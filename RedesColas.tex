
%_____________________________________________________________________________________
%
\section{Redes de Colas}
%_____________________________________________________________________________________

%_____________________________________________________________________________________
%
\subsection{Sistemas Abiertos}
%_____________________________________________________________________________________
%

Considerese un sistema con dos servidores, en los cuales los usuarios llegan de acuerdo a un proceso poisson con intensidad $\lambda_{1}$ al primer servidor, despu\'es de ser atendido se pasa a la siguiente cola en el segundo servidor. Cada servidor atiende a un usuario a la vez con tiempo exponencial con raz\'on $\mu_{i}$, para $i=1,2$. A este tipo de sistemas se les conoce como sistemas secuenciales.

Def\'inase el par $\left(n,m\right)$ como el n\'umero de usuarios en el servidor 1 y 2 respectivamente. Las ecuaciones de balance son
\begin{eqnarray}\label{Eq.Balance}
\lambda P_{0,0}&=&\mu_{2}P_{0,1}\\
\left(\lambda+\mu_{1}\right)P_{n,0}&=&\mu_{2}P_{n,1}+\lambda P_{n-1,0}\\
\left(\lambda+\mu_{2}\right)P_{0,m}&=&\mu_{2}P_{0,m+1}+\mu_{1}P_{1,m-1}\\
\left(\lambda+\mu_{1}+\mu_{2}\right)P_{n,m}&=&\mu_{2}P_{n,m+1}+\mu_{1}P_{n+1,m-1}+\lambda
P_{n-1,m}
\end{eqnarray}

Cada servidor puede ser visto como un modelo de tipo $M/M/1$, de igual manera el proceso de salida de una cola $M/M/1$ con raz\'on $\lambda$, nos permite asumir que el servidor 2 tambi\'en es una cola $M/M/1$. Adem\'as la probabilidad de que haya $n$ usuarios en el servidor 1 es
\begin{eqnarray*}
P\left\{n\textrm{ en el servidor }1\right\}&=&\left(\frac{\lambda}{\mu_{1}}\right)^{n}\left(1-\frac{\lambda}{\mu_{1}}\right)=\rho_{1}^{n}\left(1-\rho_{1}\right)\\
P\left\{m\textrm{ en el servidor }2\right\}&=&\left(\frac{\lambda}{\mu_{2}}\right)^{n}\left(1-\frac{\lambda}{\mu_{2}}\right)=\rho_{2}^{m}\left(1-\rho_{2}\right)\\
\end{eqnarray*}
Si el n\'umero de usuarios en los servidores 1 y 2 son variables aleatorias independientes, se sigue que:
\begin{equation}\label{Eq.8.16}
P_{n,m}=\rho_{1}^{n}\left(1-\rho_{1}\right)\rho_{2}^{m}\left(1-\rho_{2}\right)
\end{equation}
Verifiquemos que $P_{n,m}$ satisface las ecuaciones de balance (\ref{Eq.Balance}) Antes de eso, enunciemos unas igualdades que nos ser\'an de utilidad:
\begin{eqnarray*}
\mu_{i}\rho_{i}&=&\lambda\textrm{ para }i=1,2.\\
\lambda P_{0,0}&=&\lambda\left(1-\rho_{1}\right)\left(1-\rho_{2}\right)\\
\textrm{ y }\mu_{2} P_{0,1}&=&\mu_{2}\left(1-\rho_{1}\right)\rho_{2}\left(1-\rho_{2}\right)\Rightarrow\\
\lambda P_{0,0}&=&\mu_{2} P_{0,1}\\
\left(\lambda+\mu_{2}\right)P_{0,m}&=&\left(\lambda+\mu_{2}\right)\left(1-\rho_{1}\right)\rho_{2}^{m}\left(1-\rho_{2}\right)\\
\mu_{2}P_{0,m+1}&=&\lambda\left(1-\rho_{1}\right)\rho_{2}^{m}\left(1-\rho_{2}\right)\\
&=&\mu_{2}\left(1-\rho_{1}\right)\rho_{2}^{m}\left(1-\rho_{2}\right)\\
\mu_{1}P_{1,m-1}&=&\frac{\lambda}{\rho_{2}}\left(1-\rho_{1}\right)\rho_{2}^{m}\left(1-\rho_{2}\right)\Rightarrow\\
\left(\lambda+\mu_{2}\right)P_{0,m}&=&\mu_{2}P_{0,m+1}+\mu_{1}P_{1,m-1}\\
\left(\lambda+\mu_{1}+\mu_{2}\right)P_{n,m}&=&\left(\lambda+\mu_{1}+\mu_{2}\right)\rho^{n}\left(1-\rho_{1}\right)\rho_{2}^{m}\left(1-\rho_{2}\right)\\
\mu_{2}P_{n,m+1}&=&\mu_{2}\rho_{2}\rho_{1}^{n}\left(1-\rho_{1}\right)\rho_{2}^{m}\left(1-\rho_{2}\right)\\
\mu_{1} P_{n-1,m-1}&=&\mu_{1}\frac{\rho_{1}}{\rho_{2}}\rho_{1}^{n}\left(1-\rho_{1}\right)\rho_{2}^{m}\left(1-\rho_{2}\right)\\
\lambda P_{n-1,m}&=&\frac{\lambda}{\rho_{1}}\rho_{1}^{n}\left(1-\rho_{1}\right)\rho_{2}^{m}\left(1-\rho_{2}\right)\\
\Rightarrow\left(\lambda+\mu_{1}+\mu_{2}\right)P_{n,m}&=&\mu_{2}P_{n,m+1}+\mu_{1} P_{n-1,m-1}+\lambda P_{n-1,m}\\
\end{eqnarray*}
entonces efectivamente la ecuaci\'on (\ref{Eq.8.16}) satisface las ecuaciones de balance (\ref{Eq.Balance}). El n\'umero promedio  de usuarios en el sistema, est\'a dado por
\begin{eqnarray*}
L&=&\sum_{n,m}\left(n+m\right)P_{n,m}=\sum_{n,m}nP_{n,m}+\sum_{n,m}mP_{n,m}\\
&=&\sum_{n}\sum_{m}nP_{n,m}+\sum_{m}\sum_{n}mP_{n,m}=\sum_{n}n\sum_{m}P_{n,m}+\sum_{m}m\sum_{n}P_{n,m}\\
&=&\sum_{n}n\sum_{m}\rho_{1}^{n}\left(1-\rho_{1}\right)\rho_{2}^{m}\left(1-\rho_{2}\right)+\sum_{m}m\sum_{n}\rho_{1}^{n}\left(1-\rho_{1}\right)\rho_{2}^{m}\left(1-\rho_{2}\right)\\
&=&\sum_{n}n\rho_{1}^{n}\left(1-\rho_{1}\right)\sum_{m}\rho_{2}^{m}\left(1-\rho_{2}\right)+\sum_{m}m\rho_{2}^{m}\left(1-\rho_{2}\right)\sum_{n}\rho_{1}^{n}\left(1-\rho_{1}\right)\\
&=&\sum_{n}n\rho_{1}^{n}\left(1-\rho_{1}\right)+\sum_{m}m\rho_{2}^{m}\left(1-\rho_{2}\right)\\
&=&\frac{\lambda}{\mu_{1}-\lambda}+\frac{\lambda}{\mu_{2}-\lambda}
\end{eqnarray*}

