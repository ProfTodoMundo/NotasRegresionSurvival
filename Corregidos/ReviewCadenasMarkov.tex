\documentclass{article}
%_-_-_-_-_-_-_-_-_-_-_-_-_-_-_-_-_-_-_-_-_-_-_-_-_-_-_-
% PAQUETES A UTILIZAR
%_-_-_-_-_-_-_-_-_-_-_-_-_-_-_-_-_-_-_-_-_-_-_-_-_-_-_-
\usepackage[utf8]{inputenc}
\usepackage[spanish,english]{babel}
\usepackage{amsmath,amssymb,amsthm,amsfonts}
\usepackage{geometry}
\usepackage{hyperref}
\usepackage{fancyhdr}
\usepackage{titlesec}
\usepackage{listings}
\usepackage{graphicx,graphics}
\usepackage{multicol}
\usepackage{multirow}
\usepackage{color}
\usepackage{float} 
\usepackage{subfig}
\usepackage[figuresright]{rotating}
\usepackage{enumerate}
\usepackage{anysize} 
\usepackage{url}
\usepackage{imakeidx}
%_-_-_-_-_-_-_-_-_-_-_-_-_-_-_-_-_-_-_-_-_-_-_-_-_-_-_-
% TITULO DEL DOCUMENTO
%_-_-_-_-_-_-_-_-_-_-_-_-_-_-_-_-_-_-_-_-_-_-_-_-_-_-_-
\title{Notas sobre Sistemas de Espera\\
\small{Notes about Queueting Systems}}
\author{Carlos E. Martínez-Rodríguez}
\date{}
%_-_-_-_-_-_-_-_-_-_-_-_-_-_-_-_-_-_-_-_-_-_-_-_-_-_-_-
% MODIFICACION DE LOS MARGENES
%_-_-_-_-_-_-_-_-_-_-_-_-_-_-_-_-_-_-_-_-_-_-_-_-_-_-_-
\geometry{
  a4paper,
  left=15mm,
  right=15mm,
  left=14mm,
  right=14mm,
  top=30mm,
  bottom=30mm,
}
%_-_-_-_-_-_-_-_-_-_-_-_-_-_-_-_-_-_-_-_-_-_-_-_-_-_-_-
% CONFIGURACION DE ENCABEZADOS Y PIES DE PAG
%_-_-_-_-_-_-_-_-_-_-_-_-_-_-_-_-_-_-_-_-_-_-_-_-_-_-_-
\pagestyle{fancy}
\fancyhf{}
\fancyhead[L]{\nouppercase{\leftmark}} % Sección en el encabezado izquierdo
\fancyfoot[C]{\thepage} % Número de página centrado en el pie
\fancyfoot[L]{\tiny Carlos E. Martínez-Rodríguez} % Autor en el pie izquierdo
\fancyfoot[R]{\tiny \nouppercase{\rightmark}} % Subsection actual en el pie derecho

%_-_-_-_-_-_-_-_-_-_-_-_-_-_-_-_-_-_-_-_-_-_-_-_-_-_-_-
%_-_-_-_-_-_-_-_-_-_-_-_-_-_-_-_-_-_-_-_-_-_-_-_-_-_-_-
% Definiciones de nuevos entornos
%_-_-_-_-_-_-_-_-_-_-_-_-_-_-_-_-_-_-_-_-_-_-_-_-_-_-_-
\newtheorem{Def}{Definición}[section]
\newtheorem{Ejem}{Ejemplo}[section]
\newtheorem{Teo}{Teorema}[section]
\newtheorem{Note}{Nota}[section]
\newtheorem{Prop}{Proposición}[section]
\newtheorem{Cor}{Corolario}[section]
\newtheorem{Coro}{Corolario}[section]
\newtheorem{Lema}{Lema}[section]
\newtheorem{Lemma}{Lema}[section]
\newtheorem{Lem}{Lema}[section]
\newtheorem{Sup}{Supuestos}[section]
\newtheorem{Obs}{Observación}[section]
%_-_-_-_-_-_-_-_-_-_-_-_-_-_-_-_-_-_-_-_-_-_-_-_-_-_-_-
%NUEVOS COMANDOS
%_-_-_-_-_-_-_-_-_-_-_-_-_-_-_-_-_-_-_-_-_-_-_-_-_-_-_-
\newcommand{\nat}{\mathbb{N}}
\newcommand{\ent}{\mathbb{Z}}
\newcommand{\rea}{\mathbb{R}}
\newcommand{\Eb}{\mathbf{E}}
\newcommand{\esp}{\mathbb{E}}
\newcommand{\prob}{\mathbb{P}}
\newcommand{\indora}{\mbox{$1$\hspace{-0.8ex}$1$}}
\newcommand{\ER}{\left(E,\mathcal{E}\right)}
\newcommand{\KM}{\left(P_{s,t}\right)}
\newcommand{\PE}{\left(X_{t}\right)_{t\in I}}
\newcommand{\CM}{\mathbf{P}^{x}}
\renewcommand{\abstractname}{Resumen}
\numberwithin{equation}{section}
\newcommand{\acmclass}[1]{\noindent\textbf{ACM Class:} #1\\}
\newcommand{\mscclass}[1]{\noindent\textbf{MSC Class:} #1\\}
%_-_-_-_-_-_-_-_-_-_-_-_-_-_-_-_-_-_-_-_-_-_-_-_-_-_-_-
\makeindex

%_-_-_-_-_-_-_-_-_-_-_-_-_-_-_-_-_-_-_-_-_-_-_-_-_-_-_-
\begin{document}
%_-_-_-_-_-_-_-_-_-_-_-_-_-_-_-_-_-_-_-_-_-_-_-_-_-_-_-
\maketitle

%<<>><<>><<>><<>><<>><<>><<>><<>><<>><<>><<>><<>>
\begin{abstract}
%<<>><<>><<>><<>><<>><<>><<>><<>><<>><<>><<>><<>>
%_-_-_-_-_-_-_-_-_-_-_-_-_-_-_-_-_-_-_-_-_-_-_-_-_-_-_-

\end{abstract}

\begin{otherlanguage}{english}
\renewcommand{\abstractname}{Abstract} % Cambia "Resumen" a "Abstract"
\begin{abstract}

\end{abstract}
\end{otherlanguage}
%<<>><<>><<>><<>><<>><<>><<>><<>><<>><<>><<>><<>>

\tableofcontents
%\newpage

%<====>====<><====>====<><====>====<><====>====<><====>
%\part{Introducci\'on a Procesos Regenerativos}
%<====>====<><====>====<><====>====<><====>====<><====>
%_-_-_-_-_-_-_-_-_-_-_-_-_-_-_-_-_-_-_-_-_-_-_-_-_-_-_-
\section*{Introducción}
%_-_-_-_-_-_-_-_-_-_-_-_-_-_-_-_-_-_-_-_-_-_-_-_-_-_-_-



\begin{otherlanguage}{english}
\renewcommand{\abstractname}{Abstract} % Cambia "Resumen" a "Abstract"
\section*{Introduction}


\end{otherlanguage}

%______________________________________________________________________
\section{Ec

\begin{Teo}[Teorema de Continuidad]
Sup\'ongase que $\left\{X_{n},n=1,2,3,\ldots\right\}$ son variables aleatorias finitas, no negativas con valores enteros tales que $P\left(X_{n}=k\right)=p_{k}^{(n)}$, para $n=1,2,3,\ldots$, $k=0,1,2,\ldots$, con $\sum_{k=0}^{\infty}p_{k}^{(n)}=1$, para $n=1,2,3,\ldots$. Sea $g_{n}$ la PGF para la variable aleatoria $X_{n}$. Entonces existe una sucesi\'on $\left\{p_{k}\right\}$ tal que \begin{eqnarray*}
lim_{n\rightarrow\infty}p_{k}^{(n)}=p_{k}\textrm{ para }0<s<1.
\end{eqnarray*}
En este caso, $g\left(s\right)=\sum_{k=0}^{\infty}s^{k}p_{k}$. Adem\'as
\begin{eqnarray*}
\sum_{k=0}^{\infty}p_{k}=1\textrm{ si y s\'olo si
}lim_{s\uparrow1}g\left(s\right)=1
\end{eqnarray*}
\end{Teo}

\begin{Teo}
Sea $N$ una variable aleatoria con valores enteros no negativos finita tal que $P\left(N=k\right)=p_{k}$, para $k=0,1,2,\ldots$, y $\sum_{k=0}^{\infty}p_{k}=P\left(N<\infty\right)=1$. Sea $\Phi$ la PGF de $N$ tal que $g\left(s\right)=\esp\left[s^{N}\right]=\sum_{k=0}^{\infty}s^{k}p_{k}$ con $g\left(1\right)=1$. Si $0\leq p_{1}\leq1$ y $\esp\left[N\right]=g^{'}\left(1\right)\leq1$, entonces no existe soluci\'on  de la ecuaci\'on $g\left(s\right)=s$ en el intervalo $\left[0,1\right)$. Si $\esp\left[N\right]=g^{'}\left(1\right)>1$, lo cual implica que $0\leq p_{1}<1$, entonces existe una \'unica soluci\'on de la ecuaci\'on $g\left(s\right)=s$ en el intervalo
$\left[0,1\right)$.
\end{Teo}

\begin{Teo}
Si $X$ y $Y$ tienen PGF $G_{X}$ y $G_{Y}$ respectivamente, entonces,\[G_{X}\left(s\right)=G_{Y}\left(s\right)\] para toda $s$, si y s\'olo si \[P\left(X=k\right))=P\left(Y=k\right)\] para toda $k=0,1,\ldots,$., es decir, si y s\'olo si $X$ y $Y$ tienen la misma distribuci\'on de probabilidad.
\end{Teo}


\begin{Teo}
Para cada $n$ fijo, sea la sucesi\'oin de probabilidades $\left\{a_{0,n},a_{1,n},\ldots,\right\}$, tales que $a_{k,n}\geq0$ para toda $k=0,1,2,\ldots,$ y $\sum_{k\geq0}a_{k,n}=1$, y sea $G_{n}\left(s\right)$ la correspondiente funci\'on generadora, $G_{n}\left(s\right)=\sum_{k\geq0}a_{k,n}s^{k}$. De modo que para cada valor fijo de $k$
\begin{eqnarray*}
lim_{n\rightarrow\infty}a_{k,n}=a_{k},
\end{eqnarray*}
es decir converge en distribuci\'on, es necesario y suficiente que para cada valor fijo $s\in\left[0,\right)$,
\begin{eqnarray*}
lim_{n\rightarrow\infty}G_{n}\left(s\right)=G\left(s\right),
\end{eqnarray*}
donde $G\left(s\right)=\sum_{k\geq0}p_{k}s^{k}$, para cualquier la funci\'on generadora del l\'imite de la sucesi\'on.
\end{Teo}

\begin{Teo}[Teorema de Abel]
Sea $G\left(s\right)=\sum_{k\geq0}a_{k}s^{k}$ para cualquier $\left\{p_{0},p_{1},\ldots,\right\}$, tales que $p_{k}\geq0$ para toda $k=0,1,2,\ldots,$. Entonces $G\left(s\right)$ es continua por la derecha en $s=1$, es decir
\begin{eqnarray*}
lim_{s\uparrow1}G\left(s\right)=\sum_{k\geq0}p_{k}=G\left(\right),
\end{eqnarray*}
sin importar si la suma es finita o no.
\end{Teo}
\begin{Note}
El radio de Convergencia para cualquier PGF es $R\geq1$, entonces, el Teorema de Abel nos dice que a\'un en el peor escenario, cuando $R=1$, a\'un se puede confiar en que la PGF ser\'a continua en $s=1$, en contraste, no se puede asegurar que la PGF ser\'a continua en el l\'imite inferior $-R$, puesto que la PGF es sim\'etrica alrededor del cero: la PGF converge para todo $s\in\left(-R,R\right)$, y no lo hace para $s<-R$ o $s>R$. Adem\'as nos dice que podemos escribir $G_{X}\left(1\right)$ como una abreviaci\'on de $lim_{s\uparrow1}G_{X}\left(s\right)$.
\end{Note}

Entonces si suponemos que la diferenciaci\'on t\'ermino a t\'ermino est\'a permitida, entonces

\begin{eqnarray*}
G_{X}^{'}\left(s\right)&=&\sum_{x=1}^{\infty}xs^{x-1}p_{x}
\end{eqnarray*}

el Teorema de Abel nos dice que
\begin{eqnarray*}
\esp\left(X\right]&=&\lim_{s\uparrow1}G_{X}^{'}\left(s\right):\\
\esp\left[X\right]&=&=\sum_{x=1}^{\infty}xp_{x}=G_{X}^{'}\left(1\right)\\
&=&\lim_{s\uparrow1}G_{X}^{'}\left(s\right),
\end{eqnarray*}
dado que el Teorema de Abel se aplica a
\begin{eqnarray*}
G_{X}^{'}\left(s\right)&=&\sum_{x=1}^{\infty}xs^{x-1}p_{x},
\end{eqnarray*}
estableciendo as\'i que $G_{X}^{'}\left(s\right)$ es continua en $s=1$. Sin el Teorema de Abel no se podr\'ia asegurar que el l\'imite de $G_{X}^{'}\left(s\right)$ conforme $s\uparrow1$ sea la respuesta correcta para $\esp\left[X\right]$.

\begin{Note}
La PGF converge para todo $|s|<R$, para alg\'un $R$. De hecho la PGF converge absolutamente si $|s|<R$. La PGF adem\'as converge uniformemente en conjuntos de la forma $\left\{s:|s|<R^{'}\right\}$, donde $R^{'}<R$, es decir, $\forall\epsilon>0, \exists n_{0}\in\ent$ tal que $\forall s$, con $|s|<R^{'}$, y $\forall n\geq n_{0}$,
\begin{eqnarray*}
|\sum_{x=0}^{n}s^{x}\prob\left(X=x\right)-G_{X}\left(s\right)|<\epsilon.
\end{eqnarray*}
De hecho, la convergencia uniforme es la que nos permite diferenciar t\'ermino a t\'ermino:
\begin{eqnarray*}
G_{X}\left(s\right)=\esp\left[s^{X}\right]=\sum_{x=0}^{\infty}s^{x}\prob\left(X=x\right),
\end{eqnarray*}
y sea $s<R$.
\begin{enumerate}
\item
\begin{eqnarray*}
G_{X}^{'}\left(s\right)&=&\frac{d}{ds}\left(\sum_{x=0}^{\infty}s^{x}\prob\left(X=x\right)\right)=\sum_{x=0}^{\infty}\frac{d}{ds}\left(s^{x}\prob\left(X=x\right)\right)\\
&=&\sum_{x=0}^{n}xs^{x-1}\prob\left(X=x\right).
\end{eqnarray*}

\item\begin{eqnarray*}
\int_{a}^{b}G_{X}\left(s\right)ds&=&\int_{a}^{b}\left(\sum_{x=0}^{\infty}s^{x}\prob\left(X=x\right)\right)ds=\sum_{x=0}^{\infty}\left(\int_{a}^{b}s^{x}\prob\left(X=x\right)ds\right)\\
&=&\sum_{x=0}^{\infty}\frac{s^{x+1}}{x+1}\prob\left(X=x\right),
\end{eqnarray*}
para $-R<a<b<R$.
\end{enumerate}
\end{Note}

\begin{Teo}[Teorema de Convergencia Mon\'otona para PGF]
Sean $X$ y $X_{n}$ variables aleatorias no negativas, con valores en los enteros, finitas, tales que
\begin{eqnarray*}
lim_{n\rightarrow\infty}G_{X_{n}}\left(s\right)&=&G_{X}\left(s\right)
\end{eqnarray*}
para $0\leq s\leq1$, entonces
\begin{eqnarray*}
lim_{n\rightarrow\infty}P\left(X_{n}=k\right)=P\left(X=k\right),
\end{eqnarray*}
para $k=0,1,2,\ldots.$
\end{Teo}

El teorema anterior requiere del siguiente lema

\begin{Lemma}
Sean $a_{n,k}\in\ent^{+}$, $n\in\nat$ constantes no negativas con $\sum_{k\geq0}a_{k,n}\leq1$. Sup\'ongase que para $0\leq s\leq1$,
se tiene
\begin{eqnarray*}
a_{n}\left(s\right)&=&\sum_{k=0}^{\infty}a_{k,n}s^{k}\rightarrow
a\left(s\right)=\sum_{k=0}^{\infty}a_{k}s^{k}.
\end{eqnarray*}
Entonces
\begin{eqnarray*}
a_{0,n}\rightarrow a_{0}.
\end{eqnarray*}
\end{Lemma}


%_________________________________________________________________________
\section{Redes de Jackson}
%_________________________________________________________________________
Cuando se considera la cantidad de
usuarios que llegan a cada uno de los nodos desde fuera del
sistema m\'as los que provienen del resto de los nodos, se dice
que la red es abierta y recibe el nombre de {\em Red de Jackson Abierta}.\\

Si denotamos por $Q_{1}\left(t\right),Q_{2}\left(t\right),\ldots,Q_{K}\left(t\right)$ el n\'umero de usuarios presentes en la cola $1,2,\ldots,K$ respectivamente al tiempo $t$, entonces se tiene la colecci\'on de colas $\left\{Q_{1},Q_{2},\ldots,Q_{K}\right\}$, donde despu\'es de que el usuario es atendido en la cola $i$, se traslada a la cola $j$ con probabilidad $p_{ij}$. En caso de que un usuario decida volver a ser atendido en $i$, este permanecer\'a en la misma cola con probabilidad $p_{ii}$. Para considerar a los usuarios que entran al sistema por primera vez por $i$, m\'as aquellos que provienen de otra cola, es necesario considerar un estado adicional $0$, con probabilidad de transici\'on $p_{00}=0$, $p_{0j}\geq0$ y $p_{j0}\geq0$, para $j=1,2,\ldots,K$, entonces en general la probabilidad de transici\'on de una cola a otra puede representarse por $P=\left(p_{ij}\right)_{i,j=0}^{K}$.\\

Para el caso espec\'ifico en el que en cada una de las colas los tiempos entre arribos y los tiempos de servicio sean exponenciales con par\'ametro de intensidad $\lambda$ y media $\mu$, respectivamente, con $m$ servidores y sin restricciones en la capacidad de almacenamiento en cada una de las colas, en Chee-Hook y Boon-Hee \cite{HookHee}, cap. 6, se muestra que el n\'umero de
usuarios en las $K$ colas, en el caso estacionario, puede determinarse por la ecuaci\'on (\ref{Eq.7.5.1})  que a
continuaci\'on se presenta, adem\'as de que la distribuci\'on l\'imite de la misma es (\ref{Eq.7.5.2}).\\

El n\'umero de usuarios en las $K$ colas en su estado estacionario, ver \cite{Bhat}, se define como
\begin{equation}\label{Eq.7.5.1}
p_{q_{1}q_{2}\cdots
q_{K}}=P\left[Q_{1}=q_{1},Q_{2}=q_{2},\ldots,Q_{K}=q_{K}\right].
\end{equation}

Jackson (1957), demostr\'o que la distribuci\'on l\'imite
$p_{q_{1}q_{2}\cdots q_{K}}$ de (\ref{Eq.7.5.1}) es

\begin{equation}\label{Eq.7.5.2}
p_{q_{1}q_{2}\cdots
q_{K}}=P_{1}\left(q_{1}\right)P_{2}\left(q_{2}\right)\cdots
P_{K}\left(q_{K}\right),
\end{equation}

donde
\begin{equation}\label{Eq.7.5.3}
p_{i}\left(r\right)=\left\{\begin{array}{cc}
 p_{i}\left(0\right)\frac{\left(\gamma_{i}/\mu_{i}\right)^{r}}{r!},  & r=0,1,2,\ldots,m, \\
 p_{i}\left(0\right)\frac{\left(\gamma_{i}/\mu_{i}\right)^{r}}{m!m^{r-m}}, & r=m,m+1,\ldots .\\
\end{array}\right.
\end{equation}

y

\begin{equation}\label{Eq.7.5.4}
\gamma_{i}=\lambda_{i}+\sum p_{ji}\gamma_{j},\textrm{
}i=1,2,\ldots,K.
\end{equation}

La relaci\'on (\ref{Eq.7.5.4}) es importante puesto que considera no solamente los arribos externos si no que adem\'as permite considerar intercambio de clientes entre las distintas colas que conforman el sistema.\\

Dados $\lambda_{i}$ y $p_{ij}$, la cantidad $\gamma_{i}$ puede determinarse a partir de la ecuaci\'on (\ref{Eq.7.5.4}) de manera recursiva. Adem\'as $p_{i}\left(0\right)$ puede determinarse utilizando la condici\'on de normalidad
\[\sum_{q_{1}}\sum_{q_{2}}\cdots\sum_{q_{K}}p_{q_{1}q_{2}\cdots q_{K}}=1.\]

Sin embargo las Redes de Jackson tienen el inconveniente de que no consideran el caso en que existan tiempos de traslado entre las colas. 



\section{Resultados Adicionales}


%_______________________________________________________________________________________
\subsection{Procesos de Renovaci\'on y Regenerativos}
%_______________________________________________________________________________________




En Sigman, Thorison y Wolff \cite{Sigman2} prueban que para la existencia de un una sucesi\'on infinita no decreciente de tiempos de regeneraci\'on $\tau_{1}\leq\tau_{2}\leq\cdots$ en los cuales el proceso se regenera, basta un tiempo de regeneraci\'on $R_{1}$, donde $R_{j}=\tau_{j}-\tau_{j-1}$. Para tal efecto se requiere la existencia de un espacio de probabilidad $\left(\Omega,\mathcal{F},\prob\right)$, y proceso estoc\'astico $\textit{X}=\left\{X\left(t\right):t\geq0\right\}$ con espacio de estados $\left(S,\mathcal{R}\right)$, con $\mathcal{R}$ $\sigma$-\'algebra.

\begin{Prop}
Si existe una variable aleatoria no negativa $R_{1}$ tal que $\theta_{R1}X=_{D}X$, entonces $\left(\Omega,\mathcal{F},\prob\right)$ puede extenderse para soportar una sucesi\'on estacionaria de variables aleatorias $R=\left\{R_{k}:k\geq1\right\}$, tal que para $k\geq1$,
\begin{eqnarray*}
\theta_{k}\left(X,R\right)=_{D}\left(X,R\right).
\end{eqnarray*}

Adem\'as, para $k\geq1$, $\theta_{k}R$ es condicionalmente independiente de $\left(X,R_{1},\ldots,R_{k}\right)$, dado $\theta_{\tau k}X$.

\end{Prop}


\begin{itemize}
\item Doob en 1953 demostr\'o que el estado estacionario de un proceso de partida en un sistema de espera $M/G/\infty$, es Poisson con la misma tasa que el proceso de arribos.

\item Burke en 1968, fue el primero en demostrar que el estado estacionario de un proceso de salida de una cola $M/M/s$ es un proceso Poisson.

\item Disney en 1973 obtuvo el siguiente resultado:

\begin{Teo}
Para el sistema de espera $M/G/1/L$ con disciplina FIFO, el proceso $\textbf{I}$ es un proceso de renovaci\'on si y s\'olo si el proceso denominado longitud de la cola es estacionario y se cumple cualquiera de los siguientes casos:

\begin{itemize}
\item[a)] Los tiempos de servicio son identicamente cero;
\item[b)] $L=0$, para cualquier proceso de servicio $S$;
\item[c)] $L=1$ y $G=D$;
\item[d)] $L=\infty$ y $G=M$.
\end{itemize}
En estos casos, respectivamente, las distribuciones de interpartida $P\left\{T_{n+1}-T_{n}\leq t\right\}$ son


\begin{itemize}
\item[a)] $1-e^{-\lambda t}$, $t\geq0$;
\item[b)] $1-e^{-\lambda t}*F\left(t\right)$, $t\geq0$;
\item[c)] $1-e^{-\lambda t}*\indora_{d}\left(t\right)$, $t\geq0$;
\item[d)] $1-e^{-\lambda t}*F\left(t\right)$, $t\geq0$.
\end{itemize}
\end{Teo}


\item Finch (1959) mostr\'o que para los sistemas $M/G/1/L$, con $1\leq L\leq \infty$ con distribuciones de servicio dos veces diferenciable, solamente el sistema $M/M/1/\infty$ tiene proceso de salida de renovaci\'on estacionario.

\item King (1971) demostro que un sistema de colas estacionario $M/G/1/1$ tiene sus tiempos de interpartida sucesivas $D_{n}$ y $D_{n+1}$ son independientes, si y s\'olo si, $G=D$, en cuyo caso le proceso de salida es de renovaci\'on.

\item Disney (1973) demostr\'o que el \'unico sistema estacionario $M/G/1/L$, que tiene proceso de salida de renovaci\'on  son los sistemas $M/M/1$ y $M/D/1/1$.



\item El siguiente resultado es de Disney y Koning (1985)
\begin{Teo}
En un sistema de espera $M/G/s$, el estado estacionario del proceso de salida es un proceso Poisson para cualquier distribuci\'on de los tiempos de servicio si el sistema tiene cualquiera de las siguientes cuatro propiedades.

\begin{itemize}
\item[a)] $s=\infty$
\item[b)] La disciplina de servicio es de procesador compartido.
\item[c)] La disciplina de servicio es LCFS y preemptive resume, esto se cumple para $L<\infty$
\item[d)] $G=M$.
\end{itemize}

\end{Teo}

\item El siguiente resultado es de Alamatsaz (1983)

\begin{Teo}
En cualquier sistema de colas $GI/G/1/L$ con $1\leq L<\infty$ y distribuci\'on de interarribos $A$ y distribuci\'on de los tiempos de servicio $B$, tal que $A\left(0\right)=0$, $A\left(t\right)\left(1-B\left(t\right)\right)>0$ para alguna $t>0$ y $B\left(t\right)$ para toda $t>0$, es imposible que el proceso de salida estacionario sea de renovaci\'on.
\end{Teo}

\end{itemize}



%________________________________________________________________________
%\subsection{Procesos Regenerativos Sigman, Thorisson y Wolff \cite{Sigman1}}
%________________________________________________________________________


\begin{Def}[Definici\'on Cl\'asica]
Un proceso estoc\'astico $X=\left\{X\left(t\right):t\geq0\right\}$ es llamado regenerativo is existe una variable aleatoria $R_{1}>0$ tal que
\begin{itemize}
\item[i)] $\left\{X\left(t+R_{1}\right):t\geq0\right\}$ es independiente de $\left\{\left\{X\left(t\right):t<R_{1}\right\},\right\}$
\item[ii)] $\left\{X\left(t+R_{1}\right):t\geq0\right\}$ es estoc\'asticamente equivalente a $\left\{X\left(t\right):t>0\right\}$
\end{itemize}

Llamamos a $R_{1}$ tiempo de regeneraci\'on, y decimos que $X$ se regenera en este punto.
\end{Def}

$\left\{X\left(t+R_{1}\right)\right\}$ es regenerativo con tiempo de regeneraci\'on $R_{2}$, independiente de $R_{1}$ pero con la misma distribuci\'on que $R_{1}$. Procediendo de esta manera se obtiene una secuencia de variables aleatorias independientes e id\'enticamente distribuidas $\left\{R_{n}\right\}$ llamados longitudes de ciclo. Si definimos a $Z_{k}\equiv R_{1}+R_{2}+\cdots+R_{k}$, se tiene un proceso de renovaci\'on llamado proceso de renovaci\'on encajado para $X$.


\begin{Note}
La existencia de un primer tiempo de regeneraci\'on, $R_{1}$, implica la existencia de una sucesi\'on completa de estos tiempos $R_{1},R_{2}\ldots,$ que satisfacen la propiedad deseada \cite{Sigman2}.
\end{Note}


\begin{Note} Para la cola $GI/GI/1$ los usuarios arriban con tiempos $t_{n}$ y son atendidos con tiempos de servicio $S_{n}$, los tiempos de arribo forman un proceso de renovaci\'on  con tiempos entre arribos independientes e identicamente distribuidos (\texttt{i.i.d.})$T_{n}=t_{n}-t_{n-1}$, adem\'as los tiempos de servicio son \texttt{i.i.d.} e independientes de los procesos de arribo. Por \textit{estable} se entiende que $\esp S_{n}<\esp T_{n}<\infty$.
\end{Note}
 


\begin{Def}
Para $x$ fijo y para cada $t\geq0$, sea $I_{x}\left(t\right)=1$ si $X\left(t\right)\leq x$,  $I_{x}\left(t\right)=0$ en caso contrario, y def\'inanse los tiempos promedio
\begin{eqnarray*}
\overline{X}&=&lim_{t\rightarrow\infty}\frac{1}{t}\int_{0}^{\infty}X\left(u\right)du\\
\prob\left(X_{\infty}\leq x\right)&=&lim_{t\rightarrow\infty}\frac{1}{t}\int_{0}^{\infty}I_{x}\left(u\right)du,
\end{eqnarray*}
cuando estos l\'imites existan.
\end{Def}

Como consecuencia del teorema de Renovaci\'on-Recompensa, se tiene que el primer l\'imite  existe y es igual a la constante
\begin{eqnarray*}
\overline{X}&=&\frac{\esp\left[\int_{0}^{R_{1}}X\left(t\right)dt\right]}{\esp\left[R_{1}\right]},
\end{eqnarray*}
suponiendo que ambas esperanzas son finitas.
 
\begin{Note}
Funciones de procesos regenerativos son regenerativas, es decir, si $X\left(t\right)$ es regenerativo y se define el proceso $Y\left(t\right)$ por $Y\left(t\right)=f\left(X\left(t\right)\right)$ para alguna funci\'on Borel medible $f\left(\cdot\right)$. Adem\'as $Y$ es regenerativo con los mismos tiempos de renovaci\'on que $X$. 

En general, los tiempos de renovaci\'on, $Z_{k}$ de un proceso regenerativo no requieren ser tiempos de paro con respecto a la evoluci\'on de $X\left(t\right)$.
\end{Note} 

\begin{Note}
Una funci\'on de un proceso de Markov, usualmente no ser\'a un proceso de Markov, sin embargo ser\'a regenerativo si el proceso de Markov lo es.
\end{Note}

 
\begin{Note}
Un proceso regenerativo con media de la longitud de ciclo finita es llamado positivo recurrente.
\end{Note}


\begin{Note}
\begin{itemize}
\item[a)] Si el proceso regenerativo $X$ es positivo recurrente y tiene trayectorias muestrales no negativas, entonces la ecuaci\'on anterior es v\'alida.
\item[b)] Si $X$ es positivo recurrente regenerativo, podemos construir una \'unica versi\'on estacionaria de este proceso, $X_{e}=\left\{X_{e}\left(t\right)\right\}$, donde $X_{e}$ es un proceso estoc\'astico regenerativo y estrictamente estacionario, con distribuci\'on marginal distribuida como $X_{\infty}$
\end{itemize}
\end{Note}


%__________________________________________________________________________________________
%\subsection{Procesos Regenerativos Estacionarios - Stidham \cite{Stidham}}
%__________________________________________________________________________________________


Un proceso estoc\'astico a tiempo continuo $\left\{V\left(t\right),t\geq0\right\}$ es un proceso regenerativo si existe una sucesi\'on de variables aleatorias independientes e id\'enticamente distribuidas $\left\{X_{1},X_{2},\ldots\right\}$, sucesi\'on de renovaci\'on, tal que para cualquier conjunto de Borel $A$, 

\begin{eqnarray*}
\prob\left\{V\left(t\right)\in A|X_{1}+X_{2}+\cdots+X_{R\left(t\right)}=s,\left\{V\left(\tau\right),\tau<s\right\}\right\}=\prob\left\{V\left(t-s\right)\in A|X_{1}>t-s\right\},
\end{eqnarray*}
para todo $0\leq s\leq t$, donde $R\left(t\right)=\max\left\{X_{1}+X_{2}+\cdots+X_{j}\leq t\right\}=$n\'umero de renovaciones ({\emph{puntos de regeneraci\'on}}) que ocurren en $\left[0,t\right]$. El intervalo $\left[0,X_{1}\right)$ es llamado {\emph{primer ciclo de regeneraci\'on}} de $\left\{V\left(t \right),t\geq0\right\}$, $\left[X_{1},X_{1}+X_{2}\right)$ el {\emph{segundo ciclo de regeneraci\'on}}, y as\'i sucesivamente.

Sea $X=X_{1}$ y sea $F$ la funci\'on de distrbuci\'on de $X$


\begin{Def}
Se define el proceso estacionario, $\left\{V^{*}\left(t\right),t\geq0\right\}$, para $\left\{V\left(t\right),t\geq0\right\}$ por

\begin{eqnarray*}
\prob\left\{V\left(t\right)\in A\right\}=\frac{1}{\esp\left[X\right]}\int_{0}^{\infty}\prob\left\{V\left(t+x\right)\in A|X>x\right\}\left(1-F\left(x\right)\right)dx,
\end{eqnarray*} 
para todo $t\geq0$ y todo conjunto de Borel $A$.
\end{Def}

\begin{Def}
Una distribuci\'on se dice que es {\emph{aritm\'etica}} si todos sus puntos de incremento son m\'ultiplos de la forma $0,\lambda, 2\lambda,\ldots$ para alguna $\lambda>0$ entera.
\end{Def}


\begin{Def}
Una modificaci\'on medible de un proceso $\left\{V\left(t\right),t\geq0\right\}$, es una versi\'on de este, $\left\{V\left(t,w\right)\right\}$ conjuntamente medible para $t\geq0$ y para $w\in S$, $S$ espacio de estados para $\left\{V\left(t\right),t\geq0\right\}$.
\end{Def}

\begin{Teo}
Sea $\left\{V\left(t\right),t\geq\right\}$ un proceso regenerativo no negativo con modificaci\'on medible. Sea $\esp\left[X\right]<\infty$. Entonces el proceso estacionario dado por la ecuaci\'on anterior est\'a bien definido y tiene funci\'on de distribuci\'on independiente de $t$, adem\'as
\begin{itemize}
\item[i)] \begin{eqnarray*}
\esp\left[V^{*}\left(0\right)\right]&=&\frac{\esp\left[\int_{0}^{X}V\left(s\right)ds\right]}{\esp\left[X\right]}\end{eqnarray*}
\item[ii)] Si $\esp\left[V^{*}\left(0\right)\right]<\infty$, equivalentemente, si $\esp\left[\int_{0}^{X}V\left(s\right)ds\right]<\infty$,entonces
\begin{eqnarray*}
\frac{\int_{0}^{t}V\left(s\right)ds}{t}\rightarrow\frac{\esp\left[\int_{0}^{X}V\left(s\right)ds\right]}{\esp\left[X\right]}
\end{eqnarray*}
con probabilidad 1 y en media, cuando $t\rightarrow\infty$.
\end{itemize}
\end{Teo}

\begin{Coro}
Sea $\left\{V\left(t\right),t\geq0\right\}$ un proceso regenerativo no negativo, con modificaci\'on medible. Si $\esp <\infty$, $F$ es no-aritm\'etica, y para todo $x\geq0$, $P\left\{V\left(t\right)\leq x,C>x\right\}$ es de variaci\'on acotada como funci\'on de $t$ en cada intervalo finito $\left[0,\tau\right]$, entonces $V\left(t\right)$ converge en distribuci\'on  cuando $t\rightarrow\infty$ y $$\esp V=\frac{\esp \int_{0}^{X}V\left(s\right)ds}{\esp X}$$
Donde $V$ tiene la distribuci\'on l\'imite de $V\left(t\right)$ cuando $t\rightarrow\infty$.

\end{Coro}

Para el caso discreto se tienen resultados similares.



%______________________________________________________________________
%\subsection{Procesos de Renovaci\'on}
%______________________________________________________________________

\begin{Def}%\label{Def.Tn}
Sean $0\leq T_{1}\leq T_{2}\leq \ldots$ son tiempos aleatorios infinitos en los cuales ocurren ciertos eventos. El n\'umero de tiempos $T_{n}$ en el intervalo $\left[0,t\right)$ es

\begin{eqnarray}
N\left(t\right)=\sum_{n=1}^{\infty}\indora\left(T_{n}\leq t\right),
\end{eqnarray}
para $t\geq0$.
\end{Def}

Si se consideran los puntos $T_{n}$ como elementos de $\rea_{+}$, y $N\left(t\right)$ es el n\'umero de puntos en $\rea$. El proceso denotado por $\left\{N\left(t\right):t\geq0\right\}$, denotado por $N\left(t\right)$, es un proceso puntual en $\rea_{+}$. Los $T_{n}$ son los tiempos de ocurrencia, el proceso puntual $N\left(t\right)$ es simple si su n\'umero de ocurrencias son distintas: $0<T_{1}<T_{2}<\ldots$ casi seguramente.

\begin{Def}
Un proceso puntual $N\left(t\right)$ es un proceso de renovaci\'on si los tiempos de interocurrencia $\xi_{n}=T_{n}-T_{n-1}$, para $n\geq1$, son independientes e identicamente distribuidos con distribuci\'on $F$, donde $F\left(0\right)=0$ y $T_{0}=0$. Los $T_{n}$ son llamados tiempos de renovaci\'on, referente a la independencia o renovaci\'on de la informaci\'on estoc\'astica en estos tiempos. Los $\xi_{n}$ son los tiempos de inter-renovaci\'on, y $N\left(t\right)$ es el n\'umero de renovaciones en el intervalo $\left[0,t\right)$
\end{Def}


\begin{Note}
Para definir un proceso de renovaci\'on para cualquier contexto, solamente hay que especificar una distribuci\'on $F$, con $F\left(0\right)=0$, para los tiempos de inter-renovaci\'on. La funci\'on $F$ en turno degune las otra variables aleatorias. De manera formal, existe un espacio de probabilidad y una sucesi\'on de variables aleatorias $\xi_{1},\xi_{2},\ldots$ definidas en este con distribuci\'on $F$. Entonces las otras cantidades son $T_{n}=\sum_{k=1}^{n}\xi_{k}$ y $N\left(t\right)=\sum_{n=1}^{\infty}\indora\left(T_{n}\leq t\right)$, donde $T_{n}\rightarrow\infty$ casi seguramente por la Ley Fuerte de los Grandes Números.
\end{Note}

%___________________________________________________________________________________________
%
%\subsection{Teorema Principal de Renovaci\'on}
%___________________________________________________________________________________________
%

\begin{Note} Una funci\'on $h:\rea_{+}\rightarrow\rea$ es Directamente Riemann Integrable en los siguientes casos:
\begin{itemize}
\item[a)] $h\left(t\right)\geq0$ es decreciente y Riemann Integrable.
\item[b)] $h$ es continua excepto posiblemente en un conjunto de Lebesgue de medida 0, y $|h\left(t\right)|\leq b\left(t\right)$, donde $b$ es DRI.
\end{itemize}
\end{Note}

\begin{Teo}[Teorema Principal de Renovaci\'on]
Si $F$ es no aritm\'etica y $h\left(t\right)$ es Directamente Riemann Integrable (DRI), entonces

\begin{eqnarray*}
lim_{t\rightarrow\infty}U\star h=\frac{1}{\mu}\int_{\rea_{+}}h\left(s\right)ds.
\end{eqnarray*}
\end{Teo}

\begin{Prop}
Cualquier funci\'on $H\left(t\right)$ acotada en intervalos finitos y que es 0 para $t<0$ puede expresarse como
\begin{eqnarray*}
H\left(t\right)=U\star h\left(t\right)\textrm{,  donde }h\left(t\right)=H\left(t\right)-F\star H\left(t\right)
\end{eqnarray*}
\end{Prop}

\begin{Def}
Un proceso estoc\'astico $X\left(t\right)$ es crudamente regenerativo en un tiempo aleatorio positivo $T$ si
\begin{eqnarray*}
\esp\left[X\left(T+t\right)|T\right]=\esp\left[X\left(t\right)\right]\textrm{, para }t\geq0,\end{eqnarray*}
y con las esperanzas anteriores finitas.
\end{Def}

\begin{Prop}
Sup\'ongase que $X\left(t\right)$ es un proceso crudamente regenerativo en $T$, que tiene distribuci\'on $F$. Si $\esp\left[X\left(t\right)\right]$ es acotado en intervalos finitos, entonces
\begin{eqnarray*}
\esp\left[X\left(t\right)\right]=U\star h\left(t\right)\textrm{,  donde }h\left(t\right)=\esp\left[X\left(t\right)\indora\left(T>t\right)\right].
\end{eqnarray*}
\end{Prop}

\begin{Teo}[Regeneraci\'on Cruda]
Sup\'ongase que $X\left(t\right)$ es un proceso con valores positivo crudamente regenerativo en $T$, y def\'inase $M=\sup\left\{|X\left(t\right)|:t\leq T\right\}$. Si $T$ es no aritm\'etico y $M$ y $MT$ tienen media finita, entonces
\begin{eqnarray*}
lim_{t\rightarrow\infty}\esp\left[X\left(t\right)\right]=\frac{1}{\mu}\int_{\rea_{+}}h\left(s\right)ds,
\end{eqnarray*}
donde $h\left(t\right)=\esp\left[X\left(t\right)\indora\left(T>t\right)\right]$.
\end{Teo}

%___________________________________________________________________________________________
%
%\subsection{Propiedades de los Procesos de Renovaci\'on}
%___________________________________________________________________________________________
%

Los tiempos $T_{n}$ est\'an relacionados con los conteos de $N\left(t\right)$ por

\begin{eqnarray*}
\left\{N\left(t\right)\geq n\right\}&=&\left\{T_{n}\leq t\right\}\\
T_{N\left(t\right)}\leq &t&<T_{N\left(t\right)+1},
\end{eqnarray*}

adem\'as $N\left(T_{n}\right)=n$, y 

\begin{eqnarray*}
N\left(t\right)=\max\left\{n:T_{n}\leq t\right\}=\min\left\{n:T_{n+1}>t\right\}
\end{eqnarray*}

Por propiedades de la convoluci\'on se sabe que

\begin{eqnarray*}
P\left\{T_{n}\leq t\right\}=F^{n\star}\left(t\right)
\end{eqnarray*}
que es la $n$-\'esima convoluci\'on de $F$. Entonces 

\begin{eqnarray*}
\left\{N\left(t\right)\geq n\right\}&=&\left\{T_{n}\leq t\right\}\\
P\left\{N\left(t\right)\leq n\right\}&=&1-F^{\left(n+1\right)\star}\left(t\right)
\end{eqnarray*}

Adem\'as usando el hecho de que $\esp\left[N\left(t\right)\right]=\sum_{n=1}^{\infty}P\left\{N\left(t\right)\geq n\right\}$
se tiene que

\begin{eqnarray*}
\esp\left[N\left(t\right)\right]=\sum_{n=1}^{\infty}F^{n\star}\left(t\right)
\end{eqnarray*}

\begin{Prop}
Para cada $t\geq0$, la funci\'on generadora de momentos $\esp\left[e^{\alpha N\left(t\right)}\right]$ existe para alguna $\alpha$ en una vecindad del 0, y de aqu\'i que $\esp\left[N\left(t\right)^{m}\right]<\infty$, para $m\geq1$.
\end{Prop}


\begin{Note}
Si el primer tiempo de renovaci\'on $\xi_{1}$ no tiene la misma distribuci\'on que el resto de las $\xi_{n}$, para $n\geq2$, a $N\left(t\right)$ se le llama Proceso de Renovaci\'on retardado, donde si $\xi$ tiene distribuci\'on $G$, entonces el tiempo $T_{n}$ de la $n$-\'esima renovaci\'on tiene distribuci\'on $G\star F^{\left(n-1\right)\star}\left(t\right)$
\end{Note}


\begin{Teo}
Para una constante $\mu\leq\infty$ ( o variable aleatoria), las siguientes expresiones son equivalentes:

\begin{eqnarray}
lim_{n\rightarrow\infty}n^{-1}T_{n}&=&\mu,\textrm{ c.s.}\\
lim_{t\rightarrow\infty}t^{-1}N\left(t\right)&=&1/\mu,\textrm{ c.s.}
\end{eqnarray}
\end{Teo}


Es decir, $T_{n}$ satisface la Ley Fuerte de los Grandes N\'umeros s\'i y s\'olo s\'i $N\left/t\right)$ la cumple.


\begin{Coro}[Ley Fuerte de los Grandes N\'umeros para Procesos de Renovaci\'on]
Si $N\left(t\right)$ es un proceso de renovaci\'on cuyos tiempos de inter-renovaci\'on tienen media $\mu\leq\infty$, entonces
\begin{eqnarray}
t^{-1}N\left(t\right)\rightarrow 1/\mu,\textrm{ c.s. cuando }t\rightarrow\infty.
\end{eqnarray}

\end{Coro}


Considerar el proceso estoc\'astico de valores reales $\left\{Z\left(t\right):t\geq0\right\}$ en el mismo espacio de probabilidad que $N\left(t\right)$

\begin{Def}
Para el proceso $\left\{Z\left(t\right):t\geq0\right\}$ se define la fluctuaci\'on m\'axima de $Z\left(t\right)$ en el intervalo $\left(T_{n-1},T_{n}\right]$:
\begin{eqnarray*}
M_{n}=\sup_{T_{n-1}<t\leq T_{n}}|Z\left(t\right)-Z\left(T_{n-1}\right)|
\end{eqnarray*}
\end{Def}

\begin{Teo}
Sup\'ongase que $n^{-1}T_{n}\rightarrow\mu$ c.s. cuando $n\rightarrow\infty$, donde $\mu\leq\infty$ es una constante o variable aleatoria. Sea $a$ una constante o variable aleatoria que puede ser infinita cuando $\mu$ es finita, y considere las expresiones l\'imite:
\begin{eqnarray}
lim_{n\rightarrow\infty}n^{-1}Z\left(T_{n}\right)&=&a,\textrm{ c.s.}\\
lim_{t\rightarrow\infty}t^{-1}Z\left(t\right)&=&a/\mu,\textrm{ c.s.}
\end{eqnarray}
La segunda expresi\'on implica la primera. Conversamente, la primera implica la segunda si el proceso $Z\left(t\right)$ es creciente, o si $lim_{n\rightarrow\infty}n^{-1}M_{n}=0$ c.s.
\end{Teo}

\begin{Coro}
Si $N\left(t\right)$ es un proceso de renovaci\'on, y $\left(Z\left(T_{n}\right)-Z\left(T_{n-1}\right),M_{n}\right)$, para $n\geq1$, son variables aleatorias independientes e id\'enticamente distribuidas con media finita, entonces,
\begin{eqnarray}
lim_{t\rightarrow\infty}t^{-1}Z\left(t\right)\rightarrow\frac{\esp\left[Z\left(T_{1}\right)-Z\left(T_{0}\right)\right]}{\esp\left[T_{1}\right]},\textrm{ c.s. cuando  }t\rightarrow\infty.
\end{eqnarray}
\end{Coro}



%___________________________________________________________________________________________
%
%\subsection{Propiedades de los Procesos de Renovaci\'on}
%___________________________________________________________________________________________
%

Los tiempos $T_{n}$ est\'an relacionados con los conteos de $N\left(t\right)$ por

\begin{eqnarray*}
\left\{N\left(t\right)\geq n\right\}&=&\left\{T_{n}\leq t\right\}\\
T_{N\left(t\right)}\leq &t&<T_{N\left(t\right)+1},
\end{eqnarray*}

adem\'as $N\left(T_{n}\right)=n$, y 

\begin{eqnarray*}
N\left(t\right)=\max\left\{n:T_{n}\leq t\right\}=\min\left\{n:T_{n+1}>t\right\}
\end{eqnarray*}

Por propiedades de la convoluci\'on se sabe que

\begin{eqnarray*}
P\left\{T_{n}\leq t\right\}=F^{n\star}\left(t\right)
\end{eqnarray*}
que es la $n$-\'esima convoluci\'on de $F$. Entonces 

\begin{eqnarray*}
\left\{N\left(t\right)\geq n\right\}&=&\left\{T_{n}\leq t\right\}\\
P\left\{N\left(t\right)\leq n\right\}&=&1-F^{\left(n+1\right)\star}\left(t\right)
\end{eqnarray*}

Adem\'as usando el hecho de que $\esp\left[N\left(t\right)\right]=\sum_{n=1}^{\infty}P\left\{N\left(t\right)\geq n\right\}$
se tiene que

\begin{eqnarray*}
\esp\left[N\left(t\right)\right]=\sum_{n=1}^{\infty}F^{n\star}\left(t\right)
\end{eqnarray*}

\begin{Prop}
Para cada $t\geq0$, la funci\'on generadora de momentos $\esp\left[e^{\alpha N\left(t\right)}\right]$ existe para alguna $\alpha$ en una vecindad del 0, y de aqu\'i que $\esp\left[N\left(t\right)^{m}\right]<\infty$, para $m\geq1$.
\end{Prop}


\begin{Note}
Si el primer tiempo de renovaci\'on $\xi_{1}$ no tiene la misma distribuci\'on que el resto de las $\xi_{n}$, para $n\geq2$, a $N\left(t\right)$ se le llama Proceso de Renovaci\'on retardado, donde si $\xi$ tiene distribuci\'on $G$, entonces el tiempo $T_{n}$ de la $n$-\'esima renovaci\'on tiene distribuci\'on $G\star F^{\left(n-1\right)\star}\left(t\right)$
\end{Note}


\begin{Teo}
Para una constante $\mu\leq\infty$ ( o variable aleatoria), las siguientes expresiones son equivalentes:

\begin{eqnarray}
lim_{n\rightarrow\infty}n^{-1}T_{n}&=&\mu,\textrm{ c.s.}\\
lim_{t\rightarrow\infty}t^{-1}N\left(t\right)&=&1/\mu,\textrm{ c.s.}
\end{eqnarray}
\end{Teo}


Es decir, $T_{n}$ satisface la Ley Fuerte de los Grandes N\'umeros s\'i y s\'olo s\'i $N\left/t\right)$ la cumple.


\begin{Coro}[Ley Fuerte de los Grandes N\'umeros para Procesos de Renovaci\'on]
Si $N\left(t\right)$ es un proceso de renovaci\'on cuyos tiempos de inter-renovaci\'on tienen media $\mu\leq\infty$, entonces
\begin{eqnarray}
t^{-1}N\left(t\right)\rightarrow 1/\mu,\textrm{ c.s. cuando }t\rightarrow\infty.
\end{eqnarray}

\end{Coro}


Considerar el proceso estoc\'astico de valores reales $\left\{Z\left(t\right):t\geq0\right\}$ en el mismo espacio de probabilidad que $N\left(t\right)$

\begin{Def}
Para el proceso $\left\{Z\left(t\right):t\geq0\right\}$ se define la fluctuaci\'on m\'axima de $Z\left(t\right)$ en el intervalo $\left(T_{n-1},T_{n}\right]$:
\begin{eqnarray*}
M_{n}=\sup_{T_{n-1}<t\leq T_{n}}|Z\left(t\right)-Z\left(T_{n-1}\right)|
\end{eqnarray*}
\end{Def}

\begin{Teo}
Sup\'ongase que $n^{-1}T_{n}\rightarrow\mu$ c.s. cuando $n\rightarrow\infty$, donde $\mu\leq\infty$ es una constante o variable aleatoria. Sea $a$ una constante o variable aleatoria que puede ser infinita cuando $\mu$ es finita, y considere las expresiones l\'imite:
\begin{eqnarray}
lim_{n\rightarrow\infty}n^{-1}Z\left(T_{n}\right)&=&a,\textrm{ c.s.}\\
lim_{t\rightarrow\infty}t^{-1}Z\left(t\right)&=&a/\mu,\textrm{ c.s.}
\end{eqnarray}
La segunda expresi\'on implica la primera. Conversamente, la primera implica la segunda si el proceso $Z\left(t\right)$ es creciente, o si $lim_{n\rightarrow\infty}n^{-1}M_{n}=0$ c.s.
\end{Teo}

\begin{Coro}
Si $N\left(t\right)$ es un proceso de renovaci\'on, y $\left(Z\left(T_{n}\right)-Z\left(T_{n-1}\right),M_{n}\right)$, para $n\geq1$, son variables aleatorias independientes e id\'enticamente distribuidas con media finita, entonces,
\begin{eqnarray}
lim_{t\rightarrow\infty}t^{-1}Z\left(t\right)\rightarrow\frac{\esp\left[Z\left(T_{1}\right)-Z\left(T_{0}\right)\right]}{\esp\left[T_{1}\right]},\textrm{ c.s. cuando  }t\rightarrow\infty.
\end{eqnarray}
\end{Coro}


%___________________________________________________________________________________________
%
%\subsection{Propiedades de los Procesos de Renovaci\'on}
%___________________________________________________________________________________________
%

Los tiempos $T_{n}$ est\'an relacionados con los conteos de $N\left(t\right)$ por

\begin{eqnarray*}
\left\{N\left(t\right)\geq n\right\}&=&\left\{T_{n}\leq t\right\}\\
T_{N\left(t\right)}\leq &t&<T_{N\left(t\right)+1},
\end{eqnarray*}

adem\'as $N\left(T_{n}\right)=n$, y 

\begin{eqnarray*}
N\left(t\right)=\max\left\{n:T_{n}\leq t\right\}=\min\left\{n:T_{n+1}>t\right\}
\end{eqnarray*}

Por propiedades de la convoluci\'on se sabe que

\begin{eqnarray*}
P\left\{T_{n}\leq t\right\}=F^{n\star}\left(t\right)
\end{eqnarray*}
que es la $n$-\'esima convoluci\'on de $F$. Entonces 

\begin{eqnarray*}
\left\{N\left(t\right)\geq n\right\}&=&\left\{T_{n}\leq t\right\}\\
P\left\{N\left(t\right)\leq n\right\}&=&1-F^{\left(n+1\right)\star}\left(t\right)
\end{eqnarray*}

Adem\'as usando el hecho de que $\esp\left[N\left(t\right)\right]=\sum_{n=1}^{\infty}P\left\{N\left(t\right)\geq n\right\}$
se tiene que

\begin{eqnarray*}
\esp\left[N\left(t\right)\right]=\sum_{n=1}^{\infty}F^{n\star}\left(t\right)
\end{eqnarray*}

\begin{Prop}
Para cada $t\geq0$, la funci\'on generadora de momentos $\esp\left[e^{\alpha N\left(t\right)}\right]$ existe para alguna $\alpha$ en una vecindad del 0, y de aqu\'i que $\esp\left[N\left(t\right)^{m}\right]<\infty$, para $m\geq1$.
\end{Prop}


\begin{Note}
Si el primer tiempo de renovaci\'on $\xi_{1}$ no tiene la misma distribuci\'on que el resto de las $\xi_{n}$, para $n\geq2$, a $N\left(t\right)$ se le llama Proceso de Renovaci\'on retardado, donde si $\xi$ tiene distribuci\'on $G$, entonces el tiempo $T_{n}$ de la $n$-\'esima renovaci\'on tiene distribuci\'on $G\star F^{\left(n-1\right)\star}\left(t\right)$
\end{Note}


\begin{Teo}
Para una constante $\mu\leq\infty$ ( o variable aleatoria), las siguientes expresiones son equivalentes:

\begin{eqnarray}
lim_{n\rightarrow\infty}n^{-1}T_{n}&=&\mu,\textrm{ c.s.}\\
lim_{t\rightarrow\infty}t^{-1}N\left(t\right)&=&1/\mu,\textrm{ c.s.}
\end{eqnarray}
\end{Teo}


Es decir, $T_{n}$ satisface la Ley Fuerte de los Grandes N\'umeros s\'i y s\'olo s\'i $N\left/t\right)$ la cumple.


\begin{Coro}[Ley Fuerte de los Grandes N\'umeros para Procesos de Renovaci\'on]
Si $N\left(t\right)$ es un proceso de renovaci\'on cuyos tiempos de inter-renovaci\'on tienen media $\mu\leq\infty$, entonces
\begin{eqnarray}
t^{-1}N\left(t\right)\rightarrow 1/\mu,\textrm{ c.s. cuando }t\rightarrow\infty.
\end{eqnarray}

\end{Coro}


Considerar el proceso estoc\'astico de valores reales $\left\{Z\left(t\right):t\geq0\right\}$ en el mismo espacio de probabilidad que $N\left(t\right)$

\begin{Def}
Para el proceso $\left\{Z\left(t\right):t\geq0\right\}$ se define la fluctuaci\'on m\'axima de $Z\left(t\right)$ en el intervalo $\left(T_{n-1},T_{n}\right]$:
\begin{eqnarray*}
M_{n}=\sup_{T_{n-1}<t\leq T_{n}}|Z\left(t\right)-Z\left(T_{n-1}\right)|
\end{eqnarray*}
\end{Def}

\begin{Teo}
Sup\'ongase que $n^{-1}T_{n}\rightarrow\mu$ c.s. cuando $n\rightarrow\infty$, donde $\mu\leq\infty$ es una constante o variable aleatoria. Sea $a$ una constante o variable aleatoria que puede ser infinita cuando $\mu$ es finita, y considere las expresiones l\'imite:
\begin{eqnarray}
lim_{n\rightarrow\infty}n^{-1}Z\left(T_{n}\right)&=&a,\textrm{ c.s.}\\
lim_{t\rightarrow\infty}t^{-1}Z\left(t\right)&=&a/\mu,\textrm{ c.s.}
\end{eqnarray}
La segunda expresi\'on implica la primera. Conversamente, la primera implica la segunda si el proceso $Z\left(t\right)$ es creciente, o si $lim_{n\rightarrow\infty}n^{-1}M_{n}=0$ c.s.
\end{Teo}

\begin{Coro}
Si $N\left(t\right)$ es un proceso de renovaci\'on, y $\left(Z\left(T_{n}\right)-Z\left(T_{n-1}\right),M_{n}\right)$, para $n\geq1$, son variables aleatorias independientes e id\'enticamente distribuidas con media finita, entonces,
\begin{eqnarray}
lim_{t\rightarrow\infty}t^{-1}Z\left(t\right)\rightarrow\frac{\esp\left[Z\left(T_{1}\right)-Z\left(T_{0}\right)\right]}{\esp\left[T_{1}\right]},\textrm{ c.s. cuando  }t\rightarrow\infty.
\end{eqnarray}
\end{Coro}

%___________________________________________________________________________________________
%
%\subsection{Propiedades de los Procesos de Renovaci\'on}
%___________________________________________________________________________________________
%

Los tiempos $T_{n}$ est\'an relacionados con los conteos de $N\left(t\right)$ por

\begin{eqnarray*}
\left\{N\left(t\right)\geq n\right\}&=&\left\{T_{n}\leq t\right\}\\
T_{N\left(t\right)}\leq &t&<T_{N\left(t\right)+1},
\end{eqnarray*}

adem\'as $N\left(T_{n}\right)=n$, y 

\begin{eqnarray*}
N\left(t\right)=\max\left\{n:T_{n}\leq t\right\}=\min\left\{n:T_{n+1}>t\right\}
\end{eqnarray*}

Por propiedades de la convoluci\'on se sabe que

\begin{eqnarray*}
P\left\{T_{n}\leq t\right\}=F^{n\star}\left(t\right)
\end{eqnarray*}
que es la $n$-\'esima convoluci\'on de $F$. Entonces 

\begin{eqnarray*}
\left\{N\left(t\right)\geq n\right\}&=&\left\{T_{n}\leq t\right\}\\
P\left\{N\left(t\right)\leq n\right\}&=&1-F^{\left(n+1\right)\star}\left(t\right)
\end{eqnarray*}

Adem\'as usando el hecho de que $\esp\left[N\left(t\right)\right]=\sum_{n=1}^{\infty}P\left\{N\left(t\right)\geq n\right\}$
se tiene que

\begin{eqnarray*}
\esp\left[N\left(t\right)\right]=\sum_{n=1}^{\infty}F^{n\star}\left(t\right)
\end{eqnarray*}

\begin{Prop}
Para cada $t\geq0$, la funci\'on generadora de momentos $\esp\left[e^{\alpha N\left(t\right)}\right]$ existe para alguna $\alpha$ en una vecindad del 0, y de aqu\'i que $\esp\left[N\left(t\right)^{m}\right]<\infty$, para $m\geq1$.
\end{Prop}


\begin{Note}
Si el primer tiempo de renovaci\'on $\xi_{1}$ no tiene la misma distribuci\'on que el resto de las $\xi_{n}$, para $n\geq2$, a $N\left(t\right)$ se le llama Proceso de Renovaci\'on retardado, donde si $\xi$ tiene distribuci\'on $G$, entonces el tiempo $T_{n}$ de la $n$-\'esima renovaci\'on tiene distribuci\'on $G\star F^{\left(n-1\right)\star}\left(t\right)$
\end{Note}


\begin{Teo}
Para una constante $\mu\leq\infty$ ( o variable aleatoria), las siguientes expresiones son equivalentes:

\begin{eqnarray}
lim_{n\rightarrow\infty}n^{-1}T_{n}&=&\mu,\textrm{ c.s.}\\
lim_{t\rightarrow\infty}t^{-1}N\left(t\right)&=&1/\mu,\textrm{ c.s.}
\end{eqnarray}
\end{Teo}


Es decir, $T_{n}$ satisface la Ley Fuerte de los Grandes N\'umeros s\'i y s\'olo s\'i $N\left/t\right)$ la cumple.


\begin{Coro}[Ley Fuerte de los Grandes N\'umeros para Procesos de Renovaci\'on]
Si $N\left(t\right)$ es un proceso de renovaci\'on cuyos tiempos de inter-renovaci\'on tienen media $\mu\leq\infty$, entonces
\begin{eqnarray}
t^{-1}N\left(t\right)\rightarrow 1/\mu,\textrm{ c.s. cuando }t\rightarrow\infty.
\end{eqnarray}

\end{Coro}


Considerar el proceso estoc\'astico de valores reales $\left\{Z\left(t\right):t\geq0\right\}$ en el mismo espacio de probabilidad que $N\left(t\right)$

\begin{Def}
Para el proceso $\left\{Z\left(t\right):t\geq0\right\}$ se define la fluctuaci\'on m\'axima de $Z\left(t\right)$ en el intervalo $\left(T_{n-1},T_{n}\right]$:
\begin{eqnarray*}
M_{n}=\sup_{T_{n-1}<t\leq T_{n}}|Z\left(t\right)-Z\left(T_{n-1}\right)|
\end{eqnarray*}
\end{Def}

\begin{Teo}
Sup\'ongase que $n^{-1}T_{n}\rightarrow\mu$ c.s. cuando $n\rightarrow\infty$, donde $\mu\leq\infty$ es una constante o variable aleatoria. Sea $a$ una constante o variable aleatoria que puede ser infinita cuando $\mu$ es finita, y considere las expresiones l\'imite:
\begin{eqnarray}
lim_{n\rightarrow\infty}n^{-1}Z\left(T_{n}\right)&=&a,\textrm{ c.s.}\\
lim_{t\rightarrow\infty}t^{-1}Z\left(t\right)&=&a/\mu,\textrm{ c.s.}
\end{eqnarray}
La segunda expresi\'on implica la primera. Conversamente, la primera implica la segunda si el proceso $Z\left(t\right)$ es creciente, o si $lim_{n\rightarrow\infty}n^{-1}M_{n}=0$ c.s.
\end{Teo}

\begin{Coro}
Si $N\left(t\right)$ es un proceso de renovaci\'on, y $\left(Z\left(T_{n}\right)-Z\left(T_{n-1}\right),M_{n}\right)$, para $n\geq1$, son variables aleatorias independientes e id\'enticamente distribuidas con media finita, entonces,
\begin{eqnarray}
lim_{t\rightarrow\infty}t^{-1}Z\left(t\right)\rightarrow\frac{\esp\left[Z\left(T_{1}\right)-Z\left(T_{0}\right)\right]}{\esp\left[T_{1}\right]},\textrm{ c.s. cuando  }t\rightarrow\infty.
\end{eqnarray}
\end{Coro}
%___________________________________________________________________________________________
%
\subsection{Propiedades de los Procesos de Renovaci\'on}
%___________________________________________________________________________________________
%

Los tiempos $T_{n}$ est\'an relacionados con los conteos de $N\left(t\right)$ por

\begin{eqnarray*}
\left\{N\left(t\right)\geq n\right\}&=&\left\{T_{n}\leq t\right\}\\
T_{N\left(t\right)}\leq &t&<T_{N\left(t\right)+1},
\end{eqnarray*}

adem\'as $N\left(T_{n}\right)=n$, y 

\begin{eqnarray*}
N\left(t\right)=\max\left\{n:T_{n}\leq t\right\}=\min\left\{n:T_{n+1}>t\right\}
\end{eqnarray*}

Por propiedades de la convoluci\'on se sabe que

\begin{eqnarray*}
P\left\{T_{n}\leq t\right\}=F^{n\star}\left(t\right)
\end{eqnarray*}
que es la $n$-\'esima convoluci\'on de $F$. Entonces 

\begin{eqnarray*}
\left\{N\left(t\right)\geq n\right\}&=&\left\{T_{n}\leq t\right\}\\
P\left\{N\left(t\right)\leq n\right\}&=&1-F^{\left(n+1\right)\star}\left(t\right)
\end{eqnarray*}

Adem\'as usando el hecho de que $\esp\left[N\left(t\right)\right]=\sum_{n=1}^{\infty}P\left\{N\left(t\right)\geq n\right\}$
se tiene que

\begin{eqnarray*}
\esp\left[N\left(t\right)\right]=\sum_{n=1}^{\infty}F^{n\star}\left(t\right)
\end{eqnarray*}

\begin{Prop}
Para cada $t\geq0$, la funci\'on generadora de momentos $\esp\left[e^{\alpha N\left(t\right)}\right]$ existe para alguna $\alpha$ en una vecindad del 0, y de aqu\'i que $\esp\left[N\left(t\right)^{m}\right]<\infty$, para $m\geq1$.
\end{Prop}


\begin{Note}
Si el primer tiempo de renovaci\'on $\xi_{1}$ no tiene la misma distribuci\'on que el resto de las $\xi_{n}$, para $n\geq2$, a $N\left(t\right)$ se le llama Proceso de Renovaci\'on retardado, donde si $\xi$ tiene distribuci\'on $G$, entonces el tiempo $T_{n}$ de la $n$-\'esima renovaci\'on tiene distribuci\'on $G\star F^{\left(n-1\right)\star}\left(t\right)$
\end{Note}


\begin{Teo}
Para una constante $\mu\leq\infty$ ( o variable aleatoria), las siguientes expresiones son equivalentes:

\begin{eqnarray}
lim_{n\rightarrow\infty}n^{-1}T_{n}&=&\mu,\textrm{ c.s.}\\
lim_{t\rightarrow\infty}t^{-1}N\left(t\right)&=&1/\mu,\textrm{ c.s.}
\end{eqnarray}
\end{Teo}


Es decir, $T_{n}$ satisface la Ley Fuerte de los Grandes N\'umeros s\'i y s\'olo s\'i $N\left/t\right)$ la cumple.


\begin{Coro}[Ley Fuerte de los Grandes N\'umeros para Procesos de Renovaci\'on]
Si $N\left(t\right)$ es un proceso de renovaci\'on cuyos tiempos de inter-renovaci\'on tienen media $\mu\leq\infty$, entonces
\begin{eqnarray}
t^{-1}N\left(t\right)\rightarrow 1/\mu,\textrm{ c.s. cuando }t\rightarrow\infty.
\end{eqnarray}

\end{Coro}


Considerar el proceso estoc\'astico de valores reales $\left\{Z\left(t\right):t\geq0\right\}$ en el mismo espacio de probabilidad que $N\left(t\right)$

\begin{Def}
Para el proceso $\left\{Z\left(t\right):t\geq0\right\}$ se define la fluctuaci\'on m\'axima de $Z\left(t\right)$ en el intervalo $\left(T_{n-1},T_{n}\right]$:
\begin{eqnarray*}
M_{n}=\sup_{T_{n-1}<t\leq T_{n}}|Z\left(t\right)-Z\left(T_{n-1}\right)|
\end{eqnarray*}
\end{Def}

\begin{Teo}
Sup\'ongase que $n^{-1}T_{n}\rightarrow\mu$ c.s. cuando $n\rightarrow\infty$, donde $\mu\leq\infty$ es una constante o variable aleatoria. Sea $a$ una constante o variable aleatoria que puede ser infinita cuando $\mu$ es finita, y considere las expresiones l\'imite:
\begin{eqnarray}
lim_{n\rightarrow\infty}n^{-1}Z\left(T_{n}\right)&=&a,\textrm{ c.s.}\\
lim_{t\rightarrow\infty}t^{-1}Z\left(t\right)&=&a/\mu,\textrm{ c.s.}
\end{eqnarray}
La segunda expresi\'on implica la primera. Conversamente, la primera implica la segunda si el proceso $Z\left(t\right)$ es creciente, o si $lim_{n\rightarrow\infty}n^{-1}M_{n}=0$ c.s.
\end{Teo}

\begin{Coro}
Si $N\left(t\right)$ es un proceso de renovaci\'on, y $\left(Z\left(T_{n}\right)-Z\left(T_{n-1}\right),M_{n}\right)$, para $n\geq1$, son variables aleatorias independientes e id\'enticamente distribuidas con media finita, entonces,
\begin{eqnarray}
lim_{t\rightarrow\infty}t^{-1}Z\left(t\right)\rightarrow\frac{\esp\left[Z\left(T_{1}\right)-Z\left(T_{0}\right)\right]}{\esp\left[T_{1}\right]},\textrm{ c.s. cuando  }t\rightarrow\infty.
\end{eqnarray}
\end{Coro}


%___________________________________________________________________________________________
%
%\subsection{Funci\'on de Renovaci\'on}
%___________________________________________________________________________________________
%


\begin{Def}
Sea $h\left(t\right)$ funci\'on de valores reales en $\rea$ acotada en intervalos finitos e igual a cero para $t<0$ La ecuaci\'on de renovaci\'on para $h\left(t\right)$ y la distribuci\'on $F$ es

\begin{eqnarray}%\label{Ec.Renovacion}
H\left(t\right)=h\left(t\right)+\int_{\left[0,t\right]}H\left(t-s\right)dF\left(s\right)\textrm{,    }t\geq0,
\end{eqnarray}
donde $H\left(t\right)$ es una funci\'on de valores reales. Esto es $H=h+F\star H$. Decimos que $H\left(t\right)$ es soluci\'on de esta ecuaci\'on si satisface la ecuaci\'on, y es acotada en intervalos finitos e iguales a cero para $t<0$.
\end{Def}

\begin{Prop}
La funci\'on $U\star h\left(t\right)$ es la \'unica soluci\'on de la ecuaci\'on de renovaci\'on (\ref{Ec.Renovacion}).
\end{Prop}

\begin{Teo}[Teorema Renovaci\'on Elemental]
\begin{eqnarray*}
t^{-1}U\left(t\right)\rightarrow 1/\mu\textrm{,    cuando }t\rightarrow\infty.
\end{eqnarray*}
\end{Teo}

%___________________________________________________________________________________________
%
%\subsection{Funci\'on de Renovaci\'on}
%___________________________________________________________________________________________
%


Sup\'ongase que $N\left(t\right)$ es un proceso de renovaci\'on con distribuci\'on $F$ con media finita $\mu$.

\begin{Def}
La funci\'on de renovaci\'on asociada con la distribuci\'on $F$, del proceso $N\left(t\right)$, es
\begin{eqnarray*}
U\left(t\right)=\sum_{n=1}^{\infty}F^{n\star}\left(t\right),\textrm{   }t\geq0,
\end{eqnarray*}
donde $F^{0\star}\left(t\right)=\indora\left(t\geq0\right)$.
\end{Def}


\begin{Prop}
Sup\'ongase que la distribuci\'on de inter-renovaci\'on $F$ tiene densidad $f$. Entonces $U\left(t\right)$ tambi\'en tiene densidad, para $t>0$, y es $U^{'}\left(t\right)=\sum_{n=0}^{\infty}f^{n\star}\left(t\right)$. Adem\'as
\begin{eqnarray*}
\prob\left\{N\left(t\right)>N\left(t-\right)\right\}=0\textrm{,   }t\geq0.
\end{eqnarray*}
\end{Prop}

\begin{Def}
La Transformada de Laplace-Stieljes de $F$ est\'a dada por

\begin{eqnarray*}
\hat{F}\left(\alpha\right)=\int_{\rea_{+}}e^{-\alpha t}dF\left(t\right)\textrm{,  }\alpha\geq0.
\end{eqnarray*}
\end{Def}

Entonces

\begin{eqnarray*}
\hat{U}\left(\alpha\right)=\sum_{n=0}^{\infty}\hat{F^{n\star}}\left(\alpha\right)=\sum_{n=0}^{\infty}\hat{F}\left(\alpha\right)^{n}=\frac{1}{1-\hat{F}\left(\alpha\right)}.
\end{eqnarray*}


\begin{Prop}
La Transformada de Laplace $\hat{U}\left(\alpha\right)$ y $\hat{F}\left(\alpha\right)$ determina una a la otra de manera \'unica por la relaci\'on $\hat{U}\left(\alpha\right)=\frac{1}{1-\hat{F}\left(\alpha\right)}$.
\end{Prop}


\begin{Note}
Un proceso de renovaci\'on $N\left(t\right)$ cuyos tiempos de inter-renovaci\'on tienen media finita, es un proceso Poisson con tasa $\lambda$ si y s\'olo s\'i $\esp\left[U\left(t\right)\right]=\lambda t$, para $t\geq0$.
\end{Note}


\begin{Teo}
Sea $N\left(t\right)$ un proceso puntual simple con puntos de localizaci\'on $T_{n}$ tal que $\eta\left(t\right)=\esp\left[N\left(\right)\right]$ es finita para cada $t$. Entonces para cualquier funci\'on $f:\rea_{+}\rightarrow\rea$,
\begin{eqnarray*}
\esp\left[\sum_{n=1}^{N\left(\right)}f\left(T_{n}\right)\right]=\int_{\left(0,t\right]}f\left(s\right)d\eta\left(s\right)\textrm{,  }t\geq0,
\end{eqnarray*}
suponiendo que la integral exista. Adem\'as si $X_{1},X_{2},\ldots$ son variables aleatorias definidas en el mismo espacio de probabilidad que el proceso $N\left(t\right)$ tal que $\esp\left[X_{n}|T_{n}=s\right]=f\left(s\right)$, independiente de $n$. Entonces
\begin{eqnarray*}
\esp\left[\sum_{n=1}^{N\left(t\right)}X_{n}\right]=\int_{\left(0,t\right]}f\left(s\right)d\eta\left(s\right)\textrm{,  }t\geq0,
\end{eqnarray*} 
suponiendo que la integral exista. 
\end{Teo}

\begin{Coro}[Identidad de Wald para Renovaciones]
Para el proceso de renovaci\'on $N\left(t\right)$,
\begin{eqnarray*}
\esp\left[T_{N\left(t\right)+1}\right]=\mu\esp\left[N\left(t\right)+1\right]\textrm{,  }t\geq0,
\end{eqnarray*}  
\end{Coro}

%______________________________________________________________________
%\subsection{Procesos de Renovaci\'on}
%______________________________________________________________________

\begin{Def}%\label{Def.Tn}
Sean $0\leq T_{1}\leq T_{2}\leq \ldots$ son tiempos aleatorios infinitos en los cuales ocurren ciertos eventos. El n\'umero de tiempos $T_{n}$ en el intervalo $\left[0,t\right)$ es

\begin{eqnarray}
N\left(t\right)=\sum_{n=1}^{\infty}\indora\left(T_{n}\leq t\right),
\end{eqnarray}
para $t\geq0$.
\end{Def}

Si se consideran los puntos $T_{n}$ como elementos de $\rea_{+}$, y $N\left(t\right)$ es el n\'umero de puntos en $\rea$. El proceso denotado por $\left\{N\left(t\right):t\geq0\right\}$, denotado por $N\left(t\right)$, es un proceso puntual en $\rea_{+}$. Los $T_{n}$ son los tiempos de ocurrencia, el proceso puntual $N\left(t\right)$ es simple si su n\'umero de ocurrencias son distintas: $0<T_{1}<T_{2}<\ldots$ casi seguramente.

\begin{Def}
Un proceso puntual $N\left(t\right)$ es un proceso de renovaci\'on si los tiempos de interocurrencia $\xi_{n}=T_{n}-T_{n-1}$, para $n\geq1$, son independientes e identicamente distribuidos con distribuci\'on $F$, donde $F\left(0\right)=0$ y $T_{0}=0$. Los $T_{n}$ son llamados tiempos de renovaci\'on, referente a la independencia o renovaci\'on de la informaci\'on estoc\'astica en estos tiempos. Los $\xi_{n}$ son los tiempos de inter-renovaci\'on, y $N\left(t\right)$ es el n\'umero de renovaciones en el intervalo $\left[0,t\right)$
\end{Def}


\begin{Note}
Para definir un proceso de renovaci\'on para cualquier contexto, solamente hay que especificar una distribuci\'on $F$, con $F\left(0\right)=0$, para los tiempos de inter-renovaci\'on. La funci\'on $F$ en turno degune las otra variables aleatorias. De manera formal, existe un espacio de probabilidad y una sucesi\'on de variables aleatorias $\xi_{1},\xi_{2},\ldots$ definidas en este con distribuci\'on $F$. Entonces las otras cantidades son $T_{n}=\sum_{k=1}^{n}\xi_{k}$ y $N\left(t\right)=\sum_{n=1}^{\infty}\indora\left(T_{n}\leq t\right)$, donde $T_{n}\rightarrow\infty$ casi seguramente por la Ley Fuerte de los Grandes Números.
\end{Note}

%___________________________________________________________________________________________
%
%\subsection{Renewal and Regenerative Processes: Serfozo\cite{Serfozo}}
%___________________________________________________________________________________________
%
\begin{Def}%\label{Def.Tn}
Sean $0\leq T_{1}\leq T_{2}\leq \ldots$ son tiempos aleatorios infinitos en los cuales ocurren ciertos eventos. El n\'umero de tiempos $T_{n}$ en el intervalo $\left[0,t\right)$ es

\begin{eqnarray}
N\left(t\right)=\sum_{n=1}^{\infty}\indora\left(T_{n}\leq t\right),
\end{eqnarray}
para $t\geq0$.
\end{Def}

Si se consideran los puntos $T_{n}$ como elementos de $\rea_{+}$, y $N\left(t\right)$ es el n\'umero de puntos en $\rea$. El proceso denotado por $\left\{N\left(t\right):t\geq0\right\}$, denotado por $N\left(t\right)$, es un proceso puntual en $\rea_{+}$. Los $T_{n}$ son los tiempos de ocurrencia, el proceso puntual $N\left(t\right)$ es simple si su n\'umero de ocurrencias son distintas: $0<T_{1}<T_{2}<\ldots$ casi seguramente.

\begin{Def}
Un proceso puntual $N\left(t\right)$ es un proceso de renovaci\'on si los tiempos de interocurrencia $\xi_{n}=T_{n}-T_{n-1}$, para $n\geq1$, son independientes e identicamente distribuidos con distribuci\'on $F$, donde $F\left(0\right)=0$ y $T_{0}=0$. Los $T_{n}$ son llamados tiempos de renovaci\'on, referente a la independencia o renovaci\'on de la informaci\'on estoc\'astica en estos tiempos. Los $\xi_{n}$ son los tiempos de inter-renovaci\'on, y $N\left(t\right)$ es el n\'umero de renovaciones en el intervalo $\left[0,t\right)$
\end{Def}


\begin{Note}
Para definir un proceso de renovaci\'on para cualquier contexto, solamente hay que especificar una distribuci\'on $F$, con $F\left(0\right)=0$, para los tiempos de inter-renovaci\'on. La funci\'on $F$ en turno degune las otra variables aleatorias. De manera formal, existe un espacio de probabilidad y una sucesi\'on de variables aleatorias $\xi_{1},\xi_{2},\ldots$ definidas en este con distribuci\'on $F$. Entonces las otras cantidades son $T_{n}=\sum_{k=1}^{n}\xi_{k}$ y $N\left(t\right)=\sum_{n=1}^{\infty}\indora\left(T_{n}\leq t\right)$, donde $T_{n}\rightarrow\infty$ casi seguramente por la Ley Fuerte de los Grandes N\'umeros.
\end{Note}







Los tiempos $T_{n}$ est\'an relacionados con los conteos de $N\left(t\right)$ por

\begin{eqnarray*}
\left\{N\left(t\right)\geq n\right\}&=&\left\{T_{n}\leq t\right\}\\
T_{N\left(t\right)}\leq &t&<T_{N\left(t\right)+1},
\end{eqnarray*}

adem\'as $N\left(T_{n}\right)=n$, y 

\begin{eqnarray*}
N\left(t\right)=\max\left\{n:T_{n}\leq t\right\}=\min\left\{n:T_{n+1}>t\right\}
\end{eqnarray*}

Por propiedades de la convoluci\'on se sabe que

\begin{eqnarray*}
P\left\{T_{n}\leq t\right\}=F^{n\star}\left(t\right)
\end{eqnarray*}
que es la $n$-\'esima convoluci\'on de $F$. Entonces 

\begin{eqnarray*}
\left\{N\left(t\right)\geq n\right\}&=&\left\{T_{n}\leq t\right\}\\
P\left\{N\left(t\right)\leq n\right\}&=&1-F^{\left(n+1\right)\star}\left(t\right)
\end{eqnarray*}

Adem\'as usando el hecho de que $\esp\left[N\left(t\right)\right]=\sum_{n=1}^{\infty}P\left\{N\left(t\right)\geq n\right\}$
se tiene que

\begin{eqnarray*}
\esp\left[N\left(t\right)\right]=\sum_{n=1}^{\infty}F^{n\star}\left(t\right)
\end{eqnarray*}

\begin{Prop}
Para cada $t\geq0$, la funci\'on generadora de momentos $\esp\left[e^{\alpha N\left(t\right)}\right]$ existe para alguna $\alpha$ en una vecindad del 0, y de aqu\'i que $\esp\left[N\left(t\right)^{m}\right]<\infty$, para $m\geq1$.
\end{Prop}

\begin{Ejem}[\textbf{Proceso Poisson}]

Suponga que se tienen tiempos de inter-renovaci\'on \textit{i.i.d.} del proceso de renovaci\'on $N\left(t\right)$ tienen distribuci\'on exponencial $F\left(t\right)=q-e^{-\lambda t}$ con tasa $\lambda$. Entonces $N\left(t\right)$ es un proceso Poisson con tasa $\lambda$.

\end{Ejem}


\begin{Note}
Si el primer tiempo de renovaci\'on $\xi_{1}$ no tiene la misma distribuci\'on que el resto de las $\xi_{n}$, para $n\geq2$, a $N\left(t\right)$ se le llama Proceso de Renovaci\'on retardado, donde si $\xi$ tiene distribuci\'on $G$, entonces el tiempo $T_{n}$ de la $n$-\'esima renovaci\'on tiene distribuci\'on $G\star F^{\left(n-1\right)\star}\left(t\right)$
\end{Note}


\begin{Teo}
Para una constante $\mu\leq\infty$ ( o variable aleatoria), las siguientes expresiones son equivalentes:

\begin{eqnarray}
lim_{n\rightarrow\infty}n^{-1}T_{n}&=&\mu,\textrm{ c.s.}\\
lim_{t\rightarrow\infty}t^{-1}N\left(t\right)&=&1/\mu,\textrm{ c.s.}
\end{eqnarray}
\end{Teo}


Es decir, $T_{n}$ satisface la Ley Fuerte de los Grandes N\'umeros s\'i y s\'olo s\'i $N\left/t\right)$ la cumple.


\begin{Coro}[Ley Fuerte de los Grandes N\'umeros para Procesos de Renovaci\'on]
Si $N\left(t\right)$ es un proceso de renovaci\'on cuyos tiempos de inter-renovaci\'on tienen media $\mu\leq\infty$, entonces
\begin{eqnarray}
t^{-1}N\left(t\right)\rightarrow 1/\mu,\textrm{ c.s. cuando }t\rightarrow\infty.
\end{eqnarray}

\end{Coro}


Considerar el proceso estoc\'astico de valores reales $\left\{Z\left(t\right):t\geq0\right\}$ en el mismo espacio de probabilidad que $N\left(t\right)$

\begin{Def}
Para el proceso $\left\{Z\left(t\right):t\geq0\right\}$ se define la fluctuaci\'on m\'axima de $Z\left(t\right)$ en el intervalo $\left(T_{n-1},T_{n}\right]$:
\begin{eqnarray*}
M_{n}=\sup_{T_{n-1}<t\leq T_{n}}|Z\left(t\right)-Z\left(T_{n-1}\right)|
\end{eqnarray*}
\end{Def}

\begin{Teo}
Sup\'ongase que $n^{-1}T_{n}\rightarrow\mu$ c.s. cuando $n\rightarrow\infty$, donde $\mu\leq\infty$ es una constante o variable aleatoria. Sea $a$ una constante o variable aleatoria que puede ser infinita cuando $\mu$ es finita, y considere las expresiones l\'imite:
\begin{eqnarray}
lim_{n\rightarrow\infty}n^{-1}Z\left(T_{n}\right)&=&a,\textrm{ c.s.}\\
lim_{t\rightarrow\infty}t^{-1}Z\left(t\right)&=&a/\mu,\textrm{ c.s.}
\end{eqnarray}
La segunda expresi\'on implica la primera. Conversamente, la primera implica la segunda si el proceso $Z\left(t\right)$ es creciente, o si $lim_{n\rightarrow\infty}n^{-1}M_{n}=0$ c.s.
\end{Teo}

\begin{Coro}
Si $N\left(t\right)$ es un proceso de renovaci\'on, y $\left(Z\left(T_{n}\right)-Z\left(T_{n-1}\right),M_{n}\right)$, para $n\geq1$, son variables aleatorias independientes e id\'enticamente distribuidas con media finita, entonces,
\begin{eqnarray}
lim_{t\rightarrow\infty}t^{-1}Z\left(t\right)\rightarrow\frac{\esp\left[Z\left(T_{1}\right)-Z\left(T_{0}\right)\right]}{\esp\left[T_{1}\right]},\textrm{ c.s. cuando  }t\rightarrow\infty.
\end{eqnarray}
\end{Coro}


Sup\'ongase que $N\left(t\right)$ es un proceso de renovaci\'on con distribuci\'on $F$ con media finita $\mu$.

\begin{Def}
La funci\'on de renovaci\'on asociada con la distribuci\'on $F$, del proceso $N\left(t\right)$, es
\begin{eqnarray*}
U\left(t\right)=\sum_{n=1}^{\infty}F^{n\star}\left(t\right),\textrm{   }t\geq0,
\end{eqnarray*}
donde $F^{0\star}\left(t\right)=\indora\left(t\geq0\right)$.
\end{Def}


\begin{Prop}
Sup\'ongase que la distribuci\'on de inter-renovaci\'on $F$ tiene densidad $f$. Entonces $U\left(t\right)$ tambi\'en tiene densidad, para $t>0$, y es $U^{'}\left(t\right)=\sum_{n=0}^{\infty}f^{n\star}\left(t\right)$. Adem\'as
\begin{eqnarray*}
\prob\left\{N\left(t\right)>N\left(t-\right)\right\}=0\textrm{,   }t\geq0.
\end{eqnarray*}
\end{Prop}

\begin{Def}
La Transformada de Laplace-Stieljes de $F$ est\'a dada por

\begin{eqnarray*}
\hat{F}\left(\alpha\right)=\int_{\rea_{+}}e^{-\alpha t}dF\left(t\right)\textrm{,  }\alpha\geq0.
\end{eqnarray*}
\end{Def}

Entonces

\begin{eqnarray*}
\hat{U}\left(\alpha\right)=\sum_{n=0}^{\infty}\hat{F^{n\star}}\left(\alpha\right)=\sum_{n=0}^{\infty}\hat{F}\left(\alpha\right)^{n}=\frac{1}{1-\hat{F}\left(\alpha\right)}.
\end{eqnarray*}


\begin{Prop}
La Transformada de Laplace $\hat{U}\left(\alpha\right)$ y $\hat{F}\left(\alpha\right)$ determina una a la otra de manera \'unica por la relaci\'on $\hat{U}\left(\alpha\right)=\frac{1}{1-\hat{F}\left(\alpha\right)}$.
\end{Prop}


\begin{Note}
Un proceso de renovaci\'on $N\left(t\right)$ cuyos tiempos de inter-renovaci\'on tienen media finita, es un proceso Poisson con tasa $\lambda$ si y s\'olo s\'i $\esp\left[U\left(t\right)\right]=\lambda t$, para $t\geq0$.
\end{Note}


\begin{Teo}
Sea $N\left(t\right)$ un proceso puntual simple con puntos de localizaci\'on $T_{n}$ tal que $\eta\left(t\right)=\esp\left[N\left(\right)\right]$ es finita para cada $t$. Entonces para cualquier funci\'on $f:\rea_{+}\rightarrow\rea$,
\begin{eqnarray*}
\esp\left[\sum_{n=1}^{N\left(\right)}f\left(T_{n}\right)\right]=\int_{\left(0,t\right]}f\left(s\right)d\eta\left(s\right)\textrm{,  }t\geq0,
\end{eqnarray*}
suponiendo que la integral exista. Adem\'as si $X_{1},X_{2},\ldots$ son variables aleatorias definidas en el mismo espacio de probabilidad que el proceso $N\left(t\right)$ tal que $\esp\left[X_{n}|T_{n}=s\right]=f\left(s\right)$, independiente de $n$. Entonces
\begin{eqnarray*}
\esp\left[\sum_{n=1}^{N\left(t\right)}X_{n}\right]=\int_{\left(0,t\right]}f\left(s\right)d\eta\left(s\right)\textrm{,  }t\geq0,
\end{eqnarray*} 
suponiendo que la integral exista. 
\end{Teo}

\begin{Coro}[Identidad de Wald para Renovaciones]
Para el proceso de renovaci\'on $N\left(t\right)$,
\begin{eqnarray*}
\esp\left[T_{N\left(t\right)+1}\right]=\mu\esp\left[N\left(t\right)+1\right]\textrm{,  }t\geq0,
\end{eqnarray*}  
\end{Coro}


\begin{Def}
Sea $h\left(t\right)$ funci\'on de valores reales en $\rea$ acotada en intervalos finitos e igual a cero para $t<0$ La ecuaci\'on de renovaci\'on para $h\left(t\right)$ y la distribuci\'on $F$ es

\begin{eqnarray}%\label{Ec.Renovacion}
H\left(t\right)=h\left(t\right)+\int_{\left[0,t\right]}H\left(t-s\right)dF\left(s\right)\textrm{,    }t\geq0,
\end{eqnarray}
donde $H\left(t\right)$ es una funci\'on de valores reales. Esto es $H=h+F\star H$. Decimos que $H\left(t\right)$ es soluci\'on de esta ecuaci\'on si satisface la ecuaci\'on, y es acotada en intervalos finitos e iguales a cero para $t<0$.
\end{Def}

\begin{Prop}
La funci\'on $U\star h\left(t\right)$ es la \'unica soluci\'on de la ecuaci\'on de renovaci\'on (\ref{Ec.Renovacion}).
\end{Prop}

\begin{Teo}[Teorema Renovaci\'on Elemental]
\begin{eqnarray*}
t^{-1}U\left(t\right)\rightarrow 1/\mu\textrm{,    cuando }t\rightarrow\infty.
\end{eqnarray*}
\end{Teo}



Sup\'ongase que $N\left(t\right)$ es un proceso de renovaci\'on con distribuci\'on $F$ con media finita $\mu$.

\begin{Def}
La funci\'on de renovaci\'on asociada con la distribuci\'on $F$, del proceso $N\left(t\right)$, es
\begin{eqnarray*}
U\left(t\right)=\sum_{n=1}^{\infty}F^{n\star}\left(t\right),\textrm{   }t\geq0,
\end{eqnarray*}
donde $F^{0\star}\left(t\right)=\indora\left(t\geq0\right)$.
\end{Def}


\begin{Prop}
Sup\'ongase que la distribuci\'on de inter-renovaci\'on $F$ tiene densidad $f$. Entonces $U\left(t\right)$ tambi\'en tiene densidad, para $t>0$, y es $U^{'}\left(t\right)=\sum_{n=0}^{\infty}f^{n\star}\left(t\right)$. Adem\'as
\begin{eqnarray*}
\prob\left\{N\left(t\right)>N\left(t-\right)\right\}=0\textrm{,   }t\geq0.
\end{eqnarray*}
\end{Prop}

\begin{Def}
La Transformada de Laplace-Stieljes de $F$ est\'a dada por

\begin{eqnarray*}
\hat{F}\left(\alpha\right)=\int_{\rea_{+}}e^{-\alpha t}dF\left(t\right)\textrm{,  }\alpha\geq0.
\end{eqnarray*}
\end{Def}

Entonces

\begin{eqnarray*}
\hat{U}\left(\alpha\right)=\sum_{n=0}^{\infty}\hat{F^{n\star}}\left(\alpha\right)=\sum_{n=0}^{\infty}\hat{F}\left(\alpha\right)^{n}=\frac{1}{1-\hat{F}\left(\alpha\right)}.
\end{eqnarray*}


\begin{Prop}
La Transformada de Laplace $\hat{U}\left(\alpha\right)$ y $\hat{F}\left(\alpha\right)$ determina una a la otra de manera \'unica por la relaci\'on $\hat{U}\left(\alpha\right)=\frac{1}{1-\hat{F}\left(\alpha\right)}$.
\end{Prop}


\begin{Note}
Un proceso de renovaci\'on $N\left(t\right)$ cuyos tiempos de inter-renovaci\'on tienen media finita, es un proceso Poisson con tasa $\lambda$ si y s\'olo s\'i $\esp\left[U\left(t\right)\right]=\lambda t$, para $t\geq0$.
\end{Note}


\begin{Teo}
Sea $N\left(t\right)$ un proceso puntual simple con puntos de localizaci\'on $T_{n}$ tal que $\eta\left(t\right)=\esp\left[N\left(\right)\right]$ es finita para cada $t$. Entonces para cualquier funci\'on $f:\rea_{+}\rightarrow\rea$,
\begin{eqnarray*}
\esp\left[\sum_{n=1}^{N\left(\right)}f\left(T_{n}\right)\right]=\int_{\left(0,t\right]}f\left(s\right)d\eta\left(s\right)\textrm{,  }t\geq0,
\end{eqnarray*}
suponiendo que la integral exista. Adem\'as si $X_{1},X_{2},\ldots$ son variables aleatorias definidas en el mismo espacio de probabilidad que el proceso $N\left(t\right)$ tal que $\esp\left[X_{n}|T_{n}=s\right]=f\left(s\right)$, independiente de $n$. Entonces
\begin{eqnarray*}
\esp\left[\sum_{n=1}^{N\left(t\right)}X_{n}\right]=\int_{\left(0,t\right]}f\left(s\right)d\eta\left(s\right)\textrm{,  }t\geq0,
\end{eqnarray*} 
suponiendo que la integral exista. 
\end{Teo}

\begin{Coro}[Identidad de Wald para Renovaciones]
Para el proceso de renovaci\'on $N\left(t\right)$,
\begin{eqnarray*}
\esp\left[T_{N\left(t\right)+1}\right]=\mu\esp\left[N\left(t\right)+1\right]\textrm{,  }t\geq0,
\end{eqnarray*}  
\end{Coro}


\begin{Def}
Sea $h\left(t\right)$ funci\'on de valores reales en $\rea$ acotada en intervalos finitos e igual a cero para $t<0$ La ecuaci\'on de renovaci\'on para $h\left(t\right)$ y la distribuci\'on $F$ es

\begin{eqnarray}%\label{Ec.Renovacion}
H\left(t\right)=h\left(t\right)+\int_{\left[0,t\right]}H\left(t-s\right)dF\left(s\right)\textrm{,    }t\geq0,
\end{eqnarray}
donde $H\left(t\right)$ es una funci\'on de valores reales. Esto es $H=h+F\star H$. Decimos que $H\left(t\right)$ es soluci\'on de esta ecuaci\'on si satisface la ecuaci\'on, y es acotada en intervalos finitos e iguales a cero para $t<0$.
\end{Def}

\begin{Prop}
La funci\'on $U\star h\left(t\right)$ es la \'unica soluci\'on de la ecuaci\'on de renovaci\'on (\ref{Ec.Renovacion}).
\end{Prop}

\begin{Teo}[Teorema Renovaci\'on Elemental]
\begin{eqnarray*}
t^{-1}U\left(t\right)\rightarrow 1/\mu\textrm{,    cuando }t\rightarrow\infty.
\end{eqnarray*}
\end{Teo}


\begin{Note} Una funci\'on $h:\rea_{+}\rightarrow\rea$ es Directamente Riemann Integrable en los siguientes casos:
\begin{itemize}
\item[a)] $h\left(t\right)\geq0$ es decreciente y Riemann Integrable.
\item[b)] $h$ es continua excepto posiblemente en un conjunto de Lebesgue de medida 0, y $|h\left(t\right)|\leq b\left(t\right)$, donde $b$ es DRI.
\end{itemize}
\end{Note}

\begin{Teo}[Teorema Principal de Renovaci\'on]
Si $F$ es no aritm\'etica y $h\left(t\right)$ es Directamente Riemann Integrable (DRI), entonces

\begin{eqnarray*}
lim_{t\rightarrow\infty}U\star h=\frac{1}{\mu}\int_{\rea_{+}}h\left(s\right)ds.
\end{eqnarray*}
\end{Teo}

\begin{Prop}
Cualquier funci\'on $H\left(t\right)$ acotada en intervalos finitos y que es 0 para $t<0$ puede expresarse como
\begin{eqnarray*}
H\left(t\right)=U\star h\left(t\right)\textrm{,  donde }h\left(t\right)=H\left(t\right)-F\star H\left(t\right)
\end{eqnarray*}
\end{Prop}

\begin{Def}
Un proceso estoc\'astico $X\left(t\right)$ es crudamente regenerativo en un tiempo aleatorio positivo $T$ si
\begin{eqnarray*}
\esp\left[X\left(T+t\right)|T\right]=\esp\left[X\left(t\right)\right]\textrm{, para }t\geq0,\end{eqnarray*}
y con las esperanzas anteriores finitas.
\end{Def}

\begin{Prop}
Sup\'ongase que $X\left(t\right)$ es un proceso crudamente regenerativo en $T$, que tiene distribuci\'on $F$. Si $\esp\left[X\left(t\right)\right]$ es acotado en intervalos finitos, entonces
\begin{eqnarray*}
\esp\left[X\left(t\right)\right]=U\star h\left(t\right)\textrm{,  donde }h\left(t\right)=\esp\left[X\left(t\right)\indora\left(T>t\right)\right].
\end{eqnarray*}
\end{Prop}

\begin{Teo}[Regeneraci\'on Cruda]
Sup\'ongase que $X\left(t\right)$ es un proceso con valores positivo crudamente regenerativo en $T$, y def\'inase $M=\sup\left\{|X\left(t\right)|:t\leq T\right\}$. Si $T$ es no aritm\'etico y $M$ y $MT$ tienen media finita, entonces
\begin{eqnarray*}
lim_{t\rightarrow\infty}\esp\left[X\left(t\right)\right]=\frac{1}{\mu}\int_{\rea_{+}}h\left(s\right)ds,
\end{eqnarray*}
donde $h\left(t\right)=\esp\left[X\left(t\right)\indora\left(T>t\right)\right]$.
\end{Teo}


\begin{Note} Una funci\'on $h:\rea_{+}\rightarrow\rea$ es Directamente Riemann Integrable en los siguientes casos:
\begin{itemize}
\item[a)] $h\left(t\right)\geq0$ es decreciente y Riemann Integrable.
\item[b)] $h$ es continua excepto posiblemente en un conjunto de Lebesgue de medida 0, y $|h\left(t\right)|\leq b\left(t\right)$, donde $b$ es DRI.
\end{itemize}
\end{Note}

\begin{Teo}[Teorema Principal de Renovaci\'on]
Si $F$ es no aritm\'etica y $h\left(t\right)$ es Directamente Riemann Integrable (DRI), entonces

\begin{eqnarray*}
lim_{t\rightarrow\infty}U\star h=\frac{1}{\mu}\int_{\rea_{+}}h\left(s\right)ds.
\end{eqnarray*}
\end{Teo}

\begin{Prop}
Cualquier funci\'on $H\left(t\right)$ acotada en intervalos finitos y que es 0 para $t<0$ puede expresarse como
\begin{eqnarray*}
H\left(t\right)=U\star h\left(t\right)\textrm{,  donde }h\left(t\right)=H\left(t\right)-F\star H\left(t\right)
\end{eqnarray*}
\end{Prop}

\begin{Def}
Un proceso estoc\'astico $X\left(t\right)$ es crudamente regenerativo en un tiempo aleatorio positivo $T$ si
\begin{eqnarray*}
\esp\left[X\left(T+t\right)|T\right]=\esp\left[X\left(t\right)\right]\textrm{, para }t\geq0,\end{eqnarray*}
y con las esperanzas anteriores finitas.
\end{Def}

\begin{Prop}
Sup\'ongase que $X\left(t\right)$ es un proceso crudamente regenerativo en $T$, que tiene distribuci\'on $F$. Si $\esp\left[X\left(t\right)\right]$ es acotado en intervalos finitos, entonces
\begin{eqnarray*}
\esp\left[X\left(t\right)\right]=U\star h\left(t\right)\textrm{,  donde }h\left(t\right)=\esp\left[X\left(t\right)\indora\left(T>t\right)\right].
\end{eqnarray*}
\end{Prop}

\begin{Teo}[Regeneraci\'on Cruda]
Sup\'ongase que $X\left(t\right)$ es un proceso con valores positivo crudamente regenerativo en $T$, y def\'inase $M=\sup\left\{|X\left(t\right)|:t\leq T\right\}$. Si $T$ es no aritm\'etico y $M$ y $MT$ tienen media finita, entonces
\begin{eqnarray*}
lim_{t\rightarrow\infty}\esp\left[X\left(t\right)\right]=\frac{1}{\mu}\int_{\rea_{+}}h\left(s\right)ds,
\end{eqnarray*}
donde $h\left(t\right)=\esp\left[X\left(t\right)\indora\left(T>t\right)\right]$.
\end{Teo}

\begin{Def}
Para el proceso $\left\{\left(N\left(t\right),X\left(t\right)\right):t\geq0\right\}$, sus trayectoria muestrales en el intervalo de tiempo $\left[T_{n-1},T_{n}\right)$ est\'an descritas por
\begin{eqnarray*}
\zeta_{n}=\left(\xi_{n},\left\{X\left(T_{n-1}+t\right):0\leq t<\xi_{n}\right\}\right)
\end{eqnarray*}
Este $\zeta_{n}$ es el $n$-\'esimo segmento del proceso. El proceso es regenerativo sobre los tiempos $T_{n}$ si sus segmentos $\zeta_{n}$ son independientes e id\'enticamennte distribuidos.
\end{Def}


\begin{Note}
Si $\tilde{X}\left(t\right)$ con espacio de estados $\tilde{S}$ es regenerativo sobre $T_{n}$, entonces $X\left(t\right)=f\left(\tilde{X}\left(t\right)\right)$ tambi\'en es regenerativo sobre $T_{n}$, para cualquier funci\'on $f:\tilde{S}\rightarrow S$.
\end{Note}

\begin{Note}
Los procesos regenerativos son crudamente regenerativos, pero no al rev\'es.
\end{Note}


\begin{Note}
Un proceso estoc\'astico a tiempo continuo o discreto es regenerativo si existe un proceso de renovaci\'on  tal que los segmentos del proceso entre tiempos de renovaci\'on sucesivos son i.i.d., es decir, para $\left\{X\left(t\right):t\geq0\right\}$ proceso estoc\'astico a tiempo continuo con espacio de estados $S$, espacio m\'etrico.
\end{Note}

Para $\left\{X\left(t\right):t\geq0\right\}$ Proceso Estoc\'astico a tiempo continuo con estado de espacios $S$, que es un espacio m\'etrico, con trayectorias continuas por la derecha y con l\'imites por la izquierda c.s. Sea $N\left(t\right)$ un proceso de renovaci\'on en $\rea_{+}$ definido en el mismo espacio de probabilidad que $X\left(t\right)$, con tiempos de renovaci\'on $T$ y tiempos de inter-renovaci\'on $\xi_{n}=T_{n}-T_{n-1}$, con misma distribuci\'on $F$ de media finita $\mu$.



\begin{Def}
Para el proceso $\left\{\left(N\left(t\right),X\left(t\right)\right):t\geq0\right\}$, sus trayectoria muestrales en el intervalo de tiempo $\left[T_{n-1},T_{n}\right)$ est\'an descritas por
\begin{eqnarray*}
\zeta_{n}=\left(\xi_{n},\left\{X\left(T_{n-1}+t\right):0\leq t<\xi_{n}\right\}\right)
\end{eqnarray*}
Este $\zeta_{n}$ es el $n$-\'esimo segmento del proceso. El proceso es regenerativo sobre los tiempos $T_{n}$ si sus segmentos $\zeta_{n}$ son independientes e id\'enticamennte distribuidos.
\end{Def}

\begin{Note}
Un proceso regenerativo con media de la longitud de ciclo finita es llamado positivo recurrente.
\end{Note}

\begin{Teo}[Procesos Regenerativos]
Suponga que el proceso
\end{Teo}


\begin{Def}[Renewal Process Trinity]
Para un proceso de renovaci\'on $N\left(t\right)$, los siguientes procesos proveen de informaci\'on sobre los tiempos de renovaci\'on.
\begin{itemize}
\item $A\left(t\right)=t-T_{N\left(t\right)}$, el tiempo de recurrencia hacia atr\'as al tiempo $t$, que es el tiempo desde la \'ultima renovaci\'on para $t$.

\item $B\left(t\right)=T_{N\left(t\right)+1}-t$, el tiempo de recurrencia hacia adelante al tiempo $t$, residual del tiempo de renovaci\'on, que es el tiempo para la pr\'oxima renovaci\'on despu\'es de $t$.

\item $L\left(t\right)=\xi_{N\left(t\right)+1}=A\left(t\right)+B\left(t\right)$, la longitud del intervalo de renovaci\'on que contiene a $t$.
\end{itemize}
\end{Def}

\begin{Note}
El proceso tridimensional $\left(A\left(t\right),B\left(t\right),L\left(t\right)\right)$ es regenerativo sobre $T_{n}$, y por ende cada proceso lo es. Cada proceso $A\left(t\right)$ y $B\left(t\right)$ son procesos de MArkov a tiempo continuo con trayectorias continuas por partes en el espacio de estados $\rea_{+}$. Una expresi\'on conveniente para su distribuci\'on conjunta es, para $0\leq x<t,y\geq0$
\begin{equation}\label{NoRenovacion}
P\left\{A\left(t\right)>x,B\left(t\right)>y\right\}=
P\left\{N\left(t+y\right)-N\left((t-x)\right)=0\right\}
\end{equation}
\end{Note}

\begin{Ejem}[Tiempos de recurrencia Poisson]
Si $N\left(t\right)$ es un proceso Poisson con tasa $\lambda$, entonces de la expresi\'on (\ref{NoRenovacion}) se tiene que

\begin{eqnarray*}
\begin{array}{lc}
P\left\{A\left(t\right)>x,B\left(t\right)>y\right\}=e^{-\lambda\left(x+y\right)},&0\leq x<t,y\geq0,
\end{array}
\end{eqnarray*}
que es la probabilidad Poisson de no renovaciones en un intervalo de longitud $x+y$.

\end{Ejem}

\begin{Note}
Una cadena de Markov erg\'odica tiene la propiedad de ser estacionaria si la distribuci\'on de su estado al tiempo $0$ es su distribuci\'on estacionaria.
\end{Note}


\begin{Def}
Un proceso estoc\'astico a tiempo continuo $\left\{X\left(t\right):t\geq0\right\}$ en un espacio general es estacionario si sus distribuciones finito dimensionales son invariantes bajo cualquier  traslado: para cada $0\leq s_{1}<s_{2}<\cdots<s_{k}$ y $t\geq0$,
\begin{eqnarray*}
\left(X\left(s_{1}+t\right),\ldots,X\left(s_{k}+t\right)\right)=_{d}\left(X\left(s_{1}\right),\ldots,X\left(s_{k}\right)\right).
\end{eqnarray*}
\end{Def}

\begin{Note}
Un proceso de Markov es estacionario si $X\left(t\right)=_{d}X\left(0\right)$, $t\geq0$.
\end{Note}

Considerese el proceso $N\left(t\right)=\sum_{n}\indora\left(\tau_{n}\leq t\right)$ en $\rea_{+}$, con puntos $0<\tau_{1}<\tau_{2}<\cdots$.

\begin{Prop}
Si $N$ es un proceso puntual estacionario y $\esp\left[N\left(1\right)\right]<\infty$, entonces $\esp\left[N\left(t\right)\right]=t\esp\left[N\left(1\right)\right]$, $t\geq0$

\end{Prop}

\begin{Teo}
Los siguientes enunciados son equivalentes
\begin{itemize}
\item[i)] El proceso retardado de renovaci\'on $N$ es estacionario.

\item[ii)] EL proceso de tiempos de recurrencia hacia adelante $B\left(t\right)$ es estacionario.


\item[iii)] $\esp\left[N\left(t\right)\right]=t/\mu$,


\item[iv)] $G\left(t\right)=F_{e}\left(t\right)=\frac{1}{\mu}\int_{0}^{t}\left[1-F\left(s\right)\right]ds$
\end{itemize}
Cuando estos enunciados son ciertos, $P\left\{B\left(t\right)\leq x\right\}=F_{e}\left(x\right)$, para $t,x\geq0$.

\end{Teo}

\begin{Note}
Una consecuencia del teorema anterior es que el Proceso Poisson es el \'unico proceso sin retardo que es estacionario.
\end{Note}

\begin{Coro}
El proceso de renovaci\'on $N\left(t\right)$ sin retardo, y cuyos tiempos de inter renonaci\'on tienen media finita, es estacionario si y s\'olo si es un proceso Poisson.

\end{Coro}


%________________________________________________________________________
%\subsection{Procesos Regenerativos}
%________________________________________________________________________



\begin{Note}
Si $\tilde{X}\left(t\right)$ con espacio de estados $\tilde{S}$ es regenerativo sobre $T_{n}$, entonces $X\left(t\right)=f\left(\tilde{X}\left(t\right)\right)$ tambi\'en es regenerativo sobre $T_{n}$, para cualquier funci\'on $f:\tilde{S}\rightarrow S$.
\end{Note}

\begin{Note}
Los procesos regenerativos son crudamente regenerativos, pero no al rev\'es.
\end{Note}
%\subsection*{Procesos Regenerativos: Sigman\cite{Sigman1}}
\begin{Def}[Definici\'on Cl\'asica]
Un proceso estoc\'astico $X=\left\{X\left(t\right):t\geq0\right\}$ es llamado regenerativo is existe una variable aleatoria $R_{1}>0$ tal que
\begin{itemize}
\item[i)] $\left\{X\left(t+R_{1}\right):t\geq0\right\}$ es independiente de $\left\{\left\{X\left(t\right):t<R_{1}\right\},\right\}$
\item[ii)] $\left\{X\left(t+R_{1}\right):t\geq0\right\}$ es estoc\'asticamente equivalente a $\left\{X\left(t\right):t>0\right\}$
\end{itemize}

Llamamos a $R_{1}$ tiempo de regeneraci\'on, y decimos que $X$ se regenera en este punto.
\end{Def}

$\left\{X\left(t+R_{1}\right)\right\}$ es regenerativo con tiempo de regeneraci\'on $R_{2}$, independiente de $R_{1}$ pero con la misma distribuci\'on que $R_{1}$. Procediendo de esta manera se obtiene una secuencia de variables aleatorias independientes e id\'enticamente distribuidas $\left\{R_{n}\right\}$ llamados longitudes de ciclo. Si definimos a $Z_{k}\equiv R_{1}+R_{2}+\cdots+R_{k}$, se tiene un proceso de renovaci\'on llamado proceso de renovaci\'on encajado para $X$.




\begin{Def}
Para $x$ fijo y para cada $t\geq0$, sea $I_{x}\left(t\right)=1$ si $X\left(t\right)\leq x$,  $I_{x}\left(t\right)=0$ en caso contrario, y def\'inanse los tiempos promedio
\begin{eqnarray*}
\overline{X}&=&lim_{t\rightarrow\infty}\frac{1}{t}\int_{0}^{\infty}X\left(u\right)du\\
\prob\left(X_{\infty}\leq x\right)&=&lim_{t\rightarrow\infty}\frac{1}{t}\int_{0}^{\infty}I_{x}\left(u\right)du,
\end{eqnarray*}
cuando estos l\'imites existan.
\end{Def}

Como consecuencia del teorema de Renovaci\'on-Recompensa, se tiene que el primer l\'imite  existe y es igual a la constante
\begin{eqnarray*}
\overline{X}&=&\frac{\esp\left[\int_{0}^{R_{1}}X\left(t\right)dt\right]}{\esp\left[R_{1}\right]},
\end{eqnarray*}
suponiendo que ambas esperanzas son finitas.

\begin{Note}
\begin{itemize}
\item[a)] Si el proceso regenerativo $X$ es positivo recurrente y tiene trayectorias muestrales no negativas, entonces la ecuaci\'on anterior es v\'alida.
\item[b)] Si $X$ es positivo recurrente regenerativo, podemos construir una \'unica versi\'on estacionaria de este proceso, $X_{e}=\left\{X_{e}\left(t\right)\right\}$, donde $X_{e}$ es un proceso estoc\'astico regenerativo y estrictamente estacionario, con distribuci\'on marginal distribuida como $X_{\infty}$
\end{itemize}
\end{Note}

%________________________________________________________________________
%\subsection{Procesos Regenerativos}
%________________________________________________________________________

Para $\left\{X\left(t\right):t\geq0\right\}$ Proceso Estoc\'astico a tiempo continuo con estado de espacios $S$, que es un espacio m\'etrico, con trayectorias continuas por la derecha y con l\'imites por la izquierda c.s. Sea $N\left(t\right)$ un proceso de renovaci\'on en $\rea_{+}$ definido en el mismo espacio de probabilidad que $X\left(t\right)$, con tiempos de renovaci\'on $T$ y tiempos de inter-renovaci\'on $\xi_{n}=T_{n}-T_{n-1}$, con misma distribuci\'on $F$ de media finita $\mu$.



\begin{Def}
Para el proceso $\left\{\left(N\left(t\right),X\left(t\right)\right):t\geq0\right\}$, sus trayectoria muestrales en el intervalo de tiempo $\left[T_{n-1},T_{n}\right)$ est\'an descritas por
\begin{eqnarray*}
\zeta_{n}=\left(\xi_{n},\left\{X\left(T_{n-1}+t\right):0\leq t<\xi_{n}\right\}\right)
\end{eqnarray*}
Este $\zeta_{n}$ es el $n$-\'esimo segmento del proceso. El proceso es regenerativo sobre los tiempos $T_{n}$ si sus segmentos $\zeta_{n}$ son independientes e id\'enticamennte distribuidos.
\end{Def}


\begin{Note}
Si $\tilde{X}\left(t\right)$ con espacio de estados $\tilde{S}$ es regenerativo sobre $T_{n}$, entonces $X\left(t\right)=f\left(\tilde{X}\left(t\right)\right)$ tambi\'en es regenerativo sobre $T_{n}$, para cualquier funci\'on $f:\tilde{S}\rightarrow S$.
\end{Note}

\begin{Note}
Los procesos regenerativos son crudamente regenerativos, pero no al rev\'es.
\end{Note}

\begin{Def}[Definici\'on Cl\'asica]
Un proceso estoc\'astico $X=\left\{X\left(t\right):t\geq0\right\}$ es llamado regenerativo is existe una variable aleatoria $R_{1}>0$ tal que
\begin{itemize}
\item[i)] $\left\{X\left(t+R_{1}\right):t\geq0\right\}$ es independiente de $\left\{\left\{X\left(t\right):t<R_{1}\right\},\right\}$
\item[ii)] $\left\{X\left(t+R_{1}\right):t\geq0\right\}$ es estoc\'asticamente equivalente a $\left\{X\left(t\right):t>0\right\}$
\end{itemize}

Llamamos a $R_{1}$ tiempo de regeneraci\'on, y decimos que $X$ se regenera en este punto.
\end{Def}

$\left\{X\left(t+R_{1}\right)\right\}$ es regenerativo con tiempo de regeneraci\'on $R_{2}$, independiente de $R_{1}$ pero con la misma distribuci\'on que $R_{1}$. Procediendo de esta manera se obtiene una secuencia de variables aleatorias independientes e id\'enticamente distribuidas $\left\{R_{n}\right\}$ llamados longitudes de ciclo. Si definimos a $Z_{k}\equiv R_{1}+R_{2}+\cdots+R_{k}$, se tiene un proceso de renovaci\'on llamado proceso de renovaci\'on encajado para $X$.

\begin{Note}
Un proceso regenerativo con media de la longitud de ciclo finita es llamado positivo recurrente.
\end{Note}


\begin{Def}
Para $x$ fijo y para cada $t\geq0$, sea $I_{x}\left(t\right)=1$ si $X\left(t\right)\leq x$,  $I_{x}\left(t\right)=0$ en caso contrario, y def\'inanse los tiempos promedio
\begin{eqnarray*}
\overline{X}&=&lim_{t\rightarrow\infty}\frac{1}{t}\int_{0}^{\infty}X\left(u\right)du\\
\prob\left(X_{\infty}\leq x\right)&=&lim_{t\rightarrow\infty}\frac{1}{t}\int_{0}^{\infty}I_{x}\left(u\right)du,
\end{eqnarray*}
cuando estos l\'imites existan.
\end{Def}

Como consecuencia del teorema de Renovaci\'on-Recompensa, se tiene que el primer l\'imite  existe y es igual a la constante
\begin{eqnarray*}
\overline{X}&=&\frac{\esp\left[\int_{0}^{R_{1}}X\left(t\right)dt\right]}{\esp\left[R_{1}\right]},
\end{eqnarray*}
suponiendo que ambas esperanzas son finitas.

\begin{Note}
\begin{itemize}
\item[a)] Si el proceso regenerativo $X$ es positivo recurrente y tiene trayectorias muestrales no negativas, entonces la ecuaci\'on anterior es v\'alida.
\item[b)] Si $X$ es positivo recurrente regenerativo, podemos construir una \'unica versi\'on estacionaria de este proceso, $X_{e}=\left\{X_{e}\left(t\right)\right\}$, donde $X_{e}$ es un proceso estoc\'astico regenerativo y estrictamente estacionario, con distribuci\'on marginal distribuida como $X_{\infty}$
\end{itemize}
\end{Note}

%__________________________________________________________________________________________
%\subsection{Procesos Regenerativos Estacionarios - Stidham \cite{Stidham}}
%__________________________________________________________________________________________


Un proceso estoc\'astico a tiempo continuo $\left\{V\left(t\right),t\geq0\right\}$ es un proceso regenerativo si existe una sucesi\'on de variables aleatorias independientes e id\'enticamente distribuidas $\left\{X_{1},X_{2},\ldots\right\}$, sucesi\'on de renovaci\'on, tal que para cualquier conjunto de Borel $A$, 

\begin{eqnarray*}
\prob\left\{V\left(t\right)\in A|X_{1}+X_{2}+\cdots+X_{R\left(t\right)}=s,\left\{V\left(\tau\right),\tau<s\right\}\right\}=\prob\left\{V\left(t-s\right)\in A|X_{1}>t-s\right\},
\end{eqnarray*}
para todo $0\leq s\leq t$, donde $R\left(t\right)=\max\left\{X_{1}+X_{2}+\cdots+X_{j}\leq t\right\}=$n\'umero de renovaciones ({\emph{puntos de regeneraci\'on}}) que ocurren en $\left[0,t\right]$. El intervalo $\left[0,X_{1}\right)$ es llamado {\emph{primer ciclo de regeneraci\'on}} de $\left\{V\left(t \right),t\geq0\right\}$, $\left[X_{1},X_{1}+X_{2}\right)$ el {\emph{segundo ciclo de regeneraci\'on}}, y as\'i sucesivamente.

Sea $X=X_{1}$ y sea $F$ la funci\'on de distrbuci\'on de $X$


\begin{Def}
Se define el proceso estacionario, $\left\{V^{*}\left(t\right),t\geq0\right\}$, para $\left\{V\left(t\right),t\geq0\right\}$ por

\begin{eqnarray*}
\prob\left\{V\left(t\right)\in A\right\}=\frac{1}{\esp\left[X\right]}\int_{0}^{\infty}\prob\left\{V\left(t+x\right)\in A|X>x\right\}\left(1-F\left(x\right)\right)dx,
\end{eqnarray*} 
para todo $t\geq0$ y todo conjunto de Borel $A$.
\end{Def}

\begin{Def}
Una distribuci\'on se dice que es {\emph{aritm\'etica}} si todos sus puntos de incremento son m\'ultiplos de la forma $0,\lambda, 2\lambda,\ldots$ para alguna $\lambda>0$ entera.
\end{Def}


\begin{Def}
Una modificaci\'on medible de un proceso $\left\{V\left(t\right),t\geq0\right\}$, es una versi\'on de este, $\left\{V\left(t,w\right)\right\}$ conjuntamente medible para $t\geq0$ y para $w\in S$, $S$ espacio de estados para $\left\{V\left(t\right),t\geq0\right\}$.
\end{Def}

\begin{Teo}
Sea $\left\{V\left(t\right),t\geq\right\}$ un proceso regenerativo no negativo con modificaci\'on medible. Sea $\esp\left[X\right]<\infty$. Entonces el proceso estacionario dado por la ecuaci\'on anterior est\'a bien definido y tiene funci\'on de distribuci\'on independiente de $t$, adem\'as
\begin{itemize}
\item[i)] \begin{eqnarray*}
\esp\left[V^{*}\left(0\right)\right]&=&\frac{\esp\left[\int_{0}^{X}V\left(s\right)ds\right]}{\esp\left[X\right]}\end{eqnarray*}
\item[ii)] Si $\esp\left[V^{*}\left(0\right)\right]<\infty$, equivalentemente, si $\esp\left[\int_{0}^{X}V\left(s\right)ds\right]<\infty$,entonces
\begin{eqnarray*}
\frac{\int_{0}^{t}V\left(s\right)ds}{t}\rightarrow\frac{\esp\left[\int_{0}^{X}V\left(s\right)ds\right]}{\esp\left[X\right]}
\end{eqnarray*}
con probabilidad 1 y en media, cuando $t\rightarrow\infty$.
\end{itemize}
\end{Teo}


%__________________________________________________________________________________________
%\subsection{Procesos Regenerativos Estacionarios - Stidham \cite{Stidham}}
%__________________________________________________________________________________________


Un proceso estoc\'astico a tiempo continuo $\left\{V\left(t\right),t\geq0\right\}$ es un proceso regenerativo si existe una sucesi\'on de variables aleatorias independientes e id\'enticamente distribuidas $\left\{X_{1},X_{2},\ldots\right\}$, sucesi\'on de renovaci\'on, tal que para cualquier conjunto de Borel $A$, 

\begin{eqnarray*}
\prob\left\{V\left(t\right)\in A|X_{1}+X_{2}+\cdots+X_{R\left(t\right)}=s,\left\{V\left(\tau\right),\tau<s\right\}\right\}=\prob\left\{V\left(t-s\right)\in A|X_{1}>t-s\right\},
\end{eqnarray*}
para todo $0\leq s\leq t$, donde $R\left(t\right)=\max\left\{X_{1}+X_{2}+\cdots+X_{j}\leq t\right\}=$n\'umero de renovaciones ({\emph{puntos de regeneraci\'on}}) que ocurren en $\left[0,t\right]$. El intervalo $\left[0,X_{1}\right)$ es llamado {\emph{primer ciclo de regeneraci\'on}} de $\left\{V\left(t \right),t\geq0\right\}$, $\left[X_{1},X_{1}+X_{2}\right)$ el {\emph{segundo ciclo de regeneraci\'on}}, y as\'i sucesivamente.

Sea $X=X_{1}$ y sea $F$ la funci\'on de distrbuci\'on de $X$


\begin{Def}
Se define el proceso estacionario, $\left\{V^{*}\left(t\right),t\geq0\right\}$, para $\left\{V\left(t\right),t\geq0\right\}$ por

\begin{eqnarray*}
\prob\left\{V\left(t\right)\in A\right\}=\frac{1}{\esp\left[X\right]}\int_{0}^{\infty}\prob\left\{V\left(t+x\right)\in A|X>x\right\}\left(1-F\left(x\right)\right)dx,
\end{eqnarray*} 
para todo $t\geq0$ y todo conjunto de Borel $A$.
\end{Def}

\begin{Def}
Una distribuci\'on se dice que es {\emph{aritm\'etica}} si todos sus puntos de incremento son m\'ultiplos de la forma $0,\lambda, 2\lambda,\ldots$ para alguna $\lambda>0$ entera.
\end{Def}


\begin{Def}
Una modificaci\'on medible de un proceso $\left\{V\left(t\right),t\geq0\right\}$, es una versi\'on de este, $\left\{V\left(t,w\right)\right\}$ conjuntamente medible para $t\geq0$ y para $w\in S$, $S$ espacio de estados para $\left\{V\left(t\right),t\geq0\right\}$.
\end{Def}

\begin{Teo}
Sea $\left\{V\left(t\right),t\geq\right\}$ un proceso regenerativo no negativo con modificaci\'on medible. Sea $\esp\left[X\right]<\infty$. Entonces el proceso estacionario dado por la ecuaci\'on anterior est\'a bien definido y tiene funci\'on de distribuci\'on independiente de $t$, adem\'as
\begin{itemize}
\item[i)] \begin{eqnarray*}
\esp\left[V^{*}\left(0\right)\right]&=&\frac{\esp\left[\int_{0}^{X}V\left(s\right)ds\right]}{\esp\left[X\right]}\end{eqnarray*}
\item[ii)] Si $\esp\left[V^{*}\left(0\right)\right]<\infty$, equivalentemente, si $\esp\left[\int_{0}^{X}V\left(s\right)ds\right]<\infty$,entonces
\begin{eqnarray*}
\frac{\int_{0}^{t}V\left(s\right)ds}{t}\rightarrow\frac{\esp\left[\int_{0}^{X}V\left(s\right)ds\right]}{\esp\left[X\right]}
\end{eqnarray*}
con probabilidad 1 y en media, cuando $t\rightarrow\infty$.
\end{itemize}
\end{Teo}

Para $\left\{X\left(t\right):t\geq0\right\}$ Proceso Estoc\'astico a tiempo continuo con estado de espacios $S$, que es un espacio m\'etrico, con trayectorias continuas por la derecha y con l\'imites por la izquierda c.s. Sea $N\left(t\right)$ un proceso de renovaci\'on en $\rea_{+}$ definido en el mismo espacio de probabilidad que $X\left(t\right)$, con tiempos de renovaci\'on $T$ y tiempos de inter-renovaci\'on $\xi_{n}=T_{n}-T_{n-1}$, con misma distribuci\'on $F$ de media finita $\mu$.



Sean $T_{1},T_{2},\ldots$ los puntos donde las longitudes de las colas de la red de sistemas de visitas c\'iclicas son cero simult\'aneamente, cuando la cola $Q_{j}$ es visitada por el servidor para dar servicio, es decir, $L_{1}\left(T_{i}\right)=0,L_{2}\left(T_{i}\right)=0,\hat{L}_{1}\left(T_{i}\right)=0$ y $\hat{L}_{2}\left(T_{i}\right)=0$, a estos puntos se les denominar\'a puntos regenerativos. Sea la funci\'on generadora de momentos para $L_{i}$, el n\'umero de usuarios en la cola $Q_{i}\left(z\right)$ en cualquier momento, est\'a dada por el tiempo promedio de $z^{L_{i}\left(t\right)}$ sobre el ciclo regenerativo definido anteriormente:

\begin{eqnarray*}
Q_{i}\left(z\right)&=&\esp\left[z^{L_{i}\left(t\right)}\right]=\frac{\esp\left[\sum_{m=1}^{M_{i}}\sum_{t=\tau_{i}\left(m\right)}^{\tau_{i}\left(m+1\right)-1}z^{L_{i}\left(t\right)}\right]}{\esp\left[\sum_{m=1}^{M_{i}}\tau_{i}\left(m+1\right)-\tau_{i}\left(m\right)\right]}
\end{eqnarray*}

$M_{i}$ es un tiempo de paro en el proceso regenerativo con $\esp\left[M_{i}\right]<\infty$\footnote{En Stidham\cite{Stidham} y Heyman  se muestra que una condici\'on suficiente para que el proceso regenerativo 
estacionario sea un procesoo estacionario es que el valor esperado del tiempo del ciclo regenerativo sea finito, es decir: $\esp\left[\sum_{m=1}^{M_{i}}C_{i}^{(m)}\right]<\infty$, como cada $C_{i}^{(m)}$ contiene intervalos de r\'eplica positivos, se tiene que $\esp\left[M_{i}\right]<\infty$, adem\'as, como $M_{i}>0$, se tiene que la condici\'on anterior es equivalente a tener que $\esp\left[C_{i}\right]<\infty$,
por lo tanto una condici\'on suficiente para la existencia del proceso regenerativo est\'a dada por $\sum_{k=1}^{N}\mu_{k}<1.$}, se sigue del lema de Wald que:


\begin{eqnarray*}
\esp\left[\sum_{m=1}^{M_{i}}\sum_{t=\tau_{i}\left(m\right)}^{\tau_{i}\left(m+1\right)-1}z^{L_{i}\left(t\right)}\right]&=&\esp\left[M_{i}\right]\esp\left[\sum_{t=\tau_{i}\left(m\right)}^{\tau_{i}\left(m+1\right)-1}z^{L_{i}\left(t\right)}\right]\\
\esp\left[\sum_{m=1}^{M_{i}}\tau_{i}\left(m+1\right)-\tau_{i}\left(m\right)\right]&=&\esp\left[M_{i}\right]\esp\left[\tau_{i}\left(m+1\right)-\tau_{i}\left(m\right)\right]
\end{eqnarray*}

por tanto se tiene que


\begin{eqnarray*}
Q_{i}\left(z\right)&=&\frac{\esp\left[\sum_{t=\tau_{i}\left(m\right)}^{\tau_{i}\left(m+1\right)-1}z^{L_{i}\left(t\right)}\right]}{\esp\left[\tau_{i}\left(m+1\right)-\tau_{i}\left(m\right)\right]}
\end{eqnarray*}

observar que el denominador es simplemente la duraci\'on promedio del tiempo del ciclo.


Haciendo las siguientes sustituciones en la ecuaci\'on (\ref{Corolario2}): $n\rightarrow t-\tau_{i}\left(m\right)$, $T \rightarrow \overline{\tau}_{i}\left(m\right)-\tau_{i}\left(m\right)$, $L_{n}\rightarrow L_{i}\left(t\right)$ y $F\left(z\right)=\esp\left[z^{L_{0}}\right]\rightarrow F_{i}\left(z\right)=\esp\left[z^{L_{i}\tau_{i}\left(m\right)}\right]$, se puede ver que

\begin{eqnarray}\label{Eq.Arribos.Primera}
\esp\left[\sum_{n=0}^{T-1}z^{L_{n}}\right]=
\esp\left[\sum_{t=\tau_{i}\left(m\right)}^{\overline{\tau}_{i}\left(m\right)-1}z^{L_{i}\left(t\right)}\right]
=z\frac{F_{i}\left(z\right)-1}{z-P_{i}\left(z\right)}
\end{eqnarray}

Por otra parte durante el tiempo de intervisita para la cola $i$, $L_{i}\left(t\right)$ solamente se incrementa de manera que el incremento por intervalo de tiempo est\'a dado por la funci\'on generadora de probabilidades de $P_{i}\left(z\right)$, por tanto la suma sobre el tiempo de intervisita puede evaluarse como:

\begin{eqnarray*}
\esp\left[\sum_{t=\tau_{i}\left(m\right)}^{\tau_{i}\left(m+1\right)-1}z^{L_{i}\left(t\right)}\right]&=&\esp\left[\sum_{t=\tau_{i}\left(m\right)}^{\tau_{i}\left(m+1\right)-1}\left\{P_{i}\left(z\right)\right\}^{t-\overline{\tau}_{i}\left(m\right)}\right]=\frac{1-\esp\left[\left\{P_{i}\left(z\right)\right\}^{\tau_{i}\left(m+1\right)-\overline{\tau}_{i}\left(m\right)}\right]}{1-P_{i}\left(z\right)}\\
&=&\frac{1-I_{i}\left[P_{i}\left(z\right)\right]}{1-P_{i}\left(z\right)}
\end{eqnarray*}
por tanto

\begin{eqnarray*}
\esp\left[\sum_{t=\tau_{i}\left(m\right)}^{\tau_{i}\left(m+1\right)-1}z^{L_{i}\left(t\right)}\right]&=&
\frac{1-F_{i}\left(z\right)}{1-P_{i}\left(z\right)}
\end{eqnarray*}

Por lo tanto

\begin{eqnarray*}
Q_{i}\left(z\right)&=&\frac{\esp\left[\sum_{t=\tau_{i}\left(m\right)}^{\tau_{i}
\left(m+1\right)-1}z^{L_{i}\left(t\right)}\right]}{\esp\left[\tau_{i}\left(m+1\right)-\tau_{i}\left(m\right)\right]}\\
&=&\frac{1}{\esp\left[\tau_{i}\left(m+1\right)-\tau_{i}\left(m\right)\right]}
\left\{
\esp\left[\sum_{t=\tau_{i}\left(m\right)}^{\overline{\tau}_{i}\left(m\right)-1}
z^{L_{i}\left(t\right)}\right]
+\esp\left[\sum_{t=\overline{\tau}_{i}\left(m\right)}^{\tau_{i}\left(m+1\right)-1}
z^{L_{i}\left(t\right)}\right]\right\}\\
&=&\frac{1}{\esp\left[\tau_{i}\left(m+1\right)-\tau_{i}\left(m\right)\right]}
\left\{
z\frac{F_{i}\left(z\right)-1}{z-P_{i}\left(z\right)}+\frac{1-F_{i}\left(z\right)}
{1-P_{i}\left(z\right)}
\right\}
\end{eqnarray*}

es decir

\begin{equation}
Q_{i}\left(z\right)=\frac{1}{\esp\left[C_{i}\right]}\cdot\frac{1-F_{i}\left(z\right)}{P_{i}\left(z\right)-z}\cdot\frac{\left(1-z\right)P_{i}\left(z\right)}{1-P_{i}\left(z\right)}
\end{equation}

\begin{Teo}
Dada una Red de Sistemas de Visitas C\'iclicas (RSVC), conformada por dos Sistemas de Visitas C\'iclicas (SVC), donde cada uno de ellos consta de dos colas tipo $M/M/1$. Los dos sistemas est\'an comunicados entre s\'i por medio de la transferencia de usuarios entre las colas $Q_{1}\leftrightarrow Q_{3}$ y $Q_{2}\leftrightarrow Q_{4}$. Se definen los eventos para los procesos de arribos al tiempo $t$, $A_{j}\left(t\right)=\left\{0 \textrm{ arribos en }Q_{j}\textrm{ al tiempo }t\right\}$ para alg\'un tiempo $t\geq0$ y $Q_{j}$ la cola $j$-\'esima en la RSVC, para $j=1,2,3,4$.  Existe un intervalo $I\neq\emptyset$ tal que para $T^{*}\in I$, tal que $\prob\left\{A_{1}\left(T^{*}\right),A_{2}\left(Tt^{*}\right),
A_{3}\left(T^{*}\right),A_{4}\left(T^{*}\right)|T^{*}\in I\right\}>0$.
\end{Teo}

\begin{proof}
Sin p\'erdida de generalidad podemos considerar como base del an\'alisis a la cola $Q_{1}$ del primer sistema que conforma la RSVC.

Sea $n>0$, ciclo en el primer sistema en el que se sabe que $L_{j}\left(\overline{\tau}_{1}\left(n\right)\right)=0$, pues la pol\'itica de servicio con que atienden los servidores es la exhaustiva. Como es sabido, para trasladarse a la siguiente cola, el servidor incurre en un tiempo de traslado $r_{1}\left(n\right)>0$, entonces tenemos que $\tau_{2}\left(n\right)=\overline{\tau}_{1}\left(n\right)+r_{1}\left(n\right)$.


Definamos el intervalo $I_{1}\equiv\left[\overline{\tau}_{1}\left(n\right),\tau_{2}\left(n\right)\right]$ de longitud $\xi_{1}=r_{1}\left(n\right)$. Dado que los tiempos entre arribo son exponenciales con tasa $\tilde{\mu}_{1}=\mu_{1}+\hat{\mu}_{1}$ ($\mu_{1}$ son los arribos a $Q_{1}$ por primera vez al sistema, mientras que $\hat{\mu}_{1}$ son los arribos de traslado procedentes de $Q_{3}$) se tiene que la probabilidad del evento $A_{1}\left(t\right)$ est\'a dada por 

\begin{equation}
\prob\left\{A_{1}\left(t\right)|t\in I_{1}\left(n\right)\right\}=e^{-\tilde{\mu}_{1}\xi_{1}\left(n\right)}.
\end{equation} 

Por otra parte, para la cola $Q_{2}$, el tiempo $\overline{\tau}_{2}\left(n-1\right)$ es tal que $L_{2}\left(\overline{\tau}_{2}\left(n-1\right)\right)=0$, es decir, es el tiempo en que la cola queda totalmente vac\'ia en el ciclo anterior a $n$. Entonces tenemos un sgundo intervalo $I_{2}\equiv\left[\overline{\tau}_{2}\left(n-1\right),\tau_{2}\left(n\right)\right]$. Por lo tanto la probabilidad del evento $A_{2}\left(t\right)$ tiene probabilidad dada por

\begin{equation}
\prob\left\{A_{2}\left(t\right)|t\in I_{2}\left(n\right)\right\}=e^{-\tilde{\mu}_{2}\xi_{2}\left(n\right)},
\end{equation} 

donde $\xi_{2}\left(n\right)=\tau_{2}\left(n\right)-\overline{\tau}_{2}\left(n-1\right)$.



Entonces, se tiene que

\begin{eqnarray*}
\prob\left\{A_{1}\left(t\right),A_{2}\left(t\right)|t\in I_{1}\left(n\right)\right\}&=&
\prob\left\{A_{1}\left(t\right)|t\in I_{1}\left(n\right)\right\}
\prob\left\{A_{2}\left(t\right)|t\in I_{1}\left(n\right)\right\}\\
&\geq&
\prob\left\{A_{1}\left(t\right)|t\in I_{1}\left(n\right)\right\}
\prob\left\{A_{2}\left(t\right)|t\in I_{2}\left(n\right)\right\}\\
&=&e^{-\tilde{\mu}_{1}\xi_{1}\left(n\right)}e^{-\tilde{\mu}_{2}\xi_{2}\left(n\right)}
=e^{-\left[\tilde{\mu}_{1}\xi_{1}\left(n\right)+\tilde{\mu}_{2}\xi_{2}\left(n\right)\right]}.
\end{eqnarray*}


es decir, 

\begin{equation}
\prob\left\{A_{1}\left(t\right),A_{2}\left(t\right)|t\in I_{1}\left(n\right)\right\}
=e^{-\left[\tilde{\mu}_{1}\xi_{1}\left(n\right)+\tilde{\mu}_{2}\xi_{2}
\left(n\right)\right]}>0.
\end{equation}

En lo que respecta a la relaci\'on entre los dos SVC que conforman la RSVC, se afirma que existe $m>0$ tal que $\overline{\tau}_{3}\left(m\right)<\tau_{2}\left(n\right)<\tau_{4}\left(m\right)$.

Para $Q_{3}$ sea $I_{3}=\left[\overline{\tau}_{3}\left(m\right),\tau_{4}\left(m\right)\right]$ con longitud  $\xi_{3}\left(m\right)=r_{3}\left(m\right)$, entonces 

\begin{equation}
\prob\left\{A_{3}\left(t\right)|t\in I_{3}\left(n\right)\right\}=e^{-\tilde{\mu}_{3}\xi_{3}\left(n\right)}.
\end{equation} 

An\'alogamente que como se hizo para $Q_{2}$, tenemos que para $Q_{4}$ se tiene el intervalo $I_{4}=\left[\overline{\tau}_{4}\left(m-1\right),\tau_{4}\left(m\right)\right]$ con longitud $\xi_{4}\left(m\right)=\tau_{4}\left(m\right)-\overline{\tau}_{4}\left(m-1\right)$, entonces


\begin{equation}
\prob\left\{A_{4}\left(t\right)|t\in I_{4}\left(m\right)\right\}=e^{-\tilde{\mu}_{4}\xi_{4}\left(n\right)}.
\end{equation} 

Al igual que para el primer sistema, dado que $I_{3}\left(m\right)\subset I_{4}\left(m\right)$, se tiene que

\begin{eqnarray*}
\xi_{3}\left(m\right)\leq\xi_{4}\left(m\right)&\Leftrightarrow& -\xi_{3}\left(m\right)\geq-\xi_{4}\left(m\right)
\\
-\tilde{\mu}_{4}\xi_{3}\left(m\right)\geq-\tilde{\mu}_{4}\xi_{4}\left(m\right)&\Leftrightarrow&
e^{-\tilde{\mu}_{4}\xi_{3}\left(m\right)}\geq e^{-\tilde{\mu}_{4}\xi_{4}\left(m\right)}\\
\prob\left\{A_{4}\left(t\right)|t\in I_{3}\left(m\right)\right\}&\geq&
\prob\left\{A_{4}\left(t\right)|t\in I_{4}\left(m\right)\right\}
\end{eqnarray*}

Entonces, dado que los eventos $A_{3}$ y $A_{4}$ son independientes, se tiene que

\begin{eqnarray*}
\prob\left\{A_{3}\left(t\right),A_{4}\left(t\right)|t\in I_{3}\left(m\right)\right\}&=&
\prob\left\{A_{3}\left(t\right)|t\in I_{3}\left(m\right)\right\}
\prob\left\{A_{4}\left(t\right)|t\in I_{3}\left(m\right)\right\}\\
&\geq&
\prob\left\{A_{3}\left(t\right)|t\in I_{3}\left(n\right)\right\}
\prob\left\{A_{4}\left(t\right)|t\in I_{4}\left(n\right)\right\}\\
&=&e^{-\tilde{\mu}_{3}\xi_{3}\left(m\right)}e^{-\tilde{\mu}_{4}\xi_{4}
\left(m\right)}
=e^{-\left[\tilde{\mu}_{3}\xi_{3}\left(m\right)+\tilde{\mu}_{4}\xi_{4}
\left(m\right)\right]}.
\end{eqnarray*}


es decir, 

\begin{equation}
\prob\left\{A_{3}\left(t\right),A_{4}\left(t\right)|t\in I_{3}\left(m\right)\right\}
=e^{-\left[\tilde{\mu}_{3}\xi_{3}\left(m\right)+\tilde{\mu}_{4}\xi_{4}
\left(m\right)\right]}>0.
\end{equation}

Por construcci\'on se tiene que $I\left(n,m\right)\equiv I_{1}\left(n\right)\cap I_{3}\left(m\right)\neq\emptyset$,entonces en particular se tienen las contenciones $I\left(n,m\right)\subseteq I_{1}\left(n\right)$ y $I\left(n,m\right)\subseteq I_{3}\left(m\right)$, por lo tanto si definimos $\xi_{n,m}\equiv\ell\left(I\left(n,m\right)\right)$ tenemos que

\begin{eqnarray*}
\xi_{n,m}\leq\xi_{1}\left(n\right)\textrm{ y }\xi_{n,m}\leq\xi_{3}\left(m\right)\textrm{ entonces }
-\xi_{n,m}\geq-\xi_{1}\left(n\right)\textrm{ y }-\xi_{n,m}\leq-\xi_{3}\left(m\right)\\
\end{eqnarray*}
por lo tanto tenemos las desigualdades 



\begin{eqnarray*}
\begin{array}{ll}
-\tilde{\mu}_{1}\xi_{n,m}\geq-\tilde{\mu}_{1}\xi_{1}\left(n\right),&
-\tilde{\mu}_{2}\xi_{n,m}\geq-\tilde{\mu}_{2}\xi_{1}\left(n\right)
\geq-\tilde{\mu}_{2}\xi_{2}\left(n\right),\\
-\tilde{\mu}_{3}\xi_{n,m}\geq-\tilde{\mu}_{3}\xi_{3}\left(m\right),&
-\tilde{\mu}_{4}\xi_{n,m}\geq-\tilde{\mu}_{4}\xi_{3}\left(m\right)
\geq-\tilde{\mu}_{4}\xi_{4}\left(m\right).
\end{array}
\end{eqnarray*}

Sea $T^{*}\in I_{n,m}$, entonces dado que en particular $T^{*}\in I_{1}\left(n\right)$ se cumple con probabilidad positiva que no hay arribos a las colas $Q_{1}$ y $Q_{2}$, en consecuencia, tampoco hay usuarios de transferencia para $Q_{3}$ y $Q_{4}$, es decir, $\tilde{\mu}_{1}=\mu_{1}$, $\tilde{\mu}_{2}=\mu_{2}$, $\tilde{\mu}_{3}=\mu_{3}$, $\tilde{\mu}_{4}=\mu_{4}$, es decir, los eventos $Q_{1}$ y $Q_{3}$ son condicionalmente independientes en el intervalo $I_{n,m}$; lo mismo ocurre para las colas $Q_{2}$ y $Q_{4}$, por lo tanto tenemos que


\begin{eqnarray}
\begin{array}{l}
\prob\left\{A_{1}\left(T^{*}\right),A_{2}\left(T^{*}\right),
A_{3}\left(T^{*}\right),A_{4}\left(T^{*}\right)|T^{*}\in I_{n,m}\right\}
=\prod_{j=1}^{4}\prob\left\{A_{j}\left(T^{*}\right)|T^{*}\in I_{n,m}\right\}\\
\geq\prob\left\{A_{1}\left(T^{*}\right)|T^{*}\in I_{1}\left(n\right)\right\}
\prob\left\{A_{2}\left(T^{*}\right)|T^{*}\in I_{2}\left(n\right)\right\}
\prob\left\{A_{3}\left(T^{*}\right)|T^{*}\in I_{3}\left(m\right)\right\}
\prob\left\{A_{4}\left(T^{*}\right)|T^{*}\in I_{4}\left(m\right)\right\}\\
=e^{-\mu_{1}\xi_{1}\left(n\right)}
e^{-\mu_{2}\xi_{2}\left(n\right)}
e^{-\mu_{3}\xi_{3}\left(m\right)}
e^{-\mu_{4}\xi_{4}\left(m\right)}
=e^{-\left[\tilde{\mu}_{1}\xi_{1}\left(n\right)
+\tilde{\mu}_{2}\xi_{2}\left(n\right)
+\tilde{\mu}_{3}\xi_{3}\left(m\right)
+\tilde{\mu}_{4}\xi_{4}
\left(m\right)\right]}>0.
\end{array}
\end{eqnarray}
\end{proof}


Estos resultados aparecen en Daley (1968) \cite{Daley68} para $\left\{T_{n}\right\}$ intervalos de inter-arribo, $\left\{D_{n}\right\}$ intervalos de inter-salida y $\left\{S_{n}\right\}$ tiempos de servicio.

\begin{itemize}
\item Si el proceso $\left\{T_{n}\right\}$ es Poisson, el proceso $\left\{D_{n}\right\}$ es no correlacionado si y s\'olo si es un proceso Poisso, lo cual ocurre si y s\'olo si $\left\{S_{n}\right\}$ son exponenciales negativas.

\item Si $\left\{S_{n}\right\}$ son exponenciales negativas, $\left\{D_{n}\right\}$ es un proceso de renovaci\'on  si y s\'olo si es un proceso Poisson, lo cual ocurre si y s\'olo si $\left\{T_{n}\right\}$ es un proceso Poisson.

\item $\esp\left(D_{n}\right)=\esp\left(T_{n}\right)$.

\item Para un sistema de visitas $GI/M/1$ se tiene el siguiente teorema:

\begin{Teo}
En un sistema estacionario $GI/M/1$ los intervalos de interpartida tienen
\begin{eqnarray*}
\esp\left(e^{-\theta D_{n}}\right)&=&\mu\left(\mu+\theta\right)^{-1}\left[\delta\theta
-\mu\left(1-\delta\right)\alpha\left(\theta\right)\right]
\left[\theta-\mu\left(1-\delta\right)^{-1}\right]\\
\alpha\left(\theta\right)&=&\esp\left[e^{-\theta T_{0}}\right]\\
var\left(D_{n}\right)&=&var\left(T_{0}\right)-\left(\tau^{-1}-\delta^{-1}\right)
2\delta\left(\esp\left(S_{0}\right)\right)^{2}\left(1-\delta\right)^{-1}.
\end{eqnarray*}
\end{Teo}



\begin{Teo}
El proceso de salida de un sistema de colas estacionario $GI/M/1$ es un proceso de renovaci\'on si y s\'olo si el proceso de entrada es un proceso Poisson, en cuyo caso el proceso de salida es un proceso Poisson.
\end{Teo}


\begin{Teo}
Los intervalos de interpartida $\left\{D_{n}\right\}$ de un sistema $M/G/1$ estacionario son no correlacionados si y s\'olo si la distribuci\'on de los tiempos de servicio es exponencial negativa, es decir, el sistema es de tipo  $M/M/1$.

\end{Teo}



\end{itemize}


\subsection{Resultados para Procesos de Salida}

En Sigman, Thorison y Wolff \cite{Sigman2} prueban que para la existencia de un una sucesi\'on infinita no decreciente de tiempos de regeneraci\'on $\tau_{1}\leq\tau_{2}\leq\cdots$ en los cuales el proceso se regenera, basta un tiempo de regeneraci\'on $R_{1}$, donde $R_{j}=\tau_{j}-\tau_{j-1}$. Para tal efecto se requiere la existencia de un espacio de probabilidad $\left(\Omega,\mathcal{F},\prob\right)$, y proceso estoc\'astico $\textit{X}=\left\{X\left(t\right):t\geq0\right\}$ con espacio de estados $\left(S,\mathcal{R}\right)$, con $\mathcal{R}$ $\sigma$-\'algebra.

\begin{Prop}
Si existe una variable aleatoria no negativa $R_{1}$ tal que $\theta_{R\footnotesize{1}}X=_{D}X$, entonces $\left(\Omega,\mathcal{F},\prob\right)$ puede extenderse para soportar una sucesi\'on estacionaria de variables aleatorias $R=\left\{R_{k}:k\geq1\right\}$, tal que para $k\geq1$,
\begin{eqnarray*}
\theta_{k}\left(X,R\right)=_{D}\left(X,R\right).
\end{eqnarray*}

Adem\'as, para $k\geq1$, $\theta_{k}R$ es condicionalmente independiente de $\left(X,R_{1},\ldots,R_{k}\right)$, dado $\theta_{\tau k}X$.

\end{Prop}


\begin{itemize}
\item Doob en 1953 demostr\'o que el estado estacionario de un proceso de partida en un sistema de espera $M/G/\infty$, es Poisson con la misma tasa que el proceso de arribos.

\item Burke en 1968, fue el primero en demostrar que el estado estacionario de un proceso de salida de una cola $M/M/s$ es un proceso Poisson.

\item Disney en 1973 obtuvo el siguiente resultado:

\begin{Teo}
Para el sistema de espera $M/G/1/L$ con disciplina FIFO, el proceso $\textbf{I}$ es un proceso de renovaci\'on si y s\'olo si el proceso denominado longitud de la cola es estacionario y se cumple cualquiera de los siguientes casos:

\begin{itemize}
\item[a)] Los tiempos de servicio son identicamente cero;
\item[b)] $L=0$, para cualquier proceso de servicio $S$;
\item[c)] $L=1$ y $G=D$;
\item[d)] $L=\infty$ y $G=M$.
\end{itemize}
En estos casos, respectivamente, las distribuciones de interpartida $P\left\{T_{n+1}-T_{n}\leq t\right\}$ son


\begin{itemize}
\item[a)] $1-e^{-\lambda t}$, $t\geq0$;
\item[b)] $1-e^{-\lambda t}*F\left(t\right)$, $t\geq0$;
\item[c)] $1-e^{-\lambda t}*\indora_{d}\left(t\right)$, $t\geq0$;
\item[d)] $1-e^{-\lambda t}*F\left(t\right)$, $t\geq0$.
\end{itemize}
\end{Teo}


\item Finch (1959) mostr\'o que para los sistemas $M/G/1/L$, con $1\leq L\leq \infty$ con distribuciones de servicio dos veces diferenciable, solamente el sistema $M/M/1/\infty$ tiene proceso de salida de renovaci\'on estacionario.

\item King (1971) demostro que un sistema de colas estacionario $M/G/1/1$ tiene sus tiempos de interpartida sucesivas $D_{n}$ y $D_{n+1}$ son independientes, si y s\'olo si, $G=D$, en cuyo caso le proceso de salida es de renovaci\'on.

\item Disney (1973) demostr\'o que el \'unico sistema estacionario $M/G/1/L$, que tiene proceso de salida de renovaci\'on  son los sistemas $M/M/1$ y $M/D/1/1$.



\item El siguiente resultado es de Disney y Koning (1985)
\begin{Teo}
En un sistema de espera $M/G/s$, el estado estacionario del proceso de salida es un proceso Poisson para cualquier distribuci\'on de los tiempos de servicio si el sistema tiene cualquiera de las siguientes cuatro propiedades.

\begin{itemize}
\item[a)] $s=\infty$
\item[b)] La disciplina de servicio es de procesador compartido.
\item[c)] La disciplina de servicio es LCFS y preemptive resume, esto se cumple para $L<\infty$
\item[d)] $G=M$.
\end{itemize}

\end{Teo}

\item El siguiente resultado es de Alamatsaz (1983)

\begin{Teo}
En cualquier sistema de colas $GI/G/1/L$ con $1\leq L<\infty$ y distribuci\'on de interarribos $A$ y distribuci\'on de los tiempos de servicio $B$, tal que $A\left(0\right)=0$, $A\left(t\right)\left(1-B\left(t\right)\right)>0$ para alguna $t>0$ y $B\left(t\right)$ para toda $t>0$, es imposible que el proceso de salida estacionario sea de renovaci\'on.
\end{Teo}

\end{itemize}

Estos resultados aparecen en Daley (1968) \cite{Daley68} para $\left\{T_{n}\right\}$ intervalos de inter-arribo, $\left\{D_{n}\right\}$ intervalos de inter-salida y $\left\{S_{n}\right\}$ tiempos de servicio.

\begin{itemize}
\item Si el proceso $\left\{T_{n}\right\}$ es Poisson, el proceso $\left\{D_{n}\right\}$ es no correlacionado si y s\'olo si es un proceso Poisso, lo cual ocurre si y s\'olo si $\left\{S_{n}\right\}$ son exponenciales negativas.

\item Si $\left\{S_{n}\right\}$ son exponenciales negativas, $\left\{D_{n}\right\}$ es un proceso de renovaci\'on  si y s\'olo si es un proceso Poisson, lo cual ocurre si y s\'olo si $\left\{T_{n}\right\}$ es un proceso Poisson.

\item $\esp\left(D_{n}\right)=\esp\left(T_{n}\right)$.

\item Para un sistema de visitas $GI/M/1$ se tiene el siguiente teorema:

\begin{Teo}
En un sistema estacionario $GI/M/1$ los intervalos de interpartida tienen
\begin{eqnarray*}
\esp\left(e^{-\theta D_{n}}\right)&=&\mu\left(\mu+\theta\right)^{-1}\left[\delta\theta
-\mu\left(1-\delta\right)\alpha\left(\theta\right)\right]
\left[\theta-\mu\left(1-\delta\right)^{-1}\right]\\
\alpha\left(\theta\right)&=&\esp\left[e^{-\theta T_{0}}\right]\\
var\left(D_{n}\right)&=&var\left(T_{0}\right)-\left(\tau^{-1}-\delta^{-1}\right)
2\delta\left(\esp\left(S_{0}\right)\right)^{2}\left(1-\delta\right)^{-1}.
\end{eqnarray*}
\end{Teo}



\begin{Teo}
El proceso de salida de un sistema de colas estacionario $GI/M/1$ es un proceso de renovaci\'on si y s\'olo si el proceso de entrada es un proceso Poisson, en cuyo caso el proceso de salida es un proceso Poisson.
\end{Teo}


\begin{Teo}
Los intervalos de interpartida $\left\{D_{n}\right\}$ de un sistema $M/G/1$ estacionario son no correlacionados si y s\'olo si la distribuci\'on de los tiempos de servicio es exponencial negativa, es decir, el sistema es de tipo  $M/M/1$.

\end{Teo}



\end{itemize}
%\newpage

\section{Aplicaci\'on a Teor\'ia de Colas}



Def\'inanse los puntos de regenaraci\'on  en el proceso $\left[L_{1}\left(t\right),L_{2}\left(t\right),\ldots,L_{N}\left(t\right)\right]$. Los puntos cuando la cola $i$ es visitada y todos los $L_{j}\left(\tau_{i}\left(m\right)\right)=0$ para $i=1,2$  son puntos de regeneraci\'on. Se llama ciclo regenerativo al intervalo entre dos puntos regenerativos sucesivos.

Sea $M_{i}$  el n\'umero de ciclos de visita en un ciclo regenerativo, y sea $C_{i}^{(m)}$, para $m=1,2,\ldots,M_{i}$ la duraci\'on del $m$-\'esimo ciclo de visita en un ciclo regenerativo. Se define el ciclo del tiempo de visita promedio $\esp\left[C_{i}\right]$ como

\begin{eqnarray*}
\esp\left[C_{i}\right]&=&\frac{\esp\left[\sum_{m=1}^{M_{i}}C_{i}^{(m)}\right]}{\esp\left[M_{i}\right]}
\end{eqnarray*}




Sea la funci\'on generadora de momentos para $L_{i}$, el n\'umero de usuarios en la cola $Q_{i}\left(z\right)$ en cualquier momento, est\'a dada por el tiempo promedio de $z^{L_{i}\left(t\right)}$ sobre el ciclo regenerativo definido anteriormente:

\begin{eqnarray*}
Q_{i}\left(z\right)&=&\esp\left[z^{L_{i}\left(t\right)}\right]=\frac{\esp\left[\sum_{m=1}^{M_{i}}\sum_{t=\tau_{i}\left(m\right)}^{\tau_{i}\left(m+1\right)-1}z^{L_{i}\left(t\right)}\right]}{\esp\left[\sum_{m=1}^{M_{i}}\tau_{i}\left(m+1\right)-\tau_{i}\left(m\right)\right]}
\end{eqnarray*}

$M_{i}$ es un tiempo de paro en el proceso regenerativo con $\esp\left[M_{i}\right]<\infty$, se sigue del lema de Wald que:


\begin{eqnarray*}
\esp\left[\sum_{m=1}^{M_{i}}\sum_{t=\tau_{i}\left(m\right)}^{\tau_{i}\left(m+1\right)-1}z^{L_{i}\left(t\right)}\right]&=&\esp\left[M_{i}\right]\esp\left[\sum_{t=\tau_{i}\left(m\right)}^{\tau_{i}\left(m+1\right)-1}z^{L_{i}\left(t\right)}\right]\\
\esp\left[\sum_{m=1}^{M_{i}}\tau_{i}\left(m+1\right)-\tau_{i}\left(m\right)\right]&=&\esp\left[M_{i}\right]\esp\left[\tau_{i}\left(m+1\right)-\tau_{i}\left(m\right)\right]
\end{eqnarray*}

por tanto se tiene que


\begin{eqnarray*}
Q_{i}\left(z\right)&=&\frac{\esp\left[\sum_{t=\tau_{i}\left(m\right)}^{\tau_{i}\left(m+1\right)-1}z^{L_{i}\left(t\right)}\right]}{\esp\left[\tau_{i}\left(m+1\right)-\tau_{i}\left(m\right)\right]}
\end{eqnarray*}

observar que el denominador es simplemente la duraci\'on promedio del tiempo del ciclo.


Se puede demostrar (ver Hideaki Takagi 1986) que

\begin{eqnarray*}
\esp\left[\sum_{t=\tau_{i}\left(m\right)}^{\tau_{i}\left(m+1\right)-1}z^{L_{i}\left(t\right)}\right]=z\frac{F_{i}\left(z\right)-1}{z-P_{i}\left(z\right)}
\end{eqnarray*}

Durante el tiempo de intervisita para la cola $i$, $L_{i}\left(t\right)$ solamente se incrementa de manera que el incremento por intervalo de tiempo est\'a dado por la funci\'on generadora de probabilidades de $P_{i}\left(z\right)$, por tanto la suma sobre el tiempo de intervisita puede evaluarse como:

\begin{eqnarray*}
\esp\left[\sum_{t=\tau_{i}\left(m\right)}^{\tau_{i}\left(m+1\right)-1}z^{L_{i}\left(t\right)}\right]&=&\esp\left[\sum_{t=\tau_{i}\left(m\right)}^{\tau_{i}\left(m+1\right)-1}\left\{P_{i}\left(z\right)\right\}^{t-\overline{\tau}_{i}\left(m\right)}\right]=\frac{1-\esp\left[\left\{P_{i}\left(z\right)\right\}^{\tau_{i}\left(m+1\right)-\overline{\tau}_{i}\left(m\right)}\right]}{1-P_{i}\left(z\right)}\\
&=&\frac{1-I_{i}\left[P_{i}\left(z\right)\right]}{1-P_{i}\left(z\right)}
\end{eqnarray*}
por tanto

\begin{eqnarray*}
\esp\left[\sum_{t=\tau_{i}\left(m\right)}^{\tau_{i}\left(m+1\right)-1}z^{L_{i}\left(t\right)}\right]&=&\frac{1-F_{i}\left(z\right)}{1-P_{i}\left(z\right)}
\end{eqnarray*}

Haciendo uso de lo hasta ahora desarrollado se tiene que

\begin{eqnarray*}
Q_{i}\left(z\right)&=&\frac{1}{\esp\left[C_{i}\right]}\cdot\frac{1-F_{i}\left(z\right)}{P_{i}\left(z\right)-z}\cdot\frac{\left(1-z\right)P_{i}\left(z\right)}{1-P_{i}\left(z\right)}\\
&=&\frac{\mu_{i}\left(1-\mu_{i}\right)}{f_{i}\left(i\right)}\cdot\frac{1-F_{i}\left(z\right)}{P_{i}\left(z\right)-z}\cdot\frac{\left(1-z\right)P_{i}\left(z\right)}{1-P_{i}\left(z\right)}
\end{eqnarray*}

\begin{Def}
Sea $L_{i}^{*}$el n\'umero de usuarios en la cola $Q_{i}$ cuando es visitada por el servidor para dar servicio, entonces

\begin{eqnarray}
\esp\left[L_{i}^{*}\right]&=&f_{i}\left(i\right)\\
Var\left[L_{i}^{*}\right]&=&f_{i}\left(i,i\right)+\esp\left[L_{i}^{*}\right]-\esp\left[L_{i}^{*}\right]^{2}.
\end{eqnarray}

\end{Def}


\begin{Def}
El tiempo de intervisita $I_{i}$ es el periodo de tiempo que comienza cuando se ha completado el servicio en un ciclo y termina cuando es visitada nuevamente en el pr\'oximo ciclo. Su  duraci\'on del mismo est\'a dada por $\tau_{i}\left(m+1\right)-\overline{\tau}_{i}\left(m\right)$.
\end{Def}


Recordemos las siguientes expresiones:

\begin{eqnarray*}
S_{i}\left(z\right)&=&\esp\left[z^{\overline{\tau}_{i}\left(m\right)-\tau_{i}\left(m\right)}\right]=F_{i}\left(\theta\left(z\right)\right),\\
F\left(z\right)&=&\esp\left[z^{L_{0}}\right],\\
P\left(z\right)&=&\esp\left[z^{X_{n}}\right],\\
F_{i}\left(z\right)&=&\esp\left[z^{L_{i}\left(\tau_{i}\left(m\right)\right)}\right],
\theta_{i}\left(z\right)-zP_{i}
\end{eqnarray*}

entonces 

\begin{eqnarray*}
\esp\left[S_{i}\right]&=&\frac{\esp\left[L_{i}^{*}\right]}{1-\mu_{i}}=\frac{f_{i}\left(i\right)}{1-\mu_{i}},\\
Var\left[S_{i}\right]&=&\frac{Var\left[L_{i}^{*}\right]}{\left(1-\mu_{i}\right)^{2}}+\frac{\sigma^{2}\esp\left[L_{i}^{*}\right]}{\left(1-\mu_{i}\right)^{3}}
\end{eqnarray*}

donde recordemos que

\begin{eqnarray*}
Var\left[L_{i}^{*}\right]&=&f_{i}\left(i,i\right)+f_{i}\left(i\right)-f_{i}\left(i\right)^{2}.
\end{eqnarray*}

La duraci\'on del tiempo de intervisita es $\tau_{i}\left(m+1\right)-\overline{\tau}\left(m\right)$. Dado que el n\'umero de usuarios presentes en $Q_{i}$ al tiempo $t=\tau_{i}\left(m+1\right)$ es igual al n\'umero de arribos durante el intervalo de tiempo $\left[\overline{\tau}\left(m\right),\tau_{i}\left(m+1\right)\right]$ se tiene que


\begin{eqnarray*}
\esp\left[z_{i}^{L_{i}\left(\tau_{i}\left(m+1\right)\right)}\right]=\esp\left[\left\{P_{i}\left(z_{i}\right)\right\}^{\tau_{i}\left(m+1\right)-\overline{\tau}\left(m\right)}\right]
\end{eqnarray*}

entonces, si \begin{eqnarray*}I_{i}\left(z\right)&=&\esp\left[z^{\tau_{i}\left(m+1\right)-\overline{\tau}\left(m\right)}\right]\end{eqnarray*} se tienen que

\begin{eqnarray*}
F_{i}\left(z\right)=I_{i}\left[P_{i}\left(z\right)\right]
\end{eqnarray*}
para $i=1,2$, por tanto



\begin{eqnarray*}
\esp\left[L_{i}^{*}\right]&=&\mu_{i}\esp\left[I_{i}\right]\\
Var\left[L_{i}^{*}\right]&=&\mu_{i}^{2}Var\left[I_{i}\right]+\sigma^{2}\esp\left[I_{i}\right]
\end{eqnarray*}
para $i=1,2$, por tanto


\begin{eqnarray*}
\esp\left[I_{i}\right]&=&\frac{f_{i}\left(i\right)}{\mu_{i}},
\end{eqnarray*}
adem\'as

\begin{eqnarray*}
Var\left[I_{i}\right]&=&\frac{Var\left[L_{i}^{*}\right]}{\mu_{i}^{2}}-\frac{\sigma_{i}^{2}}{\mu_{i}^{2}}f_{i}\left(i\right).
\end{eqnarray*}


Si  $C_{i}\left(z\right)=\esp\left[z^{\overline{\tau}\left(m+1\right)-\overline{\tau}_{i}\left(m\right)}\right]$el tiempo de duraci\'on del ciclo, entonces, por lo hasta ahora establecido, se tiene que

\begin{eqnarray*}
C_{i}\left(z\right)=I_{i}\left[\theta_{i}\left(z\right)\right],
\end{eqnarray*}
entonces

\begin{eqnarray*}
\esp\left[C_{i}\right]&=&\esp\left[I_{i}\right]\esp\left[\theta_{i}\left(z\right)\right]=\frac{\esp\left[L_{i}^{*}\right]}{\mu_{i}}\frac{1}{1-\mu_{i}}=\frac{f_{i}\left(i\right)}{\mu_{i}\left(1-\mu_{i}\right)}\\
Var\left[C_{i}\right]&=&\frac{Var\left[L_{i}^{*}\right]}{\mu_{i}^{2}\left(1-\mu_{i}\right)^{2}}.
\end{eqnarray*}

Por tanto se tienen las siguientes igualdades


\begin{eqnarray*}
\esp\left[S_{i}\right]&=&\mu_{i}\esp\left[C_{i}\right],\\
\esp\left[I_{i}\right]&=&\left(1-\mu_{i}\right)\esp\left[C_{i}\right]\\
\end{eqnarray*}

derivando con respecto a $z$



\begin{eqnarray*}
\frac{d Q_{i}\left(z\right)}{d z}&=&\frac{\left(1-F_{i}\left(z\right)\right)P_{i}\left(z\right)}{\esp\left[C_{i}\right]\left(1-P_{i}\left(z\right)\right)\left(P_{i}\left(z\right)-z\right)}\\
&-&\frac{\left(1-z\right)P_{i}\left(z\right)F_{i}^{'}\left(z\right)}{\esp\left[C_{i}\right]\left(1-P_{i}\left(z\right)\right)\left(P_{i}\left(z\right)-z\right)}\\
&-&\frac{\left(1-z\right)\left(1-F_{i}\left(z\right)\right)P_{i}\left(z\right)\left(P_{i}^{'}\left(z\right)-1\right)}{\esp\left[C_{i}\right]\left(1-P_{i}\left(z\right)\right)\left(P_{i}\left(z\right)-z\right)^{2}}\\
&+&\frac{\left(1-z\right)\left(1-F_{i}\left(z\right)\right)P_{i}^{'}\left(z\right)}{\esp\left[C_{i}\right]\left(1-P_{i}\left(z\right)\right)\left(P_{i}\left(z\right)-z\right)}\\
&+&\frac{\left(1-z\right)\left(1-F_{i}\left(z\right)\right)P_{i}\left(z\right)P_{i}^{'}\left(z\right)}{\esp\left[C_{i}\right]\left(1-P_{i}\left(z\right)\right)^{2}\left(P_{i}\left(z\right)-z\right)}
\end{eqnarray*}

Calculando el l\'imite cuando $z\rightarrow1^{+}$:
\begin{eqnarray}
Q_{i}^{(1)}\left(z\right)=\lim_{z\rightarrow1^{+}}\frac{d Q_{i}\left(z\right)}{dz}&=&\lim_{z\rightarrow1}\frac{\left(1-F_{i}\left(z\right)\right)P_{i}\left(z\right)}{\esp\left[C_{i}\right]\left(1-P_{i}\left(z\right)\right)\left(P_{i}\left(z\right)-z\right)}\\
&-&\lim_{z\rightarrow1^{+}}\frac{\left(1-z\right)P_{i}\left(z\right)F_{i}^{'}\left(z\right)}{\esp\left[C_{i}\right]\left(1-P_{i}\left(z\right)\right)\left(P_{i}\left(z\right)-z\right)}\\
&-&\lim_{z\rightarrow1^{+}}\frac{\left(1-z\right)\left(1-F_{i}\left(z\right)\right)P_{i}\left(z\right)\left(P_{i}^{'}\left(z\right)-1\right)}{\esp\left[C_{i}\right]\left(1-P_{i}\left(z\right)\right)\left(P_{i}\left(z\right)-z\right)^{2}}\\
&+&\lim_{z\rightarrow1^{+}}\frac{\left(1-z\right)\left(1-F_{i}\left(z\right)\right)P_{i}^{'}\left(z\right)}{\esp\left[C_{i}\right]\left(1-P_{i}\left(z\right)\right)\left(P_{i}\left(z\right)-z\right)}\\
&+&\lim_{z\rightarrow1^{+}}\frac{\left(1-z\right)\left(1-F_{i}\left(z\right)\right)P_{i}\left(z\right)P_{i}^{'}\left(z\right)}{\esp\left[C_{i}\right]\left(1-P_{i}\left(z\right)\right)^{2}\left(P_{i}\left(z\right)-z\right)}
\end{eqnarray}

Entonces:
%______________________________________________________

\begin{eqnarray*}
\lim_{z\rightarrow1^{+}}\frac{\left(1-F_{i}\left(z\right)\right)P_{i}\left(z\right)}{\left(1-P_{i}\left(z\right)\right)\left(P_{i}\left(z\right)-z\right)}&=&\lim_{z\rightarrow1^{+}}\frac{\frac{d}{dz}\left[\left(1-F_{i}\left(z\right)\right)P_{i}\left(z\right)\right]}{\frac{d}{dz}\left[\left(1-P_{i}\left(z\right)\right)\left(-z+P_{i}\left(z\right)\right)\right]}\\
&=&\lim_{z\rightarrow1^{+}}\frac{-P_{i}\left(z\right)F_{i}^{'}\left(z\right)+\left(1-F_{i}\left(z\right)\right)P_{i}^{'}\left(z\right)}{\left(1-P_{i}\left(z\right)\right)\left(-1+P_{i}^{'}\left(z\right)\right)-\left(-z+P_{i}\left(z\right)\right)P_{i}^{'}\left(z\right)}
\end{eqnarray*}


%______________________________________________________


\begin{eqnarray*}
\lim_{z\rightarrow1^{+}}\frac{\left(1-z\right)P_{i}\left(z\right)F_{i}^{'}\left(z\right)}{\left(1-P_{i}\left(z\right)\right)\left(P_{i}\left(z\right)-z\right)}&=&\lim_{z\rightarrow1^{+}}\frac{\frac{d}{dz}\left[\left(1-z\right)P_{i}\left(z\right)F_{i}^{'}\left(z\right)\right]}{\frac{d}{dz}\left[\left(1-P_{i}\left(z\right)\right)\left(P_{i}\left(z\right)-z\right)\right]}\\
&=&\lim_{z\rightarrow1^{+}}\frac{-P_{i}\left(z\right) F_{i}^{'}\left(z\right)+(1-z) F_{i}^{'}\left(z\right) P_{i}^{'}\left(z\right)+(1-z) P_{i}\left(z\right)F_{i}^{''}\left(z\right)}{\left(1-P_{i}\left(z\right)\right)\left(-1+P_{i}^{'}\left(z\right)\right)-\left(-z+P_{i}\left(z\right)\right)P_{i}^{'}\left(z\right)}
\end{eqnarray*}


%______________________________________________________

\begin{eqnarray*}
&&\lim_{z\rightarrow1^{+}}\frac{\left(1-z\right)\left(1-F_{i}\left(z\right)\right)P_{i}\left(z\right)\left(P_{i}^{'}\left(z\right)-1\right)}{\left(1-P_{i}\left(z\right)\right)\left(P_{i}\left(z\right)-z\right)^{2}}=\lim_{z\rightarrow1^{+}}\frac{\frac{d}{dz}\left[\left(1-z\right)\left(1-F_{i}\left(z\right)\right)P_{i}\left(z\right)\left(P_{i}^{'}\left(z\right)-1\right)\right]}{\frac{d}{dz}\left[\left(1-P_{i}\left(z\right)\right)\left(P_{i}\left(z\right)-z\right)^{2}\right]}\\
&=&\lim_{z\rightarrow1^{+}}\frac{-\left(1-F_{i}\left(z\right)\right) P_{i}\left(z\right)\left(-1+P_{i}^{'}\left(z\right)\right)-(1-z) P_{i}\left(z\right)F_{i}^{'}\left(z\right)\left(-1+P_{i}^{'}\left(z\right)\right)}{2\left(1-P_{i}\left(z\right)\right)\left(-z+P_{i}\left(z\right)\right) \left(-1+P_{i}^{'}\left(z\right)\right)-\left(-z+P_{i}\left(z\right)\right)^2 P_{i}^{'}\left(z\right)}\\
&+&\lim_{z\rightarrow1^{+}}\frac{+(1-z) \left(1-F_{i}\left(z\right)\right) \left(-1+P_{i}^{'}\left(z\right)\right) P_{i}^{'}\left(z\right)}{{2\left(1-P_{i}\left(z\right)\right)\left(-z+P_{i}\left(z\right)\right) \left(-1+P_{i}^{'}\left(z\right)\right)-\left(-z+P_{i}\left(z\right)\right)^2 P_{i}^{'}\left(z\right)}}\\
&+&\lim_{z\rightarrow1^{+}}\frac{+(1-z) \left(1-F_{i}\left(z\right)\right) P_{i}\left(z\right)P_{i}^{''}\left(z\right)}{{2\left(1-P_{i}\left(z\right)\right)\left(-z+P_{i}\left(z\right)\right) \left(-1+P_{i}^{'}\left(z\right)\right)-\left(-z+P_{i}\left(z\right)\right)^2 P_{i}^{'}\left(z\right)}}
\end{eqnarray*}











%______________________________________________________
\begin{eqnarray*}
&&\lim_{z\rightarrow1^{+}}\frac{\left(1-z\right)\left(1-F_{i}\left(z\right)\right)P_{i}^{'}\left(z\right)}{\left(1-P_{i}\left(z\right)\right)\left(P_{i}\left(z\right)-z\right)}=\lim_{z\rightarrow1^{+}}\frac{\frac{d}{dz}\left[\left(1-z\right)\left(1-F_{i}\left(z\right)\right)P_{i}^{'}\left(z\right)\right]}{\frac{d}{dz}\left[\left(1-P_{i}\left(z\right)\right)\left(P_{i}\left(z\right)-z\right)\right]}\\
&=&\lim_{z\rightarrow1^{+}}\frac{-\left(1-F_{i}\left(z\right)\right) P_{i}^{'}\left(z\right)-(1-z) F_{i}^{'}\left(z\right) P_{i}^{'}\left(z\right)+(1-z) \left(1-F_{i}\left(z\right)\right) P_{i}^{''}\left(z\right)}{\left(1-P_{i}\left(z\right)\right) \left(-1+P_{i}^{'}\left(z\right)\right)-\left(-z+P_{i}\left(z\right)\right) P_{i}^{'}\left(z\right)}\frac{}{}
\end{eqnarray*}

%______________________________________________________
\begin{eqnarray*}
&&\lim_{z\rightarrow1^{+}}\frac{\left(1-z\right)\left(1-F_{i}\left(z\right)\right)P_{i}\left(z\right)P_{i}^{'}\left(z\right)}{\left(1-P_{i}\left(z\right)\right)^{2}\left(P_{i}\left(z\right)-z\right)}=\lim_{z\rightarrow1^{+}}\frac{\frac{d}{dz}\left[\left(1-z\right)\left(1-F_{i}\left(z\right)\right)P_{i}\left(z\right)P_{i}^{'}\left(z\right)\right]}{\frac{d}{dz}\left[\left(1-P_{i}\left(z\right)\right)^{2}\left(P_{i}\left(z\right)-z\right)\right]}\\
&=&\lim_{z\rightarrow1^{+}}\frac{-\left(1-F_{i}\left(z\right)\right) P_{i}\left(z\right) P_{i}^{'}\left(z\right)-(1-z) P_{i}\left(z\right) F_{i}^{'}\left(z\right)P_i'[z]}{\left(1-P_{i}\left(z\right)\right)^2 \left(-1+P_{i}^{'}\left(z\right)\right)-2 \left(1-P_{i}\left(z\right)\right) \left(-z+P_{i}\left(z\right)\right) P_{i}^{'}\left(z\right)}\\
&+&\lim_{z\rightarrow1^{+}}\frac{(1-z) \left(1-F_{i}\left(z\right)\right) P_{i}^{'}\left(z\right)^2+(1-z) \left(1-F_{i}\left(z\right)\right) P_{i}\left(z\right) P_{i}^{''}\left(z\right)}{\left(1-P_{i}\left(z\right)\right)^2 \left(-1+P_{i}^{'}\left(z\right)\right)-2 \left(1-P_{i}\left(z\right)\right) \left(-z+P_{i}\left(z\right)\right) P_{i}^{'}\left(z\right)}\\
\end{eqnarray*}



En nuestra notaci\'on $V\left(t\right)\equiv C_{i}$ y $X_{i}=C_{i}^{(m)}$ para nuestra segunda definici\'on, mientras que para la primera la notaci\'on es: $X\left(t\right)\equiv C_{i}$ y $R_{i}\equiv C_{i}^{(m)}$.


%___________________________________________________________________________________________
%\section{Tiempos de Ciclo e Intervisita}
%___________________________________________________________________________________________


\begin{Def}
Sea $L_{i}^{*}$el n\'umero de usuarios en la cola $Q_{i}$ cuando es visitada por el servidor para dar servicio, entonces

\begin{eqnarray}
\esp\left[L_{i}^{*}\right]&=&f_{i}\left(i\right)\\
Var\left[L_{i}^{*}\right]&=&f_{i}\left(i,i\right)+\esp\left[L_{i}^{*}\right]-\esp\left[L_{i}^{*}\right]^{2}.
\end{eqnarray}

\end{Def}

\begin{Def}
El tiempo de Ciclo $C_{i}$ es e periodo de tiempo que comienza cuando la cola $i$ es visitada por primera vez en un ciclo, y termina cuando es visitado nuevamente en el pr\'oximo ciclo. La duraci\'on del mismo est\'a dada por $\tau_{i}\left(m+1\right)-\tau_{i}\left(m\right)$, o equivalentemente $\overline{\tau}_{i}\left(m+1\right)-\overline{\tau}_{i}\left(m\right)$ bajo condiciones de estabilidad.
\end{Def}

\begin{Def}
El tiempo de intervisita $I_{i}$ es el periodo de tiempo que comienza cuando se ha completado el servicio en un ciclo y termina cuando es visitada nuevamente en el pr\'oximo ciclo. Su  duraci\'on del mismo est\'a dada por $\tau_{i}\left(m+1\right)-\overline{\tau}_{i}\left(m\right)$.
\end{Def}


Recordemos las siguientes expresiones:

\begin{eqnarray*}
S_{i}\left(z\right)&=&\esp\left[z^{\overline{\tau}_{i}\left(m\right)-\tau_{i}\left(m\right)}\right]=F_{i}\left(\theta\left(z\right)\right),\\
F\left(z\right)&=&\esp\left[z^{L_{0}}\right],\\
P\left(z\right)&=&\esp\left[z^{X_{n}}\right],\\
F_{i}\left(z\right)&=&\esp\left[z^{L_{i}\left(\tau_{i}\left(m\right)\right)}\right],
\theta_{i}\left(z\right)-zP_{i}
\end{eqnarray*}

entonces 

\begin{eqnarray*}
\esp\left[S_{i}\right]&=&\frac{\esp\left[L_{i}^{*}\right]}{1-\mu_{i}}=\frac{f_{i}\left(i\right)}{1-\mu_{i}},\\
Var\left[S_{i}\right]&=&\frac{Var\left[L_{i}^{*}\right]}{\left(1-\mu_{i}\right)^{2}}+\frac{\sigma^{2}\esp\left[L_{i}^{*}\right]}{\left(1-\mu_{i}\right)^{3}}
\end{eqnarray*}

donde recordemos que

\begin{eqnarray*}
Var\left[L_{i}^{*}\right]&=&f_{i}\left(i,i\right)+f_{i}\left(i\right)-f_{i}\left(i\right)^{2}.
\end{eqnarray*}

La duraci\'on del tiempo de intervisita es $\tau_{i}\left(m+1\right)-\overline{\tau}\left(m\right)$. Dado que el n\'umero de usuarios presentes en $Q_{i}$ al tiempo $t=\tau_{i}\left(m+1\right)$ es igual al n\'umero de arribos durante el intervalo de tiempo $\left[\overline{\tau}\left(m\right),\tau_{i}\left(m+1\right)\right]$ se tiene que


\begin{eqnarray*}
\esp\left[z_{i}^{L_{i}\left(\tau_{i}\left(m+1\right)\right)}\right]=\esp\left[\left\{P_{i}\left(z_{i}\right)\right\}^{\tau_{i}\left(m+1\right)-\overline{\tau}\left(m\right)}\right]
\end{eqnarray*}

entonces, si \begin{eqnarray*}I_{i}\left(z\right)&=&\esp\left[z^{\tau_{i}\left(m+1\right)-\overline{\tau}\left(m\right)}\right]\end{eqnarray*} se tienen que

\begin{eqnarray*}
F_{i}\left(z\right)=I_{i}\left[P_{i}\left(z\right)\right]
\end{eqnarray*}
para $i=1,2$, por tanto



\begin{eqnarray*}
\esp\left[L_{i}^{*}\right]&=&\mu_{i}\esp\left[I_{i}\right]\\
Var\left[L_{i}^{*}\right]&=&\mu_{i}^{2}Var\left[I_{i}\right]+\sigma^{2}\esp\left[I_{i}\right]
\end{eqnarray*}
para $i=1,2$, por tanto


\begin{eqnarray*}
\esp\left[I_{i}\right]&=&\frac{f_{i}\left(i\right)}{\mu_{i}},
\end{eqnarray*}
adem\'as

\begin{eqnarray*}
Var\left[I_{i}\right]&=&\frac{Var\left[L_{i}^{*}\right]}{\mu_{i}^{2}}-\frac{\sigma_{i}^{2}}{\mu_{i}^{2}}f_{i}\left(i\right).
\end{eqnarray*}


Si  $C_{i}\left(z\right)=\esp\left[z^{\overline{\tau}\left(m+1\right)-\overline{\tau}_{i}\left(m\right)}\right]$el tiempo de duraci\'on del ciclo, entonces, por lo hasta ahora establecido, se tiene que

\begin{eqnarray*}
C_{i}\left(z\right)=I_{i}\left[\theta_{i}\left(z\right)\right],
\end{eqnarray*}
entonces

\begin{eqnarray*}
\esp\left[C_{i}\right]&=&\esp\left[I_{i}\right]\esp\left[\theta_{i}\left(z\right)\right]=\frac{\esp\left[L_{i}^{*}\right]}{\mu_{i}}\frac{1}{1-\mu_{i}}=\frac{f_{i}\left(i\right)}{\mu_{i}\left(1-\mu_{i}\right)}\\
Var\left[C_{i}\right]&=&\frac{Var\left[L_{i}^{*}\right]}{\mu_{i}^{2}\left(1-\mu_{i}\right)^{2}}.
\end{eqnarray*}

Por tanto se tienen las siguientes igualdades


\begin{eqnarray*}
\esp\left[S_{i}\right]&=&\mu_{i}\esp\left[C_{i}\right],\\
\esp\left[I_{i}\right]&=&\left(1-\mu_{i}\right)\esp\left[C_{i}\right]\\
\end{eqnarray*}

Def\'inanse los puntos de regenaraci\'on  en el proceso $\left[L_{1}\left(t\right),L_{2}\left(t\right),\ldots,L_{N}\left(t\right)\right]$. Los puntos cuando la cola $i$ es visitada y todos los $L_{j}\left(\tau_{i}\left(m\right)\right)=0$ para $i=1,2$  son puntos de regeneraci\'on. Se llama ciclo regenerativo al intervalo entre dos puntos regenerativos sucesivos.

Sea $M_{i}$  el n\'umero de ciclos de visita en un ciclo regenerativo, y sea $C_{i}^{(m)}$, para $m=1,2,\ldots,M_{i}$ la duraci\'on del $m$-\'esimo ciclo de visita en un ciclo regenerativo. Se define el ciclo del tiempo de visita promedio $\esp\left[C_{i}\right]$ como

\begin{eqnarray*}
\esp\left[C_{i}\right]&=&\frac{\esp\left[\sum_{m=1}^{M_{i}}C_{i}^{(m)}\right]}{\esp\left[M_{i}\right]}
\end{eqnarray*}


En Stid72 y Heym82 se muestra que una condici\'on suficiente para que el proceso regenerativo 
estacionario sea un procesoo estacionario es que el valor esperado del tiempo del ciclo regenerativo sea finito:

\begin{eqnarray*}
\esp\left[\sum_{m=1}^{M_{i}}C_{i}^{(m)}\right]<\infty.
\end{eqnarray*}

como cada $C_{i}^{(m)}$ contiene intervalos de r\'eplica positivos, se tiene que $\esp\left[M_{i}\right]<\infty$, adem\'as, como $M_{i}>0$, se tiene que la condici\'on anterior es equivalente a tener que 

\begin{eqnarray*}
\esp\left[C_{i}\right]<\infty,
\end{eqnarray*}
por lo tanto una condici\'on suficiente para la existencia del proceso regenerativo est\'a dada por

\begin{eqnarray*}
\sum_{k=1}^{N}\mu_{k}<1.
\end{eqnarray*}



\begin{Note}\label{Cita1.Stidham}
En Stidham\cite{Stidham} y Heyman  se muestra que una condici\'on suficiente para que el proceso regenerativo 
estacionario sea un procesoo estacionario es que el valor esperado del tiempo del ciclo regenerativo sea finito:

\begin{eqnarray*}
\esp\left[\sum_{m=1}^{M_{i}}C_{i}^{(m)}\right]<\infty.
\end{eqnarray*}

como cada $C_{i}^{(m)}$ contiene intervalos de r\'eplica positivos, se tiene que $\esp\left[M_{i}\right]<\infty$, adem\'as, como $M_{i}>0$, se tiene que la condici\'on anterior es equivalente a tener que 

\begin{eqnarray*}
\esp\left[C_{i}\right]<\infty,
\end{eqnarray*}
por lo tanto una condici\'on suficiente para la existencia del proceso regenerativo est\'a dada por

\begin{eqnarray*}
\sum_{k=1}^{N}\mu_{k}<1.
\end{eqnarray*}
\end{Note}





\begin{thebibliography}{99}

\bibitem{ISL}
James, G., Witten, D., Hastie, T., and Tibshirani, R. (2013). \textit{An Introduction to Statistical Learning: with Applications in R}. Springer.

\bibitem{Logistic}
Hosmer, D. W., Lemeshow, S., and Sturdivant, R. X. (2013). \textit{Applied Logistic Regression} (3rd ed.). Wiley.

\bibitem{PatternRecognition}
Bishop, C. M. (2006). \textit{Pattern Recognition and Machine Learning}. Springer.

\bibitem{Harrell}
Harrell, F. E. (2015). \textit{Regression Modeling Strategies: With Applications to Linear Models, Logistic and Ordinal Regression, and Survival Analysis}. Springer.

\bibitem{RDocumentation}
R Documentation and Tutorials: \url{https://cran.r-project.org/manuals.html}

\bibitem{RBlogger}
Tutorials on R-bloggers: \url{https://www.r-bloggers.com/}

\bibitem{CourseraML}
Coursera: \textit{Machine Learning} by Andrew Ng.

\bibitem{edXDS}
edX: \textit{Data Science and Machine Learning Essentials} by Microsoft.

% Libros adicionales
\bibitem{Ross}
Ross, S. M. (2014). \textit{Introduction to Probability and Statistics for Engineers and Scientists}. Academic Press.

\bibitem{DeGroot}
DeGroot, M. H., and Schervish, M. J. (2012). \textit{Probability and Statistics} (4th ed.). Pearson.

\bibitem{Hogg}
Hogg, R. V., McKean, J., and Craig, A. T. (2019). \textit{Introduction to Mathematical Statistics} (8th ed.). Pearson.

\bibitem{Kleinbaum}
Kleinbaum, D. G., and Klein, M. (2010). \textit{Logistic Regression: A Self-Learning Text} (3rd ed.). Springer.

% Artículos y tutoriales adicionales
\bibitem{Wasserman}
Wasserman, L. (2004). \textit{All of Statistics: A Concise Course in Statistical Inference}. Springer.

\bibitem{KhanAcademy}
Probability and Statistics Tutorials on Khan Academy: \url{https://www.khanacademy.org/math/statistics-probability}

\bibitem{OnlineStatBook}
Online Statistics Education: \url{http://onlinestatbook.com/}

\bibitem{Peng}
Peng, C. Y. J., Lee, K. L., and Ingersoll, G. M. (2002). \textit{An Introduction to Logistic Regression Analysis and Reporting}. The Journal of Educational Research.

\bibitem{Agresti}
Agresti, A. (2007). \textit{An Introduction to Categorical Data Analysis} (2nd ed.). Wiley.

\bibitem{Han}
Han, J., Pei, J., and Kamber, M. (2011). \textit{Data Mining: Concepts and Techniques}. Morgan Kaufmann.

\bibitem{TowardsDataScience}
Data Cleaning and Preprocessing on Towards Data Science: \url{https://towardsdatascience.com/data-cleaning-and-preprocessing}

\bibitem{Molinaro}
Molinaro, A. M., Simon, R., and Pfeiffer, R. M. (2005). \textit{Prediction error estimation: a comparison of resampling methods}. Bioinformatics.

\bibitem{EvaluatingModels}
Evaluating Machine Learning Models on Towards Data Science: \url{https://towardsdatascience.com/evaluating-machine-learning-models}

\bibitem{LogisticRegressionGuide}
Practical Guide to Logistic Regression in R on Towards Data Science: \url{https://towardsdatascience.com/practical-guide-to-logistic-regression-in-r}

% Cursos en línea adicionales
\bibitem{CourseraStatistics}
Coursera: \textit{Statistics with R} by Duke University.

\bibitem{edXProbability}
edX: \textit{Data Science: Probability} by Harvard University.

\bibitem{CourseraLogistic}
Coursera: \textit{Logistic Regression} by Stanford University.

\bibitem{edXInference}
edX: \textit{Data Science: Inference and Modeling} by Harvard University.

\bibitem{CourseraWrangling}
Coursera: \textit{Data Science: Wrangling and Cleaning} by Johns Hopkins University.

\bibitem{edXRBasics}
edX: \textit{Data Science: R Basics} by Harvard University.

\bibitem{CourseraRegression}
Coursera: \textit{Regression Models} by Johns Hopkins University.

\bibitem{edXStatInference}
edX: \textit{Data Science: Statistical Inference} by Harvard University.

\bibitem{SurvivalAnalysis}
An Introduction to Survival Analysis on Towards Data Science: \url{https://towardsdatascience.com/an-introduction-to-survival-analysis}

\bibitem{MultinomialLogistic}
Multinomial Logistic Regression on DataCamp: \url{https://www.datacamp.com/community/tutorials/multinomial-logistic-regression-R}

\bibitem{CourseraSurvival}
Coursera: \textit{Survival Analysis} by Johns Hopkins University.

\bibitem{edXHighthroughput}
edX: \textit{Data Science: Statistical Inference and Modeling for High-throughput Experiments} by Harvard University.

\end{thebibliography}


\printindex
\end{document}
