\documentclass{article}
%_-_-_-_-_-_-_-_-_-_-_-_-_-_-_-_-_-_-_-_-_-_-_-_-_-_-_-
% PAQUETES A UTILIZAR
%_-_-_-_-_-_-_-_-_-_-_-_-_-_-_-_-_-_-_-_-_-_-_-_-_-_-_-
\usepackage[utf8]{inputenc}
\usepackage[spanish]{babel}
%\usepackage[spanish,english]{babel}
\usepackage{amsmath,amssymb,amsthm,amsfonts}
\usepackage{geometry}
\usepackage{hyperref}
\usepackage{fancyhdr}
\usepackage{titlesec}
\usepackage{listings}
\usepackage{graphicx,graphics}
\usepackage{multicol}
\usepackage{multirow}
\usepackage{color}
\usepackage{float} 
\usepackage{subfig}
\usepackage[figuresright]{rotating}
\usepackage{enumerate}
\usepackage{anysize} 
\usepackage{url}
\usepackage{imakeidx}
%_-_-_-_-_-_-_-_-_-_-_-_-_-_-_-_-_-_-_-_-_-_-_-_-_-_-_-
% TITULO DEL DOCUMENTO
%_-_-_-_-_-_-_-_-_-_-_-_-_-_-_-_-_-_-_-_-_-_-_-_-_-_-_-
\title{Modelos de Flujo: Revisión\\
\small{Renewal and Regenerative Processes: Review}}
\author{Carlos E. Martínez-Rodríguez}
\date{Julio-Agosto 2024}
%_-_-_-_-_-_-_-_-_-_-_-_-_-_-_-_-_-_-_-_-_-_-_-_-_-_-_-
% MODIFICACION DE LOS MARGENES
%_-_-_-_-_-_-_-_-_-_-_-_-_-_-_-_-_-_-_-_-_-_-_-_-_-_-_-
\geometry{
  a4paper,
  left=15mm,
  right=15mm,
  left=14mm,
  right=14mm,
  top=30mm,
  bottom=30mm,
}
%_-_-_-_-_-_-_-_-_-_-_-_-_-_-_-_-_-_-_-_-_-_-_-_-_-_-_-
% CONFIGURACION DE ENCABEZADOS Y PIES DE PAG
%_-_-_-_-_-_-_-_-_-_-_-_-_-_-_-_-_-_-_-_-_-_-_-_-_-_-_-
\pagestyle{fancy}
\fancyhf{}
\fancyfoot[C]{\thepage}
\fancyfoot[R]{\tiny \leftmark}
\fancyfoot[L]{\tiny Carlos E. Martínez-Rodríguez}
%_-_-_-_-_-_-_-_-_-_-_-_-_-_-_-_-_-_-_-_-_-_-_-_-_-_-_-
% Reducción del tamaño de letra en los pies de página
%\fancypagestyle{plain}{
  %\fancyfoot[C]{\footnotesize \thepage}
  %\fancyfoot[R]{\footnotesize \leftmark}
%  \fancyfoot[R]{\footnotesize \nouppercase{\rightmark}}
  %\fancyfoot[L]{\footnotesize Carlos E. Martínez-Rodríguez}
  %\fancyhead{}
%}
%_-_-_-_-_-_-_-_-_-_-_-_-_-_-_-_-_-_-_-_-_-_-_-_-_-_-_-
% DEFINICION DE TITULOS CORTOS
%_-_-_-_-_-_-_-_-_-_-_-_-_-_-_-_-_-_-_-_-_-_-_-_-_-_-_-
% Definir títulos cortos para las secciones
%\usepackage{shorttoc} % Incluye este paquete para usar títulos cortos
%\newcommand{\shorttitle}[2]{\section[#1]{#2}}
%\newcommand{\shorttitle}[2]{\section[#1]{#2}\sectionmark{#1}}
%_-_-_-_-_-_-_-_-_-_-_-_-_-_-_-_-_-_-_-_-_-_-_-_-_-_-_-
% Definiciones de nuevos entornos
%_-_-_-_-_-_-_-_-_-_-_-_-_-_-_-_-_-_-_-_-_-_-_-_-_-_-_-
\newtheorem{Def}{Definición}[section]
\newtheorem{Ejem}{Ejemplo}[section]
\newtheorem{Teo}{Teorema}[section]
\newtheorem{Dem}{Demostraci\'on}[section]
\newtheorem{Note}{Nota}[section]
\newtheorem{Prop}{Proposición}[section]
\newtheorem{Cor}{Corolario}[section]
\newtheorem{Lema}{Lema}[section]
\newtheorem{Lemma}{Lema}[section]
\newtheorem{Sup}{Supuestos}[section]
\newtheorem{Obs}{Observación}[section]
%_-_-_-_-_-_-_-_-_-_-_-_-_-_-_-_-_-_-_-_-_-_-_-_-_-_-_-
%NUEVOS COMANDOS
%_-_-_-_-_-_-_-_-_-_-_-_-_-_-_-_-_-_-_-_-_-_-_-_-_-_-_-
%\def\RR{\mathbb{R}}
%\def\ZZ{\mathbb{Z}}
\newcommand{\nat}{\mathbb{N}}
\newcommand{\ent}{\mathbb{Z}}
\newcommand{\rea}{\mathbb{R}}
\newcommand{\Eb}{\mathbf{E}}
\newcommand{\esp}{\mathbb{E}}
\newcommand{\prob}{\mathbb{P}}
\newcommand{\indora}{\mbox{$1$\hspace{-0.8ex}$1$}}
\newcommand{\ER}{\left(E,\mathcal{E}\right)}
\newcommand{\KM}{\left(P_{s,t}\right)}
%\newcommand{\Xt}{\left(X_{t}\right)_{t\in I}}
\newcommand{\PE}{\left(X_{t}\right)_{t\in I}}
\newcommand{\SG}{\left(P_{t}\right)}
\newcommand{\CM}{\mathbf{P}^{x}}
%\newcommand\mypar{\par\vspace{\baselineskip}}
\renewcommand{\abstractname}{Resumen}
\numberwithin{equation}{section}
%_-_-_-_-_-_-_-_-_-_-_-_-_-_-_-_-_-_-_-_-_-_-_-_-_-_-_-
\makeindex
%_-_-_-_-_-_-_-_-_-_-_-_-_-_-_-_-_-_-_-_-_-_-_-_-_-_-_-
\begin{document}
%_-_-_-_-_-_-_-_-_-_-_-_-_-_-_-_-_-_-_-_-_-_-_-_-_-_-_-
\maketitle

%<<>><<>><<>><<>><<>><<>><<>><<>><<>><<>><<>><<>>
\begin{abstract}
%<<>><<>><<>><<>><<>><<>><<>><<>><<>><<>><<>><<>>
%_-_-_-_-_-_-_-_-_-_-_-_-_-_-_-_-_-_-_-_-_-_-_-_-_-_-_-
\end{abstract}

\begin{otherlanguage}{english}
\renewcommand{\abstractname}{Abstract} % Cambia "Resumen" a "Abstract"
\begin{abstract}
\end{abstract}
\end{otherlanguage}
%<<>><<>><<>><<>><<>><<>><<>><<>><<>><<>><<>><<>>
\tableofcontents
%\newpage
%<====>====<><====>====<><====>====<><====>====<><====>
%\part{Introducci\'on a Procesos Regenerativos}
%<====>====<><====>====<><====>====<><====>====<><====>
%_-_-_-_-_-_-_-_-_-_-_-_-_-_-_-_-_-_-_-_-_-_-_-_-_-_-_-
\section*{Introducción}
%_-_-_-_-_-_-_-_-_-_-_-_-_-_-_-_-_-_-_-_-_-_-_-_-_-_-_-
\begin{otherlanguage}{english}
%\renewcommand{\abstractname}{Abstract} % Cambia "Resumen" a "Abstract"
\section*{Introduction}
\end{otherlanguage}
%<>===<>==<>===<>==<>===<>==<>===<>==<>===<>==<>===<>==<>
%_____________________________________________________________________
\section{Definiciones  B\'asicas}
%_____________________________________________________________________
\begin{Def}\index{Espacio Medible}
Sea $X$ un conjunto y $\mathcal{F}$ una $\sigma$-\'algebra de subconjuntos de $X$, la pareja $\left(X,\mathcal{F}\right)$ es llamado espacio medible. Un subconjunto $A$ de $X$ es llamado medible, o medible con respecto a $\mathcal{F}$, si $A\in\mathcal{F}$.
\end{Def}

\begin{Def}\index{Medida $\sigma$-finita}
Sea $\left(X,\mathcal{F},\mu\right)$ espacio de medida. Se dice que la medida $\mu$ es $\sigma$-finita si se puede escribir $X=\bigcup_{n\geq1}X_{n}$ con $X_{n}\in\mathcal{F}$ y $\mu\left(X_{n}\right)<\infty$.
\end{Def}

\begin{Def}\label{Cto.Borel}\index{Conjuto de Borel}
El \'algebra de Borel es la $\sigma$-\'algebra $B$ generada por los intervalos abiertos $\left(a,b\right)\in\rea$. Cualquier conjunto en $B$ es llamado {\em Conjunto de Borel}.
\end{Def}

\begin{Def}\label{Funcion.Medible}\index{Funci\'on Medible}
Una funci\'on $f:X\rightarrow\rea$, es medible si para cualquier n\'umero real $\alpha$ el conjunto \[\left\{x\in X:f\left(x\right)>\alpha\right\}\] pertenece a $X$. Equivalentemente, se dice que $f$ es medible si \[f^{-1}\left(\left(\alpha,\infty\right)\right)=\left\{x\in X:f\left(x\right)>\alpha\right\}\in\mathcal{F}.\]
\end{Def}

\begin{Def}\label{Def.Cilindros}\index{Cilindros}
Sean $\left(\Omega_{i},\mathcal{F}_{i}\right)$, $i=1,2,\ldots,$ espacios medibles y $\Omega=\prod_{i=1}^{\infty}\Omega_{i}$ el conjunto de todas las sucesiones $\left(\omega_{1},\omega_{2},\ldots,\right)$ tales que $\omega_{i}\in\Omega_{i}$, $i=1,2,\ldots,$. Si $B^{n}\subset\prod_{i=1}^{\infty}\Omega_{i}$, definimos $B_{n}=\left\{\omega\in\Omega:\left(\omega_{1},\omega_{2},\ldots,\omega_{n}\right)\in B^{n}\right\}$. Al conjunto $B_{n}$ se le llama {\em cilindro} con base $B^{n}$, el cilindro es llamado medible si $B^{n}\in\prod_{i=1}^{\infty}\mathcal{F}_{i}$.
\end{Def}


\begin{Def}\label{Def.Proc.Adaptado}\index{Proceso Adaptado}
Sea $X\left(t\right),t\geq0$ proceso estoc\'astico, el proceso es adaptado a la familia de $\sigma$-\'algebras $\mathcal{F}_{t}$, para $t\geq0$, si para $s<t$ implica que $\mathcal{F}_{s}\subset\mathcal{F}_{t}$, y $X\left(t\right)$ es $\mathcal{F}_{t}$-medible para cada $t$. Si no se especifica $\mathcal{F}_{t}$ entonces se toma $\mathcal{F}_{t}$ como $\mathcal{F}\left(X\left(s\right),s\leq t\right)$, la m\'as peque\~na $\sigma$-\'algebra de subconjuntos de $\Omega$ que hace que cada $X\left(s\right)$, con $s\leq t$ sea Borel medible.
\end{Def}


\begin{Def}\label{Def.Tiempo.Paro}\index{Tiempo de Paro}
Sea $\left\{\mathcal{F}\left(t\right),t\geq0\right\}$ familia creciente de sub $\sigma$-\'algebras. es decir, $\mathcal{F}\left(s\right)\subset\mathcal{F}\left(t\right)$ para $s\leq t$. Un tiempo de paro para $\mathcal{F}\left(t\right)$ es una funci\'on $T:\Omega\rightarrow\left[0,\infty\right]$ tal que $\left\{T\leq t\right\}\in\mathcal{F}\left(t\right)$ para cada $t\geq0$. Un tiempo de paro para el proceso estoc\'astico $X\left(t\right),t\geq0$ es un tiempo de paro para las $\sigma$-\'algebras $\mathcal{F}\left(t\right)=\mathcal{F}\left(X\left(s\right)\right)$.
\end{Def}

\begin{Def}\index{Proceso Adaptado}
Sea $X\left(t\right),t\geq0$ proceso estoc\'astico, con $\left(S,\chi\right)$ espacio de estados. Se dice que el proceso es adaptado a $\left\{\mathcal{F}\left(t\right)\right\}$, es decir, si para cualquier $s,t\in I$, $I$ conjunto de \'indices, $s<t$, se tiene que $\mathcal{F}\left(s\right)\subset\mathcal{F}\left(t\right)$ y $X\left(t\right)$ es $\mathcal{F}\left(t\right)$-medible,
\end{Def}

\begin{Def}\index{Proceso de Markov}
Sea $X\left(t\right),t\geq0$ proceso estoc\'astico, se dice que es un Proceso de Markov relativo a $\mathcal{F}\left(t\right)$ o que $\left\{X\left(t\right),\mathcal{F}\left(t\right)\right\}$ es de Markov si y s\'olo si para cualquier conjunto $B\in\chi$,  y $s,t\in I$, $s<t$ se cumple que
\begin{equation}\label{Propiedad.Markov}
P\left\{X\left(t\right)\in B|\mathcal{F}\left(s\right)\right\}=P\left\{X\left(t\right)\in B|X\left(s\right)\right\}.
\end{equation}
\end{Def}
\begin{Note}
Si se dice que $\left\{X\left(t\right)\right\}$ es un Proceso de Markov sin mencionar $\mathcal{F}\left(t\right)$, se asumir\'a que
\begin{eqnarray}
\mathcal{F}\left(t\right)=\mathcal{F}_{0}\left(t\right)=\mathcal{F}\left(X\left(r\right),r\leq t\right),
\end{eqnarray}
entonces la ecuaci\'on (\ref{Propiedad.Markov}) se puede escribir como
\begin{equation}\label{Propiedad.Markov.Equivalente}\index{Propiedad de Markov}
P\left\{X\left(t\right)\in B|X\left(r\right),r\leq s\right\} = P\left\{X\left(t\right)\in B|X\left(s\right)\right\}.
\end{equation}
\end{Note}

\begin{Teo}\index{Proceso de Markov}
Sea $\left(X_{n},\mathcal{F}_{n},n=0,1,\ldots,\right\}$ Proceso de Markov con espacio de estados $\left(S_{0},\chi_{0}\right)$ generado por una distribuici\'on inicial $P_{o}$ y probabilidad de transici\'on $p_{mn}$, para $m,n=0,1,\ldots,$ $m<n$, que por notaci\'on se escribir\'a como $p\left(m,n,x,B\right)\rightarrow p_{mn}\left(x,B\right)$. Sea $S$ tiempo de paro relativo a la $\sigma$-\'algebra $\mathcal{F}_{n}$. Sea $T$ funci\'on medible, $T:\Omega\rightarrow\left\{0,1,\ldots,\right\}$. Sup\'ongase que $T\geq S$, entonces $T$ es tiempo de paro. Si $B\in\chi_{0}$, entonces
\begin{equation}\label{Prop.Fuerte.Markov}
P\left\{X\left(T\right)\in B,T<\infty|\mathcal{F}\left(S\right)\right\} = p\left(S,T,X\left(s\right),B\right)
\end{equation}
en $\left\{T<\infty\right\}$.
\end{Teo}

%_______________________________________________________________________
\section{Modelo de Flujo}
%_______________________________________________________________________




Sup\'ongase que el sistema consta de varias colas a los cuales llegan uno o varios servidores a dar servicio a los usuarios esperando en la cola. Sea $x$ el n\'umero de usuarios en la cola esperando por servicio y $N\left(x\right)$ es el n\'umero de usuarios que son atendidos con una pol\'itica dada y fija mientras el servidor permanece dando servicio, entonces se asume que:
\begin{itemize}
\item[(S1.)]
\begin{equation}\label{S1}
lim_{x\rightarrow\infty}\esp\left[N\left(x\right)\right]=\overline{N}>0.
\end{equation}
\item[(S2.)]
\begin{equation}\label{S2}
\esp\left[N\left(x\right)\right]\leq \overline{N},
\end{equation}
para cualquier valor de $x$.
\end{itemize}

El tiempo que tarda un servidor en volver a dar servicio despu\'es de abandonar la cola inmediata anterior y llegar a la pr\'oxima se llama tiempo de traslado o de cambio  de cola, al momento de la $n$-\'esima visita del servidor a la cola $j$ se genera una sucesi\'on de variables aleatorias $\delta_{j,j+1}\left(n\right)$, independientes e id\'enticamente distribuidas, con la propiedad de que $\esp\left[\delta_{j,j+1}\left(1\right)\right]\geq0$.\\


Se define
\begin{equation}
\delta^{*}:=\sum_{j,j+1}\esp\left[\delta_{j,j+1}\left(1\right)\right].
\end{equation}

Los tiempos entre arribos a la cola $k$, son de la forma $\left\{\xi_{k}\left(n\right)\right\}_{n\geq1}$, con la propiedad de que son independientes e id\'enticamente distribuidos. Los tiempos de servicio $\left\{\eta_{k}\left(n\right)\right\}_{n\geq1}$ tienen la propiedad de ser independientes e id\'enticamente distribuidos. Para la $k$-\'esima cola se define la tasa de arribo a la como $\lambda_{k}=1/\esp\left[\xi_{k}\left(1\right)\right]$ y la tasa de servicio como $\mu_{k}=1/\esp\left[\eta_{k}\left(1\right)\right]$, finalmente se define la carga de la cola como $\rho_{k}=\lambda_{k}/\mu_{k}$, donde se pide que $\rho<1$, para garantizar la estabilidad del sistema.\\

Para el caso m\'as sencillo podemos definir un proceso de estados para la red que depende de la pol\'itica de servicio utilizada, el estado $\mathbb{X}\left(t\right)$ a cualquier tiempo $t$ puede definirse como
\begin{equation}\label{Eq.Esp.Estados}
\mathbb{X}\left(t\right)=\left(Q_{k}\left(t\right),A_{l}\left(t\right),B_{k}\left(t\right):k=1,2,\ldots,K,l\in\mathcal{A}\right),
\end{equation}

donde $Q_{k}\left(t\right)$ es la longitud de la cola $k$ para los usuarios esperando servicio, incluyendo aquellos que est\'an siendo atendidos, $B_{k}\left(t\right)$ son los tiempos de servicio residuales para los usuarios de la clase $k$ que est\'an en servicio.\\

Los tiempos entre arribos residuales, que son el tiempo que queda hasta que el pr\'oximo usuario llega a la cola para recibir servicio, se denotan por $A_{k}\left(t\right)$. Tanto $B_{k}\left(t\right)$ como $A_{k}\left(t\right)$ se suponen continuos por la derecha \cite{Dai2}.\\

Sea $\mathcal{X}$ el espacio de estados para el proceso de estados que por definici\'on es igual  al conjunto de posibles valores para el estado $\mathbb{X}\left(t\right)$, y sea $x=\left(q,a,b\right)$ un estado gen\'erico en $\mathbb{X}$, la componente $q$ determina la posici\'on del usuario en la red, $|q|$ denota la longitud total de la cola en la red.\\

Para un estado $x=\left(q,a,b\right)\in\mathbb{X}$ definimos la {\em norma} de $x$ como $\left\|x\right\|=|q|+|a|+|b|$. En \cite{Dai} se muestra que para una amplia serie de disciplinas de servicio el proceso $\mathbb{X}$ es un Proceso Fuerte de Markov, y por tanto se puede asumir que \[\left(\left(\Omega,\mathcal{F}\right),\mathcal{F}_{t},\mathbb{X}\left(t\right),\theta_{t},P_{x}\right)\] es un proceso de {\em Borel Derecho} en el espacio de estados medible $\left(\mathcal{X},\mathcal{B}_{\mathcal{X}}\right)$.\\

Sea $P^{t}\left(x,D\right)$, $D\in\mathcal{B}_{\mathbb{X}}$, $t\geq0$ probabilidad de transici\'on de $X$ definida como \[P^{t}\left(x,D\right)=P_{x}\left(\mathbb{X}\left(t\right)\in D\right).\]

\begin{Def}
Una medida no cero $\pi$ en $\left(\mathbb{X},\mathcal{B}_{\mathbb{X}}\right)$ es {\em invariante} para $X$ si $\pi$ es $\sigma$-finita y
\[\pi\left(D\right)=\int_{X}P^{t}\left(x,D\right)\pi\left(dx\right),\] para todo $D\in \mathcal{B}_{\mathbb{X}}$, con $t\geq0$.
\end{Def}

\begin{Def}
El proceso de Markov $X$ es llamado {\em Harris recurrente} si existe una medida de probabilidad $\nu$ en $\left(\mathbb{X},\mathcal{B}_{\mathbb{X}}\right)$, tal que si $\nu\left(D\right)>0$ y $D\in\mathcal{B}_{\mathbb{X}}$ \[P_{x}\left\{\tau_{D}<\infty\right\}\equiv1,\] cuando $\tau_{D}=inf\left\{t\geq0:\mathbb{X}_{t}\in D\right\}$.
\end{Def}

\begin{Def}
Un conjunto $D\in\mathcal{B}_\mathbb{X}$ es llamado peque\~no si existe un $t>0$, una medida de probabilidad $\nu$ en $\mathcal{B}_\mathbb{X}$, y un $\delta>0$ tal que \[P^{t}\left(x,A\right)\geq\delta\nu\left(A\right),\] para $x\in D,A\in\mathcal{B}_\mathbb{X}$.
\end{Def}
\begin{Note}
\begin{itemize}

\item[i)] Si $X$ es Harris recurrente, entonces existe una \'unica medida invariante $\pi$ (\cite{Getoor}).

\item[ii)] Si la medida invariante es finita, entonces puede normalizarse a una medida de probabilidad, en este caso a la medida se le llama \textbf{Harris recurrente positiva}.

\item[iii)] Cuando $X$ es Harris recurrente positivo se dice que la disciplina de servicio es estable. En este caso $\pi$ denota la ditribuci\'on estacionaria; se define \[P_{\pi}\left(\cdot\right)=\int_{X}P_{x}\left(\cdot\right)\pi\left(dx\right).\] Se utiliza $E_{\pi}$ para denotar el operador esperanza
correspondiente, as\'i, el proceso $X=\left\{\mathbb{X}\left(t\right),t\geq0\right\}$ es un proceso estrictamente estacionario bajo $P_{\pi}$.

\item[iv)] En \cite{MeynTweedie} se muestra que si $P_{x}\left\{\tau_{D}<\infty\right\}=1$ incluso para solamente un conjunto peque\~no, entonces el proceso de Harris es recurrente.
\end{itemize}
\end{Note}


Las Colas C\'iclicas se pueden describir por medio de un proceso de Markov $\left(X\left(t\right)\right)_{t\in\rea}$, donde el estado del sistema al tiempo $t\geq0$ est\'a dado por
\begin{equation}
X\left(t\right)=\left(Q\left(t\right),A\left(t\right),H\left(t\right),B\left(t\right),B^{0}\left(t\right),C\left(t\right)\right)
\end{equation}
definido en el espacio producto:
\begin{equation}
\mathcal{X}=\mathbb{Z}^{K}\times\rea_{+}^{K}\times\left(\left\{1,2,\ldots,K\right\}\times\left\{1,2,\ldots,S\right\}\right)^{M}\times\rea_{+}^{K}\times\rea_{+}^{K}\times\mathbb{Z}^{K},
\end{equation}

\begin{itemize}
\item $Q\left(t\right)=\left(Q_{k}\left(t\right),1\leq k\leq K\right)$, es el n\'umero de usuarios en la cola $k$, incluyendo aquellos que est\'an siendo atendidos provenientes de la $k$-\'esima cola.

\item $A\left(t\right)=\left(A_{k}\left(t\right),1\leq k\leq K\right)$, son los residuales de los tiempos de arribo en la cola $k$. \item $H\left(t\right)$ es el par ordenado que consiste en la cola que esta siendo atendida y la pol\'itica de servicio que se utilizar\'a.

\item $B\left(t\right)$ es el tiempo de servicio residual.

\item $B^{0}\left(t\right)$ es el tiempo residual del cambio de cola.

\item $C\left(t\right)$ indica el n\'umero de usuarios atendidos durante la visita del servidor a la cola dada en $H\left(t\right)$.
\end{itemize}

$A_{k}\left(t\right),B_{m}\left(t\right)$ y $B_{m}^{0}\left(t\right)$ se suponen continuas por la derecha y que satisfacen la propiedad fuerte de Markov, (\cite{Dai}).

Dada una condici\'on inicial $x\in\mathcal{X}$, $Q_{k}^{x}\left(t\right)$ es la longitud de la cola $k$ al tiempo $t$ y $T_{m,k}^{x}\left(t\right)$  el tiempo acumulado al tiempo $t$ que el servidor tarda en atender a los usuarios de la cola $k$. De igual manera se define $T_{m,k}^{x,0}\left(t\right)$ el tiempo acumulado al tiempo $t$ que el servidor tarda en cambiar de cola para volver a atender a los usuarios.

Para reducir la fluctuaci\'on del modelo se escala tanto el espacio como el tiempo, entonces se tiene el proceso:

\begin{eqnarray}
\overline{Q}^{x}\left(t\right)=\frac{1}{|x|}Q^{x}\left(|x|t\right),\\
\overline{T}_{m}^{x}\left(t\right)=\frac{1}{|x|}T_{m}^{x}\left(|x|t\right),\\
\overline{T}_{m}^{x,0}\left(t\right)=\frac{1}{|x|}T_{m}^{x,0}\left(|x|t\right).
\end{eqnarray}
Cualquier l\'imite $\overline{Q}\left(t\right)$ es llamado un flujo l\'imite del proceso longitud de la cola, al conjunto de todos los posibles flujos l\'imite se le llamar\'a \textbf{modelo de flujo}, (\cite{MaynDown}).

\begin{Def}
Un flujo l\'imite para un sistema de visitas bajo una disciplina de servicio espec\'ifica se define como cualquier soluci\'on
 $\left(\overline{Q}\left(\cdot\right),\overline{T}_{m}\left(\cdot\right),\overline{T}_{m}^{0}\left(\cdot\right)\right)$  de las siguientes ecuaciones, donde $\overline{Q}\left(t\right)=\left(\overline{Q}_{1}\left(t\right),\ldots,\overline{Q}_{K}\left(t\right)\right)$ y $\overline{T}\left(t\right)=\left(\overline{T}_{1}\left(t\right),\ldots,\overline{T}_{K}\left(t\right)\right)$
\begin{equation}\label{Eq.3.8}
\overline{Q}_{k}\left(t\right)=\overline{Q}_{k}\left(0\right)+\lambda_{k}t-\sum_{m=1}^{M}\mu_{k}\overline{T}_{m,k}\left(t\right)\\
\end{equation}
\begin{equation}\label{Eq.3.9}
\overline{Q}_{k}\left(t\right)\geq0\textrm{ para }k=1,2,\ldots,K,\\
\end{equation}
\begin{equation}\label{Eq.3.10}
\overline{T}_{m,k}\left(0\right)=0,\textrm{ y }\overline{T}_{m,k}\left(\cdot\right)\textrm{ es no decreciente},\\
\end{equation}
\begin{equation}\label{Eq.3.11}
\sum_{k=1}^{K}\overline{T}_{m,k}^{0}\left(t\right)+\overline{T}_{m,k}\left(t\right)=t\textrm{ para}m=1,2,\ldots,M\\
\end{equation}
\end{Def}

Al conjunto de ecuaciones dadas en (\ref{Eq.3.8})-(\ref{Eq.3.11}) se le llama {\em Modelo de flujo} y al conjunto de todas las
soluciones del modelo de flujo $\left(\overline{Q}\left(\cdot\right),\overline{T} \left(\cdot\right)\right)$ se le denotar\'a por $\mathcal{Q}$.


\begin{Def}
El modelo de flujo es estable si existe un tiempo fijo $t_{0}$ tal que $\overline{Q}\left(t\right)=0$, con $t\geq t_{0}$, para cualquier $\overline{Q}\left(\cdot\right)\in\mathcal{Q}$ que cumple con $|\overline{Q}\left(0\right)|=1$.
\end{Def}

Dada una condici\'on inicial $x\in\mathbb{X}$, sea

\begin{itemize}
\item $Q_{k}^{x}\left(t\right)$ la longitud de la cola al tiempo
$t$,

\item $T_{m,k}^{x}\left(t\right)$ el tiempo acumulado, al tiempo $t$, que tarda el servidor $m$ en atender a los usuarios de la cola $k$.

\item $T_{m,k}^{x,0}\left(t\right)$ el tiempo acumulado, al tiempo $t$, que tarda el servidor $m$ en trasladarse a otra cola a partir de la $k$-\'esima.\\
\end{itemize}

Sup\'ongase que la funci\'on $\left(\overline{Q}\left(\cdot\right),\overline{T}_{m} \left(\cdot\right),\overline{T}_{m}^{0} \left(\cdot\right)\right)$
para $m=1,2,\ldots,M$ es un punto l\'imite de
\begin{equation}\label{Eq.Punto.Limite}
\left(\frac{1}{|x|}Q^{x}\left(|x|t\right),\frac{1}{|x|}T_{m}^{x}\left(|x|t\right),\frac{1}{|x|}T_{m}^{x,0}\left(|x|t\right)\right)
\end{equation}
para $m=1,2,\ldots,M$, cuando $x\rightarrow\infty$, ver \cite{Down}. Entonces $\left(\overline{Q}\left(t\right),\overline{T}_{m} \left(t\right),\overline{T}_{m}^{0} \left(t\right)\right)$ es un flujo l\'imite del sistema. Al conjunto de todos las posibles flujos l\'imite se le llama {\emph{Modelo de Flujo}} y se le denotar\'a por $\mathcal{Q}$, ver \cite{Down, Dai, DaiSean}.\\

El modelo de flujo satisface el siguiente conjunto de ecuaciones:

\begin{equation}\label{Eq.MF.1}
\overline{Q}_{k}\left(t\right)=\overline{Q}_{k}\left(0\right)+\lambda_{k}t-\sum_{m=1}^{M}\mu_{k}\overline{T}_{m,k}\left(t\right),
\end{equation}
para $k=1,2,\ldots,K$.\\

\begin{equation}\label{Eq.MF.2}
\overline{Q}_{k}\left(t\right)\geq0\textrm{ para }k=1,2,\ldots,K.
\end{equation}

\begin{equation}\label{Eq.MF.3}
\overline{T}_{m,k}\left(0\right)=0,\textrm{ y }\overline{T}_{m,k}\left(\cdot\right)\textrm{ es no decreciente},
\end{equation}
para $k=1,2,\ldots,K$ y $m=1,2,\ldots,M$.\\
\begin{equation}\label{Eq.MF.4}
\sum_{k=1}^{K}\overline{T}_{m,k}^{0}\left(t\right)+\overline{T}_{m,k}\left(t\right)=t\textrm{ para }m=1,2,\ldots,M.
\end{equation}


\begin{Def}[Definici\'on 4.1, Dai \cite{Dai}]\label{Def.Modelo.Flujo}
Sea una disciplina de servicio espec\'ifica. Cualquier l\'imite $\left(\overline{Q}\left(\cdot\right),\overline{T}\left(\cdot\right),\overline{T}^{0}\left(\cdot\right)\right)$ en (\ref{Eq.Punto.Limite}) es un {\em flujo l\'imite} de la disciplina. Cualquier soluci\'on (\ref{Eq.MF.1})-(\ref{Eq.MF.4}) es llamado flujo soluci\'on de la disciplina.
\end{Def}

\begin{Def}
Se dice que el modelo de flujo l\'imite, modelo de flujo, de la disciplina de la cola es estable si existe una constante $\delta>0$ que depende de $\mu,\lambda$ y $P$ solamente, tal que cualquier flujo l\'imite con $|\overline{Q}\left(0\right)|+|\overline{U}|+|\overline{V}|=1$, se tiene que $\overline{Q}\left(\cdot+\delta\right)\equiv0$.
\end{Def}

Si se hace $|x|\rightarrow\infty$ sin restringir ninguna de las componentes, tambi\'en se obtienen un modelo de flujo, pero en este caso el residual de los procesos de arribo y servicio introducen un retraso:
\begin{Teo}[Teorema 4.2, Dai \cite{Dai}]\label{Tma.4.2.Dai}
Sea una disciplina fija para la cola, suponga que se cumplen las condiciones (A1)-(A3). Si el modelo de flujo l\'imite de la disciplina de la cola es estable, entonces la cadena de Markov $X$ que describe la din\'amica de la red bajo la disciplina es Harris recurrente positiva.
\end{Teo}

Ahora se procede a escalar el espacio y el tiempo para reducir la aparente fluctuaci\'on del modelo. Consid\'erese el proceso
\begin{equation}\label{Eq.3.7}
\overline{Q}^{x}\left(t\right)=\frac{1}{|x|}Q^{x}\left(|x|t\right).
\end{equation}
A este proceso se le conoce como el flujo escalado, y cualquier l\'imite $\overline{Q}^{x}\left(t\right)$ es llamado flujo l\'imite del proceso de longitud de la cola. Haciendo $|q|\rightarrow\infty$ mientras se mantiene el resto de las componentes fijas, cualquier punto l\'imite del proceso de longitud de la cola normalizado $\overline{Q}^{x}$ es soluci\'on del siguiente modelo de flujo.


\begin{Def}[Definici\'on 3.3, Dai y Meyn \cite{DaiSean}]
El modelo de flujo es estable si existe un tiempo fijo $t_{0}$ tal que $\overline{Q}\left(t\right)=0$, con $t\geq t_{0}$, para cualquier $\overline{Q}\left(\cdot\right)\in\mathcal{Q}$ que cumple con $|\overline{Q}\left(0\right)|=1$.
\end{Def}

\begin{Lemma}[Lema 3.1, Dai y Meyn \cite{DaiSean}]
Si el modelo de flujo definido por (\ref{Eq.MF.1})-(\ref{Eq.MF.4}) es estable, entonces el modelo de flujo retrasado es tambi\'en estable, es decir, existe $t_{0}>0$ tal que $\overline{Q}\left(t\right)=0$ para cualquier $t\geq t_{0}$, para cualquier soluci\'on del modelo de flujo retrasado cuya condici\'on inicial $\overline{x}$ satisface que $|\overline{x}|=|\overline{Q}\left(0\right)|+|\overline{A}\left(0\right)|+|\overline{B}\left(0\right)|\leq1$.
\end{Lemma}


Ahora ya estamos en condiciones de enunciar los resultados principales:


\begin{Teo}[Teorema 2.1, Down \cite{Down}]\label{Tma2.1.Down}
Suponga que el modelo de flujo es estable, y que se cumplen los supuestos (A1) y (A2), entonces
\begin{itemize}
\item[i)] Para alguna constante $\kappa_{p}$, y para cada condici\'on inicial $x\in X$
\begin{equation}\label{Estability.Eq1}
\limsup_{t\rightarrow\infty}\frac{1}{t}\int_{0}^{t}\esp_{x}\left[|Q\left(s\right)|^{p}\right]ds\leq\kappa_{p},
\end{equation}
donde $p$ es el entero dado en (A2).
\end{itemize}
Si adem\'as se cumple la condici\'on (A3), entonces para cada condici\'on inicial:
\begin{itemize}
\item[ii)] Los momentos transitorios convergen a su estado estacionario:
 \begin{equation}\label{Estability.Eq2}
lim_{t\rightarrow\infty}\esp_{x}\left[Q_{k}\left(t\right)^{r}\right]=\esp_{\pi}\left[Q_{k}\left(0\right)^{r}\right]\leq\kappa_{r},
\end{equation}
para $r=1,2,\ldots,p$ y $k=1,2,\ldots,K$. Donde $\pi$ es la probabilidad invariante para $X$.

\item[iii)]  El primer momento converge con raz\'on $t^{p-1}$:
\begin{equation}\label{Estability.Eq3}
lim_{t\rightarrow\infty}t^{p-1}|\esp_{x}\left[Q_{k}\left(t\right)\right]-\esp_{\pi}\left[Q_{k}\left(0\right)\right]|=0.
\end{equation}

\item[iv)] La {\em Ley Fuerte de los grandes n\'umeros} se cumple:
\begin{equation}\label{Estability.Eq4}
lim_{t\rightarrow\infty}\frac{1}{t}\int_{0}^{t}Q_{k}^{r}\left(s\right)ds=\esp_{\pi}\left[Q_{k}\left(0\right)^{r}\right],\textrm{ }\prob_{x}\textrm{-c.s.}
\end{equation}
para $r=1,2,\ldots,p$ y $k=1,2,\ldots,K$.
\end{itemize}
\end{Teo}

La contribuci\'on de Down a la teor\'ia de los {\emph {sistemas de visitas c\'iclicas}}, es la relaci\'on que hay entre la estabilidad del sistema con el comportamiento de las medidas de desempe\~no, es decir, la condici\'on suficiente para poder garantizar la convergencia del proceso de la longitud de la cola as\'i como de por los menos los dos primeros momentos adem\'as de una versi\'on de la Ley Fuerte de los Grandes N\'umeros para los
sistemas de visitas.

\begin{Teo}[Teorema 2.3, Down \cite{Down}]\label{Tma2.3.Down}
Considere el siguiente valor:
\begin{equation}\label{Eq.Rho.1serv}
\rho=\sum_{k=1}^{K}\rho_{k}+max_{1\leq j\leq K}\left(\frac{\lambda_{j}}{\sum_{s=1}^{S}p_{js}\overline{N}_{s}}\right)\delta^{*}
\end{equation}
\begin{itemize}
\item[i)] Si $\rho<1$ entonces la red es estable, es decir, se cumple el Teorema \ref{Tma2.1.Down}.

\item[ii)] Si $\rho>1$ entonces la red es inestable, es decir, se cumple el Teorema \ref{Tma2.2.Down}
\end{itemize}
\end{Teo}

\begin{Teo}
Sea $\left(X_{n},\mathcal{F}_{n},n=0,1,\ldots,\right\}$ Proceso de
Markov con espacio de estados $\left(S_{0},\chi_{0}\right)$
generado por una distribuici\'on inicial $P_{o}$ y probabilidad de
transici\'on $p_{mn}$, para $m,n=0,1,\ldots,$ $m<n$, que por
notaci\'on se escribir\'a como $p\left(m,n,x,B\right)\rightarrow
p_{mn}\left(x,B\right)$. Sea $S$ tiempo de paro relativo a la
$\sigma$-\'algebra $\mathcal{F}_{n}$. Sea $T$ funci\'on medible,
$T:\Omega\rightarrow\left\{0,1,\ldots,\right\}$. Sup\'ongase que
$T\geq S$, entonces $T$ es tiempo de paro. Si $B\in\chi_{0}$,
entonces
\begin{equation}\label{Prop.Fuerte.Markov}
P\left\{X\left(T\right)\in
B,T<\infty|\mathcal{F}\left(S\right)\right\} =
p\left(S,T,X\left(s\right),B\right)
\end{equation}
en $\left\{T<\infty\right\}$.
\end{Teo}


Sea $K$ conjunto numerable y sea $d:K\rightarrow\nat$ funci\'on.
Para $v\in K$, $M_{v}$ es un conjunto abierto de
$\rea^{d\left(v\right)}$. Entonces \[E=\cup_{v\in
K}M_{v}=\left\{\left(v,\zeta\right):v\in K,\zeta\in
M_{v}\right\}.\]

Sea $\mathcal{E}$ la clase de conjuntos medibles en $E$:
\[\mathcal{E}=\left\{\cup_{v\in K}A_{v}:A_{v}\in \mathcal{M}_{v}\right\}.\]

donde $\mathcal{M}$ son los conjuntos de Borel de $M_{v}$.
Entonces $\left(E,\mathcal{E}\right)$ es un espacio de Borel. El
estado del proceso se denotar\'a por
$\mathbf{x}_{t}=\left(v_{t},\zeta_{t}\right)$. La distribuci\'on
de $\left(\mathbf{x}_{t}\right)$ est\'a determinada por por los
siguientes objetos:

\begin{itemize}
\item[i)] Los campos vectoriales $\left(\mathcal{H}_{v},v\in
K\right)$. \item[ii)] Una funci\'on medible $\lambda:E\rightarrow
\rea_{+}$. \item[iii)] Una medida de transici\'on
$Q:\mathcal{E}\times\left(E\cup\Gamma^{*}\right)\rightarrow\left[0,1\right]$
donde
\begin{equation}
\Gamma^{*}=\cup_{v\in K}\partial^{*}M_{v}.
\end{equation}
y
\begin{equation}
\partial^{*}M_{v}=\left\{z\in\partial M_{v}:\mathbf{\mathbf{\phi}_{v}\left(t,\zeta\right)=\mathbf{z}}\textrm{ para alguna }\left(t,\zeta\right)\in\rea_{+}\times M_{v}\right\}.
\end{equation}
$\partial M_{v}$ denota  la frontera de $M_{v}$.
\end{itemize}

El campo vectorial $\left(\mathcal{H}_{v},v\in K\right)$ se supone
tal que para cada $\mathbf{z}\in M_{v}$ existe una \'unica curva
integral $\mathbf{\phi}_{v}\left(t,\zeta\right)$ que satisface la
ecuaci\'on

\begin{equation}
\frac{d}{dt}f\left(\zeta_{t}\right)=\mathcal{H}f\left(\zeta_{t}\right),
\end{equation}
con $\zeta_{0}=\mathbf{z}$, para cualquier funci\'on suave
$f:\rea^{d}\rightarrow\rea$ y $\mathcal{H}$ denota el operador
diferencial de primer orden, con $\mathcal{H}=\mathcal{H}_{v}$ y
$\zeta_{t}=\mathbf{\phi}\left(t,\mathbf{z}\right)$. Adem\'as se
supone que $\mathcal{H}_{v}$ es conservativo, es decir, las curvas
integrales est\'an definidas para todo $t>0$.

Para $\mathbf{x}=\left(v,\zeta\right)\in E$ se denota
\[t^{*}\mathbf{x}=inf\left\{t>0:\mathbf{\phi}_{v}\left(t,\zeta\right)\in\partial^{*}M_{v}\right\}\]

En lo que respecta a la funci\'on $\lambda$, se supondr\'a que
para cada $\left(v,\zeta\right)\in E$ existe un $\epsilon>0$ tal
que la funci\'on
$s\rightarrow\lambda\left(v,\phi_{v}\left(s,\zeta\right)\right)\in
E$ es integrable para $s\in\left[0,\epsilon\right)$. La medida de
transici\'on $Q\left(A;\mathbf{x}\right)$ es una funci\'on medible
de $\mathbf{x}$ para cada $A\in\mathcal{E}$, definida para
$\mathbf{x}\in E\cup\Gamma^{*}$ y es una medida de probabilidad en
$\left(E,\mathcal{E}\right)$ para cada $\mathbf{x}\in E$.

El movimiento del proceso $\left(\mathbf{x}_{t}\right)$ comenzando
en $\mathbf{x}=\left(n,\mathbf{z}\right)\in E$ se puede construir
de la siguiente manera, def\'inase la funci\'on $F$ por

\begin{equation}
F\left(t\right)=\left\{\begin{array}{ll}\\
exp\left(-\int_{0}^{t}\lambda\left(n,\phi_{n}\left(s,\mathbf{z}\right)\right)ds\right), & t<t^{*}\left(\mathbf{x}\right),\\
0, & t\geq t^{*}\left(\mathbf{x}\right)
\end{array}\right.
\end{equation}

Sea $T_{1}$ una variable aleatoria tal que
$\prob\left[T_{1}>t\right]=F\left(t\right)$, ahora sea la variable
aleatoria $\left(N,Z\right)$ con distribuici\'on
$Q\left(\cdot;\phi_{n}\left(T_{1},\mathbf{z}\right)\right)$. La
trayectoria de $\left(\mathbf{x}_{t}\right)$ para $t\leq T_{1}$
es\footnote{Revisar p\'agina 362, y 364 de Davis \cite{Davis}.}
\begin{eqnarray*}
\mathbf{x}_{t}=\left(v_{t},\zeta_{t}\right)=\left\{\begin{array}{ll}
\left(n,\phi_{n}\left(t,\mathbf{z}\right)\right), & t<T_{1},\\
\left(N,\mathbf{Z}\right), & t=t_{1}.
\end{array}\right.
\end{eqnarray*}

Comenzando en $\mathbf{x}_{T_{1}}$ se selecciona el siguiente
tiempo de intersalto $T_{2}-T_{1}$ lugar del post-salto
$\mathbf{x}_{T_{2}}$ de manera similar y as\'i sucesivamente. Este
procedimiento nos da una trayectoria determinista por partes
$\mathbf{x}_{t}$ con tiempos de salto $T_{1},T_{2},\ldots$. Bajo
las condiciones enunciadas para $\lambda,T_{1}>0$  y
$T_{1}-T_{2}>0$ para cada $i$, con probabilidad 1. Se supone que
se cumple la siguiente condici\'on.

\begin{Sup}[Supuesto 3.1, Davis \cite{Davis}]\label{Sup3.1.Davis}
Sea $N_{t}:=\sum_{t}\indora_{\left(t\geq t\right)}$ el n\'umero de
saltos en $\left[0,t\right]$. Entonces
\begin{equation}
\esp\left[N_{t}\right]<\infty\textrm{ para toda }t.
\end{equation}
\end{Sup}

es un proceso de Markov, m\'as a\'un, es un Proceso Fuerte de
Markov, es decir, la Propiedad Fuerte de Markov se cumple para
cualquier tiempo de paro.


Sea $E$ es un espacio m\'etrico separable y la m\'etrica $d$ es
compatible con la topolog\'ia.


\begin{Def}
Un espacio topol\'ogico $E$ es llamado de {\em Rad\'on} si es
homeomorfo a un subconjunto universalmente medible de un espacio
m\'etrico compacto.
\end{Def}

Equivalentemente, la definici\'on de un espacio de Rad\'on puede
encontrarse en los siguientes t\'erminos:


\begin{Def}
$E$ es un espacio de Rad\'on si cada medida finita en
$\left(E,\mathcal{B}\left(E\right)\right)$ es regular interior o
cerrada, {\em tight}.
\end{Def}

\begin{Def}
Una medida finita, $\lambda$ en la $\sigma$-\'algebra de Borel de
un espacio metrizable $E$ se dice cerrada si
\begin{equation}\label{Eq.A2.3}
\lambda\left(E\right)=sup\left\{\lambda\left(K\right):K\textrm{ es
compacto en }E\right\}.
\end{equation}
\end{Def}

El siguiente teorema nos permite tener una mejor caracterizaci\'on
de los espacios de Rad\'on:
\begin{Teo}\label{Tma.A2.2}
Sea $E$ espacio separable metrizable. Entonces $E$ es Radoniano si
y s\'olo s\'i cada medida finita en
$\left(E,\mathcal{B}\left(E\right)\right)$ es cerrada.
\end{Teo}

Sea $E$ espacio de estados, tal que $E$ es un espacio de Rad\'on,
$\mathcal{B}\left(E\right)$ $\sigma$-\'algebra de Borel en $E$,
que se denotar\'a por $\mathcal{E}$.

Sea $\left(X,\mathcal{G},\prob\right)$ espacio de probabilidad,
$I\subset\rea$ conjunto de \'indices. Sea $\mathcal{F}_{\leq t}$
la $\sigma$-\'algebra natural definida como
$\sigma\left\{f\left(X_{r}\right):r\in I, r\leq
t,f\in\mathcal{E}\right\}$. Se considerar\'a una
$\sigma$-\'algebra m\'as general, $ \left(\mathcal{G}_{t}\right)$
tal que $\left(X_{t}\right)$ sea $\mathcal{E}$-adaptado.

\begin{Def}
Una familia $\left(P_{s,t}\right)$ de kernels de Markov en
$\left(E,\mathcal{E}\right)$ indexada por pares $s,t\in I$, con
$s\leq t$ es una funci\'on de transici\'on en $\ER$, si  para todo
$r\leq s< t$ en $I$ y todo $x\in E$,
$B\in\mathcal{E}$\footnote{Ecuaci\'on de Chapman-Kolmogorov}
\begin{equation}\label{Eq.Kernels}
P_{r,t}\left(x,B\right)=\int_{E}P_{r,s}\left(x,dy\right)P_{s,t}\left(y,B\right).
\end{equation}
\end{Def}

Se dice que la funci\'on de transici\'on $\KM$ en $\ER$ es la
funci\'on de transici\'on para un proceso $\PE$  con valores en
$E$ y que satisface la propiedad de
Markov\footnote{\begin{equation}\label{Eq.1.4.S}
\prob\left\{H|\mathcal{G}_{t}\right\}=\prob\left\{H|X_{t}\right\}\textrm{
}H\in p\mathcal{F}_{\geq t}.
\end{equation}} (\ref{Eq.1.4.S}) relativa a $\left(\mathcal{G}_{t}\right)$ si

\begin{equation}\label{Eq.1.6.S}
\prob\left\{f\left(X_{t}\right)|\mathcal{G}_{s}\right\}=P_{s,t}f\left(X_{t}\right)\textrm{
}s\leq t\in I,\textrm{ }f\in b\mathcal{E}.
\end{equation}

\begin{Def}
Una familia $\left(P_{t}\right)_{t\geq0}$ de kernels de Markov en
$\ER$ es llamada {\em Semigrupo de Transici\'on de Markov} o {\em
Semigrupo de Transici\'on} si
\[P_{t+s}f\left(x\right)=P_{t}\left(P_{s}f\right)\left(x\right),\textrm{ }t,s\geq0,\textrm{ }x\in E\textrm{ }f\in b\mathcal{E}.\]
\end{Def}
\begin{Note}
Si la funci\'on de transici\'on $\KM$ es llamada homog\'enea si
$P_{s,t}=P_{t-s}$.
\end{Note}

Un proceso de Markov que satisface la ecuaci\'on (\ref{Eq.1.6.S})
con funci\'on de transici\'on homog\'enea $\left(P_{t}\right)$
tiene la propiedad caracter\'istica
\begin{equation}\label{Eq.1.8.S}
\prob\left\{f\left(X_{t+s}\right)|\mathcal{G}_{t}\right\}=P_{s}f\left(X_{t}\right)\textrm{
}t,s\geq0,\textrm{ }f\in b\mathcal{E}.
\end{equation}
La ecuaci\'on anterior es la {\em Propiedad Simple de Markov} de
$X$ relativa a $\left(P_{t}\right)$.

En este sentido el proceso $\PE$ cumple con la propiedad de Markov
(\ref{Eq.1.8.S}) relativa a
$\left(\Omega,\mathcal{G},\mathcal{G}_{t},\prob\right)$ con
semigrupo de transici\'on $\left(P_{t}\right)$.

\begin{Def}
Un proceso estoc\'astico $\PE$ definido en
$\left(\Omega,\mathcal{G},\prob\right)$ con valores en el espacio
topol\'ogico $E$ es continuo por la derecha si cada trayectoria
muestral $t\rightarrow X_{t}\left(w\right)$ es un mapeo continuo
por la derecha de $I$ en $E$.
\end{Def}

\begin{Def}[HD1]\label{Eq.2.1.S}
Un semigrupo de Markov $\left(P_{t}\right)$ en un espacio de
Rad\'on $E$ se dice que satisface la condici\'on {\em HD1} si,
dada una medida de probabilidad $\mu$ en $E$, existe una
$\sigma$-\'algebra $\mathcal{E^{*}}$ con
$\mathcal{E}\subset\mathcal{E}^{*}$ y
$P_{t}\left(b\mathcal{E}^{*}\right)\subset b\mathcal{E}^{*}$, y un
$\mathcal{E}^{*}$-proceso $E$-valuado continuo por la derecha
$\PE$ en alg\'un espacio de probabilidad filtrado
$\left(\Omega,\mathcal{G},\mathcal{G}_{t},\prob\right)$ tal que
$X=\left(\Omega,\mathcal{G},\mathcal{G}_{t},\prob\right)$ es de
Markov (Homog\'eneo) con semigrupo de transici\'on $(P_{t})$ y
distribuci\'on inicial $\mu$.
\end{Def}

Consid\'erese la colecci\'on de variables aleatorias $X_{t}$
definidas en alg\'un espacio de probabilidad, y una colecci\'on de
medidas $\mathbf{P}^{x}$ tales que
$\mathbf{P}^{x}\left\{X_{0}=x\right\}$, y bajo cualquier
$\mathbf{P}^{x}$, $X_{t}$ es de Markov con semigrupo
$\left(P_{t}\right)$. $\mathbf{P}^{x}$ puede considerarse como la
distribuci\'on condicional de $\mathbf{P}$ dado $X_{0}=x$.

\begin{Def}\label{Def.2.2.S}
Sea $E$ espacio de Rad\'on, $\SG$ semigrupo de Markov en $\ER$. La
colecci\'on
$\mathbf{X}=\left(\Omega,\mathcal{G},\mathcal{G}_{t},X_{t},\theta_{t},\CM\right)$
es un proceso $\mathcal{E}$-Markov continuo por la derecha simple,
con espacio de estados $E$ y semigrupo de transici\'on $\SG$ en
caso de que $\mathbf{X}$ satisfaga las siguientes condiciones:
\begin{itemize}
\item[i)] $\left(\Omega,\mathcal{G},\mathcal{G}_{t}\right)$ es un
espacio de medida filtrado, y $X_{t}$ es un proceso $E$-valuado
continuo por la derecha $\mathcal{E}^{*}$-adaptado a
$\left(\mathcal{G}_{t}\right)$;

\item[ii)] $\left(\theta_{t}\right)_{t\geq0}$ es una colecci\'on
de operadores {\em shift} para $X$, es decir, mapea $\Omega$ en
s\'i mismo satisfaciendo para $t,s\geq0$,

\begin{equation}\label{Eq.Shift}
\theta_{t}\circ\theta_{s}=\theta_{t+s}\textrm{ y
}X_{t}\circ\theta_{t}=X_{t+s};
\end{equation}

\item[iii)] Para cualquier $x\in E$,$\CM\left\{X_{0}=x\right\}=1$,
y el proceso $\PE$ tiene la propiedad de Markov (\ref{Eq.1.8.S})
con semigrupo de transici\'on $\SG$ relativo a
$\left(\Omega,\mathcal{G},\mathcal{G}_{t},\CM\right)$.
\end{itemize}
\end{Def}

\begin{Def}[HD2]\label{Eq.2.2.S}
Para cualquier $\alpha>0$ y cualquier $f\in S^{\alpha}$, el
proceso $t\rightarrow f\left(X_{t}\right)$ es continuo por la
derecha casi seguramente.
\end{Def}

\begin{Def}\label{Def.PD}
Un sistema
$\mathbf{X}=\left(\Omega,\mathcal{G},\mathcal{G}_{t},X_{t},\theta_{t},\CM\right)$
es un proceso derecho en el espacio de Rad\'on $E$ con semigrupo
de transici\'on $\SG$ provisto de:
\begin{itemize}
\item[i)] $\mathbf{X}$ es una realizaci\'on  continua por la
derecha, \ref{Def.2.2.S}, de $\SG$.

\item[ii)] $\mathbf{X}$ satisface la condicion HD2,
\ref{Eq.2.2.S}, relativa a $\mathcal{G}_{t}$.

\item[iii)] $\mathcal{G}_{t}$ es aumentado y continuo por la
derecha.
\end{itemize}
\end{Def}

\begin{Lema}[Lema 4.2, Dai\cite{Dai}]\label{Lema4.2}
Sea $\left\{x_{n}\right\}\subset \mathbf{X}$ con
$|x_{n}|\rightarrow\infty$, conforme $n\rightarrow\infty$. Suponga
que
\[lim_{n\rightarrow\infty}\frac{1}{|x_{n}|}U\left(0\right)=\overline{U}\]
y
\[lim_{n\rightarrow\infty}\frac{1}{|x_{n}|}V\left(0\right)=\overline{V}.\]

Entonces, conforme $n\rightarrow\infty$, casi seguramente

\begin{equation}\label{E1.4.2}
\frac{1}{|x_{n}|}\Phi^{k}\left(\left[|x_{n}|t\right]\right)\rightarrow
P_{k}^{'}t\textrm{, u.o.c.,}
\end{equation}

\begin{equation}\label{E1.4.3}
\frac{1}{|x_{n}|}E^{x_{n}}_{k}\left(|x_{n}|t\right)\rightarrow
\alpha_{k}\left(t-\overline{U}_{k}\right)^{+}\textrm{, u.o.c.,}
\end{equation}

\begin{equation}\label{E1.4.4}
\frac{1}{|x_{n}|}S^{x_{n}}_{k}\left(|x_{n}|t\right)\rightarrow
\mu_{k}\left(t-\overline{V}_{k}\right)^{+}\textrm{, u.o.c.,}
\end{equation}

donde $\left[t\right]$ es la parte entera de $t$ y
$\mu_{k}=1/m_{k}=1/\esp\left[\eta_{k}\left(1\right)\right]$.
\end{Lema}

\begin{Lema}[Lema 4.3, Dai\cite{Dai}]\label{Lema.4.3}
Sea $\left\{x_{n}\right\}\subset \mathbf{X}$ con
$|x_{n}|\rightarrow\infty$, conforme $n\rightarrow\infty$. Suponga
que
\[lim_{n\rightarrow\infty}\frac{1}{|x_{n}|}U\left(0\right)=\overline{U}_{k}\]
y
\[lim_{n\rightarrow\infty}\frac{1}{|x_{n}|}V\left(0\right)=\overline{V}_{k}.\]
\begin{itemize}
\item[a)] Conforme $n\rightarrow\infty$ casi seguramente,
\[lim_{n\rightarrow\infty}\frac{1}{|x_{n}|}U^{x_{n}}_{k}\left(|x_{n}|t\right)=\left(\overline{U}_{k}-t\right)^{+}\textrm{, u.o.c.}\]
y
\[lim_{n\rightarrow\infty}\frac{1}{|x_{n}|}V^{x_{n}}_{k}\left(|x_{n}|t\right)=\left(\overline{V}_{k}-t\right)^{+}.\]

\item[b)] Para cada $t\geq0$ fijo,
\[\left\{\frac{1}{|x_{n}|}U^{x_{n}}_{k}\left(|x_{n}|t\right),|x_{n}|\geq1\right\}\]
y
\[\left\{\frac{1}{|x_{n}|}V^{x_{n}}_{k}\left(|x_{n}|t\right),|x_{n}|\geq1\right\}\]
\end{itemize}
son uniformemente convergentes.
\end{Lema}

$S_{l}^{x}\left(t\right)$ es el n\'umero total de servicios
completados de la clase $l$, si la clase $l$ est\'a dando $t$
unidades de tiempo de servicio. Sea $T_{l}^{x}\left(x\right)$ el
monto acumulado del tiempo de servicio que el servidor
$s\left(l\right)$ gasta en los usuarios de la clase $l$ al tiempo
$t$. Entonces $S_{l}^{x}\left(T_{l}^{x}\left(t\right)\right)$ es
el n\'umero total de servicios completados para la clase $l$ al
tiempo $t$. Una fracci\'on de estos usuarios,
$\Phi_{l}^{x}\left(S_{l}^{x}\left(T_{l}^{x}\left(t\right)\right)\right)$,
se convierte en usuarios de la clase $k$.\\

Entonces, dado lo anterior, se tiene la siguiente representaci\'on
para el proceso de la longitud de la cola:\\

\begin{equation}
Q_{k}^{x}\left(t\right)=_{k}^{x}\left(0\right)+E_{k}^{x}\left(t\right)+\sum_{l=1}^{K}\Phi_{k}^{l}\left(S_{l}^{x}\left(T_{l}^{x}\left(t\right)\right)\right)-S_{k}^{x}\left(T_{k}^{x}\left(t\right)\right)
\end{equation}
para $k=1,\ldots,K$. Para $i=1,\ldots,d$, sea
\[I_{i}^{x}\left(t\right)=t-\sum_{j\in C_{i}}T_{k}^{x}\left(t\right).\]

Entonces $I_{i}^{x}\left(t\right)$ es el monto acumulado del
tiempo que el servidor $i$ ha estado desocupado al tiempo $t$. Se
est\'a asumiendo que las disciplinas satisfacen la ley de
conservaci\'on del trabajo, es decir, el servidor $i$ est\'a en
pausa solamente cuando no hay usuarios en la estaci\'on $i$.
Entonces, se tiene que

\begin{equation}
\int_{0}^{\infty}\left(\sum_{k\in
C_{i}}Q_{k}^{x}\left(t\right)\right)dI_{i}^{x}\left(t\right)=0,
\end{equation}
para $i=1,\ldots,d$.\\

Hacer
\[T^{x}\left(t\right)=\left(T_{1}^{x}\left(t\right),\ldots,T_{K}^{x}\left(t\right)\right)^{'},\]
\[I^{x}\left(t\right)=\left(I_{1}^{x}\left(t\right),\ldots,I_{K}^{x}\left(t\right)\right)^{'}\]
y
\[S^{x}\left(T^{x}\left(t\right)\right)=\left(S_{1}^{x}\left(T_{1}^{x}\left(t\right)\right),\ldots,S_{K}^{x}\left(T_{K}^{x}\left(t\right)\right)\right)^{'}.\]

Para una disciplina que cumple con la ley de conservaci\'on del
trabajo, en forma vectorial, se tiene el siguiente conjunto de
ecuaciones

\begin{equation}\label{Eq.MF.1.3}
Q^{x}\left(t\right)=Q^{x}\left(0\right)+E^{x}\left(t\right)+\sum_{l=1}^{K}\Phi^{l}\left(S_{l}^{x}\left(T_{l}^{x}\left(t\right)\right)\right)-S^{x}\left(T^{x}\left(t\right)\right),\\
\end{equation}

\begin{equation}\label{Eq.MF.2.3}
Q^{x}\left(t\right)\geq0,\\
\end{equation}

\begin{equation}\label{Eq.MF.3.3}
T^{x}\left(0\right)=0,\textrm{ y }\overline{T}^{x}\left(t\right)\textrm{ es no decreciente},\\
\end{equation}

\begin{equation}\label{Eq.MF.4.3}
I^{x}\left(t\right)=et-CT^{x}\left(t\right)\textrm{ es no
decreciente}\\
\end{equation}

\begin{equation}\label{Eq.MF.5.3}
\int_{0}^{\infty}\left(CQ^{x}\left(t\right)\right)dI_{i}^{x}\left(t\right)=0,\\
\end{equation}

\begin{equation}\label{Eq.MF.6.3}
\textrm{Condiciones adicionales en
}\left(\overline{Q}^{x}\left(\cdot\right),\overline{T}^{x}\left(\cdot\right)\right)\textrm{
espec\'ificas de la disciplina de la cola,}
\end{equation}

donde $e$ es un vector de unos de dimensi\'on $d$, $C$ es la
matriz definida por
\[C_{ik}=\left\{\begin{array}{cc}
1,& S\left(k\right)=i,\\
0,& \textrm{ en otro caso}.\\
\end{array}\right.
\]
Es necesario enunciar el siguiente Teorema que se utilizar\'a para
el Teorema \ref{Tma.4.2.Dai}:
\begin{Teo}[Teorema 4.1, Dai \cite{Dai}]
Considere una disciplina que cumpla la ley de conservaci\'on del
trabajo, para casi todas las trayectorias muestrales $\omega$ y
cualquier sucesi\'on de estados iniciales
$\left\{x_{n}\right\}\subset \mathbf{X}$, con
$|x_{n}|\rightarrow\infty$, existe una subsucesi\'on
$\left\{x_{n_{j}}\right\}$ con $|x_{n_{j}}|\rightarrow\infty$ tal
que
\begin{equation}\label{Eq.4.15}
\frac{1}{|x_{n_{j}}|}\left(Q^{x_{n_{j}}}\left(0\right),U^{x_{n_{j}}}\left(0\right),V^{x_{n_{j}}}\left(0\right)\right)\rightarrow\left(\overline{Q}\left(0\right),\overline{U},\overline{V}\right),
\end{equation}

\begin{equation}\label{Eq.4.16}
\frac{1}{|x_{n_{j}}|}\left(Q^{x_{n_{j}}}\left(|x_{n_{j}}|t\right),T^{x_{n_{j}}}\left(|x_{n_{j}}|t\right)\right)\rightarrow\left(\overline{Q}\left(t\right),\overline{T}\left(t\right)\right)\textrm{
u.o.c.}
\end{equation}

Adem\'as,
$\left(\overline{Q}\left(t\right),\overline{T}\left(t\right)\right)$
satisface las siguientes ecuaciones:
\begin{equation}\label{Eq.MF.1.3a}
\overline{Q}\left(t\right)=Q\left(0\right)+\left(\alpha
t-\overline{U}\right)^{+}-\left(I-P\right)^{'}M^{-1}\left(\overline{T}\left(t\right)-\overline{V}\right)^{+},
\end{equation}

\begin{equation}\label{Eq.MF.2.3a}
\overline{Q}\left(t\right)\geq0,\\
\end{equation}

\begin{equation}\label{Eq.MF.3.3a}
\overline{T}\left(t\right)\textrm{ es no decreciente y comienza en cero},\\
\end{equation}

\begin{equation}\label{Eq.MF.4.3a}
\overline{I}\left(t\right)=et-C\overline{T}\left(t\right)\textrm{
es no decreciente,}\\
\end{equation}

\begin{equation}\label{Eq.MF.5.3a}
\int_{0}^{\infty}\left(C\overline{Q}\left(t\right)\right)d\overline{I}\left(t\right)=0,\\
\end{equation}

\begin{equation}\label{Eq.MF.6.3a}
\textrm{Condiciones adicionales en
}\left(\overline{Q}\left(\cdot\right),\overline{T}\left(\cdot\right)\right)\textrm{
especficas de la disciplina de la cola,}
\end{equation}
\end{Teo}


En Dai \cite{Dai} se muestra que para una amplia serie de disciplinas
de servicio el proceso $X$ es un Proceso Fuerte de
Markov, y por tanto se puede asumir que


Para establecer que $X=\left\{X\left(t\right),t\geq0\right\}$ es
un Proceso Fuerte de Markov, se siguen las secciones 2.3 y 2.4 de Kaspi and Mandelbaum \cite{KaspiMandelbaum}. \\

Sea el proceso de Markov $X=\left\{X\left(t\right),t\geq0\right\}$
que describe la din\'amica de la red de colas. En lo que respecta
al supuesto (A3), en Dai y Meyn \cite{DaiSean} y Meyn y Down
\cite{MeynDown} hacen ver que este se puede sustituir por

\begin{itemize}
\item[A3')] Para el Proceso de Markov $X$, cada subconjunto
compacto de $X$ es un conjunto peque\~no.
\end{itemize}

Este supuesto es importante pues es un requisito para deducir la ergodicidad de la red.
%_________________________________________________________________________
\section{Construcci\'on de un Modelo de Flujo L\'imite}
%_________________________________________________________________________

Consideremos un caso m\'as simple para poner en contexto lo
anterior: para un sistema de visitas c\'iclicas se tiene que el
estado al tiempo $t$ es
\begin{equation}
X\left(t\right)=\left(Q\left(t\right),U\left(t\right),V\left(t\right)\right),
\end{equation}

donde $Q\left(t\right)$ es el n\'umero de usuarios formados en
cada estaci\'on. $U\left(t\right)$ es el tiempo restante antes de
que la siguiente clase $k$ de usuarios lleguen desde fuera del
sistema, $V\left(t\right)$ es el tiempo restante de servicio para
la clase $k$ de usuarios que est\'an siendo atendidos. Tanto
$U\left(t\right)$ como $V\left(t\right)$ se puede asumir que son
continuas por la derecha.

Sea
$x=\left(Q\left(0\right),U\left(0\right),V\left(0\right)\right)=\left(q,a,b\right)$,
el estado inicial de la red bajo una disciplina espec\'ifica para
la cola. Para $l\in\mathcal{E}$, donde $\mathcal{E}$ es el conjunto de clases de arribos externos, y $k=1,\ldots,K$ se define\\
\begin{eqnarray*}
E_{l}^{x}\left(t\right)&=&max\left\{r:U_{l}\left(0\right)+\xi_{l}\left(1\right)+\cdots+\xi_{l}\left(r-1\right)\leq
t\right\}\textrm{   }t\geq0,\\
S_{k}^{x}\left(t\right)&=&max\left\{r:V_{k}\left(0\right)+\eta_{k}\left(1\right)+\cdots+\eta_{k}\left(r-1\right)\leq
t\right\}\textrm{   }t\geq0.
\end{eqnarray*}

Para cada $k$ y cada $n$ se define

\begin{eqnarray*}\label{Eq.phi}
\Phi^{k}\left(n\right):=\sum_{i=1}^{n}\phi^{k}\left(i\right).
\end{eqnarray*}

donde $\phi^{k}\left(n\right)$ se define como el vector de ruta
para el $n$-\'esimo usuario de la clase $k$ que termina en la
estaci\'on $s\left(k\right)$, la $s$-\'eima componente de
$\phi^{k}\left(n\right)$ es uno si estos usuarios se convierten en
usuarios de la clase $l$ y cero en otro caso, por lo tanto
$\phi^{k}\left(n\right)$ es un vector {\em Bernoulli} de
dimensi\'on $K$ con par\'ametro $P_{k}^{'}$, donde $P_{k}$ denota
el $k$-\'esimo rengl\'on de $P=\left(P_{kl}\right)$.

Se asume que cada para cada $k$ la sucesi\'on $\phi^{k}\left(n\right)=\left\{\phi^{k}\left(n\right),n\geq1\right\}$
es independiente e id\'enticamente distribuida y que las
$\phi^{1}\left(n\right),\ldots,\phi^{K}\left(n\right)$ son
mutuamente independientes, adem\'as de independientes de los
procesos de arribo y de servicio.\\

\begin{Lema}[Lema 4.2, Dai\cite{Dai}]\label{Lema4.2}
Sea $\left\{x_{n}\right\}\subset \mathbf{X}$ con
$|x_{n}|\rightarrow\infty$, conforme $n\rightarrow\infty$. Suponga
que
\[lim_{n\rightarrow\infty}\frac{1}{|x_{n}|}U\left(0\right)=\overline{U}\]
y
\[lim_{n\rightarrow\infty}\frac{1}{|x_{n}|}V\left(0\right)=\overline{V}.\]

Entonces, conforme $n\rightarrow\infty$, casi seguramente

\begin{equation}\label{E1.4.2}
\frac{1}{|x_{n}|}\Phi^{k}\left(\left[|x_{n}|t\right]\right)\rightarrow
P_{k}^{'}t\textrm{, u.o.c.,}
\end{equation}

\begin{equation}\label{E1.4.3}
\frac{1}{|x_{n}|}E^{x_{n}}_{k}\left(|x_{n}|t\right)\rightarrow
\alpha_{k}\left(t-\overline{U}_{k}\right)^{+}\textrm{, u.o.c.,}
\end{equation}

\begin{equation}\label{E1.4.4}
\frac{1}{|x_{n}|}S^{x_{n}}_{k}\left(|x_{n}|t\right)\rightarrow
\mu_{k}\left(t-\overline{V}_{k}\right)^{+}\textrm{, u.o.c.,}
\end{equation}

donde $\left[t\right]$ es la parte entera de $t$ y
$\mu_{k}=1/m_{k}=1/\esp\left[\eta_{k}\left(1\right)\right]$.
\end{Lema}

\begin{Lema}[Lema 4.3, Dai\cite{Dai}]\label{Lema.4.3}
Sea $\left\{x_{n}\right\}\subset \mathbf{X}$ con
$|x_{n}|\rightarrow\infty$, conforme $n\rightarrow\infty$. Suponga
que
\[lim_{n\rightarrow\infty}\frac{1}{|x_{n}|}U\left(0\right)=\overline{U}_{k}\]
y
\[lim_{n\rightarrow\infty}\frac{1}{|x_{n}|}V\left(0\right)=\overline{V}_{k}.\]
\begin{itemize}
\item[a)] Conforme $n\rightarrow\infty$ casi seguramente,
\[lim_{n\rightarrow\infty}\frac{1}{|x_{n}|}U^{x_{n}}_{k}\left(|x_{n}|t\right)=\left(\overline{U}_{k}-t\right)^{+}\textrm{, u.o.c.}\]
y
\[lim_{n\rightarrow\infty}\frac{1}{|x_{n}|}V^{x_{n}}_{k}\left(|x_{n}|t\right)=\left(\overline{V}_{k}-t\right)^{+}.\]

\item[b)] Para cada $t\geq0$ fijo,
\[\left\{\frac{1}{|x_{n}|}U^{x_{n}}_{k}\left(|x_{n}|t\right),|x_{n}|\geq1\right\}\]
y
\[\left\{\frac{1}{|x_{n}|}V^{x_{n}}_{k}\left(|x_{n}|t\right),|x_{n}|\geq1\right\}\]
\end{itemize}
son uniformemente convergentes.
\end{Lema}

$S_{l}^{x}\left(t\right)$ es el n\'umero total de servicios
completados de la clase $l$, si la clase $l$ est\'a dando $t$
unidades de tiempo de servicio. Sea $T_{l}^{x}\left(x\right)$ el
monto acumulado del tiempo de servicio que el servidor
$s\left(l\right)$ gasta en los usuarios de la clase $l$ al tiempo
$t$. Entonces $S_{l}^{x}\left(T_{l}^{x}\left(t\right)\right)$ es
el n\'umero total de servicios completados para la clase $l$ al
tiempo $t$. Una fracci\'on de estos usuarios,
$\Phi_{l}^{x}\left(S_{l}^{x}\left(T_{l}^{x}\left(t\right)\right)\right)$,
se convierte en usuarios de la clase $k$.\\

Entonces, dado lo anterior, se tiene la siguiente representaci\'on
para el proceso de la longitud de la cola:\\

\begin{equation}
Q_{k}^{x}\left(t\right)=_{k}^{x}\left(0\right)+E_{k}^{x}\left(t\right)+\sum_{l=1}^{K}\Phi_{k}^{l}\left(S_{l}^{x}\left(T_{l}^{x}\left(t\right)\right)\right)-S_{k}^{x}\left(T_{k}^{x}\left(t\right)\right)
\end{equation}
para $k=1,\ldots,K$. Para $i=1,\ldots,d$, sea
\[I_{i}^{x}\left(t\right)=t-\sum_{j\in C_{i}}T_{k}^{x}\left(t\right).\]

Entonces $I_{i}^{x}\left(t\right)$ es el monto acumulado del
tiempo que el servidor $i$ ha estado desocupado al tiempo $t$. Se
est\'a asumiendo que las disciplinas satisfacen la ley de
conservaci\'on del trabajo, es decir, el servidor $i$ est\'a en
pausa solamente cuando no hay usuarios en la estaci\'on $i$.
Entonces, se tiene que

\begin{equation}
\int_{0}^{\infty}\left(\sum_{k\in
C_{i}}Q_{k}^{x}\left(t\right)\right)dI_{i}^{x}\left(t\right)=0,
\end{equation}
para $i=1,\ldots,d$.\\

Hacer
\[T^{x}\left(t\right)=\left(T_{1}^{x}\left(t\right),\ldots,T_{K}^{x}\left(t\right)\right)^{'},\]
\[I^{x}\left(t\right)=\left(I_{1}^{x}\left(t\right),\ldots,I_{K}^{x}\left(t\right)\right)^{'}\]
y
\[S^{x}\left(T^{x}\left(t\right)\right)=\left(S_{1}^{x}\left(T_{1}^{x}\left(t\right)\right),\ldots,S_{K}^{x}\left(T_{K}^{x}\left(t\right)\right)\right)^{'}.\]

Para una disciplina que cumple con la ley de conservaci\'on del
trabajo, en forma vectorial, se tiene el siguiente conjunto de
ecuaciones

\begin{equation}\label{Eq.MF.1.3}
Q^{x}\left(t\right)=Q^{x}\left(0\right)+E^{x}\left(t\right)+\sum_{l=1}^{K}\Phi^{l}\left(S_{l}^{x}\left(T_{l}^{x}\left(t\right)\right)\right)-S^{x}\left(T^{x}\left(t\right)\right),\\
\end{equation}

\begin{equation}\label{Eq.MF.2.3}
Q^{x}\left(t\right)\geq0,\\
\end{equation}

\begin{equation}\label{Eq.MF.3.3}
T^{x}\left(0\right)=0,\textrm{ y }\overline{T}^{x}\left(t\right)\textrm{ es no decreciente},\\
\end{equation}

\begin{equation}\label{Eq.MF.4.3}
I^{x}\left(t\right)=et-CT^{x}\left(t\right)\textrm{ es no
decreciente}\\
\end{equation}

\begin{equation}\label{Eq.MF.5.3}
\int_{0}^{\infty}\left(CQ^{x}\left(t\right)\right)dI_{i}^{x}\left(t\right)=0,\\
\end{equation}

\begin{equation}\label{Eq.MF.6.3}
\textrm{Condiciones adicionales en
}\left(\overline{Q}^{x}\left(\cdot\right),\overline{T}^{x}\left(\cdot\right)\right)\textrm{
espec\'ificas de la disciplina de la cola,}
\end{equation}

donde $e$ es un vector de unos de dimensi\'on $d$, $C$ es la
matriz definida por
\[C_{ik}=\left\{\begin{array}{cc}
1,& S\left(k\right)=i,\\
0,& \textrm{ en otro caso}.\\
\end{array}\right.
\]
Es necesario enunciar el siguiente Teorema que se utilizar\'a para
el Teorema \ref{Tma.4.2.Dai}:
\begin{Teo}[Teorema 4.1, Dai \cite{Dai}]
Considere una disciplina que cumpla la ley de conservaci\'on del
trabajo, para casi todas las trayectorias muestrales $\omega$ y
cualquier sucesi\'on de estados iniciales
$\left\{x_{n}\right\}\subset \mathbf{X}$, con
$|x_{n}|\rightarrow\infty$, existe una subsucesi\'on
$\left\{x_{n_{j}}\right\}$ con $|x_{n_{j}}|\rightarrow\infty$ tal
que
\begin{equation}\label{Eq.4.15}
\frac{1}{|x_{n_{j}}|}\left(Q^{x_{n_{j}}}\left(0\right),U^{x_{n_{j}}}\left(0\right),V^{x_{n_{j}}}\left(0\right)\right)\rightarrow\left(\overline{Q}\left(0\right),\overline{U},\overline{V}\right),
\end{equation}

\begin{equation}\label{Eq.4.16}
\frac{1}{|x_{n_{j}}|}\left(Q^{x_{n_{j}}}\left(|x_{n_{j}}|t\right),T^{x_{n_{j}}}\left(|x_{n_{j}}|t\right)\right)\rightarrow\left(\overline{Q}\left(t\right),\overline{T}\left(t\right)\right)\textrm{
u.o.c.}
\end{equation}

Adem\'as,
$\left(\overline{Q}\left(t\right),\overline{T}\left(t\right)\right)$
satisface las siguientes ecuaciones:
\begin{equation}\label{Eq.MF.1.3a}
\overline{Q}\left(t\right)=Q\left(0\right)+\left(\alpha
t-\overline{U}\right)^{+}-\left(I-P\right)^{'}M^{-1}\left(\overline{T}\left(t\right)-\overline{V}\right)^{+},
\end{equation}

\begin{equation}\label{Eq.MF.2.3a}
\overline{Q}\left(t\right)\geq0,\\
\end{equation}

\begin{equation}\label{Eq.MF.3.3a}
\overline{T}\left(t\right)\textrm{ es no decreciente y comienza en cero},\\
\end{equation}

\begin{equation}\label{Eq.MF.4.3a}
\overline{I}\left(t\right)=et-C\overline{T}\left(t\right)\textrm{
es no decreciente,}\\
\end{equation}

\begin{equation}\label{Eq.MF.5.3a}
\int_{0}^{\infty}\left(C\overline{Q}\left(t\right)\right)d\overline{I}\left(t\right)=0,\\
\end{equation}

\begin{equation}\label{Eq.MF.6.3a}
\textrm{Condiciones adicionales en
}\left(\overline{Q}\left(\cdot\right),\overline{T}\left(\cdot\right)\right)\textrm{
especficas de la disciplina de la cola,}
\end{equation}
\end{Teo}

\begin{Def}[Definici\'on 4.1, , Dai \cite{Dai}]
Sea una disciplina de servicio espec\'ifica. Cualquier l\'imite
$\left(\overline{Q}\left(\cdot\right),\overline{T}\left(\cdot\right)\right)$
en \ref{Eq.4.16} es un {\em flujo l\'imite} de la disciplina.
Cualquier soluci\'on (\ref{Eq.MF.1.3a})-(\ref{Eq.MF.6.3a}) es
llamado flujo soluci\'on de la disciplina. Se dice que el modelo de flujo l\'imite, modelo de flujo, de la disciplina de la cola es estable si existe una constante
$\delta>0$ que depende de $\mu,\alpha$ y $P$ solamente, tal que
cualquier flujo l\'imite con
$|\overline{Q}\left(0\right)|+|\overline{U}|+|\overline{V}|=1$, se
tiene que $\overline{Q}\left(\cdot+\delta\right)\equiv0$.
\end{Def}

\begin{Teo}[Teorema 4.2, Dai\cite{Dai}]\label{Tma.4.2.Dai}
Sea una disciplina fija para la cola, suponga que se cumplen las
condiciones (1.2)-(1.5). Si el modelo de flujo l\'imite de la
disciplina de la cola es estable, entonces la cadena de Markov $X$
que describe la din\'amica de la red bajo la disciplina es Harris
recurrente positiva.
\end{Teo}

Ahora se procede a escalar el espacio y el tiempo para reducir la
aparente fluctuaci\'on del modelo. Consid\'erese el proceso
\begin{equation}\label{Eq.3.7}
\overline{Q}^{x}\left(t\right)=\frac{1}{|x|}Q^{x}\left(|x|t\right)
\end{equation}
A este proceso se le conoce como el fluido escalado, y cualquier l\'imite $\overline{Q}^{x}\left(t\right)$ es llamado flujo l\'imite del proceso de longitud de la cola. Haciendo $|q|\rightarrow\infty$ mientras se mantiene el resto de las componentes fijas, cualquier punto l\'imite del proceso de longitud de la cola normalizado $\overline{Q}^{x}$ es soluci\'on del siguiente modelo de flujo.

\begin{Def}[Definici\'on 3.1, Dai y Meyn \cite{DaiSean}]
Un flujo l\'imite (retrasado) para una red bajo una disciplina de
servicio espec\'ifica se define como cualquier soluci\'on
 $\left(\overline{Q}\left(\cdot\right),\overline{T}\left(\cdot\right)\right)$ de las siguientes ecuaciones, donde
$\overline{Q}\left(t\right)=\left(\overline{Q}_{1}\left(t\right),\ldots,\overline{Q}_{K}\left(t\right)\right)^{'}$
y
$\overline{T}\left(t\right)=\left(\overline{T}_{1}\left(t\right),\ldots,\overline{T}_{K}\left(t\right)\right)^{'}$
\begin{equation}\label{Eq.3.8}
\overline{Q}_{k}\left(t\right)=\overline{Q}_{k}\left(0\right)+\alpha_{k}t-\mu_{k}\overline{T}_{k}\left(t\right)+\sum_{l=1}^{k}P_{lk}\mu_{l}\overline{T}_{l}\left(t\right)\\
\end{equation}
\begin{equation}\label{Eq.3.9}
\overline{Q}_{k}\left(t\right)\geq0\textrm{ para }k=1,2,\ldots,K,\\
\end{equation}
\begin{equation}\label{Eq.3.10}
\overline{T}_{k}\left(0\right)=0,\textrm{ y }\overline{T}_{k}\left(\cdot\right)\textrm{ es no decreciente},\\
\end{equation}
\begin{equation}\label{Eq.3.11}
\overline{I}_{i}\left(t\right)=t-\sum_{k\in C_{i}}\overline{T}_{k}\left(t\right)\textrm{ es no decreciente}\\
\end{equation}
\begin{equation}\label{Eq.3.12}
\overline{I}_{i}\left(\cdot\right)\textrm{ se incrementa al tiempo }t\textrm{ cuando }\sum_{k\in C_{i}}Q_{k}^{x}\left(t\right)dI_{i}^{x}\left(t\right)=0\\
\end{equation}
\begin{equation}\label{Eq.3.13}
\textrm{condiciones adicionales sobre
}\left(Q^{x}\left(\cdot\right),T^{x}\left(\cdot\right)\right)\textrm{
referentes a la disciplina de servicio}
\end{equation}
\end{Def}

Al conjunto de ecuaciones dadas en \ref{Eq.3.8}-\ref{Eq.3.13} se
le llama {\em Modelo de flujo} y al conjunto de todas las
soluciones del modelo de flujo
$\left(\overline{Q}\left(\cdot\right),\overline{T}
\left(\cdot\right)\right)$ se le denotar\'a por $\mathcal{Q}$.

Si se hace $|x|\rightarrow\infty$ sin restringir ninguna de las
componentes, tambi\'en se obtienen un modelo de flujo, pero en
este caso el residual de los procesos de arribo y servicio
introducen un retraso:

\begin{Def}[Definici\'on 3.2, Dai y Meyn \cite{DaiSean}]
El modelo de flujo retrasado de una disciplina de servicio en una
red con retraso
$\left(\overline{A}\left(0\right),\overline{B}\left(0\right)\right)\in\rea_{+}^{K+|A|}$
se define como el conjunto de ecuaciones dadas en
\ref{Eq.3.8}-\ref{Eq.3.13}, junto con la condici\'on:
\begin{equation}\label{CondAd.FluidModel}
\overline{Q}\left(t\right)=\overline{Q}\left(0\right)+\left(\alpha
t-\overline{A}\left(0\right)\right)^{+}-\left(I-P^{'}\right)M\left(\overline{T}\left(t\right)-\overline{B}\left(0\right)\right)^{+}
\end{equation}
\end{Def}

\begin{Def}[Definici\'on 3.3, Dai y Meyn \cite{DaiSean}]
El modelo de flujo es estable si existe un tiempo fijo $t_{0}$ tal
que $\overline{Q}\left(t\right)=0$, con $t\geq t_{0}$, para
cualquier $\overline{Q}\left(\cdot\right)\in\mathcal{Q}$ que
cumple con $|\overline{Q}\left(0\right)|=1$.
\end{Def}

El sigui	ente resultado se encuentra en Chen \cite{Chen}.
\begin{Lemma}[Lema 3.1, Dai y Meyn \cite{DaiSean}]
Si el modelo de flujo definido por \ref{Eq.3.8}-\ref{Eq.3.13} es
estable, entonces el modelo de flujo retrasado es tambi\'en
estable, es decir, existe $t_{0}>0$ tal que
$\overline{Q}\left(t\right)=0$ para cualquier $t\geq t_{0}$, para
cualquier soluci\'on del modelo de flujo retrasado cuya
condici\'on inicial $\overline{x}$ satisface que
$|\overline{x}|=|\overline{Q}\left(0\right)|+|\overline{A}\left(0\right)|+|\overline{B}\left(0\right)|\leq1$.
\end{Lemma}


\subsection{Modelo de Flujo Retrasado}\label{ModeloFlujoIntro}

Propiedades importantes para el modelo de flujo retrasado:

\begin{Prop}
 Sea $\left(\overline{Q},\overline{T},\overline{T}^{0}\right)$ un flujo l\'imite de \ref{Equation.4.4} y suponga que cuando $x\rightarrow\infty$ a lo largo de
una subsucesi\'on \[\left(\frac{1}{|x|}Q_{k}^{x}\left(0\right),\frac{1}{|x|}A_{k}^{x}\left(0\right),\frac{1}{|x|}B_{k}^{x}\left(0\right),\frac{1}{|x|}B_{k}^{x,0}\left(0\right)\right)\rightarrow\left(\overline{Q}_{k}\left(0\right),0,0,0\right)\] para $k=1,\ldots,K$. EL flujo l\'imite tiene las siguientes propiedades, donde las propiedades de la derivada se cumplen donde la derivada exista:
\begin{itemize}
 \item[i)] Los vectores de tiempo ocupado $\overline{T}\left(t\right)$ y $\overline{T}^{0}\left(t\right)$ son crecientes y continuas con $\overline{T}\left(0\right)=\overline{T}^{0}\left(0\right)=0$.
\item[ii)] Para todo $t\geq0$ \[\sum_{k=1}^{K}\left[\overline{T}_{k}\left(t\right)+\overline{T}_{k}^{0}\left(t\right)\right]=t\]
\item[iii)] Para todo $1\leq k\leq K$ \[\overline{Q}_{k}\left(t\right)=\overline{Q}_{k}\left(0\right)+\alpha_{k}t-\mu_{k}\overline{T}_{k}\left(t\right)\]
\item[iv)]  Para todo $1\leq k\leq K$ \[\dot{{\overline{T}}}_{k}\left(t\right)=\beta_{k}\] para $\overline{Q}_{k}\left(t\right)=0$.
\item[v)] Para todo $k,j$ \[\mu_{k}^{0}\overline{T}_{k}^{0}\left(t\right)=\mu_{j}^{0}\overline{T}_{j}^{0}\left(t\right)\]
\item[vi)]  Para todo $1\leq k\leq K$ \[\mu_{k}\dot{{\overline{T}}}_{k}\left(t\right)=l_{k}\mu_{k}^{0}\dot{{\overline{T}}}_{k}^{0}\left(t\right)\] para $\overline{Q}_{k}\left(t\right)>0$.
\end{itemize}
\end{Prop}

\begin{Lema}[Lema 3.1 \cite{Chen}]\label{Lema3.1}
Si el modelo de flujo es estable, definido por las ecuaciones (3.8)-(3.13), entonces el modelo de flujo retrasado tambi\'en es estable.
\end{Lema}

\begin{Teo}[Teorema 5.2 \cite{Chen}]\label{Tma.5.2}
Si el modelo de flujo lineal correspondiente a la red de cola es estable, entonces la red de colas es estable.
\end{Teo}

\begin{Teo}[Teorema 5.1 \cite{Chen}]\label{Tma.5.1.Chen}
La red de colas es estable si existe una constante $t_{0}$ que depende de $\left(\alpha,\mu,T,U\right)$ y $V$ que satisfagan las ecuaciones (5.1)-(5.5), $Z\left(t\right)=0$, para toda $t\geq t_{0}$.
\end{Teo}

\begin{Lema}[Lema 5.2 \cite{Gut}]\label{Lema.5.2.Gut}
Sea $\left\{\xi\left(k\right):k\in\ent\right\}$ sucesin de variables aleatorias i.i.d. con valores en
$\left(0,\infty\right)$, y sea $E\left(t\right)$ el proceso de conteo \[E\left(t\right)=max\left\{n\geq1:\xi\left(1\right)+\cdots+\xi\left(n-1\right)\leq t\right\}.\]
Si $E\left[\xi\left(1\right)\right]<\infty$, entonces para cualquier entero $r\geq1$
\begin{equation}
lim_{t\rightarrow\infty}\esp\left[\left(\frac{E\left(t\right)}{t}\right)^{r}\right]=\left(\frac{1}{E\left[\xi_{1}\right]}\right)^{r}
\end{equation}
de aqu\'i, bajo estas condiciones
\begin{itemize}
\item[a)] Para cualquier $t>0$, $sup_{t\geq\delta}\esp\left[\left(\frac{E\left(t\right)}{t}\right)^{r}\right]$
\item[b)] Las variables aleatorias $\left\{\left(\frac{E\left(t\right)}{t}\right)^{r}:t\geq1\right\}$ son uniformemente integrables.
\end{itemize}
\end{Lema}

\begin{Teo}[Teorema 5.1: Ley Fuerte para Procesos de Conteo \cite{Gut}]\label{Tma.5.1.Gut} Sea $0<\mu<\esp\left(X_{1}\right]\leq\infty$. entonces

\begin{itemize}
\item[a)] $\frac{N\left(t\right)}{t}\rightarrow\frac{1}{\mu}$
a.s., cuando $t\rightarrow\infty$.

\item[b)]$\esp\left[\frac{N\left(t\right)}{t}\right]^{r}\rightarrow\frac{1}{\mu^{r}}$, cuando $t\rightarrow\infty$ para todo $r>0$..
\end{itemize}
\end{Teo}

\begin{Prop}[Proposicin 5.1 \cite{DaiSean}]\label{Prop.5.1}
Suponga que los supuestos (A1) y (A2) se cumplen, adem\'as suponga que el modelo de flujo es estable. Entonces existe $t_{0}>0$ tal que
\begin{equation}\label{Eq.Prop.5.1}
lim_{|x|\rightarrow\infty}\frac{1}{|x|^{p+1}}\esp_{x}\left[|X\left(t_{0}|x|\right)|^{p+1}\right]=0.
\end{equation}

\end{Prop}


\begin{Prop}[Proposici\'on 5.3 \cite{DaiSean}]
Sea $X$ proceso de estados para la red de colas, y suponga que se cumplen los supuestos (A1) y (A2), entonces para alguna constante positiva $C_{p+1}<\infty$, $\delta>0$ y un conjunto compacto $C\subset X$.

\begin{equation}\label{Eq.5.4}
\esp_{x}\left[\int_{0}^{\tau_{C}\left(\delta\right)}\left(1+|X\left(t\right)|^{p}\right)dt\right]\leq C_{p+1}\left(1+|x|^{p+1}\right)
\end{equation}
\end{Prop}

\begin{Prop}[Proposici\'on 5.4 \cite{DaiSean}]
Sea $X$ un proceso de Markov Borel Derecho en $X$, sea $f:X\leftarrow\rea_{+}$ y defina para alguna $\delta>0$, y un conjunto cerrado $C\subset X$
\[V\left(x\right):=\esp_{x}\left[\int_{0}^{\tau_{C}\left(\delta\right)}f\left(X\left(t\right)\right)dt\right]\]
para $x\in X$. Si $V$ es finito en todas partes y uniformemente acotada en $C$, entonces existe $k<\infty$ tal que
\begin{equation}\label{Eq.5.11}
\frac{1}{t}\esp_{x}\left[V\left(x\right)\right]+\frac{1}{t}\int_{0}^{t}\esp_{x}\left[f\left(X\left(s\right)\right)ds\right]\leq\frac{1}{t}V\left(x\right)+k,
\end{equation}
para $x\in X$ y $t>0$.
\end{Prop}


\begin{Teo}[Teorema 5.5 \cite{DaiSean}]
Suponga que se cumplen (A1) y (A2), adem\'as suponga que el modelo de flujo es estable. Entonces existe una constante $k_{p}<\infty$ tal que
\begin{equation}\label{Eq.5.13}
\frac{1}{t}\int_{0}^{t}\esp_{x}\left[|Q\left(s\right)|^{p}\right]ds\leq k_{p}\left\{\frac{1}{t}|x|^{p+1}+1\right\}
\end{equation}
para $t\geq0$, $x\in X$. En particular para cada condici\'on inicial
\begin{equation}\label{Eq.5.14}
Limsup_{t\rightarrow\infty}\frac{1}{t}\int_{0}^{t}\esp_{x}\left[|Q\left(s\right)|^{p}\right]ds\leq k_{p}
\end{equation}
\end{Teo}

\begin{Teo}[Teorema 6.2\cite{DaiSean}]\label{Tma.6.2}
Suponga que se cumplen los supuestos (A1)-(A3) y que el modelo de flujo es estable, entonces se tiene que
\[\parallel P^{t}\left(c,\cdot\right)-\pi\left(\cdot\right)\parallel_{f_{p}}\rightarrow0\]
para $t\rightarrow\infty$ y $x\in X$. En particular para cada condici\'on inicial \[lim_{t\rightarrow\infty}\esp_{x}\left[\left|Q_{t}\right|^{p}\right]=\esp_{\pi}\left[\left|Q_{0}\right|^{p}\right]<\infty\]
\end{Teo}


\begin{Teo}[Teorema 6.3\cite{DaiSean}]\label{Tma.6.3}
Suponga que se cumplen los supuestos (A1)-(A3) y que el modelo de flujo es estable, entonces con $f\left(x\right)=f_{1}\left(x\right)$, se tiene que \[lim_{t\rightarrow\infty}t^{(p-1)\left|P^{t}\left(c,\cdot\right)-\pi\left(\cdot\right)\right|_{f}=0},\] para $x\in X$. En particular, para cada condici\'on inicial \[lim_{t\rightarrow\infty}t^{(p-1)\left|\esp_{x}\left[Q_{t}\right]-\esp_{\pi}\left[Q_{0}\right]\right|=0}.\]
\end{Teo}


Si $x$ es el n{\'u}mero de usuarios en la cola al comienzo del periodo de servicio y $N_{s}\left(x\right)=N\left(x\right)$ es el n{\'u}mero de usuarios que son atendidos con la pol{\'\i}tica $s$, {\'u}nica en nuestro caso, durante un periodo de servicio, entonces se asume que:
\begin{itemize}
\item[(S1.)]
\begin{equation}\label{S1}
lim_{x\rightarrow\infty}\esp\left[N\left(x\right)\right]=\overline{N}>0.
\end{equation}
\item[(S2.)]
\begin{equation}\label{S2}
\esp\left[N\left(x\right)\right]\leq \overline{N}, \end{equation}
para cualquier valor de $x$. \item La $n$-{\'e}sima ocurrencia va acompa{\~n}ada con el tiempo de cambio de longitud $\delta_{j,j+1}\left(n\right)$, independientes e id{\'e}nticamente distribuidas, con $\esp\left[\delta_{j,j+1}\left(1\right)\right]\geq0$. \item Se
define
\begin{equation}
\delta^{*}:=\sum_{j,j+1}\esp\left[\delta_{j,j+1}\left(1\right)\right].
\end{equation}

\item Los tiempos de inter-arribo a la cola $k$,son de la forma
$\left\{\xi_{k}\left(n\right)\right\}_{n\geq1}$, con la propiedad de que son independientes e id{\'e}nticamente distribuidos.

\item Los tiempos de servicio
$\left\{\eta_{k}\left(n\right)\right\}_{n\geq1}$ tienen la propiedad de ser independientes e id{\'e}nticamente distribuidos.

\item Se define la tasa de arribo a la $k$-{\'e}sima cola como $\lambda_{k}=1/\esp\left[\xi_{k}\left(1\right)\right]$ y adem{\'a}s se define

\item la tasa de servicio para la $k$-{\'e}sima cola como
$\mu_{k}=1/\esp\left[\eta_{k}\left(1\right)\right]$

\item tambi{\'e}n se define $\rho_{k}=\lambda_{k}/\mu_{k}$, donde es necesario que $\rho<1$ para cuestiones de estabilidad.

\item De las pol{\'\i}ticas posibles solamente consideraremos la pol{\'\i}tica cerrada (Gated).
\end{itemize}

Las Colas C\'iclicas se pueden describir por medio de un proceso de Markov $\left(X\left(t\right)\right)_{t\in\rea}$, donde el estado del sistema al tiempo $t\geq0$ est\'a dado por
\begin{equation}
X\left(t\right)=\left(Q\left(t\right),A\left(t\right),H\left(t\right),B\left(t\right),B^{0}\left(t\right),C\left(t\right)\right)
\end{equation}
definido en el espacio producto:
\begin{equation}
\mathcal{X}=\mathbb{Z}^{K}\times\rea_{+}^{K}\times\left(\left\{1,2,\ldots,K\right\}\times\left\{1,2,\ldots,S\right\}\right)^{M}\times\rea_{+}^{K}\times\rea_{+}^{K}\times\mathbb{Z}^{K},
\end{equation}

\begin{itemize}
\item $Q\left(t\right)=\left(Q_{k}\left(t\right),1\leq k\leq K\right)$, es el n\'umero de usuarios en la cola $k$, incluyendo aquellos que est\'an siendo atendidos provenientes de la $k$-\'esima cola.

\item $A\left(t\right)=\left(A_{k}\left(t\right),1\leq k\leq
 K\right)$, son los residuales de los tiempos de arribo en la cola $k$. \item $H\left(t\right)$ es el par ordenado que consiste en la cola que esta siendo atendida y la pol\'itica de servicio que se utilizar\'a.

\item $B\left(t\right)$ es el tiempo de servicio residual.

\item $B^{0}\left(t\right)$ es el tiempo residual del cambio de cola.

\item $C\left(t\right)$ indica el n\'umero de usuarios atendidos durante la visita del servidor a la cola dada en $H\left(t\right)$.
\end{itemize}

$A_{k}\left(t\right),B_{m}\left(t\right)$ y $B_{m}^{0}\left(t\right)$ se suponen continuas por la derecha y que satisfacen la propiedad fuerte de Markov, (\cite{Dai}).

\begin{itemize}
\item Los tiempos de interarribo a la cola $k$,son de la forma $\left\{\xi_{k}\left(n\right)\right\}_{n\geq1}$, con la propiedad de que son independientes e id{\'e}nticamente distribuidos.

\item Los tiempos de servicio $\left\{\eta_{k}\left(n\right)\right\}_{n\geq1}$ tienen la propiedad de ser independientes e id{\'e}nticamente distribuidos.

\item Se define la tasa de arribo a la $k$-{\'e}sima cola como $\lambda_{k}=1/\esp\left[\xi_{k}\left(1\right)\right]$ y adem{\'a}s se define

\item la tasa de servicio para la $k$-{\'e}sima cola como $\mu_{k}=1/\esp\left[\eta_{k}\left(1\right)\right]$.

\item tambi{\'e}n se define $\rho_{k}=\lambda_{k}/\mu_{k}$, donde es necesario que $\rho<1$ para cuestiones de estabilidad.

\item De las pol{\'\i}ticas posibles solamente consideraremos la pol{\'\i}tica cerrada (Gated).
\end{itemize}

Sup\'ongase que el sistema consta de varias colas a los cuales llegan uno o varios servidores a dar servicio a los usuarios
esperando en la cola. Si $x$ es el n\'umero de usuarios en la cola al comienzo del periodo de servicio y $N_{s}\left(x\right)=N\left(x\right)$ es el n\'umero de usuarios que son atendidos con la pol\'itica $s$, \'unica en nuestro caso, durante un periodo de servicio, entonces se asume que:
\begin{itemize}
\item[1)]\label{S1}$lim_{x\rightarrow\infty}\esp\left[N\left(x\right)\right]=\overline{N}>0$.
\item[2)]\label{S2}$\esp\left[N\left(x\right)\right]\leq\overline{N}$para cualquier valor de $x$.
\end{itemize}
La manera en que atiende el servidor $m$-\'esimo, en este caso en espec\'ifico solo lo ilustraremos con un s\'olo servidor, es la siguiente:
\begin{itemize}
\item Al t\'ermino de la visita a la cola $j$, el servidor se cambia a la cola $j^{'}$ con probabilidad $r_{j,j^{'}}^{m}=r_{j,j^{'}}$.

\item La $n$-\'esima ocurrencia va acompa\~nada con el tiempo de cambio de longitud $\delta_{j,j^{'}}\left(n\right)$, independientes e id\'enticamente distribuidas, con $\esp\left[\delta_{j,j^{'}}\left(1\right)\right]\geq0$.

\item Sea $\left\{p_{j}\right\}$ la distribuci\'on invariante estacionaria \'unica para la Cadena de Markov con matriz de transici\'on $\left(r_{j,j^{'}}\right)$.

\item Finalmente, se define
\begin{equation}
\delta^{*}:=\sum_{j,j^{'}}p_{j}r_{j,j^{'}}\esp\left[\delta_{j,j^{'}}\left(i\right)\right].
\end{equation}
\end{itemize}

Veamos un caso muy espec\'ifico en el cual los tiempos de arribo a cada una de las colas se comportan de acuerdo a un proceso Poisson de la forma $\left\{\xi_{k}\left(n\right)\right\}_{n\geq1}$, y los tiempos de servicio en cada una de las colas son variables aleatorias distribuidas exponencialmente e id\'enticamente distribuidas $\left\{\eta_{k}\left(n\right)\right\}_{n\geq1}$, donde ambos procesos adem\'as cumplen la condici\'on de ser independientes entre si. Para la $k$-\'esima cola se define la tasa de arribo a la como $\lambda_{k}=1/\esp\left[\xi_{k}\left(1\right)\right]$ y la tasa de servicio como $\mu_{k}=1/\esp\left[\eta_{k}\left(1\right)\right]$, finalmente se define la carga de la cola como $\rho_{k}=\lambda_{k}/\mu_{k}$, donde se pide que $\rho<1$, para garantizar la estabilidad del sistema.\\

Se denotar\'a por $Q_{k}\left(t\right)$ el n\'umero de usuarios en la cola $k$, $A_{k}\left(t\right)$ los residuales de los tiempos entre arribos a la cola $k$; para cada servidor $m$, se denota por $B_{m}\left(t\right)$ los residuales de los tiempos de servicio al tiempo $t$; $B_{m}^{0}\left(t\right)$ son los residuales de los tiempos de traslado de la cola $k$ a la pr\'oxima por atender, al tiempo $t$, finalmente sea $C_{m}\left(t\right)$ el n\'umero de usuarios atendidos durante la visita del servidor a la cola $k$ al tiempo $t$.\\

En este sentido el proceso para el sistema de visitas se puede definir como:

\begin{equation}\label{Esp.Edos.Down}
X\left(t\right)^{T}=\left(Q_{k}\left(t\right),A_{k}\left(t\right),B_{m}\left(t\right),B_{m}^{0}\left(t\right),C_{m}\left(t\right)\right)
\end{equation}
para $k=1,\ldots,K$ y $m=1,2,\ldots,M$. $X$ evoluciona en el espacio de estados: $X=\ent_{+}^{K}\times\rea_{+}^{K}\times\left(\left\{1,2,\ldots,K\right\}\times\left\{1,2,\ldots,S\right\}\right)^{M}\times\rea_{+}^{K}\times\ent_{+}^{K}$.\\

El sistema aqu\'i descrito debe de cumplir con los siguientes supuestos b\'asicos de un sistema de visitas:

Antes enunciemos los supuestos que regir\'an en la red.

\begin{itemize}
\item[A1)] $\xi_{1},\ldots,\xi_{K},\eta_{1},\ldots,\eta_{K}$ son mutuamente independientes y son sucesiones independientes e id\'enticamente distribuidas.

\item[A2)] Para alg\'un entero $p\geq1$
\begin{eqnarray*}
\esp\left[\xi_{l}\left(1\right)^{p+1}\right]<\infty\textrm{ para }l\in\mathcal{A}\textrm{ y }\\
\esp\left[\eta_{k}\left(1\right)^{p+1}\right]<\infty\textrm{ para
}k=1,\ldots,K.
\end{eqnarray*}
donde $\mathcal{A}$ es la clase de posibles arribos.

\item[A3)] Para $k=1,2,\ldots,K$ existe una funci\'on positiva $q_{k}\left(x\right)$ definida en $\rea_{+}$, y un entero $j_{k}$, tal que
\begin{eqnarray}
P\left(\xi_{k}\left(1\right)\geq x\right)>0\textrm{, para todo }x>0\\
P\left\{a\leq\sum_{i=1}^{j_{k}}\xi_{k}\left(i\right)\leq
b\right\}\geq\int_{a}^{b}q_{k}\left(x\right)dx, \textrm{ }0\leq
a<b.
\end{eqnarray}
\end{itemize}

En particular los procesos de tiempo entre arribos y de servicio considerados con fines de ilustraci\'on de la metodolog\'ia cumplen con el supuesto $A2)$ para $p=1$, es decir, ambos procesos tienen primer y segundo momento finito.

En lo que respecta al supuesto (A3), en Dai y Meyn \cite{DaiSean} hacen ver que este se puede sustituir por
\begin{itemize}
\item[A3')] Para el Proceso de Markov $X$, cada subconjunto compacto de $X$ es un conjunto peque\~no, ver definici\'on
\ref{Def.Cto.Peq.}.
\end{itemize}

Es por esta raz\'on que con la finalidad de poder hacer uso de $A3^{'})$ es necesario recurrir a los Procesos de Harris y en particular a los Procesos Harris Recurrente.


Consideremos el caso en el que se tienen varias colas a las cuales llegan uno o varios servidores para dar servicio a los usuarios que se encuentran presentes en la cola, como ya se mencion\'o hay varios tipos de pol\'iticas de servicio, incluso podr\'ia ocurrir que la manera en que atiende al resto de las colas sea distinta a como lo hizo en las anteriores.\\

Para ejemplificar los sistemas de visitas c\'iclicas se considerar\'a el caso en que a las colas los usuarios son atendidos con una s\'ola pol\'itica de servicio.\\


Si $\omega$ es el n\'umero de usuarios en la cola al comienzo del periodo de servicio y $N\left(\omega\right)$ es el n\'umero de usuarios que son atendidos con una pol\'itica en espec\'ifico durante el periodo de servicio, entonces se asume que:
\begin{itemize}
\item[1)]\label{S1}$lim_{\omega\rightarrow\infty}\esp\left[N\left(\omega\right)\right]=\overline{N}>0$;
\item[2)]\label{S2}$\esp\left[N\left(\omega\right)\right]\leq\overline{N}$
para cualquier valor de $\omega$.
\end{itemize}
La manera en que atiende el servidor $m$-\'esimo, es la siguiente:
\begin{itemize}
\item Al t\'ermino de la visita a la cola $j$, el servidor cambia a la cola $j^{'}$ con probabilidad $r_{j,j^{'}}^{m}$
\item La $n$-\'esima vez que el servidor cambia de la cola $j$ a $j'$, va acompa\~nada con el tiempo de cambio de longitud
$\delta_{j,j^{'}}^{m}\left(n\right)$, con $\delta_{j,j^{'}}^{m}\left(n\right)$, $n\geq1$, variables aleatorias independientes e id\'enticamente distribuidas, tales que $\esp\left[\delta_{j,j^{'}}^{m}\left(1\right)\right]\geq0$.

\item Sea $\left\{p_{j}^{m}\right\}$ la distribuci\'on invariante estacionaria \'unica para la Cadena de Markov con matriz de transici\'on $\left(r_{j,j^{'}}^{m}\right)$, se supone que \'esta existe.

\item Finalmente, se define el tiempo promedio total de traslado entre las colas como
\begin{equation}
\delta^{*}:=\sum_{j,j^{'}}p_{j}^{m}r_{j,j^{'}}^{m}\esp\left[\delta_{j,j^{'}}^{m}\left(i\right)\right].
\end{equation}
\end{itemize}

Consideremos el caso donde los tiempos entre arribo a cada una de las colas, $\left\{\xi_{k}\left(n\right)\right\}_{n\geq1}$ son variables aleatorias independientes a id\'enticamente distribuidas, y los tiempos de servicio en cada una de las colas se distribuyen de manera independiente e id\'enticamente distribuidas $\left\{\eta_{k}\left(n\right)\right\}_{n\geq1}$; adem\'as ambos procesos cumplen la condici\'on de ser independientes entre s\'i. Para la $k$ \'esima cola se define la tasa de arribo por $\lambda_{k}=1/\esp\left[\xi_{k}\left(1\right)\right]$ y la tasa de servicio como $\mu_{k}=1/\esp\left[\eta_{k}\left(1\right)\right]$, finalmente se define la carga de la cola como $\rho_{k}=\lambda_{k}/\mu_{k}$, donde se pide que $\rho=\sum_{k=1}^{K}\rho_{k}<1$, para garantizar la estabilidad del sistema, esto es cierto para las pol\'iticas de servicio exhaustiva y cerrada, ver Geetor \cite{Getoor}.\\

Si denotamos por
\begin{itemize}
\item $Q_{k}\left(t\right)$ el n\'umero de usuarios presentes en
la cola $k$ al tiempo $t$; \item $A_{k}\left(t\right)$ los
residuales de los tiempos entre arribos a la cola $k$; para cada
servidor $m$; \item $B_{m}\left(t\right)$ denota a los residuales
de los tiempos de servicio al tiempo $t$; \item
$B_{m}^{0}\left(t\right)$ los residuales de los tiempos de
traslado de la cola $k$ a la pr\'oxima por atender al tiempo $t$,

\item sea
$C_{m}\left(t\right)$ el n\'umero de usuarios atendidos durante la
visita del servidor a la cola $k$ al tiempo $t$.
\end{itemize}


En este sentido, el proceso para el sistema de visitas se puede definir como:

\begin{equation}\label{Esp.Edos.Down}
X\left(t\right)^{T}=\left(Q_{k}\left(t\right),A_{k}\left(t\right),B_{m}\left(t\right),B_{m}^{0}\left(t\right),C_{m}\left(t\right)\right),
\end{equation}
para $k=1,\ldots,K$ y $m=1,2,\ldots,M$, donde $T$ indica que es el transpuesto del vector que se est\'a definiendo. El proceso $X$ evoluciona en el espacio de estados: $\mathbb{X}=\ent_{+}^{K}\times\rea_{+}^{K}\times\left(\left\{1,2,\ldots,K\right\}\times\left\{1,2,\ldots,S\right\}\right)^{M}\times\rea_{+}^{K}\times\ent_{+}^{K}$.\\

El sistema aqu\'i descrito debe de cumplir con los siguientes supuestos b\'asicos de un sistema de visitas:
\begin{itemize}
\item[A1)] Los procesos $\xi_{1},\ldots,\xi_{K},\eta_{1},\ldots,\eta_{K}$ son mutuamente independientes y son sucesiones independientes e id\'enticamente
distribuidas.

\item[A2)] Para alg\'un entero $p\geq1$
\begin{eqnarray*}
\esp\left[\xi_{l}\left(1\right)^{p+1}\right]&<&\infty\textrm{ para }l=1,\ldots,K\textrm{ y }\\
\esp\left[\eta_{k}\left(1\right)^{p+1}\right]&<&\infty\textrm{
para }k=1,\ldots,K.
\end{eqnarray*}
donde $\mathcal{A}$ es la clase de posibles arribos.

\item[A3)] Para cada $k=1,2,\ldots,K$ existe una funci\'on positiva $q_{k}\left(\cdot\right)$ definida en $\rea_{+}$, y un entero $j_{k}$, tal que
\begin{eqnarray}
P\left(\xi_{k}\left(1\right)\geq x\right)&>&0\textrm{, para todo }x>0,\\
P\left\{a\leq\sum_{i=1}^{j_{k}}\xi_{k}\left(i\right)\leq
b\right\}&\geq&\int_{a}^{b}q_{k}\left(x\right)dx, \textrm{ }0\leq
a<b.
\end{eqnarray}
\end{itemize}

En lo que respecta al supuesto (A3), en Dai y Meyn \cite{DaiSean} hacen ver que este se puede sustituir por

\begin{itemize}
\item[A3')] Para el Proceso de Markov $X$, cada subconjunto compacto del espacio de estados de $X$ es un conjunto peque\~no, ver definici\'on \ref{Def.Cto.Peq.}.
\end{itemize}

Es por esta raz\'on que con la finalidad de poder hacer uso de $A3^{'})$ es necesario recurrir a los Procesos de Harris y en particular a los Procesos Harris Recurrente, ver \cite{Dai, DaiSean}.

%_______________________________________________________________________
\section{Procesos Harris Recurrente}
%_______________________________________________________________________

Por el supuesto (A1) conforme a Davis \cite{Davis}, se puede definir el proceso de saltos correspondiente de manera tal que satisfaga el supuesto (\ref{Sup3.1.Davis}), de hecho la demostraci\'on est\'a basada en la l\'inea de argumentaci\'on de Davis, (\cite{Davis}, p\'aginas 362-364).

Entonces se tiene un espacio de estados Markoviano. El espacio de Markov descrito en Dai y Meyn \cite{DaiSean}

\[\left(\Omega,\mathcal{F},\mathcal{F}_{t},X\left(t\right),\theta_{t},P_{x}\right)\] es un proceso de Borel Derecho (Sharpe \cite{Sharpe}) en el espacio de estados medible $\left(X,\mathcal{B}_{X}\right)$. El Proceso $X=\left\{X\left(t\right),t\geq0\right\}$ tiene trayectorias continuas por la derecha, est\'a definida en $\left(\Omega,\mathcal{F}\right)$ y est\'a adaptado a $\left\{\mathcal{F}_{t},t\geq0\right\}$; la colecci\'on $\left\{P_{x},x\in \mathbb{X}\right\}$ son medidas de probabilidad en $\left(\Omega,\mathcal{F}\right)$ tales que para todo $x\in \mathbb{X}$ \[P_{x}\left\{X\left(0\right)=x\right\}=1\] y \[E_{x}\left\{f\left(X\circ\theta_{t}\right)|\mathcal{F}_{t}\right\}=E_{X}\left(\tau\right)f\left(X\right)\] en $\left\{\tau<\infty\right\}$, $P_{x}$-c.s. Donde $\tau$ es un $\mathcal{F}_{t}$-tiempo de paro \[\left(X\circ\theta_{\tau}\right)\left(w\right)=\left\{X\left(\tau\left(w\right)+t,w\right),t\geq0\right\}\] y $f$ es una funci\'on de valores reales acotada y medible con la $\sigma$-algebra de Kolmogorov generada por los cilindros.\\

Sea $P^{t}\left(x,D\right)$, $D\in\mathcal{B}_{\mathbb{X}}$, $t\geq0$ probabilidad de transici\'on de $X$ definida como \[P^{t}\left(x,D\right)=P_{x}\left(X\left(t\right)\in D\right)\]


\begin{Def}
Una medida no cero $\pi$ en $\left(\mathbf{X},\mathcal{B}_{\mathbf{X}}\right)$ es {\bf invariante} para $X$ si $\pi$ es $\sigma$-finita y \[\pi\left(D\right)=\int_{\mathbf{X}}P^{t}\left(x,D\right)\pi\left(dx\right)\] para todo $D\in \mathcal{B}_{\mathbf{X}}$, con $t\geq0$.
\end{Def}

\begin{Def}
El proceso de Markov $X$ es llamado Harris recurrente si existe una medida de probabilidad $\nu$ en $\left(\mathbf{X},\mathcal{B}_{\mathbf{X}}\right)$, tal que si $\nu\left(D\right)>0$ y $D\in\mathcal{B}_{\mathbf{X}}$ \[P_{x}\left\{\tau_{D}<\infty\right\}\equiv1\] cuando $\tau_{D}=inf\left\{t\geq0:X_{t}\in D\right\}$.
\end{Def}

\begin{Note}
\begin{itemize}
\item[i)] Si $X$ es Harris recurrente, entonces existe una \'unica medida invariante $\pi$ (Getoor \cite{Getoor}).

\item[ii)] Si la medida invariante es finita, entonces puede normalizarse a una medida de probabilidad, en este caso se le
llama Proceso {\em Harris recurrente positivo}.

\item[iii)] Cuando $X$ es Harris recurrente positivo se dice que la disciplina de servicio es estable. En este caso $\pi$ denota la distribuci\'on estacionaria y hacemos \[P_{\pi}\left(\cdot\right)=\int_{\mathbf{X}}P_{x}\left(\cdot\right)\pi\left(dx\right)\] y se utiliza $E_{\pi}$ para denotar el operador esperanza correspondiente.
\end{itemize}
\end{Note}

\begin{Def}\label{Def.Cto.Peq.}
Un conjunto $D\in\mathcal{B_{\mathbf{X}}}$ es llamado peque\~no si existe un $t>0$, una medida de probabilidad $\nu$ en $\mathcal{B_{\mathbf{X}}}$, y un $\delta>0$ tal que \[P^{t}\left(x,A\right)\geq\delta\nu\left(A\right)\] para $x\in D,A\in\mathcal{B_{X}}$.
\end{Def}

La siguiente serie de resultados vienen enunciados y demostrados en Dai \cite{Dai}:
\begin{Lema}[Lema 3.1, Dai\cite{Dai}]
Sea $B$ conjunto peque\~no cerrado, supongamos que $P_{x}\left(\tau_{B}<\infty\right)\equiv1$ y que para alg\'un $\delta>0$ se cumple que
\begin{equation}\label{Eq.3.1}
\sup\esp_{x}\left[\tau_{B}\left(\delta\right)\right]<\infty,
\end{equation}
donde $\tau_{B}\left(\delta\right)=inf\left\{t\geq\delta:X\left(t\right)\in B\right\}$. Entonces, $X$ es un proceso Harris Recurrente Positivo.
\end{Lema}

\begin{Lema}[Lema 3.1, Dai \cite{Dai}]\label{Lema.3.}
Bajo el supuesto (A3), el conjunto $B=\left\{|x|\leq k\right\}$ es un conjunto peque\~no cerrado para cualquier $k>0$.
\end{Lema}

\begin{Teo}[Teorema 3.1, Dai\cite{Dai}]\label{Tma.3.1}
Si existe un $\delta>0$ tal que 
\begin{equation}
lim_{|x|\rightarrow\infty}\frac{1}{|x|}\esp|X^{x}\left(|x|\delta\right)|=0,
\end{equation}
entonces la ecuaci\'on (\ref{Eq.3.1}) se cumple para $B=\left\{|x|\leq k\right\}$ con alg\'un $k>0$. En particular, $X$ es Harris Recurrente Positivo.
\end{Teo}

\begin{Note}
En Meyn and Tweedie \cite{MeynTweedie} muestran que si $P_{x}\left\{\tau_{D}<\infty\right\}\equiv1$ incluso para solo un conjunto peque\~no, entonces el proceso es Harris Recurrente.
\end{Note}

Entonces, tenemos que el proceso $X$ es un proceso de Markov que cumple con los supuestos $A1)$-$A3)$, lo que falta de hacer es construir el Modelo de Flujo bas\'andonos en lo hasta ahora presentado.



Dado el proceso $X=\left\{X\left(t\right),t\geq0\right\}$ definido
en (\ref{Esp.Edos.Down}) que describe la din\'amica del sistema de
visitas c\'iclicas, si $U\left(t\right)$ es el residual de los
tiempos de llegada al tiempo $t$ entre dos usuarios consecutivos y
$V\left(t\right)$ es el residual de los tiempos de servicio al
tiempo $t$ para el usuario que est\'as siendo atendido por el
servidor. Sea $\mathbb{X}$ el espacio de estados que puede tomar
el proceso $X$.


\begin{Lema}[Lema 4.3, Dai\cite{Dai}]\label{Lema.4.3}
Sea $\left\{x_{n}\right\}\subset \mathbf{X}$ con
$|x_{n}|\rightarrow\infty$, conforme $n\rightarrow\infty$. Suponga
que
\[lim_{n\rightarrow\infty}\frac{1}{|x_{n}|}U\left(0\right)=\overline{U}_{k},\]
y
\[lim_{n\rightarrow\infty}\frac{1}{|x_{n}|}V\left(0\right)=\overline{V}_{k}.\]
\begin{itemize}
\item[a)] Conforme $n\rightarrow\infty$ casi seguramente,
\[lim_{n\rightarrow\infty}\frac{1}{|x_{n}|}U^{x_{n}}_{k}\left(|x_{n}|t\right)=\left(\overline{U}_{k}-t\right)^{+}\textrm{, u.o.c.}\]
y
\[lim_{n\rightarrow\infty}\frac{1}{|x_{n}|}V^{x_{n}}_{k}\left(|x_{n}|t\right)=\left(\overline{V}_{k}-t\right)^{+}.\]

\item[b)] Para cada $t\geq0$ fijo,
\[\left\{\frac{1}{|x_{n}|}U^{x_{n}}_{k}\left(|x_{n}|t\right),|x_{n}|\geq1\right\}\]
y
\[\left\{\frac{1}{|x_{n}|}V^{x_{n}}_{k}\left(|x_{n}|t\right),|x_{n}|\geq1\right\}\]
\end{itemize}
son uniformemente convergentes.
\end{Lema}

Sea $e$ es un vector de unos, $C$ es la matriz definida por
\[C_{ik}=\left\{\begin{array}{cc}
1,& S\left(k\right)=i,\\
0,& \textrm{ en otro caso}.\\
\end{array}\right.
\]
Es necesario enunciar el siguiente Teorema que se utilizar\'a para
el Teorema (\ref{Tma.4.2.Dai}):
\begin{Teo}[Teorema 4.1, Dai \cite{Dai}]
Considere una disciplina que cumpla la ley de conservaci\'on, para
casi todas las trayectorias muestrales $\omega$ y cualquier
sucesi\'on de estados iniciales $\left\{x_{n}\right\}\subset
\mathbf{X}$, con $|x_{n}|\rightarrow\infty$, existe una
subsucesi\'on $\left\{x_{n_{j}}\right\}$ con
$|x_{n_{j}}|\rightarrow\infty$ tal que
\begin{equation}\label{Eq.4.15}
\frac{1}{|x_{n_{j}}|}\left(Q^{x_{n_{j}}}\left(0\right),U^{x_{n_{j}}}\left(0\right),V^{x_{n_{j}}}\left(0\right)\right)\rightarrow\left(\overline{Q}\left(0\right),\overline{U},\overline{V}\right),
\end{equation}

\begin{equation}\label{Eq.4.16}
\frac{1}{|x_{n_{j}}|}\left(Q^{x_{n_{j}}}\left(|x_{n_{j}}|t\right),T^{x_{n_{j}}}\left(|x_{n_{j}}|t\right)\right)\rightarrow\left(\overline{Q}\left(t\right),\overline{T}\left(t\right)\right)\textrm{
u.o.c.}
\end{equation}

Adem\'as,
$\left(\overline{Q}\left(t\right),\overline{T}\left(t\right)\right)$
satisface las siguientes ecuaciones:
\begin{equation}\label{Eq.MF.1.3a}
\overline{Q}\left(t\right)=Q\left(0\right)+\left(\alpha
t-\overline{U}\right)^{+}-\left(I-P\right)^{'}M^{-1}\left(\overline{T}\left(t\right)-\overline{V}\right)^{+},
\end{equation}

\begin{equation}\label{Eq.MF.2.3a}
\overline{Q}\left(t\right)\geq0,\\
\end{equation}

\begin{equation}\label{Eq.MF.3.3a}
\overline{T}\left(t\right)\textrm{ es no decreciente y comienza en cero},\\
\end{equation}

\begin{equation}\label{Eq.MF.4.3a}
\overline{I}\left(t\right)=et-C\overline{T}\left(t\right)\textrm{
es no decreciente,}\\
\end{equation}

\begin{equation}\label{Eq.MF.5.3a}
\int_{0}^{\infty}\left(C\overline{Q}\left(t\right)\right)d\overline{I}\left(t\right)=0,\\
\end{equation}

\begin{equation}\label{Eq.MF.6.3a}
\textrm{Condiciones en
}\left(\overline{Q}\left(\cdot\right),\overline{T}\left(\cdot\right)\right)\textrm{
espec\'ificas de la disciplina de la cola,}
\end{equation}
\end{Teo}


Propiedades importantes para el modelo de flujo retrasado:

\begin{Prop}[Proposici\'on 4.2, Dai \cite{Dai}]
 Sea $\left(\overline{Q},\overline{T},\overline{T}^{0}\right)$ un flujo l\'imite de \ref{Eq.Punto.Limite}
 y suponga que cuando $x\rightarrow\infty$ a lo largo de una subsucesi\'on
\[\left(\frac{1}{|x|}Q_{k}^{x}\left(0\right),\frac{1}{|x|}A_{k}^{x}\left(0\right),\frac{1}{|x|}B_{k}^{x}\left(0\right),\frac{1}{|x|}B_{k}^{x,0}\left(0\right)\right)\rightarrow\left(\overline{Q}_{k}\left(0\right),0,0,0\right)\]
para $k=1,\ldots,K$. El flujo l\'imite tiene las siguientes
propiedades, donde las propiedades de la derivada se cumplen donde
la derivada exista:
\begin{itemize}
 \item[i)] Los vectores de tiempo ocupado $\overline{T}\left(t\right)$ y $\overline{T}^{0}\left(t\right)$ son crecientes y continuas con
$\overline{T}\left(0\right)=\overline{T}^{0}\left(0\right)=0$.
\item[ii)] Para todo $t\geq0$
\[\sum_{k=1}^{K}\left[\overline{T}_{k}\left(t\right)+\overline{T}_{k}^{0}\left(t\right)\right]=t.\]
\item[iii)] Para todo $1\leq k\leq K$
\[\overline{Q}_{k}\left(t\right)=\overline{Q}_{k}\left(0\right)+\alpha_{k}t-\mu_{k}\overline{T}_{k}\left(t\right).\]
\item[iv)]  Para todo $1\leq k\leq K$
\[\dot{{\overline{T}}}_{k}\left(t\right)=\rho_{k}\] para $\overline{Q}_{k}\left(t\right)=0$.
\item[v)] Para todo $k,j$
\[\mu_{k}^{0}\overline{T}_{k}^{0}\left(t\right)=\mu_{j}^{0}\overline{T}_{j}^{0}\left(t\right).\]
\item[vi)]  Para todo $1\leq k\leq K$
\[\mu_{k}\dot{{\overline{T}}}_{k}\left(t\right)=l_{k}\mu_{k}^{0}\dot{{\overline{T}}}_{k}^{0}\left(t\right),\] para $\overline{Q}_{k}\left(t\right)>0$.
\end{itemize}
\end{Prop}

\begin{Lema}[Lema 3.1, Chen \cite{Chen}]\label{Lema3.1}
Si el modelo de flujo es estable, definido por las ecuaciones
(3.8)-(3.13), entonces el modelo de flujo retrasado tambi\'en es
estable.
\end{Lema}

\begin{Lema}[Lema 5.2, Gut \cite{Gut}]\label{Lema.5.2.Gut}
Sea $\left\{\xi\left(k\right):k\in\ent\right\}$ sucesi\'on de
variables aleatorias i.i.d. con valores en
$\left(0,\infty\right)$, y sea $E\left(t\right)$ el proceso de
conteo
\[E\left(t\right)=max\left\{n\geq1:\xi\left(1\right)+\cdots+\xi\left(n-1\right)\leq t\right\}.\]
Si $E\left[\xi\left(1\right)\right]<\infty$, entonces para
cualquier entero $r\geq1$
\begin{equation}
lim_{t\rightarrow\infty}\esp\left[\left(\frac{E\left(t\right)}{t}\right)^{r}\right]=\left(\frac{1}{E\left[\xi_{1}\right]}\right)^{r},
\end{equation}
de aqu\'i, bajo estas condiciones
\begin{itemize}
\item[a)] Para cualquier $t>0$,
$sup_{t\geq\delta}\esp\left[\left(\frac{E\left(t\right)}{t}\right)^{r}\right]<\infty$.

\item[b)] Las variables aleatorias
$\left\{\left(\frac{E\left(t\right)}{t}\right)^{r}:t\geq1\right\}$
son uniformemente integrables.
\end{itemize}
\end{Lema}

\begin{Teo}[Teorema 5.1: Ley Fuerte para Procesos de Conteo, Gut
\cite{Gut}]\label{Tma.5.1.Gut} Sea
$0<\mu<\esp\left(X_{1}\right]\leq\infty$. entonces

\begin{itemize}
\item[a)] $\frac{N\left(t\right)}{t}\rightarrow\frac{1}{\mu}$
a.s., cuando $t\rightarrow\infty$.


\item[b)]$\esp\left[\frac{N\left(t\right)}{t}\right]^{r}\rightarrow\frac{1}{\mu^{r}}$,
cuando $t\rightarrow\infty$ para todo $r>0$.
\end{itemize}
\end{Teo}


\begin{Prop}[Proposici\'on 5.1, Dai y Sean \cite{DaiSean}]\label{Prop.5.1}
Suponga que los supuestos (A1) y (A2) se cumplen, adem\'as suponga
que el modelo de flujo es estable. Entonces existe $t_{0}>0$ tal
que
\begin{equation}\label{Eq.Prop.5.1}
lim_{|x|\rightarrow\infty}\frac{1}{|x|^{p+1}}\esp_{x}\left[|X\left(t_{0}|x|\right)|^{p+1}\right]=0.
\end{equation}

\end{Prop}


\begin{Prop}[Proposici\'on 5.3, Dai y Sean \cite{DaiSean}]\label{Prop.5.3.DaiSean}
Sea $X$ proceso de estados para la red de colas, y suponga que se
cumplen los supuestos (A1) y (A2), entonces para alguna constante
positiva $C_{p+1}<\infty$, $\delta>0$ y un conjunto compacto
$C\subset X$.

\begin{equation}\label{Eq.5.4}
\esp_{x}\left[\int_{0}^{\tau_{C}\left(\delta\right)}\left(1+|X\left(t\right)|^{p}\right)dt\right]\leq
C_{p+1}\left(1+|x|^{p+1}\right).
\end{equation}
\end{Prop}

\begin{Prop}[Proposici\'on 5.4, Dai y Sean \cite{DaiSean}]\label{Prop.5.4.DaiSean}
Sea $X$ un proceso de Markov Borel Derecho en $X$, sea
$f:X\leftarrow\rea_{+}$ y defina para alguna $\delta>0$, y un
conjunto cerrado $C\subset X$
\[V\left(x\right):=\esp_{x}\left[\int_{0}^{\tau_{C}\left(\delta\right)}f\left(X\left(t\right)\right)dt\right],\]
para $x\in X$. Si $V$ es finito en todas partes y uniformemente
acotada en $C$, entonces existe $k<\infty$ tal que
\begin{equation}\label{Eq.5.11}
\frac{1}{t}\esp_{x}\left[V\left(x\right)\right]+\frac{1}{t}\int_{0}^{t}\esp_{x}\left[f\left(X\left(s\right)\right)ds\right]\leq\frac{1}{t}V\left(x\right)+k,
\end{equation}
para $x\in X$ y $t>0$.
\end{Prop}


\begin{Teo}[Teorema 5.5, Dai y Sean  \cite{DaiSean}]
Suponga que se cumplen (A1) y (A2), adem\'as suponga que el modelo
de flujo es estable. Entonces existe una constante $k_{p}<\infty$
tal que
\begin{equation}\label{Eq.5.13}
\frac{1}{t}\int_{0}^{t}\esp_{x}\left[|Q\left(s\right)|^{p}\right]ds\leq
k_{p}\left\{\frac{1}{t}|x|^{p+1}+1\right\},
\end{equation}
para $t\geq0$, $x\in X$. En particular para cada condici\'on
inicial
\begin{equation}\label{Eq.5.14}
\limsup_{t\rightarrow\infty}\frac{1}{t}\int_{0}^{t}\esp_{x}\left[|Q\left(s\right)|^{p}\right]ds\leq
k_{p}.
\end{equation}
\end{Teo}

\begin{Teo}[Teorema 6.2 Dai y Sean \cite{DaiSean}]\label{Tma.6.2}
Suponga que se cumplen los supuestos (A1)-(A3) y que el modelo de
flujo es estable, entonces se tiene que
\[\parallel P^{t}\left(x,\cdot\right)-\pi\left(\cdot\right)\parallel_{f_{p}}\rightarrow0,\]
para $t\rightarrow\infty$ y $x\in X$. En particular para cada
condici\'on inicial
\[lim_{t\rightarrow\infty}\esp_{x}\left[\left|Q_{t}\right|^{p}\right]=\esp_{\pi}\left[\left|Q_{0}\right|^{p}\right]<\infty,\]
\end{Teo}

donde

\begin{eqnarray*}
\parallel
P^{t}\left(c,\cdot\right)-\pi\left(\cdot\right)\parallel_{f}=sup_{|g\leq
f|}|\int\pi\left(dy\right)g\left(y\right)-\int
P^{t}\left(x,dy\right)g\left(y\right)|,
\end{eqnarray*}
para $x\in\mathbb{X}$.

\begin{Teo}[Teorema 6.3, Dai y Sean \cite{DaiSean}]\label{Tma.6.3}
Suponga que se cumplen los supuestos (A1)-(A3) y que el modelo de
flujo es estable, entonces con
$f\left(x\right)=f_{1}\left(x\right)$, se tiene que
\[lim_{t\rightarrow\infty}t^{(p-1)}\left|P^{t}\left(c,\cdot\right)-\pi\left(\cdot\right)\right|_{f}=0,\]
para $x\in X$. En particular, para cada condici\'on inicial
\[lim_{t\rightarrow\infty}t^{(p-1)}\left|\esp_{x}\left[Q_{t}\right]-\esp_{\pi}\left[Q_{0}\right]\right|=0.\]
\end{Teo}



\begin{Prop}[Proposici\'on 5.1, Dai y Meyn \cite{DaiSean}]\label{Prop.5.1.DaiSean}
Suponga que los supuestos A1) y A2) son ciertos y que el modelo de
flujo es estable. Entonces existe $t_{0}>0$ tal que
\begin{equation}
lim_{|x|\rightarrow\infty}\frac{1}{|x|^{p+1}}\esp_{x}\left[|X\left(t_{0}|x|\right)|^{p+1}\right]=0.
\end{equation}
\end{Prop}


\begin{Teo}[Teorema 5.5, Dai y Meyn \cite{DaiSean}]\label{Tma.5.5.DaiSean}
Suponga que los supuestos A1) y A2) se cumplen y que el modelo de
flujo es estable. Entonces existe una constante $\kappa_{p}$ tal
que
\begin{equation}
\frac{1}{t}\int_{0}^{t}\esp_{x}\left[|Q\left(s\right)|^{p}\right]ds\leq\kappa_{p}\left\{\frac{1}{t}|x|^{p+1}+1\right\},
\end{equation}
para $t>0$ y $x\in X$. En particular, para cada condici\'on
inicial
\begin{eqnarray*}
\limsup_{t\rightarrow\infty}\frac{1}{t}\int_{0}^{t}\esp_{x}\left[|Q\left(s\right)|^{p}\right]ds\leq\kappa_{p}.
\end{eqnarray*}
\end{Teo}


\begin{Teo}[Teorema 6.4, Dai y Meyn \cite{DaiSean}]\label{Tma.6.4.DaiSean}
Suponga que se cumplen los supuestos A1), A2) y A3) y que el
modelo de flujo es estable. Sea $\nu$ cualquier distribuci\'on de
probabilidad en
$\left(\mathbb{X},\mathcal{B}_{\mathbb{X}}\right)$, y $\pi$ la
distribuci\'on estacionaria de $X$.
\begin{itemize}
\item[i)] Para cualquier $f:X\leftarrow\rea_{+}$
\begin{equation}
\lim_{t\rightarrow\infty}\frac{1}{t}\int_{o}^{t}f\left(X\left(s\right)\right)ds=\pi\left(f\right):=\int
f\left(x\right)\pi\left(dx\right),
\end{equation}
$\prob$-c.s.

\item[ii)] Para cualquier $f:X\leftarrow\rea_{+}$ con
$\pi\left(|f|\right)<\infty$, la ecuaci\'on anterior se cumple.
\end{itemize}
\end{Teo}

\begin{Teo}[Teorema 2.2, Down \cite{Down}]\label{Tma2.2.Down}
Suponga que el fluido modelo es inestable en el sentido de que
para alguna $\epsilon_{0},c_{0}\geq0$,
\begin{equation}\label{Eq.Inestability}
|Q\left(T\right)|\geq\epsilon_{0}T-c_{0}\textrm{,   }T\geq0,
\end{equation}
para cualquier condici\'on inicial $Q\left(0\right)$, con
$|Q\left(0\right)|=1$. Entonces para cualquier $0<q\leq1$, existe
$B<0$ tal que para cualquier $|x|\geq B$,
\begin{equation}
\prob_{x}\left\{\mathbb{X}\rightarrow\infty\right\}\geq q.
\end{equation}
\end{Teo}

\begin{Dem}[Teorema \ref{Tma2.1.Down}] La demostraci\'on de este
teorema se da a continuaci\'on:\\
\begin{itemize}
\item[i)] Utilizando la proposici\'on \ref{Prop.5.3.DaiSean} se
tiene que la proposici\'on \ref{Prop.5.4.DaiSean} es cierta para
$f\left(x\right)=1+|x|^{p}$.

\item[i)] es consecuencia directa del Teorema \ref{Tma.6.2}.

\item[iii)] ver la demostraci\'on dada en Dai y Sean
\cite{DaiSean} p\'aginas 1901-1902.

\item[iv)] ver Dai y Sean \cite{DaiSean} p\'aginas 1902-1903 \'o
\cite{MeynTweedie2}.
\end{itemize}
\end{Dem}


Para cada $k$ y cada $n$ se define

\numberwithin{equation}{section}
\begin{equation}
\Phi^{k}\left(n\right):=\sum_{i=1}^{n}\phi^{k}\left(i\right).
\end{equation}

suponiendo que el estado inicial de la red es
$x=\left(q,a,b\right)\in X$, entonces para cada $k$

\begin{eqnarray}
E_{k}^{x}\left(t\right):=\max\left\{n\geq0:A_{k}^{x}\left(0\right)+\psi_{k}\left(1\right)+\cdots+\psi_{k}\left(n-1\right)\leq t\right\}\\
S_{k}^{x}\left(t\right):=\max\left\{n\geq0:B_{k}^{x}\left(0\right)+\eta_{k}\left(1\right)+\cdots+\eta_{k}\left(n-1\right)\leq
t\right\}
\end{eqnarray}

Sea $T_{k}^{x}\left(t\right)$ el tiempo acumulado que el servidor
$s\left(k\right)$ ha utilizado en los usuarios de la clase $k$ en
el intervalo $\left[0,t\right]$. Entonces se tienen las siguientes
ecuaciones:

\begin{equation}
Q_{k}^{x}\left(t\right)=Q_{k}^{x}\left(0\right)+E_{k}^{x}\left(t\right)+\sum_{l=1}^{k}\Phi_{k}^{l}S_{l}^{x}\left(T_{l}^{x}\right)-S_{k}^{x}\left(T_{k}^{x}\right)\\
\end{equation}
\begin{equation}
Q^{x}\left(t\right)=\left(Q^{x}_{1}\left(t\right),\ldots,Q^{x}_{K}\left(t\right)\right)^{'}\geq0,\\
\end{equation}
\begin{equation}
T^{x}\left(t\right)=\left(T^{x}_{1}\left(t\right),\ldots,T^{x}_{K}\left(t\right)\right)^{'}\geq0,\textrm{ es no decreciente}\\
\end{equation}
\begin{equation}
I_{i}^{x}\left(t\right)=t-\sum_{k\in C_{i}}T_{k}^{x}\left(t\right)\textrm{ es no decreciente}\\
\end{equation}
\begin{equation}
\int_{0}^{\infty}\sum_{k\in C_{i}}Q_{k}^{x}\left(t\right)dI_{i}^{x}\left(t\right)=0\\
\end{equation}
\begin{equation}
\textrm{condiciones adicionales sobre
}\left(Q^{x}\left(\cdot\right),T^{x}\left(\cdot\right)\right)\textrm{
referentes a la disciplina de servicio}
\end{equation}

Para reducir la fluctuaci\'on del modelo se escala tanto el
espacio como el tiempo, entonces se tiene el proceso:

\begin{equation}
\overline{Q}^{x}\left(t\right)=\frac{1}{|x|}Q^{x}\left(|x|t\right)
\end{equation}
Cualquier l\'imite $\overline{Q}\left(t\right)$ es llamado un
flujo l\'imite del proceso longitud de la cola. Si se hace
$|q|\rightarrow\infty$ y se mantienen las componentes restantes
fijas, de la condici\'on inicial $x$, cualquier punto l\'imite del
proceso normalizado $\overline{Q}^{x}$ es una soluci\'on del
siguiente modelo de flujo, ver \cite{Dai}.

\begin{Def}
Un flujo l\'imite (retrasado) para una red bajo una disciplina de
servicio espec\'ifica se define como cualquier soluci\'on
 $\left(Q^{x}\left(\cdot\right),T^{x}\left(\cdot\right)\right)$ de las siguientes ecuaciones, donde
$\overline{Q}\left(t\right)=\left(\overline{Q}_{1}\left(t\right),\ldots,\overline{Q}_{K}\left(t\right)\right)^{'}$
y
$\overline{T}\left(t\right)=\left(\overline{T}_{1}\left(t\right),\ldots,\overline{T}_{K}\left(t\right)\right)^{'}$
\begin{equation}\label{Eq.3.8}
\overline{Q}_{k}\left(t\right)=\overline{Q}_{k}\left(0\right)+\alpha_{k}t-\mu_{k}\overline{T}_{k}\left(t\right)+\sum_{l=1}^{k}P_{lk}\mu_{l}\overline{T}_{l}\left(t\right)\\
\end{equation}
\begin{equation}\label{Eq.3.9}
\overline{Q}_{k}\left(t\right)\geq0\textrm{ para }k=1,2,\ldots,K,\\
\end{equation}
\begin{equation}\label{Eq.3.10}
\overline{T}_{k}\left(0\right)=0,\textrm{ y }\overline{T}_{k}\left(\cdot\right)\textrm{ es no decreciente},\\
\end{equation}
\begin{equation}\label{Eq.3.11}
\overline{I}_{i}\left(t\right)=t-\sum_{k\in C_{i}}\overline{T}_{k}\left(t\right)\textrm{ es no decreciente}\\
\end{equation}
\begin{equation}\label{Eq.3.12}
\overline{I}_{i}\left(\cdot\right)\textrm{ se incrementa al tiempo}t\textrm{ cuando }\sum_{k\in C_{i}}Q_{k}^{x}\left(t\right)dI_{i}^{x}\left(t\right)=0\\
\end{equation}
\begin{equation}\label{Eq.3.13}
\textrm{condiciones adicionales sobre
}\left(Q^{x}\left(\cdot\right),T^{x}\left(\cdot\right)\right)\textrm{
referentes a la disciplina de servicio}
\end{equation}
\end{Def}

Al conjunto de ecuaciones dadas en \ref{Eq.3.8}-\ref{Eq.3.13} se
le llama {\em Modelo de flujo} y al conjunto de todas las
soluciones del modelo de flujo
$\left(\overline{Q}\left(\cdot\right),\overline{T}
\left(\cdot\right)\right)$ se le denotar\'a por $\mathcal{Q}$.

Si se hace $|x|\rightarrow\infty$ sin restringir ninguna de las
componentes, tambi\'en se obtienen un modelo de flujo, pero en
este caso el residual de los procesos de arribo y servicio
introducen un retraso:

\begin{Def}
El modelo de flujo retrasado de una disciplina de servicio en una
red con retraso
$\left(\overline{A}\left(0\right),\overline{B}\left(0\right)\right)\in\rea_{+}^{K+|A|}$
se define como el conjunto de ecuaciones dadas en
\ref{Eq.3.8}-\ref{Eq.3.13}, junto con la condici\'on:
\begin{equation}\label{CondAd.FluidModel}
\overline{Q}\left(t\right)=\overline{Q}\left(0\right)+\left(\alpha
t-\overline{A}\left(0\right)\right)^{+}-\left(I-P^{'}\right)M\left(\overline{T}\left(t\right)-\overline{B}\left(0\right)\right)^{+}
\end{equation}
\end{Def}

\begin{Def}
El modelo de flujo es estable si existe un tiempo fijo $t_{0}$ tal
que $\overline{Q}\left(t\right)=0$, con $t\geq t_{0}$, para
cualquier $\overline{Q}\left(\cdot\right)\in\mathcal{Q}$ que
cumple con $|\overline{Q}\left(0\right)|=1$.
\end{Def}

El siguiente resultado se encuentra en \cite{Chen}.
\begin{Lemma}
Si el modelo de flujo definido por \ref{Eq.3.8}-\ref{Eq.3.13} es
estable, entonces el modelo de flujo retrasado es tambi\'en
estable, es decir, existe $t_{0}>0$ tal que
$\overline{Q}\left(t\right)=0$ para cualquier $t\geq t_{0}$, para
cualquier soluci\'on del modelo de flujo retrasado cuya
condici\'on inicial $\overline{x}$ satisface que
$|overline{x}|=|\overline{Q}\left(0\right)|+|\overline{A}\left(0\right)|+|\overline{B}\left(0\right)|\leq1$.
\end{Lemma}

Supuestos necesarios sobre la red

\begin{Sup}
\begin{itemize}
\item[A1)] $\psi_{1},\ldots,\psi_{K},\eta_{1},\ldots,\eta_{K}$ son
mutuamente independientes y son sucesiones independientes e
id\'enticamente distribuidas.

\item[A2)] Para alg\'un entero $p\geq1$
\begin{eqnarray*}
\esp\left[\psi_{l}\left(1\right)^{p+1}\right]<\infty\textrm{ para }l\in\mathcal{A}\textrm{ y }\\
\esp\left[\eta_{k}\left(1\right)^{p+1}\right]<\infty\textrm{ para
}k=1,\ldots,K.
\end{eqnarray*}
\item[A3)] El conjunto $\left\{x\in X:|x|=0\right\}$ es un
singleton, y para cada $k\in\mathcal{A}$, existe una funci\'on
positiva $q_{k}\left(x\right)$ definida en $\rea_{+}$, y un entero
$j_{k}$, tal que
\begin{eqnarray}
P\left(\psi_{k}\left(1\right)\geq x\right)>0\textrm{, para todo }x>0\\
P\left(\psi_{k}\left(1\right)+\ldots\psi_{k}\left(j_{k}\right)\in dx\right)\geq q_{k}\left(x\right)dx0\textrm{ y }\\
\int_{0}^{\infty}q_{k}\left(x\right)dx>0
\end{eqnarray}
\end{itemize}
\end{Sup}

El argumento dado en \cite{MaynDown} en el lema
\ref{Lema.34.MeynDown} se puede aplicar para deducir que todos los
subconjuntos compactos de $X$ son peque\~nos.Entonces la
condici\'on $A3)$ se puede generalizar a
\begin{itemize}
\item[A3')] Para el proceso de Markov $X$, cada subconjunto
compacto de $X$ es peque\~no.
\end{itemize}

\begin{Teo}\label{Tma.4.1}
Suponga que el modelo de flujo para una disciplina de servicio es
estable, y suponga adem\'as que las condiciones A1) y A2) se
satisfacen. Entonces:
\begin{itemize}
\item[i)] Para alguna constante $\kappa_{p}$, y para cada
condici\'on inicial $x\in X$
\begin{equation}
\limsup_{t\rightarrow\infty}\frac{1}{t}\int_{0}^{t}\esp_{x}\left[|Q\left(t\right)|^{p}\right]ds\leq\kappa_{p}
\end{equation}
donde $p$ es el entero dado por A2). Suponga adem\'as que A3) o A3')
se cumple, entonces la disciplina de servicio es estable y adem\'as
para cada condici\'on inicial se tiene lo siguiente: \item[ii)] Los
momentos transitorios convergen a sus valores en estado
estacionario:
\begin{equation}
\lim_{t\rightarrow\infty}\esp_{x}\left[Q_{k}\left(t\right)^{r}\right]=\esp_{\pi}\left[Q_{k}\left(0\right)^{r}\right]\leq\kappa_{r}
\end{equation}
para $r=1,\ldots,p$ y $k=1,\ldots,K$. \item[iii)] EL primer
momento converge con raz\'on $t^{p-1}$:
\begin{equation}
\lim_{t\rightarrow\infty}t^{p-1}|\esp_{x}\left[Q\left(t\right)\right]-\esp_{\pi}\left[Q\left(0\right)\right]|=0.
\end{equation}
\item[iv)] Se cumple la Ley Fuerte de los Grandes N\'umeros:
\begin{equation}
\lim_{t\rightarrow\infty}\frac{1}{t}\int_{0}^{t}Q_{k}^{r}\left(s\right)ds=\esp_{\pi}\left[Q_{k}\left(0\right)^{r}\right]
\end{equation}
$\prob$-c.s., para $r=1,\ldots,p$ y $k=1,\ldots,K$.
\end{itemize}
\end{Teo}
\begin{Dem}
La demostraci\'on de este resultado se da aplicando los teoremas
\ref{Tma.5.5}, \ref{Tma.6.2}, \ref{Tma.6.3} y \ref{Tma.6.4}
\end{Dem}


Definimos un proceso de estados para la red que depende de la
pol\'itica de servicio utilizada. Bajo cualquier {\em preemptive
buffer priority} disciplina de servicio, el estado
$\mathbb{X}\left(t\right)$ a cualquier tiempo $t$ puede definirse
como
\begin{equation}\label{Eq.Esp.Estados}
\mathbb{X}\left(t\right)=\left(Q_{k}\left(t\right),A_{l}\left(t\right),B_{k}\left(t\right):k=1,2,\ldots,K,l\in\mathcal{A}\right)
\end{equation}
donde $Q_{k}\left(t\right)$ es la longitud de la cola para los
usuarios de la clase $k$, incluyendo aquellos que est\'an siendo
atendidos, $B_{k}\left(t\right)$ son los tiempos de servicio
residuales para los usuarios de la clase $k$ que est\'an en
servicio. Los tiempos de arribo residuales, que son iguales al
tiempo que queda hasta que el pr\'oximo usuario de la clase $k$
llega, se denotan por $A_{k}\left(t\right)$. Tanto
$B_{k}\left(t\right)$ como $A_{k}\left(t\right)$ se suponen
continuos por la derecha.

Sea $\mathbb{X}$ el espacio de estados para el proceso de estados
que por definici\'on es igual  al conjunto de posibles valores
para el estado $\mathbb{X}\left(t\right)$, y sea
$x=\left(q,a,b\right)$ un estado gen\'erico en $\mathbb{X}$, la
componente $q$ determina la posici\'on del usuario en la red,
$|q|$ denota la longitud total de la cola en la red.

Para un estado $x=\left(q,a,b\right)\in\mathbb{X}$ definimos la
{\em norma} de $x$ como $\left\|x\right|=|q|+|a|+|b|$. En
\cite{Dai} se muestra que para una amplia serie de disciplinas de
servicio el proceso $\mathbb{X}$ es un Proceso Fuerte de Markov, y
por tanto se puede asumir que
\[\left(\left(\Omega,\mathcal{F}\right),\mathcal{F}_{t},\mathbb{X}\left(t\right),\theta_{t},P_{x}\right)\]
es un proceso de Borel Derecho en el espacio de estadio medible
$\left(\mathbb{X},\mathcal{B}_{\mathbb{X}}\right)$. El Proceso
$X=\left\{\mathbb{X}\left(t\right),t\geq0\right\}$ tiene
trayectorias continuas por la derecha, est definida en
$\left(\Omega,\mathcal{F}\right)$ y est adaptado a
$\left\{\mathcal{F}_{t},t\geq0\right\}$; $\left\{P_{x},x\in
X\right\}$ son medidas de probabilidad en
$\left(\Omega,\mathcal{F}\right)$ tales que para todo $x\in X$
\[P_{x}\left\{\mathbb{X}\left(0\right)=x\right\}=1\] y
\[E_{x}\left\{f\left(X\circ\theta_{t}\right)|\mathcal{F}_{t}\right\}=E_{X}\left(\tau\right)f\left(X\right)\]
en $\left\{\tau<\infty\right\}$, $P_{x}$-c.s. Donde $\tau$ es un
$\mathcal{F}_{t}$-tiempo de paro
\[\left(X\circ\theta_{\tau}\right)\left(w\right)=\left\{\mathbb{X}\left(\tau\left(w\right)+t,w\right),t\geq0\right\}\]
y $f$ es una funci\'on de valores reales acotada y medible con la
sigma algebra de Kolmogorov generada por los cilindros.

Sea $P^{t}\left(x,D\right)$, $D\in\mathcal{B}_{\mathbb{X}}$,
$t\geq0$ probabilidad de transici\'on de $X$ definida como
\[P^{t}\left(x,D\right)=P_{x}\left(\mathbb{X}\left(t\right)\in
D\right)\]

\begin{Def}
Una medida no cero $\pi$ en
$\left(\mathbb{X},\mathcal{B}_{\mathbb{X}}\right)$ es {\em
invariante} para $X$ si $\pi$ es $\sigma$-finita y
\[\pi\left(D\right)=\int_{X}P^{t}\left(x,D\right)\pi\left(dx\right)\]
para todo $D\in \mathcal{B}_{\mathbb{X}}$, con $t\geq0$.
\end{Def}

\begin{Def}
El proceso de Markov $X$ es llamado {\em Harris recurrente} si
existe una medida de probabilidad $\nu$ en
$\left(\mathbb{X},\mathcal{B}_{\mathbb{X}}\right)$, tal que si
$\nu\left(D\right)>0$ y $D\in\mathcal{B}_{\mathbb{X}}$
\[P_{x}\left\{\tau_{D}<\infty\right\}\equiv1\] cuando
$\tau_{D}=\inf\left\{t\geq0:\mathbb{X}_{t}\in D\right\}$.
\end{Def}

\begin{itemize}
\item Si $X$ es Harris recurrente, entonces una \'unica medida
invariante $\pi$ existe (\cite{Getoor}). \item Si la medida
invariante es finita, entonces puede normalizarse a una medida de
probabilidad, en este caso se le llama {\em Harris recurrente
positiva}. \item Cuando $X$ es Harris recurrente positivo se dice
que la disciplina de servicio es estable. En este caso $\pi$
denota la ditribuci\'on estacionaria y hacemos
\[P_{\pi}\left(\cdot\right)[=\int_{X}P_{x}\left(\cdot\right)\pi\left(dx\right)\]
y se utiliza $E_{\pi}$ para denotar el operador esperanza
correspondiente, as, el proceso
$X=\left\{\mathbb{X}\left(t\right),t\geq0\right\}$ es un proceso
estrictamente estacionario bajo $P_{\pi}$
\end{itemize}

\begin{Def}
Un conjunto $D\in\mathcal{B}_\mathbb{X}$ es llamado peque\~no si
existe un $t>0$, una medida de probabilidad $\nu$ en
$\mathcal{B}_\mathbb{X}$, y un $\delta>0$ tal que
\[P^{t}\left(x,A\right)\geq\delta\nu\left(A\right)\] para $x\in
D,A\in\mathcal{B}_\mathbb{X}$.\footnote{En \cite{MeynTweedie}
muestran que si $P_{x}\left\{\tau_{D}<\infty\right\}\equiv1$
solamente para uno conjunto peque\~no, entonces el proceso es
Harris recurrente}
\end{Def}

El modelo est\'a compuesto por $c$ colas de capacidad infinita,
etiquetadas de $1$ a $c$ las cuales son atendidas por $s$
servidores. Los servidores atienden de acuerdo a una cadena de
Markov independiente $\left(X^{i}_{n}\right)_{n}$ con $1\leq i\leq
s$ y $n\in\left\{1,2,\ldots,c\right\}$ con la misma matriz de
transici\'on $r_{k,l}$ y \'unica medida invariante
$\left(p_{k}\right)$. Cada servidor permanece atendiendo en la
cola un periodo llamado de visita y determinada por la pol\'itica de
servicio asignada a la cola.

Los usuarios llegan a la cola $k$ con una tasa $\lambda_{k}$ y son
atendidos a una raz\'on $\mu_{k}$. Las sucesiones de tiempos de
interarribo $\left(\tau_{k}\left(n\right)\right)_{n}$, la de
tiempos de servicio
$\left(\sigma_{k}^{i}\left(n\right)\right)_{n}$ y la de tiempos de
cambio $\left(\sigma_{k,l}^{0,i}\left(n\right)\right)_{n}$
requeridas en la cola $k$ para el servidor $i$ son sucesiones
independientes e id\'enticamente distribuidas con distribuci\'on
general independiente de $i$, con media
$\sigma_{k}=\frac{1}{\mu_{k}}$, respectivamente
$\sigma_{k,l}^{0}=\frac{1}{\mu_{k,l}^{0}}$, e independiente de las
cadenas de Markov $\left(X^{i}_{n}\right)_{n}$. Adem\'as se supone
que los tiempos de interarribo se asume son acotados, para cada
$\rho_{k}=\lambda_{k}\sigma_{k}<s$ para asegurar la estabilidad de
la cola $k$ cuando opera como una cola $M/GM/1$.


Una pol\'itica de servicio determina el n\'umero de usuarios que ser\'an
atendidos sin interrupci\'on en periodo de servicio por los
servidores que atienden a la cola. Para un solo servidor esta se
define a trav\'es de una funci\'on $f$ donde $f\left(x,a\right)$ es el
n\'umero de usuarios que son atendidos sin interrupci\'on cuando el
servidor llega a la cola y encuentra $x$ usuarios esperando dado
el tiempo transcurrido de interarribo $a$. Sea $v\left(x,a\right)$
la duraci\'on del periodo de servicio para una sola condici\'on
inicial $\left(x,a\right)$.

Las pol\'iticas de servicio consideradas satisfacen las siguientes
propiedades:

\begin{itemize}
\item[i)] Hay conservaci\'on del trabajo, es decir
\[v\left(x,a\right)=\sum_{l=1}^{f\left(x,a\right)}\sigma\left(l\right)\]
con $f\left(0,a\right)=v\left(0,a\right)=0$, donde
$\left(\sigma\left(l\right)\right)_{l}$ es una sucesi\'on
independiente e id\'enticamente distribuida de los tiempos de
servicio solicitados. \item[ii)] La selecci\'on de usuarios para se
atendidos es independiente de sus correspondientes tiempos de
servicio y del pasado hasta el inicio del periodo de servicio. As\'i
las distribuci\'on $\left(f,v\right)$ no depende del orden en el
cu\'al son atendidos los usuarios. \item[iii)] La pol\'itica de
servicio es mon\'otona en el sentido de que para cada $a\geq0$ los
n\'umeros $f\left(x,a\right)$ son mon\'otonos en distribuci\'on en $x$ y
su l\'imite en distribuci\'on cuando $x\rightarrow\infty$ es una
variable aleatoria $F^{*0}$ que no depende de $a$. \item[iv)] El
n\'umero de usuarios atendidos por cada servidor es acotado por
$f^{min}\left(x\right)$ de la longitud de la cola $x$ que adem\'as
converge mon\'otonamente en distribuci\'on a $F^{*}$ cuando
$x\rightarrow\infty$
\end{itemize}

El sistema de colas se describe por medio del proceso de Markov
$\left(X\left(t\right)\right)_{t\in\rea}$ como se define a
continuaci\'on. El estado del sistema al tiempo $t\geq0$ est\'a dado
por
\[X\left(t\right)=\left(Q\left(t\right),P\left(t\right),A\left(t\right),R\left(t\right),C\left(t\right)\right)\]
donde
\begin{itemize}
\item
$Q\left(t\right)=\left(Q_{k}\left(t\right)\right)_{k=1}^{c}$,
n\'umero de usuarios en la cola $k$ al tiempo $t$. \item
$P\left(t\right)=\left(P^{i}\left(t\right)\right)_{i=1}^{s}$, es
la posici\'on del servidor $i$. \item
$A\left(t\right)=\left(A_{k}\left(t\right)\right)_{k=1}^{c}$, es
el residual del tiempo de arribo en la cola $k$ al tiempo $t$.
\item
$R\left(t\right)=\left(R_{k}^{i}\left(t\right),R_{k,l}^{0,i}\left(t\right)\right)_{k,l,i=1}^{c,c,s}$,
el primero es el residual del tiempo de servicio del usuario
atendido por servidor $i$ en la cola $k$ al tiempo $t$, la segunda
componente es el residual del tiempo de cambio del servidor $i$ de
la cola $k$ a la cola $l$ al tiempo $t$. \item
$C\left(t\right)=\left(C_{k}^{i}\left(t\right)\right)_{k,i=1}^{c,s}$,
es la componente correspondiente a la cola $k$ y al servidor $i$
que est\'a determinada por la pol\'itica de servicio en la cola $k$
y que hace al proceso $X\left(t\right)$ un proceso de Markov.
\end{itemize}
Todos los procesos definidos arriba se suponen continuos por la
derecha.

El proceso $X$ tiene la propiedad fuerte de Markov y su espacio de
estados es el espacio producto
\[\mathcal{X}=\nat^{c}\times E^{s}\times \rea_{+}^{c}\times\rea_{+}^{cs}\times\rea_{+}^{c^{2}s}\times \mathcal{C}\] donde $E=\left\{1,2,\ldots,c\right\}^{2}\cup\left\{1,2,\ldots,c\right\}$ y $\mathcal{C}$  depende de las pol\'iticas de servicio.

Si $x$ es el n{\'u}mero de usuarios en la cola al comienzo del
periodo de servicio y $N_{s}\left(x\right)=N\left(x\right)$ es el
n{\'u}mero de usuarios que son atendidos con la pol{\'\i}tica $s$,
{\'u}nica en nuestro caso durante un periodo de servicio, entonces
se asume que:
\begin{enumerate}
\item
\begin{equation}\label{S1}
lim_{x\rightarrow\infty}\esp\left[N\left(x\right)\right]=\overline{N}>0
\end{equation}
\item
\begin{equation}\label{S2}
\esp\left[N\left(x\right)\right]\leq \overline{N} \end{equation}
para cualquier valor de $x$.
\end{enumerate}
La manera en que atiende el servidor $m$-{\'e}simo, en este caso
en espec{\'\i}fico solo lo ilustraremos con un s{\'o}lo servidor,
es la siguiente:
\begin{itemize}
\item Al t{\'e}rmino de la visita a la cola $j$, el servidor se
cambia a la cola $j^{'}$ con probabilidad
$r_{j,j^{'}}^{m}=r_{j,j^{'}}$

\item La $n$-{\'e}sima ocurencia va acompa{\~n}ada con el tiempo
de cambio de longitud $\delta_{j,j^{'}}\left(n\right)$,
independientes e id{\'e}nticamente distribuidas, con
$\esp\left[\delta_{j,j^{'}}\left(1\right)\right]\geq0$.

\item Sea $\left\{p_{j}\right\}$ la {\'u}nica distribuci{\'o}n
invariante estacionaria para la Cadena de Markov con matriz de
transici{\'o}n $\left(r_{j,j^{'}}\right)$.

\item Finalmente, se define
\begin{equation}
\delta^{*}:=\sum_{j,j^{'}}p_{j}r_{j,j^{'}}\esp\left[\delta_{j,j^{'}}\left(1\right)\right].
\end{equation}
\end{itemize}

El {\em token passing ring} es una estaci\'on de un solo servidor
con $K$ clases de usuarios. Cada clase tiene su propio regulador
en la estaci\'on. Los usuarios llegan al regulador con raz\'on
$\alpha_{k}$ y son atendidos con taza $\mu_{k}$.

La red se puede modelar como un Proceso de Markov con espacio de
estados continuo, continuo en el tiempo:
\begin{equation}
 X\left(t\right)^{T}=\left(Q_{k}\left(t\right),A_{l}\left(t\right),B_{k}\left(t\right),B_{k}^{0}\left(t\right),C\left(t\right):k=1,\ldots,K,l\in\mathcal{A}\right)
\end{equation}
donde $Q_{k}\left(t\right), B_{k}\left(t\right)$ y
$A_{k}\left(t\right)$ se define como en \ref{Eq.Esp.Estados},
$B_{k}^{0}\left(t\right)$ es el tiempo residual de cambio de la
clase $k$ a la clase $k+1\left(mod K\right)$; $C\left(t\right)$
indica el n\'umero de servicios que han sido comenzados y/o
completados durante la sesi\'on activa del buffer.

Los par\'ametros cruciales son la carga nominal de la cola $k$:
$\beta_{k}=\alpha_{k}/\mu_{k}$ y la carga total es
$\rho_{0}=\sum\beta_{k}$, la media total del tiempo de cambio en
un ciclo del token est\'a definido por
\begin{equation}
 u^{0}=\sum_{k=1}^{K}\esp\left[\eta_{k}^{0}\left(1\right)\right]=\sum_{k=1}^{K}\frac{1}{\mu_{k}^{0}}
\end{equation}

El proceso de la longitud de la cola $Q_{k}^{x}\left(t\right)$ y
el proceso de acumulaci\'on del tiempo de servicio
$T_{k}^{x}\left(t\right)$ para el buffer $k$ y para el estado
inicial $x$ se definen como antes. Sea $T_{k}^{x,0}\left(t\right)$
el tiempo acumulado al tiempo $t$ que el token tarda en cambiar
del buffer $k$ al $k+1\mod K$. Suponga que la funci\'on
$\left(\overline{Q}\left(\cdot\right),\overline{T}\left(\cdot\right),\overline{T}^{0}\left(\cdot\right)\right)$
es un punto l\'imite de
\begin{equation}\label{Eq.4.4}
\left(\frac{1}{|x|}Q^{x}\left(|x|t\right),\frac{1}{|x|}T^{x}\left(|x|t\right),\frac{1}{|x|}T^{x,0}\left(|x|t\right)\right)
\end{equation}
cuando $|x|\rightarrow\infty$. Entonces
$\left(\overline{Q}\left(t\right),\overline{T}\left(t\right),\overline{T}^{0}\left(t\right)\right)$
es un flujo l\'imite retrasado del token ring.

Propiedades importantes para el modelo de flujo retrasado

\begin{Prop}
 Sea $\left(\overline{Q},\overline{T},\overline{T}^{0}\right)$ un flujo l\'imite de \ref{Eq.4.4} y suponga que cuando $x\rightarrow\infty$ a lo largo de
una subsucesi\'on
\[\left(\frac{1}{|x|}Q_{k}^{x}\left(0\right),\frac{1}{|x|}A_{k}^{x}\left(0\right),\frac{1}{|x|}B_{k}^{x}\left(0\right),\frac{1}{|x|}B_{k}^{x,0}\left(0\right)\right)\rightarrow\left(\overline{Q}_{k}\left(0\right),0,0,0\right)\]
para $k=1,\ldots,K$. EL flujo l\'imite tiene las siguientes
propiedades, donde las propiedades de la derivada se cumplen donde
la derivada exista:
\begin{itemize}
 \item[i)] Los vectores de tiempo ocupado $\overline{T}\left(t\right)$ y $\overline{T}^{0}\left(t\right)$ son crecientes y continuas con
$\overline{T}\left(0\right)=\overline{T}^{0}\left(0\right)=0$.
\item[ii)] Para todo $t\geq0$
\[\sum_{k=1}^{K}\left[\overline{T}_{k}\left(t\right)+\overline{T}_{k}^{0}\left(t\right)\right]=t\]
\item[iii)] Para todo $1\leq k\leq K$
\[\overline{Q}_{k}\left(t\right)=\overline{Q}_{k}\left(0\right)+\alpha_{k}t-\mu_{k}\overline{T}_{k}\left(t\right)\]
\item[iv)]  Para todo $1\leq k\leq K$
\[\dot{{\overline{T}}}_{k}\left(t\right)=\beta_{k}\] para $\overline{Q}_{k}\left(t\right)=0$.
\item[v)] Para todo $k,j$
\[\mu_{k}^{0}\overline{T}_{k}^{0}\left(t\right)=\mu_{j}^{0}\overline{T}_{j}^{0}\left(t\right)\]
\item[vi)]  Para todo $1\leq k\leq K$
\[\mu_{k}\dot{{\overline{T}}}_{k}\left(t\right)=l_{k}\mu_{k}^{0}\dot{{\overline{T}}}_{k}^{0}\left(t\right)\] para $\overline{Q}_{k}\left(t\right)>0$.
\end{itemize}
\end{Prop}


\begin{Lemma}\label{Lema.34.MeynDown}
El proceso estoc\'astico $\Phi$ es un proceso de markov fuerte,
temporalmente homog\'eneo, con trayectorias muestrales continuas
por la derecha, cuyo espacio de estados $Y$ es igual a
$X\times\rea$
\end{Lemma}
\begin{Prop}
 Suponga que los supuestos A1) y A2) son ciertos y que el modelo de flujo es estable. Entonces existe $t_{0}>0$ tal que
\begin{equation}
 lim_{|x|\rightarrow\infty}\frac{1}{|x|^{p+1}}\esp_{x}\left[|X\left(t_{0}|x|\right)|^{p+1}\right]=0
\end{equation}
\end{Prop}

\begin{Lemma}\label{Lema.5.2}
 Sea $\left\{\zeta\left(k\right):k\in \mathbb{z}\right\}$ una sucesi\'on independiente e id\'enticamente distribuida que toma valores en $\left(0,\infty\right)$,
y sea
$E\left(t\right)=max\left(n\geq1:\zeta\left(1\right)+\cdots+\zeta\left(n-1\right)\leq
t\right)$. Si $\esp\left[\zeta\left(1\right)\right]<\infty$,
entonces para cualquier entero $r\geq1$
\begin{equation}
 lim_{t\rightarrow\infty}\esp\left[\left(\frac{E\left(t\right)}{t}\right)^{r}\right]=\left(\frac{1}{\esp\left[\zeta_{1}\right]}\right)^{r}.
\end{equation}
Luego, bajo estas condiciones:
\begin{itemize}
 \item[a)] para cualquier $\delta>0$, $\sup_{t\geq\delta}\esp\left[\left(\frac{E\left(t\right)}{t}\right)^{r}\right]<\infty$
\item[b)] las variables aleatorias
$\left\{\left(\frac{E\left(t\right)}{t}\right)^{r}:t\geq1\right\}$
son uniformemente integrables.
\end{itemize}
\end{Lemma}

\begin{Teo}\label{Tma.5.5}
Suponga que los supuestos A1) y A2) se cumplen y que el modelo de
flujo es estable. Entonces existe una constante $\kappa_{p}$ tal
que
\begin{equation}
\frac{1}{t}\int_{0}^{t}\esp_{x}\left[|Q\left(s\right)|^{p}\right]ds\leq\kappa_{p}\left\{\frac{1}{t}|x|^{p+1}+1\right\}
\end{equation}
para $t>0$ y $x\in X$. En particular, para cada condici\'on inicial
\begin{eqnarray*}
\limsup_{t\rightarrow\infty}\frac{1}{t}\int_{0}^{t}\esp_{x}\left[|Q\left(s\right)|^{p}\right]ds\leq\kappa_{p}.
\end{eqnarray*}
\end{Teo}

\begin{Teo}\label{Tma.6.2}
Suponga que se cumplen los supuestos A1), A2) y A3) y que el
modelo de flujo es estable. Entonces se tiene que
\begin{equation}
|\left|P^{t}\left(x,\cdot\right)-\pi\left(\cdot\right)\right||_{f_{p}}\textrm{,
}t\rightarrow\infty,x\in X.
\end{equation}
En particular para cada condici\'on inicial
\begin{eqnarray*}
\lim_{t\rightarrow\infty}\esp_{x}\left[|Q\left(t\right)|^{p}\right]=\esp_{\pi}\left[|Q\left(0\right)|^{p}\right]\leq\kappa_{r}
\end{eqnarray*}
\end{Teo}
\begin{Teo}\label{Tma.6.3}
Suponga que se cumplen los supuestos A1), A2) y A3) y que el
modelo de flujo es estable. Entonces con
$f\left(x\right)=f_{1}\left(x\right)$ se tiene
\begin{equation}
\lim_{t\rightarrow\infty}t^{p-1}|\left|P^{t}\left(x,\cdot\right)-\pi\left(\cdot\right)\right||_{f}=0.
\end{equation}
En particular para cada condici\'on inicial
\begin{eqnarray*}
\lim_{t\rightarrow\infty}t^{p-1}|\esp_{x}\left[Q\left(t\right)\right]-\esp_{\pi}\left[Q\left(0\right)\right]|=0.
\end{eqnarray*}
\end{Teo}

\begin{Teo}\label{Tma.6.4}
Suponga que se cumplen los supuestos A1), A2) y A3) y que el
modelo de flujo es estable. Sea $\nu$ cualquier distribuci\'on de
probabilidad en $\left(X,\mathcal{B}_{X}\right)$, y $\pi$ la
distribuci\'on estacionaria de $X$.
\begin{itemize}
\item[i)] Para cualquier $f:X\leftarrow\rea_{+}$
\begin{equation}
\lim_{t\rightarrow\infty}\frac{1}{t}\int_{o}^{t}f\left(X\left(s\right)\right)ds=\pi\left(f\right):=\int
f\left(x\right)\pi\left(dx\right)
\end{equation}
$\prob$-c.s. \item[ii)] Para cualquier $f:X\leftarrow\rea_{+}$ con
$\pi\left(|f|\right)<\infty$, la ecuaci\'on anterior se cumple.
\end{itemize}
\end{Teo}


Si $x$ es el n{\'u}mero de usuarios en la cola al comienzo del
periodo de servicio y $N_{s}\left(x\right)=N\left(x\right)$ es el
n{\'u}mero de usuarios que son atendidos con la pol{\'\i}tica $s$,
{\'u}nica en nuestro caso durante un periodo de servicio, entonces
se asume que:
\begin{enumerate}
\item
\begin{equation}\label{S1}
lim_{x\rightarrow\infty}\esp\left[N\left(x\right)\right]=\overline{N}>0
\end{equation}
\item
\begin{equation}\label{S2}
\esp\left[N\left(x\right)\right]\leq \overline{N} \end{equation}
para cualquier valor de $x$.
\end{enumerate}
La manera en que atiende el servidor $m$-{\'e}simo, en este caso
en espec{\'\i}fico solo lo ilustraremos con un s{\'o}lo servidor,
es la siguiente:
\begin{itemize}
\item Al t{\'e}rmino de la visita a la cola $j$, el servidor se
cambia a la cola $j^{'}$ con probabilidad
$r_{j,j^{'}}^{m}=r_{j,j^{'}}$

\item La $n$-{\'e}sima ocurencia va acompa{\~n}ada con el tiempo
de cambio de longitud $\delta_{j,j^{'}}\left(n\right)$,
independientes e id{\'e}nticamente distribuidas, con
$\esp\left[\delta_{j,j^{'}}\left(1\right)\right]\geq0$.

\item Sea $\left\{p_{j}\right\}$ la {\'u}nica distribuci{\'o}n
invariante estacionaria para la Cadena de Markov con matriz de
transici{\'o}n $\left(r_{j,j^{'}}\right)$.

\item Finalmente, se define
\begin{equation}
\delta^{*}:=\sum_{j,j^{'}}p_{j}r_{j,j^{'}}\esp\left[\delta_{j,j^{'}}\left(1\right)\right].
\end{equation}
\end{itemize}


\begin{thebibliography}{99}

\bibitem{ISL}
James, G., Witten, D., Hastie, T., and Tibshirani, R. (2013). \textit{An Introduction to Statistical Learning: with Applications in R}. Springer.

\bibitem{Logistic}
Hosmer, D. W., Lemeshow, S., and Sturdivant, R. X. (2013). \textit{Applied Logistic Regression} (3rd ed.). Wiley.

\bibitem{PatternRecognition}
Bishop, C. M. (2006). \textit{Pattern Recognition and Machine Learning}. Springer.

\bibitem{Harrell}
Harrell, F. E. (2015). \textit{Regression Modeling Strategies: With Applications to Linear Models, Logistic and Ordinal Regression, and Survival Analysis}. Springer.

\bibitem{RDocumentation}
R Documentation and Tutorials: \url{https://cran.r-project.org/manuals.html}

\bibitem{RBlogger}
Tutorials on R-bloggers: \url{https://www.r-bloggers.com/}

\bibitem{CourseraML}
Coursera: \textit{Machine Learning} by Andrew Ng.

\bibitem{edXDS}
edX: \textit{Data Science and Machine Learning Essentials} by Microsoft.

% Libros adicionales
\bibitem{Ross}
Ross, S. M. (2014). \textit{Introduction to Probability and Statistics for Engineers and Scientists}. Academic Press.

\bibitem{DeGroot}
DeGroot, M. H., and Schervish, M. J. (2012). \textit{Probability and Statistics} (4th ed.). Pearson.

\bibitem{Hogg}
Hogg, R. V., McKean, J., and Craig, A. T. (2019). \textit{Introduction to Mathematical Statistics} (8th ed.). Pearson.

\bibitem{Kleinbaum}
Kleinbaum, D. G., and Klein, M. (2010). \textit{Logistic Regression: A Self-Learning Text} (3rd ed.). Springer.

% Artículos y tutoriales adicionales
\bibitem{Wasserman}
Wasserman, L. (2004). \textit{All of Statistics: A Concise Course in Statistical Inference}. Springer.

\bibitem{KhanAcademy}
Probability and Statistics Tutorials on Khan Academy: \url{https://www.khanacademy.org/math/statistics-probability}

\bibitem{OnlineStatBook}
Online Statistics Education: \url{http://onlinestatbook.com/}

\bibitem{Peng}
Peng, C. Y. J., Lee, K. L., and Ingersoll, G. M. (2002). \textit{An Introduction to Logistic Regression Analysis and Reporting}. The Journal of Educational Research.

\bibitem{Agresti}
Agresti, A. (2007). \textit{An Introduction to Categorical Data Analysis} (2nd ed.). Wiley.

\bibitem{Han}
Han, J., Pei, J., and Kamber, M. (2011). \textit{Data Mining: Concepts and Techniques}. Morgan Kaufmann.

\bibitem{TowardsDataScience}
Data Cleaning and Preprocessing on Towards Data Science: \url{https://towardsdatascience.com/data-cleaning-and-preprocessing}

\bibitem{Molinaro}
Molinaro, A. M., Simon, R., and Pfeiffer, R. M. (2005). \textit{Prediction error estimation: a comparison of resampling methods}. Bioinformatics.

\bibitem{EvaluatingModels}
Evaluating Machine Learning Models on Towards Data Science: \url{https://towardsdatascience.com/evaluating-machine-learning-models}

\bibitem{LogisticRegressionGuide}
Practical Guide to Logistic Regression in R on Towards Data Science: \url{https://towardsdatascience.com/practical-guide-to-logistic-regression-in-r}

% Cursos en línea adicionales
\bibitem{CourseraStatistics}
Coursera: \textit{Statistics with R} by Duke University.

\bibitem{edXProbability}
edX: \textit{Data Science: Probability} by Harvard University.

\bibitem{CourseraLogistic}
Coursera: \textit{Logistic Regression} by Stanford University.

\bibitem{edXInference}
edX: \textit{Data Science: Inference and Modeling} by Harvard University.

\bibitem{CourseraWrangling}
Coursera: \textit{Data Science: Wrangling and Cleaning} by Johns Hopkins University.

\bibitem{edXRBasics}
edX: \textit{Data Science: R Basics} by Harvard University.

\bibitem{CourseraRegression}
Coursera: \textit{Regression Models} by Johns Hopkins University.

\bibitem{edXStatInference}
edX: \textit{Data Science: Statistical Inference} by Harvard University.

\bibitem{SurvivalAnalysis}
An Introduction to Survival Analysis on Towards Data Science: \url{https://towardsdatascience.com/an-introduction-to-survival-analysis}

\bibitem{MultinomialLogistic}
Multinomial Logistic Regression on DataCamp: \url{https://www.datacamp.com/community/tutorials/multinomial-logistic-regression-R}

\bibitem{CourseraSurvival}
Coursera: \textit{Survival Analysis} by Johns Hopkins University.

\bibitem{edXHighthroughput}
edX: \textit{Data Science: Statistical Inference and Modeling for High-throughput Experiments} by Harvard University.

\end{thebibliography}

\printindex
\end{document}
