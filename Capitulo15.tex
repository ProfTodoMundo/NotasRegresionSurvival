
\section{Introducci\'on}
El modelo de riesgos proporcionales de Cox, propuesto por David Cox en 1972, es una de las herramientas más utilizadas en el análisis de supervivencia. Este modelo permite evaluar el efecto de varias covariables en el tiempo hasta el evento, sin asumir una forma espec\'ifica para la distribuci\'on de los tiempos de supervivencia.

\section{Definici\'on del Modelo de Cox}
El modelo de Cox se define como:
\begin{eqnarray*}
\lambda(t \mid X) = \lambda_0(t) \exp(\beta^T X)
\end{eqnarray*}
donde:
\begin{itemize}
    \item $\lambda(t \mid X)$ es la funci\'on de riesgo en el tiempo $t$ dado el vector de covariables $X$.
    \item $\lambda_0(t)$ es la funci\'on de riesgo basal en el tiempo $t$.
    \item $\beta$ es el vector de coeficientes del modelo.
    \item $X$ es el vector de covariables.
\end{itemize}

\section{Supuesto de Proporcionalidad de Riesgos}
El modelo de Cox asume que las razones de riesgo entre dos individuos son constantes a lo largo del tiempo. Matemáticamente, si $X_i$ y $X_j$ son las covariables de dos individuos, la raz\'on de riesgos se expresa como:
\begin{eqnarray*}
\frac{\lambda(t \mid X_i)}{\lambda(t \mid X_j)} = \frac{\lambda_0(t) \exp(\beta^T X_i)}{\lambda_0(t) \exp(\beta^T X_j)} = \exp(\beta^T (X_i - X_j))
\end{eqnarray*}

\section{Estimaci\'on de los Par\'ametros}
Los par\'ametros $\beta$ se estiman utilizando el m\'etodo de m\'axima verosimilitud parcial. La funci\'on de verosimilitud parcial se define como:
\begin{eqnarray*}
L(\beta) = \prod_{i=1}^k \frac{\exp(\beta^T X_i)}{\sum_{j \in R(t_i)} \exp(\beta^T X_j)}
\end{eqnarray*}
donde $R(t_i)$ es el conjunto de individuos en riesgo en el tiempo $t_i$.

\subsection{Funci\'on de Log-Verosimilitud Parcial}
La funci\'on de log-verosimilitud parcial es:
\begin{eqnarray*}
\log L(\beta) = \sum_{i=1}^k \left(\beta^T X_i - \log \sum_{j \in R(t_i)} \exp(\beta^T X_j)\right)
\end{eqnarray*}

\subsection{Derivadas Parciales y Maximizaci\'on}
Para encontrar los estimadores de m\'axima verosimilitud, resolvemos el sistema de ecuaciones obtenido al igualar a cero las derivadas parciales de $\log L(\beta)$ con respecto a $\beta$:
\begin{eqnarray*}
\frac{\partial \log L(\beta)}{\partial \beta} = \sum_{i=1}^k \left(X_i - \frac{\sum_{j \in R(t_i)} X_j \exp(\beta^T X_j)}{\sum_{j \in R(t_i)} \exp(\beta^T X_j)}\right) = 0
\end{eqnarray*}

\section{Interpretaci\'on de los Coeficientes}
Cada coeficiente $\beta_i$ representa el logaritmo de la raz\'on de riesgos asociado con un incremento unitario en la covariable $X_i$. Un valor positivo de $\beta_i$ indica que un aumento en $X_i$ incrementa el riesgo del evento, mientras que un valor negativo indica una reducci\'on del riesgo.

\section{Evaluaci\'on del Modelo}
El modelo de Cox se eval\'ua utilizando varias t\'ecnicas, como el an\'alisis de residuos de Schoenfeld para verificar el supuesto de proporcionalidad de riesgos, y el uso de curvas de supervivencia estimadas para evaluar la bondad de ajuste.

\subsection{Residuos de Schoenfeld}
Los residuos de Schoenfeld se utilizan para evaluar la proporcionalidad de riesgos. Para cada evento en el tiempo $t_i$, el residuo de Schoenfeld para la covariable $X_j$ se define como:
\begin{eqnarray*}
r_{ij} = X_{ij} - \hat{X}_{ij}
\end{eqnarray*}
donde $\hat{X}_{ij}$ es la covariable ajustada.

\subsection{Curvas de Supervivencia Ajustadas}
Las curvas de supervivencia ajustadas se obtienen utilizando la funci\'on de riesgo basal estimada y los coeficientes del modelo. La funci\'on de supervivencia ajustada se define como:
\begin{eqnarray*}
\hat{S}(t \mid X) = \hat{S}_0(t)^{\exp(\beta^T X)}
\end{eqnarray*}
donde $\hat{S}_0(t)$ es la funci\'on de supervivencia basal estimada.

\section{Ejemplo de Aplicaci\'on del Modelo de Cox}
Consideremos un ejemplo con tres covariables: edad, sexo y tratamiento. Supongamos que los datos se ajustan a un modelo de Cox y obtenemos los siguientes coeficientes:
\begin{eqnarray*}
\hat{\beta}_{edad} = 0.02, \quad \hat{\beta}_{sexo} = -0.5, \quad \hat{\beta}_{tratamiento} = 1.2
\end{eqnarray*}

La funci\'on de riesgo ajustada se expresa como:
\begin{eqnarray*}
\lambda(t \mid X) = \lambda_0(t) \exp(0.02 \cdot \text{edad} - 0.5 \cdot \text{sexo} + 1.2 \cdot \text{tratamiento})
\end{eqnarray*}

\section{Conclusi\'on}
El modelo de riesgos proporcionales de Cox es una herramienta poderosa para analizar datos de supervivencia con m\'ultiples covariables. Su flexibilidad y la falta de suposiciones fuertes sobre la distribuci\'on de los tiempos de supervivencia lo hacen ampliamente aplicable en diversas disciplinas.

