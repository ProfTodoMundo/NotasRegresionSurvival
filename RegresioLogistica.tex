\documentclass{report}
\usepackage[utf8]{inputenc}
\usepackage{amsmath}
\usepackage{amssymb}
\usepackage{geometry}
\usepackage{hyperref}
\usepackage{fancyhdr}
\usepackage{titlesec} % Paquete para modificar títulos de secciones
\usepackage[spanish]{babel} % Paquete para definir el idioma
\usepackage{listings} % Para incluir código fuente


\title{Curso Elemental de Regresi\'on Logística y Análisis de Supervivencia}
\author{Carlos E Mart\'inez-Rodr\'iguez}
\date{Julio 2024}

\geometry{
  a4paper,
  left=25mm,
  right=25mm,
  top=30mm,
  bottom=30mm,
}

% Configuración de encabezados y pies de página
\pagestyle{fancy}
\fancyhf{}
\fancyhead[L]{\leftmark}
\fancyfoot[C]{\thepage}
\fancyfoot[R]{\rightmark}
\fancyfoot[L]{Carlos E Mart\'inez-Rodr\'iguez} % Nombre del autor en la parte inferior izquierda

% Redefinir el nombre de los capítulos
\titleformat{\chapter}[display]
  {\normalfont\huge\bfseries}
  {CAPÍTULO \thechapter}
  {20pt}
  {\Huge}

\newtheorem{Algthm}{Algoritmo}%[section]

% Configuración para la inclusión de código fuente en R
\lstset{
    language=R,
    basicstyle=\ttfamily\small,
    numbers=left,
    numberstyle=\tiny,
    stepnumber=1,
    numbersep=5pt,
    showspaces=false,
    showstringspaces=false,
    showtabs=false,
    frame=single,
    tabsize=2,
    captionpos=b,
    breaklines=true,
    breakatwhitespace=false,
    title=\lstname
}
\newtheorem{Def}{Definici\'on}[section]
\newtheorem{Ejem}{Ejemplo}[section]
\newtheorem{Teo}{Teorema}[section]
\newtheorem{Dem}{Demostraci\'on}[section]
\newtheorem{Note}{Nota}[section]
\newtheorem{Sol}{Soluci\'on}[section]
\newtheorem{Prop}{Proposici\'on}[section]
\newtheorem{Coro}{Corolario}[section]
\newtheorem{Cor}{Corolario}[section]
\newtheorem{Lemma}{Lema}[section]
\newtheorem{Lema}{Lema}[section]
\newtheorem{Lem}{Lema}[section]
\newtheorem{Sup}{Supuestos}[section]

\def\RR{\mathbb{R}}
\def\ZZ{\mathbb{Z}}
\newcommand{\nat}{\mathbb{N}}
\newcommand{\ent}{\mathbb{Z}}
\newcommand{\rea}{\mathbb{R}}
\newcommand{\esp}{\mathbb{E}}
\newcommand{\prob}{\mathbb{P}}
\newcommand{\indora}{\mbox{$1$\hspace{-0.8ex}$1$}}



\begin{document}

\maketitle

\tableofcontents

\part{INTRODUCCION}
\chapter{Introducci\'on}

\section{Descripci\'on del curso}

\subsection{Presentaci\'on}

\textbf{El curso}

\begin{itemize}
    \item Indispensable
    \item Modalidad: Presencial y Semipresencial
    \item Horas de clase: 5 teor\'ia y 5 de pr\'acticas
\end{itemize}

\subsection{Parte I. Introducci\'on a la Bioestad\'istica}

\textbf{Unidad 1. Conceptos b\'asicos de Bioestad\'istica y Metodolog\'ia de la Investigaci\'on}

Prop\'ositos: Que el/la estudiante:
\begin{enumerate}
    \item Comprenda y utilice correctamente los conceptos b\'asicos de la bioestad\'istica.
    \item Elabore bases de datos para el an\'alisis estad\'istico de la informaci\'on obtenida para sus trabajos de investigaci\'on.
\end{enumerate}

\subsection{Parte II. Estad\'istica descriptiva}

\textbf{Unidad 2. An\'alisis de la informaci\'on: tabulaci\'on y visualizaci\'on}

Prop\'ositos: Que el/la estudiante:
\begin{enumerate}
    \item Elabore e interprete correctamente representaciones tabulares y visuales.
    \item Seleccione las mejores formas de representar visualmente la informaci\'on.
\end{enumerate}

\textbf{Unidad 3. An\'alisis de la informaci\'on: medidas de tendencia central, variabilidad y localizaci\'on}

Prop\'ositos: Que el/la estudiante:
\begin{enumerate}
    \item Seleccione las medidas de tendencia central y variabilidad m\'as adecuadas de acuerdo a los objetivos del an\'alisis estad\'istico.
    \item Aplique e interprete correctamente las medidas de tendencia central, variabilidad y localizaci\'on.
\end{enumerate}

\subsection{Parte III. Estad\'istica inferencial}

\textbf{Unidad 4. La distribuci\'on normal}

Prop\'ositos: Que el/la estudiante:
\begin{enumerate}
    \item Entienda los conceptos impl\'icitos, expl\'icitos y la \textit{universalidad} de la distribuci\'on normal.
    \item Seleccione y ejecute correctamente los procedimientos num\'ericos para determinar las probabilidades correctas a partir de valores $z$ y la determinaci\'on de valores $z$ para probabilidades conocidas.
    \item Utilice correctamente las probabilidades $z$ en la resoluci\'on de problemas.
\end{enumerate}

\textbf{Unidad 5. Prueba de hip\'otesis}

Prop\'ositos: Que el/la estudiante:
\begin{enumerate}
    \item Comprenda los fundamentos te\'oricos de las pruebas de hip\'otesis.
    \item Comprenda la utilidad de las pruebas de hip\'otesis y su utilidad dentro del proceso de investigaci\'on.
    \item Comprenda y sea capaz de llevar a cabo pruebas de hip\'otesis para comparar proporciones de dos muestras.
    \item Comprenda y sea capaz de realizar pruebas de hip\'otesis para comparar las medias de dos muestras independientes.
    \item Comprenda y realice pruebas de hip\'otesis de datos apareados.
    \item Ejecute correctamente los procesos num\'ericos para obtener los estad\'isticos de prueba para la toma de decisiones.
    \item Comprenda la relaci\'on entre la decisi\'on estad\'istica de resultado del contraste de hip\'otesis y las decisiones con respecto a la pregunta de investigaci\'on y las hip\'otesis de trabajo.
\end{enumerate}

\textbf{Unidad 6. Correlaci\'on y regresi\'on lineal simple}

Prop\'ositos: Que el/la estudiante:
\begin{enumerate}
    \item Aplique correctamente el an\'alisis de correlaci\'on.
    \item Aplique correctamente el ajuste del modelo de regresi\'on lineal simple.
    \item Interprete correctamente los resultados del an\'alisis de correlaci\'on lineal simple y del modelo ajustado de regresi\'on lineal simple.
\end{enumerate}

\textbf{EVALUACI\'ON}
\begin{itemize}
    \item \textbf{Evaluaci\'on diagn\'ostica.} Se evaluar\'an conocimientos indispensables de aritm\'etica, \'algebra, geometr\'ia anal\'itica, as\'i como habilidades sobre c\'alculos aritm\'eticos, relaciones y operaciones algebraicas, construcci\'on e interpretaci\'on de gr\'aficas e interpretaci\'on de ecuaciones y aspectos indispensables de biolog\'ia humana.
    \item \textbf{Evaluaciones formativas.} Al menos tres evaluaciones formativas, que en conjunto deber\'an abarcar la totalidad del programa del curso. La forma, contenido y momento en que se realicen las evaluaciones formativas depender\'an del criterio del profesor. El resultado de cada evaluaci\'on formativa consistir\'a de la calificaci\'on de cada evaluaci\'on y las tareas (ejercicios, lecturas, etc.) comprendidos en los temas de cada evaluaci\'on.
    \item \textbf{Evaluaci\'on final.} La carpeta valdr\'a el 40\% de la calificaci\'on final y los elementos a considerar en la carpeta son: ex\'amenes, tareas y participaci\'on de cada estudiante. El porcentaje de cada elemento lo determinar\'a el profesor del curso, pero en total debe ser igual al 40\%. El examen de certificaci\'on valdr\'a el 60\% y no habr\'a estudiantes exentos. Todos los estudiantes que quieran aprobar la materia deber\'an realizar el examen de certificaci\'on (60\% de la calificaci\'on final) y contar con la carpeta (40\% de la calificaci\'on final). La calificaci\'on del examen de certificaci\'on valdr\'a el 60\%, esto es que si los estudiantes obtienen 10 de calificaci\'on en el examen de certificaci\'on, porcentualmente es el 60\% o sea 6, y si obtienen 10 de calificaci\'on en la carpeta tendr\'an el 40\% o sea 4 puntos de la calificaci\'on final, en este ejemplo la calificaci\'on final es 10.
\end{itemize}

\section{Introducci\'on}

\subsection{Definici\'on de Estad\'istica}

\begin{itemize}
    \item La Estad\'istica es una ciencia formal que estudia la recolecci\'on, an\'alisis e interpretaci\'on de datos de una muestra representativa, ya sea para ayudar en la toma de decisiones o para explicar condiciones regulares o irregulares de alg\'un fen\'omeno o estudio aplicado, de ocurrencia en forma aleatoria o condicional. 
    \item Sin embargo, la estad\'istica es m\'as que eso, es decir, es el veh\'iculo que permite llevar a cabo el proceso relacionado con la investigaci\'on cient\'ifica. 
    \item Es transversal a una amplia variedad de disciplinas, desde la f\'isica hasta las ciencias sociales, desde las ciencias de la salud hasta el control de calidad. Se usa para la toma de decisiones en \'areas de negocios o instituciones gubernamentales.
\end{itemize}

\begin{Def}
    La Estad\'istica es la ciencia cuyo objetivo es reunir una informaci\'on cuantitativa concerniente a individuos, grupos, series de hechos, etc. y deducir de ello gracias al an\'alisis de estos datos unos significados precisos o unas previsiones para el futuro.
\end{Def}

\begin{itemize}
    \item La estad\'istica, en general, es la ciencia que trata de la recopilaci\'on, organizaci\'on presentaci\'on, an\'alisis e interpretaci\'on de datos num\'ericos con el fin de realizar una toma de decisi\'on m\'as efectiva.
\end{itemize}

\subsection{Utilidad e Importancia}

\begin{itemize}
    \item Los m\'etodos estad\'isticos tradicionalmente se utilizan para prop\'ositos descriptivos, para organizar y resumir datos num\'ericos. La estad\'istica descriptiva, por ejemplo trata de la tabulaci\'on de datos, su presentaci\'on en forma gr\'afica o ilustrativa y el c\'alculo de medidas descriptivas.
    \item Ahora bien, las t\'ecnicas estad\'isticas se aplican de manera amplia en mercadotecnia, contabilidad, control de calidad y en otras actividades; estudios de consumidores; an\'alisis de resultados en deportes; administradores de instituciones; en la educaci\'on; organismos pol\'iticos; m\'edicos; y por otras personas que intervienen en la toma de decisiones.
\end{itemize}

\subsection{Historia de la Estad\'istica}

\begin{itemize}
    \item Es dif\'icil conocer los or\'igenes de la Estad\'istica. Desde los comienzos de la civilizaci\'on han existido formas sencillas de estad\'istica, pues ya se utilizaban representaciones gr\'aficas y otros s\'imbolos en pieles, rocas, palos de madera y paredes de cuevas para contar el n\'umero de personas, animales o ciertas cosas. 
    \item Su origen empieza posiblemente en la isla de Cerde\~na, donde existen monumentos prehist\'oricos pertenecientes a los Nuragas, las primeros habitantes de la isla; estos monumentos constan de bloques de basalto superpuestos sin mortero y en cuyas paredes de encontraban grabados toscos signos que han sido interpretados con mucha verosimilidad como muescas que serv\'ian para llevar la cuenta del ganado y la caza.
    \item Los babilonios usaban ya peque\~nas tablillas de arcilla para recopilar datos en tablas sobre la producci\'on agr\'icola y los g\'eneros vendidos o cambiados mediante trueque. 
    \item Otros vestigios pueden ser hallados en el antiguo Egipto, cuyos faraones lograron recopilar, hacia el a\~no 3050 antes de Cristo, prolijos datos relativos a la poblaci\'on y la riqueza del pa\'is. De acuerdo al historiador griego Her\'odoto, dicho registro de riqueza y poblaci\'on se hizo con el objetivo de preparar la construcci\'on de las pir\'amides.
    \item En el mismo Egipto, Rams\'es II hizo un censo de las tierras con el objeto de verificar un nuevo reparto. En el antiguo Israel la Biblia da referencias, en el libro de los N\'umeros, de los datos estad\'isticos obtenidos en dos recuentos de la poblaci\'on hebrea. El rey David por otra parte, orden\'o a Joab, general del ej\'ercito hacer un censo de Israel con la finalidad de conocer el n\'umero de la poblaci\'on.
    \item Tambi\'en los chinos efectuaron censos hace m\'as de cuarenta siglos. Los griegos efectuaron censos peri\'odicamente con fines tributarios, sociales (divisi\'on de tierras) y militares (c\'alculo de recursos y hombres disponibles). 
    \item La investigaci\'on hist\'orica revela que se realizaron 69 censos para calcular los impuestos, determinar los derechos de voto y ponderar la potencia guerrera.
    \item Fueron los romanos, maestros de la organizaci\'on pol\'itica, quienes mejor supieron emplear los recursos de la estad\'istica. Cada cinco a\~nos realizaban un censo de la poblaci\'on y sus funcionarios p\'ublicos ten\'ian la obligaci\'on de anotar nacimientos, defunciones y matrimonios, sin olvidar los recuentos peri\'odicos del ganado y de las riquezas contenidas en las tierras conquistadas. Para el nacimiento de Cristo suced\'ia uno de estos empadronamientos de la poblaci\'on bajo la autoridad del imperio. 
    \item Durante los mil a\~nos siguientes a la ca\'ida del imperio Romano se realizaron muy pocas operaciones Estad\'isticas, con la notable excepci\'on de las relaciones de tierras pertenecientes a la Iglesia, compiladas por Pipino el Breve en el 758 y por Carlomagno en el 762 DC. Durante el siglo IX se realizaron en Francia algunos censos parciales de siervos. En Inglaterra, Guillermo el Conquistador recopil\'o el Domesday Book o libro del Gran Catastro para el a\~no 1086, un documento de la propiedad, extensi\'on y valor de las tierras de Inglaterra. Esa obra fue el primer compendio estad\'istico de Inglaterra. 
    \item Aunque Carlomagno, en Francia; y Guillermo el Conquistador, en Inglaterra, trataron de revivir la t\'ecnica romana, los m\'etodos estad\'isticos permanecieron casi olvidados durante la Edad Media.
    \item Durante los siglos XV, XVI, y XVII, hombres como Leonardo de Vinci, Nicol\'as Cop\'ernico, Galileo, Neper, William Harvey, Sir Francis Bacon y Ren\'e Descartes, hicieron grandes operaciones al m\'etodo cient\'ifico, de tal forma que cuando se crearon los Estados Nacionales y surgi\'o como fuerza el comercio internacional exist\'ia ya un m\'etodo capaz de aplicarse a los datos econ\'omicos. 
    \item Para el a\~no 1532 empezaron a registrarse en Inglaterra las defunciones debido al temor que Enrique VII ten\'ia por la peste.  M\'as o menos por la misma \'epoca, en Francia la ley exigi\'o a los cl\'erigos registrar los bautismos, fallecimientos y matrimonios. Durante un brote de peste que apareci\'o a fines de la d\'ecada de 1500, el gobierno ingl\'es comenz\'o a publicar estad\'istica semanales de los decesos. Esa costumbre continu\'o muchos a\~nos, y en 1632 estos Bills of Mortality (Cuentas de Mortalidad) conten\'ian los nacimientos y fallecimientos por sexo.
    \item En 1662, el capit\'an John Graunt us\'o documentos que abarcaban treinta a\~nos y efectu\'o predicciones sobre el n\'umero de personas que morir\'ian de varias enfermedades y sobre las proporciones de nacimientos de varones y mujeres que cabr\'ia esperar. El trabajo de Graunt, condensado en su obra \textit{Natural and Political Observations...Made upon the Bills of Mortality}, fue un esfuerzo innovador en el an\'alisis estad\'istico. Por el a\~no 1540 el alem\'an Sebasti\'an Muster realiz\'o una compilaci\'on estad\'istica de los recursos nacionales, comprensiva de datos sobre organizaci\'on pol\'itica, instrucciones sociales, comercio y poder\'io militar. 
    \item Durante el siglo XVII aport\'o indicaciones m\'as concretas de m\'etodos de observaci\'on y an\'alisis cuantitativo y ampli\'o los campos de la inferencia y la teor\'ia Estad\'istica.
    \item Los eruditos del siglo XVII demostraron especial inter\'es por la Estad\'istica Demogr\'afica como resultado de la especulaci\'on sobre si la poblaci\'on aumentaba, decrec\'ia o permanec\'ia est\'atica. En los tiempos modernos tales m\'etodos fueron resucitados por algunos reyes que necesitaban conocer las riquezas monetarias y el potencial humano de sus respectivos pa\'ises. 
    \item El primer empleo de los datos estad\'isticos para fines ajenos a la pol\'itica tuvo lugar en 1691 y estuvo a cargo de Gaspar Neumann, un profesor alem\'an que viv\'ia en Breslau. Este investigador se propuso destruir la antigua creencia popular de que en los a\~nos terminados en siete mor\'ia m\'as gente que en los restantes, y para lograrlo hurg\'o pacientemente en los archivos parroquiales de la ciudad. Despu\'es de revisar miles de partidas de defunci\'on pudo demostrar que en tales a\~nos no fallec\'ian m\'as personas que en los dem\'as. Los procedimientos de Neumann fueron conocidos por el astr\'onomo ingl\'es Halley, descubridor del cometa que lleva su nombre, quien los aplic\'o al estudio de la vida humana. Sus c\'alculos sirvieron de base para las tablas de mortalidad que hoy utilizan todas las compa\~n\'ias de seguros. Durante el siglo XVII y principios del XVIII, matem\'aticos como Bernoulli, Francis Maseres, Lagrange y Laplace desarrollaron la teor\'ia de probabilidades. No obstante durante cierto tiempo, la teor\'ia de las probabilidades limit\'o su aplicaci\'on a los juegos de azar y hasta el siglo XVIII no comenz\'o a aplicarse a los grandes problemas cient\'ificos.
    \item Godofredo Achenwall, profesor de la Universidad de Gotinga, acu\~n\'o en 1760 la palabra estad\'istica, que extrajo del t\'ermino italiano statista (estadista). Cre\'ia, y con sobrada raz\'on, que los datos de la nueva ciencia ser\'ian el aliado m\'as eficaz del gobernante consciente. La ra\'iz remota de la palabra se halla, por otra parte, en el t\'ermino latino status, que significa estado o situaci\'on; Esta etimolog\'ia aumenta el valor intr\'inseco de la palabra, por cuanto la estad\'istica revela el sentido cuantitativo de las m\'as variadas situaciones. Jacques Qu\'etelect es quien aplica las Estad\'isticas a las ciencias sociales. Este interpret\'o la teor\'ia de la probabilidad para su uso en las ciencias sociales y resolver la aplicaci\'on del principio de promedios y de la variabilidad a los fen\'omenos sociales.
    \item Qu\'etelect fue el primero en realizar la aplicaci\'on pr\'actica de todo el m\'etodo Estad\'istico, entonces conocido, a las diversas ramas de la ciencia.
    \item Entretanto, en el per\'iodo del 1800 al 1820 se desarrollaron dos conceptos matem\'aticos fundamentales para la teor\'ia Estad\'istica; la teor\'ia de los errores de observaci\'on, aportada por Laplace y Gauss; y la teor\'ia de los m\'inimos cuadrados desarrollada por Laplace, Gauss y Legendre. A finales del siglo XIX, Sir Francis Gaston ide\'o el m\'etodo conocido por Correlaci\'on, que ten\'ia por objeto medir la influencia relativa de los factores sobre las variables.
    \item De aqu\'i parti\'o el desarrollo del coeficiente de correlaci\'on creado por Karl Pearson y otros cultivadores de la ciencia biom\'etrica como J. Pease Norton, R. H. Hooker y G. Udny Yule, que efectuaron amplios estudios sobre la medida de las relaciones.
    \item Los progresos m\'as recientes en el campo de la Estad\'istica se refieren al ulterior desarrollo del c\'alculo de probabilidades, particularmente en la rama denominada indeterminismo o relatividad, se ha demostrado que el determinismo fue reconocido en la F\'isica como resultado de las investigaciones at\'omicas y que este principio se juzga aplicable tanto a las ciencias sociales como a las f\'isicas.
\end{itemize}

\subsection{Etapas de Desarrollo de la Estad\'istica}

La historia de la estad\'istica est\'a resumida en tres grandes etapas o fases.

\begin{itemize}
    \item \textbf{Fase 1: Los Censos:} Desde el momento en que se constituye una autoridad pol\'itica, la idea de inventariar de una forma m\'as o menos regular la poblaci\'on y las riquezas existentes en el territorio est\'a ligada a la conciencia de soberan\'ia y a los primeros esfuerzos administrativos.
    \item \textbf{Fase 2: De la Descripci\'on de los Conjuntos a la Aritm\'etica Pol\'itica:} Las ideas mercantilistas extra\~nan una intensificaci\'on de este tipo de investigaci\'on. Colbert multiplica las encuestas sobre art\'iculos manufacturados, el comercio y la poblaci\'on: los intendentes del Reino env\'ian a Par\'is sus memorias. Vauban, m\'as conocido por sus fortificaciones o su Dime Royale, que es la primera propuesta de un impuesto sobre los ingresos, se se\~nala como el verdadero precursor de los sondeos. M\'as tarde, Buf\'on se preocupa de esos problemas antes de dedicarse a la historia natural. La escuela inglesa proporciona un nuevo progreso al superar la fase puramente descriptiva.
\end{itemize}

Sus tres principales representantes son Graunt, Petty y Halley. El pen\'ultimo es autor de la famosa Aritm\'etica Pol\'itica. Chaptal, ministro del interior franc\'es, publica en 1801 el primer censo general de poblaci\'on, desarrolla los estudios industriales, de las producciones y los cambios, haci\'endose sistem\'aticos durantes las dos terceras partes del siglo XIX.

\begin{itemize}
    \item \textbf{Fase 3: Estad\'istica y C\'alculo de Probabilidades:} El c\'alculo de probabilidades se incorpora r\'apidamente como un instrumento de an\'alisis extremadamente poderoso para el estudio de los fen\'omenos econ\'omicos y sociales y en general para el estudio de fen\'omenos cuyas causas son demasiados complejas para conocerlos totalmente y hacer posible su an\'alisis.
\end{itemize}

\subsection{Divisi\'on de la Estad\'istica}

La Estad\'istica para su mejor estudio se ha dividido en dos grandes ramas: \textbf{la Estad\'istica Descriptiva y la Estad\'istica Inferencial}.

\begin{itemize}
    \item \textbf{Descriptiva:} consiste sobre todo en la presentaci\'on de datos en forma de tablas y gr\'aficas. Esta comprende cualquier actividad relacionada con los datos y est\'a dise\~nada para resumir o describir los mismos sin factores pertinentes adicionales; esto es, sin intentar inferir nada que vaya m\'as all\'a de los datos, como tales.
    \item \textbf{Inferencial:} se deriva de muestras, de observaciones hechas s\'olo acerca de una parte de un conjunto numeroso de elementos y esto implica que su an\'alisis requiere de generalizaciones que van m\'as all\'a de los datos. Como consecuencia, la caracter\'istica m\'as importante del reciente crecimiento de la estad\'istica ha sido un cambio en el \'enfasis de los m\'etodos que describen a m\'etodos que sirven para hacer generalizaciones. La Estad\'istica Inferencial investiga o analiza una poblaci\'on partiendo de una muestra tomada.
\end{itemize}

\subsection{Estad\'istica Inferencial}

Los m\'etodos b\'asicos de la estad\'istica inferencial son la estimaci\'on y el contraste de hip\'otesis, que juegan un papel fundamental en la investigaci\'on. Por tanto, algunos de los objetivos que se persiguen son:

\begin{itemize}
    \item Calcular los par\'ametros de la distribuci\'on de medias o proporciones muestrales de tama\~no $n$, extra\'idas de una poblaci\'on de media y varianza conocidas.
    \item Estimar la media o la proporci\'on de una poblaci\'on a partir de la media o proporci\'on muestral.
    \item Utilizar distintos tama\~nos muestrales para controlar la confianza y el error admitido.
    \item Contrastar los resultados obtenidos a partir de muestras.
    \item Visualizar gr\'aficamente, mediante las respectivas curvas normales, las estimaciones realizadas.
\end{itemize}

En definitiva, la idea es, a partir de una poblaci\'on se extrae una muestra por algunos de los m\'etodos existentes, con la que se generan datos num\'ericos que se van a utilizar para generar estad\'isticos con los que realizar estimaciones o contrastes poblacionales. Existen dos formas de estimar par\'ametros: la \textit{estimaci\'on puntual} y la \textit{estimaci\'on por intervalo de confianza}. En la primera se busca, con base en los datos muestrales, un \'unico valor estimado para el par\'ametro. Para la segunda, se determina un intervalo dentro del cual se encuentra el valor del par\'ametro, con una probabilidad determinada.

\begin{itemize}
    \item Si el objetivo del tratamiento estad\'istico inferencial, es efectuar generalizaciones acerca de la estructura, composici\'on o comportamiento de las poblaciones no observadas, a partir de una parte de la poblaci\'on, ser\'a necesario que la parcela de poblaci\'on examinada sea representativa del total. 
    \item Por ello, la selecci\'on de la muestra requiere unos requisitos que lo garanticen, debe ser representativa y aleatoria. 
    \item Adem\'as, la cantidad de elementos que integran la muestra (el tama\~no de la muestra) depende de m\'ultiples factores, como el dinero y el tiempo disponibles para el estudio, la importancia del tema analizado, la confiabilidad que se espera de los resultados, las caracter\'isticas propias del fen\'omeno analizado, etc\'etera. 
\end{itemize}

As\'i, a partir de la muestra seleccionada se realizan algunos c\'alculos y se estima el valor de los par\'ametros de la poblaci\'on tales como la media, la varianza, la desviaci\'on est\'andar, o la forma de la distribuci\'on, etc.

\subsection{M\'etodo Estad\'istico}

El conjunto de los m\'etodos que se utilizan para medir las caracter\'isticas de la informaci\'on, para resumir los valores individuales, y para analizar los datos a fin de extraerles el m\'aximo de informaci\'on, es lo que se llama \textit{m\'etodos estad\'isticos}. Los m\'etodos de an\'alisis para la informaci\'on cuantitativa se pueden dividir en los siguientes seis pasos:

\begin{enumerate}
    \item Definici\'on del problema.
    \item Recopilaci\'on de la informaci\'on existente.
    \item Obtenci\'on de informaci\'on original.
    \item Clasificaci\'on.
    \item Presentaci\'on.
    \item An\'alisis.
\end{enumerate}

El centro de gravedad de la metodolog\'ia estad\'istica se empieza a desplazar t\'ecnicas de computaci\'on intensiva aplicadas a grandes masas de datos, y se empieza a considerar el m\'etodo estad\'istico como un proceso iterativo de b\'usqueda del modelo ideal.

Las aplicaciones en este periodo de la Estad\'istica a la Econom\'ia conducen a una disciplina con contenido propio: la Econometr\'ia. La investigaci\'on estad\'istica en problemas militares durante la segunda guerra mundial y los nuevos m\'etodos de programaci\'on matem\'atica, dan lugar a la Investigaci\'on Operativa.

\subsection{Errores Estad\'isticos Comunes}

Al momento de recopilar los datos que ser\'an procesados se es susceptible de cometer errores as\'i como durante los c\'omputos de los mismos. No obstante, hay otros errores que no tienen nada que ver con la digitaci\'on y que no son tan f\'acilmente identificables. Algunos de \'estos errores son:

\begin{itemize}
    \item \textbf{Sesgo:} Es imposible ser completamente objetivo o no tener ideas preconcebidas antes de comenzar a estudiar un problema, y existen muchas maneras en que una perspectiva o estado mental pueda influir en la recopilaci\'on y en el an\'alisis de la informaci\'on. En estos casos se dice que hay un sesgo cuando el individuo da mayor peso a los datos que apoyan su opini\'on que a aquellos que la contradicen. Un caso extremo de sesgo ser\'ia la situaci\'on donde primero se toma una decisi\'on y despu\'es se utiliza el an\'alisis estad\'istico para justificar la decisi\'on ya tomada.
    \item \textbf{Datos No Comparables:} el establecer comparaciones es una de las partes m\'as importantes del an\'alisis estad\'istico, pero es extremadamente importante que tales comparaciones se hagan entre datos que sean comparables.
    \item \textbf{Proyecci\'on descuidada de tendencias:} la proyecci\'on simplista de tendencias pasadas hacia el futuro es uno de los errores que m\'as ha desacreditado el uso del an\'alisis estad\'istico.
    \item \textbf{Muestreo Incorrecto:} en la mayor\'ia de los estudios sucede que el volumen de informaci\'on disponible es tan inmenso que se hace necesario estudiar muestras, para derivar conclusiones acerca de la poblaci\'on a que pertenece la muestra. Si la muestra se selecciona correctamente, tendr\'a b\'asicamente las mismas propiedades que la poblaci\'on de la cual fue extra\'ida; pero si el muestreo se realiza incorrectamente, entonces puede suceder que los resultados no signifiquen nada.
\end{itemize}

En resumen se puede decir que la Estad\'istica es un conjunto de procedimientos para reunir, clasificar, codificar, procesar, analizar y resumir informaci\'on num\'erica adquirida sistem\'aticamente (Ritchey, 2002). Permite hacer inferencias a partir de una muestra para extrapolarlas a una poblaci\'on. Aunque normalmente se asocia a muchos c\'alculos y operaciones aritm\'eticas, y aunque las matem\'aticas est\'an involucradas, en su mayor parte sus fundamentos y uso apropiado pueden dominarse sin hacer referencia a habilidades matem\'aticas avanzadas. 

De hecho se trata de una forma de ver la realidad basada en el an\'alisis cuidadoso de los hechos (Ritchey, 2002). Es necesaria sin embargo la sistematizaci\'on para reducir el efecto que las emociones y las experiencias individuales puedan tener al interpretar esa realidad.

De esta manera la estad\'istica se relaciona con el m\'etodo cient\'ifico complement\'andolo como herramienta de an\'alisis y, aunque la investigaci\'on cient\'ifica no requiere necesariamente de la estad\'istica, \'esta valida muchos de los resultados cuantitativos derivados de la investigaci\'on. 

La obtenci\'on del conocimiento debe hacerse de manera sistem\'atica por lo que deben planearse todos los pasos que llevan desde el planteamiento de un problema, pasando por la elaboraci\'on de hip\'otesis y la manera en que van a ser probadas; la selecci\'on de sujetos (muestreo), los escenarios, los instrumentos que se utilizar\'an para obtener los datos, definir el procedimiento que se seguir\'a para esto \'ultimo, los controles que se deben hacer para asegurar que las intervenciones son las causas m\'as probables de los cambios esperados (dise\~no); 

El tratamiento de los datos de la investigaci\'on cient\'ifica tiene varias etapas:

\begin{itemize}
    \item En la etapa de recolecci\'on de datos del m\'etodo cient\'ifico, se define a la poblaci\'on de inter\'es y se selecciona una muestra o conjunto de personas representativas de la misma, se realizan experimentos o se emplean instrumentos ya existentes o de nueva creaci\'on, para medir los atributos de inter\'es necesarios para responder a las preguntas de investigaci\'on.Durante lo que es llamado trabajo de campo se obtienen los datos en crudo, es decir las respuestas directas de los sujetos uno por uno, se codifican (se les asignan valores a las respuestas), se capturan y se verifican para ser utilizados en las siguientes etapas.
    \item En la etapa de recuento, se organizan y ordenan los datos obtenidos de la muestra. Esta ser\'a descrita en la siguiente etapa utilizando la estad\'istica descriptiva, todas las investigaciones utilizan estad\'istica descriptiva, para conocer de manera organizada y resumida las caracter\'isticas de la muestra.
    \item En la etapa de an\'alisis se utilizan las pruebas estad\'isticas (estad\'istica inferencial) y en la interpretaci\'on se acepta o rechaza la hip\'otesis nula.
\end{itemize}

En investigaci\'on, el fen\'omeno en estudio puede ser cualitativo que implicar\'ia comprenderlo y explicarlo, o cuantitativo para compararlo y hacer inferencias. Se puede decir que si se hace an\'alisis se usan m\'etodos cuantitativos y si se hace descripci\'on se usan m\'etodos cualitativos. 

Medici\'on Para poder emplear el m\'etodo estad\'istico en un estudio es necesario medir las variables. 

\begin{itemize}
    \item Medir: es asignar valores a las propiedades de los objetos bajo ciertas reglas, esas reglas son los niveles de medici\'on.
    \item Cuantificar: es asignar valores a algo tomando un patr\'on de referencia. Por ejemplo, cuantificar es ver cu\'antos hombres y cu\'antas mujeres hay.
\end{itemize}

\textbf{Variable:} es una caracter\'istica o propiedad que asume diferentes valores dentro de una poblaci\'on de inter\'es y cuya variaci\'on es susceptible de medirse.

Las variables pueden clasificarse de acuerdo al tipo de valores que puede tomar como:

\begin{itemize}
    \item Discretas o categ\'oricas.- en las que los valores se relacionan a nombres, etiquetas o categor\'ias, no existe un significado num\'erico directo.
    \item Continuas.- los valores tienen un correlato num\'erico directo, son continuos y susceptibles de fraccionarse y de poder utilizarse en operaciones aritm\'eticas.
    \item Dicot\'omica.- s\'olo tienen dos valores posibles, la caracter\'istica est\'a ausente o presente.
    \item Policot\'omica.- pueden tomar tres valores o m\'as, pueden tomarse matices diferentes, en grados, jerarqu\'ias o magnitudes continuas.
\end{itemize}

En cuanto a una clasificaci\'on estad\'istica:

\begin{itemize}
    \item Aleatoria.- Aquella en la cual desconocemos el valor porque fluct\'ua de acuerdo a un evento debido al azar.
    \item Determin\'istica.- Aquella variable de la que se conoce el valor.
    \item Independiente.- aquellas variables que son manipuladas por el investigador. Define los grupos.
    \item Dependiente.- son mediciones que ocurren durante el experimento o tratamiento (resultado de la independiente), es la que se mide y compara entre los grupos.
\end{itemize}

\textbf{Niveles de Medici\'on}

\begin{itemize}
    \item Nominal: Las propiedades de la medici\'on nominal son:
    \begin{itemize}
        \item Exhaustiva: implica a todas las opciones.
        \item A los sujetos se les asignan categor\'ias, por lo que son mutuamente excluyentes. Es decir, la variable est\'a presente o no; tiene o no una caracter\'istica.
    \end{itemize}
    \item Ordinal: Las propiedades de la medici\'on ordinal son:
    \begin{itemize}
        \item El nivel ordinal posee transitividad, por lo que se tiene la capacidad de identificar que es mejor o mayor que otra, en ese sentido se pueden establecer jerarqu\'ias.
        \item Las distancias entre un valor y otro no son iguales.
    \end{itemize}
    \item Intervalo: 
    \begin{itemize}
        \item El nivel de medici\'on intervalar requiere distancias iguales entre cada valor. Por lo general utiliza datos cuantitativos. Por ejemplo: temperatura, atributos psicol\'ogicos (CI, nivel de autoestima, pruebas de conocimientos, etc.)
        \item Las unidades de calificaci\'on son equivalentes en todos los puntos de la escala. Una escala de intervalos implica: clasificaci\'on, magnitud y unidades de tama\~nos iguales (Brown, 2000).
        \item Se pueden hacer operaciones aritm\'eticas.
        \item Cuando se le pide al sujeto que califique una situaci\'on del 0 al 10 puede tomarse como un nivel de medici\'on de intervalo, siempre y cuando se incluya el 0.
    \end{itemize}
    \item Raz\'on: 
    \begin{itemize}
        \item La escala empieza a partir del 0 absoluto, por lo tanto incluye s\'olo los n\'umeros por su valor en s\'i, por lo que no pueden existir los n\'umeros con signo negativo. Por ejemplo: Peso corporal en kg., edad en a\~nos, estatura en cm.
        \item Convencionalmente los datos que son de nivel absoluto o de raz\'on son manejados como los datos intervalares.
    \end{itemize}
\end{itemize}

\subsection{T\'erminos comunes utilizados en Estad\'istica}

\begin{itemize}
    \item \textbf{Variable:} Consideraciones que una variable son una caracter\'istica o fen\'omeno que puede tomar distintos valores.
    \item \textbf{Dato:} Mediciones o cualidades que han sido recopiladas como resultado de observaciones.
    \item \textbf{Poblaci\'on:} Se considera el \'area de la cual son extra\'idos los datos. Es decir, es el conjunto de elementos o individuos que poseen una caracter\'istica com\'un y medible acerca de lo cual se desea informaci\'on. Es tambi\'en llamado Universo.
    \item \textbf{Muestra:} Es un subconjunto de la poblaci\'on, seleccionado de acuerdo a una regla o alg\'un plan de muestreo.
    \item \textbf{Censo:} Recopilaci\'on de todos los datos (de inter\'es para la investigaci\'on) de la poblaci\'on.
    \item \textbf{Estad\'istica:} Es una funci\'on o f\'ormula que depende de los datos de la muestra (es variable).
    \item \textbf{Par\'ametro:} Caracter\'istica medible de la poblaci\'on. Es un resumen num\'erico de alguna variable observada de la poblaci\'on. Los par\'ametros normales que se estudian son: \textit{La media poblacional, Proporci\'on.}
    \item \textbf{Estimador:} Un estimador de un par\'ametro es un estad\'istico que se emplea para conocer el par\'ametro desconocido.
    \item \textbf{Estad\'istico:} Es una funci\'on de los valores de la muestra. Es una variable aleatoria, cuyos valores dependen de la muestra seleccionada. Su distribuci\'on de probabilidad, se conoce como \textit{Distribuci\'on muestral del estad\'istico}.
    \item \textbf{Estimaci\'on:} Este t\'ermino indica que a partir de lo observado en una muestra (un resumen estad\'istico con las medidas que conocemos de Descriptiva) se extrapola o generaliza dicho resultado muestral a la poblaci\'on total, de modo que lo estimado es el valor generalizado a la poblaci\'on. Consiste en la b\'usqueda del valor de los par\'ametros poblacionales objeto de estudio. Puede ser puntual o por intervalo de confianza:
    \begin{itemize}
        \item \textit{Puntual:} cuando buscamos un valor concreto. Un estimador de un par\'ametro poblacional es una funci\'on de los datos muestrales. En pocas palabras, es una f\'ormula que depende de los valores obtenidos de una muestra, para realizar estimaciones. Lo que se pretende obtener es el valor exacto de un par\'ametro.
    \end{itemize}
    \item \textit{Intervalo de confianza:} cuando determinamos un intervalo, dentro del cual se supone que va a estar el valor del par\'ametro que se busca con una cierta probabilidad. El intervalo de confianza est\'a determinado por dos valores dentro de los cuales afirmamos que est\'a el verdadero par\'ametro con cierta probabilidad. Son unos l\'imites o margen de variabilidad que damos al valor estimado, para poder afirmar, bajo un criterio de probabilidad, que el verdadero valor no los rebasar\'a.
\end{itemize}

Este intervalo contiene al par\'ametro estimado con una determinada certeza o nivel de confianza. 

En la estimaci\'on por intervalos se usan los siguientes conceptos:

\begin{itemize}
    \item Variabilidad del par\'ametro: Si no se conoce, puede obtenerse una aproximaci\'on en los datos o en un estudio piloto. Tambi\'en hay m\'etodos para calcular el tama\~no de la muestra que prescinden de este aspecto. Habitualmente se usa como medida de esta variabilidad la desviaci\'on t\'ipica poblacional.
    \item Error de la estimaci\'on: Es una medida de su precisi\'on que se corresponde con la amplitud del intervalo de confianza. Cuanta m\'as precisi\'on se desee en la estimaci\'on de un par\'ametro, m\'as estrecho deber\'a ser el intervalo de confianza y, por tanto, menor el error, y m\'as sujetos deber\'an incluirse en la muestra estudiada. 
    \item Nivel de confianza: Es la probabilidad de que el verdadero valor del par\'ametro estimado en la poblaci\'on se sit\'ue en el intervalo de confianza obtenido. El nivel de confianza se denota por $1-\alpha$
    \item $p$-value : Tambi\'en llamado nivel de significaci\'on. Es la probabilidad (en tanto por uno) de fallar en nuestra estimaci\'on, esto es, la diferencia entre la certeza (1) y el nivel de confianza $1-\alpha$. 
    \item Valor cr\'itico: Se representa por $Z_{\alpha/2}$. Es el valor de la abscisa en una determinada distribuci\'on que deja a su derecha un \'area igual a 1/2, siendo $1-\alpha$ el nivel de confianza. Normalmente los valores cr\'iticos est\'an tabulados o pueden calcularse en funci\'on de la distribuci\'on de la poblaci\'on.
\end{itemize}

Para un tama\~no fijo de la muestra, los conceptos de error y nivel de confianza van relacionados. Si admitimos un error mayor, esto es, aumentamos el tama\~no del intervalo de confianza, tenemos tambi\'en una mayor probabilidad de \'exito en nuestra estimaci\'on, es decir, un mayor nivel de confianza. Por tanto, un aspecto que debe de tenerse en cuenta es el tama\~no muestral, ya que para disminuir el error que se comente habr\'a que aumentar el tama\~no muestral. Esto se resolver\'a, para un intervalo de confianza cualquiera, despejando el tama\~no de la muestra en cualquiera de las formulas de los intervalos de confianza que veremos a continuaci\'on, a partir del error m\'aximo permitido. Los intervalos de confianza pueden ser unilaterales o bilaterales:

\begin{itemize}
    \item \textbf{Contraste de Hip\'otesis:} Consiste en determinar si es aceptable, partiendo de datos muestrales, que la caracter\'istica o el par\'ametro poblacional estudiado tome un determinado valor o est\'e dentro de unos determinados valores.
    \item \textbf{Nivel de Confianza:} Indica la proporci\'on de veces que acertar\'iamos al afirmar que el par\'ametro est\'a dentro del intervalo al seleccionar muchas muestras.
\end{itemize}

\subsection{Muestreo:} 

Una muestra es representativa en la medida que es imagen de la poblaci\'on. 

En general, podemos decir que el tama\~no de una muestra depender\'a principalmente de: \textit{Nivel de precisi\'on deseado, Recursos disponibles, Tiempo involucrado en la investigaci\'on.} Adem\'as el plan de muestreo debe considerar \textit{La poblaci\'on, Par\'ametros a medir}.

Existe una gran cantidad de tipos de muestreo. En la pr\'actica los m\'as utilizados son los siguientes:

\begin{itemize}
    \item \textbf{MUESTREO ALEATORIO SIMPLE:} Es un m\'etodo de selecci\'on de $n$ unidades extra\'idas de $N$, de tal manera que cada una de las posibles muestras tiene la misma probabilidad de ser escogida. (En la pr\'actica, se enumeran las unidades de 1 a $N$, y a continuaci\'on se seleccionan $n$ n\'umeros aleatorios entre 1 y $N$, ya sea de tablas o de alguna urna con fichas numeradas).
    \item \textbf{MUESTREO ESTRATIFICADO ALEATORIO:} Se usa cuando la poblaci\'on est\'a agrupada en pocos estratos, cada uno de ellos son muchas entidades. Este muestreo consiste en sacar una muestra aleatoria simple de cada uno de los estratos. (Generalmente, de tama\~no proporcional al estrato).
    \item \textbf{MUESTREO SISTEM\'ATICO:} Se utiliza cuando las unidades de la poblaci\'on est\'an de alguna manera totalmente ordenadas. Para seleccionar una muestra de $n$ unidades, se divide la poblaci\'on en $n$ subpoblaciones de tama\~no $K = N/n$ y se toma al azar una unidad de la $K$ primeras y de ah\'i en adelante cada $K$-\'esima unidad.
    \item \textbf{MUESTREO POR CONGLOMERADO:} Se emplea cuando la poblaci\'on est\'a dividida en grupos o conglomerados peque\~nos. Consiste en obtener una muestra aleatoria simple de conglomerados y luego CENSAR cada uno de \'estos.
    \item \textbf{MUESTREO EN DOS ETAPAS (Biet\'apico):} En este caso la muestra se toma en dos pasos:
    \begin{itemize}
        \item Seleccionar una muestra de unidades primarias, y 
        \item Seleccionar una muestra de elementos a partir de cada unidad primaria escogida.
        \item \textit{Observaci\'on:} En la realidad es posible encontrarse con situaciones en las cuales no es posible aplicar libremente un tipo de muestreo, incluso estaremos obligados a mezclarlas en ocasiones.
    \end{itemize}
\end{itemize}

Las variables se pueden clasificar en dos grandes grupos:

\begin{itemize}
    \item \textbf{Variables categ\'oricas:} Son aquellas que pueden ser representadas a trav\'es de s\'imbolos, letras, palabras, etc. Los valores que toman se denominan categor\'ias, y los elementos que pertenecen a estas categor\'ias, se consideran id\'enticos respecto a la caracter\'istica que se est\'a midiendo. Las variables categ\'oricas de dividen en dos tipos: Ordinal y Nominal.
    \begin{itemize}
        \item \textbf{Las Ordinales:} son aquellas en que las categor\'ias tienen un orden impl\'icito. Admiten grados de calidad, es decir, existe una relaci\'on total entre las categor\'ias.
        \item \textbf{Las Nominales:} son aquellas donde no existe una relaci\'on de orden.
    \end{itemize}
    \item \textbf{Variables Num\'ericas:} Son aquellas que pueden tomar valores num\'ericos exclusivamente (mediciones). Se dividen en dos tipos: Discretas y continuas.
    \begin{itemize}
        \item \textbf{Discretas:} son aquellas que toman sus valores en un conjunto finito o infinito numerable.
        \item \textbf{Continuas:} Son aquellas que toman sus valores en un subconjunto de los n\'umeros reales, es decir en un intervalo. En general para las variables continuas el hombre ha debido inventar una medida para poder establecer una medici\'on de ellas.
    \end{itemize}
\end{itemize}

El prop\'osito de esta secci\'on es solamente indicar los malos usos comunes de datos estad\'isticos, sin incluir el uso de m\'etodos estad\'isticos complicados. Un estudiante deber\'ia estar alerta en relaci\'on con estos malos usos y deber\'ia hacer un gran esfuerzo para evitarlos a fin de ser un verdadero estad\'istico.

\textbf{Datos estad\'isticos inadecuados}

Los datos estad\'isticos son usados como la materia prima para un estudio estad\'istico. Cuando los datos son inadecuados, la conclusi\'on extra\'ida del estudio de los datos se vuelve obviamente inv\'alida. Por ejemplo, supongamos que deseamos encontrar el ingreso familiar t\'ipico del a\~no pasado en la ciudad Y de 50,000 familias y tenemos una muestra consistente del ingreso de solamente tres familias: 1 mill\'on, 2 millones y no ingreso. Si sumamos el ingreso de las tres familias y dividimos el total por 3, obtenemos un promedio de 1 mill\'on.

Entonces, extraemos una conclusi\'on basada en la muestra de que el ingreso familiar promedio durante el a\~no pasado en la ciudad fue de 1 mill\'on. Es obvio que la conclusi\'on es falsa, puesto que las cifras son extremas y el tama\~no de la muestra es demasiado peque\~no; por lo tanto la muestra no es representativa. 

Hay muchas otras clases de datos inadecuados. Por ejemplo, algunos datos son respuestas inexactas de una encuesta, porque las preguntas usadas en la misma son vagas o enga\~nosas, algunos datos son toscas estimaciones porque no hay disponibles datos exactos o es demasiado costosa su obtenci\'on, y algunos datos son irrelevantes en un problema dado, porque el estudio estad\'istico no est\'a bien planeado.

\subsection{Un sesgo del usuario}

Sesgo significa que un usuario d\'e los datos perjudicialmente de m\'as \'enfasis a los hechos, los cuales son empleados para mantener su predeterminada posici\'on u opini\'on. Los estad\'isticos son frecuentemente degradados por lemas tales como: \textit{Hay tres clases de mentiras: mentiras, mentiras reprobables y estad\'istica, y Las cifras no mienten, pero los mentirosos piensan.}

Hay dos clases de sesgos: conscientes e inconscientes. Ambos son comunes en el an\'alisis estad\'istico. Hay numerosos ejemplos de sesgos conscientes. Un anunciante frecuentemente usa la estad\'istica para probar que su producto es muy superior al producto de su competidor. Un pol\'itico prefiere usar la estad\'istica para sostener su punto de vista. Gerentes y l\'ideres de trabajadores pueden simult\'aneamente situar sus respectivas cifras estad\'isticas sobre la misma tabla de trato para mostrar que sus rechazos o peticiones son justificadas. 

Es casi imposible que un sesgo inconsciente est\'e completamente ausente en un trabajo estad\'istico. En lo que respecta al ser humano, es dif\'icil obtener una actitud completamente objetiva al abordar un problema, aun cuando un cient\'ifico deber\'ia tener una mente abierta. Un estad\'istico deber\'ia estar enterado del hecho de que su interpretaci\'on de los resultados del an\'alisis estad\'istico est\'a influenciado por su propia experiencia, conocimiento y antecedentes con relaci\'on al problema dado.

\subsection{Supuestos falsos}

Es muy frecuente que un an\'alisis estad\'istico contemple supuestos. Un investigador debe ser muy cuidadoso en este hecho, para evitar que \'estos sean falsos. Los supuestos falsos pueden ser originados por:

\begin{itemize}
    \item Quien usa los datos.
    \item Quien est\'a tratando de confundir (con intencionalidad).
    \item Ignorancia.
    \item Descuido.
\end{itemize}

\section{Muestreo}

\textbf{Muestreo:} Una muestra es representativa en la medida que es imagen de la poblaci\'on.

En general, podemos decir que el tama\~no de una muestra depender\'a principalmente de:

\begin{itemize}
    \item Nivel de precisi\'on deseado.
    \item Recursos disponibles.
    \item Tiempo involucrado en la investigaci\'on.
\end{itemize}

Adem\'as el plan de muestreo debe considerar

\begin{itemize}
    \item La poblaci\'on.
    \item Par\'ametros a medir.
\end{itemize}

Existe una gran cantidad de tipos de muestreo. En la pr\'actica los m\'as utilizados son los siguientes:

\begin{itemize}
    \item \textbf{MUESTREO ALEATORIO SIMPLE:} Es un m\'etodo de selecci\'on de $n$ unidades extra\'idas de $N$, de tal manera que cada una de las posibles muestras tiene la misma probabilidad de ser escogida. (En la pr\'actica, se enumeran las unidades de $1$ a $N$, y a continuaci\'on se seleccionan $n$ n\'umeros aleatorios entre $1$ y $N$, ya sea de tablas o de alguna urna con fichas numeradas).
    \item \textbf{MUESTREO ESTRATIFICADO ALEATORIO:} Se usa cuando la poblaci\'on est\'a agrupada en pocos estratos, cada uno de ellos son muchas entidades. Este muestreo consiste en sacar una muestra aleatoria simple de cada uno de los estratos. (Generalmente, de tama\~no proporcional al estrato).
    \item \textbf{MUESTREO SISTEM\'ATICO:} Se utiliza cuando las unidades de la poblaci\'on est\'an de alguna manera totalmente ordenadas. Para seleccionar una muestra de $n$ unidades, se divide la poblaci\'on en $n$ subpoblaciones de tama\~no $K = N/n$ y se toma al azar una unidad de la $K$ primeras y de ah\'i en adelante cada $K$-\'esima unidad.
    \item \textbf{MUESTREO POR CONGLOMERADO:} Se emplea cuando la poblaci\'on est\'a dividida en grupos o conglomerados peque\~nos. Consiste en obtener una muestra aleatoria simple de conglomerados y luego CENSAR cada uno de \'estos.
    \item \textbf{MUESTREO EN DOS ETAPAS (Biet\'apico):} En este caso la muestra se toma en dos pasos:
    \begin{itemize}
        \item Seleccionar una muestra de unidades primarias, y 
        \item Seleccionar una muestra de elementos a partir de cada unidad primaria escogida.
        \item \textit{Observaci\'on:} En la realidad es posible encontrarse con situaciones en las cuales no es posible aplicar libremente un tipo de muestreo, incluso estaremos obligados a mezclarlas en ocasiones.
    \end{itemize}
\end{itemize}

Las variables se pueden clasificar en dos grandes grupos:

\begin{itemize}
    \item \textbf{Variables categ\'oricas:} Son aquellas que pueden ser representadas a trav\'es de s\'imbolos, letras, palabras, etc. Los valores que toman se denominan categor\'ias, y los elementos que pertenecen a estas categor\'ias, se consideran id\'enticos respecto a la caracter\'istica que se est\'a midiendo. Las variables categ\'oricas de dividen en dos tipos: Ordinal y Nominal.
    \begin{itemize}
        \item \textbf{Las Ordinales:} son aquellas en que las categor\'ias tienen un orden impl\'icito. Admiten grados de calidad, es decir, existe una relaci\'on total entre las categor\'ias.
        \item \textbf{Las Nominales:} son aquellas donde no existe una relaci\'on de orden.
    \end{itemize}
    \item \textbf{Variables Num\'ericas:} Son aquellas que pueden tomar valores num\'ericos exclusivamente (mediciones). Se dividen en dos tipos: Discretas y continuas.
    \begin{itemize}
        \item \textbf{Discretas:} son aquellas que toman sus valores en un conjunto finito o infinito numerable.
        \item \textbf{Continuas:} Son aquellas que toman sus valores en un subconjunto de los n\'umeros reales, es decir en un intervalo. En general para las variables continuas el hombre ha debido inventar una medida para poder establecer una medici\'on de ellas.
    \end{itemize}
\end{itemize}

El prop\'osito de esta secci\'on es solamente indicar los malos usos comunes de datos estad\'isticos, sin incluir el uso de m\'etodos estad\'isticos complicados. Un estudiante deber\'ia estar alerta en relaci\'on con estos malos usos y deber\'ia hacer un gran esfuerzo para evitarlos a fin de ser un verdadero estad\'istico.

\textbf{Datos estad\'isticos inadecuados}

Los datos estad\'isticos son usados como la materia prima para un estudio estad\'istico. Cuando los datos son inadecuados, la conclusi\'on extra\'ida del estudio de los datos se vuelve obviamente inv\'alida. Por ejemplo, supongamos que deseamos encontrar el ingreso familiar t\'ipico del a\~no pasado en la ciudad Y de 50,000 familias y tenemos una muestra consistente del ingreso de solamente tres familias: 1 mill\'on, 2 millones y no ingreso. Si sumamos el ingreso de las tres familias y dividimos el total por 3, obtenemos un promedio de 1 mill\'on.

Entonces, extraemos una conclusi\'on basada en la muestra de que el ingreso familiar promedio durante el a\~no pasado en la ciudad fue de 1 mill\'on. Es obvio que la conclusi\'on es falsa, puesto que las cifras son extremas y el tama\~no de la muestra es demasiado peque\~no; por lo tanto la muestra no es representativa. 

Hay muchas otras clases de datos inadecuados. Por ejemplo, algunos datos son respuestas inexactas de una encuesta, porque las preguntas usadas en la misma son vagas o enga\~nosas, algunos datos son toscas estimaciones porque no hay disponibles datos exactos o es demasiado costosa su obtenci\'on, y algunos datos son irrelevantes en un problema dado, porque el estudio estad\'istico no est\'a bien planeado.

\subsection{Un sesgo del usuario}

Sesgo significa que un usuario d\'e los datos perjudicialmente de m\'as \'enfasis a los hechos, los cuales son empleados para mantener su predeterminada posici\'on u opini\'on. Los estad\'isticos son frecuentemente degradados por lemas tales como: \textit{Hay tres clases de mentiras: mentiras, mentiras reprobables y estad\'istica, y Las cifras no mienten, pero los mentirosos piensan.}

Hay dos clases de sesgos: conscientes e inconscientes. Ambos son comunes en el an\'alisis estad\'istico. Hay numerosos ejemplos de sesgos conscientes. Un anunciante frecuentemente usa la estad\'istica para probar que su producto es muy superior al producto de su competidor. Un pol\'itico prefiere usar la estad\'istica para sostener su punto de vista. Gerentes y l\'ideres de trabajadores pueden simult\'aneamente situar sus respectivas cifras estad\'isticas sobre la misma tabla de trato para mostrar que sus rechazos o peticiones son justificadas. 

Es casi imposible que un sesgo inconsciente est\'e completamente ausente en un trabajo estad\'istico. En lo que respecta al ser humano, es dif\'icil obtener una actitud completamente objetiva al abordar un problema, aun cuando un cient\'ifico deber\'ia tener una mente abierta. Un estad\'istico deber\'ia estar enterado del hecho de que su interpretaci\'on de los resultados del an\'alisis estad\'istico est\'a influenciado por su propia experiencia, conocimiento y antecedentes con relaci\'on al problema dado.

\subsection{Supuestos falsos}

Es muy frecuente que un an\'alisis estad\'istico contemple supuestos. Un investigador debe ser muy cuidadoso en este hecho, para evitar que \'estos sean falsos. Los supuestos falsos pueden ser originados por:

\begin{itemize}
    \item Quien usa los datos.
    \item Quien est\'a tratando de confundir (con intencionalidad).
    \item Ignorancia.
    \item Descuido.
\end{itemize}


\section{2. Pruebas de Hipótesis}

\subsection{2.1 Tipos de errores}

\begin{itemize}
    \item Una hipótesis estadística es una afirmación acerca de la distribución de probabilidad de una variable aleatoria, a menudo involucran uno o más parámetros de la distribución.
    \item Las hipótesis son afirmaciones respecto a la población o distribución bajo estudio, no en torno a la muestra.
    \item La mayoría de las veces, la prueba de hipótesis consiste en determinar si la situación experimental ha cambiado.
    \item El interés principal es decidir sobre la veracidad o falsedad de una hipótesis, a este procedimiento se le llama \textit{prueba de hipótesis}.
    \item Si la información es consistente con la hipótesis, se concluye que esta es verdadera, de lo contrario que con base en la información, es falsa.
\end{itemize}

Una prueba de hipótesis está formada por cinco partes:
\begin{itemize}
    \item La hipótesis nula, denotada por $H_{0}$.
    \item La hipótesis alternativa, denotada por $H_{1}$.
    \item El estadístico de prueba y su valor $p$.
    \item La región de rechazo.
    \item La conclusión.
\end{itemize}

\begin{Def}
Las dos hipótesis en competencia son la \textbf{hipótesis alternativa $H_{1}$}, usualmente la que se desea apoyar, y la \textbf{hipótesis nula $H_{0}$}, opuesta a $H_{1}$.
\end{Def}
En general, es más fácil presentar evidencia de que $H_{1}$ es cierta, que demostrar que $H_{0}$ es falsa, es por eso que por lo regular se comienza suponiendo que $H_{0}$ es cierta, luego se utilizan los datos de la muestra para decidir si existe evidencia a favor de $H_{1}$, más que a favor de $H_{0}$, así se tienen dos conclusiones:
\begin{itemize}
    \item Rechazar $H_{0}$ y concluir que $H_{1}$ es verdadera.
    \item Aceptar, no rechazar, $H_{0}$ como verdadera.
\end{itemize}

\begin{Ejem}
Se desea demostrar que el salario promedio por hora en cierto lugar es distinto de $19$ usd, que es el promedio nacional. Entonces $H_{1}:\mu \neq 19$, y $H_{0}:\mu = 19$.
\end{Ejem}
A esta se le denomina \textbf{Prueba de hipótesis de dos colas}.

\begin{Ejem}
Un determinado proceso produce un promedio de $5\%$ de piezas defectuosas. Se está interesado en demostrar que un simple ajuste en una máquina reducirá $p$, la proporción de piezas defectuosas producidas en este proceso. Entonces se tiene $H_{0}: p < 0.3$ y $H_{1}: p = 0.03$. Si se puede rechazar $H_{0}$, se concluye que el proceso ajustado produce menos del $5\%$ de piezas defectuosas.
\end{Ejem}
A esta se le denomina \textbf{Prueba de hipótesis de una cola}.

La decisión de rechazar o aceptar la hipótesis nula está basada en la información contenida en una muestra proveniente de la población de interés. Esta información tiene estas formas:
\begin{itemize}
    \item \textbf{Estadístico de prueba:} un sólo número calculado a partir de la muestra.
    \item \textbf{$p$-value:} probabilidad calculada a partir del estadístico de prueba.
\end{itemize}

\begin{Def}
El $p$-value es la probabilidad de observar un estadístico de prueba tanto o más alejado del valor observado, si en realidad $H_{0}$ es verdadera.\medskip
Valores grandes del estadístico de prueba y valores pequeños de $p$ significan que se ha observado un evento muy poco probable, si $H_{0}$ en realidad es verdadera.
\end{Def}
Todo el conjunto de valores que puede tomar el estadístico de prueba se divide en dos regiones. Un conjunto, formado de valores que apoyan la hipótesis alternativa y llevan a rechazar $H_{0}$, se denomina \textbf{región de rechazo}. El otro, conformado por los valores que sustentan la hipótesis nula, se le denomina \textbf{región de aceptación}.\medskip

Cuando la región de rechazo está en la cola izquierda de la distribución, la prueba se denomina \textbf{prueba lateral izquierda}. Una prueba con región de rechazo en la cola derecha se le llama \textbf{prueba lateral derecha}.\medskip

Si el estadístico de prueba cae en la región de rechazo, entonces se rechaza $H_{0}$. Si el estadístico de prueba cae en la región de aceptación, entonces la hipótesis nula se acepta o la prueba se juzga como no concluyente.\medskip

Dependiendo del nivel de confianza que se desea agregar a las conclusiones de la prueba, y el \textbf{nivel de significancia $\alpha$}, el riesgo que está dispuesto a correr si se toma una decisión incorrecta.

\begin{Def}
Un \textbf{error de tipo I} para una prueba estadística es el error que se tiene al rechazar la hipótesis nula cuando es verdadera. El \textbf{nivel de significancia} para una prueba estadística de hipótesis es
\begin{eqnarray*}
\alpha &=& P\left\{\textrm{error tipo I}\right\} = P\left\{\textrm{rechazar equivocadamente } H_{0}\right\} \\
&=& P\left\{\textrm{rechazar } H_{0} \textrm{ cuando } H_{0} \textrm{ es verdadera}\right\}
\end{eqnarray*}
\end{Def}
Este valor $\alpha$ representa el valor máximo de riesgo tolerable de rechazar incorrectamente $H_{0}$. Una vez establecido el nivel de significancia, la región de rechazo se define para poder determinar si se rechaza $H_{0}$ con un cierto nivel de confianza.


\section{2.2 Muestras grandes: una media poblacional}
\subsection{2.2.1 Cálculo de valor $p$}


\begin{Def}
El \textbf{valor de $p$} (\textbf{$p$-value}) o nivel de significancia observado de un estadístico de prueba es el valor más pequeño de $\alpha$ para el cual $H_{0}$ se puede rechazar. El riesgo de cometer un error tipo $I$, si $H_{0}$ es rechazada con base en la información que proporciona la muestra.
\end{Def}

\begin{Note}
Valores pequeños de $p$ indican que el valor observado del estadístico de prueba se encuentra alejado del valor hipotético de $\mu$, es decir se tiene evidencia de que $H_{0}$ es falsa y por tanto debe de rechazarse.
\end{Note}

\begin{Note}
Valores grandes de $p$ indican que el estadístico de prueba observado no está alejado de la media hipotética y no apoya el rechazo de $H_{0}$.
\end{Note}

\begin{Def}
Si el valor de $p$ es menor o igual que el nivel de significancia $\alpha$, determinado previamente, entonces $H_{0}$ es rechazada y se puede concluir que los resultados son estadísticamente significativos con un nivel de confianza del $100 (1-\alpha)\%$.
\end{Def}
Es usual utilizar la siguiente clasificación de resultados:

\begin{tabular}{|c||c|l|}\hline
$p$ & $H_{0}$ & Significativa \\ \hline
$\leq 0.01$ & rechazada & \begin{tabular}[c]{@{}l@{}}Result. altamente significativos \\ y en contra de $H_{0}$\end{tabular} \\ \hline
$\leq 0.05$ & rechazada & \begin{tabular}[c]{@{}l@{}}Result. significativos \\ y en contra de $H_{0}$\end{tabular} \\ \hline
$\leq 0.10$ & rechazada & \begin{tabular}[c]{@{}l@{}}Result. posiblemente \\ significativos \\ y en contra de $H_{0}$\end{tabular} \\ \hline
$> 0.10$ & no rechazada & \begin{tabular}[c]{@{}l@{}}Result. no significativos \\ y no rechazar $H_{0}$\end{tabular} \\ \hline
\end{tabular}


\section{2. Pruebas de Hip\'otesis}
%---------------------------------------------------------
\subsection{2.1 Tipos de errores}

\begin{frame}
Una prueba de hip\'otesis est\'a formada por cinco partes
\begin{itemize}
\item La hip\'otesis nula, denotada por $H_{0}$.
\item La hip\'otesis alterativa, denorada por $H_{1}$.
\item El estad\'sitico de prueba y su valor $p$.
\item La regi\'on de rechazo.
\item La conclusi\'on.

\end{itemize}

\begin{Def}
Las dos hip\'otesis en competencias son la \textbf{hip\'otesis alternativa $H_{1}$}, usualmente la que se desea apoyar, y la \textbf{hip\'otesis nula $H_{0}$}, opuesta a $H_{1}$.
\end{Def}

En general, es m\'as f\'acil presentar evidencia de que $H_{1}$ es cierta, que demostrar 	que $H_{0}$ es falsa, es por eso que por lo regular se comienza suponiendo que $H_{0}$ es cierta, luego se utilizan los datos de la muestra para decidir si existe evidencia a favor de $H_{1}$, m\'as que a favor de $H_{0}$, as\'i se tienen dos conclusiones:
\begin{itemize}
\item Rechazar $H_{0}$ y concluir que $H_{1}$ es verdadera.
\item Aceptar, no rechazar, $H_{0}$ como verdadera.

\end{itemize}
\begin{Ejem}
Se desea demostrar que el salario promedio  por hora en cierto lugar es distinto de $19$usd, que es el promedio nacional. Entonces $H_{1}:\mu\neq19$, y $H_{0}:\mu=19$.
\end{Ejem}
A esta se le denomina \textbf{Prueba de hip\'otesis de dos colas}.


\begin{Ejem}
Un determinado proceso produce un promedio de $5\%$ de piezas defectuosas. Se est\'a interesado en demostrar que un simple ajuste en una m\'aquina reducir\'a $p$, la proporci\'on de piezas defectuosas producidas en este proceso. Entonces se tiene $H_{0}:p<0.3$ y $H_{1}:p=0.03$. Si se puede rechazar $H_{0}$, se concluye que el proceso ajustado produce menos del $5\%$ de piezas defectuosas.
\end{Ejem}
A esta se le denomina \textbf{Prueba de hip\'otesis de una cola}.

La decisi\'on de rechazar o aceptar la hip\'otesis nula est\'a basada en la informaci\'on contenida en una muestra proveniente de la poblaci\'on de inter\'es. Esta informaci\'on tiene estas formas

\begin{itemize}
\item \textbf{Estad\'sitico de prueba:} un s\'olo n\'umero calculado a partir de la muestra.

\item \textbf{$p$-value:} probabilidad calculada a partir del estad\'stico de prueba.

\end{itemize}

\begin{Def}
El $p$-value es la probabilidad de observar un estad\'istico de prueba tanto o m\'as alejado del valor obervado, si en realidad $H_{0}$ es verdadera.\medskip
Valores grandes del estad\'stica de prueba  y valores peque\~nos de $p$ significan que se ha observado un evento muy poco probable, si $H_{0}$ en realidad es verdadera.
\end{Def}

Todo el conjunto de valores que puede tomar el estad\'istico de prueba se divide en dos regiones. Un conjunto, formado de valores que apoyan la hip\'otesis alternativa y llevan a rechazar $H_{0}$, se denomina \textbf{regi\'on de rechazo}. El otro, conformado por los valores que sustentatn la hip\'otesis nula, se le denomina \textbf{regi\'on de aceptaci\'on}.\medskip


Cuando la regi\'on de rechazo est\'a en la cola izquierda de la distribuci\'on, la  prueba se denomina \textbf{prueba lateral izquierda}. Una prueba con regi\'on de rechazo en la cola derecha se le llama \textbf{prueba lateral derecha}.\medskip

Si el estad\'stico de prueba cae en la regi\'on de rechazo, entonces se rechaza $H_{0}$. Si el estad\'stico de prueba cae en la regi\'on de aceptaci\'on, entonces la hip\'otesis nula se acepta o la prueba se juzga como no concluyente.\medskip

Dependiendo del nivel de confianza que se desea agregar a las conclusiones de la prueba, y el \textbf{nivel de significancia $\alpha$}, el riesgo que est\'a dispuesto a correr si se toma una decisi\'on incorrecta.

\begin{Def}
Un \textbf{error de tipo I} para una prueba estad\'istica es el error que se tiene al rechazar la hip\'otesis nula cuando es verdadera. El \textbf{nivel de significancia} para una prueba estad\'istica de hip\'otesis es
\begin{eqnarray*}
\alpha&=&P\left\{\textrm{error tipo I}\right\}=P\left\{\textrm{rechazar equivocadamente }H_{0}\right\}\\
&=&P\left\{\textrm{rechazar }H_{0}\textrm{ cuando }H_{0}\textrm{ es verdadera}\right\}
\end{eqnarray*}

\end{Def}

Este valor $\alpha$ representa el valor m\'aximo de riesgo tolerable de rechazar incorrectamente $H_{0}$. Una vez establecido el nivel de significancia, la regi\'on de rechazo se define para poder determinar si se rechaza $H_{0}$ con un cierto nivel de confianza.





\section{2.2 Muestras grandes: una media poblacional}
\subsection{2.2.1 C\'alculo de valor $p$}
\begin{frame}

\begin{Def}
El \textbf{valor de $p$} (\textbf{$p$-value}) o nivel de significancia observado de un estad\'istico de prueba es el valor m\'as peque\~ no de $\alpha$ para el cual $H_{0}$ se puede rechazar. El riesgo de cometer un error tipo $I$, si $H_{0}$ es rechazada con base en la informaci\'on que proporciona la muestra.
\end{Def}

\begin{Note}
Valores peque\~ nos de $p$ indican 	que el valor observado del estad\'stico de prueba se encuentra alejado del valor hipot\'etico de $\mu$, es decir se tiene evidencia de que $H_{0}$ es falsa y por tanto debe de rechazarse.
\end{Note}




\begin{Note}
Valores grandes de $p$ indican que el estad\'istico de prueba observado no est\'a alejado de la medi hipot\'etica y no apoya el rechazo de $H_{0}$.
\end{Note}

\begin{Def}
Si el valor de $p$ es menor o igual que el nivel de significancia $\alpha$, determinado previamente, entonces $H_{0}$ es rechazada y se puede concluir que los resultados son estad\'isticamente significativos con un nivel de confianza del $100\left(1-\alpha\right)\%$.
\end{Def}
Es usual utilizar la siguiente clasificaci\'on de resultados




\begin{tabular}{|c||c|l|}\hline
$p$& $H_{0}$&Significativa\\\hline\hline
$p<0.01$&Rechazar &Altamente\\\hline
$0.01\leq p<0.05$ & Rechazar&Estad\'isticamente\\\hline
$0.05\leq p <0.1$ & No rechazar & Tendencia estad\'istica\\\hline
$0.01\leq p$ & No rechazar & No son estad\'isticamente\\\hline
\end{tabular}

\begin{Note}
Para determinar el valor de $p$, encontrar el \'area en la cola despu\'es del estad\'istico de prueba. Si la prueba es de una cola, este es el valor de $p$. Si es de dos colas, \'este valor encontrado es la mitad del valor de $p$. Rechazar $H_{0}$ cuando el valor de $p<\alpha$.
\end{Note}




Hay dos tipos de errores al realizar una prueba de hip\'otesis
\begin{center}
\begin{tabular}{c|cc}
& $H_{0}$ es Verdadera & $H_{0}$ es Falsa\\\hline\hline
Rechazar $H_{0}$ & Error tipo I & $\surd$\\
Aceptar $H_{0}$ & $\surd$ & Error tipo II
\end{tabular}
\end{center}
\begin{Def}
La probabilidad de cometer el error tipo II se define por $\beta$ donde
\begin{eqnarray*}
\beta&=&P\left\{\textrm{error tipo II}\right\}=P\left\{\textrm{Aceptar equivocadamente }H_{0}\right\}\\
&=&P\left\{\textrm{Aceptar }H_{0}\textrm{ cuando }H_{0}\textrm{ es falsa}\right\}
\end{eqnarray*}
\end{Def}





\begin{Note}
Cuando $H_{0}$ es falsa y $H_{1}$ es verdadera, no siempre es posible especificar un valor exacto de $\mu$, sino m\'as bien un rango de posibles valores.\medskip
En lugar de arriesgarse a tomar una decisi\'on incorrecta, es mejor conlcuir que \textit{no hay evidencia suficiente para rechazar $H_{0}$}, es decir en lugar de aceptar $H_{0}$, \textit{no rechazar $H_{0}$}.

\end{Note}
La bondad de una prueba estad\'istica se mide por el tama\~ no de $\alpha$ y $\beta$, ambas deben de ser peque\~ nas. Una manera muy efectiva de medir la potencia de la prueba es calculando el complemento del error tipo $II$:
\begin{eqnarray*}
1-\beta&= &P\left\{\textrm{Rechazar }H_{0}\textrm{ cuando }H_{0}\textrm{ es falsa}\right\}\\
&=&P\left\{\textrm{Rechazar }H_{0}\textrm{ cuando }H_{1}\textrm{ es verdadera}\right\}
\end{eqnarray*}
\begin{Def}
La \textbf{potencia de la prueba}, $1-\beta$, mide la capacidad de que la prueba funciona como se necesita.
\end{Def}





\begin{Ejem}
La producci\'on diariade una planta qu\'imica local ha promediado 880 toneladas en los \'ultimos a\~nos. A la gerente de control de calidad le gustar\'ia saber si este promedio ha cambiado en meses recientes. Ella selecciona al azar 50 d\'ias de la base de datos computarizada y calcula el promedio y la desviaci\'on est\'andar de las $n=50$  producciones como $\overline{x}=871$ toneladas y $s=21$ toneladas, respectivamente. Pruebe la hip\'otesis  apropiada usando $\alpha=0.05$.

\end{Ejem}

\begin{Sol}
La hip\'otesis nula apropiada es:
\begin{eqnarray*}
H_{0}&:& \mu=880\\
&&\textrm{ y la hip\'otesis alternativa }H_{1}\textrm{ es }\\
H_{1}&:& \mu\neq880
\end{eqnarray*}
el estimador puntual para $\mu$ es $\overline{x}$, entonces el estad\'istico de prueba es\medskip
\begin{eqnarray*}
z&=&\frac{\overline{x}-\mu_{0}}{s/\sqrt{n}}\\
&=&\frac{871-880}{21/\sqrt{50}}=-3.03
\end{eqnarray*}
\end{Sol}




\begin{Sol}
Para esta prueba de  dos colas, hay que determinar los dos valores de $z_{\alpha/2}$, es decir,  $z_{\alpha/2}=\pm1.96$, como $z>z_{\alpha/2}$, $z$ cae en la zona de rechazo, por lo tanto  la gerente puede rechazar la hip\'otesis nula y concluir que el promedio efectivamente ha cambiado.\medskip
La probabilidad de rechazar $H_{0}$ cuando esta es verdadera es de  $0.05$.


Recordemos que el valor observado del estad\'istico de prueba es $z=-3.03$, la regi\'on de rechazo m\'as peque\~na que puede usarse y todav\'ia seguir rechazando $H_{0}$ es $|z|>3.03$, \\
entonces $p=2(0.012)=0.0024$, que a su vez es menor que el nivel de significancia $\alpha$ asignado inicialmente, y adem\'as los resultados son  \textbf{altamente significativos}.


\end{Sol}





Finalmente determinemos la potencia de la prueba cuando $\mu$ en realidad es igual a $870$ toneladas.

Recordar que la regi\'on de aceptaci\'on est\'a entre $-1.96$ y $1.96$, para $\mu=880$, equivalentemente $$874.18<\overline{x}<885.82$$
$\beta$ es la probabilidad de aceptar $H_{0}$ cuando $\mu=870$, calculemos los valores de $z$ correspondientes a $874.18$ y $885.82$ \medskip
Entonces
\begin{eqnarray*}
z_{1}&=&\frac{\overline{x}-\mu}{s/\sqrt{n}}=\frac{874.18-870}{21/\sqrt{50}}=1.41\\
z_{1}&=&\frac{\overline{x}-\mu}{s/\sqrt{n}}=\frac{885.82-870}{21/\sqrt{50}}=5.33
\end{eqnarray*}
por lo tanto
\begin{eqnarray*}
\beta&=&P\left\{\textrm{aceptar }H_{0}\textrm{ cuando }H_{0}\textrm{ es falsa}\right\}\\
&=&P\left\{874.18<\mu<885.82\textrm{ cuando }\mu=870\right\}\\
&=&P\left\{1.41<z<5.33\right\}=P\left\{1.41<z\right\}\\
&=&1-0.9207=0.0793
\end{eqnarray*}
entonces, la potencia de la prueba es
$$1-\beta=1-0.0793=0.9207$$ que es la probabilidad de rechazar correctamente $H_{0}$ cuando $H_{0}$ es falsa.




Determinar la potencia de la prueba para distintos valores de $H_{1}$ y graficarlos, \textit{curva de potencia}
\begin{center}
\begin{tabular}{c||c}
$H_{1}$ & $\left(1-\beta\right)$ \\\hline 
\hline 
865 &  \\ \hline 
870 &  \\ \hline 
872 &  \\ \hline 
875 &  \\ \hline 
877 &  \\ \hline 
880 &  \\ \hline 
883 &  \\ \hline 
885 &  \\ \hline 
888 &  \\ \hline 
890 &  \\ \hline 
895 &  \\ \hline 
\end{tabular} 

\end{center}




\begin{enumerate}
\item Encontrar las regiones de rechazo para el estad\'istico $z$, para una prueba de
\begin{itemize}
\item[a) ]  dos colas para $\alpha=0.01,0.05,0.1$
\item[b) ]  una cola superior para $\alpha=0.01,0.05,0.1$
\item[c) ] una cola inferior para $\alpha=0.01,0.05,0.1$

\end{itemize}


\item Suponga que el valor del estad\'istico de prueba es 
\begin{itemize}
\item[a) ]$z=-2.41$, sacar las conclusiones correspondientes para los incisos anteriores.
\item[b) ] $z=2.16$, sacar las conclusiones correspondientes para los incisos anteriores.
\item[c) ] $z=1.15$, sacar las conclusiones correspondientes para los incisos anteriores.
\item[d) ] $z=-2.78$, sacar las conclusiones correspondientes para los incisos anteriores.
\item[e) ] $z=-1.81$, sacar las conclusiones correspondientes para los incisos anteriores.

\end{itemize}
\end{enumerate}





\begin{itemize}
\item[3. ] Encuentre el valor de $p$ para las pruebas de hip\'otesis correspondientes a los valores de $z$ del ejercicio anterior.

\item[4. ] Para las pruebas dadas en el ejercicio 2, utilice el valor de $p$, determinado en el ejercicio 3,  para determinar la significancia de los resultados.


\end{itemize}


\begin{itemize}
\item[5. ] Una muestra aleatoria de $n=45$ observaciones de una poblaci\'on con media $\overline{x}=2.4$, y desviaci\'on est\'andar $s=0.29$. Suponga que el objetivo es demostrar que la media poblacional $\mu$ excede $2.3$.
\begin{itemize}
\item[a) ] Defina la hip\'otesis nula y alternativa para la prueba.
\item[b) ] Determine la regi\'on de rechazo para un nivel de significancia de: $\alpha=0.1,0.05,0.01$.
\item[c) ] Determine el error est\'andar de la media muestral.
\item[d) ] Calcule el valor de $p$ para los estad\'isticos de prueba definidos en los incisos anteriores.
\item[e) ] Utilice el valor de $p$ pra sacar una conclusi\'on al nivel de significancia $\alpha$.
\item[f) ] Determine el valor de $\beta$ cuando $\mu=2.5$
\item[g) ] Graficar la curva de potencia para la prueba.

\end{itemize}
\end{itemize}


\subsection{2.2.2 Prueba de hip\'otesis para la diferencia entre dos medias poblacionales}



El estad\'istico que resume la informaci\'on muestral respecto a la diferencia en medias poblacionales $\left(\mu_{1}-\mu_{2}\right)$ es la diferencia de las medias muestrales $\left(\overline{x}_{1}-\overline{x}_{2}\right)$, por tanto al probar la difencia entre las medias muestrales se verifica que la diferencia real entre las medias poblacionales difiere de un valor especificado, $\left(\mu_{1}-\mu_{2}\right)=D_{0}$, se puede usar el error est\'andar de $\left(\overline{x}_{1}-\overline{x}_{2}\right)$, es decir
$$\sqrt{\frac{\sigma^{2}_{1}}{n_{1}}+\frac{\sigma^{2}_{2}}{n_{2}}}$$
cuyo estimador est\'a dado por
$$SE=\sqrt{\frac{s^{2}_{1}}{n_{1}}+\frac{s^{2}_{2}}{n_{2}}}$$
El procedimiento para muestras grandes es:



\begin{itemize}
\item[1) ] \textbf{Hip\'otesis Nula} $H_{0}:\left(\mu_{1}-\mu_{2}\right)=D_{0}$,\medskip

donde $D_{0}$ es el valor, la diferencia, espec\'ifico que se desea probar. En algunos casos se querr\'a demostrar que no hay diferencia alguna, es decir $D_{0}=0$.

\item[2) ] \textbf{Hip\'otesis Alternativa}
\begin{tabular}{cc}\hline
\textbf{Prueba de una Cola} & \textbf{Prueba de dos colas}\\\hline
$H_{1}:\left(\mu_{1}-\mu_{2}\right)>D_{0}$ & $H_{1}:\left(\mu_{1}-\mu_{2}\right)\neq D_{0}$\\ 
$H_{1}:\left(\mu_{1}-\mu_{2}\right)<D_{0}$&\\
\end{tabular}

\end{itemize}




\begin{itemize}
\item[3) ] Estad\'istico de prueba:
$$z=\frac{\left(\overline{x}_{1}-\overline{x}_{2}\right)-D_{0}}{\sqrt{\frac{s^{2}_{1}}{n_{1}}+\frac{s^{2}_{2}}{n_{2}}}}$$
\item[4) ] Regi\'on de rechazo: rechazar $H_{0}$ cuando
\begin{tabular}{cc}\hline
\textbf{Prueba de una Cola} & \textbf{Prueba de dos colas}\\\hline
$z>z_{0}$ & \\
$z<-z_{\alpha}$ cuando $H_{1}:\left(\mu_{1}-\mu_{2}\right)<D_{0}$&$z>z_{\alpha/2}$ o $z<-z_{\alpha/2}$\\
 cuando $p<\alpha$&\\
\end{tabular}


\end{itemize}





\begin{Ejem}
Para determinar si ser propietario de un autom\'ovil afecta el rendimiento acad\'emico de un estudiante, se tomaron dos muestras aleatorias de 100 estudiantes varones. El promedio de calificaciones para los $n_{1}=100$ no propietarios de un auto tuvieron un promedio y varianza de $\overline{x}_{1}=2.7$ y $s_{1}^{2}=0.36$, respectivamente, mientras que para para la segunda muestra con $n_{2}=100$ propietarios de un auto, se tiene $\overline{x}_{2}=2.54$ y $s_{2}^{2}=0.4$. Los datos presentan suficiente evidencia para indicar una diferencia en la media en el rendimiento acad\'emico entre propietarios y no propietarios de un autom\'ovil? Hacer pruebas para $\alpha=0.01,0.05$ y $\alpha=0.1$.
\end{Ejem}

\begin{Sol}
\begin{itemize}
\item Soluci\'on utilizando la t\'ecnica de regiones de rechazo:\medskip
realizando las operaciones
$z=1.84$, determinar si excede los valores de $z_{\alpha/2}$.
\item Soluci\'on utilizando el $p$-value:\medskip
Calcular el valor de $p$, la probabilidad de que $z$ sea mayor que $z=1.84$ o menor que $z=-1.84$, se tiene que $p=0.0658$. Concluir.
\end{itemize}
\end{Sol}



\begin{itemize}
\item Si el intervalo de confianza que se construye contiene el valor del par\'ametro especificado por $H_{0}$, entonces ese valor es uno de los posibles valores del par\'ametro y $H_{0}$ no debe ser rechazada.

\item Si el valor hipot\'etico se encuentra fuera de los l\'imites de confianza, la hip\'otesis nula es rechazada al nivel de significancia $\alpha$.
\end{itemize}

\begin{enumerate}
\item Del libro Mendenhall resolver los ejercicios 9.18, 9.19 y 9.20(\href{https://cu.uacm.edu.mx/nextcloud/index.php/f/202873}{Mendenhall}).

\item Del libro \href{https://cu.uacm.edu.mx/nextcloud/index.php/f/202873}{Mendenhall} resolver los ejercicios: 9.23, 9.26 y 9.28.
\end{enumerate}



\subsection{2.2.3 Prueba de Hip\'otesis para una Proporci\'on Binomial}


Para una muestra aleatoria de $n$ intentos id\'enticos, de una poblaci\'on binomial, la proporci\'on muesrtal $\hat{p}$ tiene una distribuci\'on aproximadamente normal cuando $n$ es grande, con media $p$ y error est\'andar
$$SE=\sqrt{\frac{pq}{n}}.$$
La prueba de hip\'otesis de la forma
\begin{eqnarray*}
H_{0}&:&p=p_{0}\\
H_{1}&:&p>p_{0}\textrm{, o }p<p_{0}\textrm{ o }p\neq p_{0}
\end{eqnarray*}
El estad\'istico de prueba se construye con el mejor estimador de la proporci\'on verdadera, $\hat{p}$, con el estad\'istico de prueba $z$, que se distribuye normal est\'andar.



El procedimiento es
\begin{itemize}
\item[1) ] Hip\'otesis nula: $H_{0}:p=p_{0}$
\item[2) ] Hip\'otesis alternativa
\begin{tabular}{cc}\hline
\textbf{Prueba de una Cola} & \textbf{Prueba de dos colas}\\\hline
$H_{1}:p>p_{0}$ & $p\neq p_{0}$\\
$H_{1}:p<p_{0}$ & \\
\end{tabular}
\item[3) ] Estad\'istico de prueba:
\begin{eqnarray*}
z=\frac{\hat{p}-p_{0}}{\sqrt{\frac{pq}{n}}},\hat{p}=\frac{x}{n}
\end{eqnarray*}
donde $x$ es el n\'umero de \'exitos en $n$ intentos binomiales.

\end{itemize}



\begin{itemize}
\item[4) ] Regi\'on de rechazo: rechazar $H_{0}$ cuando
\begin{tabular}{cc}\hline
\textbf{Prueba de una Cola} & \textbf{Prueba de dos colas}\\\hline
$z>z_{0}$ & \\
$z<-z_{\alpha}$ cuando $H_{1}:p<p_{0}$&$z>z_{\alpha/2}$ o $z<-z_{\alpha/2}$\\
 cuando $p<\alpha$&\\
\end{tabular}
\end{itemize}



\begin{Ejem}
A cualquier edad, alrededor del $20\%$ de los adultos de cierto pa\'is realiza actividades de acondicionamiento f\'isico al menos dos veces por semana. En una encuesta local de $n=100$ adultos de m\'as de $40$ a\ ~nos, un total de 15 personas indicaron que realizaron actividad f\'isica al menos dos veces por semana. Estos datos indican que el porcentaje de participaci\'on para adultos de m\'as de 40 a\ ~nos de edad es  considerablemente menor a la cifra del $20\%$? Calcule el valor de $p$ y \'uselo para sacar las conclusiones apropiadas.
\end{Ejem}

\begin{enumerate}
\item Resolver los ejercicios: 9.30, 9.32, 9.33, 9.35 y 9.39.
\end{enumerate}



\subsection{2.2.4 Prueba de Hip\'otesis diferencia entre dos Proporciones Binomiales}




\begin{Note}
Cuando se tienen dos muestras aleatorias independientes de dos poblaciones binomiales, el objetivo del experimento puede ser la diferencia $\left(p_{1}-p_{2}\right)$ en las proporciones de individuos u objetos que poseen una caracter\'istica especifica en las dos poblaciones. En este caso se pueden utilizar los estimadores de las dos proporciones $\left(\hat{p}_{1}-\hat{p}_{2}\right)$ con error est\'andar dado por
$$SE=\sqrt{\frac{p_{1}q_{1}}{n_{1}}+\frac{p_{2}q_{2}}{n_{2}}}$$
considerando el estad\'istico $z$ con un nivel de significancia $\left(1-\alpha\right)100\%$

\end{Note}


\begin{Note}
La hip\'otesis nula a probarse es de la forma
\begin{itemize}
\item[$H_{0}$: ] $p_{1}=p_{2}$ o equivalentemente $\left(p_{1}-p_{2}\right)=0$, contra una hip\'otesis alternativa $H_{1}$ de una o dos colas.
\end{itemize}
\end{Note}





\begin{Note}
Para estimar el error est\'andar del estad\'istico $z$, se debe de utilizar el hecho de que suponiendo que $H_{0}$ es verdadera, las dos proporciones son iguales a alg\'un valor com\'un, $p$. Para obtener el mejor estimador de $p$ es
$$p=\frac{\textrm{n\'umero total de \'exitos}}{\textrm{N\'umero total de pruebas}}=\frac{x_{1}+x_{2}}{n_{1}+n_{2}}$$
\end{Note}



\begin{itemize}
\item[1) ] \textbf{Hip\'otesis Nula:} $H_{0}:\left(p_{1}-p_{2}\right)=0$
\item[2) ] \textbf{Hip\'otesis Alternativa: } $H_{1}:$
\begin{tabular}{cc}\hline
\textbf{Prueba de una Cola} & \textbf{Prueba de dos colas}\\\hline
$H_{1}:\left(p_{1}-p_{2}\right)>0$ & $H_{1}:\left(p_{1}-p_{2}\right)\neq 0$\\ 
$H_{1}:\left(p_{1}-p_{2}\right)<0$&\\
\end{tabular}
\item[3) ] Estad\'istico de prueba:
\begin{eqnarray*}
z=\frac{\left(\hat{p}_{1}-\hat{p}_{2}\right)}{\sqrt{\frac{p_{1}q_{1}}{n_{1}}+\frac{p_{2}q_{2}}{n_{2}}}}=\frac{\left(\hat{p}_{1}-\hat{p}_{2}\right)}{\sqrt{\frac{pq}{n_{1}}+\frac{pq}{n_{2}}}}
\end{eqnarray*}
donde $\hat{p_{1}}=x_{1}/n_{1}$ y $\hat{p_{2}}=x_{2}/n_{2}$ , dado que el valor com\'un para $p_{1}$ y $p_{2}$ es $p$, entonces $\hat{p}=\frac{x_{1}+x_{2}}{n_{1}+n_{2}}$ y por tanto el estad\'istico de prueba es
\end{itemize}






\begin{eqnarray*}
z=\frac{\hat{p}_{1}-\hat{p}_{2}}{\sqrt{\hat{p}\hat{q}}\left(\frac{1}{n_{1}}+\frac{1}{n_{2}}\right)}
\end{eqnarray*}
\begin{itemize}
\item[4) ] Regi\'on de rechazo: rechazar $H_{0}$ cuando
\begin{tabular}{cc}\hline
\textbf{Prueba de una Cola} & \textbf{Prueba de dos colas}\\\hline
$z>z_{\alpha}$ & \\
$z<-z_{\alpha}$ cuando $H_{1}:p<p_{0}$&$z>z_{\alpha/2}$ o $z<-z_{\alpha/2}$\\
 cuando $p<\alpha$&\\
\end{tabular}

\end{itemize}





\begin{Ejem}
Los registros de un hospital, indican que 52 hombres de una muestra de 1000 contra 23 mujeres de una muestra de 1000 fueron ingresados por enfermedad del coraz\'on. Estos datos presentan suficiente evidencia para indicar un porcentaje m\'as alto de enfermedades del coraz\'on entre hombres ingresados al hospital?, utilizar distintos niveles de confianza de $\alpha$.

\end{Ejem}
\begin{enumerate}
\item Resolver los ejercicios 9.42

\item Resolver los ejercicios: 9.45, 9.48, 9.50
\end{enumerate}





\section{2.3 Muestras Peque\~nas}

\subsection{2.3.1 Una media poblacional}




\begin{itemize}
\item[1) ] \textbf{Hip\'otesis Nula:} $H_{0}:\mu=\mu_{0}$
\item[2) ] \textbf{Hip\'otesis Alternativa: } $H_{1}:$
\begin{tabular}{cc}\hline
\textbf{Prueba de una Cola} & \textbf{Prueba de dos colas}\\\hline
$H_{1}:\mu>\mu_{0}$ & $H_{1}:\mu\neq \mu_{0}$\\ 
$H_{1}:\mu<\mu0$&\\
\end{tabular}
\item[3) ] Estad\'istico de prueba:
\begin{eqnarray*}
t=\frac{\overline{x}-\mu_{0}}{\sqrt{\frac{s^{2}}{n}}}
\end{eqnarray*}
\item[4) ] Regi\'on de rechazo: rechazar $H_{0}$ cuando
\begin{tabular}{cc}\hline
\textbf{Prueba de una Cola} & \textbf{Prueba de dos colas}\\\hline
$t>t_{\alpha}$ & \\
$t<-t_{\alpha}$ cuando $H_{1}:\mu<mu_{0}$&$t>t_{\alpha/2}$ o $t<-t_{\alpha/2}$\\
 cuando $p<\alpha$&\\
\end{tabular}
\end{itemize}





\begin{Ejem}
Las etiquetas en latas de un gal'on de pintura por lo general indican el tiempo de secado y el \'area puede cubrir una capa. Casi todas las marcas de pintura indican que, en una capa, un gal\'on cubrir\'a entre 250 y 500 pies cuadrados, dependiento de la textura de la superficie a pintarse, un fabricante, sin embargo afirma que un gal\'on de su pintura cubrir\'a 400 pies cuadrados de \'area superficial. Para probar su afirmaci\'on, una muestra aleatoria de 10 latas de un gal\'on de pintura blanca se emple\'o para pintar 10 \'areas id\'enticas usando la misma clase de equipo. Las \'areas reales en pies cuadrados cubiertas por estos 10 galones de pintura se dan a continuac\'on:
\begin{center}
\begin{tabular}{|c|c|c|c|c|}
\hline 
310 & 311 & 412 & 368 & 447 \\ 
\hline 
376 & 303 &410 &365 & 350 \\ 
\hline 
\end{tabular} 
\end{center}
\end{Ejem}





\begin{Ejem}
Los datos presentan suficiente evidencia para indicar que el promedio de la cobertura difiere de 400 pies cuadrados? encuentre el valor de $p$ para la prueba y \'uselo para evaluar la significancia de los resultados.
\end{Ejem}
\begin{enumerate}
\item Resolver los ejercicios: 10.2, 10.3,10.5, 10.7, 10.9, 10.13 y 10.16
\end{enumerate}




\subsection{2.3.2 Diferencia entre dos medias poblacionales: M.A.I.}




\begin{Note}
Cuando los tama\ ~nos de muestra son peque\ ~nos, no se puede asegurar que las medias muestrales sean normales, pero si las poblaciones originales son normales, entonces la distribuci\'on muestral de la diferencia de las medias muestales, $\left(\overline{x}_{1}-\overline{x}_{2}\right)$, ser\'a normal con media $\left(\mu_{1}-\mu_{2}\right)$ y error est\'andar $$ES=\sqrt{\frac{\sigma_{1}^{2}}{n_{1}}+\frac{\sigma_{2}^{2}}{n_{2}}}$$

\end{Note}

\begin{itemize}
\item[1) ] \textbf{Hip\'otesis Nula} $H_{0}:\left(\mu_{1}-\mu_{2}\right)=D_{0}$,\medskip

donde $D_{0}$ es el valor, la diferencia, espec\'ifico que se desea probar. En algunos casos se querr\'a demostrar que no hay diferencia alguna, es decir $D_{0}=0$.

\item[2) ] \textbf{Hip\'otesis Alternativa}
\begin{tabular}{cc}\hline
\textbf{Prueba de una Cola} & \textbf{Prueba de dos colas}\\\hline
$H_{1}:\left(\mu_{1}-\mu_{2}\right)>D_{0}$ & $H_{1}:\left(\mu_{1}-\mu_{2}\right)\neq D_{0}$\\ 
$H_{1}:\left(\mu_{1}-\mu_{2}\right)<D_{0}$&\\
\end{tabular}

\item[3) ] Estad\'istico de prueba:
$$t=\frac{\left(\overline{x}_{1}-\overline{x}_{2}\right)-D_{0}}{\sqrt{\frac{s^{2}_{1}}{n_{1}}+\frac{s^{2}_{2}}{n_{2}}}}$$
\end{itemize}






donde $$s^{2}=\frac{\left(n_{1}-1\right)s_{1}^{2}+\left(n_{2}-1\right)s_{2}^{2}}{n_{1}+n_{2}-2}$$
\begin{itemize}

\item[4) ] Regi\'on de rechazo: rechazar $H_{0}$ cuando
\begin{tabular}{cc}\hline
\textbf{Prueba de una Cola} & \textbf{Prueba de dos colas}\\\hline
$z>z_{0}$ & \\
$z<-z_{\alpha}$ cuando $H_{1}:\left(\mu_{1}-\mu_{2}\right)<D_{0}$&$z>z_{\alpha/2}$ o $z<-z_{\alpha/2}$\\
 cuando $p<\alpha$&\\
\end{tabular}
Los valores cr\'iticos de $t$, $t_{-\alpha}$ y $t_{\alpha/2}$ est\'an basados en $\left(n_{1}+n_{2}-2\right)$ grados de libertad.


\end{itemize}




\subsection{2.3.3 Diferencia entre dos medias poblacionales: Diferencias Pareadas}






\begin{itemize}
\item[1) ] \textbf{Hip\'otesis Nula:} $H_{0}:\mu_{d}=0$
\item[2) ] \textbf{Hip\'otesis Alternativa: } $H_{1}:\mu_{d}$
\begin{tabular}{cc}\hline
\textbf{Prueba de una Cola} & \textbf{Prueba de dos colas}\\\hline
$H_{1}:\mu_{d}>0$ & $H_{1}:\mu_{d}\neq 0$\\ 
$H_{1}:\mu_{d}<0$&\\
\end{tabular}
\item[3) ] Estad\'istico de prueba:
\begin{eqnarray*}
t=\frac{\overline{d}}{\sqrt{\frac{s_{d}^{2}}{n}}}
\end{eqnarray*}
donde $n$ es el n\'umero de diferencias pareadas, $\overline{d}$ es la media de las diferencias muestrales, y $s_{d}$ es la desviaci\'on est\'andar de las diferencias muestrales.

\end{itemize}






\begin{itemize}
\item[4) ] Regi\'on de rechazo: rechazar $H_{0}$ cuando
\begin{tabular}{cc}\hline
\textbf{Prueba de una Cola} & \textbf{Prueba de dos colas}\\\hline
$t>t_{\alpha}$ & \\
$t<-t_{\alpha}$ cuando $H_{1}:\mu<mu_{0}$&$t>t_{\alpha/2}$ o $t<-t_{\alpha/2}$\\
 cuando $p<\alpha$&\\
\end{tabular}

Los valores cr\'iticos de $t$, $t_{-\alpha}$ y $t_{\alpha/2}$ est\'an basados en $\left(n_{1}+n_{2}-2\right)$ grados de libertad.

\end{itemize}






\subsection{2.3.4 Inferencias con respecto a la Varianza Poblacional}




\begin{itemize}
\item[1) ] \textbf{Hip\'otesis Nula:} $H_{0}:\sigma^{2}=\sigma^{2}_{0}$
\item[2) ] \textbf{Hip\'otesis Alternativa: } $H_{1}$
\begin{tabular}{cc}\hline
\textbf{Prueba de una Cola} & \textbf{Prueba de dos colas}\\\hline
$H_{1}:\sigma^{2}>\sigma^{2}_{0}$ & $H_{1}:\sigma^{2}\neq \sigma^{2}_{0}$\\ 
$H_{1}:\sigma^{2}<\sigma^{2}_{0}$&\\
\end{tabular}
\item[3) ] Estad\'istico de prueba:
\begin{eqnarray*}
\chi^{2}=\frac{\left(n-1\right)s^{2}}{\sigma^{2}_{0}}
\end{eqnarray*}

\end{itemize}






\begin{itemize}
\item[4) ] Regi\'on de rechazo: rechazar $H_{0}$ cuando
\begin{tabular}{cc}\hline
\textbf{Prueba de una Cola} & \textbf{Prueba de dos colas}\\\hline
$\chi^{2}>\chi^{2}_{\alpha}$ & \\
$\chi^{2}<\chi^{2}_{\left(1-\alpha\right)}$ cuando $H_{1}:\chi^{2}<\chi^{2}_{0}$&$\chi^{2}>\chi^{2}_{\alpha/2}$ o $\chi^{2}<\chi^{2}_{\left(1-\alpha/2\right)}$\\
 cuando $p<\alpha$&\\
\end{tabular}

Los valores cr\'iticos de $\chi^{2}$,est\'an basados en $\left(n_{1}+\right)$ grados de libertad.

\end{itemize}





\subsection{2.3.5 Comparaci\'on de dos varianzas poblacionales}





\begin{itemize}
\item[1) ] \textbf{Hip\'otesis Nula} $H_{0}:\left(\sigma^{2}_{1}-\sigma^{2}_{2}\right)=D_{0}$,\medskip

donde $D_{0}$ es el valor, la diferencia, espec\'ifico que se desea probar. En algunos casos se querr\'a demostrar que no hay diferencia alguna, es decir $D_{0}=0$.

\item[2) ] \textbf{Hip\'otesis Alternativa}
\begin{tabular}{cc}\hline
\textbf{Prueba de una Cola} & \textbf{Prueba de dos colas}\\\hline
$H_{1}:\left(\sigma^{2}_{1}-\sigma^{2}_{2}\right)>D_{0}$ & $H_{1}:\left(\sigma^{2}_{1}-\sigma^{2}_{2}\right)\neq D_{0}$\\ 
$H_{1}:\left(\sigma^{2}_{1}-\sigma^{2}_{2}\right)<D_{0}$&\\
\end{tabular}

\end{itemize}







\begin{itemize}
\item[3) ] Estad\'istico de prueba:
$$F=\frac{s_{1}^{2}}{s_{2}^{2}}$$
donde $s_{1}^{2}$ es la varianza muestral m\'as grande.
\item[4) ] Regi\'on de rechazo: rechazar $H_{0}$ cuando
\begin{tabular}{cc}\hline
\textbf{Prueba de una Cola} & \textbf{Prueba de dos colas}\\\hline
$F>F_{\alpha}$ & $F>F_{\alpha/2}$\\
 cuando $p<\alpha$&\\
\end{tabular}


\end{itemize}




%---------------------------------------------------------
\section{2. Pruebas de Hip\'otesis}
%---------------------------------------------------------
\subsection{2.1 Tipos de errores}





\begin{itemize}
\item Una hip\'otesis estad\'istica es una afirmaci\'on  acerca de la distribuci\'on de probabilidad de una variable aleatoria, a menudo involucran uno o m\'as par\'ametros de la distribuci\'on.

\item Las hip\'otesis son afirmaciones respecto a la poblaci\'on o distribuci\'on bajo estudio, no en torno a la muestra.

\item La mayor\'ia de las veces, la prueba de hip\'otesis consiste en determinar si la situaci \'on experimental ha cambiado

\item el inter\'es principal es decidir sobre la veracidad o falsedad de una hip\'otesis, a este procedimiento se le llama \textit{prueba de hip\'otesis}.

\item Si la informaci\'on es consistente con la hip\'otesis, se concluye que esta es verdadera, de lo contrario que con base en la informaci\'on, es falsa.

\end{itemize}







Una prueba de hip\'otesis est\'a formada por cinco partes
\begin{itemize}
\item La hip\'otesis nula, denotada por $H_{0}$.
\item La hip\'otesis alterativa, denorada por $H_{1}$.
\item El estad\'sitico de prueba y su valor $p$.
\item La regi\'on de rechazo.
\item La conclusi\'on.

\end{itemize}


\begin{Def}
Las dos hip\'otesis en competencias son la \textbf{hip\'otesis alternativa $H_{1}$}, usualmente la que se desea apoyar, y la \textbf{hip\'otesis nula $H_{0}$}, opuesta a $H_{1}$.
\end{Def}







En general, es m\'as f\'acil presentar evidencia de que $H_{1}$ es cierta, que demostrar 	que $H_{0}$ es falsa, es por eso que por lo regular se comienza suponiendo que $H_{0}$ es cierta, luego se utilizan los datos de la muestra para decidir si existe evidencia a favor de $H_{1}$, m\'as que a favor de $H_{0}$, as\'i se tienen dos conclusiones:
\begin{itemize}
\item Rechazar $H_{0}$ y concluir que $H_{1}$ es verdadera.
\item Aceptar, no rechazar, $H_{0}$ como verdadera.

\end{itemize}







\begin{Ejem}
Se desea demostrar que el salario promedio  por hora en cierto lugar es distinto de $19$usd, que es el promedio nacional. Entonces $H_{1}:\mu\neq19$, y $H_{0}:\mu=19$.
\end{Ejem}
A esta se le denomina \textbf{Prueba de hip\'otesis de dos colas}.


\begin{Ejem}
Un determinado proceso produce un promedio de $5\%$ de piezas defectuosas. Se est\'a interesado en demostrar que un simple ajuste en una m\'aquina reducir\'a $p$, la proporci\'on de piezas defectuosas producidas en este proceso. Entonces se tiene $H_{0}:p<0.3$ y $H_{1}:p=0.03$. Si se puede rechazar $H_{0}$, se concluye que el proceso ajustado produce menos del $5\%$ de piezas defectuosas.
\end{Ejem}
A esta se le denomina \textbf{Prueba de hip\'otesis de una cola}.






La decisi\'on de rechazar o aceptar la hip\'otesis nula est\'a basada en la informaci\'on contenida en una muestra proveniente de la poblaci\'on de inter\'es. Esta informaci\'on tiene estas formas

\begin{itemize}
\item \textbf{Estad\'sitico de prueba:} un s\'olo n\'umero calculado a partir de la muestra.

\item \textbf{$p$-value:} probabilidad calculada a partir del estad\'stico de prueba.

\end{itemize}






\begin{Def}
El $p$-value es la probabilidad de observar un estad\'istico de prueba tanto o m\'as alejado del valor obervado, si en realidad $H_{0}$ es verdadera.\medskip
Valores grandes del estad\'stica de prueba  y valores peque\~nos de $p$ significan que se ha observado un evento muy poco probable, si $H_{0}$ en realidad es verdadera.
\end{Def}

Todo el conjunto de valores que puede tomar el estad\'istico de prueba se divide en dos regiones. Un conjunto, formado de valores que apoyan la hip\'otesis alternativa y llevan a rechazar $H_{0}$, se denomina \textbf{regi\'on de rechazo}. El otro, conformado por los valores que sustentatn la hip\'otesis nula, se le denomina \textbf{regi\'on de aceptaci\'on}.\medskip







Cuando la regi\'on de rechazo est\'a en la cola izquierda de la distribuci\'on, la  prueba se denomina \textbf{prueba lateral izquierda}. Una prueba con regi\'on de rechazo en la cola derecha se le llama \textbf{prueba lateral derecha}.\medskip

Si el estad\'stico de prueba cae en la regi\'on de rechazo, entonces se rechaza $H_{0}$. Si el estad\'stico de prueba cae en la regi\'on de aceptaci\'on, entonces la hip\'otesis nula se acepta o la prueba se juzga como no concluyente.\medskip

Dependiendo del nivel de confianza que se desea agregar a las conclusiones de la prueba, y el \textbf{nivel de significancia $\alpha$}, el riesgo que est\'a dispuesto a correr si se toma una decisi\'on incorrecta.






\begin{Def}
Un \textbf{error de tipo I} para una prueba estad\'istica es el error que se tiene al rechazar la hip\'otesis nula cuando es verdadera. El \textbf{nivel de significancia} para una prueba estad\'istica de hip\'otesis es
\begin{eqnarray*}
\alpha&=&P\left\{\textrm{error tipo I}\right\}=P\left\{\textrm{rechazar equivocadamente }H_{0}\right\}\\
&=&P\left\{\textrm{rechazar }H_{0}\textrm{ cuando }H_{0}\textrm{ es verdadera}\right\}
\end{eqnarray*}

\end{Def}
Este valor $\alpha$ representa el valor m\'aximo de riesgo tolerable de rechazar incorrectamente $H_{0}$. Una vez establecido el nivel de significancia, la regi\'on de rechazo se define para poder determinar si se rechaza $H_{0}$ con un cierto nivel de confianza.




\section{2.2 Muestras grandes: una media poblacional}
\subsection{2.2.1 C\'alculo de valor $p$}





\begin{Def}
El \textbf{valor de $p$} (\textbf{$p$-value}) o nivel de significancia observado de un estad\'istico de prueba es el valor m\'as peque\~ no de $\alpha$ para el cual $H_{0}$ se puede rechazar. El riesgo de cometer un error tipo $I$, si $H_{0}$ es rechazada con base en la informaci\'on que proporciona la muestra.
\end{Def}

\begin{Note}
Valores peque\~ nos de $p$ indican 	que el valor observado del estad\'stico de prueba se encuentra alejado del valor hipot\'etico de $\mu$, es decir se tiene evidencia de que $H_{0}$ es falsa y por tanto debe de rechazarse.
\end{Note}









\begin{Note}
Valores grandes de $p$ indican que el estad\'istico de prueba observado no est\'a alejado de la medi hipot\'etica y no apoya el rechazo de $H_{0}$.
\end{Note}

\begin{Def}
Si el valor de $p$ es menor o igual que el nivel de significancia $\alpha$, determinado previamente, entonces $H_{0}$ es rechazada y se puede concluir que los resultados son estad\'isticamente significativos con un nivel de confianza del $100\left(1-\alpha\right)\%$.
\end{Def}
Es usual utilizar la siguiente clasificaci\'on de resultados









\begin{tabular}{|c||c|l|}\hline
$p$& $H_{0}$&Significativa\\\hline\hline
$p<0.01$&Rechazar &Altamente\\\hline
$0.01\leq p<0.05$ & Rechazar&Estad\'isticamente\\\hline
$0.05\leq p <0.1$ & No rechazar & Tendencia estad\'istica\\\hline
$0.01\leq p$ & No rechazar & No son estad\'isticamente\\\hline
\end{tabular}

\begin{Note}
Para determinar el valor de $p$, encontrar el \'area en la cola despu\'es del estad\'istico de prueba. Si la prueba es de una cola, este es el valor de $p$. Si es de dos colas, \'este valor encontrado es la mitad del valor de $p$. Rechazar $H_{0}$ cuando el valor de $p<\alpha$.
\end{Note}








Hay dos tipos de errores al realizar una prueba de hip\'otesis
\begin{center}
\begin{tabular}{c|cc}
& $H_{0}$ es Verdadera & $H_{0}$ es Falsa\\\hline\hline
Rechazar $H_{0}$ & Error tipo I & $\surd$\\
Aceptar $H_{0}$ & $\surd$ & Error tipo II
\end{tabular}
\end{center}
\begin{Def}
La probabilidad de cometer el error tipo II se define por $\beta$ donde
\begin{eqnarray*}
\beta&=&P\left\{\textrm{error tipo II}\right\}=P\left\{\textrm{Aceptar equivocadamente }H_{0}\right\}\\
&=&P\left\{\textrm{Aceptar }H_{0}\textrm{ cuando }H_{0}\textrm{ es falsa}\right\}
\end{eqnarray*}
\end{Def}







\begin{Note}
Cuando $H_{0}$ es falsa y $H_{1}$ es verdadera, no siempre es posible especificar un valor exacto de $\mu$, sino m\'as bien un rango de posibles valores.\medskip
En lugar de arriesgarse a tomar una decisi\'on incorrecta, es mejor conlcuir que \textit{no hay evidencia suficiente para rechazar $H_{0}$}, es decir en lugar de aceptar $H_{0}$, \textit{no rechazar $H_{0}$}.

\end{Note}






La bondad de una prueba estad\'istica se mide por el tama\~ no de $\alpha$ y $\beta$, ambas deben de ser peque\~ nas. Una manera muy efectiva de medir la potencia de la prueba es calculando el complemento del error tipo $II$:
\begin{eqnarray*}
1-\beta&= &P\left\{\textrm{Rechazar }H_{0}\textrm{ cuando }H_{0}\textrm{ es falsa}\right\}\\
&=&P\left\{\textrm{Rechazar }H_{0}\textrm{ cuando }H_{1}\textrm{ es verdadera}\right\}
\end{eqnarray*}
\begin{Def}
La \textbf{potencia de la prueba}, $1-\beta$, mide la capacidad de que la prueba funciona como se necesita.
\end{Def}







\begin{Ejem}
La producci\'on diariade una planta qu\'imica local ha promediado 880 toneladas en los \'ultimos a\~nos. A la gerente de control de calidad le gustar\'ia saber si este promedio ha cambiado en meses recientes. Ella selecciona al azar 50 d\'ias de la base de datos computarizada y calcula el promedio y la desviaci\'on est\'andar de las $n=50$  producciones como $\overline{x}=871$ toneladas y $s=21$ toneladas, respectivamente. Pruebe la hip\'otesis  apropiada usando $\alpha=0.05$.

\end{Ejem}

\begin{Sol}
La hip\'otesis nula apropiada es:
\begin{eqnarray*}
H_{0}&:& \mu=880\\
&&\textrm{ y la hip\'otesis alternativa }H_{1}\textrm{ es }\\
H_{1}&:& \mu\neq880
\end{eqnarray*}
el estimador puntual para $\mu$ es $\overline{x}$, entonces el estad\'istico de prueba es\medskip
\begin{eqnarray*}
z&=&\frac{\overline{x}-\mu_{0}}{s/\sqrt{n}}\\
&=&\frac{871-880}{21/\sqrt{50}}=-3.03
\end{eqnarray*}
\end{Sol}







\begin{Sol}
Para esta prueba de  dos colas, hay que determinar los dos valores de $z_{\alpha/2}$, es decir,  $z_{\alpha/2}=\pm1.96$, como $z>z_{\alpha/2}$, $z$  cae en la zona de rechazo, por lo tanto  la gerente puede rechazar la hip\'otesis nula y concluir que el promedio efectivamente ha cambiado.\medskip
La probabilidad de rechazar $H_{0}$ cuando esta es verdadera es de $0.05$.


Recordemos que el valor observado del estad\'istico de prueba es $z=-3.03$, la regi\'on de rechazo m\'as peque\~na que puede usarse y todav\'ia seguir rechazando $H_{0}$ es $|z|>3.03$, \\
entonces $p=2(0.012)=0.0024$, que a su vez es menor que el nivel de significancia $\alpha$ asignado inicialmente, y adem\'as los resultados son  \textbf{altamente significativos}.


\end{Sol}






Finalmente determinemos la potencia de la prueba cuando $\mu$ en realidad es igual a $870$ toneladas.

Recordar que la regi\'on de aceptaci\'on est\'a entre $-1.96$ y $1.96$, para $\mu=880$, equivalentemente $$874.18<\overline{x}<885.82$$
$\beta$ es la probabilidad de aceptar $H_{0}$ cuando $\mu=870$, calculemos los valores de $z$ correspondientes a $874.18$ y $885.82$ \medskip
Entonces
\begin{eqnarray*}
z_{1}&=&\frac{\overline{x}-\mu}{s/\sqrt{n}}=\frac{874.18-870}{21/\sqrt{50}}=1.41\\
z_{1}&=&\frac{\overline{x}-\mu}{s/\sqrt{n}}=\frac{885.82-870}{21/\sqrt{50}}=5.33
\end{eqnarray*}


por lo tanto
\begin{eqnarray*}
\beta&=&P\left\{\textrm{aceptar }H_{0}\textrm{ cuando }H_{0}\textrm{ es falsa}\right\}\\
&=&P\left\{874.18<\mu<885.82\textrm{ cuando }\mu=870\right\}\\
&=&P\left\{1.41<z<5.33\right\}=P\left\{1.41<z\right\}\\
&=&1-0.9207=0.0793
\end{eqnarray*}
entonces, la potencia de la prueba es
$$1-\beta=1-0.0793=0.9207$$ que es la probabilidad de rechazar correctamente $H_{0}$ cuando $H_{0}$ es falsa.






Determinar la potencia de la prueba para distintos valores de $H_{1}$ y graficarlos, \textit{curva de potencia}
\begin{center}
\begin{tabular}{c||c}
$H_{1}$ & $\left(1-\beta\right)$ \\\hline 
\hline 
865 &  \\ \hline 
870 &  \\ \hline 
872 &  \\ \hline 
875 &  \\ \hline 
877 &  \\ \hline 
880 &  \\ \hline 
883 &  \\ \hline 
885 &  \\ \hline 
888 &  \\ \hline 
890 &  \\ \hline 
895 &  \\ \hline 
\end{tabular} 

\end{center}






\begin{enumerate}
\item Encontrar las regiones de rechazo para el estad\'istico $z$, para una prueba de
\begin{itemize}
\item[a) ]  dos colas para $\alpha=0.01,0.05,0.1$
\item[b) ]  una cola superior para $\alpha=0.01,0.05,0.1$
\item[c) ] una cola inferior para $\alpha=0.01,0.05,0.1$

\end{itemize}


\item Suponga que el valor del estad\'istico de prueba es 
\begin{itemize}
\item[a) ]$z=-2.41$, sacar las conclusiones correspondientes para los incisos anteriores.
\item[b) ] $z=2.16$, sacar las conclusiones correspondientes para los incisos anteriores.
\item[c) ] $z=1.15$, sacar las conclusiones correspondientes para los incisos anteriores.
\item[d) ] $z=-2.78$, sacar las conclusiones correspondientes para los incisos anteriores.
\item[e) ] $z=-1.81$, sacar las conclusiones correspondientes para los incisos anteriores.

\end{itemize}
\end{enumerate}





\begin{itemize}
\item[3. ] Encuentre el valor de $p$ para las pruebas de hip\'otesis correspondientes a los valores de $z$ del ejercicio anterior.

\item[4. ] Para las pruebas dadas en el ejercicio 2, utilice el valor de $p$, determinado en el ejercicio 3,  para determinar la significancia de los resultados.


\end{itemize}






\begin{itemize}
\item[5. ] Una muestra aleatoria de $n=45$ observaciones de una poblaci\'on con media $\overline{x}=2.4$, y desviaci\'on est\'andar $s=0.29$. Suponga que el objetivo es demostrar que la media poblacional $\mu$ excede $2.3$.
\begin{itemize}
\item[a) ] Defina la hip\'otesis nula y alternativa para la prueba.
\item[b) ] Determine la regi\'on de rechazo para un nivel de significancia de: $\alpha=0.1,0.05,0.01$.
\item[c) ] Determine el error est\'andar de la media muestral.
\item[d) ] Calcule el valor de $p$ para los estad\'isticos de prueba definidos en los incisos anteriores.
\item[e) ] Utilice el valor de $p$ pra sacar una conclusi\'on al nivel de significancia $\alpha$.
\item[f) ] Determine el valor de $\beta$ cuando $\mu=2.5$
\item[g) ] Graficar la curva de potencia para la prueba.

\end{itemize}
\end{itemize}





\subsection{2.2.2 Prueba de hip\'otesis para la diferencia entre dos medias poblacionales}





El estad\'istico que resume la informaci\'on muestral respecto a la diferencia en medias poblacionales $\left(\mu_{1}-\mu_{2}\right)$ es la diferencia de las medias muestrales $\left(\overline{x}_{1}-\overline{x}_{2}\right)$, por tanto al probar la difencia entre las medias muestrales se verifica que la diferencia real entre las medias poblacionales difiere de un valor especificado, $\left(\mu_{1}-\mu_{2}\right)=D_{0}$, se puede usar el error est\'andar de $\left(\overline{x}_{1}-\overline{x}_{2}\right)$, es decir
$$\sqrt{\frac{\sigma^{2}_{1}}{n_{1}}+\frac{\sigma^{2}_{2}}{n_{2}}}$$
cuyo estimador est\'a dado por
$$SE=\sqrt{\frac{s^{2}_{1}}{n_{1}}+\frac{s^{2}_{2}}{n_{2}}}$$

El procedimiento para muestras grandes es:
\begin{itemize}
\item[1) ] \textbf{Hip\'otesis Nula} $H_{0}:\left(\mu_{1}-\mu_{2}\right)=D_{0}$,\medskip

donde $D_{0}$ es el valor, la diferencia, espec\'ifico que se desea probar. En algunos casos se querr\'a demostrar que no hay diferencia alguna, es decir $D_{0}=0$.

\item[2) ] \textbf{Hip\'otesis Alternativa}
\begin{tabular}{cc}\hline
\textbf{Prueba de una Cola} & \textbf{Prueba de dos colas}\\\hline
$H_{1}:\left(\mu_{1}-\mu_{2}\right)>D_{0}$ & $H_{1}:\left(\mu_{1}-\mu_{2}\right)\neq D_{0}$\\ 
$H_{1}:\left(\mu_{1}-\mu_{2}\right)<D_{0}$&\\
\end{tabular}

\end{itemize}








\begin{itemize}
\item[3) ] Estad\'istico de prueba:
$$z=\frac{\left(\overline{x}_{1}-\overline{x}_{2}\right)-D_{0}}{\sqrt{\frac{s^{2}_{1}}{n_{1}}+\frac{s^{2}_{2}}{n_{2}}}}$$
\item[4) ] Regi\'on de rechazo: rechazar $H_{0}$ cuando
\begin{tabular}{cc}\hline
\textbf{Prueba de una Cola} & \textbf{Prueba de dos colas}\\\hline
$z>z_{0}$ & \\
$z<-z_{\alpha}$ cuando $H_{1}:\left(\mu_{1}-\mu_{2}\right)<D_{0}$&$z>z_{\alpha/2}$ o $z<-z_{\alpha/2}$\\
 cuando $p<\alpha$&\\
\end{tabular}


\end{itemize}









\begin{Ejem}
Para determinar si ser propietario de un autom\'ovil afecta el rendimiento acad\'emico de un estudiante, se tomaron dos muestras aleatorias de 100 estudiantes varones. El promedio de calificaciones para los $n_{1}=100$ no propietarios de un auto tuvieron un promedio y varianza de $\overline{x}_{1}=2.7$ y $s_{1}^{2}=0.36$, respectivamente, mientras que para para la segunda muestra con $n_{2}=100$ propietarios de un auto, se tiene $\overline{x}_{2}=2.54$ y $s_{2}^{2}=0.4$. Los datos presentan suficiente evidencia para indicar una diferencia en la media en el rendimiento acad\'emico entre propietarios y no propietarios de un autom\'ovil? Hacer pruebas para $\alpha=0.01,0.05$ y $\alpha=0.1$.
\end{Ejem}







\begin{Sol}
\begin{itemize}
\item Soluci\'on utilizando la t\'ecnica de regiones de rechazo:\medskip
realizando las operaciones
$z=1.84$, determinar si excede los valores de $z_{\alpha/2}$.
\item Soluci\'on utilizando el $p$-value:\medskip
Calcular el valor de $p$, la probabilidad de que $z$ sea mayor que $z=1.84$ o menor que $z=-1.84$, se tiene que $p=0.0658$. Concluir.
\end{itemize}
\end{Sol}







\begin{itemize}
\item Si el intervalo de confianza que se construye contiene el valor del par\'ametro especificado por $H_{0}$, entonces ese valor es uno de los posibles valores del par\'ametro y $H_{0}$ no debe ser rechazada.

\item Si el valor hipot\'etico se encuentra fuera de los l\'imites de confianza, la hip\'otesis nula es rechazada al nivel de significancia $\alpha$.
\end{itemize}

\begin{enumerate}
\item Del libro Mendenhall resolver los ejercicios 9.18, 9.19 y 9.20(\href{https://cu.uacm.edu.mx/nextcloud/index.php/f/202873}{Mendenhall}).

\item Del libro \href{https://cu.uacm.edu.mx/nextcloud/index.php/f/202873}{Mendenhall} resolver los ejercicios: 9.23, 9.26 y 9.28.
\end{enumerate}







\subsection{2.2.3 Prueba de Hip\'otesis para una Proporci\'on Binomial}




Para una muestra aleatoria de $n$ intentos id\'enticos, de una poblaci\'on binomial, la proporci\'on muesrtal $\hat{p}$ tiene una distribuci\'on aproximadamente normal cuando $n$ es grande, con media $p$ y error est\'andar
$$SE=\sqrt{\frac{pq}{n}}.$$
La prueba de hip\'otesis de la forma
\begin{eqnarray*}
H_{0}&:&p=p_{0}\\
H_{1}&:&p>p_{0}\textrm{, o }p<p_{0}\textrm{ o }p\neq p_{0}
\end{eqnarray*}
El estad\'istico de prueba se construye con el mejor estimador de la proporci\'on verdadera, $\hat{p}$, con el estad\'istico de prueba $z$, que se distribuye normal est\'andar.

El procedimiento es
\begin{itemize}
\item[1) ] Hip\'otesis nula: $H_{0}:p=p_{0}$
\item[2) ] Hip\'otesis alternativa
\begin{tabular}{cc}\hline
\textbf{Prueba de una Cola} & \textbf{Prueba de dos colas}\\\hline
$H_{1}:p>p_{0}$ & $p\neq p_{0}$\\
$H_{1}:p<p_{0}$ & \\
\end{tabular}
\item[3) ] Estad\'istico de prueba:
\begin{eqnarray*}
z=\frac{\hat{p}-p_{0}}{\sqrt{\frac{pq}{n}}},\hat{p}=\frac{x}{n}
\end{eqnarray*}
donde $x$ es el n\'umero de \'exitos en $n$ intentos binomiales.

\end{itemize}







\begin{itemize}
\item[4) ] Regi\'on de rechazo: rechazar $H_{0}$ cuando
\begin{tabular}{cc}\hline
\textbf{Prueba de una Cola} & \textbf{Prueba de dos colas}\\\hline
$z>z_{0}$ & \\
$z<-z_{\alpha}$ cuando $H_{1}:p<p_{0}$&$z>z_{\alpha/2}$ o $z<-z_{\alpha/2}$\\
 cuando $p<\alpha$&\\
\end{tabular}
\end{itemize}







\begin{Ejem}
A cualquier edad, alrededor del $20\%$ de los adultos de cierto pa\'is realiza actividades de acondicionamiento f\'isico al menos dos veces por semana. En una encuesta local de $n=100$ adultos de m\'as de $40$ a\ ~nos, un total de 15 personas indicaron que realizaron actividad f\'isica al menos dos veces por semana. Estos datos indican que el porcentaje de participaci\'on para adultos de m\'as de 40 a\ ~nos de edad es  considerablemente menor a la cifra del $20\%$? Calcule el valor de $p$ y \'uselo para sacar las conclusiones apropiadas.
\end{Ejem}

\begin{enumerate}
\item Resolver los ejercicios: 9.30, 9.32, 9.33, 9.35 y 9.39.
\end{enumerate}






\subsection{2.2.4 Prueba de Hip\'otesis diferencia entre dos Proporciones Binomiales}






\begin{Note}
Cuando se tienen dos muestras aleatorias independientes de dos poblaciones binomiales, el objetivo del experimento puede ser la diferencia $\left(p_{1}-p_{2}\right)$ en las proporciones de individuos u objetos que poseen una caracter\'istica especifica en las dos poblaciones. En este caso se pueden utilizar los estimadores de las dos proporciones $\left(\hat{p}_{1}-\hat{p}_{2}\right)$ con error est\'andar dado por
$$SE=\sqrt{\frac{p_{1}q_{1}}{n_{1}}+\frac{p_{2}q_{2}}{n_{2}}}$$
considerando el estad\'istico $z$ con un nivel de significancia $\left(1-\alpha\right)100\%$

\end{Note}


\begin{Note}
La hip\'otesis nula a probarse es de la forma
\begin{itemize}
\item[$H_{0}$: ] $p_{1}=p_{2}$ o equivalentemente $\left(p_{1}-p_{2}\right)=0$, contra una hip\'otesis alternativa $H_{1}$ de una o dos colas.
\end{itemize}
\end{Note}







\begin{Note}
Para estimar el error est\'andar del estad\'istico $z$, se debe de utilizar el hecho de que suponiendo que $H_{0}$ es verdadera, las dos proporciones son iguales a alg\'un valor com\'un, $p$. Para obtener el mejor estimador de $p$ es
$$p=\frac{\textrm{n\'umero total de \'exitos}}{\textrm{N\'umero total de pruebas}}=\frac{x_{1}+x_{2}}{n_{1}+n_{2}}$$
\end{Note}



\begin{itemize}
\item[1) ] \textbf{Hip\'otesis Nula:} $H_{0}:\left(p_{1}-p_{2}\right)=0$
\item[2) ] \textbf{Hip\'otesis Alternativa: } $H_{1}:$
\begin{tabular}{cc}\hline
\textbf{Prueba de una Cola} & \textbf{Prueba de dos colas}\\\hline
$H_{1}:\left(p_{1}-p_{2}\right)>0$ & $H_{1}:\left(p_{1}-p_{2}\right)\neq 0$\\ 
$H_{1}:\left(p_{1}-p_{2}\right)<0$&\\
\end{tabular}
\item[3) ] Estad\'istico de prueba:
\begin{eqnarray*}
z=\frac{\left(\hat{p}_{1}-\hat{p}_{2}\right)}{\sqrt{\frac{p_{1}q_{1}}{n_{1}}+\frac{p_{2}q_{2}}{n_{2}}}}=\frac{\left(\hat{p}_{1}-\hat{p}_{2}\right)}{\sqrt{\frac{pq}{n_{1}}+\frac{pq}{n_{2}}}}
\end{eqnarray*}
donde $\hat{p_{1}}=x_{1}/n_{1}$ y $\hat{p_{2}}=x_{2}/n_{2}$ , dado que el valor com\'un para $p_{1}$ y $p_{2}$ es $p$, entonces $\hat{p}=\frac{x_{1}+x_{2}}{n_{1}+n_{2}}$ y por tanto el estad\'istico de prueba es
\end{itemize}








\begin{eqnarray*}
z=\frac{\hat{p}_{1}-\hat{p}_{2}}{\sqrt{\hat{p}\hat{q}}\left(\frac{1}{n_{1}}+\frac{1}{n_{2}}\right)}
\end{eqnarray*}
\begin{itemize}
\item[4) ] Regi\'on de rechazo: rechazar $H_{0}$ cuando
\begin{tabular}{cc}\hline
\textbf{Prueba de una Cola} & \textbf{Prueba de dos colas}\\\hline
$z>z_{\alpha}$ & \\
$z<-z_{\alpha}$ cuando $H_{1}:p<p_{0}$&$z>z_{\alpha/2}$ o $z<-z_{\alpha/2}$\\
 cuando $p<\alpha$&\\
\end{tabular}

\end{itemize}








\begin{Ejem}
Los registros de un hospital, indican que 52 hombres de una muestra de 1000 contra 23 mujeres de una muestra de 1000 fueron ingresados por enfermedad del coraz\'on. Estos datos presentan suficiente evidencia para indicar un porcentaje m\'as alto de enfermedades del coraz\'on entre hombres ingresados al hospital?, utilizar distintos niveles de confianza de $\alpha$.

\end{Ejem}
\begin{enumerate}
\item Resolver los ejercicios 9.42

\item Resolver los ejercicios: 9.45, 9.48, 9.50
\end{enumerate}







\section{2.3 Muestras Peque\~nas}

\subsection{2.3.1 Una media poblacional}




\begin{itemize}
\item[1) ] \textbf{Hip\'otesis Nula:} $H_{0}:\mu=\mu_{0}$
\item[2) ] \textbf{Hip\'otesis Alternativa: } $H_{1}:$
\begin{tabular}{cc}\hline
\textbf{Prueba de una Cola} & \textbf{Prueba de dos colas}\\\hline
$H_{1}:\mu>\mu_{0}$ & $H_{1}:\mu\neq \mu_{0}$\\ 
$H_{1}:\mu<\mu0$&\\
\end{tabular}
\item[3) ] Estad\'istico de prueba:
\begin{eqnarray*}
t=\frac{\overline{x}-\mu_{0}}{\sqrt{\frac{s^{2}}{n}}}
\end{eqnarray*}
\item[4) ] Regi\'on de rechazo: rechazar $H_{0}$ cuando
\begin{tabular}{cc}\hline
\textbf{Prueba de una Cola} & \textbf{Prueba de dos colas}\\\hline
$t>t_{\alpha}$ & \\
$t<-t_{\alpha}$ cuando $H_{1}:\mu<mu_{0}$&$t>t_{\alpha/2}$ o $t<-t_{\alpha/2}$\\
 cuando $p<\alpha$&\\
\end{tabular}
\end{itemize}






\begin{Ejem}
Las etiquetas en latas de un gal'on de pintura por lo general indican el tiempo de secado y el \'area puede cubrir una capa. Casi todas las marcas de pintura indican que, en una capa, un gal\'on cubrir\'a entre 250 y 500 pies cuadrados, dependiento de la textura de la superficie a pintarse, un fabricante, sin embargo afirma que un gal\'on de su pintura cubrir\'a 400 pies cuadrados de \'area superficial. Para probar su afirmaci\'on, una muestra aleatoria de 10 latas de un gal\'on de pintura blanca se emple\'o para pintar 10 \'areas id\'enticas usando la misma clase de equipo. Las \'areas reales en pies cuadrados cubiertas por estos 10 galones de pintura se dan a continuac\'on:
\begin{center}
\begin{tabular}{|c|c|c|c|c|}
\hline 
310 & 311 & 412 & 368 & 447 \\ 
\hline 
376 & 303 &410 &365 & 350 \\ 
\hline 
\end{tabular} 
\end{center}
\end{Ejem}






\begin{Ejem}
Los datos presentan suficiente evidencia para indicar que el promedio de la cobertura difiere de 400 pies cuadrados? encuentre el valor de $p$ para la prueba y \'uselo para evaluar la significancia de los resultados.
\end{Ejem}
\begin{enumerate}
\item Resolver los ejercicios: 10.2, 10.3,10.5, 10.7, 10.9, 10.13 y 10.16
\end{enumerate}





\subsection{2.3.2 Diferencia entre dos medias poblacionales: M.A.I.}



\begin{Note}
Cuando los tama\ ~nos de muestra son peque\ ~nos, no se puede asegurar que las medias muestrales sean normales, pero si las poblaciones originales son normales, entonces la distribuci\'on muestral de la diferencia de las medias muestales, $\left(\overline{x}_{1}-\overline{x}_{2}\right)$, ser\'a normal con media $\left(\mu_{1}-\mu_{2}\right)$ y error est\'andar $$ES=\sqrt{\frac{\sigma_{1}^{2}}{n_{1}}+\frac{\sigma_{2}^{2}}{n_{2}}}$$

\end{Note}

\begin{itemize}
\item[1) ] \textbf{Hip\'otesis Nula} $H_{0}:\left(\mu_{1}-\mu_{2}\right)=D_{0}$,\medskip

donde $D_{0}$ es el valor, la diferencia, espec\'ifico que se desea probar. En algunos casos se querr\'a demostrar que no hay diferencia alguna, es decir $D_{0}=0$.

\item[2) ] \textbf{Hip\'otesis Alternativa}
\begin{tabular}{cc}\hline
\textbf{Prueba de una Cola} & \textbf{Prueba de dos colas}\\\hline
$H_{1}:\left(\mu_{1}-\mu_{2}\right)>D_{0}$ & $H_{1}:\left(\mu_{1}-\mu_{2}\right)\neq D_{0}$\\ 
$H_{1}:\left(\mu_{1}-\mu_{2}\right)<D_{0}$&\\
\end{tabular}

\item[3) ] Estad\'istico de prueba:
$$t=\frac{\left(\overline{x}_{1}-\overline{x}_{2}\right)-D_{0}}{\sqrt{\frac{s^{2}_{1}}{n_{1}}+\frac{s^{2}_{2}}{n_{2}}}}$$
\end{itemize}







donde $$s^{2}=\frac{\left(n_{1}-1\right)s_{1}^{2}+\left(n_{2}-1\right)s_{2}^{2}}{n_{1}+n_{2}-2}$$
\begin{itemize}

\item[4) ] Regi\'on de rechazo: rechazar $H_{0}$ cuando
\begin{tabular}{cc}\hline
\textbf{Prueba de una Cola} & \textbf{Prueba de dos colas}\\\hline
$z>z_{0}$ & \\
$z<-z_{\alpha}$ cuando $H_{1}:\left(\mu_{1}-\mu_{2}\right)<D_{0}$&$z>z_{\alpha/2}$ o $z<-z_{\alpha/2}$\\
 cuando $p<\alpha$&\\
\end{tabular}
Los valores cr\'iticos de $t$, $t_{-\alpha}$ y $t_{\alpha/2}$ est\'an basados en $\left(n_{1}+n_{2}-2\right)$ grados de libertad.


\end{itemize}






\subsection{2.3.3 Diferencia entre dos medias poblacionales: Diferencias Pareadas}




\begin{itemize}
\item[1) ] \textbf{Hip\'otesis Nula:} $H_{0}:\mu_{d}=0$
\item[2) ] \textbf{Hip\'otesis Alternativa: } $H_{1}:\mu_{d}$
\begin{tabular}{cc}\hline
\textbf{Prueba de una Cola} & \textbf{Prueba de dos colas}\\\hline
$H_{1}:\mu_{d}>0$ & $H_{1}:\mu_{d}\neq 0$\\ 
$H_{1}:\mu_{d}<0$&\\
\end{tabular}
\item[3) ] Estad\'istico de prueba:
\begin{eqnarray*}
t=\frac{\overline{d}}{\sqrt{\frac{s_{d}^{2}}{n}}}
\end{eqnarray*}
donde $n$ es el n\'umero de diferencias pareadas, $\overline{d}$ es la media de las diferencias muestrales, y $s_{d}$ es la desviaci\'on est\'andar de las diferencias muestrales.



\end{itemize}






\begin{itemize}
\item[4) ] Regi\'on de rechazo: rechazar $H_{0}$ cuando
\begin{tabular}{cc}\hline
\textbf{Prueba de una Cola} & \textbf{Prueba de dos colas}\\\hline
$t>t_{\alpha}$ & \\
$t<-t_{\alpha}$ cuando $H_{1}:\mu<mu_{0}$&$t>t_{\alpha/2}$ o $t<-t_{\alpha/2}$\\
 cuando $p<\alpha$&\\
\end{tabular}

Los valores cr\'iticos de $t$, $t_{-\alpha}$ y $t_{\alpha/2}$ est\'an basados en $\left(n_{1}+n_{2}-2\right)$ grados de libertad.

\end{itemize}





\subsection{2.3.4 Inferencias con respecto a la Varianza Poblacional}




\begin{itemize}
\item[1) ] \textbf{Hip\'otesis Nula:} $H_{0}:\sigma^{2}=\sigma^{2}_{0}$
\item[2) ] \textbf{Hip\'otesis Alternativa: } $H_{1}$
\begin{tabular}{cc}\hline
\textbf{Prueba de una Cola} & \textbf{Prueba de dos colas}\\\hline
$H_{1}:\sigma^{2}>\sigma^{2}_{0}$ & $H_{1}:\sigma^{2}\neq \sigma^{2}_{0}$\\ 
$H_{1}:\sigma^{2}<\sigma^{2}_{0}$&\\
\end{tabular}
\item[3) ] Estad\'istico de prueba:
\begin{eqnarray*}
\chi^{2}=\frac{\left(n-1\right)s^{2}}{\sigma^{2}_{0}}
\end{eqnarray*}

\end{itemize}







\begin{itemize}
\item[4) ] Regi\'on de rechazo: rechazar $H_{0}$ cuando
\begin{tabular}{cc}\hline
\textbf{Prueba de una Cola} & \textbf{Prueba de dos colas}\\\hline
$\chi^{2}>\chi^{2}_{\alpha}$ & \\
$\chi^{2}<\chi^{2}_{\left(1-\alpha\right)}$ cuando $H_{1}:\chi^{2}<\chi^{2}_{0}$&$\chi^{2}>\chi^{2}_{\alpha/2}$ o $\chi^{2}<\chi^{2}_{\left(1-\alpha/2\right)}$\\
 cuando $p<\alpha$&\\
\end{tabular}

Los valores cr\'iticos de $\chi^{2}$,est\'an basados en $\left(n_{1}+\right)$ grados de libertad.

\end{itemize}




\subsection{2.3.5 Comparaci\'on de dos varianzas poblacionales}




\begin{itemize}
\item[1) ] \textbf{Hip\'otesis Nula} $H_{0}:\left(\sigma^{2}_{1}-\sigma^{2}_{2}\right)=D_{0}$,\medskip

donde $D_{0}$ es el valor, la diferencia, espec\'ifico que se desea probar. En algunos casos se querr\'a demostrar que no hay diferencia alguna, es decir $D_{0}=0$.

\item[2) ] \textbf{Hip\'otesis Alternativa}
\begin{tabular}{cc}\hline
\textbf{Prueba de una Cola} & \textbf{Prueba de dos colas}\\\hline
$H_{1}:\left(\sigma^{2}_{1}-\sigma^{2}_{2}\right)>D_{0}$ & $H_{1}:\left(\sigma^{2}_{1}-\sigma^{2}_{2}\right)\neq D_{0}$\\ 
$H_{1}:\left(\sigma^{2}_{1}-\sigma^{2}_{2}\right)<D_{0}$&\\
\end{tabular}

\end{itemize}


\begin{itemize}
\item[3) ] Estad\'istico de prueba:
$$F=\frac{s_{1}^{2}}{s_{2}^{2}}$$
donde $s_{1}^{2}$ es la varianza muestral m\'as grande.
\item[4) ] Regi\'on de rechazo: rechazar $H_{0}$ cuando
\begin{tabular}{cc}\hline
\textbf{Prueba de una Cola} & \textbf{Prueba de dos colas}\\\hline
$F>F_{\alpha}$ & $F>F_{\alpha/2}$\\
 cuando $p<\alpha$&\\
\end{tabular}


\end{itemize}




\section{Ejercicios}


\begin{itemize}
\item[1) ] Del libro Probabililidad y Estad\'sitica para Ingenier\'ia de Hines, Montgomery, Goldsman y Borror resolver los siguientes ejercicios: 10-9, 10-10,10-13,10-16 y 10-20.

\item[2) ] Realizar un programa en R para cada una de las secciones y subsecciones revisadas en clase, para determinar intervalos de confianza.

\item[3) ] Aplicar los programas elaborados en el ejercicio anterior a la siguiente lista:  10-39, 10-41, 10-45, 10-47, 10-48, 10-50, 10-52, 10-54, 10-56,10-57, 10-58, 10-65, 10-68, 10-72 y 10-73.

\item[4) ]  Elaborar una rutina en R que grafique las siguientes distribuciones, permitiendo variar los par\'ametros de las distribuciones: Binomial, Uniforme continua, Gamma, Beta, Exponencial, Normal y $t$-Student.

\item[5)] Presentar el primer cap\'itulo del libro del curso en formato \textit{Rnw} con su respectivo archivo \textit{pdf} generado
\end{itemize}






%---------------------------------------------------------
\section{Pruebas de Hip\'otesis}
%---------------------------------------------------------
\subsection{Tipos de errores}





%\frametitle{Prueba de Hip\'otesis}

\begin{itemize}
\item Una hip\'otesis estad\'istica es una afirmaci\'on  acerca de la distribuci\'on de probabilidad de una variable aleatoria, a menudo involucran uno o m\'as par\'ametros de la distribuci\'on.

\item Las hip\'otesis son afirmaciones respecto a la poblaci\'on o distribuci\'on bajo estudio, no en torno a la muestra.

\item La mayor\'ia de las veces, la prueba de hip\'otesis consiste en determinar si la situaci \'on experimental ha cambiado

\item el inter\'es principal es decidir sobre la veracidad o falsedad de una hip\'otesis, a este procedimiento se le llama \textit{prueba de hip\'otesis}.

\item Si la informaci\'on es consistente con la hip\'otesis, se concluye que esta es verdadera, de lo contrario que con base en la informaci\'on, es falsa.

\end{itemize}







Una prueba de hip\'otesis est\'a formada por cinco partes
\begin{itemize}
\item La hip\'otesis nula, denotada por $H_{0}$.
\item La hip\'otesis alterativa, denorada por $H_{1}$.
\item El estad\'sitico de prueba y su valor $p$.
\item La regi\'on de rechazo.
\item La conclusi\'on.

\end{itemize}

\begin{Def}
Las dos hip\'otesis en competencias son la \textbf{hip\'otesis alternativa $H_{1}$}, usualmente la que se desea apoyar, y la \textbf{hip\'otesis nula $H_{0}$}, opuesta a $H_{1}$.
\end{Def}









En general, es m\'as f\'acil presentar evidencia de que $H_{1}$ es cierta, que demostrar 	que $H_{0}$ es falsa, es por eso que por lo regular se comienza suponiendo que $H_{0}$ es cierta, luego se utilizan los datos de la muestra para decidir si existe evidencia a favor de $H_{1}$, m\'as que a favor de $H_{0}$, as\'i se tienen dos conclusiones
\begin{itemize}
\item Rechazar $H_{0}$ y concluir que $H_{1}$ es verdadera.
\item Aceptar, no rechazar, $H_{0}$ como verdadera.

\end{itemize}

\begin{Ejem}
Se desea demostrar que el salario promedio  por hora en cierto lugar es distinto de $19$usd, que es el promedio nacional. Entonces $H_{1}:\mu\neq19$, y $H_{0}:\mu=19$.
\end{Ejem}
A esta se le denomina \textbf{Prueba de hip\'otesis de dos colas}.










\begin{Ejem}
Un determinado proceso produce un promedio de $5\%$ de piezas defectuosas. Se est\'a interesado en demostrar que un simple ajuste en una m\'aquina reducir\'a $p$, la proporci\'on de piezas defectuosas producidas en este proceso. Entonces se tiene $H_{0}:p<0.3$ y $H_{1}:p=0.03$. Si se puede rechazar $H_{0}$, se concluye que el proceso ajustado produce menos del $5\%$ de piezas defectuosas.
\end{Ejem}
A esta se le denomina \textbf{Prueba de hip\'otesis de una cola}.

La decisi\'on de rechazar o aceptar la hip\'otesis nula est\'a basada en la informaci\'on contenida en una muestra proveniente de la poblaci\'on de inter\'es. Esta informaci\'on tiene estas formas




\begin{itemize}
\item \textbf{Estad\'sitico de prueba:} un s\'olo n\'umero calculado a partir de la muestra.

\item \textbf{$p$-value:} probabilidad calculada a partir del estad\'stico de prueba.
\begin{Def}
El $p$-value es la probabilidad de observar un estad\'istico de prueba tanto o m\'as alejado del valor obervado, si en realidad $H_{0}$ es verdadera.\medskip
Valores grandes del estad\'stica de prueba  y valores peque\~nos de $p$ significan que se ha observado un evento muy poco probable, si $H_{0}$ en realidad es verdadera.
\end{Def}

\end{itemize}






Todo el conjunto de valores que puede tomar el estad\'istico de prueba se divide en dos regiones. Un conjunto, formado de valores que apoyan la hip\'otesis alternativa y llevan a rechazar $H_{0}$, se denomina \textbf{regi\'on de rechazo}. El otro, conformado por los valores que sustentatn la hip\'otesis nula, se le denomina \textbf{regi\'on de aceptaci\'on}.\medskip

Cuando la regi\'on de rechazo est\'a en la cola izquierda de la distribuci\'on, la  prueba se denomina \textbf{prueba lateral izquierda}. Una prueba con regi\'on de rechazo en la cola derecha se le llama \textbf{prueba lateral derecha}.


Si el estad\'stico de prueba cae en la regi\'on de rechazo, entonces se rechaza $H_{0}$. Si el estad\'stico de prueba cae en la regi\'on de aceptaci\'on, entonces la hip\'otesis nula se acepta o la prueba se juzga como no concluyente.\medskip






Dependiendo del nivel de confianza que se desea agregar a las conclusiones de la prueba, y el \textbf{nivel de significancia $\alpha$}, el riesgo que est\'a dispuesto a correr si se toma una decisi\'on incorrecta.






\begin{Def}
Un \textbf{error de tipo I} para una prueba estad\'istica es el error que se tiene al rechazar la hip\'otesis nula cuando es verdadera. El \textbf{nivel de significancia} para una prueba estad\'istica de hip\'otesis es
\begin{eqnarray*}
\alpha&=&P\left\{\textrm{error tipo I}\right\}=P\left\{\textrm{rechazar equivocadamente }H_{0}\right\}\\
&=&P\left\{\textrm{rechazar }H_{0}\textrm{ cuando }H_{0}\textrm{ es verdadera}\right\}
\end{eqnarray*}

\end{Def}
Este valor $\alpha$ representa el valor m\'aximo de riesgo tolerable de rechazar incorrectamente $H_{0}$. Una vez establecido el nivel de significancia, la regi\'on de rechazo se define para poder determinar si se rechaza $H_{0}$ con un cierto nivel de confianza.



\section{Muestras grandes: una media poblacional}
\subsection{C\'alculo de valor $p$}





\begin{Def}
El \textbf{valor de $p$} (\textbf{$p$-value}) o nivel de significancia observado de un estad\'istico de prueba es el valor m\'as peque\~ no de $\alpha$ para el cual $H_{0}$ se puede rechazar. El riesgo de cometer un error tipo $I$, si $H_{0}$ es rechazada con base en la informaci\'on que proporciona la muestra.
\end{Def}

\begin{Note}
Valores peque\~ nos de $p$ indican 	que el valor observado del estad\'stico de prueba se encuentra alejado del valor hipot\'etico de $\mu$, es decir se tiene evidencia de que $H_{0}$ es falsa y por tanto debe de rechazarse.
\end{Note}

\begin{Note}
Valores grandes de $p$ indican que el estad\'istico de prueba observado no est\'a alejado de la medi hipot\'etica y no apoya el rechazo de $H_{0}$.
\end{Note}







\begin{Def}
Si el valor de $p$ es menor o igual que el nivel de significancia $\alpha$, determinado previamente, entonces $H_{0}$ es rechazada y se puede concluir que los resultados son estad\'isticamente significativos con un nivel de confianza del $100\left(1-\alpha\right)\%$.
\end{Def}
Es usual utilizar la siguiente clasificaci\'on de resultados


\begin{tabular}{|c||c|l|}\hline
$p$& $H_{0}$&Significativa\\\hline\hline
$p<0.01$&Rechazar &Altamente\\\hline
$0.01\leq p<0.05$ & Rechazar&Estad\'isticamente\\\hline
$0.05\leq p <0.1$ & No rechazar & Tendencia estad\'istica\\\hline
$0.01\leq p$ & No rechazar & No son estad\'isticamente\\\hline
\end{tabular}





\begin{Note}
Para determinar el valor de $p$, encontrar el \'area en la cola despu\'es del estad\'istico de prueba. Si la prueba es de una cola, este es el valor de $p$. Si es de dos colas, \'este valor encontrado es la mitad del valor de $p$. Rechazar $H_{0}$ cuando el valor de $p<\alpha$.
\end{Note}

Hay dos tipos de errores al realizar una prueba de hip\'otesis
\begin{center}
\begin{tabular}{c|cc}
& $H_{0}$ es Verdadera & $H_{0}$ es Falsa\\\hline\hline
Rechazar $H_{0}$ & Error tipo I & $\surd$\\
Aceptar $H_{0}$ & $\surd$ & Error tipo II
\end{tabular}
\end{center}






\begin{Def}
La probabilidad de cometer el error tipo II se define por $\beta$ donde
\begin{eqnarray*}
\beta&=&P\left\{\textrm{error tipo II}\right\}=P\left\{\textrm{Aceptar equivocadamente }H_{0}\right\}\\
&=&P\left\{\textrm{Aceptar }H_{0}\textrm{ cuando }H_{0}\textrm{ es falsa}\right\}
\end{eqnarray*}
\end{Def}

\begin{Note}
Cuando $H_{0}$ es falsa y $H_{1}$ es verdadera, no siempre es posible especificar un valor exacto de $\mu$, sino m\'as bien un rango de posibles valores.\medskip
En lugar de arriesgarse a tomar una decisi\'on incorrecta, es mejor conlcuir que \textit{no hay evidencia suficiente para rechazar $H_{0}$}, es decir en lugar de aceptar $H_{0}$, \textit{no rechazar $H_{0}$}.

\end{Note}






La bondad de una prueba estad\'istica se mide por el tama\~ no de $\alpha$ y $\beta$, ambas deben de ser peque\~ nas. Una manera muy efectiva de medir la potencia de la prueba es calculando el complemento del error tipo $II$:
\begin{eqnarray*}
1-\beta&= &P\left\{\textrm{Rechazar }H_{0}\textrm{ cuando }H_{0}\textrm{ es falsa}\right\}\\
&=&P\left\{\textrm{Rechazar }H_{0}\textrm{ cuando }H_{1}\textrm{ es verdadera}\right\}
\end{eqnarray*}
\begin{Def}
La \textbf{potencia de la prueba}, $1-\beta$, mide la capacidad de que la prueba funciona como se necesita.
\end{Def}







\begin{Ejem}
La producci\'on diariade una planta qu\'imica local ha promediado 880 toneladas en los \'ultimos a\~nos. A la gerente de control de calidad le gustar\'ia saber si este promedio ha cambiado en meses recientes. Ella selecciona al azar 50 d\'ias de la base de datos computarizada y calcula el promedio y la desviaci\'on est\'andar de las $n=50$  producciones como $\overline{x}=871$ toneladas y $s=21$ toneladas, respectivamente. Pruebe la hip\'otesis  apropiada usando $\alpha=0.05$.

\end{Ejem}






\begin{Sol}
La hip\'otesis nula apropiada es:

\begin{eqnarray*}
H_{0}&:& \mu=880\\
&&\textrm{ y la hip\'otesis alternativa }H_{1}\textrm{ es }\\
H_{1}&:& \mu\neq880
\end{eqnarray*}
el estimador puntual para $\mu$ es $\overline{x}$, entonces el estad\'istico de prueba es\medskip
\begin{eqnarray*}
z&=&\frac{\overline{x}-\mu_{0}}{s/\sqrt{n}}\\
&=&\frac{871-880}{21/\sqrt{50}}=-3.03
\end{eqnarray*}
\end{Sol}



\begin{Sol}
Para esta prueba de  dos colas, hay que determinar los dos valores de $z_{\alpha/2}$, es decir, $z_{\alpha/2}=\pm1.96$

\end{Sol}





%---------------------------------------------------------
\section{Pruebas de Hip\'otesis}
%---------------------------------------------------------
\subsection{Tipos de errores}





\begin{itemize}
\item Una hip\'otesis estad\'istica es una afirmaci\'on  acerca de la distribuci\'on de probabilidad de una variable aleatoria, a menudo involucran uno o m\'as par\'ametros de la distribuci\'on.

\item Las hip\'otesis son afirmaciones respecto a la poblaci\'on o distribuci\'on bajo estudio, no en torno a la muestra.

\item La mayor\'ia de las veces, la prueba de hip\'otesis consiste en determinar si la situaci \'on experimental ha cambiado

\item el inter\'es principal es decidir sobre la veracidad o falsedad de una hip\'otesis, a este procedimiento se le llama \textit{prueba de hip\'otesis}.

\item Si la informaci\'on es consistente con la hip\'otesis, se concluye que esta es verdadera, de lo contrario que con base en la informaci\'on, es falsa.

\end{itemize}










\begin{itemize}
\item La descisi\'on de aceptar o rechazar la hip\'otesis nula se basa en un estad\'istico calculado a partir de la muestra. Esto necesariamente implica la existencia de un error.


\end{itemize}



%---------------------------------------------------------
\section{Pruebas de Hip\'otesis}
%---------------------------------------------------------
\subsection{Tipos de errores}




\begin{itemize}
\item Una hip\'otesis estad\'istica es una afirmaci\'on  acerca de la distribuci\'on de probabilidad de una variable aleatoria, a menudo involucran uno o m\'as par\'ametros de la distribuci\'on.

\item Las hip\'otesis son afirmaciones respecto a la poblaci\'on o distribuci\'on bajo estudio, no en torno a la muestra.

\item La mayor\'ia de las veces, la prueba de hip\'otesis consiste en determinar si la situaci \'on experimental ha cambiado

\item el inter\'es principal es decidir sobre la veracidad o falsedad de una hip\'otesis, a este procedimiento se le llama \textit{prueba de hip\'otesis}.

\item Si la informaci\'on es consistente con la hip\'otesis, se concluye que esta es verdadera, de lo contrario que con base en la informaci\'on, es falsa.

\end{itemize}







Una prueba de hip\'otesis est\'a formada por cinco partes
\begin{itemize}
\item La hip\'otesis nula, denotada por $H_{0}$.
\item La hip\'otesis alterativa, denorada por $H_{1}$.
\item El estad\'sitico de prueba y su valor $p$.
\item La regi\'on de rechazo.
\item La conclusi\'on.

\end{itemize}

\begin{Def}
Las dos hip\'otesis en competencias son la \textbf{hip\'otesis alternativa $H_{1}$}, usualmente la que se desea apoyar, y la \textbf{hip\'otesis nula $H_{0}$}, opuesta a $H_{1}$.
\end{Def}








En general, es m\'as f\'acil presentar evidencia de que $H_{1}$ es cierta, que demostrar 	que $H_{0}$ es falsa, es por eso que por lo regular se comienza suponiendo que $H_{0}$ es cierta, luego se utilizan los datos de la muestra para decidir si existe evidencia a favor de $H_{1}$, m\'as que a favor de $H_{0}$, as\'i se tienen dos conclusiones
\begin{itemize}
\item Rechazar $H_{0}$ y concluir que $H_{1}$ es verdadera.
\item Aceptar, no rechazar, $H_{0}$ como verdadera.

\end{itemize}

\begin{Ejem}
Se desea demostrar que el salario promedio  por hora en cierto lugar es distinto de $19$usd, que es el promedio nacional. Entonces $H_{1}:\mu\neq19$, y $H_{0}:\mu=19$.
\end{Ejem}
A esta se le denomina \textbf{Prueba de hip\'otesis de dos colas}.






\begin{Ejem}
Un determinado proceso produce un promedio de $5\%$ de piezas defectuosas. Se est\'a interesado en demostrar que un simple ajuste en una m\'aquina reducir\'a $p$, la proporci\'on de piezas defectuosas producidas en este proceso. Entonces se tiene $H_{0}:p<0.3$ y $H_{1}:p=0.03$. Si se puede rechazar $H_{0}$, se concluye que el proceso ajustado produce menos del $5\%$ de piezas defectuosas.
\end{Ejem}
A esta se le denomina \textbf{Prueba de hip\'otesis de una cola}.


La decisi\'on de rechazar o aceptar la hip\'otesis nula est\'a basada en la informaci\'on contenida en una muestra proveniente de la poblaci\'on de inter\'es. Esta informaci\'on tiene estas formas

\begin{itemize}
\item \textbf{Estad\'sitico de prueba:} un s\'olo n\'umero calculado a partir de la muestra.

\item \textbf{$p$-value:} probabilidad calculada a partir del estad\'stico de prueba.
\begin{Def}
El $p$-value es la probabilidad de observar un estad\'istico de prueba tanto o m\'as alejado del valor obervado, si en realidad $H_{0}$ es verdadera.\medskip
Valores grandes del estad\'stica de prueba  y valores peque\~nos de $p$ significan que se ha observado un evento muy poco probable, si $H_{0}$ en realidad es verdadera.
\end{Def}

\end{itemize}








Todo el conjunto de valores que puede tomar el estad\'istico de prueba se divide en dos regiones. Un conjunto, formado de valores que apoyan la hip\'otesis alternativa y llevan a rechazar $H_{0}$, se denomina \textbf{regi\'on de rechazo}. El otro, conformado por los valores que sustentatn la hip\'otesis nula, se le denomina \textbf{regi\'on de aceptaci\'on}.\medskip

Cuando la regi\'on de rechazo est\'a en la cola izquierda de la distribuci\'on, la  prueba se denomina \textbf{prueba lateral izquierda}. Una prueba con regi\'on de rechazo en la cola derecha se le llama \textbf{prueba lateral derecha}.



Si el estad\'stico de prueba cae en la regi\'on de rechazo, entonces se rechaza $H_{0}$. Si el estad\'stico de prueba cae en la regi\'on de aceptaci\'on, entonces la hip\'otesis nula se acepta o la prueba se juzga como no concluyente.\medskip








Dependiendo del nivel de confianza que se desea agregar a las conclusiones de la prueba, y el \textbf{nivel de significancia $\alpha$}, el riesgo que est\'a dispuesto a correr si se toma una decisi\'on incorrecta.








\begin{Def}
Un \textbf{error de tipo I} para una prueba estad\'istica es el error que se tiene al rechazar la hip\'otesis nula cuando es verdadera. El \textbf{nivel de significancia} para una prueba estad\'istica de hip\'otesis es
\begin{eqnarray*}
\alpha&=&P\left\{\textrm{error tipo I}\right\}=P\left\{\textrm{rechazar equivocadamente }H_{0}\right\}\\
&=&P\left\{\textrm{rechazar }H_{0}\textrm{ cuando }H_{0}\textrm{ es verdadera}\right\}
\end{eqnarray*}

\end{Def}
Este valor $\alpha$ representa el valor m\'aximo de riesgo tolerable de rechazar incorrectamente $H_{0}$. Una vez establecido el nivel de significancia, la regi\'on de rechazo se define para poder determinar si se rechaza $H_{0}$ con un cierto nivel de confianza.






\section{Muestras grandes: una media poblacional}
\subsection{C\'alculo de valor $p$}






\begin{Def}
El \textbf{valor de $p$} (\textbf{$p$-value}) o nivel de significancia observado de un estad\'istico de prueba es el valor m\'as peque\~ no de $\alpha$ para el cual $H_{0}$ se puede rechazar. El riesgo de cometer un error tipo $I$, si $H_{0}$ es rechazada con base en la informaci\'on que proporciona la muestra.
\end{Def}

\begin{Note}
Valores peque\~ nos de $p$ indican 	que el valor observado del estad\'stico de prueba se encuentra alejado del valor hipot\'etico de $\mu$, es decir se tiene evidencia de que $H_{0}$ es falsa y por tanto debe de rechazarse.
\end{Note}

\begin{Note}
Valores grandes de $p$ indican que el estad\'istico de prueba observado no est\'a alejado de la medi hipot\'etica y no apoya el rechazo de $H_{0}$.
\end{Note}








\begin{Def}
Si el valor de $p$ es menor o igual que el nivel de significancia $\alpha$, determinado previamente, entonces $H_{0}$ es rechazada y se puede concluir que los resultados son estad\'isticamente significativos con un nivel de confianza del $100\left(1-\alpha\right)\%$.
\end{Def}
Es usual utilizar la siguiente clasificaci\'on de resultados


\begin{tabular}{|c||c|l|}\hline
$p$& $H_{0}$&Significativa\\\hline\hline
$p<0.01$&Rechazar &Altamente\\\hline
$0.01\leq p<0.05$ & Rechazar&Estad\'isticamente\\\hline
$0.05\leq p <0.1$ & No rechazar & Tendencia estad\'istica\\\hline
$0.01\leq p$ & No rechazar & No son estad\'isticamente\\\hline
\end{tabular}






\begin{Note}
Para determinar el valor de $p$, encontrar el \'area en la cola despu\'es del estad\'istico de prueba. Si la prueba es de una cola, este es el valor de $p$. Si es de dos colas, \'este valor encontrado es la mitad del valor de $p$. Rechazar $H_{0}$ cuando el valor de $p<\alpha$.
\end{Note}

Hay dos tipos de errores al realizar una prueba de hip\'otesis
\begin{center}
\begin{tabular}{c|cc}
& $H_{0}$ es Verdadera & $H_{0}$ es Falsa\\\hline\hline
Rechazar $H_{0}$ & Error tipo I & $\surd$\\
Aceptar $H_{0}$ & $\surd$ & Error tipo II
\end{tabular}
\end{center}






\begin{Def}
La probabilidad de cometer el error tipo II se define por $\beta$ donde
\begin{eqnarray*}
\beta&=&P\left\{\textrm{error tipo II}\right\}=P\left\{\textrm{Aceptar equivocadamente }H_{0}\right\}\\
&=&P\left\{\textrm{Aceptar }H_{0}\textrm{ cuando }H_{0}\textrm{ es falsa}\right\}
\end{eqnarray*}
\end{Def}

\begin{Note}
Cuando $H_{0}$ es falsa y $H_{1}$ es verdadera, no siempre es posible especificar un valor exacto de $\mu$, sino m\'as bien un rango de posibles valores.\medskip
En lugar de arriesgarse a tomar una decisi\'on incorrecta, es mejor conlcuir que \textit{no hay evidencia suficiente para rechazar $H_{0}$}, es decir en lugar de aceptar $H_{0}$, \textit{no rechazar $H_{0}$}.

\end{Note}






La bondad de una prueba estad\'istica se mide por el tama\~ no de $\alpha$ y $\beta$, ambas deben de ser peque\~ nas. Una manera muy efectiva de medir la potencia de la prueba es calculando el complemento del error tipo $II$:
\begin{eqnarray*}
1-\beta&= &P\left\{\textrm{Rechazar }H_{0}\textrm{ cuando }H_{0}\textrm{ es falsa}\right\}\\
&=&P\left\{\textrm{Rechazar }H_{0}\textrm{ cuando }H_{1}\textrm{ es verdadera}\right\}
\end{eqnarray*}
\begin{Def}
La \textbf{potencia de la prueba}, $1-\beta$, mide la capacidad de que la prueba funciones como se necesita.
\end{Def}




%---------------------------------------------------------
\section{Estimaci\'on por intervalos}
%---------------------------------------------------------
\subsection*{Para la media}





Recordemos que $S^{2}$ es un estimador insesgado de $\sigma^{2}$
\begin{Def}
Sean $\hat{\theta}_{1}$ y $\hat{\theta}_{2}$ dos estimadores insesgados de $\theta$, par\'ametro poblacional. Si $\sigma_{\hat{\theta}_{1}}^{2}<\sigma_{\hat{\theta}_{2}}^{2}$, decimos que $\hat{\theta}_{1}$ un estimador m\'as eficaz de $\theta$ que $\hat{\theta}_{2}$.
\end{Def}

Algunas observaciones que es preciso realizar

\begin{enumerate}
\item[a) ]Para poblaciones normales, $\overline{X}$ y $\tilde{X}$ son estimadores insesgados de $\mu$, pero con $\sigma_{\overline{X}}^{2}<\sigma_{\tilde{X}_{2}}^{2}$.
%\end{Note}

%\begin{Note}
\item[b) ]Para las estimaciones por intervalos de $\theta$, un intervalo de la forma $\hat{\theta}_{L}<\theta<\hat{\theta}_{U}$,  $\hat{\theta}_{L}$ y $\hat{\theta}_{U}$ dependen del valor de $\hat{\theta}$.
\item[c) ]Para $\sigma_{\overline{X}}^{2}=\frac{\sigma^{2}}{n}$, si $n\rightarrow\infty$, entonces $\hat{\theta}\rightarrow\mu$.
%\end{Note}
\end{enumerate}









%\begin{Note}
%Para $\sigma_{\overline{X}}^{2}=\frac{\sigma^{2}}{n}$, si $n\rightarrow\infty$, %entonces $\hat{\theta}\rightarrow\mu$.
%\end{Note}

%\begin{Note}
\begin{enumerate}
\item[d) ]Para $\hat{\theta}$ se determinan $\hat{\theta}_{L}$ y $\hat{\theta}_{U}$ de modo tal que 
\begin{eqnarray}
P\left\{\hat{\theta}_{L}<\hat{\theta}<\hat{\theta}_{U}\right\}=1-\alpha,
\end{eqnarray}
con $\alpha\in\left(0,1\right)$. Es decir, $\theta\in\left(\hat{\theta}_{L},\hat{\theta}_{U}\right)$ es un intervalo de confianza del $100\left(1-\alpha\right)\%$.

\item[e) ] De acuerdo con el TLC se espera que la distribuci\'on muestral de $\overline{X}$ se distribuye aproximadamente normal con media $\mu_{X}=\mu$ y desviaci\'on est\'andar $\sigma_{\overline{X}}=\frac{\sigma}{\sqrt{n}}$.

\end{enumerate}








Para $Z_{\alpha/2}$ se tiene $P\left\{-Z_{\alpha/2}<Z<Z_{\alpha/2}\right\}=1-\alpha$, donde $Z=\frac{\overline{X}-\mu}{\sigma/\sqrt{n}}$. Entonces
$P\left\{-Z_{\alpha/2}<\frac{\overline{X}-\mu}{\sigma/\sqrt{n}}<Z_{\alpha/2}\right\}=1-\alpha$ es equivalente a 
$P\left\{\overline{X}-Z_{\alpha/2}\frac{\sigma}{\sqrt{n}}<\mu<\overline{X}+Z_{\alpha/2}\frac{\sigma}{\sqrt{n}}\right\}=1-\alpha$ 

\begin{enumerate}
\item[f) ]Si $\overline{X}$ es la media muestral de una muestra de tama\~no $n$ de una poblaci\'on con varianza conocida $\sigma^{2}$, el intervalo de confianza de $100\left(1-\alpha\right)\%$ para $\mu$ es $\mu\in\left(\overline{x}-z_{\alpha/2}\frac{\sigma}{\sqrt{n}},\overline{x}+z_{\alpha/2}\frac{\sigma}{\sqrt{n}}\right)$.

\item[g) ] Para muestras peque\~nas de poblaciones no normales, no se puede esperar que el grado de confianza sea preciso.
\item[h) ] Para $n\geq30$, con distribuci\'on de forma no muy sesgada, se pueden tener buenos resultados.
\end{enumerate}








\begin{Teo}
Si $\overline{X}$ es un estimador de $\mu$, podemos tener $100\left(1-\alpha\right)\%$  de confianza en que el error no exceder\'a a $z_{\alpha/2}\frac{\sigma}{\sqrt{n}}$, error entre $\overline{X}$ y $\mu$.
\end{Teo}

\begin{Teo}
Si $\overline{X}$ es un estimador de $\mu$, podemos tener $100\left(1-\alpha\right)\%$  de confianza en que el error no exceder\'a una cantidad $e$ cuando el tama\~no de la muestra es $$n=\left(\frac{z_{\alpha/2}\sigma}{e}\right)^{2}.$$
\end{Teo}
\begin{Note}
Para intervalos unilaterales
$$P\left\{\frac{\overline{X}-\mu}{\sigma/\sqrt{n}}<Z_{\alpha}\right\}=1-\alpha$$
\end{Note}








equivalentemente
$$P\left\{\mu<\overline{X}+Z_{\alpha}\frac{\sigma}{\sqrt{n}}\right\}=1-\alpha.$$
Si $\overline{X}$ es la media de una muestra aleatoria de tama\~no $n$  a partir de una poblaci\'on con varianza $\sigma^{2}$, los l\'imites de confianza unilaterales del   $100\left(1-\alpha\right)\%$  de confianza para $\mu$ est\'an dados por
\begin{itemize}
\item L\'imite unilateral superior: $\overline{x}+z_{\alpha}\frac{\sigma}{\sqrt{n}}$
\item L\'imite unilateral inferior: $\overline{x}-z_{\alpha}\frac{\sigma}{\sqrt{n}}$
\end{itemize}


\begin{itemize}
\item Para $\sigma$ desconocida recordar que $T=\frac{\overline{x}-\mu}{s/\sqrt{n}}\sim t_{n-1}$, donde $s$ es la desviaci\'on est\'andar de la muestra. Entonces
\begin{eqnarray*}
P\left\{-t_{\alpha/2}<T<t_{\alpha/2}\right\}=1-\alpha,\textrm{equivalentemente}\\
P\left\{\overline{X}-t_{\alpha/2}\frac{s}{\sqrt{n}}<\mu<\overline{X}+t_{\alpha/2}\frac{s}{\sqrt{n}}\right\}=1-\alpha.
\end{eqnarray*}

\item Un intervalo de confianza del $100\left(1-\alpha\right)\%$  de confianza para $\mu$, $\sigma^{2}$ desconocida y poblaci\'on normal es $\mu\in\left(\overline{x}-t_{\alpha/2}\frac{s}{\sqrt{n}},\overline{x}+t_{\alpha/2}\frac{s}{\sqrt{n}}\right)$, donde $t_{\alpha/2}$ es una $t$-student con $\nu=n-1$ grados de libertad.
\item Los l\'imites unilaterales para $\mu$ con $\sigma$ desconocida son $\overline{X}-t_{\alpha/2}\frac{s}{\sqrt{n}}$ y $\overline{X}+t_{\alpha/2}\frac{s}{\sqrt{n}}$.
\end{itemize}










\begin{itemize}
\item Cuando la poblaci\'on no es normal, $\sigma$ desconocida y $n\geq30$, $\sigma$ se puede reemplazar por $s$ para obtener el intervalo de confianza para muestras grandes:
$$\overline{X}\pm t_{\alpha/2}\frac{s}{\sqrt{n}}.$$

\item El estimador de $\overline{X}$ de $\mu$,  $\sigma$ desconocida, la varianza de $\sigma_{\overline{X}}^{2}=\frac{\sigma^{2}}{n}$, el error est\'andar de $\overline{X}$ es $\sigma/\sqrt{n}$.

\item Si $\sigma$ es desconocida y la poblaci\'on es normal, $s\rightarrow\sigma$ y se incluye el error est\'andar $s/\sqrt{n}$, entonces $$\overline{x}\pm t_{\alpha/2}\frac{s}{\sqrt{n}}.$$
\end{itemize}




%---------------------------------------------------------
\subsection*{Intervalos de confianza sobre la varianza}
%---------------------------------------------------------





Supongamos que  $X$ se distribuye normal $\left(\mu,\sigma^{2}\right)$, desconocidas. Sea $X_{1},X_{2},\ldots,X_{n}$ muestra aleatoria de tama\~no $n$ , $s^{2}$ la varianza muestral.

Se sabe que $X^{2}=\frac{\left(n-1\right)s^{2}}{\sigma^{2}}$ se distribuye $\chi^{2}_{n-1}$ grados de libertad. Su intervalo de confianza es
\begin{eqnarray}
\begin{array}{l}
P\left\{\chi^{2}_{1-\frac{\alpha}{2},n-1}\leq\chi^{2}\leq\chi^{2}_{\frac{\alpha}{2},n-1}\right\}=1-\alpha\\
P\left\{\chi^{2}_{1-\frac{\alpha}{2},n-1}\leq\frac{\left(n-1\right)s^{2}}{\sigma^{2}}\leq\chi^{2}_{\frac{\alpha}{2},n-1}\right\}=1-\alpha\\
P\left\{\frac{\left(n-1\right)s^{2}}{\chi^{2}_{\frac{\alpha}{2},n-1}}\leq\sigma^{2}\leq\frac{\left(n-1\right)s^{2}}{\chi^{2}_{1-\frac{\alpha}{2},n-1}}\right\}=1-\alpha
\end{array}
\end{eqnarray}
es decir








\begin{eqnarray}
\sigma^{2}\in\left[\frac{\left(n-1\right)s^{2}}{\chi^{2}_{\frac{\alpha}{2},n-1}},\frac{\left(n-1\right)s^{2}}{\chi^{2}_{1-\frac{\alpha}{2},n-1}}\right]
\end{eqnarray}
los intervalos unilaterales son
\begin{eqnarray}
\sigma^{2}\in\left[\frac{\left(n-1\right)s^{2}}{\chi^{2}_{\frac{\alpha}{2},n-1}},\infty\right]-
\end{eqnarray}
\begin{eqnarray}
\sigma^{2}\in\left[-\infty,\frac{\left(n-1\right)s^{2}}{\chi^{2}_{1-\frac{\alpha}{2},n-1}}\right]
\end{eqnarray}




%---------------------------------------------------------
\subsection*{Intervalos de confianza para proporciones}
%---------------------------------------------------------




Supongamos que se tienen una muestra de tama\~no $n$ de una poblaci\'on grande pero finita, y supongamos que $X$, $X\leq n$, pertenecen a la clase de inter\'es, entonces $$\hat{p}=\frac{\overline{X}}{n}$$ es el estimador puntual de la proporci\'on de la poblaci\'on que pertenece a dicha clase.

$n$ y $p$ son los par\'ametros de la distribuci\'on binomial, entonces $\hat{p}\sim N\left(p,\frac{p\left(1-p\right)}{n}\right)$ aproximadamente si $p$ es distinto de $0$ y $1$; o si $n$ es suficientemente grande. Entonces
\begin{eqnarray*}
Z=\frac{\hat{p}-p}{\sqrt{\frac{p\left(1-p\right)}{n}}}\sim N\left(0,1\right),\textrm{aproximadamente.}
\end{eqnarray*}
 
 
entonces
\begin{eqnarray*}
1-\alpha&=&P\left\{-z_{\alpha/2}\leq\frac{\hat{p}-p}{\sqrt{\frac{p\left(1-p\right)}{n}}}\leq z_{\alpha/2}\right\}\\
&=&P\left\{\hat{p}-z_{\alpha/2}\sqrt{\frac{p\left(1-p\right)}{n}}\leq p\leq \hat{p}+z_{\alpha/2}\sqrt{\frac{p\left(1-p\right)}{n}}\right\}
\end{eqnarray*}
con $\sqrt{\frac{p\left(1-p\right)}{n}}$ error est\'andar del estimador puntual $p$. Una soluci\'on para determinar el intervalo de confianza del par\'ametro $p$ (desconocido) es







\begin{eqnarray*}
1-\alpha=P\left\{\hat{p}-z_{\alpha/2}\sqrt{\frac{\hat{p}\left(1-\hat{p}\right)}{n}}\leq p\leq \hat{p}+z_{\alpha/2}\sqrt{\frac{\hat{p}\left(1-\hat{p}\right)}{n}}\right\}
\end{eqnarray*}
entonces los intervalos de confianza, tanto unilaterales como de dos colas son: 
\begin{itemize}
\item $p\in \left(\hat{p}-z_{\alpha/2}\sqrt{\frac{\hat{p}\left(1-\hat{p}\right)}{n}},\hat{p}+z_{\alpha/2}\sqrt{\frac{\hat{p}\left(1-\hat{p}\right)}{n}}\right)$

\item $p\in \left(-\infty,\hat{p}+z_{\alpha/2}\sqrt{\frac{\hat{p}\left(1-\hat{p}\right)}{n}}\right)$

\item $p\in \left(\hat{p}-z_{\alpha/2}\sqrt{\frac{\hat{p}\left(1-\hat{p}\right)}{n}},\infty\right)$

\end{itemize}
para minimizar el error est\'andar, se propone que el tama\~no de la muestra sea $n= \left(\frac{z_{\alpha/2}}{E}\right)^{2}p\left(1-p\right)$, donde $E=\mid p-\hat{p}\mid$.



%---------------------------------------------------------
\section{Intervalos de confianza para dos muestras}
%---------------------------------------------------------
\subsection*{Varianzas conocidas}
%---------------------------------------------------------




Sean $X_{1}$ y $X_{2}$ variables aleatorias independientes. $X_{1}$ con media desconocida $\mu_{1}$ y varianza conocida $\sigma_{1}^{2}$; y $X_{2}$ con media desconocida $\mu_{2}$ y varianza conocida $\sigma_{2}^{2}$. Se busca encontrar un intervalo de confianza de $100\left(1-\alpha\right)\%$ de la diferencia entre medias $\mu_{1}$ y $\mu_{2}$.\medskip

Sean $X_{11},X_{12},\ldots,X_{1n_{1}}$ muestra aleatoria de $n_{1}$ observaciones de $X_{1}$, y sean $X_{21},X_{22},\ldots,X_{2n_{2}}$ muestra aleatoria de $n_{2}$ observaciones de $X_{2}$.\medskip

Sean $\overline{X}_{1}$ y $\overline{X}_{2}$, medias muestrales, entonces el estad\'sitico 
\begin{eqnarray}
Z=\frac{\left(\overline{X}_{1}-\overline{X}_{2}\right)-\left(\mu_{1}-\mu_{2}\right)}{\sqrt{\frac{\sigma_{1}^{2}}{n_{1}}+\frac{\sigma_{2}^{2}}{n_{2}}}}\sim N\left(0,1\right),\end{eqnarray}
si $X_{1}$ y $X_{2}$ son normales o aproximadamente normales si se aplican las condiciones del Teorema de L\'imite Central respectivamente. 







Entonces se tiene
\begin{eqnarray*}
1-\alpha&=& P\left\{-Z_{\alpha/2}\leq Z\leq Z_{\alpha/2}\right\}\\
&=&P\left\{-Z_{\alpha/2}\leq \frac{\left(\overline{X}_{1}-\overline{X}_{2}\right)-\left(\mu_{1}-\mu_{2}\right)}{\sqrt{\frac{\sigma_{1}^{2}}{n_{1}}+\frac{\sigma_{2}^{2}}{n_{2}}}}\leq Z_{\alpha/2}\right\}\\
&=&P\left\{\left(\overline{X}_{1}-\overline{X}_{2}\right)-Z_{\alpha/2}\sqrt{\frac{\sigma_{1}^{2}}{n_{1}}+\frac{\sigma_{2}^{2}}{n_{2}}}\leq \mu_{1}-\mu_{2}\leq\right.\\
&&\left. \left(\overline{X}_{1}-\overline{X}_{2}\right)+Z_{\alpha/2}\sqrt{\frac{\sigma_{1}^{2}}{n_{1}}+\frac{\sigma_{2}^{2}}{n_{2}}}\right\}
\end{eqnarray*}

Entonces los intervalos de confianza unilaterales y de dos colas al $\left(1-\alpha\right)\%$ de confianza son 







\begin{itemize}
\item $\mu_{1}-\mu_{2}\in \left[\left(\overline{X}_{1}-\overline{X}_{2}\right)-Z_{\alpha/2}\sqrt{\frac{\sigma_{1}^{2}}{n_{1}}+\frac{\sigma_{2}^{2}}{n_{2}}},\left(\overline{X}_{1}-\overline{X}_{2}\right)+Z_{\alpha/2}\sqrt{\frac{\sigma_{1}^{2}}{n_{1}}+\frac{\sigma_{2}^{2}}{n_{2}}}\right]$

\item $\mu_{1}-\mu_{2}\in \left[-\infty,\left(\overline{X}_{1}-\overline{X}_{2}\right)+Z_{\alpha/2}\sqrt{\frac{\sigma_{1}^{2}}{n_{1}}+\frac{\sigma_{2}^{2}}{n_{2}}}\right]$

\item $\mu_{1}-\mu_{2}\in \left[\left(\overline{X}_{1}-\overline{X}_{2}\right)-Z_{\alpha/2}\sqrt{\frac{\sigma_{1}^{2}}{n_{1}}+\frac{\sigma_{2}^{2}}{n_{2}}},\infty\right]$

\end{itemize}








\begin{Note}
Si $\sigma_{1}$ y $\sigma_{2}$ son conocidas, o por lo menos se conoce una aproximaci\'on, y los tama\~nos de las muestras $n_{1}$ y $n_{2}$ son iguales, $n_{1}=n_{2}=n$, se puede determinar el tama\~no de la muestra para que el error al estimar $\mu_{1}-\mu_{2}$ usando $\overline{X}_{1}-\overline{X}_{2}$ sea menor que $E$ (valor del error deseado) al $\left(1-\alpha\right)\%$ de confianza. El tama\~no $n$ de la muestra requerido para cada muestra es
\begin{eqnarray*}
n=\left(\frac{Z_{\alpha/2}}{E}\right)^{2}\left(\sigma_{1}^{2}+\sigma_{2}^{2}\right).
\end{eqnarray*}

\end{Note}





\subsection*{Varianzas desconocidas}






\begin{itemize}
\item Si $n_{1},n_{2}\geq30$ se pueden utilizar los intervalos de la distribuci\'on normal para varianza conocida


\item Si $n_{1},n_{2}$ son muestras peque\~nas, supongase que las poblaciones para $X_{1}$ y $X_{2}$ son normales con varianzas desconocidas y con base en el intervalo de confianza para distribuciones $t$-student
\end{itemize}




\subsubsection*{$\sigma_{1}^{2}=\sigma_{2}^{2}=\sigma$}


Supongamos que $X_{1}$ es una variable aleatoria con media $\mu_{1}$ y varianza $\sigma_{1}^{2}$, $X_{2}$ es una variable aleatoria con media $\mu_{2}$ y varianza $\sigma_{2}^{2}$. Todos los par\'ametros son desconocidos. Sin embargo sup\'ongase que es razonable considerar que $\sigma_{1}^{2}=\sigma_{2}^{2}=\sigma^{2}$.\medskip

Nuevamente sean $X_{1}$ y $X_{2}$ variables aleatorias independientes. $X_{1}$ con media desconocida $\mu_{1}$ y varianza muestral $S_{1}^{2}$; y $X_{2}$ con media desconocida $\mu_{2}$ y varianza muestral $S_{2}^{2}$. Dado que $S_{1}^{2}$ y $S_{2}^{2}$ son estimadores de $\sigma_{1}^{2}$, se propone el estimador $S$ de $\sigma^{2}$ como 

\begin{eqnarray*}
S_{p}^{2}=\frac{\left(n_{1}-1\right)S_{1}^{2}+\left(n_{2}-1\right)S_{2}^{2}}{n_{1}+n_{2}-2},
\end{eqnarray*}
entonces, el estad\'istico para $\mu_{1}-\mu_{2}$ es

\begin{eqnarray*}
t_{\nu}=\frac{\left(\overline{X}_{1}-\overline{X}_{2}\right)-\left(\mu_{1}-\mu_{2}\right)}{S_{p}\sqrt{\frac{1}{n_{1}}+\frac{1}{n_{2}}}}
\end{eqnarray*}
donde $t_{\nu}$ es una $t$ de student con $\nu=n_{1}+n_{2}-2$ grados de libertad.\medskip

Por lo tanto







\begin{eqnarray*}
1-\alpha=P\left\{-t_{\alpha/2,\nu}\leq t\leq t_{\alpha/2,\nu}\right\}\\
=P\left\{\left(\overline{X}_{1}-\overline{X}_{2}\right)-t_{\alpha/2,\nu}S_{p}\sqrt{\frac{1}{n_{1}}+\frac{1}{n_{2}}}\leq \right.\\
\left.t\leq\left(\overline{X}_{1}-\overline{X}_{2}\right)+ t_{\alpha/2,\nu}S_{p}\sqrt{\frac{1}{n_{1}}+\frac{1}{n_{2}}}\right\}
\end{eqnarray*}

luego, los intervalos de confianza del $\left(1-\alpha\right)\%$ para $\mu_{1}-|mu_{2}$ son 
\begin{itemize}
\item $\mu_{1}-\mu_{2}\in\left[\left(\overline{X}_{1}-\overline{X}_{2}\right)- t_{\alpha/2,\nu}S_{p}\sqrt{\frac{1}{n_{1}}+\frac{1}{n_{2}}},\left(\overline{X}_{1}-\overline{X}_{2}\right)+ t_{\alpha/2,\nu}S_{p}\sqrt{\frac{1}{n_{1}}+\frac{1}{n_{2}}}\right]$


\item $\mu_{1}-\mu_{2}\in\left[-\infty,\left(\overline{X}_{1}-\overline{X}_{2}\right)+ t_{\alpha/2,\nu}S_{p}\sqrt{\frac{1}{n_{1}}+\frac{1}{n_{2}}}\right]$

\item $\mu_{1}-\mu_{2}\in\left[\left(\overline{X}_{1}-\overline{X}_{2}\right)- t_{\alpha/2,\nu}S_{p}\sqrt{\frac{1}{n_{1}}+\frac{1}{n_{2}}},\infty\right]$


\end{itemize}





\subsubsection*{$\sigma_{1}^{2}\neq\sigma_{2}^{2}$}





Si no se tiene certeza de que $\sigma_{1}^{2}=\sigma_{2}^{2}$, se propone el estad\'istico
\begin{eqnarray}
t^{*}=\frac{\left(\overline{X}_{1}-\overline{X}_{2}\right)-\left(\mu_{1}-\mu_{2}\right)}{\sqrt{\frac{S_{1}^{2}}{n_{1}}+\frac{S_{2}^{2}}{n_{2}}}}
\end{eqnarray}
que se distribuye $t$-student con $\nu$ grados de libertad, donde

\begin{eqnarray*}
\nu=\frac{\left(\frac{S_{1}^{2}}{n_{1}}+\frac{S_{2}^{2}}{n_{2}}\right)^{2}}{\frac{S_{1}^{2}/n_{1}}{n_{1}+1}+\frac{S_{2}^{2}/n_{2}}{n_{2}+1}}-2
\end{eqnarray*}


Entonces el intervalo de confianza de aproximadamente el $100\left(1-\alpha\right)\%$ para $\mu_{1}-\mu_{2}$ con $\sigma_{1}^{2}\neq\sigma_{2}^{2}$ es
\begin{eqnarray*}
\mu_{1}-\mu_{2}\in\left[\left(\overline{X}_{1}-\overline{X}_{2}\right)-t_{\alpha/2,\nu}\sqrt{\frac{S_{1}^{2}}{n_{1}}+\frac{S_{2}^{2}}{n_{2}}},\right.\\
\left.\left(\overline{X}_{1}-\overline{X}_{2}\right)+t_{\alpha/2,\nu}\sqrt{\frac{S_{1}^{2}}{n_{1}}+\frac{S_{2}^{2}}{n_{2}}}\right]
\end{eqnarray*}






\section{Intervalos de confianza para raz\'on de Varianzas}





Supongamos que se toman dos muestras aleatorias independientes de las dos poblaciones de inter\'es.\medskip

Sean $X_{1}$ y $X_{2}$ variables normales independientes con medias desconocidas $\mu_{1}$ y $\mu_{2}$ y varianzas desconocidas $\sigma_{1}^{2}$ y $\sigma_{2}^{2}$ respectivamente. Se busca un intervalo de confianza de $100\left(1-\alpha\right)\%$ para $\sigma_{1}^{2}/\sigma_{2}^{2}$.\medskip
Supongamos $n_{1}$ y $n_{2}$ muestras aleatorias de $X_{1}$ y $X_{2}$ y sean $S_{1}^{2}$ y $S_{2}^{2}$ varianzas muestralres. Se sabe que 
$$F=\frac{S_{2}^{2}/\sigma_{2}^{2}}{S_{1}^{2}/\sigma_{1}^{2}}$$
se distribuye $F$ con $n_{2}-1$ y $n_{1}-1$ grados de libertad.


Por lo tanto
\begin{eqnarray*}
P\left\{F_{1-\frac{\alpha}{2},n_{2}-1,n_{1}-1}\leq F\leq F_{\frac{\alpha}{2},n_{2}-1,n_{1}-1}\right\}=1-\alpha\\
P\left\{F_{1-\frac{\alpha}{2},n_{2}-1,n_{1}-1}\leq \frac{S_{2}^{2}/\sigma_{2}^{2}}{S_{1}^{2}/\sigma_{1}^{2}}\leq F_{\frac{\alpha}{2},n_{2}-1,n_{1}-1}\right\}=1-\alpha
\end{eqnarray*}
por lo tanto
\begin{eqnarray*}
P\left\{\frac{S_{1}^{2}}{S_{2}^{2}}F_{1-\frac{\alpha}{2},n_{2}-1,n_{1}-1}\leq \frac{\sigma_{1}^{2}}{\sigma_{2}^{2}}\leq \frac{S_{1}^{2}}{S_{2}^{2}}F_{\frac{\alpha}{2},n_{2}-1,n_{1}-1}\right\}=1-\alpha\\
\end{eqnarray*}
entonces








\begin{eqnarray*}
\frac{\sigma_{1}^{2}}{\sigma_{2}^{2}}\in \left[\frac{S_{1}^{2}}{S_{2}^{2}}F_{1-\frac{\alpha}{2},n_{2}-1,n_{1}-1}, \frac{S_{1}^{2}}{S_{2}^{2}}F_{\frac{\alpha}{2},n_{2}-1,n_{1}-1}\right]
\end{eqnarray*}
donde
\begin{eqnarray*}
F_{1-\frac{\alpha}{2},n_{2}-1,n_{1}-1}=\frac{1}{F_{\frac{\alpha}{2},n_{2}-1,n_{1}-1}}
\end{eqnarray*}





\section{Intervalos de confianza para diferencia de proporciones}





Sean dos proporciones de inter\'es $p_{1}$ y $p_{2}$. Se busca un intervalo para $p_{1}-p_{2}$ al $100\left(1-\alpha\right)\%$.\medskip

Sean dos muestras independientes de tama\~no $n_{1}$ y $n_{2}$ de poblaciones infinitas de modo que $X_{1}$ y $X_{2}$ variables aleatorias binomiales independientes con par\'ametros $\left(n_{1},p_{1}\right)$ y $\left(n_{2},p_{2}\right)$.\medskip

$X_{1}$ y $X_{2}$ son  el n\'umero de observaciones que pertenecen a la clase de inter\'es correspondientes. Entonces $\hat{p}_{1}=\frac{X_{1}}{n_{1}}$ y $\hat{p}_{2}=\frac{X_{2}}{n_{2}}$ son estimadores de $p_{1}$ y $p_{2}$ respectivamente. Supongamos que se cumple la aproximaci\'on  normal a la binomial, entonces




\begin{eqnarray*}
Z=\frac{\left(\hat{p}_{1}-\hat{p}_{2}\right)-\left(p_{1}-p_{2}\right)}{\sqrt{\frac{p_{1}\left(1-p_{1}\right)}{n_{1}}-\frac{p_{2}\left(1-p_{2}\right)}{n_{2}}}}\sim N\left(0,1\right)\textrm{aproximadamente}
\end{eqnarray*}
entonces

\begin{eqnarray*}
\left(\hat{p}_{1}-\hat{p}_{2}\right)-Z_{\frac{\alpha}{2}}\sqrt{\frac{\hat{p}_{1}\left(1-\hat{p}_{1}\right)}{n_{1}}+\frac{\hat{p}_{2}\left(1-\hat{p}_{2}\right)}{n_{2}}}\leq p_{1}-p_{2}\\
\leq\left(\hat{p}_{1}-\hat{p}_{2}\right)+Z_{\frac{\alpha}{2}}\sqrt{\frac{\hat{p}_{1}\left(1-\hat{p}_{1}\right)}{n_{1}}-\frac{\hat{p}_{2}\left(1-\hat{p}_{2}\right)}{n_{2}}}
\end{eqnarray*}




\begin{itemize}
\item Una hip\'otesis estad\'istica es una afirmaci\'on  acerca de la distribuci\'on de probabilidad de una variable aleatoria, a menudo involucran uno o m\'as par\'ametros de la distribuci\'on.

\item Las hip\'otesis son afirmaciones respecto a la poblaci\'on o distribuci\'on bajo estudio, no en torno a la muestra.

\item La mayor\'ia de las veces, la prueba de hip\'otesis consiste en determinar si la situaci \'on experimental ha cambiado

\item el inter\'es principal es decidir sobre la veracidad o falsedad de una hip\'otesis, a este procedimiento se le llama \textit{prueba de hip\'otesis}.

\item Si la informaci\'on es consistente con la hip\'otesis, se concluye que esta es verdadera, de lo contrario que con base en la informaci\'on, es falsa.

\end{itemize}




%---------------------------------------------------------
\section{An\'alisis de Regresion Lineal (RL)}
%---------------------------------------------------------



\begin{Note}
\begin{itemize}
\item En muchos problemas hay dos o m\'as variables relacionadas, para medir el grado de relaci\'on se utiliza el \textbf{an\'alisis de regresi\'on}. 
\item Supongamos que se tiene una \'unica variable dependiente, $y$, y varias  variables independientes, $x_{1},x_{2},\ldots,x_{n}$.

\item  La variable $y$ es una varaible aleatoria, y las variables independientes pueden ser distribuidas independiente o conjuntamente. 

\item A la relaci\'on entre estas variables se le denomina modelo regresi\'on de $y$ en $x_{1},x_{2},\ldots,x_{n}$, por ejemplo $y=\phi\left(x_{1},x_{2},\ldots,x_{n}\right)$, lo que se busca es una funci\'on que mejor aproxime a $\phi\left(\cdot\right)$.

\end{itemize}

\end{Note}



%---------------------------------------------------------
\section{An\'alisis de Regresion Lineal (RL)}
%---------------------------------------------------------


\begin{Note}
\begin{itemize}
\item En muchos problemas hay dos o m\'as variables relacionadas, para medir el grado de relaci\'on se utiliza el \textbf{an\'alisis de regresi\'on}. 
\item Supongamos que se tiene una \'unica variable dependiente, $y$, y varias  variables independientes, $x_{1},x_{2},\ldots,x_{n}$.

\item  La variable $y$ es una varaible aleatoria, y las variables independientes pueden ser distribuidas independiente o conjuntamente. 

\item A la relaci\'on entre estas variables se le denomina modelo regresi\'on de $y$ en $x_{1},x_{2},\ldots,x_{n}$, por ejemplo $y=\phi\left(x_{1},x_{2},\ldots,x_{n}\right)$, lo que se busca es una funci\'on que mejor aproxime a $\phi\left(\cdot\right)$.

\end{itemize}

\end{Note}





\subsection{Regresi\'on Lineal Simple (RLS)}




Supongamos que de momento solamente se tienen una variable independiente $x$, para la variable de respuesta $y$. Y supongamos que la relaci\'on que hay entre $x$ y $y$ es una l\'inea recta, y que para cada observaci\'on de $x$, $y$ es una variable aleatoria.

El valor esperado de $y$ para cada valor de $x$ es
\begin{eqnarray}
E\left(y|x\right)=\beta_{0}+\beta_{1}x
\end{eqnarray}
$\beta_{0}$ es la ordenada al or\'igen y  $\beta_{1}$ la pendiente de la recta en cuesti\'on, ambas constantes desconocidas. 

Supongamos que cada observaci\'on $y$ se puede describir por el modelo
\begin{eqnarray}\label{Modelo.Regresion}
y=\beta_{0}+\beta_{1}x+\epsilon
\end{eqnarray}

donde $\epsilon$ es un error aleatorio con media cero y varianza $\sigma^{2}$. Para cada valor $y_{i}$ se tiene $\epsilon_{i}$ variables aleatorias no correlacionadas, cuando se incluyen en el modelo \ref{Modelo.Regresion}, este se le llama \textit{modelo de regresi\'on lineal simple}.






Suponga que se tienen $n$ pares de observaciones $\left(x_{1},y_{1}\right),\left(x_{2},y_{2}\right),\ldots,\left(x_{n},y_{n}\right)$,  estos datos pueden utilizarse para estimar los valores de $\beta_{0}$ y $\beta_{1}$. Esta estimaci\'on es por el \textbf{m\'etodos de m\'inimos cuadrados}.

Entonces la ecuaci\'on (\ref{Modelo.Regresion}) se puede reescribir como
\begin{eqnarray}\label{Modelo.Regresion.dos}
y_{i}=\beta_{0}+\beta_{1}x_{i}+\epsilon_{i},\textrm{ para }i=1,2,\ldots,n.
\end{eqnarray}
Si consideramos la suma de los cuadrados de los errores aleatorios, es decir, el cuadrado de la diferencia entre las observaciones con la recta de regresi\'on 
\begin{eqnarray}
L=\sum_{i=1}^{n}\epsilon^{2}=\sum_{i=1}^{n}\left(y_{i}-\beta_{0}-\beta_{1}x_{i}\right)^{2}
\end{eqnarray}





Para obtener los estimadores por m\'inimos cuadrados de $\beta_{0}$ y $\beta_{1}$,  $\hat{\beta}_{0}$ y $\hat{\beta}_{1}$, es preciso calcular las derivadas parciales con respecto a $\beta_{0}$ y $\beta_{1}$,  igualar a cero y  resolver el sistema de ecuaciones lineales resultante:
\begin{eqnarray*}
\frac{\partial L}{\partial \beta_{0}}=0\\
\frac{\partial L}{\partial \beta_{1}}=0\\
\end{eqnarray*}
evaluando en $\hat{\beta}_{0}$ y $\hat{\beta}_{1}$, se tiene 

\begin{eqnarray*}
-2\sum_{i=1}^{n}\left(y_{i}-\hat{\beta}_{0}-\hat{\beta}_{1}x_{i}\right)&=&0\\
-2\sum_{i=1}^{n}\left(y_{i}-\hat{\beta}_{0}-\hat{\beta}_{1}x_{i}\right)x_{i}&=&0
\end{eqnarray*}
simplificando
\begin{eqnarray*}
n\hat{\beta}_{0}+\hat{\beta}_{1}\sum_{i=1}^{n}x_{i}&=&\sum_{i=1}^{n}y_{i}\\
\hat{\beta}_{0}\sum_{i=1}^{n}x_{i}+\hat{\beta}_{1}\sum_{i=1}^{n}x_{i}^{2}&=&\sum_{i=1}^{n}x_{i}y_{i}
\end{eqnarray*}





Las ecuaciones anteriores se les denominan \textit{ecuaciones normales de m\'inimos cuadrados} con soluci\'on
\begin{eqnarray}
\hat{\beta}_{0}&=&\overline{y}-\hat{\beta}_{1}\overline{x}\\
\hat{\beta}_{1}&=&\frac{\sum_{i=1}^{n}x_{i}y_{i}-\frac{1}{n}\left(\sum_{i=1}^{n}y_{i}\right)\left(\sum_{i=1}^{n}x_{i}\right)}{\sum_{i=1}^{n}x_{i}^{2}-\frac{1}{n}\left(\sum_{i=1}^{n}x_{i}\right)^{2}}
\end{eqnarray}
entonces el modelo de regresi\'on lineal simple ajustado es
\begin{eqnarray}
\hat{y}=\hat{\beta}_{0}+\hat{\beta}_{1}x
\end{eqnarray}

Se intrduce la siguiente notaci\'on
\begin{eqnarray}
S_{xx}&=&\sum_{i=1}^{n}\left(x_{i}-\overline{x}\right)^{2}=\sum_{i=1}^{n}x_{i}^{2}-\frac{1}{n}\left(\sum_{i=1}^{n}x_{i}\right)^{2}\\
S_{xy}&=&\sum_{i=1}^{n}y_{i}\left(x_{i}-\overline{x}\right)=\sum_{i=1}^{n}x_{i}y_{i}-\frac{1}{n}\left(\sum_{i=1}^{n}x_{i}\right)\left(\sum_{i=1}^{n}y_{i}\right)
\end{eqnarray}
y por tanto





\begin{eqnarray}
\hat{\beta}_{1}=\frac{S_{xy}}{S_{xx}}
\end{eqnarray}




\subsection{Regresi\'on Lineal Simple (RLS)}





Supongamos que de momento solamente se tienen una variable independiente $x$, para la variable de respuesta $y$. Y supongamos que la relaci\'on que hay entre $x$ y $y$ es una l\'inea recta, y que para cada observaci\'on de $x$, $y$ es una variable aleatoria.

El valor esperado de $y$ para cada valor de $x$ es
\begin{eqnarray}
E\left(y|x\right)=\beta_{0}+\beta_{1}x
\end{eqnarray}
$\beta_{0}$ es la ordenada al or\'igen y  $\beta_{1}$ la pendiente de la recta en cuesti\'on, ambas constantes desconocidas. 

Supongamos que cada observaci\'on $y$ se puede describir por el modelo
\begin{eqnarray}\label{Modelo.Regresion}
y=\beta_{0}+\beta_{1}x+\epsilon
\end{eqnarray}





donde $\epsilon$ es un error aleatorio con media cero y varianza $\sigma^{2}$. Para cada valor $y_{i}$ se tiene $\epsilon_{i}$ variables aleatorias no correlacionadas, cuando se incluyen en el modelo \ref{Modelo.Regresion}, este se le llama \textit{modelo de regresi\'on lineal simple}.


Suponga que se tienen $n$ pares de observaciones $\left(x_{1},y_{1}\right),\left(x_{2},y_{2}\right),\ldots,\left(x_{n},y_{n}\right)$,  estos datos pueden utilizarse para estimar los valores de $\beta_{0}$ y $\beta_{1}$. Esta estimaci\'on es por el \textbf{m\'etodos de m\'inimos cuadrados}.


Entonces la ecuaci\'on (\ref{Modelo.Regresion}) se puede reescribir como
\begin{eqnarray}\label{Modelo.Regresion.dos}
y_{i}=\beta_{0}+\beta_{1}x_{i}+\epsilon_{i},\textrm{ para }i=1,2,\ldots,n.
\end{eqnarray}
Si consideramos la suma de los cuadrados de los errores aleatorios, es decir, el cuadrado de la diferencia entre las observaciones con la recta de regresi\'on 
\begin{eqnarray}
L=\sum_{i=1}^{n}\epsilon^{2}=\sum_{i=1}^{n}\left(y_{i}-\beta_{0}-\beta_{1}x_{i}\right)^{2}
\end{eqnarray}





Para obtener los estimadores por m\'inimos cuadrados de $\beta_{0}$ y $\beta_{1}$, $\hat{\beta}_{0}$ y $\hat{\beta}_{1}$, es preciso calcular las derivadas parciales con respecto a $\beta_{0}$ y $\beta_{1}$,  igualar a cero y  resolver el sistema de ecuaciones lineales resultante:
\begin{eqnarray*}
\frac{\partial L}{\partial \beta_{0}}=0\\
\frac{\partial L}{\partial \beta_{1}}=0\\
\end{eqnarray*}
evaluando en $\hat{\beta}_{0}$ y $\hat{\beta}_{1}$, se tiene

\begin{eqnarray*}
-2\sum_{i=1}^{n}\left(y_{i}-\hat{\beta}_{0}-\hat{\beta}_{1}x_{i}\right)&=&0\\
-2\sum_{i=1}^{n}\left(y_{i}-\hat{\beta}_{0}-\hat{\beta}_{1}x_{i}\right)x_{i}&=&0
\end{eqnarray*}
simplificando
\begin{eqnarray*}
n\hat{\beta}_{0}+\hat{\beta}_{1}\sum_{i=1}^{n}x_{i}&=&\sum_{i=1}^{n}y_{i}\\
\hat{\beta}_{0}\sum_{i=1}^{n}x_{i}+\hat{\beta}_{1}\sum_{i=1}^{n}x_{i}^{2}&=&\sum_{i=1}^{n}x_{i}y_{i}
\end{eqnarray*}





Las ecuaciones anteriores se les denominan \textit{ecuaciones normales de m\'inimos cuadrados} con soluci\'on
\begin{eqnarray}
\hat{\beta}_{0}&=&\overline{y}-\hat{\beta}_{1}\overline{x}\\
\hat{\beta}_{1}&=&\frac{\sum_{i=1}^{n}x_{i}y_{i}-\frac{1}{n}\left(\sum_{i=1}^{n}y_{i}\right)\left(\sum_{i=1}^{n}x_{i}\right)}{\sum_{i=1}^{n}x_{i}^{2}-\frac{1}{n}\left(\sum_{i=1}^{n}x_{i}\right)^{2}}
\end{eqnarray}
entonces el modelo de regresi\'on lineal simple ajustado es
\begin{eqnarray}
\hat{y}=\hat{\beta}_{0}+\hat{\beta}_{1}x
\end{eqnarray}

Se intrduce la siguiente notaci\'on
\begin{eqnarray}
S_{xx}&=&\sum_{i=1}^{n}\left(x_{i}-\overline{x}\right)^{2}=\sum_{i=1}^{n}x_{i}^{2}-\frac{1}{n}\left(\sum_{i=1}^{n}x_{i}\right)^{2}\\
S_{xy}&=&\sum_{i=1}^{n}y_{i}\left(x_{i}-\overline{x}\right)=\sum_{i=1}^{n}x_{i}y_{i}-\frac{1}{n}\left(\sum_{i=1}^{n}x_{i}\right)\left(\sum_{i=1}^{n}y_{i}\right)
\end{eqnarray}
y por tanto
\begin{eqnarray}
\hat{\beta}_{1}=\frac{S_{xy}}{S_{xx}}
\end{eqnarray}




%---------------------------------------------------------
\section{3. An\'alisis de Regresion Lineal (RL)}
%---------------------------------------------------------




\begin{Note}
\begin{itemize}
\item En muchos problemas hay dos o m\'as variables relacionadas, para medir el grado de relaci\'on se utiliza el \textbf{an\'alisis de regresi\'on}. 
\item Supongamos que se tiene una \'unica variable dependiente, $y$, y varias  variables independientes, $x_{1},x_{2},\ldots,x_{n}$.

\item  La variable $y$ es una varaible aleatoria, y las variables independientes pueden ser distribuidas independiente o conjuntamente. 

\end{itemize}

\end{Note}




\subsection{3.1 Regresi\'on Lineal Simple (RLS)}





\begin{itemize}

\item A la relaci\'on entre estas variables se le denomina modelo regresi\'on de $y$ en $x_{1},x_{2},\ldots,x_{n}$, por ejemplo $y=\phi\left(x_{1},x_{2},\ldots,x_{n}\right)$, lo que se busca es una funci\'on que mejor aproxime a $\phi\left(\cdot\right)$.

\end{itemize}

Supongamos que de momento solamente se tienen una variable independiente $x$, para la variable de respuesta $y$. Y supongamos que la relaci\'on que hay entre $x$ y $y$ es una l\'inea recta, y que para cada observaci\'on de $x$, $y$ es una variable aleatoria.

El valor esperado de $y$ para cada valor de $x$ es
\begin{eqnarray}
E\left(y|x\right)=\beta_{0}+\beta_{1}x
\end{eqnarray}
$\beta_{0}$ es la ordenada al or\'igen y $\beta_{1}$ la pendiente de la recta en cuesti\'on, ambas constantes desconocidas. 





\subsection{3.2 M\'etodo de M\'inimos Cuadrados}




Supongamos que cada observaci\'on $y$ se puede describir por el modelo
\begin{eqnarray}\label{Modelo.Regresion}
y=\beta_{0}+\beta_{1}x+\epsilon
\end{eqnarray}
donde $\epsilon$ es un error aleatorio con media cero y varianza $\sigma^{2}$. Para cada valor $y_{i}$ se tiene $\epsilon_{i}$ variables aleatorias no correlacionadas, cuando se incluyen en el modelo \ref{Modelo.Regresion}, este se le llama \textit{modelo de regresi\'on lineal simple}.


Suponga que se tienen $n$ pares de observaciones $\left(x_{1},y_{1}\right),\left(x_{2},y_{2}\right),\ldots,\left(x_{n},y_{n}\right)$, estos datos pueden utilizarse para estimar los valores de $\beta_{0}$ y $\beta_{1}$. Esta estimaci\'on es por el \textbf{m\'etodos de m\'inimos cuadrados}.







Entonces la ecuaci\'on \ref{Modelo.Regresion} se puede reescribir como
\begin{eqnarray}\label{Modelo.Regresion.dos}
y_{i}=\beta_{0}+\beta_{1}x_{i}+\epsilon_{i},\textrm{ para }i=1,2,\ldots,n.
\end{eqnarray}
Si consideramos la suma de los cuadrados de los errores aleatorios, es decir, el cuadrado de la diferencia entre las observaciones con la recta de regresi\'on
\begin{eqnarray}
L=\sum_{i=1}^{n}\epsilon^{2}=\sum_{i=1}^{n}\left(y_{i}-\beta_{0}-\beta_{1}x_{i}\right)^{2}
\end{eqnarray}






Para obtener los estimadores por m\'inimos cuadrados de $\beta_{0}$ y $\beta_{1}$, $\hat{\beta}_{0}$ y $\hat{\beta}_{1}$, es preciso calcular las derivadas parciales con respecto a $\beta_{0}$ y $\beta_{1}$, igualar a cero y resolver el sistema de ecuaciones lineales resultante:
\begin{eqnarray*}
\frac{\partial L}{\partial \beta_{0}}=0\\
\frac{\partial L}{\partial \beta_{1}}=0\\
\end{eqnarray*}
evaluando en $\hat{\beta}_{0}$ y $\hat{\beta}_{1}$, se tiene

\begin{eqnarray*}
-2\sum_{i=1}^{n}\left(y_{i}-\hat{\beta}_{0}-\hat{\beta}_{1}x_{i}\right)&=&0\\
-2\sum_{i=1}^{n}\left(y_{i}-\hat{\beta}_{0}-\hat{\beta}_{1}x_{i}\right)x_{i}&=&0
\end{eqnarray*}
simplificando
\begin{eqnarray*}
n\hat{\beta}_{0}+\hat{\beta}_{1}\sum_{i=1}^{n}x_{i}&=&\sum_{i=1}^{n}y_{i}\\
\hat{\beta}_{0}\sum_{i=1}^{n}x_{i}+\hat{\beta}_{1}\sum_{i=1}^{n}x_{i}^{2}&=&\sum_{i=1}^{n}x_{i}y_{i}
\end{eqnarray*}
Las ecuaciones anteriores se les denominan \textit{ecuaciones normales de m\'inimos cuadrados} con soluci\'on
\begin{eqnarray}\label{Ecs.Estimadores.Regresion}
\hat{\beta}_{0}&=&\overline{y}-\hat{\beta}_{1}\overline{x}\\
\hat{\beta}_{1}&=&\frac{\sum_{i=1}^{n}x_{i}y_{i}-\frac{1}{n}\left(\sum_{i=1}^{n}y_{i}\right)\left(\sum_{i=1}^{n}x_{i}\right)}{\sum_{i=1}^{n}x_{i}^{2}-\frac{1}{n}\left(\sum_{i=1}^{n}x_{i}\right)^{2}}
\end{eqnarray}





entonces el modelo de regresi\'on lineal simple ajustado es
\begin{eqnarray}
\hat{y}=\hat{\beta}_{0}+\hat{\beta}_{1}x
\end{eqnarray}
Se intrduce la siguiente notaci\'on
\begin{eqnarray}
S_{xx}&=&\sum_{i=1}^{n}\left(x_{i}-\overline{x}\right)^{2}=\sum_{i=1}^{n}x_{i}^{2}-\frac{1}{n}\left(\sum_{i=1}^{n}x_{i}\right)^{2}\\
S_{xy}&=&\sum_{i=1}^{n}y_{i}\left(x_{i}-\overline{x}\right)=\sum_{i=1}^{n}x_{i}y_{i}-\frac{1}{n}\left(\sum_{i=1}^{n}x_{i}\right)\left(\sum_{i=1}^{n}y_{i}\right)
\end{eqnarray}
y por tanto
\begin{eqnarray}
\hat{\beta}_{1}=\frac{S_{xy}}{S_{xx}}
\end{eqnarray}




\subsection{3.3 Propiedades de los Estimadores $\hat{\beta}_{0}$ y $\hat{\beta}_{1}$}


\begin{Note}
\begin{itemize}
\item Las propiedades estad\'isticas de los estimadores de m\'inimos cuadrados son \'utiles para evaluar la suficiencia del modelo.

\item Dado que $\hat{\beta}_{0}$ y  $\hat{\beta}_{1}$ son combinaciones lineales de las variables aleatorias $y_{i}$, tambi\'en resultan ser variables aleatorias.
\end{itemize}
\end{Note}
A saber
\begin{eqnarray*}
E\left(\hat{\beta}_{1}\right)&=&E\left(\frac{S_{xy}}{S_{xx}}\right)=\frac{1}{S_{xx}}E\left(\sum_{i=1}^{n}y_{i}\left(x_{i}-\overline{x}\right)\right)
\end{eqnarray*}
\begin{eqnarray*}
&=&\frac{1}{S_{xx}}E\left(\sum_{i=1}^{n}\left(\beta_{0}+\beta_{1}x_{i}+\epsilon_{i}\right)\left(x_{i}-\overline{x}\right)\right)\\
&=&\frac{1}{S_{xx}}\left[\beta_{0}E\left(\sum_{k=1}^{n}\left(x_{k}-\overline{x}\right)\right)+E\left(\beta_{1}\sum_{k=1}^{n}x_{k}\left(x_{k}-\overline{x}\right)\right)\right.\\
&+&\left.E\left(\sum_{k=1}^{n}\epsilon_{k}\left(x_{k}-\overline{x}\right)\right)\right]=\frac{1}{S_{xx}}\beta_{1}S_{xx}=\beta_{1}
\end{eqnarray*}






por lo tanto 
\begin{equation}\label{Esperanza.Beta.1}
E\left(\hat{\beta}_{1}\right)=\beta_{1}
\end{equation}
\begin{Note}
Es decir, $\hat{\beta_{1}}$ es un estimador insesgado.
\end{Note}
Ahora calculemos la varianza:
\begin{eqnarray*}
V\left(\hat{\beta}_{1}\right)&=&V\left(\frac{S_{xy}}{S_{xx}}\right)=\frac{1}{S_{xx}^{2}}V\left(\sum_{k=1}^{n}y_{k}\left(x_{k}-\overline{x}\right)\right)\\
&=&\frac{1}{S_{xx}^{2}}\sum_{k=1}^{n}V\left(y_{k}\left(x_{k}-\overline{x}\right)\right)=\frac{1}{S_{xx}^{2}}\sum_{k=1}^{n}\sigma^{2}\left(x_{k}-\overline{x}\right)^{2}\\
&=&\frac{\sigma^{2}}{S_{xx}^{2}}\sum_{k=1}^{n}\left(x_{k}-\overline{x}\right)^{2}=\frac{\sigma^{2}}{S_{xx}}
\end{eqnarray*}






por lo tanto
\begin{equation}\label{Varianza.Beta.1}
V\left(\hat{\beta}_{1}\right)=\frac{\sigma^{2}}{S_{xx}}
\end{equation}
\begin{Prop}
\begin{eqnarray*}
E\left(\hat{\beta}_{0}\right)&=&\beta_{0},\\
V\left(\hat{\beta}_{0}\right)&=&\sigma^{2}\left(\frac{1}{n}+\frac{\overline{x}^{2}}{S_{xx}}\right),\\
Cov\left(\hat{\beta}_{0},\hat{\beta}_{1}\right)&=&-\frac{\sigma^{2}\overline{x}}{S_{xx}}.
\end{eqnarray*}
\end{Prop}
Para estimar $\sigma^{2}$ es preciso definir la diferencia entre la observaci\'on $y_{k}$, y el valor predecido $\hat{y}_{k}$, es decir
\begin{eqnarray*}
e_{k}=y_{k}-\hat{y}_{k},\textrm{ se le denomina \textbf{residuo}.}
\end{eqnarray*}
La suma de los cuadrados de los errores de los reisduos, \textit{suma de cuadrados del error}
\begin{eqnarray}
SC_{E}=\sum_{k=1}^{n}e_{k}^{2}=\sum_{k=1}^{n}\left(y_{k}-\hat{y}_{k}\right)^{2}
\end{eqnarray}






sustituyendo $\hat{y}_{k}=\hat{\beta}_{0}+\hat{\beta}_{1}x_{k}$ se obtiene
\begin{eqnarray*}
SC_{E}&=&\sum_{k=1}^{n}y_{k}^{2}-n\overline{y}^{2}-\hat{\beta}_{1}S_{xy}=S_{yy}-\hat{\beta}_{1}S_{xy},\\
E\left(SC_{E}\right)&=&\left(n-2\right)\sigma^{2},\textrm{ por lo tanto}\\
\hat{\sigma}^{2}&=&\frac{SC_{E}}{n-2}=MC_{E}\textrm{ es un estimador insesgado de }\sigma^{2}.
\end{eqnarray*}



\subsection{3.4 Prueba de Hip\'otesis en RLS}




\begin{itemize}
\item Para evaluar la suficiencia del modelo de regresi\'on lineal simple, es necesario lleva a cabo una prueba de hip\'otesis respecto de los par\'ametros del modelo as\'i como de la construcci\'on de intervalos de confianza.

\item Para poder realizar la prueba de hip\'otesis sobre la pendiente y la ordenada al or\'igen de la recta de regresi\'on es necesario hacer el supuesto de que el error $\epsilon_{i}$ se distribuye normalmente, es decir $\epsilon_{i} \sim N\left(0,\sigma^{2}\right)$.

\end{itemize}






Suponga que se desea probar la hip\'otesis de que la pendiente es igual a una constante, $\beta_{0,1}$ las hip\'otesis Nula y Alternativa son:
\begin{centering}
\begin{itemize}
\item[$H_{0}$: ] $\beta_{1}=\beta_{1,0}$,

\item[$H_{1}$: ]$\beta_{1}\neq\beta_{1,0}$.

\end{itemize}
donde dado que las $\epsilon_{i} \sim N\left(0,\sigma^{2}\right)$, se tiene que $y_{i}$ son variables aleatorias normales $N\left(\beta_{0}+\beta_{1}x_{1},\sigma^{2}\right)$. 
\end{centering}





De las ecuaciones (\ref{Ecs.Estimadores.Regresion}) se desprende que $\hat{\beta}_{1}$ es combinaci\'on lineal de variables aleatorias normales independientes, es decir, $\hat{\beta}_{1}\sim N\left(\beta_{1},\sigma^{2}/S_{xx}\right)$, recordar las ecuaciones (\ref{Esperanza.Beta.1}) y (\ref{Varianza.Beta.1}).
Entonces se tiene que el estad\'istico de prueba apropiado es
\begin{equation}\label{Estadistico.Beta.1}
t_{0}=\frac{\hat{\beta}_{1}-\hat{\beta}_{1,0}}{\sqrt{MC_{E}/S_{xx}}}
\end{equation}
que se distribuye $t$ con $n-2$ grados de libertad bajo $H_{0}:\beta_{1}=\beta_{1,0}$. Se rechaza $H_{0}$ si 
\begin{equation}\label{Zona.Rechazo.Beta.1}
t_{0}|>t_{\alpha/2,n-2}.
\end{equation}







Para $\beta_{0}$ se puede proceder de manera an\'aloga para
\begin{itemize}
\item[$H_{0}:$] $\beta_{0}=\beta_{0,0}$,
\item[$H_{1}:$] $\beta_{0}\neq\beta_{0,0}$,
\end{itemize}
con $\hat{\beta}_{0}\sim N\left(\beta_{0},\sigma^{2}\left(\frac{1}{n}+\frac{\overline{x}^{2}}{S_{xx}}\right)\right)$, por lo tanto
\begin{equation}\label{Estadistico.Beta.0}
t_{0}=\frac{\hat{\beta}_{0}-\beta_{0,0}}{MC_{E}\left[\frac{1}{n}+\frac{\overline{x}^{2}}{S_{xx}}\right]},
\end{equation}
con el que rechazamos la hip\'otesis nula si
\begin{equation}\label{Zona.Rechazo.Beta.0}
t_{0}|>t_{\alpha/2,n-2}.
\end{equation}






\begin{itemize}
\item No rechazar $H_{0}:\beta_{1}=0$ es equivalente a decir que no hay relaci\'on lineal entre $x$ y $y$.
\item Alternativamente, si $H_{0}:\beta_{1}=0$ se rechaza, esto implica que $x$ explica la variabilidad de $y$, es decir, podr\'ia significar que la l\'inea recta esel modelo adecuado.
\end{itemize}
El procedimiento de prueba para $H_{0}:\beta_{1}=0$ puede realizarse de la siguiente manera:
\begin{eqnarray*}
S_{yy}=\sum_{k=1}^{n}\left(y_{k}-\overline{y}\right)^{2}=\sum_{k=1}^{n}\left(\hat{y}_{k}-\overline{y}\right)^{2}+\sum_{k=1}^{n}\left(y_{k}-\hat{y}_{k}\right)^{2}
\end{eqnarray*}

\begin{eqnarray*}
S_{yy}&=&\sum_{k=1}^{n}\left(y_{k}-\overline{y}\right)^{2}=\sum_{k=1}^{n}\left(y_{k}-\hat{y}_{k}+\hat{y}_{k}-\overline{y}\right)^{2}\\
&=&\sum_{k=1}^{n}\left[\left(\hat{y}_{k}-\overline{y}\right)+\left(y_{k}-\hat{y}_{k}\right)\right]^{2}\\
&=&\sum_{k=1}^{n}\left[\left(\hat{y}_{k}-\overline{y}\right)^{2}+2\left(\hat{y}_{k}-\overline{y}\right)\left(y_{k}-\hat{y}_{k}\right)+\left(y_{k}-\hat{y}_{k}\right)^{2}\right]\\
&=&\sum_{k=1}^{n}\left(\hat{y}_{k}-\overline{y}\right)^{2}+2\sum_{k=1}^{n}\left(\hat{y}_{k}-\overline{y}\right)\left(y_{k}-\hat{y}_{k}\right)+\sum_{k=1}^{n}\left(y_{k}-\hat{y}_{k}\right)^{2}
\end{eqnarray*}






\begin{eqnarray*}
&&\sum_{k=1}^{n}\left(\hat{y}_{k}-\overline{y}\right)\left(y_{k}-\hat{y}_{k}\right)=\sum_{k=1}^{n}\hat{y}_{k}\left(y_{k}-\hat{y}_{k}\right)-\sum_{k=1}^{n}\overline{y}\left(y_{k}-\hat{y}_{k}\right)\\
&=&\sum_{k=1}^{n}\hat{y}_{k}\left(y_{k}-\hat{y}_{k}\right)-\overline{y}\sum_{k=1}^{n}\left(y_{k}-\hat{y}_{k}\right)\\
&=&\sum_{k=1}^{n}\left(\hat{\beta}_{0}+\hat{\beta}_{1}x_{k}\right)\left(y_{k}-\hat{\beta}_{0}-\hat{\beta}_{1}x_{k}\right)-\overline{y}\sum_{k=1}^{n}\left(y_{k}-\hat{\beta}_{0}-\hat{\beta}_{1}x_{k}\right)
\end{eqnarray*}

\begin{eqnarray*}
&=&\sum_{k=1}^{n}\hat{\beta}_{0}\left(y_{k}-\hat{\beta}_{0}-\hat{\beta}_{1}x_{k}\right)+\sum_{k=1}^{n}\hat{\beta}_{1}x_{k}\left(y_{k}-\hat{\beta}_{0}-\hat{\beta}_{1}x_{k}\right)\\
&-&\overline{y}\sum_{k=1}^{n}\left(y_{k}-\hat{\beta}_{0}-\hat{\beta}_{1}x_{k}\right)\\
&=&\hat{\beta}_{0}\sum_{k=1}^{n}\left(y_{k}-\hat{\beta}_{0}-\hat{\beta}_{1}x_{k}\right)+\hat{\beta}_{1}\sum_{k=1}^{n}x_{k}\left(y_{k}-\hat{\beta}_{0}-\hat{\beta}_{1}x_{k}\right)\\
&-&\overline{y}\sum_{k=1}^{n}\left(y_{k}-\hat{\beta}_{0}-\hat{\beta}_{1}x_{k}\right)=0+0+0=0.
\end{eqnarray*}






Por lo tanto, efectivamente se tiene
\begin{equation}\label{Suma.Total.Cuadrados}
S_{yy}=\sum_{k=1}^{n}\left(\hat{y}_{k}-\overline{y}\right)^{2}+\sum_{k=1}^{n}\left(y_{k}-\hat{y}_{k}\right)^{2},
\end{equation}
donde se hacen las definiciones
\begin{eqnarray}
SC_{E}&=&\sum_{k=1}^{n}\left(\hat{y}_{k}-\overline{y}\right)^{2}\cdots\textrm{Suma de Cuadrados del Error}\\
SC_{R}&=&\sum_{k=1}^{n}\left(y_{k}-\hat{y}_{k}\right)^{2}\cdots\textrm{ Suma de Regresi\'on de Cuadrados}
\end{eqnarray}
Por lo tanto la ecuaci\'on (\ref{Suma.Total.Cuadrados}) se puede reescribir como 
\begin{equation}\label{Suma.Total.Cuadrados.Dos}
S_{yy}=SC_{R}+SC_{E}
\end{equation}
recordemos que $SC_{E}=S_{yy}-\hat{\beta}_{1}S_{xy}$
\begin{eqnarray*}
S_{yy}&=&SC_{R}+\left( S_{yy}-\hat{\beta}_{1}S_{xy}\right)\\
S_{xy}&=&\frac{1}{\hat{\beta}_{1}}SC_{R}
\end{eqnarray*}
$S_{xy}$ tiene $n-1$ grados de libertad y $SC_{R}$ y $SC_{E}$ tienen 1 y $n-2$ grados de libertad respectivamente.






\begin{Prop}
\begin{equation}
E\left(SC_{R}\right)=\sigma^{2}+\beta_{1}S_{xx}
\end{equation}
adem\'as, $SC_{E}$ y $SC_{R}$ son independientes.
\end{Prop}
Recordemos que $\hat{\beta}_{1}=\frac{S_{xy}}{S_{xx}}$. Para $H_{0}:\beta_{1}=0$ verdadera,
\begin{eqnarray*}
F_{0}=\frac{SC_{R}/1}{SC_{E}/(n-2)}=\frac{MC_{R}}{MC_{E}}
\end{eqnarray*}
se distribuye $F_{1,n-2}$, y se rechazar\'ia $H_{0}$ si $F_{0}>F_{\alpha,1,n-2}$.

El procedimiento de prueba de hip\'otesis puede presentarse como la tabla de an\'alisis de varianza siguiente\medskip






\begin{tabular}{lcccc}\hline
Fuente de & Suma de  &  Grados de  & Media  & $F_{0}$ \\ 
 variaci\'on & Cuadrados & Libertad & Cuadr\'atica & \\\hline
 Regresi\'on & $SC_{R}$ & 1 & $MC_{R}$  & $MC_{R}/MC_{E}$\\
 Error Residual & $SC_{E}$ & $n-2$ & $MC_{E}$ & \\\hline
 Total & $S_{yy}$ & $n-1$ & & \\\hline
\end{tabular} 

La prueba para la significaci\'on de la regresi\'on puede desarrollarse bas\'andose en la expresi\'on (\ref{Estadistico.Beta.1}), con $\hat{\beta}_{1,0}=0$, es decir
\begin{equation}\label{Estadistico.Beta.1.Cero}
t_{0}=\frac{\hat{\beta}_{1}}{\sqrt{MC_{E}/S_{xx}}}
\end{equation}





Elevando al cuadrado ambos t\'erminos:
\begin{eqnarray*}
t_{0}^{2}=\frac{\hat{\beta}_{1}^{2}S_{xx}}{MC_{E}}=\frac{\hat{\beta}_{1}S_{xy}}{MC_{E}}=\frac{MC_{R}}{MC_{E}}
\end{eqnarray*}
Observar que $t_{0}^{2}=F_{0}$, por tanto la prueba que se utiliza para $t_{0}$ es la misma que para $F_{0}$.




\subsection{Estimaci\'on de Intervalos en RLS}




\begin{itemize}
\item Adem\'as de la estimaci\'on puntual para los par\'ametros $\beta_{1}$ y $\beta_{0}$, es posible obtener estimaciones del intervalo de confianza de estos par\'ametros.

\item El ancho de estos intervalos de confianza es una medida de la calidad total de la recta de regresi\'on.

\end{itemize}






Si los $\epsilon_{k}$ se distribuyen normal e independientemente, entonces
\begin{eqnarray*}
\begin{array}{ccc}
\frac{\left(\hat{\beta}_{1}-\beta_{1}\right)}{\sqrt{\frac{MC_{E}}{S_{xx}}}}&y &\frac{\left(\hat{\beta}_{0}-\beta_{0}\right)}{\sqrt{MC_{E}\left(\frac{1}{n}+\frac{\overline{x}^{2}}{S_{xx}}\right)}}
\end{array}
\end{eqnarray*}
se distribuyen $t$ con $n-2$ grados de libertad. Por tanto un intervalo de confianza de $100\left(1-\alpha\right)\%$ para $\beta_{1}$ est\'a dado por
\begin{eqnarray}
\hat{\beta}_{1}-t_{\alpha/2,n-2}\sqrt{\frac{MC_{E}}{S_{xx}}}\leq \beta_{1}\leq\hat{\beta}_{1}+t_{\alpha/2,n-2}\sqrt{\frac{MC_{E}}{S_{xx}}}.
\end{eqnarray}
De igual manera, para $\beta_{0}$ un intervalo de confianza al $100\left(1-\alpha\right)\%$ es
\small{
\begin{eqnarray}
\begin{array}{l}
\hat{\beta}_{0}-t_{\alpha/2,n-2}\sqrt{MC_{E}\left(\frac{1}{n}+\frac{\overline{x}^{2}}{S_{xx}}\right)}\leq\beta_{0}\leq\hat{\beta}_{0}+t_{\alpha/2,n-2}\sqrt{MC_{E}\left(\frac{1}{n}+\frac{\overline{x}^{2}}{S_{xx}}\right)}
\end{array}
\end{eqnarray}}



%---------------------------------------------------------
\section{3. An\'alisis de Regresion Lineal (RL)}
%---------------------------------------------------------



\begin{Note}
\begin{itemize}
\item En muchos problemas hay dos o m\'as variables relacionadas, para medir el grado de relaci\'on se utiliza el \textbf{an\'alisis de regresi\'on}. 
\item Supongamos que se tiene una \'unica variable dependiente, $y$, y varias  variables independientes, $x_{1},x_{2},\ldots,x_{n}$.

\item  La variable $y$ es una varaible aleatoria, y las variables independientes pueden ser distribuidas independiente o conjuntamente. 

\end{itemize}

\end{Note}





\subsection{3.1 Regresi\'on Lineal Simple (RLS)}





\begin{itemize}

\item A la relaci\'on entre estas variables se le denomina modelo regresi\'on de $y$ en $x_{1},x_{2},\ldots,x_{n}$, por ejemplo $y=\phi\left(x_{1},x_{2},\ldots,x_{n}\right)$, lo que se busca es una funci\'on que mejor aproxime a $\phi\left(\cdot\right)$.

\end{itemize}

Supongamos que de momento solamente se tienen una variable independiente $x$, para la variable de respuesta $y$. Y supongamos que la relaci\'on que hay entre $x$ y $y$ es una l\'inea recta, y que para cada observaci\'on de $x$, $y$ es una variable aleatoria.

El valor esperado de $y$ para cada valor de $x$ es
\begin{eqnarray}
E\left(y|x\right)=\beta_{0}+\beta_{1}x
\end{eqnarray}
$\beta_{0}$ es la ordenada al or\'igen y $\beta_{1}$ la pendiente de la recta en cuesti\'on, ambas constantes desconocidas. 




\subsection{3.2 M\'etodo de M\'inimos Cuadrados}




Supongamos que cada observaci\'on $y$ se puede describir por el modelo
\begin{eqnarray}\label{Modelo.Regresion}
y=\beta_{0}+\beta_{1}x+\epsilon
\end{eqnarray}
donde $\epsilon$ es un error aleatorio con media cero y varianza $\sigma^{2}$. Para cada valor $y_{i}$ se tiene $\epsilon_{i}$ variables aleatorias no correlacionadas, cuando se incluyen en el modelo \ref{Modelo.Regresion}, este se le llama \textit{modelo de regresi\'on lineal simple}.


Suponga que se tienen $n$ pares de observaciones $\left(x_{1},y_{1}\right),\left(x_{2},y_{2}\right),\ldots,\left(x_{n},y_{n}\right)$, estos datos pueden utilizarse para estimar los valores de $\beta_{0}$ y $\beta_{1}$. Esta estimaci\'on es por el \textbf{m\'etodos de m\'inimos cuadrados}.









Entonces la ecuaci\'on \ref{Modelo.Regresion} se puede reescribir como
\begin{eqnarray}\label{Modelo.Regresion.dos}
y_{i}=\beta_{0}+\beta_{1}x_{i}+\epsilon_{i},\textrm{ para }i=1,2,\ldots,n.
\end{eqnarray}
Si consideramos la suma de los cuadrados de los errores aleatorios, es decir, el cuadrado de la diferencia entre las observaciones con la recta de regresi\'on
\begin{eqnarray}
L=\sum_{i=1}^{n}\epsilon^{2}=\sum_{i=1}^{n}\left(y_{i}-\beta_{0}-\beta_{1}x_{i}\right)^{2}
\end{eqnarray}
Para obtener los estimadores por m\'inimos cuadrados de $\beta_{0}$ y $\beta_{1}$, $\hat{\beta}_{0}$ y $\hat{\beta}_{1}$, es preciso calcular las derivadas parciales con respecto a $\beta_{0}$ y $\beta_{1}$, igualar a cero y resolver el sistema de ecuaciones lineales resultante:
\begin{eqnarray*}
\frac{\partial L}{\partial \beta_{0}}=0\\
\frac{\partial L}{\partial \beta_{1}}=0\\
\end{eqnarray*}





evaluando en $\hat{\beta}_{0}$ y $\hat{\beta}_{1}$, se tiene
\begin{eqnarray*}
-2\sum_{i=1}^{n}\left(y_{i}-\hat{\beta}_{0}-\hat{\beta}_{1}x_{i}\right)&=&0\\
-2\sum_{i=1}^{n}\left(y_{i}-\hat{\beta}_{0}-\hat{\beta}_{1}x_{i}\right)x_{i}&=&0
\end{eqnarray*}
simplificando
\begin{eqnarray*}
n\hat{\beta}_{0}+\hat{\beta}_{1}\sum_{i=1}^{n}x_{i}&=&\sum_{i=1}^{n}y_{i}\\
\hat{\beta}_{0}\sum_{i=1}^{n}x_{i}+\hat{\beta}_{1}\sum_{i=1}^{n}x_{i}^{2}&=&\sum_{i=1}^{n}x_{i}y_{i}
\end{eqnarray*}
Las ecuaciones anteriores se les denominan \textit{ecuaciones normales de m\'inimos cuadrados} con soluci\'on
\begin{eqnarray}\label{Ecs.Estimadores.Regresion}
\hat{\beta}_{0}&=&\overline{y}-\hat{\beta}_{1}\overline{x}\\
\hat{\beta}_{1}&=&\frac{\sum_{i=1}^{n}x_{i}y_{i}-\frac{1}{n}\left(\sum_{i=1}^{n}y_{i}\right)\left(\sum_{i=1}^{n}x_{i}\right)}{\sum_{i=1}^{n}x_{i}^{2}-\frac{1}{n}\left(\sum_{i=1}^{n}x_{i}\right)^{2}}
\end{eqnarray}





entonces el modelo de regresi\'on lineal simple ajustado es
\begin{eqnarray}
\hat{y}=\hat{\beta}_{0}+\hat{\beta}_{1}x
\end{eqnarray}
Se intrduce la siguiente notaci\'on
\begin{eqnarray}
S_{xx}&=&\sum_{i=1}^{n}\left(x_{i}-\overline{x}\right)^{2}=\sum_{i=1}^{n}x_{i}^{2}-\frac{1}{n}\left(\sum_{i=1}^{n}x_{i}\right)^{2}\\
S_{xy}&=&\sum_{i=1}^{n}y_{i}\left(x_{i}-\overline{x}\right)=\sum_{i=1}^{n}x_{i}y_{i}-\frac{1}{n}\left(\sum_{i=1}^{n}x_{i}\right)\left(\sum_{i=1}^{n}y_{i}\right)
\end{eqnarray}
y por tanto
\begin{eqnarray}
\hat{\beta}_{1}=\frac{S_{xy}}{S_{xx}}
\end{eqnarray}





\subsection{3.3 Propiedades de los Estimadores $\hat{\beta}_{0}$ y $\hat{\beta}_{1}$}





\begin{Note}
\begin{itemize}
\item Las propiedades estad\'isticas de los estimadores de m\'inimos cuadrados son \'utiles para evaluar la suficiencia del modelo.

\item Dado que $\hat{\beta}_{0}$ y  $\hat{\beta}_{1}$ son combinaciones lineales de las variables aleatorias $y_{i}$, tambi\'en resultan ser variables aleatorias.
\end{itemize}
\end{Note}
A saber
\begin{eqnarray*}
E\left(\hat{\beta}_{1}\right)&=&E\left(\frac{S_{xy}}{S_{xx}}\right)=\frac{1}{S_{xx}}E\left(\sum_{i=1}^{n}y_{i}\left(x_{i}-\overline{x}\right)\right)
\end{eqnarray*}






\begin{eqnarray*}
&=&\frac{1}{S_{xx}}E\left(\sum_{i=1}^{n}\left(\beta_{0}+\beta_{1}x_{i}+\epsilon_{i}\right)\left(x_{i}-\overline{x}\right)\right)\\
&=&\frac{1}{S_{xx}}\left[\beta_{0}E\left(\sum_{k=1}^{n}\left(x_{k}-\overline{x}\right)\right)+E\left(\beta_{1}\sum_{k=1}^{n}x_{k}\left(x_{k}-\overline{x}\right)\right)\right.\\
&+&\left.E\left(\sum_{k=1}^{n}\epsilon_{k}\left(x_{k}-\overline{x}\right)\right)\right]=\frac{1}{S_{xx}}\beta_{1}S_{xx}=\beta_{1}
\end{eqnarray*}
por lo tanto 
\begin{equation}\label{Esperanza.Beta.1}
E\left(\hat{\beta}_{1}\right)=\beta_{1}
\end{equation}






\begin{Note}
Es decir, $\hat{\beta_{1}}$ es un estimador insesgado.
\end{Note}
Ahora calculemos la varianza:
\begin{eqnarray*}
V\left(\hat{\beta}_{1}\right)&=&V\left(\frac{S_{xy}}{S_{xx}}\right)=\frac{1}{S_{xx}^{2}}V\left(\sum_{k=1}^{n}y_{k}\left(x_{k}-\overline{x}\right)\right)\\
&=&\frac{1}{S_{xx}^{2}}\sum_{k=1}^{n}V\left(y_{k}\left(x_{k}-\overline{x}\right)\right)=\frac{1}{S_{xx}^{2}}\sum_{k=1}^{n}\sigma^{2}\left(x_{k}-\overline{x}\right)^{2}\\
&=&\frac{\sigma^{2}}{S_{xx}^{2}}\sum_{k=1}^{n}\left(x_{k}-\overline{x}\right)^{2}=\frac{\sigma^{2}}{S_{xx}}
\end{eqnarray*}







por lo tanto
\begin{equation}\label{Varianza.Beta.1}
V\left(\hat{\beta}_{1}\right)=\frac{\sigma^{2}}{S_{xx}}
\end{equation}
\begin{Prop}
\begin{eqnarray*}
E\left(\hat{\beta}_{0}\right)&=&\beta_{0},\\
V\left(\hat{\beta}_{0}\right)&=&\sigma^{2}\left(\frac{1}{n}+\frac{\overline{x}^{2}}{S_{xx}}\right),\\
Cov\left(\hat{\beta}_{0},\hat{\beta}_{1}\right)&=&-\frac{\sigma^{2}\overline{x}}{S_{xx}}.
\end{eqnarray*}
\end{Prop}

Para estimar $\sigma^{2}$ es preciso definir la diferencia entre la observaci\'on $y_{k}$, y el valor predecido $\hat{y}_{k}$, es decir
\begin{eqnarray*}
e_{k}=y_{k}-\hat{y}_{k},\textrm{ se le denomina \textbf{residuo}.}
\end{eqnarray*}
La suma de los cuadrados de los errores de los reisduos, \textit{suma de cuadrados del error}
\begin{eqnarray}
SC_{E}=\sum_{k=1}^{n}e_{k}^{2}=\sum_{k=1}^{n}\left(y_{k}-\hat{y}_{k}\right)^{2}
\end{eqnarray}






sustituyendo $\hat{y}_{k}=\hat{\beta}_{0}+\hat{\beta}_{1}x_{k}$ se obtiene
\begin{eqnarray*}
SC_{E}&=&\sum_{k=1}^{n}y_{k}^{2}-n\overline{y}^{2}-\hat{\beta}_{1}S_{xy}=S_{yy}-\hat{\beta}_{1}S_{xy},\\
E\left(SC_{E}\right)&=&\left(n-2\right)\sigma^{2},\textrm{ por lo tanto}\\
\hat{\sigma}^{2}&=&\frac{SC_{E}}{n-2}=MC_{E}\textrm{ es un estimador insesgado de }\sigma^{2}.
\end{eqnarray*}






%\end{document}
\subsection{3.4 Prueba de Hip\'otesis en RLS}




\begin{itemize}
\item Para evaluar la suficiencia del modelo de regresi\'on lineal simple, es necesario lleva a cabo una prueba de hip\'otesis respecto de los par\'ametros del modelo as\'i como de la construcci\'on de intervalos de confianza.

\item Para poder realizar la prueba de hip\'otesis sobre la pendiente y la ordenada al or\'igen de la recta de regresi\'on es necesario hacer el supuesto de que el error $\epsilon_{i}$ se distribuye normalmente, es decir $\epsilon_{i} \sim N\left(0,\sigma^{2}\right)$.

\end{itemize}


Suponga que se desea probar la hip\'otesis de que la pendiente es igual a una constante, $\beta_{0,1}$ las hip\'otesis Nula y Alternativa son:
\begin{centering}
\begin{itemize}
\item[$H_{0}$: ] $\beta_{1}=\beta_{1,0}$,

\item[$H_{1}$: ]$\beta_{1}\neq\beta_{1,0}$.

\end{itemize}

donde dado que las $\epsilon_{i} \sim N\left(0,\sigma^{2}\right)$, se tiene que $y_{i}$ son variables aleatorias normales $N\left(\beta_{0}+\beta_{1}x_{1},\sigma^{2}\right)$. 
\end{centering}




De las ecuaciones (\ref{Ecs.Estimadores.Regresion}) se desprende que $\hat{\beta}_{1}$ es combinaci\'on lineal de variables aleatorias normales independientes, es decir, $\hat{\beta}_{1}\sim N\left(\beta_{1},\sigma^{2}/S_{xx}\right)$, recordar las ecuaciones (\ref{Esperanza.Beta.1}) y (\ref{Varianza.Beta.1}).

Entonces se tiene que el estad\'istico de prueba apropiado es
\begin{equation}\label{Estadistico.Beta.1}
t_{0}=\frac{\hat{\beta}_{1}-\hat{\beta}_{1,0}}{\sqrt{MC_{E}/S_{xx}}}
\end{equation}
que se distribuye $t$ con $n-2$ grados de libertad bajo $H_{0}:\beta_{1}=\beta_{1,0}$. Se rechaza $H_{0}$ si 
\begin{equation}\label{Zona.Rechazo.Beta.1}
t_{0}|>t_{\alpha/2,n-2}.
\end{equation}







Para $\beta_{0}$ se puede proceder de manera an\'aloga para
\begin{itemize}
\item[$H_{0}:$] $\beta_{0}=\beta_{0,0}$,
\item[$H_{1}:$] $\beta_{0}\neq\beta_{0,0}$,
\end{itemize}
con $\hat{\beta}_{0}\sim N\left(\beta_{0},\sigma^{2}\left(\frac{1}{n}+\frac{\overline{x}^{2}}{S_{xx}}\right)\right)$, por lo tanto
\begin{equation}\label{Estadistico.Beta.0}
t_{0}=\frac{\hat{\beta}_{0}-\beta_{0,0}}{MC_{E}\left[\frac{1}{n}+\frac{\overline{x}^{2}}{S_{xx}}\right]},
\end{equation}
con el que rechazamos la hip\'otesis nula si
\begin{equation}\label{Zona.Rechazo.Beta.0}
t_{0}|>t_{\alpha/2,n-2}.
\end{equation}








\begin{itemize}
\item No rechazar $H_{0}:\beta_{1}=0$ es equivalente a decir que no hay relaci\'on lineal entre $x$ y $y$.
\item Alternativamente, si $H_{0}:\beta_{1}=0$ se rechaza, esto implica que $x$ explica la variabilidad de $y$, es decir, podr\'ia significar que la l\'inea recta esel modelo adecuado.
\end{itemize}
El procedimiento de prueba para $H_{0}:\beta_{1}=0$ puede realizarse de la siguiente manera:
\begin{eqnarray*}
S_{yy}=\sum_{k=1}^{n}\left(y_{k}-\overline{y}\right)^{2}=\sum_{k=1}^{n}\left(\hat{y}_{k}-\overline{y}\right)^{2}+\sum_{k=1}^{n}\left(y_{k}-\hat{y}_{k}\right)^{2}
\end{eqnarray*}






\begin{eqnarray*}
S_{yy}&=&\sum_{k=1}^{n}\left(y_{k}-\overline{y}\right)^{2}=\sum_{k=1}^{n}\left(y_{k}-\hat{y}_{k}+\hat{y}_{k}-\overline{y}\right)^{2}\\
&=&\sum_{k=1}^{n}\left[\left(\hat{y}_{k}-\overline{y}\right)+\left(y_{k}-\hat{y}_{k}\right)\right]^{2}\\
&=&\sum_{k=1}^{n}\left[\left(\hat{y}_{k}-\overline{y}\right)^{2}+2\left(\hat{y}_{k}-\overline{y}\right)\left(y_{k}-\hat{y}_{k}\right)+\left(y_{k}-\hat{y}_{k}\right)^{2}\right]\\
&=&\sum_{k=1}^{n}\left(\hat{y}_{k}-\overline{y}\right)^{2}+2\sum_{k=1}^{n}\left(\hat{y}_{k}-\overline{y}\right)\left(y_{k}-\hat{y}_{k}\right)+\sum_{k=1}^{n}\left(y_{k}-\hat{y}_{k}\right)^{2}
\end{eqnarray*}

\begin{eqnarray*}
&&\sum_{k=1}^{n}\left(\hat{y}_{k}-\overline{y}\right)\left(y_{k}-\hat{y}_{k}\right)=\sum_{k=1}^{n}\hat{y}_{k}\left(y_{k}-\hat{y}_{k}\right)-\sum_{k=1}^{n}\overline{y}\left(y_{k}-\hat{y}_{k}\right)\\
&=&\sum_{k=1}^{n}\hat{y}_{k}\left(y_{k}-\hat{y}_{k}\right)-\overline{y}\sum_{k=1}^{n}\left(y_{k}-\hat{y}_{k}\right)\\
&=&\sum_{k=1}^{n}\left(\hat{\beta}_{0}+\hat{\beta}_{1}x_{k}\right)\left(y_{k}-\hat{\beta}_{0}-\hat{\beta}_{1}x_{k}\right)-\overline{y}\sum_{k=1}^{n}\left(y_{k}-\hat{\beta}_{0}-\hat{\beta}_{1}x_{k}\right)
\end{eqnarray*}





\begin{eqnarray*}
&=&\sum_{k=1}^{n}\hat{\beta}_{0}\left(y_{k}-\hat{\beta}_{0}-\hat{\beta}_{1}x_{k}\right)+\sum_{k=1}^{n}\hat{\beta}_{1}x_{k}\left(y_{k}-\hat{\beta}_{0}-\hat{\beta}_{1}x_{k}\right)\\
&-&\overline{y}\sum_{k=1}^{n}\left(y_{k}-\hat{\beta}_{0}-\hat{\beta}_{1}x_{k}\right)\\
&=&\hat{\beta}_{0}\sum_{k=1}^{n}\left(y_{k}-\hat{\beta}_{0}-\hat{\beta}_{1}x_{k}\right)+\hat{\beta}_{1}\sum_{k=1}^{n}x_{k}\left(y_{k}-\hat{\beta}_{0}-\hat{\beta}_{1}x_{k}\right)\\
&-&\overline{y}\sum_{k=1}^{n}\left(y_{k}-\hat{\beta}_{0}-\hat{\beta}_{1}x_{k}\right)=0+0+0=0.
\end{eqnarray*}

Por lo tanto, efectivamente se tiene
\begin{equation}\label{Suma.Total.Cuadrados}
S_{yy}=\sum_{k=1}^{n}\left(\hat{y}_{k}-\overline{y}\right)^{2}+\sum_{k=1}^{n}\left(y_{k}-\hat{y}_{k}\right)^{2},
\end{equation}
donde se hacen las definiciones
\begin{eqnarray}
SC_{E}&=&\sum_{k=1}^{n}\left(\hat{y}_{k}-\overline{y}\right)^{2}\cdots\textrm{Suma de Cuadrados del Error}\\
SC_{R}&=&\sum_{k=1}^{n}\left(y_{k}-\hat{y}_{k}\right)^{2}\cdots\textrm{ Suma de Regresi\'on de Cuadrados}
\end{eqnarray}





Por lo tanto la ecuaci\'on (\ref{Suma.Total.Cuadrados}) se puede reescribir como 
\begin{equation}\label{Suma.Total.Cuadrados.Dos}
S_{yy}=SC_{R}+SC_{E}
\end{equation}
recordemos que $SC_{E}=S_{yy}-\hat{\beta}_{1}S_{xy}$
\begin{eqnarray*}
S_{yy}&=&SC_{R}+\left( S_{yy}-\hat{\beta}_{1}S_{xy}\right)\\
S_{xy}&=&\frac{1}{\hat{\beta}_{1}}SC_{R}
\end{eqnarray*}
$S_{xy}$ tiene $n-1$ grados de libertad y $SC_{R}$ y $SC_{E}$ tienen 1 y $n-2$ grados de libertad respectivamente.

\begin{Prop}
\begin{equation}
E\left(SC_{R}\right)=\sigma^{2}+\beta_{1}S_{xx}
\end{equation}
adem\'as, $SC_{E}$ y $SC_{R}$ son independientes.
\end{Prop}




Recordemos que $\hat{\beta}_{1}=\frac{S_{xy}}{S_{xx}}$. Para $H_{0}:\beta_{1}=0$ verdadera,
\begin{eqnarray*}
F_{0}=\frac{SC_{R}/1}{SC_{E}/(n-2)}=\frac{MC_{R}}{MC_{E}}
\end{eqnarray*}
se distribuye $F_{1,n-2}$, y se rechazar\'ia $H_{0}$ si $F_{0}>F_{\alpha,1,n-2}$.
									
El procedimiento de prueba de hip\'otesis puede presentarse como la tabla de an\'alisis de varianza siguiente\medskip

\begin{tabular}{lcccc}\hline
Fuente de & Suma de  &  Grados de  & Media  & $F_{0}$ \\ 
 variaci\'on & Cuadrados & Libertad & Cuadr\'atica & \\\hline
 Regresi\'on & $SC_{R}$ & 1 & $MC_{R}$  & $MC_{R}/MC_{E}$\\
 Error Residual & $SC_{E}$ & $n-2$ & $MC_{E}$ & \\\hline
 Total & $S_{yy}$ & $n-1$ & & \\\hline
\end{tabular} 





La prueba para la significaci\'on de la regresi\'on puede desarrollarse bas\'andose en la expresi\'on (\ref{Estadistico.Beta.1}), con $\hat{\beta}_{1,0}=0$, es decir
\begin{equation}\label{Estadistico.Beta.1.Cero}
t_{0}=\frac{\hat{\beta}_{1}}{\sqrt{MC_{E}/S_{xx}}}
\end{equation}
Elevando al cuadrado ambos t\'erminos:
\begin{eqnarray*}
t_{0}^{2}=\frac{\hat{\beta}_{1}^{2}S_{xx}}{MC_{E}}=\frac{\hat{\beta}_{1}S_{xy}}{MC_{E}}=\frac{MC_{R}}{MC_{E}}
\end{eqnarray*}
Observar que $t_{0}^{2}=F_{0}$, por tanto la prueba que se utiliza para $t_{0}$ es la misma que para $F_{0}$.





\subsection{Estimaci\'on de Intervalos en RLS}



\begin{itemize}
\item Adem\'as de la estimaci\'on puntual para los par\'ametros $\beta_{1}$ y $\beta_{0}$, es posible obtener estimaciones del intervalo de confianza de estos par\'ametros.

\item El ancho de estos intervalos de confianza es una medida de la calidad total de la recta de regresi\'on.

\end{itemize}

Si los $\epsilon_{k}$ se distribuyen normal e independientemente, entonces
\begin{eqnarray*}
\begin{array}{ccc}
\frac{\left(\hat{\beta}_{1}-\beta_{1}\right)}{\sqrt{\frac{MC_{E}}{S_{xx}}}}&y &\frac{\left(\hat{\beta}_{0}-\beta_{0}\right)}{\sqrt{MC_{E}\left(\frac{1}{n}+\frac{\overline{x}^{2}}{S_{xx}}\right)}}
\end{array}
\end{eqnarray*}
se distribuyen $t$ con $n-2$ grados de libertad. Por tanto un intervalo de confianza de $100\left(1-\alpha\right)\%$ para $\beta_{1}$ est\'a dado por




\begin{eqnarray}
\hat{\beta}_{1}-t_{\alpha/2,n-2}\sqrt{\frac{MC_{E}}{S_{xx}}}\leq \beta_{1}\leq\hat{\beta}_{1}+t_{\alpha/2,n-2}\sqrt{\frac{MC_{E}}{S_{xx}}}.
\end{eqnarray}
De igual manera, para $\beta_{0}$ un intervalo de confianza al $100\left(1-\alpha\right)\%$ es
\small{
\begin{eqnarray}
\begin{array}{l}
\hat{\beta}_{0}-t_{\alpha/2,n-2}\sqrt{MC_{E}\left(\frac{1}{n}+\frac{\overline{x}^{2}}{S_{xx}}\right)}\leq\beta_{0}\leq\hat{\beta}_{0}+t_{\alpha/2,n-2}\sqrt{MC_{E}\left(\frac{1}{n}+\frac{\overline{x}^{2}}{S_{xx}}\right)}
\end{array}
\end{eqnarray}}


%\end{document}
\subsection{Predicci\'on}


Supongamos que se tiene un valor $x_{0}$ de inter\'es, entonces la estimaci\'on puntual de este nuevo valor
\begin{equation}
\hat{y}_{0}=\hat{\beta}_{0}+\hat{\beta}_{1}x_{0}
\end{equation}
\begin{Note}
Esta nueva observaci\'on es independiente de las utilizadas para obtener el modelo de regresi\'on, por tanto, el intervalo en torno a la recta de regresi\'on es inapropiado, puesto que se basa \'unicamente en los datos empleados para ajustar el modelo de regresi\'on.\\

El intervalo de confianza en torno a la recta de regresi\'on se refiere a la respuesta media verdadera $x=x_{0}$, no a observaciones futuras.
\end{Note}






Sea $y_{0}$ la observaci\'on futura en $x=x_{0}$, y sea $\hat{y}_{0}$ dada en la ecuaci\'on anterior, el estimador de $y_{0}$. Si se define la variable aleatoria $$w=y_{0}-\hat{y}_{0},$$ esta se distribuye normalmente con media cero y varianza $$V\left(w\right)=\sigma^{2}\left[1+\frac{1}{n}+\frac{\left(x-x_{0}\right)^2}{S_{xx}}\right]$$
dado que $y_{0}$ es independiente de $\hat{y}_{0}$, por lo tanto el intervalo de predicci\'on al nivel $\alpha$ para futuras observaciones $x_{0}$ es


\begin{eqnarray*}
\hat{y}_{0}-t_{\alpha/2,n-2}\sqrt{MC_{E}\left[1+\frac{1}{n}+\frac{\left(x-x_{0}\right)^2}{S_{xx}}\right]}\leq y_{0}\\
\leq \hat{y}_{0}+t_{\alpha/2,n-2}\sqrt{MC_{E}\left[1+\frac{1}{n}+\frac{\left(x-x_{0}\right)^2}{S_{xx}}\right]}.
\end{eqnarray*}





\subsection{Prueba de falta de ajuste}
%\frametitle{Falta de ajuste}
Es com\'un encontrar que el modelo ajustado no satisface totalmente el modelo necesario para los datos, en este caso es preciso saber qu\'e tan bueno es el modelo propuesto. Para esto se propone la siguiente prueba de hip\'otesis:
\begin{itemize}
\item[$H_{0}:$ ]El modelo propuesto se ajusta adecuademente a los datos.
\item[$H_{1}:$ ]El modelo NO se ajusta a los datos.
\end{itemize}
La prueba implica dividir la suma de cuadrados del eror o del residuo en las siguientes dos componentes:
\begin{eqnarray*}
SC_{E}=SC_{EP}+SC_{FDA}
\end{eqnarray*}



%\frametitle{Falta de ajuste}
donde $SC_{EP}$ es la suma de cuadrados atribuibles al error puro, y $SC_{FDA}$ es la suma de cuadrados atribuible a la falta de ajuste del modelo.


%%\frametitle{Falta de ajuste}
%


\subsection{Coeficiente de Determinaci\'on}



La cantidad
\begin{equation}
R^{2}=\frac{SC_{R}}{S_{yy}}=1-\frac{SC_{E}}{S_{yy}}
\end{equation}
se denomina coeficiente de determinaci\'on y se utiliza para saber si el modelo de regresi\'on es suficiente o no. Se puede demostrar que $0\leq R^{2}\leq1$, una manera de interpretar este valor es que si $R^{2}=k$, entonces el modelo de regresi\'on explica el $k*100\%$ de la variabilidad en los datos.
$R^{2}$ 
\begin{itemize}
\item No mide la magnitud de la pendiente de la recta de regresi\'on
\item Un valor grande de $R^{2}$ no implica una pendiente empinada.
\item No mide la suficiencia del modelo.
\item Valores grandes de $R^{2}$ no implican necesariamente que el modelo de regresi\'on proporcionar\'a predicciones precisas para futuras observaciones.
\end{itemize}





\part{PRIMERA PARTE: Regresi\'on Log\'istica}


\chapter{D\'ia 1: Introducci\'on}

\section{Conceptos Básicos}

La regresión logística es una técnica de modelado estadístico utilizada para predecir la probabilidad de un evento binario (es decir, un evento que tiene dos posibles resultados) en función de una o más variables independientes. Es ampliamente utilizada en diversas disciplinas, como medicina, economía, biología, y ciencias sociales, para analizar y predecir resultados binarios.

Un modelo de regresión logística tiene la forma de una ecuación que describe cómo una variable dependiente binaria $Y$ (que puede tomar los valores $0$ o $1$) está relacionada con una o más variables independientes $X_1, X_2, \ldots, X_n$. A diferencia de la regresión lineal, que predice un valor continuo, la regresión logística predice una probabilidad que puede ser interpretada como la probabilidad de que $Y=1$ dado un conjunto de valores para $X_1, X_2, \ldots, X_n$.

\section{Regresión Lineal}

La regresión lineal es una técnica de modelado estadístico utilizada para predecir el valor de una variable dependiente continua en función de una o más variables independientes.

\subsection*{Modelo Lineal}

El modelo de regresión lineal tiene la forma:
\begin{equation}
Y = \beta_0 + \beta_1 X_1 + \beta_2 X_2 + \ldots + \beta_n X_n + \epsilon
\end{equation}
donde:
\begin{itemize}
    \item $Y$ es la variable dependiente.
    \item $\beta_0$ es la intersección con el eje $Y$ o término constante.
    \item $\beta_1, \beta_2, \ldots, \beta_n$ son los coeficientes que representan la relación entre las variables independientes y la variable dependiente.
    \item $X_1, X_2, \ldots, X_n$ son las variables independientes.
    \item $\epsilon$ es el término de error, que representa la desviación de los datos observados de los valores predichos por el modelo.
\end{itemize}

\subsection*{Mínimos Cuadrados Ordinarios (OLS)}

El objetivo de la regresión lineal es encontrar los valores de los coeficientes $\beta_0, \beta_1, \ldots, \beta_n$ que minimicen la suma de los cuadrados de las diferencias entre los valores observados y los valores predichos. Este método se conoce como mínimos cuadrados ordinarios (OLS, por sus siglas en inglés).

La función de costo que se minimiza es:
\begin{equation}
J\left(\beta_0, \beta_1, \ldots, \beta_n\right) = \sum_{i=1}^{n}\left(y_i - \hat{y}_i\right)^2
\end{equation}
donde:
\begin{itemize}
    \item $y_i$ es el valor observado de la variable dependiente para la $i$-ésima observación.
    \item $\hat{y}_i$ es el valor predicho por el modelo para la $i$-ésima observación, dado por:
    \begin{equation}
    \hat{y}_i = \beta_0 + \beta_1 x_{i1} + \beta_2 x_{i2} + \ldots + \beta_n x_{in}
    \end{equation}
\end{itemize}

Para encontrar los valores óptimos de los coeficientes, se toman las derivadas parciales de la función de costo con respecto a cada coeficiente y se igualan a cero:
\begin{equation}
\frac{\partial J}{\partial \beta_j} = 0 \quad \text{para } j = 0, 1, \ldots, n
\end{equation}

Resolviendo este sistema de ecuaciones, se obtienen los valores de los coeficientes que minimizan la función de costo.

\section{Regresión Logística}

La deducción de la fórmula de la regresión logística comienza con la necesidad de modelar la probabilidad de un evento binario. Queremos encontrar una función que relacione las variables independientes con la probabilidad de que la variable dependiente tome el valor $1$.

\subsection*{Probabilidad y Odds}

La probabilidad de que el evento ocurra, $P(Y=1)$, se denota como $p$. La probabilidad de que el evento no ocurra, $P(Y=0)$, es $1-p$. Los \textit{odds} (chances) de que ocurra el evento se definen como:
\begin{equation}
\text{odds} = \frac{p}{1-p}
\end{equation}
Los \textit{odds} nos indican cuántas veces más probable es que ocurra el evento frente a que no ocurra.

\subsection*{Transformación Logit}

Para simplificar el modelado de los \textit{odds}, aplicamos el logaritmo natural, obteniendo la función logit:
\begin{equation}
\text{logit}(p) = \log\left(\frac{p}{1-p}\right)
\end{equation}
La transformación logit es útil porque convierte el rango de la probabilidad (0, 1) al rango de números reales $\left(-\infty, \infty\right)$.

\subsection*{Modelo Lineal en el Espacio Logit}

La idea clave de la regresión logística es modelar la transformación logit de la probabilidad como una combinación lineal de las variables independientes:
\begin{equation}
\text{logit}(p) = \log\left(\frac{p}{1-p}\right) = \beta_0 + \beta_1 X_1 + \beta_2 X_2 + \ldots + \beta_n X_n
\end{equation}
Aquí, $\beta_0$ es el intercepto y $\beta_1, \beta_2, \ldots, \beta_n$ son los coeficientes asociados con las variables independientes $X_1, X_2, \ldots, X_n$.

\subsection*{Invertir la Transformación Logit}

Para expresar $p$ en función de una combinación lineal de las variables independientes, invertimos la transformación logit. Partimos de la ecuación:
\begin{equation}
\log\left(\frac{p}{1-p}\right) = \beta_0 + \beta_1 X_1 + \beta_2 X_2 + \ldots + \beta_n X_n
\end{equation}
Aplicamos la exponenciación a ambos lados:
\begin{equation}
\frac{p}{1-p} = e^{\beta_0 + \beta_1 X_1 + \beta_2 X_2 + \ldots + \beta_n X_n}
\end{equation}
Despejamos $p$:
\begin{equation}
p = \frac{e^{\beta_0 + \beta_1 X_1 + \beta_2 X_2 + \ldots + \beta_n X_n}}{1 + e^{\beta_0 + \beta_1 X_1 + \beta_2 X_2 + \ldots + \beta_n X_n}}
\end{equation}

\subsection*{Función Logística}

La expresión final que obtenemos es conocida como la función logística:
\begin{equation}
p = \frac{1}{1 + e^{-\left(\beta_0 + \beta_1 X_1 + \beta_2 X_2 + \ldots + \beta_n X_n\right)}}
\end{equation}
Esta función describe cómo las variables independientes se relacionan con la probabilidad de que el evento de interés ocurra. Los coeficientes $\beta_0, \beta_1, \ldots, \beta_n$ se estiman a partir de los datos utilizando el método de máxima verosimilitud.

\section{Método de Máxima Verosimilitud}

En la regresión logística, los coeficientes del modelo se estiman utilizando el método de máxima verosimilitud. Este método busca encontrar los valores de los coeficientes que maximicen la probabilidad de observar los datos dados los valores de las variables independientes.

\subsection*{Función de Verosimilitud}

Para un conjunto de $n$ observaciones, la función de verosimilitud $L$ se define como el producto de las probabilidades individuales de observar cada dato:
\begin{equation}
L(\beta_0, \beta_1, \ldots, \beta_n) = \prod_{i=1}^{n} p_i^{y_i} (1 - p_i)^{1 - y_i}
\end{equation}
donde $y_i$ es el valor observado de la variable dependiente para la $i$-ésima observación y $p_i$ es la probabilidad predicha de que $Y_i = 1$. Aquí, $p_i$ es dado por la función logística:
\begin{equation}
p_i = \frac{1}{1 + e^{-(\beta_0 + \beta_1 X_{i1} + \beta_2 X_{i2} + \ldots + \beta_n X_{in})}}
\end{equation}

\subsection*{Función de Log-Verosimilitud}

Para simplificar los cálculos, trabajamos con el logaritmo de la función de verosimilitud, conocido como la función de log-verosimilitud. Tomar el logaritmo convierte el producto en una suma:
\begin{equation}
\log L(\beta_0, \beta_1, \ldots, \beta_n) = \sum_{i=1}^{n} \left[ y_i \log(p_i) + (1 - y_i) \log(1 - p_i) \right]
\end{equation}

Sustituyendo $p_i$:
\begin{equation}
\log L(\beta_0, \beta_1, \ldots, \beta_n) = \sum_{i=1}^{n} \left[ y_i (\beta_0 + \beta_1 X_{i1} + \beta_2 X_{i2} + \ldots + \beta_n X_{in}) - \log(1 + e^{\beta_0 + \beta_1 X_{i1} + \beta_2 X_{i2} + \ldots + \beta_n X_{in}}) \right]
\end{equation}

\subsection*{Maximización de la Log-Verosimilitud}

El objetivo es encontrar los valores de $\beta_0, \beta_1, \ldots, \beta_n$ que maximicen la función de log-verosimilitud. Esto se hace derivando la función de log-verosimilitud con respecto a cada uno de los coeficientes y encontrando los puntos críticos.

Para $\beta_j$, la derivada parcial de la función de log-verosimilitud es:
\begin{equation}
\frac{\partial \log L}{\partial \beta_j} = \sum_{i=1}^{n} \left[ y_i X_{ij} - \frac{X_{ij} e^{\beta_0 + \beta_1 X_{i1} + \beta_2 X_{i2} + \ldots + \beta_n X_{in}}}{1 + e^{\beta_0 + \beta_1 X_{i1} + \beta_2 X_{i2} + \ldots + \beta_n X_{in}}} \right]
\end{equation}

Esto se simplifica a:
\begin{equation}
\frac{\partial \log L}{\partial \beta_j} = \sum_{i=1}^{n} X_{ij} (y_i - p_i)
\end{equation}

Para maximizar la log-verosimilitud, resolvemos el sistema de ecuaciones $\frac{\partial \log L}{\partial \beta_j} = 0$ para todos los $j$ de 0 a $n$. Este sistema de ecuaciones no tiene una solución analítica cerrada, por lo que se resuelve numéricamente utilizando métodos iterativos como el algoritmo de Newton-Raphson.

\subsection*{Método de Newton-Raphson}

El método de Newton-Raphson es un algoritmo iterativo que se utiliza para encontrar las raíces de una función. En el contexto de la regresión logística, se utiliza para maximizar la función de log-verosimilitud encontrando los valores de los coeficientes $\beta_0, \beta_1, \ldots, \beta_n$. El método de Newton-Raphson se basa en una aproximación de segundo orden de la función objetivo. Dado un valor inicial de los coeficientes $\beta^{(0)}$, se iterativamente actualiza el valor de los coeficientes utilizando la fórmula:
\begin{equation}
\beta^{(k+1)} = \beta^{(k)} - \left[ \mathbf{H}(\beta^{(k)}) \right]^{-1} \mathbf{g}(\beta^{(k)})
\end{equation}
donde:
\begin{itemize}
    \item $\beta^{(k)}$ es el vector de coeficientes en la $k$-ésima iteración.
    \item $\mathbf{H}(\beta^{(k)})$ es la matriz Hessiana (matriz de segundas derivadas) evaluada en $\beta^{(k)}$.
    \item $\mathbf{g}(\beta^{(k)})$ es el gradiente (vector de primeras derivadas) evaluado en $\beta^{(k)}$.
\end{itemize}

\subsubsection*{Gradiente}

El gradiente de la función de log-verosimilitud con respecto a los coeficientes $\beta$ es:
\begin{equation}
\mathbf{g}(\beta) = \frac{\partial \log L}{\partial \beta} = \sum_{i=1}^{n} \mathbf{X}_i (y_i - p_i)
\end{equation}
donde $\mathbf{X}_i$ es el vector de valores de las variables independientes para la $i$-ésima observación.

\subsubsection*{Hessiana}

La matriz Hessiana de la función de log-verosimilitud con respecto a los coeficientes $\beta$ es:
\begin{equation}
\mathbf{H}(\beta) = \frac{\partial^2 \log L}{\partial \beta \partial \beta^T} = -\sum_{i=1}^{n} p_i (1 - p_i) \mathbf{X}_i \mathbf{X}_i^T
\end{equation}

\subsubsection*{Algoritmo Newton-Raphson}

El algoritmo Newton-Raphson para la regresión logística se puede resumir en los siguientes pasos:
\begin{Algthm}
\begin{enumerate}
    \item Inicializar el vector de coeficientes $\beta^{(0)}$ (por ejemplo, con ceros o valores pequeños aleatorios).
    \item Calcular el gradiente $\mathbf{g}(\beta^{(k)})$ y la matriz Hessiana $\mathbf{H}(\beta^{(k)})$ en la iteración $k$.
    \item Actualizar los coeficientes utilizando la fórmula:
    \begin{equation}
    \beta^{(k+1)} = \beta^{(k)} - \left[ \mathbf{H}(\beta^{(k)}) \right]^{-1} \mathbf{g}(\beta^{(k)})
    \end{equation}
    \item Repetir los pasos 2 y 3 hasta que la diferencia entre $\beta^{(k+1)}$ y $\beta^{(k)}$ sea menor que un umbral predefinido (criterio de convergencia).
\end{enumerate}
\end{Algthm}

\section{Representaci\'on vectorial}

En el contexto de la regresión logística, los vectores $X_1, X_2, \ldots, X_n$ representan las variables independientes. Cada $X_j$ es un vector columna que contiene los valores de la variable independiente $j$ para cada una de las $n$ observaciones. Es decir,

\begin{equation}
X_j = \begin{bmatrix}
x_{1j} \\
x_{2j} \\
\vdots \\
x_{nj}
\end{bmatrix}
\end{equation}

Para simplificar la notación y los cálculos, a menudo combinamos todos los vectores de variables independientes en una única matriz de diseño $\mathbf{X}$ de tamaño $n \times (k+1)$, donde $n$ es el número de observaciones y $k+1$ es el número de variables independientes más el término de intercepto. La primera columna de $\mathbf{X}$ corresponde a un vector de unos para el término de intercepto, y las demás columnas corresponden a los valores de las variables independientes:

\begin{equation}
\mathbf{X} = \begin{bmatrix}
1 & x_{11} & x_{12} & \ldots & x_{1k} \\
1 & x_{21} & x_{22} & \ldots & x_{2k} \\
\vdots & \vdots & \vdots & \ddots & \vdots \\
1 & x_{n1} & x_{n2} & \ldots & x_{nk}
\end{bmatrix}
\end{equation}

De esta forma, el modelo logit puede ser escrito de manera compacta utilizando la notación matricial:

\begin{equation}
\text{logit}(p) = \log\left(\frac{p}{1-p}\right) = \mathbf{X} \boldsymbol{\beta}
\end{equation}

donde $\boldsymbol{\beta}$ es el vector de coeficientes:

\begin{equation}
\boldsymbol{\beta} = \begin{bmatrix}
\beta_0 \\
\beta_1 \\
\beta_2 \\
\vdots \\
\beta_k
\end{bmatrix}
\end{equation}

Así, la probabilidad $p$ se puede expresar como:

\begin{equation}
p = \frac{1}{1 + e^{-\mathbf{X} \boldsymbol{\beta}}}
\end{equation}

Esta notación matricial simplifica la implementación y la derivación de los estimadores de los coeficientes en la regresión logística.

\subsection*{Estimación de los Coeficientes}

Para estimar los coeficientes $\boldsymbol{\beta}$ en la regresión logística, se utiliza el método de máxima verosimilitud. La función de verosimilitud $L(\boldsymbol{\beta})$ se define como el producto de las probabilidades de las observaciones dadas las variables independientes:

\begin{equation}
L(\boldsymbol{\beta}) = \prod_{i=1}^{n} p_i^{y_i} (1 - p_i)^{1 - y_i}
\end{equation}

donde $y_i$ es el valor observado de la variable dependiente para la $i$-ésima observación, y $p_i$ es la probabilidad predicha de que $Y_i = 1$.

La función de log-verosimilitud, que es más fácil de maximizar, se obtiene tomando el logaritmo natural de la función de verosimilitud:

\begin{equation}
\log L(\boldsymbol{\beta}) = \sum_{i=1}^{n} \left[ y_i \log(p_i) + (1 - y_i) \log(1 - p_i) \right]
\end{equation}

Sustituyendo $p_i = \frac{1}{1 + e^{-\mathbf{X}_i \boldsymbol{\beta}}}$, donde $\mathbf{X}_i$ es la $i$-ésima fila de la matriz de diseño $\mathbf{X}$, obtenemos:

\begin{equation}
\log L(\boldsymbol{\beta}) = \sum_{i=1}^{n} \left[ y_i (\mathbf{X}_i \boldsymbol{\beta}) - \log(1 + e^{\mathbf{X}_i \boldsymbol{\beta}}) \right]
\end{equation}

Para encontrar los valores de $\boldsymbol{\beta}$ que maximizan la función de log-verosimilitud, se utiliza un algoritmo iterativo como el método de Newton-Raphson. Este método requiere calcular el gradiente y la matriz Hessiana de la función de log-verosimilitud.

\subsubsection*{Gradiente}

El gradiente de la función de log-verosimilitud con respecto a $\boldsymbol{\beta}$ es:

\begin{equation}
\nabla \log L(\boldsymbol{\beta}) = \mathbf{X}^T (\mathbf{y} - \mathbf{p})
\end{equation}

donde $\mathbf{y}$ es el vector de valores observados y $\mathbf{p}$ es el vector de probabilidades predichas.

\subsubsection*{Matriz Hessiana}

La matriz Hessiana de la función de log-verosimilitud es:

\begin{equation}
\mathbf{H}(\boldsymbol{\beta}) = -\mathbf{X}^T \mathbf{W} \mathbf{X}
\end{equation}

donde $\mathbf{W}$ es una matriz diagonal de pesos con elementos $w_i = p_i (1 - p_i)$.

\subsubsection*{Método de Newton-Raphson}

El método de Newton-Raphson actualiza los coeficientes $\boldsymbol{\beta}$ de la siguiente manera:

\begin{equation}
\boldsymbol{\beta}^{(t+1)} = \boldsymbol{\beta}^{(t)} - [\mathbf{H}(\boldsymbol{\beta}^{(t)})]^{-1} \nabla \log L(\boldsymbol{\beta}^{(t)})
\end{equation}

Iterando este proceso hasta que la diferencia entre $\boldsymbol{\beta}^{(t+1)}$ y $\boldsymbol{\beta}^{(t)}$ sea menor que un umbral predefinido, se obtienen los estimadores de máxima verosimilitud para los coeficientes de la regresión logística.



\chapter{Elementos de Probabilidad}
\section{Introducci\'on}

Los fundamentos de probabilidad y estad\'istica son esenciales para comprender y aplicar t\'ecnicas de an\'alisis de datos y modelado estad\'istico, incluyendo la regresi\'on lineal y log\'istica. Este cap\'itulo proporciona una revisi\'on de los conceptos clave en probabilidad y estad\'istica que son relevantes para estos m\'etodos.

\section{Probabilidad}

La probabilidad es una medida de la incertidumbre o el grado de creencia en la ocurrencia de un evento. Los conceptos fundamentales incluyen:

\subsection{Espacio Muestral y Eventos}

El espacio muestral, denotado como $S$, es el conjunto de todos los posibles resultados de un experimento aleatorio. Un evento es un subconjunto del espacio muestral. Por ejemplo, si lanzamos un dado, el espacio muestral es:
\begin{eqnarray*}
S = \{1, 2, 3, 4, 5, 6\}
\end{eqnarray*}
Un evento podr\'ia ser obtener un n\'umero par:
\begin{eqnarray*}
E = \{2, 4, 6\}
\end{eqnarray*}

\subsection{Definiciones de Probabilidad}

Existen varias definiciones de probabilidad, incluyendo la probabilidad cl\'asica, la probabilidad frecuentista y la probabilidad bayesiana.

\subsubsection{Probabilidad Cl\'asica}

La probabilidad cl\'asica se define como el n\'umero de resultados favorables dividido por el n\'umero total de resultados posibles:
\begin{eqnarray*}
P(E) = \frac{|E|}{|S|}
\end{eqnarray*}
donde $|E|$ es el n\'umero de elementos en el evento $E$ y $|S|$ es el n\'umero de elementos en el espacio muestral $S$.

\subsubsection{Probabilidad Frecuentista}

La probabilidad frecuentista se basa en la frecuencia relativa de ocurrencia de un evento en un gran n\'umero de repeticiones del experimento:
\begin{eqnarray*}
P(E) = \lim_{n \to \infty} \frac{n_E}{n}
\end{eqnarray*}
donde $n_E$ es el n\'umero de veces que ocurre el evento $E$ y $n$ es el n\'umero total de repeticiones del experimento.

\subsubsection{Probabilidad Bayesiana}

La probabilidad bayesiana se interpreta como un grado de creencia actualizado a medida que se dispone de nueva informaci\'on. Se basa en el teorema de Bayes:
\begin{eqnarray*}
P(A|B) = \frac{P(B|A)P(A)}{P(B)}
\end{eqnarray*}
donde $P(A|B)$ es la probabilidad de $A$ dado $B$, $P(B|A)$ es la probabilidad de $B$ dado $A$, $P(A)$ y $P(B)$ son las probabilidades de $A$ y $B$ respectivamente.

\section{Estad\'istica Bayesiana}

La estad\'istica bayesiana proporciona un enfoque coherente para el an\'alisis de datos basado en el teorema de Bayes. Los conceptos fundamentales incluyen:

\subsection{Prior y Posterior}

\subsubsection{Distribuci\'on Prior}

La distribuci\'on prior (apriori) representa nuestra creencia sobre los par\'ametros antes de observar los datos. Es una distribuci\'on de probabilidad que refleja nuestra incertidumbre inicial sobre los par\'ametros. Por ejemplo, si creemos que un par\'ametro $\theta$ sigue una distribuci\'on normal con media $\mu_0$ y varianza $\sigma_0^2$, nuestra prior ser\'ia:
\begin{eqnarray*}
P(\theta) = \frac{1}{\sqrt{2\pi\sigma_0^2}} e^{-\frac{(\theta-\mu_0)^2}{2\sigma_0^2}}
\end{eqnarray*}

\subsubsection{Verosimilitud}

La verosimilitud (likelihood) es la probabilidad de observar los datos dados los par\'ametros. Es una funci\'on de los par\'ametros $\theta$ dada una muestra de datos $X$:
\begin{eqnarray*}
L(\theta; X) = P(X|\theta)
\end{eqnarray*}
donde $X$ son los datos observados y $\theta$ son los par\'ametros del modelo.

\subsubsection{Distribuci\'on Posterior}

La distribuci\'on posterior (a posteriori) combina la informaci\'on de la prior y la verosimilitud utilizando el teorema de Bayes. Representa nuestra creencia sobre los par\'ametros despu\'es de observar los datos:
\begin{eqnarray*}
P(\theta|X) = \frac{P(X|\theta)P(\theta)}{P(X)}
\end{eqnarray*}
donde $P(\theta|X)$ es la distribuci\'on posterior, $P(X|\theta)$ es la verosimilitud, $P(\theta)$ es la prior y $P(X)$ es la probabilidad marginal de los datos.

La probabilidad marginal de los datos $P(X)$ se puede calcular como:
\begin{eqnarray*}
P(X) = \int_{\Theta} P(X|\theta)P(\theta) d\theta
\end{eqnarray*}
donde $\Theta$ es el espacio de todos los posibles valores del par\'ametro $\theta$.

\section{Distribuciones de Probabilidad}

Las distribuciones de probabilidad describen c\'omo se distribuyen los valores de una variable aleatoria. Existen distribuciones de probabilidad discretas y continuas.

\subsection{Distribuciones Discretas}

Una variable aleatoria discreta toma un n\'umero finito o contable de valores. Algunas distribuciones discretas comunes incluyen:

\subsubsection{Distribuci\'on Binomial}

La distribuci\'on binomial describe el n\'umero de \'exitos en una serie de ensayos de Bernoulli independientes y con la misma probabilidad de \'exito. La funci\'on de probabilidad es:
\begin{eqnarray*}
P(X = k) = \binom{n}{k} p^k (1-p)^{n-k}
\end{eqnarray*}
donde $X$ es el n\'umero de \'exitos, $n$ es el n\'umero de ensayos, $p$ es la probabilidad de \'exito en cada ensayo, y $\binom{n}{k}$ es el coeficiente binomial.

La funci\'on generadora de momentos (MGF) para la distribuci\'on binomial es:
\begin{eqnarray*}
M_X(t) = \left( 1 - p + pe^t \right)^n
\end{eqnarray*}

El valor esperado y la varianza de una variable aleatoria binomial son:
\begin{eqnarray*}
E(X) &=& np \\
\text{Var}(X) &=& np(1-p)
\end{eqnarray*}

\subsubsection{Distribuci\'on de Poisson}

La distribuci\'on de Poisson describe el n\'umero de eventos que ocurren en un intervalo de tiempo fijo o en un \'area fija. La funci\'on de probabilidad es:
\begin{eqnarray*}
P(X = k) = \frac{\lambda^k e^{-\lambda}}{k!}
\end{eqnarray*}
donde $X$ es el n\'umero de eventos, $\lambda$ es la tasa media de eventos por intervalo, y $k$ es el n\'umero de eventos observados.

La funci\'on generadora de momentos (MGF) para la distribuci\'on de Poisson es:
\begin{eqnarray*}
M_X(t) = e^{\lambda (e^t - 1)}
\end{eqnarray*}

El valor esperado y la varianza de una variable aleatoria de Poisson son:
\begin{eqnarray*}
E(X) &=& \lambda \\
\text{Var}(X) &=& \lambda
\end{eqnarray*}

\subsection{Distribuciones Continuas}

Una variable aleatoria continua toma un n\'umero infinito de valores en un intervalo continuo. Algunas distribuciones continuas comunes incluyen:

\subsubsection{Distribuci\'on Normal}

La distribuci\'on normal, tambi\'en conocida como distribuci\'on gaussiana, es una de las distribuciones m\'as importantes en estad\'istica. La funci\'on de densidad de probabilidad es:
\begin{eqnarray*}
f(x) = \frac{1}{\sqrt{2\pi\sigma^2}} e^{-\frac{(x-\mu)^2}{2\sigma^2}}
\end{eqnarray*}
donde $x$ es un valor de la variable aleatoria, $\mu$ es la media, y $\sigma$ es la desviaci\'on est\'andar.

La funci\'on generadora de momentos (MGF) para la distribuci\'on normal es:
\begin{eqnarray*}
M_X(t) = e^{\mu t + \frac{1}{2} \sigma^2 t^2}
\end{eqnarray*}

El valor esperado y la varianza de una variable aleatoria normal son:
\begin{eqnarray*}
E(X) &=& \mu \\
\text{Var}(X) &=& \sigma^2
\end{eqnarray*}

\subsubsection{Distribuci\'on Exponencial}

La distribuci\'on exponencial describe el tiempo entre eventos en un proceso de Poisson. La funci\'on de densidad de probabilidad es:
\begin{eqnarray*}
f(x) = \lambda e^{-\lambda x}
\end{eqnarray*}
donde $x$ es el tiempo entre eventos y $\lambda$ es la tasa media de eventos.

La funci\'on generadora de momentos (MGF) para la distribuci\'on exponencial es:
\begin{eqnarray*}
M_X(t) = \frac{\lambda}{\lambda - t}, \quad \text{para } t < \lambda
\end{eqnarray*}

El valor esperado y la varianza de una variable aleatoria exponencial son:
\begin{eqnarray*}
E(X) &=& \frac{1}{\lambda} \\
\text{Var}(X) &=& \frac{1}{\lambda^2}
\end{eqnarray*}

\section{Estad\'istica Descriptiva}

La estad\'istica descriptiva resume y describe las caracter\'isticas de un conjunto de datos. Incluye medidas de tendencia central, medidas de dispersi\'on y medidas de forma.

\subsection{Medidas de Tendencia Central}

Las medidas de tendencia central incluyen la media, la mediana y la moda.

\subsubsection{Media}

La media aritm\'etica es la suma de los valores dividida por el n\'umero de valores:
\begin{eqnarray*}
\bar{x} = \frac{1}{n} \sum_{i=1}^{n} x_i
\end{eqnarray*}
donde $x_i$ son los valores de la muestra y $n$ es el tama\~no de la muestra.

\subsubsection{Mediana}

La mediana es el valor medio cuando los datos est\'an ordenados. Si el n\'umero de valores es impar, la mediana es el valor central. Si es par, es el promedio de los dos valores centrales.

\subsubsection{Moda}

La moda es el valor que ocurre con mayor frecuencia en un conjunto de datos.

\subsection{Medidas de Dispersi\'on}

Las medidas de dispersi\'on incluyen el rango, la varianza y la desviaci\'on est\'andar.

\subsubsection{Rango}

El rango es la diferencia entre el valor m\'aximo y el valor m\'inimo de los datos:
\begin{eqnarray*}
Rango = x_{\text{max}} - x_{\text{min}}
\end{eqnarray*}

\subsubsection{Varianza}

La varianza es la media de los cuadrados de las diferencias entre los valores y la media:
\begin{eqnarray*}
\sigma^2 = \frac{1}{n} \sum_{i=1}^{n} (x_i - \bar{x})^2
\end{eqnarray*}

\subsubsection{Desviaci\'on Est\'andar}

La desviaci\'on est\'andar es la ra\'iz cuadrada de la varianza:
\begin{eqnarray*}
\sigma = \sqrt{\frac{1}{n} \sum_{i=1}^{n} (x_i - \bar{x})^2}
\end{eqnarray*}

\section{Inferencia Estad\'istica}

La inferencia estad\'istica es el proceso de sacar conclusiones sobre una poblaci\'on a partir de una muestra. Incluye la estimaci\'on de par\'ametros y la prueba de hip\'otesis.

\subsection{Estimaci\'on de Par\'ametros}

La estimaci\'on de par\'ametros implica el uso de datos muestrales para estimar los par\'ametros de una poblaci\'on.

\subsubsection{Estimador Puntual}

Un estimador puntual proporciona un \'unico valor como estimaci\'on de un par\'ametro de la poblaci\'on. Por ejemplo, la media muestral $\bar{x}$ es un estimador puntual de la media poblacional $\mu$. Otros ejemplos de estimadores puntuales son:

\begin{itemize}
    \item \textbf{Mediana muestral ($\tilde{x}$)}: Estimador de la mediana poblacional.
    \item \textbf{Varianza muestral ($s^2$)}: Estimador de la varianza poblacional $\sigma^2$, definido como:
    \begin{eqnarray*}
    s^2 = \frac{1}{n-1} \sum_{i=1}^{n} (x_i - \bar{x})^2
    \end{eqnarray*}
    \item \textbf{Desviaci\'on est\'andar muestral ($s$)}: Estimador de la desviaci\'on est\'andar poblacional $\sigma$, definido como:
    \begin{eqnarray*}
    s = \sqrt{s^2}
    \end{eqnarray*}
\end{itemize}

\subsubsection{Propiedades de los Estimadores Puntuales}

Los estimadores puntuales deben cumplir ciertas propiedades deseables, como:

\begin{itemize}
    \item \textbf{Insesgadez}: Un estimador es insesgado si su valor esperado es igual al valor del par\'ametro que estima.
    \begin{eqnarray*}
    E(\hat{\theta}) = \theta
    \end{eqnarray*}
    \item \textbf{Consistencia}: Un estimador es consistente si converge en probabilidad al valor del par\'ametro a medida que el tama\~no de la muestra tiende a infinito.
    \item \textbf{Eficiencia}: Un estimador es eficiente si tiene la varianza m\'as baja entre todos los estimadores insesgados.
\end{itemize}

\subsubsection{Estimador por Intervalo}

Un estimador por intervalo proporciona un rango de valores dentro del cual se espera que se encuentre el par\'ametro poblacional con un cierto nivel de confianza. Por ejemplo, un intervalo de confianza para la media es:
\begin{eqnarray*}
\left( \bar{x} - z \frac{\sigma}{\sqrt{n}}, \bar{x} + z \frac{\sigma}{\sqrt{n}} \right)
\end{eqnarray*}
donde $z$ es el valor cr\'itico correspondiente al nivel de confianza deseado, $\sigma$ es la desviaci\'on est\'andar poblacional y $n$ es el tama\~no de la muestra.

\subsection{Prueba de Hip\'otesis}

La prueba de hip\'otesis es un procedimiento para decidir si una afirmaci\'on sobre un par\'ametro poblacional es consistente con los datos muestrales.

\subsubsection{Hip\'otesis Nula y Alternativa}

La hip\'otesis nula ($H_0$) es la afirmaci\'on que se somete a prueba, y la hip\'otesis alternativa ($H_a$) es la afirmaci\'on que se acepta si se rechaza la hip\'otesis nula.

\subsubsection{Nivel de Significancia}

El nivel de significancia ($\alpha$) es la probabilidad de rechazar la hip\'otesis nula cuando es verdadera. Un valor com\'unmente utilizado es $\alpha = 0.05$.

\subsubsection{Estad\'istico de Prueba}

El estad\'istico de prueba es una medida calculada a partir de los datos muestrales que se utiliza para decidir si se rechaza la hip\'otesis nula. Por ejemplo, en una prueba $t$ para la media:
\begin{eqnarray*}
t = \frac{\bar{x} - \mu_0}{s / \sqrt{n}}
\end{eqnarray*}
donde $\bar{x}$ es la media muestral, $\mu_0$ es la media poblacional bajo la hip\'otesis nula, $s$ es la desviaci\'on est\'andar muestral y $n$ es el tama\~no de la muestra.

\subsubsection{P-valor}

El p-valor es la probabilidad de obtener un valor del estad\'istico de prueba al menos tan extremo como el observado, bajo la suposici\'on de que la hip\'otesis nula es verdadera. Si el p-valor es menor que el nivel de significancia $\alpha$, se rechaza la hip\'otesis nula. El p-valor se interpreta de la siguiente manera:

\begin{itemize}
    \item \textbf{P-valor bajo (p < 0.05)}: Evidencia suficiente para rechazar la hip\'otesis nula.
    \item \textbf{P-valor alto (p > 0.05)}: No hay suficiente evidencia para rechazar la hip\'otesis nula.
\end{itemize}

\subsubsection{Tipos de Errores}

En la prueba de hip\'otesis, se pueden cometer dos tipos de errores:

\begin{itemize}
    \item \textbf{Error Tipo I ($\alpha$)}: Rechazar la hip\'otesis nula cuando es verdadera.
    \item \textbf{Error Tipo II ($\beta$)}: No rechazar la hip\'otesis nula cuando es falsa.
\end{itemize}

\subsubsection{Tabla de Errores en la Prueba de Hip\'otesis}

A continuaci\'on se presenta una tabla que muestra los posibles resultados en una prueba de hip\'otesis, incluyendo los falsos positivos (error tipo I) y los falsos negativos (error tipo II):

\begin{table}[h]
\centering
\begin{tabular}{|c|c|c|}
\hline
 & \textbf{Hip\'otesis Nula Verdadera} & \textbf{Hip\'otesis Nula Falsa} \\
\hline
\textbf{Rechazar $H_0$} & Error Tipo I ($\alpha$) & Aceptar $H_a$ \\
\hline
\textbf{No Rechazar $H_0$} & Aceptar $H_0$ & Error Tipo II ($\beta$) \\
\hline
\end{tabular}
\caption{Resultados de la Prueba de Hip\'otesis}
\label{tab:hypothesis_testing}
\end{table}



\chapter{Matem\'aticas Detr\'as de la Regresi\'on Logística}
\section{Introducci\'on}

La regresi\'on log\'istica es una t\'ecnica de modelado estad\'istico utilizada para predecir la probabilidad de un evento binario en funci\'on de una o m\'as variables independientes. Este cap\'itulo profundiza en las matem\'aticas subyacentes a la regresi\'on log\'istica, incluyendo la funci\'on log\'istica, la funci\'on de verosimilitud, y los m\'etodos para estimar los coeficientes del modelo.

\section{Funci\'on Log\'istica}

La funci\'on log\'istica es la base de la regresi\'on log\'istica. Esta funci\'on transforma una combinaci\'on lineal de variables independientes en una probabilidad.

\subsection{Definici\'on}

La funci\'on log\'istica se define como:
\begin{eqnarray*}
p = \frac{1}{1 + e^{-(\beta_0 + \beta_1 X_1 + \beta_2 X_2 + \ldots + \beta_n X_n)}}
\end{eqnarray*}
donde $p$ es la probabilidad de que el evento ocurra, $\beta_0, \beta_1, \ldots, \beta_n$ son los coeficientes del modelo, y $X_1, X_2, \ldots, X_n$ son las variables independientes.

\subsection{Propiedades}

La funci\'on log\'istica tiene varias propiedades importantes:
\begin{itemize}
    \item \textbf{Rango}: La funci\'on log\'istica siempre produce un valor entre 0 y 1, lo que la hace adecuada para modelar probabilidades.
    \item \textbf{Monoton\'ia}: La funci\'on es mon\'otona creciente, lo que significa que a medida que la combinaci\'on lineal de variables independientes aumenta, la probabilidad tambi\'en aumenta.
    \item \textbf{Simetr\'ia}: La funci\'on log\'istica es sim\'etrica en torno a $p = 0.5$.
\end{itemize}

\section{Funci\'on de Verosimilitud}

La funci\'on de verosimilitud se utiliza para estimar los coeficientes del modelo de regresi\'on log\'istica. Esta funci\'on mide la probabilidad de observar los datos dados los coeficientes del modelo.

\subsection{Definici\'on}

Para un conjunto de $n$ observaciones, la funci\'on de verosimilitud $L$ se define como el producto de las probabilidades individuales de observar cada dato:
\begin{eqnarray*}
L(\beta_0, \beta_1, \ldots, \beta_n) = \prod_{i=1}^{n} p_i^{y_i} (1 - p_i)^{1 - y_i}
\end{eqnarray*}
donde $y_i$ es el valor observado de la variable dependiente para la $i$-\'esima observaci\'on y $p_i$ es la probabilidad predicha de que $Y_i = 1$.

\subsection{Funci\'on de Log-Verosimilitud}

Para simplificar los c\'alculos, trabajamos con el logaritmo de la funci\'on de verosimilitud, conocido como la funci\'on de log-verosimilitud. Tomar el logaritmo convierte el producto en una suma:
\begin{eqnarray*}
\log L(\beta_0, \beta_1, \ldots, \beta_n) = \sum_{i=1}^{n} \left[ y_i \log(p_i) + (1 - y_i) \log(1 - p_i) \right]
\end{eqnarray*}

Sustituyendo $p_i$:
\begin{eqnarray*}
\log L(\beta_0, \beta_1, \ldots, \beta_n) = \sum_{i=1}^{n} \left[ y_i (\beta_0 + \beta_1 X_{i1} + \beta_2 X_{i2} + \ldots + \beta_n X_{in}) - \log(1 + e^{\beta_0 + \beta_1 X_{i1} + \beta_2 X_{i2} + \ldots + \beta_n X_{in}}) \right]
\end{eqnarray*}

\section{Estimaci\'on de Coeficientes}

Los coeficientes del modelo de regresi\'on log\'istica se estiman maximizando la funci\'on de log-verosimilitud. Este proceso generalmente se realiza mediante m\'etodos iterativos como el algoritmo de Newton-Raphson.

\subsection{Gradiente y Hessiana}

Para maximizar la funci\'on de log-verosimilitud, necesitamos calcular su gradiente y su matriz Hessiana.

\subsubsection{Gradiente}

El gradiente de la funci\'on de log-verosimilitud con respecto a los coeficientes $\beta$ es:
\begin{eqnarray*}
\mathbf{g}(\beta) = \frac{\partial \log L}{\partial \beta} = \sum_{i=1}^{n} \mathbf{X}_i (y_i - p_i)
\end{eqnarray*}
donde $\mathbf{X}_i$ es el vector de valores de las variables independientes para la $i$-\'esima observaci\'on.

\subsubsection{Hessiana}

La matriz Hessiana de la funci\'on de log-verosimilitud con respecto a los coeficientes $\beta$ es:
\begin{eqnarray*}
\mathbf{H}(\beta) = \frac{\partial^2 \log L}{\partial \beta \partial \beta^T} = -\sum_{i=1}^{n} p_i (1 - p_i) \mathbf{X}_i \mathbf{X}_i^T
\end{eqnarray*}

\subsection{Algoritmo Newton-Raphson}

El algoritmo Newton-Raphson se utiliza para encontrar los valores de los coeficientes que maximizan la funci\'on de log-verosimilitud. El algoritmo se puede resumir en los siguientes pasos:
\begin{enumerate}
    \item Inicializar el vector de coeficientes $\beta^{(0)}$ (por ejemplo, con ceros o valores peque\~nos aleatorios).
    \item Calcular el gradiente $\mathbf{g}(\beta^{(k)})$ y la matriz Hessiana $\mathbf{H}(\beta^{(k)})$ en la iteraci\'on $k$.
    \item Actualizar los coeficientes utilizando la f\'ormula:
    \begin{eqnarray*}
    \beta^{(k+1)} = \beta^{(k)} - \left[ \mathbf{H}(\beta^{(k)}) \right]^{-1} \mathbf{g}(\beta^{(k)})
    \end{eqnarray*}
    \item Repetir los pasos 2 y 3 hasta que la diferencia entre $\beta^{(k+1)}$ y $\beta^{(k)}$ sea menor que un umbral predefinido (criterio de convergencia).
\end{enumerate}

\section{Validaci\'on del Modelo}

Una vez que se han estimado los coeficientes del modelo de regresi\'on log\'istica, es importante validar el modelo para asegurarse de que proporciona predicciones precisas.

\subsection{Curva ROC y AUC}

La curva ROC (Receiver Operating Characteristic) es una herramienta gr\'afica utilizada para evaluar el rendimiento de un modelo de clasificaci\'on binaria. El \'area bajo la curva (AUC) mide la capacidad del modelo para distinguir entre las clases.

\subsection{Matriz de Confusi\'on}

La matriz de confusi\'on es una tabla que resume el rendimiento de un modelo de clasificaci\'on al comparar las predicciones del modelo con los valores reales. Los t\'erminos en la matriz de confusi\'on incluyen verdaderos positivos, falsos positivos, verdaderos negativos y falsos negativos.



\chapter{Preparaci\'on de Datos y Selecci\'on de Variables}


\section{Introducci\'on}

La preparaci\'on de datos y la selecci\'on de variables son pasos cruciales en el proceso de modelado estad\'istico. Un modelo bien preparado y con las variables adecuadas puede mejorar significativamente la precisi\'on y la interpretabilidad del modelo. Este cap\'itulo proporciona una revisi\'on detallada de las t\'ecnicas de limpieza de datos, tratamiento de datos faltantes, codificaci\'on de variables categ\'oricas y selecci\'on de variables.

\section{Importancia de la Preparaci\'on de Datos}

La calidad de los datos es fundamental para el \'exito de cualquier an\'alisis estad\'istico. Los datos sin limpiar pueden llevar a modelos inexactos y conclusiones err\'oneas. La preparaci\'on de datos incluye varias etapas:
\begin{itemize}
    \item Limpieza de datos
    \item Tratamiento de datos faltantes
    \item Codificaci\'on de variables categ\'oricas
    \item Selecci\'on y transformaci\'on de variables
\end{itemize}

\section{Limpieza de Datos}

La limpieza de datos es el proceso de detectar y corregir (o eliminar) los datos incorrectos, incompletos o irrelevantes. Este proceso incluye:
\begin{itemize}
    \item Eliminaci\'on de duplicados
    \item Correcci\'on de errores tipogr\'aficos
    \item Consistencia de formato
    \item Tratamiento de valores extremos (outliers)
\end{itemize}

\section{Tratamiento de Datos Faltantes}

Los datos faltantes son un problema com\'un en los conjuntos de datos y pueden afectar la calidad de los modelos. Hay varias estrategias para manejar los datos faltantes:
\begin{itemize}
    \item \textbf{Eliminaci\'on de Datos Faltantes}: Se eliminan las filas o columnas con datos faltantes.
    \item \textbf{Imputaci\'on}: Se reemplazan los valores faltantes con estimaciones, como la media, la mediana o la moda.
    \item \textbf{Modelos Predictivos}: Se utilizan modelos predictivos para estimar los valores faltantes.
\end{itemize}

\subsection{Imputaci\'on de la Media}

Una t\'ecnica com\'un es reemplazar los valores faltantes con la media de la variable. Esto se puede hacer de la siguiente manera:
\begin{eqnarray*}
x_i = \begin{cases} 
      x_i & \text{si } x_i \text{ no es faltante} \\
      \bar{x} & \text{si } x_i \text{ es faltante}
   \end{cases}
\end{eqnarray*}
donde $\bar{x}$ es la media de la variable.

\section{Codificaci\'on de Variables Categ\'oricas}

Las variables categ\'oricas deben ser convertidas a un formato num\'erico antes de ser usadas en un modelo de regresi\'on log\'istica. Hay varias t\'ecnicas para codificar variables categ\'oricas:

\subsection{Codificaci\'on One-Hot}

La codificaci\'on one-hot crea una columna binaria para cada categor\'ia. Por ejemplo, si tenemos una variable categ\'orica con tres categor\'ias (A, B, C), se crean tres columnas:
\begin{eqnarray*}
\text{A} &=& [1, 0, 0] \\
\text{B} &=& [0, 1, 0] \\
\text{C} &=& [0, 0, 1]
\end{eqnarray*}

\subsection{Codificaci\'on Ordinal}

La codificaci\'on ordinal asigna un valor entero \'unico a cada categor\'ia, preservando el orden natural de las categor\'ias. Por ejemplo:
\begin{eqnarray*}
\text{Bajo} &=& 1 \\
\text{Medio} &=& 2 \\
\text{Alto} &=& 3
\end{eqnarray*}

\section{Selecci\'on de Variables}

La selecci\'on de variables es el proceso de elegir las variables m\'as relevantes para el modelo. Existen varias t\'ecnicas para la selecci\'on de variables:

\subsection{M\'etodos de Filtrado}

Los m\'etodos de filtrado seleccionan variables basadas en criterios estad\'isticos, como la correlaci\'on o la chi-cuadrado. Algunas t\'ecnicas comunes incluyen:
\begin{itemize}
    \item \textbf{An\'alisis de Correlaci\'on}: Se seleccionan variables con alta correlaci\'on con la variable dependiente y baja correlaci\'on entre ellas.
    \item \textbf{Pruebas de Chi-cuadrado}: Se utilizan para variables categ\'oricas para determinar la asociaci\'on entre la variable independiente y la variable dependiente.
\end{itemize}

\subsection{M\'etodos de Wrapper}

Los m\'etodos de wrapper eval\'uan m\'ultiples combinaciones de variables y seleccionan la combinaci\'on que optimiza el rendimiento del modelo. Ejemplos incluyen:
\begin{itemize}
    \item \textbf{Selecci\'on hacia Adelante}: Comienza con un modelo vac\'io y agrega variables una por una, seleccionando la variable que mejora m\'as el modelo en cada paso.
    \item \textbf{Selecci\'on hacia Atr\'as}: Comienza con todas las variables y elimina una por una, removiendo la variable que tiene el menor impacto en el modelo en cada paso.
    \item \textbf{Selecci\'on Paso a Paso}: Combina la selecci\'on hacia adelante y hacia atr\'as, agregando y eliminando variables seg\'un sea necesario.
\end{itemize}

\subsection{M\'etodos Basados en Modelos}

Los m\'etodos basados en modelos utilizan t\'ecnicas de regularizaci\'on como Lasso y Ridge para seleccionar variables. Estas t\'ecnicas a\~naden un t\'ermino de penalizaci\'on a la funci\'on de costo para evitar el sobreajuste.

\subsubsection{Regresi\'on Lasso}

La regresi\'on Lasso (Least Absolute Shrinkage and Selection Operator) a\~nade una penalizaci\'on $L_1$ a la funci\'on de costo:
\begin{eqnarray*}
J(\beta) = \sum_{i=1}^{n} (y_i - \hat{y}_i)^2 + \lambda \sum_{j=1}^{p} |\beta_j|
\end{eqnarray*}
donde $\lambda$ es el par\'ametro de regularizaci\'on que controla la cantidad de penalizaci\'on.

\subsubsection{Regresi\'on Ridge}

La regresi\'on Ridge a\~nade una penalizaci\'on $L_2$ a la funci\'on de costo:
\begin{eqnarray*}
J(\beta) = \sum_{i=1}^{n} (y_i - \hat{y}_i)^2 + \lambda \sum_{j=1}^{p} \beta_j^2
\end{eqnarray*}
donde $\lambda$ es el par\'ametro de regularizaci\'on.

\section{Implementaci\'on en R}

\subsection{Limpieza de Datos}

Para ilustrar la limpieza de datos en R, considere el siguiente conjunto de datos:
\begin{verbatim}
data <- data.frame(
  var1 = c(1, 2, 3, NA, 5),
  var2 = c("A", "B", "A", "B", "A"),
  var3 = c(10, 15, 10, 20, 25)
)

# Eliminaci\'on de filas con datos faltantes
data_clean <- na.omit(data)

# Imputaci\'on de la media
data$var1[is.na(data$var1)] <- mean(data$var1, na.rm = TRUE)
\end{verbatim}

\subsection{Codificaci\'on de Variables Categ\'oricas}

Para codificar variables categ\'oricas, utilice la funci\'on `model.matrix`:
\begin{verbatim}
data <- data.frame(
  var1 = c(1, 2, 3, 4, 5),
  var2 = c("A", "B", "A", "B", "A")
)

# Codificaci\'on one-hot
data_onehot <- model.matrix(~ var2 - 1, data = data)
\end{verbatim}

\subsection{Selecci\'on de Variables}

Para la selecci\'on de variables, utilice el paquete `caret`:
\begin{verbatim}
library(caret)

# Dividir los datos en conjuntos de entrenamiento y prueba
set.seed(123)
trainIndex <- createDataPartition(data$var1, p = .8, 
                                  list = FALSE, 
                                  times = 1)
dataTrain <- data[trainIndex,]
dataTest <- data[-trainIndex,]

# Modelo de regresi\'on log\'istica
model <- train(var1 ~ ., data = dataTrain, method = "glm", family = "binomial")

# Selecci\'on de variables
model <- step(model, direction = "both")
summary(model)
\end{verbatim}



\chapter{Evaluaci\'on del Modelo y Validaci\'on Cruzada}


\section{Introducción}

Evaluar la calidad y el rendimiento de un modelo de regresión logística es crucial para asegurar que las predicciones sean precisas y útiles. Este capítulo se centra en las técnicas y métricas utilizadas para evaluar modelos de clasificación binaria, así como en la validación cruzada, una técnica para evaluar la generalización del modelo.

\section{Métricas de Evaluación del Modelo}

Las métricas de evaluación permiten cuantificar la precisión y el rendimiento de un modelo. Algunas de las métricas más comunes incluyen:

\subsection{Curva ROC y AUC}

La curva ROC (Receiver Operating Characteristic) es una representación gráfica de la sensibilidad (verdaderos positivos) frente a 1 - especificidad (falsos positivos). El área bajo la curva (AUC) mide la capacidad del modelo para distinguir entre las clases.

\begin{eqnarray*}
\text{Sensibilidad} &=& \frac{\text{TP}}{\text{TP} + \text{FN}} \\
\text{Especificidad} &=& \frac{\text{TN}}{\text{TN} + \text{FP}}
\end{eqnarray*}

\subsection{Matriz de Confusión}

La matriz de confusión es una tabla que muestra el rendimiento del modelo comparando las predicciones con los valores reales. Los términos incluyen:
\begin{itemize}
    \item \textbf{Verdaderos Positivos (TP)}: Predicciones correctas de la clase positiva.
    \item \textbf{Falsos Positivos (FP)}: Predicciones incorrectas de la clase positiva.
    \item \textbf{Verdaderos Negativos (TN)}: Predicciones correctas de la clase negativa.
    \item \textbf{Falsos Negativos (FN)}: Predicciones incorrectas de la clase negativa.
\end{itemize}

\begin{table}[h]
\centering
\begin{tabular}{|c|c|c|}
\hline
 & \textbf{Predicción Positiva} & \textbf{Predicción Negativa} \\
\hline
\textbf{Real Positiva} & TP & FN \\
\hline
\textbf{Real Negativa} & FP & TN \\
\hline
\end{tabular}
\caption{Matriz de Confusión}
\label{tab:confusion_matrix}
\end{table}

\subsection{Precisión, Recall y F1-Score}

\begin{eqnarray*}
\text{Precisión} &=& \frac{\text{TP}}{\text{TP} + \text{FP}} \\
\text{Recall} &=& \frac{\text{TP}}{\text{TP} + \text{FN}} \\
\text{F1-Score} &=& 2 \cdot \frac{\text{Precisión} \cdot \text{Recall}}{\text{Precisión} + \text{Recall}}
\end{eqnarray*}

\subsection{Log-Loss}

La pérdida logarítmica (Log-Loss) mide la precisión de las probabilidades predichas. La fórmula es:
\begin{eqnarray*}
\text{Log-Loss} = -\frac{1}{n} \sum_{i=1}^{n} \left[ y_i \log(p_i) + (1 - y_i) \log(1 - p_i) \right]
\end{eqnarray*}
donde $y_i$ son los valores reales y $p_i$ son las probabilidades predichas.

\section{Validación Cruzada}

La validación cruzada es una técnica para evaluar la capacidad de generalización de un modelo. Existen varios tipos de validación cruzada:

\subsection{K-Fold Cross-Validation}

En K-Fold Cross-Validation, los datos se dividen en K subconjuntos. El modelo se entrena K veces, cada vez utilizando K-1 subconjuntos para el entrenamiento y el subconjunto restante para la validación.

\begin{eqnarray*}
\text{Error Medio} = \frac{1}{K} \sum_{k=1}^{K} \text{Error}_k
\end{eqnarray*}

\subsection{Leave-One-Out Cross-Validation (LOOCV)}

En LOOCV, cada observación se usa una vez como conjunto de validación y las restantes como conjunto de entrenamiento. Este método es computacionalmente costoso pero útil para conjuntos de datos pequeños.

\section{Ajuste y Sobreajuste del Modelo}

El ajuste adecuado del modelo es crucial para evitar el sobreajuste (overfitting) y el subajuste (underfitting).

\subsection{Sobreajuste}

El sobreajuste ocurre cuando un modelo se ajusta demasiado bien a los datos de entrenamiento, capturando ruido y patrones irrelevantes. Los síntomas incluyen una alta precisión en el entrenamiento y baja precisión en la validación.

\subsection{Subajuste}

El subajuste ocurre cuando un modelo no captura los patrones subyacentes de los datos. Los síntomas incluyen baja precisión tanto en el entrenamiento como en la validación.

\subsection{Regularización}

La regularización es una técnica para prevenir el sobreajuste añadiendo un término de penalización a la función de costo. Las técnicas comunes incluyen:
\begin{itemize}
    \item \textbf{Regresión Lasso (L1)}
    \item \textbf{Regresión Ridge (L2)}
\end{itemize}

\section{Implementación en R}

\subsection{Evaluación del Modelo}

\begin{verbatim}
# Cargar el paquete necesario
library(caret)

# Dividir los datos en conjuntos de entrenamiento y prueba
set.seed(123)
trainIndex <- createDataPartition(data$var1, p = .8, 
                                  list = FALSE, 
                                  times = 1)
dataTrain <- data[trainIndex,]
dataTest <- data[-trainIndex,]

# Entrenar el modelo de regresión logística
model <- train(var1 ~ ., data = dataTrain, method = "glm", family = "binomial")

# Predicciones en el conjunto de prueba
predictions <- predict(model, dataTest)

# Matriz de confusión
confusionMatrix(predictions, dataTest$var1)
\end{verbatim}

\subsection{Validación Cruzada}

\begin{verbatim}
# K-Fold Cross-Validation
control <- trainControl(method = "cv", number = 10)
model_cv <- train(var1 ~ ., data = dataTrain, method = "glm", 
                  family = "binomial", trControl = control)

# Evaluación del modelo con validación cruzada
print(model_cv)
\end{verbatim}



\chapter{Diagn\'ostico del Modelo y Ajuste de Par\'ametros}


\section{Introducci\'on}

El diagn\'ostico del modelo y el ajuste de par\'ametros son pasos esenciales para mejorar la precisi\'on y la robustez de los modelos de regresi\'on log\'istica. Este cap\'itulo se enfoca en las t\'ecnicas para diagnosticar problemas en los modelos y en m\'etodos para ajustar los par\'ametros de manera \'optima.

\section{Diagn\'ostico del Modelo}

El diagn\'ostico del modelo implica evaluar el rendimiento del modelo y detectar posibles problemas, como el sobreajuste, la multicolinealidad y la influencia de puntos de datos individuales.

\subsection{Residuos}

Los residuos son las diferencias entre los valores observados y los valores predichos por el modelo. El an\'alisis de residuos puede revelar patrones que indican problemas con el modelo.

\begin{eqnarray*}
\text{Residuo}_i = y_i - \hat{y}_i
\end{eqnarray*}

\subsubsection{Residuos Estudiantizados}

Los residuos estudiantizados se ajustan por la variabilidad del residuo y se utilizan para detectar outliers.

\begin{eqnarray*}
r_i = \frac{\text{Residuo}_i}{\hat{\sigma} \sqrt{1 - h_i}}
\end{eqnarray*}
donde $h_i$ es el leverage del punto de datos.

\subsection{Influencia}

La influencia mide el impacto de un punto de datos en los coeficientes del modelo. Los puntos con alta influencia pueden distorsionar el modelo.

\subsubsection{Distancia de Cook}

La distancia de Cook es una medida de la influencia de un punto de datos en los coeficientes del modelo.

\begin{eqnarray*}
D_i = \frac{r_i^2}{p} \cdot \frac{h_i}{1 - h_i}
\end{eqnarray*}
donde $p$ es el n\'umero de par\'ametros en el modelo.

\subsection{Multicolinealidad}

La multicolinealidad ocurre cuando dos o m\'as variables independientes est\'an altamente correlacionadas. Esto puede inflar las varianzas de los coeficientes y hacer que el modelo sea inestable.

\subsubsection{Factor de Inflaci\'on de la Varianza (VIF)}

El VIF mide cu\'anto se inflan las varianzas de los coeficientes debido a la multicolinealidad.

\begin{eqnarray*}
\text{VIF}_j = \frac{1}{1 - R_j^2}
\end{eqnarray*}
donde $R_j^2$ es el coeficiente de determinaci\'on de la regresi\'on de la variable $j$ contra todas las dem\'as variables.

\section{Ajuste de Par\'ametros}

El ajuste de par\'ametros implica seleccionar los valores \'optimos para los hiperpar\'ametros del modelo. Esto puede mejorar el rendimiento y prevenir el sobreajuste.

\subsection{Grid Search}

El grid search es un m\'etodo exhaustivo para ajustar los par\'ametros. Se define una rejilla de posibles valores de par\'ametros y se eval\'ua el rendimiento del modelo para cada combinaci\'on.

\subsection{Random Search}

El random search selecciona aleatoriamente combinaciones de valores de par\'ametros dentro de un rango especificado. Es menos exhaustivo que el grid search, pero puede ser m\'as eficiente.

\subsection{Bayesian Optimization}

La optimizaci\'on bayesiana utiliza modelos probabil\'isticos para seleccionar iterativamente los valores de par\'ametros m\'as prometedores.

\section{Implementaci\'on en R}

\subsection{Diagn\'ostico del Modelo}

\begin{verbatim}
# Cargar el paquete necesario
library(car)

# Residuos estudentizados
dataTrain$resid <- rstudent(model)
hist(dataTrain$resid, breaks = 20, main = "Residuos Estudentizados")

# Distancia de Cook
dataTrain$cook <- cooks.distance(model)
plot(dataTrain$cook, type = "h", main = "Distancia de Cook")

# Factor de Inflaci\'on de la Varianza
vif_values <- vif(model)
print(vif_values)
\end{verbatim}

\subsection{Ajuste de Par\'ametros}

\begin{verbatim}
# Grid Search con caret
control <- trainControl(method = "cv", number = 10)
tune_grid <- expand.grid(.alpha = c(0, 0.5, 1), .lambda = seq(0.01, 0.1, by = 0.01))

model_tune <- train(var1 ~ ., data = dataTrain, method = "glmnet", 
                    trControl = control, tuneGrid = tune_grid)

print(model_tune)
\end{verbatim}



\chapter{Interpretaci\'on de los Resultados}

\section{Introducci\'on}

Interpretar correctamente los resultados de un modelo de regresi\'on log\'istica es esencial para tomar decisiones informadas. Este cap\'itulo se centra en la interpretaci\'on de los coeficientes del modelo, las odds ratios, los intervalos de confianza y la significancia estad\'istica.

\section{Coeficientes de Regresi\'on Log\'istica}

Los coeficientes de regresi\'on log\'istica representan la relaci\'on entre las variables independientes y la variable dependiente en t\'erminos de log-odds. 

\subsection{Interpretaci\'on de los Coeficientes}

Cada coeficiente $\beta_j$ en el modelo de regresi\'on log\'istica se interpreta como el cambio en el log-odds de la variable dependiente por unidad de cambio en la variable independiente $X_j$.

\begin{eqnarray*}
\log\left(\frac{p}{1-p}\right) = \beta_0 + \beta_1 X_1 + \beta_2 X_2 + \ldots + \beta_n X_n
\end{eqnarray*}

\subsection{Signo de los Coeficientes}

\begin{itemize}
    \item \textbf{Coeficiente Positivo}: Un coeficiente positivo indica que un aumento en la variable independiente est\'a asociado con un aumento en el log-odds de la variable dependiente.
    \item \textbf{Coeficiente Negativo}: Un coeficiente negativo indica que un aumento en la variable independiente est\'a asociado con una disminuci\'on en el log-odds de la variable dependiente.
\end{itemize}

\section{Odds Ratios}

Las odds ratios proporcionan una interpretaci\'on m\'as intuitiva de los coeficientes de regresi\'on log\'istica. La odds ratio para una variable independiente $X_j$ se calcula como $e^{\beta_j}$.

\subsection{C\'alculo de las Odds Ratios}

\begin{eqnarray*}
\text{OR}_j = e^{\beta_j}
\end{eqnarray*}

\subsection{Interpretaci\'on de las Odds Ratios}

\begin{itemize}
    \item \textbf{OR > 1}: Un OR mayor que 1 indica que un aumento en la variable independiente est\'a asociado con un aumento en las odds de la variable dependiente.
    \item \textbf{OR < 1}: Un OR menor que 1 indica que un aumento en la variable independiente est\'a asociado con una disminuci\'on en las odds de la variable dependiente.
    \item \textbf{OR = 1}: Un OR igual a 1 indica que la variable independiente no tiene efecto sobre las odds de la variable dependiente.
\end{itemize}

\section{Intervalos de Confianza}

Los intervalos de confianza proporcionan una medida de la incertidumbre asociada con los estimadores de los coeficientes. Un intervalo de confianza del 95\% para un coeficiente $\beta_j$ indica que, en el 95\% de las muestras, el intervalo contendr\'a el valor verdadero de $\beta_j$.

\subsection{C\'alculo de los Intervalos de Confianza}

Para calcular un intervalo de confianza del 95\% para un coeficiente $\beta_j$, utilizamos la f\'ormula:
\begin{eqnarray*}
\beta_j \pm 1.96 \cdot \text{SE}(\beta_j)
\end{eqnarray*}
donde $\text{SE}(\beta_j)$ es el error est\'andar de $\beta_j$.

\section{Significancia Estad\'istica}

La significancia estad\'istica se utiliza para determinar si los coeficientes del modelo son significativamente diferentes de cero. Esto se eval\'ua mediante pruebas de hip\'otesis.

\subsection{Prueba de Hip\'otesis}

Para cada coeficiente $\beta_j$, la hip\'otesis nula $H_0$ es que $\beta_j = 0$. La hip\'otesis alternativa $H_a$ es que $\beta_j \neq 0$.

\subsection{P-valor}

El p-valor indica la probabilidad de obtener un coeficiente tan extremo como el observado, asumiendo que la hip\'otesis nula es verdadera. Un p-valor menor que el nivel de significancia $\alpha$ (t\'ipicamente 0.05) indica que podemos rechazar la hip\'otesis nula.

\section{Implementaci\'on en R}

\subsection{C\'alculo de Coeficientes y Odds Ratios}

\begin{verbatim}
# Cargar el paquete necesario
library(broom)

# Entrenar el modelo de regresi\'on log\'istica
model <- glm(var1 ~ ., data = dataTrain, family = "binomial")

# Coeficientes del modelo
coef(model)

# Odds ratios
exp(coef(model))
\end{verbatim}

\subsection{Intervalos de Confianza}

\begin{verbatim}
# Intervalos de confianza para los coeficientes
confint(model)

# Intervalos de confianza para las odds ratios
exp(confint(model))
\end{verbatim}

\subsection{P-valores y Significancia Estad\'istica}

\begin{verbatim}
# Resumen del modelo con p-valores
summary(model)
\end{verbatim}



\chapter{Regresi\'on Logística Multinomial y An\'alisis de Supervivencia}

\section{Introducci\'on}

La regresi\'on log\'istica multinomial y el an\'alisis de supervivencia son extensiones de la regresi\'on log\'istica binaria. Este cap\'itulo se enfoca en las t\'ecnicas y aplicaciones de estos m\'etodos avanzados.

\section{Regresi\'on Log\'istica Multinomial}

La regresi\'on log\'istica multinomial se utiliza cuando la variable dependiente tiene m\'as de dos categor\'ias.

\subsection{Modelo Multinomial}

El modelo de regresi\'on log\'istica multinomial generaliza el modelo binario para manejar m\'ultiples categor\'ias. La probabilidad de que una observaci\'on pertenezca a la categor\'ia $k$ se expresa como:

\begin{eqnarray*}
P(Y = k) = \frac{e^{\beta_{0k} + \beta_{1k} X_1 + \ldots + \beta_{nk} X_n}}{\sum_{j=1}^{K} e^{\beta_{0j} + \beta_{1j} X_1 + \ldots + \beta_{nj} X_n}}
\end{eqnarray*}

\subsection{Estimaci\'on de Par\'ametros}

Los coeficientes del modelo multinomial se estiman utilizando m\'axima verosimilitud, similar a la regresi\'on log\'istica binaria.

\section{An\'alisis de Supervivencia}

El an\'alisis de supervivencia se utiliza para modelar el tiempo hasta que ocurre un evento de inter\'es, como la muerte o la falla de un componente.

\subsection{Funci\'on de Supervivencia}

La funci\'on de supervivencia $S(t)$ describe la probabilidad de que una observaci\'on sobreviva m\'as all\'a del tiempo $t$:

\begin{eqnarray*}
S(t) = P(T > t)
\end{eqnarray*}

\subsection{Modelo de Riesgos Proporcionales de Cox}

El modelo de Cox es un modelo de regresi\'on semiparam\'etrico utilizado para analizar datos de supervivencia:

\begin{eqnarray*}
h(t|X) = h_0(t) e^{\beta_1 X_1 + \ldots + \beta_p X_p}
\end{eqnarray*}
donde $h(t|X)$ es la tasa de riesgo en el tiempo $t$ dado el vector de covariables $X$ y $h_0(t)$ es la tasa de riesgo basal.

\section{Implementaci\'on en R}

\subsection{Regresi\'on Log\'istica Multinomial}

\begin{verbatim}
# Cargar el paquete necesario
library(nnet)

# Entrenar el modelo de regresi\'on log\'istica multinomial
model_multinom <- multinom(var1 ~ ., data = dataTrain)

# Resumen del modelo
summary(model_multinom)
\end{verbatim}

\subsection{An\'alisis de Supervivencia}

\begin{verbatim}
# Cargar el paquete necesario
library(survival)

# Crear el objeto de supervivencia
surv_object <- Surv(time = data$time, event = data$status)

# Ajustar el modelo de Cox
model_cox <- coxph(surv_object ~ var1 + var2, data = data)

# Resumen del modelo
summary(model_cox)
\end{verbatim}



\chapter{Implementaci\'on de Regresi\'on Logística en Datos Reales}
\section{Introducci\'on}

Implementar un modelo de regresi\'on log\'istica en datos reales implica varias etapas, desde la limpieza de datos hasta la evaluaci\'on y validaci\'on del modelo. Este cap\'itulo presenta un ejemplo pr\'actico de la implementaci\'on de un modelo de regresi\'on log\'istica utilizando un conjunto de datos real.

\section{Conjunto de Datos}

Para este ejemplo, utilizaremos un conjunto de datos disponible p\'ublicamente que contiene informaci\'on sobre clientes bancarios. El objetivo es predecir si un cliente suscribir\'a un dep\'osito a plazo fijo.

\section{Preparaci\'on de Datos}

\subsection{Carga y Exploraci\'on de Datos}

Primero, cargamos y exploramos el conjunto de datos para entender su estructura y contenido.

\begin{verbatim}
# Cargar el paquete necesario
library(dplyr)

# Cargar el conjunto de datos
data <- read.csv("bank.csv")

# Explorar los datos
str(data)
summary(data)
\end{verbatim}

\subsection{Limpieza de Datos}

El siguiente paso es limpiar los datos, lo que incluye tratar los valores faltantes y eliminar las duplicidades.

\begin{verbatim}
# Eliminar duplicados
data <- data %>% distinct()

# Imputar valores faltantes (si existen)
data <- data %>% mutate_if(is.numeric, ~ifelse(is.na(.), mean(., na.rm = TRUE), .))
\end{verbatim}

\subsection{Codificaci\'on de Variables Categ\'oricas}

Convertimos las variables categ\'oricas en variables num\'ericas utilizando la codificaci\'on one-hot.

\begin{verbatim}
# Codificaci\'on one-hot de variables categ\'oricas
data <- data %>% mutate(across(where(is.factor), ~ as.numeric(as.factor(.))))
\end{verbatim}

\section{Divisi\'on de Datos}

Dividimos los datos en conjuntos de entrenamiento y prueba.

\begin{verbatim}
# Dividir los datos en conjuntos de entrenamiento y prueba
set.seed(123)
trainIndex <- createDataPartition(data$y, p = .8, list = FALSE, times = 1)
dataTrain <- data[trainIndex,]
dataTest <- data[-trainIndex,]
\end{verbatim}

\section{Entrenamiento del Modelo}

Entrenamos un modelo de regresi\'on log\'istica utilizando el conjunto de entrenamiento.

\begin{verbatim}
# Entrenar el modelo de regresi\'on log\'istica
model <- glm(y ~ ., data = dataTrain, family = "binomial")

# Resumen del modelo
summary(model)
\end{verbatim}

\section{Evaluaci\'on del Modelo}

Evaluamos el rendimiento del modelo utilizando el conjunto de prueba.

\begin{verbatim}
# Predicciones en el conjunto de prueba
predictions <- predict(model, dataTest, type = "response")

# Convertir probabilidades a etiquetas
predicted_labels <- ifelse(predictions > 0.5, 1, 0)

# Matriz de confusi\'on
confusionMatrix(predicted_labels, dataTest$y)
\end{verbatim}

\section{Interpretaci\'on de los Resultados}

Interpretamos los coeficientes del modelo y las odds ratios.

\begin{verbatim}
# Coeficientes del modelo
coef(model)

# Odds ratios
exp(coef(model))
\end{verbatim}



\chapter{Resumen y Proyecto Final}
\section{Resumen de Conceptos Clave}

En este curso, hemos cubierto una variedad de conceptos y t\'ecnicas esenciales para la regresi\'on log\'istica. Los conceptos clave incluyen:

\begin{itemize}
    \item \textbf{Fundamentos de Probabilidad y Estad\'istica}: Comprensi\'on de distribuciones de probabilidad, medidas de tendencia central y dispersi\'on, inferencia estad\'istica y pruebas de hip\'otesis.
    \item \textbf{Regresi\'on Log\'istica}: Modelo de regresi\'on log\'istica binaria y multinomial, interpretaci\'on de coeficientes y odds ratios, m\'etodos de estimaci\'on y validaci\'on.
    \item \textbf{Preparaci\'on de Datos}: Limpieza de datos, tratamiento de valores faltantes, codificaci\'on de variables categ\'oricas y selecci\'on de variables.
    \item \textbf{Evaluaci\'on del Modelo}: Curva ROC, AUC, matriz de confusi\'on, precisi\'on, recall, F1-score y validaci\'on cruzada.
    \item \textbf{Diagn\'ostico del Modelo}: An\'alisis de residuos, influencia, multicolinealidad y ajuste de par\'ametros.
    \item \textbf{An\'alisis de Supervivencia}: Modelos de supervivencia, funci\'on de supervivencia y modelos de riesgos proporcionales de Cox.
\end{itemize}

\section{Buenas Pr\'acticas}

Al implementar modelos de regresi\'on log\'istica, es importante seguir buenas pr\'acticas para garantizar la precisi\'on y la robustez de los modelos. Algunas buenas pr\'acticas incluyen:

\begin{itemize}
    \item \textbf{Exploraci\'on y Preparaci\'on de Datos}: Realizar un an\'alisis exploratorio exhaustivo y preparar los datos adecuadamente antes de construir el modelo.
    \item \textbf{Evaluaci\'on y Validaci\'on del Modelo}: Utilizar m\'etricas adecuadas para evaluar el rendimiento del modelo y validar el modelo utilizando t\'ecnicas como la validaci\'on cruzada.
    \item \textbf{Interpretaci\'on de Resultados}: Interpretar correctamente los coeficientes del modelo y las odds ratios, y comunicar los resultados de manera clara y concisa.
    \item \textbf{Revisi\'on y Ajuste del Modelo}: Diagnosticar problemas en el modelo y ajustar los par\'ametros para mejorar el rendimiento.
\end{itemize}

\section{Proyecto Final}

Para aplicar los conceptos y t\'ecnicas aprendidos en este curso, te proponemos realizar un proyecto final utilizando un conjunto de datos de tu elecci\'on. El proyecto debe incluir las siguientes etapas:

\subsection{Selecci\'on del Conjunto de Datos}

Elige un conjunto de datos relevante que contenga una variable dependiente binaria o multinomial y varias variables independientes.

\subsection{Exploraci\'on y Preparaci\'on de Datos}

Realiza un an\'alisis exploratorio de los datos y prepara los datos para el modelado. Esto incluye la limpieza de datos, el tratamiento de valores faltantes y la codificaci\'on de variables categ\'oricas.

\subsection{Entrenamiento y Evaluaci\'on del Modelo}

Entrena un modelo de regresi\'on log\'istica utilizando el conjunto de datos preparado y eval\'ua su rendimiento utilizando m\'etricas apropiadas.

\subsection{Interpretaci\'on de Resultados}

Interpreta los coeficientes del modelo y las odds ratios, y proporciona una explicaci\'on clara de los resultados.

\subsection{Presentaci\'on del Proyecto}

Presenta tu proyecto en un informe detallado que incluya la descripci\'on del conjunto de datos, los pasos de preparaci\'on y modelado, los resultados del modelo y las conclusiones.




\part{SEGUNDA PARTE}
\chapter{Introducci\'on al An\'alisis de Supervivencia}

\section{Conceptos Básicos}
El análisis de supervivencia es una rama de la estad\'istica que se ocupa del análisis del tiempo que transcurre hasta que ocurre un evento de inter\'es, com\'unmente referido como "tiempo de falla". Este campo es ampliamente utilizado en medicina, biolog\'ia, ingenier\'ia, ciencias sociales, y otros campos.

\section{Definici\'on de Eventos y Tiempos}
En el análisis de supervivencia, un "evento" se refiere a la ocurrencia de un evento espec\'ifico, como la muerte, la falla de un componente, la reca\'ida de una enfermedad, etc. El "tiempo de supervivencia" es el tiempo que transcurre desde un punto de inicio definido hasta la ocurrencia del evento.

\section{Censura}
La censura ocurre cuando la informaci\'on completa sobre el tiempo hasta el evento no está disponible para todos los individuos en el estudio. Hay tres tipos principales de censura:
\begin{itemize}
    \item \textbf{Censura a la derecha:} Ocurre cuando el evento de inter\'es no se ha observado para algunos sujetos antes del final del estudio.
    \item \textbf{Censura a la izquierda:} Ocurre cuando el evento de inter\'es ocurri\'o antes del inicio del periodo de observaci\'on.
    \item \textbf{Censura por intervalo:} Ocurre cuando el evento de inter\'es se sabe que ocurri\'o en un intervalo de tiempo, pero no se conoce el momento exacto.
\end{itemize}

\section{Funci\'on de Supervivencia}
La funci\'on de supervivencia, $S(t)$, se define como la probabilidad de que un individuo sobreviva más allá de un tiempo $t$. Matemáticamente, se expresa como:
\begin{eqnarray*}
S(t) = P(T > t)
\end{eqnarray*}
donde $T$ es una variable aleatoria que representa el tiempo hasta el evento. La funci\'on de supervivencia tiene las siguientes propiedades:
\begin{itemize}
    \item $S(0) = 1$: Esto indica que al inicio (tiempo $t=0$), la probabilidad de haber experimentado el evento es cero, por lo tanto, la supervivencia es del 100%.
    \item $\lim_{t \to \infty} S(t) = 0$: A medida que el tiempo tiende al infinito, la probabilidad de que cualquier individuo a\'un no haya experimentado el evento tiende a cero.
    \item $S(t)$ es una funci\'on no creciente: Esto significa que a medida que el tiempo avanza, la probabilidad de supervivencia no aumenta.
\end{itemize}

\section{Funci\'on de Densidad de Probabilidad}
La funci\'on de densidad de probabilidad $f(t)$ describe la probabilidad de que el evento ocurra en un instante de tiempo espec\'ifico. Se define como:
\begin{eqnarray*}
f(t) = \frac{dF(t)}{dt}
\end{eqnarray*}
donde $F(t)$ es la funci\'on de distribuci\'on acumulada, $F(t) = P(T \leq t)$. La relaci\'on entre $S(t)$ y $f(t)$ es:
\begin{eqnarray*}
f(t) = -\frac{dS(t)}{dt}
\end{eqnarray*}

\section{Funci\'on de Riesgo}
La funci\'on de riesgo, $\lambda(t)$, tambi\'en conocida como funci\'on de tasa de fallas o hazard rate, se define como la tasa instant\'anea de ocurrencia del evento en el tiempo $t$, dado que el individuo ha sobrevivido hasta el tiempo $t$. Matem\'aticamente, se expresa como:
\begin{eqnarray*}
\lambda(t) = \lim_{\Delta t \to 0} \frac{P(t \leq T < t + \Delta t \mid T \geq t)}{\Delta t}
\end{eqnarray*}
Esto se puede reescribir usando $f(t)$ y $S(t)$ como:
\begin{eqnarray*}
\lambda(t) = \frac{f(t)}{S(t)}
\end{eqnarray*}

\section{Relaci\'on entre Funci\'on de Supervivencia y Funci\'on de Riesgo}
La funci\'on de supervivencia y la funci\'on de riesgo est\'an relacionadas a trav\'es de la siguiente ecuaci\'on:
\begin{eqnarray*}
S(t) = \exp\left(-\int_0^t \lambda(u) \, du\right)
\end{eqnarray*}
Esta f\'ormula se deriva del hecho de que la funci\'on de supervivencia es la probabilidad acumulativa de no haber experimentado el evento hasta el tiempo $t$, y $\lambda(t)$ es la tasa instant\'anea de ocurrencia del evento.

La funci\'on de riesgo tambi\'en puede ser expresada como:
\begin{eqnarray*}
\lambda(t) = -\frac{d}{dt} \log S(t)
\end{eqnarray*}

\section{Deducci\'on de la Funci\'on de Supervivencia}
La relaci\'on entre la funci\'on de supervivencia y la funci\'on de riesgo se puede deducir integrando la funci\'on de riesgo:
\begin{eqnarray*}
S(t) &=& \exp\left(-\int_0^t \lambda(u) \, du\right) \\
\log S(t) &=& -\int_0^t \lambda(u) \, du \\
\frac{d}{dt} \log S(t) &=& -\lambda(t) \\
\lambda(t) &=& -\frac{d}{dt} \log S(t)
\end{eqnarray*}

\section{Ejemplo de C\'alculo}
Supongamos que tenemos una muestra de tiempos de supervivencia $T_1, T_2, \ldots, T_n$. Podemos estimar la funci\'on de supervivencia emp\'irica como:
\begin{eqnarray*}
\hat{S}(t) = \frac{\text{N\'umero de individuos que sobreviven m\'as all\'a de } t}{\text{N\'umero total de individuos en riesgo en } t}
\end{eqnarray*}
y la funci\'on de riesgo emp\'irica como:
\begin{eqnarray*}
\hat{\lambda}(t) = \frac{\text{N\'umero de eventos en } t}{\text{N\'umero de individuos en riesgo en } t}
\end{eqnarray*}

\section{Conclusi\'on}
El an\'alisis de supervivencia es una herramienta poderosa para analizar datos de tiempo hasta evento. Entender los conceptos b\'asicos como la funci\'on de supervivencia y la funci\'on de riesgo es fundamental para el an\'alisis m\'as avanzado.


\chapter{Funci\'on de Supervivencia y Funci\'on de Riesgo}
\section{Introducci\'on}
Este cap\'itulo profundiza en la definici\'on y propiedades de la funci\'on de supervivencia y la funci\'on de riesgo, dos conceptos fundamentales en el análisis de supervivencia. Entender estas funciones y su relaci\'on es crucial para modelar y analizar datos de tiempo hasta evento.

\section{Funci\'on de Supervivencia}
La funci\'on de supervivencia, $S(t)$, describe la probabilidad de que un individuo sobreviva más allá de un tiempo $t$. Formalmente, se define como:
\begin{eqnarray*}
S(t) = P(T > t)
\end{eqnarray*}
donde $T$ es una variable aleatoria que representa el tiempo hasta el evento.

\subsection{Propiedades de la Funci\'on de Supervivencia}
La funci\'on de supervivencia tiene varias propiedades importantes:
\begin{itemize}
    \item $S(0) = 1$: Indica que la probabilidad de haber experimentado el evento en el tiempo 0 es cero.
    \item $\lim_{t \to \infty} S(t) = 0$: A medida que el tiempo tiende al infinito, la probabilidad de supervivencia tiende a cero.
    \item $S(t)$ es una funci\'on no creciente: A medida que el tiempo avanza, la probabilidad de supervivencia no aumenta.
\end{itemize}

\subsection{Derivaci\'on de $S(t)$}
Si la funci\'on de densidad de probabilidad $f(t)$ del tiempo de supervivencia $T$ es conocida, la funci\'on de supervivencia puede derivarse como:
\begin{eqnarray*}
S(t) &=& P(T > t) \\
     &=& 1 - P(T \leq t) \\
     &=& 1 - F(t) \\
     &=& 1 - \int_0^t f(u) \, du
\end{eqnarray*}
donde $F(t)$ es la funci\'on de distribuci\'on acumulada.

\subsection{Ejemplo de Cálculo de $S(t)$}
Consideremos un ejemplo donde el tiempo de supervivencia $T$ sigue una distribuci\'on exponencial con tasa $\lambda$. La funci\'on de densidad de probabilidad $f(t)$ es:
\begin{eqnarray*}
f(t) = \lambda e^{-\lambda t}, \quad t \geq 0
\end{eqnarray*}
La funci\'on de distribuci\'on acumulada $F(t)$ es:
\begin{eqnarray*}
F(t) = \int_0^t \lambda e^{-\lambda u} \, du = 1 - e^{-\lambda t}
\end{eqnarray*}
Por lo tanto, la funci\'on de supervivencia $S(t)$ es:
\begin{eqnarray*}
S(t) = 1 - F(t) = e^{-\lambda t}
\end{eqnarray*}

\section{Funci\'on de Riesgo}
La funci\'on de riesgo, $\lambda(t)$, proporciona la tasa instant\'anea de ocurrencia del evento en el tiempo $t$, dado que el individuo ha sobrevivido hasta el tiempo $t$. Matem\'aticamente, se define como:
\begin{eqnarray*}
\lambda(t) = \lim_{\Delta t \to 0} \frac{P(t \leq T < t + \Delta t \mid T \geq t)}{\Delta t}
\end{eqnarray*}

\subsection{Relaci\'on entre $\lambda(t)$ y $f(t)$}
La funci\'on de riesgo se puede relacionar con la funci\'on de densidad de probabilidad $f(t)$ y la funci\'on de supervivencia $S(t)$ de la siguiente manera:
\begin{eqnarray*}
\lambda(t) &=& \frac{f(t)}{S(t)}
\end{eqnarray*}

\subsection{Derivaci\'on de $\lambda(t)$}
La derivaci\'on de $\lambda(t)$ se basa en la definici\'on condicional de la probabilidad:
\begin{eqnarray*}
\lambda(t) &=& \lim_{\Delta t \to 0} \frac{P(t \leq T < t + \Delta t \mid T \geq t)}{\Delta t} \\
           &=& \lim_{\Delta t \to 0} \frac{\frac{P(t \leq T < t + \Delta t \text{ y } T \geq t)}{P(T \geq t)}}{\Delta t} \\
           &=& \lim_{\Delta t \to 0} \frac{\frac{P(t \leq T < t + \Delta t)}{P(T \geq t)}}{\Delta t} \\
           &=& \frac{f(t)}{S(t)}
\end{eqnarray*}

\section{Relaci\'on entre Funci\'on de Supervivencia y Funci\'on de Riesgo}
La funci\'on de supervivencia y la funci\'on de riesgo est\'an estrechamente relacionadas. La relaci\'on se expresa mediante la siguiente ecuaci\'on:
\begin{eqnarray*}
S(t) = \exp\left(-\int_0^t \lambda(u) \, du\right)
\end{eqnarray*}

\subsection{Deducci\'on de la Relaci\'on}
Para deducir esta relaci\'on, consideramos la derivada logar\'itmica de la funci\'on de supervivencia:
\begin{eqnarray*}
S(t) &=& \exp\left(-\int_0^t \lambda(u) \, du\right) \\
\log S(t) &=& -\int_0^t \lambda(u) \, du \\
\frac{d}{dt} \log S(t) &=& -\lambda(t) \\
\lambda(t) &=& -\frac{d}{dt} \log S(t)
\end{eqnarray*}

\section{Interpretaci\'on de la Funci\'on de Riesgo}
La funci\'on de riesgo, $\lambda(t)$, se interpreta como la tasa instant\'anea de ocurrencia del evento por unidad de tiempo, dado que el individuo ha sobrevivido hasta el tiempo $t$. Es una medida local del riesgo de falla en un instante espec\'ifico.

\subsection{Ejemplo de C\'alculo de $\lambda(t)$}
Consideremos nuevamente el caso donde el tiempo de supervivencia $T$ sigue una distribuci\'on exponencial con tasa $\lambda$. La funci\'on de densidad de probabilidad $f(t)$ es:
\begin{eqnarray*}
f(t) = \lambda e^{-\lambda t}
\end{eqnarray*}
La funci\'on de supervivencia $S(t)$ es:
\begin{eqnarray*}
S(t) = e^{-\lambda t}
\end{eqnarray*}
La funci\'on de riesgo $\lambda(t)$ se calcula como:
\begin{eqnarray*}
\lambda(t) &=& \frac{f(t)}{S(t)} \\
           &=& \frac{\lambda e^{-\lambda t}}{e^{-\lambda t}} \\
           &=& \lambda
\end{eqnarray*}
En este caso, $\lambda(t)$ es constante y igual a $\lambda$, lo que es una caracter\'istica de la distribuci\'on exponencial.

\section{Funciones de Riesgo Acumulada y Media Residual}
La funci\'on de riesgo acumulada $H(t)$ se define como:
\begin{eqnarray*}
H(t) = \int_0^t \lambda(u) \, du
\end{eqnarray*}
Esta funci\'on proporciona la suma acumulada de la tasa de riesgo hasta el tiempo $t$.

La funci\'on de vida media residual $e(t)$ se define como la esperanza del tiempo de vida restante dado que el individuo ha sobrevivido hasta el tiempo $t$:
\begin{eqnarray*}
e(t) = \mathbb{E}[T - t \mid T > t] = \int_t^\infty S(u) \, du
\end{eqnarray*}

\section{Ejemplo de C\'alculo de Funci\'on de Riesgo Acumulada y Vida Media Residual}
Consideremos nuevamente la distribuci\'on exponencial con tasa $\lambda$. La funci\'on de riesgo acumulada $H(t)$ es:
\begin{eqnarray*}
H(t) &=& \int_0^t \lambda \, du \\
     &=& \lambda t
\end{eqnarray*}

La funci\'on de vida media residual $e(t)$ es:
\begin{eqnarray*}
e(t) &=& \int_t^\infty e^{-\lambda u} \, du \\
     &=& \left[ \frac{-1}{\lambda} e^{-\lambda u} \right]_t^\infty \\
     &=& \frac{1}{\lambda} e^{-\lambda t} \\
     &=& \frac{1}{\lambda}
\end{eqnarray*}
En este caso, la vida media residual es constante e igual a $\frac{1}{\lambda}$, otra caracter\'istica de la distribuci\'on exponencial.

\section{Conclusi\'on}
La funci\'on de supervivencia y la funci\'on de riesgo son herramientas fundamentales en el an\'alisis de supervivencia. Entender su definici\'on, propiedades, y la relaci\'on entre ellas es esencial para modelar y analizar correctamente los datos de tiempo hasta evento. Las funciones de riesgo acumulada y vida media residual proporcionan informaci\'on adicional sobre la din\'amica del riesgo a lo largo del tiempo.



\chapter{Estimador de Kaplan-Meier}

\section{Introducci\'on}
El estimador de Kaplan-Meier, tambi\'en conocido como la funci\'on de supervivencia emp\'irica, es una herramienta no param\'etrica para estimar la funci\'on de supervivencia a partir de datos censurados. Este m\'etodo es especialmente \'util cuando los tiempos de evento están censurados a la derecha.

\section{Definici\'on del Estimador de Kaplan-Meier}
El estimador de Kaplan-Meier se define como:
\begin{eqnarray*}
\hat{S}(t) = \prod_{t_i \leq t} \left(1 - \frac{d_i}{n_i}\right)
\end{eqnarray*}
donde:
\begin{itemize}
    \item $t_i$ es el tiempo del $i$-\'esimo evento,
    \item $d_i$ es el n\'umero de eventos que ocurren en $t_i$,
    \item $n_i$ es el n\'umero de individuos en riesgo justo antes de $t_i$.
\end{itemize}

\section{Propiedades del Estimador de Kaplan-Meier}
El estimador de Kaplan-Meier tiene las siguientes propiedades:
\begin{itemize}
    \item Es una funci\'on escalonada que disminuye en los tiempos de los eventos observados.
    \item Puede manejar datos censurados a la derecha.
    \item Proporciona una estimaci\'on no param\'etrica de la funci\'on de supervivencia.
\end{itemize}

\subsection{Funci\'on Escalonada}
La funci\'on escalonada del estimador de Kaplan-Meier significa que $\hat{S}(t)$ permanece constante entre los tiempos de los eventos y disminuye en los tiempos de los eventos. Matem\'aticamente, si $t_i$ es el tiempo del $i$-\'esimo evento, entonces:
\begin{eqnarray*}
\hat{S}(t) = \hat{S}(t_i) \quad \text{para} \ t_i \leq t < t_{i+1}
\end{eqnarray*}

\subsection{Manejo de Datos Censurados}
El estimador de Kaplan-Meier maneja datos censurados a la derecha al ajustar la estimaci\'on de la funci\'on de supervivencia s\'olo en los tiempos en que ocurren eventos. Si un individuo es censurado antes de experimentar el evento, no contribuye a la disminuci\'on de $\hat{S}(t)$ en el tiempo de censura. Esto asegura que la censura no sesga la estimaci\'on de la supervivencia.

\subsection{Estimaci\'on No Param\'etrica}
El estimador de Kaplan-Meier es no param\'etrico porque no asume ninguna forma espec\'ifica para la distribuci\'on de los tiempos de supervivencia. En cambio, utiliza la informaci\'on emp\'irica disponible para estimar la funci\'on de supervivencia.

\section{Deducci\'on del Estimador de Kaplan-Meier}
La deducci\'on del estimador de Kaplan-Meier se basa en el principio de probabilidad condicional. Consideremos un conjunto de tiempos de supervivencia observados $t_1, t_2, \ldots, t_k$ con eventos en cada uno de estos tiempos. El estimador de la probabilidad de supervivencia m\'as all\'a del tiempo $t$ es el producto de las probabilidades de sobrevivir m\'as all\'a de cada uno de los tiempos de evento observados hasta $t$.

\subsection{Probabilidad Condicional}
La probabilidad de sobrevivir m\'as all\'a de $t_i$, dado que el individuo ha sobrevivido justo antes de $t_i$, es:
\begin{eqnarray*}
P(T > t_i \mid T \geq t_i) = 1 - \frac{d_i}{n_i}
\end{eqnarray*}
donde $d_i$ es el n\'umero de eventos en $t_i$ y $n_i$ es el n\'umero de individuos en riesgo justo antes de $t_i$.

\subsection{Producto de Probabilidades Condicionales}
La probabilidad de sobrevivir m\'as all\'a de un tiempo $t$ cualquiera, dada la secuencia de tiempos de evento, es el producto de las probabilidades condicionales de sobrevivir m\'as all\'a de cada uno de los tiempos de evento observados hasta $t$. As\'i, el estimador de Kaplan-Meier se obtiene como:
\begin{eqnarray*}
\hat{S}(t) = \prod_{t_i \leq t} \left(1 - \frac{d_i}{n_i}\right)
\end{eqnarray*}

\section{Ejemplo de C\'alculo}
Supongamos que tenemos los siguientes tiempos de supervivencia observados para cinco individuos: 2, 3, 5, 7, 8. Supongamos adem\'as que tenemos censura a la derecha en el tiempo 10. Los tiempos de evento y el n\'umero de individuos en riesgo justo antes de cada evento son:

\begin{table}[h]
\centering
\begin{tabular}{|c|c|c|}
\hline
Tiempo ($t_i$) & Eventos ($d_i$) & En Riesgo ($n_i$) \\
\hline
2 & 1 & 5 \\
3 & 1 & 4 \\
5 & 1 & 3 \\
7 & 1 & 2 \\
8 & 1 & 1 \\
\hline
\end{tabular}
\caption{Ejemplo de c\'alculo del estimador de Kaplan-Meier}
\end{table}

Usando estos datos, el estimador de Kaplan-Meier se calcula como:
\begin{eqnarray*}
\hat{S}(2) &=& 1 - \frac{1}{5} = 0.8 \\
\hat{S}(3) &=& 0.8 \times \left(1 - \frac{1}{4}\right) = 0.8 \times 0.75 = 0.6 \\
\hat{S}(5) &=& 0.6 \times \left(1 - \frac{1}{3}\right) = 0.6 \times 0.6667 = 0.4 \\
\hat{S}(7) &=& 0.4 \times \left(1 - \frac{1}{2}\right) = 0.4 \times 0.5 = 0.2 \\
\hat{S}(8) &=& 0.2 \times \left(1 - \frac{1}{1}\right) = 0.2 \times 0 = 0 \\
\end{eqnarray*}

\section{Intervalos de Confianza para el Estimador de Kaplan-Meier}
Para calcular intervalos de confianza para el estimador de Kaplan-Meier, se puede usar la transformaci\'on logar\'itmica y la aproximaci\'on normal. Un intervalo de confianza aproximado para $\log(-\log(\hat{S}(t)))$ se obtiene como:
\begin{eqnarray*}
\log(-\log(\hat{S}(t))) \pm z_{\alpha/2} \sqrt{\frac{1}{d_i(n_i - d_i)}}
\end{eqnarray*}
donde $z_{\alpha/2}$ es el percentil correspondiente de la distribuci\'on normal est\'andar.

\section{Transformaci\'on Logar\'itmica Inversa}
La transformaci\'on logar\'itmica inversa se utiliza para obtener los l\'imites del intervalo de confianza para $S(t)$:
\begin{eqnarray*}
\hat{S}(t) = \exp\left(-\exp\left(\log(-\log(\hat{S}(t))) \pm z_{\alpha/2} \sqrt{\frac{1}{d_i(n_i - d_i)}}\right)\right)
\end{eqnarray*}

\section{C\'alculo Detallado de Intervalos de Confianza}
Para un c\'alculo m\'as detallado de los intervalos de confianza, consideremos un tiempo espec\'ifico $t_j$. La varianza del estimador de Kaplan-Meier en $t_j$ se puede estimar usando Greenwood's formula:
\begin{eqnarray*}
\text{Var}(\hat{S}(t_j)) = \hat{S}(t_j)^2 \sum_{t_i \leq t_j} \frac{d_i}{n_i(n_i - d_i)}
\end{eqnarray*}
El intervalo de confianza aproximado para $\hat{S}(t_j)$ es entonces:
\begin{eqnarray*}
\hat{S}(t_j) \pm z_{\alpha/2} \sqrt{\text{Var}(\hat{S}(t_j))}
\end{eqnarray*}

\section{Ejemplo de Intervalo de Confianza}
Supongamos que en el ejemplo anterior queremos calcular el intervalo de confianza para $\hat{S}(3)$. Primero, calculamos la varianza:
\begin{eqnarray*}
\text{Var}(\hat{S}(3)) &=& \hat{S}(3)^2 \left( \frac{1}{5 \times 4} + \frac{1}{4 \times 3} \right) \\
                       &=& 0.6^2 \left( \frac{1}{20} + \frac{1}{12} \right) \\
                       &=& 0.36 \left( 0.05 + 0.0833 \right) \\
                       &=& 0.36 \times 0.1333 \\
                       &=& 0.048
\end{eqnarray*}
El intervalo de confianza es entonces:
\begin{eqnarray*}
0.6 \pm 1.96 \sqrt{0.048} = 0.6 \pm 1.96 \times 0.219 = 0.6 \pm 0.429
\end{eqnarray*}
Por lo tanto, el intervalo de confianza para $\hat{S}(3)$ es aproximadamente $(0.171, 1.029)$. Dado que una probabilidad no puede exceder 1, ajustamos el intervalo a $(0.171, 1.0)$.

\section{Interpretaci\'on del Estimador de Kaplan-Meier}
El estimador de Kaplan-Meier proporciona una estimaci\'on emp\'irica de la funci\'on de supervivencia que es f\'acil de interpretar y calcular. Su capacidad para manejar datos censurados lo hace especialmente \'util en estudios de supervivencia.

\section{Conclusi\'on}
El estimador de Kaplan-Meier es una herramienta poderosa para estimar la funci\'on de supervivencia en presencia de datos censurados. Su c\'alculo es relativamente sencillo y proporciona una estimaci\'on no param\'etrica robusta de la supervivencia a lo largo del tiempo. La interpretaci\'on adecuada de este estimador y su intervalo de confianza asociado es fundamental para el an\'alisis de datos de supervivencia.



\chapter{Comparaci\'on de Curvas de Supervivencia}

\section{Introducci\'on}
Comparar curvas de supervivencia es crucial para determinar si existen diferencias significativas en las tasas de supervivencia entre diferentes grupos. Las pruebas de hip\'otesis, como el test de log-rank, son herramientas comunes para esta comparaci\'on.

\section{Test de Log-rank}
El test de log-rank se utiliza para comparar las curvas de supervivencia de dos o más grupos. La hip\'otesis nula es que no hay diferencia en las funciones de riesgo entre los grupos.

\subsection{F\'ormula del Test de Log-rank}
El estad\'istico del test de log-rank se define como:
\begin{eqnarray*}
\chi^2 = \frac{\left(\sum_{i=1}^k (O_i - E_i)\right)^2}{\sum_{i=1}^k V_i}
\end{eqnarray*}
donde:
\begin{itemize}
    \item $O_i$ es el n\'umero observado de eventos en el grupo $i$.
    \item $E_i$ es el n\'umero esperado de eventos en el grupo $i$.
    \item $V_i$ es la varianza del n\'umero de eventos en el grupo $i$.
\end{itemize}

\subsection{Cálculo de $E_i$ y $V_i$}
El n\'umero esperado de eventos $E_i$ y la varianza $V_i$ se calculan como:
\begin{eqnarray*}
E_i &=& \frac{d_i \cdot n_i}{n} \\
V_i &=& \frac{d_i \cdot (n - d_i) \cdot n_i \cdot (n - n_i)}{n^2 \cdot (n - 1)}
\end{eqnarray*}
donde:
\begin{itemize}
    \item $d_i$ es el n\'umero total de eventos en el grupo $i$.
    \item $n_i$ es el n\'umero de individuos en riesgo en el grupo $i$.
    \item $n$ es el n\'umero total de individuos en todos los grupos.
\end{itemize}

\section{Ejemplo de C\'alculo del Test de Log-rank}
Supongamos que tenemos dos grupos con los siguientes datos de eventos:

\begin{table}[h]
\centering
\begin{tabular}{|c|c|c|c|}
\hline
Grupo & Tiempo ($t_i$) & Eventos ($O_i$) & En Riesgo ($n_i$) \\
\hline
1 & 2 & 1 & 5 \\
1 & 4 & 1 & 4 \\
2 & 3 & 1 & 4 \\
2 & 5 & 1 & 3 \\
\hline
\end{tabular}
\caption{Ejemplo de datos para el test de log-rank}
\end{table}

Calculemos $E_i$ y $V_i$ para cada grupo:

\begin{eqnarray*}
E_1 &=& \frac{2 \cdot 5}{9} + \frac{2 \cdot 4}{8} = \frac{10}{9} + \frac{8}{8} = 1.11 + 1 = 2.11 \\
V_1 &=& \frac{2 \cdot 7 \cdot 5 \cdot 4}{81 \cdot 8} = \frac{2 \cdot 7 \cdot 5 \cdot 4}{648} = \frac{280}{648} = 0.432 \\
E_2 &=& \frac{2 \cdot 4}{9} + \frac{2 \cdot 3}{8} = \frac{8}{9} + \frac{6}{8} = 0.89 + 0.75 = 1.64 \\
V_2 &=& \frac{2 \cdot 7 \cdot 4 \cdot 4}{81 \cdot 8} = \frac{2 \cdot 7 \cdot 4 \cdot 4}{648} = \frac{224}{648} = 0.346 \\
\end{eqnarray*}

El estad\'istico de log-rank se calcula como:
\begin{eqnarray*}
\chi^2 &=& \frac{\left((1 - 2.11) + (1 - 1.64)\right)^2}{0.432 + 0.346} \\
       &=& \frac{\left(-1.11 - 0.64\right)^2}{0.778} \\
       &=& \frac{3.04}{0.778} \\
       &=& 3.91
\end{eqnarray*}

El valor p se puede obtener comparando $\chi^2$ con una distribuci\'on $\chi^2$ con un grado de libertad (dado que estamos comparando dos grupos).

\section{Interpretaci\'on del Test de Log-rank}
Un valor p peque\~no (generalmente menos de 0.05) indica que hay una diferencia significativa en las curvas de supervivencia entre los grupos. Un valor p grande sugiere que no hay suficiente evidencia para rechazar la hip\'otesis nula de que las curvas de supervivencia son iguales.

\section{Pruebas Alternativas}
Adem\'as del test de log-rank, existen otras pruebas para comparar curvas de supervivencia, como el test de Wilcoxon (Breslow), que da m\'as peso a los eventos en tiempos tempranos.

\section{Conclusi\'on}
El test de log-rank es una herramienta esencial para comparar curvas de supervivencia entre diferentes grupos. Su c\'alculo se basa en la diferencia entre los eventos observados y esperados en cada grupo, y su interpretaci\'on puede ayudar a identificar diferencias significativas en la supervivencia.



\chapter{Modelos de Riesgos Proporcionales de Cox}

\section{Introducci\'on}
El modelo de riesgos proporcionales de Cox, propuesto por David Cox en 1972, es una de las herramientas más utilizadas en el análisis de supervivencia. Este modelo permite evaluar el efecto de varias covariables en el tiempo hasta el evento, sin asumir una forma espec\'ifica para la distribuci\'on de los tiempos de supervivencia.

\section{Definici\'on del Modelo de Cox}
El modelo de Cox se define como:
\begin{eqnarray*}
\lambda(t \mid X) = \lambda_0(t) \exp(\beta^T X)
\end{eqnarray*}
donde:
\begin{itemize}
    \item $\lambda(t \mid X)$ es la funci\'on de riesgo en el tiempo $t$ dado el vector de covariables $X$.
    \item $\lambda_0(t)$ es la funci\'on de riesgo basal en el tiempo $t$.
    \item $\beta$ es el vector de coeficientes del modelo.
    \item $X$ es el vector de covariables.
\end{itemize}

\section{Supuesto de Proporcionalidad de Riesgos}
El modelo de Cox asume que las razones de riesgo entre dos individuos son constantes a lo largo del tiempo. Matemáticamente, si $X_i$ y $X_j$ son las covariables de dos individuos, la raz\'on de riesgos se expresa como:
\begin{eqnarray*}
\frac{\lambda(t \mid X_i)}{\lambda(t \mid X_j)} = \frac{\lambda_0(t) \exp(\beta^T X_i)}{\lambda_0(t) \exp(\beta^T X_j)} = \exp(\beta^T (X_i - X_j))
\end{eqnarray*}

\section{Estimaci\'on de los Par\'ametros}
Los par\'ametros $\beta$ se estiman utilizando el m\'etodo de m\'axima verosimilitud parcial. La funci\'on de verosimilitud parcial se define como:
\begin{eqnarray*}
L(\beta) = \prod_{i=1}^k \frac{\exp(\beta^T X_i)}{\sum_{j \in R(t_i)} \exp(\beta^T X_j)}
\end{eqnarray*}
donde $R(t_i)$ es el conjunto de individuos en riesgo en el tiempo $t_i$.

\subsection{Funci\'on de Log-Verosimilitud Parcial}
La funci\'on de log-verosimilitud parcial es:
\begin{eqnarray*}
\log L(\beta) = \sum_{i=1}^k \left(\beta^T X_i - \log \sum_{j \in R(t_i)} \exp(\beta^T X_j)\right)
\end{eqnarray*}

\subsection{Derivadas Parciales y Maximizaci\'on}
Para encontrar los estimadores de m\'axima verosimilitud, resolvemos el sistema de ecuaciones obtenido al igualar a cero las derivadas parciales de $\log L(\beta)$ con respecto a $\beta$:
\begin{eqnarray*}
\frac{\partial \log L(\beta)}{\partial \beta} = \sum_{i=1}^k \left(X_i - \frac{\sum_{j \in R(t_i)} X_j \exp(\beta^T X_j)}{\sum_{j \in R(t_i)} \exp(\beta^T X_j)}\right) = 0
\end{eqnarray*}

\section{Interpretaci\'on de los Coeficientes}
Cada coeficiente $\beta_i$ representa el logaritmo de la raz\'on de riesgos asociado con un incremento unitario en la covariable $X_i$. Un valor positivo de $\beta_i$ indica que un aumento en $X_i$ incrementa el riesgo del evento, mientras que un valor negativo indica una reducci\'on del riesgo.

\section{Evaluaci\'on del Modelo}
El modelo de Cox se eval\'ua utilizando varias t\'ecnicas, como el an\'alisis de residuos de Schoenfeld para verificar el supuesto de proporcionalidad de riesgos, y el uso de curvas de supervivencia estimadas para evaluar la bondad de ajuste.

\subsection{Residuos de Schoenfeld}
Los residuos de Schoenfeld se utilizan para evaluar la proporcionalidad de riesgos. Para cada evento en el tiempo $t_i$, el residuo de Schoenfeld para la covariable $X_j$ se define como:
\begin{eqnarray*}
r_{ij} = X_{ij} - \hat{X}_{ij}
\end{eqnarray*}
donde $\hat{X}_{ij}$ es la covariable ajustada.

\subsection{Curvas de Supervivencia Ajustadas}
Las curvas de supervivencia ajustadas se obtienen utilizando la funci\'on de riesgo basal estimada y los coeficientes del modelo. La funci\'on de supervivencia ajustada se define como:
\begin{eqnarray*}
\hat{S}(t \mid X) = \hat{S}_0(t)^{\exp(\beta^T X)}
\end{eqnarray*}
donde $\hat{S}_0(t)$ es la funci\'on de supervivencia basal estimada.

\section{Ejemplo de Aplicaci\'on del Modelo de Cox}
Consideremos un ejemplo con tres covariables: edad, sexo y tratamiento. Supongamos que los datos se ajustan a un modelo de Cox y obtenemos los siguientes coeficientes:
\begin{eqnarray*}
\hat{\beta}_{edad} = 0.02, \quad \hat{\beta}_{sexo} = -0.5, \quad \hat{\beta}_{tratamiento} = 1.2
\end{eqnarray*}

La funci\'on de riesgo ajustada se expresa como:
\begin{eqnarray*}
\lambda(t \mid X) = \lambda_0(t) \exp(0.02 \cdot \text{edad} - 0.5 \cdot \text{sexo} + 1.2 \cdot \text{tratamiento})
\end{eqnarray*}

\section{Conclusi\'on}
El modelo de riesgos proporcionales de Cox es una herramienta poderosa para analizar datos de supervivencia con m\'ultiples covariables. Su flexibilidad y la falta de suposiciones fuertes sobre la distribuci\'on de los tiempos de supervivencia lo hacen ampliamente aplicable en diversas disciplinas.



\chapter{Diagn\'ostico y Validaci\'on de Modelos de Cox}

\section{Introducci\'on}
Una vez ajustado un modelo de Cox, es crucial realizar diagn\'osticos y validaciones para asegurar que el modelo es apropiado y que los supuestos subyacentes son válidos. Esto incluye la verificaci\'on del supuesto de proporcionalidad de riesgos y la evaluaci\'on del ajuste del modelo.

\section{Supuesto de Proporcionalidad de Riesgos}
El supuesto de proporcionalidad de riesgos implica que la raz\'on de riesgos entre dos individuos es constante a lo largo del tiempo. Si este supuesto no se cumple, las inferencias hechas a partir del modelo pueden ser incorrectas.

\subsection{Residuos de Schoenfeld}
Los residuos de Schoenfeld se utilizan para evaluar la proporcionalidad de riesgos. Para cada evento en el tiempo $t_i$, el residuo de Schoenfeld para la covariable $X_j$ se define como:
\begin{eqnarray*}
r_{ij} = X_{ij} - \hat{X}_{ij}
\end{eqnarray*}
donde $\hat{X}_{ij}$ es la covariable ajustada. Si los residuos de Schoenfeld no muestran una tendencia sistemática cuando se trazan contra el tiempo, el supuesto de proporcionalidad de riesgos es razonable.

\section{Bondad de Ajuste}
La bondad de ajuste del modelo de Cox se eval\'ua comparando las curvas de supervivencia observadas y ajustadas, y utilizando estad\'isticas de ajuste global.

\subsection{Curvas de Supervivencia Ajustadas}
Las curvas de supervivencia aaustadas se obtienen utilizando la funci\'on de riesgo basal estimada y los coeficientes del modelo. La funci\'on de supervivencia ajustada se define como:
\begin{eqnarray*}
\hat{S}(t \mid X) = \hat{S}_0(t)^{\exp(\beta^T X)}
\end{eqnarray*}
donde $\hat{S}_0(t)$ es la funci\'on de supervivencia basal estimada. Comparar estas curvas con las curvas de Kaplan-Meier para diferentes niveles de las covariables puede proporcionar una validaci\'on visual del ajuste del modelo.

\subsection{Estad\'isticas de Ajuste Global}
Las estad\'isticas de ajuste global, como el test de la desviaci\'on y el test de la bondad de ajuste de Grambsch y Therneau, se utilizan para evaluar el ajuste global del modelo de Cox.

\section{Diagn\'ostico de Influencia}
El diagn\'ostico de influencia identifica observaciones individuales que tienen un gran impacto en los estimados del modelo. Los residuos de devianza y los residuos de martingala se utilizan com\'unmente para este prop\'osito.

\subsection{Residuos de Deviance}
Los residuos de deviance se definen como:
\begin{eqnarray*}
D_i = \text{sign}(O_i - E_i) \sqrt{-2 \left(O_i \log \frac{O_i}{E_i} - (O_i - E_i)\right)}
\end{eqnarray*}
donde $O_i$ es el n\'umero observado de eventos y $E_i$ es el n\'umero esperado de eventos. Observaciones con residuos de deviance grandes en valor absoluto pueden ser influyentes.

\subsection{Residuos de Martingala}
Los residuos de martingala se definen como:
\begin{eqnarray*}
M_i = O_i - E_i
\end{eqnarray*}
donde $O_i$ es el n\'umero observado de eventos y $E_i$ es el n\'umero esperado de eventos. Los residuos de martingala se utilizan para detectar observaciones que no se ajustan bien al modelo.

\section{Ejemplo de Diagn\'ostico}
Consideremos un modelo de Cox ajustado con las covariables edad, sexo y tratamiento. Para diagnosticar la influencia de observaciones individuales, calculamos los residuos de deviance y martingala para cada observaci\'on.

\begin{table}[h]
\centering
\begin{tabular}{|c|c|c|c|c|}
\hline
Observaci\'on & Edad & Sexo & Tratamiento & Residuo de Deviance \\
\hline
1 & 50 & 0 & 1 & 1.2 \\
2 & 60 & 1 & 0 & -0.5 \\
3 & 45 & 0 & 1 & -1.8 \\
4 & 70 & 1 & 0 & 0.3 \\
\hline
\end{tabular}
\caption{Residuos de deviance para observaciones individuales}
\end{table}

Observaciones con residuos de deviance grandes en valor absoluto (como la observaci\'on 3) pueden ser influyentes y requieren una revisi\'on adicional.

\section{Conclusi\'on}
El diagn\'ostico y la validaci\'on son pasos cr\'iticos en el an\'slisis de modelos de Cox. Evaluar el supuesto de proporcionalidad de riesgos, la bondad de ajuste y la influencia de observaciones individuales asegura que las inferencias y conclusiones derivadas del modelo sean v\'slidas y fiables.



\chapter{Modelos Acelerados de Fallos}
\section{Introducci\'on}
Los modelos acelerados de fallos (AFT) son una alternativa a los modelos de riesgos proporcionales de Cox. En lugar de asumir que las covariables afectan la tasa de riesgo, los modelos AFT asumen que las covariables multiplican el tiempo de supervivencia por una constante.

\section{Definici\'on del Modelo AFT}
Un modelo AFT se expresa como:
\begin{eqnarray*}
T = T_0 \exp(\beta^T X)
\end{eqnarray*}
donde:
\begin{itemize}
    \item $T$ es el tiempo de supervivencia observado.
    \item $T_0$ es el tiempo de supervivencia bajo condiciones basales.
    \item $\beta$ es el vector de coeficientes del modelo.
    \item $X$ es el vector de covariables.
\end{itemize}

\subsection{Transformaci\'on Logar\'itmica}
El modelo AFT se puede transformar logar\'itmicamente para obtener una forma lineal:
\begin{eqnarray*}
\log(T) = \log(T_0) + \beta^T X
\end{eqnarray*}

\section{Estimaci\'on de los Parámetros}
Los parámetros del modelo AFT se estiman utilizando el m\'etodo de máxima verosimilitud. La funci\'on de verosimilitud se define como:
\begin{eqnarray*}
L(\beta) = \prod_{i=1}^n f(t_i \mid X_i; \beta)
\end{eqnarray*}
donde $f(t_i \mid X_i; \beta)$ es la funci\'on de densidad de probabilidad del tiempo de supervivencia $t_i$ dado el vector de covariables $X_i$ y los par\'ametros $\beta$.

\subsection{Funci\'on de Log-Verosimilitud}
La funci\'on de log-verosimilitud es:
\begin{eqnarray*}
\log L(\beta) = \sum_{i=1}^n \log f(t_i \mid X_i; \beta)
\end{eqnarray*}

\subsection{Maximizaci\'on de la Verosimilitud}
Los estimadores de m\'axima verosimilitud se obtienen resolviendo el sistema de ecuaciones obtenido al igualar a cero las derivadas parciales de $\log L(\beta)$ con respecto a $\beta$:
\begin{eqnarray*}
\frac{\partial \log L(\beta)}{\partial \beta} = 0
\end{eqnarray*}

\section{Distribuciones Comunes en Modelos AFT}
En los modelos AFT, el tiempo de supervivencia $T$ puede seguir varias distribuciones comunes, como la exponencial, Weibull, log-normal y log-log\'istica. Cada una de estas distribuciones tiene diferentes propiedades y aplicaciones.

\subsection{Modelo Exponencial AFT}
En un modelo exponencial AFT, el tiempo de supervivencia $T$ sigue una distribuci\'on exponencial con par\'ametro $\lambda$:
\begin{eqnarray*}
f(t) = \lambda \exp(-\lambda t)
\end{eqnarray*}
La funci\'on de supervivencia es:
\begin{eqnarray*}
S(t) = \exp(-\lambda t)
\end{eqnarray*}
La transformaci\'on logar\'itmica del tiempo de supervivencia es:
\begin{eqnarray*}
\log(T) = \log\left(\frac{1}{\lambda}\right) + \beta^T X
\end{eqnarray*}

\subsection{Modelo Weibull AFT}
En un modelo Weibull AFT, el tiempo de supervivencia $T$ sigue una distribuci\'on Weibull con par\'ametros $\lambda$ y $k$:
\begin{eqnarray*}
f(t) = \lambda k t^{k-1} \exp(-\lambda t^k)
\end{eqnarray*}
La funci\'on de supervivencia es:
\begin{eqnarray*}
S(t) = \exp(-\lambda t^k)
\end{eqnarray*}
La transformaci\'on logar\'itmica del tiempo de supervivencia es:
\begin{eqnarray*}
\log(T) = \log\left(\left(\frac{1}{\lambda}\right)^{1/k}\right) + \frac{\beta^T X}{k}
\end{eqnarray*}

\section{Interpretaci\'on de los Coeficientes}
En los modelos AFT, los coeficientes $\beta_i$ se interpretan como factores multiplicativos del tiempo de supervivencia. Un valor positivo de $\beta_i$ indica que un aumento en la covariable $X_i$ incrementa el tiempo de supervivencia, mientras que un valor negativo indica una reducci\'on del tiempo de supervivencia.

\section{Ejemplo de Aplicaci\'on del Modelo AFT}
Consideremos un ejemplo con tres covariables: edad, sexo y tratamiento. Supongamos que los datos se ajustan a un modelo Weibull AFT y obtenemos los siguientes coeficientes:
\begin{eqnarray*}
\hat{\beta}_{edad} = -0.02, \quad \hat{\beta}_{sexo} = 0.5, \quad \hat{\beta}_{tratamiento} = -1.2
\end{eqnarray*}

La funci\'on de supervivencia ajustada se expresa como:
\begin{eqnarray*}
S(t \mid X) = \exp\left(-\left(\frac{t \exp(-0.02 \cdot \text{edad} + 0.5 \cdot \text{sexo} - 1.2 \cdot \text{tratamiento})}{\lambda}\right)^k\right)
\end{eqnarray*}

\section{Conclusi\'on}
Los modelos AFT proporcionan una alternativa flexible a los modelos de riesgos proporcionales de Cox. Su enfoque en la multiplicaci\'on del tiempo de supervivencia por una constante permite una interpretaci\'on intuitiva y aplicaciones en diversas \'areas.



\chapter{An\'alisis Multivariado de Supervivencia}

\section{Introducci\'on}
El análisis multivariado de supervivencia extiende los modelos de supervivencia para incluir m\'ultiples covariables, permitiendo evaluar su efecto simultáneo sobre el tiempo hasta el evento. Los modelos de Cox y AFT son com\'unmente utilizados en este contexto.

\section{Modelo de Cox Multivariado}
El modelo de Cox multivariado se define como:
\begin{eqnarray*}
\lambda(t \mid X) = \lambda_0(t) \exp(\beta^T X)
\end{eqnarray*}
donde $X$ es un vector de covariables.

\subsection{Estimaci\'on de los Parámetros}
Los parámetros $\beta$ se estiman utilizando el m\'etodo de máxima verosimilitud parcial, como se discuti\'o anteriormente. La funci\'on de verosimilitud parcial se maximiza para obtener los estimadores de los coeficientes.

\section{Modelo AFT Multivariado}
El modelo AFT multivariado se expresa como:
\begin{eqnarray*}
T = T_0 \exp(\beta^T X)
\end{eqnarray*}

\subsection{Estimaci\'on de los Par\'ametros}
Los par\'ametros $\beta$ se estiman utilizando el m\'etodo de m\'axima verosimilitud, similar al caso univariado. La funci\'on de verosimilitud se maximiza para obtener los estimadores de los coeficientes.

\section{Interacci\'on y Efectos No Lineales}
En el an\'alisis multivariado, es importante considerar la posibilidad de interacciones entre covariables y efectos no lineales. Estos se pueden incluir en los modelos extendiendo las funciones de riesgo o supervivencia.

\subsection{Interacciones}
Las interacciones entre covariables se pueden modelar a\~nadiendo t\'erminos de interacci\'on en el modelo:
\begin{eqnarray*}
\lambda(t \mid X) = \lambda_0(t) \exp(\beta_1 X_1 + \beta_2 X_2 + \beta_3 X_1 X_2)
\end{eqnarray*}
donde $X_1 X_2$ es el t\'ermino de interacci\'on.

\subsection{Efectos No Lineales}
Los efectos no lineales se pueden modelar utilizando funciones no lineales de las covariables, como polinomios o splines:
\begin{eqnarray*}
\lambda(t \mid X) = \lambda_0(t) \exp(\beta_1 X + \beta_2 X^2)
\end{eqnarray*}

\section{Selecci\'on de Variables}
La selecci\'on de variables es crucial en el an\'alisis multivariado para evitar el sobreajuste y mejorar la interpretabilidad del modelo. M\'etodos como la regresi\'on hacia atr\'as, la regresi\'on hacia adelante y la selecci\'on por criterios de informaci\'on (AIC, BIC) son com\'unmente utilizados.

\subsection{Regresi\'on Hacia Atr\'as}
La regresi\'on hacia atr\'as comienza con todas las covariables en el modelo y elimina iterativamente la covariable menos significativa hasta que todas las covariables restantes sean significativas.

\subsection{Regresi\'on Hacia Adelante}
La regresi\'on hacia adelante comienza con un modelo vac\'io y a\~nade iterativamente la covariable m\'as significativa hasta que no se pueda a\~nadir ninguna covariable adicional significativa.

\subsection{Criterios de Informaci\'on}
Los criterios de informaci\'on, como el AIC (Akaike Information Criterion) y el BIC (Bayesian Information Criterion), se utilizan para seleccionar el modelo que mejor se ajusta a los datos con la menor complejidad posible:
\begin{eqnarray*}
AIC &=& -2 \log L + 2k \\
BIC &=& -2 \log L + k \log n
\end{eqnarray*}
donde $L$ es la funci\'on de verosimilitud del modelo, $k$ es el n\'umero de par\'ametros en el modelo y $n$ es el tama\~no de la muestra.

\section{Ejemplo de An\'alisis Multivariado}
Consideremos un ejemplo con tres covariables: edad, sexo y tratamiento. Ajustamos un modelo de Cox multivariado y obtenemos los siguientes coeficientes:
\begin{eqnarray*}
\hat{\beta}_{edad} = 0.03, \quad \hat{\beta}_{sexo} = -0.6, \quad \hat{\beta}_{tratamiento} = 1.5
\end{eqnarray*}

La funci\'on de riesgo ajustada se expresa como:
\begin{eqnarray*}
\lambda(t \mid X) = \lambda_0(t) \exp(0.03 \cdot \text{edad} - 0.6 \cdot \text{sexo} + 1.5 \cdot \text{tratamiento})
\end{eqnarray*}

\section{Conclusi\'on}
El an\'alisis multivariado de supervivencia permite evaluar el efecto conjunto de m\'ultiples covariables sobre el tiempo hasta el evento. La inclusi\'on de interacciones y efectos no lineales, junto con la selecci\'on adecuada de variables, mejora la precisi\'on y la interpretabilidad de los modelos de supervivencia.



\chapter{Supervivencia en Datos Complicados}

\section{Introducci\'on}
El análisis de supervivencia en datos complicados se refiere a la evaluaci\'on de datos de supervivencia que presentan desaf\'ios adicionales, como la censura por intervalo, datos truncados y datos con m\'ultiples tipos de eventos. Estos escenarios requieren m\'etodos avanzados para un análisis adecuado.

\section{Censura por Intervalo}
La censura por intervalo ocurre cuando el evento de inter\'es se sabe que ocurri\'o dentro de un intervalo de tiempo, pero no se conoce el momento exacto. Esto es com\'un en estudios donde las observaciones se realizan en puntos de tiempo discretos.

\subsection{Modelo para Datos Censurados por Intervalo}
Para datos censurados por intervalo, la funci\'on de verosimilitud se modifica para incluir la probabilidad de que el evento ocurra dentro de un intervalo:
\begin{eqnarray*}
L(\beta) = \prod_{i=1}^n P(T_i \in [L_i, U_i] \mid X_i; \beta)
\end{eqnarray*}
donde $[L_i, U_i]$ es el intervalo de tiempo durante el cual se sabe que ocurri\'o el evento para el individuo $i$.

\section{Datos Truncados}
Los datos truncados ocurren cuando los tiempos de supervivencia est\'an sujetos a un umbral, y solo se observan los individuos cuyos tiempos de supervivencia superan (o est\'an por debajo de) ese umbral. Existen dos tipos principales de truncamiento: truncamiento a la izquierda y truncamiento a la derecha.

\subsection{Modelo para Datos Truncados}
Para datos truncados a la izquierda, la funci\'on de verosimilitud se ajusta para considerar solo los individuos que superan el umbral de truncamiento:
\begin{eqnarray*}
L(\beta) = \prod_{i=1}^n \frac{f(t_i \mid X_i; \beta)}{1 - F(L_i \mid X_i; \beta)}
\end{eqnarray*}
donde $L_i$ es el umbral de truncamiento para el individuo $i$.

\section{An\'alisis de Competing Risks}
En estudios donde pueden ocurrir m\'ultiples tipos de eventos (competing risks), es crucial modelar adecuadamente el riesgo asociado con cada tipo de evento. La probabilidad de ocurrencia de cada evento compite con las probabilidades de ocurrencia de otros eventos.

\subsection{Modelo de Competing Risks}
Para un an\'alisis de competing risks, la funci\'on de riesgo se descompone en funciones de riesgo espec\'ificas para cada tipo de evento:
\begin{eqnarray*}
\lambda(t) = \sum_{j=1}^m \lambda_j(t)
\end{eqnarray*}
donde $\lambda_j(t)$ es la funci\'on de riesgo para el evento $j$.

\section{M\'etodos de Imputaci\'on}
Los m\'etodos de imputaci\'on se utilizan para manejar datos faltantes o censurados en estudios de supervivencia. La imputaci\'on m\'ultiple es un enfoque com\'un que crea m\'ultiples conjuntos de datos completos imputando valores faltantes varias veces y luego combina los resultados.

\subsection{Imputaci\'on M\'ultiple}
La imputaci\'on m\'ultiple para datos de supervivencia se realiza en tres pasos:
\begin{enumerate}
    \item Imputar los valores faltantes m\'ultiples veces para crear varios conjuntos de datos completos.
    \item Analizar cada conjunto de datos completo por separado utilizando m\'etodos de supervivencia est\'andar.
    \item Combinar los resultados de los an\'alisis separados para obtener estimaciones y varianzas combinadas.
\end{enumerate}

\section{Ejemplo de An\'alisis con Datos Complicados}
Consideremos un estudio con datos censurados por intervalo y competing risks. Ajustamos un modelo para los datos censurados por intervalo y obtenemos los siguientes coeficientes para las covariables edad y tratamiento:
\begin{eqnarray*}
\hat{\beta}_{edad} = 0.04, \quad \hat{\beta}_{tratamiento} = -0.8
\end{eqnarray*}

La funci\'on de supervivencia ajustada se expresa como:
\begin{eqnarray*}
S(t \mid X) = \exp\left(-\left(\frac{t \exp(0.04 \cdot \text{edad} - 0.8 \cdot \text{tratamiento})}{\lambda}\right)^k\right)
\end{eqnarray*}

\section{Conclusi\'on}
El an\'alisis de supervivencia en datos complicados requiere m\'etodos avanzados para manejar censura por intervalo, datos truncados y competing risks. La aplicaci\'on de modelos adecuados y m\'etodos de imputaci\'on asegura un an\'alisis preciso y completo de estos datos complejos.



\chapter{Proyecto Final y Revisi\'on}

\section{Introducci\'on}
El proyecto final proporciona una oportunidad para aplicar los conceptos y t\'ecnicas aprendidas en el curso de análisis de supervivencia. Este cap\'itulo incluye una gu\'ia para desarrollar un proyecto de análisis de supervivencia y una revisi\'on de los conceptos clave.

\section{Desarrollo del Proyecto}
El proyecto final debe incluir los siguientes componentes:
\begin{enumerate}
    \item Definici\'on del problema: Identificar la pregunta de investigaci\'on y los objetivos del análisis de supervivencia.
    \item Descripci\'on de los datos: Presentar los datos utilizados, incluyendo las covariables y la estructura de los datos.
    \item Análisis exploratorio: Realizar un análisis descriptivo de los datos, incluyendo la censura y la distribuci\'on de los tiempos de supervivencia.
    \item Ajuste del modelo: Ajustar modelos de supervivencia adecuados (Kaplan-Meier, Cox, AFT) y evaluar su bondad de ajuste.
    \item Diagn\'ostico del modelo: Realizar diagn\'osticos para evaluar los supuestos del modelo y la influencia de observaciones individuales.
    \item Interpretaci\'on de resultados: Interpretar los coeficientes del modelo y las curvas de supervivencia ajustadas.
    \item Conclusiones: Resumir los hallazgos del análisis y proporcionar recomendaciones basadas en los resultados.
\end{enumerate}

\section{Revisi\'on de Conceptos Clave}
Una revisi\'on de los conceptos clave del an\'alisis de supervivencia incluye:
\begin{itemize}
    \item \textbf{Funci\'on de Supervivencia:} Define la probabilidad de sobrevivir m\'as all\'a de un tiempo espec\'ifico.
    \item \textbf{Funci\'on de Riesgo:} Define la tasa instant\'anea de ocurrencia del evento.
    \item \textbf{Estimador de Kaplan-Meier:} Proporciona una estimaci\'on no param\'etrica de la funci\'on de supervivencia.
    \item \textbf{Test de Log-rank:} Compara curvas de supervivencia entre diferentes grupos.
    \item \textbf{Modelo de Cox:} Eval\'ua el efecto de m\'ultiples covariables sobre el tiempo hasta el evento, asumiendo proporcionalidad de riesgos.
    \item \textbf{Modelos AFT:} Modelan el efecto de las covariables multiplicando el tiempo de supervivencia por una constante.
    \item \textbf{An\'alisis Multivariado:} Considera interacciones y efectos no lineales entre m\'ultiples covariables.
    \item \textbf{Supervivencia en Datos Complicados:} Maneja censura por intervalo, datos truncados y competing risks.
\end{itemize}

\section{Ejemplo de Proyecto Final}
A continuaci\'on se presenta un ejemplo de estructura de un proyecto final de an\'alisis de supervivencia:

\subsection{Definici\'on del Problema}
Analizar el efecto del tratamiento y la edad sobre la supervivencia de pacientes con una enfermedad espec\'ifica.

\subsection{Descripci\'on de los Datos}
Datos de supervivencia de 100 pacientes, con covariables: edad, sexo y tipo de tratamiento. Los tiempos de supervivencia est\'an censurados a la derecha.

\subsection{An\'alisis Exploratorio}
Realizar histogramas y curvas de Kaplan-Meier para explorar la distribuci\'on de los tiempos de supervivencia y la censura.

\subsection{Ajuste del Modelo}
Ajustar un modelo de Cox y un modelo AFT con las covariables edad y tratamiento.

\subsection{Diagn\'ostico del Modelo}
Evaluar la proporcionalidad de riesgos y realizar an\'alisis de residuos para identificar observaciones influyentes.

\subsection{Interpretaci\'on de Resultados}
Interpretar los coeficientes del modelo y las curvas de supervivencia ajustadas para diferentes niveles de las covariables.

\subsection{Conclusiones}
Resumir los hallazgos y proporcionar recomendaciones sobre el efecto del tratamiento y la edad en la supervivencia de los pacientes.

\section{Conclusi\'on}
El proyecto final es una oportunidad para aplicar los conocimientos adquiridos en un contexto pr\'actico. La revisi\'on de los conceptos clave y la aplicaci\'on de t\'ecnicas adecuadas de an\'alisis de supervivencia aseguran un an\'alisis riguroso y significativo.



\part{TERCERA PARTE}
\chapter{Probabilidad Avanzada}
%__________________________________________________________________________
%
\section{Cadenas de Markov}
%_____________________________________________________________________________________
%
\subsection{Estacionareidad}
%_____________________________________________________________________________________
%

Sea $v=\left(v_{i}\right)_{i\in E}$ medida no negativa en $E$, podemos definir una nueva medida $v\prob$ que asigna masa $\sum_{i\in E}v_{i}p_{ij}$ a cada estado $j$.\smallskip

\begin{Def}
La medida $v$ es estacionaria si $v_{i}<\infty$ para toda $i$ y adem\'as $v\prob=v$.\smallskip
\end{Def}
En el caso de que $v$ sea distribuci\'on, independientemente de que sea estacionaria o no, se cumple con

\begin{eqnarray*}
\prob_{v}\left[X_{1}=j\right]=\sum_{i\in E}\prob_{v}\left[X_{0}=i\right]p_{ij}=\sum_{i\in E}v_{i}p_{ij}=\left(vP\right)_{j}
\end{eqnarray*}

\begin{Teo}
Supongamos que $v$ es una distribuci\'on estacionaria. Entonces
\begin{itemize}
\item[i)] La cadena es estrictamente estacionaria con respecto a $\prob_{v}$, es decir, $\prob_{v}$-distribuci\'on de $\left\{X_{n},X_{n+1},\ldots\right\}$ no depende de $n$;
\item[ii)] Existe un aversi\'on estrictamente estacionaria $\left\{X_{n}\right\}_{n\in Z}$ de la cadena con doble tiempo infinito y $\prob\left(X_{n}=i\right)=v_{i}$ para toda $n\in Z$.
\end{itemize}
\end{Teo}
\begin{Teo}
Sea $i$ estado fijo, recurrente. Entonces una medida estacionaria $v$ puede definirse haciendo que $v_{j}$ sea el n\'umero esperado de visitas a $j$ entre dos visitas consecutivas $i$,
\begin{equation}\label{Eq.3.1}
v_{j}=\esp_{i}\sum_{n=0}^{\tau(i)-1}\indora\left(X_{n}=i\right)=\sum_{n=0}^{\infty}\prob_{i}\left[X_{n}=j,\tau(i)>n\right]
\end{equation}
\end{Teo}
\begin{Teo}\label{Teo.3.3}
Si la cadena es irreducible y recurrente, entonces una medida estacionaria $v$ existe, satisface $0<v_{j}<\infty$ para toda $j$ y es \'unica salvo factores multiplicativos, es decir, si $v,v^{*}$ son estacionarias, entonces $c=cv^{*}$ para alguna $c\in\left(0,\infty\right)$.
\end{Teo}
\begin{Cor}\label{Cor.3.5}
Si la cadena es irreducible y positiva recurrente, existe una \'unica distribuci\'on estacionaria $\pi$ dada por
\begin{equation}
\pi_{j}=\frac{1}{\esp_{i}\tau_{i}}\esp_{i}\sum_{n=0}^{\tau\left(i\right)-1}\indora\left(X_{n}=j\right)=\frac{1}{\esp_{j}\tau\left(j\right)}.
\end{equation}
\end{Cor}
\begin{Cor}\label{Cor.3.6}
Cualquier cadena de Markov irreducible con un espacio de estados finito es positiva recurrente.
\end{Cor}
%
\subsection{Teor\'ia Erg\'odica}
%_____________________________________________________________________________________
%

\begin{Lema}
Sea $\left\{X_{n}\right\}$ cadena irreducible y se $F$ subconjunto finito del espacio de estados. Entonces la cadena es positiva recurrente si $\esp_{i}\tau\left(F\right)<\infty$ para todo $i\in F$.
\end{Lema}

\begin{Prop}
Sea $\left\{X_{n}\right\}$ cadena irreducible y transiente o cero recurrente, entonces $p_{ij}^{n}\rightarrow0$ conforme $n\rightarrow\infty$ para cualquier $i,j\in E$, $E$ espacio de estados.
\end{Prop}
Utilizando el teorema (2.2) y el corolario ref{Cor.3.5}, se demuestra el siguiente resultado importante.

\begin{Teo}
Sea $\left\{X_{n}\right\}$ cadena irreducible y aperi\'odica positiva recurrente, y sea $\pi=\left\{\pi_{j}\right\}_{j\in E}$ la distribuci\'on estacionaria. Entonces $p_{ij}^{n}\rightarrow\pi_{j}$ para todo $i,j$.
\end{Teo}
\begin{Def}\label{Def.Ergodicidad}
Una cadena irreducible aperiodica, positiva recurrente con medida estacionaria $v$, es llamada {\em erg\'odica}.
\end{Def}
\begin{Prop}\label{Prop.4.4}
Sea $\left\{X_{n}\right\}$ cadena irreducible y recurrente con medida estacionaria $v$, entocnes para todo $i,j,k,l\in E$
\begin{equation}
\frac{\sum_{n=0}^{m}p_{ij}^{n}}{\sum_{n=0}^{m}p_{lk}^{n}}\rightarrow\frac{v_{j}}{v_{k}}\textrm{,    }m\rightarrow\infty
\end{equation}
\end{Prop}
\begin{Lema}\label{Lema.4.5}
La matriz $\widetilde{P}$ con elementos $\widetilde{p}_{ij}=\frac{v_{ji}p_{ji}}{v_{i}}$ es una matriz de transici\'on. Adem\'as, el $i$-\'esimo elementos $\widetilde{p}_{ij}^{m}$ de la matriz potencia $\widetilde{P}^{m}$ est\'a dada por $\widetilde{p}_{ij}^{m}=\frac{v_{ji}p_{ji}^{m}}{v_{i}}$.
\end{Lema}

\begin{Lema}
Def\'inase $N_{i}^{m}=\sum_{n=0}^{m}\indora\left(X_{n}=i\right)$ como el n\'umero de visitas a $i$ antes del tiempo $m$. Entonces si la cadena es reducible y recurrente, $lim_{m\rightarrow\infty}\frac{\esp_{j}N_{i}^{m}}{\esp_{k}N_{i}^{m}}=1$ para todo $j,k\in E$.
\end{Lema}

%_____________________________________________________________________________________
%
\subsection{Funciones Arm\'onicas, Recurrencia y Transitoriedad}
%_____________________________________________________________________________________
%

\begin{Def}\label{Def.Armonica}
Una funci\'on Arm\'onica es el eigenvector derecho $h$ de $P$ corrrespondiente al eigenvalor 1.
\end{Def}
\begin{eqnarray*}
Ph=h\Leftrightarrow h\left(i\right)=\sum_{j\in E}p_{ij}h\left(j\right)=\esp_{i}h\left(X_{1}\right)=\esp\left[h\left(X_{n+1}\right)|X_{n}=i\right].
\end{eqnarray*}
es decir, $\left\{h\left(X_{n}\right)\right\}$ es martingala.
\begin{Prop}\label{Prop.5.2}
Sea $\left\{X_{n}\right\}$ cadena irreducible  y sea $i$ estado fijo arbitrario. Entonces la cadena es transitoria s\'i y s\'olo si existe una funci\'on no cero, acotada $h:E-\left\{i\right\}\rightarrow\rea$ que satisface
\begin{equation}\label{Eq.5.1}
h\left(j\right)=\sum_{k\neq i}p_{jk}h\left(k\right)\textrm{   para  }j\neq i.
\end{equation}
\end{Prop}
\begin{Prop}\label{Prop.5.3}
Supongamos que la cadena es irreducible y sea $E_{0}$ un subconjunto finito del espacio de estados, entonces
\begin{itemize}
\item[i)]
\item[ii)]
\end{itemize}
\end{Prop}

\begin{Prop}\label{Prop.5.4}
Suponga que la cadena es irreducible y sea $E_{0}$ un subconjunto finito de $E$ tal que se cumple la ecuaci\'on 5.2 para alguna funci\'on $h$ acotada que satisface $h\left(i\right)<h\left(j\right)$ para alg\'un estado $i\notin E_{0}$ y todo $j\in E_{0}$. Entonces la cadena es transitoria.
\end{Prop}


%_____________________________________________________________________________________
%
\section{Procesos de Markov de Saltos}
%_____________________________________________________________________________________
%
\subsection{Estructura B\'asica de los Procesos Markovianos de Saltos}
%_____________________________________________________________________________________
%

\begin{itemize}
\item Sea $E$ espacio discreto de estados, finito o numerable, y sea $\left\{X_{t}\right\}$ un proceso de Markov con espacio de estados $E$. Una medida $\mu$ en $E$ definida por sus probabilidades puntuales $\mu_{i}$, escribimos $p_{ij}^{t}=P^{t}\left(i,\left\{j\right\}\right)=P_{i}\left(X_{t}=j\right)$.\smallskip

\item El monto del tiempo gastado en cada estado es positivo, de modo tal que las trayectorias muestrales son constantes por partes. Para un proceso de saltos denotamos por los tiempos de saltos a $S_{0}=0<S_{1}<S_{2}\cdots$, los tiempos entre saltos consecutivos $T_{n}=S_{n+1}-S_{n}$ y la secuencia de estados visitados por $Y_{0},Y_{1},\ldots$, as\'i las trayectorias muestrales son constantes entre $S_{n}$ consecutivos, continua por la derecha, es decir, $X_{S_{n}}=Y_{n}$.
\item La descripci\'on de un modelo pr\'actico est\'a dado usualmente en t\'erminos de las intensidades $\lambda\left(i\right)$ y las probabilidades de salto $q_{ij}$ m\'as que en t\'erminos de la matriz de transici\'on $P^{t}$.
\item Sup\'ongase de ahora en adelante que $q_{ii}=0$ cuando $\lambda\left(i\right)>0$
\end{itemize}

%_____________________________________________________________________________________
%
\subsection{Matriz Intensidad}
%_____________________________________________________________________________________
%


\begin{Def}
La matriz intensidad $\Lambda=\left(\lambda\left(i,j\right)\right)_{i,j\in E}$ del proceso de saltos $\left\{X_{t}\right\}_{t\geq0}$ est\'a dada por
\begin{eqnarray*}
\lambda\left(i,j\right)&=&\lambda\left(i\right)q_{i,j}\textrm{,    }j\neq i\\
\lambda\left(i,i\right)&=&-\lambda\left(i\right)\\
\end{eqnarray*}

\begin{Prop}\label{Prop.3.1}
Una matriz $E\times E$, $\Lambda$ es la matriz de intensidad de un proceso markoviano de saltos $\left\{X_{t}\right\}_{t\geq0}$ si y s\'olo si
\begin{eqnarray*}
\lambda\left(i,i\right)\leq0\textrm{,  }\lambda\left(i,j\right)\textrm{,   }i\neq j\textrm{,  }\sum_{j\in E}\lambda\left(i,j\right)=0.
\end{eqnarray*}
Adem\'as, $\Lambda$ est\'a en correspondencia uno a uno con la distribuci\'on del proceso minimal dado por el teorema 3.1.
\end{Prop}

\end{Def}
Para el caso particular de la Cola $M/M/1$, la matr\'iz de itensidad est\'a dada por
\begin{eqnarray*}
\Lambda=\left[\begin{array}{cccccc}
-\beta & \beta & 0 &0 &0& \cdots\\
\delta & -\beta-\delta & \beta & 0 & 0 &\cdots\\
0 & \delta & -\beta-\delta & \beta & 0 &\cdots\\
\vdots & & & & & \ddots\\
\end{array}\right]
\end{eqnarray*}

%____________________________________________________________________________
\subsection{Medidas Estacionarias}
%____________________________________________________________________________
%

\begin{Def}
Una medida $v\neq0$ es estacionaria si $0\leq v_{j}<\infty$, $vP^{t}=v$ para toda $t$.
\end{Def}

\begin{Teo}\label{Teo.4.2}
Supongamos que $\left\{X_{t}\right\}$ es irreducible recurrente en $E$. Entonces existe una y s\'olo una, salvo m\'ultiplos, medida estacionaria $v$. Esta $v$ tiene la propiedad de que $0< v_{j}<\infty$ para todo $j$ y puede encontrarse en cualquiera de las siguientes formas

\begin{itemize}
\item[i)] Para alg\'un estado $i$, fijo pero arbitrario, $v_{j}$ es el tiempo esperado utilizado en $j$ entre dos llegadas consecutivas al estado $i$;
\begin{equation}\label{Eq.4.2}
v_{j}=\esp_{i}\int_{0}^{w\left(i\right)}\indora\left(X_{t}=j\right)dt
\end{equation}
con $w\left(i\right)=\inf\left\{t>0:X_{t}=i,X_{t^{-}}=\lim_{s\uparrow t}X_{s}\neq i\right\}$.
\item[ii)] $v_{j}=\frac{\mu_{j}}{\lambda\left(j\right)}$, donde $\mu$ es estacionaria para $\left\{Y_{n}\right\}$.
\item[iii)] como soluci\'on de $v\Lambda=0$.
\end{itemize}
\end{Teo}
%____________________________________________________________________________
\subsection{Criterios de Ergodicidad}
%____________________________________________________________________________
%

\begin{Def}
Un proceso irreducible recurrente con medida estacionaria de masa finita es llamado erg\'odico.
\end{Def}

\begin{Teo}\label{Teo.4.3}
Un proceso de Markov de saltos irreducible no explosivo es erg\'odico si y s\'olo si se puede encontrar una soluci\'on, de probabilidad, $\pi$, con $|\pi|=1$ y $0\leq\pi_{j}\leq1$, a $\pi\Lambda=0$. En este caso $\pi$ es la distribuci\'on estacionaria.
\end{Teo}

\begin{Cor}\label{Cor.4.4}
Una condici\'on suficiente para la ergodicidad de un proceso irreducible es la existencia de una probabilidad $\pi$ que resuelva el sistema $\pi\Lambda=0$ y que adem\'as tenga la propiedad de que $\sum\pi_{j}\lambda\left(j\right)<\infty$.
\end{Cor}
\begin{Prop}
Si el proceso es erg\'odico, entonces existe una versi\'on estrictamente estacionaria $\left\{X_{t}\right\}_{-\infty<t<\infty}$con doble tiempo infinito.
\end{Prop}

\begin{Teo}
Si $\left\{X_{t}\right\}$ es erg\'odico y $\pi$ es la distribuci\'on estacionaria, entonces para todo $i,j$, $p_{ij}^{t}\rightarrow\pi_{j}$ cuando $t\rightarrow\infty$.
\end{Teo}

\begin{Cor}
Si $\left\{X_{t}\right\}$ es irreducible recurente pero no erg\'odica, es decir $|v|=\infty$, entonces $p_{ij}^{t}\rightarrow0$ para todo $i,j\in E$.
\end{Cor}

\begin{Cor}
Para cualquier proceso Markoviano de Saltos minimal, irreducible o no, los l\'imites $li_{t\rightarrow\infty}p_{ij}^{t}$ existe.
\end{Cor}

%_____________________________________________________________________________________
%
\section{Notaci\'on Kendall-Lee}
%_____________________________________________________________________________________
%


A partir de este momento se har\'an las siguientes consideraciones: Si $t_{n}$ es el tiempo aleatorio en el que llega al sistema el $n$-\'esimo cliente, para $n=1,2,\ldots$, $t_{0}=0$ y $t_{0}<t_{1}<\cdots$ se definen los tiempos entre arribos  $\tau_{n}=t_{n}-t_{n-1}$ para $n=1,2,\ldots$, variables aleatorias independientes e id\'enticamente distribuidas. Los tiempos entre arribos tienen un valor medio $E\left(\tau\right)$ finito y positivo $\frac{1}{\beta}$, es decir, $\beta$ se puede ver como la tasa o intensidad promedio de arribos al sistema por unidad de tiempo. Adem\'as se supondr\'a que los servidores son identicos y si $s$ denota la variable aleatoria que describe el tiempo de servicio, entonces $E\left(s\right)=\frac{1}{\delta}$, $\delta$ es la tasa promedio de servicio por servidor.

La notaci\'on de Kendall-Lee es una forma abreviada de describir un sisema de espera con las siguientes componentes:
\begin{itemize}
\item[i)] {\em\bf Fuente}: Poblaci\'on de clientes potenciales del sistema, esta puede ser finita o infinita.
\item[ii)] {\em\bf Proceso de Arribos}: Proceso determinado por la funci\'on de distribuci\'on $A\left(t\right)=P\left\{\tau\leq t\right\}$ de los tiempos entre arribos.
\end{itemize}

Adem\'as tenemos las siguientes igualdades
\begin{equation}\label{Eq.0.1}
N\left(t\right)=N_{q}\left(t\right)+N_{s}\left(s\right)
\end{equation}
donde
\begin{itemize}
\item $N\left(t\right)$ es el n\'umero de clientes en el sistema al tiempo $t$.
\item $N_{q}\left(t\right)$ es el n\'umero de clientes en la cola al tiempo $t$
\item $N_{s}\left(t\right)$ es el n\'umero de clientes recibiendo servicio en el tiempo $t$.
\end{itemize}

Bajo la hip\'otesis de estacionareidad, es decir, las caracter\'isticas de funcionamiento del sistema se han estabilizado en valores independientes del tiempo, entonces
\begin{equation}
N=N_{q}+N_{s}.
\end{equation}

Los valores medios de las cantidades anteriores se escriben como $L=E\left(N\right)$, $L_{q}=E\left(N_{q}\right)$ y $L_{s}=E\left(N_{s}\right)$, entonces de la ecuaci\'on \ref{Eq.0.1} se obtiene

\begin{equation}
L=L_{q}+L_{s}
\end{equation}

Si $q$ es el tiempo que pasa un cliente en la cola antes de recibir servicio, y W es el tiempo total que un cliente pasa en el sistema, entonces
\[w=q+s\]
por lo tanto
\[W=W_{q}+W_{s},\]
donde $W=E\left(w\right)$, $W_{q}=E\left(q\right)$ y $W_{s}=E\left(s\right)=\frac{1}{\delta}$.

La intensidad de tr\'afico se define como
\begin{equation}
\rho=\frac{E\left(s\right)}{E\left(\tau\right)}=\frac{\beta}{\delta}.
\end{equation}

La utilizaci\'on por servidor es
\begin{equation}
u=\frac{\rho}{c}=\frac{\beta}{c\delta}.
\end{equation}
donde $c$ es el n\'umero de servidores.
Esta notaci\'on es una forma abreviada de describir un sistema de espera con componentes dados a continuaci\'on, la notaci\'on es
\begin{equation}\label{Notacion.K.L.}
A/S/c/K/F/d
\end{equation}
Cada una de las letras describe:
\begin{itemize}
\item $A$ es la distribuci\'on de los tiempos entre arribos.
\item $S$ es la distribuci\'on del tiempo de servicio.
\item $c$ es el n\'umero de servidores.
\item $K$ es la capacidad del sistema.
\item $F$ es el n\'umero de individuos en la fuente.
\item $d$ es la disciplina del servicio
\end{itemize}
Usualmente se acostumbra suponer que $K=\infty$, $F=\infty$ y $d=FIFO$, es decir, First In First Out.
Las distribuciones usuales para $A$ y $B$ son:
\begin{itemize}
\item $GI$ para la distribuci\'on general de los tiempos entre arribos.
\item $G$ distribuci\'on general del tiempo de servicio.
\item $M$ Distribuci\'on exponencial para $A$ o $S$.
\item $E_{K}$ Distribuci\'on Erlang-$K$, para $A$ o $S$.
\item $D$ tiempos entre arribos o de servicio constantes, es decir, deterministicos.
\end{itemize}

%_____________________________________________________________________________________
%
\section{Procesos de Nacimiento y Muerte}
%_____________________________________________________________________________________
%
\subsection{Procesos de Nacimiento y Muerte Generales}
%_____________________________________________________________________________________
%

Por un proceso de nacimiento y muerte se entiende un proceso de saltos de markov $\left\{X_{t}\right\}_{t\geq0}$ con espacio de estados a lo m\'as numerable, con la propiedad de que s\'olo puede ir al estado $n+1$ o al estado $n-1$, es decir, su matriz de intensidad es de la forma
\begin{eqnarray*}
\Lambda=\left[\begin{array}{cccccc}
-\beta_{0} & \beta_{0} & 0 &0 &0& \cdots\\
\delta_{1} & -\beta_{1}-\delta_{1} & \beta_{1} & 0 & 0 &\cdots\\
0 & \delta_{2} & -\beta_{2}-\delta_{2} & \beta_{2} & 0 &\cdots\\
\vdots & & & & & \ddots\\
\end{array}\right]
\end{eqnarray*}
donde $\beta_{n}$ son las intensidades de nacimiento y $\delta_{n}$ las intensidades de muerte, o tambi\'en se puede ver como a $X_{t}$ el n\'umero de usuarios en una cola al tiempo $t$, un salto hacia arriba corresponde a la llegada de un nuevo usuario y un salto hacia abajo como al abandono de un usuario despu\'es de haber recibido su servicio.

La cadena de saltos $\left\{Y_{n}\right\}$ tiene matriz de transici\'on dada por
\begin{eqnarray*}
Q=\left[\begin{array}{cccccc}
0 & 1 & 0 &0 &0& \cdots\\
q_{1} & 0 & p_{1} & 0 & 0 &\cdots\\
0 & q_{2} & 0 & p_{2} & 0 &\cdots\\
\vdots & & & & & \ddots\\
\end{array}\right]
\end{eqnarray*}
donde $p_{n}=\frac{\beta_{n}}{\beta_{n}+\delta_{n}}$ y $q_{n}=1-p_{n}=\frac{\delta_{n}}{\beta_{n}+\delta_{n}}$, donde adem\'as se asumne por el momento que $p_{n}$ no puede tomar el valor $0$ \'o $1$ para cualquier valor de $n$.

%____________________________________________________________________________


\begin{Prop}\label{Prop.2.1}
La recurrencia de $\left\{X_{t}\right\}$, o equivalentemente de $\left\{Y_{n}\right\}$ es equivalente a
\begin{equation}\label{Eq.2.1}
\sum_{n=1}^{\infty}\frac{\delta_{1}\cdots\delta_{n}}{\beta_{1}\cdots\beta_{n}}=\sum_{n=1}^{\infty}\frac{q_{1}\cdots q_{n}}{p_{1}\cdots p_{n}}=\infty
\end{equation}
\end{Prop}

\begin{Lema}\label{Lema.2.2}
Independientemente de la recurrencia o transitorieadad, existe una y s\'olo una, salvo m\'ultiplos, soluci\'on a $v\Lambda=0$, dada por
\begin{equation}\label{Eq.2.2}
v_{n}=\frac{\beta_{0}\cdots\beta_{n-1}}{\delta_{1}\cdots\delta_{n}}v_{0}
\end{equation}
para $n=1,2,\ldots$.
\end{Lema}


\begin{Cor}\label{Cor.2.3}
En el caso recurrente, la medida estacionaria $\mu$ para $\left\{Y_{n}\right\}$ est\'a dada por
\begin{equation}\label{Eq.}
\mu_{n}=\frac{p_{1}\cdots p_{n-1}}{q_{1}\cdots q_{n}}\mu_{0}
\end{equation}
para $n=1,2,\ldots$.
\end{Cor}

Se define a $S=1+\sum_{n=1}^{\infty}\frac{\beta_{0}\beta_{1}\cdots\beta_{n-1}}{\delta_{1}\delta_{2}\cdots\delta_{n}}$

\begin{Cor}\label{Cor.2.4}
$\left\{X_{t}\right\}$ es erg\'odica si y s\'olo si la ecuaci\'on (\ref{Eq.2.1}) se cumple y adem\'as $S<\infty$, en cuyo caso la distribuci\'on erg\'odica, $\pi$, est\'a dada por
\begin{equation}\label{Eq.2.4}
\pi_{0}=\frac{1}{S}\textrm{,     }\pi_{n}=\frac{1}{S}\frac{\beta_{0}\cdots\beta_{n-1}}{\delta_{1}\cdots\delta_{n}}
\end{equation}
para $n=1,2,\ldots$.
\end{Cor}
%_____________________________________________________________________________________
%
\subsection{Cola M/M/1}
%_____________________________________________________________________________________
%



Este modelo corresponde a un proceso de nacimiento y muerte con $\beta_{n}=\beta$ y $\delta_{n}=\delta$ independiente del valor de $n$. La intensidad de tr\'afico $\rho=\frac{\beta}{\delta}$, implica que el criterio de recurrencia (ecuaci\'on \ref{Eq.2.1}) quede de la forma:
\begin{eqnarray*}
1+\sum_{n=1}^{\infty}\rho^{-n}=\infty.
\end{eqnarray*}
Equivalentemente el proceso es recurrente si y s\'olo si
\begin{eqnarray*}
\sum_{n\geq1}\left(\frac{\beta}{\delta}\right)^{n}<\infty\Leftrightarrow \frac{\beta}{\delta}<1
\end{eqnarray*}
Entonces
$S=\frac{\delta}{\delta-\beta}$, luego por la ecuaci\'on \ref{Eq.2.4} se tiene que
\begin{eqnarray*}
\pi_{0}&=&\frac{\delta-\beta}{\delta}=1-\frac{\beta}{\delta}\\
\pi_{n}&=&\pi_{0}\left(\frac{\beta}{\delta}\right)^{n}=\left(1-\frac{\beta}{\delta}\right)\left(\frac{\beta}{\delta}\right)^{n}=\left(1-\rho\right)\rho^{n}
\end{eqnarray*}


Lo cual nos lleva a la siguiente

\begin{Prop}
La cola $M/M/1$ con intendisad de tr\'afico $\rho$, es recurrente si y s\'olo si $\rho\leq1$.
\end{Prop}

Entonces por el corolario \ref{Cor.2.3}

\begin{Prop}
La cola $M/M/1$ con intensidad de tr\'afico $\rho$ es erg\'odica si y s\'olo si $\rho<1$. En cuyo caso, la distribuci\'on de equilibrio $\pi$ de la longitud de la cola es geom\'etrica, $\pi_{n}=\left(1-\rho\right)\rho^{n}$, para $n=1,2,\ldots$.
\end{Prop}
De la proposici\'on anterior se desprenden varios hechos importantes.
\begin{enumerate}
\item $\prob\left[X_{t}=0\right]=\pi_{0}=1-\rho$, es decir, la probabilidad de que el sistema se encuentre ocupado.
\item De las propiedades de la distribuci\'on Geom\'etrica se desprende que
\begin{enumerate}
\item $\esp\left[X_{t}\right]=\frac{\rho}{1-\rho}$,
\item $Var\left[X_{t}\right]=\frac{\rho}{\left(1-\rho\right)^{2}}$.
\end{enumerate}
\end{enumerate}

Si $L$ es el n\'umero esperado de clientes en el sistema, incluyendo los que est\'an siendo atendidos, entonces
\begin{eqnarray*}
L=\frac{\rho}{1-\rho}
\end{eqnarray*}
Si adem\'as $W$ es el tiempo total del cliente en la cola: $W=W_{q}+W_{s}$
$\rho=\frac{\esp\left[s\right]}{\esp\left[\tau\right]}=\beta W_{s}$, puesto que $W_{s}=\esp\left[s\right]$ y $\esp\left[\tau\right]=\frac{1}{\delta}$. Por la f\'ormula de Little $L=\lambda W$
\begin{eqnarray*}
W&=&\frac{L}{\beta}=\frac{\frac{\rho}{1-\rho}}{\beta}=\frac{\rho}{\delta}\frac{1}{1-\rho}=\frac{W_{s}}{1-\rho}\\
&=&\frac{1}{\delta\left(1-\rho\right)}=\frac{1}{\delta-\beta}
\end{eqnarray*}
luego entonces
\begin{eqnarray*}
W_{q}&=&W-W_{s}=\frac{1}{\delta-\beta}-\frac{1}{\delta}=\frac{\beta}{\delta(\delta-\beta)}\\
&=&\frac{\rho}{1-\rho}\frac{1}{\delta}=\esp\left[s\right]\frac{\rho}{1-\rho}
\end{eqnarray*}
Entonces
\begin{eqnarray*}
L_{q}=\beta W_{q}=\frac{\rho^{2}}{1-\rho}.
\end{eqnarray*}
Finalmente
\begin{Prop}
\begin{enumerate}
\item $W\left(t\right)=1-e^{-\frac{t}{W}}$.
\item $W_{q}\left(t\right)=1-\rho\exp^{-\frac{t}{W}}$.
\end{enumerate}
donde $W=\esp(w)$.
\end{Prop}


%_____________________________________________________________________________________
%
\subsection{Cola $M/M/\infty$}
%_____________________________________________________________________________________
%

Este tipo de modelos se utilizan para estimar el n\'umero de l\'ineas en uso en una gran red comunicaci\'on o para estimar valores en los sistemas $M/M/c$ o $M/M/c/c$, en el se puede pensar que siempre hay un servidor disponible para cada cliente que llega.\smallskip

Se puede considerar como un proceso de nacimiento y muerte con par\'ametros $\beta_{n}=\beta$ y $\mu_{n}=n\mu$ para $n=0,1,2,\ldots$, entonces por la ecuaci\'on \ref{Eq.2.4} se tiene que
\begin{eqnarray*}\label{MMinf.pi}
\pi_{0}=e^{\rho}\\
\pi_{n}=e^{-\rho}\frac{\rho^{n}}{n!}
\end{eqnarray*}
Entonces, el n\'umero promedio de servidores ocupados es equivalente a considerar el n\'umero de clientes en el  sistema, es decir,
\begin{eqnarray*}
L=\esp\left[N\right]=\rho\\
Var\left[N\right]=\rho
\end{eqnarray*}

Adem\'as se tiene que $W_{q}=0$ y $L_{q}=0$.\smallskip

El tiempo promedio en el sistema es el tiempo promedio de servicio, es decir, $W=\esp\left[s\right]=\frac{1}{\delta}$.
Resumiendo, tenemos la sisuguiente proposici\'on:
\begin{Prop}
La cola $M/M/\infty$ es erg\'odica para todos los valores de $\eta$. La distribuci\'on de equilibrio $\pi$ es Poisson con media $\eta$, $\pi_{n}=\frac{e^{-n}\eta^{n}}{n!}$.
\end{Prop}

%_____________________________________________________________________________________
%
\subsection{Cola M/M/m}
%_____________________________________________________________________________________
%

Este sistema considera $m$ servidores id\'enticos, con tiempos entre arribos y de servicio exponenciales con medias $\esp\left[\tau\right]=\frac{1}{\beta}$ y $\esp\left[s\right]=\frac{1}{\delta}$. definimos ahora la utilizaci\'on por servidor como $u=\frac{\rho}{m}$ que tambi\'en se puede interpretar como la fracci\'on de tiempo promedio que cada servidor est\'a ocupado.\smallskip

La cola $M/M/m$ se puede considerar como un proceso de nacimiento y muerte con par\'ametros: $\beta_{n}=\beta$ para $n=0,1,2,\ldots$ y $\delta_{n}=\left\{\begin{array}{cc}
n\delta & n=0,1,\ldots,m-1\\
c\delta & n=m,\ldots\\
\end{array}\right.$

entonces  la condici\'on de recurrencia se va a cumplir s\'i y s\'olo si $\sum_{n\geq1}\frac{\beta_{0}\cdots\beta_{n-1}}{\delta_{1}\cdots\delta_{n}}<\infty$, equivalentemente se debe de cumplir que

\begin{eqnarray*}
S&=&1+\sum_{n\geq1}\frac{\beta_{0}\cdots\beta_{n-1}}{\delta_{1}\cdots\delta_{n}}=\sum_{n=0}^{m-1}\frac{\beta_{0}\cdots\beta_{n-1}}{\delta_{1}\cdots\delta_{n}}+\sum_{n=0}^{\infty}\frac{\beta_{0}\cdots\beta_{n-1}}{\delta_{1}\cdots\delta_{n}}\\
&=&\sum_{n=0}^{m-1}\frac{\beta^{n}}{n!\delta^{n}}+\sum_{n=0}^{\infty}\frac{\rho^{m}}{m!}u^{n}
\end{eqnarray*}
converja, lo cual ocurre si $u<1$, en este caso
\begin{eqnarray*}
S=\sum_{n=0}^{m-1}\frac{\rho^{n}}{n!}+\frac{\rho^{m}}{m!}\left(1-u\right)
\end{eqnarray*}
luego, para este caso se tiene que

\begin{eqnarray*}
\pi_{0}&=&\frac{1}{S}\\
\pi_{n}&=&\left\{\begin{array}{cc}
\pi_{0}\frac{\rho^{n}}{n!} & n=0,1,\ldots,m-1\\
\pi_{0}\frac{\rho^{n}}{m!m^{n-m}}& n=m,\ldots\\
\end{array}\right.
\end{eqnarray*}
Al igual que se hizo antes, determinaremos los valores de $L_{q},W_{q},W$ y $L$:
\begin{eqnarray*}
L_{q}&=&\esp\left[N_{q}\right]=\sum_{n=0}^{\infty}\left(n-m\right)\pi_{n}=\sum_{n=0}^{\infty}n\pi_{n+m}\\
&=&\sum_{n=0}^{\infty}n\pi_{0}\frac{\rho^{n+m}}{m!m^{n+m}}=\pi_{0}\frac{\rho^{m}}{m!}\sum_{n=0}^{\infty}nu^{n}=\pi_{0}\frac{u\rho^{m}}{m!}\sum_{n=0}^{\infty}\frac{d}{du}u^{n}\\
&=&\pi_{0}\frac{u\rho^{m}}{m!}\frac{d}{du}\sum_{n=0}^{\infty}u^{n}=\pi_{0}\frac{u\rho^{m}}{m!}\frac{d}{du}\left(\frac{1}{1-u}\right)=\pi_{0}\frac{u\rho^{m}}{m!}\frac{1}{\left(1-u\right)^{2}}
\end{eqnarray*}
es decir
\begin{equation}
L_{q}=\frac{u\pi_{0}\rho^{m}}{m!\left(1-u\right)^{2}}
\end{equation}
luego
\begin{equation}
W_{q}=\frac{L_{q}}{\beta}
\end{equation}
\begin{equation}
W=W_{q}+\frac{1}{\delta}
\end{equation}
Si definimos $C\left(m,\rho\right)=\frac{\pi_{0}\rho^{m}}{m!\left(1-u\right)}=\frac{\pi_{m}}{1-u}$, que es la probabilidad de que un cliente que llegue al sistema tenga que esperar en la cola. Entonces podemos reescribir las ecuaciones reci\'en enunciadas:

\begin{eqnarray*}
L_{q}&=&\frac{C\left(m,\rho\right)u}{1-u}\\
W_{q}&=&\frac{C\left(m,\rho\right)\esp\left[s\right]}{m\left(1-u\right)}\\
\end{eqnarray*}
\begin{Prop}
La cola $M/M/m$ con intensidad de tr\'afico $\rho$ es erg\'odica si y s\'olo si $\rho<1$. En este caso la distribuci\'on erg\'odica $\pi$ est\'a dada por
\begin{eqnarray*}
\pi_{n}=\left\{\begin{array}{cc}
\frac{1}{S}\frac{\eta^{n}}{n!} & 0\leq n\leq m\\
\frac{1}{S}\frac{\eta^{m}}{m!}\rho^{n-m} & m\leq n<\infty\\
\end{array}\right.
\end{eqnarray*}
\end{Prop}
\begin{Prop}
Para $t\geq0$
\begin{itemize}
\item[a)]$W_{q}\left(t\right)=1-C\left(m,\rho\right)e^{-c\delta t\left(1-u\right)}$
\item[b)]\begin{eqnarray*}
W\left(t\right)=\left\{\begin{array}{cc}
1+e^{-\delta t}\frac{\rho-m+W_{q}\left(0\right)}{m-1-\rho}+e^{-m\delta t\left(1-u\right)}\frac{C\left(m,\rho\right)}{m-1-\rho} & \rho\neq m-1\\
1-\left(1+C\left(m,\rho\right)\delta t\right)e^{-\delta t} & \rho=m-1\\
\end{array}\right.
\end{eqnarray*}
\end{itemize}
\end{Prop}
%_____________________________________________________________________________________
%
\subsection{Cola M/G/1}
%_____________________________________________________________________________________
%

Consideremos un sistema de espera con un servidor, en el que los tiempos entre arribos son exponenciales, y los tiempos de servicio tienen una distribuci\'on general $G$. Sea $N\left(t\right)_{t\geq0}$ el n\'umero de clientes en el sistema al tiempo $t$, y sean $t_{1}<t_{2}<\dots$ los tiempos sucesivos en los que los clientes completan su servicio y salen del sistema.

La sucesi\'on $\left\{X_{n}\right\}$ definida por $X_{n}=N\left(t_{n}\right)$ es una cadena de Markov, en espec\'ifico es la Cadena encajada del proceso de llegadas de usuarios. Sea $U_{n}$ el n\'umero de clientes que llegan al sistema durante el tiempo de servicio del $n$-\'esimo cliente, entonces se tiene que

\begin{eqnarray*}
X_{n+1}=\left\{\begin{array}{cc}
X_{n}-1+U_{n+1} & \textrm{si }X_{n}\geq1,\\
U_{n+1} & \textrm{si }X_{n}=0\\
\end{array}\right.
\end{eqnarray*}

Dado que los procesos de arribos de los usuarios es Poisson con par\'ametro $\lambda$, la probabilidad condicional de que lleguen $j$ clientes al sistema dado que el tiempo de servicio es $s=t$, resulta:
\begin{eqnarray*}
\prob\left\{U=j|s=t\right\}=e^{-\lambda t}\frac{\left(\lambda t\right)^{j}}{j!}\textrm{,   }j=0,1,\ldots
\end{eqnarray*}
por el teorema de la probabilidad total se tiene que
\begin{equation}
a_{j}=\prob\left\{U=j\right\}=\int_{0}^{\infty}\prob\left\{U=j|s=t\right\}dG\left(t\right)=\int_{0}^{\infty}e^{-\lambda t}\frac{\left(\lambda t\right)^{j}}{j!}dG\left(t\right)
\end{equation}
donde $G$ es la distribuci\'on de los tiempos de servicio. Las probabilidades de transici\'on de la cadena est\'an dadas por
\begin{equation}
p_{0j}=\prob\left\{U_{n+1}=j\right\}=a_{j}\textrm{, para }j=0,1,\ldots
\end{equation}
y para $i\geq1$

\begin{equation}
p_{ij}=\left\{\begin{array}{cc}
\prob\left\{U_{n+1}=j-i+1\right\}=a_{j-i+1}&\textrm{, para }j\geq i-1\\
0 & j<i-1\\
\end{array}
\right.
\end{equation}
Entonces la matriz de transici\'on es:
\begin{eqnarray*}
P=\left[\begin{array}{ccccc}
a_{0} & a_{1} & a_{2} & a_{3} & \cdots\\
a_{0} & a_{1} & a_{2} & a_{3} & \cdots\\
0 & a_{0} & a_{1} & a_{2} & \cdots\\
0 & 0 & a_{0} & a_{1} & \cdots\\
\vdots & \vdots & \cdots & \ddots &\vdots\\
\end{array}
\right]
\end{eqnarray*}
Sea $\rho=\sum_{n=0}na_{n}$, entonces se tiene el siguiente teorema:
\begin{Teo}
La cadena encajada $\left\{X_{n}\right\}$ es
\begin{itemize}
\item[a)] Recurrente positiva si $\rho<1$,
\item[b)] Transitoria si $\rho>1$,
\item[c)] Recurrente nula si $\rho=1$.
\end{itemize}
\end{Teo}
Recordemos que si la cadena de Markov $\left\{X_{n}\right\}$ tiene una distribuci\'on estacionaria entonces existe una distribuci\'on de probabilidad $\pi=\left(\pi_{0},\pi_{1},\ldots,\right)$, con $\pi_{i}\geq0$ y $\sum_{i\geq1}\pi_{i}=1$ tal que satisface la ecuaci\'on $\pi=\pi P$, equivalentemente
\begin{equation}\label{Eq.18.9}
\pi_{j}=\sum_{i=0}^{\infty}\pi_{k}p_{ij},\textrm{ para }j=0,1,2,\ldots
\end{equation}
que se puede ver como
\begin{equation}\label{Eq.19.6}
\pi_{j}=\pi_{0}a_{j}+\sum_{i=1}^{j+1}\pi_{i}a_{j-i+1}\textrm{, para }j=0,1,\ldots
\end{equation}
si definimos\[\pi\left(z\right)=\sum_{j=0}^{\infty}\pi_{j}z^{j}\] y \[A\left(z\right)=\sum_{j=0}^{\infty}a_{j}z^{j}\] con $|z_{j}|\leq1$.
Si la ecuaci\'on \ref{Eq.19.6} la multiplicamos por $z^{j}$ y sumando sobre $j$, se tiene que
\begin{eqnarray*}
\sum_{j=0}^{\infty}\pi_{j}z^{j}&=&\sum_{j=0}^{\infty}\pi_{0}a_{j}z^{j}+\sum_{j=0}^{\infty}\sum_{i=1}^{j+1}\pi_{i}a_{j-i+1}z^{j}\\
&=&\pi_{0}\sum_{j=0}^{\infty}a_{j}z^{j}+\sum_{j=0}^{\infty}a_{j}z^{j}\sum_{i=1}^{\infty}\pi_{i}a_{i-1}\\
&=&\pi_{0}A\left(z\right)+A\left(z\right)\left(\frac{\pi\left(z\right)-\pi_{0}}{z}\right)\\
\end{eqnarray*}
es decir,

\begin{equation}
\pi\left(z\right)=\pi_{0}A\left(z\right)+A\left(z\right)\left(\frac{\pi\left(z\right)-\pi_{0}}{z}\right)\Leftrightarrow\pi\left(z\right)=\frac{\pi_{0}A\left(z\right)\left(z-1\right)}{z-A\left(z\right)}
\end{equation}
Si $z\rightarrow1$, entonces $A\left(z\right)\rightarrow A\left(1\right)=1$, y adem\'as $A^{'}\left(z\right)\rightarrow A^{'}\left(1\right)=\rho$. Si aplicamos la Regla de L'Hospital se tiene que
\begin{eqnarray*}
\sum_{j=0}^{\infty}\pi_{j}=lim_{z\rightarrow1^{-}}\pi\left(z\right)=\pi_{0}lim_{z\rightarrow1^{-}}\frac{z-1}{z-A\left(z\right)}=\frac{\pi_{0}}{1-\rho}
\end{eqnarray*}

Retomando,
\begin{eqnarray*}
a_{j}=\prob\left\{U=j\right\}=\int_{0}^{\infty}e^{-\lambda t}\frac{\left(\lambda t\right)^{n}}{n!}dG\left(t\right)\textrm{, para }n=0,1,2,\ldots
\end{eqnarray*}
entonces
\begin{eqnarray*}
\rho&=&\sum_{n=0}^{\infty}na_{n}=\sum_{n=0}^{\infty}n\int_{0}^{\infty}e^{-\lambda t}\frac{\left(\lambda t\right)^{n}}{n!}dG\left(t\right)\\
&=&\int_{0}^{\infty}\sum_{n=0}^{\infty}ne^{-\lambda t}\frac{\left(\lambda t\right)^{n}}{n!}dG\left(t\right)=\int_{0}^{\infty}\lambda tdG\left(t\right)=\lambda\esp\left[s\right]
\end{eqnarray*}
Adem\'as, se tiene que $\rho=\beta\esp\left[s\right]=\frac{\beta}{\delta}$ y la distribuci\'on estacionaria est\'a dada por
\begin{eqnarray}
\pi_{j}&=&\pi_{0}a_{j}+\sum_{i=1}^{j+1}\pi_{i}a_{j-i+1}\textrm{, para }j=0,1,\ldots\\
\pi_{0}&=&1-\rho
\end{eqnarray}
Por otra parte se tiene que

\begin{equation}
L=\pi^{'}\left(1\right)=\rho+\frac{A^{''}\left(1\right)}{2\left(1-\rho\right)}
\end{equation}
%____________________________________________________________________________
pero $A^{''}\left(1\right)=\sum_{n=1}n\left(n-1\right)a_{n}= \esp\left[U^{2}\right]-\esp\left[U\right]$, $\esp\left[U\right]=\rho$ y $\esp\left[U^{2}\right]=\lambda^{2}\esp\left[s^{2}\right]+\rho$. Por lo tanto $L=\rho+\frac{\beta^{2}\esp\left[s^{2}\right]}{2\left(1-\rho\right)}$.

De las f\'ormulas de Little, se tiene que $W=E\left(w\right)=\frac{L}{\beta}$, tambi\'en el tiempo de espera en la cola
\begin{equation}
W_{q}=\esp\left(q\right)=\esp\left(w\right)-\esp\left(s\right)=\frac{L}{\beta}-\frac{1}{\delta},
\end{equation}
adem\'as el n\'umero promedio de clientes en la cola es
\begin{equation}
L_{q}=\esp\left(N_{q}\right)=\beta W_{q}=L-\rho
\end{equation}
%_____________________________________________________________________________________
%
\section{Redes de Colas}
%_____________________________________________________________________________________
%
%\begin{frame}\frametitle{Redes de Colas}
%\end{frame}

%_____________________________________________________________________________________
%
\section{Estacionareidad}
%_____________________________________________________________________________________
%}

Sea $v=\left(v_{i}\right)_{i\in E}$ medida no negativa en $E$,
podemos definir una nueva medida $v\prob$ que asigna masa
$\sum_{i\in E}v_{i}p_{ij}$ a cada estado $j$.\smallskip

\begin{Def}
La medida $v$ es estacionaria si $v_{i}<\infty$ para toda $i$ y
adem\'as $v\prob=v$.\smallskip
\end{Def}
En el caso de que $v$ sea distribuci\'on, independientemente de que
sea estacionaria o no, se cumple con

\begin{eqnarray*}
\prob_{v}\left[X_{1}=j\right]=\sum_{i\in
E}\prob_{v}\left[X_{0}=i\right]p_{ij}=\sum_{i\in
E}v_{i}p_{ij}=\left(vP\right)_{j}
\end{eqnarray*}

\begin{Teo}
Supongamos que $v$ es una distribuci\'on estacionaria. Entonces
\begin{itemize}
\item[i)] La cadena es estrictamente estacionaria con respecto a
$\prob_{v}$, es decir, $\prob_{v}$-distribuci\'on de
$\left\{X_{n},X_{n+1},\ldots\right\}$ no depende de $n$;
\item[ii)] Existe una versi\'on estrictamente estacionaria
$\left\{X_{n}\right\}_{n\in Z}$ de la cadena con doble tiempo
infinito y $\prob\left(X_{n}=i\right)=v_{i}$ para toda $n\in Z$.
\end{itemize}
\end{Teo}
\begin{Teo}
Sea $i$ estado fijo, recurrente. Entonces una medida estacionaria
$v$ puede definirse haciendo que $v_{j}$ sea el n\'umero esperado de
visitas a $j$ entre dos visitas consecutivas $i$,
\begin{equation}\label{Eq.3.1}
v_{j}=\esp_{i}\sum_{n=0}^{\tau(i)-1}\indora\left(X_{n}=i\right)=\sum_{n=0}^{\infty}\prob_{i}\left[X_{n}=j,\tau(i)>n\right]
\end{equation}
\end{Teo}
\begin{Teo}\label{Teo.3.3}
Si la cadena es irreducible y recurrente, entonces una medida
estacionaria $v$ existe, satisface $0<v_{j}<\infty$ para toda $j$
y es \'unica salvo factores multiplicativos, es decir, si $v,v^{*}$
son estacionarias, entonces $c=cv^{*}$ para alguna
$c\in\left(0,\infty\right)$.
\end{Teo}
\begin{Cor}\label{Cor.3.5}
Si la cadena es irreducible y positiva recurrente, existe una
\'unica distribuci\'on estacionaria $\pi$ dada por
\begin{equation}
\pi_{j}=\frac{1}{\esp_{i}\tau_{i}}\esp_{i}\sum_{n=0}^{\tau\left(i\right)-1}\indora\left(X_{n}=j\right)=\frac{1}{\esp_{j}\tau\left(j\right)}.
\end{equation}
\end{Cor}
\begin{Cor}\label{Cor.3.6}
Cualquier cadena de Markov irreducible con un espacio de estados
finito es positiva recurrente.
\end{Cor}
\begin{Lema}
Sea $\left\{X_{n}\right\}$ cadena irreducible y se $F$ subconjunto
finito del espacio de estados. Entonces la cadena es positiva
recurrente si $\esp_{i}\tau\left(F\right)<\infty$ para todo $i\in
F$.
\end{Lema}

\begin{Prop}
Sea $\left\{X_{n}\right\}$ cadena irreducible y transiente o cero
recurrente, entonces $p_{ij}^{n}\rightarrow0$ conforme
$n\rightarrow\infty$ para cualquier $i,j\in E$, $E$ espacio de
estados.
\end{Prop}
Utilizando el teorema (2.2) y el corolario ref{Cor.3.5}, se
demuestra el siguiente resultado importante.

\begin{Teo}
Sea $\left\{X_{n}\right\}$ cadena irreducible y aperi\'odica
positiva recurrente, y sea $\pi=\left\{\pi_{j}\right\}_{j\in E}$
\end{Teo}



%_____________________________________________________________________________________
%
\section{Procesos de Markov de Saltos}
%_____________________________________________________________________________________
%


Sea $E$ espacio discreto de estados, finito o numerable, y sea
$\left\{X_{t}\right\}$ un proceso de Markov con espacio de estados
$E$. Una medida $\mu$ en $E$ definida por sus probabilidades
puntuales $\mu_{i}$, escribimos
$p_{ij}^{t}=P^{t}\left(i,\left\{j\right\}\right)=P_{i}\left(X_{t}=j\right)$.\smallskip

El monto del tiempo gastado en cada estado es positivo, de modo
tal que las trayectorias muestrales son constantes por partes.
Para un proceso de saltos denotamos por los tiempos de saltos a
$S_{0}=0<S_{1}<S_{2}\cdots$, los tiempos entre saltos consecutivos
$T_{n}=S_{n+1}-S_{n}$ y la secuencia de estados visitados por
$Y_{0},Y_{1},\ldots$, as\'i las trayectorias muestrales son
constantes entre $S_{n}$ consecutivos, continua por la derecha, es
decir, $X_{S_{n}}=Y_{n}$.
\begin{Teo}
Cualquier Proceso de Markov de Saltos satisface la Propiedad
Fuerte de Markov
\end{Teo}

%\begin{Teo}Consid\'erese un proceso de Saltos de Markov, entonces la distribuci\'on conjunta de la sucesi\'on $\left\{Y_{n}\right\}_{n\in\nat}$ de los estados visitados, antes de explotar, y $\left\{T_{n}\right\}_{n\in\nat}$\end{Teo}




\begin{Def}
Una medida $v\neq0$ es estacionaria si $0\leq v_{j}<\infty$,
$vP^{t}=v$ para toda $t$.
\end{Def}

\begin{Teo}\label{Teo.4.2}
Supongamos que $\left\{X_{t}\right\}$ es irreducible recurrente en
$E$. Entonces existe una y s\'olo una, salvo m\'ultiplos, medida
estacionaria $v$. Esta $v$ tiene la propiedad de que $0<
v_{j}<\infty$ para todo $j$ y puede encontrarse en cualquiera de
las siguientes formas

\begin{itemize}
\item[i)] Para alg\'un estado $i$, fijo pero arbitrario, $v_{j}$ es
el tiempo esperado utilizado en $j$ entre dos llegadas
consecutivas al estado $i$;
\begin{equation}\label{Eq.4.2}
v_{j}=\esp_{i}\int_{0}^{w\left(i\right)}\indora\left(X_{t}=j\right)dt
\end{equation}
con
$w\left(i\right)=\inf\left\{t>0:X_{t}=i,X_{t^{-}}=\lim_{s\uparrow
t}X_{s}\neq i\right\}$. \item[ii)]
$v_{j}=\frac{\mu_{j}}{\lambda\left(j\right)}$, donde $\mu$ es
estacionaria para $\left\{Y_{n}\right\}$. \item[iii)] como
soluci\'on de $v\Lambda=0$.
\end{itemize}
\end{Teo}
\begin{Def}
Un proceso irreducible recurrente con medida estacionaria de masa
finita es llamado erg\'odico.
\end{Def}

\begin{Teo}\label{Teo.4.3}
Un proceso de Markov de saltos irreducible no explosivo es
erg\'odico si y s\'olo si se puede encontrar una soluci\'on, de
probabilidad, $\pi$, con $|\pi|=1$ y $0\leq\pi_{j}\leq1$, a
$\pi\Lambda=0$. En este caso $\pi$ es la distribuci\'on
estacionaria.
\end{Teo}

\begin{Cor}\label{Cor.4.4}
Una condici\'on suficiente para la ergodicidad de un proceso
irreducible es la existencia de una probabilidad $\pi$ que
resuelva el sistema $\pi\Lambda=0$ y que adem\'as tenga la propiedad
de que $\sum\pi_{j}\lambda\left(j\right)$.
\end{Cor}
%_____________________________________________________________________________________
%
\section{Notaci\'on Kendall-Lee}
%_____________________________________________________________________________________
%

A partir de este momento se har\'an las siguientes consideraciones:
Si $t_{n}$ es el tiempo aleatorio en el que llega al sistema el
$n$-\'esimo cliente, para $n=1,2,\ldots$, $t_{0}=0$ y
$t_{0}<t_{1}<\cdots$ se definen los tiempos entre arribos
$\tau_{n}=t_{n}-t_{n-1}$ para $n=1,2,\ldots$, variables aleatorias
independientes e id\'enticamente distribuidas. Los tiempos entre
arribos tienen un valor medio $E\left(\tau\right)$ finito y
positivo $\frac{1}{\beta}$, es decir, $\beta$ se puede ver como la
tasa o intensidad promedio de arribos al sistema por unidad de
tiempo. Adem\'as se supondr\'a que los servidores son identicos y si
$s$ denota la variable aleatoria que describe el tiempo de
servicio, entonces $E\left(s\right)=\frac{1}{\delta}$, $\delta$ es
la tasa promedio de servicio por servidor.

La notaci\'on de Kendall-Lee es una forma abreviada de describir un
sisema de espera con las siguientes componentes:
\begin{itemize}
\item[i)] {\em\bf Fuente}: Poblaci\'on de clientes potenciales del
sistema, esta puede ser finita o infinita. \item[ii)] {\em\bf
Proceso de Arribos}: Proceso determinado por la funci\'on de
distribuci\'on $A\left(t\right)=P\left\{\tau\leq t\right\}$ de los
tiempos entre arribos.
\end{itemize}

Adem\'as tenemos las siguientes igualdades
\begin{equation}\label{Eq.0.1}
N\left(t\right)=N_{q}\left(t\right)+N_{s}\left(s\right)
\end{equation}
donde
\begin{itemize}
\item $N\left(t\right)$ es el n\'umero de clientes en el sistema al
tiempo $t$. \item $N_{q}\left(t\right)$ es el n\'umero de clientes
en la cola al tiempo $t$ \item $N_{s}\left(t\right)$ es el n\'umero
de clientes recibiendo servicio en el tiempo $t$.
\end{itemize}

Bajo la hip\'otesis de estacionareidad, es decir, las
caracter\'isticas de funcionamiento del sistema se han estabilizado
en valores independientes del tiempo, entonces
\begin{equation}
N=N_{q}+N_{s}.
\end{equation}

Los valores medios de las cantidades anteriores se escriben como
$L=E\left(N\right)$, $L_{q}=E\left(N_{q}\right)$ y
$L_{s}=E\left(N_{s}\right)$, entonces de la ecuaci\'on \ref{Eq.0.1}
se obtiene

\begin{equation}
L=L_{q}+L_{s}
\end{equation}
Si $q$ es el tiempo que pasa un cliente en la cola antes de
recibir servicio, y W es el tiempo total que un cliente pasa en el
sistema, entonces
\[w=q+s\]
por lo tanto
\[W=W_{q}+W_{s},\]
donde $W=E\left(w\right)$, $W_{q}=E\left(q\right)$ y
$W_{s}=E\left(s\right)=\frac{1}{\delta}$.

La intensidad de tr\'afico se define como
\begin{equation}
\rho=\frac{E\left(s\right)}{E\left(\tau\right)}=\frac{\beta}{\delta}.
\end{equation}

La utilizaci\'on por servidor es
\begin{equation}
u=\frac{\rho}{c}=\frac{\beta}{c\delta}.
\end{equation}
donde $c$ es el n\'umero de servidores.

%_____________________________________________________________________________________
%
\section{Procesos de Nacimiento y Muerte (Teor\'ia)}
%_____________________________________________________________________________________
%

\begin{Prop}\label{Prop.2.1}
La recurrencia de $\left\{X_{t}\right\}$, o equivalentemente de
$\left\{Y_{n}\right\}$ es equivalente a
\begin{equation}\label{Eq.2.1}
\sum_{n=1}^{\infty}\frac{\delta_{1}\cdots\delta_{n}}{\beta_{1}\cdots\beta_{n}}=\sum_{n=1}^{\infty}\frac{q_{1}\cdots
q_{n}}{p_{1}\cdots p_{n}}=\infty
\end{equation}
\end{Prop}

\begin{Lema}\label{Lema.2.2}
Independientemente de la recurrencia o transitorieadad, existe una
y s\'olo una, salvo m\'ultiplos, soluci\'on a $v\Lambda=0$, dada por
\begin{equation}\label{Eq.2.2}
v_{n}=\frac{\beta_{0}\cdots\beta_{n-1}}{\delta_{1}\cdots\delta_{n}}v_{0}
\end{equation}
para $n=1,2,\ldots$.
\end{Lema}





\begin{Cor}\label{Cor.2.3}
En el caso recurrente, la medida estacionaria $\mu$ para
$\left\{Y_{n}\right\}$ est\'a dada por
\begin{equation}
\mu_{n}=\frac{p_{1}\cdots p_{n-1}}{q_{1}\cdots q_{n}}\mu_{0}
\end{equation}
para $n=1,2,\ldots$.
\end{Cor}

Se define a
$S=1+\sum_{n=1}^{\infty}\frac{\beta_{0}\beta_{1}\cdots\beta_{n-1}}{\delta_{1}\delta_{2}\cdots\delta_{n}}$

\begin{Cor}\label{Cor.2.4}
$\left\{X_{t}\right\}$ es erg\'odica si y s\'olo si la ecuaci\'on
(\ref{Eq.2.1}) se cumple y adem\'as $S<\infty$, en cuyo caso la
distribuci\'on erg\'odica, $\pi$, est\'a dada por
\begin{equation}\label{Eq.2.4}
\pi_{0}=\frac{1}{S}\textrm{,
}\pi_{n}=\frac{1}{S}\frac{\beta_{0}\cdots\beta_{n-1}}{\delta_{1}\cdots\delta_{n}}
\end{equation}
para $n=1,2,\ldots$.
\end{Cor}


%_____________________________________________________________________________________
%
\section{Procesos de Nacimiento y Muerte como Modelos de Colas}
%_____________________________________________________________________________________
%

%_____________________________________________________________________________________
%
\subsection{Cola $M/M/1$}
%_____________________________________________________________________________________
%




Este modelo corresponde a un proceso de nacimiento y muerte con
$\beta_{n}=\beta$ y $\delta_{n}=\delta$ independiente del valor de
$n$. La intensidad de tr\'afico $\rho=\frac{\beta}{\delta}$, implica
que el criterio de recurrencia (ecuaci\'on \ref{Eq.2.1}) quede de la
forma:
\begin{eqnarray*}
1+\sum_{n=1}^{\infty}\rho^{-n}=\infty.
\end{eqnarray*}
Equivalentemente el proceso es recurrente si y s\'olo si
\begin{eqnarray*}
\sum_{n\geq1}\left(\frac{\beta}{\delta}\right)^{n}<\infty\Leftrightarrow
\frac{\beta}{\delta}<1
\end{eqnarray*}
Entonces $S=\frac{\delta}{\delta-\beta}$, luego por la ecuaci\'on
\ref{Eq.2.4} se tiene que
\begin{eqnarray*}
\pi_{0}&=&\frac{\delta-\beta}{\delta}=1-\frac{\beta}{\delta}\\
\pi_{n}&=&\pi_{0}\left(\frac{\beta}{\delta}\right)^{n}=\left(1-\frac{\beta}{\delta}\right)\left(\frac{\beta}{\delta}\right)^{n}=\left(1-\rho\right)\rho^{n}
\end{eqnarray*}



Lo cual nos lleva a la siguiente

\begin{Prop}
La cola $M/M/1$ con intendisad de tr\'afico $\rho$, es recurrente si
y s\'olo si $\rho\leq1$.
\end{Prop}

Entonces por el corolario \ref{Cor.2.3}

\begin{Prop}
La cola $M/M/1$ con intensidad de tr\'afico $\rho$ es erg\'odica si y
s\'olo si $\rho<1$. En cuyo caso, la distribuci\'on de equilibrio
$\pi$ de la longitud de la cola es geom\'etrica,
$\pi_{n}=\left(1-\rho\right)\rho^{n}$, para $n=1,2,\ldots$.
\end{Prop}
De la proposici\'on anterior se desprenden varios hechos
importantes.
\begin{enumerate}
\item $\prob\left[X_{t}=0\right]=\pi_{0}=1-\rho$, es decir, la
probabilidad de que el sistema se encuentre ocupado. \item De las
propiedades de la distribuci\'on Geom\'etrica se desprende que
\begin{enumerate}
\item $\esp\left[X_{t}\right]=\frac{\rho}{1-\rho}$, \item
$Var\left[X_{t}\right]=\frac{\rho}{\left(1-\rho\right)^{2}}$.
\end{enumerate}
\end{enumerate}



Si $L$ es el n\'umero esperado de clientes en el sistema, incluyendo
los que est\'an siendo atendidos, entonces
\begin{eqnarray*}
L=\frac{\rho}{1-\rho}
\end{eqnarray*}
Si adem\'as $W$ es el tiempo total del cliente en la cola:
$W=W_{q}+W_{s}$
$\rho=\frac{\esp\left[s\right]}{\esp\left[\tau\right]}=\beta
W_{s}$, puesto que $W_{s}=\esp\left[s\right]$ y
$\esp\left[\tau\right]=\frac{1}{\delta}$. Por la f\'ormula de Little
$L=\lambda W$
\begin{eqnarray*}
W&=&\frac{L}{\beta}=\frac{\frac{\rho}{1-\rho}}{\beta}=\frac{\rho}{\delta}\frac{1}{1-\rho}=\frac{W_{s}}{1-\rho}\\
&=&\frac{1}{\delta\left(1-\rho\right)}=\frac{1}{\delta-\beta}
\end{eqnarray*}
luego entonces
\begin{eqnarray*}
W_{q}&=&W-W_{s}=\frac{1}{\delta-\beta}-\frac{1}{\delta}=\frac{\beta}{\delta(\delta-\beta)}\\
&=&\frac{\rho}{1-\rho}\frac{1}{\delta}=\esp\left[s\right]\frac{\rho}{1-\rho}
\end{eqnarray*}


Entonces
\begin{eqnarray*}
L_{q}=\beta W_{q}=\frac{\rho^{2}}{1-\rho}.
\end{eqnarray*}
Finalmente
\begin{Prop}
\begin{enumerate}
\item $W\left(t\right)=1-e^{-\frac{t}{W}}$. \item
$W_{q}\left(t\right)=1-\rho\exp^{-\frac{t}{W}}$.
\end{enumerate}
donde $W=\esp(w)$.
\end{Prop}


%_____________________________________________________________________________________
%
\section{Notaci\'on Kendall-Lee, segunda parte}
%_____________________________________________________________________________________
%


Esta notaci\'on es una forma abreviada de describir un sistema de
espera con componentes dados a continuaci\'on, la notaci\'on es
\begin{equation}\label{Notacion.K.L.}
A/S/c/K/F/d
\end{equation}
Cada una de las letras describe:
\begin{itemize}
\item $A$ es la distribuci\'on de los tiempos entre arribos. \item
$S$ es la distribuci\'on del tiempo de servicio. \item $c$ es el
n\'umero de servidores. \item $K$ es la capacidad del sistema. \item
$F$ es el n\'umero de individuos en la fuente. \item $d$ es la
disciplina del servicio
\end{itemize}
Usualmente se acostumbra suponer que $K=\infty$, $F=\infty$ y
$d=FIFO$, es decir, First In First Out.
Las distribuciones usuales para $A$ y $B$ son:
\begin{itemize}
\item $GI$ para la distribuci\'on general de los tiempos entre
arribos. \item $G$ distribuci\'on general del tiempo de servicio.
\item $M$ Distribuci\'on exponencial para $A$ o $S$. \item $E_{K}$
Distribuci\'on Erlang-$K$, para $A$ o $S$. \item $D$ tiempos entre
arribos o de servicio constantes, es decir, deterministicos.
\end{itemize}


%_____________________________________________________________________________________
%
\section{Procesos de Nacimiento y Muerte como Modelos de Colas(Continuaci\'on)}
%_____________________________________________________________________________________
%


%_____________________________________________________________________________________
%
\subsection{Cola $M/M/\infty$}
%_____________________________________________________________________________________
%


Este modelo corresponde al caso en que $\beta_{n}=\beta$ y
$\delta_{n}=n\delta$, en este caso el par\'ametro de inter\'es
$\eta=\frac{\beta}{\delta}$, luego, la ecuaci\'on \ref{Eq.2.1} queda
de la forma:

\begin{eqnarray*}
\sum_{n=1}^{\infty}\frac{\delta_{1}\cdots\delta_{n}}{\beta_{1}\cdots\beta_{n}}=\sum_{n=1}^{\infty}n!\eta^{-n}=\infty\\
\end{eqnarray*}
con $S=1+\sum_{n=1}^{\infty}\frac{\eta^{n}}{n!}=e$

\begin{Prop}
La cola $M/M/\infty$ es erg\'odica para todos los valores de $\eta$.
La distribuci\'on de equilibrio $\pi$ es Poisson con media $\eta$,
$\pi_{n}=\frac{e^{-n}\eta^{n}}{n!}$.
\end{Prop}


%_____________________________________________________________________________________
%
\subsection{Cola $M/M/m$}
%_____________________________________________________________________________________
%


Para este caso $\beta_{n}=\beta$ y
$\delta_{n}=m\left(n\right)\delta$, donde $m\left(n\right)$ es el
n\'umero de servidores ocupados en el estado $n$, es decir,
$m\left(n\right)=m$, para $n\geq m$ y $m\left(n\right)=m$ para
$1\leq n\leq m$. La intensidad de tr\'afico es
$\rho=\frac{\beta}{m\delta}$ y $\frac{\beta_{n}}{\delta_{n}}=\rho$
para $n\geq m$. As\'i, al igual que en el caso $m=1$, la ecuaci\'on
\ref{Eq.2.1} y la recurrencia se cumplen si y s\'olo si
$\sum_{n=1}^{\infty}\rho^{-n}=\infty$, es decir, cuando
$\rho\leq1$. Similarmente, con $\eta=\frac{\beta}{\delta}$ se
tiene que
\begin{eqnarray*}
S&=&1+\sum_{n\geq1}\frac{\beta_{0}\cdots\beta_{n-1}}{\delta_{1}\cdots\delta_{n}}=\sum_{n\geq0}\frac{\eta^{n}}{n!}+\frac{\eta^{m}}{m!}\sum_{n\geq0}\rho^{n}\\
&=&\sum_{n=0}^{m-1}\frac{\eta^{n}}{n!}+\frac{\eta^{m}}{m!}\left(1-\rho\right)^{-1}
\end{eqnarray*}
es finita si y s\'olo si $\rho<1$, por tanto se tiene la siguiente

\begin{Prop}
La cola $M/M/m$ con intensidad de tr\'afico $\rho$ es erg\'odica si y
s\'olo si $\rho<1$. En este caso la distribuci\'on erg\'odica $\pi$ est\'a
dada por
\begin{eqnarray*}
\pi_{n}=\left\{\begin{array}{cc}
\frac{1}{S}\frac{\eta^{n}}{n!} & 0\leq n\leq m\\
\frac{1}{S}\frac{\eta^{m}}{m!}\rho^{n-m} & m\leq n<\infty\\
\end{array}\right.
\end{eqnarray*}
\end{Prop}


%_____________________________________________________________________________________
%
\section{Cadenas de Markov}
%_____________________________________________________________________________________
%
\subsection{Estacionareidad}
%_____________________________________________________________________________________
%}


Sea $v=\left(v_{i}\right)_{i\in E}$ medida no negativa en $E$,
podemos definir una nueva medida $v\prob$ que asigna masa
$\sum_{i\in E}v_{i}p_{ij}$ a cada estado $j$.\smallskip

\begin{Def}
La medida $v$ es estacionaria si $v_{i}<\infty$ para toda $i$ y
adem\'as $v\prob=v$.\smallskip
\end{Def}
En el caso de que $v$ sea distribuci\'on, independientemente de que
sea estacionaria o no, se cumple con

\begin{eqnarray*}
\prob_{v}\left[X_{1}=j\right]=\sum_{i\in
E}\prob_{v}\left[X_{0}=i\right]p_{ij}=\sum_{i\in
E}v_{i}p_{ij}=\left(vP\right)_{j}
\end{eqnarray*}


\begin{Teo}
Supongamos que $v$ es una distribuci\'on estacionaria. Entonces
\begin{itemize}
\item[i)] La cadena es estrictamente estacionaria con respecto a
$\prob_{v}$, es decir, $\prob_{v}$-distribuci\'on de
$\left\{X_{n},X_{n+1},\ldots\right\}$ no depende de $n$;
\item[ii)] Existe un aversi\'on estrictamente estacionaria
$\left\{X_{n}\right\}_{n\in Z}$ de la cadena con doble tiempo
infinito y $\prob\left(X_{n}=i\right)=v_{i}$ para toda $n\in Z$.
\end{itemize}
\end{Teo}
\begin{Teo}
Sea $i$ estado fijo, recurrente. Entonces una medida estacionaria
$v$ puede definirse haciendo que $v_{j}$ sea el n\'umero esperado de
visitas a $j$ entre dos visitas consecutivas $i$,
\begin{equation}\label{Eq.3.1}
v_{j}=\esp_{i}\sum_{n=0}^{\tau(i)-1}\indora\left(X_{n}=i\right)=\sum_{n=0}^{\infty}\prob_{i}\left[X_{n}=j,\tau(i)>n\right]
\end{equation}
\end{Teo}


\begin{Teo}\label{Teo.3.3}
Si la cadena es irreducible y recurrente, entonces una medida
estacionaria $v$ existe, satisface $0<v_{j}<\infty$ para toda $j$
y es \'unica salvo factores multiplicativos, es decir, si $v,v^{*}$
son estacionarias, entonces $c=cv^{*}$ para alguna
$c\in\left(0,\infty\right)$.
\end{Teo}
\begin{Cor}\label{Cor.3.5}
Si la cadena es irreducible y positiva recurrente, existe una
\'unica distribuci\'on estacionaria $\pi$ dada por
\begin{equation}
\pi_{j}=\frac{1}{\esp_{i}\tau_{i}}\esp_{i}\sum_{n=0}^{\tau\left(i\right)-1}\indora\left(X_{n}=j\right)=\frac{1}{\esp_{j}\tau\left(j\right)}.
\end{equation}
\end{Cor}
\begin{Cor}\label{Cor.3.6}
Cualquier cadena de Markov irreducible con un espacio de estados
finito es positiva recurrente.
\end{Cor}

%_____________________________________________________________________________________
%
\subsection{Teor\'ia Erg\'odica}
%_____________________________________________________________________________________
%

\begin{Lema}
Sea $\left\{X_{n}\right\}$ cadena irreducible y se $F$ subconjunto
finito del espacio de estados. Entonces la cadena es positiva
recurrente si $\esp_{i}\tau\left(F\right)<\infty$ para todo $i\in
F$.
\end{Lema}

\begin{Prop}
Sea $\left\{X_{n}\right\}$ cadena irreducible y transiente o cero
recurrente, entonces $p_{ij}^{n}\rightarrow0$ conforme
$n\rightarrow\infty$ para cualquier $i,j\in E$, $E$ espacio de
estados.
\end{Prop}
Utilizando el teorema (2.2) y el corolario ref{Cor.3.5}, se
demuestra el siguiente resultado importante.

\begin{Teo}
Sea $\left\{X_{n}\right\}$ cadena irreducible y aperi\'odica
positiva recurrente, y sea $\pi=\left\{\pi_{j}\right\}_{j\in E}$
la distribuci\'on estacionaria. Entonces
$p_{ij}^{n}\rightarrow\pi_{j}$ para todo $i,j$.
\end{Teo}
\begin{Def}\label{Def.Ergodicidad}
Una cadena irreducible aperiodica, positiva recurrente con medida
estacionaria $v$, es llamada {\em erg\'odica}.
\end{Def}

\begin{Prop}\label{Prop.4.4}
Sea $\left\{X_{n}\right\}$ cadena irreducible y recurrente con
medida estacionaria $v$, entocnes para todo $i,j,k,l\in E$
\begin{equation}
\frac{\sum_{n=0}^{m}p_{ij}^{n}}{\sum_{n=0}^{m}p_{lk}^{n}}\rightarrow\frac{v_{j}}{v_{k}}\textrm{,
}m\rightarrow\infty
\end{equation}
\end{Prop}
\begin{Lema}\label{Lema.4.5}
La matriz $\widetilde{P}$ con elementos
$\widetilde{p}_{ij}=\frac{v_{ji}p_{ji}}{v_{i}}$ es una matriz de
transici\'on. Adem\'as, el $i$-\'esimo elementos
$\widetilde{p}_{ij}^{m}$ de la matriz potencia $\widetilde{P}^{m}$
est\'a dada por
$\widetilde{p}_{ij}^{m}=\frac{v_{ji}p_{ji}^{m}}{v_{i}}$.
\end{Lema}

\begin{Lema}
Def\'inase $N_{i}^{m}=\sum_{n=0}^{m}\indora\left(X_{n}=i\right)$
como el n\'umero de visitas a $i$ antes del tiempo $m$. Entonces si
la cadena es reducible y recurrente,
$lim_{m\rightarrow\infty}\frac{\esp_{j}N_{i}^{m}}{\esp_{k}N_{i}^{m}}=1$
para todo $j,k\in E$.
\end{Lema}

%_____________________________________________________________________________________
%
\subsection{Funciones Arm\'onicas, Recurrencia y Transitoriedad}
%_____________________________________________________________________________________
%

\begin{Def}\label{Def.Armonica}
Una funci\'on Arm\'onica es el eigenvector derecho $h$ de $P$
corrrespondiente al eigenvalor 1.
\end{Def}
\begin{eqnarray*}
Ph=h\Leftrightarrow h\left(i\right)=\sum_{j\in
E}p_{ij}h\left(j\right)=\esp_{i}h\left(X_{1}\right)=\esp\left[h\left(X_{n+1}\right)|X_{n}=i\right].
\end{eqnarray*}
es decir, $\left\{h\left(X_{n}\right)\right\}$ es martingala.
\begin{Prop}\label{Prop.5.2}
Sea $\left\{X_{n}\right\}$ cadena irreducible  y sea $i$ estado
fijo arbitrario. Entonces la cadena es transitoria s\'i y s\'olo si
existe una funci\'on no cero, acotada
$h:E-\left\{i\right\}\rightarrow\rea$ que satisface
\begin{equation}\label{Eq.5.1}
h\left(j\right)=\sum_{k\neq i}p_{jk}h\left(k\right)\textrm{   para
}j\neq i.
\end{equation}
\end{Prop}




%_____________________________________________________________________________________
%
\section{Procesos de Markov de Saltos}
%_____________________________________________________________________________________
%


Sea $E$ espacio discreto de estados, finito o numerable, y sea
$\left\{X_{t}\right\}$ un proceso de Markov con espacio de estados
$E$. Una medida $\mu$ en $E$ definida por sus probabilidades
puntuales $\mu_{i}$, escribimos
$p_{ij}^{t}=P^{t}\left(i,\left\{j\right\}\right)=P_{i}\left(X_{t}=j\right)$.\smallskip

El monto del tiempo gastado en cada estado es positivo, de modo
tal que las trayectorias muestrales son constantes por partes.
Para un proceso de saltos denotamos por los tiempos de saltos a
$S_{0}=0<S_{1}<S_{2}\cdots$, los tiempos entre saltos consecutivos
$T_{n}=S_{n+1}-S_{n}$ y la secuencia de estados visitados por
$Y_{0},Y_{1},\ldots$, as\'i las trayectorias muestrales son
constantes entre $S_{n}$ consecutivos, continua por la derecha, es
decir, $X_{S_{n}}=Y_{n}$.

\begin{Teo}
Cualquier Proceso de Markov de Saltos satisface la Propiedad
Fuerte de Markov
\end{Teo}

%\begin{Teo}Consid\'erese un proceso de Saltos de Markov, entonces la distribuci\'on conjunta de la sucesi\'on $\left\{Y_{n}\right\}_{n\in\nat}$ de los estados visitados, antes de explotar, y $\left\{T_{n}\right\}_{n\in\nat}$\end{Teo}



\begin{Def}
Una medida $v\neq0$ es estacionaria si $0\leq v_{j}<\infty$,
$vP^{t}=v$ para toda $t$.
\end{Def}

\begin{Teo}\label{Teo.4.2}
Supongamos que $\left\{X_{t}\right\}$ es irreducible recurrente en
$E$. Entonces existe una y s\'olo una, salvo m\'ultiplos, medida
estacionaria $v$. Esta $v$ tiene la propiedad de que $0<
v_{j}<\infty$ para todo $j$ y puede encontrarse en cualquiera de
las siguientes formas

\begin{itemize}
\item[i)] Para alg\'un estado $i$, fijo pero arbitrario, $v_{j}$ es
el tiempo esperado utilizado en $j$ entre dos llegadas
consecutivas al estado $i$;
\begin{equation}\label{Eq.4.2}
v_{j}=\esp_{i}\int_{0}^{w\left(i\right)}\indora\left(X_{t}=j\right)dt
\end{equation}
con
$w\left(i\right)=\inf\left\{t>0:X_{t}=i,X_{t^{-}}=\lim_{s\uparrow
t}X_{s}\neq i\right\}$. \item[ii)]
$v_{j}=\frac{\mu_{j}}{\lambda\left(j\right)}$, donde $\mu$ es
estacionaria para $\left\{Y_{n}\right\}$. \item[iii)] como
soluci\'on de $v\Lambda=0$.
\end{itemize}
\end{Teo}

\begin{Def}
Un proceso irreducible recurrente con medida estacionaria de masa
finita es llamado erg\'odico.
\end{Def}

\begin{Teo}\label{Teo.4.3}
Un proceso de Markov de saltos irreducible no explosivo es
erg\'odico si y s\'olo si se puede encontrar una soluci\'on, de
probabilidad, $\pi$, con $|\pi|=1$ y $0\leq\pi_{j}\leq1$, a
$\pi\Lambda=0$. En este caso $\pi$ es la distribuci\'on
estacionaria.
\end{Teo}

\begin{Cor}\label{Cor.4.4}
Una condici\'on suficiente para la ergodicidad de un proceso
irreducible es la existencia de una probabilidad $\pi$ que
resuelva el sistema $\pi\Lambda=0$ y que adem\'as tenga la propiedad
de que $\sum\pi_{j}\lambda\left(j\right)$.
\end{Cor}

%_____________________________________________________________________________________
%
\section{Notaci\'on Kendall-Lee}
%_____________________________________________________________________________________
%
\subsection{Primera parte}

A partir de este momento se har\'an las siguientes consideraciones:
Si $t_{n}$ es el tiempo aleatorio en el que llega al sistema el
$n$-\'esimo cliente, para $n=1,2,\ldots$, $t_{0}=0$ y
$t_{0}<t_{1}<\cdots$ se definen los tiempos entre arribos
$\tau_{n}=t_{n}-t_{n-1}$ para $n=1,2,\ldots$, variables aleatorias
independientes e id\'enticamente distribuidas. Los tiempos entre
arribos tienen un valor medio $E\left(\tau\right)$ finito y
positivo $\frac{1}{\beta}$, es decir, $\beta$ se puede ver como la
tasa o intensidad promedio de arribos al sistema por unidad de
tiempo. Adem\'as se supondr\'a que los servidores son identicos y si
$s$ denota la variable aleatoria que describe el tiempo de
servicio, entonces $E\left(s\right)=\frac{1}{\delta}$, $\delta$ es
la tasa promedio de servicio por servidor.

La notaci\'on de Kendall-Lee es una forma abreviada de describir un
sisema de espera con las siguientes componentes:
\begin{itemize}
\item[i)] {\em\bf Fuente}: Poblaci\'on de clientes potenciales del
sistema, esta puede ser finita o infinita. \item[ii)] {\em\bf
Proceso de Arribos}: Proceso determinado por la funci\'on de
distribuci\'on $A\left(t\right)=P\left\{\tau\leq t\right\}$ de los
tiempos entre arribos.
\end{itemize}

Adem\'as tenemos las siguientes igualdades
\begin{equation}\label{Eq.0.1}
N\left(t\right)=N_{q}\left(t\right)+N_{s}\left(s\right)
\end{equation}
donde
\begin{itemize}
\item $N\left(t\right)$ es el n\'umero de clientes en el sistema al
tiempo $t$. \item $N_{q}\left(t\right)$ es el n\'umero de clientes
en la cola al tiempo $t$ \item $N_{s}\left(t\right)$ es el n\'umero
de clientes recibiendo servicio en el tiempo $t$.
\end{itemize}

Bajo la hip\'otesis de estacionareidad, es decir, las
caracter\'isticas de funcionamiento del sistema se han estabilizado
en valores independientes del tiempo, entonces
\begin{equation}
N=N_{q}+N_{s}.
\end{equation}

Los valores medios de las cantidades anteriores se escriben como
$L=E\left(N\right)$, $L_{q}=E\left(N_{q}\right)$ y
$L_{s}=E\left(N_{s}\right)$, entonces de la ecuaci\'on \ref{Eq.0.1}
se obtiene

\begin{equation}
L=L_{q}+L_{s}
\end{equation}
Si $q$ es el tiempo que pasa un cliente en la cola antes de
recibir servicio, y W es el tiempo total que un cliente pasa en el
sistema, entonces
\[w=q+s\]
por lo tanto
\[W=W_{q}+W_{s},\]
donde $W=E\left(w\right)$, $W_{q}=E\left(q\right)$ y
$W_{s}=E\left(s\right)=\frac{1}{\delta}$.

La intensidad de tr\'afico se define como
\begin{equation}
\rho=\frac{E\left(s\right)}{E\left(\tau\right)}=\frac{\beta}{\delta}.
\end{equation}

La utilizaci\'on por servidor es
\begin{equation}
u=\frac{\rho}{c}=\frac{\beta}{c\delta}.
\end{equation}
donde $c$ es el n\'umero de servidores.

%_____________________________________________________________________________________
%
\subsection{Segunda parte}
%_____________________________________________________________________________________
%

Esta notaci\'on es una forma abreviada de describir un sistema de
espera con componentes dados a continuaci\'on, la notaci\'on es
\begin{equation}\label{Notacion.K.L.}
A/S/c/K/F/d
\end{equation}
Cada una de las letras describe:
\begin{itemize}
\item $A$ es la distribuci\'on de los tiempos entre arribos. \item
$S$ es la distribuci\'on del tiempo de servicio. \item $c$ es el
n\'umero de servidores. \item $K$ es la capacidad del sistema. \item
$F$ es el n\'umero de individuos en la fuente. \item $d$ es la
disciplina del servicio
\end{itemize}
Usualmente se acostumbra suponer que $K=\infty$, $F=\infty$ y
$d=FIFO$, es decir, First In First Out.

Las distribuciones usuales para $A$ y $B$ son:
\begin{itemize}
\item $GI$ para la distribuci\'on general de los tiempos entre
arribos. \item $G$ distribuci\'on general del tiempo de servicio.
\item $M$ Distribuci\'on exponencial para $A$ o $S$. \item $E_{K}$
Distribuci\'on Erlang-$K$, para $A$ o $S$. \item $D$ tiempos entre
arribos o de servicio constantes, es decir, deterministicos.
\end{itemize}

%_____________________________________________________________________________________
%
\section{Procesos de Nacimiento y Muerte}
%_____________________________________________________________________________________
%

\begin{Prop}\label{Prop.2.1}
La recurrencia de $\left\{X_{t}\right\}$, o equivalentemente de
$\left\{Y_{n}\right\}$ es equivalente a
\begin{equation}\label{Eq.2.1}
\sum_{n=1}^{\infty}\frac{\delta_{1}\cdots\delta_{n}}{\beta_{1}\cdots\beta_{n}}=\sum_{n=1}^{\infty}\frac{q_{1}\cdots
q_{n}}{p_{1}\cdots p_{n}}=\infty
\end{equation}
\end{Prop}

\begin{Lema}\label{Lema.2.2}
Independientemente de la recurrencia o transitorieadad, existe una
y s\'olo una, salvo m\'ultiplos, soluci\'on a $v\Lambda=0$, dada por
\begin{equation}\label{Eq.2.2}
v_{n}=\frac{\beta_{0}\cdots\beta_{n-1}}{\delta_{1}\cdots\delta_{n}}v_{0}
\end{equation}
para $n=1,2,\ldots$.
\end{Lema}


\begin{Cor}\label{Cor.2.3}
En el caso recurrente, la medida estacionaria $\mu$ para
$\left\{Y_{n}\right\}$ est\'a dada por
\begin{equation}\label{Eq.}
\mu_{n}=\frac{p_{1}\cdots p_{n-1}}{q_{1}\cdots q_{n}}\mu_{0}
\end{equation}
para $n=1,2,\ldots$.
\end{Cor}

Se define a
$S=1+\sum_{n=1}^{\infty}\frac{\beta_{0}\beta_{1}\cdots\beta_{n-1}}{\delta_{1}\delta_{2}\cdots\delta_{n}}$

\begin{Cor}\label{Cor.2.4}
$\left\{X_{t}\right\}$ es erg\'odica si y s\'olo si la ecuaci\'on
(\ref{Eq.2.1}) se cumple y adem\'as $S<\infty$, en cuyo caso la
distribuci\'on erg\'odica, $\pi$, est\'a dada por
\begin{equation}\label{Eq.2.4}
\pi_{0}=\frac{1}{S}\textrm{,
}\pi_{n}=\frac{1}{S}\frac{\beta_{0}\cdots\beta_{n-1}}{\delta_{1}\cdots\delta_{n}}
\end{equation}
para $n=1,2,\ldots$.
\end{Cor}

%_____________________________________________________________________________________
%
\subsection{Cola $M/M/1$}
%_____________________________________________________________________________________
%



Este modelo corresponde a un proceso de nacimiento y muerte con
$\beta_{n}=\beta$ y $\delta_{n}=\delta$ independiente del valor de
$n$. La intensidad de tr\'afico $\rho=\frac{\beta}{\delta}$, implica
que el criterio de recurrencia (ecuaci\'on \ref{Eq.2.1}) quede de la
forma:
\begin{eqnarray*}
1+\sum_{n=1}^{\infty}\rho^{-n}=\infty.
\end{eqnarray*}
Equivalentemente el proceso es recurrente si y s\'olo si
\begin{eqnarray*}
\sum_{n\geq1}\left(\frac{\beta}{\delta}\right)^{n}<\infty\Leftrightarrow
\frac{\beta}{\delta}<1
\end{eqnarray*}
Entonces $S=\frac{\delta}{\delta-\beta}$, luego por la ecuaci\'on
\ref{Eq.2.4} se tiene que
\begin{eqnarray*}
\pi_{0}&=&\frac{\delta-\beta}{\delta}=1-\frac{\beta}{\delta}\\
\pi_{n}&=&\pi_{0}\left(\frac{\beta}{\delta}\right)^{n}=\left(1-\frac{\beta}{\delta}\right)\left(\frac{\beta}{\delta}\right)^{n}=\left(1-\rho\right)\rho^{n}
\end{eqnarray*}


Lo cual nos lleva a la siguiente

\begin{Prop}
La cola $M/M/1$ con intendisad de tr\'afico $\rho$, es recurrente si
y s\'olo si $\rho\leq1$.
\end{Prop}

Entonces por el corolario \ref{Cor.2.3}

\begin{Prop}
La cola $M/M/1$ con intensidad de tr\'afico $\rho$ es erg\'odica si y
s\'olo si $\rho<1$. En cuyo caso, la distribuci\'on de equilibrio
$\pi$ de la longitud de la cola es geom\'etrica,
$\pi_{n}=\left(1-\rho\right)\rho^{n}$, para $n=1,2,\ldots$.
\end{Prop}
De la proposici\'on anterior se desprenden varios hechos
importantes.
\begin{enumerate}
\item $\prob\left[X_{t}=0\right]=\pi_{0}=1-\rho$, es decir, la
probabilidad de que el sistema se encuentre ocupado. \item De las
propiedades de la distribuci\'on Geom\'etrica se desprende que
\begin{enumerate}
\item $\esp\left[X_{t}\right]=\frac{\rho}{1-\rho}$, \item
$Var\left[X_{t}\right]=\frac{\rho}{\left(1-\rho\right)^{2}}$.
\end{enumerate}
\end{enumerate}

Si $L$ es el n\'umero esperado de clientes en el sistema, incluyendo
los que est\'an siendo atendidos, entonces
\begin{eqnarray*}
L=\frac{\rho}{1-\rho}
\end{eqnarray*}
Si adem\'as $W$ es el tiempo total del cliente en la cola:
$W=W_{q}+W_{s}$
$\rho=\frac{\esp\left[s\right]}{\esp\left[\tau\right]}=\beta
W_{s}$, puesto que $W_{s}=\esp\left[s\right]$ y
$\esp\left[\tau\right]=\frac{1}{\delta}$. Por la f\'ormula de Little
$L=\lambda W$
\begin{eqnarray*}
W&=&\frac{L}{\beta}=\frac{\frac{\rho}{1-\rho}}{\beta}=\frac{\rho}{\delta}\frac{1}{1-\rho}=\frac{W_{s}}{1-\rho}\\
&=&\frac{1}{\delta\left(1-\rho\right)}=\frac{1}{\delta-\beta}
\end{eqnarray*}
luego entonces
\begin{eqnarray*}
W_{q}&=&W-W_{s}=\frac{1}{\delta-\beta}-\frac{1}{\delta}=\frac{\beta}{\delta(\delta-\beta)}\\
&=&\frac{\rho}{1-\rho}\frac{1}{\delta}=\esp\left[s\right]\frac{\rho}{1-\rho}
\end{eqnarray*}

Entonces
\begin{eqnarray*}
L_{q}=\beta W_{q}=\frac{\rho^{2}}{1-\rho}.
\end{eqnarray*}
Finalmente
\begin{Prop}
\begin{enumerate}
\item $W\left(t\right)=1-e^{-\frac{t}{W}}$. \item
$W_{q}\left(t\right)=1-\rho\exp^{-\frac{t}{W}}$.
\end{enumerate}
donde $W=\esp(w)$.
\end{Prop}

%_____________________________________________________________________________________
%
\subsection{Cola $M/M/\infty$}
%_____________________________________________________________________________________
%

Este tipo de modelos se utilizan para estimar el n\'umero de l\'ineas
en uso en una gran red comunicaci\'on o para estimar valores en los
sistemas $M/M/c$ o $M/M/c/c$, en el se puede pensar que siempre
hay un servidor disponible para cada cliente que llega.\smallskip

Se puede considerar como un proceso de nacimiento y muerte con
par\'ametros $\beta_{n}=\beta$ y $\mu_{n}=n\mu$ para
$n=0,1,2,\ldots$, entonces por la ecuaci\'on \ref{Eq.2.4} se tiene
que
\begin{eqnarray*}\label{MMinf.pi}
\pi_{0}=e^{\rho}\\
\pi_{n}=e^{-\rho}\frac{\rho^{n}}{n!}
\end{eqnarray*}
Entonces, el n\'umero promedio de servidores ocupados es equivalente
a considerar el n\'umero de clientes en el  sistema, es decir,
\begin{eqnarray*}
L=\esp\left[N\right]=\rho\\
Var\left[N\right]=\rho
\end{eqnarray*}

Adem\'as se tiene que $W_{q}=0$ y $L_{q}=0$.\smallskip

El tiempo promedio en el sistema es el tiempo promedio de
servicio, es decir, $W=\esp\left[s\right]=\frac{1}{\delta}$.
Resumiendo, tenemos la sisuguiente proposici\'on:
\begin{Prop}
La cola $M/M/\infty$ es erg\'odica para todos los valores de $\eta$.
La distribuci\'on de equilibrio $\pi$ es Poisson con media $\eta$,
$\pi_{n}=\frac{e^{-n}\eta^{n}}{n!}$.
\end{Prop}

%_____________________________________________________________________________________
%
\subsection{Cola $M/M/m$}
%_____________________________________________________________________________________
%

Este sistema considera $m$ servidores id\'enticos, con tiempos entre
arribos y de servicio exponenciales con medias
$\esp\left[\tau\right]=\frac{1}{\beta}$ y
$\esp\left[s\right]=\frac{1}{\delta}$. definimos ahora la
utilizaci\'on por servidor como $u=\frac{\rho}{m}$ que tambi\'en se
puede interpretar como la fracci\'on de tiempo promedio que cada
servidor est\'a ocupado.\smallskip

La cola $M/M/m$ se puede considerar como un proceso de nacimiento
y muerte con par\'ametros: $\beta_{n}=\beta$ para $n=0,1,2,\ldots$ y
$\delta_{n}=\left\{\begin{array}{cc}
n\delta & n=0,1,\ldots,m-1\\
c\delta & n=m,\ldots\\
\end{array}\right.$

entonces  la condici\'on de recurrencia se va a cumplir s\'i y s\'olo si
$\sum_{n\geq1}\frac{\beta_{0}\cdots\beta_{n-1}}{\delta_{1}\cdots\delta_{n}}<\infty$,
equivalentemente se debe de cumplir que

\begin{eqnarray*}
S&=&1+\sum_{n\geq1}\frac{\beta_{0}\cdots\beta_{n-1}}{\delta_{1}\cdots\delta_{n}}=\sum_{n=0}^{m-1}\frac{\beta_{0}\cdots\beta_{n-1}}{\delta_{1}\cdots\delta_{n}}+\sum_{n=0}^{\infty}\frac{\beta_{0}\cdots\beta_{n-1}}{\delta_{1}\cdots\delta_{n}}\\
&=&\sum_{n=0}^{m-1}\frac{\beta^{n}}{n!\delta^{n}}+\sum_{n=0}^{\infty}\frac{\rho^{m}}{m!}u^{n}
\end{eqnarray*}
converja, lo cual ocurre si $u<1$, en este caso

\begin{eqnarray*}
S=\sum_{n=0}^{m-1}\frac{\rho^{n}}{n!}+\frac{\rho^{m}}{m!}\left(1-u\right)
\end{eqnarray*}
luego, para este caso se tiene que

\begin{eqnarray*}
\pi_{0}&=&\frac{1}{S}\\
\pi_{n}&=&\left\{\begin{array}{cc}
\pi_{0}\frac{\rho^{n}}{n!} & n=0,1,\ldots,m-1\\
\pi_{0}\frac{\rho^{n}}{m!m^{n-m}}& n=m,\ldots\\
\end{array}\right.
\end{eqnarray*}
Al igual que se hizo antes, determinaremos los valores de
$L_{q},W_{q},W$ y $L$:
\begin{eqnarray*}
L_{q}&=&\esp\left[N_{q}\right]=\sum_{n=0}^{\infty}\left(n-m\right)\pi_{n}=\sum_{n=0}^{\infty}n\pi_{n+m}\\
&=&\sum_{n=0}^{\infty}n\pi_{0}\frac{\rho^{n+m}}{m!m^{n+m}}=\pi_{0}\frac{\rho^{m}}{m!}\sum_{n=0}^{\infty}nu^{n}=\pi_{0}\frac{u\rho^{m}}{m!}\sum_{n=0}^{\infty}\frac{d}{du}u^{n}\\
&=&\pi_{0}\frac{u\rho^{m}}{m!}\frac{d}{du}\sum_{n=0}^{\infty}u^{n}=\pi_{0}\frac{u\rho^{m}}{m!}\frac{d}{du}\left(\frac{1}{1-u}\right)=\pi_{0}\frac{u\rho^{m}}{m!}\frac{1}{\left(1-u\right)^{2}}
\end{eqnarray*}

es decir
\begin{equation}
L_{q}=\frac{u\pi_{0}\rho^{m}}{m!\left(1-u\right)^{2}}
\end{equation}
luego
\begin{equation}
W_{q}=\frac{L_{q}}{\beta}
\end{equation}
\begin{equation}
W=W_{q}+\frac{1}{\delta}
\end{equation}
Si definimos
$C\left(m,\rho\right)=\frac{\pi_{0}\rho^{m}}{m!\left(1-u\right)}=\frac{\pi_{m}}{1-u}$,
que es la probabilidad de que un cliente que llegue al sistema
tenga que esperar en la cola. Entonces podemos reescribir las
ecuaciones reci\'en enunciadas:

\begin{eqnarray*}
L_{q}&=&\frac{C\left(m,\rho\right)u}{1-u}\\
W_{q}&=&\frac{C\left(m,\rho\right)\esp\left[s\right]}{m\left(1-u\right)}\\
\end{eqnarray*}


\begin{Prop}
La cola $M/M/m$ con intensidad de tr\'afico $\rho$ es erg\'odica si y
s\'olo si $\rho<1$. En este caso la distribuci\'on erg\'odica $\pi$ est\'a
dada por
\begin{eqnarray*}
\pi_{n}=\left\{\begin{array}{cc}
\frac{1}{S}\frac{\eta^{n}}{n!} & 0\leq n\leq m\\
\frac{1}{S}\frac{\eta^{m}}{m!}\rho^{n-m} & m\leq n<\infty\\
\end{array}\right.
\end{eqnarray*}
\end{Prop}
\begin{Prop}
Para $t\geq0$
\begin{itemize}
\item[a)]$W_{q}\left(t\right)=1-C\left(m,\rho\right)e^{-c\delta
t\left(1-u\right)}$ \item[b)]\begin{eqnarray*}
W\left(t\right)=\left\{\begin{array}{cc}
1+e^{-\delta t}\frac{\rho-m+W_{q}\left(0\right)}{m-1-\rho}+e^{-m\delta t\left(1-u\right)}\frac{C\left(m,\rho\right)}{m-1-\rho} & \rho\neq m-1\\
1-\left(1+C\left(m,\rho\right)\delta t\right)e^{-\delta t} & \rho=m-1\\
\end{array}\right.
\end{eqnarray*}
\end{itemize}

\end{Prop}

%_____________________________________________________________________________________
%
%\subsection{Cola $M/M/1/K$}
%_____________________________________________________________________________________
%
%_____________________________________________________________________________________
%
\section{Ejemplo de Cadena de Markov para dos Estados}
%_____________________________________________________________________________________
%

Supongamos que se tiene la siguiente cadena:
\begin{equation}
\left(\begin{array}{cc}
1-q & q\\
p & 1-p\\
\end{array}
\right)
\end{equation}
Si $P\left[X_{0}=0\right]=\pi_{0}(0)=a$ y
$P\left[X_{0}=1\right]=\pi_{0}(1)=b=1-\pi_{0}(0)$, con $a+b=1$,
entonces despu\'es de un procedimiento m\'as o menos corto se tiene
que:

\begin{eqnarray*}
P\left[X_{n}=0\right]=\frac{p}{p+q}+\left(1-p-q\right)^{n}\left(a-\frac{p}{p+q}\right)\\
P\left[X_{n}=1\right]=\frac{q}{p+q}+\left(1-p-q\right)^{n}\left(b-\frac{q}{p+q}\right)\\
\end{eqnarray*}
donde, como $0<p,q<1$, se tiene que $|1-p-q|<1$, entonces
$\left(1-p-q\right)^{n}\rightarrow 0$ cuando $n\rightarrow\infty$.
Por lo tanto
\begin{eqnarray*}
lim_{n\rightarrow\infty}P\left[X_{n}=0\right]=\frac{p}{p+q}\\
lim_{n\rightarrow\infty}P\left[X_{n}=1\right]=\frac{q}{p+q}\\
\end{eqnarray*}
Si hacemos $v=\left(\frac{p}{p+q},\frac{q}{p+q}\right)$, entonces
\begin{eqnarray*}
\left(\frac{p}{p+q},\frac{q}{p+q}\right)\left(\begin{array}{cc}
1-q & q\\
p & 1-p\\
\end{array}\right)
\end{eqnarray*}


%_____________________________________________________________________________________
%
\section{Cadenas de Markov}
%_____________________________________________________________________________________
%
\subsection{Estacionareidad}
%_____________________________________________________________________________________
%}

Sea $v=\left(v_{i}\right)_{i\in E}$ medida no negativa en $E$,
podemos definir una nueva medida $v\prob$ que asigna masa
$\sum_{i\in E}v_{i}p_{ij}$ a cada estado $j$.\smallskip

\begin{Def}
La medida $v$ es estacionaria si $v_{i}<\infty$ para toda $i$ y
adem\'as $v\prob=v$.\smallskip
\end{Def}
En el caso de que $v$ sea distribuci\'on, independientemente de que
sea estacionaria o no, se cumple con

\begin{eqnarray*}
\prob_{v}\left[X_{1}=j\right]=\sum_{i\in
E}\prob_{v}\left[X_{0}=i\right]p_{ij}=\sum_{i\in
E}v_{i}p_{ij}=\left(vP\right)_{j}
\end{eqnarray*}

\begin{Teo}
Supongamos que $v$ es una distribuci\'on estacionaria. Entonces
\begin{itemize}
\item[i)] La cadena es estrictamente estacionaria con respecto a
$\prob_{v}$, es decir, $\prob_{v}$-distribuci\'on de
$\left\{X_{n},X_{n+1},\ldots\right\}$ no depende de $n$;
\item[ii)] Existe un aversi\'on estrictamente estacionaria
$\left\{X_{n}\right\}_{n\in Z}$ de la cadena con doble tiempo
infinito y $\prob\left(X_{n}=i\right)=v_{i}$ para toda $n\in Z$.
\end{itemize}
\end{Teo}
\begin{Teo}
Sea $i$ estado fijo, recurrente. Entonces una medida estacionaria
$v$ puede definirse haciendo que $v_{j}$ sea el n\'umero esperado de
visitas a $j$ entre dos visitas consecutivas $i$,
\begin{equation}\label{Eq.3.1}
v_{j}=\esp_{i}\sum_{n=0}^{\tau(i)-1}\indora\left(X_{n}=i\right)=\sum_{n=0}^{\infty}\prob_{i}\left[X_{n}=j,\tau(i)>n\right]
\end{equation}
\end{Teo}

\begin{Teo}\label{Teo.3.3}
Si la cadena es irreducible y recurrente, entonces una medida
estacionaria $v$ existe, satisface $0<v_{j}<\infty$ para toda $j$
y es \'unica salvo factores multiplicativos, es decir, si $v,v^{*}$
son estacionarias, entonces $c=cv^{*}$ para alguna
$c\in\left(0,\infty\right)$.
\end{Teo}
\begin{Cor}\label{Cor.3.5}
Si la cadena es irreducible y positiva recurrente, existe una
\'unica distribuci\'on estacionaria $\pi$ dada por
\begin{equation}
\pi_{j}=\frac{1}{\esp_{i}\tau_{i}}\esp_{i}\sum_{n=0}^{\tau\left(i\right)-1}\indora\left(X_{n}=j\right)=\frac{1}{\esp_{j}\tau\left(j\right)}.
\end{equation}
\end{Cor}
\begin{Cor}\label{Cor.3.6}
Cualquier cadena de Markov irreducible con un espacio de estados
finito es positiva recurrente.
\end{Cor}

%_____________________________________________________________________________________
%
\subsection{Teor\'ia Erg\'odica}
%_____________________________________________________________________________________
%

\begin{Lema}
Sea $\left\{X_{n}\right\}$ cadena irreducible y se $F$ subconjunto
finito del espacio de estados. Entonces la cadena es positiva
recurrente si $\esp_{i}\tau\left(F\right)<\infty$ para todo $i\in
F$.
\end{Lema}

\begin{Prop}
Sea $\left\{X_{n}\right\}$ cadena irreducible y transiente o cero
recurrente, entonces $p_{ij}^{n}\rightarrow0$ conforme
$n\rightarrow\infty$ para cualquier $i,j\in E$, $E$ espacio de
estados.
\end{Prop}
Utilizando el teorema (2.2) y el corolario ref{Cor.3.5}, se
demuestra el siguiente resultado importante.

\begin{Teo}
Sea $\left\{X_{n}\right\}$ cadena irreducible y aperi\'odica
positiva recurrente, y sea $\pi=\left\{\pi_{j}\right\}_{j\in E}$
la distribuci\'on estacionaria. Entonces
$p_{ij}^{n}\rightarrow\pi_{j}$ para todo $i,j$.
\end{Teo}
\begin{Def}\label{Def.Ergodicidad}
Una cadena irreducible aperiodica, positiva recurrente con medida
estacionaria $v$, es llamada {\em erg\'odica}.
\end{Def}

\begin{Prop}\label{Prop.4.4}
Sea $\left\{X_{n}\right\}$ cadena irreducible y recurrente con
medida estacionaria $v$, entocnes para todo $i,j,k,l\in E$
\begin{equation}
\frac{\sum_{n=0}^{m}p_{ij}^{n}}{\sum_{n=0}^{m}p_{lk}^{n}}\rightarrow\frac{v_{j}}{v_{k}}\textrm{,
}m\rightarrow\infty
\end{equation}
\end{Prop}
\begin{Lema}\label{Lema.4.5}
La matriz $\widetilde{P}$ con elementos
$\widetilde{p}_{ij}=\frac{v_{ji}p_{ji}}{v_{i}}$ es una matriz de
transici\'on. Adem\'as, el $i$-\'esimo elementos
$\widetilde{p}_{ij}^{m}$ de la matriz potencia $\widetilde{P}^{m}$
est\'a dada por
$\widetilde{p}_{ij}^{m}=\frac{v_{ji}p_{ji}^{m}}{v_{i}}$.
\end{Lema}

\begin{Lema}
Def\'inase $N_{i}^{m}=\sum_{n=0}^{m}\indora\left(X_{n}=i\right)$
como el n\'umero de visitas a $i$ antes del tiempo $m$. Entonces si
la cadena es reducible y recurrente,
$lim_{m\rightarrow\infty}\frac{\esp_{j}N_{i}^{m}}{\esp_{k}N_{i}^{m}}=1$
para todo $j,k\in E$.
\end{Lema}

%_____________________________________________________________________________________
%
\subsection{Funciones Arm\'onicas, Recurrencia y Transitoriedad}
%_____________________________________________________________________________________
%

\begin{Def}\label{Def.Armonica}
Una funci\'on Arm\'onica es el eigenvector derecho $h$ de $P$
corrrespondiente al eigenvalor 1.
\end{Def}
\begin{eqnarray*}
Ph=h\Leftrightarrow h\left(i\right)=\sum_{j\in
E}p_{ij}h\left(j\right)=\esp_{i}h\left(X_{1}\right)=\esp\left[h\left(X_{n+1}\right)|X_{n}=i\right].
\end{eqnarray*}
es decir, $\left\{h\left(X_{n}\right)\right\}$ es martingala.
\begin{Prop}\label{Prop.5.2}
Sea $\left\{X_{n}\right\}$ cadena irreducible  y sea $i$ estado
fijo arbitrario. Entonces la cadena es transitoria s\'i y s\'olo si
existe una funci\'on no cero, acotada
$h:E-\left\{i\right\}\rightarrow\rea$ que satisface
\begin{equation}\label{Eq.5.1}
h\left(j\right)=\sum_{k\neq i}p_{jk}h\left(k\right)\textrm{   para
}j\neq i.
\end{equation}
\end{Prop}

\begin{Prop}\label{Prop.5.3}
Supongamos que la cadena es irreducible y sea $E_{0}$ un
subconjunto finito del espacio de estados, entonces
\begin{itemize}
\item[i)] \item[ii)]
\end{itemize}
\end{Prop}

\begin{Prop}\label{Prop.5.4}
Suponga que la cadena es irreducible y sea $E_{0}$ un subconjunto
finito de $E$ tal que se cumple la ecuaci\'on 5.2 para alguna
funci\'on $h$ acotada que satisface
$h\left(i\right)<h\left(j\right)$ para alg\'un estado $i\notin
E_{0}$ y todo $j\in E_{0}$. Entonces la cadena es transitoria.
\end{Prop}



%_____________________________________________________________________________________
%
\section{Procesos de Markov de Saltos}
%_____________________________________________________________________________________
%
\subsection{Estructura B\'asica de los Procesos Markovianos de Saltos}
%_____________________________________________________________________________________
%

\begin{itemize}
\item Sea $E$ espacio discreto de estados, finito o numerable, y
sea $\left\{X_{t}\right\}$ un proceso de Markov con espacio de
estados $E$. Una medida $\mu$ en $E$ definida por sus
probabilidades puntuales $\mu_{i}$, escribimos
$p_{ij}^{t}=P^{t}\left(i,\left\{j\right\}\right)=P_{i}\left(X_{t}=j\right)$.\smallskip

\item El monto del tiempo gastado en cada estado es positivo, de
modo tal que las trayectorias muestrales son constantes por
partes. Para un proceso de saltos denotamos por los tiempos de
saltos a $S_{0}=0<S_{1}<S_{2}\cdots$, los tiempos entre saltos
consecutivos $T_{n}=S_{n+1}-S_{n}$ y la secuencia de estados
visitados por $Y_{0},Y_{1},\ldots$, as\'i las trayectorias
muestrales son constantes entre $S_{n}$ consecutivos, continua por
la derecha, es decir, $X_{S_{n}}=Y_{n}$. \item La descripci\'on de
un modelo pr\'actico est\'a dado usualmente en t\'erminos de las
intensidades $\lambda\left(i\right)$ y las probabilidades de salto
$q_{ij}$ m\'as que en t\'erminos de la matriz de transici\'on $P^{t}$.
\item Sup\'ongase de ahora en adelante que $q_{ii}=0$ cuando
$\lambda\left(i\right)>0$
\end{itemize}

%_____________________________________________________________________________________
%
\subsection{Matriz Intensidad}
%_____________________________________________________________________________________
%

\begin{Def}
La matriz intensidad
$\Lambda=\left(\lambda\left(i,j\right)\right)_{i,j\in E}$ del
proceso de saltos $\left\{X_{t}\right\}_{t\geq0}$ est\'a dada por
\begin{eqnarray*}
\lambda\left(i,j\right)&=&\lambda\left(i\right)q_{i,j}\textrm{,    }j\neq i\\
\lambda\left(i,i\right)&=&-\lambda\left(i\right)\\
\end{eqnarray*}

\begin{Prop}\label{Prop.3.1}
Una matriz $E\times E$, $\Lambda$ es la matriz de intensidad de un
proceso markoviano de saltos $\left\{X_{t}\right\}_{t\geq0}$ si y
s\'olo si
\begin{eqnarray*}
\lambda\left(i,i\right)\leq0\textrm{,
}\lambda\left(i,j\right)\textrm{,   }i\neq j\textrm{,  }\sum_{j\in
E}\lambda\left(i,j\right)=0.
\end{eqnarray*}
Adem\'as, $\Lambda$ est\'a en correspondencia uno a uno con la
distribuci\'on del proceso minimal dado por el teorema 3.1.
\end{Prop}

\end{Def}


Para el caso particular de la Cola $M/M/1$, la matr\'iz de itensidad
est\'a dada por
\begin{eqnarray*}
\Lambda=\left[\begin{array}{cccccc}
-\beta & \beta & 0 &0 &0& \cdots\\
\delta & -\beta-\delta & \beta & 0 & 0 &\cdots\\
0 & \delta & -\beta-\delta & \beta & 0 &\cdots\\
\vdots & & & & & \ddots\\
\end{array}\right]
\end{eqnarray*}

%____________________________________________________________________________
\subsection{Medidas Estacionarias}
%____________________________________________________________________________
%


\begin{Def}
Una medida $v\neq0$ es estacionaria si $0\leq v_{j}<\infty$,
$vP^{t}=v$ para toda $t$.
\end{Def}

\begin{Teo}\label{Teo.4.2}
Supongamos que $\left\{X_{t}\right\}$ es irreducible recurrente en
$E$. Entonces existe una y s\'olo una, salvo m\'ultiplos, medida
estacionaria $v$. Esta $v$ tiene la propiedad de que $0<
v_{j}<\infty$ para todo $j$ y puede encontrarse en cualquiera de
las siguientes formas

\begin{itemize}
\item[i)] Para alg\'un estado $i$, fijo pero arbitrario, $v_{j}$ es
el tiempo esperado utilizado en $j$ entre dos llegadas
consecutivas al estado $i$;
\begin{equation}\label{Eq.4.2}
v_{j}=\esp_{i}\int_{0}^{w\left(i\right)}\indora\left(X_{t}=j\right)dt
\end{equation}
con
$w\left(i\right)=\inf\left\{t>0:X_{t}=i,X_{t^{-}}=\lim_{s\uparrow
t}X_{s}\neq i\right\}$. \item[ii)]
$v_{j}=\frac{\mu_{j}}{\lambda\left(j\right)}$, donde $\mu$ es
estacionaria para $\left\{Y_{n}\right\}$. \item[iii)] como
soluci\'on de $v\Lambda=0$.
\end{itemize}
\end{Teo}
%____________________________________________________________________________
\subsection{Criterios de Ergodicidad}
%____________________________________________________________________________
%

\begin{Def}
Un proceso irreducible recurrente con medida estacionaria de masa
finita es llamado erg\'odico.
\end{Def}

\begin{Teo}\label{Teo.4.3}
Un proceso de Markov de saltos irreducible no explosivo es
erg\'odico si y s\'olo si se puede encontrar una soluci\'on, de
probabilidad, $\pi$, con $|\pi|=1$ y $0\leq\pi_{j}\leq1$, a
$\pi\Lambda=0$. En este caso $\pi$ es la distribuci\'on
estacionaria.
\end{Teo}

\begin{Cor}\label{Cor.4.4}
Una condici\'on suficiente para la ergodicidad de un proceso
irreducible es la existencia de una probabilidad $\pi$ que
resuelva el sistema $\pi\Lambda=0$ y que adem\'as tenga la propiedad
de que $\sum\pi_{j}\lambda\left(j\right)<\infty$.
\end{Cor}
\begin{Prop}
Si el proceso es erg\'odico, entonces existe una versi\'on
estrictamente estacionaria
$\left\{X_{t}\right\}_{-\infty<t<\infty}$con doble tiempo
infinito.
\end{Prop}

\begin{Teo}
Si $\left\{X_{t}\right\}$ es erg\'odico y $\pi$ es la distribuci\'on
estacionaria, entonces para todo $i,j$,
$p_{ij}^{t}\rightarrow\pi_{j}$ cuando $t\rightarrow\infty$.
\end{Teo}

\begin{Cor}
Si $\left\{X_{t}\right\}$ es irreducible recurente pero no
erg\'odica, es decir $|v|=\infty$, entonces $p_{ij}^{t}\rightarrow0$
para todo $i,j\in E$.
\end{Cor}

\begin{Cor}
Para cualquier proceso Markoviano de Saltos minimal, irreducible o
no, los l\'imites $li_{t\rightarrow\infty}p_{ij}^{t}$ existe.
\end{Cor}

%_____________________________________________________________________________________
%
\section{Notaci\'on Kendall-Lee}
%_____________________________________________________________________________________
%
\subsection{Primera parte}

A partir de este momento se har\'an las siguientes consideraciones:
Si $t_{n}$ es el tiempo aleatorio en el que llega al sistema el
$n$-\'esimo cliente, para $n=1,2,\ldots$, $t_{0}=0$ y
$t_{0}<t_{1}<\cdots$ se definen los tiempos entre arribos
$\tau_{n}=t_{n}-t_{n-1}$ para $n=1,2,\ldots$, variables aleatorias
independientes e id\'enticamente distribuidas. Los tiempos entre
arribos tienen un valor medio $E\left(\tau\right)$ finito y
positivo $\frac{1}{\beta}$, es decir, $\beta$ se puede ver como la
tasa o intensidad promedio de arribos al sistema por unidad de
tiempo. Adem\'as se supondr\'a que los servidores son identicos y si
$s$ denota la variable aleatoria que describe el tiempo de
servicio, entonces $E\left(s\right)=\frac{1}{\delta}$, $\delta$ es
la tasa promedio de servicio por servidor.

La notaci\'on de Kendall-Lee es una forma abreviada de describir un
sisema de espera con las siguientes componentes:
\begin{itemize}
\item[i)] {\em\bf Fuente}: Poblaci\'on de clientes potenciales del
sistema, esta puede ser finita o infinita. \item[ii)] {\em\bf
Proceso de Arribos}: Proceso determinado por la funci\'on de
distribuci\'on $A\left(t\right)=P\left\{\tau\leq t\right\}$ de los
tiempos entre arribos.
\end{itemize}


Adem\'as tenemos las siguientes igualdades
\begin{equation}\label{Eq.0.1}
N\left(t\right)=N_{q}\left(t\right)+N_{s}\left(s\right)
\end{equation}
donde
\begin{itemize}
\item $N\left(t\right)$ es el n\'umero de clientes en el sistema al
tiempo $t$. \item $N_{q}\left(t\right)$ es el n\'umero de clientes
en la cola al tiempo $t$ \item $N_{s}\left(t\right)$ es el n\'umero
de clientes recibiendo servicio en el tiempo $t$.
\end{itemize}

Bajo la hip\'otesis de estacionareidad, es decir, las
caracter\'isticas de funcionamiento del sistema se han estabilizado
en valores independientes del tiempo, entonces
\begin{equation}
N=N_{q}+N_{s}.
\end{equation}

Los valores medios de las cantidades anteriores se escriben como
$L=E\left(N\right)$, $L_{q}=E\left(N_{q}\right)$ y
$L_{s}=E\left(N_{s}\right)$, entonces de la ecuaci\'on \ref{Eq.0.1}
se obtiene

\begin{equation}
L=L_{q}+L_{s}
\end{equation}

Si $q$ es el tiempo que pasa un cliente en la cola antes de
recibir servicio, y W es el tiempo total que un cliente pasa en el
sistema, entonces
\[w=q+s\]
por lo tanto
\[W=W_{q}+W_{s},\]
donde $W=E\left(w\right)$, $W_{q}=E\left(q\right)$ y
$W_{s}=E\left(s\right)=\frac{1}{\delta}$.

La intensidad de tr\'afico se define como
\begin{equation}
\rho=\frac{E\left(s\right)}{E\left(\tau\right)}=\frac{\beta}{\delta}.
\end{equation}

La utilizaci\'on por servidor es
\begin{equation}
u=\frac{\rho}{c}=\frac{\beta}{c\delta}.
\end{equation}
donde $c$ es el n\'umero de servidores.

%_____________________________________________________________________________________
%
\subsection{M\'as sobre la notaci\'on {\em Kendall-Lee}}
%_____________________________________________________________________________________
%

Esta notaci\'on es una forma abreviada de describir un sistema de
espera con componentes dados a continuaci\'on, la notaci\'on es
\begin{equation}\label{Notacion.K.L.}
A/S/c/K/F/d
\end{equation}
Cada una de las letras describe:
\begin{itemize}
\item $A$ es la distribuci\'on de los tiempos entre arribos. \item
$S$ es la distribuci\'on del tiempo de servicio. \item $c$ es el
n\'umero de servidores. \item $K$ es la capacidad del sistema. \item
$F$ es el n\'umero de individuos en la fuente. \item $d$ es la
disciplina del servicio
\end{itemize}
Usualmente se acostumbra suponer que $K=\infty$, $F=\infty$ y
$d=FIFO$, es decir, First In First Out.

Las distribuciones usuales para $A$ y $B$ son:
\begin{itemize}
\item $GI$ para la distribuci\'on general de los tiempos entre
arribos. \item $G$ distribuci\'on general del tiempo de servicio.
\item $M$ Distribuci\'on exponencial para $A$ o $S$. \item $E_{K}$
Distribuci\'on Erlang-$K$, para $A$ o $S$. \item $D$ tiempos entre
arribos o de servicio constantes, es decir, deterministicos.
\end{itemize}

%_____________________________________________________________________________________
%
\section{Procesos de Nacimiento y Muerte}
%_____________________________________________________________________________________
%
\subsection{Procesos de Nacimiento y Muerte Generales}
%_____________________________________________________________________________________
%

Por un proceso de nacimiento y muerte se entiende un proceso de
saltos de markov $\left\{X_{t}\right\}_{t\geq0}$ con espacio de
estados a lo m\'as numerable, con la propiedad de que s\'olo puede ir
al estado $n+1$ o al estado $n-1$, es decir, su matriz de
intensidad es de la forma
\begin{eqnarray*}
\Lambda=\left[\begin{array}{cccccc}
-\beta_{0} & \beta_{0} & 0 &0 &0& \cdots\\
\delta_{1} & -\beta_{1}-\delta_{1} & \beta_{1} & 0 & 0 &\cdots\\
0 & \delta_{2} & -\beta_{2}-\delta_{2} & \beta_{2} & 0 &\cdots\\
\vdots & & & & & \ddots\\
\end{array}\right]
\end{eqnarray*}
donde $\beta_{n}$ son las intensidades de nacimiento y
$\delta_{n}$ las intensidades de muerte, o tambi\'en se puede ver
como a $X_{t}$ el n\'umero de usuarios en una cola al tiempo $t$, un
salto hacia arriba corresponde a la llegada de un nuevo usuario y
un salto hacia abajo como al abandono de un usuario despu\'es de
haber recibido su servicio.

La cadena de saltos $\left\{Y_{n}\right\}$ tiene matriz de
transici\'on dada por
\begin{eqnarray*}
Q=\left[\begin{array}{cccccc}
0 & 1 & 0 &0 &0& \cdots\\
q_{1} & 0 & p_{1} & 0 & 0 &\cdots\\
0 & q_{2} & 0 & p_{2} & 0 &\cdots\\
\vdots & & & & & \ddots\\
\end{array}\right]
\end{eqnarray*}
donde $p_{n}=\frac{\beta_{n}}{\beta_{n}+\delta_{n}}$ y
$q_{n}=1-p_{n}=\frac{\delta_{n}}{\beta_{n}+\delta_{n}}$, donde
adem\'as se asumne por el momento que $p_{n}$ no puede tomar el
valor $0$ \'o $1$ para cualquier valor de $n$.

\begin{Prop}\label{Prop.2.1}
La recurrencia de $\left\{X_{t}\right\}$, o equivalentemente de
$\left\{Y_{n}\right\}$ es equivalente a
\begin{equation}\label{Eq.2.1}
\sum_{n=1}^{\infty}\frac{\delta_{1}\cdots\delta_{n}}{\beta_{1}\cdots\beta_{n}}=\sum_{n=1}^{\infty}\frac{q_{1}\cdots
q_{n}}{p_{1}\cdots p_{n}}=\infty
\end{equation}
\end{Prop}

\begin{Lema}\label{Lema.2.2}
Independientemente de la recurrencia o transitorieadad, existe una
y s\'olo una, salvo m\'ultiplos, soluci\'on a $v\Lambda=0$, dada por
\begin{equation}\label{Eq.2.2}
v_{n}=\frac{\beta_{0}\cdots\beta_{n-1}}{\delta_{1}\cdots\delta_{n}}v_{0}
\end{equation}
para $n=1,2,\ldots$.
\end{Lema}




\begin{Cor}\label{Cor.2.3}
En el caso recurrente, la medida estacionaria $\mu$ para
$\left\{Y_{n}\right\}$ est\'a dada por
\begin{equation}\label{Eq.}
\mu_{n}=\frac{p_{1}\cdots p_{n-1}}{q_{1}\cdots q_{n}}\mu_{0}
\end{equation}
para $n=1,2,\ldots$.
\end{Cor}

Se define a
$S=1+\sum_{n=1}^{\infty}\frac{\beta_{0}\beta_{1}\cdots\beta_{n-1}}{\delta_{1}\delta_{2}\cdots\delta_{n}}$

\begin{Cor}\label{Cor.2.4}
$\left\{X_{t}\right\}$ es erg\'odica si y s\'olo si la ecuaci\'on
(\ref{Eq.2.1}) se cumple y adem\'as $S<\infty$, en cuyo caso la
distribuci\'on erg\'odica, $\pi$, est\'a dada por
\begin{equation}\label{Eq.2.4}
\pi_{0}=\frac{1}{S}\textrm{,
}\pi_{n}=\frac{1}{S}\frac{\beta_{0}\cdots\beta_{n-1}}{\delta_{1}\cdots\delta_{n}}
\end{equation}
para $n=1,2,\ldots$.
\end{Cor}

%_____________________________________________________________________________________
%
\subsection{Cola M/M/1}
%_____________________________________________________________________________________
%


Este modelo corresponde a un proceso de nacimiento y muerte con
$\beta_{n}=\beta$ y $\delta_{n}=\delta$ independiente del valor de
$n$. La intensidad de tr\'afico $\rho=\frac{\beta}{\delta}$, implica
que el criterio de recurrencia (ecuaci\'on \ref{Eq.2.1}) quede de la
forma:
\begin{eqnarray*}
1+\sum_{n=1}^{\infty}\rho^{-n}=\infty.
\end{eqnarray*}
Equivalentemente el proceso es recurrente si y s\'olo si
\begin{eqnarray*}
\sum_{n\geq1}\left(\frac{\beta}{\delta}\right)^{n}<\infty\Leftrightarrow
\frac{\beta}{\delta}<1
\end{eqnarray*}
Entonces $S=\frac{\delta}{\delta-\beta}$, luego por la ecuaci\'on
\ref{Eq.2.4} se tiene que
\begin{eqnarray*}
\pi_{0}&=&\frac{\delta-\beta}{\delta}=1-\frac{\beta}{\delta}\\
\pi_{n}&=&\pi_{0}\left(\frac{\beta}{\delta}\right)^{n}=\left(1-\frac{\beta}{\delta}\right)\left(\frac{\beta}{\delta}\right)^{n}=\left(1-\rho\right)\rho^{n}
\end{eqnarray*}


Lo cual nos lleva a la siguiente

\begin{Prop}
La cola $M/M/1$ con intendisad de tr\'afico $\rho$, es recurrente si
y s\'olo si $\rho\leq1$.
\end{Prop}

Entonces por el corolario \ref{Cor.2.3}

\begin{Prop}
La cola $M/M/1$ con intensidad de tr\'afico $\rho$ es erg\'odica si y
s\'olo si $\rho<1$. En cuyo caso, la distribuci\'on de equilibrio
$\pi$ de la longitud de la cola es geom\'etrica,
$\pi_{n}=\left(1-\rho\right)\rho^{n}$, para $n=1,2,\ldots$.
\end{Prop}
De la proposici\'on anterior se desprenden varios hechos
importantes.
\begin{enumerate}
\item $\prob\left[X_{t}=0\right]=\pi_{0}=1-\rho$, es decir, la
probabilidad de que el sistema se encuentre ocupado. \item De las
propiedades de la distribuci\'on Geom\'etrica se desprende que
\begin{enumerate}
\item $\esp\left[X_{t}\right]=\frac{\rho}{1-\rho}$, \item
$Var\left[X_{t}\right]=\frac{\rho}{\left(1-\rho\right)^{2}}$.
\end{enumerate}
\end{enumerate}

Si $L$ es el n\'umero esperado de clientes en el sistema, incluyendo
los que est\'an siendo atendidos, entonces
\begin{eqnarray*}
L=\frac{\rho}{1-\rho}
\end{eqnarray*}
Si adem\'as $W$ es el tiempo total del cliente en la cola:
$W=W_{q}+W_{s}$
$\rho=\frac{\esp\left[s\right]}{\esp\left[\tau\right]}=\beta
W_{s}$, puesto que $W_{s}=\esp\left[s\right]$ y
$\esp\left[\tau\right]=\frac{1}{\delta}$. Por la f\'ormula de Little
$L=\lambda W$
\begin{eqnarray*}
W&=&\frac{L}{\beta}=\frac{\frac{\rho}{1-\rho}}{\beta}=\frac{\rho}{\delta}\frac{1}{1-\rho}=\frac{W_{s}}{1-\rho}\\
&=&\frac{1}{\delta\left(1-\rho\right)}=\frac{1}{\delta-\beta}
\end{eqnarray*}
luego entonces
\begin{eqnarray*}
W_{q}&=&W-W_{s}=\frac{1}{\delta-\beta}-\frac{1}{\delta}=\frac{\beta}{\delta(\delta-\beta)}\\
&=&\frac{\rho}{1-\rho}\frac{1}{\delta}=\esp\left[s\right]\frac{\rho}{1-\rho}
\end{eqnarray*}

Entonces
\begin{eqnarray*}
L_{q}=\beta W_{q}=\frac{\rho^{2}}{1-\rho}.
\end{eqnarray*}
Finalmente
\begin{Prop}
\begin{enumerate}
\item $W\left(t\right)=1-e^{-\frac{t}{W}}$. \item
$W_{q}\left(t\right)=1-\rho\exp^{-\frac{t}{W}}$.
\end{enumerate}
donde $W=\esp(w)$.
\end{Prop}

%_____________________________________________________________________________________
%
\subsection{Cola $M/M/\infty$}
%_____________________________________________________________________________________
%

Este tipo de modelos se utilizan para estimar el n\'umero de l\'ineas
en uso en una gran red comunicaci\'on o para estimar valores en los
sistemas $M/M/c$ o $M/M/c/c$, en el se puede pensar que siempre
hay un servidor disponible para cada cliente que llega.\smallskip

Se puede considerar como un proceso de nacimiento y muerte con
par\'ametros $\beta_{n}=\beta$ y $\mu_{n}=n\mu$ para
$n=0,1,2,\ldots$, entonces por la ecuaci\'on \ref{Eq.2.4} se tiene
que
\begin{eqnarray*}\label{MMinf.pi}
\pi_{0}=e^{\rho}\\
\pi_{n}=e^{-\rho}\frac{\rho^{n}}{n!}
\end{eqnarray*}
Entonces, el n\'umero promedio de servidores ocupados es equivalente
a considerar el n\'umero de clientes en el  sistema, es decir,
\begin{eqnarray*}
L=\esp\left[N\right]=\rho\\
Var\left[N\right]=\rho
\end{eqnarray*}


Adem\'as se tiene que $W_{q}=0$ y $L_{q}=0$.\smallskip

El tiempo promedio en el sistema es el tiempo promedio de
servicio, es decir, $W=\esp\left[s\right]=\frac{1}{\delta}$.
Resumiendo, tenemos la sisuguiente proposici\'on:
\begin{Prop}
La cola $M/M/\infty$ es erg\'odica para todos los valores de $\eta$.
La distribuci\'on de equilibrio $\pi$ es Poisson con media $\eta$,
$\pi_{n}=\frac{e^{-n}\eta^{n}}{n!}$.
\end{Prop}

%_____________________________________________________________________________________
%
\subsection{Cola M/M/m}
%_____________________________________________________________________________________
%

Este sistema considera $m$ servidores id\'enticos, con tiempos entre
arribos y de servicio exponenciales con medias
$\esp\left[\tau\right]=\frac{1}{\beta}$ y
$\esp\left[s\right]=\frac{1}{\delta}$. definimos ahora la
utilizaci\'on por servidor como $u=\frac{\rho}{m}$ que tambi\'en se
puede interpretar como la fracci\'on de tiempo promedio que cada
servidor est\'a ocupado.\smallskip

La cola $M/M/m$ se puede considerar como un proceso de nacimiento
y muerte con par\'ametros: $\beta_{n}=\beta$ para $n=0,1,2,\ldots$ y
$\delta_{n}=\left\{\begin{array}{cc}
n\delta & n=0,1,\ldots,m-1\\
c\delta & n=m,\ldots\\
\end{array}\right.$

entonces  la condici\'on de recurrencia se va a cumplir s\'i y s\'olo si
$\sum_{n\geq1}\frac{\beta_{0}\cdots\beta_{n-1}}{\delta_{1}\cdots\delta_{n}}<\infty$,
equivalentemente se debe de cumplir que

\begin{eqnarray*}
S&=&1+\sum_{n\geq1}\frac{\beta_{0}\cdots\beta_{n-1}}{\delta_{1}\cdots\delta_{n}}=\sum_{n=0}^{m-1}\frac{\beta_{0}\cdots\beta_{n-1}}{\delta_{1}\cdots\delta_{n}}+\sum_{n=0}^{\infty}\frac{\beta_{0}\cdots\beta_{n-1}}{\delta_{1}\cdots\delta_{n}}\\
&=&\sum_{n=0}^{m-1}\frac{\beta^{n}}{n!\delta^{n}}+\sum_{n=0}^{\infty}\frac{\rho^{m}}{m!}u^{n}
\end{eqnarray*}
converja, lo cual ocurre si $u<1$, en este caso

\begin{eqnarray*}
S=\sum_{n=0}^{m-1}\frac{\rho^{n}}{n!}+\frac{\rho^{m}}{m!}\left(1-u\right)
\end{eqnarray*}
luego, para este caso se tiene que

\begin{eqnarray*}
\pi_{0}&=&\frac{1}{S}\\
\pi_{n}&=&\left\{\begin{array}{cc}
\pi_{0}\frac{\rho^{n}}{n!} & n=0,1,\ldots,m-1\\
\pi_{0}\frac{\rho^{n}}{m!m^{n-m}}& n=m,\ldots\\
\end{array}\right.
\end{eqnarray*}
Al igual que se hizo antes, determinaremos los valores de
$L_{q},W_{q},W$ y $L$:
\begin{eqnarray*}
L_{q}&=&\esp\left[N_{q}\right]=\sum_{n=0}^{\infty}\left(n-m\right)\pi_{n}=\sum_{n=0}^{\infty}n\pi_{n+m}\\
&=&\sum_{n=0}^{\infty}n\pi_{0}\frac{\rho^{n+m}}{m!m^{n+m}}=\pi_{0}\frac{\rho^{m}}{m!}\sum_{n=0}^{\infty}nu^{n}=\pi_{0}\frac{u\rho^{m}}{m!}\sum_{n=0}^{\infty}\frac{d}{du}u^{n}\\
&=&\pi_{0}\frac{u\rho^{m}}{m!}\frac{d}{du}\sum_{n=0}^{\infty}u^{n}=\pi_{0}\frac{u\rho^{m}}{m!}\frac{d}{du}\left(\frac{1}{1-u}\right)=\pi_{0}\frac{u\rho^{m}}{m!}\frac{1}{\left(1-u\right)^{2}}
\end{eqnarray*}
es decir
\begin{equation}
L_{q}=\frac{u\pi_{0}\rho^{m}}{m!\left(1-u\right)^{2}}
\end{equation}
luego
\begin{equation}
W_{q}=\frac{L_{q}}{\beta}
\end{equation}
\begin{equation}
W=W_{q}+\frac{1}{\delta}
\end{equation}
Si definimos
$C\left(m,\rho\right)=\frac{\pi_{0}\rho^{m}}{m!\left(1-u\right)}=\frac{\pi_{m}}{1-u}$,
que es la probabilidad de que un cliente que llegue al sistema
tenga que esperar en la cola. Entonces podemos reescribir las
ecuaciones reci\'en enunciadas:

\begin{eqnarray*}
L_{q}&=&\frac{C\left(m,\rho\right)u}{1-u}\\
W_{q}&=&\frac{C\left(m,\rho\right)\esp\left[s\right]}{m\left(1-u\right)}\\
\end{eqnarray*}

\begin{Prop}
La cola $M/M/m$ con intensidad de tr\'afico $\rho$ es erg\'odica si y
s\'olo si $\rho<1$. En este caso la distribuci\'on erg\'odica $\pi$ est\'a
dada por
\begin{eqnarray*}
\pi_{n}=\left\{\begin{array}{cc}
\frac{1}{S}\frac{\eta^{n}}{n!} & 0\leq n\leq m\\
\frac{1}{S}\frac{\eta^{m}}{m!}\rho^{n-m} & m\leq n<\infty\\
\end{array}\right.
\end{eqnarray*}
\end{Prop}
\begin{Prop}
Para $t\geq0$
\begin{itemize}
\item[a)]$W_{q}\left(t\right)=1-C\left(m,\rho\right)e^{-c\delta
t\left(1-u\right)}$ \item[b)]\begin{eqnarray*}
W\left(t\right)=\left\{\begin{array}{cc}
1+e^{-\delta t}\frac{\rho-m+W_{q}\left(0\right)}{m-1-\rho}+e^{-m\delta t\left(1-u\right)}\frac{C\left(m,\rho\right)}{m-1-\rho} & \rho\neq m-1\\
1-\left(1+C\left(m,\rho\right)\delta t\right)e^{-\delta t} & \rho=m-1\\
\end{array}\right.
\end{eqnarray*}
\end{itemize}
\end{Prop}

%_____________________________________________________________________________________
%
\subsection{Cola M/G/1}
%_____________________________________________________________________________________
%

Consideremos un sistema de espera con un servidor, en el que los
tiempos entre arribos son exponenciales, y los tiempos de servicio
tienen una distribuci\'on general $G$. Sea
$N\left(t\right)_{t\geq0}$ el n\'umero de clientes en el sistema al
tiempo $t$, y sean $t_{1}<t_{2}<\dots$ los tiempos sucesivos en
los que los clientes completan su servicio y salen del sistema.

La sucesi\'on $\left\{X_{n}\right\}$ definida por
$X_{n}=N\left(t\right)$ es una cadena de Markov, en espec\'ifico es
la Cadena encajada del proceso de llegadas de usuarios. Sea
$U_{n}$ el n\'umero de clientes que llegan al sistema durante el
tiempo de servicio del $n$-\'esimo cliente, entonces se tiene que

\begin{eqnarray*}
X_{n+1}=\left\{\begin{array}{cc}
X_{n}-1+U_{n+1} & \textrm{si }X_{n}\geq1,\\
U_{n+1} & \textrm{si }X_{n}=0\\
\end{array}\right.
\end{eqnarray*}

Dado que los procesos de arribos de los usuarios es Poisson con
par\'ametro $\lambda$, la probabilidad condicional de que lleguen
$j$ clientes al sistema dado que el tiempo de servicio es $s=t$,
resulta:
\begin{eqnarray*}
\prob\left\{U=j|s=t\right\}=e^{-\lambda t}\frac{\left(\lambda
t\right)^{j}}{j!}\textrm{,   }j=0,1,\ldots
\end{eqnarray*}
por el teorema de la probabilidad total se tiene que
\begin{equation}
a_{j}=\prob\left\{U=j\right\}=\int_{0}^{\infty}\prob\left\{U=j|s=t\right\}dG\left(t\right)=\int_{0}^{\infty}e^{-\lambda
t}\frac{\left(\lambda t\right)^{j}}{j!}dG\left(t\right)
\end{equation}
donde $G$ es la distribuci\'on de los tiempos de servicio. Las
probabilidades de transici\'on de la cadena est\'an dadas por
\begin{equation}
p_{0j}=\prob\left\{U_{n+1}=j\right\}=a_{j}\textrm{, para
}j=0,1,\ldots
\end{equation}
y para $i\geq1$
\begin{equation}
p_{ij}=\left\{\begin{array}{cc}
\prob\left\{U_{n+1}=j-i+1\right\}=a_{j-i+1}&\textrm{, para }j\geq i-1\\
0 & j<i-1\\
\end{array}
\right.
\end{equation}
Sea $\rho=\sum_{n=0}na_{n}$, entonces se tiene el siguiente
teorema:
\begin{Teo}
La cadena encajada $\left\{X_{n}\right\}$ es
\begin{itemize}
\item[a)] Recurrente positiva si $\rho<1$, \item[b)] Transitoria
si $\rho>1$, \item[c)] Recurrente nula si $\rho=1$.
\end{itemize}
\end{Teo}
Adem\'as, se tiene que
$\rho=\beta\esp\left[s\right]=\frac{\beta}{\delta}$ y la
distribuci\'on estacionaria est\'a dada por
\begin{eqnarray}
\pi_{j}&=&\pi_{0}a_{j}+\sum_{i=1}^{j+1}\pi_{i}a_{j-i+1}\textrm{, para }j=0,1,\ldots\\
\pi_{0}&=&1-\rho
\end{eqnarray}
Adem\'as se tiene que

\begin{equation}
L=\pi^{'}\left(1\right)=\rho+\frac{A^{''}\left(1\right)}{2\left(1-\rho\right)}
\end{equation}


pero $A^{''}\left(1\right)=\sum_{n=1}n\left(n-1\right)a_{n}=
\esp\left[U^{2}\right]-\esp\left[U\right]$,
$\esp\left[U\right]=\rho$ y
$\esp\left[U^{2}\right]=\lambda^{2}\esp\left[s^{2}\right]+\rho$.
Por lo tanto
$L=\rho+\frac{\beta^{2}\esp\left[s^{2}\right]}{2\left(1-\rho\right)}$.

De las f\'ormulas de Little, se tiene que
$W=E\left(w\right)=\frac{L}{\beta}$, tambi\'en el tiempo de espera
en la cola
\begin{equation}
W_{q}=\esp\left(q\right)=\esp\left(w\right)-\esp\left(s\right)=\frac{L}{\beta}-\frac{1}{\delta},
\end{equation}
adem\'as el n\'umero promedio de clientes en la cola es
\begin{equation}
L_{q}=\esp\left(N_{q}\right)=\beta W_{q}=L-\rho
\end{equation}

%_____________________________________________________________________________________
%
\subsection{Cola M/M/m/m}
%_____________________________________________________________________________________
%

Consideremos un sistema con $m$ servidores id\'enticos, pero ahora
cada uno es de capacidad finita $m$. Si todos los servidores se
encuentran ocupados, el siguiente usuario en llegar se pierde pues
no se le deja esperar a que reciba servicio. Este tipo de sistemas
pueden verse como un proceso de nacimiento y muerte con
\begin{eqnarray*}
\beta_{n}=\left\{\begin{array}{cc}
\beta & n=0,1,2,\ldots,m-1\\
0 & n\geq m\\
\end{array}
\right.
\end{eqnarray*}

\begin{eqnarray*}
\delta_{n}=\left\{\begin{array}{cc}
n\delta & n=0,1,2,\ldots,m-1\\
0 & n\geq m\\
\end{array}
\right.
\end{eqnarray*}
El proceso tiene epacio de estados finitos,
$S=\left\{0,1,\ldots,m\right\}$, entonces de las ecuaciones que
determinan la distribuci\'on estacionaria se tiene que
\begin{equation}\label{Eq.13.1}
\pi_{n}=\left\{\begin{array}{cc}
\pi_{0}\frac{\rho^{n}}{n!} & n=0,1,2,\ldots,m\\
0 & n\geq m\\
\end{array}
\right.
\end{equation}
y ademas
\begin{equation}
\pi_{0}=\left(\sum_{n=0}^{m}\frac{\rho^{n}}{n!}\right)^{-1}
\end{equation}
A la ecuaci\'on \ref{Eq.13.1} se le llama {\em distribuci\'on
truncada}.

Si definimos
$\pi_{m}=B\left(m,\rho\right)=\pi_{0}\frac{\rho^{m}}{m!}$,
$\pi_{m}$ representa la probabilidad de que todos los servidores
se encuentren ocupados, y tambi\'en se le conoce como {\em f\'ormula
de p\'erdida de Erlang}.

Necesariamente en este caso el tiempo de espera en la cola $W_{q}$
y el n\'umero promedio de clientes en la cola $L_{q}$ deben de ser
cero puesto que no se permite esperar para recibir servicio, m\'as
a\'un el tiempode espera en el sistema y el tiempo de serivcio
tienen la misma distribuci\'on, es decir,
\[W\left(t\right)=\prob\left\{w\leq t\right\}=1-e^{-\mu t}\], en
particular
\[W=\esp\left[w\right]=\esp\left[s\right]=\frac{1}{\delta}\]
Por otra parte, el n\'umero esperado de clientes en el sistema es
\begin{eqnarray*}
L&=&\esp\left[N\right]=\sum_{n=0}^{m}n\pi_{n}=\pi_{0}\rho\sum_{n=0}^{m}\frac{\rho^{n-1}}{\left(n-1\right)!}\\
&=&\pi_{0}\rho\sum_{n=0}^{m-1}\frac{\rho^{n}}{n!}
\end{eqnarray*}
entonces, se tiene que
\begin{equation}
L=\rho\left(1-B\left(m,\rho\right)\right)=\esp\left[s\right]\left(1-B\left(m,\rho\right)\right).
\end{equation}
Adem\'as
\begin{equation}
\delta_{q}=\delta\left(1-B\left(m,\rho\right)\right)
\end{equation}
representa la tasa promedio efectiva de arribos al sistema.


%_____________________________________________________________________________________
%
%_____________________________________________________________________________________
%
\section{Cadenas de Markov}
%_____________________________________________________________________________________
%
\subsection{Estacionareidad}
%_____________________________________________________________________________________
%}

Sea $v=\left(v_{i}\right)_{i\in E}$ medida no negativa en $E$,
podemos definir una nueva medida $v\prob$ que asigna masa
$\sum_{i\in E}v_{i}p_{ij}$ a cada estado $j$.\smallskip

\begin{Def}
La medida $v$ es estacionaria si $v_{i}<\infty$ para toda $i$ y
adem\'as $v\prob=v$.\smallskip
\end{Def}
En el caso de que $v$ sea distribuci\'on, independientemente de que
sea estacionaria o no, se cumple con

\begin{eqnarray*}
\prob_{v}\left[X_{1}=j\right]=\sum_{i\in
E}\prob_{v}\left[X_{0}=i\right]p_{ij}=\sum_{i\in
E}v_{i}p_{ij}=\left(vP\right)_{j}
\end{eqnarray*}

\begin{Teo}
Supongamos que $v$ es una distribuci\'on estacionaria. Entonces
\begin{itemize}
\item[i)] La cadena es estrictamente estacionaria con respecto a
$\prob_{v}$, es decir, $\prob_{v}$-distribuci\'on de
$\left\{X_{n},X_{n+1},\ldots\right\}$ no depende de $n$;
\item[ii)] Existe un aversi\'on estrictamente estacionaria
$\left\{X_{n}\right\}_{n\in Z}$ de la cadena con doble tiempo
infinito y $\prob\left(X_{n}=i\right)=v_{i}$ para toda $n\in Z$.
\end{itemize}
\end{Teo}
\begin{Teo}
Sea $i$ estado fijo, recurrente. Entonces una medida estacionaria
$v$ puede definirse haciendo que $v_{j}$ sea el n\'umero esperado de
visitas a $j$ entre dos visitas consecutivas $i$,
\begin{equation}\label{Eq.3.1}
v_{j}=\esp_{i}\sum_{n=0}^{\tau(i)-1}\indora\left(X_{n}=i\right)=\sum_{n=0}^{\infty}\prob_{i}\left[X_{n}=j,\tau(i)>n\right]
\end{equation}
\end{Teo}
\begin{Teo}\label{Teo.3.3}
Si la cadena es irreducible y recurrente, entonces una medida
estacionaria $v$ existe, satisface $0<v_{j}<\infty$ para toda $j$
y es \'unica salvo factores multiplicativos, es decir, si $v,v^{*}$
son estacionarias, entonces $c=cv^{*}$ para alguna
$c\in\left(0,\infty\right)$.
\end{Teo}
\begin{Cor}\label{Cor.3.5}
Si la cadena es irreducible y positiva recurrente, existe una
\'unica distribuci\'on estacionaria $\pi$ dada por
\begin{equation}
\pi_{j}=\frac{1}{\esp_{i}\tau_{i}}\esp_{i}\sum_{n=0}^{\tau\left(i\right)-1}\indora\left(X_{n}=j\right)=\frac{1}{\esp_{j}\tau\left(j\right)}.
\end{equation}
\end{Cor}
\begin{Cor}\label{Cor.3.6}
Cualquier cadena de Markov irreducible con un espacio de estados
finito es positiva recurrente.
\end{Cor}

%_____________________________________________________________________________________
%
\subsection{Teor\'ia Erg\'odica}
%_____________________________________________________________________________________
%

\begin{Lema}
Sea $\left\{X_{n}\right\}$ cadena irreducible y se $F$ subconjunto
finito del espacio de estados. Entonces la cadena es positiva
recurrente si $\esp_{i}\tau\left(F\right)<\infty$ para todo $i\in
F$.
\end{Lema}

\begin{Prop}
Sea $\left\{X_{n}\right\}$ cadena irreducible y transiente o cero
recurrente, entonces $p_{ij}^{n}\rightarrow0$ conforme
$n\rightarrow\infty$ para cualquier $i,j\in E$, $E$ espacio de
estados.
\end{Prop}
Utilizando el teorema (2.2) y el corolario ref{Cor.3.5}, se
demuestra el siguiente resultado importante.

\begin{Teo}
Sea $\left\{X_{n}\right\}$ cadena irreducible y aperi\'odica
positiva recurrente, y sea $\pi=\left\{\pi_{j}\right\}_{j\in E}$
la distribuci\'on estacionaria. Entonces
$p_{ij}^{n}\rightarrow\pi_{j}$ para todo $i,j$.
\end{Teo}
\begin{Def}\label{Def.Ergodicidad}
Una cadena irreducible aperiodica, positiva recurrente con medida
estacionaria $v$, es llamada {\em erg\'odica}.
\end{Def}

\begin{Prop}\label{Prop.4.4}
Sea $\left\{X_{n}\right\}$ cadena irreducible y recurrente con
medida estacionaria $v$, entocnes para todo $i,j,k,l\in E$
\begin{equation}
\frac{\sum_{n=0}^{m}p_{ij}^{n}}{\sum_{n=0}^{m}p_{lk}^{n}}\rightarrow\frac{v_{j}}{v_{k}}\textrm{,
}m\rightarrow\infty
\end{equation}
\end{Prop}
\begin{Lema}\label{Lema.4.5}
La matriz $\widetilde{P}$ con elementos
$\widetilde{p}_{ij}=\frac{v_{ji}p_{ji}}{v_{i}}$ es una matriz de
transici\'on. Adem\'as, el $i$-\'esimo elementos
$\widetilde{p}_{ij}^{m}$ de la matriz potencia $\widetilde{P}^{m}$
est\'a dada por
$\widetilde{p}_{ij}^{m}=\frac{v_{ji}p_{ji}^{m}}{v_{i}}$.
\end{Lema}

\begin{Lema}
Def\'inase $N_{i}^{m}=\sum_{n=0}^{m}\indora\left(X_{n}=i\right)$
como el n\'umero de visitas a $i$ antes del tiempo $m$. Entonces si
la cadena es reducible y recurrente,
$lim_{m\rightarrow\infty}\frac{\esp_{j}N_{i}^{m}}{\esp_{k}N_{i}^{m}}=1$
para todo $j,k\in E$.
\end{Lema}

%_____________________________________________________________________________________
%
\subsection{Funciones Arm\'onicas, Recurrencia y Transitoriedad}
%_____________________________________________________________________________________
%

\begin{Def}\label{Def.Armonica}
Una funci\'on Arm\'onica es el eigenvector derecho $h$ de $P$
corrrespondiente al eigenvalor 1.
\end{Def}
\begin{eqnarray*}
Ph=h\Leftrightarrow h\left(i\right)=\sum_{j\in
E}p_{ij}h\left(j\right)=\esp_{i}h\left(X_{1}\right)=\esp\left[h\left(X_{n+1}\right)|X_{n}=i\right].
\end{eqnarray*}
es decir, $\left\{h\left(X_{n}\right)\right\}$ es martingala.
\begin{Prop}\label{Prop.5.2}
Sea $\left\{X_{n}\right\}$ cadena irreducible  y sea $i$ estado
fijo arbitrario. Entonces la cadena es transitoria s\'i y s\'olo si
existe una funci\'on no cero, acotada
$h:E-\left\{i\right\}\rightarrow\rea$ que satisface
\begin{equation}\label{Eq.5.1}
h\left(j\right)=\sum_{k\neq i}p_{jk}h\left(k\right)\textrm{   para
}j\neq i.
\end{equation}
\end{Prop}

\begin{Prop}\label{Prop.5.3}
Supongamos que la cadena es irreducible y sea $E_{0}$ un
subconjunto finito del espacio de estados, entonces
\begin{itemize}
\item[i)] \item[ii)]
\end{itemize}
\end{Prop}

\begin{Prop}\label{Prop.5.4}
Suponga que la cadena es irreducible y sea $E_{0}$ un subconjunto
finito de $E$ tal que se cumple la ecuaci\'on 5.2 para alguna
funci\'on $h$ acotada que satisface
$h\left(i\right)<h\left(j\right)$ para alg\'un estado $i\notin
E_{0}$ y todo $j\in E_{0}$. Entonces la cadena es transitoria.
\end{Prop}



%_____________________________________________________________________________________
%
\section{Procesos de Markov de Saltos}
%_____________________________________________________________________________________
%
\subsection{Estructura B\'asica de los Procesos Markovianos de Saltos}
%_____________________________________________________________________________________
%

\begin{itemize}
\item Sea $E$ espacio discreto de estados, finito o numerable, y
sea $\left\{X_{t}\right\}$ un proceso de Markov con espacio de
estados $E$. Una medida $\mu$ en $E$ definida por sus
probabilidades puntuales $\mu_{i}$, escribimos
$p_{ij}^{t}=P^{t}\left(i,\left\{j\right\}\right)=P_{i}\left(X_{t}=j\right)$.\smallskip

\item El monto del tiempo gastado en cada estado es positivo, de
modo tal que las trayectorias muestrales son constantes por
partes. Para un proceso de saltos denotamos por los tiempos de
saltos a $S_{0}=0<S_{1}<S_{2}\cdots$, los tiempos entre saltos
consecutivos $T_{n}=S_{n+1}-S_{n}$ y la secuencia de estados
visitados por $Y_{0},Y_{1},\ldots$, as\'i las trayectorias
muestrales son constantes entre $S_{n}$ consecutivos, continua por
la derecha, es decir, $X_{S_{n}}=Y_{n}$. \item La descripci\'on de
un modelo pr\'actico est\'a dado usualmente en t\'erminos de las
intensidades $\lambda\left(i\right)$ y las probabilidades de salto
$q_{ij}$ m\'as que en t\'erminos de la matriz de transici\'on $P^{t}$.
\item Sup\'ongase de ahora en adelante que $q_{ii}=0$ cuando
$\lambda\left(i\right)>0$
\end{itemize}

%_____________________________________________________________________________________
%
\subsection{Matriz Intensidad}
%_____________________________________________________________________________________
%


\begin{Def}
La matriz intensidad
$\Lambda=\left(\lambda\left(i,j\right)\right)_{i,j\in E}$ del
proceso de saltos $\left\{X_{t}\right\}_{t\geq0}$ est\'a dada por
\begin{eqnarray*}
\lambda\left(i,j\right)&=&\lambda\left(i\right)q_{i,j}\textrm{,    }j\neq i\\
\lambda\left(i,i\right)&=&-\lambda\left(i\right)\\
\end{eqnarray*}

\begin{Prop}\label{Prop.3.1}
Una matriz $E\times E$, $\Lambda$ es la matriz de intensidad de un
proceso markoviano de saltos $\left\{X_{t}\right\}_{t\geq0}$ si y
s\'olo si
\begin{eqnarray*}
\lambda\left(i,i\right)\leq0\textrm{,
}\lambda\left(i,j\right)\textrm{,   }i\neq j\textrm{,  }\sum_{j\in
E}\lambda\left(i,j\right)=0.
\end{eqnarray*}
Adem\'as, $\Lambda$ est\'a en correspondencia uno a uno con la
distribuci\'on del proceso minimal dado por el teorema 3.1.
\end{Prop}

\end{Def}

Para el caso particular de la Cola $M/M/1$, la matr\'iz de itensidad
est\'a dada por
\begin{eqnarray*}
\Lambda=\left[\begin{array}{cccccc}
-\beta & \beta & 0 &0 &0& \cdots\\
\delta & -\beta-\delta & \beta & 0 & 0 &\cdots\\
0 & \delta & -\beta-\delta & \beta & 0 &\cdots\\
\vdots & & & & & \ddots\\
\end{array}\right]
\end{eqnarray*}

%____________________________________________________________________________
\subsection{Medidas Estacionarias}
%____________________________________________________________________________
%


\begin{Def}
Una medida $v\neq0$ es estacionaria si $0\leq v_{j}<\infty$,
$vP^{t}=v$ para toda $t$.
\end{Def}

\begin{Teo}\label{Teo.4.2}
Supongamos que $\left\{X_{t}\right\}$ es irreducible recurrente en
$E$. Entonces existe una y s\'olo una, salvo m\'ultiplos, medida
estacionaria $v$. Esta $v$ tiene la propiedad de que $0<
v_{j}<\infty$ para todo $j$ y puede encontrarse en cualquiera de
las siguientes formas

\begin{itemize}
\item[i)] Para alg\'un estado $i$, fijo pero arbitrario, $v_{j}$ es
el tiempo esperado utilizado en $j$ entre dos llegadas
consecutivas al estado $i$;
\begin{equation}\label{Eq.4.2}
v_{j}=\esp_{i}\int_{0}^{w\left(i\right)}\indora\left(X_{t}=j\right)dt
\end{equation}
con
$w\left(i\right)=\inf\left\{t>0:X_{t}=i,X_{t^{-}}=\lim_{s\uparrow
t}X_{s}\neq i\right\}$. \item[ii)]
$v_{j}=\frac{\mu_{j}}{\lambda\left(j\right)}$, donde $\mu$ es
estacionaria para $\left\{Y_{n}\right\}$. \item[iii)] como
soluci\'on de $v\Lambda=0$.
\end{itemize}
\end{Teo}

%____________________________________________________________________________
\subsection{Criterios de Ergodicidad}
%____________________________________________________________________________
%

\begin{Def}
Un proceso irreducible recurrente con medida estacionaria de masa
finita es llamado erg\'odico.
\end{Def}

\begin{Teo}\label{Teo.4.3}
Un proceso de Markov de saltos irreducible no explosivo es
erg\'odico si y s\'olo si se puede encontrar una soluci\'on, de
probabilidad, $\pi$, con $|\pi|=1$ y $0\leq\pi_{j}\leq1$, a
$\pi\Lambda=0$. En este caso $\pi$ es la distribuci\'on
estacionaria.
\end{Teo}

\begin{Cor}\label{Cor.4.4}
Una condici\'on suficiente para la ergodicidad de un proceso
irreducible es la existencia de una probabilidad $\pi$ que
resuelva el sistema $\pi\Lambda=0$ y que adem\'as tenga la propiedad
de que $\sum\pi_{j}\lambda\left(j\right)<\infty$.
\end{Cor}
\begin{Prop}
Si el proceso es erg\'odico, entonces existe una versi\'on
estrictamente estacionaria
$\left\{X_{t}\right\}_{-\infty<t<\infty}$con doble tiempo
infinito.
\end{Prop}

\begin{Teo}
Si $\left\{X_{t}\right\}$ es erg\'odico y $\pi$ es la distribuci\'on
estacionaria, entonces para todo $i,j$,
$p_{ij}^{t}\rightarrow\pi_{j}$ cuando $t\rightarrow\infty$.
\end{Teo}

\begin{Cor}
Si $\left\{X_{t}\right\}$ es irreducible recurente pero no
erg\'odica, es decir $|v|=\infty$, entonces $p_{ij}^{t}\rightarrow0$
para todo $i,j\in E$.
\end{Cor}

\begin{Cor}
Para cualquier proceso Markoviano de Saltos minimal, irreducible o
no, los l\'imites $li_{t\rightarrow\infty}p_{ij}^{t}$ existe.
\end{Cor}

%_____________________________________________________________________________________
%
\section{Notaci\'on Kendall-Lee}
%_____________________________________________________________________________________
%


A partir de este momento se har\'an las siguientes consideraciones:
Si $t_{n}$ es el tiempo aleatorio en el que llega al sistema el
$n$-\'esimo cliente, para $n=1,2,\ldots$, $t_{0}=0$ y
$t_{0}<t_{1}<\cdots$ se definen los tiempos entre arribos
$\tau_{n}=t_{n}-t_{n-1}$ para $n=1,2,\ldots$, variables aleatorias
independientes e id\'enticamente distribuidas. Los tiempos entre
arribos tienen un valor medio $E\left(\tau\right)$ finito y
positivo $\frac{1}{\beta}$, es decir, $\beta$ se puede ver como la
tasa o intensidad promedio de arribos al sistema por unidad de
tiempo. Adem\'as se supondr\'a que los servidores son identicos y si
$s$ denota la variable aleatoria que describe el tiempo de
servicio, entonces $E\left(s\right)=\frac{1}{\delta}$, $\delta$ es
la tasa promedio de servicio por servidor.

La notaci\'on de Kendall-Lee es una forma abreviada de describir un
sisema de espera con las siguientes componentes:
\begin{itemize}
\item[i)] {\em\bf Fuente}: Poblaci\'on de clientes potenciales del
sistema, esta puede ser finita o infinita. \item[ii)] {\em\bf
Proceso de Arribos}: Proceso determinado por la funci\'on de
distribuci\'on $A\left(t\right)=P\left\{\tau\leq t\right\}$ de los
tiempos entre arribos.
\end{itemize}


Adem\'as tenemos las siguientes igualdades
\begin{equation}\label{Eq.0.1}
N\left(t\right)=N_{q}\left(t\right)+N_{s}\left(s\right)
\end{equation}
donde
\begin{itemize}
\item $N\left(t\right)$ es el n\'umero de clientes en el sistema al
tiempo $t$. \item $N_{q}\left(t\right)$ es el n\'umero de clientes
en la cola al tiempo $t$ \item $N_{s}\left(t\right)$ es el n\'umero
de clientes recibiendo servicio en el tiempo $t$.
\end{itemize}

Bajo la hip\'otesis de estacionareidad, es decir, las
caracter\'isticas de funcionamiento del sistema se han estabilizado
en valores independientes del tiempo, entonces
\begin{equation}
N=N_{q}+N_{s}.
\end{equation}

Los valores medios de las cantidades anteriores se escriben como
$L=E\left(N\right)$, $L_{q}=E\left(N_{q}\right)$ y
$L_{s}=E\left(N_{s}\right)$, entonces de la ecuaci\'on \ref{Eq.0.1}
se obtiene

\begin{equation}
L=L_{q}+L_{s}
\end{equation}

Si $q$ es el tiempo que pasa un cliente en la cola antes de
recibir servicio, y W es el tiempo total que un cliente pasa en el
sistema, entonces
\[w=q+s\]
por lo tanto
\[W=W_{q}+W_{s},\]
donde $W=E\left(w\right)$, $W_{q}=E\left(q\right)$ y
$W_{s}=E\left(s\right)=\frac{1}{\delta}$.

La intensidad de tr\'afico se define como
\begin{equation}
\rho=\frac{E\left(s\right)}{E\left(\tau\right)}=\frac{\beta}{\delta}.
\end{equation}

La utilizaci\'on por servidor es
\begin{equation}
u=\frac{\rho}{c}=\frac{\beta}{c\delta}.
\end{equation}
donde $c$ es el n\'umero de servidores.

Esta notaci\'on es una forma abreviada de describir un sistema de
espera con componentes dados a continuaci\'on, la notaci\'on es
\begin{equation}\label{Notacion.K.L.}
A/S/c/K/F/d
\end{equation}
Cada una de las letras describe:
\begin{itemize}
\item $A$ es la distribuci\'on de los tiempos entre arribos. \item
$S$ es la distribuci\'on del tiempo de servicio. \item $c$ es el
n\'umero de servidores. \item $K$ es la capacidad del sistema. \item
$F$ es el n\'umero de individuos en la fuente. \item $d$ es la
disciplina del servicio
\end{itemize}
Usualmente se acostumbra suponer que $K=\infty$, $F=\infty$ y
$d=FIFO$, es decir, First In First Out.

Las distribuciones usuales para $A$ y $B$ son:
\begin{itemize}
\item $GI$ para la distribuci\'on general de los tiempos entre
arribos. \item $G$ distribuci\'on general del tiempo de servicio.
\item $M$ Distribuci\'on exponencial para $A$ o $S$. \item $E_{K}$
Distribuci\'on Erlang-$K$, para $A$ o $S$. \item $D$ tiempos entre
arribos o de servicio constantes, es decir, deterministicos.
\end{itemize}

%_____________________________________________________________________________________
%
\section{Procesos de Nacimiento y Muerte}
%_____________________________________________________________________________________
%
\subsection{Procesos de Nacimiento y Muerte Generales}
%_____________________________________________________________________________________
%

Por un proceso de nacimiento y muerte se entiende un proceso de
saltos de markov $\left\{X_{t}\right\}_{t\geq0}$ con espacio de
estados a lo m\'as numerable, con la propiedad de que s\'olo puede ir
al estado $n+1$ o al estado $n-1$, es decir, su matriz de
intensidad es de la forma
\begin{eqnarray*}
\Lambda=\left[\begin{array}{cccccc}
-\beta_{0} & \beta_{0} & 0 &0 &0& \cdots\\
\delta_{1} & -\beta_{1}-\delta_{1} & \beta_{1} & 0 & 0 &\cdots\\
0 & \delta_{2} & -\beta_{2}-\delta_{2} & \beta_{2} & 0 &\cdots\\
\vdots & & & & & \ddots\\
\end{array}\right]
\end{eqnarray*}
donde $\beta_{n}$ son las intensidades de nacimiento y
$\delta_{n}$ las intensidades de muerte, o tambi\'en se puede ver
como a $X_{t}$ el n\'umero de usuarios en una cola al tiempo $t$, un
salto hacia arriba corresponde a la llegada de un nuevo usuario y
un salto hacia abajo como al abandono de un usuario despu\'es de
haber recibido su servicio.

La cadena de saltos $\left\{Y_{n}\right\}$ tiene matriz de
transici\'on dada por
\begin{eqnarray*}
Q=\left[\begin{array}{cccccc}
0 & 1 & 0 &0 &0& \cdots\\
q_{1} & 0 & p_{1} & 0 & 0 &\cdots\\
0 & q_{2} & 0 & p_{2} & 0 &\cdots\\
\vdots & & & & & \ddots\\
\end{array}\right]
\end{eqnarray*}
donde $p_{n}=\frac{\beta_{n}}{\beta_{n}+\delta_{n}}$ y
$q_{n}=1-p_{n}=\frac{\delta_{n}}{\beta_{n}+\delta_{n}}$, donde
adem\'as se asumne por el momento que $p_{n}$ no puede tomar el
valor $0$ \'o $1$ para cualquier valor de $n$.

\begin{Prop}\label{Prop.2.1}
La recurrencia de $\left\{X_{t}\right\}$, o equivalentemente de
$\left\{Y_{n}\right\}$ es equivalente a
\begin{equation}\label{Eq.2.1}
\sum_{n=1}^{\infty}\frac{\delta_{1}\cdots\delta_{n}}{\beta_{1}\cdots\beta_{n}}=\sum_{n=1}^{\infty}\frac{q_{1}\cdots
q_{n}}{p_{1}\cdots p_{n}}=\infty
\end{equation}
\end{Prop}

\begin{Lema}\label{Lema.2.2}
Independientemente de la recurrencia o transitorieadad, existe una
y s\'olo una, salvo m\'ultiplos, soluci\'on a $v\Lambda=0$, dada por
\begin{equation}\label{Eq.2.2}
v_{n}=\frac{\beta_{0}\cdots\beta_{n-1}}{\delta_{1}\cdots\delta_{n}}v_{0}
\end{equation}
para $n=1,2,\ldots$.
\end{Lema}

\begin{Cor}\label{Cor.2.3}
En el caso recurrente, la medida estacionaria $\mu$ para
$\left\{Y_{n}\right\}$ est\'a dada por
\begin{equation}\label{Eq.}
\mu_{n}=\frac{p_{1}\cdots p_{n-1}}{q_{1}\cdots q_{n}}\mu_{0}
\end{equation}
para $n=1,2,\ldots$.
\end{Cor}

Se define a
$S=1+\sum_{n=1}^{\infty}\frac{\beta_{0}\beta_{1}\cdots\beta_{n-1}}{\delta_{1}\delta_{2}\cdots\delta_{n}}$

\begin{Cor}\label{Cor.2.4}
$\left\{X_{t}\right\}$ es erg\'odica si y s\'olo si la ecuaci\'on
(\ref{Eq.2.1}) se cumple y adem\'as $S<\infty$, en cuyo caso la
distribuci\'on erg\'odica, $\pi$, est\'a dada por
\begin{equation}\label{Eq.2.4}
\pi_{0}=\frac{1}{S}\textrm{,
}\pi_{n}=\frac{1}{S}\frac{\beta_{0}\cdots\beta_{n-1}}{\delta_{1}\cdots\delta_{n}}
\end{equation}
para $n=1,2,\ldots$.
\end{Cor}

%_____________________________________________________________________________________
%
\subsection{Cola M/M/1}
%_____________________________________________________________________________________
%


Este modelo corresponde a un proceso de nacimiento y muerte con
$\beta_{n}=\beta$ y $\delta_{n}=\delta$ independiente del valor de
$n$. La intensidad de tr\'afico $\rho=\frac{\beta}{\delta}$, implica
que el criterio de recurrencia (ecuaci\'on \ref{Eq.2.1}) quede de la
forma:
\begin{eqnarray*}
1+\sum_{n=1}^{\infty}\rho^{-n}=\infty.
\end{eqnarray*}
Equivalentemente el proceso es recurrente si y s\'olo si
\begin{eqnarray*}
\sum_{n\geq1}\left(\frac{\beta}{\delta}\right)^{n}<\infty\Leftrightarrow
\frac{\beta}{\delta}<1
\end{eqnarray*}
Entonces $S=\frac{\delta}{\delta-\beta}$, luego por la ecuaci\'on
\ref{Eq.2.4} se tiene que
\begin{eqnarray*}
\pi_{0}&=&\frac{\delta-\beta}{\delta}=1-\frac{\beta}{\delta}\\
\pi_{n}&=&\pi_{0}\left(\frac{\beta}{\delta}\right)^{n}=\left(1-\frac{\beta}{\delta}\right)\left(\frac{\beta}{\delta}\right)^{n}=\left(1-\rho\right)\rho^{n}
\end{eqnarray*}

Lo cual nos lleva a la siguiente

\begin{Prop}
La cola $M/M/1$ con intendisad de tr\'afico $\rho$, es recurrente si
y s\'olo si $\rho\leq1$.
\end{Prop}

Entonces por el corolario \ref{Cor.2.3}

\begin{Prop}
La cola $M/M/1$ con intensidad de tr\'afico $\rho$ es erg\'odica si y
s\'olo si $\rho<1$. En cuyo caso, la distribuci\'on de equilibrio
$\pi$ de la longitud de la cola es geom\'etrica,
$\pi_{n}=\left(1-\rho\right)\rho^{n}$, para $n=1,2,\ldots$.
\end{Prop}
De la proposici\'on anterior se desprenden varios hechos
importantes.
\begin{enumerate}
\item $\prob\left[X_{t}=0\right]=\pi_{0}=1-\rho$, es decir, la
probabilidad de que el sistema se encuentre ocupado. \item De las
propiedades de la distribuci\'on Geom\'etrica se desprende que
\begin{enumerate}
\item $\esp\left[X_{t}\right]=\frac{\rho}{1-\rho}$, \item
$Var\left[X_{t}\right]=\frac{\rho}{\left(1-\rho\right)^{2}}$.
\end{enumerate}
\end{enumerate}

Si $L$ es el n\'umero esperado de clientes en el sistema, incluyendo
los que est\'an siendo atendidos, entonces
\begin{eqnarray*}
L=\frac{\rho}{1-\rho}
\end{eqnarray*}
Si adem\'as $W$ es el tiempo total del cliente en la cola:
$W=W_{q}+W_{s}$
$\rho=\frac{\esp\left[s\right]}{\esp\left[\tau\right]}=\beta
W_{s}$, puesto que $W_{s}=\esp\left[s\right]$ y
$\esp\left[\tau\right]=\frac{1}{\delta}$. Por la f\'ormula de Little
$L=\lambda W$
\begin{eqnarray*}
W&=&\frac{L}{\beta}=\frac{\frac{\rho}{1-\rho}}{\beta}=\frac{\rho}{\delta}\frac{1}{1-\rho}=\frac{W_{s}}{1-\rho}\\
&=&\frac{1}{\delta\left(1-\rho\right)}=\frac{1}{\delta-\beta}
\end{eqnarray*}
luego entonces
\begin{eqnarray*}
W_{q}&=&W-W_{s}=\frac{1}{\delta-\beta}-\frac{1}{\delta}=\frac{\beta}{\delta(\delta-\beta)}\\
&=&\frac{\rho}{1-\rho}\frac{1}{\delta}=\esp\left[s\right]\frac{\rho}{1-\rho}
\end{eqnarray*}
Entonces
\begin{eqnarray*}
L_{q}=\beta W_{q}=\frac{\rho^{2}}{1-\rho}.
\end{eqnarray*}
Finalmente
\begin{Prop}
\begin{enumerate}
\item $W\left(t\right)=1-e^{-\frac{t}{W}}$. \item
$W_{q}\left(t\right)=1-\rho\exp^{-\frac{t}{W}}$.
\end{enumerate}
donde $W=\esp(w)$.
\end{Prop}


%_____________________________________________________________________________________
%
\subsection{Cola $M/M/\infty$}
%_____________________________________________________________________________________
%

Este tipo de modelos se utilizan para estimar el n\'umero de l\'ineas
en uso en una gran red comunicaci\'on o para estimar valores en los
sistemas $M/M/c$ o $M/M/c/c$, en el se puede pensar que siempre
hay un servidor disponible para cada cliente que llega.\smallskip

Se puede considerar como un proceso de nacimiento y muerte con
par\'ametros $\beta_{n}=\beta$ y $\mu_{n}=n\mu$ para
$n=0,1,2,\ldots$, entonces por la ecuaci\'on \ref{Eq.2.4} se tiene
que
\begin{eqnarray*}\label{MMinf.pi}
\pi_{0}=e^{\rho}\\
\pi_{n}=e^{-\rho}\frac{\rho^{n}}{n!}
\end{eqnarray*}
Entonces, el n\'umero promedio de servidores ocupados es equivalente
a considerar el n\'umero de clientes en el  sistema, es decir,
\begin{eqnarray*}
L=\esp\left[N\right]=\rho\\
Var\left[N\right]=\rho
\end{eqnarray*}

Adem\'as se tiene que $W_{q}=0$ y $L_{q}=0$.\smallskip

El tiempo promedio en el sistema es el tiempo promedio de
servicio, es decir, $W=\esp\left[s\right]=\frac{1}{\delta}$.
Resumiendo, tenemos la sisuguiente proposici\'on:
\begin{Prop}
La cola $M/M/\infty$ es erg\'odica para todos los valores de $\eta$.
La distribuci\'on de equilibrio $\pi$ es Poisson con media $\eta$,
$\pi_{n}=\frac{e^{-n}\eta^{n}}{n!}$.
\end{Prop}

%_____________________________________________________________________________________
%
\subsection{Cola M/M/m}
%_____________________________________________________________________________________
%

Este sistema considera $m$ servidores id\'enticos, con tiempos entre
arribos y de servicio exponenciales con medias
$\esp\left[\tau\right]=\frac{1}{\beta}$ y
$\esp\left[s\right]=\frac{1}{\delta}$. definimos ahora la
utilizaci\'on por servidor como $u=\frac{\rho}{m}$ que tambi\'en se
puede interpretar como la fracci\'on de tiempo promedio que cada
servidor est\'a ocupado.\smallskip

La cola $M/M/m$ se puede considerar como un proceso de nacimiento
y muerte con par\'ametros: $\beta_{n}=\beta$ para $n=0,1,2,\ldots$ y
$\delta_{n}=\left\{\begin{array}{cc}
n\delta & n=0,1,\ldots,m-1\\
c\delta & n=m,\ldots\\
\end{array}\right.$

entonces  la condici\'on de recurrencia se va a cumplir s\'i y s\'olo si
$\sum_{n\geq1}\frac{\beta_{0}\cdots\beta_{n-1}}{\delta_{1}\cdots\delta_{n}}<\infty$,
equivalentemente se debe de cumplir que

\begin{eqnarray*}
S&=&1+\sum_{n\geq1}\frac{\beta_{0}\cdots\beta_{n-1}}{\delta_{1}\cdots\delta_{n}}=\sum_{n=0}^{m-1}\frac{\beta_{0}\cdots\beta_{n-1}}{\delta_{1}\cdots\delta_{n}}+\sum_{n=0}^{\infty}\frac{\beta_{0}\cdots\beta_{n-1}}{\delta_{1}\cdots\delta_{n}}\\
&=&\sum_{n=0}^{m-1}\frac{\beta^{n}}{n!\delta^{n}}+\sum_{n=0}^{\infty}\frac{\rho^{m}}{m!}u^{n}
\end{eqnarray*}
converja, lo cual ocurre si $u<1$, en este caso

\begin{eqnarray*}
S=\sum_{n=0}^{m-1}\frac{\rho^{n}}{n!}+\frac{\rho^{m}}{m!}\left(1-u\right)
\end{eqnarray*}
luego, para este caso se tiene que

\begin{eqnarray*}
\pi_{0}&=&\frac{1}{S}\\
\pi_{n}&=&\left\{\begin{array}{cc}
\pi_{0}\frac{\rho^{n}}{n!} & n=0,1,\ldots,m-1\\
\pi_{0}\frac{\rho^{n}}{m!m^{n-m}}& n=m,\ldots\\
\end{array}\right.
\end{eqnarray*}
Al igual que se hizo antes, determinaremos los valores de
$L_{q},W_{q},W$ y $L$:
\begin{eqnarray*}
L_{q}&=&\esp\left[N_{q}\right]=\sum_{n=0}^{\infty}\left(n-m\right)\pi_{n}=\sum_{n=0}^{\infty}n\pi_{n+m}\\
&=&\sum_{n=0}^{\infty}n\pi_{0}\frac{\rho^{n+m}}{m!m^{n+m}}=\pi_{0}\frac{\rho^{m}}{m!}\sum_{n=0}^{\infty}nu^{n}=\pi_{0}\frac{u\rho^{m}}{m!}\sum_{n=0}^{\infty}\frac{d}{du}u^{n}\\
&=&\pi_{0}\frac{u\rho^{m}}{m!}\frac{d}{du}\sum_{n=0}^{\infty}u^{n}=\pi_{0}\frac{u\rho^{m}}{m!}\frac{d}{du}\left(\frac{1}{1-u}\right)=\pi_{0}\frac{u\rho^{m}}{m!}\frac{1}{\left(1-u\right)^{2}}
\end{eqnarray*}

es decir
\begin{equation}
L_{q}=\frac{u\pi_{0}\rho^{m}}{m!\left(1-u\right)^{2}}
\end{equation}
luego
\begin{equation}
W_{q}=\frac{L_{q}}{\beta}
\end{equation}
\begin{equation}
W=W_{q}+\frac{1}{\delta}
\end{equation}
Si definimos
$C\left(m,\rho\right)=\frac{\pi_{0}\rho^{m}}{m!\left(1-u\right)}=\frac{\pi_{m}}{1-u}$,
que es la probabilidad de que un cliente que llegue al sistema
tenga que esperar en la cola. Entonces podemos reescribir las
ecuaciones reci\'en enunciadas:

\begin{eqnarray*}
L_{q}&=&\frac{C\left(m,\rho\right)u}{1-u}\\
W_{q}&=&\frac{C\left(m,\rho\right)\esp\left[s\right]}{m\left(1-u\right)}\\
\end{eqnarray*}

\begin{Prop}
La cola $M/M/m$ con intensidad de tr\'afico $\rho$ es erg\'odica si y
s\'olo si $\rho<1$. En este caso la distribuci\'on erg\'odica $\pi$ est\'a
dada por
\begin{eqnarray*}
\pi_{n}=\left\{\begin{array}{cc}
\frac{1}{S}\frac{\eta^{n}}{n!} & 0\leq n\leq m\\
\frac{1}{S}\frac{\eta^{m}}{m!}\rho^{n-m} & m\leq n<\infty\\
\end{array}\right.
\end{eqnarray*}
\end{Prop}
\begin{Prop}
Para $t\geq0$
\begin{itemize}
\item[a)]$W_{q}\left(t\right)=1-C\left(m,\rho\right)e^{-c\delta
t\left(1-u\right)}$ \item[b)]\begin{eqnarray*}
W\left(t\right)=\left\{\begin{array}{cc}
1+e^{-\delta t}\frac{\rho-m+W_{q}\left(0\right)}{m-1-\rho}+e^{-m\delta t\left(1-u\right)}\frac{C\left(m,\rho\right)}{m-1-\rho} & \rho\neq m-1\\
1-\left(1+C\left(m,\rho\right)\delta t\right)e^{-\delta t} & \rho=m-1\\
\end{array}\right.
\end{eqnarray*}
\end{itemize}
\end{Prop}

%_____________________________________________________________________________________
%
\subsection{Cola M/G/1}
%_____________________________________________________________________________________
%
Consideremos un sistema de espera con un servidor, en el que los
tiempos entre arribos son exponenciales, y los tiempos de servicio
tienen una distribuci\'on general $G$. Sea
$N\left(t\right)_{t\geq0}$ el n\'umero de clientes en el sistema al
tiempo $t$, y sean $t_{1}<t_{2}<\dots$ los tiempos sucesivos en
los que los clientes completan su servicio y salen del sistema.

La sucesi\'on $\left\{X_{n}\right\}$ definida por
$X_{n}=N\left(t_{n}\right)$ es una cadena de Markov, en espec\'ifico
es la Cadena encajada del proceso de llegadas de usuarios. Sea
$U_{n}$ el n\'umero de clientes que llegan al sistema durante el
tiempo de servicio del $n$-\'esimo cliente, entonces se tiene que

\begin{eqnarray*}
X_{n+1}=\left\{\begin{array}{cc}
X_{n}-1+U_{n+1} & \textrm{si }X_{n}\geq1,\\
U_{n+1} & \textrm{si }X_{n}=0\\
\end{array}\right.
\end{eqnarray*}

Dado que los procesos de arribos de los usuarios es Poisson con
par\'ametro $\lambda$, la probabilidad condicional de que lleguen
$j$ clientes al sistema dado que el tiempo de servicio es $s=t$,
resulta:
\begin{eqnarray*}
\prob\left\{U=j|s=t\right\}=e^{-\lambda t}\frac{\left(\lambda
t\right)^{j}}{j!}\textrm{,   }j=0,1,\ldots
\end{eqnarray*}
por el teorema de la probabilidad total se tiene que
\begin{equation}
a_{j}=\prob\left\{U=j\right\}=\int_{0}^{\infty}\prob\left\{U=j|s=t\right\}dG\left(t\right)=\int_{0}^{\infty}e^{-\lambda
t}\frac{\left(\lambda t\right)^{j}}{j!}dG\left(t\right)
\end{equation}
donde $G$ es la distribuci\'on de los tiempos de servicio. Las
probabilidades de transici\'on de la cadena est\'an dadas por
\begin{equation}
p_{0j}=\prob\left\{U_{n+1}=j\right\}=a_{j}\textrm{, para
}j=0,1,\ldots
\end{equation}
y para $i\geq1$


\begin{equation}
p_{ij}=\left\{\begin{array}{cc}
\prob\left\{U_{n+1}=j-i+1\right\}=a_{j-i+1}&\textrm{, para }j\geq i-1\\
0 & j<i-1\\
\end{array}
\right.
\end{equation}
Entonces la matriz de transici\'on es:
\begin{eqnarray*}
P=\left[\begin{array}{ccccc}
a_{0} & a_{1} & a_{2} & a_{3} & \cdots\\
a_{0} & a_{1} & a_{2} & a_{3} & \cdots\\
0 & a_{0} & a_{1} & a_{2} & \cdots\\
0 & 0 & a_{0} & a_{1} & \cdots\\
\vdots & \vdots & \cdots & \ddots &\vdots\\
\end{array}
\right]
\end{eqnarray*}
Sea $\rho=\sum_{n=0}na_{n}$, entonces se tiene el siguiente
teorema:
\begin{Teo}
La cadena encajada $\left\{X_{n}\right\}$ es
\begin{itemize}
\item[a)] Recurrente positiva si $\rho<1$, \item[b)] Transitoria
si $\rho>1$, \item[c)] Recurrente nula si $\rho=1$.
\end{itemize}
\end{Teo}

Recordemos que si la cadena de Markov $\left\{X_{n}\right\}$ tiene
una distribuci\'on estacionaria entonces existe una distribuci\'on de
probabilidad $\pi=\left(\pi_{0},\pi_{1},\ldots,\right)$, con
$\pi_{i}\geq0$ y $\sum_{i\geq1}\pi_{i}=1$ tal que satisface la
ecuaci\'on $\pi=\pi P$, equivalentemente
\begin{equation}\label{Eq.18.9}
\pi_{j}=\sum_{i=0}^{\infty}\pi_{k}p_{ij},\textrm{ para
}j=0,1,2,\ldots
\end{equation}
que se puede ver como
\begin{equation}\label{Eq.19.6}
\pi_{j}=\pi_{0}a_{j}+\sum_{i=1}^{j+1}\pi_{i}a_{j-i+1}\textrm{,
para }j=0,1,\ldots
\end{equation}
si definimos\[\pi\left(z\right)=\sum_{j=0}^{\infty}\pi_{j}z^{j}\]
y \[A\left(z\right)=\sum_{j=0}^{\infty}a_{j}z^{j}\] con
$|z_{j}|\leq1$.

Si la ecuaci\'on \ref{Eq.19.6} la multiplicamos por $z^{j}$ y
sumando sobre $j$, se tiene que
\begin{eqnarray*}
\sum_{j=0}^{\infty}\pi_{j}z^{j}&=&\sum_{j=0}^{\infty}\pi_{0}a_{j}z^{j}+\sum_{j=0}^{\infty}\sum_{i=1}^{j+1}\pi_{i}a_{j-i+1}z^{j}\\
&=&\pi_{0}\sum_{j=0}^{\infty}a_{j}z^{j}+\sum_{j=0}^{\infty}a_{j}z^{j}\sum_{i=1}^{\infty}\pi_{i}a_{i-1}\\
&=&\pi_{0}A\left(z\right)+A\left(z\right)\left(\frac{\pi\left(z\right)-\pi_{0}}{z}\right)\\
\end{eqnarray*}
es decir,

\begin{equation}
\pi\left(z\right)=\pi_{0}A\left(z\right)+A\left(z\right)\left(\frac{\pi\left(z\right)-\pi_{0}}{z}\right)\Leftrightarrow\pi\left(z\right)=\frac{\pi_{0}A\left(z\right)\left(z-1\right)}{z-A\left(z\right)}
\end{equation}

Si $z\rightarrow1$, entonces $A\left(z\right)\rightarrow
A\left(1\right)=1$, y adem\'as $A^{'}\left(z\right)\rightarrow
A^{'}\left(1\right)=\rho$. Si aplicamos la Regla de L'Hospital se
tiene que
\begin{eqnarray*}
\sum_{j=0}^{\infty}\pi_{j}=lim_{z\rightarrow1^{-}}\pi\left(z\right)=\pi_{0}lim_{z\rightarrow1^{-}}\frac{z-1}{z-A\left(z\right)}=\frac{\pi_{0}}{1-\rho}
\end{eqnarray*}

Retomando,
\begin{eqnarray*}
a_{j}=\prob\left\{U=j\right\}=\int_{0}^{\infty}e^{-\lambda
t}\frac{\left(\lambda t\right)^{n}}{n!}dG\left(t\right)\textrm{,
para }n=0,1,2,\ldots
\end{eqnarray*}
entonces
\begin{eqnarray*}
\rho&=&\sum_{n=0}^{\infty}na_{n}=\sum_{n=0}^{\infty}n\int_{0}^{\infty}e^{-\lambda t}\frac{\left(\lambda t\right)^{n}}{n!}dG\left(t\right)\\
&=&\int_{0}^{\infty}\sum_{n=0}^{\infty}ne^{-\lambda
t}\frac{\left(\lambda
t\right)^{n}}{n!}dG\left(t\right)=\int_{0}^{\infty}\lambda
tdG\left(t\right)=\lambda\esp\left[s\right]
\end{eqnarray*}

Adem\'as, se tiene que
$\rho=\beta\esp\left[s\right]=\frac{\beta}{\delta}$ y la
distribuci\'on estacionaria est\'a dada por
\begin{eqnarray}
\pi_{j}&=&\pi_{0}a_{j}+\sum_{i=1}^{j+1}\pi_{i}a_{j-i+1}\textrm{, para }j=0,1,\ldots\\
\pi_{0}&=&1-\rho
\end{eqnarray}
Por otra parte se tiene que

\begin{equation}
L=\pi^{'}\left(1\right)=\rho+\frac{A^{''}\left(1\right)}{2\left(1-\rho\right)}
\end{equation}


pero $A^{''}\left(1\right)=\sum_{n=1}n\left(n-1\right)a_{n}=
\esp\left[U^{2}\right]-\esp\left[U\right]$,
$\esp\left[U\right]=\rho$ y
$\esp\left[U^{2}\right]=\lambda^{2}\esp\left[s^{2}\right]+\rho$.
Por lo tanto
$L=\rho+\frac{\beta^{2}\esp\left[s^{2}\right]}{2\left(1-\rho\right)}$.

De las f\'ormulas de Little, se tiene que
$W=E\left(w\right)=\frac{L}{\beta}$, tambi\'en el tiempo de espera
en la cola
\begin{equation}
W_{q}=\esp\left(q\right)=\esp\left(w\right)-\esp\left(s\right)=\frac{L}{\beta}-\frac{1}{\delta},
\end{equation}
adem\'as el n\'umero promedio de clientes en la cola es
\begin{equation}
L_{q}=\esp\left(N_{q}\right)=\beta W_{q}=L-\rho
\end{equation}

%_____________________________________________________________________________________
%
\section{Redes de Colas}
%_____________________________________________________________________________________

%_____________________________________________________________________________________
%
\subsection{Sistemas Abiertos}
%_____________________________________________________________________________________
%

Considerese un sistema con dos servidores, en los cuales los
usuarios llegan de acuerdo a un proceso poisson con intensidad
$\lambda_{1}$ al primer servidor, despu\'es de ser atendido se pasa
a la siguiente cola en el segundo servidor. Cada servidor atiende
a un usuario a la vez con tiempo exponencial con raz\'on $\mu_{i}$,
para $i=1,2$. A este tipo de sistemas se les conoce como siemas
secuenciales.

Def\'inase el par $\left(n,m\right)$ como el n\'umero de usuarios en
el servidor 1 y 2 respectivamente. Las ecuaciones de balance son
\begin{eqnarray}\label{Eq.Balance}
\lambda P_{0,0}&=&\mu_{2}P_{0,1}\\
\left(\lambda+\mu_{1}\right)P_{n,0}&=&\mu_{2}P_{n,1}+\lambda P_{n-1,0}\\
\left(\lambda+\mu_{2}\right)P_{0,m}&=&\mu_{2}P_{0,m+1}+\mu_{1}P_{1,m-1}\\
\left(\lambda+\mu_{1}+\mu_{2}\right)P_{n,m}&=&\mu_{2}P_{n,m+1}+\mu_{1}P_{n+1,m-1}+\lambda
P_{n-1,m}
\end{eqnarray}

Cada servidor puede ser visto como un modelo de tipo $M/M/1$, de
igual manera el proceso de salida de una cola $M/M/1$ con raz\'on
$\lambda$, nos permite asumir que el servidor 2 tambi\'en es una
cola $M/M/1$. Adem\'as la probabilidad de que haya $n$ usuarios en
el servidor 1 es
\begin{eqnarray*}
P\left\{n\textrm{ en el servidor }1\right\}&=&\left(\frac{\lambda}{\mu_{1}}\right)^{n}\left(1-\frac{\lambda}{\mu_{1}}\right)=\rho_{1}^{n}\left(1-\rho_{1}\right)\\
P\left\{m\textrm{ en el servidor }2\right\}&=&\left(\frac{\lambda}{\mu_{2}}\right)^{n}\left(1-\frac{\lambda}{\mu_{2}}\right)=\rho_{2}^{m}\left(1-\rho_{2}\right)\\
\end{eqnarray*}
Si el n\'umero de usuarios en los servidores 1 y 2 son variables
aleatorias independientes, se sigue que:
\begin{equation}\label{Eq.8.16}
P_{n,m}=\rho_{1}^{n}\left(1-\rho_{1}\right)\rho_{2}^{m}\left(1-\rho_{2}\right)
\end{equation}
Verifiquemos que $P_{n,m}$ satisface las ecuaciones de balance
(\ref{Eq.Balance}) Antes de eso, enunciemos unas igualdades que
nos ser\'an de utilidad:
\begin{eqnarray*}
\mu_{i}\rho_{i}=\lambda\textrm{ para }i=1,2.
\end{eqnarray*}

\begin{eqnarray*}
\lambda P_{0,0}&=&\lambda\left(1-\rho_{1}\right)\left(1-\rho_{2}\right)\\
\textrm{ y }\mu_{2} P_{0,1}&=&\mu_{2}\left(1-\rho_{1}\right)\rho_{2}\left(1-\rho_{2}\right)\\
\Rightarrow\lambda P_{0,0}&=&\mu_{2} P_{0,1}\\
\left(\lambda+\mu_{2}\right)P_{0,m}&=&\left(\lambda+\mu_{2}\right)\left(1-\rho_{1}\right)\rho_{2}^{m}\left(1-\rho_{2}\right)\\
\mu_{2}P_{0,m+1}&=&\lambda\left(1-\rho_{1}\right)\rho_{2}^{m}\left(1-\rho_{2}\right)\\
&=&\mu_{2}\left(1-\rho_{1}\right)\rho_{2}^{m}\left(1-\rho_{2}\right)\\
\mu_{1}P_{1,m-1}&=&\frac{\lambda}{\rho_{2}}\left(1-\rho_{1}\right)\rho_{2}^{m}\left(1-\rho_{2}\right)\\
\Rightarrow\left(\lambda+\mu_{2}\right)P_{0,m}&=&\mu_{2}P_{0,m+1}+\mu_{1}P_{1,m-1}\\
\end{eqnarray*}
%_________________________________________________________
\begin{eqnarray*}
\left(\lambda+\mu_{1}+\mu_{2}\right)P_{n,m}&=&\left(\lambda+\mu_{1}+\mu_{2}\right)\rho^{n}\left(1-\rho_{1}\right)\rho_{2}^{m}\left(1-\rho_{2}\right)\\
\mu_{2}P_{n,m+1}&=&\mu_{2}\rho_{2}\rho_{1}^{n}\left(1-\rho_{1}\right)\rho_{2}^{m}\left(1-\rho_{2}\right)\\
\mu_{1} P_{n-1,m-1}&=&\mu_{1}\frac{\rho_{1}}{\rho_{2}}\rho_{1}^{n}\left(1-\rho_{1}\right)\rho_{2}^{m}\left(1-\rho_{2}\right)\\
\lambda P_{n-1,m}&=&\frac{\lambda}{\rho_{1}}\rho_{1}^{n}\left(1-\rho_{1}\right)\rho_{2}^{m}\left(1-\rho_{2}\right)\\
\Rightarrow\left(\lambda+\mu_{1}+\mu_{2}\right)P_{n,m}&=&\mu_{2}P_{n,m+1}+\mu_{1} P_{n-1,m-1}+\lambda P_{n-1,m}\\
\end{eqnarray*}
entonces efectivamente la ecuaci\'on (\ref{Eq.8.16}) satisface las
ecuaciones de balance (\ref{Eq.Balance}). El n\'umero promedio  de
usuarios en el sistema, est\'a dado por
\begin{eqnarray*}
L&=&\sum_{n,m}\left(n+m\right)P_{n,m}=\sum_{n,m}nP_{n,m}+\sum_{n,m}mP_{n,m}\\
&=&\sum_{n}\sum_{m}nP_{n,m}+\sum_{m}\sum_{n}mP_{n,m}=\sum_{n}n\sum_{m}P_{n,m}+\sum_{m}m\sum_{n}P_{n,m}\\
\end{eqnarray*}

\begin{eqnarray*}
&=&\sum_{n}n\sum_{m}\rho_{1}^{n}\left(1-\rho_{1}\right)\rho_{2}^{m}\left(1-\rho_{2}\right)+\sum_{m}m\sum_{n}\rho_{1}^{n}\left(1-\rho_{1}\right)\rho_{2}^{m}\left(1-\rho_{2}\right)\\
&=&\sum_{n}n\rho_{1}^{n}\left(1-\rho_{1}\right)\sum_{m}\rho_{2}^{m}\left(1-\rho_{2}\right)+\sum_{m}m\rho_{2}^{m}\left(1-\rho_{2}\right)\sum_{n}\rho_{1}^{n}\left(1-\rho_{1}\right)\\
&=&\sum_{n}n\rho_{1}^{n}\left(1-\rho_{1}\right)+\sum_{m}m\rho_{2}^{m}\left(1-\rho_{2}\right)\\
&=&\frac{\lambda}{\mu_{1}-\lambda}+\frac{\lambda}{\mu_{2}-\lambda}
\end{eqnarray*}


%_____________________________________________________________________________________
%
\section{Ejemplo de Cadena de Markov para dos Estados}
%_____________________________________________________________________________________
%

Supongamos que se tiene la siguiente cadena:
\begin{equation}
\left(\begin{array}{cc}
1-q & q\\
p & 1-p\\
\end{array}
\right)
\end{equation}
Si $P\left[X_{0}=0\right]=\pi_{0}(0)=a$ y
$P\left[X_{0}=1\right]=\pi_{0}(1)=b=1-\pi_{0}(0)$, con $a+b=1$,
entonces despu\'es de un procedimiento m\'as o menos corto se tiene
que:

\begin{eqnarray*}
P\left[X_{n}=0\right]=\frac{p}{p+q}+\left(1-p-q\right)^{n}\left(a-\frac{p}{p+q}\right)\\
P\left[X_{n}=1\right]=\frac{q}{p+q}+\left(1-p-q\right)^{n}\left(b-\frac{q}{p+q}\right)\\
\end{eqnarray*}
donde, como $0<p,q<1$, se tiene que $|1-p-q|<1$, entonces
$\left(1-p-q\right)^{n}\rightarrow 0$ cuando $n\rightarrow\infty$.
Por lo tanto
\begin{eqnarray*}
lim_{n\rightarrow\infty}P\left[X_{n}=0\right]=\frac{p}{p+q}\\
lim_{n\rightarrow\infty}P\left[X_{n}=1\right]=\frac{q}{p+q}\\
\end{eqnarray*}
Si hacemos $v=\left(\frac{p}{p+q},\frac{q}{p+q}\right)$, entonces
\begin{eqnarray*}
\left(\frac{p}{p+q},\frac{q}{p+q}\right)\left(\begin{array}{cc}
1-q & q\\
p & 1-p\\
\end{array}\right)
\end{eqnarray*}


%____________________________________________________________________
\section{Cadenas de Markov: Estacionareidad}

\begin{Teo}seserseraeraer

\end{Teo}

%____________________________________________________________________
\section{Teor\'ia Erg\'odica}

\begin{Teo}
Supongamos que $\left\{X_{t}\right\}_{t\geq0}$ es irreducible
recurrente en $E$. Entonces existe una y s\'olo una, salvo
m\'ultiplos, medida estacionaria $\nu$. Esta $\nu$ tiene la
propiedad de que $0\leq\nu_{j}<\infty$ para toda $j$ y puede
encontrarse en las siguentes formas:
\begin{itemize}
\item[i)] Para alg\'un estado fijo pero arbitrario, $i$, $\nu_{j}$
es el tiempo esperado utilizado en $j$ entre dos llegas
consecutivas al estado $i$;
\begin{equation}\label{Eq.4.2}
\nu_{j}=\esp_{i}\int_{0}^{\omega\left(i\right)}\indora\left(X_{t}=j\right)dt,
\end{equation}
con
$\omega\left(i\right)=inf\left\{t>0:X_{t}=i,X_{t^{-}}=\lim_{s\uparrow
t}X_{s}\neq i\right\}$ \item [ii)]
$\nu_{j}=\frac{\mu_{j}}{\lambda\left(j\right)}$, donde $\mu$ es
estacionaria para $\left\{Y_{n}\right\}$; \item[iii)] como
soluci\'on de $\nu\Lambda=0$.
\end{itemize}
\end{Teo}
%_____________________________________________________________________
\section{Queueing Theory at Markovian Level}

\subsection{General Death Birth Processes}


Consideremos un estado que comienza en el estado $x_{0}$ al tiempo
$0$, supongamos que el sistema permanece en $x_{0}$ hasta alg\'un
tiempo positivo $\tau_{1}$, tiempo en el que el sistema salta a un
nuevo estado $x_{1}\neq x_{0}$. Puede ocurrir que el sistema
permanezca en $x_{0}$ de manera indefinida, en este caso hacemos
$\tau_{1}=\infty$. Si $\tau_{1}$ es finito, el sistema
permanecer\'a en $x_{1}$ hasta $\tau_{2}$, y as\'i sucesivamente.
Sea
\begin{equation}
X\left(t\right)=\left\{\begin{array}{cc}
x_{0} & 0\leq t<\tau_{1}\\
x_{1} & \tau_{1}\leq t<\tau_{2}\\
x_{2} & \tau_{2}\leq t<\tau_{3}\\
\vdots &\\
\end{array}\right.
\end{equation}

A este proceso  se le llama {\em proceso de salto}. Si
\begin{equation}
lim_{n\rightarrow\infty}\tau_{n}=\left\{\begin{array}{cc}
<\infty & X_{t}\textrm{ explota}\\
=\infty & X_{t}\textrm{ no explota}\\
\end{array}\right.
\end{equation}

Un proceso puro de saltos es un proceso de saltos que satisface la
propiedad de Markov.

\begin{Prop}
Un proceso de saltos es Markoviano si y s\'olo si todos los
estados no absorbentes $x$ son tales que
\begin{eqnarray*}
P_{x}\left(\tau_{1}>t+s|\tau_{1}>s\right)=P_{x}\left(\tau_{1}>t\right)
\end{eqnarray*}
para $s,t\geq0$, equivalentemente

\begin{equation}\label{Eq.5}
\frac{1-F_{x}\left(t+s\right)}{1-F_{x}\left(s\right)}=1-F_{x}\left(t\right).
\end{equation}

\end{Prop}

\begin{Note}
Una distribuci\'on $F_{x}$ satisface la ecuaci\'on (\ref{Eq.5}) si
y s\'olo si es una funci\'on de distribuci\'on exponencial para
todos los estados no absorbentes $x$.
\end{Note}

Por un proceso de nacimiento y muerte se entiende un proceso de
Markov de Saltos, $\left\{X_{t}\right\}_{t\geq0}$ en $E=\nat$ tal
que del estado $n$ s\'olo se puede mover a $n-1$ o $n+1$, es
decir, la matriz intensidad es de la forma:

\begin{equation}
\Lambda=\left(\begin{array}{ccccc}
-\beta_{0}&\beta_{0} & 0 & 0 & \ldots\\
\delta_{1}&-\beta_{1}-\delta_{1} & \beta_{1}&0&\ldots\\
0&\delta_{2}&-\beta_{2}-\delta_{2} & \beta_{2}&\ldots\\
\vdots & & & \ddots &
\end{array}\right)
\end{equation}

donde $\beta_{n}$ son las probabilidades de nacimiento y
$\delta_{n}$ las probabilidades de muerte.

La matriz de transici\'on es
\begin{equation}
Q=\left(\begin{array}{ccccc}
0& 1 & 0 & 0 & \ldots\\
q_{1}&0 & p_{1}&0&\ldots\\
0&q_{2}&0& p_{2}&\ldots\\
\vdots & & & \ddots &
\end{array}\right)
\end{equation}
con $p_{n}=\frac{\beta_{n}}{\beta_{n}+\delta_{n}}$ y
$q_{n}=\frac{\delta_{n}}{\beta_{n}+\delta_{n}}$

\begin{Prop}
La recurrencia de un Proceso Markoviano de Saltos
$\left\{X_{t}\right\}_{t\geq0}$ con espacio de estados numerable,
o equivalentemente de la cadena encajada $\left\{Y_{n}\right\}$ es
equivalente a
\begin{equation}\label{Eq.2.1}
\sum_{n=1}^{\infty}\frac{\delta_{1}\cdots\delta_{n}}{\beta_{1}\cdots\beta_{n}}=\sum_{n=1}^{\infty}\frac{q_{1}\cdots
q_{n}}{p_{1}\cdots p_{n}}=\infty
\end{equation}
\end{Prop}

\begin{Lem}
Independientemente de la recurrencia o transitoriedad de la
cadena, hay una y s\'olo una, salvo m\'ultiplos, soluci\'on $\nu$
a $\nu\Lambda=0$, dada por
\begin{equation}\label{Eq.2.2}
\nu_{n}=\frac{\beta_{0}\cdots\beta_{n-1}}{\delta_{1}\cdots\delta_{n}}\nu_{0}
\end{equation}
\end{Lem}

\begin{Cor}\label{Corolario2.3}
En el caso recurrente, la medida estacionaria $\mu$ para
$\left\{Y_{n}\right\}$ est\'a dada por
\begin{equation}\label{Eq.2.3}
\mu_{n}=\frac{p_{1}\cdots p_{n-1}}{q_{1}\cdots q_{n}}\mu_{0}
\end{equation}
para $n=1,2,\ldots$
\end{Cor}

\begin{Def}
Una medida $\nu$ es estacionaria si $0\leq\nu_{j}<\infty$ y para
toda $t$ se cumple que $\nu P^{t}=nu$.
\end{Def}


\begin{Def}
Un proceso irreducible recurrente con medida estacionaria con masa
finita es llamado erg\'odico.
\end{Def}

\begin{Teo}\label{Teo4.3}
Un Proceso de Saltos de Markov irreducible no explosivo es
erg\'odico si y s\'olo si uno puede encontrar una soluci\'on
$\pi$ de probabilidad, $|\pi|=1$, $0\leq\pi_{j}\leq1$ para
$\nu\Lambda=0$. En este caso $\pi$ es la distribuci\'on
estacionaria.
\end{Teo}
\begin{Cor}\label{Corolario2.4}
$\left\{X_{t}\right\}_{t\geq0}$ es erg\'odica si y s\'olo si
(\ref{Eq.2.1}) se cumple y $S<\infty$, en cuyo caso la
distribuci\'on estacionaria $\pi$ est\'a dada por

\begin{equation}\label{Eq.2.4}
\pi_{0}=\frac{1}{S}\textrm{,
}\pi_{n}=\frac{1}{S}\frac{\beta_{0}\cdots\beta_{n-1}}{\delta_{1}\cdots\delta_{n}}\textrm{,
}n=1,2,\ldots
\end{equation}
\end{Cor}

\section{Birth-Death Processes as Queueing Models}

\subsection{Cola M/M/1}
\begin{Prop}
La cola M/M/1 con intensidad de tr\'afico $\rho$ es recurrente si
y s\'olo si $\rho\leq1$
\end{Prop}

\begin{Prop}
La cola M/M/1 con intensidad de tr\'afica $\rho$ es ergodica si y
s\'olo si $\rho<1$. En este caso, la distribuci\'on de equilibrio
$\pi$ de la longitud de la cola es geom\'etrica,
$\pi_{n}=\left(1-\rho\right)\rho^{n}$, para $n=0,1,2,\ldots$.
\end{Prop}



%____________________________________________________________________________

\subsection{Cola con Infinidad de Servidores}

Este caso corresponde a $\beta_{n}=\beta$ y $\delta_{n}=n\delta$.
El par\'ametro de inter\'es es $\eta=\frac{\beta}{\delta}$, de
donde se obtiene:
\begin{eqnarray*}
\sum_{n\geq0}\frac{\delta_{1}\cdots\delta_{n}}{\beta_{1}\cdots\beta_{n}}=\sum_{n=1}^{\infty}n!\eta^{n}=\infty,\\
S=1+\sum_{n=1}^{\infty}\frac{\eta^{n}}{n!}=e^{n}.
\end{eqnarray*}
\begin{Prop}
La cola $M/M/\infty$ es ergodica para todos los valores de $\eta$.
La distribuci\'on de equilibrio $\pi$ es Poisson con media $\eta$,
$\pi_{n}=\frac{e^{-n}\eta}{n!}$
\end{Prop}
\subsection{Cola M/M/m}

En este caso $\beta_{n}=\beta$ y
$\delta_{n}=m\left(n\right)\delta$, donde $m\left(n\right)=n$,
$1\leq n\leq m$. La intensidad de tr\'afico es
$\rho=\frac{\beta}{m\delta}$, se tiene entonces que
$\beta_{n}/\delta_{n}=\rho$ para $n\geq m$. As\'i, para el caso
$m=1$,
%___________________________________________________________________________________________
%
\section{Procesos Regenerativos}
%_____________________________________________________

Si $x$ es el n{\'u}mero de usuarios en la cola al comienzo del
periodo de servicio y $N_{s}\left(x\right)=N\left(x\right)$ es el
n{\'u}mero de usuarios que son atendidos con la pol{\'\i}tica $s$,
{\'u}nica en nuestro caso, durante un periodo de servicio,
entonces se asume que:
\begin{itemize}
\item[(S1.)]
\begin{equation}\label{S1}
lim_{x\rightarrow\infty}\esp\left[N\left(x\right)\right]=\overline{N}>0.
\end{equation}
\item[(S2.)]
\begin{equation}\label{S2}
\esp\left[N\left(x\right)\right]\leq \overline{N}, \end{equation}
para cualquier valor de $x$. \item La $n$-{\'e}sima ocurrencia va
acompa{\~n}ada con el tiempo de cambio de longitud
$\delta_{j,j+1}\left(n\right)$, independientes e id{\'e}nticamente
distribuidas, con
$\esp\left[\delta_{j,j+1}\left(1\right)\right]\geq0$. \item Se
define
\begin{equation}
\delta^{*}:=\sum_{j,j+1}\esp\left[\delta_{j,j+1}\left(1\right)\right].
\end{equation}

\item Los tiempos de inter-arribo a la cola $k$,son de la forma
$\left\{\xi_{k}\left(n\right)\right\}_{n\geq1}$, con la propiedad
de que son independientes e id{\'e}nticamente distribuidos.

\item Los tiempos de servicio
$\left\{\eta_{k}\left(n\right)\right\}_{n\geq1}$ tienen la
propiedad de ser independientes e id{\'e}nticamente distribuidos.

\item Se define la tasa de arribo a la $k$-{\'e}sima cola como
$\lambda_{k}=1/\esp\left[\xi_{k}\left(1\right)\right]$ y
adem{\'a}s se define

\item la tasa de servicio para la $k$-{\'e}sima cola como
$\mu_{k}=1/\esp\left[\eta_{k}\left(1\right)\right]$

\item tambi{\'e}n se define $\rho_{k}=\lambda_{k}/\mu_{k}$, donde
es necesario que $\rho<1$ para cuestiones de estabilidad.

\item De las pol{\'\i}ticas posibles solamente consideraremos la
pol{\'\i}tica cerrada (Gated).
\end{itemize}

Las Colas C\'iclicas se pueden describir por medio de un proceso
de Markov $\left(X\left(t\right)\right)_{t\in\rea}$, donde el
estado del sistema al tiempo $t\geq0$ est\'a dado por
\begin{equation}
X\left(t\right)=\left(Q\left(t\right),A\left(t\right),H\left(t\right),B\left(t\right),B^{0}\left(t\right),C\left(t\right)\right)
\end{equation}
definido en el espacio producto:
\begin{equation}
\mathcal{X}=\mathbb{Z}^{K}\times\rea_{+}^{K}\times\left(\left\{1,2,\ldots,K\right\}\times\left\{1,2,\ldots,S\right\}\right)^{M}\times\rea_{+}^{K}\times\rea_{+}^{K}\times\mathbb{Z}^{K},
\end{equation}

\begin{itemize}
\item $Q\left(t\right)=\left(Q_{k}\left(t\right),1\leq k\leq
K\right)$, es el n\'umero de usuarios en la cola $k$, incluyendo
aquellos que est\'an siendo atendidos provenientes de la
$k$-\'esima cola.

\item $A\left(t\right)=\left(A_{k}\left(t\right),1\leq k\leq
K\right)$, son los residuales de los tiempos de arribo en la cola
$k$. \item $H\left(t\right)$ es el par ordenado que consiste en la
cola que esta siendo atendida y la pol\'itica de servicio que se
utilizar\'a.

\item $B\left(t\right)$ es el tiempo de servicio residual.

\item $B^{0}\left(t\right)$ es el tiempo residual del cambio de
cola.

\item $C\left(t\right)$ indica el n\'umero de usuarios atendidos
durante la visita del servidor a la cola dada en
$H\left(t\right)$.
\end{itemize}

$A_{k}\left(t\right),B_{m}\left(t\right)$ y
$B_{m}^{0}\left(t\right)$ se suponen continuas por la derecha y
que satisfacen la propiedad fuerte de Markov, (\cite{Dai})

\begin{itemize}
\item Los tiempos de interarribo a la cola $k$,son de la forma
$\left\{\xi_{k}\left(n\right)\right\}_{n\geq1}$, con la propiedad
de que son independientes e id{\'e}nticamente distribuidos.

\item Los tiempos de servicio
$\left\{\eta_{k}\left(n\right)\right\}_{n\geq1}$ tienen la
propiedad de ser independientes e id{\'e}nticamente distribuidos.

\item Se define la tasa de arribo a la $k$-{\'e}sima cola como
$\lambda_{k}=1/\esp\left[\xi_{k}\left(1\right)\right]$ y
adem{\'a}s se define

\item la tasa de servicio para la $k$-{\'e}sima cola como
$\mu_{k}=1/\esp\left[\eta_{k}\left(1\right)\right]$

\item tambi{\'e}n se define $\rho_{k}=\lambda_{k}/\mu_{k}$, donde
es necesario que $\rho<1$ para cuestiones de estabilidad.

\item De las pol{\'\i}ticas posibles solamente consideraremos la
pol{\'\i}tica cerrada (Gated).
\end{itemize}

%\section{Preliminares}



Sup\'ongase que el sistema consta de varias colas a los cuales
llegan uno o varios servidores a dar servicio a los usuarios
esperando en la cola.\\


Si $x$ es el n\'umero de usuarios en la cola al comienzo del
periodo de servicio y $N_{s}\left(x\right)=N\left(x\right)$ es el
n\'umero de usuarios que son atendidos con la pol\'itica $s$,
\'unica en nuestro caso, durante un periodo de servicio, entonces
se asume que:
\begin{itemize}
\item[1)]\label{S1}$lim_{x\rightarrow\infty}\esp\left[N\left(x\right)\right]=\overline{N}>0$
\item[2)]\label{S2}$\esp\left[N\left(x\right)\right]\leq\overline{N}$para
cualquier valor de $x$.
\end{itemize}
La manera en que atiende el servidor $m$-\'esimo, en este caso en
espec\'ifico solo lo ilustraremos con un s\'olo servidor, es la
siguiente:
\begin{itemize}
\item Al t\'ermino de la visita a la cola $j$, el servidor se
cambia a la cola $j^{'}$ con probabilidad
$r_{j,j^{'}}^{m}=r_{j,j^{'}}$

\item La $n$-\'esima ocurrencia va acompa\~nada con el tiempo de
cambio de longitud $\delta_{j,j^{'}}\left(n\right)$,
independientes e id\'enticamente distribuidas, con
$\esp\left[\delta_{j,j^{'}}\left(1\right)\right]\geq0$.

\item Sea $\left\{p_{j}\right\}$ la distribuci\'on invariante
estacionaria \'unica para la Cadena de Markov con matriz de
transici\'on $\left(r_{j,j^{'}}\right)$.

\item Finalmente, se define
\begin{equation}
\delta^{*}:=\sum_{j,j^{'}}p_{j}r_{j,j^{'}}\esp\left[\delta_{j,j^{'}}\left(i\right)\right].
\end{equation}
\end{itemize}

Veamos un caso muy espec\'ifico en el cual los tiempos de arribo a cada una de las colas se comportan de acuerdo a un proceso Poisson de la forma
$\left\{\xi_{k}\left(n\right)\right\}_{n\geq1}$, y los tiempos de servicio en cada una de las colas son variables aleatorias distribuidas exponencialmente e id\'enticamente distribuidas
$\left\{\eta_{k}\left(n\right)\right\}_{n\geq1}$, donde ambos procesos adem\'as cumplen la condici\'on de ser independientes entre si. Para la $k$-\'esima cola se define la tasa de arribo a la como
$\lambda_{k}=1/\esp\left[\xi_{k}\left(1\right)\right]$ y la tasa
de servicio como
$\mu_{k}=1/\esp\left[\eta_{k}\left(1\right)\right]$, finalmente se
define la carga de la cola como $\rho_{k}=\lambda_{k}/\mu_{k}$,
donde se pide que $\rho<1$, para garantizar la estabilidad del sistema.\\

Se denotar\'a por $Q_{k}\left(t\right)$ el n\'umero de usuarios en la cola $k$,
$A_{k}\left(t\right)$ los residuales de los tiempos entre arribos a la cola $k$;
para cada servidor $m$, se denota por $B_{m}\left(t\right)$ los residuales de los tiempos de servicio al tiempo $t$; $B_{m}^{0}\left(t\right)$ son los residuales de los tiempos de traslado de la cola $k$ a la pr\'oxima por atender, al tiempo $t$, finalmente sea $C_{m}\left(t\right)$ el n\'umero de usuarios atendidos durante la visita del servidor a la cola $k$ al tiempo $t$.\\


En este sentido el proceso para el sistema de visitas se puede definir como:

\begin{equation}\label{Esp.Edos.Down}
X\left(t\right)^{T}=\left(Q_{k}\left(t\right),A_{k}\left(t\right),B_{m}\left(t\right),B_{m}^{0}\left(t\right),C_{m}\left(t\right)\right)
\end{equation}
para $k=1,\ldots,K$ y $m=1,2,\ldots,M$. $X$ evoluciona en el
espacio de estados:
$X=\ent_{+}^{K}\times\rea_{+}^{K}\times\left(\left\{1,2,\ldots,K\right\}\times\left\{1,2,\ldots,S\right\}\right)^{M}\times\rea_{+}^{K}\times\ent_{+}^{K}$.\\

El sistema aqu\'i descrito debe de cumplir con los siguientes supuestos b\'asicos de un sistema de visitas:

Antes enunciemos los supuestos que regir\'an en la red.

\begin{itemize}
\item[A1)] $\xi_{1},\ldots,\xi_{K},\eta_{1},\ldots,\eta_{K}$ son
mutuamente independientes y son sucesiones independientes e
id\'enticamente distribuidas.

\item[A2)] Para alg\'un entero $p\geq1$
\begin{eqnarray*}
\esp\left[\xi_{l}\left(1\right)^{p+1}\right]<\infty\textrm{ para }l\in\mathcal{A}\textrm{ y }\\
\esp\left[\eta_{k}\left(1\right)^{p+1}\right]<\infty\textrm{ para
}k=1,\ldots,K.
\end{eqnarray*}
donde $\mathcal{A}$ es la clase de posibles arribos.

\item[A3)] Para $k=1,2,\ldots,K$ existe una funci\'on positiva
$q_{k}\left(x\right)$ definida en $\rea_{+}$, y un entero $j_{k}$,
tal que
\begin{eqnarray}
P\left(\xi_{k}\left(1\right)\geq x\right)>0\textrm{, para todo }x>0\\
P\left\{a\leq\sum_{i=1}^{j_{k}}\xi_{k}\left(i\right)\leq
b\right\}\geq\int_{a}^{b}q_{k}\left(x\right)dx, \textrm{ }0\leq
a<b.
\end{eqnarray}
\end{itemize}

En particular los procesos de tiempo entre arribos y de servicio
considerados con fines de ilustraci\'on de la metodolog\'ia
cumplen con el supuesto $A2)$ para $p=1$, es decir, ambos procesos
tienen primer y segundo momento finito.

En lo que respecta al supuesto (A3), en Dai y Meyn \cite{DaiSean}
hacen ver que este se puede sustituir por

\begin{itemize}
\item[A3')] Para el Proceso de Markov $X$, cada subconjunto
compacto de $X$ es un conjunto peque\~no, ver definici\'on
\ref{Def.Cto.Peq.}.
\end{itemize}

Es por esta raz\'on que con la finalidad de poder hacer uso de
$A3^{'})$ es necesario recurrir a los Procesos de Harris y en
particular a los Procesos Harris Recurrente:
%_______________________________________________________________________
\subsection{Procesos Harris Recurrente}
%_______________________________________________________________________

Por el supuesto (A1) conforme a Davis \cite{Davis}, se puede
definir el proceso de saltos correspondiente de manera tal que
satisfaga el supuesto (\ref{Sup3.1.Davis}), de hecho la
demostraci\'on est\'a basada en la l\'inea de argumentaci\'on de
Davis, (\cite{Davis}, p\'aginas 362-364).

Entonces se tiene un espacio de estados Markoviano. El espacio de
Markov descrito en Dai y Meyn \cite{DaiSean}

\[\left(\Omega,\mathcal{F},\mathcal{F}_{t},X\left(t\right),\theta_{t},P_{x}\right)\]
es un proceso de Borel Derecho (Sharpe \cite{Sharpe}) en el
espacio de estados medible $\left(X,\mathcal{B}_{X}\right)$. El
Proceso $X=\left\{X\left(t\right),t\geq0\right\}$ tiene
trayectorias continuas por la derecha, est\'a definida en
$\left(\Omega,\mathcal{F}\right)$ y est\'a adaptado a
$\left\{\mathcal{F}_{t},t\geq0\right\}$; la colecci\'on
$\left\{P_{x},x\in \mathbb{X}\right\}$ son medidas de probabilidad
en $\left(\Omega,\mathcal{F}\right)$ tales que para todo $x\in
\mathbb{X}$
\[P_{x}\left\{X\left(0\right)=x\right\}=1\] y
\[E_{x}\left\{f\left(X\circ\theta_{t}\right)|\mathcal{F}_{t}\right\}=E_{X}\left(\tau\right)f\left(X\right)\]
en $\left\{\tau<\infty\right\}$, $P_{x}$-c.s. Donde $\tau$ es un
$\mathcal{F}_{t}$-tiempo de paro
\[\left(X\circ\theta_{\tau}\right)\left(w\right)=\left\{X\left(\tau\left(w\right)+t,w\right),t\geq0\right\}\]
y $f$ es una funci\'on de valores reales acotada y medible con la
$\sigma$-algebra de Kolmogorov generada por los cilindros.\\

Sea $P^{t}\left(x,D\right)$, $D\in\mathcal{B}_{\mathbb{X}}$,
$t\geq0$ probabilidad de transici\'on de $X$ definida como
\[P^{t}\left(x,D\right)=P_{x}\left(X\left(t\right)\in
D\right)\]


\begin{Def}
Una medida no cero $\pi$ en
$\left(\mathbf{X},\mathcal{B}_{\mathbf{X}}\right)$ es {\bf
invariante} para $X$ si $\pi$ es $\sigma$-finita y
\[\pi\left(D\right)=\int_{\mathbf{X}}P^{t}\left(x,D\right)\pi\left(dx\right)\]
para todo $D\in \mathcal{B}_{\mathbf{X}}$, con $t\geq0$.
\end{Def}

\begin{Def}
El proceso de Markov $X$ es llamado Harris recurrente si existe
una medida de probabilidad $\nu$ en
$\left(\mathbf{X},\mathcal{B}_{\mathbf{X}}\right)$, tal que si
$\nu\left(D\right)>0$ y $D\in\mathcal{B}_{\mathbf{X}}$
\[P_{x}\left\{\tau_{D}<\infty\right\}\equiv1\] cuando
$\tau_{D}=inf\left\{t\geq0:X_{t}\in D\right\}$.
\end{Def}

\begin{Note}
\begin{itemize}
\item[i)] Si $X$ es Harris recurrente, entonces existe una \'unica
medida invariante $\pi$ (Getoor \cite{Getoor}).

\item[ii)] Si la medida invariante es finita, entonces puede
normalizarse a una medida de probabilidad, en este caso se le
llama Proceso {\em Harris recurrente positivo}.


\item[iii)] Cuando $X$ es Harris recurrente positivo se dice que
la disciplina de servicio es estable. En este caso $\pi$ denota la
distribuci\'on estacionaria y hacemos
\[P_{\pi}\left(\cdot\right)=\int_{\mathbf{X}}P_{x}\left(\cdot\right)\pi\left(dx\right)\]
y se utiliza $E_{\pi}$ para denotar el operador esperanza
correspondiente.
\end{itemize}
\end{Note}

\begin{Def}\label{Def.Cto.Peq.}
Un conjunto $D\in\mathcal{B_{\mathbf{X}}}$ es llamado peque\~no si
existe un $t>0$, una medida de probabilidad $\nu$ en
$\mathcal{B_{\mathbf{X}}}$, y un $\delta>0$ tal que
\[P^{t}\left(x,A\right)\geq\delta\nu\left(A\right)\] para $x\in
D,A\in\mathcal{B_{X}}$.
\end{Def}

La siguiente serie de resultados vienen enunciados y demostrados
en Dai \cite{Dai}:
\begin{Lema}[Lema 3.1, Dai\cite{Dai}]
Sea $B$ conjunto peque\~no cerrado, supongamos que
$P_{x}\left(\tau_{B}<\infty\right)\equiv1$ y que para alg\'un
$\delta>0$ se cumple que
\begin{equation}\label{Eq.3.1}
\sup\esp_{x}\left[\tau_{B}\left(\delta\right)\right]<\infty,
\end{equation}
donde
$\tau_{B}\left(\delta\right)=inf\left\{t\geq\delta:X\left(t\right)\in
B\right\}$. Entonces, $X$ es un proceso Harris Recurrente
Positivo.
\end{Lema}

\begin{Lema}[Lema 3.1, Dai \cite{Dai}]\label{Lema.3.}
Bajo el supuesto (A3), el conjunto $B=\left\{|x|\leq k\right\}$ es
un conjunto peque\~no cerrado para cualquier $k>0$.
\end{Lema}

\begin{Teo}[Teorema 3.1, Dai\cite{Dai}]\label{Tma.3.1}
Si existe un $\delta>0$ tal que
\begin{equation}
lim_{|x|\rightarrow\infty}\frac{1}{|x|}\esp|X^{x}\left(|x|\delta\right)|=0,
\end{equation}
entonces la ecuaci\'on (\ref{Eq.3.1}) se cumple para
$B=\left\{|x|\leq k\right\}$ con alg\'un $k>0$. En particular, $X$
es Harris Recurrente Positivo.
\end{Teo}

\begin{Note}
En Meyn and Tweedie \cite{MeynTweedie} muestran que si
$P_{x}\left\{\tau_{D}<\infty\right\}\equiv1$ incluso para solo un
conjunto peque\~no, entonces el proceso es Harris Recurrente.
\end{Note}

Entonces, tenemos que el proceso $X$ es un proceso de Markov que
cumple con los supuestos $A1)$-$A3)$, lo que falta de hacer es
construir el Modelo de Flujo bas\'andonos en lo hasta ahora
presentado.
%_______________________________________________________________________
\subsection{Modelo de Flujo}
%_______________________________________________________________________

Dada una condici\'on inicial $x\in\textrm{X}$, sea
$Q_{k}^{x}\left(t\right)$ la longitud de la cola al tiempo $t$,
$T_{m,k}^{x}\left(t\right)$ el tiempo acumulado, al tiempo $t$,
que tarda el servidor $m$ en atender a los usuarios de la cola
$k$. Finalmente sea $T_{m,k}^{x,0}\left(t\right)$ el tiempo
acumulado, al tiempo $t$, que tarda el servidor $m$ en trasladarse
a otra cola a partir de la $k$-\'esima.\\

Sup\'ongase que la funci\'on
$\left(\overline{Q}\left(\cdot\right),\overline{T}_{m}
\left(\cdot\right),\overline{T}_{m}^{0} \left(\cdot\right)\right)$
para $m=1,2,\ldots,M$ es un punto l\'imite de
\begin{equation}\label{Eq.Punto.Limite}
\left(\frac{1}{|x|}Q^{x}\left(|x|t\right),\frac{1}{|x|}T_{m}^{x}\left(|x|t\right),\frac{1}{|x|}T_{m}^{x,0}\left(|x|t\right)\right)
\end{equation}
para $m=1,2,\ldots,M$, cuando $x\rightarrow\infty$. Entonces
$\left(\overline{Q}\left(t\right),\overline{T}_{m}
\left(t\right),\overline{T}_{m}^{0} \left(t\right)\right)$ es un
flujo l\'imite del sistema. Al conjunto de todos las posibles
flujos l\'imite se le llama \textbf{Modelo de Flujo}.\\

El modelo de flujo satisface el siguiente conjunto de ecuaciones:

\begin{equation}\label{Eq.MF.1}
\overline{Q}_{k}\left(t\right)=\overline{Q}_{k}\left(0\right)+\lambda_{k}t-\sum_{m=1}^{M}\mu_{k}\overline{T}_{m,k}\left(t\right)\\
\end{equation}
para $k=1,2,\ldots,K$.\\
\begin{equation}\label{Eq.MF.2}
\overline{Q}_{k}\left(t\right)\geq0\textrm{ para
}k=1,2,\ldots,K,\\
\end{equation}

\begin{equation}\label{Eq.MF.3}
\overline{T}_{m,k}\left(0\right)=0,\textrm{ y }\overline{T}_{m,k}\left(\cdot\right)\textrm{ es no decreciente},\\
\end{equation}
para $k=1,2,\ldots,K$ y $m=1,2,\ldots,M$,\\
\begin{equation}\label{Eq.MF.4}
\sum_{k=1}^{K}\overline{T}_{m,k}^{0}\left(t\right)+\overline{T}_{m,k}\left(t\right)=t\textrm{
para }m=1,2,\ldots,M.\\
\end{equation}

De acuerdo a Dai \cite{Dai}, se tiene que el conjunto de posibles
l\'imites
$\left(\overline{Q}\left(\cdot\right),\overline{T}\left(\cdot\right),\overline{T}^{0}\left(\cdot\right)\right)$,
en el sentido de que deben de satisfacer las ecuaciones
(\ref{Eq.MF.1})-(\ref{Eq.MF.4}), se le llama {\em Modelo de
Flujo}.


\begin{Def}[Definici\'on 4.1, , Dai \cite{Dai}]\label{Def.Modelo.Flujo}
Sea una disciplina de servicio espec\'ifica. Cualquier l\'imite
$\left(\overline{Q}\left(\cdot\right),\overline{T}\left(\cdot\right)\right)$
en (\ref{Eq.Punto.Limite}) es un {\em flujo l\'imite} de la
disciplina. Cualquier soluci\'on (\ref{Eq.MF.1})-(\ref{Eq.MF.4})
es llamado flujo soluci\'on de la disciplina. Se dice que el
modelo de flujo l\'imite, modelo de flujo, de la disciplina de la
cola es estable si existe una constante $\delta>0$ que depende de
$\mu,\lambda$ y $P$ solamente, tal que cualquier flujo l\'imite
con
$|\overline{Q}\left(0\right)|+|\overline{U}|+|\overline{V}|=1$, se
tiene que $\overline{Q}\left(\cdot+\delta\right)\equiv0$.
\end{Def}

Al conjunto de ecuaciones dadas en \ref{Eq.MF.1}-\ref{Eq.MF.4} se
le llama {\em Modelo de flujo} y al conjunto de todas las
soluciones del modelo de flujo
$\left(\overline{Q}\left(\cdot\right),\overline{T}
\left(\cdot\right)\right)$ se le denotar\'a por $\mathcal{Q}$.

Si se hace $|x|\rightarrow\infty$ sin restringir ninguna de las
componentes, tambi\'en se obtienen un modelo de flujo, pero en
este caso el residual de los procesos de arribo y servicio
introducen un retraso:
\begin{Teo}[Teorema 4.2, Dai\cite{Dai}]\label{Tma.4.2.Dai}
Sea una disciplina fija para la cola, suponga que se cumplen las
condiciones (A1))-(A3)). Si el modelo de flujo l\'imite de la
disciplina de la cola es estable, entonces la cadena de Markov $X$
que describe la din\'amica de la red bajo la disciplina es Harris
recurrente positiva.
\end{Teo}

Ahora se procede a escalar el espacio y el tiempo para reducir la
aparente fluctuaci\'on del modelo. Consid\'erese el proceso
\begin{equation}\label{Eq.3.7}
\overline{Q}^{x}\left(t\right)=\frac{1}{|x|}Q^{x}\left(|x|t\right)
\end{equation}
A este proceso se le conoce como el fluido escalado, y cualquier
l\'imite $\overline{Q}^{x}\left(t\right)$ es llamado flujo
l\'imite del proceso de longitud de la cola. Haciendo
$|q|\rightarrow\infty$ mientras se mantiene el resto de las
componentes fijas, cualquier punto l\'imite del proceso de
longitud de la cola normalizado $\overline{Q}^{x}$ es soluci\'on
del siguiente modelo de flujo.


\begin{Def}[Definici\'on 3.3, Dai y Meyn \cite{DaiSean}]
El modelo de flujo es estable si existe un tiempo fijo $t_{0}$ tal
que $\overline{Q}\left(t\right)=0$, con $t\geq t_{0}$, para
cualquier $\overline{Q}\left(\cdot\right)\in\mathcal{Q}$ que
cumple con $|\overline{Q}\left(0\right)|=1$.
\end{Def}

El siguiente resultado se encuentra en Chen \cite{Chen}.
\begin{Lemma}[Lema 3.1, Dai y Meyn \cite{DaiSean}]
Si el modelo de flujo definido por \ref{Eq.MF.1}-\ref{Eq.MF.4} es
estable, entonces el modelo de flujo retrasado es tambi\'en
estable, es decir, existe $t_{0}>0$ tal que
$\overline{Q}\left(t\right)=0$ para cualquier $t\geq t_{0}$, para
cualquier soluci\'on del modelo de flujo retrasado cuya
condici\'on inicial $\overline{x}$ satisface que
$|\overline{x}|=|\overline{Q}\left(0\right)|+|\overline{A}\left(0\right)|+|\overline{B}\left(0\right)|\leq1$.
\end{Lemma}


Ahora ya estamos en condiciones de enunciar los resultados principales:


\begin{Teo}[Teorema 2.1, Down \cite{Down}]\label{Tma2.1.Down}
Suponga que el modelo de flujo es estable, y que se cumplen los supuestos (A1) y (A2), entonces
\begin{itemize}
\item[i)] Para alguna constante $\kappa_{p}$, y para cada
condici\'on inicial $x\in X$
\begin{equation}\label{Estability.Eq1}
limsup_{t\rightarrow\infty}\frac{1}{t}\int_{0}^{t}\esp_{x}\left[|Q\left(s\right)|^{p}\right]ds\leq\kappa_{p},
\end{equation}
donde $p$ es el entero dado en (A2).
\end{itemize}
Si adem\'as se cumple la condici\'on (A3), entonces para cada
condici\'on inicial:
\begin{itemize}
\item[ii)] Los momentos transitorios convergen a su estado
estacionario:
 \begin{equation}\label{Estability.Eq2}
lim_{t\rightarrow\infty}\esp_{x}\left[Q_{k}\left(t\right)^{r}\right]=\esp_{\pi}\left[Q_{k}\left(0\right)^{r}\right]\leq\kappa_{r},
\end{equation}
para $r=1,2,\ldots,p$ y $k=1,2,\ldots,K$. Donde $\pi$ es la
probabilidad invariante para $\mathbf{X}$.

\item[iii)]  El primer momento converge con raz\'on $t^{p-1}$:
\begin{equation}\label{Estability.Eq3}
lim_{t\rightarrow\infty}t^{p-1}|\esp_{x}\left[Q_{k}\left(t\right)\right]-\esp_{\pi}\left[Q_{k}\left(0\right)\right]=0.
\end{equation}

\item[iv)] La {\em Ley Fuerte de los grandes n\'umeros} se cumple:
\begin{equation}\label{Estability.Eq4}
lim_{t\rightarrow\infty}\frac{1}{t}\int_{0}^{t}Q_{k}^{r}\left(s\right)ds=\esp_{\pi}\left[Q_{k}\left(0\right)^{r}\right],\textrm{
}\prob_{x}\textrm{-c.s.}
\end{equation}
para $r=1,2,\ldots,p$ y $k=1,2,\ldots,K$.
\end{itemize}
\end{Teo}

La contribuci\'on de Down a la teor\'ia de los Sistemas de Visitas
C\'iclicas, es la relaci\'on que hay entre la estabilidad del
sistema con el comportamiento de las medidas de desempe\~no, es
decir, la condici\'on suficiente para poder garantizar la
convergencia del proceso de la longitud de la cola as\'i como de
por los menos los dos primeros momentos adem\'as de una versi\'on
de la Ley Fuerte de los Grandes N\'umeros para los sistemas de
visitas.


\begin{Teo}[Teorema 2.3, Down \cite{Down}]\label{Tma2.3.Down}
Considere el siguiente valor:
\begin{equation}\label{Eq.Rho.1serv}
\rho=\sum_{k=1}^{K}\rho_{k}+max_{1\leq j\leq K}\left(\frac{\lambda_{j}}{\sum_{s=1}^{S}p_{js}\overline{N}_{s}}\right)\delta^{*}
\end{equation}
\begin{itemize}
\item[i)] Si $\rho<1$ entonces la red es estable, es decir, se cumple el teorema \ref{Tma2.1.Down}.

\item[ii)] Si $\rho<1$ entonces la red es inestable, es decir, se cumple el teorema \ref{Tma2.2.Down}
\end{itemize}
\end{Teo}

\begin{Teo}
Sea $\left(X_{n},\mathcal{F}_{n},n=0,1,\ldots,\right\}$ Proceso de
Markov con espacio de estados $\left(S_{0},\chi_{0}\right)$
generado por una distribuici\'on inicial $P_{o}$ y probabilidad de
transici\'on $p_{mn}$, para $m,n=0,1,\ldots,$ $m<n$, que por
notaci\'on se escribir\'a como $p\left(m,n,x,B\right)\rightarrow
p_{mn}\left(x,B\right)$. Sea $S$ tiempo de paro relativo a la
$\sigma$-\'algebra $\mathcal{F}_{n}$. Sea $T$ funci\'on medible,
$T:\Omega\rightarrow\left\{0,1,\ldots,\right\}$. Sup\'ongase que
$T\geq S$, entonces $T$ es tiempo de paro. Si $B\in\chi_{0}$,
entonces
\begin{equation}\label{Prop.Fuerte.Markov}
P\left\{X\left(T\right)\in
B,T<\infty|\mathcal{F}\left(S\right)\right\} =
p\left(S,T,X\left(s\right),B\right)
\end{equation}
en $\left\{T<\infty\right\}$.
\end{Teo}


Sea $K$ conjunto numerable y sea $d:K\rightarrow\nat$ funci\'on.
Para $v\in K$, $M_{v}$ es un conjunto abierto de
$\rea^{d\left(v\right)}$. Entonces \[E=\cup_{v\in
K}M_{v}=\left\{\left(v,\zeta\right):v\in K,\zeta\in
M_{v}\right\}.\]

Sea $\mathcal{E}$ la clase de conjuntos medibles en $E$:
\[\mathcal{E}=\left\{\cup_{v\in K}A_{v}:A_{v}\in \mathcal{M}_{v}\right\}.\]

donde $\mathcal{M}$ son los conjuntos de Borel de $M_{v}$.
Entonces $\left(E,\mathcal{E}\right)$ es un espacio de Borel. El
estado del proceso se denotar\'a por
$\mathbf{x}_{t}=\left(v_{t},\zeta_{t}\right)$. La distribuci\'on
de $\left(\mathbf{x}_{t}\right)$ est\'a determinada por por los
siguientes objetos:

\begin{itemize}
\item[i)] Los campos vectoriales $\left(\mathcal{H}_{v},v\in
K\right)$. \item[ii)] Una funci\'on medible $\lambda:E\rightarrow
\rea_{+}$. \item[iii)] Una medida de transici\'on
$Q:\mathcal{E}\times\left(E\cup\Gamma^{*}\right)\rightarrow\left[0,1\right]$
donde
\begin{equation}
\Gamma^{*}=\cup_{v\in K}\partial^{*}M_{v}.
\end{equation}
y
\begin{equation}
\partial^{*}M_{v}=\left\{z\in\partial M_{v}:\mathbf{\mathbf{\phi}_{v}\left(t,\zeta\right)=\mathbf{z}}\textrm{ para alguna }\left(t,\zeta\right)\in\rea_{+}\times M_{v}\right\}.
\end{equation}
$\partial M_{v}$ denota  la frontera de $M_{v}$.
\end{itemize}

El campo vectorial $\left(\mathcal{H}_{v},v\in K\right)$ se supone
tal que para cada $\mathbf{z}\in M_{v}$ existe una \'unica curva
integral $\mathbf{\phi}_{v}\left(t,\zeta\right)$ que satisface la
ecuaci\'on

\begin{equation}
\frac{d}{dt}f\left(\zeta_{t}\right)=\mathcal{H}f\left(\zeta_{t}\right),
\end{equation}
con $\zeta_{0}=\mathbf{z}$, para cualquier funci\'on suave
$f:\rea^{d}\rightarrow\rea$ y $\mathcal{H}$ denota el operador
diferencial de primer orden, con $\mathcal{H}=\mathcal{H}_{v}$ y
$\zeta_{t}=\mathbf{\phi}\left(t,\mathbf{z}\right)$. Adem\'as se
supone que $\mathcal{H}_{v}$ es conservativo, es decir, las curvas
integrales est\'an definidas para todo $t>0$.

Para $\mathbf{x}=\left(v,\zeta\right)\in E$ se denota
\[t^{*}\mathbf{x}=inf\left\{t>0:\mathbf{\phi}_{v}\left(t,\zeta\right)\in\partial^{*}M_{v}\right\}\]

En lo que respecta a la funci\'on $\lambda$, se supondr\'a que
para cada $\left(v,\zeta\right)\in E$ existe un $\epsilon>0$ tal
que la funci\'on
$s\rightarrow\lambda\left(v,\phi_{v}\left(s,\zeta\right)\right)\in
E$ es integrable para $s\in\left[0,\epsilon\right)$. La medida de
transici\'on $Q\left(A;\mathbf{x}\right)$ es una funci\'on medible
de $\mathbf{x}$ para cada $A\in\mathcal{E}$, definida para
$\mathbf{x}\in E\cup\Gamma^{*}$ y es una medida de probabilidad en
$\left(E,\mathcal{E}\right)$ para cada $\mathbf{x}\in E$.

El movimiento del proceso $\left(\mathbf{x}_{t}\right)$ comenzando
en $\mathbf{x}=\left(n,\mathbf{z}\right)\in E$ se puede construir
de la siguiente manera, def\'inase la funci\'on $F$ por

\begin{equation}
F\left(t\right)=\left\{\begin{array}{ll}\\
exp\left(-\int_{0}^{t}\lambda\left(n,\phi_{n}\left(s,\mathbf{z}\right)\right)ds\right), & t<t^{*}\left(\mathbf{x}\right),\\
0, & t\geq t^{*}\left(\mathbf{x}\right)
\end{array}\right.
\end{equation}

Sea $T_{1}$ una variable aleatoria tal que
$\prob\left[T_{1}>t\right]=F\left(t\right)$, ahora sea la variable
aleatoria $\left(N,Z\right)$ con distribuici\'on
$Q\left(\cdot;\phi_{n}\left(T_{1},\mathbf{z}\right)\right)$. La
trayectoria de $\left(\mathbf{x}_{t}\right)$ para $t\leq T_{1}$
es\footnote{Revisar p\'agina 362, y 364 de Davis \cite{Davis}.}
\begin{eqnarray*}
\mathbf{x}_{t}=\left(v_{t},\zeta_{t}\right)=\left\{\begin{array}{ll}
\left(n,\phi_{n}\left(t,\mathbf{z}\right)\right), & t<T_{1},\\
\left(N,\mathbf{Z}\right), & t=t_{1}.
\end{array}\right.
\end{eqnarray*}

Comenzando en $\mathbf{x}_{T_{1}}$ se selecciona el siguiente
tiempo de intersalto $T_{2}-T_{1}$ lugar del post-salto
$\mathbf{x}_{T_{2}}$ de manera similar y as\'i sucesivamente. Este
procedimiento nos da una trayectoria determinista por partes
$\mathbf{x}_{t}$ con tiempos de salto $T_{1},T_{2},\ldots$. Bajo
las condiciones enunciadas para $\lambda,T_{1}>0$  y
$T_{1}-T_{2}>0$ para cada $i$, con probabilidad 1. Se supone que
se cumple la siguiente condici\'on.

\begin{Sup}[Supuesto 3.1, Davis \cite{Davis}]\label{Sup3.1.Davis}
Sea $N_{t}:=\sum_{t}\indora_{\left(t\geq t\right)}$ el n\'umero de
saltos en $\left[0,t\right]$. Entonces
\begin{equation}
\esp\left[N_{t}\right]<\infty\textrm{ para toda }t.
\end{equation}
\end{Sup}

es un proceso de Markov, m\'as a\'un, es un Proceso Fuerte de
Markov, es decir, la Propiedad Fuerte de Markov se cumple para
cualquier tiempo de paro.


Sea $E$ es un espacio m\'etrico separable y la m\'etrica $d$ es
compatible con la topolog\'ia.


\begin{Def}
Un espacio topol\'ogico $E$ es llamado de {\em Rad\'on} si es
homeomorfo a un subconjunto universalmente medible de un espacio
m\'etrico compacto.
\end{Def}

Equivalentemente, la definici\'on de un espacio de Rad\'on puede
encontrarse en los siguientes t\'erminos:


\begin{Def}
$E$ es un espacio de Rad\'on si cada medida finita en
$\left(E,\mathcal{B}\left(E\right)\right)$ es regular interior o
cerrada, {\em tight}.
\end{Def}

\begin{Def}
Una medida finita, $\lambda$ en la $\sigma$-\'algebra de Borel de
un espacio metrizable $E$ se dice cerrada si
\begin{equation}\label{Eq.A2.3}
\lambda\left(E\right)=sup\left\{\lambda\left(K\right):K\textrm{ es
compacto en }E\right\}.
\end{equation}
\end{Def}

El siguiente teorema nos permite tener una mejor caracterizaci\'on
de los espacios de Rad\'on:
\begin{Teo}\label{Tma.A2.2}
Sea $E$ espacio separable metrizable. Entonces $E$ es Radoniano si
y s\'olo s\'i cada medida finita en
$\left(E,\mathcal{B}\left(E\right)\right)$ es cerrada.
\end{Teo}

Sea $E$ espacio de estados, tal que $E$ es un espacio de Rad\'on,
$\mathcal{B}\left(E\right)$ $\sigma$-\'algebra de Borel en $E$,
que se denotar\'a por $\mathcal{E}$.

Sea $\left(X,\mathcal{G},\prob\right)$ espacio de probabilidad,
$I\subset\rea$ conjunto de \'indices. Sea $\mathcal{F}_{\leq t}$
la $\sigma$-\'algebra natural definida como
$\sigma\left\{f\left(X_{r}\right):r\in I, r\leq
t,f\in\mathcal{E}\right\}$. Se considerar\'a una
$\sigma$-\'algebra m\'as general, $ \left(\mathcal{G}_{t}\right)$
tal que $\left(X_{t}\right)$ sea $\mathcal{E}$-adaptado.

\begin{Def}
Una familia $\left(P_{s,t}\right)$ de kernels de Markov en
$\left(E,\mathcal{E}\right)$ indexada por pares $s,t\in I$, con
$s\leq t$ es una funci\'on de transici\'on en $\ER$, si  para todo
$r\leq s< t$ en $I$ y todo $x\in E$,
$B\in\mathcal{E}$\footnote{Ecuaci\'on de Chapman-Kolmogorov}
\begin{equation}\label{Eq.Kernels}
P_{r,t}\left(x,B\right)=\int_{E}P_{r,s}\left(x,dy\right)P_{s,t}\left(y,B\right).
\end{equation}
\end{Def}

Se dice que la funci\'on de transici\'on $\KM$ en $\ER$ es la
funci\'on de transici\'on para un proceso $\PE$  con valores en
$E$ y que satisface la propiedad de
Markov\footnote{\begin{equation}\label{Eq.1.4.S}
\prob\left\{H|\mathcal{G}_{t}\right\}=\prob\left\{H|X_{t}\right\}\textrm{
}H\in p\mathcal{F}_{\geq t}.
\end{equation}} (\ref{Eq.1.4.S}) relativa a $\left(\mathcal{G}_{t}\right)$ si

\begin{equation}\label{Eq.1.6.S}
\prob\left\{f\left(X_{t}\right)|\mathcal{G}_{s}\right\}=P_{s,t}f\left(X_{t}\right)\textrm{
}s\leq t\in I,\textrm{ }f\in b\mathcal{E}.
\end{equation}

\begin{Def}
Una familia $\left(P_{t}\right)_{t\geq0}$ de kernels de Markov en
$\ER$ es llamada {\em Semigrupo de Transici\'on de Markov} o {\em
Semigrupo de Transici\'on} si
\[P_{t+s}f\left(x\right)=P_{t}\left(P_{s}f\right)\left(x\right),\textrm{ }t,s\geq0,\textrm{ }x\in E\textrm{ }f\in b\mathcal{E}.\]
\end{Def}
\begin{Note}
Si la funci\'on de transici\'on $\KM$ es llamada homog\'enea si
$P_{s,t}=P_{t-s}$.
\end{Note}

Un proceso de Markov que satisface la ecuaci\'on (\ref{Eq.1.6.S})
con funci\'on de transici\'on homog\'enea $\left(P_{t}\right)$
tiene la propiedad caracter\'istica
\begin{equation}\label{Eq.1.8.S}
\prob\left\{f\left(X_{t+s}\right)|\mathcal{G}_{t}\right\}=P_{s}f\left(X_{t}\right)\textrm{
}t,s\geq0,\textrm{ }f\in b\mathcal{E}.
\end{equation}
La ecuaci\'on anterior es la {\em Propiedad Simple de Markov} de
$X$ relativa a $\left(P_{t}\right)$.

En este sentido el proceso $\PE$ cumple con la propiedad de Markov
(\ref{Eq.1.8.S}) relativa a
$\left(\Omega,\mathcal{G},\mathcal{G}_{t},\prob\right)$ con
semigrupo de transici\'on $\left(P_{t}\right)$.

\begin{Def}
Un proceso estoc\'astico $\PE$ definido en
$\left(\Omega,\mathcal{G},\prob\right)$ con valores en el espacio
topol\'ogico $E$ es continuo por la derecha si cada trayectoria
muestral $t\rightarrow X_{t}\left(w\right)$ es un mapeo continuo
por la derecha de $I$ en $E$.
\end{Def}

\begin{Def}[HD1]\label{Eq.2.1.S}
Un semigrupo de Markov $\left(P_{t}\right)$ en un espacio de
Rad\'on $E$ se dice que satisface la condici\'on {\em HD1} si,
dada una medida de probabilidad $\mu$ en $E$, existe una
$\sigma$-\'algebra $\mathcal{E^{*}}$ con
$\mathcal{E}\subset\mathcal{E}^{*}$ y
$P_{t}\left(b\mathcal{E}^{*}\right)\subset b\mathcal{E}^{*}$, y un
$\mathcal{E}^{*}$-proceso $E$-valuado continuo por la derecha
$\PE$ en alg\'un espacio de probabilidad filtrado
$\left(\Omega,\mathcal{G},\mathcal{G}_{t},\prob\right)$ tal que
$X=\left(\Omega,\mathcal{G},\mathcal{G}_{t},\prob\right)$ es de
Markov (Homog\'eneo) con semigrupo de transici\'on $(P_{t})$ y
distribuci\'on inicial $\mu$.
\end{Def}

Consid\'erese la colecci\'on de variables aleatorias $X_{t}$
definidas en alg\'un espacio de probabilidad, y una colecci\'on de
medidas $\mathbf{P}^{x}$ tales que
$\mathbf{P}^{x}\left\{X_{0}=x\right\}$, y bajo cualquier
$\mathbf{P}^{x}$, $X_{t}$ es de Markov con semigrupo
$\left(P_{t}\right)$. $\mathbf{P}^{x}$ puede considerarse como la
distribuci\'on condicional de $\mathbf{P}$ dado $X_{0}=x$.

\begin{Def}\label{Def.2.2.S}
Sea $E$ espacio de Rad\'on, $\SG$ semigrupo de Markov en $\ER$. La
colecci\'on
$\mathbf{X}=\left(\Omega,\mathcal{G},\mathcal{G}_{t},X_{t},\theta_{t},\CM\right)$
es un proceso $\mathcal{E}$-Markov continuo por la derecha simple,
con espacio de estados $E$ y semigrupo de transici\'on $\SG$ en
caso de que $\mathbf{X}$ satisfaga las siguientes condiciones:
\begin{itemize}
\item[i)] $\left(\Omega,\mathcal{G},\mathcal{G}_{t}\right)$ es un
espacio de medida filtrado, y $X_{t}$ es un proceso $E$-valuado
continuo por la derecha $\mathcal{E}^{*}$-adaptado a
$\left(\mathcal{G}_{t}\right)$;

\item[ii)] $\left(\theta_{t}\right)_{t\geq0}$ es una colecci\'on
de operadores {\em shift} para $X$, es decir, mapea $\Omega$ en
s\'i mismo satisfaciendo para $t,s\geq0$,

\begin{equation}\label{Eq.Shift}
\theta_{t}\circ\theta_{s}=\theta_{t+s}\textrm{ y
}X_{t}\circ\theta_{t}=X_{t+s};
\end{equation}

\item[iii)] Para cualquier $x\in E$,$\CM\left\{X_{0}=x\right\}=1$,
y el proceso $\PE$ tiene la propiedad de Markov (\ref{Eq.1.8.S})
con semigrupo de transici\'on $\SG$ relativo a
$\left(\Omega,\mathcal{G},\mathcal{G}_{t},\CM\right)$.
\end{itemize}
\end{Def}

\begin{Def}[HD2]\label{Eq.2.2.S}
Para cualquier $\alpha>0$ y cualquier $f\in S^{\alpha}$, el
proceso $t\rightarrow f\left(X_{t}\right)$ es continuo por la
derecha casi seguramente.
\end{Def}

\begin{Def}\label{Def.PD}
Un sistema
$\mathbf{X}=\left(\Omega,\mathcal{G},\mathcal{G}_{t},X_{t},\theta_{t},\CM\right)$
es un proceso derecho en el espacio de Rad\'on $E$ con semigrupo
de transici\'on $\SG$ provisto de:
\begin{itemize}
\item[i)] $\mathbf{X}$ es una realizaci\'on  continua por la
derecha, \ref{Def.2.2.S}, de $\SG$.

\item[ii)] $\mathbf{X}$ satisface la condicion HD2,
\ref{Eq.2.2.S}, relativa a $\mathcal{G}_{t}$.

\item[iii)] $\mathcal{G}_{t}$ es aumentado y continuo por la
derecha.
\end{itemize}
\end{Def}

\begin{Lema}[Lema 4.2, Dai\cite{Dai}]\label{Lema4.2}
Sea $\left\{x_{n}\right\}\subset \mathbf{X}$ con
$|x_{n}|\rightarrow\infty$, conforme $n\rightarrow\infty$. Suponga
que
\[lim_{n\rightarrow\infty}\frac{1}{|x_{n}|}U\left(0\right)=\overline{U}\]
y
\[lim_{n\rightarrow\infty}\frac{1}{|x_{n}|}V\left(0\right)=\overline{V}.\]

Entonces, conforme $n\rightarrow\infty$, casi seguramente

\begin{equation}\label{E1.4.2}
\frac{1}{|x_{n}|}\Phi^{k}\left(\left[|x_{n}|t\right]\right)\rightarrow
P_{k}^{'}t\textrm{, u.o.c.,}
\end{equation}

\begin{equation}\label{E1.4.3}
\frac{1}{|x_{n}|}E^{x_{n}}_{k}\left(|x_{n}|t\right)\rightarrow
\alpha_{k}\left(t-\overline{U}_{k}\right)^{+}\textrm{, u.o.c.,}
\end{equation}

\begin{equation}\label{E1.4.4}
\frac{1}{|x_{n}|}S^{x_{n}}_{k}\left(|x_{n}|t\right)\rightarrow
\mu_{k}\left(t-\overline{V}_{k}\right)^{+}\textrm{, u.o.c.,}
\end{equation}

donde $\left[t\right]$ es la parte entera de $t$ y
$\mu_{k}=1/m_{k}=1/\esp\left[\eta_{k}\left(1\right)\right]$.
\end{Lema}

\begin{Lema}[Lema 4.3, Dai\cite{Dai}]\label{Lema.4.3}
Sea $\left\{x_{n}\right\}\subset \mathbf{X}$ con
$|x_{n}|\rightarrow\infty$, conforme $n\rightarrow\infty$. Suponga
que
\[lim_{n\rightarrow\infty}\frac{1}{|x_{n}|}U\left(0\right)=\overline{U}_{k}\]
y
\[lim_{n\rightarrow\infty}\frac{1}{|x_{n}|}V\left(0\right)=\overline{V}_{k}.\]
\begin{itemize}
\item[a)] Conforme $n\rightarrow\infty$ casi seguramente,
\[lim_{n\rightarrow\infty}\frac{1}{|x_{n}|}U^{x_{n}}_{k}\left(|x_{n}|t\right)=\left(\overline{U}_{k}-t\right)^{+}\textrm{, u.o.c.}\]
y
\[lim_{n\rightarrow\infty}\frac{1}{|x_{n}|}V^{x_{n}}_{k}\left(|x_{n}|t\right)=\left(\overline{V}_{k}-t\right)^{+}.\]

\item[b)] Para cada $t\geq0$ fijo,
\[\left\{\frac{1}{|x_{n}|}U^{x_{n}}_{k}\left(|x_{n}|t\right),|x_{n}|\geq1\right\}\]
y
\[\left\{\frac{1}{|x_{n}|}V^{x_{n}}_{k}\left(|x_{n}|t\right),|x_{n}|\geq1\right\}\]
\end{itemize}
son uniformemente convergentes.
\end{Lema}

$S_{l}^{x}\left(t\right)$ es el n\'umero total de servicios
completados de la clase $l$, si la clase $l$ est\'a dando $t$
unidades de tiempo de servicio. Sea $T_{l}^{x}\left(x\right)$ el
monto acumulado del tiempo de servicio que el servidor
$s\left(l\right)$ gasta en los usuarios de la clase $l$ al tiempo
$t$. Entonces $S_{l}^{x}\left(T_{l}^{x}\left(t\right)\right)$ es
el n\'umero total de servicios completados para la clase $l$ al
tiempo $t$. Una fracci\'on de estos usuarios,
$\Phi_{l}^{x}\left(S_{l}^{x}\left(T_{l}^{x}\left(t\right)\right)\right)$,
se convierte en usuarios de la clase $k$.\\

Entonces, dado lo anterior, se tiene la siguiente representaci\'on
para el proceso de la longitud de la cola:\\

\begin{equation}
Q_{k}^{x}\left(t\right)=_{k}^{x}\left(0\right)+E_{k}^{x}\left(t\right)+\sum_{l=1}^{K}\Phi_{k}^{l}\left(S_{l}^{x}\left(T_{l}^{x}\left(t\right)\right)\right)-S_{k}^{x}\left(T_{k}^{x}\left(t\right)\right)
\end{equation}
para $k=1,\ldots,K$. Para $i=1,\ldots,d$, sea
\[I_{i}^{x}\left(t\right)=t-\sum_{j\in C_{i}}T_{k}^{x}\left(t\right).\]

Entonces $I_{i}^{x}\left(t\right)$ es el monto acumulado del
tiempo que el servidor $i$ ha estado desocupado al tiempo $t$. Se
est\'a asumiendo que las disciplinas satisfacen la ley de
conservaci\'on del trabajo, es decir, el servidor $i$ est\'a en
pausa solamente cuando no hay usuarios en la estaci\'on $i$.
Entonces, se tiene que

\begin{equation}
\int_{0}^{\infty}\left(\sum_{k\in
C_{i}}Q_{k}^{x}\left(t\right)\right)dI_{i}^{x}\left(t\right)=0,
\end{equation}
para $i=1,\ldots,d$.\\

Hacer
\[T^{x}\left(t\right)=\left(T_{1}^{x}\left(t\right),\ldots,T_{K}^{x}\left(t\right)\right)^{'},\]
\[I^{x}\left(t\right)=\left(I_{1}^{x}\left(t\right),\ldots,I_{K}^{x}\left(t\right)\right)^{'}\]
y
\[S^{x}\left(T^{x}\left(t\right)\right)=\left(S_{1}^{x}\left(T_{1}^{x}\left(t\right)\right),\ldots,S_{K}^{x}\left(T_{K}^{x}\left(t\right)\right)\right)^{'}.\]

Para una disciplina que cumple con la ley de conservaci\'on del
trabajo, en forma vectorial, se tiene el siguiente conjunto de
ecuaciones

\begin{equation}\label{Eq.MF.1.3}
Q^{x}\left(t\right)=Q^{x}\left(0\right)+E^{x}\left(t\right)+\sum_{l=1}^{K}\Phi^{l}\left(S_{l}^{x}\left(T_{l}^{x}\left(t\right)\right)\right)-S^{x}\left(T^{x}\left(t\right)\right),\\
\end{equation}

\begin{equation}\label{Eq.MF.2.3}
Q^{x}\left(t\right)\geq0,\\
\end{equation}

\begin{equation}\label{Eq.MF.3.3}
T^{x}\left(0\right)=0,\textrm{ y }\overline{T}^{x}\left(t\right)\textrm{ es no decreciente},\\
\end{equation}

\begin{equation}\label{Eq.MF.4.3}
I^{x}\left(t\right)=et-CT^{x}\left(t\right)\textrm{ es no
decreciente}\\
\end{equation}

\begin{equation}\label{Eq.MF.5.3}
\int_{0}^{\infty}\left(CQ^{x}\left(t\right)\right)dI_{i}^{x}\left(t\right)=0,\\
\end{equation}

\begin{equation}\label{Eq.MF.6.3}
\textrm{Condiciones adicionales en
}\left(\overline{Q}^{x}\left(\cdot\right),\overline{T}^{x}\left(\cdot\right)\right)\textrm{
espec\'ificas de la disciplina de la cola,}
\end{equation}

donde $e$ es un vector de unos de dimensi\'on $d$, $C$ es la
matriz definida por
\[C_{ik}=\left\{\begin{array}{cc}
1,& S\left(k\right)=i,\\
0,& \textrm{ en otro caso}.\\
\end{array}\right.
\]
Es necesario enunciar el siguiente Teorema que se utilizar\'a para
el Teorema \ref{Tma.4.2.Dai}:
\begin{Teo}[Teorema 4.1, Dai \cite{Dai}]
Considere una disciplina que cumpla la ley de conservaci\'on del
trabajo, para casi todas las trayectorias muestrales $\omega$ y
cualquier sucesi\'on de estados iniciales
$\left\{x_{n}\right\}\subset \mathbf{X}$, con
$|x_{n}|\rightarrow\infty$, existe una subsucesi\'on
$\left\{x_{n_{j}}\right\}$ con $|x_{n_{j}}|\rightarrow\infty$ tal
que
\begin{equation}\label{Eq.4.15}
\frac{1}{|x_{n_{j}}|}\left(Q^{x_{n_{j}}}\left(0\right),U^{x_{n_{j}}}\left(0\right),V^{x_{n_{j}}}\left(0\right)\right)\rightarrow\left(\overline{Q}\left(0\right),\overline{U},\overline{V}\right),
\end{equation}

\begin{equation}\label{Eq.4.16}
\frac{1}{|x_{n_{j}}|}\left(Q^{x_{n_{j}}}\left(|x_{n_{j}}|t\right),T^{x_{n_{j}}}\left(|x_{n_{j}}|t\right)\right)\rightarrow\left(\overline{Q}\left(t\right),\overline{T}\left(t\right)\right)\textrm{
u.o.c.}
\end{equation}

Adem\'as,
$\left(\overline{Q}\left(t\right),\overline{T}\left(t\right)\right)$
satisface las siguientes ecuaciones:
\begin{equation}\label{Eq.MF.1.3a}
\overline{Q}\left(t\right)=Q\left(0\right)+\left(\alpha
t-\overline{U}\right)^{+}-\left(I-P\right)^{'}M^{-1}\left(\overline{T}\left(t\right)-\overline{V}\right)^{+},
\end{equation}

\begin{equation}\label{Eq.MF.2.3a}
\overline{Q}\left(t\right)\geq0,\\
\end{equation}

\begin{equation}\label{Eq.MF.3.3a}
\overline{T}\left(t\right)\textrm{ es no decreciente y comienza en cero},\\
\end{equation}

\begin{equation}\label{Eq.MF.4.3a}
\overline{I}\left(t\right)=et-C\overline{T}\left(t\right)\textrm{
es no decreciente,}\\
\end{equation}

\begin{equation}\label{Eq.MF.5.3a}
\int_{0}^{\infty}\left(C\overline{Q}\left(t\right)\right)d\overline{I}\left(t\right)=0,\\
\end{equation}

\begin{equation}\label{Eq.MF.6.3a}
\textrm{Condiciones adicionales en
}\left(\overline{Q}\left(\cdot\right),\overline{T}\left(\cdot\right)\right)\textrm{
especficas de la disciplina de la cola,}
\end{equation}
\end{Teo}


Propiedades importantes para el modelo de flujo retrasado:

\begin{Prop}
 Sea $\left(\overline{Q},\overline{T},\overline{T}^{0}\right)$ un flujo l\'imite de \ref{Eq.4.4} y suponga que cuando $x\rightarrow\infty$ a lo largo de
una subsucesi\'on
\[\left(\frac{1}{|x|}Q_{k}^{x}\left(0\right),\frac{1}{|x|}A_{k}^{x}\left(0\right),\frac{1}{|x|}B_{k}^{x}\left(0\right),\frac{1}{|x|}B_{k}^{x,0}\left(0\right)\right)\rightarrow\left(\overline{Q}_{k}\left(0\right),0,0,0\right)\]
para $k=1,\ldots,K$. EL flujo l\'imite tiene las siguientes
propiedades, donde las propiedades de la derivada se cumplen donde
la derivada exista:
\begin{itemize}
 \item[i)] Los vectores de tiempo ocupado $\overline{T}\left(t\right)$ y $\overline{T}^{0}\left(t\right)$ son crecientes y continuas con
$\overline{T}\left(0\right)=\overline{T}^{0}\left(0\right)=0$.
\item[ii)] Para todo $t\geq0$
\[\sum_{k=1}^{K}\left[\overline{T}_{k}\left(t\right)+\overline{T}_{k}^{0}\left(t\right)\right]=t\]
\item[iii)] Para todo $1\leq k\leq K$
\[\overline{Q}_{k}\left(t\right)=\overline{Q}_{k}\left(0\right)+\alpha_{k}t-\mu_{k}\overline{T}_{k}\left(t\right)\]
\item[iv)]  Para todo $1\leq k\leq K$
\[\dot{{\overline{T}}}_{k}\left(t\right)=\beta_{k}\] para $\overline{Q}_{k}\left(t\right)=0$.
\item[v)] Para todo $k,j$
\[\mu_{k}^{0}\overline{T}_{k}^{0}\left(t\right)=\mu_{j}^{0}\overline{T}_{j}^{0}\left(t\right)\]
\item[vi)]  Para todo $1\leq k\leq K$
\[\mu_{k}\dot{{\overline{T}}}_{k}\left(t\right)=l_{k}\mu_{k}^{0}\dot{{\overline{T}}}_{k}^{0}\left(t\right)\] para $\overline{Q}_{k}\left(t\right)>0$.
\end{itemize}
\end{Prop}

\begin{Lema}[Lema 3.1 \cite{Chen}]\label{Lema3.1}
Si el modelo de flujo es estable, definido por las ecuaciones
(3.8)-(3.13), entonces el modelo de flujo retrasado tambi\'en es
estable.
\end{Lema}

\begin{Teo}[Teorema 5.1 \cite{Chen}]\label{Tma.5.1.Chen}
La red de colas es estable si existe una constante $t_{0}$ que
depende de $\left(\alpha,\mu,T,U\right)$ y $V$ que satisfagan las
ecuaciones (5.1)-(5.5), $Z\left(t\right)=0$, para toda $t\geq
t_{0}$.
\end{Teo}



\begin{Lema}[Lema 5.2 \cite{Gut}]\label{Lema.5.2.Gut}
Sea $\left\{\xi\left(k\right):k\in\ent\right\}$ sucesi\'on de
variables aleatorias i.i.d. con valores en
$\left(0,\infty\right)$, y sea $E\left(t\right)$ el proceso de
conteo
\[E\left(t\right)=max\left\{n\geq1:\xi\left(1\right)+\cdots+\xi\left(n-1\right)\leq t\right\}.\]
Si $E\left[\xi\left(1\right)\right]<\infty$, entonces para
cualquier entero $r\geq1$
\begin{equation}
lim_{t\rightarrow\infty}\esp\left[\left(\frac{E\left(t\right)}{t}\right)^{r}\right]=\left(\frac{1}{E\left[\xi_{1}\right]}\right)^{r}
\end{equation}
de aqu\'i, bajo estas condiciones
\begin{itemize}
\item[a)] Para cualquier $t>0$,
$sup_{t\geq\delta}\esp\left[\left(\frac{E\left(t\right)}{t}\right)^{r}\right]$

\item[b)] Las variables aleatorias
$\left\{\left(\frac{E\left(t\right)}{t}\right)^{r}:t\geq1\right\}$
son uniformemente integrables.
\end{itemize}
\end{Lema}

\begin{Teo}[Teorema 5.1: Ley Fuerte para Procesos de Conteo
\cite{Gut}]\label{Tma.5.1.Gut} Sea
$0<\mu<\esp\left(X_{1}\right]\leq\infty$. entonces

\begin{itemize}
\item[a)] $\frac{N\left(t\right)}{t}\rightarrow\frac{1}{\mu}$
a.s., cuando $t\rightarrow\infty$.


\item[b)]$\esp\left[\frac{N\left(t\right)}{t}\right]^{r}\rightarrow\frac{1}{\mu^{r}}$,
cuando $t\rightarrow\infty$ para todo $r>0$..
\end{itemize}
\end{Teo}


\begin{Prop}[Proposici\'on 5.1 \cite{DaiSean}]\label{Prop.5.1}
Suponga que los supuestos (A1) y (A2) se cumplen, adem\'as suponga
que el modelo de flujo es estable. Entonces existe $t_{0}>0$ tal
que
\begin{equation}\label{Eq.Prop.5.1}
lim_{|x|\rightarrow\infty}\frac{1}{|x|^{p+1}}\esp_{x}\left[|X\left(t_{0}|x|\right)|^{p+1}\right]=0.
\end{equation}

\end{Prop}


\begin{Prop}[Proposici\'on 5.3 \cite{DaiSean}]
Sea $X$ proceso de estados para la red de colas, y suponga que se
cumplen los supuestos (A1) y (A2), entonces para alguna constante
positiva $C_{p+1}<\infty$, $\delta>0$ y un conjunto compacto
$C\subset X$.

\begin{equation}\label{Eq.5.4}
\esp_{x}\left[\int_{0}^{\tau_{C}\left(\delta\right)}\left(1+|X\left(t\right)|^{p}\right)dt\right]\leq
C_{p+1}\left(1+|x|^{p+1}\right)
\end{equation}
\end{Prop}

\begin{Prop}[Proposici\'on 5.4 \cite{DaiSean}]
Sea $X$ un proceso de Markov Borel Derecho en $X$, sea
$f:X\leftarrow\rea_{+}$ y defina para alguna $\delta>0$, y un
conjunto cerrado $C\subset X$
\[V\left(x\right):=\esp_{x}\left[\int_{0}^{\tau_{C}\left(\delta\right)}f\left(X\left(t\right)\right)dt\right]\]
para $x\in X$. Si $V$ es finito en todas partes y uniformemente
acotada en $C$, entonces existe $k<\infty$ tal que
\begin{equation}\label{Eq.5.11}
\frac{1}{t}\esp_{x}\left[V\left(x\right)\right]+\frac{1}{t}\int_{0}^{t}\esp_{x}\left[f\left(X\left(s\right)\right)ds\right]\leq\frac{1}{t}V\left(x\right)+k,
\end{equation}
para $x\in X$ y $t>0$.
\end{Prop}


\begin{Teo}[Teorema 5.5 \cite{DaiSean}]
Suponga que se cumplen (A1) y (A2), adem\'as suponga que el modelo
de flujo es estable. Entonces existe una constante $k_{p}<\infty$
tal que
\begin{equation}\label{Eq.5.13}
\frac{1}{t}\int_{0}^{t}\esp_{x}\left[|Q\left(s\right)|^{p}\right]ds\leq
k_{p}\left\{\frac{1}{t}|x|^{p+1}+1\right\}
\end{equation}
para $t\geq0$, $x\in X$. En particular para cada condici\'on
inicial
\begin{equation}\label{Eq.5.14}
Limsup_{t\rightarrow\infty}\frac{1}{t}\int_{0}^{t}\esp_{x}\left[|Q\left(s\right)|^{p}\right]ds\leq
k_{p}
\end{equation}
\end{Teo}

\begin{Teo}[Teorema 6.2 \cite{DaiSean}]\label{Tma.6.2}
Suponga que se cumplen los supuestos (A1)-(A3) y que el modelo de
flujo es estable, entonces se tiene que
\[\parallel P^{t}\left(c,\cdot\right)-\pi\left(\cdot\right)\parallel_{f_{p}}\rightarrow0\]
para $t\rightarrow\infty$ y $x\in X$. En particular para cada
condici\'on inicial
\[lim_{t\rightarrow\infty}\esp_{x}\left[\left|Q_{t}\right|^{p}\right]=\esp_{\pi}\left[\left|Q_{0}\right|^{p}\right]<\infty\]
\end{Teo}


\begin{Teo}[Teorema 6.3 \cite{DaiSean}]\label{Tma.6.3}
Suponga que se cumplen los supuestos (A1)-(A3) y que el modelo de
flujo es estable, entonces con
$f\left(x\right)=f_{1}\left(x\right)$, se tiene que
\[lim_{t\rightarrow\infty}t^{(p-1)\left|P^{t}\left(c,\cdot\right)-\pi\left(\cdot\right)\right|_{f}=0},\]
para $x\in X$. En particular, para cada condici\'on inicial
\[lim_{t\rightarrow\infty}t^{(p-1)}\left|\esp_{x}\left[Q_{t}\right]-\esp_{\pi}\left[Q_{0}\right]\right|=0.\]
\end{Teo}



\begin{Prop}[Proposici\'on 5.1, Dai y Meyn \cite{DaiSean}]\label{Prop.5.1.DaiSean}
Suponga que los supuestos A1) y A2) son ciertos y que el modelo de
flujo es estable. Entonces existe $t_{0}>0$ tal que
\begin{equation}
lim_{|x|\rightarrow\infty}\frac{1}{|x|^{p+1}}\esp_{x}\left[|X\left(t_{0}|x|\right)|^{p+1}\right]=0
\end{equation}
\end{Prop}

\begin{Lemma}[Lema 5.2, Dai y Meyn, \cite{DaiSean}]\label{Lema.5.2.DaiSean}
 Sea $\left\{\zeta\left(k\right):k\in \mathbb{z}\right\}$ una sucesi\'on independiente e id\'enticamente distribuida que toma valores en $\left(0,\infty\right)$,
y sea
$E\left(t\right)=max\left(n\geq1:\zeta\left(1\right)+\cdots+\zeta\left(n-1\right)\leq
t\right)$. Si $\esp\left[\zeta\left(1\right)\right]<\infty$,
entonces para cualquier entero $r\geq1$
\begin{equation}
 lim_{t\rightarrow\infty}\esp\left[\left(\frac{E\left(t\right)}{t}\right)^{r}\right]=\left(\frac{1}{\esp\left[\zeta_{1}\right]}\right)^{r}.
\end{equation}
Luego, bajo estas condiciones:
\begin{itemize}
 \item[a)] para cualquier $\delta>0$, $\sup_{t\geq\delta}\esp\left[\left(\frac{E\left(t\right)}{t}\right)^{r}\right]<\infty$
\item[b)] las variables aleatorias
$\left\{\left(\frac{E\left(t\right)}{t}\right)^{r}:t\geq1\right\}$
son uniformemente integrables.
\end{itemize}
\end{Lemma}

\begin{Teo}[Teorema 5.5, Dai y Meyn \cite{DaiSean}]\label{Tma.5.5.DaiSean}
Suponga que los supuestos A1) y A2) se cumplen y que el modelo de
flujo es estable. Entonces existe una constante $\kappa_{p}$ tal
que
\begin{equation}
\frac{1}{t}\int_{0}^{t}\esp_{x}\left[|Q\left(s\right)|^{p}\right]ds\leq\kappa_{p}\left\{\frac{1}{t}|x|^{p+1}+1\right\}
\end{equation}
para $t>0$ y $x\in X$. En particular, para cada condici\'on
inicial
\begin{eqnarray*}
\limsup_{t\rightarrow\infty}\frac{1}{t}\int_{0}^{t}\esp_{x}\left[|Q\left(s\right)|^{p}\right]ds\leq\kappa_{p}.
\end{eqnarray*}
\end{Teo}

\begin{Teo}[Teorema 6.2, Dai y Meyn \cite{DaiSean}]\label{Tma.6.2.DaiSean}
Suponga que se cumplen los supuestos A1), A2) y A3) y que el
modelo de flujo es estable. Entonces se tiene que
\begin{equation}
\left\|P^{t}\left(x,\cdot\right)-\pi\left(\cdot\right)\right\|_{f_{p}}\textrm{,
}t\rightarrow\infty,x\in X.
\end{equation}
En particular para cada condici\'on inicial
\begin{eqnarray*}
\lim_{t\rightarrow\infty}\esp_{x}\left[|Q\left(t\right)|^{p}\right]=\esp_{\pi}\left[|Q\left(0\right)|^{p}\right]\leq\kappa_{r}
\end{eqnarray*}
\end{Teo}
\begin{Teo}[Teorema 6.3, Dai y Meyn \cite{DaiSean}]\label{Tma.6.3.DaiSean}
Suponga que se cumplen los supuestos A1), A2) y A3) y que el
modelo de flujo es estable. Entonces con
$f\left(x\right)=f_{1}\left(x\right)$ se tiene
\begin{equation}
\lim_{t\rightarrow\infty}t^{p-1}\left\|P^{t}\left(x,\cdot\right)-\pi\left(\cdot\right)\right\|_{f}=0.
\end{equation}
En particular para cada condici\'on inicial
\begin{eqnarray*}
\lim_{t\rightarrow\infty}t^{p-1}|\esp_{x}\left[Q\left(t\right)\right]-\esp_{\pi}\left[Q\left(0\right)\right]|=0.
\end{eqnarray*}
\end{Teo}

\begin{Teo}[Teorema 6.4, Dai y Meyn, \cite{DaiSean}]\label{Tma.6.4.DaiSean}
Suponga que se cumplen los supuestos A1), A2) y A3) y que el
modelo de flujo es estable. Sea $\nu$ cualquier distribuci\'on de
probabilidad en $\left(X,\mathcal{B}_{X}\right)$, y $\pi$ la
distribuci\'on estacionaria de $X$.
\begin{itemize}
\item[i)] Para cualquier $f:X\leftarrow\rea_{+}$
\begin{equation}
\lim_{t\rightarrow\infty}\frac{1}{t}\int_{o}^{t}f\left(X\left(s\right)\right)ds=\pi\left(f\right):=\int
f\left(x\right)\pi\left(dx\right)
\end{equation}
$\prob$-c.s.

\item[ii)] Para cualquier $f:X\leftarrow\rea_{+}$ con
$\pi\left(|f|\right)<\infty$, la ecuaci\'on anterior se cumple.
\end{itemize}
\end{Teo}

\begin{Teo}[Teorema 2.2, Down \cite{Down}]\label{Tma2.2.Down}
Suponga que el fluido modelo es inestable en el sentido de que
para alguna $\epsilon_{0},c_{0}\geq0$,
\begin{equation}\label{Eq.Inestability}
|Q\left(T\right)|\geq\epsilon_{0}T-c_{0}\textrm{,   }T\geq0,
\end{equation}
para cualquier condici\'on inicial $Q\left(0\right)$, con
$|Q\left(0\right)|=1$. Entonces para cualquier $0<q\leq1$, existe
$B<0$ tal que para cualquier $|x|\geq B$,
\begin{equation}
\prob_{x}\left\{\mathbb{X}\rightarrow\infty\right\}\geq q.
\end{equation}
\end{Teo}



\begin{Def}
Sea $X$ un conjunto y $\mathcal{F}$ una $\sigma$-\'algebra de
subconjuntos de $X$, la pareja $\left(X,\mathcal{F}\right)$ es
llamado espacio medible. Un subconjunto $A$ de $X$ es llamado
medible, o medible con respecto a $\mathcal{F}$, si
$A\in\mathcal{F}$.
\end{Def}

\begin{Def}
Sea $\left(X,\mathcal{F},\mu\right)$ espacio de medida. Se dice
que la medida $\mu$ es $\sigma$-finita si se puede escribir
$X=\bigcup_{n\geq1}X_{n}$ con $X_{n}\in\mathcal{F}$ y
$\mu\left(X_{n}\right)<\infty$.
\end{Def}

\begin{Def}\label{Cto.Borel}
Sea $X$ el conjunto de los n\'umeros reales $\rea$. El \'algebra
de Borel es la $\sigma$-\'algebra $B$ generada por los intervalos
abiertos $\left(a,b\right)\in\rea$. Cualquier conjunto en $B$ es
llamado {\em Conjunto de Borel}.
\end{Def}

\begin{Def}\label{Funcion.Medible}
Una funci\'on $f:X\rightarrow\rea$, es medible si para cualquier
n\'umero real $\alpha$ el conjunto
\[\left\{x\in X:f\left(x\right)>\alpha\right\}\]
pertenece a $\mathcal{F}$. Equivalentemente, se dice que $f$ es
medible si
\[f^{-1}\left(\left(\alpha,\infty\right)\right)=\left\{x\in X:f\left(x\right)>\alpha\right\}\in\mathcal{F}.\]
\end{Def}


\begin{Def}\label{Def.Cilindros}
Sean $\left(\Omega_{i},\mathcal{F}_{i}\right)$, $i=1,2,\ldots,$
espacios medibles y $\Omega=\prod_{i=1}^{\infty}\Omega_{i}$ el
conjunto de todas las sucesiones
$\left(\omega_{1},\omega_{2},\ldots,\right)$ tales que
$\omega_{i}\in\Omega_{i}$, $i=1,2,\ldots,$. Si
$B^{n}\subset\prod_{i=1}^{\infty}\Omega_{i}$, definimos
$B_{n}=\left\{\omega\in\Omega:\left(\omega_{1},\omega_{2},\ldots,\omega_{n}\right)\in
B^{n}\right\}$. Al conjunto $B_{n}$ se le llama {\em cilindro} con
base $B^{n}$, el cilindro es llamado medible si
$B^{n}\in\prod_{i=1}^{\infty}\mathcal{F}_{i}$.
\end{Def}


\begin{Def}\label{Def.Proc.Adaptado}[TSP, Ash \cite{RBA}]
Sea $X\left(t\right),t\geq0$ proceso estoc\'astico, el proceso es
adaptado a la familia de $\sigma$-\'algebras $\mathcal{F}_{t}$,
para $t\geq0$, si para $s<t$ implica que
$\mathcal{F}_{s}\subset\mathcal{F}_{t}$, y $X\left(t\right)$ es
$\mathcal{F}_{t}$-medible para cada $t$. Si no se especifica
$\mathcal{F}_{t}$ entonces se toma $\mathcal{F}_{t}$ como
$\mathcal{F}\left(X\left(s\right),s\leq t\right)$, la m\'as
peque\~na $\sigma$-\'algebra de subconjuntos de $\Omega$ que hace
que cada $X\left(s\right)$, con $s\leq t$ sea Borel medible.
\end{Def}


\begin{Def}\label{Def.Tiempo.Paro}[TSP, Ash \cite{RBA}]
Sea $\left\{\mathcal{F}\left(t\right),t\geq0\right\}$ familia
creciente de sub $\sigma$-\'algebras. es decir,
$\mathcal{F}\left(s\right)\subset\mathcal{F}\left(t\right)$ para
$s\leq t$. Un tiempo de paro para $\mathcal{F}\left(t\right)$ es
una funci\'on $T:\Omega\rightarrow\left[0,\infty\right]$ tal que
$\left\{T\leq t\right\}\in\mathcal{F}\left(t\right)$ para cada
$t\geq0$. Un tiempo de paro para el proceso estoc\'astico
$X\left(t\right),t\geq0$ es un tiempo de paro para las
$\sigma$-\'algebras
$\mathcal{F}\left(t\right)=\mathcal{F}\left(X\left(s\right)\right)$.
\end{Def}

\begin{Def}
Sea $X\left(t\right),t\geq0$ proceso estoc\'astico, con
$\left(S,\chi\right)$ espacio de estados. Se dice que el proceso
es adaptado a $\left\{\mathcal{F}\left(t\right)\right\}$, es
decir, si para cualquier $s,t\in I$, $I$ conjunto de \'indices,
$s<t$, se tiene que
$\mathcal{F}\left(s\right)\subset\mathcal{F}\left(t\right)$ y
$X\left(t\right)$ es $\mathcal{F}\left(t\right)$-medible,
\end{Def}

\begin{Def}
Sea $X\left(t\right),t\geq0$ proceso estoc\'astico, se dice que es
un Proceso de Markov relativo a $\mathcal{F}\left(t\right)$ o que
$\left\{X\left(t\right),\mathcal{F}\left(t\right)\right\}$ es de
Markov si y s\'olo si para cualquier conjunto $B\in\chi$,  y
$s,t\in I$, $s<t$ se cumple que
\begin{equation}\label{Prop.Markov}
P\left\{X\left(t\right)\in
B|\mathcal{F}\left(s\right)\right\}=P\left\{X\left(t\right)\in
B|X\left(s\right)\right\}.
\end{equation}
\end{Def}
\begin{Note}
Si se dice que $\left\{X\left(t\right)\right\}$ es un Proceso de
Markov sin mencionar $\mathcal{F}\left(t\right)$, se asumir\'a que
\begin{eqnarray*}
\mathcal{F}\left(t\right)=\mathcal{F}_{0}\left(t\right)=\mathcal{F}\left(X\left(r\right),r\leq
t\right),
\end{eqnarray*}
entonces la ecuaci\'on (\ref{Prop.Markov}) se puede escribir como
\begin{equation}
P\left\{X\left(t\right)\in B|X\left(r\right),r\leq s\right\} =
P\left\{X\left(t\right)\in B|X\left(s\right)\right\}
\end{equation}
\end{Note}
%_______________________________________________________________
\subsection{Procesos de Estados de Markov}
%_______________________________________________________________

\begin{Teo}
Sea $\left(X_{n},\mathcal{F}_{n},n=0,1,\ldots,\right\}$ Proceso de
Markov con espacio de estados $\left(S_{0},\chi_{0}\right)$
generado por una distribuici\'on inicial $P_{o}$ y probabilidad de
transici\'on $p_{mn}$, para $m,n=0,1,\ldots,$ $m<n$, que por
notaci\'on se escribir\'a como $p\left(m,n,x,B\right)\rightarrow
p_{mn}\left(x,B\right)$. Sea $S$ tiempo de paro relativo a la
$\sigma$-\'algebra $\mathcal{F}_{n}$. Sea $T$ funci\'on medible,
$T:\Omega\rightarrow\left\{0,1,\ldots,\right\}$. Sup\'ongase que
$T\geq S$, entonces $T$ es tiempo de paro. Si $B\in\chi_{0}$,
entonces
\begin{equation}\label{Prop.Fuerte.Markov}
P\left\{X\left(T\right)\in
B,T<\infty|\mathcal{F}\left(S\right)\right\} =
p\left(S,T,X\left(s\right),B\right)
\end{equation}
en $\left\{T<\infty\right\}$.
\end{Teo}


Sea $K$ conjunto numerable y sea $d:K\rightarrow\nat$ funci\'on.
Para $v\in K$, $M_{v}$ es un conjunto abierto de
$\rea^{d\left(v\right)}$. Entonces \[E=\bigcup_{v\in
K}M_{v}=\left\{\left(v,\zeta\right):v\in K,\zeta\in
M_{v}\right\}.\]

Sea $\mathcal{E}$ la clase de conjuntos medibles en $E$:
\[\mathcal{E}=\left\{\bigcup_{v\in K}A_{v}:A_{v}\in \mathcal{M}_{v}\right\}.\]

donde $\mathcal{M}$ son los conjuntos de Borel de $M_{v}$.
Entonces $\left(E,\mathcal{E}\right)$ es un espacio de Borel. El
estado del proceso se denotar\'a por
$\mathbf{x}_{t}=\left(v_{t},\zeta_{t}\right)$. La distribuci\'on
de $\left(\mathbf{x}_{t}\right)$ est\'a determinada por por los
siguientes objetos:

\begin{itemize}
\item[i)] Los campos vectoriales $\left(\mathcal{H}_{v},v\in
K\right)$. \item[ii)] Una funci\'on medible $\lambda:E\rightarrow
\rea_{+}$. \item[iii)] Una medida de transici\'on
$Q:\mathcal{E}\times\left(E\cup\Gamma^{*}\right)\rightarrow\left[0,1\right]$
donde
\begin{equation}
\Gamma^{*}=\bigcup_{v\in K}\partial^{*}M_{v}.
\end{equation}
y
\begin{equation}
\partial^{*}M_{v}=\left\{z\in\partial M_{v}:\mathbf{\mathbf{\phi}_{v}\left(t,\zeta\right)=\mathbf{z}}\textrm{ para alguna }\left(t,\zeta\right)\in\rea_{+}\times M_{v}\right\}.
\end{equation}
$\partial M_{v}$ denota  la frontera de $M_{v}$.
\end{itemize}

El campo vectorial $\left(\mathcal{H}_{v},v\in K\right)$ se supone
tal que para cada $\mathbf{z}\in M_{v}$ existe una \'unica curva
integral $\mathbf{\phi}_{v}\left(t,\zeta\right)$ que satisface la
ecuaci\'on

\begin{equation}
\frac{d}{dt}f\left(\zeta_{t}\right)=\mathcal{H}f\left(\zeta_{t}\right),
\end{equation}
con $\zeta_{0}=\mathbf{z}$, para cualquier funci\'on suave
$f:\rea^{d}\rightarrow\rea$ y $\mathcal{H}$ denota el operador
diferencial de primer orden, con $\mathcal{H}=\mathcal{H}_{v}$ y
$\zeta_{t}=\mathbf{\phi}\left(t,\mathbf{z}\right)$. Adem\'as se
supone que $\mathcal{H}_{v}$ es conservativo, es decir, las curvas
integrales est\'an definidas para todo $t>0$.

Para $\mathbf{x}=\left(v,\zeta\right)\in E$ se denota
\[t^{*}\mathbf{x}=inf\left\{t>0:\mathbf{\phi}_{v}\left(t,\zeta\right)\in\partial^{*}M_{v}\right\}\]

En lo que respecta a la funci\'on $\lambda$, se supondr\'a que
para cada $\left(v,\zeta\right)\in E$ existe un $\epsilon>0$ tal
que la funci\'on
$s\rightarrow\lambda\left(v,\phi_{v}\left(s,\zeta\right)\right)\in
E$ es integrable para $s\in\left[0,\epsilon\right)$. La medida de
transici\'on $Q\left(A;\mathbf{x}\right)$ es una funci\'on medible
de $\mathbf{x}$ para cada $A\in\mathcal{E}$, definida para
$\mathbf{x}\in E\cup\Gamma^{*}$ y es una medida de probabilidad en
$\left(E,\mathcal{E}\right)$ para cada $\mathbf{x}\in E$.

El movimiento del proceso $\left(\mathbf{x}_{t}\right)$ comenzando
en $\mathbf{x}=\left(n,\mathbf{z}\right)\in E$ se puede construir
de la siguiente manera, def\'inase la funci\'on $F$ por

\begin{equation}
F\left(t\right)=\left\{\begin{array}{ll}\\
exp\left(-\int_{0}^{t}\lambda\left(n,\phi_{n}\left(s,\mathbf{z}\right)\right)ds\right), & t<t^{*}\left(\mathbf{x}\right),\\
0, & t\geq t^{*}\left(\mathbf{x}\right)
\end{array}\right.
\end{equation}

Sea $T_{1}$ una variable aleatoria tal que
$\prob\left[T_{1}>t\right]=F\left(t\right)$, ahora sea la variable
aleatoria $\left(N,Z\right)$ con distribuici\'on
$Q\left(\cdot;\phi_{n}\left(T_{1},\mathbf{z}\right)\right)$. La
trayectoria de $\left(\mathbf{x}_{t}\right)$ para $t\leq T_{1}$ es
\begin{eqnarray*}
\mathbf{x}_{t}=\left(v_{t},\zeta_{t}\right)=\left\{\begin{array}{ll}
\left(n,\phi_{n}\left(t,\mathbf{z}\right)\right), & t<T_{1},\\
\left(N,\mathbf{Z}\right), & t=t_{1}.
\end{array}\right.
\end{eqnarray*}

Comenzando en $\mathbf{x}_{T_{1}}$ se selecciona el siguiente
tiempo de intersalto $T_{2}-T_{1}$ lugar del post-salto
$\mathbf{x}_{T_{2}}$ de manera similar y as\'i sucesivamente. Este
procedimiento nos da una trayectoria determinista por partes
$\mathbf{x}_{t}$ con tiempos de salto $T_{1},T_{2},\ldots$. Bajo
las condiciones enunciadas para $\lambda,T_{1}>0$  y
$T_{1}-T_{2}>0$ para cada $i$, con probabilidad 1. Se supone que
se cumple la siguiente condici\'on.

\begin{Sup}[Supuesto 3.1, Davis \cite{Davis}]\label{Sup3.1.Davis}
Sea $N_{t}:=\sum_{t}\indora_{\left(t\geq t\right)}$ el n\'umero de
saltos en $\left[0,t\right]$. Entonces
\begin{equation}
\esp\left[N_{t}\right]<\infty\textrm{ para toda }t.
\end{equation}
\end{Sup}

es un proceso de Markov, m\'as a\'un, es un Proceso Fuerte de
Markov, es decir, la Propiedad Fuerte de Markov\footnote{Revisar
p\'agina 362, y 364 de Davis \cite{Davis}.} se cumple para
cualquier tiempo de paro.
%_________________________________________________________________________
%\renewcommand{\refname}{PROCESOS ESTOC\'ASTICOS}
%\renewcommand{\appendixname}{PROCESOS ESTOC\'ASTICOS}
%\renewcommand{\appendixtocname}{PROCESOS ESTOC\'ASTICOS}
%\renewcommand{\appendixpagename}{PROCESOS ESTOC\'ASTICOS}
%\appendix
%\clearpage % o \cleardoublepage
%\addappheadtotoc
%\appendixpage
%_________________________________________________________________________
\subsection{Teor\'ia General de Procesos Estoc\'asticos}
%_________________________________________________________________________
En esta secci\'on se har\'an las siguientes consideraciones: $E$
es un espacio m\'etrico separable y la m\'etrica $d$ es compatible
con la topolog\'ia.

\begin{Def}
Una medida finita, $\lambda$ en la $\sigma$-\'algebra de Borel de
un espacio metrizable $E$ se dice cerrada si
\begin{equation}\label{Eq.A2.3}
\lambda\left(E\right)=sup\left\{\lambda\left(K\right):K\textrm{ es
compacto en }E\right\}.
\end{equation}
\end{Def}

\begin{Def}
$E$ es un espacio de Rad\'on si cada medida finita en
$\left(E,\mathcal{B}\left(E\right)\right)$ es regular interior o cerrada,
{\em tight}.
\end{Def}


El siguiente teorema nos permite tener una mejor caracterizaci\'on de los espacios de Rad\'on:
\begin{Teo}\label{Tma.A2.2}
Sea $E$ espacio separable metrizable. Entonces $E$ es de Rad\'on
si y s\'olo s\'i cada medida finita en
$\left(E,\mathcal{B}\left(E\right)\right)$ es cerrada.
\end{Teo}

%_________________________________________________________________________________________
\subsection{Propiedades de Markov}
%_________________________________________________________________________________________

Sea $E$ espacio de estados, tal que $E$ es un espacio de Rad\'on, $\mathcal{B}\left(E\right)$ $\sigma$-\'algebra de Borel en $E$, que se denotar\'a por $\mathcal{E}$.

Sea $\left(X,\mathcal{G},\prob\right)$ espacio de probabilidad,
$I\subset\rea$ conjunto de índices. Sea $\mathcal{F}_{\leq t}$ la
$\sigma$-\'algebra natural definida como
$\sigma\left\{f\left(X_{r}\right):r\in I, r\leq
t,f\in\mathcal{E}\right\}$. Se considerar\'a una
$\sigma$-\'algebra m\'as general\footnote{qu\'e se quiere decir
con el t\'ermino: m\'as general?}, $ \left(\mathcal{G}_{t}\right)$
tal que $\left(X_{t}\right)$ sea $\mathcal{E}$-adaptado.

\begin{Def}
Una familia $\left(P_{s,t}\right)$ de kernels de Markov en $\left(E,\mathcal{E}\right)$ indexada por pares $s,t\in I$, con $s\leq t$ es una funci\'on de transici\'on en $\ER$, si  para todo $r\leq s< t$ en $I$ y todo $x\in E$, $B\in\mathcal{E}$
\begin{equation}\label{Eq.Kernels}
P_{r,t}\left(x,B\right)=\int_{E}P_{r,s}\left(x,dy\right)P_{s,t}\left(y,B\right)\footnote{Ecuaci\'on de Chapman-Kolmogorov}.
\end{equation}
\end{Def}

Se dice que la funci\'on de transici\'on $\KM$ en $\ER$ es la funci\'on de transici\'on para un proceso $\PE$  con valores en $E$ y que satisface la propiedad de Markov\footnote{\begin{equation}\label{Eq.1.4.S}
\prob\left\{H|\mathcal{G}_{t}\right\}=\prob\left\{H|X_{t}\right\}\textrm{ }H\in p\mathcal{F}_{\geq t}.
\end{equation}} (\ref{Eq.1.4.S}) relativa a $\left(\mathcal{G}_{t}\right)$ si

\begin{equation}\label{Eq.1.6.S}
\prob\left\{f\left(X_{t}\right)|\mathcal{G}_{s}\right\}=P_{s,t}f\left(X_{t}\right)\textrm{ }s\leq t\in I,\textrm{ }f\in b\mathcal{E}.
\end{equation}

\begin{Def}
Una familia $\left(P_{t}\right)_{t\geq0}$ de kernels de Markov en $\ER$ es llamada {\em Semigrupo de Transici\'on de Markov} o {\em Semigrupo de Transici\'on} si
\[P_{t+s}f\left(x\right)=P_{t}\left(P_{s}f\right)\left(x\right),\textrm{ }t,s\geq0,\textrm{ }x\in E\textrm{ }f\in b\mathcal{E}\footnote{Definir los t\'ermino $b\mathcal{E}$ y $p\mathcal{E}$}.\]
\end{Def}
\begin{Note}
Si la funci\'on de transici\'on $\KM$ es llamada homog\'enea si $P_{s,t}=P_{t-s}$.
\end{Note}

Un proceso de Markov que satisface la ecuaci\'on (\ref{Eq.1.6.S}) con funci\'on de transici\'on homog\'enea $\left(P_{t}\right)$ tiene la propiedad caracter\'istica
\begin{equation}\label{Eq.1.8.S}
\prob\left\{f\left(X_{t+s}\right)|\mathcal{G}_{t}\right\}=P_{s}f\left(X_{t}\right)\textrm{ }t,s\geq0,\textrm{ }f\in b\mathcal{E}.
\end{equation}
La ecuaci\'on anterior es la {\em Propiedad Simple de Markov} de $X$ relativa a $\left(P_{t}\right)$.

En este sentido el proceso $\PE$ cumple con la propiedad de Markov (\ref{Eq.1.8.S}) relativa a $\left(\Omega,\mathcal{G},\mathcal{G}_{t},\prob\right)$ con semigrupo de transici\'on $\left(P_{t}\right)$.
%_________________________________________________________________________________________
\subsection{Primer Condici\'on de Regularidad}
%_________________________________________________________________________________________
%\newcommand{\EM}{\left(\Omega,\mathcal{G},\prob\right)}
%\newcommand{\E4}{\left(\Omega,\mathcal{G},\mathcal{G}_{t},\prob\right)}
\begin{Def}
Un proceso estoc\'astico $\PE$ definido en
$\left(\Omega,\mathcal{G},\prob\right)$ con valores en el espacio
topol\'ogico $E$ es continuo por la derecha si cada trayectoria
muestral $t\rightarrow X_{t}\left(w\right)$ es un mapeo continuo
por la derecha de $I$ en $E$.
\end{Def}

\begin{Def}[HD1]\label{Eq.2.1.S}
Un semigrupo de Markov $\left(P_{t}\right)$ en un espacio de
Rad\'on $E$ se dice que satisface la condici\'on {\em HD1} si,
dada una medida de probabilidad $\mu$ en $E$, existe una
$\sigma$-\'algebra $\mathcal{E^{*}}$ con
$\mathcal{E}\subset\mathcal{E}^{*}$ y
$P_{t}\left(b\mathcal{E}^{*}\right)\subset b\mathcal{E}^{*}$, y un
$\mathcal{E}^{*}$-proceso $E$-valuado continuo por la derecha
$\PE$ en alg\'un espacio de probabilidad filtrado
$\left(\Omega,\mathcal{G},\mathcal{G}_{t},\prob\right)$ tal que
$X=\left(\Omega,\mathcal{G},\mathcal{G}_{t},\prob\right)$ es de
Markov (Homog\'eneo) con semigrupo de transici\'on $(P_{t})$ y
distribuci\'on inicial $\mu$.
\end{Def}

Consid\'erese la colecci\'on de variables aleatorias $X_{t}$
definidas en alg\'un espacio de probabilidad, y una colecci\'on de
medidas $\mathbf{P}^{x}$ tales que
$\mathbf{P}^{x}\left\{X_{0}=x\right\}$, y bajo cualquier
$\mathbf{P}^{x}$, $X_{t}$ es de Markov con semigrupo
$\left(P_{t}\right)$. $\mathbf{P}^{x}$ puede considerarse como la
distribuci\'on condicional de $\mathbf{P}$ dado $X_{0}=x$.

\begin{Def}\label{Def.2.2.S}
Sea $E$ espacio de Rad\'on, $\SG$ semigrupo de Markov en $\ER$. La colecci\'on $\mathbf{X}=\left(\Omega,\mathcal{G},\mathcal{G}_{t},X_{t},\theta_{t},\CM\right)$ es un proceso $\mathcal{E}$-Markov continuo por la derecha simple, con espacio de estados $E$ y semigrupo de transici\'on $\SG$ en caso de que $\mathbf{X}$ satisfaga las siguientes condiciones:
\begin{itemize}
\item[i)] $\left(\Omega,\mathcal{G},\mathcal{G}_{t}\right)$ es un espacio de medida filtrado, y $X_{t}$ es un proceso $E$-valuado continuo por la derecha $\mathcal{E}^{*}$-adaptado a $\left(\mathcal{G}_{t}\right)$;

\item[ii)] $\left(\theta_{t}\right)_{t\geq0}$ es una colecci\'on de operadores {\em shift} para $X$, es decir, mapea $\Omega$ en s\'i mismo satisfaciendo para $t,s\geq0$,

\begin{equation}\label{Eq.Shift}
\theta_{t}\circ\theta_{s}=\theta_{t+s}\textrm{ y }X_{t}\circ\theta_{t}=X_{t+s};
\end{equation}

\item[iii)] Para cualquier $x\in E$,$\CM\left\{X_{0}=x\right\}=1$, y el proceso $\PE$ tiene la propiedad de Markov (\ref{Eq.1.8.S}) con semigrupo de transici\'on $\SG$ relativo a $\left(\Omega,\mathcal{G},\mathcal{G}_{t},\CM\right)$.
\end{itemize}
\end{Def}

\begin{Def}[HD2]\label{Eq.2.2.S}
Para cualquier $\alpha>0$ y cualquier $f\in S^{\alpha}$, el proceso $t\rightarrow f\left(X_{t}\right)$ es continuo por la derecha casi seguramente.
\end{Def}

\begin{Def}\label{Def.PD}
Un sistema $\mathbf{X}=\left(\Omega,\mathcal{G},\mathcal{G}_{t},X_{t},\theta_{t},\CM\right)$ es un proceso derecho en el espacio de Rad\'on $E$ con semigrupo de transici\'on $\SG$ provisto de:
\begin{itemize}
\item[i)] $\mathbf{X}$ es una realizaci\'on  continua por la derecha, \ref{Def.2.2.S}, de $\SG$.

\item[ii)] $\mathbf{X}$ satisface la condicion HD2, \ref{Eq.2.2.S}, relativa a $\mathcal{G}_{t}$.

\item[iii)] $\mathcal{G}_{t}$ es aumentado y continuo por la derecha.
\end{itemize}
\end{Def}


%_________________________________________________________________________
%\renewcommand{\refname}{MODELO DE FLUJO}
%\renewcommand{\appendixname}{MODELO DE FLUJO}
%\renewcommand{\appendixtocname}{MODELO DE FLUJO}
%\renewcommand{\appendixpagename}{MODELO DE FLUJO}
%\appendix
%\clearpage % o \cleardoublepage
%\addappheadtotoc
%\appendixpage

\subsection{Construcci\'on del Modelo de Flujo}


\begin{Lema}[Lema 4.2, Dai\cite{Dai}]\label{Lema4.2}
Sea $\left\{x_{n}\right\}\subset \mathbf{X}$ con
$|x_{n}|\rightarrow\infty$, conforme $n\rightarrow\infty$. Suponga
que
\[lim_{n\rightarrow\infty}\frac{1}{|x_{n}|}U\left(0\right)=\overline{U}\]
y
\[lim_{n\rightarrow\infty}\frac{1}{|x_{n}|}V\left(0\right)=\overline{V}.\]

Entonces, conforme $n\rightarrow\infty$, casi seguramente

\begin{equation}\label{E1.4.2}
\frac{1}{|x_{n}|}\Phi^{k}\left(\left[|x_{n}|t\right]\right)\rightarrow
P_{k}^{'}t\textrm{, u.o.c.,}
\end{equation}

\begin{equation}\label{E1.4.3}
\frac{1}{|x_{n}|}E^{x_{n}}_{k}\left(|x_{n}|t\right)\rightarrow
\alpha_{k}\left(t-\overline{U}_{k}\right)^{+}\textrm{, u.o.c.,}
\end{equation}

\begin{equation}\label{E1.4.4}
\frac{1}{|x_{n}|}S^{x_{n}}_{k}\left(|x_{n}|t\right)\rightarrow
\mu_{k}\left(t-\overline{V}_{k}\right)^{+}\textrm{, u.o.c.,}
\end{equation}

donde $\left[t\right]$ es la parte entera de $t$ y
$\mu_{k}=1/m_{k}=1/\esp\left[\eta_{k}\left(1\right)\right]$.
\end{Lema}

\begin{Lema}[Lema 4.3, Dai\cite{Dai}]\label{Lema.4.3}
Sea $\left\{x_{n}\right\}\subset \mathbf{X}$ con
$|x_{n}|\rightarrow\infty$, conforme $n\rightarrow\infty$. Suponga
que
\[lim_{n\rightarrow\infty}\frac{1}{|x_{n}|}U_{k}\left(0\right)=\overline{U}_{k}\]
y
\[lim_{n\rightarrow\infty}\frac{1}{|x_{n}|}V_{k}\left(0\right)=\overline{V}_{k}.\]
\begin{itemize}
\item[a)] Conforme $n\rightarrow\infty$ casi seguramente,
\[lim_{n\rightarrow\infty}\frac{1}{|x_{n}|}U^{x_{n}}_{k}\left(|x_{n}|t\right)=\left(\overline{U}_{k}-t\right)^{+}\textrm{, u.o.c.}\]
y
\[lim_{n\rightarrow\infty}\frac{1}{|x_{n}|}V^{x_{n}}_{k}\left(|x_{n}|t\right)=\left(\overline{V}_{k}-t\right)^{+}.\]

\item[b)] Para cada $t\geq0$ fijo,
\[\left\{\frac{1}{|x_{n}|}U^{x_{n}}_{k}\left(|x_{n}|t\right),|x_{n}|\geq1\right\}\]
y
\[\left\{\frac{1}{|x_{n}|}V^{x_{n}}_{k}\left(|x_{n}|t\right),|x_{n}|\geq1\right\}\]
\end{itemize}
son uniformemente convergentes.
\end{Lema}

Sea $S_{l}^{x}\left(t\right)$ el n\'umero total de servicios
completados de la clase $l$, si la clase $l$ est\'a dando $t$
unidades de tiempo de servicio. Sea $T_{l}^{x}\left(x\right)$ el
monto acumulado del tiempo de servicio que el servidor
$s\left(l\right)$ gasta en los usuarios de la clase $l$ al tiempo
$t$. Entonces $S_{l}^{x}\left(T_{l}^{x}\left(t\right)\right)$ es
el n\'umero total de servicios completados para la clase $l$ al
tiempo $t$. Una fracci\'on de estos usuarios,
$\Phi_{k}^{x}\left(S_{l}^{x}\left(T_{l}^{x}\left(t\right)\right)\right)$,
se convierte en usuarios de la clase $k$.\\

Entonces, dado lo anterior, se tiene la siguiente representaci\'on
para el proceso de la longitud de la cola:\\

\begin{equation}
Q_{k}^{x}\left(t\right)=Q_{k}^{x}\left(0\right)+E_{k}^{x}\left(t\right)+\sum_{l=1}^{K}\Phi_{k}^{l}\left(S_{l}^{x}\left(T_{l}^{x}\left(t\right)\right)\right)-S_{k}^{x}\left(T_{k}^{x}\left(t\right)\right)
\end{equation}
para $k=1,\ldots,K$. Para $i=1,\ldots,d$, sea
\[I_{i}^{x}\left(t\right)=t-\sum_{j\in C_{i}}T_{k}^{x}\left(t\right).\]

Entonces $I_{i}^{x}\left(t\right)$ es el monto acumulado del
tiempo que el servidor $i$ ha estado desocupado al tiempo $t$. Se
est\'a asumiendo que las disciplinas satisfacen la ley de
conservaci\'on del trabajo, es decir, el servidor $i$ est\'a en
pausa solamente cuando no hay usuarios en la estaci\'on $i$.
Entonces, se tiene que

\begin{equation}
\int_{0}^{\infty}\left(\sum_{k\in
C_{i}}Q_{k}^{x}\left(t\right)\right)dI_{i}^{x}\left(t\right)=0,
\end{equation}
para $i=1,\ldots,d$.\\

Hacer
\[T^{x}\left(t\right)=\left(T_{1}^{x}\left(t\right),\ldots,T_{K}^{x}\left(t\right)\right)^{'},\]
\[I^{x}\left(t\right)=\left(I_{1}^{x}\left(t\right),\ldots,I_{K}^{x}\left(t\right)\right)^{'}\]
y
\[S^{x}\left(T^{x}\left(t\right)\right)=\left(S_{1}^{x}\left(T_{1}^{x}\left(t\right)\right),\ldots,S_{K}^{x}\left(T_{K}^{x}\left(t\right)\right)\right)^{'}.\]

Para una disciplina que cumple con la ley de conservaci\'on del
trabajo, en forma vectorial, se tiene el siguiente conjunto de
ecuaciones

\begin{equation}\label{Eq.MF.1.3}
Q^{x}\left(t\right)=Q^{x}\left(0\right)+E^{x}\left(t\right)+\sum_{l=1}^{K}\Phi^{l}\left(S_{l}^{x}\left(T_{l}^{x}\left(t\right)\right)\right)-S^{x}\left(T^{x}\left(t\right)\right),\\
\end{equation}

\begin{equation}\label{Eq.MF.2.3}
Q^{x}\left(t\right)\geq0,\\
\end{equation}

\begin{equation}\label{Eq.MF.3.3}
T^{x}\left(0\right)=0,\textrm{ y }\overline{T}^{x}\left(t\right)\textrm{ es no decreciente},\\
\end{equation}

\begin{equation}\label{Eq.MF.4.3}
I^{x}\left(t\right)=et-CT^{x}\left(t\right)\textrm{ es no
decreciente}\\
\end{equation}

\begin{equation}\label{Eq.MF.5.3}
\int_{0}^{\infty}\left(CQ^{x}\left(t\right)\right)dI_{i}^{x}\left(t\right)=0,\\
\end{equation}

\begin{equation}\label{Eq.MF.6.3}
\textrm{Condiciones adicionales en
}\left(\overline{Q}^{x}\left(\cdot\right),\overline{T}^{x}\left(\cdot\right)\right)\textrm{
espec\'ificas de la disciplina de la cola,}
\end{equation}

donde $e$ es un vector de unos de dimensi\'on $d$, $C$ es la
matriz definida por
\[C_{ik}=\left\{\begin{array}{cc}
1,& S\left(k\right)=i,\\
0,& \textrm{ en otro caso}.\\
\end{array}\right.
\]
Es necesario enunciar el siguiente Teorema que se utilizar\'a para
el Teorema \ref{Tma.4.2.Dai}:
\begin{Teo}[Teorema 4.1, Dai \cite{Dai}]
Considere una disciplina que cumpla la ley de conservaci\'on del
trabajo, para casi todas las trayectorias muestrales $\omega$ y
cualquier sucesi\'on de estados iniciales
$\left\{x_{n}\right\}\subset \mathbf{X}$, con
$|x_{n}|\rightarrow\infty$, existe una subsucesi\'on
$\left\{x_{n_{j}}\right\}$ con $|x_{n_{j}}|\rightarrow\infty$ tal
que
\begin{equation}\label{Eq.4.15}
\frac{1}{|x_{n_{j}}|}\left(Q^{x_{n_{j}}}\left(0\right),U^{x_{n_{j}}}\left(0\right),V^{x_{n_{j}}}\left(0\right)\right)\rightarrow\left(\overline{Q}\left(0\right),\overline{U},\overline{V}\right),
\end{equation}

\begin{equation}\label{Eq.4.16}
\frac{1}{|x_{n_{j}}|}\left(Q^{x_{n_{j}}}\left(|x_{n_{j}}|t\right),T^{x_{n_{j}}}\left(|x_{n_{j}}|t\right)\right)\rightarrow\left(\overline{Q}\left(t\right),\overline{T}\left(t\right)\right)\textrm{
u.o.c.}
\end{equation}

Adem\'as,
$\left(\overline{Q}\left(t\right),\overline{T}\left(t\right)\right)$
satisface las siguientes ecuaciones:
\begin{equation}\label{Eq.MF.1.3a}
\overline{Q}\left(t\right)=Q\left(0\right)+\left(\alpha
t-\overline{U}\right)^{+}-\left(I-P\right)^{'}M^{-1}\left(\overline{T}\left(t\right)-\overline{V}\right)^{+},
\end{equation}

\begin{equation}\label{Eq.MF.2.3a}
\overline{Q}\left(t\right)\geq0,\\
\end{equation}

\begin{equation}\label{Eq.MF.3.3a}
\overline{T}\left(t\right)\textrm{ es no decreciente y comienza en cero},\\
\end{equation}

\begin{equation}\label{Eq.MF.4.3a}
\overline{I}\left(t\right)=et-C\overline{T}\left(t\right)\textrm{
es no decreciente,}\\
\end{equation}

\begin{equation}\label{Eq.MF.5.3a}
\int_{0}^{\infty}\left(C\overline{Q}\left(t\right)\right)d\overline{I}\left(t\right)=0,\\
\end{equation}

\begin{equation}\label{Eq.MF.6.3a}
\textrm{Condiciones adicionales en
}\left(\overline{Q}\left(\cdot\right),\overline{T}\left(\cdot\right)\right)\textrm{
especficas de la disciplina de la cola,}
\end{equation}
\end{Teo}


Propiedades importantes para el modelo de flujo retrasado:

\begin{Prop}
 Sea $\left(\overline{Q},\overline{T},\overline{T}^{0}\right)$ un flujo l\'imite de \ref{Eq.4.4} y suponga que cuando $x\rightarrow\infty$ a lo largo de
una subsucesi\'on
\[\left(\frac{1}{|x|}Q_{k}^{x}\left(0\right),\frac{1}{|x|}A_{k}^{x}\left(0\right),\frac{1}{|x|}B_{k}^{x}\left(0\right),\frac{1}{|x|}B_{k}^{x,0}\left(0\right)\right)\rightarrow\left(\overline{Q}_{k}\left(0\right),0,0,0\right)\]
para $k=1,\ldots,K$. EL flujo l\'imite tiene las siguientes
propiedades, donde las propiedades de la derivada se cumplen donde
la derivada exista:
\begin{itemize}
 \item[i)] Los vectores de tiempo ocupado $\overline{T}\left(t\right)$ y $\overline{T}^{0}\left(t\right)$ son crecientes y continuas con
$\overline{T}\left(0\right)=\overline{T}^{0}\left(0\right)=0$.
\item[ii)] Para todo $t\geq0$
\[\sum_{k=1}^{K}\left[\overline{T}_{k}\left(t\right)+\overline{T}_{k}^{0}\left(t\right)\right]=t\]
\item[iii)] Para todo $1\leq k\leq K$
\[\overline{Q}_{k}\left(t\right)=\overline{Q}_{k}\left(0\right)+\alpha_{k}t-\mu_{k}\overline{T}_{k}\left(t\right)\]
\item[iv)]  Para todo $1\leq k\leq K$
\[\dot{{\overline{T}}}_{k}\left(t\right)=\beta_{k}\] para $\overline{Q}_{k}\left(t\right)=0$.
\item[v)] Para todo $k,j$
\[\mu_{k}^{0}\overline{T}_{k}^{0}\left(t\right)=\mu_{j}^{0}\overline{T}_{j}^{0}\left(t\right)\]
\item[vi)]  Para todo $1\leq k\leq K$
\[\mu_{k}\dot{{\overline{T}}}_{k}\left(t\right)=l_{k}\mu_{k}^{0}\dot{{\overline{T}}}_{k}^{0}\left(t\right)\] para $\overline{Q}_{k}\left(t\right)>0$.
\end{itemize}
\end{Prop}

\begin{Teo}[Teorema 5.1: Ley Fuerte para Procesos de Conteo
\cite{Gut}]\label{Tma.5.1.Gut} Sea
$0<\mu<\esp\left(X_{1}\right]\leq\infty$. entonces

\begin{itemize}
\item[a)] $\frac{N\left(t\right)}{t}\rightarrow\frac{1}{\mu}$
a.s., cuando $t\rightarrow\infty$.


\item[b)]$\esp\left[\frac{N\left(t\right)}{t}\right]^{r}\rightarrow\frac{1}{\mu^{r}}$,
cuando $t\rightarrow\infty$ para todo $r>0$..
\end{itemize}
\end{Teo}


\begin{Prop}[Proposici\'on 5.3 \cite{DaiSean}]
Sea $X$ proceso de estados para la red de colas, y suponga que se
cumplen los supuestos (A1) y (A2), entonces para alguna constante
positiva $C_{p+1}<\infty$, $\delta>0$ y un conjunto compacto
$C\subset X$.

\begin{equation}\label{Eq.5.4}
\esp_{x}\left[\int_{0}^{\tau_{C}\left(\delta\right)}\left(1+|X\left(t\right)|^{p}\right)dt\right]\leq
C_{p+1}\left(1+|x|^{p+1}\right)
\end{equation}
\end{Prop}

\begin{Prop}[Proposici\'on 5.4 \cite{DaiSean}]
Sea $X$ un proceso de Markov Borel Derecho en $X$, sea
$f:X\leftarrow\rea_{+}$ y defina para alguna $\delta>0$, y un
conjunto cerrado $C\subset X$
\[V\left(x\right):=\esp_{x}\left[\int_{0}^{\tau_{C}\left(\delta\right)}f\left(X\left(t\right)\right)dt\right]\]
para $x\in X$. Si $V$ es finito en todas partes y uniformemente
acotada en $C$, entonces existe $k<\infty$ tal que
\begin{equation}\label{Eq.5.11}
\frac{1}{t}\esp_{x}\left[V\left(x\right)\right]+\frac{1}{t}\int_{0}^{t}\esp_{x}\left[f\left(X\left(s\right)\right)ds\right]\leq\frac{1}{t}V\left(x\right)+k,
\end{equation}
para $x\in X$ y $t>0$.
\end{Prop}


%_________________________________________________________________________
%\renewcommand{\refname}{Ap\'endice D}
%\renewcommand{\appendixname}{ESTABILIDAD}
%\renewcommand{\appendixtocname}{ESTABILIDAD}
%\renewcommand{\appendixpagename}{ESTABILIDAD}
%\appendix
%\clearpage % o \cleardoublepage
%\addappheadtotoc
%\appendixpage

\subsection{Estabilidad}

\begin{Def}[Definici\'on 3.2, Dai y Meyn \cite{DaiSean}]
El modelo de flujo retrasado de una disciplina de servicio en una
red con retraso
$\left(\overline{A}\left(0\right),\overline{B}\left(0\right)\right)\in\rea_{+}^{K+|A|}$
se define como el conjunto de ecuaciones dadas en
\ref{Eq.3.8}-\ref{Eq.3.13}, junto con la condici\'on:
\begin{equation}\label{CondAd.FluidModel}
\overline{Q}\left(t\right)=\overline{Q}\left(0\right)+\left(\alpha
t-\overline{A}\left(0\right)\right)^{+}-\left(I-P^{'}\right)M\left(\overline{T}\left(t\right)-\overline{B}\left(0\right)\right)^{+}
\end{equation}
\end{Def}

entonces si el modelo de flujo retrasado tambi\'en es estable:


\begin{Def}[Definici\'on 3.1, Dai y Meyn \cite{DaiSean}]
Un flujo l\'imite (retrasado) para una red bajo una disciplina de
servicio espec\'ifica se define como cualquier soluci\'on
 $\left(\overline{Q}\left(\cdot\right),\overline{T}\left(\cdot\right)\right)$ de las siguientes ecuaciones, donde
$\overline{Q}\left(t\right)=\left(\overline{Q}_{1}\left(t\right),\ldots,\overline{Q}_{K}\left(t\right)\right)^{'}$
y
$\overline{T}\left(t\right)=\left(\overline{T}_{1}\left(t\right),\ldots,\overline{T}_{K}\left(t\right)\right)^{'}$
\begin{equation}\label{Eq.3.8}
\overline{Q}_{k}\left(t\right)=\overline{Q}_{k}\left(0\right)+\alpha_{k}t-\mu_{k}\overline{T}_{k}\left(t\right)+\sum_{l=1}^{k}P_{lk}\mu_{l}\overline{T}_{l}\left(t\right)\\
\end{equation}
\begin{equation}\label{Eq.3.9}
\overline{Q}_{k}\left(t\right)\geq0\textrm{ para }k=1,2,\ldots,K,\\
\end{equation}
\begin{equation}\label{Eq.3.10}
\overline{T}_{k}\left(0\right)=0,\textrm{ y }\overline{T}_{k}\left(\cdot\right)\textrm{ es no decreciente},\\
\end{equation}
\begin{equation}\label{Eq.3.11}
\overline{I}_{i}\left(t\right)=t-\sum_{k\in C_{i}}\overline{T}_{k}\left(t\right)\textrm{ es no decreciente}\\
\end{equation}
\begin{equation}\label{Eq.3.12}
\overline{I}_{i}\left(\cdot\right)\textrm{ se incrementa al tiempo }t\textrm{ cuando }\sum_{k\in C_{i}}Q_{k}^{x}\left(t\right)dI_{i}^{x}\left(t\right)=0\\
\end{equation}
\begin{equation}\label{Eq.3.13}
\textrm{condiciones adicionales sobre
}\left(Q^{x}\left(\cdot\right),T^{x}\left(\cdot\right)\right)\textrm{
referentes a la disciplina de servicio}
\end{equation}
\end{Def}

\begin{Lema}[Lema 3.1 \cite{Chen}]\label{Lema3.1}
Si el modelo de flujo es estable, definido por las ecuaciones
(3.8)-(3.13), entonces el modelo de flujo retrasado tambin es
estable.
\end{Lema}

\begin{Teo}[Teorema 5.1 \cite{Chen}]\label{Tma.5.1.Chen}
La red de colas es estable si existe una constante $t_{0}$ que
depende de $\left(\alpha,\mu,T,U\right)$ y $V$ que satisfagan las
ecuaciones (5.1)-(5.5), $Z\left(t\right)=0$, para toda $t\geq
t_{0}$.
\end{Teo}

\begin{Prop}[Proposici\'on 5.1, Dai y Meyn \cite{DaiSean}]\label{Prop.5.1.DaiSean}
Suponga que los supuestos A1) y A2) son ciertos y que el modelo de flujo es estable. Entonces existe $t_{0}>0$ tal que
\begin{equation}
lim_{|x|\rightarrow\infty}\frac{1}{|x|^{p+1}}\esp_{x}\left[|X\left(t_{0}|x|\right)|^{p+1}\right]=0
\end{equation}
\end{Prop}

\begin{Lemma}[Lema 5.2, Dai y Meyn \cite{DaiSean}]\label{Lema.5.2.DaiSean}
 Sea $\left\{\zeta\left(k\right):k\in \mathbb{z}\right\}$ una sucesi\'on independiente e id\'enticamente distribuida que toma valores en $\left(0,\infty\right)$,
y sea
$E\left(t\right)=max\left(n\geq1:\zeta\left(1\right)+\cdots+\zeta\left(n-1\right)\leq
t\right)$. Si $\esp\left[\zeta\left(1\right)\right]<\infty$,
entonces para cualquier entero $r\geq1$
\begin{equation}
 lim_{t\rightarrow\infty}\esp\left[\left(\frac{E\left(t\right)}{t}\right)^{r}\right]=\left(\frac{1}{\esp\left[\zeta_{1}\right]}\right)^{r}.
\end{equation}
Luego, bajo estas condiciones:
\begin{itemize}
 \item[a)] para cualquier $\delta>0$, $\sup_{t\geq\delta}\esp\left[\left(\frac{E\left(t\right)}{t}\right)^{r}\right]<\infty$
\item[b)] las variables aleatorias
$\left\{\left(\frac{E\left(t\right)}{t}\right)^{r}:t\geq1\right\}$
son uniformemente integrables.
\end{itemize}
\end{Lemma}

\begin{Teo}[Teorema 5.5, Dai y Meyn \cite{DaiSean}]\label{Tma.5.5.DaiSean}
Suponga que los supuestos A1) y A2) se cumplen y que el modelo de
flujo es estable. Entonces existe una constante $\kappa_{p}$ tal
que
\begin{equation}
\frac{1}{t}\int_{0}^{t}\esp_{x}\left[|Q\left(s\right)|^{p}\right]ds\leq\kappa_{p}\left\{\frac{1}{t}|x|^{p+1}+1\right\}
\end{equation}
para $t>0$ y $x\in X$. En particular, para cada condici\'on
inicial
\begin{eqnarray*}
\limsup_{t\rightarrow\infty}\frac{1}{t}\int_{0}^{t}\esp_{x}\left[|Q\left(s\right)|^{p}\right]ds\leq\kappa_{p}.
\end{eqnarray*}
\end{Teo}

\begin{Teo}[Teorema 6.2, Dai y Meyn \cite{DaiSean}]\label{Tma.6.2.DaiSean}
Suponga que se cumplen los supuestos A1), A2) y A3) y que el
modelo de flujo es estable. Entonces se tiene que
\begin{equation}
\left\|P^{t}\left(x,\cdot\right)-\pi\left(\cdot\right)\right\|_{f_{p}}\textrm{,
}t\rightarrow\infty,x\in X.
\end{equation}
En particular para cada condici\'on inicial
\begin{eqnarray*}
\lim_{t\rightarrow\infty}\esp_{x}\left[|Q\left(t\right)|^{p}\right]=\esp_{\pi}\left[|Q\left(0\right)|^{p}\right]\leq\kappa_{r}
\end{eqnarray*}
\end{Teo}
\begin{Teo}[Teorema 6.3, Dai y Meyn \cite{DaiSean}]\label{Tma.6.3.DaiSean}
Suponga que se cumplen los supuestos A1), A2) y A3) y que el
modelo de flujo es estable. Entonces con
$f\left(x\right)=f_{1}\left(x\right)$ se tiene
\begin{equation}
\lim_{t\rightarrow\infty}t^{p-1}\left\|P^{t}\left(x,\cdot\right)-\pi\left(\cdot\right)\right\|_{f}=0.
\end{equation}
En particular para cada condici\'on inicial
\begin{eqnarray*}
\lim_{t\rightarrow\infty}t^{p-1}|\esp_{x}\left[Q\left(t\right)\right]-\esp_{\pi}\left[Q\left(0\right)\right]|=0.
\end{eqnarray*}
\end{Teo}

\begin{Teo}[Teorema 6.4, Dai y Meyn \cite{DaiSean}]\label{Tma.6.4.DaiSean}
Suponga que se cumplen los supuestos A1), A2) y A3) y que el
modelo de flujo es estable. Sea $\nu$ cualquier distribuci\'on de
probabilidad en $\left(X,\mathcal{B}_{X}\right)$, y $\pi$ la
distribuci\'on estacionaria de $X$.
\begin{itemize}
\item[i)] Para cualquier $f:X\leftarrow\rea_{+}$
\begin{equation}
\lim_{t\rightarrow\infty}\frac{1}{t}\int_{o}^{t}f\left(X\left(s\right)\right)ds=\pi\left(f\right):=\int
f\left(x\right)\pi\left(dx\right)
\end{equation}
$\prob$-c.s.

\item[ii)] Para cualquier $f:X\leftarrow\rea_{+}$ con
$\pi\left(|f|\right)<\infty$, la ecuaci\'on anterior se cumple.
\end{itemize}
\end{Teo}

\begin{Teo}[Teorema 2.2, Down \cite{Down}]\label{Tma2.2.Down}
Suponga que el fluido modelo es inestable en el sentido de que
para alguna $\epsilon_{0},c_{0}\geq0$,
\begin{equation}\label{Eq.Inestability}
|Q\left(T\right)|\geq\epsilon_{0}T-c_{0}\textrm{,   }T\geq0,
\end{equation}
para cualquier condici\'on inicial $Q\left(0\right)$, con
$|Q\left(0\right)|=1$. Entonces para cualquier $0<q\leq1$, existe
$B<0$ tal que para cualquier $|x|\geq B$,
\begin{equation}
\prob_{x}\left\{\mathbb{X}\rightarrow\infty\right\}\geq q.
\end{equation}
\end{Teo}


\begin{Def}
Sea $X$ un conjunto y $\mathcal{F}$ una $\sigma$-\'algebra de
subconjuntos de $X$, la pareja $\left(X,\mathcal{F}\right)$ es
llamado espacio medible. Un subconjunto $A$ de $X$ es llamado
medible, o medible con respecto a $\mathcal{F}$, si
$A\in\mathcal{F}$.
\end{Def}

\begin{Def}
Sea $\left(X,\mathcal{F},\mu\right)$ espacio de medida. Se dice
que la medida $\mu$ es $\sigma$-finita si se puede escribir
$X=\bigcup_{n\geq1}X_{n}$ con $X_{n}\in\mathcal{F}$ y
$\mu\left(X_{n}\right)<\infty$.
\end{Def}

\begin{Def}\label{Cto.Borel}
Sea $X$ el conjunto de los \'umeros reales $\rea$. El \'algebra de
Borel es la $\sigma$-\'algebra $B$ generada por los intervalos
abiertos $\left(a,b\right)\in\rea$. Cualquier conjunto en $B$ es
llamado {\em Conjunto de Borel}.
\end{Def}

\begin{Def}\label{Funcion.Medible}
Una funci\'on $f:X\rightarrow\rea$, es medible si para cualquier
n\'umero real $\alpha$ el conjunto
\[\left\{x\in X:f\left(x\right)>\alpha\right\}\]
pertenece a $X$. Equivalentemente, se dice que $f$ es medible si
\[f^{-1}\left(\left(\alpha,\infty\right)\right)=\left\{x\in X:f\left(x\right)>\alpha\right\}\in\mathcal{F}.\]
\end{Def}


\begin{Def}\label{Def.Cilindros}
Sean $\left(\Omega_{i},\mathcal{F}_{i}\right)$, $i=1,2,\ldots,$
espacios medibles y $\Omega=\prod_{i=1}^{\infty}\Omega_{i}$ el
conjunto de todas las sucesiones
$\left(\omega_{1},\omega_{2},\ldots,\right)$ tales que
$\omega_{i}\in\Omega_{i}$, $i=1,2,\ldots,$. Si
$B^{n}\subset\prod_{i=1}^{\infty}\Omega_{i}$, definimos
$B_{n}=\left\{\omega\in\Omega:\left(\omega_{1},\omega_{2},\ldots,\omega_{n}\right)\in
B^{n}\right\}$. Al conjunto $B_{n}$ se le llama {\em cilindro} con
base $B^{n}$, el cilindro es llamado medible si
$B^{n}\in\prod_{i=1}^{\infty}\mathcal{F}_{i}$.
\end{Def}


\begin{Def}\label{Def.Proc.Adaptado}[TSP, Ash \cite{RBA}]
Sea $X\left(t\right),t\geq0$ proceso estoc\'astico, el proceso es
adaptado a la familia de $\sigma$-\'algebras $\mathcal{F}_{t}$,
para $t\geq0$, si para $s<t$ implica que
$\mathcal{F}_{s}\subset\mathcal{F}_{t}$, y $X\left(t\right)$ es
$\mathcal{F}_{t}$-medible para cada $t$. Si no se especifica
$\mathcal{F}_{t}$ entonces se toma $\mathcal{F}_{t}$ como
$\mathcal{F}\left(X\left(s\right),s\leq t\right)$, la m\'as
peque\~na $\sigma$-\'algebra de subconjuntos de $\Omega$ que hace
que cada $X\left(s\right)$, con $s\leq t$ sea Borel medible.
\end{Def}


\begin{Def}\label{Def.Tiempo.Paro}[TSP, Ash \cite{RBA}]
Sea $\left\{\mathcal{F}\left(t\right),t\geq0\right\}$ familia
creciente de sub $\sigma$-\'algebras. es decir,
$\mathcal{F}\left(s\right)\subset\mathcal{F}\left(t\right)$ para
$s\leq t$. Un tiempo de paro para $\mathcal{F}\left(t\right)$ es
una funci\'on $T:\Omega\rightarrow\left[0,\infty\right]$ tal que
$\left\{T\leq t\right\}\in\mathcal{F}\left(t\right)$ para cada
$t\geq0$. Un tiempo de paro para el proceso estoc\'astico
$X\left(t\right),t\geq0$ es un tiempo de paro para las
$\sigma$-\'algebras
$\mathcal{F}\left(t\right)=\mathcal{F}\left(X\left(s\right)\right)$.
\end{Def}

\begin{Def}
Sea $X\left(t\right),t\geq0$ proceso estoc\'astico, con
$\left(S,\chi\right)$ espacio de estados. Se dice que el proceso
es adaptado a $\left\{\mathcal{F}\left(t\right)\right\}$, es
decir, si para cualquier $s,t\in I$, $I$ conjunto de \'indices,
$s<t$, se tiene que
$\mathcal{F}\left(s\right)\subset\mathcal{F}\left(t\right)$ y
$X\left(t\right)$ es $\mathcal{F}\left(t\right)$-medible,
\end{Def}

\begin{Def}
Sea $X\left(t\right),t\geq0$ proceso estoc\'astico, se dice que es
un Proceso de Markov relativo a $\mathcal{F}\left(t\right)$ o que
$\left\{X\left(t\right),\mathcal{F}\left(t\right)\right\}$ es de
Markov si y s\'olo si para cualquier conjunto $B\in\chi$,  y
$s,t\in I$, $s<t$ se cumple que
\begin{equation}\label{Prop.Markov}
P\left\{X\left(t\right)\in
B|\mathcal{F}\left(s\right)\right\}=P\left\{X\left(t\right)\in
B|X\left(s\right)\right\}.
\end{equation}
\end{Def}
\begin{Note}
Si se dice que $\left\{X\left(t\right)\right\}$ es un Proceso de
Markov sin mencionar $\mathcal{F}\left(t\right)$, se asumir\'a que
\begin{eqnarray*}
\mathcal{F}\left(t\right)=\mathcal{F}_{0}\left(t\right)=\mathcal{F}\left(X\left(r\right),r\leq
t\right),
\end{eqnarray*}
entonces la ecuaci\'on (\ref{Prop.Markov}) se puede escribir como
\begin{equation}
P\left\{X\left(t\right)\in B|X\left(r\right),r\leq s\right\} =
P\left\{X\left(t\right)\in B|X\left(s\right)\right\}
\end{equation}
\end{Note}

\begin{Teo}
Sea $\left(X_{n},\mathcal{F}_{n},n=0,1,\ldots,\right\}$ Proceso de
Markov con espacio de estados $\left(S_{0},\chi_{0}\right)$
generado por una distribuici\'on inicial $P_{o}$ y probabilidad de
transici\'on $p_{mn}$, para $m,n=0,1,\ldots,$ $m<n$, que por
notaci\'on se escribir\'a como $p\left(m,n,x,B\right)\rightarrow
p_{mn}\left(x,B\right)$. Sea $S$ tiempo de paro relativo a la
$\sigma$-\'algebra $\mathcal{F}_{n}$. Sea $T$ funci\'on medible,
$T:\Omega\rightarrow\left\{0,1,\ldots,\right\}$. Sup\'ongase que
$T\geq S$, entonces $T$ es tiempo de paro. Si $B\in\chi_{0}$,
entonces
\begin{equation}\label{Prop.Fuerte.Markov}
P\left\{X\left(T\right)\in
B,T<\infty|\mathcal{F}\left(S\right)\right\} =
p\left(S,T,X\left(s\right),B\right)
\end{equation}
en $\left\{T<\infty\right\}$.
\end{Teo}


Sea $K$ conjunto numerable y sea $d:K\rightarrow\nat$ funci\'on.
Para $v\in K$, $M_{v}$ es un conjunto abierto de
$\rea^{d\left(v\right)}$. Entonces \[E=\cup_{v\in
K}M_{v}=\left\{\left(v,\zeta\right):v\in K,\zeta\in
M_{v}\right\}.\]

Sea $\mathcal{E}$ la clase de conjuntos medibles en $E$:
\[\mathcal{E}=\left\{\cup_{v\in K}A_{v}:A_{v}\in \mathcal{M}_{v}\right\}.\]

donde $\mathcal{M}$ son los conjuntos de Borel de $M_{v}$.
Entonces $\left(E,\mathcal{E}\right)$ es un espacio de Borel. El
estado del proceso se denotar\'a por
$\mathbf{x}_{t}=\left(v_{t},\zeta_{t}\right)$. La distribuci\'on
de $\left(\mathbf{x}_{t}\right)$ est\'a determinada por por los
siguientes objetos:

\begin{itemize}
\item[i)] Los campos vectoriales $\left(\mathcal{H}_{v},v\in
K\right)$. \item[ii)] Una funci\'on medible $\lambda:E\rightarrow
\rea_{+}$. \item[iii)] Una medida de transici\'on
$Q:\mathcal{E}\times\left(E\cup\Gamma^{*}\right)\rightarrow\left[0,1\right]$
donde
\begin{equation}
\Gamma^{*}=\cup_{v\in K}\partial^{*}M_{v}.
\end{equation}
y
\begin{equation}
\partial^{*}M_{v}=\left\{z\in\partial M_{v}:\mathbf{\mathbf{\phi}_{v}\left(t,\zeta\right)=\mathbf{z}}\textrm{ para alguna }\left(t,\zeta\right)\in\rea_{+}\times M_{v}\right\}.
\end{equation}
$\partial M_{v}$ denota  la frontera de $M_{v}$.
\end{itemize}

El campo vectorial $\left(\mathcal{H}_{v},v\in K\right)$ se supone
tal que para cada $\mathbf{z}\in M_{v}$ existe una \'unica curva
integral $\mathbf{\phi}_{v}\left(t,\zeta\right)$ que satisface la
ecuaci\'on

\begin{equation}
\frac{d}{dt}f\left(\zeta_{t}\right)=\mathcal{H}f\left(\zeta_{t}\right),
\end{equation}
con $\zeta_{0}=\mathbf{z}$, para cualquier funci\'on suave
$f:\rea^{d}\rightarrow\rea$ y $\mathcal{H}$ denota el operador
diferencial de primer orden, con $\mathcal{H}=\mathcal{H}_{v}$ y
$\zeta_{t}=\mathbf{\phi}\left(t,\mathbf{z}\right)$. Adem\'as se
supone que $\mathcal{H}_{v}$ es conservativo, es decir, las curvas
integrales est\'an definidas para todo $t>0$.

Para $\mathbf{x}=\left(v,\zeta\right)\in E$ se denota
\[t^{*}\mathbf{x}=inf\left\{t>0:\mathbf{\phi}_{v}\left(t,\zeta\right)\in\partial^{*}M_{v}\right\}\]

En lo que respecta a la funci\'on $\lambda$, se supondr\'a que
para cada $\left(v,\zeta\right)\in E$ existe un $\epsilon>0$ tal
que la funci\'on
$s\rightarrow\lambda\left(v,\phi_{v}\left(s,\zeta\right)\right)\in
E$ es integrable para $s\in\left[0,\epsilon\right)$. La medida de
transici\'on $Q\left(A;\mathbf{x}\right)$ es una funci\'on medible
de $\mathbf{x}$ para cada $A\in\mathcal{E}$, definida para
$\mathbf{x}\in E\cup\Gamma^{*}$ y es una medida de probabilidad en
$\left(E,\mathcal{E}\right)$ para cada $\mathbf{x}\in E$.

El movimiento del proceso $\left(\mathbf{x}_{t}\right)$ comenzando
en $\mathbf{x}=\left(n,\mathbf{z}\right)\in E$ se puede construir
de la siguiente manera, def\'inase la funci\'on $F$ por

\begin{equation}
F\left(t\right)=\left\{\begin{array}{ll}\\
exp\left(-\int_{0}^{t}\lambda\left(n,\phi_{n}\left(s,\mathbf{z}\right)\right)ds\right), & t<t^{*}\left(\mathbf{x}\right),\\
0, & t\geq t^{*}\left(\mathbf{x}\right)
\end{array}\right.
\end{equation}

Sea $T_{1}$ una variable aleatoria tal que
$\prob\left[T_{1}>t\right]=F\left(t\right)$, ahora sea la variable
aleatoria $\left(N,Z\right)$ con distribuici\'on
$Q\left(\cdot;\phi_{n}\left(T_{1},\mathbf{z}\right)\right)$. La
trayectoria de $\left(\mathbf{x}_{t}\right)$ para $t\leq T_{1}$
es\footnote{Revisar p\'agina 362, y 364 de Davis \cite{Davis}.}
\begin{eqnarray*}
\mathbf{x}_{t}=\left(v_{t},\zeta_{t}\right)=\left\{\begin{array}{ll}
\left(n,\phi_{n}\left(t,\mathbf{z}\right)\right), & t<T_{1},\\
\left(N,\mathbf{Z}\right), & t=t_{1}.
\end{array}\right.
\end{eqnarray*}

Comenzando en $\mathbf{x}_{T_{1}}$ se selecciona el siguiente
tiempo de intersalto $T_{2}-T_{1}$ lugar del post-salto
$\mathbf{x}_{T_{2}}$ de manera similar y as\'i sucesivamente. Este
procedimiento nos da una trayectoria determinista por partes
$\mathbf{x}_{t}$ con tiempos de salto $T_{1},T_{2},\ldots$. Bajo
las condiciones enunciadas para $\lambda,T_{1}>0$  y
$T_{1}-T_{2}>0$ para cada $i$, con probabilidad 1. Se supone que
se cumple la siquiente condici\'on.

\begin{Sup}[Supuesto 3.1, Davis \cite{Davis}]\label{Sup3.1.Davis}
Sea $N_{t}:=\sum_{t}\indora_{\left(t\geq t\right)}$ el n\'umero de
saltos en $\left[0,t\right]$. Entonces
\begin{equation}
\esp\left[N_{t}\right]<\infty\textrm{ para toda }t.
\end{equation}
\end{Sup}

es un proceso de Markov, m\'as a\'un, es un Proceso Fuerte de
Markov, es decir, la Propiedad Fuerte de Markov se cumple para
cualquier tiempo de paro.
%_________________________________________________________________________

En esta secci\'on se har\'an las siguientes consideraciones: $E$
es un espacio m\'etrico separable y la m\'etrica $d$ es compatible
con la topolog\'ia.


\begin{Def}
Un espacio topol\'ogico $E$ es llamado {\em Luisin} si es
homeomorfo a un subconjunto de Borel de un espacio m\'etrico
compacto.
\end{Def}

\begin{Def}
Un espacio topol\'ogico $E$ es llamado de {\em Rad\'on} si es
homeomorfo a un subconjunto universalmente medible de un espacio
m\'etrico compacto.
\end{Def}

Equivalentemente, la definici\'on de un espacio de Rad\'on puede
encontrarse en los siguientes t\'erminos:


\begin{Def}
$E$ es un espacio de Rad\'on si cada medida finita en
$\left(E,\mathcal{B}\left(E\right)\right)$ es regular interior o cerrada,
{\em tight}.
\end{Def}

\begin{Def}
Una medida finita, $\lambda$ en la $\sigma$-\'algebra de Borel de
un espacio metrizable $E$ se dice cerrada si
\begin{equation}\label{Eq.A2.3}
\lambda\left(E\right)=sup\left\{\lambda\left(K\right):K\textrm{ es
compacto en }E\right\}.
\end{equation}
\end{Def}

El siguiente teorema nos permite tener una mejor caracterizaci\'on de los espacios de Rad\'on:
\begin{Teo}\label{Tma.A2.2}
Sea $E$ espacio separable metrizable. Entonces $E$ es Radoniano si y s\'olo s\'i cada medida finita en $\left(E,\mathcal{B}\left(E\right)\right)$ es cerrada.
\end{Teo}

%_________________________________________________________________________________________
\subsection{Propiedades de Markov}
%_________________________________________________________________________________________

Sea $E$ espacio de estados, tal que $E$ es un espacio de Rad\'on, $\mathcal{B}\left(E\right)$ $\sigma$-\'algebra de Borel en $E$, que se denotar\'a por $\mathcal{E}$.

Sea $\left(X,\mathcal{G},\prob\right)$ espacio de probabilidad, $I\subset\rea$ conjunto de índices. Sea $\mathcal{F}_{\leq t}$ la $\sigma$-\'algebra natural definida como $\sigma\left\{f\left(X_{r}\right):r\in I, rleq t,f\in\mathcal{E}\right\}$. Se considerar\'a una $\sigma$-\'algebra m\'as general, $ \left(\mathcal{G}_{t}\right)$ tal que $\left(X_{t}\right)$ sea $\mathcal{E}$-adaptado.

\begin{Def}
Una familia $\left(P_{s,t}\right)$ de kernels de Markov en $\left(E,\mathcal{E}\right)$ indexada por pares $s,t\in I$, con $s\leq t$ es una funci\'on de transici\'on en $\ER$, si  para todo $r\leq s< t$ en $I$ y todo $x\in E$, $B\in\mathcal{E}$
\begin{equation}\label{Eq.Kernels}
P_{r,t}\left(x,B\right)=\int_{E}P_{r,s}\left(x,dy\right)P_{s,t}\left(y,B\right)\footnote{Ecuaci\'on de Chapman-Kolmogorov}.
\end{equation}
\end{Def}

Se dice que la funci\'on de transici\'on $\KM$ en $\ER$ es la funci\'on de transici\'on para un proceso $\PE$  con valores en $E$ y que satisface la propiedad de Markov\footnote{\begin{equation}\label{Eq.1.4.S}
\prob\left\{H|\mathcal{G}_{t}\right\}=\prob\left\{H|X_{t}\right\}\textrm{ }H\in p\mathcal{F}_{\geq t}.
\end{equation}} (\ref{Eq.1.4.S}) relativa a $\left(\mathcal{G}_{t}\right)$ si 

\begin{equation}\label{Eq.1.6.S}
\prob\left\{f\left(X_{t}\right)|\mathcal{G}_{s}\right\}=P_{s,t}f\left(X_{t}\right)\textrm{ }s\leq t\in I,\textrm{ }f\in b\mathcal{E}.
\end{equation}

\begin{Def}
Una familia $\left(P_{t}\right)_{t\geq0}$ de kernels de Markov en $\ER$ es llamada {\em Semigrupo de Transici\'on de Markov} o {\em Semigrupo de Transici\'on} si
\[P_{t+s}f\left(x\right)=P_{t}\left(P_{s}f\right)\left(x\right),\textrm{ }t,s\geq0,\textrm{ }x\in E\textrm{ }f\in b\mathcal{E}.\]
\end{Def}
\begin{Note}
Si la funci\'on de transici\'on $\KM$ es llamada homog\'enea si $P_{s,t}=P_{t-s}$.
\end{Note}

Un proceso de Markov que satisface la ecuaci\'on (\ref{Eq.1.6.S}) con funci\'on de transici\'on homog\'enea $\left(P_{t}\right)$ tiene la propiedad caracter\'istica
\begin{equation}\label{Eq.1.8.S}
\prob\left\{f\left(X_{t+s}\right)|\mathcal{G}_{t}\right\}=P_{s}f\left(X_{t}\right)\textrm{ }t,s\geq0,\textrm{ }f\in b\mathcal{E}.
\end{equation}
La ecuaci\'on anterior es la {\em Propiedad Simple de Markov} de $X$ relativa a $\left(P_{t}\right)$.

En este sentido el proceso $\PE$ cumple con la propiedad de Markov (\ref{Eq.1.8.S}) relativa a $\left(\Omega,\mathcal{G},\mathcal{G}_{t},\prob\right)$ con semigrupo de transici\'on $\left(P_{t}\right)$.
%_________________________________________________________________________________________
\subsection{Primer Condici\'on de Regularidad}
%_________________________________________________________________________________________
%\newcommand{\EM}{\left(\Omega,\mathcal{G},\prob\right)}
%\newcommand{\E4}{\left(\Omega,\mathcal{G},\mathcal{G}_{t},\prob\right)}
\begin{Def}
Un proceso estoc\'astico $\PE$ definido en $\left(\Omega,\mathcal{G},\prob\right)$ con valores en el espacio topol\'ogico $E$ es continuo por la derecha si cada trayectoria muestral $t\rightarrow X_{t}\left(w\right)$ es un mapeo continuo por la derecha de $I$ en $E$.
\end{Def}

\begin{Def}[HD1]\label{Eq.2.1.S}
Un semigrupo de Markov $\left/P_{t}\right)$ en un espacio de Rad\'on $E$ se dice que satisface la condici\'on {\em HD1} si, dada una medida de probabilidad $\mu$ en $E$, existe una $\sigma$-\'algebra $\mathcal{E^{*}}$ con $\mathcal{E}\subset\mathcal{E}$ y $P_{t}\left(b\mathcal{E}^{*}\right)\subset b\mathcal{E}^{*}$, y un $\mathcal{E}^{*}$-proceso $E$-valuado continuo por la derecha $\PE$ en alg\'un espacio de probabilidad filtrado $\left(\Omega,\mathcal{G},\mathcal{G}_{t},\prob\right)$ tal que $X=\left(\Omega,\mathcal{G},\mathcal{G}_{t},\prob\right)$ es de Markov (Homog\'eneo) con semigrupo de transici\'on $(P_{t})$ y distribuci\'on inicial $\mu$.
\end{Def}

Considerese la colecci\'on de variables aleatorias $X_{t}$ definidas en alg\'un espacio de probabilidad, y una colecci\'on de medidas $\mathbf{P}^{x}$ tales que $\mathbf{P}^{x}\left\{X_{0}=x\right\}$, y bajo cualquier $\mathbf{P}^{x}$, $X_{t}$ es de Markov con semigrupo $\left(P_{t}\right)$. $\mathbf{P}^{x}$ puede considerarse como la distribuci\'on condicional de $\mathbf{P}$ dado $X_{0}=x$.

\begin{Def}\label{Def.2.2.S}
Sea $E$ espacio de Rad\'on, $\SG$ semigrupo de Markov en $\ER$. La colecci\'on $\mathbf{X}=\left(\Omega,\mathcal{G},\mathcal{G}_{t},X_{t},\theta_{t},\CM\right)$ es un proceso $\mathcal{E}$-Markov continuo por la derecha simple, con espacio de estados $E$ y semigrupo de transici\'on $\SG$ en caso de que $\mathbf{X}$ satisfaga las siguientes condiciones:
\begin{itemize}
\item[i)] $\left(\Omega,\mathcal{G},\mathcal{G}_{t}\right)$ es un espacio de medida filtrado, y $X_{t}$ es un proceso $E$-valuado continuo por la derecha $\mathcal{E}^{*}$-adaptado a $\left(\mathcal{G}_{t}\right)$;

\item[ii)] $\left(\theta_{t}\right)_{t\geq0}$ es una colecci\'on de operadores {\em shift} para $X$, es decir, mapea $\Omega$ en s\'i mismo satisfaciendo para $t,s\geq0$,

\begin{equation}\label{Eq.Shift}
\theta_{t}\circ\theta_{s}=\theta_{t+s}\textrm{ y }X_{t}\circ\theta_{t}=X_{t+s};
\end{equation}

\item[iii)] Para cualquier $x\in E$,$\CM\left\{X_{0}=x\right\}=1$, y el proceso $\PE$ tiene la propiedad de Markov (\ref{Eq.1.8.S}) con semigrupo de transici\'on $\SG$ relativo a $\left(\Omega,\mathcal{G},\mathcal{G}_{t},\CM\right)$.
\end{itemize}
\end{Def}

\begin{Def}[HD2]\label{Eq.2.2.S}
Para cualquier $\alpha>0$ y cualquier $f\in S^{\alpha}$, el proceso $t\rightarrow f\left(X_{t}\right)$ es continuo por la derecha casi seguramente.
\end{Def}

\begin{Def}\label{Def.PD}
Un sistema $\mathbf{X}=\left(\Omega,\mathcal{G},\mathcal{G}_{t},X_{t},\theta_{t},\CM\right)$ es un proceso derecho en el espacio de Rad\'on $E$ con semigrupo de transici\'on $\SG$ provisto de:
\begin{itemize}
\item[i)] $\mathbf{X}$ es una realizaci\'on  continua por la derecha, \ref{Def.2.2.S}, de $\SG$.

\item[ii)] $\mathbf{X}$ satisface la condicion HD2, \ref{Eq.2.2.S}, relativa a $\mathcal{G}_{t}$.

\item[iii)] $\mathcal{G}_{t}$ es aumentado y continuo por la derecha.
\end{itemize}
\end{Def}




\begin{Lema}[Lema 4.2, Dai\cite{Dai}]\label{Lema4.2}
Sea $\left\{x_{n}\right\}\subset \mathbf{X}$ con
$|x_{n}|\rightarrow\infty$, conforme $n\rightarrow\infty$. Suponga
que
\[lim_{n\rightarrow\infty}\frac{1}{|x_{n}|}U\left(0\right)=\overline{U}\]
y
\[lim_{n\rightarrow\infty}\frac{1}{|x_{n}|}V\left(0\right)=\overline{V}.\]

Entonces, conforme $n\rightarrow\infty$, casi seguramente

\begin{equation}\label{E1.4.2}
\frac{1}{|x_{n}|}\Phi^{k}\left(\left[|x_{n}|t\right]\right)\rightarrow
P_{k}^{'}t\textrm{, u.o.c.,}
\end{equation}

\begin{equation}\label{E1.4.3}
\frac{1}{|x_{n}|}E^{x_{n}}_{k}\left(|x_{n}|t\right)\rightarrow
\alpha_{k}\left(t-\overline{U}_{k}\right)^{+}\textrm{, u.o.c.,}
\end{equation}

\begin{equation}\label{E1.4.4}
\frac{1}{|x_{n}|}S^{x_{n}}_{k}\left(|x_{n}|t\right)\rightarrow
\mu_{k}\left(t-\overline{V}_{k}\right)^{+}\textrm{, u.o.c.,}
\end{equation}

donde $\left[t\right]$ es la parte entera de $t$ y
$\mu_{k}=1/m_{k}=1/\esp\left[\eta_{k}\left(1\right)\right]$.
\end{Lema}

\begin{Lema}[Lema 4.3, Dai\cite{Dai}]\label{Lema.4.3}
Sea $\left\{x_{n}\right\}\subset \mathbf{X}$ con
$|x_{n}|\rightarrow\infty$, conforme $n\rightarrow\infty$. Suponga
que
\[lim_{n\rightarrow\infty}\frac{1}{|x_{n}|}U\left(0\right)=\overline{U}_{k}\]
y
\[lim_{n\rightarrow\infty}\frac{1}{|x_{n}|}V\left(0\right)=\overline{V}_{k}.\]
\begin{itemize}
\item[a)] Conforme $n\rightarrow\infty$ casi seguramente,
\[lim_{n\rightarrow\infty}\frac{1}{|x_{n}|}U^{x_{n}}_{k}\left(|x_{n}|t\right)=\left(\overline{U}_{k}-t\right)^{+}\textrm{, u.o.c.}\]
y
\[lim_{n\rightarrow\infty}\frac{1}{|x_{n}|}V^{x_{n}}_{k}\left(|x_{n}|t\right)=\left(\overline{V}_{k}-t\right)^{+}.\]

\item[b)] Para cada $t\geq0$ fijo,
\[\left\{\frac{1}{|x_{n}|}U^{x_{n}}_{k}\left(|x_{n}|t\right),|x_{n}|\geq1\right\}\]
y
\[\left\{\frac{1}{|x_{n}|}V^{x_{n}}_{k}\left(|x_{n}|t\right),|x_{n}|\geq1\right\}\]
\end{itemize}
son uniformemente convergentes.
\end{Lema}

$S_{l}^{x}\left(t\right)$ es el n\'umero total de servicios
completados de la clase $l$, si la clase $l$ est\'a dando $t$
unidades de tiempo de servicio. Sea $T_{l}^{x}\left(x\right)$ el
monto acumulado del tiempo de servicio que el servidor
$s\left(l\right)$ gasta en los usuarios de la clase $l$ al tiempo
$t$. Entonces $S_{l}^{x}\left(T_{l}^{x}\left(t\right)\right)$ es
el n\'umero total de servicios completados para la clase $l$ al
tiempo $t$. Una fracci\'on de estos usuarios,
$\Phi_{l}^{x}\left(S_{l}^{x}\left(T_{l}^{x}\left(t\right)\right)\right)$,
se convierte en usuarios de la clase $k$.\\

Entonces, dado lo anterior, se tiene la siguiente representaci\'on
para el proceso de la longitud de la cola:\\

\begin{equation}
Q_{k}^{x}\left(t\right)=_{k}^{x}\left(0\right)+E_{k}^{x}\left(t\right)+\sum_{l=1}^{K}\Phi_{k}^{l}\left(S_{l}^{x}\left(T_{l}^{x}\left(t\right)\right)\right)-S_{k}^{x}\left(T_{k}^{x}\left(t\right)\right)
\end{equation}
para $k=1,\ldots,K$. Para $i=1,\ldots,d$, sea
\[I_{i}^{x}\left(t\right)=t-\sum_{j\in C_{i}}T_{k}^{x}\left(t\right).\]

Entonces $I_{i}^{x}\left(t\right)$ es el monto acumulado del
tiempo que el servidor $i$ ha estado desocupado al tiempo $t$. Se
est\'a asumiendo que las disciplinas satisfacen la ley de
conservaci\'on del trabajo, es decir, el servidor $i$ est\'a en
pausa solamente cuando no hay usuarios en la estaci\'on $i$.
Entonces, se tiene que

\begin{equation}
\int_{0}^{\infty}\left(\sum_{k\in
C_{i}}Q_{k}^{x}\left(t\right)\right)dI_{i}^{x}\left(t\right)=0,
\end{equation}
para $i=1,\ldots,d$.\\

Hacer
\[T^{x}\left(t\right)=\left(T_{1}^{x}\left(t\right),\ldots,T_{K}^{x}\left(t\right)\right)^{'},\]
\[I^{x}\left(t\right)=\left(I_{1}^{x}\left(t\right),\ldots,I_{K}^{x}\left(t\right)\right)^{'}\]
y
\[S^{x}\left(T^{x}\left(t\right)\right)=\left(S_{1}^{x}\left(T_{1}^{x}\left(t\right)\right),\ldots,S_{K}^{x}\left(T_{K}^{x}\left(t\right)\right)\right)^{'}.\]

Para una disciplina que cumple con la ley de conservaci\'on del
trabajo, en forma vectorial, se tiene el siguiente conjunto de
ecuaciones

\begin{equation}\label{Eq.MF.1.3}
Q^{x}\left(t\right)=Q^{x}\left(0\right)+E^{x}\left(t\right)+\sum_{l=1}^{K}\Phi^{l}\left(S_{l}^{x}\left(T_{l}^{x}\left(t\right)\right)\right)-S^{x}\left(T^{x}\left(t\right)\right),\\
\end{equation}

\begin{equation}\label{Eq.MF.2.3}
Q^{x}\left(t\right)\geq0,\\
\end{equation}

\begin{equation}\label{Eq.MF.3.3}
T^{x}\left(0\right)=0,\textrm{ y }\overline{T}^{x}\left(t\right)\textrm{ es no decreciente},\\
\end{equation}

\begin{equation}\label{Eq.MF.4.3}
I^{x}\left(t\right)=et-CT^{x}\left(t\right)\textrm{ es no
decreciente}\\
\end{equation}

\begin{equation}\label{Eq.MF.5.3}
\int_{0}^{\infty}\left(CQ^{x}\left(t\right)\right)dI_{i}^{x}\left(t\right)=0,\\
\end{equation}

\begin{equation}\label{Eq.MF.6.3}
\textrm{Condiciones adicionales en
}\left(\overline{Q}^{x}\left(\cdot\right),\overline{T}^{x}\left(\cdot\right)\right)\textrm{
espec\'ificas de la disciplina de la cola,}
\end{equation}

donde $e$ es un vector de unos de dimensi\'on $d$, $C$ es la
matriz definida por
\[C_{ik}=\left\{\begin{array}{cc}
1,& S\left(k\right)=i,\\
0,& \textrm{ en otro caso}.\\
\end{array}\right.
\]
Es necesario enunciar el siguiente Teorema que se utilizar\'a para
el Teorema \ref{Tma.4.2.Dai}:
\begin{Teo}[Teorema 4.1, Dai \cite{Dai}]
Considere una disciplina que cumpla la ley de conservaci\'on del
trabajo, para casi todas las trayectorias muestrales $\omega$ y
cualquier sucesi\'on de estados iniciales
$\left\{x_{n}\right\}\subset \mathbf{X}$, con
$|x_{n}|\rightarrow\infty$, existe una subsucesi\'on
$\left\{x_{n_{j}}\right\}$ con $|x_{n_{j}}|\rightarrow\infty$ tal
que
\begin{equation}\label{Eq.4.15}
\frac{1}{|x_{n_{j}}|}\left(Q^{x_{n_{j}}}\left(0\right),U^{x_{n_{j}}}\left(0\right),V^{x_{n_{j}}}\left(0\right)\right)\rightarrow\left(\overline{Q}\left(0\right),\overline{U},\overline{V}\right),
\end{equation}

\begin{equation}\label{Eq.4.16}
\frac{1}{|x_{n_{j}}|}\left(Q^{x_{n_{j}}}\left(|x_{n_{j}}|t\right),T^{x_{n_{j}}}\left(|x_{n_{j}}|t\right)\right)\rightarrow\left(\overline{Q}\left(t\right),\overline{T}\left(t\right)\right)\textrm{
u.o.c.}
\end{equation}

Adem\'as,
$\left(\overline{Q}\left(t\right),\overline{T}\left(t\right)\right)$
satisface las siguientes ecuaciones:
\begin{equation}\label{Eq.MF.1.3a}
\overline{Q}\left(t\right)=Q\left(0\right)+\left(\alpha
t-\overline{U}\right)^{+}-\left(I-P\right)^{'}M^{-1}\left(\overline{T}\left(t\right)-\overline{V}\right)^{+},
\end{equation}

\begin{equation}\label{Eq.MF.2.3a}
\overline{Q}\left(t\right)\geq0,\\
\end{equation}

\begin{equation}\label{Eq.MF.3.3a}
\overline{T}\left(t\right)\textrm{ es no decreciente y comienza en cero},\\
\end{equation}

\begin{equation}\label{Eq.MF.4.3a}
\overline{I}\left(t\right)=et-C\overline{T}\left(t\right)\textrm{
es no decreciente,}\\
\end{equation}

\begin{equation}\label{Eq.MF.5.3a}
\int_{0}^{\infty}\left(C\overline{Q}\left(t\right)\right)d\overline{I}\left(t\right)=0,\\
\end{equation}

\begin{equation}\label{Eq.MF.6.3a}
\textrm{Condiciones adicionales en
}\left(\overline{Q}\left(\cdot\right),\overline{T}\left(\cdot\right)\right)\textrm{
especficas de la disciplina de la cola,}
\end{equation}
\end{Teo}

\begin{Def}[Definici\'on 4.1, , Dai \cite{Dai}]
Sea una disciplina de servicio espec\'ifica. Cualquier l\'imite
$\left(\overline{Q}\left(\cdot\right),\overline{T}\left(\cdot\right)\right)$
en \ref{Eq.4.16} es un {\em flujo l\'imite} de la disciplina.
Cualquier soluci\'on (\ref{Eq.MF.1.3a})-(\ref{Eq.MF.6.3a}) es
llamado flujo soluci\'on de la disciplina. Se dice que el modelo de flujo l\'imite, modelo de flujo, de la disciplina de la cola es estable si existe una constante
$\delta>0$ que depende de $\mu,\alpha$ y $P$ solamente, tal que
cualquier flujo l\'imite con
$|\overline{Q}\left(0\right)|+|\overline{U}|+|\overline{V}|=1$, se
tiene que $\overline{Q}\left(\cdot+\delta\right)\equiv0$.
\end{Def}

\begin{Teo}[Teorema 4.2, Dai\cite{Dai}]\label{Tma.4.2.Dai}
Sea una disciplina fija para la cola, suponga que se cumplen las
condiciones (1.2)-(1.5). Si el modelo de flujo l\'imite de la
disciplina de la cola es estable, entonces la cadena de Markov $X$
que describe la din\'amica de la red bajo la disciplina es Harris
recurrente positiva.
\end{Teo}

Ahora se procede a escalar el espacio y el tiempo para reducir la
aparente fluctuaci\'on del modelo. Consid\'erese el proceso
\begin{equation}\label{Eq.3.7}
\overline{Q}^{x}\left(t\right)=\frac{1}{|x|}Q^{x}\left(|x|t\right)
\end{equation}
A este proceso se le conoce como el fluido escalado, y cualquier l\'imite $\overline{Q}^{x}\left(t\right)$ es llamado flujo l\'imite del proceso de longitud de la cola. Haciendo $|q|\rightarrow\infty$ mientras se mantiene el resto de las componentes fijas, cualquier punto l\'imite del proceso de longitud de la cola normalizado $\overline{Q}^{x}$ es soluci\'on del siguiente modelo de flujo.

Al conjunto de ecuaciones dadas en \ref{Eq.3.8}-\ref{Eq.3.13} se
le llama {\em Modelo de flujo} y al conjunto de todas las
soluciones del modelo de flujo
$\left(\overline{Q}\left(\cdot\right),\overline{T}
\left(\cdot\right)\right)$ se le denotar\'a por $\mathcal{Q}$.

Si se hace $|x|\rightarrow\infty$ sin restringir ninguna de las
componentes, tambi\'en se obtienen un modelo de flujo, pero en
este caso el residual de los procesos de arribo y servicio
introducen un retraso:

\begin{Def}[Definici\'on 3.3, Dai y Meyn \cite{DaiSean}]
El modelo de flujo es estable si existe un tiempo fijo $t_{0}$ tal
que $\overline{Q}\left(t\right)=0$, con $t\geq t_{0}$, para
cualquier $\overline{Q}\left(\cdot\right)\in\mathcal{Q}$ que
cumple con $|\overline{Q}\left(0\right)|=1$.
\end{Def}

El siguiente resultado se encuentra en Chen \cite{Chen}.
\begin{Lemma}[Lema 3.1, Dai y Meyn \cite{DaiSean}]
Si el modelo de flujo definido por \ref{Eq.3.8}-\ref{Eq.3.13} es
estable, entonces el modelo de flujo retrasado es tambi\'en
estable, es decir, existe $t_{0}>0$ tal que
$\overline{Q}\left(t\right)=0$ para cualquier $t\geq t_{0}$, para
cualquier soluci\'on del modelo de flujo retrasado cuya
condici\'on inicial $\overline{x}$ satisface que
$|\overline{x}|=|\overline{Q}\left(0\right)|+|\overline{A}\left(0\right)|+|\overline{B}\left(0\right)|\leq1$.
\end{Lemma}


Propiedades importantes para el modelo de flujo retrasado:

\begin{Prop}
 Sea $\left(\overline{Q},\overline{T},\overline{T}^{0}\right)$ un flujo l\'imite de \ref{Eq.4.4} y suponga que cuando $x\rightarrow\infty$ a lo largo de
una subsucesi\'on
\[\left(\frac{1}{|x|}Q_{k}^{x}\left(0\right),\frac{1}{|x|}A_{k}^{x}\left(0\right),\frac{1}{|x|}B_{k}^{x}\left(0\right),\frac{1}{|x|}B_{k}^{x,0}\left(0\right)\right)\rightarrow\left(\overline{Q}_{k}\left(0\right),0,0,0\right)\]
para $k=1,\ldots,K$. EL flujo l\'imite tiene las siguientes
propiedades, donde las propiedades de la derivada se cumplen donde
la derivada exista:
\begin{itemize}
 \item[i)] Los vectores de tiempo ocupado $\overline{T}\left(t\right)$ y $\overline{T}^{0}\left(t\right)$ son crecientes y continuas con
$\overline{T}\left(0\right)=\overline{T}^{0}\left(0\right)=0$.
\item[ii)] Para todo $t\geq0$
\[\sum_{k=1}^{K}\left[\overline{T}_{k}\left(t\right)+\overline{T}_{k}^{0}\left(t\right)\right]=t\]
\item[iii)] Para todo $1\leq k\leq K$
\[\overline{Q}_{k}\left(t\right)=\overline{Q}_{k}\left(0\right)+\alpha_{k}t-\mu_{k}\overline{T}_{k}\left(t\right)\]
\item[iv)]  Para todo $1\leq k\leq K$
\[\dot{{\overline{T}}}_{k}\left(t\right)=\beta_{k}\] para $\overline{Q}_{k}\left(t\right)=0$.
\item[v)] Para todo $k,j$
\[\mu_{k}^{0}\overline{T}_{k}^{0}\left(t\right)=\mu_{j}^{0}\overline{T}_{j}^{0}\left(t\right)\]
\item[vi)]  Para todo $1\leq k\leq K$
\[\mu_{k}\dot{{\overline{T}}}_{k}\left(t\right)=l_{k}\mu_{k}^{0}\dot{{\overline{T}}}_{k}^{0}\left(t\right)\] para $\overline{Q}_{k}\left(t\right)>0$.
\end{itemize}
\end{Prop}

\begin{Lema}[Lema 3.1 \cite{Chen}]\label{Lema3.1}
Si el modelo de flujo es estable, definido por las ecuaciones
(3.8)-(3.13), entonces el modelo de flujo retrasado tambin es
estable.
\end{Lema}

\begin{Teo}[Teorema 5.2 \cite{Chen}]\label{Tma.5.2}
Si el modelo de flujo lineal correspondiente a la red de cola es
estable, entonces la red de colas es estable.
\end{Teo}

\begin{Teo}[Teorema 5.1 \cite{Chen}]\label{Tma.5.1.Chen}
La red de colas es estable si existe una constante $t_{0}$ que
depende de $\left(\alpha,\mu,T,U\right)$ y $V$ que satisfagan las
ecuaciones (5.1)-(5.5), $Z\left(t\right)=0$, para toda $t\geq
t_{0}$.
\end{Teo}



\begin{Lema}[Lema 5.2 \cite{Gut}]\label{Lema.5.2.Gut}
Sea $\left\{\xi\left(k\right):k\in\ent\right\}$ sucesin de
variables aleatorias i.i.d. con valores en
$\left(0,\infty\right)$, y sea $E\left(t\right)$ el proceso de
conteo
\[E\left(t\right)=max\left\{n\geq1:\xi\left(1\right)+\cdots+\xi\left(n-1\right)\leq t\right\}.\]
Si $E\left[\xi\left(1\right)\right]<\infty$, entonces para
cualquier entero $r\geq1$
\begin{equation}
lim_{t\rightarrow\infty}\esp\left[\left(\frac{E\left(t\right)}{t}\right)^{r}\right]=\left(\frac{1}{E\left[\xi_{1}\right]}\right)^{r}
\end{equation}
de aqu, bajo estas condiciones
\begin{itemize}
\item[a)] Para cualquier $t>0$,
$sup_{t\geq\delta}\esp\left[\left(\frac{E\left(t\right)}{t}\right)^{r}\right]$

\item[b)] Las variables aleatorias
$\left\{\left(\frac{E\left(t\right)}{t}\right)^{r}:t\geq1\right\}$
son uniformemente integrables.
\end{itemize}
\end{Lema}

\begin{Teo}[Teorema 5.1: Ley Fuerte para Procesos de Conteo
\cite{Gut}]\label{Tma.5.1.Gut} Sea
$0<\mu<\esp\left(X_{1}\right]\leq\infty$. entonces

\begin{itemize}
\item[a)] $\frac{N\left(t\right)}{t}\rightarrow\frac{1}{\mu}$
a.s., cuando $t\rightarrow\infty$.


\item[b)]$\esp\left[\frac{N\left(t\right)}{t}\right]^{r}\rightarrow\frac{1}{\mu^{r}}$,
cuando $t\rightarrow\infty$ para todo $r>0$..
\end{itemize}
\end{Teo}


\begin{Prop}[Proposicin 5.1 \cite{DaiSean}]\label{Prop.5.1}
Suponga que los supuestos (A1) y (A2) se cumplen, adems suponga
que el modelo de flujo es estable. Entonces existe $t_{0}>0$ tal
que
\begin{equation}\label{Eq.Prop.5.1}
lim_{|x|\rightarrow\infty}\frac{1}{|x|^{p+1}}\esp_{x}\left[|X\left(t_{0}|x|\right)|^{p+1}\right]=0.
\end{equation}

\end{Prop}


\begin{Prop}[Proposici\'on 5.3 \cite{DaiSean}]
Sea $X$ proceso de estados para la red de colas, y suponga que se
cumplen los supuestos (A1) y (A2), entonces para alguna constante
positiva $C_{p+1}<\infty$, $\delta>0$ y un conjunto compacto
$C\subset X$.

\begin{equation}\label{Eq.5.4}
\esp_{x}\left[\int_{0}^{\tau_{C}\left(\delta\right)}\left(1+|X\left(t\right)|^{p}\right)dt\right]\leq
C_{p+1}\left(1+|x|^{p+1}\right)
\end{equation}
\end{Prop}

\begin{Prop}[Proposici\'on 5.4 \cite{DaiSean}]
Sea $X$ un proceso de Markov Borel Derecho en $X$, sea
$f:X\leftarrow\rea_{+}$ y defina para alguna $\delta>0$, y un
conjunto cerrado $C\subset X$
\[V\left(x\right):=\esp_{x}\left[\int_{0}^{\tau_{C}\left(\delta\right)}f\left(X\left(t\right)\right)dt\right]\]
para $x\in X$. Si $V$ es finito en todas partes y uniformemente
acotada en $C$, entonces existe $k<\infty$ tal que
\begin{equation}\label{Eq.5.11}
\frac{1}{t}\esp_{x}\left[V\left(x\right)\right]+\frac{1}{t}\int_{0}^{t}\esp_{x}\left[f\left(X\left(s\right)\right)ds\right]\leq\frac{1}{t}V\left(x\right)+k,
\end{equation}
para $x\in X$ y $t>0$.
\end{Prop}


\begin{Teo}[Teorema 5.5 \cite{DaiSean}]
Suponga que se cumplen (A1) y (A2), adems suponga que el modelo
de flujo es estable. Entonces existe una constante $k_{p}<\infty$
tal que
\begin{equation}\label{Eq.5.13}
\frac{1}{t}\int_{0}^{t}\esp_{x}\left[|Q\left(s\right)|^{p}\right]ds\leq
k_{p}\left\{\frac{1}{t}|x|^{p+1}+1\right\}
\end{equation}
para $t\geq0$, $x\in X$. En particular para cada condici\'on inicial
\begin{equation}\label{Eq.5.14}
Limsup_{t\rightarrow\infty}\frac{1}{t}\int_{0}^{t}\esp_{x}\left[|Q\left(s\right)|^{p}\right]ds\leq
k_{p}
\end{equation}
\end{Teo}

\begin{Teo}[Teorema 6.2\cite{DaiSean}]\label{Tma.6.2}
Suponga que se cumplen los supuestos (A1)-(A3) y que el modelo de
flujo es estable, entonces se tiene que
\[\parallel P^{t}\left(c,\cdot\right)-\pi\left(\cdot\right)\parallel_{f_{p}}\rightarrow0\]
para $t\rightarrow\infty$ y $x\in X$. En particular para cada
condicin inicial
\[lim_{t\rightarrow\infty}\esp_{x}\left[\left|Q_{t}\right|^{p}\right]=\esp_{\pi}\left[\left|Q_{0}\right|^{p}\right]<\infty\]
\end{Teo}


\begin{Teo}[Teorema 6.3\cite{DaiSean}]\label{Tma.6.3}
Suponga que se cumplen los supuestos (A1)-(A3) y que el modelo de
flujo es estable, entonces con
$f\left(x\right)=f_{1}\left(x\right)$, se tiene que
\[lim_{t\rightarrow\infty}t^{(p-1)\left|P^{t}\left(c,\cdot\right)-\pi\left(\cdot\right)\right|_{f}=0},\]
para $x\in X$. En particular, para cada condicin inicial
\[lim_{t\rightarrow\infty}t^{(p-1)\left|\esp_{x}\left[Q_{t}\right]-\esp_{\pi}\left[Q_{0}\right]\right|=0}.\]
\end{Teo}


\begin{Prop}[Proposici\'on 5.1, Dai y Meyn \cite{DaiSean}]\label{Prop.5.1.DaiSean}
Suponga que los supuestos A1) y A2) son ciertos y que el modelo de flujo es estable. Entonces existe $t_{0}>0$ tal que
\begin{equation}
lim_{|x|\rightarrow\infty}\frac{1}{|x|^{p+1}}\esp_{x}\left[|X\left(t_{0}|x|\right)|^{p+1}\right]=0
\end{equation}
\end{Prop}

\begin{Lemma}[Lema 5.2, Dai y Meyn \cite{DaiSean}]\label{Lema.5.2.DaiSean}
 Sea $\left\{\zeta\left(k\right):k\in \mathbb{z}\right\}$ una sucesi\'on independiente e id\'enticamente distribuida que toma valores en $\left(0,\infty\right)$,
y sea
$E\left(t\right)=max\left(n\geq1:\zeta\left(1\right)+\cdots+\zeta\left(n-1\right)\leq
t\right)$. Si $\esp\left[\zeta\left(1\right)\right]<\infty$,
entonces para cualquier entero $r\geq1$
\begin{equation}
 lim_{t\rightarrow\infty}\esp\left[\left(\frac{E\left(t\right)}{t}\right)^{r}\right]=\left(\frac{1}{\esp\left[\zeta_{1}\right]}\right)^{r}.
\end{equation}
Luego, bajo estas condiciones:
\begin{itemize}
 \item[a)] para cualquier $\delta>0$, $\sup_{t\geq\delta}\esp\left[\left(\frac{E\left(t\right)}{t}\right)^{r}\right]<\infty$
\item[b)] las variables aleatorias
$\left\{\left(\frac{E\left(t\right)}{t}\right)^{r}:t\geq1\right\}$
son uniformemente integrables.
\end{itemize}
\end{Lemma}

\begin{Teo}[Teorema 5.5, Dai y Meyn \cite{DaiSean}]\label{Tma.5.5.DaiSean}
Suponga que los supuestos A1) y A2) se cumplen y que el modelo de
flujo es estable. Entonces existe una constante $\kappa_{p}$ tal
que
\begin{equation}
\frac{1}{t}\int_{0}^{t}\esp_{x}\left[|Q\left(s\right)|^{p}\right]ds\leq\kappa_{p}\left\{\frac{1}{t}|x|^{p+1}+1\right\}
\end{equation}
para $t>0$ y $x\in X$. En particular, para cada condici\'on
inicial
\begin{eqnarray*}
\limsup_{t\rightarrow\infty}\frac{1}{t}\int_{0}^{t}\esp_{x}\left[|Q\left(s\right)|^{p}\right]ds\leq\kappa_{p}.
\end{eqnarray*}
\end{Teo}

\begin{Teo}[Teorema 6.2, Dai y Meyn \cite{DaiSean}]\label{Tma.6.2.DaiSean}
Suponga que se cumplen los supuestos A1), A2) y A3) y que el
modelo de flujo es estable. Entonces se tiene que
\begin{equation}
\left\|P^{t}\left(x,\cdot\right)-\pi\left(\cdot\right)\right\|_{f_{p}}\textrm{,
}t\rightarrow\infty,x\in X.
\end{equation}
En particular para cada condici\'on inicial
\begin{eqnarray*}
\lim_{t\rightarrow\infty}\esp_{x}\left[|Q\left(t\right)|^{p}\right]=\esp_{\pi}\left[|Q\left(0\right)|^{p}\right]\leq\kappa_{r}
\end{eqnarray*}
\end{Teo}
\begin{Teo}[Teorema 6.3, Dai y Meyn \cite{DaiSean}]\label{Tma.6.3.DaiSean}
Suponga que se cumplen los supuestos A1), A2) y A3) y que el
modelo de flujo es estable. Entonces con
$f\left(x\right)=f_{1}\left(x\right)$ se tiene
\begin{equation}
\lim_{t\rightarrow\infty}t^{p-1}\left\|P^{t}\left(x,\cdot\right)-\pi\left(\cdot\right)\right\|_{f}=0.
\end{equation}
En particular para cada condici\'on inicial
\begin{eqnarray*}
\lim_{t\rightarrow\infty}t^{p-1}|\esp_{x}\left[Q\left(t\right)\right]-\esp_{\pi}\left[Q\left(0\right)\right]|=0.
\end{eqnarray*}
\end{Teo}

\begin{Teo}[Teorema 6.4, Dai y Meyn \cite{DaiSean}]\label{Tma.6.4.DaiSean}
Suponga que se cumplen los supuestos A1), A2) y A3) y que el
modelo de flujo es estable. Sea $\nu$ cualquier distribuci\'on de
probabilidad en $\left(X,\mathcal{B}_{X}\right)$, y $\pi$ la
distribuci\'on estacionaria de $X$.
\begin{itemize}
\item[i)] Para cualquier $f:X\leftarrow\rea_{+}$
\begin{equation}
\lim_{t\rightarrow\infty}\frac{1}{t}\int_{o}^{t}f\left(X\left(s\right)\right)ds=\pi\left(f\right):=\int
f\left(x\right)\pi\left(dx\right)
\end{equation}
$\prob$-c.s.

\item[ii)] Para cualquier $f:X\leftarrow\rea_{+}$ con
$\pi\left(|f|\right)<\infty$, la ecuaci\'on anterior se cumple.
\end{itemize}
\end{Teo}

\begin{Teo}[Teorema 2.2, Down \cite{Down}]\label{Tma2.2.Down}
Suponga que el fluido modelo es inestable en el sentido de que
para alguna $\epsilon_{0},c_{0}\geq0$,
\begin{equation}\label{Eq.Inestability}
|Q\left(T\right)|\geq\epsilon_{0}T-c_{0}\textrm{,   }T\geq0,
\end{equation}
para cualquier condici\'on inicial $Q\left(0\right)$, con
$|Q\left(0\right)|=1$. Entonces para cualquier $0<q\leq1$, existe
$B<0$ tal que para cualquier $|x|\geq B$,
\begin{equation}
\prob_{x}\left\{\mathbb{X}\rightarrow\infty\right\}\geq q.
\end{equation}
\end{Teo}



Es necesario hacer los siguientes supuestos sobre el
comportamiento del sistema de visitas c\'iclicas:
\begin{itemize}
\item Los tiempos de interarribo a la $k$-\'esima cola, son de la
forma $\left\{\xi_{k}\left(n\right)\right\}_{n\geq1}$, con la
propiedad de que son independientes e id{\'e}nticamente
distribuidos,
\item Los tiempos de servicio
$\left\{\eta_{k}\left(n\right)\right\}_{n\geq1}$ tienen la
propiedad de ser independientes e id{\'e}nticamente distribuidos,
\item Se define la tasa de arribo a la $k$-{\'e}sima cola como
$\lambda_{k}=1/\esp\left[\xi_{k}\left(1\right)\right]$,
\item la tasa de servicio para la $k$-{\'e}sima cola se define
como $\mu_{k}=1/\esp\left[\eta_{k}\left(1\right)\right]$,
\item tambi{\'e}n se define $\rho_{k}:=\lambda_{k}/\mu_{k}$, la
intensidad de tr\'afico del sistema o carga de la red, donde es
necesario que $\rho<1$ para cuestiones de estabilidad.
\end{itemize}



%_________________________________________________________________________
\subsection{Procesos Fuerte de Markov}
%_________________________________________________________________________
En Dai \cite{Dai} se muestra que para una amplia serie de disciplinas
de servicio el proceso $X$ es un Proceso Fuerte de
Markov, y por tanto se puede asumir que


Para establecer que $X=\left\{X\left(t\right),t\geq0\right\}$ es
un Proceso Fuerte de Markov, se siguen las secciones 2.3 y 2.4 de Kaspi and Mandelbaum \cite{KaspiMandelbaum}. \\

%______________________________________________________________
\subsubsection{Construcci\'on de un Proceso Determinista por partes, Davis
\cite{Davis}}.
%______________________________________________________________

%_________________________________________________________________________
\subsection{Procesos Harris Recurrentes Positivos}
%_________________________________________________________________________
Sea el proceso de Markov $X=\left\{X\left(t\right),t\geq0\right\}$
que describe la din\'amica de la red de colas. En lo que respecta
al supuesto (A3), en Dai y Meyn \cite{DaiSean} y Meyn y Down
\cite{MeynDown} hacen ver que este se puede sustituir por

\begin{itemize}
\item[A3')] Para el Proceso de Markov $X$, cada subconjunto
compacto de $X$ es un conjunto peque\~no.
\end{itemize}

Este supuesto es importante pues es un requisito para deducir la ergodicidad de la red.

%_________________________________________________________________________
\subsection{Construcci\'on de un Modelo de Flujo L\'imite}
%_________________________________________________________________________

Consideremos un caso m\'as simple para poner en contexto lo
anterior: para un sistema de visitas c\'iclicas se tiene que el
estado al tiempo $t$ es
\begin{equation}
X\left(t\right)=\left(Q\left(t\right),U\left(t\right),V\left(t\right)\right),
\end{equation}

donde $Q\left(t\right)$ es el n\'umero de usuarios formados en
cada estaci\'on. $U\left(t\right)$ es el tiempo restante antes de
que la siguiente clase $k$ de usuarios lleguen desde fuera del
sistema, $V\left(t\right)$ es el tiempo restante de servicio para
la clase $k$ de usuarios que est\'an siendo atendidos. Tanto
$U\left(t\right)$ como $V\left(t\right)$ se puede asumir que son
continuas por la derecha.

Sea
$x=\left(Q\left(0\right),U\left(0\right),V\left(0\right)\right)=\left(q,a,b\right)$,
el estado inicial de la red bajo una disciplina espec\'ifica para
la cola. Para $l\in\mathcal{E}$, donde $\mathcal{E}$ es el conjunto de clases de arribos externos, y $k=1,\ldots,K$ se define\\
\begin{eqnarray*}
E_{l}^{x}\left(t\right)&=&max\left\{r:U_{l}\left(0\right)+\xi_{l}\left(1\right)+\cdots+\xi_{l}\left(r-1\right)\leq
t\right\}\textrm{   }t\geq0,\\
S_{k}^{x}\left(t\right)&=&max\left\{r:V_{k}\left(0\right)+\eta_{k}\left(1\right)+\cdots+\eta_{k}\left(r-1\right)\leq
t\right\}\textrm{   }t\geq0.
\end{eqnarray*}

Para cada $k$ y cada $n$ se define

\begin{eqnarray*}\label{Eq.phi}
\Phi^{k}\left(n\right):=\sum_{i=1}^{n}\phi^{k}\left(i\right).
\end{eqnarray*}

donde $\phi^{k}\left(n\right)$ se define como el vector de ruta
para el $n$-\'esimo usuario de la clase $k$ que termina en la
estaci\'on $s\left(k\right)$, la $s$-\'eima componente de
$\phi^{k}\left(n\right)$ es uno si estos usuarios se convierten en
usuarios de la clase $l$ y cero en otro caso, por lo tanto
$\phi^{k}\left(n\right)$ es un vector {\em Bernoulli} de
dimensi\'on $K$ con par\'ametro $P_{k}^{'}$, donde $P_{k}$ denota
el $k$-\'esimo rengl\'on de $P=\left(P_{kl}\right)$.

Se asume que cada para cada $k$ la sucesi\'on $\phi^{k}\left(n\right)=\left\{\phi^{k}\left(n\right),n\geq1\right\}$
es independiente e id\'enticamente distribuida y que las
$\phi^{1}\left(n\right),\ldots,\phi^{K}\left(n\right)$ son
mutuamente independientes, adem\'as de independientes de los
procesos de arribo y de servicio.\\

\begin{Lema}[Lema 4.2, Dai\cite{Dai}]\label{Lema4.2}
Sea $\left\{x_{n}\right\}\subset \mathbf{X}$ con
$|x_{n}|\rightarrow\infty$, conforme $n\rightarrow\infty$. Suponga
que
\[lim_{n\rightarrow\infty}\frac{1}{|x_{n}|}U\left(0\right)=\overline{U}\]
y
\[lim_{n\rightarrow\infty}\frac{1}{|x_{n}|}V\left(0\right)=\overline{V}.\]

Entonces, conforme $n\rightarrow\infty$, casi seguramente

\begin{equation}\label{E1.4.2}
\frac{1}{|x_{n}|}\Phi^{k}\left(\left[|x_{n}|t\right]\right)\rightarrow
P_{k}^{'}t\textrm{, u.o.c.,}
\end{equation}

\begin{equation}\label{E1.4.3}
\frac{1}{|x_{n}|}E^{x_{n}}_{k}\left(|x_{n}|t\right)\rightarrow
\alpha_{k}\left(t-\overline{U}_{k}\right)^{+}\textrm{, u.o.c.,}
\end{equation}

\begin{equation}\label{E1.4.4}
\frac{1}{|x_{n}|}S^{x_{n}}_{k}\left(|x_{n}|t\right)\rightarrow
\mu_{k}\left(t-\overline{V}_{k}\right)^{+}\textrm{, u.o.c.,}
\end{equation}

donde $\left[t\right]$ es la parte entera de $t$ y
$\mu_{k}=1/m_{k}=1/\esp\left[\eta_{k}\left(1\right)\right]$.
\end{Lema}

\begin{Lema}[Lema 4.3, Dai\cite{Dai}]\label{Lema.4.3}
Sea $\left\{x_{n}\right\}\subset \mathbf{X}$ con
$|x_{n}|\rightarrow\infty$, conforme $n\rightarrow\infty$. Suponga
que
\[lim_{n\rightarrow\infty}\frac{1}{|x_{n}|}U\left(0\right)=\overline{U}_{k}\]
y
\[lim_{n\rightarrow\infty}\frac{1}{|x_{n}|}V\left(0\right)=\overline{V}_{k}.\]
\begin{itemize}
\item[a)] Conforme $n\rightarrow\infty$ casi seguramente,
\[lim_{n\rightarrow\infty}\frac{1}{|x_{n}|}U^{x_{n}}_{k}\left(|x_{n}|t\right)=\left(\overline{U}_{k}-t\right)^{+}\textrm{, u.o.c.}\]
y
\[lim_{n\rightarrow\infty}\frac{1}{|x_{n}|}V^{x_{n}}_{k}\left(|x_{n}|t\right)=\left(\overline{V}_{k}-t\right)^{+}.\]

\item[b)] Para cada $t\geq0$ fijo,
\[\left\{\frac{1}{|x_{n}|}U^{x_{n}}_{k}\left(|x_{n}|t\right),|x_{n}|\geq1\right\}\]
y
\[\left\{\frac{1}{|x_{n}|}V^{x_{n}}_{k}\left(|x_{n}|t\right),|x_{n}|\geq1\right\}\]
\end{itemize}
son uniformemente convergentes.
\end{Lema}

$S_{l}^{x}\left(t\right)$ es el n\'umero total de servicios
completados de la clase $l$, si la clase $l$ est\'a dando $t$
unidades de tiempo de servicio. Sea $T_{l}^{x}\left(x\right)$ el
monto acumulado del tiempo de servicio que el servidor
$s\left(l\right)$ gasta en los usuarios de la clase $l$ al tiempo
$t$. Entonces $S_{l}^{x}\left(T_{l}^{x}\left(t\right)\right)$ es
el n\'umero total de servicios completados para la clase $l$ al
tiempo $t$. Una fracci\'on de estos usuarios,
$\Phi_{l}^{x}\left(S_{l}^{x}\left(T_{l}^{x}\left(t\right)\right)\right)$,
se convierte en usuarios de la clase $k$.\\

Entonces, dado lo anterior, se tiene la siguiente representaci\'on
para el proceso de la longitud de la cola:\\

\begin{equation}
Q_{k}^{x}\left(t\right)=_{k}^{x}\left(0\right)+E_{k}^{x}\left(t\right)+\sum_{l=1}^{K}\Phi_{k}^{l}\left(S_{l}^{x}\left(T_{l}^{x}\left(t\right)\right)\right)-S_{k}^{x}\left(T_{k}^{x}\left(t\right)\right)
\end{equation}
para $k=1,\ldots,K$. Para $i=1,\ldots,d$, sea
\[I_{i}^{x}\left(t\right)=t-\sum_{j\in C_{i}}T_{k}^{x}\left(t\right).\]

Entonces $I_{i}^{x}\left(t\right)$ es el monto acumulado del
tiempo que el servidor $i$ ha estado desocupado al tiempo $t$. Se
est\'a asumiendo que las disciplinas satisfacen la ley de
conservaci\'on del trabajo, es decir, el servidor $i$ est\'a en
pausa solamente cuando no hay usuarios en la estaci\'on $i$.
Entonces, se tiene que

\begin{equation}
\int_{0}^{\infty}\left(\sum_{k\in
C_{i}}Q_{k}^{x}\left(t\right)\right)dI_{i}^{x}\left(t\right)=0,
\end{equation}
para $i=1,\ldots,d$.\\

Hacer
\[T^{x}\left(t\right)=\left(T_{1}^{x}\left(t\right),\ldots,T_{K}^{x}\left(t\right)\right)^{'},\]
\[I^{x}\left(t\right)=\left(I_{1}^{x}\left(t\right),\ldots,I_{K}^{x}\left(t\right)\right)^{'}\]
y
\[S^{x}\left(T^{x}\left(t\right)\right)=\left(S_{1}^{x}\left(T_{1}^{x}\left(t\right)\right),\ldots,S_{K}^{x}\left(T_{K}^{x}\left(t\right)\right)\right)^{'}.\]

Para una disciplina que cumple con la ley de conservaci\'on del
trabajo, en forma vectorial, se tiene el siguiente conjunto de
ecuaciones

\begin{equation}\label{Eq.MF.1.3}
Q^{x}\left(t\right)=Q^{x}\left(0\right)+E^{x}\left(t\right)+\sum_{l=1}^{K}\Phi^{l}\left(S_{l}^{x}\left(T_{l}^{x}\left(t\right)\right)\right)-S^{x}\left(T^{x}\left(t\right)\right),\\
\end{equation}

\begin{equation}\label{Eq.MF.2.3}
Q^{x}\left(t\right)\geq0,\\
\end{equation}

\begin{equation}\label{Eq.MF.3.3}
T^{x}\left(0\right)=0,\textrm{ y }\overline{T}^{x}\left(t\right)\textrm{ es no decreciente},\\
\end{equation}

\begin{equation}\label{Eq.MF.4.3}
I^{x}\left(t\right)=et-CT^{x}\left(t\right)\textrm{ es no
decreciente}\\
\end{equation}

\begin{equation}\label{Eq.MF.5.3}
\int_{0}^{\infty}\left(CQ^{x}\left(t\right)\right)dI_{i}^{x}\left(t\right)=0,\\
\end{equation}

\begin{equation}\label{Eq.MF.6.3}
\textrm{Condiciones adicionales en
}\left(\overline{Q}^{x}\left(\cdot\right),\overline{T}^{x}\left(\cdot\right)\right)\textrm{
espec\'ificas de la disciplina de la cola,}
\end{equation}

donde $e$ es un vector de unos de dimensi\'on $d$, $C$ es la
matriz definida por
\[C_{ik}=\left\{\begin{array}{cc}
1,& S\left(k\right)=i,\\
0,& \textrm{ en otro caso}.\\
\end{array}\right.
\]
Es necesario enunciar el siguiente Teorema que se utilizar\'a para
el Teorema \ref{Tma.4.2.Dai}:
\begin{Teo}[Teorema 4.1, Dai \cite{Dai}]
Considere una disciplina que cumpla la ley de conservaci\'on del
trabajo, para casi todas las trayectorias muestrales $\omega$ y
cualquier sucesi\'on de estados iniciales
$\left\{x_{n}\right\}\subset \mathbf{X}$, con
$|x_{n}|\rightarrow\infty$, existe una subsucesi\'on
$\left\{x_{n_{j}}\right\}$ con $|x_{n_{j}}|\rightarrow\infty$ tal
que
\begin{equation}\label{Eq.4.15}
\frac{1}{|x_{n_{j}}|}\left(Q^{x_{n_{j}}}\left(0\right),U^{x_{n_{j}}}\left(0\right),V^{x_{n_{j}}}\left(0\right)\right)\rightarrow\left(\overline{Q}\left(0\right),\overline{U},\overline{V}\right),
\end{equation}

\begin{equation}\label{Eq.4.16}
\frac{1}{|x_{n_{j}}|}\left(Q^{x_{n_{j}}}\left(|x_{n_{j}}|t\right),T^{x_{n_{j}}}\left(|x_{n_{j}}|t\right)\right)\rightarrow\left(\overline{Q}\left(t\right),\overline{T}\left(t\right)\right)\textrm{
u.o.c.}
\end{equation}

Adem\'as,
$\left(\overline{Q}\left(t\right),\overline{T}\left(t\right)\right)$
satisface las siguientes ecuaciones:
\begin{equation}\label{Eq.MF.1.3a}
\overline{Q}\left(t\right)=Q\left(0\right)+\left(\alpha
t-\overline{U}\right)^{+}-\left(I-P\right)^{'}M^{-1}\left(\overline{T}\left(t\right)-\overline{V}\right)^{+},
\end{equation}

\begin{equation}\label{Eq.MF.2.3a}
\overline{Q}\left(t\right)\geq0,\\
\end{equation}

\begin{equation}\label{Eq.MF.3.3a}
\overline{T}\left(t\right)\textrm{ es no decreciente y comienza en cero},\\
\end{equation}

\begin{equation}\label{Eq.MF.4.3a}
\overline{I}\left(t\right)=et-C\overline{T}\left(t\right)\textrm{
es no decreciente,}\\
\end{equation}

\begin{equation}\label{Eq.MF.5.3a}
\int_{0}^{\infty}\left(C\overline{Q}\left(t\right)\right)d\overline{I}\left(t\right)=0,\\
\end{equation}

\begin{equation}\label{Eq.MF.6.3a}
\textrm{Condiciones adicionales en
}\left(\overline{Q}\left(\cdot\right),\overline{T}\left(\cdot\right)\right)\textrm{
especficas de la disciplina de la cola,}
\end{equation}
\end{Teo}

\begin{Def}[Definici\'on 4.1, , Dai \cite{Dai}]
Sea una disciplina de servicio espec\'ifica. Cualquier l\'imite
$\left(\overline{Q}\left(\cdot\right),\overline{T}\left(\cdot\right)\right)$
en \ref{Eq.4.16} es un {\em flujo l\'imite} de la disciplina.
Cualquier soluci\'on (\ref{Eq.MF.1.3a})-(\ref{Eq.MF.6.3a}) es
llamado flujo soluci\'on de la disciplina. Se dice que el modelo de flujo l\'imite, modelo de flujo, de la disciplina de la cola es estable si existe una constante
$\delta>0$ que depende de $\mu,\alpha$ y $P$ solamente, tal que
cualquier flujo l\'imite con
$|\overline{Q}\left(0\right)|+|\overline{U}|+|\overline{V}|=1$, se
tiene que $\overline{Q}\left(\cdot+\delta\right)\equiv0$.
\end{Def}

\begin{Teo}[Teorema 4.2, Dai\cite{Dai}]\label{Tma.4.2.Dai}
Sea una disciplina fija para la cola, suponga que se cumplen las
condiciones (1.2)-(1.5). Si el modelo de flujo l\'imite de la
disciplina de la cola es estable, entonces la cadena de Markov $X$
que describe la din\'amica de la red bajo la disciplina es Harris
recurrente positiva.
\end{Teo}

Ahora se procede a escalar el espacio y el tiempo para reducir la
aparente fluctuaci\'on del modelo. Consid\'erese el proceso
\begin{equation}\label{Eq.3.7}
\overline{Q}^{x}\left(t\right)=\frac{1}{|x|}Q^{x}\left(|x|t\right)
\end{equation}
A este proceso se le conoce como el fluido escalado, y cualquier l\'imite $\overline{Q}^{x}\left(t\right)$ es llamado flujo l\'imite del proceso de longitud de la cola. Haciendo $|q|\rightarrow\infty$ mientras se mantiene el resto de las componentes fijas, cualquier punto l\'imite del proceso de longitud de la cola normalizado $\overline{Q}^{x}$ es soluci\'on del siguiente modelo de flujo.

\begin{Def}[Definici\'on 3.1, Dai y Meyn \cite{DaiSean}]
Un flujo l\'imite (retrasado) para una red bajo una disciplina de
servicio espec\'ifica se define como cualquier soluci\'on
 $\left(\overline{Q}\left(\cdot\right),\overline{T}\left(\cdot\right)\right)$ de las siguientes ecuaciones, donde
$\overline{Q}\left(t\right)=\left(\overline{Q}_{1}\left(t\right),\ldots,\overline{Q}_{K}\left(t\right)\right)^{'}$
y
$\overline{T}\left(t\right)=\left(\overline{T}_{1}\left(t\right),\ldots,\overline{T}_{K}\left(t\right)\right)^{'}$
\begin{equation}\label{Eq.3.8}
\overline{Q}_{k}\left(t\right)=\overline{Q}_{k}\left(0\right)+\alpha_{k}t-\mu_{k}\overline{T}_{k}\left(t\right)+\sum_{l=1}^{k}P_{lk}\mu_{l}\overline{T}_{l}\left(t\right)\\
\end{equation}
\begin{equation}\label{Eq.3.9}
\overline{Q}_{k}\left(t\right)\geq0\textrm{ para }k=1,2,\ldots,K,\\
\end{equation}
\begin{equation}\label{Eq.3.10}
\overline{T}_{k}\left(0\right)=0,\textrm{ y }\overline{T}_{k}\left(\cdot\right)\textrm{ es no decreciente},\\
\end{equation}
\begin{equation}\label{Eq.3.11}
\overline{I}_{i}\left(t\right)=t-\sum_{k\in C_{i}}\overline{T}_{k}\left(t\right)\textrm{ es no decreciente}\\
\end{equation}
\begin{equation}\label{Eq.3.12}
\overline{I}_{i}\left(\cdot\right)\textrm{ se incrementa al tiempo }t\textrm{ cuando }\sum_{k\in C_{i}}Q_{k}^{x}\left(t\right)dI_{i}^{x}\left(t\right)=0\\
\end{equation}
\begin{equation}\label{Eq.3.13}
\textrm{condiciones adicionales sobre
}\left(Q^{x}\left(\cdot\right),T^{x}\left(\cdot\right)\right)\textrm{
referentes a la disciplina de servicio}
\end{equation}
\end{Def}

Al conjunto de ecuaciones dadas en \ref{Eq.3.8}-\ref{Eq.3.13} se
le llama {\em Modelo de flujo} y al conjunto de todas las
soluciones del modelo de flujo
$\left(\overline{Q}\left(\cdot\right),\overline{T}
\left(\cdot\right)\right)$ se le denotar\'a por $\mathcal{Q}$.

Si se hace $|x|\rightarrow\infty$ sin restringir ninguna de las
componentes, tambi\'en se obtienen un modelo de flujo, pero en
este caso el residual de los procesos de arribo y servicio
introducen un retraso:

\begin{Def}[Definici\'on 3.2, Dai y Meyn \cite{DaiSean}]
El modelo de flujo retrasado de una disciplina de servicio en una
red con retraso
$\left(\overline{A}\left(0\right),\overline{B}\left(0\right)\right)\in\rea_{+}^{K+|A|}$
se define como el conjunto de ecuaciones dadas en
\ref{Eq.3.8}-\ref{Eq.3.13}, junto con la condici\'on:
\begin{equation}\label{CondAd.FluidModel}
\overline{Q}\left(t\right)=\overline{Q}\left(0\right)+\left(\alpha
t-\overline{A}\left(0\right)\right)^{+}-\left(I-P^{'}\right)M\left(\overline{T}\left(t\right)-\overline{B}\left(0\right)\right)^{+}
\end{equation}
\end{Def}

\begin{Def}[Definici\'on 3.3, Dai y Meyn \cite{DaiSean}]
El modelo de flujo es estable si existe un tiempo fijo $t_{0}$ tal
que $\overline{Q}\left(t\right)=0$, con $t\geq t_{0}$, para
cualquier $\overline{Q}\left(\cdot\right)\in\mathcal{Q}$ que
cumple con $|\overline{Q}\left(0\right)|=1$.
\end{Def}

El siguiente resultado se encuentra en Chen \cite{Chen}.
\begin{Lemma}[Lema 3.1, Dai y Meyn \cite{DaiSean}]
Si el modelo de flujo definido por \ref{Eq.3.8}-\ref{Eq.3.13} es
estable, entonces el modelo de flujo retrasado es tambi\'en
estable, es decir, existe $t_{0}>0$ tal que
$\overline{Q}\left(t\right)=0$ para cualquier $t\geq t_{0}$, para
cualquier soluci\'on del modelo de flujo retrasado cuya
condici\'on inicial $\overline{x}$ satisface que
$|\overline{x}|=|\overline{Q}\left(0\right)|+|\overline{A}\left(0\right)|+|\overline{B}\left(0\right)|\leq1$.
\end{Lemma}

%_________________________________________________________________________
\subsection{Modelo de Visitas C\'iclicas con un Servidor: Estabilidad}
%_________________________________________________________________________

%_________________________________________________________________________
\subsection{Teorema 2.1}
%_________________________________________________________________________



El resultado principal de Down \cite{Down} que relaciona la estabilidad del modelo de flujo con la estabilidad del sistema original

\begin{Teo}[Teorema 2.1, Down \cite{Down}]\label{Tma.2.1.Down}
Suponga que el modelo de flujo es estable, y que se cumplen los supuestos (A1) y (A2), entonces
\begin{itemize}
\item[i)] Para alguna constante $\kappa_{p}$, y para cada
condici\'on inicial $x\in X$
\begin{equation}\label{Estability.Eq1}
lim_{t\rightarrow\infty}\sup\frac{1}{t}\int_{0}^{t}\esp_{x}\left[|Q\left(s\right)|^{p}\right]ds\leq\kappa_{p},
\end{equation}
donde $p$ es el entero dado en (A2). Si adem\'as se cumple
la condici\'on (A3), entonces para cada condici\'on inicial:

\item[ii)] Los momentos transitorios convergen a su estado estacionario:
 \begin{equation}\label{Estability.Eq2}
lim_{t\rightarrow\infty}\esp_{x}\left[Q_{k}\left(t\right)^{r}\right]=\esp_{\pi}\left[Q_{k}\left(0\right)^{r}\right]\leq\kappa_{r},
\end{equation}
para $r=1,2,\ldots,p$ y $k=1,2,\ldots,K$. Donde $\pi$ es la
probabilidad invariante para $\mathbf{X}$.

\item[iii)]  El primer momento converge con raz\'on $t^{p-1}$:
\begin{equation}\label{Estability.Eq3}
lim_{t\rightarrow\infty}t^{p-1}|\esp_{x}\left[Q_{k}\left(t\right)\right]-\esp_{\pi}\left[Q\left(0\right)\right]=0.
\end{equation}

\item[iv)] La {\em Ley Fuerte de los grandes n\'umeros} se cumple:
\begin{equation}\label{Estability.Eq4}
lim_{t\rightarrow\infty}\frac{1}{t}\int_{0}^{t}Q_{k}^{r}\left(s\right)ds=\esp_{\pi}\left[Q_{k}\left(0\right)^{r}\right],\textrm{
}\prob_{x}\textrm{-c.s.}
\end{equation}
para $r=1,2,\ldots,p$ y $k=1,2,\ldots,K$.
\end{itemize}
\end{Teo}


\begin{Prop}[Proposici\'on 5.1, Dai y Meyn \cite{DaiSean}]\label{Prop.5.1.DaiSean}
Suponga que los supuestos A1) y A2) son ciertos y que el modelo de flujo es estable. Entonces existe $t_{0}>0$ tal que
\begin{equation}
lim_{|x|\rightarrow\infty}\frac{1}{|x|^{p+1}}\esp_{x}\left[|X\left(t_{0}|x|\right)|^{p+1}\right]=0
\end{equation}
\end{Prop}

\begin{Lemma}[Lema 5.2, Dai y Meyn \cite{DaiSean}]\label{Lema.5.2.DaiSean}
 Sea $\left\{\zeta\left(k\right):k\in \mathbb{z}\right\}$ una sucesi\'on independiente e id\'enticamente distribuida que toma valores en $\left(0,\infty\right)$,
y sea
$E\left(t\right)=max\left(n\geq1:\zeta\left(1\right)+\cdots+\zeta\left(n-1\right)\leq
t\right)$. Si $\esp\left[\zeta\left(1\right)\right]<\infty$,
entonces para cualquier entero $r\geq1$
\begin{equation}
 lim_{t\rightarrow\infty}\esp\left[\left(\frac{E\left(t\right)}{t}\right)^{r}\right]=\left(\frac{1}{\esp\left[\zeta_{1}\right]}\right)^{r}.
\end{equation}
Luego, bajo estas condiciones:
\begin{itemize}
 \item[a)] para cualquier $\delta>0$, $\sup_{t\geq\delta}\esp\left[\left(\frac{E\left(t\right)}{t}\right)^{r}\right]<\infty$
\item[b)] las variables aleatorias
$\left\{\left(\frac{E\left(t\right)}{t}\right)^{r}:t\geq1\right\}$
son uniformemente integrables.
\end{itemize}
\end{Lemma}

\begin{Teo}[Teorema 5.5, Dai y Meyn \cite{DaiSean}]\label{Tma.5.5.DaiSean}
Suponga que los supuestos A1) y A2) se cumplen y que el modelo de
flujo es estable. Entonces existe una constante $\kappa_{p}$ tal
que
\begin{equation}
\frac{1}{t}\int_{0}^{t}\esp_{x}\left[|Q\left(s\right)|^{p}\right]ds\leq\kappa_{p}\left\{\frac{1}{t}|x|^{p+1}+1\right\}
\end{equation}
para $t>0$ y $x\in X$. En particular, para cada condici\'on
inicial
\begin{eqnarray*}
\limsup_{t\rightarrow\infty}\frac{1}{t}\int_{0}^{t}\esp_{x}\left[|Q\left(s\right)|^{p}\right]ds\leq\kappa_{p}.
\end{eqnarray*}
\end{Teo}

\begin{Teo}[Teorema 6.2, Dai y Meyn \cite{DaiSean}]\label{Tma.6.2.DaiSean}
Suponga que se cumplen los supuestos A1), A2) y A3) y que el
modelo de flujo es estable. Entonces se tiene que
\begin{equation}
\left\|P^{t}\left(x,\cdot\right)-\pi\left(\cdot\right)\right\|_{f_{p}}\textrm{,
}t\rightarrow\infty,x\in X.
\end{equation}
En particular para cada condici\'on inicial
\begin{eqnarray*}
\lim_{t\rightarrow\infty}\esp_{x}\left[|Q\left(t\right)|^{p}\right]=\esp_{\pi}\left[|Q\left(0\right)|^{p}\right]\leq\kappa_{r}
\end{eqnarray*}
\end{Teo}
\begin{Teo}[Teorema 6.3, Dai y Meyn \cite{DaiSean}]\label{Tma.6.3.DaiSean}
Suponga que se cumplen los supuestos A1), A2) y A3) y que el
modelo de flujo es estable. Entonces con
$f\left(x\right)=f_{1}\left(x\right)$ se tiene
\begin{equation}
\lim_{t\rightarrow\infty}t^{p-1}\left\|P^{t}\left(x,\cdot\right)-\pi\left(\cdot\right)\right\|_{f}=0.
\end{equation}
En particular para cada condici\'on inicial
\begin{eqnarray*}
\lim_{t\rightarrow\infty}t^{p-1}|\esp_{x}\left[Q\left(t\right)\right]-\esp_{\pi}\left[Q\left(0\right)\right]|=0.
\end{eqnarray*}
\end{Teo}

\begin{Teo}[Teorema 6.4, Dai y Meyn \cite{DaiSean}]\label{Tma.6.4.DaiSean}
Suponga que se cumplen los supuestos A1), A2) y A3) y que el
modelo de flujo es estable. Sea $\nu$ cualquier distribuci\'on de
probabilidad en $\left(X,\mathcal{B}_{X}\right)$, y $\pi$ la
distribuci\'on estacionaria de $X$.
\begin{itemize}
\item[i)] Para cualquier $f:X\leftarrow\rea_{+}$
\begin{equation}
\lim_{t\rightarrow\infty}\frac{1}{t}\int_{o}^{t}f\left(X\left(s\right)\right)ds=\pi\left(f\right):=\int
f\left(x\right)\pi\left(dx\right)
\end{equation}
$\prob$-c.s.

\item[ii)] Para cualquier $f:X\leftarrow\rea_{+}$ con
$\pi\left(|f|\right)<\infty$, la ecuaci\'on anterior se cumple.
\end{itemize}
\end{Teo}

%_________________________________________________________________________
\subsection{Teorema 2.2}
%_________________________________________________________________________

\begin{Teo}[Teorema 2.2, Down \cite{Down}]\label{Tma2.2.Down}
Suponga que el fluido modelo es inestable en el sentido de que
para alguna $\epsilon_{0},c_{0}\geq0$,
\begin{equation}\label{Eq.Inestability}
|Q\left(T\right)|\geq\epsilon_{0}T-c_{0}\textrm{,   }T\geq0,
\end{equation}
para cualquier condici\'on inicial $Q\left(0\right)$, con
$|Q\left(0\right)|=1$. Entonces para cualquier $0<q\leq1$, existe
$B<0$ tal que para cualquier $|x|\geq B$,
\begin{equation}
\prob_{x}\left\{\mathbb{X}\rightarrow\infty\right\}\geq q.
\end{equation}
\end{Teo}

%_________________________________________________________________________
\subsection{Teorema 2.3}
%_________________________________________________________________________
\begin{Teo}[Teorema 2.3, Down \cite{Down}]\label{Tma2.3.Down}
Considere el siguiente valor:
\begin{equation}\label{Eq.Rho.1serv}
\rho=\sum_{k=1}^{K}\rho_{k}+max_{1\leq j\leq K}\left(\frac{\lambda_{j}}{\sum_{s=1}^{S}p_{js}\overline{N}_{s}}\right)\delta^{*}
\end{equation}
\begin{itemize}
\item[i)] Si $\rho<1$ entonces la red es estable, es decir, se cumple el teorema \ref{Tma.2.1.Down}.

\item[ii)] Si $\rho<1$ entonces la red es inestable, es decir, se cumple el teorema \ref{Tma2.2.Down}
\end{itemize}
\end{Teo}
%_____________________________________________________________________
\subsection{Definiciones  B\'asicas}
%_____________________________________________________________________
\begin{Def}
Sea $X$ un conjunto y $\mathcal{F}$ una $\sigma$-\'algebra de
subconjuntos de $X$, la pareja $\left(X,\mathcal{F}\right)$ es
llamado espacio medible. Un subconjunto $A$ de $X$ es llamado
medible, o medible con respecto a $\mathcal{F}$, si
$A\in\mathcal{F}$.
\end{Def}

\begin{Def}
Sea $\left(X,\mathcal{F},\mu\right)$ espacio de medida. Se dice
que la medida $\mu$ es $\sigma$-finita si se puede escribir
$X=\bigcup_{n\geq1}X_{n}$ con $X_{n}\in\mathcal{F}$ y
$\mu\left(X_{n}\right)<\infty$.
\end{Def}

\begin{Def}\label{Cto.Borel}
Sea $X$ el conjunto de los \'umeros reales $\rea$. El \'algebra de
Borel es la $\sigma$-\'algebra $B$ generada por los intervalos
abiertos $\left(a,b\right)\in\rea$. Cualquier conjunto en $B$ es
llamado {\em Conjunto de Borel}.
\end{Def}

\begin{Def}\label{Funcion.Medible}
Una funci\'on $f:X\rightarrow\rea$, es medible si para cualquier
n\'umero real $\alpha$ el conjunto
\[\left\{x\in X:f\left(x\right)>\alpha\right\}\]
pertenece a $X$. Equivalentemente, se dice que $f$ es medible si
\[f^{-1}\left(\left(\alpha,\infty\right)\right)=\left\{x\in X:f\left(x\right)>\alpha\right\}\in\mathcal{F}.\]
\end{Def}


\begin{Def}\label{Def.Cilindros}
Sean $\left(\Omega_{i},\mathcal{F}_{i}\right)$, $i=1,2,\ldots,$
espacios medibles y $\Omega=\prod_{i=1}^{\infty}\Omega_{i}$ el
conjunto de todas las sucesiones
$\left(\omega_{1},\omega_{2},\ldots,\right)$ tales que
$\omega_{i}\in\Omega_{i}$, $i=1,2,\ldots,$. Si
$B^{n}\subset\prod_{i=1}^{\infty}\Omega_{i}$, definimos
$B_{n}=\left\{\omega\in\Omega:\left(\omega_{1},\omega_{2},\ldots,\omega_{n}\right)\in
B^{n}\right\}$. Al conjunto $B_{n}$ se le llama {\em cilindro} con
base $B^{n}$, el cilindro es llamado medible si
$B^{n}\in\prod_{i=1}^{\infty}\mathcal{F}_{i}$.
\end{Def}


\begin{Def}\label{Def.Proc.Adaptado}[TSP, Ash \cite{RBA}]
Sea $X\left(t\right),t\geq0$ proceso estoc\'astico, el proceso es
adaptado a la familia de $\sigma$-\'algebras $\mathcal{F}_{t}$,
para $t\geq0$, si para $s<t$ implica que
$\mathcal{F}_{s}\subset\mathcal{F}_{t}$, y $X\left(t\right)$ es
$\mathcal{F}_{t}$-medible para cada $t$. Si no se especifica
$\mathcal{F}_{t}$ entonces se toma $\mathcal{F}_{t}$ como
$\mathcal{F}\left(X\left(s\right),s\leq t\right)$, la m\'as
peque\~na $\sigma$-\'algebra de subconjuntos de $\Omega$ que hace
que cada $X\left(s\right)$, con $s\leq t$ sea Borel medible.
\end{Def}


\begin{Def}\label{Def.Tiempo.Paro}[TSP, Ash \cite{RBA}]
Sea $\left\{\mathcal{F}\left(t\right),t\geq0\right\}$ familia
creciente de sub $\sigma$-\'algebras. es decir,
$\mathcal{F}\left(s\right)\subset\mathcal{F}\left(t\right)$ para
$s\leq t$. Un tiempo de paro para $\mathcal{F}\left(t\right)$ es
una funci\'on $T:\Omega\rightarrow\left[0,\infty\right]$ tal que
$\left\{T\leq t\right\}\in\mathcal{F}\left(t\right)$ para cada
$t\geq0$. Un tiempo de paro para el proceso estoc\'astico
$X\left(t\right),t\geq0$ es un tiempo de paro para las
$\sigma$-\'algebras
$\mathcal{F}\left(t\right)=\mathcal{F}\left(X\left(s\right)\right)$.
\end{Def}

\begin{Def}
Sea $X\left(t\right),t\geq0$ proceso estoc\'astico, con
$\left(S,\chi\right)$ espacio de estados. Se dice que el proceso
es adaptado a $\left\{\mathcal{F}\left(t\right)\right\}$, es
decir, si para cualquier $s,t\in I$, $I$ conjunto de \'indices,
$s<t$, se tiene que
$\mathcal{F}\left(s\right)\subset\mathcal{F}\left(t\right)$ y
$X\left(t\right)$ es $\mathcal{F}\left(t\right)$-medible,
\end{Def}

\begin{Def}
Sea $X\left(t\right),t\geq0$ proceso estoc\'astico, se dice que es
un Proceso de Markov relativo a $\mathcal{F}\left(t\right)$ o que
$\left\{X\left(t\right),\mathcal{F}\left(t\right)\right\}$ es de
Markov si y s\'olo si para cualquier conjunto $B\in\chi$,  y
$s,t\in I$, $s<t$ se cumple que
\begin{equation}\label{Prop.Markov}
P\left\{X\left(t\right)\in
B|\mathcal{F}\left(s\right)\right\}=P\left\{X\left(t\right)\in
B|X\left(s\right)\right\}.
\end{equation}
\end{Def}
\begin{Note}
Si se dice que $\left\{X\left(t\right)\right\}$ es un Proceso de
Markov sin mencionar $\mathcal{F}\left(t\right)$, se asumir\'a que
\begin{eqnarray*}
\mathcal{F}\left(t\right)=\mathcal{F}_{0}\left(t\right)=\mathcal{F}\left(X\left(r\right),r\leq
t\right),
\end{eqnarray*}
entonces la ecuaci\'on (\ref{Prop.Markov}) se puede escribir como
\begin{equation}
P\left\{X\left(t\right)\in B|X\left(r\right),r\leq s\right\} =
P\left\{X\left(t\right)\in B|X\left(s\right)\right\}
\end{equation}
\end{Note}

\begin{Teo}
Sea $\left(X_{n},\mathcal{F}_{n},n=0,1,\ldots,\right\}$ Proceso de
Markov con espacio de estados $\left(S_{0},\chi_{0}\right)$
generado por una distribuici\'on inicial $P_{o}$ y probabilidad de
transici\'on $p_{mn}$, para $m,n=0,1,\ldots,$ $m<n$, que por
notaci\'on se escribir\'a como $p\left(m,n,x,B\right)\rightarrow
p_{mn}\left(x,B\right)$. Sea $S$ tiempo de paro relativo a la
$\sigma$-\'algebra $\mathcal{F}_{n}$. Sea $T$ funci\'on medible,
$T:\Omega\rightarrow\left\{0,1,\ldots,\right\}$. Sup\'ongase que
$T\geq S$, entonces $T$ es tiempo de paro. Si $B\in\chi_{0}$,
entonces
\begin{equation}\label{Prop.Fuerte.Markov}
P\left\{X\left(T\right)\in
B,T<\infty|\mathcal{F}\left(S\right)\right\} =
p\left(S,T,X\left(s\right),B\right)
\end{equation}
en $\left\{T<\infty\right\}$.
\end{Teo}

Propiedades importantes para el modelo de flujo retrasado:

\begin{Prop}
 Sea $\left(\overline{Q},\overline{T},\overline{T}^{0}\right)$ un flujo l\'imite de \ref{Equation.4.4} y suponga que cuando $x\rightarrow\infty$ a lo largo de
una subsucesi\'on
\[\left(\frac{1}{|x|}Q_{k}^{x}\left(0\right),\frac{1}{|x|}A_{k}^{x}\left(0\right),\frac{1}{|x|}B_{k}^{x}\left(0\right),\frac{1}{|x|}B_{k}^{x,0}\left(0\right)\right)\rightarrow\left(\overline{Q}_{k}\left(0\right),0,0,0\right)\]
para $k=1,\ldots,K$. EL flujo l\'imite tiene las siguientes
propiedades, donde las propiedades de la derivada se cumplen donde
la derivada exista:
\begin{itemize}
 \item[i)] Los vectores de tiempo ocupado $\overline{T}\left(t\right)$ y $\overline{T}^{0}\left(t\right)$ son crecientes y continuas con
$\overline{T}\left(0\right)=\overline{T}^{0}\left(0\right)=0$.
\item[ii)] Para todo $t\geq0$
\[\sum_{k=1}^{K}\left[\overline{T}_{k}\left(t\right)+\overline{T}_{k}^{0}\left(t\right)\right]=t\]
\item[iii)] Para todo $1\leq k\leq K$
\[\overline{Q}_{k}\left(t\right)=\overline{Q}_{k}\left(0\right)+\alpha_{k}t-\mu_{k}\overline{T}_{k}\left(t\right)\]
\item[iv)]  Para todo $1\leq k\leq K$
\[\dot{{\overline{T}}}_{k}\left(t\right)=\beta_{k}\] para $\overline{Q}_{k}\left(t\right)=0$.
\item[v)] Para todo $k,j$
\[\mu_{k}^{0}\overline{T}_{k}^{0}\left(t\right)=\mu_{j}^{0}\overline{T}_{j}^{0}\left(t\right)\]
\item[vi)]  Para todo $1\leq k\leq K$
\[\mu_{k}\dot{{\overline{T}}}_{k}\left(t\right)=l_{k}\mu_{k}^{0}\dot{{\overline{T}}}_{k}^{0}\left(t\right)\] para $\overline{Q}_{k}\left(t\right)>0$.
\end{itemize}
\end{Prop}

\begin{Lema}[Lema 3.1 \cite{Chen}]\label{Lema3.1}
Si el modelo de flujo es estable, definido por las ecuaciones
(3.8)-(3.13), entonces el modelo de flujo retrasado tambin es
estable.
\end{Lema}

\begin{Teo}[Teorema 5.2 \cite{Chen}]\label{Tma.5.2}
Si el modelo de flujo lineal correspondiente a la red de cola es
estable, entonces la red de colas es estable.
\end{Teo}

\begin{Teo}[Teorema 5.1 \cite{Chen}]\label{Tma.5.1.Chen}
La red de colas es estable si existe una constante $t_{0}$ que
depende de $\left(\alpha,\mu,T,U\right)$ y $V$ que satisfagan las
ecuaciones (5.1)-(5.5), $Z\left(t\right)=0$, para toda $t\geq
t_{0}$.
\end{Teo}



\begin{Lema}[Lema 5.2 \cite{Gut}]\label{Lema.5.2.Gut}
Sea $\left\{\xi\left(k\right):k\in\ent\right\}$ sucesin de
variables aleatorias i.i.d. con valores en
$\left(0,\infty\right)$, y sea $E\left(t\right)$ el proceso de
conteo
\[E\left(t\right)=max\left\{n\geq1:\xi\left(1\right)+\cdots+\xi\left(n-1\right)\leq t\right\}.\]
Si $E\left[\xi\left(1\right)\right]<\infty$, entonces para
cualquier entero $r\geq1$
\begin{equation}
lim_{t\rightarrow\infty}\esp\left[\left(\frac{E\left(t\right)}{t}\right)^{r}\right]=\left(\frac{1}{E\left[\xi_{1}\right]}\right)^{r}
\end{equation}
de aqu, bajo estas condiciones
\begin{itemize}
\item[a)] Para cualquier $t>0$,
$sup_{t\geq\delta}\esp\left[\left(\frac{E\left(t\right)}{t}\right)^{r}\right]$

\item[b)] Las variables aleatorias
$\left\{\left(\frac{E\left(t\right)}{t}\right)^{r}:t\geq1\right\}$
son uniformemente integrables.
\end{itemize}
\end{Lema}

\begin{Teo}[Teorema 5.1: Ley Fuerte para Procesos de Conteo
\cite{Gut}]\label{Tma.5.1.Gut} Sea
$0<\mu<\esp\left(X_{1}\right]\leq\infty$. entonces

\begin{itemize}
\item[a)] $\frac{N\left(t\right)}{t}\rightarrow\frac{1}{\mu}$
a.s., cuando $t\rightarrow\infty$.


\item[b)]$\esp\left[\frac{N\left(t\right)}{t}\right]^{r}\rightarrow\frac{1}{\mu^{r}}$,
cuando $t\rightarrow\infty$ para todo $r>0$..
\end{itemize}
\end{Teo}


\begin{Prop}[Proposicin 5.1 \cite{DaiSean}]\label{Prop.5.1}
Suponga que los supuestos (A1) y (A2) se cumplen, adems suponga
que el modelo de flujo es estable. Entonces existe $t_{0}>0$ tal
que
\begin{equation}\label{Eq.Prop.5.1}
lim_{|x|\rightarrow\infty}\frac{1}{|x|^{p+1}}\esp_{x}\left[|X\left(t_{0}|x|\right)|^{p+1}\right]=0.
\end{equation}

\end{Prop}


\begin{Prop}[Proposici\'on 5.3 \cite{DaiSean}]
Sea $X$ proceso de estados para la red de colas, y suponga que se
cumplen los supuestos (A1) y (A2), entonces para alguna constante
positiva $C_{p+1}<\infty$, $\delta>0$ y un conjunto compacto
$C\subset X$.

\begin{equation}\label{Eq.5.4}
\esp_{x}\left[\int_{0}^{\tau_{C}\left(\delta\right)}\left(1+|X\left(t\right)|^{p}\right)dt\right]\leq
C_{p+1}\left(1+|x|^{p+1}\right)
\end{equation}
\end{Prop}

\begin{Prop}[Proposici\'on 5.4 \cite{DaiSean}]
Sea $X$ un proceso de Markov Borel Derecho en $X$, sea
$f:X\leftarrow\rea_{+}$ y defina para alguna $\delta>0$, y un
conjunto cerrado $C\subset X$
\[V\left(x\right):=\esp_{x}\left[\int_{0}^{\tau_{C}\left(\delta\right)}f\left(X\left(t\right)\right)dt\right]\]
para $x\in X$. Si $V$ es finito en todas partes y uniformemente
acotada en $C$, entonces existe $k<\infty$ tal que
\begin{equation}\label{Eq.5.11}
\frac{1}{t}\esp_{x}\left[V\left(x\right)\right]+\frac{1}{t}\int_{0}^{t}\esp_{x}\left[f\left(X\left(s\right)\right)ds\right]\leq\frac{1}{t}V\left(x\right)+k,
\end{equation}
para $x\in X$ y $t>0$.
\end{Prop}


\begin{Teo}[Teorema 5.5 \cite{DaiSean}]
Suponga que se cumplen (A1) y (A2), adems suponga que el modelo
de flujo es estable. Entonces existe una constante $k_{p}<\infty$
tal que
\begin{equation}\label{Eq.5.13}
\frac{1}{t}\int_{0}^{t}\esp_{x}\left[|Q\left(s\right)|^{p}\right]ds\leq
k_{p}\left\{\frac{1}{t}|x|^{p+1}+1\right\}
\end{equation}
para $t\geq0$, $x\in X$. En particular para cada condicin inicial
\begin{equation}\label{Eq.5.14}
Limsup_{t\rightarrow\infty}\frac{1}{t}\int_{0}^{t}\esp_{x}\left[|Q\left(s\right)|^{p}\right]ds\leq
k_{p}
\end{equation}
\end{Teo}

\begin{Teo}[Teorema 6.2\cite{DaiSean}]\label{Tma.6.2}
Suponga que se cumplen los supuestos (A1)-(A3) y que el modelo de
flujo es estable, entonces se tiene que
\[\parallel P^{t}\left(c,\cdot\right)-\pi\left(\cdot\right)\parallel_{f_{p}}\rightarrow0\]
para $t\rightarrow\infty$ y $x\in X$. En particular para cada
condicin inicial
\[lim_{t\rightarrow\infty}\esp_{x}\left[\left|Q_{t}\right|^{p}\right]=\esp_{\pi}\left[\left|Q_{0}\right|^{p}\right]<\infty\]
\end{Teo}


\begin{Teo}[Teorema 6.3\cite{DaiSean}]\label{Tma.6.3}
Suponga que se cumplen los supuestos (A1)-(A3) y que el modelo de
flujo es estable, entonces con
$f\left(x\right)=f_{1}\left(x\right)$, se tiene que
\[lim_{t\rightarrow\infty}t^{(p-1)\left|P^{t}\left(c,\cdot\right)-\pi\left(\cdot\right)\right|_{f}=0},\]
para $x\in X$. En particular, para cada condicin inicial
\[lim_{t\rightarrow\infty}t^{(p-1)\left|\esp_{x}\left[Q_{t}\right]-\esp_{\pi}\left[Q_{0}\right]\right|=0}.\]
\end{Teo}



Si $x$ es el n{\'u}mero de usuarios en la cola al comienzo del
periodo de servicio y $N_{s}\left(x\right)=N\left(x\right)$ es el
n{\'u}mero de usuarios que son atendidos con la pol{\'\i}tica $s$,
{\'u}nica en nuestro caso, durante un periodo de servicio,
entonces se asume que:
\begin{itemize}
\item[(S1.)]
\begin{equation}\label{S1}
lim_{x\rightarrow\infty}\esp\left[N\left(x\right)\right]=\overline{N}>0.
\end{equation}
\item[(S2.)]
\begin{equation}\label{S2}
\esp\left[N\left(x\right)\right]\leq \overline{N}, \end{equation}
para cualquier valor de $x$. \item La $n$-{\'e}sima ocurrencia va
acompa{\~n}ada con el tiempo de cambio de longitud
$\delta_{j,j+1}\left(n\right)$, independientes e id{\'e}nticamente
distribuidas, con
$\esp\left[\delta_{j,j+1}\left(1\right)\right]\geq0$. \item Se
define
\begin{equation}
\delta^{*}:=\sum_{j,j+1}\esp\left[\delta_{j,j+1}\left(1\right)\right].
\end{equation}

\item Los tiempos de inter-arribo a la cola $k$,son de la forma
$\left\{\xi_{k}\left(n\right)\right\}_{n\geq1}$, con la propiedad
de que son independientes e id{\'e}nticamente distribuidos.

\item Los tiempos de servicio
$\left\{\eta_{k}\left(n\right)\right\}_{n\geq1}$ tienen la
propiedad de ser independientes e id{\'e}nticamente distribuidos.

\item Se define la tasa de arribo a la $k$-{\'e}sima cola como
$\lambda_{k}=1/\esp\left[\xi_{k}\left(1\right)\right]$ y
adem{\'a}s se define

\item la tasa de servicio para la $k$-{\'e}sima cola como
$\mu_{k}=1/\esp\left[\eta_{k}\left(1\right)\right]$

\item tambi{\'e}n se define $\rho_{k}=\lambda_{k}/\mu_{k}$, donde
es necesario que $\rho<1$ para cuestiones de estabilidad.

\item De las pol{\'\i}ticas posibles solamente consideraremos la
pol{\'\i}tica cerrada (Gated).
\end{itemize}

Las Colas C\'iclicas se pueden describir por medio de un proceso
de Markov $\left(X\left(t\right)\right)_{t\in\rea}$, donde el
estado del sistema al tiempo $t\geq0$ est\'a dado por
\begin{equation}
X\left(t\right)=\left(Q\left(t\right),A\left(t\right),H\left(t\right),B\left(t\right),B^{0}\left(t\right),C\left(t\right)\right)
\end{equation}
definido en el espacio producto:
\begin{equation}
\mathcal{X}=\mathbb{Z}^{K}\times\rea_{+}^{K}\times\left(\left\{1,2,\ldots,K\right\}\times\left\{1,2,\ldots,S\right\}\right)^{M}\times\rea_{+}^{K}\times\rea_{+}^{K}\times\mathbb{Z}^{K},
\end{equation}

\begin{itemize}
\item $Q\left(t\right)=\left(Q_{k}\left(t\right),1\leq k\leq
K\right)$, es el n\'umero de usuarios en la cola $k$, incluyendo
aquellos que est\'an siendo atendidos provenientes de la
$k$-\'esima cola.

\item $A\left(t\right)=\left(A_{k}\left(t\right),1\leq k\leq
K\right)$, son los residuales de los tiempos de arribo en la cola
$k$. \item $H\left(t\right)$ es el par ordenado que consiste en la
cola que esta siendo atendida y la pol\'itica de servicio que se
utilizar\'a.

\item $B\left(t\right)$ es el tiempo de servicio residual.

\item $B^{0}\left(t\right)$ es el tiempo residual del cambio de
cola.

\item $C\left(t\right)$ indica el n\'umero de usuarios atendidos
durante la visita del servidor a la cola dada en
$H\left(t\right)$.
\end{itemize}

$A_{k}\left(t\right),B_{m}\left(t\right)$ y
$B_{m}^{0}\left(t\right)$ se suponen continuas por la derecha y
que satisfacen la propiedad fuerte de Markov, (\cite{Dai})

\begin{itemize}
\item Los tiempos de interarribo a la cola $k$,son de la forma
$\left\{\xi_{k}\left(n\right)\right\}_{n\geq1}$, con la propiedad
de que son independientes e id{\'e}nticamente distribuidos.

\item Los tiempos de servicio
$\left\{\eta_{k}\left(n\right)\right\}_{n\geq1}$ tienen la
propiedad de ser independientes e id{\'e}nticamente distribuidos.

\item Se define la tasa de arribo a la $k$-{\'e}sima cola como
$\lambda_{k}=1/\esp\left[\xi_{k}\left(1\right)\right]$ y
adem{\'a}s se define

\item la tasa de servicio para la $k$-{\'e}sima cola como
$\mu_{k}=1/\esp\left[\eta_{k}\left(1\right)\right]$

\item tambi{\'e}n se define $\rho_{k}=\lambda_{k}/\mu_{k}$, donde
es necesario que $\rho<1$ para cuestiones de estabilidad.

\item De las pol{\'\i}ticas posibles solamente consideraremos la
pol{\'\i}tica cerrada (Gated).
\end{itemize}


%_____________________________________________________


\subsection{Preliminares}



Sup\'ongase que el sistema consta de varias colas a los cuales
llegan uno o varios servidores a dar servicio a los usuarios
esperando en la cola.\\


Si $x$ es el n\'umero de usuarios en la cola al comienzo del
periodo de servicio y $N_{s}\left(x\right)=N\left(x\right)$ es el
n\'umero de usuarios que son atendidos con la pol\'itica $s$,
\'unica en nuestro caso, durante un periodo de servicio, entonces
se asume que:
\begin{itemize}
\item[1)]\label{S1}$lim_{x\rightarrow\infty}\esp\left[N\left(x\right)\right]=\overline{N}>0$
\item[2)]\label{S2}$\esp\left[N\left(x\right)\right]\leq\overline{N}$para
cualquier valor de $x$.
\end{itemize}
La manera en que atiende el servidor $m$-\'esimo, en este caso en
espec\'ifico solo lo ilustraremos con un s\'olo servidor, es la
siguiente:
\begin{itemize}
\item Al t\'ermino de la visita a la cola $j$, el servidor se
cambia a la cola $j^{'}$ con probabilidad
$r_{j,j^{'}}^{m}=r_{j,j^{'}}$

\item La $n$-\'esima ocurrencia va acompa\~nada con el tiempo de
cambio de longitud $\delta_{j,j^{'}}\left(n\right)$,
independientes e id\'enticamente distribuidas, con
$\esp\left[\delta_{j,j^{'}}\left(1\right)\right]\geq0$.

\item Sea $\left\{p_{j}\right\}$ la distribuci\'on invariante
estacionaria \'unica para la Cadena de Markov con matriz de
transici\'on $\left(r_{j,j^{'}}\right)$.

\item Finalmente, se define
\begin{equation}
\delta^{*}:=\sum_{j,j^{'}}p_{j}r_{j,j^{'}}\esp\left[\delta_{j,j^{'}}\left(i\right)\right].
\end{equation}
\end{itemize}

Veamos un caso muy espec\'ifico en el cual los tiempos de arribo a cada una de las colas se comportan de acuerdo a un proceso Poisson de la forma
$\left\{\xi_{k}\left(n\right)\right\}_{n\geq1}$, y los tiempos de servicio en cada una de las colas son variables aleatorias distribuidas exponencialmente e id\'enticamente distribuidas
$\left\{\eta_{k}\left(n\right)\right\}_{n\geq1}$, donde ambos procesos adem\'as cumplen la condici\'on de ser independientes entre si. Para la $k$-\'esima cola se define la tasa de arribo a la como
$\lambda_{k}=1/\esp\left[\xi_{k}\left(1\right)\right]$ y la tasa
de servicio como
$\mu_{k}=1/\esp\left[\eta_{k}\left(1\right)\right]$, finalmente se
define la carga de la cola como $\rho_{k}=\lambda_{k}/\mu_{k}$,
donde se pide que $\rho<1$, para garantizar la estabilidad del sistema.\\

Se denotar\'a por $Q_{k}\left(t\right)$ el n\'umero de usuarios en la cola $k$,
$A_{k}\left(t\right)$ los residuales de los tiempos entre arribos a la cola $k$;
para cada servidor $m$, se denota por $B_{m}\left(t\right)$ los residuales de los tiempos de servicio al tiempo $t$; $B_{m}^{0}\left(t\right)$ son los residuales de los tiempos de traslado de la cola $k$ a la pr\'oxima por atender, al tiempo $t$, finalmente sea $C_{m}\left(t\right)$ el n\'umero de usuarios atendidos durante la visita del servidor a la cola $k$ al tiempo $t$.\\


En este sentido el proceso para el sistema de visitas se puede definir como:

\begin{equation}\label{Esp.Edos.Down}
X\left(t\right)^{T}=\left(Q_{k}\left(t\right),A_{k}\left(t\right),B_{m}\left(t\right),B_{m}^{0}\left(t\right),C_{m}\left(t\right)\right)
\end{equation}
para $k=1,\ldots,K$ y $m=1,2,\ldots,M$. $X$ evoluciona en el
espacio de estados:
$X=\ent_{+}^{K}\times\rea_{+}^{K}\times\left(\left\{1,2,\ldots,K\right\}\times\left\{1,2,\ldots,S\right\}\right)^{M}\times\rea_{+}^{K}\times\ent_{+}^{K}$.\\

El sistema aqu\'i descrito debe de cumplir con los siguientes supuestos b\'asicos de un sistema de visitas:

Antes enunciemos los supuestos que regir\'an en la red.

\begin{itemize}
\item[A1)] $\xi_{1},\ldots,\xi_{K},\eta_{1},\ldots,\eta_{K}$ son
mutuamente independientes y son sucesiones independientes e
id\'enticamente distribuidas.

\item[A2)] Para alg\'un entero $p\geq1$
\begin{eqnarray*}
\esp\left[\xi_{l}\left(1\right)^{p+1}\right]<\infty\textrm{ para }l\in\mathcal{A}\textrm{ y }\\
\esp\left[\eta_{k}\left(1\right)^{p+1}\right]<\infty\textrm{ para
}k=1,\ldots,K.
\end{eqnarray*}
donde $\mathcal{A}$ es la clase de posibles arribos.

\item[A3)] Para $k=1,2,\ldots,K$ existe una funci\'on positiva
$q_{k}\left(x\right)$ definida en $\rea_{+}$, y un entero $j_{k}$,
tal que
\begin{eqnarray}
P\left(\xi_{k}\left(1\right)\geq x\right)>0\textrm{, para todo }x>0\\
P\left\{a\leq\sum_{i=1}^{j_{k}}\xi_{k}\left(i\right)\leq
b\right\}\geq\int_{a}^{b}q_{k}\left(x\right)dx, \textrm{ }0\leq
a<b.
\end{eqnarray}
\end{itemize}

En particular los procesos de tiempo entre arribos y de servicio
considerados con fines de ilustraci\'on de la metodolog\'ia
cumplen con el supuesto $A2)$ para $p=1$, es decir, ambos procesos
tienen primer y segundo momento finito.

En lo que respecta al supuesto (A3), en Dai y Meyn \cite{DaiSean}
hacen ver que este se puede sustituir por

\begin{itemize}
\item[A3')] Para el Proceso de Markov $X$, cada subconjunto
compacto de $X$ es un conjunto peque\~no, ver definici\'on
\ref{Def.Cto.Peq.}.
\end{itemize}

Es por esta raz\'on que con la finalidad de poder hacer uso de
$A3^{'})$ es necesario recurrir a los Procesos de Harris y en
particular a los Procesos Harris Recurrente:
%_______________________________________________________________________
\subsection{Procesos Harris Recurrente}
%_______________________________________________________________________

Por el supuesto (A1) conforme a Davis \cite{Davis}, se puede
definir el proceso de saltos correspondiente de manera tal que
satisfaga el supuesto (\ref{Sup3.1.Davis}), de hecho la
demostraci\'on est\'a basada en la l\'inea de argumentaci\'on de
Davis, (\cite{Davis}, p\'aginas 362-364).

Entonces se tiene un espacio de estados Markoviano. El espacio de
Markov descrito en Dai y Meyn \cite{DaiSean}

\[\left(\Omega,\mathcal{F},\mathcal{F}_{t},X\left(t\right),\theta_{t},P_{x}\right)\]
es un proceso de Borel Derecho (Sharpe \cite{Sharpe}) en el
espacio de estados medible $\left(X,\mathcal{B}_{X}\right)$. El
Proceso $X=\left\{X\left(t\right),t\geq0\right\}$ tiene
trayectorias continuas por la derecha, est\'a definida en
$\left(\Omega,\mathcal{F}\right)$ y est\'a adaptado a
$\left\{\mathcal{F}_{t},t\geq0\right\}$; la colecci\'on
$\left\{P_{x},x\in \mathbb{X}\right\}$ son medidas de probabilidad
en $\left(\Omega,\mathcal{F}\right)$ tales que para todo $x\in
\mathbb{X}$
\[P_{x}\left\{X\left(0\right)=x\right\}=1\] y
\[E_{x}\left\{f\left(X\circ\theta_{t}\right)|\mathcal{F}_{t}\right\}=E_{X}\left(\tau\right)f\left(X\right)\]
en $\left\{\tau<\infty\right\}$, $P_{x}$-c.s. Donde $\tau$ es un
$\mathcal{F}_{t}$-tiempo de paro
\[\left(X\circ\theta_{\tau}\right)\left(w\right)=\left\{X\left(\tau\left(w\right)+t,w\right),t\geq0\right\}\]
y $f$ es una funci\'on de valores reales acotada y medible con la
$\sigma$-algebra de Kolmogorov generada por los cilindros.\\

Sea $P^{t}\left(x,D\right)$, $D\in\mathcal{B}_{\mathbb{X}}$,
$t\geq0$ probabilidad de transici\'on de $X$ definida como
\[P^{t}\left(x,D\right)=P_{x}\left(X\left(t\right)\in
D\right)\]


\begin{Def}
Una medida no cero $\pi$ en
$\left(\mathbf{X},\mathcal{B}_{\mathbf{X}}\right)$ es {\bf
invariante} para $X$ si $\pi$ es $\sigma$-finita y
\[\pi\left(D\right)=\int_{\mathbf{X}}P^{t}\left(x,D\right)\pi\left(dx\right)\]
para todo $D\in \mathcal{B}_{\mathbf{X}}$, con $t\geq0$.
\end{Def}

\begin{Def}
El proceso de Markov $X$ es llamado Harris recurrente si existe
una medida de probabilidad $\nu$ en
$\left(\mathbf{X},\mathcal{B}_{\mathbf{X}}\right)$, tal que si
$\nu\left(D\right)>0$ y $D\in\mathcal{B}_{\mathbf{X}}$
\[P_{x}\left\{\tau_{D}<\infty\right\}\equiv1\] cuando
$\tau_{D}=inf\left\{t\geq0:X_{t}\in D\right\}$.
\end{Def}

\begin{Note}
\begin{itemize}
\item[i)] Si $X$ es Harris recurrente, entonces existe una \'unica
medida invariante $\pi$ (Getoor \cite{Getoor}).

\item[ii)] Si la medida invariante es finita, entonces puede
normalizarse a una medida de probabilidad, en este caso se le
llama Proceso {\em Harris recurrente positivo}.


\item[iii)] Cuando $X$ es Harris recurrente positivo se dice que
la disciplina de servicio es estable. En este caso $\pi$ denota la
distribuci\'on estacionaria y hacemos
\[P_{\pi}\left(\cdot\right)=\int_{\mathbf{X}}P_{x}\left(\cdot\right)\pi\left(dx\right)\]
y se utiliza $E_{\pi}$ para denotar el operador esperanza
correspondiente.
\end{itemize}
\end{Note}

\begin{Def}\label{Def.Cto.Peq.}
Un conjunto $D\in\mathcal{B_{\mathbf{X}}}$ es llamado peque\~no si
existe un $t>0$, una medida de probabilidad $\nu$ en
$\mathcal{B_{\mathbf{X}}}$, y un $\delta>0$ tal que
\[P^{t}\left(x,A\right)\geq\delta\nu\left(A\right)\] para $x\in
D,A\in\mathcal{B_{X}}$.
\end{Def}

La siguiente serie de resultados vienen enunciados y demostrados
en Dai \cite{Dai}:
\begin{Lema}[Lema 3.1, Dai\cite{Dai}]
Sea $B$ conjunto peque\~no cerrado, supongamos que
$P_{x}\left(\tau_{B}<\infty\right)\equiv1$ y que para alg\'un
$\delta>0$ se cumple que
\begin{equation}\label{Eq.3.1}
\sup\esp_{x}\left[\tau_{B}\left(\delta\right)\right]<\infty,
\end{equation}
donde
$\tau_{B}\left(\delta\right)=inf\left\{t\geq\delta:X\left(t\right)\in
B\right\}$. Entonces, $X$ es un proceso Harris Recurrente
Positivo.
\end{Lema}

\begin{Lema}[Lema 3.1, Dai \cite{Dai}]\label{Lema.3.}
Bajo el supuesto (A3), el conjunto $B=\left\{|x|\leq k\right\}$ es
un conjunto peque\~no cerrado para cualquier $k>0$.
\end{Lema}

\begin{Teo}[Teorema 3.1, Dai\cite{Dai}]\label{Tma.3.1}
Si existe un $\delta>0$ tal que
\begin{equation}
lim_{|x|\rightarrow\infty}\frac{1}{|x|}\esp|X^{x}\left(|x|\delta\right)|=0,
\end{equation}
entonces la ecuaci\'on (\ref{Eq.3.1}) se cumple para
$B=\left\{|x|\leq k\right\}$ con alg\'un $k>0$. En particular, $X$
es Harris Recurrente Positivo.
\end{Teo}

\begin{Note}
En Meyn and Tweedie \cite{MeynTweedie} muestran que si
$P_{x}\left\{\tau_{D}<\infty\right\}\equiv1$ incluso para solo un
conjunto peque\~no, entonces el proceso es Harris Recurrente.
\end{Note}

Entonces, tenemos que el proceso $X$ es un proceso de Markov que
cumple con los supuestos $A1)$-$A3)$, lo que falta de hacer es
construir el Modelo de Flujo bas\'andonos en lo hasta ahora
presentado.
%_______________________________________________________________________
\subsection{Modelo de Flujo}
%_______________________________________________________________________

Dada una condici\'on inicial $x\in\textrm{X}$, sea
$Q_{k}^{x}\left(t\right)$ la longitud de la cola al tiempo $t$,
$T_{m,k}^{x}\left(t\right)$ el tiempo acumulado, al tiempo $t$,
que tarda el servidor $m$ en atender a los usuarios de la cola
$k$. Finalmente sea $T_{m,k}^{x,0}\left(t\right)$ el tiempo
acumulado, al tiempo $t$, que tarda el servidor $m$ en trasladarse
a otra cola a partir de la $k$-\'esima.\\

Sup\'ongase que la funci\'on
$\left(\overline{Q}\left(\cdot\right),\overline{T}_{m}
\left(\cdot\right),\overline{T}_{m}^{0} \left(\cdot\right)\right)$
para $m=1,2,\ldots,M$ es un punto l\'imite de
\begin{equation}\label{Eq.Punto.Limite}
\left(\frac{1}{|x|}Q^{x}\left(|x|t\right),\frac{1}{|x|}T_{m}^{x}\left(|x|t\right),\frac{1}{|x|}T_{m}^{x,0}\left(|x|t\right)\right)
\end{equation}
para $m=1,2,\ldots,M$, cuando $x\rightarrow\infty$. Entonces
$\left(\overline{Q}\left(t\right),\overline{T}_{m}
\left(t\right),\overline{T}_{m}^{0} \left(t\right)\right)$ es un
flujo l\'imite del sistema. Al conjunto de todos las posibles
flujos l\'imite se le llama \textbf{Modelo de Flujo}.\\

El modelo de flujo satisface el siguiente conjunto de ecuaciones:

\begin{equation}\label{Eq.MF.1}
\overline{Q}_{k}\left(t\right)=\overline{Q}_{k}\left(0\right)+\lambda_{k}t-\sum_{m=1}^{M}\mu_{k}\overline{T}_{m,k}\left(t\right)\\
\end{equation}
para $k=1,2,\ldots,K$.\\
\begin{equation}\label{Eq.MF.2}
\overline{Q}_{k}\left(t\right)\geq0\textrm{ para
}k=1,2,\ldots,K,\\
\end{equation}

\begin{equation}\label{Eq.MF.3}
\overline{T}_{m,k}\left(0\right)=0,\textrm{ y }\overline{T}_{m,k}\left(\cdot\right)\textrm{ es no decreciente},\\
\end{equation}
para $k=1,2,\ldots,K$ y $m=1,2,\ldots,M$,\\
\begin{equation}\label{Eq.MF.4}
\sum_{k=1}^{K}\overline{T}_{m,k}^{0}\left(t\right)+\overline{T}_{m,k}\left(t\right)=t\textrm{
para }m=1,2,\ldots,M.\\
\end{equation}

De acuerdo a Dai \cite{Dai}, se tiene que el conjunto de posibles
l\'imites
$\left(\overline{Q}\left(\cdot\right),\overline{T}\left(\cdot\right),\overline{T}^{0}\left(\cdot\right)\right)$,
en el sentido de que deben de satisfacer las ecuaciones
(\ref{Eq.MF.1})-(\ref{Eq.MF.4}), se le llama {\em Modelo de
Flujo}.


\begin{Def}[Definici\'on 4.1, , Dai \cite{Dai}]\label{Def.Modelo.Flujo}
Sea una disciplina de servicio espec\'ifica. Cualquier l\'imite
$\left(\overline{Q}\left(\cdot\right),\overline{T}\left(\cdot\right)\right)$
en (\ref{Eq.Punto.Limite}) es un {\em flujo l\'imite} de la
disciplina. Cualquier soluci\'on (\ref{Eq.MF.1})-(\ref{Eq.MF.4})
es llamado flujo soluci\'on de la disciplina. Se dice que el
modelo de flujo l\'imite, modelo de flujo, de la disciplina de la
cola es estable si existe una constante $\delta>0$ que depende de
$\mu,\lambda$ y $P$ solamente, tal que cualquier flujo l\'imite
con
$|\overline{Q}\left(0\right)|+|\overline{U}|+|\overline{V}|=1$, se
tiene que $\overline{Q}\left(\cdot+\delta\right)\equiv0$.
\end{Def}

Al conjunto de ecuaciones dadas en \ref{Eq.MF.1}-\ref{Eq.MF.4} se
le llama {\em Modelo de flujo} y al conjunto de todas las
soluciones del modelo de flujo
$\left(\overline{Q}\left(\cdot\right),\overline{T}
\left(\cdot\right)\right)$ se le denotar\'a por $\mathcal{Q}$.

Si se hace $|x|\rightarrow\infty$ sin restringir ninguna de las
componentes, tambi\'en se obtienen un modelo de flujo, pero en
este caso el residual de los procesos de arribo y servicio
introducen un retraso:
\begin{Teo}[Teorema 4.2, Dai\cite{Dai}]\label{Tma.4.2.Dai}
Sea una disciplina fija para la cola, suponga que se cumplen las
condiciones (A1))-(A3)). Si el modelo de flujo l\'imite de la
disciplina de la cola es estable, entonces la cadena de Markov $X$
que describe la din\'amica de la red bajo la disciplina es Harris
recurrente positiva.
\end{Teo}

Ahora se procede a escalar el espacio y el tiempo para reducir la
aparente fluctuaci\'on del modelo. Consid\'erese el proceso
\begin{equation}\label{Eq.3.7}
\overline{Q}^{x}\left(t\right)=\frac{1}{|x|}Q^{x}\left(|x|t\right)
\end{equation}
A este proceso se le conoce como el fluido escalado, y cualquier
l\'imite $\overline{Q}^{x}\left(t\right)$ es llamado flujo
l\'imite del proceso de longitud de la cola. Haciendo
$|q|\rightarrow\infty$ mientras se mantiene el resto de las
componentes fijas, cualquier punto l\'imite del proceso de
longitud de la cola normalizado $\overline{Q}^{x}$ es soluci\'on
del siguiente modelo de flujo.


\begin{Def}[Definici\'on 3.3, Dai y Meyn \cite{DaiSean}]
El modelo de flujo es estable si existe un tiempo fijo $t_{0}$ tal
que $\overline{Q}\left(t\right)=0$, con $t\geq t_{0}$, para
cualquier $\overline{Q}\left(\cdot\right)\in\mathcal{Q}$ que
cumple con $|\overline{Q}\left(0\right)|=1$.
\end{Def}

El siguiente resultado se encuentra en Chen \cite{Chen}.
\begin{Lemma}[Lema 3.1, Dai y Meyn \cite{DaiSean}]
Si el modelo de flujo definido por \ref{Eq.MF.1}-\ref{Eq.MF.4} es
estable, entonces el modelo de flujo retrasado es tambi\'en
estable, es decir, existe $t_{0}>0$ tal que
$\overline{Q}\left(t\right)=0$ para cualquier $t\geq t_{0}$, para
cualquier soluci\'on del modelo de flujo retrasado cuya
condici\'on inicial $\overline{x}$ satisface que
$|\overline{x}|=|\overline{Q}\left(0\right)|+|\overline{A}\left(0\right)|+|\overline{B}\left(0\right)|\leq1$.
\end{Lemma}


Ahora ya estamos en condiciones de enunciar los resultados principales:


\begin{Teo}[Teorema 2.1, Down \cite{Down}]\label{Tma2.1.Down}
Suponga que el modelo de flujo es estable, y que se cumplen los supuestos (A1) y (A2), entonces
\begin{itemize}
\item[i)] Para alguna constante $\kappa_{p}$, y para cada
condici\'on inicial $x\in X$
\begin{equation}\label{Estability.Eq1}
limsup_{t\rightarrow\infty}\frac{1}{t}\int_{0}^{t}\esp_{x}\left[|Q\left(s\right)|^{p}\right]ds\leq\kappa_{p},
\end{equation}
donde $p$ es el entero dado en (A2).
\end{itemize}
Si adem\'as se cumple la condici\'on (A3), entonces para cada
condici\'on inicial:
\begin{itemize}
\item[ii)] Los momentos transitorios convergen a su estado
estacionario:
 \begin{equation}\label{Estability.Eq2}
lim_{t\rightarrow\infty}\esp_{x}\left[Q_{k}\left(t\right)^{r}\right]=\esp_{\pi}\left[Q_{k}\left(0\right)^{r}\right]\leq\kappa_{r},
\end{equation}
para $r=1,2,\ldots,p$ y $k=1,2,\ldots,K$. Donde $\pi$ es la
probabilidad invariante para $\mathbf{X}$.

\item[iii)]  El primer momento converge con raz\'on $t^{p-1}$:
\begin{equation}\label{Estability.Eq3}
lim_{t\rightarrow\infty}t^{p-1}|\esp_{x}\left[Q_{k}\left(t\right)\right]-\esp_{\pi}\left[Q_{k}\left(0\right)\right]=0.
\end{equation}

\item[iv)] La {\em Ley Fuerte de los grandes n\'umeros} se cumple:
\begin{equation}\label{Estability.Eq4}
lim_{t\rightarrow\infty}\frac{1}{t}\int_{0}^{t}Q_{k}^{r}\left(s\right)ds=\esp_{\pi}\left[Q_{k}\left(0\right)^{r}\right],\textrm{
}\prob_{x}\textrm{-c.s.}
\end{equation}
para $r=1,2,\ldots,p$ y $k=1,2,\ldots,K$.
\end{itemize}
\end{Teo}

La contribuci\'on de Down a la teor\'ia de los Sistemas de Visitas
C\'iclicas, es la relaci\'on que hay entre la estabilidad del
sistema con el comportamiento de las medidas de desempe\~no, es
decir, la condici\'on suficiente para poder garantizar la
convergencia del proceso de la longitud de la cola as\'i como de
por los menos los dos primeros momentos adem\'as de una versi\'on
de la Ley Fuerte de los Grandes N\'umeros para los sistemas de
visitas.


\begin{Teo}[Teorema 2.3, Down \cite{Down}]\label{Tma2.3.Down}
Considere el siguiente valor:
\begin{equation}\label{Eq.Rho.1serv}
\rho=\sum_{k=1}^{K}\rho_{k}+max_{1\leq j\leq K}\left(\frac{\lambda_{j}}{\sum_{s=1}^{S}p_{js}\overline{N}_{s}}\right)\delta^{*}
\end{equation}
\begin{itemize}
\item[i)] Si $\rho<1$ entonces la red es estable, es decir, se cumple el teorema \ref{Tma2.1.Down}.

\item[ii)] Si $\rho<1$ entonces la red es inestable, es decir, se cumple el teorema \ref{Tma2.2.Down}
\end{itemize}
\end{Teo}

\begin{Teo}
Sea $\left(X_{n},\mathcal{F}_{n},n=0,1,\ldots,\right\}$ Proceso de
Markov con espacio de estados $\left(S_{0},\chi_{0}\right)$
generado por una distribuici\'on inicial $P_{o}$ y probabilidad de
transici\'on $p_{mn}$, para $m,n=0,1,\ldots,$ $m<n$, que por
notaci\'on se escribir\'a como $p\left(m,n,x,B\right)\rightarrow
p_{mn}\left(x,B\right)$. Sea $S$ tiempo de paro relativo a la
$\sigma$-\'algebra $\mathcal{F}_{n}$. Sea $T$ funci\'on medible,
$T:\Omega\rightarrow\left\{0,1,\ldots,\right\}$. Sup\'ongase que
$T\geq S$, entonces $T$ es tiempo de paro. Si $B\in\chi_{0}$,
entonces
\begin{equation}\label{Prop.Fuerte.Markov}
P\left\{X\left(T\right)\in
B,T<\infty|\mathcal{F}\left(S\right)\right\} =
p\left(S,T,X\left(s\right),B\right)
\end{equation}
en $\left\{T<\infty\right\}$.
\end{Teo}


Sea $K$ conjunto numerable y sea $d:K\rightarrow\nat$ funci\'on.
Para $v\in K$, $M_{v}$ es un conjunto abierto de
$\rea^{d\left(v\right)}$. Entonces \[E=\cup_{v\in
K}M_{v}=\left\{\left(v,\zeta\right):v\in K,\zeta\in
M_{v}\right\}.\]

Sea $\mathcal{E}$ la clase de conjuntos medibles en $E$:
\[\mathcal{E}=\left\{\cup_{v\in K}A_{v}:A_{v}\in \mathcal{M}_{v}\right\}.\]

donde $\mathcal{M}$ son los conjuntos de Borel de $M_{v}$.
Entonces $\left(E,\mathcal{E}\right)$ es un espacio de Borel. El
estado del proceso se denotar\'a por
$\mathbf{x}_{t}=\left(v_{t},\zeta_{t}\right)$. La distribuci\'on
de $\left(\mathbf{x}_{t}\right)$ est\'a determinada por por los
siguientes objetos:

\begin{itemize}
\item[i)] Los campos vectoriales $\left(\mathcal{H}_{v},v\in
K\right)$. \item[ii)] Una funci\'on medible $\lambda:E\rightarrow
\rea_{+}$. \item[iii)] Una medida de transici\'on
$Q:\mathcal{E}\times\left(E\cup\Gamma^{*}\right)\rightarrow\left[0,1\right]$
donde
\begin{equation}
\Gamma^{*}=\cup_{v\in K}\partial^{*}M_{v}.
\end{equation}
y
\begin{equation}
\partial^{*}M_{v}=\left\{z\in\partial M_{v}:\mathbf{\mathbf{\phi}_{v}\left(t,\zeta\right)=\mathbf{z}}\textrm{ para alguna }\left(t,\zeta\right)\in\rea_{+}\times M_{v}\right\}.
\end{equation}
$\partial M_{v}$ denota  la frontera de $M_{v}$.
\end{itemize}

El campo vectorial $\left(\mathcal{H}_{v},v\in K\right)$ se supone
tal que para cada $\mathbf{z}\in M_{v}$ existe una \'unica curva
integral $\mathbf{\phi}_{v}\left(t,\zeta\right)$ que satisface la
ecuaci\'on

\begin{equation}
\frac{d}{dt}f\left(\zeta_{t}\right)=\mathcal{H}f\left(\zeta_{t}\right),
\end{equation}
con $\zeta_{0}=\mathbf{z}$, para cualquier funci\'on suave
$f:\rea^{d}\rightarrow\rea$ y $\mathcal{H}$ denota el operador
diferencial de primer orden, con $\mathcal{H}=\mathcal{H}_{v}$ y
$\zeta_{t}=\mathbf{\phi}\left(t,\mathbf{z}\right)$. Adem\'as se
supone que $\mathcal{H}_{v}$ es conservativo, es decir, las curvas
integrales est\'an definidas para todo $t>0$.

Para $\mathbf{x}=\left(v,\zeta\right)\in E$ se denota
\[t^{*}\mathbf{x}=inf\left\{t>0:\mathbf{\phi}_{v}\left(t,\zeta\right)\in\partial^{*}M_{v}\right\}\]

En lo que respecta a la funci\'on $\lambda$, se supondr\'a que
para cada $\left(v,\zeta\right)\in E$ existe un $\epsilon>0$ tal
que la funci\'on
$s\rightarrow\lambda\left(v,\phi_{v}\left(s,\zeta\right)\right)\in
E$ es integrable para $s\in\left[0,\epsilon\right)$. La medida de
transici\'on $Q\left(A;\mathbf{x}\right)$ es una funci\'on medible
de $\mathbf{x}$ para cada $A\in\mathcal{E}$, definida para
$\mathbf{x}\in E\cup\Gamma^{*}$ y es una medida de probabilidad en
$\left(E,\mathcal{E}\right)$ para cada $\mathbf{x}\in E$.

El movimiento del proceso $\left(\mathbf{x}_{t}\right)$ comenzando
en $\mathbf{x}=\left(n,\mathbf{z}\right)\in E$ se puede construir
de la siguiente manera, def\'inase la funci\'on $F$ por

\begin{equation}
F\left(t\right)=\left\{\begin{array}{ll}\\
exp\left(-\int_{0}^{t}\lambda\left(n,\phi_{n}\left(s,\mathbf{z}\right)\right)ds\right), & t<t^{*}\left(\mathbf{x}\right),\\
0, & t\geq t^{*}\left(\mathbf{x}\right)
\end{array}\right.
\end{equation}

Sea $T_{1}$ una variable aleatoria tal que
$\prob\left[T_{1}>t\right]=F\left(t\right)$, ahora sea la variable
aleatoria $\left(N,Z\right)$ con distribuici\'on
$Q\left(\cdot;\phi_{n}\left(T_{1},\mathbf{z}\right)\right)$. La
trayectoria de $\left(\mathbf{x}_{t}\right)$ para $t\leq T_{1}$
es\footnote{Revisar p\'agina 362, y 364 de Davis \cite{Davis}.}
\begin{eqnarray*}
\mathbf{x}_{t}=\left(v_{t},\zeta_{t}\right)=\left\{\begin{array}{ll}
\left(n,\phi_{n}\left(t,\mathbf{z}\right)\right), & t<T_{1},\\
\left(N,\mathbf{Z}\right), & t=t_{1}.
\end{array}\right.
\end{eqnarray*}

Comenzando en $\mathbf{x}_{T_{1}}$ se selecciona el siguiente
tiempo de intersalto $T_{2}-T_{1}$ lugar del post-salto
$\mathbf{x}_{T_{2}}$ de manera similar y as\'i sucesivamente. Este
procedimiento nos da una trayectoria determinista por partes
$\mathbf{x}_{t}$ con tiempos de salto $T_{1},T_{2},\ldots$. Bajo
las condiciones enunciadas para $\lambda,T_{1}>0$  y
$T_{1}-T_{2}>0$ para cada $i$, con probabilidad 1. Se supone que
se cumple la siguiente condici\'on.

\begin{Sup}[Supuesto 3.1, Davis \cite{Davis}]\label{Sup3.1.Davis}
Sea $N_{t}:=\sum_{t}\indora_{\left(t\geq t\right)}$ el n\'umero de
saltos en $\left[0,t\right]$. Entonces
\begin{equation}
\esp\left[N_{t}\right]<\infty\textrm{ para toda }t.
\end{equation}
\end{Sup}

es un proceso de Markov, m\'as a\'un, es un Proceso Fuerte de
Markov, es decir, la Propiedad Fuerte de Markov se cumple para
cualquier tiempo de paro.


Sea $E$ es un espacio m\'etrico separable y la m\'etrica $d$ es
compatible con la topolog\'ia.


\begin{Def}
Un espacio topol\'ogico $E$ es llamado de {\em Rad\'on} si es
homeomorfo a un subconjunto universalmente medible de un espacio
m\'etrico compacto.
\end{Def}

Equivalentemente, la definici\'on de un espacio de Rad\'on puede
encontrarse en los siguientes t\'erminos:


\begin{Def}
$E$ es un espacio de Rad\'on si cada medida finita en
$\left(E,\mathcal{B}\left(E\right)\right)$ es regular interior o
cerrada, {\em tight}.
\end{Def}

\begin{Def}
Una medida finita, $\lambda$ en la $\sigma$-\'algebra de Borel de
un espacio metrizable $E$ se dice cerrada si
\begin{equation}\label{Eq.A2.3}
\lambda\left(E\right)=sup\left\{\lambda\left(K\right):K\textrm{ es
compacto en }E\right\}.
\end{equation}
\end{Def}

El siguiente teorema nos permite tener una mejor caracterizaci\'on
de los espacios de Rad\'on:
\begin{Teo}\label{Tma.A2.2}
Sea $E$ espacio separable metrizable. Entonces $E$ es Radoniano si
y s\'olo s\'i cada medida finita en
$\left(E,\mathcal{B}\left(E\right)\right)$ es cerrada.
\end{Teo}

Sea $E$ espacio de estados, tal que $E$ es un espacio de Rad\'on,
$\mathcal{B}\left(E\right)$ $\sigma$-\'algebra de Borel en $E$,
que se denotar\'a por $\mathcal{E}$.

Sea $\left(X,\mathcal{G},\prob\right)$ espacio de probabilidad,
$I\subset\rea$ conjunto de \'indices. Sea $\mathcal{F}_{\leq t}$
la $\sigma$-\'algebra natural definida como
$\sigma\left\{f\left(X_{r}\right):r\in I, r\leq
t,f\in\mathcal{E}\right\}$. Se considerar\'a una
$\sigma$-\'algebra m\'as general, $ \left(\mathcal{G}_{t}\right)$
tal que $\left(X_{t}\right)$ sea $\mathcal{E}$-adaptado.

\begin{Def}
Una familia $\left(P_{s,t}\right)$ de kernels de Markov en
$\left(E,\mathcal{E}\right)$ indexada por pares $s,t\in I$, con
$s\leq t$ es una funci\'on de transici\'on en $\ER$, si  para todo
$r\leq s< t$ en $I$ y todo $x\in E$,
$B\in\mathcal{E}$\footnote{Ecuaci\'on de Chapman-Kolmogorov}
\begin{equation}\label{Eq.Kernels}
P_{r,t}\left(x,B\right)=\int_{E}P_{r,s}\left(x,dy\right)P_{s,t}\left(y,B\right).
\end{equation}
\end{Def}

Se dice que la funci\'on de transici\'on $\KM$ en $\ER$ es la
funci\'on de transici\'on para un proceso $\PE$  con valores en
$E$ y que satisface la propiedad de
Markov\footnote{\begin{equation}\label{Eq.1.4.S}
\prob\left\{H|\mathcal{G}_{t}\right\}=\prob\left\{H|X_{t}\right\}\textrm{
}H\in p\mathcal{F}_{\geq t}.
\end{equation}} (\ref{Eq.1.4.S}) relativa a $\left(\mathcal{G}_{t}\right)$ si

\begin{equation}\label{Eq.1.6.S}
\prob\left\{f\left(X_{t}\right)|\mathcal{G}_{s}\right\}=P_{s,t}f\left(X_{t}\right)\textrm{
}s\leq t\in I,\textrm{ }f\in b\mathcal{E}.
\end{equation}

\begin{Def}
Una familia $\left(P_{t}\right)_{t\geq0}$ de kernels de Markov en
$\ER$ es llamada {\em Semigrupo de Transici\'on de Markov} o {\em
Semigrupo de Transici\'on} si
\[P_{t+s}f\left(x\right)=P_{t}\left(P_{s}f\right)\left(x\right),\textrm{ }t,s\geq0,\textrm{ }x\in E\textrm{ }f\in b\mathcal{E}.\]
\end{Def}
\begin{Note}
Si la funci\'on de transici\'on $\KM$ es llamada homog\'enea si
$P_{s,t}=P_{t-s}$.
\end{Note}

Un proceso de Markov que satisface la ecuaci\'on (\ref{Eq.1.6.S})
con funci\'on de transici\'on homog\'enea $\left(P_{t}\right)$
tiene la propiedad caracter\'istica
\begin{equation}\label{Eq.1.8.S}
\prob\left\{f\left(X_{t+s}\right)|\mathcal{G}_{t}\right\}=P_{s}f\left(X_{t}\right)\textrm{
}t,s\geq0,\textrm{ }f\in b\mathcal{E}.
\end{equation}
La ecuaci\'on anterior es la {\em Propiedad Simple de Markov} de
$X$ relativa a $\left(P_{t}\right)$.

En este sentido el proceso $\PE$ cumple con la propiedad de Markov
(\ref{Eq.1.8.S}) relativa a
$\left(\Omega,\mathcal{G},\mathcal{G}_{t},\prob\right)$ con
semigrupo de transici\'on $\left(P_{t}\right)$.

\begin{Def}
Un proceso estoc\'astico $\PE$ definido en
$\left(\Omega,\mathcal{G},\prob\right)$ con valores en el espacio
topol\'ogico $E$ es continuo por la derecha si cada trayectoria
muestral $t\rightarrow X_{t}\left(w\right)$ es un mapeo continuo
por la derecha de $I$ en $E$.
\end{Def}

\begin{Def}[HD1]\label{Eq.2.1.S}
Un semigrupo de Markov $\left(P_{t}\right)$ en un espacio de
Rad\'on $E$ se dice que satisface la condici\'on {\em HD1} si,
dada una medida de probabilidad $\mu$ en $E$, existe una
$\sigma$-\'algebra $\mathcal{E^{*}}$ con
$\mathcal{E}\subset\mathcal{E}^{*}$ y
$P_{t}\left(b\mathcal{E}^{*}\right)\subset b\mathcal{E}^{*}$, y un
$\mathcal{E}^{*}$-proceso $E$-valuado continuo por la derecha
$\PE$ en alg\'un espacio de probabilidad filtrado
$\left(\Omega,\mathcal{G},\mathcal{G}_{t},\prob\right)$ tal que
$X=\left(\Omega,\mathcal{G},\mathcal{G}_{t},\prob\right)$ es de
Markov (Homog\'eneo) con semigrupo de transici\'on $(P_{t})$ y
distribuci\'on inicial $\mu$.
\end{Def}

Consid\'erese la colecci\'on de variables aleatorias $X_{t}$
definidas en alg\'un espacio de probabilidad, y una colecci\'on de
medidas $\mathbf{P}^{x}$ tales que
$\mathbf{P}^{x}\left\{X_{0}=x\right\}$, y bajo cualquier
$\mathbf{P}^{x}$, $X_{t}$ es de Markov con semigrupo
$\left(P_{t}\right)$. $\mathbf{P}^{x}$ puede considerarse como la
distribuci\'on condicional de $\mathbf{P}$ dado $X_{0}=x$.

\begin{Def}\label{Def.2.2.S}
Sea $E$ espacio de Rad\'on, $\SG$ semigrupo de Markov en $\ER$. La
colecci\'on
$\mathbf{X}=\left(\Omega,\mathcal{G},\mathcal{G}_{t},X_{t},\theta_{t},\CM\right)$
es un proceso $\mathcal{E}$-Markov continuo por la derecha simple,
con espacio de estados $E$ y semigrupo de transici\'on $\SG$ en
caso de que $\mathbf{X}$ satisfaga las siguientes condiciones:
\begin{itemize}
\item[i)] $\left(\Omega,\mathcal{G},\mathcal{G}_{t}\right)$ es un
espacio de medida filtrado, y $X_{t}$ es un proceso $E$-valuado
continuo por la derecha $\mathcal{E}^{*}$-adaptado a
$\left(\mathcal{G}_{t}\right)$;

\item[ii)] $\left(\theta_{t}\right)_{t\geq0}$ es una colecci\'on
de operadores {\em shift} para $X$, es decir, mapea $\Omega$ en
s\'i mismo satisfaciendo para $t,s\geq0$,

\begin{equation}\label{Eq.Shift}
\theta_{t}\circ\theta_{s}=\theta_{t+s}\textrm{ y
}X_{t}\circ\theta_{t}=X_{t+s};
\end{equation}

\item[iii)] Para cualquier $x\in E$,$\CM\left\{X_{0}=x\right\}=1$,
y el proceso $\PE$ tiene la propiedad de Markov (\ref{Eq.1.8.S})
con semigrupo de transici\'on $\SG$ relativo a
$\left(\Omega,\mathcal{G},\mathcal{G}_{t},\CM\right)$.
\end{itemize}
\end{Def}

\begin{Def}[HD2]\label{Eq.2.2.S}
Para cualquier $\alpha>0$ y cualquier $f\in S^{\alpha}$, el
proceso $t\rightarrow f\left(X_{t}\right)$ es continuo por la
derecha casi seguramente.
\end{Def}

\begin{Def}\label{Def.PD}
Un sistema
$\mathbf{X}=\left(\Omega,\mathcal{G},\mathcal{G}_{t},X_{t},\theta_{t},\CM\right)$
es un proceso derecho en el espacio de Rad\'on $E$ con semigrupo
de transici\'on $\SG$ provisto de:
\begin{itemize}
\item[i)] $\mathbf{X}$ es una realizaci\'on  continua por la
derecha, \ref{Def.2.2.S}, de $\SG$.

\item[ii)] $\mathbf{X}$ satisface la condicion HD2,
\ref{Eq.2.2.S}, relativa a $\mathcal{G}_{t}$.

\item[iii)] $\mathcal{G}_{t}$ es aumentado y continuo por la
derecha.
\end{itemize}
\end{Def}

\begin{Lema}[Lema 4.2, Dai\cite{Dai}]\label{Lema4.2}
Sea $\left\{x_{n}\right\}\subset \mathbf{X}$ con
$|x_{n}|\rightarrow\infty$, conforme $n\rightarrow\infty$. Suponga
que
\[lim_{n\rightarrow\infty}\frac{1}{|x_{n}|}U\left(0\right)=\overline{U}\]
y
\[lim_{n\rightarrow\infty}\frac{1}{|x_{n}|}V\left(0\right)=\overline{V}.\]

Entonces, conforme $n\rightarrow\infty$, casi seguramente

\begin{equation}\label{E1.4.2}
\frac{1}{|x_{n}|}\Phi^{k}\left(\left[|x_{n}|t\right]\right)\rightarrow
P_{k}^{'}t\textrm{, u.o.c.,}
\end{equation}

\begin{equation}\label{E1.4.3}
\frac{1}{|x_{n}|}E^{x_{n}}_{k}\left(|x_{n}|t\right)\rightarrow
\alpha_{k}\left(t-\overline{U}_{k}\right)^{+}\textrm{, u.o.c.,}
\end{equation}

\begin{equation}\label{E1.4.4}
\frac{1}{|x_{n}|}S^{x_{n}}_{k}\left(|x_{n}|t\right)\rightarrow
\mu_{k}\left(t-\overline{V}_{k}\right)^{+}\textrm{, u.o.c.,}
\end{equation}

donde $\left[t\right]$ es la parte entera de $t$ y
$\mu_{k}=1/m_{k}=1/\esp\left[\eta_{k}\left(1\right)\right]$.
\end{Lema}

\begin{Lema}[Lema 4.3, Dai\cite{Dai}]\label{Lema.4.3}
Sea $\left\{x_{n}\right\}\subset \mathbf{X}$ con
$|x_{n}|\rightarrow\infty$, conforme $n\rightarrow\infty$. Suponga
que
\[lim_{n\rightarrow\infty}\frac{1}{|x_{n}|}U\left(0\right)=\overline{U}_{k}\]
y
\[lim_{n\rightarrow\infty}\frac{1}{|x_{n}|}V\left(0\right)=\overline{V}_{k}.\]
\begin{itemize}
\item[a)] Conforme $n\rightarrow\infty$ casi seguramente,
\[lim_{n\rightarrow\infty}\frac{1}{|x_{n}|}U^{x_{n}}_{k}\left(|x_{n}|t\right)=\left(\overline{U}_{k}-t\right)^{+}\textrm{, u.o.c.}\]
y
\[lim_{n\rightarrow\infty}\frac{1}{|x_{n}|}V^{x_{n}}_{k}\left(|x_{n}|t\right)=\left(\overline{V}_{k}-t\right)^{+}.\]

\item[b)] Para cada $t\geq0$ fijo,
\[\left\{\frac{1}{|x_{n}|}U^{x_{n}}_{k}\left(|x_{n}|t\right),|x_{n}|\geq1\right\}\]
y
\[\left\{\frac{1}{|x_{n}|}V^{x_{n}}_{k}\left(|x_{n}|t\right),|x_{n}|\geq1\right\}\]
\end{itemize}
son uniformemente convergentes.
\end{Lema}

$S_{l}^{x}\left(t\right)$ es el n\'umero total de servicios
completados de la clase $l$, si la clase $l$ est\'a dando $t$
unidades de tiempo de servicio. Sea $T_{l}^{x}\left(x\right)$ el
monto acumulado del tiempo de servicio que el servidor
$s\left(l\right)$ gasta en los usuarios de la clase $l$ al tiempo
$t$. Entonces $S_{l}^{x}\left(T_{l}^{x}\left(t\right)\right)$ es
el n\'umero total de servicios completados para la clase $l$ al
tiempo $t$. Una fracci\'on de estos usuarios,
$\Phi_{l}^{x}\left(S_{l}^{x}\left(T_{l}^{x}\left(t\right)\right)\right)$,
se convierte en usuarios de la clase $k$.\\

Entonces, dado lo anterior, se tiene la siguiente representaci\'on
para el proceso de la longitud de la cola:\\

\begin{equation}
Q_{k}^{x}\left(t\right)=_{k}^{x}\left(0\right)+E_{k}^{x}\left(t\right)+\sum_{l=1}^{K}\Phi_{k}^{l}\left(S_{l}^{x}\left(T_{l}^{x}\left(t\right)\right)\right)-S_{k}^{x}\left(T_{k}^{x}\left(t\right)\right)
\end{equation}
para $k=1,\ldots,K$. Para $i=1,\ldots,d$, sea
\[I_{i}^{x}\left(t\right)=t-\sum_{j\in C_{i}}T_{k}^{x}\left(t\right).\]

Entonces $I_{i}^{x}\left(t\right)$ es el monto acumulado del
tiempo que el servidor $i$ ha estado desocupado al tiempo $t$. Se
est\'a asumiendo que las disciplinas satisfacen la ley de
conservaci\'on del trabajo, es decir, el servidor $i$ est\'a en
pausa solamente cuando no hay usuarios en la estaci\'on $i$.
Entonces, se tiene que

\begin{equation}
\int_{0}^{\infty}\left(\sum_{k\in
C_{i}}Q_{k}^{x}\left(t\right)\right)dI_{i}^{x}\left(t\right)=0,
\end{equation}
para $i=1,\ldots,d$.\\

Hacer
\[T^{x}\left(t\right)=\left(T_{1}^{x}\left(t\right),\ldots,T_{K}^{x}\left(t\right)\right)^{'},\]
\[I^{x}\left(t\right)=\left(I_{1}^{x}\left(t\right),\ldots,I_{K}^{x}\left(t\right)\right)^{'}\]
y
\[S^{x}\left(T^{x}\left(t\right)\right)=\left(S_{1}^{x}\left(T_{1}^{x}\left(t\right)\right),\ldots,S_{K}^{x}\left(T_{K}^{x}\left(t\right)\right)\right)^{'}.\]

Para una disciplina que cumple con la ley de conservaci\'on del
trabajo, en forma vectorial, se tiene el siguiente conjunto de
ecuaciones

\begin{equation}\label{Eq.MF.1.3}
Q^{x}\left(t\right)=Q^{x}\left(0\right)+E^{x}\left(t\right)+\sum_{l=1}^{K}\Phi^{l}\left(S_{l}^{x}\left(T_{l}^{x}\left(t\right)\right)\right)-S^{x}\left(T^{x}\left(t\right)\right),\\
\end{equation}

\begin{equation}\label{Eq.MF.2.3}
Q^{x}\left(t\right)\geq0,\\
\end{equation}

\begin{equation}\label{Eq.MF.3.3}
T^{x}\left(0\right)=0,\textrm{ y }\overline{T}^{x}\left(t\right)\textrm{ es no decreciente},\\
\end{equation}

\begin{equation}\label{Eq.MF.4.3}
I^{x}\left(t\right)=et-CT^{x}\left(t\right)\textrm{ es no
decreciente}\\
\end{equation}

\begin{equation}\label{Eq.MF.5.3}
\int_{0}^{\infty}\left(CQ^{x}\left(t\right)\right)dI_{i}^{x}\left(t\right)=0,\\
\end{equation}

\begin{equation}\label{Eq.MF.6.3}
\textrm{Condiciones adicionales en
}\left(\overline{Q}^{x}\left(\cdot\right),\overline{T}^{x}\left(\cdot\right)\right)\textrm{
espec\'ificas de la disciplina de la cola,}
\end{equation}

donde $e$ es un vector de unos de dimensi\'on $d$, $C$ es la
matriz definida por
\[C_{ik}=\left\{\begin{array}{cc}
1,& S\left(k\right)=i,\\
0,& \textrm{ en otro caso}.\\
\end{array}\right.
\]
Es necesario enunciar el siguiente Teorema que se utilizar\'a para
el Teorema \ref{Tma.4.2.Dai}:
\begin{Teo}[Teorema 4.1, Dai \cite{Dai}]
Considere una disciplina que cumpla la ley de conservaci\'on del
trabajo, para casi todas las trayectorias muestrales $\omega$ y
cualquier sucesi\'on de estados iniciales
$\left\{x_{n}\right\}\subset \mathbf{X}$, con
$|x_{n}|\rightarrow\infty$, existe una subsucesi\'on
$\left\{x_{n_{j}}\right\}$ con $|x_{n_{j}}|\rightarrow\infty$ tal
que
\begin{equation}\label{Eq.4.15}
\frac{1}{|x_{n_{j}}|}\left(Q^{x_{n_{j}}}\left(0\right),U^{x_{n_{j}}}\left(0\right),V^{x_{n_{j}}}\left(0\right)\right)\rightarrow\left(\overline{Q}\left(0\right),\overline{U},\overline{V}\right),
\end{equation}

\begin{equation}\label{Eq.4.16}
\frac{1}{|x_{n_{j}}|}\left(Q^{x_{n_{j}}}\left(|x_{n_{j}}|t\right),T^{x_{n_{j}}}\left(|x_{n_{j}}|t\right)\right)\rightarrow\left(\overline{Q}\left(t\right),\overline{T}\left(t\right)\right)\textrm{
u.o.c.}
\end{equation}

Adem\'as,
$\left(\overline{Q}\left(t\right),\overline{T}\left(t\right)\right)$
satisface las siguientes ecuaciones:
\begin{equation}\label{Eq.MF.1.3a}
\overline{Q}\left(t\right)=Q\left(0\right)+\left(\alpha
t-\overline{U}\right)^{+}-\left(I-P\right)^{'}M^{-1}\left(\overline{T}\left(t\right)-\overline{V}\right)^{+},
\end{equation}

\begin{equation}\label{Eq.MF.2.3a}
\overline{Q}\left(t\right)\geq0,\\
\end{equation}

\begin{equation}\label{Eq.MF.3.3a}
\overline{T}\left(t\right)\textrm{ es no decreciente y comienza en cero},\\
\end{equation}

\begin{equation}\label{Eq.MF.4.3a}
\overline{I}\left(t\right)=et-C\overline{T}\left(t\right)\textrm{
es no decreciente,}\\
\end{equation}

\begin{equation}\label{Eq.MF.5.3a}
\int_{0}^{\infty}\left(C\overline{Q}\left(t\right)\right)d\overline{I}\left(t\right)=0,\\
\end{equation}

\begin{equation}\label{Eq.MF.6.3a}
\textrm{Condiciones adicionales en
}\left(\overline{Q}\left(\cdot\right),\overline{T}\left(\cdot\right)\right)\textrm{
especficas de la disciplina de la cola,}
\end{equation}
\end{Teo}


Propiedades importantes para el modelo de flujo retrasado:

\begin{Prop}
 Sea $\left(\overline{Q},\overline{T},\overline{T}^{0}\right)$ un flujo l\'imite de \ref{Eq.4.4} y suponga que cuando $x\rightarrow\infty$ a lo largo de
una subsucesi\'on
\[\left(\frac{1}{|x|}Q_{k}^{x}\left(0\right),\frac{1}{|x|}A_{k}^{x}\left(0\right),\frac{1}{|x|}B_{k}^{x}\left(0\right),\frac{1}{|x|}B_{k}^{x,0}\left(0\right)\right)\rightarrow\left(\overline{Q}_{k}\left(0\right),0,0,0\right)\]
para $k=1,\ldots,K$. EL flujo l\'imite tiene las siguientes
propiedades, donde las propiedades de la derivada se cumplen donde
la derivada exista:
\begin{itemize}
 \item[i)] Los vectores de tiempo ocupado $\overline{T}\left(t\right)$ y $\overline{T}^{0}\left(t\right)$ son crecientes y continuas con
$\overline{T}\left(0\right)=\overline{T}^{0}\left(0\right)=0$.
\item[ii)] Para todo $t\geq0$
\[\sum_{k=1}^{K}\left[\overline{T}_{k}\left(t\right)+\overline{T}_{k}^{0}\left(t\right)\right]=t\]
\item[iii)] Para todo $1\leq k\leq K$
\[\overline{Q}_{k}\left(t\right)=\overline{Q}_{k}\left(0\right)+\alpha_{k}t-\mu_{k}\overline{T}_{k}\left(t\right)\]
\item[iv)]  Para todo $1\leq k\leq K$
\[\dot{{\overline{T}}}_{k}\left(t\right)=\beta_{k}\] para $\overline{Q}_{k}\left(t\right)=0$.
\item[v)] Para todo $k,j$
\[\mu_{k}^{0}\overline{T}_{k}^{0}\left(t\right)=\mu_{j}^{0}\overline{T}_{j}^{0}\left(t\right)\]
\item[vi)]  Para todo $1\leq k\leq K$
\[\mu_{k}\dot{{\overline{T}}}_{k}\left(t\right)=l_{k}\mu_{k}^{0}\dot{{\overline{T}}}_{k}^{0}\left(t\right)\] para $\overline{Q}_{k}\left(t\right)>0$.
\end{itemize}
\end{Prop}

\begin{Lema}[Lema 3.1 \cite{Chen}]\label{Lema3.1}
Si el modelo de flujo es estable, definido por las ecuaciones
(3.8)-(3.13), entonces el modelo de flujo retrasado tambi\'en es
estable.
\end{Lema}

\begin{Teo}[Teorema 5.1 \cite{Chen}]\label{Tma.5.1.Chen}
La red de colas es estable si existe una constante $t_{0}$ que
depende de $\left(\alpha,\mu,T,U\right)$ y $V$ que satisfagan las
ecuaciones (5.1)-(5.5), $Z\left(t\right)=0$, para toda $t\geq
t_{0}$.
\end{Teo}



\begin{Lema}[Lema 5.2 \cite{Gut}]\label{Lema.5.2.Gut}
Sea $\left\{\xi\left(k\right):k\in\ent\right\}$ sucesi\'on de
variables aleatorias i.i.d. con valores en
$\left(0,\infty\right)$, y sea $E\left(t\right)$ el proceso de
conteo
\[E\left(t\right)=max\left\{n\geq1:\xi\left(1\right)+\cdots+\xi\left(n-1\right)\leq t\right\}.\]
Si $E\left[\xi\left(1\right)\right]<\infty$, entonces para
cualquier entero $r\geq1$
\begin{equation}
lim_{t\rightarrow\infty}\esp\left[\left(\frac{E\left(t\right)}{t}\right)^{r}\right]=\left(\frac{1}{E\left[\xi_{1}\right]}\right)^{r}
\end{equation}
de aqu\'i, bajo estas condiciones
\begin{itemize}
\item[a)] Para cualquier $t>0$,
$sup_{t\geq\delta}\esp\left[\left(\frac{E\left(t\right)}{t}\right)^{r}\right]$

\item[b)] Las variables aleatorias
$\left\{\left(\frac{E\left(t\right)}{t}\right)^{r}:t\geq1\right\}$
son uniformemente integrables.
\end{itemize}
\end{Lema}

\begin{Teo}[Teorema 5.1: Ley Fuerte para Procesos de Conteo
\cite{Gut}]\label{Tma.5.1.Gut} Sea
$0<\mu<\esp\left(X_{1}\right]\leq\infty$. entonces

\begin{itemize}
\item[a)] $\frac{N\left(t\right)}{t}\rightarrow\frac{1}{\mu}$
a.s., cuando $t\rightarrow\infty$.


\item[b)]$\esp\left[\frac{N\left(t\right)}{t}\right]^{r}\rightarrow\frac{1}{\mu^{r}}$,
cuando $t\rightarrow\infty$ para todo $r>0$..
\end{itemize}
\end{Teo}


\begin{Prop}[Proposici\'on 5.1 \cite{DaiSean}]\label{Prop.5.1}
Suponga que los supuestos (A1) y (A2) se cumplen, adem\'as suponga
que el modelo de flujo es estable. Entonces existe $t_{0}>0$ tal
que
\begin{equation}\label{Eq.Prop.5.1}
lim_{|x|\rightarrow\infty}\frac{1}{|x|^{p+1}}\esp_{x}\left[|X\left(t_{0}|x|\right)|^{p+1}\right]=0.
\end{equation}

\end{Prop}


\begin{Prop}[Proposici\'on 5.3 \cite{DaiSean}]
Sea $X$ proceso de estados para la red de colas, y suponga que se
cumplen los supuestos (A1) y (A2), entonces para alguna constante
positiva $C_{p+1}<\infty$, $\delta>0$ y un conjunto compacto
$C\subset X$.

\begin{equation}\label{Eq.5.4}
\esp_{x}\left[\int_{0}^{\tau_{C}\left(\delta\right)}\left(1+|X\left(t\right)|^{p}\right)dt\right]\leq
C_{p+1}\left(1+|x|^{p+1}\right)
\end{equation}
\end{Prop}

\begin{Prop}[Proposici\'on 5.4 \cite{DaiSean}]
Sea $X$ un proceso de Markov Borel Derecho en $X$, sea
$f:X\leftarrow\rea_{+}$ y defina para alguna $\delta>0$, y un
conjunto cerrado $C\subset X$
\[V\left(x\right):=\esp_{x}\left[\int_{0}^{\tau_{C}\left(\delta\right)}f\left(X\left(t\right)\right)dt\right]\]
para $x\in X$. Si $V$ es finito en todas partes y uniformemente
acotada en $C$, entonces existe $k<\infty$ tal que
\begin{equation}\label{Eq.5.11}
\frac{1}{t}\esp_{x}\left[V\left(x\right)\right]+\frac{1}{t}\int_{0}^{t}\esp_{x}\left[f\left(X\left(s\right)\right)ds\right]\leq\frac{1}{t}V\left(x\right)+k,
\end{equation}
para $x\in X$ y $t>0$.
\end{Prop}


\begin{Teo}[Teorema 5.5 \cite{DaiSean}]
Suponga que se cumplen (A1) y (A2), adem\'as suponga que el modelo
de flujo es estable. Entonces existe una constante $k_{p}<\infty$
tal que
\begin{equation}\label{Eq.5.13}
\frac{1}{t}\int_{0}^{t}\esp_{x}\left[|Q\left(s\right)|^{p}\right]ds\leq
k_{p}\left\{\frac{1}{t}|x|^{p+1}+1\right\}
\end{equation}
para $t\geq0$, $x\in X$. En particular para cada condici\'on
inicial
\begin{equation}\label{Eq.5.14}
Limsup_{t\rightarrow\infty}\frac{1}{t}\int_{0}^{t}\esp_{x}\left[|Q\left(s\right)|^{p}\right]ds\leq
k_{p}
\end{equation}
\end{Teo}

\begin{Teo}[Teorema 6.2 \cite{DaiSean}]\label{Tma.6.2}
Suponga que se cumplen los supuestos (A1)-(A3) y que el modelo de
flujo es estable, entonces se tiene que
\[\parallel P^{t}\left(c,\cdot\right)-\pi\left(\cdot\right)\parallel_{f_{p}}\rightarrow0\]
para $t\rightarrow\infty$ y $x\in X$. En particular para cada
condici\'on inicial
\[lim_{t\rightarrow\infty}\esp_{x}\left[\left|Q_{t}\right|^{p}\right]=\esp_{\pi}\left[\left|Q_{0}\right|^{p}\right]<\infty\]
\end{Teo}


\begin{Teo}[Teorema 6.3 \cite{DaiSean}]\label{Tma.6.3}
Suponga que se cumplen los supuestos (A1)-(A3) y que el modelo de
flujo es estable, entonces con
$f\left(x\right)=f_{1}\left(x\right)$, se tiene que
\[lim_{t\rightarrow\infty}t^{(p-1)\left|P^{t}\left(c,\cdot\right)-\pi\left(\cdot\right)\right|_{f}=0},\]
para $x\in X$. En particular, para cada condici\'on inicial
\[lim_{t\rightarrow\infty}t^{(p-1)}\left|\esp_{x}\left[Q_{t}\right]-\esp_{\pi}\left[Q_{0}\right]\right|=0.\]
\end{Teo}



\begin{Prop}[Proposici\'on 5.1, Dai y Meyn \cite{DaiSean}]\label{Prop.5.1.DaiSean}
Suponga que los supuestos A1) y A2) son ciertos y que el modelo de
flujo es estable. Entonces existe $t_{0}>0$ tal que
\begin{equation}
lim_{|x|\rightarrow\infty}\frac{1}{|x|^{p+1}}\esp_{x}\left[|X\left(t_{0}|x|\right)|^{p+1}\right]=0
\end{equation}
\end{Prop}

\begin{Lemma}[Lema 5.2, Dai y Meyn, \cite{DaiSean}]\label{Lema.5.2.DaiSean}
 Sea $\left\{\zeta\left(k\right):k\in \mathbb{z}\right\}$ una sucesi\'on independiente e id\'enticamente distribuida que toma valores en $\left(0,\infty\right)$,
y sea
$E\left(t\right)=max\left(n\geq1:\zeta\left(1\right)+\cdots+\zeta\left(n-1\right)\leq
t\right)$. Si $\esp\left[\zeta\left(1\right)\right]<\infty$,
entonces para cualquier entero $r\geq1$
\begin{equation}
 lim_{t\rightarrow\infty}\esp\left[\left(\frac{E\left(t\right)}{t}\right)^{r}\right]=\left(\frac{1}{\esp\left[\zeta_{1}\right]}\right)^{r}.
\end{equation}
Luego, bajo estas condiciones:
\begin{itemize}
 \item[a)] para cualquier $\delta>0$, $\sup_{t\geq\delta}\esp\left[\left(\frac{E\left(t\right)}{t}\right)^{r}\right]<\infty$
\item[b)] las variables aleatorias
$\left\{\left(\frac{E\left(t\right)}{t}\right)^{r}:t\geq1\right\}$
son uniformemente integrables.
\end{itemize}
\end{Lemma}

\begin{Teo}[Teorema 5.5, Dai y Meyn \cite{DaiSean}]\label{Tma.5.5.DaiSean}
Suponga que los supuestos A1) y A2) se cumplen y que el modelo de
flujo es estable. Entonces existe una constante $\kappa_{p}$ tal
que
\begin{equation}
\frac{1}{t}\int_{0}^{t}\esp_{x}\left[|Q\left(s\right)|^{p}\right]ds\leq\kappa_{p}\left\{\frac{1}{t}|x|^{p+1}+1\right\}
\end{equation}
para $t>0$ y $x\in X$. En particular, para cada condici\'on
inicial
\begin{eqnarray*}
\limsup_{t\rightarrow\infty}\frac{1}{t}\int_{0}^{t}\esp_{x}\left[|Q\left(s\right)|^{p}\right]ds\leq\kappa_{p}.
\end{eqnarray*}
\end{Teo}

\begin{Teo}[Teorema 6.2, Dai y Meyn \cite{DaiSean}]\label{Tma.6.2.DaiSean}
Suponga que se cumplen los supuestos A1), A2) y A3) y que el
modelo de flujo es estable. Entonces se tiene que
\begin{equation}
\left\|P^{t}\left(x,\cdot\right)-\pi\left(\cdot\right)\right\|_{f_{p}}\textrm{,
}t\rightarrow\infty,x\in X.
\end{equation}
En particular para cada condici\'on inicial
\begin{eqnarray*}
\lim_{t\rightarrow\infty}\esp_{x}\left[|Q\left(t\right)|^{p}\right]=\esp_{\pi}\left[|Q\left(0\right)|^{p}\right]\leq\kappa_{r}
\end{eqnarray*}
\end{Teo}
\begin{Teo}[Teorema 6.3, Dai y Meyn \cite{DaiSean}]\label{Tma.6.3.DaiSean}
Suponga que se cumplen los supuestos A1), A2) y A3) y que el
modelo de flujo es estable. Entonces con
$f\left(x\right)=f_{1}\left(x\right)$ se tiene
\begin{equation}
\lim_{t\rightarrow\infty}t^{p-1}\left\|P^{t}\left(x,\cdot\right)-\pi\left(\cdot\right)\right\|_{f}=0.
\end{equation}
En particular para cada condici\'on inicial
\begin{eqnarray*}
\lim_{t\rightarrow\infty}t^{p-1}|\esp_{x}\left[Q\left(t\right)\right]-\esp_{\pi}\left[Q\left(0\right)\right]|=0.
\end{eqnarray*}
\end{Teo}

\begin{Teo}[Teorema 6.4, Dai y Meyn, \cite{DaiSean}]\label{Tma.6.4.DaiSean}
Suponga que se cumplen los supuestos A1), A2) y A3) y que el
modelo de flujo es estable. Sea $\nu$ cualquier distribuci\'on de
probabilidad en $\left(X,\mathcal{B}_{X}\right)$, y $\pi$ la
distribuci\'on estacionaria de $X$.
\begin{itemize}
\item[i)] Para cualquier $f:X\leftarrow\rea_{+}$
\begin{equation}
\lim_{t\rightarrow\infty}\frac{1}{t}\int_{o}^{t}f\left(X\left(s\right)\right)ds=\pi\left(f\right):=\int
f\left(x\right)\pi\left(dx\right)
\end{equation}
$\prob$-c.s.

\item[ii)] Para cualquier $f:X\leftarrow\rea_{+}$ con
$\pi\left(|f|\right)<\infty$, la ecuaci\'on anterior se cumple.
\end{itemize}
\end{Teo}

\begin{Teo}[Teorema 2.2, Down \cite{Down}]\label{Tma2.2.Down}
Suponga que el fluido modelo es inestable en el sentido de que
para alguna $\epsilon_{0},c_{0}\geq0$,
\begin{equation}\label{Eq.Inestability}
|Q\left(T\right)|\geq\epsilon_{0}T-c_{0}\textrm{,   }T\geq0,
\end{equation}
para cualquier condici\'on inicial $Q\left(0\right)$, con
$|Q\left(0\right)|=1$. Entonces para cualquier $0<q\leq1$, existe
$B<0$ tal que para cualquier $|x|\geq B$,
\begin{equation}
\prob_{x}\left\{\mathbb{X}\rightarrow\infty\right\}\geq q.
\end{equation}
\end{Teo}



\begin{Def}
Sea $X$ un conjunto y $\mathcal{F}$ una $\sigma$-\'algebra de
subconjuntos de $X$, la pareja $\left(X,\mathcal{F}\right)$ es
llamado espacio medible. Un subconjunto $A$ de $X$ es llamado
medible, o medible con respecto a $\mathcal{F}$, si
$A\in\mathcal{F}$.
\end{Def}

\begin{Def}
Sea $\left(X,\mathcal{F},\mu\right)$ espacio de medida. Se dice
que la medida $\mu$ es $\sigma$-finita si se puede escribir
$X=\bigcup_{n\geq1}X_{n}$ con $X_{n}\in\mathcal{F}$ y
$\mu\left(X_{n}\right)<\infty$.
\end{Def}

\begin{Def}\label{Cto.Borel}
Sea $X$ el conjunto de los n\'umeros reales $\rea$. El \'algebra
de Borel es la $\sigma$-\'algebra $B$ generada por los intervalos
abiertos $\left(a,b\right)\in\rea$. Cualquier conjunto en $B$ es
llamado {\em Conjunto de Borel}.
\end{Def}

\begin{Def}\label{Funcion.Medible}
Una funci\'on $f:X\rightarrow\rea$, es medible si para cualquier
n\'umero real $\alpha$ el conjunto
\[\left\{x\in X:f\left(x\right)>\alpha\right\}\]
pertenece a $\mathcal{F}$. Equivalentemente, se dice que $f$ es
medible si
\[f^{-1}\left(\left(\alpha,\infty\right)\right)=\left\{x\in X:f\left(x\right)>\alpha\right\}\in\mathcal{F}.\]
\end{Def}


\begin{Def}\label{Def.Cilindros}
Sean $\left(\Omega_{i},\mathcal{F}_{i}\right)$, $i=1,2,\ldots,$
espacios medibles y $\Omega=\prod_{i=1}^{\infty}\Omega_{i}$ el
conjunto de todas las sucesiones
$\left(\omega_{1},\omega_{2},\ldots,\right)$ tales que
$\omega_{i}\in\Omega_{i}$, $i=1,2,\ldots,$. Si
$B^{n}\subset\prod_{i=1}^{\infty}\Omega_{i}$, definimos
$B_{n}=\left\{\omega\in\Omega:\left(\omega_{1},\omega_{2},\ldots,\omega_{n}\right)\in
B^{n}\right\}$. Al conjunto $B_{n}$ se le llama {\em cilindro} con
base $B^{n}$, el cilindro es llamado medible si
$B^{n}\in\prod_{i=1}^{\infty}\mathcal{F}_{i}$.
\end{Def}


\begin{Def}\label{Def.Proc.Adaptado}[TSP, Ash \cite{RBA}]
Sea $X\left(t\right),t\geq0$ proceso estoc\'astico, el proceso es
adaptado a la familia de $\sigma$-\'algebras $\mathcal{F}_{t}$,
para $t\geq0$, si para $s<t$ implica que
$\mathcal{F}_{s}\subset\mathcal{F}_{t}$, y $X\left(t\right)$ es
$\mathcal{F}_{t}$-medible para cada $t$. Si no se especifica
$\mathcal{F}_{t}$ entonces se toma $\mathcal{F}_{t}$ como
$\mathcal{F}\left(X\left(s\right),s\leq t\right)$, la m\'as
peque\~na $\sigma$-\'algebra de subconjuntos de $\Omega$ que hace
que cada $X\left(s\right)$, con $s\leq t$ sea Borel medible.
\end{Def}


\begin{Def}\label{Def.Tiempo.Paro}[TSP, Ash \cite{RBA}]
Sea $\left\{\mathcal{F}\left(t\right),t\geq0\right\}$ familia
creciente de sub $\sigma$-\'algebras. es decir,
$\mathcal{F}\left(s\right)\subset\mathcal{F}\left(t\right)$ para
$s\leq t$. Un tiempo de paro para $\mathcal{F}\left(t\right)$ es
una funci\'on $T:\Omega\rightarrow\left[0,\infty\right]$ tal que
$\left\{T\leq t\right\}\in\mathcal{F}\left(t\right)$ para cada
$t\geq0$. Un tiempo de paro para el proceso estoc\'astico
$X\left(t\right),t\geq0$ es un tiempo de paro para las
$\sigma$-\'algebras
$\mathcal{F}\left(t\right)=\mathcal{F}\left(X\left(s\right)\right)$.
\end{Def}

\begin{Def}
Sea $X\left(t\right),t\geq0$ proceso estoc\'astico, con
$\left(S,\chi\right)$ espacio de estados. Se dice que el proceso
es adaptado a $\left\{\mathcal{F}\left(t\right)\right\}$, es
decir, si para cualquier $s,t\in I$, $I$ conjunto de \'indices,
$s<t$, se tiene que
$\mathcal{F}\left(s\right)\subset\mathcal{F}\left(t\right)$ y
$X\left(t\right)$ es $\mathcal{F}\left(t\right)$-medible,
\end{Def}

\begin{Def}
Sea $X\left(t\right),t\geq0$ proceso estoc\'astico, se dice que es
un Proceso de Markov relativo a $\mathcal{F}\left(t\right)$ o que
$\left\{X\left(t\right),\mathcal{F}\left(t\right)\right\}$ es de
Markov si y s\'olo si para cualquier conjunto $B\in\chi$,  y
$s,t\in I$, $s<t$ se cumple que
\begin{equation}\label{Prop.Markov}
P\left\{X\left(t\right)\in
B|\mathcal{F}\left(s\right)\right\}=P\left\{X\left(t\right)\in
B|X\left(s\right)\right\}.
\end{equation}
\end{Def}
\begin{Note}
Si se dice que $\left\{X\left(t\right)\right\}$ es un Proceso de
Markov sin mencionar $\mathcal{F}\left(t\right)$, se asumir\'a que
\begin{eqnarray*}
\mathcal{F}\left(t\right)=\mathcal{F}_{0}\left(t\right)=\mathcal{F}\left(X\left(r\right),r\leq
t\right),
\end{eqnarray*}
entonces la ecuaci\'on (\ref{Prop.Markov}) se puede escribir como
\begin{equation}
P\left\{X\left(t\right)\in B|X\left(r\right),r\leq s\right\} =
P\left\{X\left(t\right)\in B|X\left(s\right)\right\}
\end{equation}
\end{Note}
%_______________________________________________________________
\subsection{Procesos de Estados de Markov}
%_______________________________________________________________

\begin{Teo}
Sea $\left(X_{n},\mathcal{F}_{n},n=0,1,\ldots,\right\}$ Proceso de
Markov con espacio de estados $\left(S_{0},\chi_{0}\right)$
generado por una distribuici\'on inicial $P_{o}$ y probabilidad de
transici\'on $p_{mn}$, para $m,n=0,1,\ldots,$ $m<n$, que por
notaci\'on se escribir\'a como $p\left(m,n,x,B\right)\rightarrow
p_{mn}\left(x,B\right)$. Sea $S$ tiempo de paro relativo a la
$\sigma$-\'algebra $\mathcal{F}_{n}$. Sea $T$ funci\'on medible,
$T:\Omega\rightarrow\left\{0,1,\ldots,\right\}$. Sup\'ongase que
$T\geq S$, entonces $T$ es tiempo de paro. Si $B\in\chi_{0}$,
entonces
\begin{equation}\label{Prop.Fuerte.Markov}
P\left\{X\left(T\right)\in
B,T<\infty|\mathcal{F}\left(S\right)\right\} =
p\left(S,T,X\left(s\right),B\right)
\end{equation}
en $\left\{T<\infty\right\}$.
\end{Teo}


Sea $K$ conjunto numerable y sea $d:K\rightarrow\nat$ funci\'on.
Para $v\in K$, $M_{v}$ es un conjunto abierto de
$\rea^{d\left(v\right)}$. Entonces \[E=\bigcup_{v\in
K}M_{v}=\left\{\left(v,\zeta\right):v\in K,\zeta\in
M_{v}\right\}.\]

Sea $\mathcal{E}$ la clase de conjuntos medibles en $E$:
\[\mathcal{E}=\left\{\bigcup_{v\in K}A_{v}:A_{v}\in \mathcal{M}_{v}\right\}.\]

donde $\mathcal{M}$ son los conjuntos de Borel de $M_{v}$.
Entonces $\left(E,\mathcal{E}\right)$ es un espacio de Borel. El
estado del proceso se denotar\'a por
$\mathbf{x}_{t}=\left(v_{t},\zeta_{t}\right)$. La distribuci\'on
de $\left(\mathbf{x}_{t}\right)$ est\'a determinada por por los
siguientes objetos:

\begin{itemize}
\item[i)] Los campos vectoriales $\left(\mathcal{H}_{v},v\in
K\right)$. \item[ii)] Una funci\'on medible $\lambda:E\rightarrow
\rea_{+}$. \item[iii)] Una medida de transici\'on
$Q:\mathcal{E}\times\left(E\cup\Gamma^{*}\right)\rightarrow\left[0,1\right]$
donde
\begin{equation}
\Gamma^{*}=\bigcup_{v\in K}\partial^{*}M_{v}.
\end{equation}
y
\begin{equation}
\partial^{*}M_{v}=\left\{z\in\partial M_{v}:\mathbf{\mathbf{\phi}_{v}\left(t,\zeta\right)=\mathbf{z}}\textrm{ para alguna }\left(t,\zeta\right)\in\rea_{+}\times M_{v}\right\}.
\end{equation}
$\partial M_{v}$ denota  la frontera de $M_{v}$.
\end{itemize}

El campo vectorial $\left(\mathcal{H}_{v},v\in K\right)$ se supone
tal que para cada $\mathbf{z}\in M_{v}$ existe una \'unica curva
integral $\mathbf{\phi}_{v}\left(t,\zeta\right)$ que satisface la
ecuaci\'on

\begin{equation}
\frac{d}{dt}f\left(\zeta_{t}\right)=\mathcal{H}f\left(\zeta_{t}\right),
\end{equation}
con $\zeta_{0}=\mathbf{z}$, para cualquier funci\'on suave
$f:\rea^{d}\rightarrow\rea$ y $\mathcal{H}$ denota el operador
diferencial de primer orden, con $\mathcal{H}=\mathcal{H}_{v}$ y
$\zeta_{t}=\mathbf{\phi}\left(t,\mathbf{z}\right)$. Adem\'as se
supone que $\mathcal{H}_{v}$ es conservativo, es decir, las curvas
integrales est\'an definidas para todo $t>0$.

Para $\mathbf{x}=\left(v,\zeta\right)\in E$ se denota
\[t^{*}\mathbf{x}=inf\left\{t>0:\mathbf{\phi}_{v}\left(t,\zeta\right)\in\partial^{*}M_{v}\right\}\]

En lo que respecta a la funci\'on $\lambda$, se supondr\'a que
para cada $\left(v,\zeta\right)\in E$ existe un $\epsilon>0$ tal
que la funci\'on
$s\rightarrow\lambda\left(v,\phi_{v}\left(s,\zeta\right)\right)\in
E$ es integrable para $s\in\left[0,\epsilon\right)$. La medida de
transici\'on $Q\left(A;\mathbf{x}\right)$ es una funci\'on medible
de $\mathbf{x}$ para cada $A\in\mathcal{E}$, definida para
$\mathbf{x}\in E\cup\Gamma^{*}$ y es una medida de probabilidad en
$\left(E,\mathcal{E}\right)$ para cada $\mathbf{x}\in E$.

El movimiento del proceso $\left(\mathbf{x}_{t}\right)$ comenzando
en $\mathbf{x}=\left(n,\mathbf{z}\right)\in E$ se puede construir
de la siguiente manera, def\'inase la funci\'on $F$ por

\begin{equation}
F\left(t\right)=\left\{\begin{array}{ll}\\
exp\left(-\int_{0}^{t}\lambda\left(n,\phi_{n}\left(s,\mathbf{z}\right)\right)ds\right), & t<t^{*}\left(\mathbf{x}\right),\\
0, & t\geq t^{*}\left(\mathbf{x}\right)
\end{array}\right.
\end{equation}

Sea $T_{1}$ una variable aleatoria tal que
$\prob\left[T_{1}>t\right]=F\left(t\right)$, ahora sea la variable
aleatoria $\left(N,Z\right)$ con distribuici\'on
$Q\left(\cdot;\phi_{n}\left(T_{1},\mathbf{z}\right)\right)$. La
trayectoria de $\left(\mathbf{x}_{t}\right)$ para $t\leq T_{1}$ es
\begin{eqnarray*}
\mathbf{x}_{t}=\left(v_{t},\zeta_{t}\right)=\left\{\begin{array}{ll}
\left(n,\phi_{n}\left(t,\mathbf{z}\right)\right), & t<T_{1},\\
\left(N,\mathbf{Z}\right), & t=t_{1}.
\end{array}\right.
\end{eqnarray*}

Comenzando en $\mathbf{x}_{T_{1}}$ se selecciona el siguiente
tiempo de intersalto $T_{2}-T_{1}$ lugar del post-salto
$\mathbf{x}_{T_{2}}$ de manera similar y as\'i sucesivamente. Este
procedimiento nos da una trayectoria determinista por partes
$\mathbf{x}_{t}$ con tiempos de salto $T_{1},T_{2},\ldots$. Bajo
las condiciones enunciadas para $\lambda,T_{1}>0$  y
$T_{1}-T_{2}>0$ para cada $i$, con probabilidad 1. Se supone que
se cumple la siguiente condici\'on.

\begin{Sup}[Supuesto 3.1, Davis \cite{Davis}]\label{Sup3.1.Davis}
Sea $N_{t}:=\sum_{t}\indora_{\left(t\geq t\right)}$ el n\'umero de
saltos en $\left[0,t\right]$. Entonces
\begin{equation}
\esp\left[N_{t}\right]<\infty\textrm{ para toda }t.
\end{equation}
\end{Sup}

es un proceso de Markov, m\'as a\'un, es un Proceso Fuerte de
Markov, es decir, la Propiedad Fuerte de Markov\footnote{Revisar
p\'agina 362, y 364 de Davis \cite{Davis}.} se cumple para
cualquier tiempo de paro.
%_________________________________________________________________________
%\renewcommand{\refname}{PROCESOS ESTOC\'ASTICOS}
%\renewcommand{\appendixname}{PROCESOS ESTOC\'ASTICOS}
%\renewcommand{\appendixtocname}{PROCESOS ESTOC\'ASTICOS}
%\renewcommand{\appendixpagename}{PROCESOS ESTOC\'ASTICOS}
%\appendix
%\clearpage % o \cleardoublepage
%\addappheadtotoc
%\appendixpage
%_________________________________________________________________________
\subsection{Teor\'ia General de Procesos Estoc\'asticos}
%_________________________________________________________________________
En esta secci\'on se har\'an las siguientes consideraciones: $E$
es un espacio m\'etrico separable y la m\'etrica $d$ es compatible
con la topolog\'ia.

\begin{Def}
Una medida finita, $\lambda$ en la $\sigma$-\'algebra de Borel de
un espacio metrizable $E$ se dice cerrada si
\begin{equation}\label{Eq.A2.3}
\lambda\left(E\right)=sup\left\{\lambda\left(K\right):K\textrm{ es
compacto en }E\right\}.
\end{equation}
\end{Def}

\begin{Def}
$E$ es un espacio de Rad\'on si cada medida finita en
$\left(E,\mathcal{B}\left(E\right)\right)$ es regular interior o cerrada,
{\em tight}.
\end{Def}


El siguiente teorema nos permite tener una mejor caracterizaci\'on de los espacios de Rad\'on:
\begin{Teo}\label{Tma.A2.2}
Sea $E$ espacio separable metrizable. Entonces $E$ es de Rad\'on
si y s\'olo s\'i cada medida finita en
$\left(E,\mathcal{B}\left(E\right)\right)$ es cerrada.
\end{Teo}

%_________________________________________________________________________________________
\subsection{Propiedades de Markov}
%_________________________________________________________________________________________

Sea $E$ espacio de estados, tal que $E$ es un espacio de Rad\'on, $\mathcal{B}\left(E\right)$ $\sigma$-\'algebra de Borel en $E$, que se denotar\'a por $\mathcal{E}$.

Sea $\left(X,\mathcal{G},\prob\right)$ espacio de probabilidad,
$I\subset\rea$ conjunto de índices. Sea $\mathcal{F}_{\leq t}$ la
$\sigma$-\'algebra natural definida como
$\sigma\left\{f\left(X_{r}\right):r\in I, r\leq
t,f\in\mathcal{E}\right\}$. Se considerar\'a una
$\sigma$-\'algebra m\'as general\footnote{qu\'e se quiere decir
con el t\'ermino: m\'as general?}, $ \left(\mathcal{G}_{t}\right)$
tal que $\left(X_{t}\right)$ sea $\mathcal{E}$-adaptado.

\begin{Def}
Una familia $\left(P_{s,t}\right)$ de kernels de Markov en $\left(E,\mathcal{E}\right)$ indexada por pares $s,t\in I$, con $s\leq t$ es una funci\'on de transici\'on en $\ER$, si  para todo $r\leq s< t$ en $I$ y todo $x\in E$, $B\in\mathcal{E}$
\begin{equation}\label{Eq.Kernels}
P_{r,t}\left(x,B\right)=\int_{E}P_{r,s}\left(x,dy\right)P_{s,t}\left(y,B\right)\footnote{Ecuaci\'on de Chapman-Kolmogorov}.
\end{equation}
\end{Def}

Se dice que la funci\'on de transici\'on $\KM$ en $\ER$ es la funci\'on de transici\'on para un proceso $\PE$  con valores en $E$ y que satisface la propiedad de Markov\footnote{\begin{equation}\label{Eq.1.4.S}
\prob\left\{H|\mathcal{G}_{t}\right\}=\prob\left\{H|X_{t}\right\}\textrm{ }H\in p\mathcal{F}_{\geq t}.
\end{equation}} (\ref{Eq.1.4.S}) relativa a $\left(\mathcal{G}_{t}\right)$ si

\begin{equation}\label{Eq.1.6.S}
\prob\left\{f\left(X_{t}\right)|\mathcal{G}_{s}\right\}=P_{s,t}f\left(X_{t}\right)\textrm{ }s\leq t\in I,\textrm{ }f\in b\mathcal{E}.
\end{equation}

\begin{Def}
Una familia $\left(P_{t}\right)_{t\geq0}$ de kernels de Markov en $\ER$ es llamada {\em Semigrupo de Transici\'on de Markov} o {\em Semigrupo de Transici\'on} si
\[P_{t+s}f\left(x\right)=P_{t}\left(P_{s}f\right)\left(x\right),\textrm{ }t,s\geq0,\textrm{ }x\in E\textrm{ }f\in b\mathcal{E}\footnote{Definir los t\'ermino $b\mathcal{E}$ y $p\mathcal{E}$}.\]
\end{Def}
\begin{Note}
Si la funci\'on de transici\'on $\KM$ es llamada homog\'enea si $P_{s,t}=P_{t-s}$.
\end{Note}

Un proceso de Markov que satisface la ecuaci\'on (\ref{Eq.1.6.S}) con funci\'on de transici\'on homog\'enea $\left(P_{t}\right)$ tiene la propiedad caracter\'istica
\begin{equation}\label{Eq.1.8.S}
\prob\left\{f\left(X_{t+s}\right)|\mathcal{G}_{t}\right\}=P_{s}f\left(X_{t}\right)\textrm{ }t,s\geq0,\textrm{ }f\in b\mathcal{E}.
\end{equation}
La ecuaci\'on anterior es la {\em Propiedad Simple de Markov} de $X$ relativa a $\left(P_{t}\right)$.

En este sentido el proceso $\PE$ cumple con la propiedad de Markov (\ref{Eq.1.8.S}) relativa a $\left(\Omega,\mathcal{G},\mathcal{G}_{t},\prob\right)$ con semigrupo de transici\'on $\left(P_{t}\right)$.
%_________________________________________________________________________________________
\subsection{Primer Condici\'on de Regularidad}
%_________________________________________________________________________________________
%\newcommand{\EM}{\left(\Omega,\mathcal{G},\prob\right)}
%\newcommand{\E4}{\left(\Omega,\mathcal{G},\mathcal{G}_{t},\prob\right)}
\begin{Def}
Un proceso estoc\'astico $\PE$ definido en
$\left(\Omega,\mathcal{G},\prob\right)$ con valores en el espacio
topol\'ogico $E$ es continuo por la derecha si cada trayectoria
muestral $t\rightarrow X_{t}\left(w\right)$ es un mapeo continuo
por la derecha de $I$ en $E$.
\end{Def}

\begin{Def}[HD1]\label{Eq.2.1.S}
Un semigrupo de Markov $\left(P_{t}\right)$ en un espacio de
Rad\'on $E$ se dice que satisface la condici\'on {\em HD1} si,
dada una medida de probabilidad $\mu$ en $E$, existe una
$\sigma$-\'algebra $\mathcal{E^{*}}$ con
$\mathcal{E}\subset\mathcal{E}^{*}$ y
$P_{t}\left(b\mathcal{E}^{*}\right)\subset b\mathcal{E}^{*}$, y un
$\mathcal{E}^{*}$-proceso $E$-valuado continuo por la derecha
$\PE$ en alg\'un espacio de probabilidad filtrado
$\left(\Omega,\mathcal{G},\mathcal{G}_{t},\prob\right)$ tal que
$X=\left(\Omega,\mathcal{G},\mathcal{G}_{t},\prob\right)$ es de
Markov (Homog\'eneo) con semigrupo de transici\'on $(P_{t})$ y
distribuci\'on inicial $\mu$.
\end{Def}

Consid\'erese la colecci\'on de variables aleatorias $X_{t}$
definidas en alg\'un espacio de probabilidad, y una colecci\'on de
medidas $\mathbf{P}^{x}$ tales que
$\mathbf{P}^{x}\left\{X_{0}=x\right\}$, y bajo cualquier
$\mathbf{P}^{x}$, $X_{t}$ es de Markov con semigrupo
$\left(P_{t}\right)$. $\mathbf{P}^{x}$ puede considerarse como la
distribuci\'on condicional de $\mathbf{P}$ dado $X_{0}=x$.

\begin{Def}\label{Def.2.2.S}
Sea $E$ espacio de Rad\'on, $\SG$ semigrupo de Markov en $\ER$. La colecci\'on $\mathbf{X}=\left(\Omega,\mathcal{G},\mathcal{G}_{t},X_{t},\theta_{t},\CM\right)$ es un proceso $\mathcal{E}$-Markov continuo por la derecha simple, con espacio de estados $E$ y semigrupo de transici\'on $\SG$ en caso de que $\mathbf{X}$ satisfaga las siguientes condiciones:
\begin{itemize}
\item[i)] $\left(\Omega,\mathcal{G},\mathcal{G}_{t}\right)$ es un espacio de medida filtrado, y $X_{t}$ es un proceso $E$-valuado continuo por la derecha $\mathcal{E}^{*}$-adaptado a $\left(\mathcal{G}_{t}\right)$;

\item[ii)] $\left(\theta_{t}\right)_{t\geq0}$ es una colecci\'on de operadores {\em shift} para $X$, es decir, mapea $\Omega$ en s\'i mismo satisfaciendo para $t,s\geq0$,

\begin{equation}\label{Eq.Shift}
\theta_{t}\circ\theta_{s}=\theta_{t+s}\textrm{ y }X_{t}\circ\theta_{t}=X_{t+s};
\end{equation}

\item[iii)] Para cualquier $x\in E$,$\CM\left\{X_{0}=x\right\}=1$, y el proceso $\PE$ tiene la propiedad de Markov (\ref{Eq.1.8.S}) con semigrupo de transici\'on $\SG$ relativo a $\left(\Omega,\mathcal{G},\mathcal{G}_{t},\CM\right)$.
\end{itemize}
\end{Def}

\begin{Def}[HD2]\label{Eq.2.2.S}
Para cualquier $\alpha>0$ y cualquier $f\in S^{\alpha}$, el proceso $t\rightarrow f\left(X_{t}\right)$ es continuo por la derecha casi seguramente.
\end{Def}

\begin{Def}\label{Def.PD}
Un sistema $\mathbf{X}=\left(\Omega,\mathcal{G},\mathcal{G}_{t},X_{t},\theta_{t},\CM\right)$ es un proceso derecho en el espacio de Rad\'on $E$ con semigrupo de transici\'on $\SG$ provisto de:
\begin{itemize}
\item[i)] $\mathbf{X}$ es una realizaci\'on  continua por la derecha, \ref{Def.2.2.S}, de $\SG$.

\item[ii)] $\mathbf{X}$ satisface la condicion HD2, \ref{Eq.2.2.S}, relativa a $\mathcal{G}_{t}$.

\item[iii)] $\mathcal{G}_{t}$ es aumentado y continuo por la derecha.
\end{itemize}
\end{Def}


%_________________________________________________________________________
%\renewcommand{\refname}{MODELO DE FLUJO}
%\renewcommand{\appendixname}{MODELO DE FLUJO}
%\renewcommand{\appendixtocname}{MODELO DE FLUJO}
%\renewcommand{\appendixpagename}{MODELO DE FLUJO}
%\appendix
%\clearpage % o \cleardoublepage
%\addappheadtotoc
%\appendixpage

\subsection{Construcci\'on del Modelo de Flujo}


\begin{Lema}[Lema 4.2, Dai\cite{Dai}]\label{Lema4.2}
Sea $\left\{x_{n}\right\}\subset \mathbf{X}$ con
$|x_{n}|\rightarrow\infty$, conforme $n\rightarrow\infty$. Suponga
que
\[lim_{n\rightarrow\infty}\frac{1}{|x_{n}|}U\left(0\right)=\overline{U}\]
y
\[lim_{n\rightarrow\infty}\frac{1}{|x_{n}|}V\left(0\right)=\overline{V}.\]

Entonces, conforme $n\rightarrow\infty$, casi seguramente

\begin{equation}\label{E1.4.2}
\frac{1}{|x_{n}|}\Phi^{k}\left(\left[|x_{n}|t\right]\right)\rightarrow
P_{k}^{'}t\textrm{, u.o.c.,}
\end{equation}

\begin{equation}\label{E1.4.3}
\frac{1}{|x_{n}|}E^{x_{n}}_{k}\left(|x_{n}|t\right)\rightarrow
\alpha_{k}\left(t-\overline{U}_{k}\right)^{+}\textrm{, u.o.c.,}
\end{equation}

\begin{equation}\label{E1.4.4}
\frac{1}{|x_{n}|}S^{x_{n}}_{k}\left(|x_{n}|t\right)\rightarrow
\mu_{k}\left(t-\overline{V}_{k}\right)^{+}\textrm{, u.o.c.,}
\end{equation}

donde $\left[t\right]$ es la parte entera de $t$ y
$\mu_{k}=1/m_{k}=1/\esp\left[\eta_{k}\left(1\right)\right]$.
\end{Lema}

\begin{Lema}[Lema 4.3, Dai\cite{Dai}]\label{Lema.4.3}
Sea $\left\{x_{n}\right\}\subset \mathbf{X}$ con
$|x_{n}|\rightarrow\infty$, conforme $n\rightarrow\infty$. Suponga
que
\[lim_{n\rightarrow\infty}\frac{1}{|x_{n}|}U_{k}\left(0\right)=\overline{U}_{k}\]
y
\[lim_{n\rightarrow\infty}\frac{1}{|x_{n}|}V_{k}\left(0\right)=\overline{V}_{k}.\]
\begin{itemize}
\item[a)] Conforme $n\rightarrow\infty$ casi seguramente,
\[lim_{n\rightarrow\infty}\frac{1}{|x_{n}|}U^{x_{n}}_{k}\left(|x_{n}|t\right)=\left(\overline{U}_{k}-t\right)^{+}\textrm{, u.o.c.}\]
y
\[lim_{n\rightarrow\infty}\frac{1}{|x_{n}|}V^{x_{n}}_{k}\left(|x_{n}|t\right)=\left(\overline{V}_{k}-t\right)^{+}.\]

\item[b)] Para cada $t\geq0$ fijo,
\[\left\{\frac{1}{|x_{n}|}U^{x_{n}}_{k}\left(|x_{n}|t\right),|x_{n}|\geq1\right\}\]
y
\[\left\{\frac{1}{|x_{n}|}V^{x_{n}}_{k}\left(|x_{n}|t\right),|x_{n}|\geq1\right\}\]
\end{itemize}
son uniformemente convergentes.
\end{Lema}

Sea $S_{l}^{x}\left(t\right)$ el n\'umero total de servicios
completados de la clase $l$, si la clase $l$ est\'a dando $t$
unidades de tiempo de servicio. Sea $T_{l}^{x}\left(x\right)$ el
monto acumulado del tiempo de servicio que el servidor
$s\left(l\right)$ gasta en los usuarios de la clase $l$ al tiempo
$t$. Entonces $S_{l}^{x}\left(T_{l}^{x}\left(t\right)\right)$ es
el n\'umero total de servicios completados para la clase $l$ al
tiempo $t$. Una fracci\'on de estos usuarios,
$\Phi_{k}^{x}\left(S_{l}^{x}\left(T_{l}^{x}\left(t\right)\right)\right)$,
se convierte en usuarios de la clase $k$.\\

Entonces, dado lo anterior, se tiene la siguiente representaci\'on
para el proceso de la longitud de la cola:\\

\begin{equation}
Q_{k}^{x}\left(t\right)=Q_{k}^{x}\left(0\right)+E_{k}^{x}\left(t\right)+\sum_{l=1}^{K}\Phi_{k}^{l}\left(S_{l}^{x}\left(T_{l}^{x}\left(t\right)\right)\right)-S_{k}^{x}\left(T_{k}^{x}\left(t\right)\right)
\end{equation}
para $k=1,\ldots,K$. Para $i=1,\ldots,d$, sea
\[I_{i}^{x}\left(t\right)=t-\sum_{j\in C_{i}}T_{k}^{x}\left(t\right).\]

Entonces $I_{i}^{x}\left(t\right)$ es el monto acumulado del
tiempo que el servidor $i$ ha estado desocupado al tiempo $t$. Se
est\'a asumiendo que las disciplinas satisfacen la ley de
conservaci\'on del trabajo, es decir, el servidor $i$ est\'a en
pausa solamente cuando no hay usuarios en la estaci\'on $i$.
Entonces, se tiene que

\begin{equation}
\int_{0}^{\infty}\left(\sum_{k\in
C_{i}}Q_{k}^{x}\left(t\right)\right)dI_{i}^{x}\left(t\right)=0,
\end{equation}
para $i=1,\ldots,d$.\\

Hacer
\[T^{x}\left(t\right)=\left(T_{1}^{x}\left(t\right),\ldots,T_{K}^{x}\left(t\right)\right)^{'},\]
\[I^{x}\left(t\right)=\left(I_{1}^{x}\left(t\right),\ldots,I_{K}^{x}\left(t\right)\right)^{'}\]
y
\[S^{x}\left(T^{x}\left(t\right)\right)=\left(S_{1}^{x}\left(T_{1}^{x}\left(t\right)\right),\ldots,S_{K}^{x}\left(T_{K}^{x}\left(t\right)\right)\right)^{'}.\]

Para una disciplina que cumple con la ley de conservaci\'on del
trabajo, en forma vectorial, se tiene el siguiente conjunto de
ecuaciones

\begin{equation}\label{Eq.MF.1.3}
Q^{x}\left(t\right)=Q^{x}\left(0\right)+E^{x}\left(t\right)+\sum_{l=1}^{K}\Phi^{l}\left(S_{l}^{x}\left(T_{l}^{x}\left(t\right)\right)\right)-S^{x}\left(T^{x}\left(t\right)\right),\\
\end{equation}

\begin{equation}\label{Eq.MF.2.3}
Q^{x}\left(t\right)\geq0,\\
\end{equation}

\begin{equation}\label{Eq.MF.3.3}
T^{x}\left(0\right)=0,\textrm{ y }\overline{T}^{x}\left(t\right)\textrm{ es no decreciente},\\
\end{equation}

\begin{equation}\label{Eq.MF.4.3}
I^{x}\left(t\right)=et-CT^{x}\left(t\right)\textrm{ es no
decreciente}\\
\end{equation}

\begin{equation}\label{Eq.MF.5.3}
\int_{0}^{\infty}\left(CQ^{x}\left(t\right)\right)dI_{i}^{x}\left(t\right)=0,\\
\end{equation}

\begin{equation}\label{Eq.MF.6.3}
\textrm{Condiciones adicionales en
}\left(\overline{Q}^{x}\left(\cdot\right),\overline{T}^{x}\left(\cdot\right)\right)\textrm{
espec\'ificas de la disciplina de la cola,}
\end{equation}

donde $e$ es un vector de unos de dimensi\'on $d$, $C$ es la
matriz definida por
\[C_{ik}=\left\{\begin{array}{cc}
1,& S\left(k\right)=i,\\
0,& \textrm{ en otro caso}.\\
\end{array}\right.
\]
Es necesario enunciar el siguiente Teorema que se utilizar\'a para
el Teorema \ref{Tma.4.2.Dai}:
\begin{Teo}[Teorema 4.1, Dai \cite{Dai}]
Considere una disciplina que cumpla la ley de conservaci\'on del
trabajo, para casi todas las trayectorias muestrales $\omega$ y
cualquier sucesi\'on de estados iniciales
$\left\{x_{n}\right\}\subset \mathbf{X}$, con
$|x_{n}|\rightarrow\infty$, existe una subsucesi\'on
$\left\{x_{n_{j}}\right\}$ con $|x_{n_{j}}|\rightarrow\infty$ tal
que
\begin{equation}\label{Eq.4.15}
\frac{1}{|x_{n_{j}}|}\left(Q^{x_{n_{j}}}\left(0\right),U^{x_{n_{j}}}\left(0\right),V^{x_{n_{j}}}\left(0\right)\right)\rightarrow\left(\overline{Q}\left(0\right),\overline{U},\overline{V}\right),
\end{equation}

\begin{equation}\label{Eq.4.16}
\frac{1}{|x_{n_{j}}|}\left(Q^{x_{n_{j}}}\left(|x_{n_{j}}|t\right),T^{x_{n_{j}}}\left(|x_{n_{j}}|t\right)\right)\rightarrow\left(\overline{Q}\left(t\right),\overline{T}\left(t\right)\right)\textrm{
u.o.c.}
\end{equation}

Adem\'as,
$\left(\overline{Q}\left(t\right),\overline{T}\left(t\right)\right)$
satisface las siguientes ecuaciones:
\begin{equation}\label{Eq.MF.1.3a}
\overline{Q}\left(t\right)=Q\left(0\right)+\left(\alpha
t-\overline{U}\right)^{+}-\left(I-P\right)^{'}M^{-1}\left(\overline{T}\left(t\right)-\overline{V}\right)^{+},
\end{equation}

\begin{equation}\label{Eq.MF.2.3a}
\overline{Q}\left(t\right)\geq0,\\
\end{equation}

\begin{equation}\label{Eq.MF.3.3a}
\overline{T}\left(t\right)\textrm{ es no decreciente y comienza en cero},\\
\end{equation}

\begin{equation}\label{Eq.MF.4.3a}
\overline{I}\left(t\right)=et-C\overline{T}\left(t\right)\textrm{
es no decreciente,}\\
\end{equation}

\begin{equation}\label{Eq.MF.5.3a}
\int_{0}^{\infty}\left(C\overline{Q}\left(t\right)\right)d\overline{I}\left(t\right)=0,\\
\end{equation}

\begin{equation}\label{Eq.MF.6.3a}
\textrm{Condiciones adicionales en
}\left(\overline{Q}\left(\cdot\right),\overline{T}\left(\cdot\right)\right)\textrm{
especficas de la disciplina de la cola,}
\end{equation}
\end{Teo}


Propiedades importantes para el modelo de flujo retrasado:

\begin{Prop}
 Sea $\left(\overline{Q},\overline{T},\overline{T}^{0}\right)$ un flujo l\'imite de \ref{Eq.4.4} y suponga que cuando $x\rightarrow\infty$ a lo largo de
una subsucesi\'on
\[\left(\frac{1}{|x|}Q_{k}^{x}\left(0\right),\frac{1}{|x|}A_{k}^{x}\left(0\right),\frac{1}{|x|}B_{k}^{x}\left(0\right),\frac{1}{|x|}B_{k}^{x,0}\left(0\right)\right)\rightarrow\left(\overline{Q}_{k}\left(0\right),0,0,0\right)\]
para $k=1,\ldots,K$. EL flujo l\'imite tiene las siguientes
propiedades, donde las propiedades de la derivada se cumplen donde
la derivada exista:
\begin{itemize}
 \item[i)] Los vectores de tiempo ocupado $\overline{T}\left(t\right)$ y $\overline{T}^{0}\left(t\right)$ son crecientes y continuas con
$\overline{T}\left(0\right)=\overline{T}^{0}\left(0\right)=0$.
\item[ii)] Para todo $t\geq0$
\[\sum_{k=1}^{K}\left[\overline{T}_{k}\left(t\right)+\overline{T}_{k}^{0}\left(t\right)\right]=t\]
\item[iii)] Para todo $1\leq k\leq K$
\[\overline{Q}_{k}\left(t\right)=\overline{Q}_{k}\left(0\right)+\alpha_{k}t-\mu_{k}\overline{T}_{k}\left(t\right)\]
\item[iv)]  Para todo $1\leq k\leq K$
\[\dot{{\overline{T}}}_{k}\left(t\right)=\beta_{k}\] para $\overline{Q}_{k}\left(t\right)=0$.
\item[v)] Para todo $k,j$
\[\mu_{k}^{0}\overline{T}_{k}^{0}\left(t\right)=\mu_{j}^{0}\overline{T}_{j}^{0}\left(t\right)\]
\item[vi)]  Para todo $1\leq k\leq K$
\[\mu_{k}\dot{{\overline{T}}}_{k}\left(t\right)=l_{k}\mu_{k}^{0}\dot{{\overline{T}}}_{k}^{0}\left(t\right)\] para $\overline{Q}_{k}\left(t\right)>0$.
\end{itemize}
\end{Prop}

\begin{Teo}[Teorema 5.1: Ley Fuerte para Procesos de Conteo
\cite{Gut}]\label{Tma.5.1.Gut} Sea
$0<\mu<\esp\left(X_{1}\right]\leq\infty$. entonces

\begin{itemize}
\item[a)] $\frac{N\left(t\right)}{t}\rightarrow\frac{1}{\mu}$
a.s., cuando $t\rightarrow\infty$.


\item[b)]$\esp\left[\frac{N\left(t\right)}{t}\right]^{r}\rightarrow\frac{1}{\mu^{r}}$,
cuando $t\rightarrow\infty$ para todo $r>0$..
\end{itemize}
\end{Teo}


\begin{Prop}[Proposici\'on 5.3 \cite{DaiSean}]
Sea $X$ proceso de estados para la red de colas, y suponga que se
cumplen los supuestos (A1) y (A2), entonces para alguna constante
positiva $C_{p+1}<\infty$, $\delta>0$ y un conjunto compacto
$C\subset X$.

\begin{equation}\label{Eq.5.4}
\esp_{x}\left[\int_{0}^{\tau_{C}\left(\delta\right)}\left(1+|X\left(t\right)|^{p}\right)dt\right]\leq
C_{p+1}\left(1+|x|^{p+1}\right)
\end{equation}
\end{Prop}

\begin{Prop}[Proposici\'on 5.4 \cite{DaiSean}]
Sea $X$ un proceso de Markov Borel Derecho en $X$, sea
$f:X\leftarrow\rea_{+}$ y defina para alguna $\delta>0$, y un
conjunto cerrado $C\subset X$
\[V\left(x\right):=\esp_{x}\left[\int_{0}^{\tau_{C}\left(\delta\right)}f\left(X\left(t\right)\right)dt\right]\]
para $x\in X$. Si $V$ es finito en todas partes y uniformemente
acotada en $C$, entonces existe $k<\infty$ tal que
\begin{equation}\label{Eq.5.11}
\frac{1}{t}\esp_{x}\left[V\left(x\right)\right]+\frac{1}{t}\int_{0}^{t}\esp_{x}\left[f\left(X\left(s\right)\right)ds\right]\leq\frac{1}{t}V\left(x\right)+k,
\end{equation}
para $x\in X$ y $t>0$.
\end{Prop}


%_________________________________________________________________________
%\renewcommand{\refname}{Ap\'endice D}
%\renewcommand{\appendixname}{ESTABILIDAD}
%\renewcommand{\appendixtocname}{ESTABILIDAD}
%\renewcommand{\appendixpagename}{ESTABILIDAD}
%\appendix
%\clearpage % o \cleardoublepage
%\addappheadtotoc
%\appendixpage

\subsection{Estabilidad}

\begin{Def}[Definici\'on 3.2, Dai y Meyn \cite{DaiSean}]
El modelo de flujo retrasado de una disciplina de servicio en una
red con retraso
$\left(\overline{A}\left(0\right),\overline{B}\left(0\right)\right)\in\rea_{+}^{K+|A|}$
se define como el conjunto de ecuaciones dadas en
\ref{Eq.3.8}-\ref{Eq.3.13}, junto con la condici\'on:
\begin{equation}\label{CondAd.FluidModel}
\overline{Q}\left(t\right)=\overline{Q}\left(0\right)+\left(\alpha
t-\overline{A}\left(0\right)\right)^{+}-\left(I-P^{'}\right)M\left(\overline{T}\left(t\right)-\overline{B}\left(0\right)\right)^{+}
\end{equation}
\end{Def}

entonces si el modelo de flujo retrasado tambi\'en es estable:


\begin{Def}[Definici\'on 3.1, Dai y Meyn \cite{DaiSean}]
Un flujo l\'imite (retrasado) para una red bajo una disciplina de
servicio espec\'ifica se define como cualquier soluci\'on
 $\left(\overline{Q}\left(\cdot\right),\overline{T}\left(\cdot\right)\right)$ de las siguientes ecuaciones, donde
$\overline{Q}\left(t\right)=\left(\overline{Q}_{1}\left(t\right),\ldots,\overline{Q}_{K}\left(t\right)\right)^{'}$
y
$\overline{T}\left(t\right)=\left(\overline{T}_{1}\left(t\right),\ldots,\overline{T}_{K}\left(t\right)\right)^{'}$
\begin{equation}\label{Eq.3.8}
\overline{Q}_{k}\left(t\right)=\overline{Q}_{k}\left(0\right)+\alpha_{k}t-\mu_{k}\overline{T}_{k}\left(t\right)+\sum_{l=1}^{k}P_{lk}\mu_{l}\overline{T}_{l}\left(t\right)\\
\end{equation}
\begin{equation}\label{Eq.3.9}
\overline{Q}_{k}\left(t\right)\geq0\textrm{ para }k=1,2,\ldots,K,\\
\end{equation}
\begin{equation}\label{Eq.3.10}
\overline{T}_{k}\left(0\right)=0,\textrm{ y }\overline{T}_{k}\left(\cdot\right)\textrm{ es no decreciente},\\
\end{equation}
\begin{equation}\label{Eq.3.11}
\overline{I}_{i}\left(t\right)=t-\sum_{k\in C_{i}}\overline{T}_{k}\left(t\right)\textrm{ es no decreciente}\\
\end{equation}
\begin{equation}\label{Eq.3.12}
\overline{I}_{i}\left(\cdot\right)\textrm{ se incrementa al tiempo }t\textrm{ cuando }\sum_{k\in C_{i}}Q_{k}^{x}\left(t\right)dI_{i}^{x}\left(t\right)=0\\
\end{equation}
\begin{equation}\label{Eq.3.13}
\textrm{condiciones adicionales sobre
}\left(Q^{x}\left(\cdot\right),T^{x}\left(\cdot\right)\right)\textrm{
referentes a la disciplina de servicio}
\end{equation}
\end{Def}

\begin{Lema}[Lema 3.1 \cite{Chen}]\label{Lema3.1}
Si el modelo de flujo es estable, definido por las ecuaciones
(3.8)-(3.13), entonces el modelo de flujo retrasado tambin es
estable.
\end{Lema}

\begin{Teo}[Teorema 5.1 \cite{Chen}]\label{Tma.5.1.Chen}
La red de colas es estable si existe una constante $t_{0}$ que
depende de $\left(\alpha,\mu,T,U\right)$ y $V$ que satisfagan las
ecuaciones (5.1)-(5.5), $Z\left(t\right)=0$, para toda $t\geq
t_{0}$.
\end{Teo}

\begin{Prop}[Proposici\'on 5.1, Dai y Meyn \cite{DaiSean}]\label{Prop.5.1.DaiSean}
Suponga que los supuestos A1) y A2) son ciertos y que el modelo de flujo es estable. Entonces existe $t_{0}>0$ tal que
\begin{equation}
lim_{|x|\rightarrow\infty}\frac{1}{|x|^{p+1}}\esp_{x}\left[|X\left(t_{0}|x|\right)|^{p+1}\right]=0
\end{equation}
\end{Prop}

\begin{Lemma}[Lema 5.2, Dai y Meyn \cite{DaiSean}]\label{Lema.5.2.DaiSean}
 Sea $\left\{\zeta\left(k\right):k\in \mathbb{z}\right\}$ una sucesi\'on independiente e id\'enticamente distribuida que toma valores en $\left(0,\infty\right)$,
y sea
$E\left(t\right)=max\left(n\geq1:\zeta\left(1\right)+\cdots+\zeta\left(n-1\right)\leq
t\right)$. Si $\esp\left[\zeta\left(1\right)\right]<\infty$,
entonces para cualquier entero $r\geq1$
\begin{equation}
 lim_{t\rightarrow\infty}\esp\left[\left(\frac{E\left(t\right)}{t}\right)^{r}\right]=\left(\frac{1}{\esp\left[\zeta_{1}\right]}\right)^{r}.
\end{equation}
Luego, bajo estas condiciones:
\begin{itemize}
 \item[a)] para cualquier $\delta>0$, $\sup_{t\geq\delta}\esp\left[\left(\frac{E\left(t\right)}{t}\right)^{r}\right]<\infty$
\item[b)] las variables aleatorias
$\left\{\left(\frac{E\left(t\right)}{t}\right)^{r}:t\geq1\right\}$
son uniformemente integrables.
\end{itemize}
\end{Lemma}

\begin{Teo}[Teorema 5.5, Dai y Meyn \cite{DaiSean}]\label{Tma.5.5.DaiSean}
Suponga que los supuestos A1) y A2) se cumplen y que el modelo de
flujo es estable. Entonces existe una constante $\kappa_{p}$ tal
que
\begin{equation}
\frac{1}{t}\int_{0}^{t}\esp_{x}\left[|Q\left(s\right)|^{p}\right]ds\leq\kappa_{p}\left\{\frac{1}{t}|x|^{p+1}+1\right\}
\end{equation}
para $t>0$ y $x\in X$. En particular, para cada condici\'on
inicial
\begin{eqnarray*}
\limsup_{t\rightarrow\infty}\frac{1}{t}\int_{0}^{t}\esp_{x}\left[|Q\left(s\right)|^{p}\right]ds\leq\kappa_{p}.
\end{eqnarray*}
\end{Teo}

\begin{Teo}[Teorema 6.2, Dai y Meyn \cite{DaiSean}]\label{Tma.6.2.DaiSean}
Suponga que se cumplen los supuestos A1), A2) y A3) y que el
modelo de flujo es estable. Entonces se tiene que
\begin{equation}
\left\|P^{t}\left(x,\cdot\right)-\pi\left(\cdot\right)\right\|_{f_{p}}\textrm{,
}t\rightarrow\infty,x\in X.
\end{equation}
En particular para cada condici\'on inicial
\begin{eqnarray*}
\lim_{t\rightarrow\infty}\esp_{x}\left[|Q\left(t\right)|^{p}\right]=\esp_{\pi}\left[|Q\left(0\right)|^{p}\right]\leq\kappa_{r}
\end{eqnarray*}
\end{Teo}
\begin{Teo}[Teorema 6.3, Dai y Meyn \cite{DaiSean}]\label{Tma.6.3.DaiSean}
Suponga que se cumplen los supuestos A1), A2) y A3) y que el
modelo de flujo es estable. Entonces con
$f\left(x\right)=f_{1}\left(x\right)$ se tiene
\begin{equation}
\lim_{t\rightarrow\infty}t^{p-1}\left\|P^{t}\left(x,\cdot\right)-\pi\left(\cdot\right)\right\|_{f}=0.
\end{equation}
En particular para cada condici\'on inicial
\begin{eqnarray*}
\lim_{t\rightarrow\infty}t^{p-1}|\esp_{x}\left[Q\left(t\right)\right]-\esp_{\pi}\left[Q\left(0\right)\right]|=0.
\end{eqnarray*}
\end{Teo}

\begin{Teo}[Teorema 6.4, Dai y Meyn \cite{DaiSean}]\label{Tma.6.4.DaiSean}
Suponga que se cumplen los supuestos A1), A2) y A3) y que el
modelo de flujo es estable. Sea $\nu$ cualquier distribuci\'on de
probabilidad en $\left(X,\mathcal{B}_{X}\right)$, y $\pi$ la
distribuci\'on estacionaria de $X$.
\begin{itemize}
\item[i)] Para cualquier $f:X\leftarrow\rea_{+}$
\begin{equation}
\lim_{t\rightarrow\infty}\frac{1}{t}\int_{o}^{t}f\left(X\left(s\right)\right)ds=\pi\left(f\right):=\int
f\left(x\right)\pi\left(dx\right)
\end{equation}
$\prob$-c.s.

\item[ii)] Para cualquier $f:X\leftarrow\rea_{+}$ con
$\pi\left(|f|\right)<\infty$, la ecuaci\'on anterior se cumple.
\end{itemize}
\end{Teo}

\begin{Teo}[Teorema 2.2, Down \cite{Down}]\label{Tma2.2.Down}
Suponga que el fluido modelo es inestable en el sentido de que
para alguna $\epsilon_{0},c_{0}\geq0$,
\begin{equation}\label{Eq.Inestability}
|Q\left(T\right)|\geq\epsilon_{0}T-c_{0}\textrm{,   }T\geq0,
\end{equation}
para cualquier condici\'on inicial $Q\left(0\right)$, con
$|Q\left(0\right)|=1$. Entonces para cualquier $0<q\leq1$, existe
$B<0$ tal que para cualquier $|x|\geq B$,
\begin{equation}
\prob_{x}\left\{\mathbb{X}\rightarrow\infty\right\}\geq q.
\end{equation}
\end{Teo}


\begin{Def}
Sea $X$ un conjunto y $\mathcal{F}$ una $\sigma$-\'algebra de
subconjuntos de $X$, la pareja $\left(X,\mathcal{F}\right)$ es
llamado espacio medible. Un subconjunto $A$ de $X$ es llamado
medible, o medible con respecto a $\mathcal{F}$, si
$A\in\mathcal{F}$.
\end{Def}

\begin{Def}
Sea $\left(X,\mathcal{F},\mu\right)$ espacio de medida. Se dice
que la medida $\mu$ es $\sigma$-finita si se puede escribir
$X=\bigcup_{n\geq1}X_{n}$ con $X_{n}\in\mathcal{F}$ y
$\mu\left(X_{n}\right)<\infty$.
\end{Def}

\begin{Def}\label{Cto.Borel}
Sea $X$ el conjunto de los \'umeros reales $\rea$. El \'algebra de
Borel es la $\sigma$-\'algebra $B$ generada por los intervalos
abiertos $\left(a,b\right)\in\rea$. Cualquier conjunto en $B$ es
llamado {\em Conjunto de Borel}.
\end{Def}

\begin{Def}\label{Funcion.Medible}
Una funci\'on $f:X\rightarrow\rea$, es medible si para cualquier
n\'umero real $\alpha$ el conjunto
\[\left\{x\in X:f\left(x\right)>\alpha\right\}\]
pertenece a $X$. Equivalentemente, se dice que $f$ es medible si
\[f^{-1}\left(\left(\alpha,\infty\right)\right)=\left\{x\in X:f\left(x\right)>\alpha\right\}\in\mathcal{F}.\]
\end{Def}


\begin{Def}\label{Def.Cilindros}
Sean $\left(\Omega_{i},\mathcal{F}_{i}\right)$, $i=1,2,\ldots,$
espacios medibles y $\Omega=\prod_{i=1}^{\infty}\Omega_{i}$ el
conjunto de todas las sucesiones
$\left(\omega_{1},\omega_{2},\ldots,\right)$ tales que
$\omega_{i}\in\Omega_{i}$, $i=1,2,\ldots,$. Si
$B^{n}\subset\prod_{i=1}^{\infty}\Omega_{i}$, definimos
$B_{n}=\left\{\omega\in\Omega:\left(\omega_{1},\omega_{2},\ldots,\omega_{n}\right)\in
B^{n}\right\}$. Al conjunto $B_{n}$ se le llama {\em cilindro} con
base $B^{n}$, el cilindro es llamado medible si
$B^{n}\in\prod_{i=1}^{\infty}\mathcal{F}_{i}$.
\end{Def}


\begin{Def}\label{Def.Proc.Adaptado}[TSP, Ash \cite{RBA}]
Sea $X\left(t\right),t\geq0$ proceso estoc\'astico, el proceso es
adaptado a la familia de $\sigma$-\'algebras $\mathcal{F}_{t}$,
para $t\geq0$, si para $s<t$ implica que
$\mathcal{F}_{s}\subset\mathcal{F}_{t}$, y $X\left(t\right)$ es
$\mathcal{F}_{t}$-medible para cada $t$. Si no se especifica
$\mathcal{F}_{t}$ entonces se toma $\mathcal{F}_{t}$ como
$\mathcal{F}\left(X\left(s\right),s\leq t\right)$, la m\'as
peque\~na $\sigma$-\'algebra de subconjuntos de $\Omega$ que hace
que cada $X\left(s\right)$, con $s\leq t$ sea Borel medible.
\end{Def}


\begin{Def}\label{Def.Tiempo.Paro}[TSP, Ash \cite{RBA}]
Sea $\left\{\mathcal{F}\left(t\right),t\geq0\right\}$ familia
creciente de sub $\sigma$-\'algebras. es decir,
$\mathcal{F}\left(s\right)\subset\mathcal{F}\left(t\right)$ para
$s\leq t$. Un tiempo de paro para $\mathcal{F}\left(t\right)$ es
una funci\'on $T:\Omega\rightarrow\left[0,\infty\right]$ tal que
$\left\{T\leq t\right\}\in\mathcal{F}\left(t\right)$ para cada
$t\geq0$. Un tiempo de paro para el proceso estoc\'astico
$X\left(t\right),t\geq0$ es un tiempo de paro para las
$\sigma$-\'algebras
$\mathcal{F}\left(t\right)=\mathcal{F}\left(X\left(s\right)\right)$.
\end{Def}

\begin{Def}
Sea $X\left(t\right),t\geq0$ proceso estoc\'astico, con
$\left(S,\chi\right)$ espacio de estados. Se dice que el proceso
es adaptado a $\left\{\mathcal{F}\left(t\right)\right\}$, es
decir, si para cualquier $s,t\in I$, $I$ conjunto de \'indices,
$s<t$, se tiene que
$\mathcal{F}\left(s\right)\subset\mathcal{F}\left(t\right)$ y
$X\left(t\right)$ es $\mathcal{F}\left(t\right)$-medible,
\end{Def}

\begin{Def}
Sea $X\left(t\right),t\geq0$ proceso estoc\'astico, se dice que es
un Proceso de Markov relativo a $\mathcal{F}\left(t\right)$ o que
$\left\{X\left(t\right),\mathcal{F}\left(t\right)\right\}$ es de
Markov si y s\'olo si para cualquier conjunto $B\in\chi$,  y
$s,t\in I$, $s<t$ se cumple que
\begin{equation}\label{Prop.Markov}
P\left\{X\left(t\right)\in
B|\mathcal{F}\left(s\right)\right\}=P\left\{X\left(t\right)\in
B|X\left(s\right)\right\}.
\end{equation}
\end{Def}
\begin{Note}
Si se dice que $\left\{X\left(t\right)\right\}$ es un Proceso de
Markov sin mencionar $\mathcal{F}\left(t\right)$, se asumir\'a que
\begin{eqnarray*}
\mathcal{F}\left(t\right)=\mathcal{F}_{0}\left(t\right)=\mathcal{F}\left(X\left(r\right),r\leq
t\right),
\end{eqnarray*}
entonces la ecuaci\'on (\ref{Prop.Markov}) se puede escribir como
\begin{equation}
P\left\{X\left(t\right)\in B|X\left(r\right),r\leq s\right\} =
P\left\{X\left(t\right)\in B|X\left(s\right)\right\}
\end{equation}
\end{Note}

\begin{Teo}
Sea $\left(X_{n},\mathcal{F}_{n},n=0,1,\ldots,\right\}$ Proceso de
Markov con espacio de estados $\left(S_{0},\chi_{0}\right)$
generado por una distribuici\'on inicial $P_{o}$ y probabilidad de
transici\'on $p_{mn}$, para $m,n=0,1,\ldots,$ $m<n$, que por
notaci\'on se escribir\'a como $p\left(m,n,x,B\right)\rightarrow
p_{mn}\left(x,B\right)$. Sea $S$ tiempo de paro relativo a la
$\sigma$-\'algebra $\mathcal{F}_{n}$. Sea $T$ funci\'on medible,
$T:\Omega\rightarrow\left\{0,1,\ldots,\right\}$. Sup\'ongase que
$T\geq S$, entonces $T$ es tiempo de paro. Si $B\in\chi_{0}$,
entonces
\begin{equation}\label{Prop.Fuerte.Markov}
P\left\{X\left(T\right)\in
B,T<\infty|\mathcal{F}\left(S\right)\right\} =
p\left(S,T,X\left(s\right),B\right)
\end{equation}
en $\left\{T<\infty\right\}$.
\end{Teo}


Sea $K$ conjunto numerable y sea $d:K\rightarrow\nat$ funci\'on.
Para $v\in K$, $M_{v}$ es un conjunto abierto de
$\rea^{d\left(v\right)}$. Entonces \[E=\cup_{v\in
K}M_{v}=\left\{\left(v,\zeta\right):v\in K,\zeta\in
M_{v}\right\}.\]

Sea $\mathcal{E}$ la clase de conjuntos medibles en $E$:
\[\mathcal{E}=\left\{\cup_{v\in K}A_{v}:A_{v}\in \mathcal{M}_{v}\right\}.\]

donde $\mathcal{M}$ son los conjuntos de Borel de $M_{v}$.
Entonces $\left(E,\mathcal{E}\right)$ es un espacio de Borel. El
estado del proceso se denotar\'a por
$\mathbf{x}_{t}=\left(v_{t},\zeta_{t}\right)$. La distribuci\'on
de $\left(\mathbf{x}_{t}\right)$ est\'a determinada por por los
siguientes objetos:

\begin{itemize}
\item[i)] Los campos vectoriales $\left(\mathcal{H}_{v},v\in
K\right)$. \item[ii)] Una funci\'on medible $\lambda:E\rightarrow
\rea_{+}$. \item[iii)] Una medida de transici\'on
$Q:\mathcal{E}\times\left(E\cup\Gamma^{*}\right)\rightarrow\left[0,1\right]$
donde
\begin{equation}
\Gamma^{*}=\cup_{v\in K}\partial^{*}M_{v}.
\end{equation}
y
\begin{equation}
\partial^{*}M_{v}=\left\{z\in\partial M_{v}:\mathbf{\mathbf{\phi}_{v}\left(t,\zeta\right)=\mathbf{z}}\textrm{ para alguna }\left(t,\zeta\right)\in\rea_{+}\times M_{v}\right\}.
\end{equation}
$\partial M_{v}$ denota  la frontera de $M_{v}$.
\end{itemize}

El campo vectorial $\left(\mathcal{H}_{v},v\in K\right)$ se supone
tal que para cada $\mathbf{z}\in M_{v}$ existe una \'unica curva
integral $\mathbf{\phi}_{v}\left(t,\zeta\right)$ que satisface la
ecuaci\'on

\begin{equation}
\frac{d}{dt}f\left(\zeta_{t}\right)=\mathcal{H}f\left(\zeta_{t}\right),
\end{equation}
con $\zeta_{0}=\mathbf{z}$, para cualquier funci\'on suave
$f:\rea^{d}\rightarrow\rea$ y $\mathcal{H}$ denota el operador
diferencial de primer orden, con $\mathcal{H}=\mathcal{H}_{v}$ y
$\zeta_{t}=\mathbf{\phi}\left(t,\mathbf{z}\right)$. Adem\'as se
supone que $\mathcal{H}_{v}$ es conservativo, es decir, las curvas
integrales est\'an definidas para todo $t>0$.

Para $\mathbf{x}=\left(v,\zeta\right)\in E$ se denota
\[t^{*}\mathbf{x}=inf\left\{t>0:\mathbf{\phi}_{v}\left(t,\zeta\right)\in\partial^{*}M_{v}\right\}\]

En lo que respecta a la funci\'on $\lambda$, se supondr\'a que
para cada $\left(v,\zeta\right)\in E$ existe un $\epsilon>0$ tal
que la funci\'on
$s\rightarrow\lambda\left(v,\phi_{v}\left(s,\zeta\right)\right)\in
E$ es integrable para $s\in\left[0,\epsilon\right)$. La medida de
transici\'on $Q\left(A;\mathbf{x}\right)$ es una funci\'on medible
de $\mathbf{x}$ para cada $A\in\mathcal{E}$, definida para
$\mathbf{x}\in E\cup\Gamma^{*}$ y es una medida de probabilidad en
$\left(E,\mathcal{E}\right)$ para cada $\mathbf{x}\in E$.

El movimiento del proceso $\left(\mathbf{x}_{t}\right)$ comenzando
en $\mathbf{x}=\left(n,\mathbf{z}\right)\in E$ se puede construir
de la siguiente manera, def\'inase la funci\'on $F$ por

\begin{equation}
F\left(t\right)=\left\{\begin{array}{ll}\\
exp\left(-\int_{0}^{t}\lambda\left(n,\phi_{n}\left(s,\mathbf{z}\right)\right)ds\right), & t<t^{*}\left(\mathbf{x}\right),\\
0, & t\geq t^{*}\left(\mathbf{x}\right)
\end{array}\right.
\end{equation}

Sea $T_{1}$ una variable aleatoria tal que
$\prob\left[T_{1}>t\right]=F\left(t\right)$, ahora sea la variable
aleatoria $\left(N,Z\right)$ con distribuici\'on
$Q\left(\cdot;\phi_{n}\left(T_{1},\mathbf{z}\right)\right)$. La
trayectoria de $\left(\mathbf{x}_{t}\right)$ para $t\leq T_{1}$
es\footnote{Revisar p\'agina 362, y 364 de Davis \cite{Davis}.}
\begin{eqnarray*}
\mathbf{x}_{t}=\left(v_{t},\zeta_{t}\right)=\left\{\begin{array}{ll}
\left(n,\phi_{n}\left(t,\mathbf{z}\right)\right), & t<T_{1},\\
\left(N,\mathbf{Z}\right), & t=t_{1}.
\end{array}\right.
\end{eqnarray*}

Comenzando en $\mathbf{x}_{T_{1}}$ se selecciona el siguiente
tiempo de intersalto $T_{2}-T_{1}$ lugar del post-salto
$\mathbf{x}_{T_{2}}$ de manera similar y as\'i sucesivamente. Este
procedimiento nos da una trayectoria determinista por partes
$\mathbf{x}_{t}$ con tiempos de salto $T_{1},T_{2},\ldots$. Bajo
las condiciones enunciadas para $\lambda,T_{1}>0$  y
$T_{1}-T_{2}>0$ para cada $i$, con probabilidad 1. Se supone que
se cumple la siquiente condici\'on.

\begin{Sup}[Supuesto 3.1, Davis \cite{Davis}]\label{Sup3.1.Davis}
Sea $N_{t}:=\sum_{t}\indora_{\left(t\geq t\right)}$ el n\'umero de
saltos en $\left[0,t\right]$. Entonces
\begin{equation}
\esp\left[N_{t}\right]<\infty\textrm{ para toda }t.
\end{equation}
\end{Sup}

es un proceso de Markov, m\'as a\'un, es un Proceso Fuerte de
Markov, es decir, la Propiedad Fuerte de Markov se cumple para
cualquier tiempo de paro.
%_________________________________________________________________________

En esta secci\'on se har\'an las siguientes consideraciones: $E$
es un espacio m\'etrico separable y la m\'etrica $d$ es compatible
con la topolog\'ia.


\begin{Def}
Un espacio topol\'ogico $E$ es llamado {\em Luisin} si es
homeomorfo a un subconjunto de Borel de un espacio m\'etrico
compacto.
\end{Def}

\begin{Def}
Un espacio topol\'ogico $E$ es llamado de {\em Rad\'on} si es
homeomorfo a un subconjunto universalmente medible de un espacio
m\'etrico compacto.
\end{Def}

Equivalentemente, la definici\'on de un espacio de Rad\'on puede
encontrarse en los siguientes t\'erminos:


\begin{Def}
$E$ es un espacio de Rad\'on si cada medida finita en
$\left(E,\mathcal{B}\left(E\right)\right)$ es regular interior o cerrada,
{\em tight}.
\end{Def}

\begin{Def}
Una medida finita, $\lambda$ en la $\sigma$-\'algebra de Borel de
un espacio metrizable $E$ se dice cerrada si
\begin{equation}\label{Eq.A2.3}
\lambda\left(E\right)=sup\left\{\lambda\left(K\right):K\textrm{ es
compacto en }E\right\}.
\end{equation}
\end{Def}

El siguiente teorema nos permite tener una mejor caracterizaci\'on de los espacios de Rad\'on:
\begin{Teo}\label{Tma.A2.2}
Sea $E$ espacio separable metrizable. Entonces $E$ es Radoniano si y s\'olo s\'i cada medida finita en $\left(E,\mathcal{B}\left(E\right)\right)$ es cerrada.
\end{Teo}

%_________________________________________________________________________________________
\subsection{Propiedades de Markov}
%_________________________________________________________________________________________

Sea $E$ espacio de estados, tal que $E$ es un espacio de Rad\'on, $\mathcal{B}\left(E\right)$ $\sigma$-\'algebra de Borel en $E$, que se denotar\'a por $\mathcal{E}$.

Sea $\left(X,\mathcal{G},\prob\right)$ espacio de probabilidad, $I\subset\rea$ conjunto de índices. Sea $\mathcal{F}_{\leq t}$ la $\sigma$-\'algebra natural definida como $\sigma\left\{f\left(X_{r}\right):r\in I, rleq t,f\in\mathcal{E}\right\}$. Se considerar\'a una $\sigma$-\'algebra m\'as general, $ \left(\mathcal{G}_{t}\right)$ tal que $\left(X_{t}\right)$ sea $\mathcal{E}$-adaptado.

\begin{Def}
Una familia $\left(P_{s,t}\right)$ de kernels de Markov en $\left(E,\mathcal{E}\right)$ indexada por pares $s,t\in I$, con $s\leq t$ es una funci\'on de transici\'on en $\ER$, si  para todo $r\leq s< t$ en $I$ y todo $x\in E$, $B\in\mathcal{E}$
\begin{equation}\label{Eq.Kernels}
P_{r,t}\left(x,B\right)=\int_{E}P_{r,s}\left(x,dy\right)P_{s,t}\left(y,B\right)\footnote{Ecuaci\'on de Chapman-Kolmogorov}.
\end{equation}
\end{Def}

Se dice que la funci\'on de transici\'on $\KM$ en $\ER$ es la funci\'on de transici\'on para un proceso $\PE$  con valores en $E$ y que satisface la propiedad de Markov\footnote{\begin{equation}\label{Eq.1.4.S}
\prob\left\{H|\mathcal{G}_{t}\right\}=\prob\left\{H|X_{t}\right\}\textrm{ }H\in p\mathcal{F}_{\geq t}.
\end{equation}} (\ref{Eq.1.4.S}) relativa a $\left(\mathcal{G}_{t}\right)$ si 

\begin{equation}\label{Eq.1.6.S}
\prob\left\{f\left(X_{t}\right)|\mathcal{G}_{s}\right\}=P_{s,t}f\left(X_{t}\right)\textrm{ }s\leq t\in I,\textrm{ }f\in b\mathcal{E}.
\end{equation}

\begin{Def}
Una familia $\left(P_{t}\right)_{t\geq0}$ de kernels de Markov en $\ER$ es llamada {\em Semigrupo de Transici\'on de Markov} o {\em Semigrupo de Transici\'on} si
\[P_{t+s}f\left(x\right)=P_{t}\left(P_{s}f\right)\left(x\right),\textrm{ }t,s\geq0,\textrm{ }x\in E\textrm{ }f\in b\mathcal{E}.\]
\end{Def}
\begin{Note}
Si la funci\'on de transici\'on $\KM$ es llamada homog\'enea si $P_{s,t}=P_{t-s}$.
\end{Note}

Un proceso de Markov que satisface la ecuaci\'on (\ref{Eq.1.6.S}) con funci\'on de transici\'on homog\'enea $\left(P_{t}\right)$ tiene la propiedad caracter\'istica
\begin{equation}\label{Eq.1.8.S}
\prob\left\{f\left(X_{t+s}\right)|\mathcal{G}_{t}\right\}=P_{s}f\left(X_{t}\right)\textrm{ }t,s\geq0,\textrm{ }f\in b\mathcal{E}.
\end{equation}
La ecuaci\'on anterior es la {\em Propiedad Simple de Markov} de $X$ relativa a $\left(P_{t}\right)$.

En este sentido el proceso $\PE$ cumple con la propiedad de Markov (\ref{Eq.1.8.S}) relativa a $\left(\Omega,\mathcal{G},\mathcal{G}_{t},\prob\right)$ con semigrupo de transici\'on $\left(P_{t}\right)$.
%_________________________________________________________________________________________
\subsection{Primer Condici\'on de Regularidad}
%_________________________________________________________________________________________
%\newcommand{\EM}{\left(\Omega,\mathcal{G},\prob\right)}
%\newcommand{\E4}{\left(\Omega,\mathcal{G},\mathcal{G}_{t},\prob\right)}
\begin{Def}
Un proceso estoc\'astico $\PE$ definido en $\left(\Omega,\mathcal{G},\prob\right)$ con valores en el espacio topol\'ogico $E$ es continuo por la derecha si cada trayectoria muestral $t\rightarrow X_{t}\left(w\right)$ es un mapeo continuo por la derecha de $I$ en $E$.
\end{Def}

\begin{Def}[HD1]\label{Eq.2.1.S}
Un semigrupo de Markov $\left/P_{t}\right)$ en un espacio de Rad\'on $E$ se dice que satisface la condici\'on {\em HD1} si, dada una medida de probabilidad $\mu$ en $E$, existe una $\sigma$-\'algebra $\mathcal{E^{*}}$ con $\mathcal{E}\subset\mathcal{E}$ y $P_{t}\left(b\mathcal{E}^{*}\right)\subset b\mathcal{E}^{*}$, y un $\mathcal{E}^{*}$-proceso $E$-valuado continuo por la derecha $\PE$ en alg\'un espacio de probabilidad filtrado $\left(\Omega,\mathcal{G},\mathcal{G}_{t},\prob\right)$ tal que $X=\left(\Omega,\mathcal{G},\mathcal{G}_{t},\prob\right)$ es de Markov (Homog\'eneo) con semigrupo de transici\'on $(P_{t})$ y distribuci\'on inicial $\mu$.
\end{Def}

Considerese la colecci\'on de variables aleatorias $X_{t}$ definidas en alg\'un espacio de probabilidad, y una colecci\'on de medidas $\mathbf{P}^{x}$ tales que $\mathbf{P}^{x}\left\{X_{0}=x\right\}$, y bajo cualquier $\mathbf{P}^{x}$, $X_{t}$ es de Markov con semigrupo $\left(P_{t}\right)$. $\mathbf{P}^{x}$ puede considerarse como la distribuci\'on condicional de $\mathbf{P}$ dado $X_{0}=x$.

\begin{Def}\label{Def.2.2.S}
Sea $E$ espacio de Rad\'on, $\SG$ semigrupo de Markov en $\ER$. La colecci\'on $\mathbf{X}=\left(\Omega,\mathcal{G},\mathcal{G}_{t},X_{t},\theta_{t},\CM\right)$ es un proceso $\mathcal{E}$-Markov continuo por la derecha simple, con espacio de estados $E$ y semigrupo de transici\'on $\SG$ en caso de que $\mathbf{X}$ satisfaga las siguientes condiciones:
\begin{itemize}
\item[i)] $\left(\Omega,\mathcal{G},\mathcal{G}_{t}\right)$ es un espacio de medida filtrado, y $X_{t}$ es un proceso $E$-valuado continuo por la derecha $\mathcal{E}^{*}$-adaptado a $\left(\mathcal{G}_{t}\right)$;

\item[ii)] $\left(\theta_{t}\right)_{t\geq0}$ es una colecci\'on de operadores {\em shift} para $X$, es decir, mapea $\Omega$ en s\'i mismo satisfaciendo para $t,s\geq0$,

\begin{equation}\label{Eq.Shift}
\theta_{t}\circ\theta_{s}=\theta_{t+s}\textrm{ y }X_{t}\circ\theta_{t}=X_{t+s};
\end{equation}

\item[iii)] Para cualquier $x\in E$,$\CM\left\{X_{0}=x\right\}=1$, y el proceso $\PE$ tiene la propiedad de Markov (\ref{Eq.1.8.S}) con semigrupo de transici\'on $\SG$ relativo a $\left(\Omega,\mathcal{G},\mathcal{G}_{t},\CM\right)$.
\end{itemize}
\end{Def}

\begin{Def}[HD2]\label{Eq.2.2.S}
Para cualquier $\alpha>0$ y cualquier $f\in S^{\alpha}$, el proceso $t\rightarrow f\left(X_{t}\right)$ es continuo por la derecha casi seguramente.
\end{Def}

\begin{Def}\label{Def.PD}
Un sistema $\mathbf{X}=\left(\Omega,\mathcal{G},\mathcal{G}_{t},X_{t},\theta_{t},\CM\right)$ es un proceso derecho en el espacio de Rad\'on $E$ con semigrupo de transici\'on $\SG$ provisto de:
\begin{itemize}
\item[i)] $\mathbf{X}$ es una realizaci\'on  continua por la derecha, \ref{Def.2.2.S}, de $\SG$.

\item[ii)] $\mathbf{X}$ satisface la condicion HD2, \ref{Eq.2.2.S}, relativa a $\mathcal{G}_{t}$.

\item[iii)] $\mathcal{G}_{t}$ es aumentado y continuo por la derecha.
\end{itemize}
\end{Def}




\begin{Lema}[Lema 4.2, Dai\cite{Dai}]\label{Lema4.2}
Sea $\left\{x_{n}\right\}\subset \mathbf{X}$ con
$|x_{n}|\rightarrow\infty$, conforme $n\rightarrow\infty$. Suponga
que
\[lim_{n\rightarrow\infty}\frac{1}{|x_{n}|}U\left(0\right)=\overline{U}\]
y
\[lim_{n\rightarrow\infty}\frac{1}{|x_{n}|}V\left(0\right)=\overline{V}.\]

Entonces, conforme $n\rightarrow\infty$, casi seguramente

\begin{equation}\label{E1.4.2}
\frac{1}{|x_{n}|}\Phi^{k}\left(\left[|x_{n}|t\right]\right)\rightarrow
P_{k}^{'}t\textrm{, u.o.c.,}
\end{equation}

\begin{equation}\label{E1.4.3}
\frac{1}{|x_{n}|}E^{x_{n}}_{k}\left(|x_{n}|t\right)\rightarrow
\alpha_{k}\left(t-\overline{U}_{k}\right)^{+}\textrm{, u.o.c.,}
\end{equation}

\begin{equation}\label{E1.4.4}
\frac{1}{|x_{n}|}S^{x_{n}}_{k}\left(|x_{n}|t\right)\rightarrow
\mu_{k}\left(t-\overline{V}_{k}\right)^{+}\textrm{, u.o.c.,}
\end{equation}

donde $\left[t\right]$ es la parte entera de $t$ y
$\mu_{k}=1/m_{k}=1/\esp\left[\eta_{k}\left(1\right)\right]$.
\end{Lema}

\begin{Lema}[Lema 4.3, Dai\cite{Dai}]\label{Lema.4.3}
Sea $\left\{x_{n}\right\}\subset \mathbf{X}$ con
$|x_{n}|\rightarrow\infty$, conforme $n\rightarrow\infty$. Suponga
que
\[lim_{n\rightarrow\infty}\frac{1}{|x_{n}|}U\left(0\right)=\overline{U}_{k}\]
y
\[lim_{n\rightarrow\infty}\frac{1}{|x_{n}|}V\left(0\right)=\overline{V}_{k}.\]
\begin{itemize}
\item[a)] Conforme $n\rightarrow\infty$ casi seguramente,
\[lim_{n\rightarrow\infty}\frac{1}{|x_{n}|}U^{x_{n}}_{k}\left(|x_{n}|t\right)=\left(\overline{U}_{k}-t\right)^{+}\textrm{, u.o.c.}\]
y
\[lim_{n\rightarrow\infty}\frac{1}{|x_{n}|}V^{x_{n}}_{k}\left(|x_{n}|t\right)=\left(\overline{V}_{k}-t\right)^{+}.\]

\item[b)] Para cada $t\geq0$ fijo,
\[\left\{\frac{1}{|x_{n}|}U^{x_{n}}_{k}\left(|x_{n}|t\right),|x_{n}|\geq1\right\}\]
y
\[\left\{\frac{1}{|x_{n}|}V^{x_{n}}_{k}\left(|x_{n}|t\right),|x_{n}|\geq1\right\}\]
\end{itemize}
son uniformemente convergentes.
\end{Lema}

$S_{l}^{x}\left(t\right)$ es el n\'umero total de servicios
completados de la clase $l$, si la clase $l$ est\'a dando $t$
unidades de tiempo de servicio. Sea $T_{l}^{x}\left(x\right)$ el
monto acumulado del tiempo de servicio que el servidor
$s\left(l\right)$ gasta en los usuarios de la clase $l$ al tiempo
$t$. Entonces $S_{l}^{x}\left(T_{l}^{x}\left(t\right)\right)$ es
el n\'umero total de servicios completados para la clase $l$ al
tiempo $t$. Una fracci\'on de estos usuarios,
$\Phi_{l}^{x}\left(S_{l}^{x}\left(T_{l}^{x}\left(t\right)\right)\right)$,
se convierte en usuarios de la clase $k$.\\

Entonces, dado lo anterior, se tiene la siguiente representaci\'on
para el proceso de la longitud de la cola:\\

\begin{equation}
Q_{k}^{x}\left(t\right)=_{k}^{x}\left(0\right)+E_{k}^{x}\left(t\right)+\sum_{l=1}^{K}\Phi_{k}^{l}\left(S_{l}^{x}\left(T_{l}^{x}\left(t\right)\right)\right)-S_{k}^{x}\left(T_{k}^{x}\left(t\right)\right)
\end{equation}
para $k=1,\ldots,K$. Para $i=1,\ldots,d$, sea
\[I_{i}^{x}\left(t\right)=t-\sum_{j\in C_{i}}T_{k}^{x}\left(t\right).\]

Entonces $I_{i}^{x}\left(t\right)$ es el monto acumulado del
tiempo que el servidor $i$ ha estado desocupado al tiempo $t$. Se
est\'a asumiendo que las disciplinas satisfacen la ley de
conservaci\'on del trabajo, es decir, el servidor $i$ est\'a en
pausa solamente cuando no hay usuarios en la estaci\'on $i$.
Entonces, se tiene que

\begin{equation}
\int_{0}^{\infty}\left(\sum_{k\in
C_{i}}Q_{k}^{x}\left(t\right)\right)dI_{i}^{x}\left(t\right)=0,
\end{equation}
para $i=1,\ldots,d$.\\

Hacer
\[T^{x}\left(t\right)=\left(T_{1}^{x}\left(t\right),\ldots,T_{K}^{x}\left(t\right)\right)^{'},\]
\[I^{x}\left(t\right)=\left(I_{1}^{x}\left(t\right),\ldots,I_{K}^{x}\left(t\right)\right)^{'}\]
y
\[S^{x}\left(T^{x}\left(t\right)\right)=\left(S_{1}^{x}\left(T_{1}^{x}\left(t\right)\right),\ldots,S_{K}^{x}\left(T_{K}^{x}\left(t\right)\right)\right)^{'}.\]

Para una disciplina que cumple con la ley de conservaci\'on del
trabajo, en forma vectorial, se tiene el siguiente conjunto de
ecuaciones

\begin{equation}\label{Eq.MF.1.3}
Q^{x}\left(t\right)=Q^{x}\left(0\right)+E^{x}\left(t\right)+\sum_{l=1}^{K}\Phi^{l}\left(S_{l}^{x}\left(T_{l}^{x}\left(t\right)\right)\right)-S^{x}\left(T^{x}\left(t\right)\right),\\
\end{equation}

\begin{equation}\label{Eq.MF.2.3}
Q^{x}\left(t\right)\geq0,\\
\end{equation}

\begin{equation}\label{Eq.MF.3.3}
T^{x}\left(0\right)=0,\textrm{ y }\overline{T}^{x}\left(t\right)\textrm{ es no decreciente},\\
\end{equation}

\begin{equation}\label{Eq.MF.4.3}
I^{x}\left(t\right)=et-CT^{x}\left(t\right)\textrm{ es no
decreciente}\\
\end{equation}

\begin{equation}\label{Eq.MF.5.3}
\int_{0}^{\infty}\left(CQ^{x}\left(t\right)\right)dI_{i}^{x}\left(t\right)=0,\\
\end{equation}

\begin{equation}\label{Eq.MF.6.3}
\textrm{Condiciones adicionales en
}\left(\overline{Q}^{x}\left(\cdot\right),\overline{T}^{x}\left(\cdot\right)\right)\textrm{
espec\'ificas de la disciplina de la cola,}
\end{equation}

donde $e$ es un vector de unos de dimensi\'on $d$, $C$ es la
matriz definida por
\[C_{ik}=\left\{\begin{array}{cc}
1,& S\left(k\right)=i,\\
0,& \textrm{ en otro caso}.\\
\end{array}\right.
\]
Es necesario enunciar el siguiente Teorema que se utilizar\'a para
el Teorema \ref{Tma.4.2.Dai}:
\begin{Teo}[Teorema 4.1, Dai \cite{Dai}]
Considere una disciplina que cumpla la ley de conservaci\'on del
trabajo, para casi todas las trayectorias muestrales $\omega$ y
cualquier sucesi\'on de estados iniciales
$\left\{x_{n}\right\}\subset \mathbf{X}$, con
$|x_{n}|\rightarrow\infty$, existe una subsucesi\'on
$\left\{x_{n_{j}}\right\}$ con $|x_{n_{j}}|\rightarrow\infty$ tal
que
\begin{equation}\label{Eq.4.15}
\frac{1}{|x_{n_{j}}|}\left(Q^{x_{n_{j}}}\left(0\right),U^{x_{n_{j}}}\left(0\right),V^{x_{n_{j}}}\left(0\right)\right)\rightarrow\left(\overline{Q}\left(0\right),\overline{U},\overline{V}\right),
\end{equation}

\begin{equation}\label{Eq.4.16}
\frac{1}{|x_{n_{j}}|}\left(Q^{x_{n_{j}}}\left(|x_{n_{j}}|t\right),T^{x_{n_{j}}}\left(|x_{n_{j}}|t\right)\right)\rightarrow\left(\overline{Q}\left(t\right),\overline{T}\left(t\right)\right)\textrm{
u.o.c.}
\end{equation}

Adem\'as,
$\left(\overline{Q}\left(t\right),\overline{T}\left(t\right)\right)$
satisface las siguientes ecuaciones:
\begin{equation}\label{Eq.MF.1.3a}
\overline{Q}\left(t\right)=Q\left(0\right)+\left(\alpha
t-\overline{U}\right)^{+}-\left(I-P\right)^{'}M^{-1}\left(\overline{T}\left(t\right)-\overline{V}\right)^{+},
\end{equation}

\begin{equation}\label{Eq.MF.2.3a}
\overline{Q}\left(t\right)\geq0,\\
\end{equation}

\begin{equation}\label{Eq.MF.3.3a}
\overline{T}\left(t\right)\textrm{ es no decreciente y comienza en cero},\\
\end{equation}

\begin{equation}\label{Eq.MF.4.3a}
\overline{I}\left(t\right)=et-C\overline{T}\left(t\right)\textrm{
es no decreciente,}\\
\end{equation}

\begin{equation}\label{Eq.MF.5.3a}
\int_{0}^{\infty}\left(C\overline{Q}\left(t\right)\right)d\overline{I}\left(t\right)=0,\\
\end{equation}

\begin{equation}\label{Eq.MF.6.3a}
\textrm{Condiciones adicionales en
}\left(\overline{Q}\left(\cdot\right),\overline{T}\left(\cdot\right)\right)\textrm{
especficas de la disciplina de la cola,}
\end{equation}
\end{Teo}

\begin{Def}[Definici\'on 4.1, , Dai \cite{Dai}]
Sea una disciplina de servicio espec\'ifica. Cualquier l\'imite
$\left(\overline{Q}\left(\cdot\right),\overline{T}\left(\cdot\right)\right)$
en \ref{Eq.4.16} es un {\em flujo l\'imite} de la disciplina.
Cualquier soluci\'on (\ref{Eq.MF.1.3a})-(\ref{Eq.MF.6.3a}) es
llamado flujo soluci\'on de la disciplina. Se dice que el modelo de flujo l\'imite, modelo de flujo, de la disciplina de la cola es estable si existe una constante
$\delta>0$ que depende de $\mu,\alpha$ y $P$ solamente, tal que
cualquier flujo l\'imite con
$|\overline{Q}\left(0\right)|+|\overline{U}|+|\overline{V}|=1$, se
tiene que $\overline{Q}\left(\cdot+\delta\right)\equiv0$.
\end{Def}

\begin{Teo}[Teorema 4.2, Dai\cite{Dai}]\label{Tma.4.2.Dai}
Sea una disciplina fija para la cola, suponga que se cumplen las
condiciones (1.2)-(1.5). Si el modelo de flujo l\'imite de la
disciplina de la cola es estable, entonces la cadena de Markov $X$
que describe la din\'amica de la red bajo la disciplina es Harris
recurrente positiva.
\end{Teo}

Ahora se procede a escalar el espacio y el tiempo para reducir la
aparente fluctuaci\'on del modelo. Consid\'erese el proceso
\begin{equation}\label{Eq.3.7}
\overline{Q}^{x}\left(t\right)=\frac{1}{|x|}Q^{x}\left(|x|t\right)
\end{equation}
A este proceso se le conoce como el fluido escalado, y cualquier l\'imite $\overline{Q}^{x}\left(t\right)$ es llamado flujo l\'imite del proceso de longitud de la cola. Haciendo $|q|\rightarrow\infty$ mientras se mantiene el resto de las componentes fijas, cualquier punto l\'imite del proceso de longitud de la cola normalizado $\overline{Q}^{x}$ es soluci\'on del siguiente modelo de flujo.

Al conjunto de ecuaciones dadas en \ref{Eq.3.8}-\ref{Eq.3.13} se
le llama {\em Modelo de flujo} y al conjunto de todas las
soluciones del modelo de flujo
$\left(\overline{Q}\left(\cdot\right),\overline{T}
\left(\cdot\right)\right)$ se le denotar\'a por $\mathcal{Q}$.

Si se hace $|x|\rightarrow\infty$ sin restringir ninguna de las
componentes, tambi\'en se obtienen un modelo de flujo, pero en
este caso el residual de los procesos de arribo y servicio
introducen un retraso:

\begin{Def}[Definici\'on 3.3, Dai y Meyn \cite{DaiSean}]
El modelo de flujo es estable si existe un tiempo fijo $t_{0}$ tal
que $\overline{Q}\left(t\right)=0$, con $t\geq t_{0}$, para
cualquier $\overline{Q}\left(\cdot\right)\in\mathcal{Q}$ que
cumple con $|\overline{Q}\left(0\right)|=1$.
\end{Def}

El siguiente resultado se encuentra en Chen \cite{Chen}.
\begin{Lemma}[Lema 3.1, Dai y Meyn \cite{DaiSean}]
Si el modelo de flujo definido por \ref{Eq.3.8}-\ref{Eq.3.13} es
estable, entonces el modelo de flujo retrasado es tambi\'en
estable, es decir, existe $t_{0}>0$ tal que
$\overline{Q}\left(t\right)=0$ para cualquier $t\geq t_{0}$, para
cualquier soluci\'on del modelo de flujo retrasado cuya
condici\'on inicial $\overline{x}$ satisface que
$|\overline{x}|=|\overline{Q}\left(0\right)|+|\overline{A}\left(0\right)|+|\overline{B}\left(0\right)|\leq1$.
\end{Lemma}


Propiedades importantes para el modelo de flujo retrasado:

\begin{Prop}
 Sea $\left(\overline{Q},\overline{T},\overline{T}^{0}\right)$ un flujo l\'imite de \ref{Eq.4.4} y suponga que cuando $x\rightarrow\infty$ a lo largo de
una subsucesi\'on
\[\left(\frac{1}{|x|}Q_{k}^{x}\left(0\right),\frac{1}{|x|}A_{k}^{x}\left(0\right),\frac{1}{|x|}B_{k}^{x}\left(0\right),\frac{1}{|x|}B_{k}^{x,0}\left(0\right)\right)\rightarrow\left(\overline{Q}_{k}\left(0\right),0,0,0\right)\]
para $k=1,\ldots,K$. EL flujo l\'imite tiene las siguientes
propiedades, donde las propiedades de la derivada se cumplen donde
la derivada exista:
\begin{itemize}
 \item[i)] Los vectores de tiempo ocupado $\overline{T}\left(t\right)$ y $\overline{T}^{0}\left(t\right)$ son crecientes y continuas con
$\overline{T}\left(0\right)=\overline{T}^{0}\left(0\right)=0$.
\item[ii)] Para todo $t\geq0$
\[\sum_{k=1}^{K}\left[\overline{T}_{k}\left(t\right)+\overline{T}_{k}^{0}\left(t\right)\right]=t\]
\item[iii)] Para todo $1\leq k\leq K$
\[\overline{Q}_{k}\left(t\right)=\overline{Q}_{k}\left(0\right)+\alpha_{k}t-\mu_{k}\overline{T}_{k}\left(t\right)\]
\item[iv)]  Para todo $1\leq k\leq K$
\[\dot{{\overline{T}}}_{k}\left(t\right)=\beta_{k}\] para $\overline{Q}_{k}\left(t\right)=0$.
\item[v)] Para todo $k,j$
\[\mu_{k}^{0}\overline{T}_{k}^{0}\left(t\right)=\mu_{j}^{0}\overline{T}_{j}^{0}\left(t\right)\]
\item[vi)]  Para todo $1\leq k\leq K$
\[\mu_{k}\dot{{\overline{T}}}_{k}\left(t\right)=l_{k}\mu_{k}^{0}\dot{{\overline{T}}}_{k}^{0}\left(t\right)\] para $\overline{Q}_{k}\left(t\right)>0$.
\end{itemize}
\end{Prop}

\begin{Lema}[Lema 3.1 \cite{Chen}]\label{Lema3.1}
Si el modelo de flujo es estable, definido por las ecuaciones
(3.8)-(3.13), entonces el modelo de flujo retrasado tambin es
estable.
\end{Lema}

\begin{Teo}[Teorema 5.2 \cite{Chen}]\label{Tma.5.2}
Si el modelo de flujo lineal correspondiente a la red de cola es
estable, entonces la red de colas es estable.
\end{Teo}

\begin{Teo}[Teorema 5.1 \cite{Chen}]\label{Tma.5.1.Chen}
La red de colas es estable si existe una constante $t_{0}$ que
depende de $\left(\alpha,\mu,T,U\right)$ y $V$ que satisfagan las
ecuaciones (5.1)-(5.5), $Z\left(t\right)=0$, para toda $t\geq
t_{0}$.
\end{Teo}



\begin{Lema}[Lema 5.2 \cite{Gut}]\label{Lema.5.2.Gut}
Sea $\left\{\xi\left(k\right):k\in\ent\right\}$ sucesin de
variables aleatorias i.i.d. con valores en
$\left(0,\infty\right)$, y sea $E\left(t\right)$ el proceso de
conteo
\[E\left(t\right)=max\left\{n\geq1:\xi\left(1\right)+\cdots+\xi\left(n-1\right)\leq t\right\}.\]
Si $E\left[\xi\left(1\right)\right]<\infty$, entonces para
cualquier entero $r\geq1$
\begin{equation}
lim_{t\rightarrow\infty}\esp\left[\left(\frac{E\left(t\right)}{t}\right)^{r}\right]=\left(\frac{1}{E\left[\xi_{1}\right]}\right)^{r}
\end{equation}
de aqu, bajo estas condiciones
\begin{itemize}
\item[a)] Para cualquier $t>0$,
$sup_{t\geq\delta}\esp\left[\left(\frac{E\left(t\right)}{t}\right)^{r}\right]$

\item[b)] Las variables aleatorias
$\left\{\left(\frac{E\left(t\right)}{t}\right)^{r}:t\geq1\right\}$
son uniformemente integrables.
\end{itemize}
\end{Lema}

\begin{Teo}[Teorema 5.1: Ley Fuerte para Procesos de Conteo
\cite{Gut}]\label{Tma.5.1.Gut} Sea
$0<\mu<\esp\left(X_{1}\right]\leq\infty$. entonces

\begin{itemize}
\item[a)] $\frac{N\left(t\right)}{t}\rightarrow\frac{1}{\mu}$
a.s., cuando $t\rightarrow\infty$.


\item[b)]$\esp\left[\frac{N\left(t\right)}{t}\right]^{r}\rightarrow\frac{1}{\mu^{r}}$,
cuando $t\rightarrow\infty$ para todo $r>0$..
\end{itemize}
\end{Teo}


\begin{Prop}[Proposicin 5.1 \cite{DaiSean}]\label{Prop.5.1}
Suponga que los supuestos (A1) y (A2) se cumplen, adems suponga
que el modelo de flujo es estable. Entonces existe $t_{0}>0$ tal
que
\begin{equation}\label{Eq.Prop.5.1}
lim_{|x|\rightarrow\infty}\frac{1}{|x|^{p+1}}\esp_{x}\left[|X\left(t_{0}|x|\right)|^{p+1}\right]=0.
\end{equation}

\end{Prop}


\begin{Prop}[Proposici\'on 5.3 \cite{DaiSean}]
Sea $X$ proceso de estados para la red de colas, y suponga que se
cumplen los supuestos (A1) y (A2), entonces para alguna constante
positiva $C_{p+1}<\infty$, $\delta>0$ y un conjunto compacto
$C\subset X$.

\begin{equation}\label{Eq.5.4}
\esp_{x}\left[\int_{0}^{\tau_{C}\left(\delta\right)}\left(1+|X\left(t\right)|^{p}\right)dt\right]\leq
C_{p+1}\left(1+|x|^{p+1}\right)
\end{equation}
\end{Prop}

\begin{Prop}[Proposici\'on 5.4 \cite{DaiSean}]
Sea $X$ un proceso de Markov Borel Derecho en $X$, sea
$f:X\leftarrow\rea_{+}$ y defina para alguna $\delta>0$, y un
conjunto cerrado $C\subset X$
\[V\left(x\right):=\esp_{x}\left[\int_{0}^{\tau_{C}\left(\delta\right)}f\left(X\left(t\right)\right)dt\right]\]
para $x\in X$. Si $V$ es finito en todas partes y uniformemente
acotada en $C$, entonces existe $k<\infty$ tal que
\begin{equation}\label{Eq.5.11}
\frac{1}{t}\esp_{x}\left[V\left(x\right)\right]+\frac{1}{t}\int_{0}^{t}\esp_{x}\left[f\left(X\left(s\right)\right)ds\right]\leq\frac{1}{t}V\left(x\right)+k,
\end{equation}
para $x\in X$ y $t>0$.
\end{Prop}


\begin{Teo}[Teorema 5.5 \cite{DaiSean}]
Suponga que se cumplen (A1) y (A2), adems suponga que el modelo
de flujo es estable. Entonces existe una constante $k_{p}<\infty$
tal que
\begin{equation}\label{Eq.5.13}
\frac{1}{t}\int_{0}^{t}\esp_{x}\left[|Q\left(s\right)|^{p}\right]ds\leq
k_{p}\left\{\frac{1}{t}|x|^{p+1}+1\right\}
\end{equation}
para $t\geq0$, $x\in X$. En particular para cada condici\'on inicial
\begin{equation}\label{Eq.5.14}
Limsup_{t\rightarrow\infty}\frac{1}{t}\int_{0}^{t}\esp_{x}\left[|Q\left(s\right)|^{p}\right]ds\leq
k_{p}
\end{equation}
\end{Teo}

\begin{Teo}[Teorema 6.2\cite{DaiSean}]\label{Tma.6.2}
Suponga que se cumplen los supuestos (A1)-(A3) y que el modelo de
flujo es estable, entonces se tiene que
\[\parallel P^{t}\left(c,\cdot\right)-\pi\left(\cdot\right)\parallel_{f_{p}}\rightarrow0\]
para $t\rightarrow\infty$ y $x\in X$. En particular para cada
condicin inicial
\[lim_{t\rightarrow\infty}\esp_{x}\left[\left|Q_{t}\right|^{p}\right]=\esp_{\pi}\left[\left|Q_{0}\right|^{p}\right]<\infty\]
\end{Teo}


\begin{Teo}[Teorema 6.3\cite{DaiSean}]\label{Tma.6.3}
Suponga que se cumplen los supuestos (A1)-(A3) y que el modelo de
flujo es estable, entonces con
$f\left(x\right)=f_{1}\left(x\right)$, se tiene que
\[lim_{t\rightarrow\infty}t^{(p-1)\left|P^{t}\left(c,\cdot\right)-\pi\left(\cdot\right)\right|_{f}=0},\]
para $x\in X$. En particular, para cada condicin inicial
\[lim_{t\rightarrow\infty}t^{(p-1)\left|\esp_{x}\left[Q_{t}\right]-\esp_{\pi}\left[Q_{0}\right]\right|=0}.\]
\end{Teo}


\begin{Prop}[Proposici\'on 5.1, Dai y Meyn \cite{DaiSean}]\label{Prop.5.1.DaiSean}
Suponga que los supuestos A1) y A2) son ciertos y que el modelo de flujo es estable. Entonces existe $t_{0}>0$ tal que
\begin{equation}
lim_{|x|\rightarrow\infty}\frac{1}{|x|^{p+1}}\esp_{x}\left[|X\left(t_{0}|x|\right)|^{p+1}\right]=0
\end{equation}
\end{Prop}

\begin{Lemma}[Lema 5.2, Dai y Meyn \cite{DaiSean}]\label{Lema.5.2.DaiSean}
 Sea $\left\{\zeta\left(k\right):k\in \mathbb{z}\right\}$ una sucesi\'on independiente e id\'enticamente distribuida que toma valores en $\left(0,\infty\right)$,
y sea
$E\left(t\right)=max\left(n\geq1:\zeta\left(1\right)+\cdots+\zeta\left(n-1\right)\leq
t\right)$. Si $\esp\left[\zeta\left(1\right)\right]<\infty$,
entonces para cualquier entero $r\geq1$
\begin{equation}
 lim_{t\rightarrow\infty}\esp\left[\left(\frac{E\left(t\right)}{t}\right)^{r}\right]=\left(\frac{1}{\esp\left[\zeta_{1}\right]}\right)^{r}.
\end{equation}
Luego, bajo estas condiciones:
\begin{itemize}
 \item[a)] para cualquier $\delta>0$, $\sup_{t\geq\delta}\esp\left[\left(\frac{E\left(t\right)}{t}\right)^{r}\right]<\infty$
\item[b)] las variables aleatorias
$\left\{\left(\frac{E\left(t\right)}{t}\right)^{r}:t\geq1\right\}$
son uniformemente integrables.
\end{itemize}
\end{Lemma}

\begin{Teo}[Teorema 5.5, Dai y Meyn \cite{DaiSean}]\label{Tma.5.5.DaiSean}
Suponga que los supuestos A1) y A2) se cumplen y que el modelo de
flujo es estable. Entonces existe una constante $\kappa_{p}$ tal
que
\begin{equation}
\frac{1}{t}\int_{0}^{t}\esp_{x}\left[|Q\left(s\right)|^{p}\right]ds\leq\kappa_{p}\left\{\frac{1}{t}|x|^{p+1}+1\right\}
\end{equation}
para $t>0$ y $x\in X$. En particular, para cada condici\'on
inicial
\begin{eqnarray*}
\limsup_{t\rightarrow\infty}\frac{1}{t}\int_{0}^{t}\esp_{x}\left[|Q\left(s\right)|^{p}\right]ds\leq\kappa_{p}.
\end{eqnarray*}
\end{Teo}

\begin{Teo}[Teorema 6.2, Dai y Meyn \cite{DaiSean}]\label{Tma.6.2.DaiSean}
Suponga que se cumplen los supuestos A1), A2) y A3) y que el
modelo de flujo es estable. Entonces se tiene que
\begin{equation}
\left\|P^{t}\left(x,\cdot\right)-\pi\left(\cdot\right)\right\|_{f_{p}}\textrm{,
}t\rightarrow\infty,x\in X.
\end{equation}
En particular para cada condici\'on inicial
\begin{eqnarray*}
\lim_{t\rightarrow\infty}\esp_{x}\left[|Q\left(t\right)|^{p}\right]=\esp_{\pi}\left[|Q\left(0\right)|^{p}\right]\leq\kappa_{r}
\end{eqnarray*}
\end{Teo}
\begin{Teo}[Teorema 6.3, Dai y Meyn \cite{DaiSean}]\label{Tma.6.3.DaiSean}
Suponga que se cumplen los supuestos A1), A2) y A3) y que el
modelo de flujo es estable. Entonces con
$f\left(x\right)=f_{1}\left(x\right)$ se tiene
\begin{equation}
\lim_{t\rightarrow\infty}t^{p-1}\left\|P^{t}\left(x,\cdot\right)-\pi\left(\cdot\right)\right\|_{f}=0.
\end{equation}
En particular para cada condici\'on inicial
\begin{eqnarray*}
\lim_{t\rightarrow\infty}t^{p-1}|\esp_{x}\left[Q\left(t\right)\right]-\esp_{\pi}\left[Q\left(0\right)\right]|=0.
\end{eqnarray*}
\end{Teo}

\begin{Teo}[Teorema 6.4, Dai y Meyn \cite{DaiSean}]\label{Tma.6.4.DaiSean}
Suponga que se cumplen los supuestos A1), A2) y A3) y que el
modelo de flujo es estable. Sea $\nu$ cualquier distribuci\'on de
probabilidad en $\left(X,\mathcal{B}_{X}\right)$, y $\pi$ la
distribuci\'on estacionaria de $X$.
\begin{itemize}
\item[i)] Para cualquier $f:X\leftarrow\rea_{+}$
\begin{equation}
\lim_{t\rightarrow\infty}\frac{1}{t}\int_{o}^{t}f\left(X\left(s\right)\right)ds=\pi\left(f\right):=\int
f\left(x\right)\pi\left(dx\right)
\end{equation}
$\prob$-c.s.

\item[ii)] Para cualquier $f:X\leftarrow\rea_{+}$ con
$\pi\left(|f|\right)<\infty$, la ecuaci\'on anterior se cumple.
\end{itemize}
\end{Teo}

\begin{Teo}[Teorema 2.2, Down \cite{Down}]\label{Tma2.2.Down}
Suponga que el fluido modelo es inestable en el sentido de que
para alguna $\epsilon_{0},c_{0}\geq0$,
\begin{equation}\label{Eq.Inestability}
|Q\left(T\right)|\geq\epsilon_{0}T-c_{0}\textrm{,   }T\geq0,
\end{equation}
para cualquier condici\'on inicial $Q\left(0\right)$, con
$|Q\left(0\right)|=1$. Entonces para cualquier $0<q\leq1$, existe
$B<0$ tal que para cualquier $|x|\geq B$,
\begin{equation}
\prob_{x}\left\{\mathbb{X}\rightarrow\infty\right\}\geq q.
\end{equation}
\end{Teo}



Es necesario hacer los siguientes supuestos sobre el
comportamiento del sistema de visitas c\'iclicas:
\begin{itemize}
\item Los tiempos de interarribo a la $k$-\'esima cola, son de la
forma $\left\{\xi_{k}\left(n\right)\right\}_{n\geq1}$, con la
propiedad de que son independientes e id{\'e}nticamente
distribuidos,
\item Los tiempos de servicio
$\left\{\eta_{k}\left(n\right)\right\}_{n\geq1}$ tienen la
propiedad de ser independientes e id{\'e}nticamente distribuidos,
\item Se define la tasa de arribo a la $k$-{\'e}sima cola como
$\lambda_{k}=1/\esp\left[\xi_{k}\left(1\right)\right]$,
\item la tasa de servicio para la $k$-{\'e}sima cola se define
como $\mu_{k}=1/\esp\left[\eta_{k}\left(1\right)\right]$,
\item tambi{\'e}n se define $\rho_{k}:=\lambda_{k}/\mu_{k}$, la
intensidad de tr\'afico del sistema o carga de la red, donde es
necesario que $\rho<1$ para cuestiones de estabilidad.
\end{itemize}



%_________________________________________________________________________
\subsection{Procesos de Estados Markoviano para el Sistema}
%_________________________________________________________________________

%_________________________________________________________________________
\subsection{Procesos Fuerte de Markov}
%_________________________________________________________________________
En Dai \cite{Dai} se muestra que para una amplia serie de disciplinas
de servicio el proceso $X$ es un Proceso Fuerte de
Markov, y por tanto se puede asumir que


Para establecer que $X=\left\{X\left(t\right),t\geq0\right\}$ es
un Proceso Fuerte de Markov, se siguen las secciones 2.3 y 2.4 de Kaspi and Mandelbaum \cite{KaspiMandelbaum}. \\

%______________________________________________________________
\subsubsection{Construcci\'on de un Proceso Determinista por partes, Davis
\cite{Davis}}.
%______________________________________________________________

%_________________________________________________________________________
\subsection{Procesos Harris Recurrentes Positivos}
%_________________________________________________________________________
Sea el proceso de Markov $X=\left\{X\left(t\right),t\geq0\right\}$
que describe la din\'amica de la red de colas. En lo que respecta
al supuesto (A3), en Dai y Meyn \cite{DaiSean} y Meyn y Down
\cite{MeynDown} hacen ver que este se puede sustituir por

\begin{itemize}
\item[A3')] Para el Proceso de Markov $X$, cada subconjunto
compacto de $X$ es un conjunto peque\~no.
\end{itemize}

Este supuesto es importante pues es un requisito para deducir la ergodicidad de la red.

%_________________________________________________________________________
\subsection{Construcci\'on de un Modelo de Flujo L\'imite}
%_________________________________________________________________________

Consideremos un caso m\'as simple para poner en contexto lo
anterior: para un sistema de visitas c\'iclicas se tiene que el
estado al tiempo $t$ es
\begin{equation}
X\left(t\right)=\left(Q\left(t\right),U\left(t\right),V\left(t\right)\right),
\end{equation}

donde $Q\left(t\right)$ es el n\'umero de usuarios formados en
cada estaci\'on. $U\left(t\right)$ es el tiempo restante antes de
que la siguiente clase $k$ de usuarios lleguen desde fuera del
sistema, $V\left(t\right)$ es el tiempo restante de servicio para
la clase $k$ de usuarios que est\'an siendo atendidos. Tanto
$U\left(t\right)$ como $V\left(t\right)$ se puede asumir que son
continuas por la derecha.

Sea
$x=\left(Q\left(0\right),U\left(0\right),V\left(0\right)\right)=\left(q,a,b\right)$,
el estado inicial de la red bajo una disciplina espec\'ifica para
la cola. Para $l\in\mathcal{E}$, donde $\mathcal{E}$ es el conjunto de clases de arribos externos, y $k=1,\ldots,K$ se define\\
\begin{eqnarray*}
E_{l}^{x}\left(t\right)&=&max\left\{r:U_{l}\left(0\right)+\xi_{l}\left(1\right)+\cdots+\xi_{l}\left(r-1\right)\leq
t\right\}\textrm{   }t\geq0,\\
S_{k}^{x}\left(t\right)&=&max\left\{r:V_{k}\left(0\right)+\eta_{k}\left(1\right)+\cdots+\eta_{k}\left(r-1\right)\leq
t\right\}\textrm{   }t\geq0.
\end{eqnarray*}

Para cada $k$ y cada $n$ se define

\begin{eqnarray*}\label{Eq.phi}
\Phi^{k}\left(n\right):=\sum_{i=1}^{n}\phi^{k}\left(i\right).
\end{eqnarray*}

donde $\phi^{k}\left(n\right)$ se define como el vector de ruta
para el $n$-\'esimo usuario de la clase $k$ que termina en la
estaci\'on $s\left(k\right)$, la $s$-\'eima componente de
$\phi^{k}\left(n\right)$ es uno si estos usuarios se convierten en
usuarios de la clase $l$ y cero en otro caso, por lo tanto
$\phi^{k}\left(n\right)$ es un vector {\em Bernoulli} de
dimensi\'on $K$ con par\'ametro $P_{k}^{'}$, donde $P_{k}$ denota
el $k$-\'esimo rengl\'on de $P=\left(P_{kl}\right)$.

Se asume que cada para cada $k$ la sucesi\'on $\phi^{k}\left(n\right)=\left\{\phi^{k}\left(n\right),n\geq1\right\}$
es independiente e id\'enticamente distribuida y que las
$\phi^{1}\left(n\right),\ldots,\phi^{K}\left(n\right)$ son
mutuamente independientes, adem\'as de independientes de los
procesos de arribo y de servicio.\\

\begin{Lema}[Lema 4.2, Dai\cite{Dai}]\label{Lema4.2}
Sea $\left\{x_{n}\right\}\subset \mathbf{X}$ con
$|x_{n}|\rightarrow\infty$, conforme $n\rightarrow\infty$. Suponga
que
\[lim_{n\rightarrow\infty}\frac{1}{|x_{n}|}U\left(0\right)=\overline{U}\]
y
\[lim_{n\rightarrow\infty}\frac{1}{|x_{n}|}V\left(0\right)=\overline{V}.\]

Entonces, conforme $n\rightarrow\infty$, casi seguramente

\begin{equation}\label{E1.4.2}
\frac{1}{|x_{n}|}\Phi^{k}\left(\left[|x_{n}|t\right]\right)\rightarrow
P_{k}^{'}t\textrm{, u.o.c.,}
\end{equation}

\begin{equation}\label{E1.4.3}
\frac{1}{|x_{n}|}E^{x_{n}}_{k}\left(|x_{n}|t\right)\rightarrow
\alpha_{k}\left(t-\overline{U}_{k}\right)^{+}\textrm{, u.o.c.,}
\end{equation}

\begin{equation}\label{E1.4.4}
\frac{1}{|x_{n}|}S^{x_{n}}_{k}\left(|x_{n}|t\right)\rightarrow
\mu_{k}\left(t-\overline{V}_{k}\right)^{+}\textrm{, u.o.c.,}
\end{equation}

donde $\left[t\right]$ es la parte entera de $t$ y
$\mu_{k}=1/m_{k}=1/\esp\left[\eta_{k}\left(1\right)\right]$.
\end{Lema}

\begin{Lema}[Lema 4.3, Dai\cite{Dai}]\label{Lema.4.3}
Sea $\left\{x_{n}\right\}\subset \mathbf{X}$ con
$|x_{n}|\rightarrow\infty$, conforme $n\rightarrow\infty$. Suponga
que
\[lim_{n\rightarrow\infty}\frac{1}{|x_{n}|}U\left(0\right)=\overline{U}_{k}\]
y
\[lim_{n\rightarrow\infty}\frac{1}{|x_{n}|}V\left(0\right)=\overline{V}_{k}.\]
\begin{itemize}
\item[a)] Conforme $n\rightarrow\infty$ casi seguramente,
\[lim_{n\rightarrow\infty}\frac{1}{|x_{n}|}U^{x_{n}}_{k}\left(|x_{n}|t\right)=\left(\overline{U}_{k}-t\right)^{+}\textrm{, u.o.c.}\]
y
\[lim_{n\rightarrow\infty}\frac{1}{|x_{n}|}V^{x_{n}}_{k}\left(|x_{n}|t\right)=\left(\overline{V}_{k}-t\right)^{+}.\]

\item[b)] Para cada $t\geq0$ fijo,
\[\left\{\frac{1}{|x_{n}|}U^{x_{n}}_{k}\left(|x_{n}|t\right),|x_{n}|\geq1\right\}\]
y
\[\left\{\frac{1}{|x_{n}|}V^{x_{n}}_{k}\left(|x_{n}|t\right),|x_{n}|\geq1\right\}\]
\end{itemize}
son uniformemente convergentes.
\end{Lema}

$S_{l}^{x}\left(t\right)$ es el n\'umero total de servicios
completados de la clase $l$, si la clase $l$ est\'a dando $t$
unidades de tiempo de servicio. Sea $T_{l}^{x}\left(x\right)$ el
monto acumulado del tiempo de servicio que el servidor
$s\left(l\right)$ gasta en los usuarios de la clase $l$ al tiempo
$t$. Entonces $S_{l}^{x}\left(T_{l}^{x}\left(t\right)\right)$ es
el n\'umero total de servicios completados para la clase $l$ al
tiempo $t$. Una fracci\'on de estos usuarios,
$\Phi_{l}^{x}\left(S_{l}^{x}\left(T_{l}^{x}\left(t\right)\right)\right)$,
se convierte en usuarios de la clase $k$.\\

Entonces, dado lo anterior, se tiene la siguiente representaci\'on
para el proceso de la longitud de la cola:\\

\begin{equation}
Q_{k}^{x}\left(t\right)=_{k}^{x}\left(0\right)+E_{k}^{x}\left(t\right)+\sum_{l=1}^{K}\Phi_{k}^{l}\left(S_{l}^{x}\left(T_{l}^{x}\left(t\right)\right)\right)-S_{k}^{x}\left(T_{k}^{x}\left(t\right)\right)
\end{equation}
para $k=1,\ldots,K$. Para $i=1,\ldots,d$, sea
\[I_{i}^{x}\left(t\right)=t-\sum_{j\in C_{i}}T_{k}^{x}\left(t\right).\]

Entonces $I_{i}^{x}\left(t\right)$ es el monto acumulado del
tiempo que el servidor $i$ ha estado desocupado al tiempo $t$. Se
est\'a asumiendo que las disciplinas satisfacen la ley de
conservaci\'on del trabajo, es decir, el servidor $i$ est\'a en
pausa solamente cuando no hay usuarios en la estaci\'on $i$.
Entonces, se tiene que

\begin{equation}
\int_{0}^{\infty}\left(\sum_{k\in
C_{i}}Q_{k}^{x}\left(t\right)\right)dI_{i}^{x}\left(t\right)=0,
\end{equation}
para $i=1,\ldots,d$.\\

Hacer
\[T^{x}\left(t\right)=\left(T_{1}^{x}\left(t\right),\ldots,T_{K}^{x}\left(t\right)\right)^{'},\]
\[I^{x}\left(t\right)=\left(I_{1}^{x}\left(t\right),\ldots,I_{K}^{x}\left(t\right)\right)^{'}\]
y
\[S^{x}\left(T^{x}\left(t\right)\right)=\left(S_{1}^{x}\left(T_{1}^{x}\left(t\right)\right),\ldots,S_{K}^{x}\left(T_{K}^{x}\left(t\right)\right)\right)^{'}.\]

Para una disciplina que cumple con la ley de conservaci\'on del
trabajo, en forma vectorial, se tiene el siguiente conjunto de
ecuaciones

\begin{equation}\label{Eq.MF.1.3}
Q^{x}\left(t\right)=Q^{x}\left(0\right)+E^{x}\left(t\right)+\sum_{l=1}^{K}\Phi^{l}\left(S_{l}^{x}\left(T_{l}^{x}\left(t\right)\right)\right)-S^{x}\left(T^{x}\left(t\right)\right),\\
\end{equation}

\begin{equation}\label{Eq.MF.2.3}
Q^{x}\left(t\right)\geq0,\\
\end{equation}

\begin{equation}\label{Eq.MF.3.3}
T^{x}\left(0\right)=0,\textrm{ y }\overline{T}^{x}\left(t\right)\textrm{ es no decreciente},\\
\end{equation}

\begin{equation}\label{Eq.MF.4.3}
I^{x}\left(t\right)=et-CT^{x}\left(t\right)\textrm{ es no
decreciente}\\
\end{equation}

\begin{equation}\label{Eq.MF.5.3}
\int_{0}^{\infty}\left(CQ^{x}\left(t\right)\right)dI_{i}^{x}\left(t\right)=0,\\
\end{equation}

\begin{equation}\label{Eq.MF.6.3}
\textrm{Condiciones adicionales en
}\left(\overline{Q}^{x}\left(\cdot\right),\overline{T}^{x}\left(\cdot\right)\right)\textrm{
espec\'ificas de la disciplina de la cola,}
\end{equation}

donde $e$ es un vector de unos de dimensi\'on $d$, $C$ es la
matriz definida por
\[C_{ik}=\left\{\begin{array}{cc}
1,& S\left(k\right)=i,\\
0,& \textrm{ en otro caso}.\\
\end{array}\right.
\]
Es necesario enunciar el siguiente Teorema que se utilizar\'a para
el Teorema \ref{Tma.4.2.Dai}:
\begin{Teo}[Teorema 4.1, Dai \cite{Dai}]
Considere una disciplina que cumpla la ley de conservaci\'on del
trabajo, para casi todas las trayectorias muestrales $\omega$ y
cualquier sucesi\'on de estados iniciales
$\left\{x_{n}\right\}\subset \mathbf{X}$, con
$|x_{n}|\rightarrow\infty$, existe una subsucesi\'on
$\left\{x_{n_{j}}\right\}$ con $|x_{n_{j}}|\rightarrow\infty$ tal
que
\begin{equation}\label{Eq.4.15}
\frac{1}{|x_{n_{j}}|}\left(Q^{x_{n_{j}}}\left(0\right),U^{x_{n_{j}}}\left(0\right),V^{x_{n_{j}}}\left(0\right)\right)\rightarrow\left(\overline{Q}\left(0\right),\overline{U},\overline{V}\right),
\end{equation}

\begin{equation}\label{Eq.4.16}
\frac{1}{|x_{n_{j}}|}\left(Q^{x_{n_{j}}}\left(|x_{n_{j}}|t\right),T^{x_{n_{j}}}\left(|x_{n_{j}}|t\right)\right)\rightarrow\left(\overline{Q}\left(t\right),\overline{T}\left(t\right)\right)\textrm{
u.o.c.}
\end{equation}

Adem\'as,
$\left(\overline{Q}\left(t\right),\overline{T}\left(t\right)\right)$
satisface las siguientes ecuaciones:
\begin{equation}\label{Eq.MF.1.3a}
\overline{Q}\left(t\right)=Q\left(0\right)+\left(\alpha
t-\overline{U}\right)^{+}-\left(I-P\right)^{'}M^{-1}\left(\overline{T}\left(t\right)-\overline{V}\right)^{+},
\end{equation}

\begin{equation}\label{Eq.MF.2.3a}
\overline{Q}\left(t\right)\geq0,\\
\end{equation}

\begin{equation}\label{Eq.MF.3.3a}
\overline{T}\left(t\right)\textrm{ es no decreciente y comienza en cero},\\
\end{equation}

\begin{equation}\label{Eq.MF.4.3a}
\overline{I}\left(t\right)=et-C\overline{T}\left(t\right)\textrm{
es no decreciente,}\\
\end{equation}

\begin{equation}\label{Eq.MF.5.3a}
\int_{0}^{\infty}\left(C\overline{Q}\left(t\right)\right)d\overline{I}\left(t\right)=0,\\
\end{equation}

\begin{equation}\label{Eq.MF.6.3a}
\textrm{Condiciones adicionales en
}\left(\overline{Q}\left(\cdot\right),\overline{T}\left(\cdot\right)\right)\textrm{
especficas de la disciplina de la cola,}
\end{equation}
\end{Teo}

\begin{Def}[Definici\'on 4.1, , Dai \cite{Dai}]
Sea una disciplina de servicio espec\'ifica. Cualquier l\'imite
$\left(\overline{Q}\left(\cdot\right),\overline{T}\left(\cdot\right)\right)$
en \ref{Eq.4.16} es un {\em flujo l\'imite} de la disciplina.
Cualquier soluci\'on (\ref{Eq.MF.1.3a})-(\ref{Eq.MF.6.3a}) es
llamado flujo soluci\'on de la disciplina. Se dice que el modelo de flujo l\'imite, modelo de flujo, de la disciplina de la cola es estable si existe una constante
$\delta>0$ que depende de $\mu,\alpha$ y $P$ solamente, tal que
cualquier flujo l\'imite con
$|\overline{Q}\left(0\right)|+|\overline{U}|+|\overline{V}|=1$, se
tiene que $\overline{Q}\left(\cdot+\delta\right)\equiv0$.
\end{Def}

\begin{Teo}[Teorema 4.2, Dai\cite{Dai}]\label{Tma.4.2.Dai}
Sea una disciplina fija para la cola, suponga que se cumplen las
condiciones (1.2)-(1.5). Si el modelo de flujo l\'imite de la
disciplina de la cola es estable, entonces la cadena de Markov $X$
que describe la din\'amica de la red bajo la disciplina es Harris
recurrente positiva.
\end{Teo}

Ahora se procede a escalar el espacio y el tiempo para reducir la
aparente fluctuaci\'on del modelo. Consid\'erese el proceso
\begin{equation}\label{Eq.3.7}
\overline{Q}^{x}\left(t\right)=\frac{1}{|x|}Q^{x}\left(|x|t\right)
\end{equation}
A este proceso se le conoce como el fluido escalado, y cualquier l\'imite $\overline{Q}^{x}\left(t\right)$ es llamado flujo l\'imite del proceso de longitud de la cola. Haciendo $|q|\rightarrow\infty$ mientras se mantiene el resto de las componentes fijas, cualquier punto l\'imite del proceso de longitud de la cola normalizado $\overline{Q}^{x}$ es soluci\'on del siguiente modelo de flujo.

\begin{Def}[Definici\'on 3.1, Dai y Meyn \cite{DaiSean}]
Un flujo l\'imite (retrasado) para una red bajo una disciplina de
servicio espec\'ifica se define como cualquier soluci\'on
 $\left(\overline{Q}\left(\cdot\right),\overline{T}\left(\cdot\right)\right)$ de las siguientes ecuaciones, donde
$\overline{Q}\left(t\right)=\left(\overline{Q}_{1}\left(t\right),\ldots,\overline{Q}_{K}\left(t\right)\right)^{'}$
y
$\overline{T}\left(t\right)=\left(\overline{T}_{1}\left(t\right),\ldots,\overline{T}_{K}\left(t\right)\right)^{'}$
\begin{equation}\label{Eq.3.8}
\overline{Q}_{k}\left(t\right)=\overline{Q}_{k}\left(0\right)+\alpha_{k}t-\mu_{k}\overline{T}_{k}\left(t\right)+\sum_{l=1}^{k}P_{lk}\mu_{l}\overline{T}_{l}\left(t\right)\\
\end{equation}
\begin{equation}\label{Eq.3.9}
\overline{Q}_{k}\left(t\right)\geq0\textrm{ para }k=1,2,\ldots,K,\\
\end{equation}
\begin{equation}\label{Eq.3.10}
\overline{T}_{k}\left(0\right)=0,\textrm{ y }\overline{T}_{k}\left(\cdot\right)\textrm{ es no decreciente},\\
\end{equation}
\begin{equation}\label{Eq.3.11}
\overline{I}_{i}\left(t\right)=t-\sum_{k\in C_{i}}\overline{T}_{k}\left(t\right)\textrm{ es no decreciente}\\
\end{equation}
\begin{equation}\label{Eq.3.12}
\overline{I}_{i}\left(\cdot\right)\textrm{ se incrementa al tiempo }t\textrm{ cuando }\sum_{k\in C_{i}}Q_{k}^{x}\left(t\right)dI_{i}^{x}\left(t\right)=0\\
\end{equation}
\begin{equation}\label{Eq.3.13}
\textrm{condiciones adicionales sobre
}\left(Q^{x}\left(\cdot\right),T^{x}\left(\cdot\right)\right)\textrm{
referentes a la disciplina de servicio}
\end{equation}
\end{Def}

Al conjunto de ecuaciones dadas en \ref{Eq.3.8}-\ref{Eq.3.13} se
le llama {\em Modelo de flujo} y al conjunto de todas las
soluciones del modelo de flujo
$\left(\overline{Q}\left(\cdot\right),\overline{T}
\left(\cdot\right)\right)$ se le denotar\'a por $\mathcal{Q}$.

Si se hace $|x|\rightarrow\infty$ sin restringir ninguna de las
componentes, tambi\'en se obtienen un modelo de flujo, pero en
este caso el residual de los procesos de arribo y servicio
introducen un retraso:

\begin{Def}[Definici\'on 3.2, Dai y Meyn \cite{DaiSean}]
El modelo de flujo retrasado de una disciplina de servicio en una
red con retraso
$\left(\overline{A}\left(0\right),\overline{B}\left(0\right)\right)\in\rea_{+}^{K+|A|}$
se define como el conjunto de ecuaciones dadas en
\ref{Eq.3.8}-\ref{Eq.3.13}, junto con la condici\'on:
\begin{equation}\label{CondAd.FluidModel}
\overline{Q}\left(t\right)=\overline{Q}\left(0\right)+\left(\alpha
t-\overline{A}\left(0\right)\right)^{+}-\left(I-P^{'}\right)M\left(\overline{T}\left(t\right)-\overline{B}\left(0\right)\right)^{+}
\end{equation}
\end{Def}

\begin{Def}[Definici\'on 3.3, Dai y Meyn \cite{DaiSean}]
El modelo de flujo es estable si existe un tiempo fijo $t_{0}$ tal
que $\overline{Q}\left(t\right)=0$, con $t\geq t_{0}$, para
cualquier $\overline{Q}\left(\cdot\right)\in\mathcal{Q}$ que
cumple con $|\overline{Q}\left(0\right)|=1$.
\end{Def}

El siguiente resultado se encuentra en Chen \cite{Chen}.
\begin{Lemma}[Lema 3.1, Dai y Meyn \cite{DaiSean}]
Si el modelo de flujo definido por \ref{Eq.3.8}-\ref{Eq.3.13} es
estable, entonces el modelo de flujo retrasado es tambi\'en
estable, es decir, existe $t_{0}>0$ tal que
$\overline{Q}\left(t\right)=0$ para cualquier $t\geq t_{0}$, para
cualquier soluci\'on del modelo de flujo retrasado cuya
condici\'on inicial $\overline{x}$ satisface que
$|\overline{x}|=|\overline{Q}\left(0\right)|+|\overline{A}\left(0\right)|+|\overline{B}\left(0\right)|\leq1$.
\end{Lemma}

%_________________________________________________________________________
\subsection{Modelo de Visitas C\'iclicas con un Servidor: Estabilidad}
%_________________________________________________________________________

%_________________________________________________________________________
\subsection{Teorema 2.1}
%_________________________________________________________________________



El resultado principal de Down \cite{Down} que relaciona la estabilidad del modelo de flujo con la estabilidad del sistema original

\begin{Teo}[Teorema 2.1, Down \cite{Down}]\label{Tma.2.1.Down}
Suponga que el modelo de flujo es estable, y que se cumplen los supuestos (A1) y (A2), entonces
\begin{itemize}
\item[i)] Para alguna constante $\kappa_{p}$, y para cada
condici\'on inicial $x\in X$
\begin{equation}\label{Estability.Eq1}
lim_{t\rightarrow\infty}\sup\frac{1}{t}\int_{0}^{t}\esp_{x}\left[|Q\left(s\right)|^{p}\right]ds\leq\kappa_{p},
\end{equation}
donde $p$ es el entero dado en (A2). Si adem\'as se cumple
la condici\'on (A3), entonces para cada condici\'on inicial:

\item[ii)] Los momentos transitorios convergen a su estado estacionario:
 \begin{equation}\label{Estability.Eq2}
lim_{t\rightarrow\infty}\esp_{x}\left[Q_{k}\left(t\right)^{r}\right]=\esp_{\pi}\left[Q_{k}\left(0\right)^{r}\right]\leq\kappa_{r},
\end{equation}
para $r=1,2,\ldots,p$ y $k=1,2,\ldots,K$. Donde $\pi$ es la
probabilidad invariante para $\mathbf{X}$.

\item[iii)]  El primer momento converge con raz\'on $t^{p-1}$:
\begin{equation}\label{Estability.Eq3}
lim_{t\rightarrow\infty}t^{p-1}|\esp_{x}\left[Q_{k}\left(t\right)\right]-\esp_{\pi}\left[Q\left(0\right)\right]=0.
\end{equation}

\item[iv)] La {\em Ley Fuerte de los grandes n\'umeros} se cumple:
\begin{equation}\label{Estability.Eq4}
lim_{t\rightarrow\infty}\frac{1}{t}\int_{0}^{t}Q_{k}^{r}\left(s\right)ds=\esp_{\pi}\left[Q_{k}\left(0\right)^{r}\right],\textrm{
}\prob_{x}\textrm{-c.s.}
\end{equation}
para $r=1,2,\ldots,p$ y $k=1,2,\ldots,K$.
\end{itemize}
\end{Teo}


\begin{Prop}[Proposici\'on 5.1, Dai y Meyn \cite{DaiSean}]\label{Prop.5.1.DaiSean}
Suponga que los supuestos A1) y A2) son ciertos y que el modelo de flujo es estable. Entonces existe $t_{0}>0$ tal que
\begin{equation}
lim_{|x|\rightarrow\infty}\frac{1}{|x|^{p+1}}\esp_{x}\left[|X\left(t_{0}|x|\right)|^{p+1}\right]=0
\end{equation}
\end{Prop}

\begin{Lemma}[Lema 5.2, Dai y Meyn \cite{DaiSean}]\label{Lema.5.2.DaiSean}
 Sea $\left\{\zeta\left(k\right):k\in \mathbb{z}\right\}$ una sucesi\'on independiente e id\'enticamente distribuida que toma valores en $\left(0,\infty\right)$,
y sea
$E\left(t\right)=max\left(n\geq1:\zeta\left(1\right)+\cdots+\zeta\left(n-1\right)\leq
t\right)$. Si $\esp\left[\zeta\left(1\right)\right]<\infty$,
entonces para cualquier entero $r\geq1$
\begin{equation}
 lim_{t\rightarrow\infty}\esp\left[\left(\frac{E\left(t\right)}{t}\right)^{r}\right]=\left(\frac{1}{\esp\left[\zeta_{1}\right]}\right)^{r}.
\end{equation}
Luego, bajo estas condiciones:
\begin{itemize}
 \item[a)] para cualquier $\delta>0$, $\sup_{t\geq\delta}\esp\left[\left(\frac{E\left(t\right)}{t}\right)^{r}\right]<\infty$
\item[b)] las variables aleatorias
$\left\{\left(\frac{E\left(t\right)}{t}\right)^{r}:t\geq1\right\}$
son uniformemente integrables.
\end{itemize}
\end{Lemma}

\begin{Teo}[Teorema 5.5, Dai y Meyn \cite{DaiSean}]\label{Tma.5.5.DaiSean}
Suponga que los supuestos A1) y A2) se cumplen y que el modelo de
flujo es estable. Entonces existe una constante $\kappa_{p}$ tal
que
\begin{equation}
\frac{1}{t}\int_{0}^{t}\esp_{x}\left[|Q\left(s\right)|^{p}\right]ds\leq\kappa_{p}\left\{\frac{1}{t}|x|^{p+1}+1\right\}
\end{equation}
para $t>0$ y $x\in X$. En particular, para cada condici\'on
inicial
\begin{eqnarray*}
\limsup_{t\rightarrow\infty}\frac{1}{t}\int_{0}^{t}\esp_{x}\left[|Q\left(s\right)|^{p}\right]ds\leq\kappa_{p}.
\end{eqnarray*}
\end{Teo}

\begin{Teo}[Teorema 6.2, Dai y Meyn \cite{DaiSean}]\label{Tma.6.2.DaiSean}
Suponga que se cumplen los supuestos A1), A2) y A3) y que el
modelo de flujo es estable. Entonces se tiene que
\begin{equation}
\left\|P^{t}\left(x,\cdot\right)-\pi\left(\cdot\right)\right\|_{f_{p}}\textrm{,
}t\rightarrow\infty,x\in X.
\end{equation}
En particular para cada condici\'on inicial
\begin{eqnarray*}
\lim_{t\rightarrow\infty}\esp_{x}\left[|Q\left(t\right)|^{p}\right]=\esp_{\pi}\left[|Q\left(0\right)|^{p}\right]\leq\kappa_{r}
\end{eqnarray*}
\end{Teo}
\begin{Teo}[Teorema 6.3, Dai y Meyn \cite{DaiSean}]\label{Tma.6.3.DaiSean}
Suponga que se cumplen los supuestos A1), A2) y A3) y que el
modelo de flujo es estable. Entonces con
$f\left(x\right)=f_{1}\left(x\right)$ se tiene
\begin{equation}
\lim_{t\rightarrow\infty}t^{p-1}\left\|P^{t}\left(x,\cdot\right)-\pi\left(\cdot\right)\right\|_{f}=0.
\end{equation}
En particular para cada condici\'on inicial
\begin{eqnarray*}
\lim_{t\rightarrow\infty}t^{p-1}|\esp_{x}\left[Q\left(t\right)\right]-\esp_{\pi}\left[Q\left(0\right)\right]|=0.
\end{eqnarray*}
\end{Teo}

\begin{Teo}[Teorema 6.4, Dai y Meyn \cite{DaiSean}]\label{Tma.6.4.DaiSean}
Suponga que se cumplen los supuestos A1), A2) y A3) y que el
modelo de flujo es estable. Sea $\nu$ cualquier distribuci\'on de
probabilidad en $\left(X,\mathcal{B}_{X}\right)$, y $\pi$ la
distribuci\'on estacionaria de $X$.
\begin{itemize}
\item[i)] Para cualquier $f:X\leftarrow\rea_{+}$
\begin{equation}
\lim_{t\rightarrow\infty}\frac{1}{t}\int_{o}^{t}f\left(X\left(s\right)\right)ds=\pi\left(f\right):=\int
f\left(x\right)\pi\left(dx\right)
\end{equation}
$\prob$-c.s.

\item[ii)] Para cualquier $f:X\leftarrow\rea_{+}$ con
$\pi\left(|f|\right)<\infty$, la ecuaci\'on anterior se cumple.
\end{itemize}
\end{Teo}

%_________________________________________________________________________
\subsection{Teorema 2.2}
%_________________________________________________________________________

\begin{Teo}[Teorema 2.2, Down \cite{Down}]\label{Tma2.2.Down}
Suponga que el fluido modelo es inestable en el sentido de que
para alguna $\epsilon_{0},c_{0}\geq0$,
\begin{equation}\label{Eq.Inestability}
|Q\left(T\right)|\geq\epsilon_{0}T-c_{0}\textrm{,   }T\geq0,
\end{equation}
para cualquier condici\'on inicial $Q\left(0\right)$, con
$|Q\left(0\right)|=1$. Entonces para cualquier $0<q\leq1$, existe
$B<0$ tal que para cualquier $|x|\geq B$,
\begin{equation}
\prob_{x}\left\{\mathbb{X}\rightarrow\infty\right\}\geq q.
\end{equation}
\end{Teo}

%_________________________________________________________________________
\subsection{Teorema 2.3}
%_________________________________________________________________________
\begin{Teo}[Teorema 2.3, Down \cite{Down}]\label{Tma2.3.Down}
Considere el siguiente valor:
\begin{equation}\label{Eq.Rho.1serv}
\rho=\sum_{k=1}^{K}\rho_{k}+max_{1\leq j\leq K}\left(\frac{\lambda_{j}}{\sum_{s=1}^{S}p_{js}\overline{N}_{s}}\right)\delta^{*}
\end{equation}
\begin{itemize}
\item[i)] Si $\rho<1$ entonces la red es estable, es decir, se cumple el teorema \ref{Tma.2.1.Down}.

\item[ii)] Si $\rho<1$ entonces la red es inestable, es decir, se cumple el teorema \ref{Tma2.2.Down}
\end{itemize}
\end{Teo}
\newpage
%_____________________________________________________________________
\subsection{Definiciones  B\'asicas}
%_____________________________________________________________________
\begin{Def}
Sea $X$ un conjunto y $\mathcal{F}$ una $\sigma$-\'algebra de
subconjuntos de $X$, la pareja $\left(X,\mathcal{F}\right)$ es
llamado espacio medible. Un subconjunto $A$ de $X$ es llamado
medible, o medible con respecto a $\mathcal{F}$, si
$A\in\mathcal{F}$.
\end{Def}

\begin{Def}
Sea $\left(X,\mathcal{F},\mu\right)$ espacio de medida. Se dice
que la medida $\mu$ es $\sigma$-finita si se puede escribir
$X=\bigcup_{n\geq1}X_{n}$ con $X_{n}\in\mathcal{F}$ y
$\mu\left(X_{n}\right)<\infty$.
\end{Def}

\begin{Def}\label{Cto.Borel}
Sea $X$ el conjunto de los \'umeros reales $\rea$. El \'algebra de
Borel es la $\sigma$-\'algebra $B$ generada por los intervalos
abiertos $\left(a,b\right)\in\rea$. Cualquier conjunto en $B$ es
llamado {\em Conjunto de Borel}.
\end{Def}

\begin{Def}\label{Funcion.Medible}
Una funci\'on $f:X\rightarrow\rea$, es medible si para cualquier
n\'umero real $\alpha$ el conjunto
\[\left\{x\in X:f\left(x\right)>\alpha\right\}\]
pertenece a $X$. Equivalentemente, se dice que $f$ es medible si
\[f^{-1}\left(\left(\alpha,\infty\right)\right)=\left\{x\in X:f\left(x\right)>\alpha\right\}\in\mathcal{F}.\]
\end{Def}


\begin{Def}\label{Def.Cilindros}
Sean $\left(\Omega_{i},\mathcal{F}_{i}\right)$, $i=1,2,\ldots,$
espacios medibles y $\Omega=\prod_{i=1}^{\infty}\Omega_{i}$ el
conjunto de todas las sucesiones
$\left(\omega_{1},\omega_{2},\ldots,\right)$ tales que
$\omega_{i}\in\Omega_{i}$, $i=1,2,\ldots,$. Si
$B^{n}\subset\prod_{i=1}^{\infty}\Omega_{i}$, definimos
$B_{n}=\left\{\omega\in\Omega:\left(\omega_{1},\omega_{2},\ldots,\omega_{n}\right)\in
B^{n}\right\}$. Al conjunto $B_{n}$ se le llama {\em cilindro} con
base $B^{n}$, el cilindro es llamado medible si
$B^{n}\in\prod_{i=1}^{\infty}\mathcal{F}_{i}$.
\end{Def}


\begin{Def}\label{Def.Proc.Adaptado}[TSP, Ash \cite{RBA}]
Sea $X\left(t\right),t\geq0$ proceso estoc\'astico, el proceso es
adaptado a la familia de $\sigma$-\'algebras $\mathcal{F}_{t}$,
para $t\geq0$, si para $s<t$ implica que
$\mathcal{F}_{s}\subset\mathcal{F}_{t}$, y $X\left(t\right)$ es
$\mathcal{F}_{t}$-medible para cada $t$. Si no se especifica
$\mathcal{F}_{t}$ entonces se toma $\mathcal{F}_{t}$ como
$\mathcal{F}\left(X\left(s\right),s\leq t\right)$, la m\'as
peque\~na $\sigma$-\'algebra de subconjuntos de $\Omega$ que hace
que cada $X\left(s\right)$, con $s\leq t$ sea Borel medible.
\end{Def}


\begin{Def}\label{Def.Tiempo.Paro}[TSP, Ash \cite{RBA}]
Sea $\left\{\mathcal{F}\left(t\right),t\geq0\right\}$ familia
creciente de sub $\sigma$-\'algebras. es decir,
$\mathcal{F}\left(s\right)\subset\mathcal{F}\left(t\right)$ para
$s\leq t$. Un tiempo de paro para $\mathcal{F}\left(t\right)$ es
una funci\'on $T:\Omega\rightarrow\left[0,\infty\right]$ tal que
$\left\{T\leq t\right\}\in\mathcal{F}\left(t\right)$ para cada
$t\geq0$. Un tiempo de paro para el proceso estoc\'astico
$X\left(t\right),t\geq0$ es un tiempo de paro para las
$\sigma$-\'algebras
$\mathcal{F}\left(t\right)=\mathcal{F}\left(X\left(s\right)\right)$.
\end{Def}

\begin{Def}
Sea $X\left(t\right),t\geq0$ proceso estoc\'astico, con
$\left(S,\chi\right)$ espacio de estados. Se dice que el proceso
es adaptado a $\left\{\mathcal{F}\left(t\right)\right\}$, es
decir, si para cualquier $s,t\in I$, $I$ conjunto de \'indices,
$s<t$, se tiene que
$\mathcal{F}\left(s\right)\subset\mathcal{F}\left(t\right)$ y
$X\left(t\right)$ es $\mathcal{F}\left(t\right)$-medible,
\end{Def}

\begin{Def}
Sea $X\left(t\right),t\geq0$ proceso estoc\'astico, se dice que es
un Proceso de Markov relativo a $\mathcal{F}\left(t\right)$ o que
$\left\{X\left(t\right),\mathcal{F}\left(t\right)\right\}$ es de
Markov si y s\'olo si para cualquier conjunto $B\in\chi$,  y
$s,t\in I$, $s<t$ se cumple que
\begin{equation}\label{Prop.Markov}
P\left\{X\left(t\right)\in
B|\mathcal{F}\left(s\right)\right\}=P\left\{X\left(t\right)\in
B|X\left(s\right)\right\}.
\end{equation}
\end{Def}
\begin{Note}
Si se dice que $\left\{X\left(t\right)\right\}$ es un Proceso de
Markov sin mencionar $\mathcal{F}\left(t\right)$, se asumir\'a que
\begin{eqnarray*}
\mathcal{F}\left(t\right)=\mathcal{F}_{0}\left(t\right)=\mathcal{F}\left(X\left(r\right),r\leq
t\right),
\end{eqnarray*}
entonces la ecuaci\'on (\ref{Prop.Markov}) se puede escribir como
\begin{equation}
P\left\{X\left(t\right)\in B|X\left(r\right),r\leq s\right\} =
P\left\{X\left(t\right)\in B|X\left(s\right)\right\}
\end{equation}
\end{Note}

\begin{Teo}
Sea $\left(X_{n},\mathcal{F}_{n},n=0,1,\ldots,\right\}$ Proceso de
Markov con espacio de estados $\left(S_{0},\chi_{0}\right)$
generado por una distribuici\'on inicial $P_{o}$ y probabilidad de
transici\'on $p_{mn}$, para $m,n=0,1,\ldots,$ $m<n$, que por
notaci\'on se escribir\'a como $p\left(m,n,x,B\right)\rightarrow
p_{mn}\left(x,B\right)$. Sea $S$ tiempo de paro relativo a la
$\sigma$-\'algebra $\mathcal{F}_{n}$. Sea $T$ funci\'on medible,
$T:\Omega\rightarrow\left\{0,1,\ldots,\right\}$. Sup\'ongase que
$T\geq S$, entonces $T$ es tiempo de paro. Si $B\in\chi_{0}$,
entonces
\begin{equation}\label{Prop.Fuerte.Markov}
P\left\{X\left(T\right)\in
B,T<\infty|\mathcal{F}\left(S\right)\right\} =
p\left(S,T,X\left(s\right),B\right)
\end{equation}
en $\left\{T<\infty\right\}$.
\end{Teo}

Propiedades importantes para el modelo de flujo retrasado:

\begin{Prop}
 Sea $\left(\overline{Q},\overline{T},\overline{T}^{0}\right)$ un flujo l\'imite de \ref{Equation.4.4} y suponga que cuando $x\rightarrow\infty$ a lo largo de
una subsucesi\'on
\[\left(\frac{1}{|x|}Q_{k}^{x}\left(0\right),\frac{1}{|x|}A_{k}^{x}\left(0\right),\frac{1}{|x|}B_{k}^{x}\left(0\right),\frac{1}{|x|}B_{k}^{x,0}\left(0\right)\right)\rightarrow\left(\overline{Q}_{k}\left(0\right),0,0,0\right)\]
para $k=1,\ldots,K$. EL flujo l\'imite tiene las siguientes
propiedades, donde las propiedades de la derivada se cumplen donde
la derivada exista:
\begin{itemize}
 \item[i)] Los vectores de tiempo ocupado $\overline{T}\left(t\right)$ y $\overline{T}^{0}\left(t\right)$ son crecientes y continuas con
$\overline{T}\left(0\right)=\overline{T}^{0}\left(0\right)=0$.
\item[ii)] Para todo $t\geq0$
\[\sum_{k=1}^{K}\left[\overline{T}_{k}\left(t\right)+\overline{T}_{k}^{0}\left(t\right)\right]=t\]
\item[iii)] Para todo $1\leq k\leq K$
\[\overline{Q}_{k}\left(t\right)=\overline{Q}_{k}\left(0\right)+\alpha_{k}t-\mu_{k}\overline{T}_{k}\left(t\right)\]
\item[iv)]  Para todo $1\leq k\leq K$
\[\dot{{\overline{T}}}_{k}\left(t\right)=\beta_{k}\] para $\overline{Q}_{k}\left(t\right)=0$.
\item[v)] Para todo $k,j$
\[\mu_{k}^{0}\overline{T}_{k}^{0}\left(t\right)=\mu_{j}^{0}\overline{T}_{j}^{0}\left(t\right)\]
\item[vi)]  Para todo $1\leq k\leq K$
\[\mu_{k}\dot{{\overline{T}}}_{k}\left(t\right)=l_{k}\mu_{k}^{0}\dot{{\overline{T}}}_{k}^{0}\left(t\right)\] para $\overline{Q}_{k}\left(t\right)>0$.
\end{itemize}
\end{Prop}

\begin{Lema}[Lema 3.1 \cite{Chen}]\label{Lema3.1}
Si el modelo de flujo es estable, definido por las ecuaciones
(3.8)-(3.13), entonces el modelo de flujo retrasado tambin es
estable.
\end{Lema}

\begin{Teo}[Teorema 5.2 \cite{Chen}]\label{Tma.5.2}
Si el modelo de flujo lineal correspondiente a la red de cola es
estable, entonces la red de colas es estable.
\end{Teo}

\begin{Teo}[Teorema 5.1 \cite{Chen}]\label{Tma.5.1.Chen}
La red de colas es estable si existe una constante $t_{0}$ que
depende de $\left(\alpha,\mu,T,U\right)$ y $V$ que satisfagan las
ecuaciones (5.1)-(5.5), $Z\left(t\right)=0$, para toda $t\geq
t_{0}$.
\end{Teo}



\begin{Lema}[Lema 5.2 \cite{Gut}]\label{Lema.5.2.Gut}
Sea $\left\{\xi\left(k\right):k\in\ent\right\}$ sucesin de
variables aleatorias i.i.d. con valores en
$\left(0,\infty\right)$, y sea $E\left(t\right)$ el proceso de
conteo
\[E\left(t\right)=max\left\{n\geq1:\xi\left(1\right)+\cdots+\xi\left(n-1\right)\leq t\right\}.\]
Si $E\left[\xi\left(1\right)\right]<\infty$, entonces para
cualquier entero $r\geq1$
\begin{equation}
lim_{t\rightarrow\infty}\esp\left[\left(\frac{E\left(t\right)}{t}\right)^{r}\right]=\left(\frac{1}{E\left[\xi_{1}\right]}\right)^{r}
\end{equation}
de aqu, bajo estas condiciones
\begin{itemize}
\item[a)] Para cualquier $t>0$,
$sup_{t\geq\delta}\esp\left[\left(\frac{E\left(t\right)}{t}\right)^{r}\right]$

\item[b)] Las variables aleatorias
$\left\{\left(\frac{E\left(t\right)}{t}\right)^{r}:t\geq1\right\}$
son uniformemente integrables.
\end{itemize}
\end{Lema}

\begin{Teo}[Teorema 5.1: Ley Fuerte para Procesos de Conteo
\cite{Gut}]\label{Tma.5.1.Gut} Sea
$0<\mu<\esp\left(X_{1}\right]\leq\infty$. entonces

\begin{itemize}
\item[a)] $\frac{N\left(t\right)}{t}\rightarrow\frac{1}{\mu}$
a.s., cuando $t\rightarrow\infty$.


\item[b)]$\esp\left[\frac{N\left(t\right)}{t}\right]^{r}\rightarrow\frac{1}{\mu^{r}}$,
cuando $t\rightarrow\infty$ para todo $r>0$..
\end{itemize}
\end{Teo}


\begin{Prop}[Proposicin 5.1 \cite{DaiSean}]\label{Prop.5.1}
Suponga que los supuestos (A1) y (A2) se cumplen, adems suponga
que el modelo de flujo es estable. Entonces existe $t_{0}>0$ tal
que
\begin{equation}\label{Eq.Prop.5.1}
lim_{|x|\rightarrow\infty}\frac{1}{|x|^{p+1}}\esp_{x}\left[|X\left(t_{0}|x|\right)|^{p+1}\right]=0.
\end{equation}

\end{Prop}


\begin{Prop}[Proposici\'on 5.3 \cite{DaiSean}]
Sea $X$ proceso de estados para la red de colas, y suponga que se
cumplen los supuestos (A1) y (A2), entonces para alguna constante
positiva $C_{p+1}<\infty$, $\delta>0$ y un conjunto compacto
$C\subset X$.

\begin{equation}\label{Eq.5.4}
\esp_{x}\left[\int_{0}^{\tau_{C}\left(\delta\right)}\left(1+|X\left(t\right)|^{p}\right)dt\right]\leq
C_{p+1}\left(1+|x|^{p+1}\right)
\end{equation}
\end{Prop}

\begin{Prop}[Proposici\'on 5.4 \cite{DaiSean}]
Sea $X$ un proceso de Markov Borel Derecho en $X$, sea
$f:X\leftarrow\rea_{+}$ y defina para alguna $\delta>0$, y un
conjunto cerrado $C\subset X$
\[V\left(x\right):=\esp_{x}\left[\int_{0}^{\tau_{C}\left(\delta\right)}f\left(X\left(t\right)\right)dt\right]\]
para $x\in X$. Si $V$ es finito en todas partes y uniformemente
acotada en $C$, entonces existe $k<\infty$ tal que
\begin{equation}\label{Eq.5.11}
\frac{1}{t}\esp_{x}\left[V\left(x\right)\right]+\frac{1}{t}\int_{0}^{t}\esp_{x}\left[f\left(X\left(s\right)\right)ds\right]\leq\frac{1}{t}V\left(x\right)+k,
\end{equation}
para $x\in X$ y $t>0$.
\end{Prop}


\begin{Teo}[Teorema 5.5 \cite{DaiSean}]
Suponga que se cumplen (A1) y (A2), adems suponga que el modelo
de flujo es estable. Entonces existe una constante $k_{p}<\infty$
tal que
\begin{equation}\label{Eq.5.13}
\frac{1}{t}\int_{0}^{t}\esp_{x}\left[|Q\left(s\right)|^{p}\right]ds\leq
k_{p}\left\{\frac{1}{t}|x|^{p+1}+1\right\}
\end{equation}
para $t\geq0$, $x\in X$. En particular para cada condicin inicial
\begin{equation}\label{Eq.5.14}
Limsup_{t\rightarrow\infty}\frac{1}{t}\int_{0}^{t}\esp_{x}\left[|Q\left(s\right)|^{p}\right]ds\leq
k_{p}
\end{equation}
\end{Teo}

\begin{Teo}[Teorema 6.2\cite{DaiSean}]\label{Tma.6.2}
Suponga que se cumplen los supuestos (A1)-(A3) y que el modelo de
flujo es estable, entonces se tiene que
\[\parallel P^{t}\left(c,\cdot\right)-\pi\left(\cdot\right)\parallel_{f_{p}}\rightarrow0\]
para $t\rightarrow\infty$ y $x\in X$. En particular para cada
condicin inicial
\[lim_{t\rightarrow\infty}\esp_{x}\left[\left|Q_{t}\right|^{p}\right]=\esp_{\pi}\left[\left|Q_{0}\right|^{p}\right]<\infty\]
\end{Teo}


\begin{Teo}[Teorema 6.3\cite{DaiSean}]\label{Tma.6.3}
Suponga que se cumplen los supuestos (A1)-(A3) y que el modelo de
flujo es estable, entonces con
$f\left(x\right)=f_{1}\left(x\right)$, se tiene que
\[lim_{t\rightarrow\infty}t^{(p-1)\left|P^{t}\left(c,\cdot\right)-\pi\left(\cdot\right)\right|_{f}=0},\]
para $x\in X$. En particular, para cada condicin inicial
\[lim_{t\rightarrow\infty}t^{(p-1)\left|\esp_{x}\left[Q_{t}\right]-\esp_{\pi}\left[Q_{0}\right]\right|=0}.\]
\end{Teo}



%_____________________________________________________________________________________
\subsection{Proceso de Estados Markoviano para el Sistema}
%_________________________________________________________________________


Sean $Q_{k}\left(t\right)$ el n\'umero de usuarios en la cola $k$,
$A_{k}\left(t\right)$ el tiempo residual de arribos a la cola $k$,
para cada servidor $m$, sea $H_{m}\left(t\right)$ par ordenado que
consiste en la cola que est\'a siendo atendida y la pol\'itica de
servicio que se est\'a utilizando. $B_{m}\left(t\right)$ los
tiempos de servicio residuales, $B_{m}^{0}\left(t\right)$ el
tiempo residual de traslado, $C_{m}\left(t\right)$ el n\'umero de
usuarios atendidos durante la visita del servidor a la cola dada
en $H_{m}\left(t\right)$.

El proceso para el sistema de visitas se puede definir como:

\begin{equation}\label{Esp.Edos.Down}
X\left(t\right)^{T}=\left(Q_{k}\left(t\right),A_{k}\left(t\right),B_{m}\left(t\right),B_{m}^{0}\left(t\right),C_{m}\left(t\right)\right)
\end{equation}
para $k=1,\ldots,K$ y $m=1,2,\ldots,M$. $X$ evoluciona en el
espacio de estados:
$X=\ent_{+}^{K}\times\rea_{+}^{K}\times\left(\left\{1,2,\ldots,K\right\}\times\left\{1,2,\ldots,S\right\}\right)^{M}\times\rea_{+}^{K}\times\rea_{+}^{K}\times\ent_{+}^{K}$.\\

Antes enunciemos los supuestos que regir\'an en la red.


\begin{itemize}
\item[A1)] $\xi_{1},\ldots,\xi_{K},\eta_{1},\ldots,\eta_{K}$ son
mutuamente independientes y son sucesiones independientes e
id\'enticamente distribuidas.

\item[A2)] Para alg\'un entero $p\geq1$
\begin{eqnarray*}
\esp\left[\xi_{l}\left(1\right)^{p+1}\right]<\infty\textrm{ para }l\in\mathcal{A}\textrm{ y }\\
\esp\left[\eta_{k}\left(1\right)^{p+1}\right]<\infty\textrm{ para
}k=1,\ldots,K.
\end{eqnarray*}
donde $\mathcal{A}$ es la clase de posibles arribos.

\item[A3)] Para $k=1,2,\ldots,K$ existe una funci\'on positiva
$q_{k}\left(x\right)$ definida en $\rea_{+}$, y un entero $j_{k}$,
tal que
\begin{eqnarray}
P\left(\xi_{k}\left(1\right)\geq x\right)>0\textrm{, para todo }x>0\\
P\left(\xi_{k}\left(1\right)+\ldots\xi_{k}\left(j_{k}\right)\in dx\right)\geq q_{k}\left(x\right)dx0\textrm{ y }\\
\int_{0}^{\infty}q_{k}\left(x\right)dx>0
\end{eqnarray}
\end{itemize}
%_________________________________________________________________________
\subsection{Procesos Fuerte de Markov}
%_________________________________________________________________________

En Dai \cite{Dai} se muestra que para una amplia serie de
disciplinas de servicio el proceso $X$ es un Proceso Fuerte de
Markov, y por tanto se puede asumir que
\[\left(\left(\Omega,\mathcal{F}\right),\mathcal{F}_{t},X\left(t\right),\theta_{t},P_{x}\right)\]
es un proceso de Borel Derecho, Sharpe \cite{Sharpe}, en el
espacio de estados medible
$\left(X,\mathcal{B}_{X}\right)$.


Se har\'an las siguientes consideraciones: $E$ es un espacio
m\'etrico separable.


\begin{Def}
Un espacio topol\'ogico $E$ es llamado {\em Luisin} si es
homeomorfo a un subconjunto de Borel de un espacio m\'etrico
compacto.
\end{Def}

\begin{Def}
Un espacio topol\'ogico $E$ es llamado de {\em Rad\'on} si es
homeomorfo a un subconjunto universalmente medible de un espacio
m\'etrico compacto.
\end{Def}

Equivalentemente, la definici\'on de un espacio de Rad\'on puede
encontrarse en los siguientes t\'erminos:

\begin{Def}
$E$ es un espacio de Rad\'on si cada medida finita en
$\left(E,\mathcal{B}\left(E\right)\right)$ es regular interior o
cerrada, {\em tight}.
\end{Def}

\begin{Def}
Una medida finita, $\lambda$ en la $\sigma$-\'algebra de Borel de
un espacio metrizable $E$ se dice cerrada si
\begin{equation}\label{Eq.A2.3}
\lambda\left(E\right)=sup\left\{\lambda\left(K\right):K\textrm{ es
compacto en }E\right\}.
\end{equation}
\end{Def}

El siguiente teorema nos permite tener una mejor caracterizaci\'on
de los espacios de Rad\'on:
\begin{Teo}\label{Tma.A2.2}
Sea $E$ espacio separable metrizable. Entonces $E$ es Radoniano si
y s\'olo s\'i cada medida finita en
$\left(E,\mathcal{B}\left(E\right)\right)$ es cerrada.
\end{Teo}

%_________________________________________________________________________________________
\subsection{Propiedades de Markov}
%_________________________________________________________________________________________

Sea $E$ espacio de estados, tal que $E$ es un espacio de Rad\'on,
$\mathcal{B}\left(E\right)$ $\sigma$-\'algebra de Borel en $E$,
que se denotar\'a por $\mathcal{E}$.

Sea $\left(X,\mathcal{G},\prob\right)$ espacio de probabilidad,
$I\subset\rea$ conjunto de índices. Sea $\mathcal{F}_{\leq
t}$ la $\sigma$-\'algebra natural definida como
$\sigma\left\{f\left(X_{r}\right):r\in I, r\leq
t,f\in\mathcal{E}\right\}$. Se considerar\'a una
$\sigma$-\'algebra m\'as general, $ \left(\mathcal{G}_{t}\right)$
tal que $\left(X_{t}\right)$ sea $\mathcal{E}$-adaptado.

\begin{Def}
Una familia $\left(P_{s,t}\right)$ de kernels de Markov en
$\left(E,\mathcal{E}\right)$ indexada por pares $s,t\in I$, con
$s\leq t$ es una funci\'on de transici\'on en $\ER$, si  para todo
$r\leq s< t$ en $I$ y todo $x\in E$, $B\in\mathcal{E}$
\begin{equation}\label{Eq.Kernels}
P_{r,t}\left(x,B\right)=\int_{E}P_{r,s}\left(x,dy\right)P_{s,t}\left(y,B\right)\footnote{Ecuaci\'on
de Chapman-Kolmogorov}.
\end{equation}
\end{Def}

Se dice que la funci\'on de transici\'on $\KM$ en $\ER$ es la
funci\'on de transici\'on para un proceso $\PE$  con valores en
$E$ y que satisface la propiedad de
Markov \footnote{\begin{equation}\label{Eq.1.4.S}
\prob\left\{H|\mathcal{G}_{t}\right\}=\prob\left\{H|X_{t}\right\}\textrm{
}H\in p\mathcal{F}_{\geq t}.
\end{equation}} (\ref{Eq.1.4.S}) relativa a $\left(\mathcal{G}_{t}\right)$ si

\begin{equation}\label{Eq.1.6.S}
\prob\left\{f\left(X_{t}\right)|\mathcal{G}_{s}\right\}=P_{s,t}f\left(X_{t}\right)\textrm{
}s\leq t\in I,\textrm{ }f\in b\mathcal{E}.
\end{equation}

\begin{Def}
Una familia $\left(P_{t}\right)_{t\geq0}$ de kernels de Markov en
$\ER$ es llamada {\em Semigrupo de Transici\'on de Markov} o {\em
Semigrupo de Transici\'on} si
\[P_{t+s}f\left(x\right)=P_{t}\left(P_{s}f\right)\left(x\right),\textrm{ }t,s\geq0,\textrm{ }x\in E\textrm{ }f\in b\mathcal{E}.\]
\end{Def}

\begin{Note}
Si la funci\'on de transici\'on $\KM$ es llamada homog\'enea si
$P_{s,t}=P_{t-s}$.
\end{Note}


Un proceso de Markov que satisface la ecuaci\'on (\ref{Eq.1.6.S})
con funci\'on de transici\'on homog\'enea $\left(P_{t}\right)$
tiene la propiedad caracter\'istica
\begin{equation}\label{Eq.1.8.S}
\prob\left\{f\left(X_{t+s}\right)|\mathcal{G}_{t}\right\}=P_{s}f\left(X_{t}\right)\textrm{
}t,s\geq0,\textrm{ }f\in b\mathcal{E}.
\end{equation}
La ecuaci\'on anterior es la {\em Propiedad Simple de Markov} de
$X$ relativa a $\left(P_{t}\right)$.

En este sentido el proceso $\PE$ cumple con la propiedad de Markov
(\ref{Eq.1.8.S}) relativa a
$\left(\Omega,\mathcal{G},\mathcal{G}_{t},\prob\right)$ con
semigrupo de transici\'on $\left(P_{t}\right)$.

%_________________________________________________________________________________________
\subsection{Primer Condici\'on de Regularidad}
%_________________________________________________________________________________________


\begin{Def}
Un proceso estoc\'astico $\PE$ definido en
$\left(\Omega,\mathcal{G},\prob\right)$ con valores en el espacio
topol\'ogico $E$ es continuo por la derecha si cada trayectoria
muestral $t\rightarrow X_{t}\left(w\right)$ es un mapeo continuo
por la derecha de $I$ en $E$.
\end{Def}

\begin{Def}[HD1]\label{Eq.2.1.S}
Un semigrupo de Markov $\left/P_{t}\right)$ en un espacio de
Rad\'on $E$ se dice que satisface la condici\'on {\em HD1} si,
dada una medida de probabilidad $\mu$ en $E$, existe una
$\sigma$-\'algebra $\mathcal{E^{*}}$ con
$\mathcal{E}\subset\mathcal{E}$ y
$P_{t}\left(b\mathcal{E}^{*}\right)\subset b\mathcal{E}^{*}$, y un
$\mathcal{E}^{*}$-proceso $E$-valuado continuo por la derecha
$\PE$ en alg\'un espacio de probabilidad filtrado
$\left(\Omega,\mathcal{G},\mathcal{G}_{t},\prob\right)$ tal que
$X=\left(\Omega,\mathcal{G},\mathcal{G}_{t},\prob\right)$ es de
Markov (Homog\'eneo) con semigrupo de transici\'on $(P_{t})$ y
distribuci\'on inicial $\mu$.
\end{Def}
Consid\'erese la colecci\'on de variables aleatorias $X_{t}$
definidas en alg\'un espacio de probabilidad, y una colecci\'on de
medidas $\mathbf{P}^{x}$ tales que
$\mathbf{P}^{x}\left\{X_{0}=x\right\}$, y bajo cualquier
$\mathbf{P}^{x}$, $X_{t}$ es de Markov con semigrupo
$\left(P_{t}\right)$. $\mathbf{P}^{x}$ puede considerarse como la
distribuci\'on condicional de $\mathbf{P}$ dado $X_{0}=x$.

\begin{Def}\label{Def.2.2.S}
Sea $E$ espacio de Rad\'on, $\SG$ semigrupo de Markov en $\ER$. La
colecci\'on
$\mathbf{X}=\left(\Omega,\mathcal{G},\mathcal{G}_{t},X_{t},\theta_{t},\CM\right)$
es un proceso $\mathcal{E}$-Markov continuo por la derecha simple,
con espacio de estados $E$ y semigrupo de transici\'on $\SG$ en
caso de que $\mathbf{X}$ satisfaga las siguientes
condiciones:
\begin{itemize}
\item[i)] $\left(\Omega,\mathcal{G},\mathcal{G}_{t}\right)$ es un
espacio de medida filtrado, y $X_{t}$ es un proceso $E$-valuado
continuo por la derecha $\mathcal{E}^{*}$-adaptado a
$\left(\mathcal{G}_{t}\right)$;

\item[ii)] $\left(\theta_{t}\right)_{t\geq0}$ es una colecci\'on
de operadores {\em shift} para $X$, es decir, mapea $\Omega$ en
s\'i mismo satisfaciendo para $t,s\geq0$,

\begin{equation}\label{Eq.Shift}
\theta_{t}\circ\theta_{s}=\theta_{t+s}\textrm{ y
}X_{t}\circ\theta_{t}=X_{t+s};
\end{equation}

\item[iii)] Para cualquier $x\in E$,$\CM\left\{X_{0}=x\right\}=1$,
y el proceso $\PE$ tiene la propiedad de Markov (\ref{Eq.1.8.S})
con semigrupo de transici\'on $\SG$ relativo a
$\left(\Omega,\mathcal{G},\mathcal{G}_{t},\CM\right)$.
\end{itemize}
\end{Def}


\begin{Def}[HD2]\label{Eq.2.2.S}
Para cualquier $\alpha>0$ y cualquier $f\in S^{\alpha}$, el
proceso $t\rightarrow f\left(X_{t}\right)$ es continuo por la
derecha casi seguramente.
\end{Def}

\begin{Def}\label{Def.PD}
Un sistema
$\mathbf{X}=\left(\Omega,\mathcal{G},\mathcal{G}_{t},X_{t},\theta_{t},\CM\right)$
es un proceso derecho en el espacio de Rad\'on $E$ con semigrupo
de transici\'on $\SG$ provisto de:
\begin{itemize}
\item[i)] $\mathbf{X}$ es una realizaci\'on  continua por la
derecha, \ref{Def.2.2.S}, de $\SG$.

\item[ii)] $\mathbf{X}$ satisface la condicion HD2,
\ref{Eq.2.2.S}, relativa a $\mathcal{G}_{t}$.

\item[iii)] $\mathcal{G}_{t}$ es aumentado y continuo por la
derecha.
\end{itemize}
\end{Def}


\begin{Def}
Sea $X$ un conjunto y $\mathcal{F}$ una $\sigma$-\'algebra de
subconjuntos de $X$, la pareja $\left(X,\mathcal{F}\right)$ es
llamado espacio medible. Un subconjunto $A$ de $X$ es llamado
medible, o medible con respecto a $\mathcal{F}$, si
$A\in\mathcal{F}$.
\end{Def}

\begin{Def}
Sea $\left(X,\mathcal{F},\mu\right)$ espacio de medida. Se dice
que la medida $\mu$ es $\sigma$-finita si se puede escribir
$X=\bigcup_{n\geq1}X_{n}$ con $X_{n}\in\mathcal{F}$ y
$\mu\left(X_{n}\right)<\infty$.
\end{Def}

\begin{Def}\label{Cto.Borel}
Sea $X$ el conjunto de los \'umeros reales $\rea$. El \'algebra de
Borel es la $\sigma$-\'algebra $B$ generada por los intervalos
abiertos $\left(a,b\right)\in\rea$. Cualquier conjunto en $B$ es
llamado {\em Conjunto de Borel}.
\end{Def}

\begin{Def}\label{Funcion.Medible}
Una funci\'on $f:X\rightarrow\rea$, es medible si para cualquier
n\'umero real $\alpha$ el conjunto
\[\left\{x\in X:f\left(x\right)>\alpha\right\}\]
pertenece a $X$. Equivalentemente, se dice que $f$ es medible si
\[f^{-1}\left(\left(\alpha,\infty\right)\right)=\left\{x\in X:f\left(x\right)>\alpha\right\}\in\mathcal{F}.\]
\end{Def}


\begin{Def}\label{Def.Cilindros}
Sean $\left(\Omega_{i},\mathcal{F}_{i}\right)$, $i=1,2,\ldots,$
espacios medibles y $\Omega=\prod_{i=1}^{\infty}\Omega_{i}$ el
conjunto de todas las sucesiones
$\left(\omega_{1},\omega_{2},\ldots,\right)$ tales que
$\omega_{i}\in\Omega_{i}$, $i=1,2,\ldots,$. Si
$B^{n}\subset\prod_{i=1}^{\infty}\Omega_{i}$, definimos
$B_{n}=\left\{\omega\in\Omega:\left(\omega_{1},\omega_{2},\ldots,\omega_{n}\right)\in
B^{n}\right\}$. Al conjunto $B_{n}$ se le llama {\em cilindro} con
base $B^{n}$, el cilindro es llamado medible si
$B^{n}\in\prod_{i=1}^{\infty}\mathcal{F}_{i}$.
\end{Def}


\begin{Def}\label{Def.Proc.Adaptado}[TSP, Ash \cite{RBA}]
Sea $X\left(t\right),t\geq0$ proceso estoc\'astico, el proceso es
adaptado a la familia de $\sigma$-\'algebras $\mathcal{F}_{t}$,
para $t\geq0$, si para $s<t$ implica que
$\mathcal{F}_{s}\subset\mathcal{F}_{t}$, y $X\left(t\right)$ es
$\mathcal{F}_{t}$-medible para cada $t$. Si no se especifica
$\mathcal{F}_{t}$ entonces se toma $\mathcal{F}_{t}$ como
$\mathcal{F}\left(X\left(s\right),s\leq t\right)$, la m\'as
peque\~na $\sigma$-\'algebra de subconjuntos de $\Omega$ que hace
que cada $X\left(s\right)$, con $s\leq t$ sea Borel medible.
\end{Def}


\begin{Def}\label{Def.Tiempo.Paro}[TSP, Ash \cite{RBA}]
Sea $\left\{\mathcal{F}\left(t\right),t\geq0\right\}$ familia
creciente de sub $\sigma$-\'algebras. es decir,
$\mathcal{F}\left(s\right)\subset\mathcal{F}\left(t\right)$ para
$s\leq t$. Un tiempo de paro para $\mathcal{F}\left(t\right)$ es
una funci\'on $T:\Omega\rightarrow\left[0,\infty\right]$ tal que
$\left\{T\leq t\right\}\in\mathcal{F}\left(t\right)$ para cada
$t\geq0$. Un tiempo de paro para el proceso estoc\'astico
$X\left(t\right),t\geq0$ es un tiempo de paro para las
$\sigma$-\'algebras
$\mathcal{F}\left(t\right)=\mathcal{F}\left(X\left(s\right)\right)$.
\end{Def}

\begin{Def}
Sea $X\left(t\right),t\geq0$ proceso estoc\'astico, con
$\left(S,\chi\right)$ espacio de estados. Se dice que el proceso
es adaptado a $\left\{\mathcal{F}\left(t\right)\right\}$, es
decir, si para cualquier $s,t\in I$, $I$ conjunto de \'indices,
$s<t$, se tiene que
$\mathcal{F}\left(s\right)\subset\mathcal{F}\left(t\right)$ y
$X\left(t\right)$ es $\mathcal{F}\left(t\right)$-medible,
\end{Def}

\begin{Def}
Sea $X\left(t\right),t\geq0$ proceso estoc\'astico, se dice que es
un Proceso de Markov relativo a $\mathcal{F}\left(t\right)$ o que
$\left\{X\left(t\right),\mathcal{F}\left(t\right)\right\}$ es de
Markov si y s\'olo si para cualquier conjunto $B\in\chi$,  y
$s,t\in I$, $s<t$ se cumple que
\begin{equation}\label{Prop.Markov}
P\left\{X\left(t\right)\in
B|\mathcal{F}\left(s\right)\right\}=P\left\{X\left(t\right)\in
B|X\left(s\right)\right\}.
\end{equation}
\end{Def}
\begin{Note}
Si se dice que $\left\{X\left(t\right)\right\}$ es un Proceso de
Markov sin mencionar $\mathcal{F}\left(t\right)$, se asumir\'a que
\begin{eqnarray*}
\mathcal{F}\left(t\right)=\mathcal{F}_{0}\left(t\right)=\mathcal{F}\left(X\left(r\right),r\leq
t\right),
\end{eqnarray*}
entonces la ecuaci\'on (\ref{Prop.Markov}) se puede escribir como
\begin{equation}
P\left\{X\left(t\right)\in B|X\left(r\right),r\leq s\right\} =
P\left\{X\left(t\right)\in B|X\left(s\right)\right\}
\end{equation}
\end{Note}

\begin{Teo}
Sea $\left(X_{n},\mathcal{F}_{n},n=0,1,\ldots,\right\}$ Proceso de
Markov con espacio de estados $\left(S_{0},\chi_{0}\right)$
generado por una distribuici\'on inicial $P_{o}$ y probabilidad de
transici\'on $p_{mn}$, para $m,n=0,1,\ldots,$ $m<n$, que por
notaci\'on se escribir\'a como $p\left(m,n,x,B\right)\rightarrow
p_{mn}\left(x,B\right)$. Sea $S$ tiempo de paro relativo a la
$\sigma$-\'algebra $\mathcal{F}_{n}$. Sea $T$ funci\'on medible,
$T:\Omega\rightarrow\left\{0,1,\ldots,\right\}$. Sup\'ongase que
$T\geq S$, entonces $T$ es tiempo de paro. Si $B\in\chi_{0}$,
entonces
\begin{equation}\label{Prop.Fuerte.Markov}
P\left\{X\left(T\right)\in
B,T<\infty|\mathcal{F}\left(S\right)\right\} =
p\left(S,T,X\left(s\right),B\right)
\end{equation}
en $\left\{T<\infty\right\}$.
\end{Teo}


Sea $K$ conjunto numerable y sea $d:K\rightarrow\nat$ funci\'on.
Para $v\in K$, $M_{v}$ es un conjunto abierto de
$\rea^{d\left(v\right)}$. Entonces \[E=\cup_{v\in
K}M_{v}=\left\{\left(v,\zeta\right):v\in K,\zeta\in
M_{v}\right\}.\]

Sea $\mathcal{E}$ la clase de conjuntos medibles en $E$:
\[\mathcal{E}=\left\{\cup_{v\in K}A_{v}:A_{v}\in \mathcal{M}_{v}\right\}.\]

donde $\mathcal{M}$ son los conjuntos de Borel de $M_{v}$.
Entonces $\left(E,\mathcal{E}\right)$ es un espacio de Borel. El
estado del proceso se denotar\'a por
$\mathbf{x}_{t}=\left(v_{t},\zeta_{t}\right)$. La distribuci\'on
de $\left(\mathbf{x}_{t}\right)$ est\'a determinada por por los
siguientes objetos:

\begin{itemize}
\item[i)] Los campos vectoriales $\left(\mathcal{H}_{v},v\in
K\right)$. \item[ii)] Una funci\'on medible $\lambda:E\rightarrow
\rea_{+}$. \item[iii)] Una medida de transici\'on
$Q:\mathcal{E}\times\left(E\cup\Gamma^{*}\right)\rightarrow\left[0,1\right]$
donde
\begin{equation}
\Gamma^{*}=\cup_{v\in K}\partial^{*}M_{v}.
\end{equation}
y
\begin{equation}
\partial^{*}M_{v}=\left\{z\in\partial M_{v}:\mathbf{\mathbf{\phi}_{v}\left(t,\zeta\right)=\mathbf{z}}\textrm{ para alguna }\left(t,\zeta\right)\in\rea_{+}\times M_{v}\right\}.
\end{equation}
$\partial M_{v}$ denota  la frontera de $M_{v}$.
\end{itemize}

El campo vectorial $\left(\mathcal{H}_{v},v\in K\right)$ se supone
tal que para cada $\mathbf{z}\in M_{v}$ existe una \'unica curva
integral $\mathbf{\phi}_{v}\left(t,\zeta\right)$ que satisface la
ecuaci\'on

\begin{equation}
\frac{d}{dt}f\left(\zeta_{t}\right)=\mathcal{H}f\left(\zeta_{t}\right),
\end{equation}
con $\zeta_{0}=\mathbf{z}$, para cualquier funci\'on suave
$f:\rea^{d}\rightarrow\rea$ y $\mathcal{H}$ denota el operador
diferencial de primer orden, con $\mathcal{H}=\mathcal{H}_{v}$ y
$\zeta_{t}=\mathbf{\phi}\left(t,\mathbf{z}\right)$. Adem\'as se
supone que $\mathcal{H}_{v}$ es conservativo, es decir, las curvas
integrales est\'an definidas para todo $t>0$.

Para $\mathbf{x}=\left(v,\zeta\right)\in E$ se denota
\[t^{*}\mathbf{x}=inf\left\{t>0:\mathbf{\phi}_{v}\left(t,\zeta\right)\in\partial^{*}M_{v}\right\}\]

En lo que respecta a la funci\'on $\lambda$, se supondr\'a que
para cada $\left(v,\zeta\right)\in E$ existe un $\epsilon>0$ tal
que la funci\'on
$s\rightarrow\lambda\left(v,\phi_{v}\left(s,\zeta\right)\right)\in
E$ es integrable para $s\in\left[0,\epsilon\right)$. La medida de
transici\'on $Q\left(A;\mathbf{x}\right)$ es una funci\'on medible
de $\mathbf{x}$ para cada $A\in\mathcal{E}$, definida para
$\mathbf{x}\in E\cup\Gamma^{*}$ y es una medida de probabilidad en
$\left(E,\mathcal{E}\right)$ para cada $\mathbf{x}\in E$.

El movimiento del proceso $\left(\mathbf{x}_{t}\right)$ comenzando
en $\mathbf{x}=\left(n,\mathbf{z}\right)\in E$ se puede construir
de la siguiente manera, def\'inase la funci\'on $F$ por

\begin{equation}
F\left(t\right)=\left\{\begin{array}{ll}\\
exp\left(-\int_{0}^{t}\lambda\left(n,\phi_{n}\left(s,\mathbf{z}\right)\right)ds\right), & t<t^{*}\left(\mathbf{x}\right),\\
0, & t\geq t^{*}\left(\mathbf{x}\right)
\end{array}\right.
\end{equation}

Sea $T_{1}$ una variable aleatoria tal que
$\prob\left[T_{1}>t\right]=F\left(t\right)$, ahora sea la variable
aleatoria $\left(N,Z\right)$ con distribuici\'on
$Q\left(\cdot;\phi_{n}\left(T_{1},\mathbf{z}\right)\right)$. La
trayectoria de $\left(\mathbf{x}_{t}\right)$ para $t\leq T_{1}$
es\footnote{Revisar p\'agina 362, y 364 de Davis \cite{Davis}.}
\begin{eqnarray*}
\mathbf{x}_{t}=\left(v_{t},\zeta_{t}\right)=\left\{\begin{array}{ll}
\left(n,\phi_{n}\left(t,\mathbf{z}\right)\right), & t<T_{1},\\
\left(N,\mathbf{Z}\right), & t=t_{1}.
\end{array}\right.
\end{eqnarray*}

Comenzando en $\mathbf{x}_{T_{1}}$ se selecciona el siguiente
tiempo de intersalto $T_{2}-T_{1}$ lugar del post-salto
$\mathbf{x}_{T_{2}}$ de manera similar y as\'i sucesivamente. Este
procedimiento nos da una trayectoria determinista por partes
$\mathbf{x}_{t}$ con tiempos de salto $T_{1},T_{2},\ldots$. Bajo
las condiciones enunciadas para $\lambda,T_{1}>0$  y
$T_{1}-T_{2}>0$ para cada $i$, con probabilidad 1. Se supone que
se cumple la siquiente condici\'on.

\begin{Sup}[Supuesto 3.1, Davis \cite{Davis}]\label{Sup3.1.Davis}
Sea $N_{t}:=\sum_{t}\indora_{\left(t\geq t\right)}$ el n\'umero de
saltos en $\left[0,t\right]$. Entonces
\begin{equation}
\esp\left[N_{t}\right]<\infty\textrm{ para toda }t.
\end{equation}
\end{Sup}

es un proceso de Markov, m\'as a\'un, es un Proceso Fuerte de
Markov, es decir, la Propiedad Fuerte de Markov se cumple para
cualquier tiempo de paro.
%_________________________________________________________________________

En esta secci\'on se har\'an las siguientes consideraciones: $E$
es un espacio m\'etrico separable y la m\'etrica $d$ es compatible
con la topolog\'ia.


\begin{Def}
Un espacio topol\'ogico $E$ es llamado {\em Luisin} si es
homeomorfo a un subconjunto de Borel de un espacio m\'etrico
compacto.
\end{Def}

\begin{Def}
Un espacio topol\'ogico $E$ es llamado de {\em Rad\'on} si es
homeomorfo a un subconjunto universalmente medible de un espacio
m\'etrico compacto.
\end{Def}

Equivalentemente, la definici\'on de un espacio de Rad\'on puede
encontrarse en los siguientes t\'erminos:


\begin{Def}
$E$ es un espacio de Rad\'on si cada medida finita en
$\left(E,\mathcal{B}\left(E\right)\right)$ es regular interior o cerrada,
{\em tight}.
\end{Def}

\begin{Def}
Una medida finita, $\lambda$ en la $\sigma$-\'algebra de Borel de
un espacio metrizable $E$ se dice cerrada si
\begin{equation}\label{Eq.A2.3}
\lambda\left(E\right)=sup\left\{\lambda\left(K\right):K\textrm{ es
compacto en }E\right\}.
\end{equation}
\end{Def}

El siguiente teorema nos permite tener una mejor caracterizaci\'on de los espacios de Rad\'on:
\begin{Teo}\label{Tma.A2.2}
Sea $E$ espacio separable metrizable. Entonces $E$ es Radoniano si y s\'olo s\'i cada medida finita en $\left(E,\mathcal{B}\left(E\right)\right)$ es cerrada.
\end{Teo}

%_________________________________________________________________________________________
\subsection{Propiedades de Markov}
%_________________________________________________________________________________________

Sea $E$ espacio de estados, tal que $E$ es un espacio de Rad\'on, $\mathcal{B}\left(E\right)$ $\sigma$-\'algebra de Borel en $E$, que se denotar\'a por $\mathcal{E}$.

Sea $\left(X,\mathcal{G},\prob\right)$ espacio de probabilidad, $I\subset\rea$ conjunto de índices. Sea $\mathcal{F}_{\leq t}$ la $\sigma$-\'algebra natural definida como $\sigma\left\{f\left(X_{r}\right):r\in I, rleq t,f\in\mathcal{E}\right\}$. Se considerar\'a una $\sigma$-\'algebra m\'as general, $ \left(\mathcal{G}_{t}\right)$ tal que $\left(X_{t}\right)$ sea $\mathcal{E}$-adaptado.

\begin{Def}
Una familia $\left(P_{s,t}\right)$ de kernels de Markov en $\left(E,\mathcal{E}\right)$ indexada por pares $s,t\in I$, con $s\leq t$ es una funci\'on de transici\'on en $\ER$, si  para todo $r\leq s< t$ en $I$ y todo $x\in E$, $B\in\mathcal{E}$
\begin{equation}\label{Eq.Kernels}
P_{r,t}\left(x,B\right)=\int_{E}P_{r,s}\left(x,dy\right)P_{s,t}\left(y,B\right)\footnote{Ecuaci\'on de Chapman-Kolmogorov}.
\end{equation}
\end{Def}

Se dice que la funci\'on de transici\'on $\KM$ en $\ER$ es la funci\'on de transici\'on para un proceso $\PE$  con valores en $E$ y que satisface la propiedad de Markov\footnote{\begin{equation}\label{Eq.1.4.S}
\prob\left\{H|\mathcal{G}_{t}\right\}=\prob\left\{H|X_{t}\right\}\textrm{ }H\in p\mathcal{F}_{\geq t}.
\end{equation}} (\ref{Eq.1.4.S}) relativa a $\left(\mathcal{G}_{t}\right)$ si 

\begin{equation}\label{Eq.1.6.S}
\prob\left\{f\left(X_{t}\right)|\mathcal{G}_{s}\right\}=P_{s,t}f\left(X_{t}\right)\textrm{ }s\leq t\in I,\textrm{ }f\in b\mathcal{E}.
\end{equation}

\begin{Def}
Una familia $\left(P_{t}\right)_{t\geq0}$ de kernels de Markov en $\ER$ es llamada {\em Semigrupo de Transici\'on de Markov} o {\em Semigrupo de Transici\'on} si
\[P_{t+s}f\left(x\right)=P_{t}\left(P_{s}f\right)\left(x\right),\textrm{ }t,s\geq0,\textrm{ }x\in E\textrm{ }f\in b\mathcal{E}.\]
\end{Def}
\begin{Note}
Si la funci\'on de transici\'on $\KM$ es llamada homog\'enea si $P_{s,t}=P_{t-s}$.
\end{Note}

Un proceso de Markov que satisface la ecuaci\'on (\ref{Eq.1.6.S}) con funci\'on de transici\'on homog\'enea $\left(P_{t}\right)$ tiene la propiedad caracter\'istica
\begin{equation}\label{Eq.1.8.S}
\prob\left\{f\left(X_{t+s}\right)|\mathcal{G}_{t}\right\}=P_{s}f\left(X_{t}\right)\textrm{ }t,s\geq0,\textrm{ }f\in b\mathcal{E}.
\end{equation}
La ecuaci\'on anterior es la {\em Propiedad Simple de Markov} de $X$ relativa a $\left(P_{t}\right)$.

En este sentido el proceso $\PE$ cumple con la propiedad de Markov (\ref{Eq.1.8.S}) relativa a $\left(\Omega,\mathcal{G},\mathcal{G}_{t},\prob\right)$ con semigrupo de transici\'on $\left(P_{t}\right)$.
%_________________________________________________________________________________________
\subsection{Primer Condici\'on de Regularidad}
%_________________________________________________________________________________________
%\newcommand{\EM}{\left(\Omega,\mathcal{G},\prob\right)}
%\newcommand{\E4}{\left(\Omega,\mathcal{G},\mathcal{G}_{t},\prob\right)}
\begin{Def}
Un proceso estoc\'astico $\PE$ definido en $\left(\Omega,\mathcal{G},\prob\right)$ con valores en el espacio topol\'ogico $E$ es continuo por la derecha si cada trayectoria muestral $t\rightarrow X_{t}\left(w\right)$ es un mapeo continuo por la derecha de $I$ en $E$.
\end{Def}

\begin{Def}[HD1]\label{Eq.2.1.S}
Un semigrupo de Markov $\left/P_{t}\right)$ en un espacio de Rad\'on $E$ se dice que satisface la condici\'on {\em HD1} si, dada una medida de probabilidad $\mu$ en $E$, existe una $\sigma$-\'algebra $\mathcal{E^{*}}$ con $\mathcal{E}\subset\mathcal{E}$ y $P_{t}\left(b\mathcal{E}^{*}\right)\subset b\mathcal{E}^{*}$, y un $\mathcal{E}^{*}$-proceso $E$-valuado continuo por la derecha $\PE$ en alg\'un espacio de probabilidad filtrado $\left(\Omega,\mathcal{G},\mathcal{G}_{t},\prob\right)$ tal que $X=\left(\Omega,\mathcal{G},\mathcal{G}_{t},\prob\right)$ es de Markov (Homog\'eneo) con semigrupo de transici\'on $(P_{t})$ y distribuci\'on inicial $\mu$.
\end{Def}

Considerese la colecci\'on de variables aleatorias $X_{t}$ definidas en alg\'un espacio de probabilidad, y una colecci\'on de medidas $\mathbf{P}^{x}$ tales que $\mathbf{P}^{x}\left\{X_{0}=x\right\}$, y bajo cualquier $\mathbf{P}^{x}$, $X_{t}$ es de Markov con semigrupo $\left(P_{t}\right)$. $\mathbf{P}^{x}$ puede considerarse como la distribuci\'on condicional de $\mathbf{P}$ dado $X_{0}=x$.

\begin{Def}\label{Def.2.2.S}
Sea $E$ espacio de Rad\'on, $\SG$ semigrupo de Markov en $\ER$. La colecci\'on $\mathbf{X}=\left(\Omega,\mathcal{G},\mathcal{G}_{t},X_{t},\theta_{t},\CM\right)$ es un proceso $\mathcal{E}$-Markov continuo por la derecha simple, con espacio de estados $E$ y semigrupo de transici\'on $\SG$ en caso de que $\mathbf{X}$ satisfaga las siguientes condiciones:
\begin{itemize}
\item[i)] $\left(\Omega,\mathcal{G},\mathcal{G}_{t}\right)$ es un espacio de medida filtrado, y $X_{t}$ es un proceso $E$-valuado continuo por la derecha $\mathcal{E}^{*}$-adaptado a $\left(\mathcal{G}_{t}\right)$;

\item[ii)] $\left(\theta_{t}\right)_{t\geq0}$ es una colecci\'on de operadores {\em shift} para $X$, es decir, mapea $\Omega$ en s\'i mismo satisfaciendo para $t,s\geq0$,

\begin{equation}\label{Eq.Shift}
\theta_{t}\circ\theta_{s}=\theta_{t+s}\textrm{ y }X_{t}\circ\theta_{t}=X_{t+s};
\end{equation}

\item[iii)] Para cualquier $x\in E$,$\CM\left\{X_{0}=x\right\}=1$, y el proceso $\PE$ tiene la propiedad de Markov (\ref{Eq.1.8.S}) con semigrupo de transici\'on $\SG$ relativo a $\left(\Omega,\mathcal{G},\mathcal{G}_{t},\CM\right)$.
\end{itemize}
\end{Def}

\begin{Def}[HD2]\label{Eq.2.2.S}
Para cualquier $\alpha>0$ y cualquier $f\in S^{\alpha}$, el proceso $t\rightarrow f\left(X_{t}\right)$ es continuo por la derecha casi seguramente.
\end{Def}

\begin{Def}\label{Def.PD}
Un sistema $\mathbf{X}=\left(\Omega,\mathcal{G},\mathcal{G}_{t},X_{t},\theta_{t},\CM\right)$ es un proceso derecho en el espacio de Rad\'on $E$ con semigrupo de transici\'on $\SG$ provisto de:
\begin{itemize}
\item[i)] $\mathbf{X}$ es una realizaci\'on  continua por la derecha, \ref{Def.2.2.S}, de $\SG$.

\item[ii)] $\mathbf{X}$ satisface la condicion HD2, \ref{Eq.2.2.S}, relativa a $\mathcal{G}_{t}$.

\item[iii)] $\mathcal{G}_{t}$ es aumentado y continuo por la derecha.
\end{itemize}
\end{Def}



\begin{Lema}[Lema 4.2, Dai\cite{Dai}]\label{Lema4.2}
Sea $\left\{x_{n}\right\}\subset \mathbf{X}$ con
$|x_{n}|\rightarrow\infty$, conforme $n\rightarrow\infty$. Suponga
que
\[lim_{n\rightarrow\infty}\frac{1}{|x_{n}|}U\left(0\right)=\overline{U}\]
y
\[lim_{n\rightarrow\infty}\frac{1}{|x_{n}|}V\left(0\right)=\overline{V}.\]

Entonces, conforme $n\rightarrow\infty$, casi seguramente

\begin{equation}\label{E1.4.2}
\frac{1}{|x_{n}|}\Phi^{k}\left(\left[|x_{n}|t\right]\right)\rightarrow
P_{k}^{'}t\textrm{, u.o.c.,}
\end{equation}

\begin{equation}\label{E1.4.3}
\frac{1}{|x_{n}|}E^{x_{n}}_{k}\left(|x_{n}|t\right)\rightarrow
\alpha_{k}\left(t-\overline{U}_{k}\right)^{+}\textrm{, u.o.c.,}
\end{equation}

\begin{equation}\label{E1.4.4}
\frac{1}{|x_{n}|}S^{x_{n}}_{k}\left(|x_{n}|t\right)\rightarrow
\mu_{k}\left(t-\overline{V}_{k}\right)^{+}\textrm{, u.o.c.,}
\end{equation}

donde $\left[t\right]$ es la parte entera de $t$ y
$\mu_{k}=1/m_{k}=1/\esp\left[\eta_{k}\left(1\right)\right]$.
\end{Lema}

\begin{Lema}[Lema 4.3, Dai\cite{Dai}]\label{Lema.4.3}
Sea $\left\{x_{n}\right\}\subset \mathbf{X}$ con
$|x_{n}|\rightarrow\infty$, conforme $n\rightarrow\infty$. Suponga
que
\[lim_{n\rightarrow\infty}\frac{1}{|x_{n}|}U\left(0\right)=\overline{U}_{k}\]
y
\[lim_{n\rightarrow\infty}\frac{1}{|x_{n}|}V\left(0\right)=\overline{V}_{k}.\]
\begin{itemize}
\item[a)] Conforme $n\rightarrow\infty$ casi seguramente,
\[lim_{n\rightarrow\infty}\frac{1}{|x_{n}|}U^{x_{n}}_{k}\left(|x_{n}|t\right)=\left(\overline{U}_{k}-t\right)^{+}\textrm{, u.o.c.}\]
y
\[lim_{n\rightarrow\infty}\frac{1}{|x_{n}|}V^{x_{n}}_{k}\left(|x_{n}|t\right)=\left(\overline{V}_{k}-t\right)^{+}.\]

\item[b)] Para cada $t\geq0$ fijo,
\[\left\{\frac{1}{|x_{n}|}U^{x_{n}}_{k}\left(|x_{n}|t\right),|x_{n}|\geq1\right\}\]
y
\[\left\{\frac{1}{|x_{n}|}V^{x_{n}}_{k}\left(|x_{n}|t\right),|x_{n}|\geq1\right\}\]
\end{itemize}
son uniformemente convergentes.
\end{Lema}

$S_{l}^{x}\left(t\right)$ es el n\'umero total de servicios
completados de la clase $l$, si la clase $l$ est\'a dando $t$
unidades de tiempo de servicio. Sea $T_{l}^{x}\left(x\right)$ el
monto acumulado del tiempo de servicio que el servidor
$s\left(l\right)$ gasta en los usuarios de la clase $l$ al tiempo
$t$. Entonces $S_{l}^{x}\left(T_{l}^{x}\left(t\right)\right)$ es
el n\'umero total de servicios completados para la clase $l$ al
tiempo $t$. Una fracci\'on de estos usuarios,
$\Phi_{l}^{x}\left(S_{l}^{x}\left(T_{l}^{x}\left(t\right)\right)\right)$,
se convierte en usuarios de la clase $k$.\\

Entonces, dado lo anterior, se tiene la siguiente representaci\'on
para el proceso de la longitud de la cola:\\

\begin{equation}
Q_{k}^{x}\left(t\right)=_{k}^{x}\left(0\right)+E_{k}^{x}\left(t\right)+\sum_{l=1}^{K}\Phi_{k}^{l}\left(S_{l}^{x}\left(T_{l}^{x}\left(t\right)\right)\right)-S_{k}^{x}\left(T_{k}^{x}\left(t\right)\right)
\end{equation}
para $k=1,\ldots,K$. Para $i=1,\ldots,d$, sea
\[I_{i}^{x}\left(t\right)=t-\sum_{j\in C_{i}}T_{k}^{x}\left(t\right).\]

Entonces $I_{i}^{x}\left(t\right)$ es el monto acumulado del
tiempo que el servidor $i$ ha estado desocupado al tiempo $t$. Se
est\'a asumiendo que las disciplinas satisfacen la ley de
conservaci\'on del trabajo, es decir, el servidor $i$ est\'a en
pausa solamente cuando no hay usuarios en la estaci\'on $i$.
Entonces, se tiene que

\begin{equation}
\int_{0}^{\infty}\left(\sum_{k\in
C_{i}}Q_{k}^{x}\left(t\right)\right)dI_{i}^{x}\left(t\right)=0,
\end{equation}
para $i=1,\ldots,d$.\\

Hacer
\[T^{x}\left(t\right)=\left(T_{1}^{x}\left(t\right),\ldots,T_{K}^{x}\left(t\right)\right)^{'},\]
\[I^{x}\left(t\right)=\left(I_{1}^{x}\left(t\right),\ldots,I_{K}^{x}\left(t\right)\right)^{'}\]
y
\[S^{x}\left(T^{x}\left(t\right)\right)=\left(S_{1}^{x}\left(T_{1}^{x}\left(t\right)\right),\ldots,S_{K}^{x}\left(T_{K}^{x}\left(t\right)\right)\right)^{'}.\]

Para una disciplina que cumple con la ley de conservaci\'on del
trabajo, en forma vectorial, se tiene el siguiente conjunto de
ecuaciones

\begin{equation}\label{Eq.MF.1.3}
Q^{x}\left(t\right)=Q^{x}\left(0\right)+E^{x}\left(t\right)+\sum_{l=1}^{K}\Phi^{l}\left(S_{l}^{x}\left(T_{l}^{x}\left(t\right)\right)\right)-S^{x}\left(T^{x}\left(t\right)\right),\\
\end{equation}

\begin{equation}\label{Eq.MF.2.3}
Q^{x}\left(t\right)\geq0,\\
\end{equation}

\begin{equation}\label{Eq.MF.3.3}
T^{x}\left(0\right)=0,\textrm{ y }\overline{T}^{x}\left(t\right)\textrm{ es no decreciente},\\
\end{equation}

\begin{equation}\label{Eq.MF.4.3}
I^{x}\left(t\right)=et-CT^{x}\left(t\right)\textrm{ es no
decreciente}\\
\end{equation}

\begin{equation}\label{Eq.MF.5.3}
\int_{0}^{\infty}\left(CQ^{x}\left(t\right)\right)dI_{i}^{x}\left(t\right)=0,\\
\end{equation}

\begin{equation}\label{Eq.MF.6.3}
\textrm{Condiciones adicionales en
}\left(\overline{Q}^{x}\left(\cdot\right),\overline{T}^{x}\left(\cdot\right)\right)\textrm{
espec\'ificas de la disciplina de la cola,}
\end{equation}

donde $e$ es un vector de unos de dimensi\'on $d$, $C$ es la
matriz definida por
\[C_{ik}=\left\{\begin{array}{cc}
1,& S\left(k\right)=i,\\
0,& \textrm{ en otro caso}.\\
\end{array}\right.
\]
Es necesario enunciar el siguiente Teorema que se utilizar\'a para
el Teorema \ref{Tma.4.2.Dai}:
\begin{Teo}[Teorema 4.1, Dai \cite{Dai}]
Considere una disciplina que cumpla la ley de conservaci\'on del
trabajo, para casi todas las trayectorias muestrales $\omega$ y
cualquier sucesi\'on de estados iniciales
$\left\{x_{n}\right\}\subset \mathbf{X}$, con
$|x_{n}|\rightarrow\infty$, existe una subsucesi\'on
$\left\{x_{n_{j}}\right\}$ con $|x_{n_{j}}|\rightarrow\infty$ tal
que
\begin{equation}\label{Eq.4.15}
\frac{1}{|x_{n_{j}}|}\left(Q^{x_{n_{j}}}\left(0\right),U^{x_{n_{j}}}\left(0\right),V^{x_{n_{j}}}\left(0\right)\right)\rightarrow\left(\overline{Q}\left(0\right),\overline{U},\overline{V}\right),
\end{equation}

\begin{equation}\label{Eq.4.16}
\frac{1}{|x_{n_{j}}|}\left(Q^{x_{n_{j}}}\left(|x_{n_{j}}|t\right),T^{x_{n_{j}}}\left(|x_{n_{j}}|t\right)\right)\rightarrow\left(\overline{Q}\left(t\right),\overline{T}\left(t\right)\right)\textrm{
u.o.c.}
\end{equation}

Adem\'as,
$\left(\overline{Q}\left(t\right),\overline{T}\left(t\right)\right)$
satisface las siguientes ecuaciones:
\begin{equation}\label{Eq.MF.1.3a}
\overline{Q}\left(t\right)=Q\left(0\right)+\left(\alpha
t-\overline{U}\right)^{+}-\left(I-P\right)^{'}M^{-1}\left(\overline{T}\left(t\right)-\overline{V}\right)^{+},
\end{equation}

\begin{equation}\label{Eq.MF.2.3a}
\overline{Q}\left(t\right)\geq0,\\
\end{equation}

\begin{equation}\label{Eq.MF.3.3a}
\overline{T}\left(t\right)\textrm{ es no decreciente y comienza en cero},\\
\end{equation}

\begin{equation}\label{Eq.MF.4.3a}
\overline{I}\left(t\right)=et-C\overline{T}\left(t\right)\textrm{
es no decreciente,}\\
\end{equation}

\begin{equation}\label{Eq.MF.5.3a}
\int_{0}^{\infty}\left(C\overline{Q}\left(t\right)\right)d\overline{I}\left(t\right)=0,\\
\end{equation}

\begin{equation}\label{Eq.MF.6.3a}
\textrm{Condiciones adicionales en
}\left(\overline{Q}\left(\cdot\right),\overline{T}\left(\cdot\right)\right)\textrm{
especficas de la disciplina de la cola,}
\end{equation}
\end{Teo}

\begin{Def}[Definici\'on 4.1, , Dai \cite{Dai}]
Sea una disciplina de servicio espec\'ifica. Cualquier l\'imite
$\left(\overline{Q}\left(\cdot\right),\overline{T}\left(\cdot\right)\right)$
en \ref{Eq.4.16} es un {\em flujo l\'imite} de la disciplina.
Cualquier soluci\'on (\ref{Eq.MF.1.3a})-(\ref{Eq.MF.6.3a}) es
llamado flujo soluci\'on de la disciplina. Se dice que el modelo de flujo l\'imite, modelo de flujo, de la disciplina de la cola es estable si existe una constante
$\delta>0$ que depende de $\mu,\alpha$ y $P$ solamente, tal que
cualquier flujo l\'imite con
$|\overline{Q}\left(0\right)|+|\overline{U}|+|\overline{V}|=1$, se
tiene que $\overline{Q}\left(\cdot+\delta\right)\equiv0$.
\end{Def}

\begin{Teo}[Teorema 4.2, Dai\cite{Dai}]\label{Tma.4.2.Dai}
Sea una disciplina fija para la cola, suponga que se cumplen las
condiciones (1.2)-(1.5). Si el modelo de flujo l\'imite de la
disciplina de la cola es estable, entonces la cadena de Markov $X$
que describe la din\'amica de la red bajo la disciplina es Harris
recurrente positiva.
\end{Teo}

Ahora se procede a escalar el espacio y el tiempo para reducir la
aparente fluctuaci\'on del modelo. Consid\'erese el proceso
\begin{equation}\label{Eq.3.7}
\overline{Q}^{x}\left(t\right)=\frac{1}{|x|}Q^{x}\left(|x|t\right)
\end{equation}
A este proceso se le conoce como el fluido escalado, y cualquier l\'imite $\overline{Q}^{x}\left(t\right)$ es llamado flujo l\'imite del proceso de longitud de la cola. Haciendo $|q|\rightarrow\infty$ mientras se mantiene el resto de las componentes fijas, cualquier punto l\'imite del proceso de longitud de la cola normalizado $\overline{Q}^{x}$ es soluci\'on del siguiente modelo de flujo.

Al conjunto de ecuaciones dadas en \ref{Eq.3.8}-\ref{Eq.3.13} se
le llama {\em Modelo de flujo} y al conjunto de todas las
soluciones del modelo de flujo
$\left(\overline{Q}\left(\cdot\right),\overline{T}
\left(\cdot\right)\right)$ se le denotar\'a por $\mathcal{Q}$.

Si se hace $|x|\rightarrow\infty$ sin restringir ninguna de las
componentes, tambi\'en se obtienen un modelo de flujo, pero en
este caso el residual de los procesos de arribo y servicio
introducen un retraso:

\begin{Def}[Definici\'on 3.3, Dai y Meyn \cite{DaiSean}]
El modelo de flujo es estable si existe un tiempo fijo $t_{0}$ tal
que $\overline{Q}\left(t\right)=0$, con $t\geq t_{0}$, para
cualquier $\overline{Q}\left(\cdot\right)\in\mathcal{Q}$ que
cumple con $|\overline{Q}\left(0\right)|=1$.
\end{Def}

El siguiente resultado se encuentra en Chen \cite{Chen}.
\begin{Lemma}[Lema 3.1, Dai y Meyn \cite{DaiSean}]
Si el modelo de flujo definido por \ref{Eq.3.8}-\ref{Eq.3.13} es
estable, entonces el modelo de flujo retrasado es tambi\'en
estable, es decir, existe $t_{0}>0$ tal que
$\overline{Q}\left(t\right)=0$ para cualquier $t\geq t_{0}$, para
cualquier soluci\'on del modelo de flujo retrasado cuya
condici\'on inicial $\overline{x}$ satisface que
$|\overline{x}|=|\overline{Q}\left(0\right)|+|\overline{A}\left(0\right)|+|\overline{B}\left(0\right)|\leq1$.
\end{Lemma}


Propiedades importantes para el modelo de flujo retrasado:

\begin{Prop}
 Sea $\left(\overline{Q},\overline{T},\overline{T}^{0}\right)$ un flujo l\'imite de \ref{Eq.4.4} y suponga que cuando $x\rightarrow\infty$ a lo largo de
una subsucesi\'on
\[\left(\frac{1}{|x|}Q_{k}^{x}\left(0\right),\frac{1}{|x|}A_{k}^{x}\left(0\right),\frac{1}{|x|}B_{k}^{x}\left(0\right),\frac{1}{|x|}B_{k}^{x,0}\left(0\right)\right)\rightarrow\left(\overline{Q}_{k}\left(0\right),0,0,0\right)\]
para $k=1,\ldots,K$. EL flujo l\'imite tiene las siguientes
propiedades, donde las propiedades de la derivada se cumplen donde
la derivada exista:
\begin{itemize}
 \item[i)] Los vectores de tiempo ocupado $\overline{T}\left(t\right)$ y $\overline{T}^{0}\left(t\right)$ son crecientes y continuas con
$\overline{T}\left(0\right)=\overline{T}^{0}\left(0\right)=0$.
\item[ii)] Para todo $t\geq0$
\[\sum_{k=1}^{K}\left[\overline{T}_{k}\left(t\right)+\overline{T}_{k}^{0}\left(t\right)\right]=t\]
\item[iii)] Para todo $1\leq k\leq K$
\[\overline{Q}_{k}\left(t\right)=\overline{Q}_{k}\left(0\right)+\alpha_{k}t-\mu_{k}\overline{T}_{k}\left(t\right)\]
\item[iv)]  Para todo $1\leq k\leq K$
\[\dot{{\overline{T}}}_{k}\left(t\right)=\beta_{k}\] para $\overline{Q}_{k}\left(t\right)=0$.
\item[v)] Para todo $k,j$
\[\mu_{k}^{0}\overline{T}_{k}^{0}\left(t\right)=\mu_{j}^{0}\overline{T}_{j}^{0}\left(t\right)\]
\item[vi)]  Para todo $1\leq k\leq K$
\[\mu_{k}\dot{{\overline{T}}}_{k}\left(t\right)=l_{k}\mu_{k}^{0}\dot{{\overline{T}}}_{k}^{0}\left(t\right)\] para $\overline{Q}_{k}\left(t\right)>0$.
\end{itemize}
\end{Prop}

\begin{Lema}[Lema 3.1 \cite{Chen}]\label{Lema3.1}
Si el modelo de flujo es estable, definido por las ecuaciones
(3.8)-(3.13), entonces el modelo de flujo retrasado tambin es
estable.
\end{Lema}

\begin{Teo}[Teorema 5.2 \cite{Chen}]\label{Tma.5.2}
Si el modelo de flujo lineal correspondiente a la red de cola es
estable, entonces la red de colas es estable.
\end{Teo}

\begin{Teo}[Teorema 5.1 \cite{Chen}]\label{Tma.5.1.Chen}
La red de colas es estable si existe una constante $t_{0}$ que
depende de $\left(\alpha,\mu,T,U\right)$ y $V$ que satisfagan las
ecuaciones (5.1)-(5.5), $Z\left(t\right)=0$, para toda $t\geq
t_{0}$.
\end{Teo}



\begin{Lema}[Lema 5.2 \cite{Gut}]\label{Lema.5.2.Gut}
Sea $\left\{\xi\left(k\right):k\in\ent\right\}$ sucesin de
variables aleatorias i.i.d. con valores en
$\left(0,\infty\right)$, y sea $E\left(t\right)$ el proceso de
conteo
\[E\left(t\right)=max\left\{n\geq1:\xi\left(1\right)+\cdots+\xi\left(n-1\right)\leq t\right\}.\]
Si $E\left[\xi\left(1\right)\right]<\infty$, entonces para
cualquier entero $r\geq1$
\begin{equation}
lim_{t\rightarrow\infty}\esp\left[\left(\frac{E\left(t\right)}{t}\right)^{r}\right]=\left(\frac{1}{E\left[\xi_{1}\right]}\right)^{r}
\end{equation}
de aqu, bajo estas condiciones
\begin{itemize}
\item[a)] Para cualquier $t>0$,
$sup_{t\geq\delta}\esp\left[\left(\frac{E\left(t\right)}{t}\right)^{r}\right]$

\item[b)] Las variables aleatorias
$\left\{\left(\frac{E\left(t\right)}{t}\right)^{r}:t\geq1\right\}$
son uniformemente integrables.
\end{itemize}
\end{Lema}

\begin{Teo}[Teorema 5.1: Ley Fuerte para Procesos de Conteo
\cite{Gut}]\label{Tma.5.1.Gut} Sea
$0<\mu<\esp\left(X_{1}\right]\leq\infty$. entonces

\begin{itemize}
\item[a)] $\frac{N\left(t\right)}{t}\rightarrow\frac{1}{\mu}$
a.s., cuando $t\rightarrow\infty$.


\item[b)]$\esp\left[\frac{N\left(t\right)}{t}\right]^{r}\rightarrow\frac{1}{\mu^{r}}$,
cuando $t\rightarrow\infty$ para todo $r>0$..
\end{itemize}
\end{Teo}


\begin{Prop}[Proposicin 5.1 \cite{DaiSean}]\label{Prop.5.1}
Suponga que los supuestos (A1) y (A2) se cumplen, adems suponga
que el modelo de flujo es estable. Entonces existe $t_{0}>0$ tal
que
\begin{equation}\label{Eq.Prop.5.1}
lim_{|x|\rightarrow\infty}\frac{1}{|x|^{p+1}}\esp_{x}\left[|X\left(t_{0}|x|\right)|^{p+1}\right]=0.
\end{equation}

\end{Prop}


\begin{Prop}[Proposici\'on 5.3 \cite{DaiSean}]
Sea $X$ proceso de estados para la red de colas, y suponga que se
cumplen los supuestos (A1) y (A2), entonces para alguna constante
positiva $C_{p+1}<\infty$, $\delta>0$ y un conjunto compacto
$C\subset X$.

\begin{equation}\label{Eq.5.4}
\esp_{x}\left[\int_{0}^{\tau_{C}\left(\delta\right)}\left(1+|X\left(t\right)|^{p}\right)dt\right]\leq
C_{p+1}\left(1+|x|^{p+1}\right)
\end{equation}
\end{Prop}

\begin{Prop}[Proposici\'on 5.4 \cite{DaiSean}]
Sea $X$ un proceso de Markov Borel Derecho en $X$, sea
$f:X\leftarrow\rea_{+}$ y defina para alguna $\delta>0$, y un
conjunto cerrado $C\subset X$
\[V\left(x\right):=\esp_{x}\left[\int_{0}^{\tau_{C}\left(\delta\right)}f\left(X\left(t\right)\right)dt\right]\]
para $x\in X$. Si $V$ es finito en todas partes y uniformemente
acotada en $C$, entonces existe $k<\infty$ tal que
\begin{equation}\label{Eq.5.11}
\frac{1}{t}\esp_{x}\left[V\left(x\right)\right]+\frac{1}{t}\int_{0}^{t}\esp_{x}\left[f\left(X\left(s\right)\right)ds\right]\leq\frac{1}{t}V\left(x\right)+k,
\end{equation}
para $x\in X$ y $t>0$.
\end{Prop}


\begin{Teo}[Teorema 5.5 \cite{DaiSean}]
Suponga que se cumplen (A1) y (A2), adems suponga que el modelo
de flujo es estable. Entonces existe una constante $k_{p}<\infty$
tal que
\begin{equation}\label{Eq.5.13}
\frac{1}{t}\int_{0}^{t}\esp_{x}\left[|Q\left(s\right)|^{p}\right]ds\leq
k_{p}\left\{\frac{1}{t}|x|^{p+1}+1\right\}
\end{equation}
para $t\geq0$, $x\in X$. En particular para cada condici\'on inicial
\begin{equation}\label{Eq.5.14}
Limsup_{t\rightarrow\infty}\frac{1}{t}\int_{0}^{t}\esp_{x}\left[|Q\left(s\right)|^{p}\right]ds\leq
k_{p}
\end{equation}
\end{Teo}

\begin{Teo}[Teorema 6.2\cite{DaiSean}]\label{Tma.6.2}
Suponga que se cumplen los supuestos (A1)-(A3) y que el modelo de
flujo es estable, entonces se tiene que
\[\parallel P^{t}\left(c,\cdot\right)-\pi\left(\cdot\right)\parallel_{f_{p}}\rightarrow0\]
para $t\rightarrow\infty$ y $x\in X$. En particular para cada
condicin inicial
\[lim_{t\rightarrow\infty}\esp_{x}\left[\left|Q_{t}\right|^{p}\right]=\esp_{\pi}\left[\left|Q_{0}\right|^{p}\right]<\infty\]
\end{Teo}


\begin{Teo}[Teorema 6.3\cite{DaiSean}]\label{Tma.6.3}
Suponga que se cumplen los supuestos (A1)-(A3) y que el modelo de
flujo es estable, entonces con
$f\left(x\right)=f_{1}\left(x\right)$, se tiene que
\[lim_{t\rightarrow\infty}t^{(p-1)\left|P^{t}\left(c,\cdot\right)-\pi\left(\cdot\right)\right|_{f}=0},\]
para $x\in X$. En particular, para cada condicin inicial
\[lim_{t\rightarrow\infty}t^{(p-1)\left|\esp_{x}\left[Q_{t}\right]-\esp_{\pi}\left[Q_{0}\right]\right|=0}.\]
\end{Teo}



\begin{Prop}[Proposici\'on 5.1, Dai y Meyn \cite{DaiSean}]\label{Prop.5.1.DaiSean}
Suponga que los supuestos A1) y A2) son ciertos y que el modelo de flujo es estable. Entonces existe $t_{0}>0$ tal que
\begin{equation}
lim_{|x|\rightarrow\infty}\frac{1}{|x|^{p+1}}\esp_{x}\left[|X\left(t_{0}|x|\right)|^{p+1}\right]=0
\end{equation}
\end{Prop}

\begin{Lemma}[Lema 5.2, Dai y Meyn \cite{DaiSean}]\label{Lema.5.2.DaiSean}
 Sea $\left\{\zeta\left(k\right):k\in \mathbb{z}\right\}$ una sucesi\'on independiente e id\'enticamente distribuida que toma valores en $\left(0,\infty\right)$,
y sea
$E\left(t\right)=max\left(n\geq1:\zeta\left(1\right)+\cdots+\zeta\left(n-1\right)\leq
t\right)$. Si $\esp\left[\zeta\left(1\right)\right]<\infty$,
entonces para cualquier entero $r\geq1$
\begin{equation}
 lim_{t\rightarrow\infty}\esp\left[\left(\frac{E\left(t\right)}{t}\right)^{r}\right]=\left(\frac{1}{\esp\left[\zeta_{1}\right]}\right)^{r}.
\end{equation}
Luego, bajo estas condiciones:
\begin{itemize}
 \item[a)] para cualquier $\delta>0$, $\sup_{t\geq\delta}\esp\left[\left(\frac{E\left(t\right)}{t}\right)^{r}\right]<\infty$
\item[b)] las variables aleatorias
$\left\{\left(\frac{E\left(t\right)}{t}\right)^{r}:t\geq1\right\}$
son uniformemente integrables.
\end{itemize}
\end{Lemma}

\begin{Teo}[Teorema 5.5, Dai y Meyn \cite{DaiSean}]\label{Tma.5.5.DaiSean}
Suponga que los supuestos A1) y A2) se cumplen y que el modelo de
flujo es estable. Entonces existe una constante $\kappa_{p}$ tal
que
\begin{equation}
\frac{1}{t}\int_{0}^{t}\esp_{x}\left[|Q\left(s\right)|^{p}\right]ds\leq\kappa_{p}\left\{\frac{1}{t}|x|^{p+1}+1\right\}
\end{equation}
para $t>0$ y $x\in X$. En particular, para cada condici\'on
inicial
\begin{eqnarray*}
\limsup_{t\rightarrow\infty}\frac{1}{t}\int_{0}^{t}\esp_{x}\left[|Q\left(s\right)|^{p}\right]ds\leq\kappa_{p}.
\end{eqnarray*}
\end{Teo}

\begin{Teo}[Teorema 6.2, Dai y Meyn \cite{DaiSean}]\label{Tma.6.2.DaiSean}
Suponga que se cumplen los supuestos A1), A2) y A3) y que el
modelo de flujo es estable. Entonces se tiene que
\begin{equation}
\left\|P^{t}\left(x,\cdot\right)-\pi\left(\cdot\right)\right\|_{f_{p}}\textrm{,
}t\rightarrow\infty,x\in X.
\end{equation}
En particular para cada condici\'on inicial
\begin{eqnarray*}
\lim_{t\rightarrow\infty}\esp_{x}\left[|Q\left(t\right)|^{p}\right]=\esp_{\pi}\left[|Q\left(0\right)|^{p}\right]\leq\kappa_{r}
\end{eqnarray*}
\end{Teo}
\begin{Teo}[Teorema 6.3, Dai y Meyn \cite{DaiSean}]\label{Tma.6.3.DaiSean}
Suponga que se cumplen los supuestos A1), A2) y A3) y que el
modelo de flujo es estable. Entonces con
$f\left(x\right)=f_{1}\left(x\right)$ se tiene
\begin{equation}
\lim_{t\rightarrow\infty}t^{p-1}\left\|P^{t}\left(x,\cdot\right)-\pi\left(\cdot\right)\right\|_{f}=0.
\end{equation}
En particular para cada condici\'on inicial
\begin{eqnarray*}
\lim_{t\rightarrow\infty}t^{p-1}|\esp_{x}\left[Q\left(t\right)\right]-\esp_{\pi}\left[Q\left(0\right)\right]|=0.
\end{eqnarray*}
\end{Teo}

\begin{Teo}[Teorema 6.4, Dai y Meyn \cite{DaiSean}]\label{Tma.6.4.DaiSean}
Suponga que se cumplen los supuestos A1), A2) y A3) y que el
modelo de flujo es estable. Sea $\nu$ cualquier distribuci\'on de
probabilidad en $\left(X,\mathcal{B}_{X}\right)$, y $\pi$ la
distribuci\'on estacionaria de $X$.
\begin{itemize}
\item[i)] Para cualquier $f:X\leftarrow\rea_{+}$
\begin{equation}
\lim_{t\rightarrow\infty}\frac{1}{t}\int_{o}^{t}f\left(X\left(s\right)\right)ds=\pi\left(f\right):=\int
f\left(x\right)\pi\left(dx\right)
\end{equation}
$\prob$-c.s.

\item[ii)] Para cualquier $f:X\leftarrow\rea_{+}$ con
$\pi\left(|f|\right)<\infty$, la ecuaci\'on anterior se cumple.
\end{itemize}
\end{Teo}

\begin{Teo}[Teorema 2.2, Down \cite{Down}]\label{Tma2.2.Down}
Suponga que el fluido modelo es inestable en el sentido de que
para alguna $\epsilon_{0},c_{0}\geq0$,
\begin{equation}\label{Eq.Inestability}
|Q\left(T\right)|\geq\epsilon_{0}T-c_{0}\textrm{,   }T\geq0,
\end{equation}
para cualquier condici\'on inicial $Q\left(0\right)$, con
$|Q\left(0\right)|=1$. Entonces para cualquier $0<q\leq1$, existe
$B<0$ tal que para cualquier $|x|\geq B$,
\begin{equation}
\prob_{x}\left\{\mathbb{X}\rightarrow\infty\right\}\geq q.
\end{equation}
\end{Teo}


%_________________________________________________________________________
\subsection{Supuestos}
%_________________________________________________________________________
Consideremos el caso en el que se tienen varias colas a las cuales
llegan uno o varios servidores para dar servicio a los usuarios
que se encuentran presentes en la cola, como ya se mencion\'o hay
varios tipos de pol\'iticas de servicio, incluso podr\'ia ocurrir
que la manera en que atiende al resto de las colas sea distinta a
como lo hizo en las anteriores.\\

Para ejemplificar los sistemas de visitas c\'iclicas se
considerar\'a el caso en que a las colas los usuarios son atendidos con
una s\'ola pol\'itica de servicio.\\


Si $\omega$ es el n\'umero de usuarios en la cola al comienzo del
periodo de servicio y $N\left(\omega\right)$ es el n\'umero de
usuarios que son atendidos con una pol\'itica en espec\'ifico
durante el periodo de servicio, entonces se asume que:
\begin{itemize}
\item[1)]\label{S1}$lim_{\omega\rightarrow\infty}\esp\left[N\left(\omega\right)\right]=\overline{N}>0$;
\item[2)]\label{S2}$\esp\left[N\left(\omega\right)\right]\leq\overline{N}$
para cualquier valor de $\omega$.
\end{itemize}
La manera en que atiende el servidor $m$-\'esimo, es la siguiente:
\begin{itemize}
\item Al t\'ermino de la visita a la cola $j$, el servidor cambia
a la cola $j^{'}$ con probabilidad $r_{j,j^{'}}^{m}$

\item La $n$-\'esima vez que el servidor cambia de la cola $j$ a
$j'$, va acompa\~nada con el tiempo de cambio de longitud
$\delta_{j,j^{'}}^{m}\left(n\right)$, con
$\delta_{j,j^{'}}^{m}\left(n\right)$, $n\geq1$, variables
aleatorias independientes e id\'enticamente distribuidas, tales
que $\esp\left[\delta_{j,j^{'}}^{m}\left(1\right)\right]\geq0$.

\item Sea $\left\{p_{j}^{m}\right\}$ la distribuci\'on invariante
estacionaria \'unica para la Cadena de Markov con matriz de
transici\'on $\left(r_{j,j^{'}}^{m}\right)$, se supone que \'esta
existe.

\item Finalmente, se define el tiempo promedio total de traslado
entre las colas como
\begin{equation}
\delta^{*}:=\sum_{j,j^{'}}p_{j}^{m}r_{j,j^{'}}^{m}\esp\left[\delta_{j,j^{'}}^{m}\left(i\right)\right].
\end{equation}
\end{itemize}

Consideremos el caso donde los tiempos entre arribo a cada una de
las colas, $\left\{\xi_{k}\left(n\right)\right\}_{n\geq1}$ son
variables aleatorias independientes a id\'enticamente
distribuidas, y los tiempos de servicio en cada una de las colas
se distribuyen de manera independiente e id\'enticamente
distribuidas $\left\{\eta_{k}\left(n\right)\right\}_{n\geq1}$;
adem\'as ambos procesos cumplen la condici\'on de ser
independientes entre s\'i. Para la $k$-\'esima cola se define la
tasa de arribo por
$\lambda_{k}=1/\esp\left[\xi_{k}\left(1\right)\right]$ y la tasa
de servicio como
$\mu_{k}=1/\esp\left[\eta_{k}\left(1\right)\right]$, finalmente se
define la carga de la cola como $\rho_{k}=\lambda_{k}/\mu_{k}$,
donde se pide que $\rho=\sum_{k=1}^{K}\rho_{k}<1$, para garantizar
la estabilidad del sistema, esto es cierto para las pol\'iticas de
servicio exhaustiva y cerrada, ver Geetor \cite{Getoor}.\\

Si denotamos por
\begin{itemize}
\item $Q_{k}\left(t\right)$ el n\'umero de usuarios presentes en
la cola $k$ al tiempo $t$; \item $A_{k}\left(t\right)$ los
residuales de los tiempos entre arribos a la cola $k$; para cada
servidor $m$; \item $B_{m}\left(t\right)$ denota a los residuales
de los tiempos de servicio al tiempo $t$; \item
$B_{m}^{0}\left(t\right)$ los residuales de los tiempos de
traslado de la cola $k$ a la pr\'oxima por atender al tiempo $t$,

\item sea
$C_{m}\left(t\right)$ el n\'umero de usuarios atendidos durante la
visita del servidor a la cola $k$ al tiempo $t$.
\end{itemize}


En este sentido, el proceso para el sistema de visitas se puede
definir como:

\begin{equation}\label{Esp.Edos.Down}
X\left(t\right)^{T}=\left(Q_{k}\left(t\right),A_{k}\left(t\right),B_{m}\left(t\right),B_{m}^{0}\left(t\right),C_{m}\left(t\right)\right),
\end{equation}
para $k=1,\ldots,K$ y $m=1,2,\ldots,M$, donde $T$ indica que es el
transpuesto del vector que se est\'a definiendo. El proceso $X$
evoluciona en el espacio de estados:
$\mathbb{X}=\ent_{+}^{K}\times\rea_{+}^{K}\times\left(\left\{1,2,\ldots,K\right\}\times\left\{1,2,\ldots,S\right\}\right)^{M}\times\rea_{+}^{K}\times\ent_{+}^{K}$.\\

El sistema aqu\'i descrito debe de cumplir con los siguientes supuestos b\'asicos de un sistema de visitas:
%__________________________________________________________________________
\subsubsection{Supuestos B\'asicos}
%__________________________________________________________________________
\begin{itemize}
\item[A1)] Los procesos
$\xi_{1},\ldots,\xi_{K},\eta_{1},\ldots,\eta_{K}$ son mutuamente
independientes y son sucesiones independientes e id\'enticamente
distribuidas.

\item[A2)] Para alg\'un entero $p\geq1$
\begin{eqnarray*}
\esp\left[\xi_{l}\left(1\right)^{p+1}\right]&<&\infty\textrm{ para }l=1,\ldots,K\textrm{ y }\\
\esp\left[\eta_{k}\left(1\right)^{p+1}\right]&<&\infty\textrm{
para }k=1,\ldots,K.
\end{eqnarray*}
donde $\mathcal{A}$ es la clase de posibles arribos.

\item[A3)] Para cada $k=1,2,\ldots,K$ existe una funci\'on
positiva $q_{k}\left(\cdot\right)$ definida en $\rea_{+}$, y un
entero $j_{k}$, tal que
\begin{eqnarray}
P\left(\xi_{k}\left(1\right)\geq x\right)&>&0\textrm{, para todo }x>0,\\
P\left\{a\leq\sum_{i=1}^{j_{k}}\xi_{k}\left(i\right)\leq
b\right\}&\geq&\int_{a}^{b}q_{k}\left(x\right)dx, \textrm{ }0\leq
a<b.
\end{eqnarray}
\end{itemize}

En lo que respecta al supuesto (A3), en Dai y Meyn \cite{DaiSean}
hacen ver que este se puede sustituir por

\begin{itemize}
\item[A3')] Para el Proceso de Markov $X$, cada subconjunto
compacto del espacio de estados de $X$ es un conjunto peque\~no,
ver definici\'on \ref{Def.Cto.Peq.}.
\end{itemize}

Es por esta raz\'on que con la finalidad de poder hacer uso de
$A3^{'})$ es necesario recurrir a los Procesos de Harris y en
particular a los Procesos Harris Recurrente, ver \cite{Dai,
DaiSean}.
%_______________________________________________________________________
\subsection{Procesos Harris Recurrente}
%_______________________________________________________________________

Por el supuesto (A1) conforme a Davis \cite{Davis}, se puede
definir el proceso de saltos correspondiente de manera tal que
satisfaga el supuesto (A3'), de hecho la demostraci\'on est\'a
basada en la l\'inea de argumentaci\'on de Davis, \cite{Davis},
p\'aginas 362-364.\\

Entonces se tiene un espacio de estados en el cual el proceso $X$
satisface la Propiedad Fuerte de Markov, ver Dai y Meyn
\cite{DaiSean}, dado por

\[\left(\Omega,\mathcal{F},\mathcal{F}_{t},X\left(t\right),\theta_{t},P_{x}\right),\]
adem\'as de ser un proceso de Borel Derecho (Sharpe \cite{Sharpe})
en el espacio de estados medible
$\left(\mathbb{X},\mathcal{B}_\mathbb{X}\right)$. El Proceso
$X=\left\{X\left(t\right),t\geq0\right\}$ tiene trayectorias
continuas por la derecha, est\'a definido en
$\left(\Omega,\mathcal{F}\right)$ y est\'a adaptado a
$\left\{\mathcal{F}_{t},t\geq0\right\}$; la colecci\'on
$\left\{P_{x},x\in \mathbb{X}\right\}$ son medidas de probabilidad
en $\left(\Omega,\mathcal{F}\right)$ tales que para todo $x\in
\mathbb{X}$
\[P_{x}\left\{X\left(0\right)=x\right\}=1,\] y
\[E_{x}\left\{f\left(X\circ\theta_{t}\right)|\mathcal{F}_{t}\right\}=E_{X}\left(\tau\right)f\left(X\right),\]
en $\left\{\tau<\infty\right\}$, $P_{x}$-c.s., con $\theta_{t}$
definido como el operador shift.


Donde $\tau$ es un $\mathcal{F}_{t}$-tiempo de paro
\[\left(X\circ\theta_{\tau}\right)\left(w\right)=\left\{X\left(\tau\left(w\right)+t,w\right),t\geq0\right\},\]
y $f$ es una funci\'on de valores reales acotada y medible, ver \cite{Dai, KaspiMandelbaum}.\\

Sea $P^{t}\left(x,D\right)$, $D\in\mathcal{B}_{\mathbb{X}}$,
$t\geq0$ la probabilidad de transici\'on de $X$ queda definida
como:
\[P^{t}\left(x,D\right)=P_{x}\left(X\left(t\right)\in
D\right).\]


\begin{Def}
Una medida no cero $\pi$ en
$\left(\mathbb{X},\mathcal{B}_{\mathbb{X}}\right)$ es invariante
para $X$ si $\pi$ es $\sigma$-finita y
\[\pi\left(D\right)=\int_{\mathbb{X}}P^{t}\left(x,D\right)\pi\left(dx\right),\]
para todo $D\in \mathcal{B}_{\mathbb{X}}$, con $t\geq0$.
\end{Def}

\begin{Def}
El proceso de Markov $X$ es llamado Harris Recurrente si existe
una medida de probabilidad $\nu$ en
$\left(\mathbb{X},\mathcal{B}_{\mathbb{X}}\right)$, tal que si
$\nu\left(D\right)>0$ y $D\in\mathcal{B}_{\mathbb{X}}$
\[P_{x}\left\{\tau_{D}<\infty\right\}\equiv1,\] cuando
$\tau_{D}=inf\left\{t\geq0:X_{t}\in D\right\}$.
\end{Def}

\begin{Note}
\begin{itemize}
\item[i)] Si $X$ es Harris recurrente, entonces existe una \'unica
medida invariante $\pi$ (Getoor \cite{Getoor}).

\item[ii)] Si la medida invariante es finita, entonces puede
normalizarse a una medida de probabilidad, en este caso al proceso
$X$ se le llama Harris recurrente positivo.


\item[iii)] Cuando $X$ es Harris recurrente positivo se dice que
la disciplina de servicio es estable. En este caso $\pi$ denota la
distribuci\'on estacionaria y hacemos
\[P_{\pi}\left(\cdot\right)=\int_{\mathbf{X}}P_{x}\left(\cdot\right)\pi\left(dx\right),\]
y se utiliza $E_{\pi}$ para denotar el operador esperanza
correspondiente, ver \cite{DaiSean}.
\end{itemize}
\end{Note}

\begin{Def}\label{Def.Cto.Peq.}
Un conjunto $D\in\mathcal{B_{\mathbb{X}}}$ es llamado peque\~no si
existe un $t>0$, una medida de probabilidad $\nu$ en
$\mathcal{B_{\mathbb{X}}}$, y un $\delta>0$ tal que
\[P^{t}\left(x,A\right)\geq\delta\nu\left(A\right),\] para $x\in
D,A\in\mathcal{B_{\mathbb{X}}}$.
\end{Def}

La siguiente serie de resultados vienen enunciados y demostrados
en Dai \cite{Dai}:
\begin{Lema}[Lema 3.1, Dai \cite{Dai}]
Sea $B$ conjunto peque\~no cerrado, supongamos que
$P_{x}\left(\tau_{B}<\infty\right)\equiv1$ y que para alg\'un
$\delta>0$ se cumple que
\begin{equation}\label{Eq.3.1}
\sup\esp_{x}\left[\tau_{B}\left(\delta\right)\right]<\infty,
\end{equation}
donde
$\tau_{B}\left(\delta\right)=inf\left\{t\geq\delta:X\left(t\right)\in
B\right\}$. Entonces, $X$ es un proceso Harris recurrente
positivo.
\end{Lema}

\begin{Lema}[Lema 3.1, Dai \cite{Dai}]\label{Lema.3.}
Bajo el supuesto (A3), el conjunto
$B=\left\{x\in\mathbb{X}:|x|\leq k\right\}$ es un conjunto
peque\~no cerrado para cualquier $k>0$.
\end{Lema}

\begin{Teo}[Teorema 3.1, Dai \cite{Dai}]\label{Tma.3.1}
Si existe un $\delta>0$ tal que
\begin{equation}
lim_{|x|\rightarrow\infty}\frac{1}{|x|}\esp|X^{x}\left(|x|\delta\right)|=0,
\end{equation}
donde $X^{x}$ se utiliza para denotar que el proceso $X$ comienza
a partir de $x$, entonces la ecuaci\'on (\ref{Eq.3.1}) se cumple
para $B=\left\{x\in\mathbb{X}:|x|\leq k\right\}$ con alg\'un
$k>0$. En particular, $X$ es Harris recurrente positivo.
\end{Teo}

Entonces, tenemos que el proceso $X$ es un proceso de Markov que
cumple con los supuestos $A1)$-$A3)$, lo que falta de hacer es
construir el Modelo de Flujo bas\'andonos en lo hasta ahora
presentado.
%_______________________________________________________________________
\subsection{Modelo de Flujo}
%_______________________________________________________________________

Dada una condici\'on inicial $x\in\mathbb{X}$, sea

\begin{itemize}
\item $Q_{k}^{x}\left(t\right)$ la longitud de la cola al tiempo
$t$,

\item $T_{m,k}^{x}\left(t\right)$ el tiempo acumulado, al tiempo
$t$, que tarda el servidor $m$ en atender a los usuarios de la
cola $k$.

\item $T_{m,k}^{x,0}\left(t\right)$ el tiempo acumulado, al tiempo
$t$, que tarda el servidor $m$ en trasladarse a otra cola a partir de la $k$-\'esima.\\
\end{itemize}

Sup\'ongase que la funci\'on
$\left(\overline{Q}\left(\cdot\right),\overline{T}_{m}
\left(\cdot\right),\overline{T}_{m}^{0} \left(\cdot\right)\right)$
para $m=1,2,\ldots,M$ es un punto l\'imite de
\begin{equation}\label{Eq.Punto.Limite}
\left(\frac{1}{|x|}Q^{x}\left(|x|t\right),\frac{1}{|x|}T_{m}^{x}\left(|x|t\right),\frac{1}{|x|}T_{m}^{x,0}\left(|x|t\right)\right)
\end{equation}
para $m=1,2,\ldots,M$, cuando $x\rightarrow\infty$, ver
\cite{Down}. Entonces
$\left(\overline{Q}\left(t\right),\overline{T}_{m}
\left(t\right),\overline{T}_{m}^{0} \left(t\right)\right)$ es un
flujo l\'imite del sistema. Al conjunto de todos las posibles
flujos l\'imite se le llama {\emph{Modelo de Flujo}} y se le
denotar\'a por $\mathcal{Q}$, ver \cite{Down, Dai, DaiSean}.\\

El modelo de flujo satisface el siguiente conjunto de ecuaciones:

\begin{equation}\label{Eq.MF.1}
\overline{Q}_{k}\left(t\right)=\overline{Q}_{k}\left(0\right)+\lambda_{k}t-\sum_{m=1}^{M}\mu_{k}\overline{T}_{m,k}\left(t\right),\\
\end{equation}
para $k=1,2,\ldots,K$.\\
\begin{equation}\label{Eq.MF.2}
\overline{Q}_{k}\left(t\right)\geq0\textrm{ para
}k=1,2,\ldots,K.\\
\end{equation}

\begin{equation}\label{Eq.MF.3}
\overline{T}_{m,k}\left(0\right)=0,\textrm{ y }\overline{T}_{m,k}\left(\cdot\right)\textrm{ es no decreciente},\\
\end{equation}
para $k=1,2,\ldots,K$ y $m=1,2,\ldots,M$.\\
\begin{equation}\label{Eq.MF.4}
\sum_{k=1}^{K}\overline{T}_{m,k}^{0}\left(t\right)+\overline{T}_{m,k}\left(t\right)=t\textrm{
para }m=1,2,\ldots,M.\\
\end{equation}


\begin{Def}[Definici\'on 4.1, Dai \cite{Dai}]\label{Def.Modelo.Flujo}
Sea una disciplina de servicio espec\'ifica. Cualquier l\'imite
$\left(\overline{Q}\left(\cdot\right),\overline{T}\left(\cdot\right),\overline{T}^{0}\left(\cdot\right)\right)$
en (\ref{Eq.Punto.Limite}) es un {\em flujo l\'imite} de la
disciplina. Cualquier soluci\'on (\ref{Eq.MF.1})-(\ref{Eq.MF.4})
es llamado flujo soluci\'on de la disciplina.
\end{Def}

\begin{Def}
Se dice que el modelo de flujo l\'imite, modelo de flujo, de la
disciplina de la cola es estable si existe una constante
$\delta>0$ que depende de $\mu,\lambda$ y $P$ solamente, tal que
cualquier flujo l\'imite con
$|\overline{Q}\left(0\right)|+|\overline{U}|+|\overline{V}|=1$, se
tiene que $\overline{Q}\left(\cdot+\delta\right)\equiv0$.
\end{Def}

Si se hace $|x|\rightarrow\infty$ sin restringir ninguna de las
componentes, tambi\'en se obtienen un modelo de flujo, pero en
este caso el residual de los procesos de arribo y servicio
introducen un retraso:
\begin{Teo}[Teorema 4.2, Dai \cite{Dai}]\label{Tma.4.2.Dai}
Sea una disciplina fija para la cola, suponga que se cumplen las
condiciones (A1)-(A3). Si el modelo de flujo l\'imite de la
disciplina de la cola es estable, entonces la cadena de Markov $X$
que describe la din\'amica de la red bajo la disciplina es Harris
recurrente positiva.
\end{Teo}

Ahora se procede a escalar el espacio y el tiempo para reducir la
aparente fluctuaci\'on del modelo. Consid\'erese el proceso
\begin{equation}\label{Eq.3.7}
\overline{Q}^{x}\left(t\right)=\frac{1}{|x|}Q^{x}\left(|x|t\right).
\end{equation}
A este proceso se le conoce como el flujo escalado, y cualquier
l\'imite $\overline{Q}^{x}\left(t\right)$ es llamado flujo
l\'imite del proceso de longitud de la cola. Haciendo
$|q|\rightarrow\infty$ mientras se mantiene el resto de las
componentes fijas, cualquier punto l\'imite del proceso de
longitud de la cola normalizado $\overline{Q}^{x}$ es soluci\'on
del siguiente modelo de flujo.


\begin{Def}[Definici\'on 3.3, Dai y Meyn \cite{DaiSean}]
El modelo de flujo es estable si existe un tiempo fijo $t_{0}$ tal
que $\overline{Q}\left(t\right)=0$, con $t\geq t_{0}$, para
cualquier $\overline{Q}\left(\cdot\right)\in\mathcal{Q}$ que
cumple con $|\overline{Q}\left(0\right)|=1$.
\end{Def}

\begin{Lemma}[Lema 3.1, Dai y Meyn \cite{DaiSean}]
Si el modelo de flujo definido por (\ref{Eq.MF.1})-(\ref{Eq.MF.4})
es estable, entonces el modelo de flujo retrasado es tambi\'en
estable, es decir, existe $t_{0}>0$ tal que
$\overline{Q}\left(t\right)=0$ para cualquier $t\geq t_{0}$, para
cualquier soluci\'on del modelo de flujo retrasado cuya
condici\'on inicial $\overline{x}$ satisface que
$|\overline{x}|=|\overline{Q}\left(0\right)|+|\overline{A}\left(0\right)|+|\overline{B}\left(0\right)|\leq1$.
\end{Lemma}


Ahora ya estamos en condiciones de enunciar los resultados principales:


\begin{Teo}[Teorema 2.1, Down \cite{Down}]\label{Tma2.1.Down}
Suponga que el modelo de flujo es estable, y que se cumplen los supuestos (A1) y (A2), entonces
\begin{itemize}
\item[i)] Para alguna constante $\kappa_{p}$, y para cada
condici\'on inicial $x\in X$
\begin{equation}\label{Estability.Eq1}
\limsup_{t\rightarrow\infty}\frac{1}{t}\int_{0}^{t}\esp_{x}\left[|Q\left(s\right)|^{p}\right]ds\leq\kappa_{p},
\end{equation}
donde $p$ es el entero dado en (A2).
\end{itemize}
Si adem\'as se cumple la condici\'on (A3), entonces para cada
condici\'on inicial:
\begin{itemize}
\item[ii)] Los momentos transitorios convergen a su estado
estacionario:
 \begin{equation}\label{Estability.Eq2}
lim_{t\rightarrow\infty}\esp_{x}\left[Q_{k}\left(t\right)^{r}\right]=\esp_{\pi}\left[Q_{k}\left(0\right)^{r}\right]\leq\kappa_{r},
\end{equation}
para $r=1,2,\ldots,p$ y $k=1,2,\ldots,K$. Donde $\pi$ es la
probabilidad invariante para $X$.

\item[iii)]  El primer momento converge con raz\'on $t^{p-1}$:
\begin{equation}\label{Estability.Eq3}
lim_{t\rightarrow\infty}t^{p-1}|\esp_{x}\left[Q_{k}\left(t\right)\right]-\esp_{\pi}\left[Q_{k}\left(0\right)\right]|=0.
\end{equation}

\item[iv)] La {\em Ley Fuerte de los grandes n\'umeros} se cumple:
\begin{equation}\label{Estability.Eq4}
lim_{t\rightarrow\infty}\frac{1}{t}\int_{0}^{t}Q_{k}^{r}\left(s\right)ds=\esp_{\pi}\left[Q_{k}\left(0\right)^{r}\right],\textrm{
}\prob_{x}\textrm{-c.s.}
\end{equation}
para $r=1,2,\ldots,p$ y $k=1,2,\ldots,K$.
\end{itemize}
\end{Teo}

La contribuci\'on de Down a la teor\'ia de los {\emph {sistemas de
visitas c\'iclicas}}, es la relaci\'on que hay entre la
estabilidad del sistema con el comportamiento de las medidas de
desempe\~no, es decir, la condici\'on suficiente para poder
garantizar la convergencia del proceso de la longitud de la cola
as\'i como de por los menos los dos primeros momentos adem\'as de
una versi\'on de la Ley Fuerte de los Grandes N\'umeros para los
sistemas de visitas.


\begin{Teo}[Teorema 2.3, Down \cite{Down}]\label{Tma2.3.Down}
Considere el siguiente valor:
\begin{equation}\label{Eq.Rho.1serv}
\rho=\sum_{k=1}^{K}\rho_{k}+max_{1\leq j\leq K}\left(\frac{\lambda_{j}}{\sum_{s=1}^{S}p_{js}\overline{N}_{s}}\right)\delta^{*}
\end{equation}
\begin{itemize}
\item[i)] Si $\rho<1$ entonces la red es estable, es decir, se
cumple el Teorema \ref{Tma2.1.Down}.

\item[ii)] Si $\rho>1$ entonces la red es inestable, es decir, se
cumple el Teorema \ref{Tma2.2.Down}
\end{itemize}
\end{Teo}




Dado el proceso $X=\left\{X\left(t\right),t\geq0\right\}$ definido
en (\ref{Esp.Edos.Down}) que describe la din\'amica del sistema de
visitas c\'iclicas, si $U\left(t\right)$ es el residual de los
tiempos de llegada al tiempo $t$ entre dos usuarios consecutivos y
$V\left(t\right)$ es el residual de los tiempos de servicio al
tiempo $t$ para el usuario que est\'as siendo atendido por el
servidor. Sea $\mathbb{X}$ el espacio de estados que puede tomar
el proceso $X$.


\begin{Lema}[Lema 4.3, Dai\cite{Dai}]\label{Lema.4.3}
Sea $\left\{x_{n}\right\}\subset \mathbf{X}$ con
$|x_{n}|\rightarrow\infty$, conforme $n\rightarrow\infty$. Suponga
que
\[lim_{n\rightarrow\infty}\frac{1}{|x_{n}|}U\left(0\right)=\overline{U}_{k},\]
y
\[lim_{n\rightarrow\infty}\frac{1}{|x_{n}|}V\left(0\right)=\overline{V}_{k}.\]
\begin{itemize}
\item[a)] Conforme $n\rightarrow\infty$ casi seguramente,
\[lim_{n\rightarrow\infty}\frac{1}{|x_{n}|}U^{x_{n}}_{k}\left(|x_{n}|t\right)=\left(\overline{U}_{k}-t\right)^{+}\textrm{, u.o.c.}\]
y
\[lim_{n\rightarrow\infty}\frac{1}{|x_{n}|}V^{x_{n}}_{k}\left(|x_{n}|t\right)=\left(\overline{V}_{k}-t\right)^{+}.\]

\item[b)] Para cada $t\geq0$ fijo,
\[\left\{\frac{1}{|x_{n}|}U^{x_{n}}_{k}\left(|x_{n}|t\right),|x_{n}|\geq1\right\}\]
y
\[\left\{\frac{1}{|x_{n}|}V^{x_{n}}_{k}\left(|x_{n}|t\right),|x_{n}|\geq1\right\}\]
\end{itemize}
son uniformemente convergentes.
\end{Lema}

Sea $e$ es un vector de unos, $C$ es la matriz definida por
\[C_{ik}=\left\{\begin{array}{cc}
1,& S\left(k\right)=i,\\
0,& \textrm{ en otro caso}.\\
\end{array}\right.
\]
Es necesario enunciar el siguiente Teorema que se utilizar\'a para
el Teorema (\ref{Tma.4.2.Dai}):
\begin{Teo}[Teorema 4.1, Dai \cite{Dai}]
Considere una disciplina que cumpla la ley de conservaci\'on, para
casi todas las trayectorias muestrales $\omega$ y cualquier
sucesi\'on de estados iniciales $\left\{x_{n}\right\}\subset
\mathbf{X}$, con $|x_{n}|\rightarrow\infty$, existe una
subsucesi\'on $\left\{x_{n_{j}}\right\}$ con
$|x_{n_{j}}|\rightarrow\infty$ tal que
\begin{equation}\label{Eq.4.15}
\frac{1}{|x_{n_{j}}|}\left(Q^{x_{n_{j}}}\left(0\right),U^{x_{n_{j}}}\left(0\right),V^{x_{n_{j}}}\left(0\right)\right)\rightarrow\left(\overline{Q}\left(0\right),\overline{U},\overline{V}\right),
\end{equation}

\begin{equation}\label{Eq.4.16}
\frac{1}{|x_{n_{j}}|}\left(Q^{x_{n_{j}}}\left(|x_{n_{j}}|t\right),T^{x_{n_{j}}}\left(|x_{n_{j}}|t\right)\right)\rightarrow\left(\overline{Q}\left(t\right),\overline{T}\left(t\right)\right)\textrm{
u.o.c.}
\end{equation}

Adem\'as,
$\left(\overline{Q}\left(t\right),\overline{T}\left(t\right)\right)$
satisface las siguientes ecuaciones:
\begin{equation}\label{Eq.MF.1.3a}
\overline{Q}\left(t\right)=Q\left(0\right)+\left(\alpha
t-\overline{U}\right)^{+}-\left(I-P\right)^{'}M^{-1}\left(\overline{T}\left(t\right)-\overline{V}\right)^{+},
\end{equation}

\begin{equation}\label{Eq.MF.2.3a}
\overline{Q}\left(t\right)\geq0,\\
\end{equation}

\begin{equation}\label{Eq.MF.3.3a}
\overline{T}\left(t\right)\textrm{ es no decreciente y comienza en cero},\\
\end{equation}

\begin{equation}\label{Eq.MF.4.3a}
\overline{I}\left(t\right)=et-C\overline{T}\left(t\right)\textrm{
es no decreciente,}\\
\end{equation}

\begin{equation}\label{Eq.MF.5.3a}
\int_{0}^{\infty}\left(C\overline{Q}\left(t\right)\right)d\overline{I}\left(t\right)=0,\\
\end{equation}

\begin{equation}\label{Eq.MF.6.3a}
\textrm{Condiciones en
}\left(\overline{Q}\left(\cdot\right),\overline{T}\left(\cdot\right)\right)\textrm{
espec\'ificas de la disciplina de la cola,}
\end{equation}
\end{Teo}


Propiedades importantes para el modelo de flujo retrasado:

\begin{Prop}[Proposici\'on 4.2, Dai \cite{Dai}]
 Sea $\left(\overline{Q},\overline{T},\overline{T}^{0}\right)$ un flujo l\'imite de \ref{Eq.Punto.Limite}
 y suponga que cuando $x\rightarrow\infty$ a lo largo de una subsucesi\'on
\[\left(\frac{1}{|x|}Q_{k}^{x}\left(0\right),\frac{1}{|x|}A_{k}^{x}\left(0\right),\frac{1}{|x|}B_{k}^{x}\left(0\right),\frac{1}{|x|}B_{k}^{x,0}\left(0\right)\right)\rightarrow\left(\overline{Q}_{k}\left(0\right),0,0,0\right)\]
para $k=1,\ldots,K$. El flujo l\'imite tiene las siguientes
propiedades, donde las propiedades de la derivada se cumplen donde
la derivada exista:
\begin{itemize}
 \item[i)] Los vectores de tiempo ocupado $\overline{T}\left(t\right)$ y $\overline{T}^{0}\left(t\right)$ son crecientes y continuas con
$\overline{T}\left(0\right)=\overline{T}^{0}\left(0\right)=0$.
\item[ii)] Para todo $t\geq0$
\[\sum_{k=1}^{K}\left[\overline{T}_{k}\left(t\right)+\overline{T}_{k}^{0}\left(t\right)\right]=t.\]
\item[iii)] Para todo $1\leq k\leq K$
\[\overline{Q}_{k}\left(t\right)=\overline{Q}_{k}\left(0\right)+\alpha_{k}t-\mu_{k}\overline{T}_{k}\left(t\right).\]
\item[iv)]  Para todo $1\leq k\leq K$
\[\dot{{\overline{T}}}_{k}\left(t\right)=\rho_{k}\] para $\overline{Q}_{k}\left(t\right)=0$.
\item[v)] Para todo $k,j$
\[\mu_{k}^{0}\overline{T}_{k}^{0}\left(t\right)=\mu_{j}^{0}\overline{T}_{j}^{0}\left(t\right).\]
\item[vi)]  Para todo $1\leq k\leq K$
\[\mu_{k}\dot{{\overline{T}}}_{k}\left(t\right)=l_{k}\mu_{k}^{0}\dot{{\overline{T}}}_{k}^{0}\left(t\right),\] para $\overline{Q}_{k}\left(t\right)>0$.
\end{itemize}
\end{Prop}

\begin{Lema}[Lema 3.1, Chen \cite{Chen}]\label{Lema3.1}
Si el modelo de flujo es estable, definido por las ecuaciones
(3.8)-(3.13), entonces el modelo de flujo retrasado tambi\'en es
estable.
\end{Lema}

\begin{Lema}[Lema 5.2, Gut \cite{Gut}]\label{Lema.5.2.Gut}
Sea $\left\{\xi\left(k\right):k\in\ent\right\}$ sucesi\'on de
variables aleatorias i.i.d. con valores en
$\left(0,\infty\right)$, y sea $E\left(t\right)$ el proceso de
conteo
\[E\left(t\right)=max\left\{n\geq1:\xi\left(1\right)+\cdots+\xi\left(n-1\right)\leq t\right\}.\]
Si $E\left[\xi\left(1\right)\right]<\infty$, entonces para
cualquier entero $r\geq1$
\begin{equation}
lim_{t\rightarrow\infty}\esp\left[\left(\frac{E\left(t\right)}{t}\right)^{r}\right]=\left(\frac{1}{E\left[\xi_{1}\right]}\right)^{r},
\end{equation}
de aqu\'i, bajo estas condiciones
\begin{itemize}
\item[a)] Para cualquier $t>0$,
$sup_{t\geq\delta}\esp\left[\left(\frac{E\left(t\right)}{t}\right)^{r}\right]<\infty$.

\item[b)] Las variables aleatorias
$\left\{\left(\frac{E\left(t\right)}{t}\right)^{r}:t\geq1\right\}$
son uniformemente integrables.
\end{itemize}
\end{Lema}

\begin{Teo}[Teorema 5.1: Ley Fuerte para Procesos de Conteo, Gut
\cite{Gut}]\label{Tma.5.1.Gut} Sea
$0<\mu<\esp\left(X_{1}\right]\leq\infty$. entonces

\begin{itemize}
\item[a)] $\frac{N\left(t\right)}{t}\rightarrow\frac{1}{\mu}$
a.s., cuando $t\rightarrow\infty$.


\item[b)]$\esp\left[\frac{N\left(t\right)}{t}\right]^{r}\rightarrow\frac{1}{\mu^{r}}$,
cuando $t\rightarrow\infty$ para todo $r>0$.
\end{itemize}
\end{Teo}


\begin{Prop}[Proposici\'on 5.1, Dai y Sean \cite{DaiSean}]\label{Prop.5.1}
Suponga que los supuestos (A1) y (A2) se cumplen, adem\'as suponga
que el modelo de flujo es estable. Entonces existe $t_{0}>0$ tal
que
\begin{equation}\label{Eq.Prop.5.1}
lim_{|x|\rightarrow\infty}\frac{1}{|x|^{p+1}}\esp_{x}\left[|X\left(t_{0}|x|\right)|^{p+1}\right]=0.
\end{equation}

\end{Prop}


\begin{Prop}[Proposici\'on 5.3, Dai y Sean \cite{DaiSean}]\label{Prop.5.3.DaiSean}
Sea $X$ proceso de estados para la red de colas, y suponga que se
cumplen los supuestos (A1) y (A2), entonces para alguna constante
positiva $C_{p+1}<\infty$, $\delta>0$ y un conjunto compacto
$C\subset X$.

\begin{equation}\label{Eq.5.4}
\esp_{x}\left[\int_{0}^{\tau_{C}\left(\delta\right)}\left(1+|X\left(t\right)|^{p}\right)dt\right]\leq
C_{p+1}\left(1+|x|^{p+1}\right).
\end{equation}
\end{Prop}

\begin{Prop}[Proposici\'on 5.4, Dai y Sean \cite{DaiSean}]\label{Prop.5.4.DaiSean}
Sea $X$ un proceso de Markov Borel Derecho en $X$, sea
$f:X\leftarrow\rea_{+}$ y defina para alguna $\delta>0$, y un
conjunto cerrado $C\subset X$
\[V\left(x\right):=\esp_{x}\left[\int_{0}^{\tau_{C}\left(\delta\right)}f\left(X\left(t\right)\right)dt\right],\]
para $x\in X$. Si $V$ es finito en todas partes y uniformemente
acotada en $C$, entonces existe $k<\infty$ tal que
\begin{equation}\label{Eq.5.11}
\frac{1}{t}\esp_{x}\left[V\left(x\right)\right]+\frac{1}{t}\int_{0}^{t}\esp_{x}\left[f\left(X\left(s\right)\right)ds\right]\leq\frac{1}{t}V\left(x\right)+k,
\end{equation}
para $x\in X$ y $t>0$.
\end{Prop}


\begin{Teo}[Teorema 5.5, Dai y Sean  \cite{DaiSean}]
Suponga que se cumplen (A1) y (A2), adem\'as suponga que el modelo
de flujo es estable. Entonces existe una constante $k_{p}<\infty$
tal que
\begin{equation}\label{Eq.5.13}
\frac{1}{t}\int_{0}^{t}\esp_{x}\left[|Q\left(s\right)|^{p}\right]ds\leq
k_{p}\left\{\frac{1}{t}|x|^{p+1}+1\right\},
\end{equation}
para $t\geq0$, $x\in X$. En particular para cada condici\'on
inicial
\begin{equation}\label{Eq.5.14}
\limsup_{t\rightarrow\infty}\frac{1}{t}\int_{0}^{t}\esp_{x}\left[|Q\left(s\right)|^{p}\right]ds\leq
k_{p}.
\end{equation}
\end{Teo}

\begin{Teo}[Teorema 6.2 Dai y Sean \cite{DaiSean}]\label{Tma.6.2}
Suponga que se cumplen los supuestos (A1)-(A3) y que el modelo de
flujo es estable, entonces se tiene que
\[\parallel P^{t}\left(x,\cdot\right)-\pi\left(\cdot\right)\parallel_{f_{p}}\rightarrow0,\]
para $t\rightarrow\infty$ y $x\in X$. En particular para cada
condici\'on inicial
\[lim_{t\rightarrow\infty}\esp_{x}\left[\left|Q_{t}\right|^{p}\right]=\esp_{\pi}\left[\left|Q_{0}\right|^{p}\right]<\infty,\]
\end{Teo}

donde

\begin{eqnarray*}
\parallel
P^{t}\left(c,\cdot\right)-\pi\left(\cdot\right)\parallel_{f}=sup_{|g\leq
f|}|\int\pi\left(dy\right)g\left(y\right)-\int
P^{t}\left(x,dy\right)g\left(y\right)|,
\end{eqnarray*}
para $x\in\mathbb{X}$.

\begin{Teo}[Teorema 6.3, Dai y Sean \cite{DaiSean}]\label{Tma.6.3}
Suponga que se cumplen los supuestos (A1)-(A3) y que el modelo de
flujo es estable, entonces con
$f\left(x\right)=f_{1}\left(x\right)$, se tiene que
\[lim_{t\rightarrow\infty}t^{(p-1)}\left|P^{t}\left(c,\cdot\right)-\pi\left(\cdot\right)\right|_{f}=0,\]
para $x\in X$. En particular, para cada condici\'on inicial
\[lim_{t\rightarrow\infty}t^{(p-1)}\left|\esp_{x}\left[Q_{t}\right]-\esp_{\pi}\left[Q_{0}\right]\right|=0.\]
\end{Teo}



\begin{Prop}[Proposici\'on 5.1, Dai y Meyn \cite{DaiSean}]\label{Prop.5.1.DaiSean}
Suponga que los supuestos A1) y A2) son ciertos y que el modelo de
flujo es estable. Entonces existe $t_{0}>0$ tal que
\begin{equation}
lim_{|x|\rightarrow\infty}\frac{1}{|x|^{p+1}}\esp_{x}\left[|X\left(t_{0}|x|\right)|^{p+1}\right]=0.
\end{equation}
\end{Prop}


\begin{Teo}[Teorema 5.5, Dai y Meyn \cite{DaiSean}]\label{Tma.5.5.DaiSean}
Suponga que los supuestos A1) y A2) se cumplen y que el modelo de
flujo es estable. Entonces existe una constante $\kappa_{p}$ tal
que
\begin{equation}
\frac{1}{t}\int_{0}^{t}\esp_{x}\left[|Q\left(s\right)|^{p}\right]ds\leq\kappa_{p}\left\{\frac{1}{t}|x|^{p+1}+1\right\},
\end{equation}
para $t>0$ y $x\in X$. En particular, para cada condici\'on
inicial
\begin{eqnarray*}
\limsup_{t\rightarrow\infty}\frac{1}{t}\int_{0}^{t}\esp_{x}\left[|Q\left(s\right)|^{p}\right]ds\leq\kappa_{p}.
\end{eqnarray*}
\end{Teo}


\begin{Teo}[Teorema 6.4, Dai y Meyn \cite{DaiSean}]\label{Tma.6.4.DaiSean}
Suponga que se cumplen los supuestos A1), A2) y A3) y que el
modelo de flujo es estable. Sea $\nu$ cualquier distribuci\'on de
probabilidad en
$\left(\mathbb{X},\mathcal{B}_{\mathbb{X}}\right)$, y $\pi$ la
distribuci\'on estacionaria de $X$.
\begin{itemize}
\item[i)] Para cualquier $f:X\leftarrow\rea_{+}$
\begin{equation}
\lim_{t\rightarrow\infty}\frac{1}{t}\int_{o}^{t}f\left(X\left(s\right)\right)ds=\pi\left(f\right):=\int
f\left(x\right)\pi\left(dx\right),
\end{equation}
$\prob$-c.s.

\item[ii)] Para cualquier $f:X\leftarrow\rea_{+}$ con
$\pi\left(|f|\right)<\infty$, la ecuaci\'on anterior se cumple.
\end{itemize}
\end{Teo}

\begin{Teo}[Teorema 2.2, Down \cite{Down}]\label{Tma2.2.Down}
Suponga que el fluido modelo es inestable en el sentido de que
para alguna $\epsilon_{0},c_{0}\geq0$,
\begin{equation}\label{Eq.Inestability}
|Q\left(T\right)|\geq\epsilon_{0}T-c_{0}\textrm{,   }T\geq0,
\end{equation}
para cualquier condici\'on inicial $Q\left(0\right)$, con
$|Q\left(0\right)|=1$. Entonces para cualquier $0<q\leq1$, existe
$B<0$ tal que para cualquier $|x|\geq B$,
\begin{equation}
\prob_{x}\left\{\mathbb{X}\rightarrow\infty\right\}\geq q.
\end{equation}
\end{Teo}

\begin{Dem}[Teorema \ref{Tma2.1.Down}] La demostraci\'on de este
teorema se da a continuaci\'on:\\
\begin{itemize}
\item[i)] Utilizando la proposici\'on \ref{Prop.5.3.DaiSean} se
tiene que la proposici\'on \ref{Prop.5.4.DaiSean} es cierta para
$f\left(x\right)=1+|x|^{p}$.

\item[i)] es consecuencia directa del Teorema \ref{Tma.6.2}.

\item[iii)] ver la demostraci\'on dada en Dai y Sean
\cite{DaiSean} p\'aginas 1901-1902.

\item[iv)] ver Dai y Sean \cite{DaiSean} p\'aginas 1902-1903 \'o
\cite{MeynTweedie2}.
\end{itemize}
\end{Dem}

%_____________________________________________________________________
\subsubsection{Modelo de Flujo y Estabilidad}
%_____________________________________________________________________

Para cada $k$ y cada $n$ se define

\numberwithin{equation}{section}
\begin{equation}
\Phi^{k}\left(n\right):=\sum_{i=1}^{n}\phi^{k}\left(i\right).
\end{equation}

suponiendo que el estado inicial de la red es
$x=\left(q,a,b\right)\in X$, entonces para cada $k$

\begin{eqnarray}
E_{k}^{x}\left(t\right):=\max\left\{n\geq0:A_{k}^{x}\left(0\right)+\psi_{k}\left(1\right)+\cdots+\psi_{k}\left(n-1\right)\leq t\right\}\\
S_{k}^{x}\left(t\right):=\max\left\{n\geq0:B_{k}^{x}\left(0\right)+\eta_{k}\left(1\right)+\cdots+\eta_{k}\left(n-1\right)\leq
t\right\}
\end{eqnarray}

Sea $T_{k}^{x}\left(t\right)$ el tiempo acumulado que el servidor
$s\left(k\right)$ ha utilizado en los usuarios de la clase $k$ en
el intervalo $\left[0,t\right]$. Entonces se tienen las siguientes
ecuaciones:

\begin{equation}
Q_{k}^{x}\left(t\right)=Q_{k}^{x}\left(0\right)+E_{k}^{x}\left(t\right)+\sum_{l=1}^{k}\Phi_{k}^{l}S_{l}^{x}\left(T_{l}^{x}\right)-S_{k}^{x}\left(T_{k}^{x}\right)\\
\end{equation}
\begin{equation}
Q^{x}\left(t\right)=\left(Q^{x}_{1}\left(t\right),\ldots,Q^{x}_{K}\left(t\right)\right)^{'}\geq0,\\
\end{equation}
\begin{equation}
T^{x}\left(t\right)=\left(T^{x}_{1}\left(t\right),\ldots,T^{x}_{K}\left(t\right)\right)^{'}\geq0,\textrm{ es no decreciente}\\
\end{equation}
\begin{equation}
I_{i}^{x}\left(t\right)=t-\sum_{k\in C_{i}}T_{k}^{x}\left(t\right)\textrm{ es no decreciente}\\
\end{equation}
\begin{equation}
\int_{0}^{\infty}\sum_{k\in C_{i}}Q_{k}^{x}\left(t\right)dI_{i}^{x}\left(t\right)=0\\
\end{equation}
\begin{equation}
\textrm{condiciones adicionales sobre
}\left(Q^{x}\left(\cdot\right),T^{x}\left(\cdot\right)\right)\textrm{
referentes a la disciplina de servicio}
\end{equation}

Para reducir la fluctuaci\'on del modelo se escala tanto el
espacio como el tiempo, entonces se tiene el proceso:

\begin{equation}
\overline{Q}^{x}\left(t\right)=\frac{1}{|x|}Q^{x}\left(|x|t\right)
\end{equation}
Cualquier l\'imite $\overline{Q}\left(t\right)$ es llamado un
flujo l\'imite del proceso longitud de la cola. Si se hace
$|q|\rightarrow\infty$ y se mantienen las componentes restantes
fijas, de la condici\'on inicial $x$, cualquier punto l\'imite del
proceso normalizado $\overline{Q}^{x}$ es una soluci\'on del
siguiente modelo de flujo, ver \cite{Dai}.

\begin{Def}
Un flujo l\'imite (retrasado) para una red bajo una disciplina de
servicio espec\'ifica se define como cualquier soluci\'on
 $\left(Q^{x}\left(\cdot\right),T^{x}\left(\cdot\right)\right)$ de las siguientes ecuaciones, donde
$\overline{Q}\left(t\right)=\left(\overline{Q}_{1}\left(t\right),\ldots,\overline{Q}_{K}\left(t\right)\right)^{'}$
y
$\overline{T}\left(t\right)=\left(\overline{T}_{1}\left(t\right),\ldots,\overline{T}_{K}\left(t\right)\right)^{'}$
\begin{equation}\label{Eq.3.8}
\overline{Q}_{k}\left(t\right)=\overline{Q}_{k}\left(0\right)+\alpha_{k}t-\mu_{k}\overline{T}_{k}\left(t\right)+\sum_{l=1}^{k}P_{lk}\mu_{l}\overline{T}_{l}\left(t\right)\\
\end{equation}
\begin{equation}\label{Eq.3.9}
\overline{Q}_{k}\left(t\right)\geq0\textrm{ para }k=1,2,\ldots,K,\\
\end{equation}
\begin{equation}\label{Eq.3.10}
\overline{T}_{k}\left(0\right)=0,\textrm{ y }\overline{T}_{k}\left(\cdot\right)\textrm{ es no decreciente},\\
\end{equation}
\begin{equation}\label{Eq.3.11}
\overline{I}_{i}\left(t\right)=t-\sum_{k\in C_{i}}\overline{T}_{k}\left(t\right)\textrm{ es no decreciente}\\
\end{equation}
\begin{equation}\label{Eq.3.12}
\overline{I}_{i}\left(\cdot\right)\textrm{ se incrementa al tiempo}t\textrm{ cuando }\sum_{k\in C_{i}}Q_{k}^{x}\left(t\right)dI_{i}^{x}\left(t\right)=0\\
\end{equation}
\begin{equation}\label{Eq.3.13}
\textrm{condiciones adicionales sobre
}\left(Q^{x}\left(\cdot\right),T^{x}\left(\cdot\right)\right)\textrm{
referentes a la disciplina de servicio}
\end{equation}
\end{Def}

Al conjunto de ecuaciones dadas en \ref{Eq.3.8}-\ref{Eq.3.13} se
le llama {\em Modelo de flujo} y al conjunto de todas las
soluciones del modelo de flujo
$\left(\overline{Q}\left(\cdot\right),\overline{T}
\left(\cdot\right)\right)$ se le denotar\'a por $\mathcal{Q}$.

Si se hace $|x|\rightarrow\infty$ sin restringir ninguna de las
componentes, tambi\'en se obtienen un modelo de flujo, pero en
este caso el residual de los procesos de arribo y servicio
introducen un retraso:

\begin{Def}
El modelo de flujo retrasado de una disciplina de servicio en una
red con retraso
$\left(\overline{A}\left(0\right),\overline{B}\left(0\right)\right)\in\rea_{+}^{K+|A|}$
se define como el conjunto de ecuaciones dadas en
\ref{Eq.3.8}-\ref{Eq.3.13}, junto con la condici\'on:
\begin{equation}\label{CondAd.FluidModel}
\overline{Q}\left(t\right)=\overline{Q}\left(0\right)+\left(\alpha
t-\overline{A}\left(0\right)\right)^{+}-\left(I-P^{'}\right)M\left(\overline{T}\left(t\right)-\overline{B}\left(0\right)\right)^{+}
\end{equation}
\end{Def}

\begin{Def}
El modelo de flujo es estable si existe un tiempo fijo $t_{0}$ tal
que $\overline{Q}\left(t\right)=0$, con $t\geq t_{0}$, para
cualquier $\overline{Q}\left(\cdot\right)\in\mathcal{Q}$ que
cumple con $|\overline{Q}\left(0\right)|=1$.
\end{Def}

El siguiente resultado se encuentra en \cite{Chen}.
\begin{Lemma}
Si el modelo de flujo definido por \ref{Eq.3.8}-\ref{Eq.3.13} es
estable, entonces el modelo de flujo retrasado es tambi\'en
estable, es decir, existe $t_{0}>0$ tal que
$\overline{Q}\left(t\right)=0$ para cualquier $t\geq t_{0}$, para
cualquier soluci\'on del modelo de flujo retrasado cuya
condici\'on inicial $\overline{x}$ satisface que
$|overline{x}|=|\overline{Q}\left(0\right)|+|\overline{A}\left(0\right)|+|\overline{B}\left(0\right)|\leq1$.
\end{Lemma}

%_____________________________________________________________________
\subsubsection{Resultados principales}
%_____________________________________________________________________
Supuestos necesarios sobre la red

\begin{Sup}
\begin{itemize}
\item[A1)] $\psi_{1},\ldots,\psi_{K},\eta_{1},\ldots,\eta_{K}$ son
mutuamente independientes y son sucesiones independientes e
id\'enticamente distribuidas.

\item[A2)] Para alg\'un entero $p\geq1$
\begin{eqnarray*}
\esp\left[\psi_{l}\left(1\right)^{p+1}\right]<\infty\textrm{ para }l\in\mathcal{A}\textrm{ y }\\
\esp\left[\eta_{k}\left(1\right)^{p+1}\right]<\infty\textrm{ para
}k=1,\ldots,K.
\end{eqnarray*}
\item[A3)] El conjunto $\left\{x\in X:|x|=0\right\}$ es un
singleton, y para cada $k\in\mathcal{A}$, existe una funci\'on
positiva $q_{k}\left(x\right)$ definida en $\rea_{+}$, y un entero
$j_{k}$, tal que
\begin{eqnarray}
P\left(\psi_{k}\left(1\right)\geq x\right)>0\textrm{, para todo }x>0\\
P\left(\psi_{k}\left(1\right)+\ldots\psi_{k}\left(j_{k}\right)\in dx\right)\geq q_{k}\left(x\right)dx0\textrm{ y }\\
\int_{0}^{\infty}q_{k}\left(x\right)dx>0
\end{eqnarray}
\end{itemize}
\end{Sup}

El argumento dado en \cite{MaynDown} en el lema
\ref{Lema.34.MeynDown} se puede aplicar para deducir que todos los
subconjuntos compactos de $X$ son peque\~nos.Entonces la
condici\'on $A3)$ se puede generalizar a
\begin{itemize}
\item[A3')] Para el proceso de Markov $X$, cada subconjunto
compacto de $X$ es peque\~no.
\end{itemize}

\begin{Teo}\label{Tma.4.1}
Suponga que el modelo de flujo para una disciplina de servicio es
estable, y suponga adem\'as que las condiciones A1) y A2) se
satisfacen. Entonces:
\begin{itemize}
\item[i)] Para alguna constante $\kappa_{p}$, y para cada
condici\'on inicial $x\in X$
\begin{equation}
\limsup_{t\rightarrow\infty}\frac{1}{t}\int_{0}^{t}\esp_{x}\left[|Q\left(t\right)|^{p}\right]ds\leq\kappa_{p}
\end{equation}
donde $p$ es el entero dado por A2). Suponga adem\'as que A3) o A3')
se cumple, entonces la disciplina de servicio es estable y adem\'as
para cada condici\'on inicial se tiene lo siguiente: \item[ii)] Los
momentos transitorios convergen a sus valores en estado
estacionario:
\begin{equation}
\lim_{t\rightarrow\infty}\esp_{x}\left[Q_{k}\left(t\right)^{r}\right]=\esp_{\pi}\left[Q_{k}\left(0\right)^{r}\right]\leq\kappa_{r}
\end{equation}
para $r=1,\ldots,p$ y $k=1,\ldots,K$. \item[iii)] EL primer
momento converge con raz\'on $t^{p-1}$:
\begin{equation}
\lim_{t\rightarrow\infty}t^{p-1}|\esp_{x}\left[Q\left(t\right)\right]-\esp_{\pi}\left[Q\left(0\right)\right]|=0.
\end{equation}
\item[iv)] Se cumple la Ley Fuerte de los Grandes N\'umeros:
\begin{equation}
\lim_{t\rightarrow\infty}\frac{1}{t}\int_{0}^{t}Q_{k}^{r}\left(s\right)ds=\esp_{\pi}\left[Q_{k}\left(0\right)^{r}\right]
\end{equation}
$\prob$-c.s., para $r=1,\ldots,p$ y $k=1,\ldots,K$.
\end{itemize}
\end{Teo}
\begin{Dem}
La demostraci\'on de este resultado se da aplicando los teoremas
\ref{Tma.5.5}, \ref{Tma.6.2}, \ref{Tma.6.3} y \ref{Tma.6.4}
\end{Dem}

%_____________________________________________________________________
\subsubsection{Definiciones Generales}
%_____________________________________________________________________
Definimos un proceso de estados para la red que depende de la
pol\'itica de servicio utilizada. Bajo cualquier {\em preemptive
buffer priority} disciplina de servicio, el estado
$\mathbb{X}\left(t\right)$ a cualquier tiempo $t$ puede definirse
como
\begin{equation}\label{Eq.Esp.Estados}
\mathbb{X}\left(t\right)=\left(Q_{k}\left(t\right),A_{l}\left(t\right),B_{k}\left(t\right):k=1,2,\ldots,K,l\in\mathcal{A}\right)
\end{equation}
donde $Q_{k}\left(t\right)$ es la longitud de la cola para los
usuarios de la clase $k$, incluyendo aquellos que est\'an siendo
atendidos, $B_{k}\left(t\right)$ son los tiempos de servicio
residuales para los usuarios de la clase $k$ que est\'an en
servicio. Los tiempos de arribo residuales, que son iguales al
tiempo que queda hasta que el pr\'oximo usuario de la clase $k$
llega, se denotan por $A_{k}\left(t\right)$. Tanto
$B_{k}\left(t\right)$ como $A_{k}\left(t\right)$ se suponen
continuos por la derecha.

Sea $\mathbb{X}$ el espacio de estados para el proceso de estados
que por definici\'on es igual  al conjunto de posibles valores
para el estado $\mathbb{X}\left(t\right)$, y sea
$x=\left(q,a,b\right)$ un estado gen\'erico en $\mathbb{X}$, la
componente $q$ determina la posici\'on del usuario en la red,
$|q|$ denota la longitud total de la cola en la red.

Para un estado $x=\left(q,a,b\right)\in\mathbb{X}$ definimos la
{\em norma} de $x$ como $\left\|x\right|=|q|+|a|+|b|$. En
\cite{Dai} se muestra que para una amplia serie de disciplinas de
servicio el proceso $\mathbb{X}$ es un Proceso Fuerte de Markov, y
por tanto se puede asumir que
\[\left(\left(\Omega,\mathcal{F}\right),\mathcal{F}_{t},\mathbb{X}\left(t\right),\theta_{t},P_{x}\right)\]
es un proceso de Borel Derecho en el espacio de estadio medible
$\left(\mathbb{X},\mathcal{B}_{\mathbb{X}}\right)$. El Proceso
$X=\left\{\mathbb{X}\left(t\right),t\geq0\right\}$ tiene
trayectorias continuas por la derecha, est definida en
$\left(\Omega,\mathcal{F}\right)$ y est adaptado a
$\left\{\mathcal{F}_{t},t\geq0\right\}$; $\left\{P_{x},x\in
X\right\}$ son medidas de probabilidad en
$\left(\Omega,\mathcal{F}\right)$ tales que para todo $x\in X$
\[P_{x}\left\{\mathbb{X}\left(0\right)=x\right\}=1\] y
\[E_{x}\left\{f\left(X\circ\theta_{t}\right)|\mathcal{F}_{t}\right\}=E_{X}\left(\tau\right)f\left(X\right)\]
en $\left\{\tau<\infty\right\}$, $P_{x}$-c.s. Donde $\tau$ es un
$\mathcal{F}_{t}$-tiempo de paro
\[\left(X\circ\theta_{\tau}\right)\left(w\right)=\left\{\mathbb{X}\left(\tau\left(w\right)+t,w\right),t\geq0\right\}\]
y $f$ es una funci\'on de valores reales acotada y medible con la
sigma algebra de Kolmogorov generada por los cilindros.

Sea $P^{t}\left(x,D\right)$, $D\in\mathcal{B}_{\mathbb{X}}$,
$t\geq0$ probabilidad de transici\'on de $X$ definida como
\[P^{t}\left(x,D\right)=P_{x}\left(\mathbb{X}\left(t\right)\in
D\right)\]

\begin{Def}
Una medida no cero $\pi$ en
$\left(\mathbb{X},\mathcal{B}_{\mathbb{X}}\right)$ es {\em
invariante} para $X$ si $\pi$ es $\sigma$-finita y
\[\pi\left(D\right)=\int_{X}P^{t}\left(x,D\right)\pi\left(dx\right)\]
para todo $D\in \mathcal{B}_{\mathbb{X}}$, con $t\geq0$.
\end{Def}

\begin{Def}
El proceso de Markov $X$ es llamado {\em Harris recurrente} si
existe una medida de probabilidad $\nu$ en
$\left(\mathbb{X},\mathcal{B}_{\mathbb{X}}\right)$, tal que si
$\nu\left(D\right)>0$ y $D\in\mathcal{B}_{\mathbb{X}}$
\[P_{x}\left\{\tau_{D}<\infty\right\}\equiv1\] cuando
$\tau_{D}=\inf\left\{t\geq0:\mathbb{X}_{t}\in D\right\}$.
\end{Def}

\begin{itemize}
\item Si $X$ es Harris recurrente, entonces una \'unica medida
invariante $\pi$ existe (\cite{Getoor}). \item Si la medida
invariante es finita, entonces puede normalizarse a una medida de
probabilidad, en este caso se le llama {\em Harris recurrente
positiva}. \item Cuando $X$ es Harris recurrente positivo se dice
que la disciplina de servicio es estable. En este caso $\pi$
denota la ditribuci\'on estacionaria y hacemos
\[P_{\pi}\left(\cdot\right)[=\int_{X}P_{x}\left(\cdot\right)\pi\left(dx\right)\]
y se utiliza $E_{\pi}$ para denotar el operador esperanza
correspondiente, as, el proceso
$X=\left\{\mathbb{X}\left(t\right),t\geq0\right\}$ es un proceso
estrictamente estacionario bajo $P_{\pi}$
\end{itemize}

\begin{Def}
Un conjunto $D\in\mathcal{B}_\mathbb{X}$ es llamado peque\~no si
existe un $t>0$, una medida de probabilidad $\nu$ en
$\mathcal{B}_\mathbb{X}$, y un $\delta>0$ tal que
\[P^{t}\left(x,A\right)\geq\delta\nu\left(A\right)\] para $x\in
D,A\in\mathcal{B}_\mathbb{X}$.\footnote{En \cite{MeynTweedie}
muestran que si $P_{x}\left\{\tau_{D}<\infty\right\}\equiv1$
solamente para uno conjunto peque\~no, entonces el proceso es
Harris recurrente}
\end{Def}

%_____________________________________________________________________
\subsubsection{Definiciones y Descripci\'on del Modelo}
%________________________________________________________________________
El modelo est\'a compuesto por $c$ colas de capacidad infinita,
etiquetadas de $1$ a $c$ las cuales son atendidas por $s$
servidores. Los servidores atienden de acuerdo a una cadena de
Markov independiente $\left(X^{i}_{n}\right)_{n}$ con $1\leq i\leq
s$ y $n\in\left\{1,2,\ldots,c\right\}$ con la misma matriz de
transici\'on $r_{k,l}$ y \'unica medida invariante
$\left(p_{k}\right)$. Cada servidor permanece atendiendo en la
cola un periodo llamado de visita y determinada por la pol\'itica de
servicio asignada a la cola.

Los usuarios llegan a la cola $k$ con una tasa $\lambda_{k}$ y son
atendidos a una raz\'on $\mu_{k}$. Las sucesiones de tiempos de
interarribo $\left(\tau_{k}\left(n\right)\right)_{n}$, la de
tiempos de servicio
$\left(\sigma_{k}^{i}\left(n\right)\right)_{n}$ y la de tiempos de
cambio $\left(\sigma_{k,l}^{0,i}\left(n\right)\right)_{n}$
requeridas en la cola $k$ para el servidor $i$ son sucesiones
independientes e id\'enticamente distribuidas con distribuci\'on
general independiente de $i$, con media
$\sigma_{k}=\frac{1}{\mu_{k}}$, respectivamente
$\sigma_{k,l}^{0}=\frac{1}{\mu_{k,l}^{0}}$, e independiente de las
cadenas de Markov $\left(X^{i}_{n}\right)_{n}$. Adem\'as se supone
que los tiempos de interarribo se asume son acotados, para cada
$\rho_{k}=\lambda_{k}\sigma_{k}<s$ para asegurar la estabilidad de
la cola $k$ cuando opera como una cola $M/GM/1$.
%________________________________________________________________________
\subsubsection{Pol\'iticas de Servicio}
%_____________________________________________________________________
Una pol\'itica de servicio determina el n\'umero de usuarios que ser\'an
atendidos sin interrupci\'on en periodo de servicio por los
servidores que atienden a la cola. Para un solo servidor esta se
define a trav\'es de una funci\'on $f$ donde $f\left(x,a\right)$ es el
n\'umero de usuarios que son atendidos sin interrupci\'on cuando el
servidor llega a la cola y encuentra $x$ usuarios esperando dado
el tiempo transcurrido de interarribo $a$. Sea $v\left(x,a\right)$
la duraci\'on del periodo de servicio para una sola condici\'on
inicial $\left(x,a\right)$.

Las pol\'iticas de servicio consideradas satisfacen las siguientes
propiedades:

\begin{itemize}
\item[i)] Hay conservaci\'on del trabajo, es decir
\[v\left(x,a\right)=\sum_{l=1}^{f\left(x,a\right)}\sigma\left(l\right)\]
con $f\left(0,a\right)=v\left(0,a\right)=0$, donde
$\left(\sigma\left(l\right)\right)_{l}$ es una sucesi\'on
independiente e id\'enticamente distribuida de los tiempos de
servicio solicitados. \item[ii)] La selecci\'on de usuarios para se
atendidos es independiente de sus correspondientes tiempos de
servicio y del pasado hasta el inicio del periodo de servicio. As\'i
las distribuci\'on $\left(f,v\right)$ no depende del orden en el
cu\'al son atendidos los usuarios. \item[iii)] La pol\'itica de
servicio es mon\'otona en el sentido de que para cada $a\geq0$ los
n\'umeros $f\left(x,a\right)$ son mon\'otonos en distribuci\'on en $x$ y
su l\'imite en distribuci\'on cuando $x\rightarrow\infty$ es una
variable aleatoria $F^{*0}$ que no depende de $a$. \item[iv)] El
n\'umero de usuarios atendidos por cada servidor es acotado por
$f^{min}\left(x\right)$ de la longitud de la cola $x$ que adem\'as
converge mon\'otonamente en distribuci\'on a $F^{*}$ cuando
$x\rightarrow\infty$
\end{itemize}
%________________________________________________________________________
\subsubsection{Proceso de Estados}
%_____________________________________________________________________
El sistema de colas se describe por medio del proceso de Markov
$\left(X\left(t\right)\right)_{t\in\rea}$ como se define a
continuaci\'on. El estado del sistema al tiempo $t\geq0$ est\'a dado
por
\[X\left(t\right)=\left(Q\left(t\right),P\left(t\right),A\left(t\right),R\left(t\right),C\left(t\right)\right)\]
donde
\begin{itemize}
\item
$Q\left(t\right)=\left(Q_{k}\left(t\right)\right)_{k=1}^{c}$,
n\'umero de usuarios en la cola $k$ al tiempo $t$. \item
$P\left(t\right)=\left(P^{i}\left(t\right)\right)_{i=1}^{s}$, es
la posici\'on del servidor $i$. \item
$A\left(t\right)=\left(A_{k}\left(t\right)\right)_{k=1}^{c}$, es
el residual del tiempo de arribo en la cola $k$ al tiempo $t$.
\item
$R\left(t\right)=\left(R_{k}^{i}\left(t\right),R_{k,l}^{0,i}\left(t\right)\right)_{k,l,i=1}^{c,c,s}$,
el primero es el residual del tiempo de servicio del usuario
atendido por servidor $i$ en la cola $k$ al tiempo $t$, la segunda
componente es el residual del tiempo de cambio del servidor $i$ de
la cola $k$ a la cola $l$ al tiempo $t$. \item
$C\left(t\right)=\left(C_{k}^{i}\left(t\right)\right)_{k,i=1}^{c,s}$,
es la componente correspondiente a la cola $k$ y al servidor $i$
que est\'a determinada por la pol\'itica de servicio en la cola $k$
y que hace al proceso $X\left(t\right)$ un proceso de Markov.
\end{itemize}
Todos los procesos definidos arriba se suponen continuos por la
derecha.

El proceso $X$ tiene la propiedad fuerte de Markov y su espacio de
estados es el espacio producto
\[\mathcal{X}=\nat^{c}\times E^{s}\times \rea_{+}^{c}\times\rea_{+}^{cs}\times\rea_{+}^{c^{2}s}\times \mathcal{C}\] donde $E=\left\{1,2,\ldots,c\right\}^{2}\cup\left\{1,2,\ldots,c\right\}$ y $\mathcal{C}$  depende de las pol\'iticas de servicio.

%_____________________________________________________________________________________
\subsubsection{Introducci{\'o}n}
%_____________________________________________________________________________________
%


Si $x$ es el n{\'u}mero de usuarios en la cola al comienzo del
periodo de servicio y $N_{s}\left(x\right)=N\left(x\right)$ es el
n{\'u}mero de usuarios que son atendidos con la pol{\'\i}tica $s$,
{\'u}nica en nuestro caso durante un periodo de servicio, entonces
se asume que:
\begin{enumerate}
\item
\begin{equation}\label{S1}
lim_{x\rightarrow\infty}\esp\left[N\left(x\right)\right]=\overline{N}>0
\end{equation}
\item
\begin{equation}\label{S2}
\esp\left[N\left(x\right)\right]\leq \overline{N} \end{equation}
para cualquier valor de $x$.
\end{enumerate}
La manera en que atiende el servidor $m$-{\'e}simo, en este caso
en espec{\'\i}fico solo lo ilustraremos con un s{\'o}lo servidor,
es la siguiente:
\begin{itemize}
\item Al t{\'e}rmino de la visita a la cola $j$, el servidor se
cambia a la cola $j^{'}$ con probabilidad
$r_{j,j^{'}}^{m}=r_{j,j^{'}}$

\item La $n$-{\'e}sima ocurencia va acompa{\~n}ada con el tiempo
de cambio de longitud $\delta_{j,j^{'}}\left(n\right)$,
independientes e id{\'e}nticamente distribuidas, con
$\esp\left[\delta_{j,j^{'}}\left(1\right)\right]\geq0$.

\item Sea $\left\{p_{j}\right\}$ la {\'u}nica distribuci{\'o}n
invariante estacionaria para la Cadena de Markov con matriz de
transici{\'o}n $\left(r_{j,j^{'}}\right)$.

\item Finalmente, se define
\begin{equation}
\delta^{*}:=\sum_{j,j^{'}}p_{j}r_{j,j^{'}}\esp\left[\delta_{j,j^{'}}\left(1\right)\right].
\end{equation}
\end{itemize}
%_____________________________________________________________________
\subsubsection{Colas C\'iclicas}
%_____________________________________________________________________
El {\em token passing ring} es una estaci\'on de un solo servidor
con $K$ clases de usuarios. Cada clase tiene su propio regulador
en la estaci\'on. Los usuarios llegan al regulador con raz\'on
$\alpha_{k}$ y son atendidos con taza $\mu_{k}$.

La red se puede modelar como un Proceso de Markov con espacio de
estados continuo, continuo en el tiempo:
\begin{equation}
 X\left(t\right)^{T}=\left(Q_{k}\left(t\right),A_{l}\left(t\right),B_{k}\left(t\right),B_{k}^{0}\left(t\right),C\left(t\right):k=1,\ldots,K,l\in\mathcal{A}\right)
\end{equation}
donde $Q_{k}\left(t\right), B_{k}\left(t\right)$ y
$A_{k}\left(t\right)$ se define como en \ref{Eq.Esp.Estados},
$B_{k}^{0}\left(t\right)$ es el tiempo residual de cambio de la
clase $k$ a la clase $k+1\left(mod K\right)$; $C\left(t\right)$
indica el n\'umero de servicios que han sido comenzados y/o
completados durante la sesi\'on activa del buffer.

Los par\'ametros cruciales son la carga nominal de la cola $k$:
$\beta_{k}=\alpha_{k}/\mu_{k}$ y la carga total es
$\rho_{0}=\sum\beta_{k}$, la media total del tiempo de cambio en
un ciclo del token est\'a definido por
\begin{equation}
 u^{0}=\sum_{k=1}^{K}\esp\left[\eta_{k}^{0}\left(1\right)\right]=\sum_{k=1}^{K}\frac{1}{\mu_{k}^{0}}
\end{equation}

El proceso de la longitud de la cola $Q_{k}^{x}\left(t\right)$ y
el proceso de acumulaci\'on del tiempo de servicio
$T_{k}^{x}\left(t\right)$ para el buffer $k$ y para el estado
inicial $x$ se definen como antes. Sea $T_{k}^{x,0}\left(t\right)$
el tiempo acumulado al tiempo $t$ que el token tarda en cambiar
del buffer $k$ al $k+1\mod K$. Suponga que la funci\'on
$\left(\overline{Q}\left(\cdot\right),\overline{T}\left(\cdot\right),\overline{T}^{0}\left(\cdot\right)\right)$
es un punto l\'imite de
\begin{equation}\label{Eq.4.4}
\left(\frac{1}{|x|}Q^{x}\left(|x|t\right),\frac{1}{|x|}T^{x}\left(|x|t\right),\frac{1}{|x|}T^{x,0}\left(|x|t\right)\right)
\end{equation}
cuando $|x|\rightarrow\infty$. Entonces
$\left(\overline{Q}\left(t\right),\overline{T}\left(t\right),\overline{T}^{0}\left(t\right)\right)$
es un flujo l\'imite retrasado del token ring.

Propiedades importantes para el modelo de flujo retrasado

\begin{Prop}
 Sea $\left(\overline{Q},\overline{T},\overline{T}^{0}\right)$ un flujo l\'imite de \ref{Eq.4.4} y suponga que cuando $x\rightarrow\infty$ a lo largo de
una subsucesi\'on
\[\left(\frac{1}{|x|}Q_{k}^{x}\left(0\right),\frac{1}{|x|}A_{k}^{x}\left(0\right),\frac{1}{|x|}B_{k}^{x}\left(0\right),\frac{1}{|x|}B_{k}^{x,0}\left(0\right)\right)\rightarrow\left(\overline{Q}_{k}\left(0\right),0,0,0\right)\]
para $k=1,\ldots,K$. EL flujo l\'imite tiene las siguientes
propiedades, donde las propiedades de la derivada se cumplen donde
la derivada exista:
\begin{itemize}
 \item[i)] Los vectores de tiempo ocupado $\overline{T}\left(t\right)$ y $\overline{T}^{0}\left(t\right)$ son crecientes y continuas con
$\overline{T}\left(0\right)=\overline{T}^{0}\left(0\right)=0$.
\item[ii)] Para todo $t\geq0$
\[\sum_{k=1}^{K}\left[\overline{T}_{k}\left(t\right)+\overline{T}_{k}^{0}\left(t\right)\right]=t\]
\item[iii)] Para todo $1\leq k\leq K$
\[\overline{Q}_{k}\left(t\right)=\overline{Q}_{k}\left(0\right)+\alpha_{k}t-\mu_{k}\overline{T}_{k}\left(t\right)\]
\item[iv)]  Para todo $1\leq k\leq K$
\[\dot{{\overline{T}}}_{k}\left(t\right)=\beta_{k}\] para $\overline{Q}_{k}\left(t\right)=0$.
\item[v)] Para todo $k,j$
\[\mu_{k}^{0}\overline{T}_{k}^{0}\left(t\right)=\mu_{j}^{0}\overline{T}_{j}^{0}\left(t\right)\]
\item[vi)]  Para todo $1\leq k\leq K$
\[\mu_{k}\dot{{\overline{T}}}_{k}\left(t\right)=l_{k}\mu_{k}^{0}\dot{{\overline{T}}}_{k}^{0}\left(t\right)\] para $\overline{Q}_{k}\left(t\right)>0$.
\end{itemize}
\end{Prop}

%_____________________________________________________________________
\subsubsection{Resultados Previos}
%_____________________________________________________________________

\begin{Lemma}\label{Lema.34.MeynDown}
El proceso estoc\'astico $\Phi$ es un proceso de markov fuerte,
temporalmente homog\'eneo, con trayectorias muestrales continuas
por la derecha, cuyo espacio de estados $Y$ es igual a
$X\times\rea$
\end{Lemma}
\begin{Prop}
 Suponga que los supuestos A1) y A2) son ciertos y que el modelo de flujo es estable. Entonces existe $t_{0}>0$ tal que
\begin{equation}
 lim_{|x|\rightarrow\infty}\frac{1}{|x|^{p+1}}\esp_{x}\left[|X\left(t_{0}|x|\right)|^{p+1}\right]=0
\end{equation}
\end{Prop}

\begin{Lemma}\label{Lema.5.2}
 Sea $\left\{\zeta\left(k\right):k\in \mathbb{z}\right\}$ una sucesi\'on independiente e id\'enticamente distribuida que toma valores en $\left(0,\infty\right)$,
y sea
$E\left(t\right)=max\left(n\geq1:\zeta\left(1\right)+\cdots+\zeta\left(n-1\right)\leq
t\right)$. Si $\esp\left[\zeta\left(1\right)\right]<\infty$,
entonces para cualquier entero $r\geq1$
\begin{equation}
 lim_{t\rightarrow\infty}\esp\left[\left(\frac{E\left(t\right)}{t}\right)^{r}\right]=\left(\frac{1}{\esp\left[\zeta_{1}\right]}\right)^{r}.
\end{equation}
Luego, bajo estas condiciones:
\begin{itemize}
 \item[a)] para cualquier $\delta>0$, $\sup_{t\geq\delta}\esp\left[\left(\frac{E\left(t\right)}{t}\right)^{r}\right]<\infty$
\item[b)] las variables aleatorias
$\left\{\left(\frac{E\left(t\right)}{t}\right)^{r}:t\geq1\right\}$
son uniformemente integrables.
\end{itemize}
\end{Lemma}

\begin{Teo}\label{Tma.5.5}
Suponga que los supuestos A1) y A2) se cumplen y que el modelo de
flujo es estable. Entonces existe una constante $\kappa_{p}$ tal
que
\begin{equation}
\frac{1}{t}\int_{0}^{t}\esp_{x}\left[|Q\left(s\right)|^{p}\right]ds\leq\kappa_{p}\left\{\frac{1}{t}|x|^{p+1}+1\right\}
\end{equation}
para $t>0$ y $x\in X$. En particular, para cada condici\'on inicial
\begin{eqnarray*}
\limsup_{t\rightarrow\infty}\frac{1}{t}\int_{0}^{t}\esp_{x}\left[|Q\left(s\right)|^{p}\right]ds\leq\kappa_{p}.
\end{eqnarray*}
\end{Teo}

\begin{Teo}\label{Tma.6.2}
Suponga que se cumplen los supuestos A1), A2) y A3) y que el
modelo de flujo es estable. Entonces se tiene que
\begin{equation}
|\left|P^{t}\left(x,\cdot\right)-\pi\left(\cdot\right)\right||_{f_{p}}\textrm{,
}t\rightarrow\infty,x\in X.
\end{equation}
En particular para cada condici\'on inicial
\begin{eqnarray*}
\lim_{t\rightarrow\infty}\esp_{x}\left[|Q\left(t\right)|^{p}\right]=\esp_{\pi}\left[|Q\left(0\right)|^{p}\right]\leq\kappa_{r}
\end{eqnarray*}
\end{Teo}
\begin{Teo}\label{Tma.6.3}
Suponga que se cumplen los supuestos A1), A2) y A3) y que el
modelo de flujo es estable. Entonces con
$f\left(x\right)=f_{1}\left(x\right)$ se tiene
\begin{equation}
\lim_{t\rightarrow\infty}t^{p-1}|\left|P^{t}\left(x,\cdot\right)-\pi\left(\cdot\right)\right||_{f}=0.
\end{equation}
En particular para cada condici\'on inicial
\begin{eqnarray*}
\lim_{t\rightarrow\infty}t^{p-1}|\esp_{x}\left[Q\left(t\right)\right]-\esp_{\pi}\left[Q\left(0\right)\right]|=0.
\end{eqnarray*}
\end{Teo}

\begin{Teo}\label{Tma.6.4}
Suponga que se cumplen los supuestos A1), A2) y A3) y que el
modelo de flujo es estable. Sea $\nu$ cualquier distribuci\'on de
probabilidad en $\left(X,\mathcal{B}_{X}\right)$, y $\pi$ la
distribuci\'on estacionaria de $X$.
\begin{itemize}
\item[i)] Para cualquier $f:X\leftarrow\rea_{+}$
\begin{equation}
\lim_{t\rightarrow\infty}\frac{1}{t}\int_{o}^{t}f\left(X\left(s\right)\right)ds=\pi\left(f\right):=\int
f\left(x\right)\pi\left(dx\right)
\end{equation}
$\prob$-c.s. \item[ii)] Para cualquier $f:X\leftarrow\rea_{+}$ con
$\pi\left(|f|\right)<\infty$, la ecuaci\'on anterior se cumple.
\end{itemize}
\end{Teo}

%_____________________________________________________________________________________
%
\subsubsection{Teorema de Estabilidad: Descripci{\'o}n}
%_____________________________________________________________________________________
%


Si $x$ es el n{\'u}mero de usuarios en la cola al comienzo del
periodo de servicio y $N_{s}\left(x\right)=N\left(x\right)$ es el
n{\'u}mero de usuarios que son atendidos con la pol{\'\i}tica $s$,
{\'u}nica en nuestro caso durante un periodo de servicio, entonces
se asume que:
\begin{enumerate}
\item
\begin{equation}\label{S1}
lim_{x\rightarrow\infty}\esp\left[N\left(x\right)\right]=\overline{N}>0
\end{equation}
\item
\begin{equation}\label{S2}
\esp\left[N\left(x\right)\right]\leq \overline{N} \end{equation}
para cualquier valor de $x$.
\end{enumerate}
La manera en que atiende el servidor $m$-{\'e}simo, en este caso
en espec{\'\i}fico solo lo ilustraremos con un s{\'o}lo servidor,
es la siguiente:
\begin{itemize}
\item Al t{\'e}rmino de la visita a la cola $j$, el servidor se
cambia a la cola $j^{'}$ con probabilidad
$r_{j,j^{'}}^{m}=r_{j,j^{'}}$

\item La $n$-{\'e}sima ocurencia va acompa{\~n}ada con el tiempo
de cambio de longitud $\delta_{j,j^{'}}\left(n\right)$,
independientes e id{\'e}nticamente distribuidas, con
$\esp\left[\delta_{j,j^{'}}\left(1\right)\right]\geq0$.

\item Sea $\left\{p_{j}\right\}$ la {\'u}nica distribuci{\'o}n
invariante estacionaria para la Cadena de Markov con matriz de
transici{\'o}n $\left(r_{j,j^{'}}\right)$.

\item Finalmente, se define
\begin{equation}
\delta^{*}:=\sum_{j,j^{'}}p_{j}r_{j,j^{'}}\esp\left[\delta_{j,j^{'}}\left(1\right)\right].
\end{equation}
\end{itemize}

%_________________________________________________________________________
\subsection{Supuestos}
%_________________________________________________________________________
Consideremos el caso en el que se tienen varias colas a las cuales
llegan uno o varios servidores para dar servicio a los usuarios
que se encuentran presentes en la cola, como ya se mencion\'o hay
varios tipos de pol\'iticas de servicio, incluso podr\'ia ocurrir
que la manera en que atiende al resto de las colas sea distinta a
como lo hizo en las anteriores.\\

Para ejemplificar los sistemas de visitas c\'iclicas se
considerar\'a el caso en que a las colas los usuarios son atendidos con
una s\'ola pol\'itica de servicio.\\



Si $\omega$ es el n\'umero de usuarios en la cola al comienzo del
periodo de servicio y $N\left(\omega\right)$ es el n\'umero de
usuarios que son atendidos con una pol\'itica en espec\'ifico
durante el periodo de servicio, entonces se asume que:
\begin{itemize}
\item[1)]\label{S1}$lim_{\omega\rightarrow\infty}\esp\left[N\left(\omega\right)\right]=\overline{N}>0$;
\item[2)]\label{S2}$\esp\left[N\left(\omega\right)\right]\leq\overline{N}$
para cualquier valor de $\omega$.
\end{itemize}
La manera en que atiende el servidor $m$-\'esimo, es la siguiente:
\begin{itemize}
\item Al t\'ermino de la visita a la cola $j$, el servidor cambia
a la cola $j^{'}$ con probabilidad $r_{j,j^{'}}^{m}$

\item La $n$-\'esima vez que el servidor cambia de la cola $j$ a
$j'$, va acompa\~nada con el tiempo de cambio de longitud
$\delta_{j,j^{'}}^{m}\left(n\right)$, con
$\delta_{j,j^{'}}^{m}\left(n\right)$, $n\geq1$, variables
aleatorias independientes e id\'enticamente distribuidas, tales
que $\esp\left[\delta_{j,j^{'}}^{m}\left(1\right)\right]\geq0$.

\item Sea $\left\{p_{j}^{m}\right\}$ la distribuci\'on invariante
estacionaria \'unica para la Cadena de Markov con matriz de
transici\'on $\left(r_{j,j^{'}}^{m}\right)$, se supone que \'esta
existe.

\item Finalmente, se define el tiempo promedio total de traslado
entre las colas como
\begin{equation}
\delta^{*}:=\sum_{j,j^{'}}p_{j}^{m}r_{j,j^{'}}^{m}\esp\left[\delta_{j,j^{'}}^{m}\left(i\right)\right].
\end{equation}
\end{itemize}

Consideremos el caso donde los tiempos entre arribo a cada una de
las colas, $\left\{\xi_{k}\left(n\right)\right\}_{n\geq1}$ son
variables aleatorias independientes a id\'enticamente
distribuidas, y los tiempos de servicio en cada una de las colas
se distribuyen de manera independiente e id\'enticamente
distribuidas $\left\{\eta_{k}\left(n\right)\right\}_{n\geq1}$;
adem\'as ambos procesos cumplen la condici\'on de ser
independientes entre s\'i. Para la $k$-\'esima cola se define la
tasa de arribo por
$\lambda_{k}=1/\esp\left[\xi_{k}\left(1\right)\right]$ y la tasa
de servicio como
$\mu_{k}=1/\esp\left[\eta_{k}\left(1\right)\right]$, finalmente se
define la carga de la cola como $\rho_{k}=\lambda_{k}/\mu_{k}$,
donde se pide que $\rho=\sum_{k=1}^{K}\rho_{k}<1$, para garantizar
la estabilidad del sistema, esto es cierto para las pol\'iticas de
servicio exhaustiva y cerrada, ver Geetor \cite{Getoor}.\\

Si denotamos por
\begin{itemize}
\item $Q_{k}\left(t\right)$ el n\'umero de usuarios presentes en
la cola $k$ al tiempo $t$; \item $A_{k}\left(t\right)$ los
residuales de los tiempos entre arribos a la cola $k$; para cada
servidor $m$; \item $B_{m}\left(t\right)$ denota a los residuales
de los tiempos de servicio al tiempo $t$; \item
$B_{m}^{0}\left(t\right)$ los residuales de los tiempos de
traslado de la cola $k$ a la pr\'oxima por atender al tiempo $t$,

\item sea
$C_{m}\left(t\right)$ el n\'umero de usuarios atendidos durante la
visita del servidor a la cola $k$ al tiempo $t$.
\end{itemize}


En este sentido, el proceso para el sistema de visitas se puede
definir como:

\begin{equation}\label{Esp.Edos.Down}
X\left(t\right)^{T}=\left(Q_{k}\left(t\right),A_{k}\left(t\right),B_{m}\left(t\right),B_{m}^{0}\left(t\right),C_{m}\left(t\right)\right),
\end{equation}
para $k=1,\ldots,K$ y $m=1,2,\ldots,M$, donde $T$ indica que es el
transpuesto del vector que se est\'a definiendo. El proceso $X$
evoluciona en el espacio de estados:
$\mathbb{X}=\ent_{+}^{K}\times\rea_{+}^{K}\times\left(\left\{1,2,\ldots,K\right\}\times\left\{1,2,\ldots,S\right\}\right)^{M}\times\rea_{+}^{K}\times\ent_{+}^{K}$.\\

El sistema aqu\'i descrito debe de cumplir con los siguientes supuestos b\'asicos de un sistema de visitas:
%__________________________________________________________________________
\subsubsection{Supuestos B\'asicos}
%__________________________________________________________________________
\begin{itemize}
\item[A1)] Los procesos
$\xi_{1},\ldots,\xi_{K},\eta_{1},\ldots,\eta_{K}$ son mutuamente
independientes y son sucesiones independientes e id\'enticamente
distribuidas.

\item[A2)] Para alg\'un entero $p\geq1$
\begin{eqnarray*}
\esp\left[\xi_{l}\left(1\right)^{p+1}\right]&<&\infty\textrm{ para }l=1,\ldots,K\textrm{ y }\\
\esp\left[\eta_{k}\left(1\right)^{p+1}\right]&<&\infty\textrm{
para }k=1,\ldots,K.
\end{eqnarray*}
donde $\mathcal{A}$ es la clase de posibles arribos.

\item[A3)] Para cada $k=1,2,\ldots,K$ existe una funci\'on
positiva $q_{k}\left(\cdot\right)$ definida en $\rea_{+}$, y un
entero $j_{k}$, tal que
\begin{eqnarray}
P\left(\xi_{k}\left(1\right)\geq x\right)&>&0\textrm{, para todo }x>0,\\
P\left\{a\leq\sum_{i=1}^{j_{k}}\xi_{k}\left(i\right)\leq
b\right\}&\geq&\int_{a}^{b}q_{k}\left(x\right)dx, \textrm{ }0\leq
a<b.
\end{eqnarray}
\end{itemize}

En lo que respecta al supuesto (A3), en Dai y Meyn \cite{DaiSean}
hacen ver que este se puede sustituir por

\begin{itemize}
\item[A3')] Para el Proceso de Markov $X$, cada subconjunto
compacto del espacio de estados de $X$ es un conjunto peque\~no,
ver definici\'on \ref{Def.Cto.Peq.}.
\end{itemize}

Es por esta raz\'on que con la finalidad de poder hacer uso de
$A3^{'})$ es necesario recurrir a los Procesos de Harris y en
particular a los Procesos Harris Recurrente, ver \cite{Dai,
DaiSean}.
%_______________________________________________________________________
\subsection{Procesos Harris Recurrente}
%_______________________________________________________________________

Por el supuesto (A1) conforme a Davis \cite{Davis}, se puede
definir el proceso de saltos correspondiente de manera tal que
satisfaga el supuesto (A3'), de hecho la demostraci\'on est\'a
basada en la l\'inea de argumentaci\'on de Davis, \cite{Davis},
p\'aginas 362-364.\\

Entonces se tiene un espacio de estados en el cual el proceso $X$
satisface la Propiedad Fuerte de Markov, ver Dai y Meyn
\cite{DaiSean}, dado por

\[\left(\Omega,\mathcal{F},\mathcal{F}_{t},X\left(t\right),\theta_{t},P_{x}\right),\]
adem\'as de ser un proceso de Borel Derecho (Sharpe \cite{Sharpe})
en el espacio de estados medible
$\left(\mathbb{X},\mathcal{B}_\mathbb{X}\right)$. El Proceso
$X=\left\{X\left(t\right),t\geq0\right\}$ tiene trayectorias
continuas por la derecha, est\'a definido en
$\left(\Omega,\mathcal{F}\right)$ y est\'a adaptado a
$\left\{\mathcal{F}_{t},t\geq0\right\}$; la colecci\'on
$\left\{P_{x},x\in \mathbb{X}\right\}$ son medidas de probabilidad
en $\left(\Omega,\mathcal{F}\right)$ tales que para todo $x\in
\mathbb{X}$
\[P_{x}\left\{X\left(0\right)=x\right\}=1,\] y
\[E_{x}\left\{f\left(X\circ\theta_{t}\right)|\mathcal{F}_{t}\right\}=E_{X}\left(\tau\right)f\left(X\right),\]
en $\left\{\tau<\infty\right\}$, $P_{x}$-c.s., con $\theta_{t}$
definido como el operador shift.


Donde $\tau$ es un $\mathcal{F}_{t}$-tiempo de paro
\[\left(X\circ\theta_{\tau}\right)\left(w\right)=\left\{X\left(\tau\left(w\right)+t,w\right),t\geq0\right\},\]
y $f$ es una funci\'on de valores reales acotada y medible, ver \cite{Dai, KaspiMandelbaum}.\\

Sea $P^{t}\left(x,D\right)$, $D\in\mathcal{B}_{\mathbb{X}}$,
$t\geq0$ la probabilidad de transici\'on de $X$ queda definida
como:
\[P^{t}\left(x,D\right)=P_{x}\left(X\left(t\right)\in
D\right).\]


\begin{Def}
Una medida no cero $\pi$ en
$\left(\mathbb{X},\mathcal{B}_{\mathbb{X}}\right)$ es invariante
para $X$ si $\pi$ es $\sigma$-finita y
\[\pi\left(D\right)=\int_{\mathbb{X}}P^{t}\left(x,D\right)\pi\left(dx\right),\]
para todo $D\in \mathcal{B}_{\mathbb{X}}$, con $t\geq0$.
\end{Def}

\begin{Def}
El proceso de Markov $X$ es llamado Harris Recurrente si existe
una medida de probabilidad $\nu$ en
$\left(\mathbb{X},\mathcal{B}_{\mathbb{X}}\right)$, tal que si
$\nu\left(D\right)>0$ y $D\in\mathcal{B}_{\mathbb{X}}$
\[P_{x}\left\{\tau_{D}<\infty\right\}\equiv1,\] cuando
$\tau_{D}=inf\left\{t\geq0:X_{t}\in D\right\}$.
\end{Def}

\begin{Note}
\begin{itemize}
\item[i)] Si $X$ es Harris recurrente, entonces existe una \'unica
medida invariante $\pi$ (Getoor \cite{Getoor}).

\item[ii)] Si la medida invariante es finita, entonces puede
normalizarse a una medida de probabilidad, en este caso al proceso
$X$ se le llama Harris recurrente positivo.


\item[iii)] Cuando $X$ es Harris recurrente positivo se dice que
la disciplina de servicio es estable. En este caso $\pi$ denota la
distribuci\'on estacionaria y hacemos
\[P_{\pi}\left(\cdot\right)=\int_{\mathbf{X}}P_{x}\left(\cdot\right)\pi\left(dx\right),\]
y se utiliza $E_{\pi}$ para denotar el operador esperanza
correspondiente, ver \cite{DaiSean}.
\end{itemize}
\end{Note}

\begin{Def}\label{Def.Cto.Peq.}
Un conjunto $D\in\mathcal{B_{\mathbb{X}}}$ es llamado peque\~no si
existe un $t>0$, una medida de probabilidad $\nu$ en
$\mathcal{B_{\mathbb{X}}}$, y un $\delta>0$ tal que
\[P^{t}\left(x,A\right)\geq\delta\nu\left(A\right),\] para $x\in
D,A\in\mathcal{B_{\mathbb{X}}}$.
\end{Def}

La siguiente serie de resultados vienen enunciados y demostrados
en Dai \cite{Dai}:
\begin{Lema}[Lema 3.1, Dai \cite{Dai}]
Sea $B$ conjunto peque\~no cerrado, supongamos que
$P_{x}\left(\tau_{B}<\infty\right)\equiv1$ y que para alg\'un
$\delta>0$ se cumple que
\begin{equation}\label{Eq.3.1}
\sup\esp_{x}\left[\tau_{B}\left(\delta\right)\right]<\infty,
\end{equation}
donde
$\tau_{B}\left(\delta\right)=inf\left\{t\geq\delta:X\left(t\right)\in
B\right\}$. Entonces, $X$ es un proceso Harris recurrente
positivo.
\end{Lema}

\begin{Lema}[Lema 3.1, Dai \cite{Dai}]\label{Lema.3.}
Bajo el supuesto (A3), el conjunto
$B=\left\{x\in\mathbb{X}:|x|\leq k\right\}$ es un conjunto
peque\~no cerrado para cualquier $k>0$.
\end{Lema}

\begin{Teo}[Teorema 3.1, Dai \cite{Dai}]\label{Tma.3.1}
Si existe un $\delta>0$ tal que
\begin{equation}
lim_{|x|\rightarrow\infty}\frac{1}{|x|}\esp|X^{x}\left(|x|\delta\right)|=0,
\end{equation}
donde $X^{x}$ se utiliza para denotar que el proceso $X$ comienza
a partir de $x$, entonces la ecuaci\'on (\ref{Eq.3.1}) se cumple
para $B=\left\{x\in\mathbb{X}:|x|\leq k\right\}$ con alg\'un
$k>0$. En particular, $X$ es Harris recurrente positivo.
\end{Teo}

Entonces, tenemos que el proceso $X$ es un proceso de Markov que
cumple con los supuestos $A1)$-$A3)$, lo que falta de hacer es
construir el Modelo de Flujo bas\'andonos en lo hasta ahora
presentado.
%_______________________________________________________________________
\subsection{Modelo de Flujo}
%_______________________________________________________________________

Dada una condici\'on inicial $x\in\mathbb{X}$, sea

\begin{itemize}
\item $Q_{k}^{x}\left(t\right)$ la longitud de la cola al tiempo
$t$,

\item $T_{m,k}^{x}\left(t\right)$ el tiempo acumulado, al tiempo
$t$, que tarda el servidor $m$ en atender a los usuarios de la
cola $k$.

\item $T_{m,k}^{x,0}\left(t\right)$ el tiempo acumulado, al tiempo
$t$, que tarda el servidor $m$ en trasladarse a otra cola a partir de la $k$-\'esima.\\
\end{itemize}

Sup\'ongase que la funci\'on
$\left(\overline{Q}\left(\cdot\right),\overline{T}_{m}
\left(\cdot\right),\overline{T}_{m}^{0} \left(\cdot\right)\right)$
para $m=1,2,\ldots,M$ es un punto l\'imite de
\begin{equation}\label{Eq.Punto.Limite}
\left(\frac{1}{|x|}Q^{x}\left(|x|t\right),\frac{1}{|x|}T_{m}^{x}\left(|x|t\right),\frac{1}{|x|}T_{m}^{x,0}\left(|x|t\right)\right)
\end{equation}
para $m=1,2,\ldots,M$, cuando $x\rightarrow\infty$, ver
\cite{Down}. Entonces
$\left(\overline{Q}\left(t\right),\overline{T}_{m}
\left(t\right),\overline{T}_{m}^{0} \left(t\right)\right)$ es un
flujo l\'imite del sistema. Al conjunto de todos las posibles
flujos l\'imite se le llama {\emph{Modelo de Flujo}} y se le
denotar\'a por $\mathcal{Q}$, ver \cite{Down, Dai, DaiSean}.\\

El modelo de flujo satisface el siguiente conjunto de ecuaciones:

\begin{equation}\label{Eq.MF.1}
\overline{Q}_{k}\left(t\right)=\overline{Q}_{k}\left(0\right)+\lambda_{k}t-\sum_{m=1}^{M}\mu_{k}\overline{T}_{m,k}\left(t\right),\\
\end{equation}
para $k=1,2,\ldots,K$.\\
\begin{equation}\label{Eq.MF.2}
\overline{Q}_{k}\left(t\right)\geq0\textrm{ para
}k=1,2,\ldots,K.\\
\end{equation}

\begin{equation}\label{Eq.MF.3}
\overline{T}_{m,k}\left(0\right)=0,\textrm{ y }\overline{T}_{m,k}\left(\cdot\right)\textrm{ es no decreciente},\\
\end{equation}
para $k=1,2,\ldots,K$ y $m=1,2,\ldots,M$.\\
\begin{equation}\label{Eq.MF.4}
\sum_{k=1}^{K}\overline{T}_{m,k}^{0}\left(t\right)+\overline{T}_{m,k}\left(t\right)=t\textrm{
para }m=1,2,\ldots,M.\\
\end{equation}


\begin{Def}[Definici\'on 4.1, Dai \cite{Dai}]\label{Def.Modelo.Flujo}
Sea una disciplina de servicio espec\'ifica. Cualquier l\'imite
$\left(\overline{Q}\left(\cdot\right),\overline{T}\left(\cdot\right),\overline{T}^{0}\left(\cdot\right)\right)$
en (\ref{Eq.Punto.Limite}) es un {\em flujo l\'imite} de la
disciplina. Cualquier soluci\'on (\ref{Eq.MF.1})-(\ref{Eq.MF.4})
es llamado flujo soluci\'on de la disciplina.
\end{Def}

\begin{Def}
Se dice que el modelo de flujo l\'imite, modelo de flujo, de la
disciplina de la cola es estable si existe una constante
$\delta>0$ que depende de $\mu,\lambda$ y $P$ solamente, tal que
cualquier flujo l\'imite con
$|\overline{Q}\left(0\right)|+|\overline{U}|+|\overline{V}|=1$, se
tiene que $\overline{Q}\left(\cdot+\delta\right)\equiv0$.
\end{Def}

Si se hace $|x|\rightarrow\infty$ sin restringir ninguna de las
componentes, tambi\'en se obtienen un modelo de flujo, pero en
este caso el residual de los procesos de arribo y servicio
introducen un retraso:
\begin{Teo}[Teorema 4.2, Dai \cite{Dai}]\label{Tma.4.2.Dai}
Sea una disciplina fija para la cola, suponga que se cumplen las
condiciones (A1)-(A3). Si el modelo de flujo l\'imite de la
disciplina de la cola es estable, entonces la cadena de Markov $X$
que describe la din\'amica de la red bajo la disciplina es Harris
recurrente positiva.
\end{Teo}

Ahora se procede a escalar el espacio y el tiempo para reducir la
aparente fluctuaci\'on del modelo. Consid\'erese el proceso
\begin{equation}\label{Eq.3.7}
\overline{Q}^{x}\left(t\right)=\frac{1}{|x|}Q^{x}\left(|x|t\right).
\end{equation}
A este proceso se le conoce como el flujo escalado, y cualquier
l\'imite $\overline{Q}^{x}\left(t\right)$ es llamado flujo
l\'imite del proceso de longitud de la cola. Haciendo
$|q|\rightarrow\infty$ mientras se mantiene el resto de las
componentes fijas, cualquier punto l\'imite del proceso de
longitud de la cola normalizado $\overline{Q}^{x}$ es soluci\'on
del siguiente modelo de flujo.


\begin{Def}[Definici\'on 3.3, Dai y Meyn \cite{DaiSean}]
El modelo de flujo es estable si existe un tiempo fijo $t_{0}$ tal
que $\overline{Q}\left(t\right)=0$, con $t\geq t_{0}$, para
cualquier $\overline{Q}\left(\cdot\right)\in\mathcal{Q}$ que
cumple con $|\overline{Q}\left(0\right)|=1$.
\end{Def}

\begin{Lemma}[Lema 3.1, Dai y Meyn \cite{DaiSean}]
Si el modelo de flujo definido por (\ref{Eq.MF.1})-(\ref{Eq.MF.4})
es estable, entonces el modelo de flujo retrasado es tambi\'en
estable, es decir, existe $t_{0}>0$ tal que
$\overline{Q}\left(t\right)=0$ para cualquier $t\geq t_{0}$, para
cualquier soluci\'on del modelo de flujo retrasado cuya
condici\'on inicial $\overline{x}$ satisface que
$|\overline{x}|=|\overline{Q}\left(0\right)|+|\overline{A}\left(0\right)|+|\overline{B}\left(0\right)|\leq1$.
\end{Lemma}


Ahora ya estamos en condiciones de enunciar los resultados principales:


\begin{Teo}[Teorema 2.1, Down \cite{Down}]\label{Tma2.1.Down}
Suponga que el modelo de flujo es estable, y que se cumplen los supuestos (A1) y (A2), entonces
\begin{itemize}
\item[i)] Para alguna constante $\kappa_{p}$, y para cada
condici\'on inicial $x\in X$
\begin{equation}\label{Estability.Eq1}
\limsup_{t\rightarrow\infty}\frac{1}{t}\int_{0}^{t}\esp_{x}\left[|Q\left(s\right)|^{p}\right]ds\leq\kappa_{p},
\end{equation}
donde $p$ es el entero dado en (A2).
\end{itemize}
Si adem\'as se cumple la condici\'on (A3), entonces para cada
condici\'on inicial:
\begin{itemize}
\item[ii)] Los momentos transitorios convergen a su estado
estacionario:
 \begin{equation}\label{Estability.Eq2}
lim_{t\rightarrow\infty}\esp_{x}\left[Q_{k}\left(t\right)^{r}\right]=\esp_{\pi}\left[Q_{k}\left(0\right)^{r}\right]\leq\kappa_{r},
\end{equation}
para $r=1,2,\ldots,p$ y $k=1,2,\ldots,K$. Donde $\pi$ es la
probabilidad invariante para $X$.

\item[iii)]  El primer momento converge con raz\'on $t^{p-1}$:
\begin{equation}\label{Estability.Eq3}
lim_{t\rightarrow\infty}t^{p-1}|\esp_{x}\left[Q_{k}\left(t\right)\right]-\esp_{\pi}\left[Q_{k}\left(0\right)\right]|=0.
\end{equation}

\item[iv)] La {\em Ley Fuerte de los grandes n\'umeros} se cumple:
\begin{equation}\label{Estability.Eq4}
lim_{t\rightarrow\infty}\frac{1}{t}\int_{0}^{t}Q_{k}^{r}\left(s\right)ds=\esp_{\pi}\left[Q_{k}\left(0\right)^{r}\right],\textrm{
}\prob_{x}\textrm{-c.s.}
\end{equation}
para $r=1,2,\ldots,p$ y $k=1,2,\ldots,K$.
\end{itemize}
\end{Teo}

La contribuci\'on de Down a la teor\'ia de los {\emph {sistemas de
visitas c\'iclicas}}, es la relaci\'on que hay entre la
estabilidad del sistema con el comportamiento de las medidas de
desempe\~no, es decir, la condici\'on suficiente para poder
garantizar la convergencia del proceso de la longitud de la cola
as\'i como de por los menos los dos primeros momentos adem\'as de
una versi\'on de la Ley Fuerte de los Grandes N\'umeros para los
sistemas de visitas.


\begin{Teo}[Teorema 2.3, Down \cite{Down}]\label{Tma2.3.Down}
Considere el siguiente valor:
\begin{equation}\label{Eq.Rho.1serv}
\rho=\sum_{k=1}^{K}\rho_{k}+max_{1\leq j\leq K}\left(\frac{\lambda_{j}}{\sum_{s=1}^{S}p_{js}\overline{N}_{s}}\right)\delta^{*}
\end{equation}
\begin{itemize}
\item[i)] Si $\rho<1$ entonces la red es estable, es decir, se
cumple el Teorema \ref{Tma2.1.Down}.

\item[ii)] Si $\rho>1$ entonces la red es inestable, es decir, se
cumple el Teorema \ref{Tma2.2.Down}
\end{itemize}
\end{Teo}



%_________________________________________________________________________
\subsection{Modelo de Flujo}
%_________________________________________________________________________
Sup\'ongase que el sistema consta de varias colas a los cuales
llegan uno o varios servidores a dar servicio a los usuarios
esperando en la cola.\\


Sea $x$ el n\'umero de usuarios en la cola esperando por servicio
y $N\left(x\right)$ es el n\'umero de usuarios que son atendidos
con una pol\'itica dada y fija mientras el servidor permanece
dando servicio, entonces se asume que:
\begin{itemize}
\item[(S1.)]
\begin{equation}\label{S1}
lim_{x\rightarrow\infty}\esp\left[N\left(x\right)\right]=\overline{N}>0.
\end{equation}
\item[(S2.)]
\begin{equation}\label{S2}
\esp\left[N\left(x\right)\right]\leq \overline{N},
\end{equation}

para cualquier valor de $x$.
\end{itemize}

El tiempo que tarda un servidor en volver a dar servicio despu\'es
de abandonar la cola inmediata anterior y llegar a la pr\'oxima se
llama tiempo de traslado o de cambio  de cola, al momento de la
$n$-\'esima visita del servidor a la cola $j$ se genera una
sucesi\'on de variables aleatorias $\delta_{j,j+1}\left(n\right)$,
independientes e id\'enticamente distribuidas, con la propiedad de
que $\esp\left[\delta_{j,j+1}\left(1\right)\right]\geq0$.\\


Se define
\begin{equation}
\delta^{*}:=\sum_{j,j+1}\esp\left[\delta_{j,j+1}\left(1\right)\right].
\end{equation}
%\begin{figure}[H]
%\centering
%\includegraphics[width=7cm]{switchovertime.jpg}
%\caption{Sistema de Visitas C\'iclicas}
%\end{figure}

Los tiempos entre arribos a la cola $k$, son de la forma
$\left\{\xi_{k}\left(n\right)\right\}_{n\geq1}$, con la propiedad
de que son independientes e id\'enticamente distribuidos. Los
tiempos de servicio
$\left\{\eta_{k}\left(n\right)\right\}_{n\geq1}$ tienen la
propiedad de ser independientes e id\'enticamente distribuidos.
Para la $k$-\'esima cola se define la tasa de arribo a la como
$\lambda_{k}=1/\esp\left[\xi_{k}\left(1\right)\right]$ y la tasa
de servicio como
$\mu_{k}=1/\esp\left[\eta_{k}\left(1\right)\right]$, finalmente se
define la carga de la cola como $\rho_{k}=\lambda_{k}/\mu_{k}$,
donde se pide que $\rho<1$, para garantizar la estabilidad del sistema.\\

%_____________________________________________________________________
%\subsubsection{Proceso de Estados}
%_____________________________________________________________________

Para el caso m\'as sencillo podemos definir un proceso de estados
para la red que depende de la pol\'itica de servicio utilizada, el
estado $\mathbb{X}\left(t\right)$ a cualquier tiempo $t$ puede
definirse como
\begin{equation}\label{Eq.Esp.Estados}
\mathbb{X}\left(t\right)=\left(Q_{k}\left(t\right),A_{l}\left(t\right),B_{k}\left(t\right):k=1,2,\ldots,K,l\in\mathcal{A}\right),
\end{equation}

donde $Q_{k}\left(t\right)$ es la longitud de la cola $k$ para los
usuarios esperando servicio, incluyendo aquellos que est\'an
siendo atendidos, $B_{k}\left(t\right)$ son los tiempos de
servicio residuales para los usuarios de la clase $k$ que est\'an
en servicio.\\

Los tiempos entre arribos residuales, que son el tiempo que queda
hasta que el pr\'oximo usuario llega a la cola para recibir
servicio, se denotan por $A_{k}\left(t\right)$. Tanto
$B_{k}\left(t\right)$ como $A_{k}\left(t\right)$ se suponen
continuos por la derecha \cite{Dai2}.\\

Sea $\mathcal{X}$ el espacio de estados para el proceso de estados
que por definici\'on es igual  al conjunto de posibles valores
para el estado $\mathbb{X}\left(t\right)$, y sea
$x=\left(q,a,b\right)$ un estado gen\'erico en $\mathbb{X}$, la
componente $q$ determina la posici\'on del usuario en la red,
$|q|$ denota la longitud total de la cola en la red.\\

Para un estado $x=\left(q,a,b\right)\in\mathbb{X}$ definimos la
{\em norma} de $x$ como $\left\|x\right\|=|q|+|a|+|b|$. En
\cite{Dai} se muestra que para una amplia serie de disciplinas de
servicio el proceso $\mathbb{X}$ es un Proceso Fuerte de Markov, y
por tanto se puede asumir que
\[\left(\left(\Omega,\mathcal{F}\right),\mathcal{F}_{t},\mathbb{X}\left(t\right),\theta_{t},P_{x}\right)\]
es un proceso de {\em Borel Derecho} en el espacio de estados
medible $\left(\mathcal{X},\mathcal{B}_{\mathcal{X}}\right)$.\\

Sea $P^{t}\left(x,D\right)$, $D\in\mathcal{B}_{\mathbb{X}}$,
$t\geq0$ probabilidad de transici\'on de $X$ definida como
\[P^{t}\left(x,D\right)=P_{x}\left(\mathbb{X}\left(t\right)\in
D\right).\]

\begin{Def}
Una medida no cero $\pi$ en
$\left(\mathbb{X},\mathcal{B}_{\mathbb{X}}\right)$ es {\em
invariante} para $X$ si $\pi$ es $\sigma$-finita y
\[\pi\left(D\right)=\int_{X}P^{t}\left(x,D\right)\pi\left(dx\right),\]
para todo $D\in \mathcal{B}_{\mathbb{X}}$, con $t\geq0$.
\end{Def}

\begin{Def}
El proceso de Markov $X$ es llamado {\em Harris recurrente} si
existe una medida de probabilidad $\nu$ en
$\left(\mathbb{X},\mathcal{B}_{\mathbb{X}}\right)$, tal que si
$\nu\left(D\right)>0$ y $D\in\mathcal{B}_{\mathbb{X}}$
\[P_{x}\left\{\tau_{D}<\infty\right\}\equiv1,\] cuando
$\tau_{D}=inf\left\{t\geq0:\mathbb{X}_{t}\in D\right\}$.
\end{Def}

\begin{Def}
Un conjunto $D\in\mathcal{B}_\mathbb{X}$ es llamado peque\~no si
existe un $t>0$, una medida de probabilidad $\nu$ en
$\mathcal{B}_\mathbb{X}$, y un $\delta>0$ tal que
\[P^{t}\left(x,A\right)\geq\delta\nu\left(A\right),\] para $x\in
D,A\in\mathcal{B}_\mathbb{X}$.
\end{Def}
\begin{Note}
\begin{itemize}

\item[i)] Si $X$ es Harris recurrente, entonces existe una \'unica medida
invariante $\pi$ (\cite{Getoor}).

\item[ii)] Si la medida invariante es finita, entonces puede
normalizarse a una medida de probabilidad, en este caso a la
medida se le llama \textbf{Harris recurrente positiva}.

\item[iii)] Cuando $X$ es Harris recurrente positivo se dice que
la disciplina de servicio es estable. En este caso $\pi$ denota la
ditribuci\'on estacionaria; se define
\[P_{\pi}\left(\cdot\right)=\int_{X}P_{x}\left(\cdot\right)\pi\left(dx\right).\]
Se utiliza $E_{\pi}$ para denotar el operador esperanza
correspondiente, as\'i, el proceso
$X=\left\{\mathbb{X}\left(t\right),t\geq0\right\}$ es un proceso
estrictamente estacionario bajo $P_{\pi}$.

\item[iv)] En \cite{MeynTweedie} se muestra que si
$P_{x}\left\{\tau_{D}<\infty\right\}=1$ incluso para solamente un
conjunto peque\~no, entonces el proceso de Harris es recurrente.
\end{itemize}
\end{Note}


%_________________________________________________________________________
%\newpage
%_________________________________________________________________________
%\subsection{Modelo de Flujo}
%_____________________________________________________________________
Las Colas C\'iclicas se pueden describir por medio de un proceso
de Markov $\left(X\left(t\right)\right)_{t\in\rea}$, donde el
estado del sistema al tiempo $t\geq0$ est\'a dado por
\begin{equation}
X\left(t\right)=\left(Q\left(t\right),A\left(t\right),H\left(t\right),B\left(t\right),B^{0}\left(t\right),C\left(t\right)\right)
\end{equation}
definido en el espacio producto:
\begin{equation}
\mathcal{X}=\mathbb{Z}^{K}\times\rea_{+}^{K}\times\left(\left\{1,2,\ldots,K\right\}\times\left\{1,2,\ldots,S\right\}\right)^{M}\times\rea_{+}^{K}\times\rea_{+}^{K}\times\mathbb{Z}^{K},
\end{equation}

\begin{itemize}
\item $Q\left(t\right)=\left(Q_{k}\left(t\right),1\leq k\leq
K\right)$, es el n\'umero de usuarios en la cola $k$, incluyendo
aquellos que est\'an siendo atendidos provenientes de la
$k$-\'esima cola.

\item $A\left(t\right)=\left(A_{k}\left(t\right),1\leq k\leq
K\right)$, son los residuales de los tiempos de arribo en la cola
$k$. \item $H\left(t\right)$ es el par ordenado que consiste en la
cola que esta siendo atendida y la pol\'itica de servicio que se
utilizar\'a.

\item $B\left(t\right)$ es el tiempo de servicio residual.

\item $B^{0}\left(t\right)$ es el tiempo residual del cambio de
cola.

\item $C\left(t\right)$ indica el n\'umero de usuarios atendidos
durante la visita del servidor a la cola dada en
$H\left(t\right)$.
\end{itemize}

$A_{k}\left(t\right),B_{m}\left(t\right)$ y
$B_{m}^{0}\left(t\right)$ se suponen continuas por la derecha y
que satisfacen la propiedad fuerte de Markov, (\cite{Dai}).

Dada una condici\'on inicial $x\in\mathcal{X}$, $Q_{k}^{x}\left(t\right)$ es la longitud de la cola $k$ al tiempo $t$
y $T_{m,k}^{x}\left(t\right)$  el tiempo acumulado al tiempo $t$ que el servidor tarda en atender a los usuarios de la cola $k$.
De igual manera se define $T_{m,k}^{x,0}\left(t\right)$ el tiempo acumulado al tiempo $t$ que el servidor tarda en
cambiar de cola para volver a atender a los usuarios.

Para reducir la fluctuaci\'on del modelo se escala tanto el espacio como el tiempo, entonces se
tiene el proceso:

\begin{eqnarray}
\overline{Q}^{x}\left(t\right)=\frac{1}{|x|}Q^{x}\left(|x|t\right),\\
\overline{T}_{m}^{x}\left(t\right)=\frac{1}{|x|}T_{m}^{x}\left(|x|t\right),\\
\overline{T}_{m}^{x,0}\left(t\right)=\frac{1}{|x|}T_{m}^{x,0}\left(|x|t\right).
\end{eqnarray}
Cualquier l\'imite $\overline{Q}\left(t\right)$ es llamado un
flujo l\'imite del proceso longitud de la cola, al conjunto de todos los posibles flujos l\'imite
se le llamar\'a \textbf{modelo de flujo}, (\cite{MaynDown}).

\begin{Def}
Un flujo l\'imite para un sistema de visitas bajo una disciplina de
servicio espec\'ifica se define como cualquier soluci\'on
 $\left(\overline{Q}\left(\cdot\right),\overline{T}_{m}\left(\cdot\right),\overline{T}_{m}^{0}\left(\cdot\right)\right)$
 de las siguientes ecuaciones, donde
$\overline{Q}\left(t\right)=\left(\overline{Q}_{1}\left(t\right),\ldots,\overline{Q}_{K}\left(t\right)\right)$
y
$\overline{T}\left(t\right)=\left(\overline{T}_{1}\left(t\right),\ldots,\overline{T}_{K}\left(t\right)\right)$
\begin{equation}\label{Eq.3.8}
\overline{Q}_{k}\left(t\right)=\overline{Q}_{k}\left(0\right)+\lambda_{k}t-\sum_{m=1}^{M}\mu_{k}\overline{T}_{m,k}\left(t\right)\\
\end{equation}
\begin{equation}\label{Eq.3.9}
\overline{Q}_{k}\left(t\right)\geq0\textrm{ para }k=1,2,\ldots,K,\\
\end{equation}
\begin{equation}\label{Eq.3.10}
\overline{T}_{m,k}\left(0\right)=0,\textrm{ y }\overline{T}_{m,k}\left(\cdot\right)\textrm{ es no decreciente},\\
\end{equation}
\begin{equation}\label{Eq.3.11}
\sum_{k=1}^{K}\overline{T}_{m,k}^{0}\left(t\right)+\overline{T}_{m,k}\left(t\right)=t\textrm{ para}m=1,2,\ldots,M\\
\end{equation}
\end{Def}

Al conjunto de ecuaciones dadas en (\ref{Eq.3.8})-(\ref{Eq.3.11}) se
le llama {\em Modelo de flujo} y al conjunto de todas las
soluciones del modelo de flujo
$\left(\overline{Q}\left(\cdot\right),\overline{T}
\left(\cdot\right)\right)$ se le denotar\'a por $\mathcal{Q}$.


\begin{Def}
El modelo de flujo es estable si existe un tiempo fijo $t_{0}$ tal
que $\overline{Q}\left(t\right)=0$, con $t\geq t_{0}$, para
cualquier $\overline{Q}\left(\cdot\right)\in\mathcal{Q}$ que
cumple con $|\overline{Q}\left(0\right)|=1$.
\end{Def}


%_____________________________________________________________________________________
%
\subsection{Estabilidad de los Sistemas de Visitas C\'iclicas}
%_________________________________________________________________________

Es necesario realizar los siguientes supuestos, ver (\cite{Dai2}) y (\cite{DaiSean}):

\begin{itemize}
\item[A1)] $\xi_{1},\ldots,\xi_{K},\eta_{1},\ldots,\eta_{K}$ son
mutuamente independientes y son sucesiones independientes e
id\'enticamente distribuidas.

\item[A2)] Para alg\'un entero $p\geq1$
\begin{eqnarray*}
\esp\left[\xi_{k}\left(1\right)^{p+1}\right]&<&\infty\textrm{ para }l\in\mathcal{A}\textrm{ y }\\
\esp\left[\eta_{k}\left(1\right)^{p+1}\right]&<&\infty\textrm{ para
}k=1,\ldots,K.
\end{eqnarray*}
\item[A3)] El conjunto $\left\{x\in X:|x|=0\right\}$ es un
singleton, y para cada $k\in\mathcal{A}$, existe una funci\'on
positiva $q_{k}\left(x\right)$ definida en $\rea_{+}$, y un entero
$j_{k}$, tal que
\begin{eqnarray}
P\left(\xi_{k}\left(1\right)\geq x\right)&>&0\textrm{, para todo }x>0\\
P\left(\xi_{k}\left(1\right)+\ldots\xi_{k}\left(j_{k}\right)\in dx\right)&\geq& q_{k}\left(x\right)dx0\textrm{ y }\\
\int_{0}^{\infty}q_{k}\left(x\right)dx>0
\end{eqnarray}
\end{itemize}


En \cite{MaynDown} ser da un argumento para deducir que todos los
subconjuntos compactos de $X$ son peque\~nos. Entonces la
condici\'on A3) se puede generalizar a
\begin{itemize}
\item[A3')] Para el proceso de Markov $X$, cada subconjunto
compacto de $X$ es peque\~no.
\end{itemize}

\begin{Teo}\label{Tma2.1}
Suponga que el modelo de flujo para una disciplina de servicio es
estable, y suponga adem\'as que las condiciones A1) y A2) se
satisfacen. Entonces:
\begin{itemize}
\item[i)] Para alguna constante $\kappa_{p}$, y para cada
condici\'on inicial $x\in X$
\begin{equation}
\limsup_{t\rightarrow\infty}\frac{1}{t}\int_{0}^{t}\esp_{x}\left[|Q\left(t\right)|^{p}\right]ds\leq\kappa_{p}
\end{equation}
donde $p$ es el entero dado por A2).
\end{itemize}

Suponga adem\'as que A3) o A3')
se cumple, entonces la disciplina de servicio es estable y adem\'as
para cada condici\'on inicial se tiene lo siguiente:
\begin{itemize}

\item[ii)] Los momentos transitorios convergen a sus valores en estado
estacionario:
\begin{equation}
\lim_{t\rightarrow\infty}\esp_{x}\left[Q_{k}\left(t\right)^{r}\right]=\esp_{\pi}\left[Q_{k}\left(0\right)^{r}\right]\leq\kappa_{r}
\end{equation}
para $r=1,\ldots,p$ y $k=1,\ldots,K$. \item[iii)] El primer
momento converge con raz\'on $t^{p-1}$:
\begin{equation}
\lim_{t\rightarrow\infty}t^{p-1}|\esp_{x}\left[Q\left(t\right)\right]-\esp_{\pi}\left[Q\left(0\right)\right]|=0.
\end{equation}
\item[iv)] Se cumple la Ley Fuerte de los Grandes N\'umeros:
\begin{equation}
\lim_{t\rightarrow\infty}\frac{1}{t}\int_{0}^{t}Q_{k}^{r}\left(s\right)ds=\esp_{\pi}\left[Q_{k}\left(0\right)^{r}\right]
\end{equation}
$\prob$-c.s., para $r=1,\ldots,p$ y $k=1,\ldots,K$.
\end{itemize}
\end{Teo}


\begin{Teo}\label{Tma2.2}
Suponga que el fluido modelo es inestable en el sentido de que
para alguna $\epsilon_{0},c_{0}\geq0$,
\begin{equation}\label{Eq.Inestability}
|Q\left(T\right)|\geq\epsilon_{0}T-c_{0}\textrm{, con }T\geq0,
\end{equation}
para cualquier condici\'on inicial $Q\left(0\right)$, con
$|Q\left(0\right)|=1$. Entonces para cualquier $0<q\leq1$, existe
$B<0$ tal que para cualquier $|x|\geq B$,
\begin{equation}
\prob_{x}\left\{\mathbb{X}\rightarrow\infty\right\}\geq q.
\end{equation}
\end{Teo}

%_____________________________________________________________________________________
%

%_____________________________________________________________________
\subsection{Resultados principales}
%_____________________________________________________________________
En el caso particular de un modelo con un solo servidor, $M=1$, se
tiene que si se define
\begin{Def}\label{Def.Ro}
\begin{equation}\label{RoM1}
\rho=\sum_{k=1}^{K}\rho_{k}+\max_{1\leq j\leq
K}\left(\frac{\lambda_{j}}{\overline{N}}\right)\delta^{*}.
\end{equation}
\end{Def}
entonces

\begin{Teo}\label{Teo.Down}
\begin{itemize}
\item[i)] Si $\rho<1$, entonces la red es estable, es decir el teorema
(\ref{Tma2.1}) se cumple. \item[ii)] De lo contrario, es decir, si
$\rho>1$ entonces la red es inestable, es decir, el teorema
(\ref{Tma2.2}).
\end{itemize}
\end{Teo}



%_________________________________________________________________________
\subsection{Supuestos}
%_________________________________________________________________________
Consideremos el caso en el que se tienen varias colas a las cuales
llegan uno o varios servidores para dar servicio a los usuarios
que se encuentran presentes en la cola, como ya se mencion\'o hay
varios tipos de pol\'iticas de servicio, incluso podr\'ia ocurrir
que la manera en que atiende al resto de las colas sea distinta a
como lo hizo en las anteriores.\\

Para ejemplificar los sistemas de visitas c\'iclicas se
considerar\'a el caso en que a las colas los usuarios son atendidos con
una s\'ola pol\'itica de servicio.\\


Si $\omega$ es el n\'umero de usuarios en la cola al comienzo del
periodo de servicio y $N\left(\omega\right)$ es el n\'umero de
usuarios que son atendidos con una pol\'itica en espec\'ifico
durante el periodo de servicio, entonces se asume que:
\begin{itemize}
\item[1)]\label{S1}$lim_{\omega\rightarrow\infty}\esp\left[N\left(\omega\right)\right]=\overline{N}>0$;
\item[2)]\label{S2}$\esp\left[N\left(\omega\right)\right]\leq\overline{N}$
para cualquier valor de $\omega$.
\end{itemize}
La manera en que atiende el servidor $m$-\'esimo, es la siguiente:
\begin{itemize}
\item Al t\'ermino de la visita a la cola $j$, el servidor cambia
a la cola $j^{'}$ con probabilidad $r_{j,j^{'}}^{m}$

\item La $n$-\'esima vez que el servidor cambia de la cola $j$ a
$j'$, va acompa\~nada con el tiempo de cambio de longitud
$\delta_{j,j^{'}}^{m}\left(n\right)$, con
$\delta_{j,j^{'}}^{m}\left(n\right)$, $n\geq1$, variables
aleatorias independientes e id\'enticamente distribuidas, tales
que $\esp\left[\delta_{j,j^{'}}^{m}\left(1\right)\right]\geq0$.

\item Sea $\left\{p_{j}^{m}\right\}$ la distribuci\'on invariante
estacionaria \'unica para la Cadena de Markov con matriz de
transici\'on $\left(r_{j,j^{'}}^{m}\right)$, se supone que \'esta
existe.

\item Finalmente, se define el tiempo promedio total de traslado
entre las colas como
\begin{equation}
\delta^{*}:=\sum_{j,j^{'}}p_{j}^{m}r_{j,j^{'}}^{m}\esp\left[\delta_{j,j^{'}}^{m}\left(i\right)\right].
\end{equation}
\end{itemize}

Consideremos el caso donde los tiempos entre arribo a cada una de
las colas, $\left\{\xi_{k}\left(n\right)\right\}_{n\geq1}$ son
variables aleatorias independientes a id\'enticamente
distribuidas, y los tiempos de servicio en cada una de las colas
se distribuyen de manera independiente e id\'enticamente
distribuidas $\left\{\eta_{k}\left(n\right)\right\}_{n\geq1}$;
adem\'as ambos procesos cumplen la condici\'on de ser
independientes entre s\'i. Para la $k$-\'esima cola se define la
tasa de arribo por
$\lambda_{k}=1/\esp\left[\xi_{k}\left(1\right)\right]$ y la tasa
de servicio como
$\mu_{k}=1/\esp\left[\eta_{k}\left(1\right)\right]$, finalmente se
define la carga de la cola como $\rho_{k}=\lambda_{k}/\mu_{k}$,
donde se pide que $\rho=\sum_{k=1}^{K}\rho_{k}<1$, para garantizar
la estabilidad del sistema, esto es cierto para las pol\'iticas de
servicio exhaustiva y cerrada, ver Geetor \cite{Getoor}.\\

Si denotamos por
\begin{itemize}
\item $Q_{k}\left(t\right)$ el n\'umero de usuarios presentes en
la cola $k$ al tiempo $t$; \item $A_{k}\left(t\right)$ los
residuales de los tiempos entre arribos a la cola $k$; para cada
servidor $m$; \item $B_{m}\left(t\right)$ denota a los residuales
de los tiempos de servicio al tiempo $t$; \item
$B_{m}^{0}\left(t\right)$ los residuales de los tiempos de
traslado de la cola $k$ a la pr\'oxima por atender al tiempo $t$,

\item sea
$C_{m}\left(t\right)$ el n\'umero de usuarios atendidos durante la
visita del servidor a la cola $k$ al tiempo $t$.
\end{itemize}


En este sentido, el proceso para el sistema de visitas se puede
definir como:

\begin{equation}\label{Esp.Edos.Down}
X\left(t\right)^{T}=\left(Q_{k}\left(t\right),A_{k}\left(t\right),B_{m}\left(t\right),B_{m}^{0}\left(t\right),C_{m}\left(t\right)\right),
\end{equation}
para $k=1,\ldots,K$ y $m=1,2,\ldots,M$, donde $T$ indica que es el
transpuesto del vector que se est\'a definiendo. El proceso $X$
evoluciona en el espacio de estados:
$\mathbb{X}=\ent_{+}^{K}\times\rea_{+}^{K}\times\left(\left\{1,2,\ldots,K\right\}\times\left\{1,2,\ldots,S\right\}\right)^{M}\times\rea_{+}^{K}\times\ent_{+}^{K}$.\\

El sistema aqu\'i descrito debe de cumplir con los siguientes supuestos b\'asicos de un sistema de visitas:
%__________________________________________________________________________
\subsubsection{Supuestos B\'asicos}
%__________________________________________________________________________
\begin{itemize}
\item[A1)] Los procesos
$\xi_{1},\ldots,\xi_{K},\eta_{1},\ldots,\eta_{K}$ son mutuamente
independientes y son sucesiones independientes e id\'enticamente
distribuidas.

\item[A2)] Para alg\'un entero $p\geq1$
\begin{eqnarray*}
\esp\left[\xi_{l}\left(1\right)^{p+1}\right]&<&\infty\textrm{ para }l=1,\ldots,K\textrm{ y }\\
\esp\left[\eta_{k}\left(1\right)^{p+1}\right]&<&\infty\textrm{
para }k=1,\ldots,K.
\end{eqnarray*}
donde $\mathcal{A}$ es la clase de posibles arribos.

\item[A3)] Para cada $k=1,2,\ldots,K$ existe una funci\'on
positiva $q_{k}\left(\cdot\right)$ definida en $\rea_{+}$, y un
entero $j_{k}$, tal que
\begin{eqnarray}
P\left(\xi_{k}\left(1\right)\geq x\right)&>&0\textrm{, para todo }x>0,\\
P\left\{a\leq\sum_{i=1}^{j_{k}}\xi_{k}\left(i\right)\leq
b\right\}&\geq&\int_{a}^{b}q_{k}\left(x\right)dx, \textrm{ }0\leq
a<b.
\end{eqnarray}
\end{itemize}

En lo que respecta al supuesto (A3), en Dai y Meyn \cite{DaiSean}
hacen ver que este se puede sustituir por

\begin{itemize}
\item[A3')] Para el Proceso de Markov $X$, cada subconjunto
compacto del espacio de estados de $X$ es un conjunto peque\~no,
ver definici\'on \ref{Def.Cto.Peq.}.
\end{itemize}

Es por esta raz\'on que con la finalidad de poder hacer uso de
$A3^{'})$ es necesario recurrir a los Procesos de Harris y en
particular a los Procesos Harris Recurrente, ver \cite{Dai,
DaiSean}.
%_______________________________________________________________________
\subsection{Procesos Harris Recurrente}
%_______________________________________________________________________

Por el supuesto (A1) conforme a Davis \cite{Davis}, se puede
definir el proceso de saltos correspondiente de manera tal que
satisfaga el supuesto (A3'), de hecho la demostraci\'on est\'a
basada en la l\'inea de argumentaci\'on de Davis, \cite{Davis},
p\'aginas 362-364.\\

Entonces se tiene un espacio de estados en el cual el proceso $X$
satisface la Propiedad Fuerte de Markov, ver Dai y Meyn
\cite{DaiSean}, dado por

\[\left(\Omega,\mathcal{F},\mathcal{F}_{t},X\left(t\right),\theta_{t},P_{x}\right),\]
adem\'as de ser un proceso de Borel Derecho (Sharpe \cite{Sharpe})
en el espacio de estados medible
$\left(\mathbb{X},\mathcal{B}_\mathbb{X}\right)$. El Proceso
$X=\left\{X\left(t\right),t\geq0\right\}$ tiene trayectorias
continuas por la derecha, est\'a definido en
$\left(\Omega,\mathcal{F}\right)$ y est\'a adaptado a
$\left\{\mathcal{F}_{t},t\geq0\right\}$; la colecci\'on
$\left\{P_{x},x\in \mathbb{X}\right\}$ son medidas de probabilidad
en $\left(\Omega,\mathcal{F}\right)$ tales que para todo $x\in
\mathbb{X}$
\[P_{x}\left\{X\left(0\right)=x\right\}=1,\] y
\[E_{x}\left\{f\left(X\circ\theta_{t}\right)|\mathcal{F}_{t}\right\}=E_{X}\left(\tau\right)f\left(X\right),\]
en $\left\{\tau<\infty\right\}$, $P_{x}$-c.s., con $\theta_{t}$
definido como el operador shift.


Donde $\tau$ es un $\mathcal{F}_{t}$-tiempo de paro
\[\left(X\circ\theta_{\tau}\right)\left(w\right)=\left\{X\left(\tau\left(w\right)+t,w\right),t\geq0\right\},\]
y $f$ es una funci\'on de valores reales acotada y medible, ver \cite{Dai, KaspiMandelbaum}.\\

Sea $P^{t}\left(x,D\right)$, $D\in\mathcal{B}_{\mathbb{X}}$,
$t\geq0$ la probabilidad de transici\'on de $X$ queda definida
como:
\[P^{t}\left(x,D\right)=P_{x}\left(X\left(t\right)\in
D\right).\]


\begin{Def}
Una medida no cero $\pi$ en
$\left(\mathbb{X},\mathcal{B}_{\mathbb{X}}\right)$ es invariante
para $X$ si $\pi$ es $\sigma$-finita y
\[\pi\left(D\right)=\int_{\mathbb{X}}P^{t}\left(x,D\right)\pi\left(dx\right),\]
para todo $D\in \mathcal{B}_{\mathbb{X}}$, con $t\geq0$.
\end{Def}

\begin{Def}
El proceso de Markov $X$ es llamado Harris Recurrente si existe
una medida de probabilidad $\nu$ en
$\left(\mathbb{X},\mathcal{B}_{\mathbb{X}}\right)$, tal que si
$\nu\left(D\right)>0$ y $D\in\mathcal{B}_{\mathbb{X}}$
\[P_{x}\left\{\tau_{D}<\infty\right\}\equiv1,\] cuando
$\tau_{D}=inf\left\{t\geq0:X_{t}\in D\right\}$.
\end{Def}

\begin{Note}
\begin{itemize}
\item[i)] Si $X$ es Harris recurrente, entonces existe una \'unica
medida invariante $\pi$ (Getoor \cite{Getoor}).

\item[ii)] Si la medida invariante es finita, entonces puede
normalizarse a una medida de probabilidad, en este caso al proceso
$X$ se le llama Harris recurrente positivo.


\item[iii)] Cuando $X$ es Harris recurrente positivo se dice que
la disciplina de servicio es estable. En este caso $\pi$ denota la
distribuci\'on estacionaria y hacemos
\[P_{\pi}\left(\cdot\right)=\int_{\mathbf{X}}P_{x}\left(\cdot\right)\pi\left(dx\right),\]
y se utiliza $E_{\pi}$ para denotar el operador esperanza
correspondiente, ver \cite{DaiSean}.
\end{itemize}
\end{Note}

\begin{Def}\label{Def.Cto.Peq.}
Un conjunto $D\in\mathcal{B_{\mathbb{X}}}$ es llamado peque\~no si
existe un $t>0$, una medida de probabilidad $\nu$ en
$\mathcal{B_{\mathbb{X}}}$, y un $\delta>0$ tal que
\[P^{t}\left(x,A\right)\geq\delta\nu\left(A\right),\] para $x\in
D,A\in\mathcal{B_{\mathbb{X}}}$.
\end{Def}

La siguiente serie de resultados vienen enunciados y demostrados
en Dai \cite{Dai}:
\begin{Lema}[Lema 3.1, Dai \cite{Dai}]
Sea $B$ conjunto peque\~no cerrado, supongamos que
$P_{x}\left(\tau_{B}<\infty\right)\equiv1$ y que para alg\'un
$\delta>0$ se cumple que
\begin{equation}\label{Eq.3.1}
\sup\esp_{x}\left[\tau_{B}\left(\delta\right)\right]<\infty,
\end{equation}
donde
$\tau_{B}\left(\delta\right)=inf\left\{t\geq\delta:X\left(t\right)\in
B\right\}$. Entonces, $X$ es un proceso Harris recurrente
positivo.
\end{Lema}

\begin{Lema}[Lema 3.1, Dai \cite{Dai}]\label{Lema.3.}
Bajo el supuesto (A3), el conjunto
$B=\left\{x\in\mathbb{X}:|x|\leq k\right\}$ es un conjunto
peque\~no cerrado para cualquier $k>0$.
\end{Lema}

\begin{Teo}[Teorema 3.1, Dai \cite{Dai}]\label{Tma.3.1}
Si existe un $\delta>0$ tal que
\begin{equation}
lim_{|x|\rightarrow\infty}\frac{1}{|x|}\esp|X^{x}\left(|x|\delta\right)|=0,
\end{equation}
donde $X^{x}$ se utiliza para denotar que el proceso $X$ comienza
a partir de $x$, entonces la ecuaci\'on (\ref{Eq.3.1}) se cumple
para $B=\left\{x\in\mathbb{X}:|x|\leq k\right\}$ con alg\'un
$k>0$. En particular, $X$ es Harris recurrente positivo.
\end{Teo}

Entonces, tenemos que el proceso $X$ es un proceso de Markov que
cumple con los supuestos $A1)$-$A3)$, lo que falta de hacer es
construir el Modelo de Flujo bas\'andonos en lo hasta ahora
presentado.
%_______________________________________________________________________
\subsection{Modelo de Flujo}
%_______________________________________________________________________

Dada una condici\'on inicial $x\in\mathbb{X}$, sea

\begin{itemize}
\item $Q_{k}^{x}\left(t\right)$ la longitud de la cola al tiempo
$t$,

\item $T_{m,k}^{x}\left(t\right)$ el tiempo acumulado, al tiempo
$t$, que tarda el servidor $m$ en atender a los usuarios de la
cola $k$.

\item $T_{m,k}^{x,0}\left(t\right)$ el tiempo acumulado, al tiempo
$t$, que tarda el servidor $m$ en trasladarse a otra cola a partir de la $k$-\'esima.\\
\end{itemize}

Sup\'ongase que la funci\'on
$\left(\overline{Q}\left(\cdot\right),\overline{T}_{m}
\left(\cdot\right),\overline{T}_{m}^{0} \left(\cdot\right)\right)$
para $m=1,2,\ldots,M$ es un punto l\'imite de
\begin{equation}\label{Eq.Punto.Limite}
\left(\frac{1}{|x|}Q^{x}\left(|x|t\right),\frac{1}{|x|}T_{m}^{x}\left(|x|t\right),\frac{1}{|x|}T_{m}^{x,0}\left(|x|t\right)\right)
\end{equation}
para $m=1,2,\ldots,M$, cuando $x\rightarrow\infty$, ver
\cite{Down}. Entonces
$\left(\overline{Q}\left(t\right),\overline{T}_{m}
\left(t\right),\overline{T}_{m}^{0} \left(t\right)\right)$ es un
flujo l\'imite del sistema. Al conjunto de todos las posibles
flujos l\'imite se le llama {\emph{Modelo de Flujo}} y se le
denotar\'a por $\mathcal{Q}$, ver \cite{Down, Dai, DaiSean}.\\

El modelo de flujo satisface el siguiente conjunto de ecuaciones:

\begin{equation}\label{Eq.MF.1}
\overline{Q}_{k}\left(t\right)=\overline{Q}_{k}\left(0\right)+\lambda_{k}t-\sum_{m=1}^{M}\mu_{k}\overline{T}_{m,k}\left(t\right),\\
\end{equation}
para $k=1,2,\ldots,K$.\\
\begin{equation}\label{Eq.MF.2}
\overline{Q}_{k}\left(t\right)\geq0\textrm{ para
}k=1,2,\ldots,K.\\
\end{equation}

\begin{equation}\label{Eq.MF.3}
\overline{T}_{m,k}\left(0\right)=0,\textrm{ y }\overline{T}_{m,k}\left(\cdot\right)\textrm{ es no decreciente},\\
\end{equation}
para $k=1,2,\ldots,K$ y $m=1,2,\ldots,M$.\\
\begin{equation}\label{Eq.MF.4}
\sum_{k=1}^{K}\overline{T}_{m,k}^{0}\left(t\right)+\overline{T}_{m,k}\left(t\right)=t\textrm{
para }m=1,2,\ldots,M.\\
\end{equation}


\begin{Def}[Definici\'on 4.1, Dai \cite{Dai}]\label{Def.Modelo.Flujo}
Sea una disciplina de servicio espec\'ifica. Cualquier l\'imite
$\left(\overline{Q}\left(\cdot\right),\overline{T}\left(\cdot\right),\overline{T}^{0}\left(\cdot\right)\right)$
en (\ref{Eq.Punto.Limite}) es un {\em flujo l\'imite} de la
disciplina. Cualquier soluci\'on (\ref{Eq.MF.1})-(\ref{Eq.MF.4})
es llamado flujo soluci\'on de la disciplina.
\end{Def}

\begin{Def}
Se dice que el modelo de flujo l\'imite, modelo de flujo, de la
disciplina de la cola es estable si existe una constante
$\delta>0$ que depende de $\mu,\lambda$ y $P$ solamente, tal que
cualquier flujo l\'imite con
$|\overline{Q}\left(0\right)|+|\overline{U}|+|\overline{V}|=1$, se
tiene que $\overline{Q}\left(\cdot+\delta\right)\equiv0$.
\end{Def}

Si se hace $|x|\rightarrow\infty$ sin restringir ninguna de las
componentes, tambi\'en se obtienen un modelo de flujo, pero en
este caso el residual de los procesos de arribo y servicio
introducen un retraso:
\begin{Teo}[Teorema 4.2, Dai \cite{Dai}]\label{Tma.4.2.Dai}
Sea una disciplina fija para la cola, suponga que se cumplen las
condiciones (A1)-(A3). Si el modelo de flujo l\'imite de la
disciplina de la cola es estable, entonces la cadena de Markov $X$
que describe la din\'amica de la red bajo la disciplina es Harris
recurrente positiva.
\end{Teo}

Ahora se procede a escalar el espacio y el tiempo para reducir la
aparente fluctuaci\'on del modelo. Consid\'erese el proceso
\begin{equation}\label{Eq.3.7}
\overline{Q}^{x}\left(t\right)=\frac{1}{|x|}Q^{x}\left(|x|t\right).
\end{equation}
A este proceso se le conoce como el flujo escalado, y cualquier
l\'imite $\overline{Q}^{x}\left(t\right)$ es llamado flujo
l\'imite del proceso de longitud de la cola. Haciendo
$|q|\rightarrow\infty$ mientras se mantiene el resto de las
componentes fijas, cualquier punto l\'imite del proceso de
longitud de la cola normalizado $\overline{Q}^{x}$ es soluci\'on
del siguiente modelo de flujo.


\begin{Def}[Definici\'on 3.3, Dai y Meyn \cite{DaiSean}]
El modelo de flujo es estable si existe un tiempo fijo $t_{0}$ tal
que $\overline{Q}\left(t\right)=0$, con $t\geq t_{0}$, para
cualquier $\overline{Q}\left(\cdot\right)\in\mathcal{Q}$ que
cumple con $|\overline{Q}\left(0\right)|=1$.
\end{Def}

\begin{Lemma}[Lema 3.1, Dai y Meyn \cite{DaiSean}]
Si el modelo de flujo definido por (\ref{Eq.MF.1})-(\ref{Eq.MF.4})
es estable, entonces el modelo de flujo retrasado es tambi\'en
estable, es decir, existe $t_{0}>0$ tal que
$\overline{Q}\left(t\right)=0$ para cualquier $t\geq t_{0}$, para
cualquier soluci\'on del modelo de flujo retrasado cuya
condici\'on inicial $\overline{x}$ satisface que
$|\overline{x}|=|\overline{Q}\left(0\right)|+|\overline{A}\left(0\right)|+|\overline{B}\left(0\right)|\leq1$.
\end{Lemma}


Ahora ya estamos en condiciones de enunciar los resultados principales:


\begin{Teo}[Teorema 2.1, Down \cite{Down}]\label{Tma2.1.Down}
Suponga que el modelo de flujo es estable, y que se cumplen los supuestos (A1) y (A2), entonces
\begin{itemize}
\item[i)] Para alguna constante $\kappa_{p}$, y para cada
condici\'on inicial $x\in X$
\begin{equation}\label{Estability.Eq1}
\limsup_{t\rightarrow\infty}\frac{1}{t}\int_{0}^{t}\esp_{x}\left[|Q\left(s\right)|^{p}\right]ds\leq\kappa_{p},
\end{equation}
donde $p$ es el entero dado en (A2).
\end{itemize}
Si adem\'as se cumple la condici\'on (A3), entonces para cada
condici\'on inicial:
\begin{itemize}
\item[ii)] Los momentos transitorios convergen a su estado
estacionario:
 \begin{equation}\label{Estability.Eq2}
lim_{t\rightarrow\infty}\esp_{x}\left[Q_{k}\left(t\right)^{r}\right]=\esp_{\pi}\left[Q_{k}\left(0\right)^{r}\right]\leq\kappa_{r},
\end{equation}
para $r=1,2,\ldots,p$ y $k=1,2,\ldots,K$. Donde $\pi$ es la
probabilidad invariante para $X$.

\item[iii)]  El primer momento converge con raz\'on $t^{p-1}$:
\begin{equation}\label{Estability.Eq3}
lim_{t\rightarrow\infty}t^{p-1}|\esp_{x}\left[Q_{k}\left(t\right)\right]-\esp_{\pi}\left[Q_{k}\left(0\right)\right]|=0.
\end{equation}

\item[iv)] La {\em Ley Fuerte de los grandes n\'umeros} se cumple:
\begin{equation}\label{Estability.Eq4}
lim_{t\rightarrow\infty}\frac{1}{t}\int_{0}^{t}Q_{k}^{r}\left(s\right)ds=\esp_{\pi}\left[Q_{k}\left(0\right)^{r}\right],\textrm{
}\prob_{x}\textrm{-c.s.}
\end{equation}
para $r=1,2,\ldots,p$ y $k=1,2,\ldots,K$.
\end{itemize}
\end{Teo}

La contribuci\'on de Down a la teor\'ia de los {\emph {sistemas de
visitas c\'iclicas}}, es la relaci\'on que hay entre la
estabilidad del sistema con el comportamiento de las medidas de
desempe\~no, es decir, la condici\'on suficiente para poder
garantizar la convergencia del proceso de la longitud de la cola
as\'i como de por los menos los dos primeros momentos adem\'as de
una versi\'on de la Ley Fuerte de los Grandes N\'umeros para los
sistemas de visitas.


\begin{Teo}[Teorema 2.3, Down \cite{Down}]\label{Tma2.3.Down}
Considere el siguiente valor:
\begin{equation}\label{Eq.Rho.1serv}
\rho=\sum_{k=1}^{K}\rho_{k}+max_{1\leq j\leq K}\left(\frac{\lambda_{j}}{\sum_{s=1}^{S}p_{js}\overline{N}_{s}}\right)\delta^{*}
\end{equation}
\begin{itemize}
\item[i)] Si $\rho<1$ entonces la red es estable, es decir, se
cumple el Teorema \ref{Tma2.1.Down}.

\item[ii)] Si $\rho>1$ entonces la red es inestable, es decir, se
cumple el Teorema \ref{Tma2.2.Down}
\end{itemize}
\end{Teo}

%_________________________________________________________________________
%\section{DESARROLLO DEL TEMA Y/O METODOLOG\'IA}
%_________________________________________________________________________
\subsection{Supuestos}
%_________________________________________________________________________
Consideremos el caso en el que se tienen varias colas a las cuales
llegan uno o varios servidores para dar servicio a los usuarios
que se encuentran presentes en la cola, como ya se mencion\'o hay
varios tipos de pol\'iticas de servicio, incluso podr\'ia ocurrir
que la manera en que atiende al resto de las colas sea distinta a
como lo hizo en las anteriores.\\

Para ejemplificar los sistemas de visitas c\'iclicas se
considerar\'a el caso en que a las colas los usuarios son atendidos con
una s\'ola pol\'itica de servicio.\\


Si $\omega$ es el n\'umero de usuarios en la cola al comienzo del
periodo de servicio y $N\left(\omega\right)$ es el n\'umero de
usuarios que son atendidos con una pol\'itica en espec\'ifico
durante el periodo de servicio, entonces se asume que:
\begin{itemize}
\item[1)]\label{S1}$lim_{\omega\rightarrow\infty}\esp\left[N\left(\omega\right)\right]=\overline{N}>0$;
\item[2)]\label{S2}$\esp\left[N\left(\omega\right)\right]\leq\overline{N}$
para cualquier valor de $\omega$.
\end{itemize}
La manera en que atiende el servidor $m$-\'esimo, es la siguiente:
\begin{itemize}
\item Al t\'ermino de la visita a la cola $j$, el servidor cambia
a la cola $j^{'}$ con probabilidad $r_{j,j^{'}}^{m}$

\item La $n$-\'esima vez que el servidor cambia de la cola $j$ a
$j'$, va acompa\~nada con el tiempo de cambio de longitud
$\delta_{j,j^{'}}^{m}\left(n\right)$, con
$\delta_{j,j^{'}}^{m}\left(n\right)$, $n\geq1$, variables
aleatorias independientes e id\'enticamente distribuidas, tales
que $\esp\left[\delta_{j,j^{'}}^{m}\left(1\right)\right]\geq0$.

\item Sea $\left\{p_{j}^{m}\right\}$ la distribuci\'on invariante
estacionaria \'unica para la Cadena de Markov con matriz de
transici\'on $\left(r_{j,j^{'}}^{m}\right)$, se supone que \'esta
existe.

\item Finalmente, se define el tiempo promedio total de traslado
entre las colas como
\begin{equation}
\delta^{*}:=\sum_{j,j^{'}}p_{j}^{m}r_{j,j^{'}}^{m}\esp\left[\delta_{j,j^{'}}^{m}\left(i\right)\right].
\end{equation}
\end{itemize}

Consideremos el caso donde los tiempos entre arribo a cada una de
las colas, $\left\{\xi_{k}\left(n\right)\right\}_{n\geq1}$ son
variables aleatorias independientes a id\'enticamente
distribuidas, y los tiempos de servicio en cada una de las colas
se distribuyen de manera independiente e id\'enticamente
distribuidas $\left\{\eta_{k}\left(n\right)\right\}_{n\geq1}$;
adem\'as ambos procesos cumplen la condici\'on de ser
independientes entre s\'i. Para la $k$-\'esima cola se define la
tasa de arribo por
$\lambda_{k}=1/\esp\left[\xi_{k}\left(1\right)\right]$ y la tasa
de servicio como
$\mu_{k}=1/\esp\left[\eta_{k}\left(1\right)\right]$, finalmente se
define la carga de la cola como $\rho_{k}=\lambda_{k}/\mu_{k}$,
donde se pide que $\rho=\sum_{k=1}^{K}\rho_{k}<1$, para garantizar
la estabilidad del sistema, esto es cierto para las pol\'iticas de
servicio exhaustiva y cerrada, ver Geetor \cite{Getoor}.\\

Si denotamos por
\begin{itemize}
\item $Q_{k}\left(t\right)$ el n\'umero de usuarios presentes en
la cola $k$ al tiempo $t$; \item $A_{k}\left(t\right)$ los
residuales de los tiempos entre arribos a la cola $k$; para cada
servidor $m$; \item $B_{m}\left(t\right)$ denota a los residuales
de los tiempos de servicio al tiempo $t$; \item
$B_{m}^{0}\left(t\right)$ los residuales de los tiempos de
traslado de la cola $k$ a la pr\'oxima por atender al tiempo $t$,

\item sea
$C_{m}\left(t\right)$ el n\'umero de usuarios atendidos durante la
visita del servidor a la cola $k$ al tiempo $t$.
\end{itemize}


En este sentido, el proceso para el sistema de visitas se puede
definir como:

\begin{equation}\label{Esp.Edos.Down}
X\left(t\right)^{T}=\left(Q_{k}\left(t\right),A_{k}\left(t\right),B_{m}\left(t\right),B_{m}^{0}\left(t\right),C_{m}\left(t\right)\right),
\end{equation}
para $k=1,\ldots,K$ y $m=1,2,\ldots,M$, donde $T$ indica que es el
transpuesto del vector que se est\'a definiendo. El proceso $X$
evoluciona en el espacio de estados:
$\mathbb{X}=\ent_{+}^{K}\times\rea_{+}^{K}\times\left(\left\{1,2,\ldots,K\right\}\times\left\{1,2,\ldots,S\right\}\right)^{M}\times\rea_{+}^{K}\times\ent_{+}^{K}$.\\

El sistema aqu\'i descrito debe de cumplir con los siguientes supuestos b\'asicos de un sistema de visitas:
%__________________________________________________________________________
\subsubsection{Supuestos B\'asicos}
%__________________________________________________________________________
\begin{itemize}
\item[A1)] Los procesos
$\xi_{1},\ldots,\xi_{K},\eta_{1},\ldots,\eta_{K}$ son mutuamente
independientes y son sucesiones independientes e id\'enticamente
distribuidas.

\item[A2)] Para alg\'un entero $p\geq1$
\begin{eqnarray*}
\esp\left[\xi_{l}\left(1\right)^{p+1}\right]&<&\infty\textrm{ para }l=1,\ldots,K\textrm{ y }\\
\esp\left[\eta_{k}\left(1\right)^{p+1}\right]&<&\infty\textrm{
para }k=1,\ldots,K.
\end{eqnarray*}
donde $\mathcal{A}$ es la clase de posibles arribos.

\item[A3)] Para cada $k=1,2,\ldots,K$ existe una funci\'on
positiva $q_{k}\left(\cdot\right)$ definida en $\rea_{+}$, y un
entero $j_{k}$, tal que
\begin{eqnarray}
P\left(\xi_{k}\left(1\right)\geq x\right)&>&0\textrm{, para todo }x>0,\\
P\left\{a\leq\sum_{i=1}^{j_{k}}\xi_{k}\left(i\right)\leq
b\right\}&\geq&\int_{a}^{b}q_{k}\left(x\right)dx, \textrm{ }0\leq
a<b.
\end{eqnarray}
\end{itemize}

En lo que respecta al supuesto (A3), en Dai y Meyn \cite{DaiSean}
hacen ver que este se puede sustituir por

\begin{itemize}
\item[A3')] Para el Proceso de Markov $X$, cada subconjunto
compacto del espacio de estados de $X$ es un conjunto peque\~no,
ver definici\'on \ref{Def.Cto.Peq.}.
\end{itemize}

Es por esta raz\'on que con la finalidad de poder hacer uso de
$A3^{'})$ es necesario recurrir a los Procesos de Harris y en
particular a los Procesos Harris Recurrente, ver \cite{Dai,
DaiSean}.
%_______________________________________________________________________
\subsection{Procesos Harris Recurrente}
%_______________________________________________________________________

Por el supuesto (A1) conforme a Davis \cite{Davis}, se puede
definir el proceso de saltos correspondiente de manera tal que
satisfaga el supuesto (A3'), de hecho la demostraci\'on est\'a
basada en la l\'inea de argumentaci\'on de Davis, \cite{Davis},
p\'aginas 362-364.\\

Entonces se tiene un espacio de estados en el cual el proceso $X$
satisface la Propiedad Fuerte de Markov, ver Dai y Meyn
\cite{DaiSean}, dado por

\[\left(\Omega,\mathcal{F},\mathcal{F}_{t},X\left(t\right),\theta_{t},P_{x}\right),\]
adem\'as de ser un proceso de Borel Derecho (Sharpe \cite{Sharpe})
en el espacio de estados medible
$\left(\mathbb{X},\mathcal{B}_\mathbb{X}\right)$. El Proceso
$X=\left\{X\left(t\right),t\geq0\right\}$ tiene trayectorias
continuas por la derecha, est\'a definido en
$\left(\Omega,\mathcal{F}\right)$ y est\'a adaptado a
$\left\{\mathcal{F}_{t},t\geq0\right\}$; la colecci\'on
$\left\{P_{x},x\in \mathbb{X}\right\}$ son medidas de probabilidad
en $\left(\Omega,\mathcal{F}\right)$ tales que para todo $x\in
\mathbb{X}$
\[P_{x}\left\{X\left(0\right)=x\right\}=1,\] y
\[E_{x}\left\{f\left(X\circ\theta_{t}\right)|\mathcal{F}_{t}\right\}=E_{X}\left(\tau\right)f\left(X\right),\]
en $\left\{\tau<\infty\right\}$, $P_{x}$-c.s., con $\theta_{t}$
definido como el operador shift.


Donde $\tau$ es un $\mathcal{F}_{t}$-tiempo de paro
\[\left(X\circ\theta_{\tau}\right)\left(w\right)=\left\{X\left(\tau\left(w\right)+t,w\right),t\geq0\right\},\]
y $f$ es una funci\'on de valores reales acotada y medible, ver \cite{Dai, KaspiMandelbaum}.\\

Sea $P^{t}\left(x,D\right)$, $D\in\mathcal{B}_{\mathbb{X}}$,
$t\geq0$ la probabilidad de transici\'on de $X$ queda definida
como:
\[P^{t}\left(x,D\right)=P_{x}\left(X\left(t\right)\in
D\right).\]


\begin{Def}
Una medida no cero $\pi$ en
$\left(\mathbb{X},\mathcal{B}_{\mathbb{X}}\right)$ es invariante
para $X$ si $\pi$ es $\sigma$-finita y
\[\pi\left(D\right)=\int_{\mathbb{X}}P^{t}\left(x,D\right)\pi\left(dx\right),\]
para todo $D\in \mathcal{B}_{\mathbb{X}}$, con $t\geq0$.
\end{Def}

\begin{Def}
El proceso de Markov $X$ es llamado Harris Recurrente si existe
una medida de probabilidad $\nu$ en
$\left(\mathbb{X},\mathcal{B}_{\mathbb{X}}\right)$, tal que si
$\nu\left(D\right)>0$ y $D\in\mathcal{B}_{\mathbb{X}}$
\[P_{x}\left\{\tau_{D}<\infty\right\}\equiv1,\] cuando
$\tau_{D}=inf\left\{t\geq0:X_{t}\in D\right\}$.
\end{Def}

\begin{Note}
\begin{itemize}
\item[i)] Si $X$ es Harris recurrente, entonces existe una \'unica
medida invariante $\pi$ (Getoor \cite{Getoor}).

\item[ii)] Si la medida invariante es finita, entonces puede
normalizarse a una medida de probabilidad, en este caso al proceso
$X$ se le llama Harris recurrente positivo.


\item[iii)] Cuando $X$ es Harris recurrente positivo se dice que
la disciplina de servicio es estable. En este caso $\pi$ denota la
distribuci\'on estacionaria y hacemos
\[P_{\pi}\left(\cdot\right)=\int_{\mathbf{X}}P_{x}\left(\cdot\right)\pi\left(dx\right),\]
y se utiliza $E_{\pi}$ para denotar el operador esperanza
correspondiente, ver \cite{DaiSean}.
\end{itemize}
\end{Note}

\begin{Def}\label{Def.Cto.Peq.}
Un conjunto $D\in\mathcal{B_{\mathbb{X}}}$ es llamado peque\~no si
existe un $t>0$, una medida de probabilidad $\nu$ en
$\mathcal{B_{\mathbb{X}}}$, y un $\delta>0$ tal que
\[P^{t}\left(x,A\right)\geq\delta\nu\left(A\right),\] para $x\in
D,A\in\mathcal{B_{\mathbb{X}}}$.
\end{Def}

La siguiente serie de resultados vienen enunciados y demostrados
en Dai \cite{Dai}:
\begin{Lema}[Lema 3.1, Dai \cite{Dai}]
Sea $B$ conjunto peque\~no cerrado, supongamos que
$P_{x}\left(\tau_{B}<\infty\right)\equiv1$ y que para alg\'un
$\delta>0$ se cumple que
\begin{equation}\label{Eq.3.1}
\sup\esp_{x}\left[\tau_{B}\left(\delta\right)\right]<\infty,
\end{equation}
donde
$\tau_{B}\left(\delta\right)=inf\left\{t\geq\delta:X\left(t\right)\in
B\right\}$. Entonces, $X$ es un proceso Harris recurrente
positivo.
\end{Lema}

\begin{Lema}[Lema 3.1, Dai \cite{Dai}]\label{Lema.3.}
Bajo el supuesto (A3), el conjunto
$B=\left\{x\in\mathbb{X}:|x|\leq k\right\}$ es un conjunto
peque\~no cerrado para cualquier $k>0$.
\end{Lema}

\begin{Teo}[Teorema 3.1, Dai \cite{Dai}]\label{Tma.3.1}
Si existe un $\delta>0$ tal que
\begin{equation}
lim_{|x|\rightarrow\infty}\frac{1}{|x|}\esp|X^{x}\left(|x|\delta\right)|=0,
\end{equation}
donde $X^{x}$ se utiliza para denotar que el proceso $X$ comienza
a partir de $x$, entonces la ecuaci\'on (\ref{Eq.3.1}) se cumple
para $B=\left\{x\in\mathbb{X}:|x|\leq k\right\}$ con alg\'un
$k>0$. En particular, $X$ es Harris recurrente positivo.
\end{Teo}

Entonces, tenemos que el proceso $X$ es un proceso de Markov que
cumple con los supuestos $A1)$-$A3)$, lo que falta de hacer es
construir el Modelo de Flujo bas\'andonos en lo hasta ahora
presentado.
%_______________________________________________________________________
\subsection{Modelo de Flujo}
%_______________________________________________________________________

Dada una condici\'on inicial $x\in\mathbb{X}$, sea

\begin{itemize}
\item $Q_{k}^{x}\left(t\right)$ la longitud de la cola al tiempo
$t$,

\item $T_{m,k}^{x}\left(t\right)$ el tiempo acumulado, al tiempo
$t$, que tarda el servidor $m$ en atender a los usuarios de la
cola $k$.

\item $T_{m,k}^{x,0}\left(t\right)$ el tiempo acumulado, al tiempo
$t$, que tarda el servidor $m$ en trasladarse a otra cola a partir de la $k$-\'esima.\\
\end{itemize}

Sup\'ongase que la funci\'on
$\left(\overline{Q}\left(\cdot\right),\overline{T}_{m}
\left(\cdot\right),\overline{T}_{m}^{0} \left(\cdot\right)\right)$
para $m=1,2,\ldots,M$ es un punto l\'imite de
\begin{equation}\label{Eq.Punto.Limite}
\left(\frac{1}{|x|}Q^{x}\left(|x|t\right),\frac{1}{|x|}T_{m}^{x}\left(|x|t\right),\frac{1}{|x|}T_{m}^{x,0}\left(|x|t\right)\right)
\end{equation}
para $m=1,2,\ldots,M$, cuando $x\rightarrow\infty$, ver
\cite{Down}. Entonces
$\left(\overline{Q}\left(t\right),\overline{T}_{m}
\left(t\right),\overline{T}_{m}^{0} \left(t\right)\right)$ es un
flujo l\'imite del sistema. Al conjunto de todos las posibles
flujos l\'imite se le llama {\emph{Modelo de Flujo}} y se le
denotar\'a por $\mathcal{Q}$, ver \cite{Down, Dai, DaiSean}.\\

El modelo de flujo satisface el siguiente conjunto de ecuaciones:

\begin{equation}\label{Eq.MF.1}
\overline{Q}_{k}\left(t\right)=\overline{Q}_{k}\left(0\right)+\lambda_{k}t-\sum_{m=1}^{M}\mu_{k}\overline{T}_{m,k}\left(t\right),\\
\end{equation}
para $k=1,2,\ldots,K$.\\
\begin{equation}\label{Eq.MF.2}
\overline{Q}_{k}\left(t\right)\geq0\textrm{ para
}k=1,2,\ldots,K.\\
\end{equation}

\begin{equation}\label{Eq.MF.3}
\overline{T}_{m,k}\left(0\right)=0,\textrm{ y }\overline{T}_{m,k}\left(\cdot\right)\textrm{ es no decreciente},\\
\end{equation}
para $k=1,2,\ldots,K$ y $m=1,2,\ldots,M$.\\
\begin{equation}\label{Eq.MF.4}
\sum_{k=1}^{K}\overline{T}_{m,k}^{0}\left(t\right)+\overline{T}_{m,k}\left(t\right)=t\textrm{
para }m=1,2,\ldots,M.\\
\end{equation}


\begin{Def}[Definici\'on 4.1, Dai \cite{Dai}]\label{Def.Modelo.Flujo}
Sea una disciplina de servicio espec\'ifica. Cualquier l\'imite
$\left(\overline{Q}\left(\cdot\right),\overline{T}\left(\cdot\right),\overline{T}^{0}\left(\cdot\right)\right)$
en (\ref{Eq.Punto.Limite}) es un {\em flujo l\'imite} de la
disciplina. Cualquier soluci\'on (\ref{Eq.MF.1})-(\ref{Eq.MF.4})
es llamado flujo soluci\'on de la disciplina.
\end{Def}

\begin{Def}
Se dice que el modelo de flujo l\'imite, modelo de flujo, de la
disciplina de la cola es estable si existe una constante
$\delta>0$ que depende de $\mu,\lambda$ y $P$ solamente, tal que
cualquier flujo l\'imite con
$|\overline{Q}\left(0\right)|+|\overline{U}|+|\overline{V}|=1$, se
tiene que $\overline{Q}\left(\cdot+\delta\right)\equiv0$.
\end{Def}

Si se hace $|x|\rightarrow\infty$ sin restringir ninguna de las
componentes, tambi\'en se obtienen un modelo de flujo, pero en
este caso el residual de los procesos de arribo y servicio
introducen un retraso:
\begin{Teo}[Teorema 4.2, Dai \cite{Dai}]\label{Tma.4.2.Dai}
Sea una disciplina fija para la cola, suponga que se cumplen las
condiciones (A1)-(A3). Si el modelo de flujo l\'imite de la
disciplina de la cola es estable, entonces la cadena de Markov $X$
que describe la din\'amica de la red bajo la disciplina es Harris
recurrente positiva.
\end{Teo}

Ahora se procede a escalar el espacio y el tiempo para reducir la
aparente fluctuaci\'on del modelo. Consid\'erese el proceso
\begin{equation}\label{Eq.3.7}
\overline{Q}^{x}\left(t\right)=\frac{1}{|x|}Q^{x}\left(|x|t\right).
\end{equation}
A este proceso se le conoce como el flujo escalado, y cualquier
l\'imite $\overline{Q}^{x}\left(t\right)$ es llamado flujo
l\'imite del proceso de longitud de la cola. Haciendo
$|q|\rightarrow\infty$ mientras se mantiene el resto de las
componentes fijas, cualquier punto l\'imite del proceso de
longitud de la cola normalizado $\overline{Q}^{x}$ es soluci\'on
del siguiente modelo de flujo.


\begin{Def}[Definici\'on 3.3, Dai y Meyn \cite{DaiSean}]
El modelo de flujo es estable si existe un tiempo fijo $t_{0}$ tal
que $\overline{Q}\left(t\right)=0$, con $t\geq t_{0}$, para
cualquier $\overline{Q}\left(\cdot\right)\in\mathcal{Q}$ que
cumple con $|\overline{Q}\left(0\right)|=1$.
\end{Def}

\begin{Lemma}[Lema 3.1, Dai y Meyn \cite{DaiSean}]
Si el modelo de flujo definido por (\ref{Eq.MF.1})-(\ref{Eq.MF.4})
es estable, entonces el modelo de flujo retrasado es tambi\'en
estable, es decir, existe $t_{0}>0$ tal que
$\overline{Q}\left(t\right)=0$ para cualquier $t\geq t_{0}$, para
cualquier soluci\'on del modelo de flujo retrasado cuya
condici\'on inicial $\overline{x}$ satisface que
$|\overline{x}|=|\overline{Q}\left(0\right)|+|\overline{A}\left(0\right)|+|\overline{B}\left(0\right)|\leq1$.
\end{Lemma}


Ahora ya estamos en condiciones de enunciar los resultados principales:


\begin{Teo}[Teorema 2.1, Down \cite{Down}]\label{Tma2.1.Down}
Suponga que el modelo de flujo es estable, y que se cumplen los supuestos (A1) y (A2), entonces
\begin{itemize}
\item[i)] Para alguna constante $\kappa_{p}$, y para cada
condici\'on inicial $x\in X$
\begin{equation}\label{Estability.Eq1}
\limsup_{t\rightarrow\infty}\frac{1}{t}\int_{0}^{t}\esp_{x}\left[|Q\left(s\right)|^{p}\right]ds\leq\kappa_{p},
\end{equation}
donde $p$ es el entero dado en (A2).
\end{itemize}
Si adem\'as se cumple la condici\'on (A3), entonces para cada
condici\'on inicial:
\begin{itemize}
\item[ii)] Los momentos transitorios convergen a su estado
estacionario:
 \begin{equation}\label{Estability.Eq2}
lim_{t\rightarrow\infty}\esp_{x}\left[Q_{k}\left(t\right)^{r}\right]=\esp_{\pi}\left[Q_{k}\left(0\right)^{r}\right]\leq\kappa_{r},
\end{equation}
para $r=1,2,\ldots,p$ y $k=1,2,\ldots,K$. Donde $\pi$ es la
probabilidad invariante para $X$.

\item[iii)]  El primer momento converge con raz\'on $t^{p-1}$:
\begin{equation}\label{Estability.Eq3}
lim_{t\rightarrow\infty}t^{p-1}|\esp_{x}\left[Q_{k}\left(t\right)\right]-\esp_{\pi}\left[Q_{k}\left(0\right)\right]|=0.
\end{equation}

\item[iv)] La {\em Ley Fuerte de los grandes n\'umeros} se cumple:
\begin{equation}\label{Estability.Eq4}
lim_{t\rightarrow\infty}\frac{1}{t}\int_{0}^{t}Q_{k}^{r}\left(s\right)ds=\esp_{\pi}\left[Q_{k}\left(0\right)^{r}\right],\textrm{
}\prob_{x}\textrm{-c.s.}
\end{equation}
para $r=1,2,\ldots,p$ y $k=1,2,\ldots,K$.
\end{itemize}
\end{Teo}

La contribuci\'on de Down a la teor\'ia de los {\emph {sistemas de
visitas c\'iclicas}}, es la relaci\'on que hay entre la
estabilidad del sistema con el comportamiento de las medidas de
desempe\~no, es decir, la condici\'on suficiente para poder
garantizar la convergencia del proceso de la longitud de la cola
as\'i como de por los menos los dos primeros momentos adem\'as de
una versi\'on de la Ley Fuerte de los Grandes N\'umeros para los
sistemas de visitas.


\begin{Teo}[Teorema 2.3, Down \cite{Down}]\label{Tma2.3.Down}
Considere el siguiente valor:
\begin{equation}\label{Eq.Rho.1serv}
\rho=\sum_{k=1}^{K}\rho_{k}+max_{1\leq j\leq K}\left(\frac{\lambda_{j}}{\sum_{s=1}^{S}p_{js}\overline{N}_{s}}\right)\delta^{*}
\end{equation}
\begin{itemize}
\item[i)] Si $\rho<1$ entonces la red es estable, es decir, se
cumple el Teorema \ref{Tma2.1.Down}.

\item[ii)] Si $\rho>1$ entonces la red es inestable, es decir, se
cumple el Teorema \ref{Tma2.2.Down}
\end{itemize}
\end{Teo}



El sistema aqu\'i descrito debe de cumplir con los siguientes supuestos b\'asicos de un sistema de visitas:
%__________________________________________________________________________
\subsubsection{Supuestos B\'asicos}
%__________________________________________________________________________
\begin{itemize}
\item[A1)] Los procesos
$\xi_{1},\ldots,\xi_{K},\eta_{1},\ldots,\eta_{K}$ son mutuamente
independientes y son sucesiones independientes e id\'enticamente
distribuidas.

\item[A2)] Para alg\'un entero $p\geq1$
\begin{eqnarray*}
\esp\left[\xi_{l}\left(1\right)^{p+1}\right]&<&\infty\textrm{ para }l=1,\ldots,K\textrm{ y }\\
\esp\left[\eta_{k}\left(1\right)^{p+1}\right]&<&\infty\textrm{
para }k=1,\ldots,K.
\end{eqnarray*}
donde $\mathcal{A}$ es la clase de posibles arribos.

\item[A3)] Para cada $k=1,2,\ldots,K$ existe una funci\'on
positiva $q_{k}\left(\cdot\right)$ definida en $\rea_{+}$, y un
entero $j_{k}$, tal que
\begin{eqnarray}
P\left(\xi_{k}\left(1\right)\geq x\right)&>&0\textrm{, para todo }x>0,\\
P\left\{a\leq\sum_{i=1}^{j_{k}}\xi_{k}\left(i\right)\leq
b\right\}&\geq&\int_{a}^{b}q_{k}\left(x\right)dx, \textrm{ }0\leq
a<b.
\end{eqnarray}
\end{itemize}

En lo que respecta al supuesto (A3), en Dai y Meyn \cite{DaiSean}
hacen ver que este se puede sustituir por

\begin{itemize}
\item[A3')] Para el Proceso de Markov $X$, cada subconjunto
compacto del espacio de estados de $X$ es un conjunto peque\~no,
ver definici\'on \ref{Def.Cto.Peq.}.
\end{itemize}

Es por esta raz\'on que con la finalidad de poder hacer uso de
$A3^{'})$ es necesario recurrir a los Procesos de Harris y en
particular a los Procesos Harris Recurrente, ver \cite{Dai,
DaiSean}.
%_______________________________________________________________________
\subsection{Procesos Harris Recurrente}
%_______________________________________________________________________

Por el supuesto (A1) conforme a Davis \cite{Davis}, se puede
definir el proceso de saltos correspondiente de manera tal que
satisfaga el supuesto (A3'), de hecho la demostraci\'on est\'a
basada en la l\'inea de argumentaci\'on de Davis, \cite{Davis},
p\'aginas 362-364.\\

Entonces se tiene un espacio de estados en el cual el proceso $X$
satisface la Propiedad Fuerte de Markov, ver Dai y Meyn
\cite{DaiSean}, dado por

\[\left(\Omega,\mathcal{F},\mathcal{F}_{t},X\left(t\right),\theta_{t},P_{x}\right),\]
adem\'as de ser un proceso de Borel Derecho (Sharpe \cite{Sharpe})
en el espacio de estados medible
$\left(\mathbb{X},\mathcal{B}_\mathbb{X}\right)$. El Proceso
$X=\left\{X\left(t\right),t\geq0\right\}$ tiene trayectorias
continuas por la derecha, est\'a definido en
$\left(\Omega,\mathcal{F}\right)$ y est\'a adaptado a
$\left\{\mathcal{F}_{t},t\geq0\right\}$; la colecci\'on
$\left\{P_{x},x\in \mathbb{X}\right\}$ son medidas de probabilidad
en $\left(\Omega,\mathcal{F}\right)$ tales que para todo $x\in
\mathbb{X}$
\[P_{x}\left\{X\left(0\right)=x\right\}=1,\] y
\[E_{x}\left\{f\left(X\circ\theta_{t}\right)|\mathcal{F}_{t}\right\}=E_{X}\left(\tau\right)f\left(X\right),\]
en $\left\{\tau<\infty\right\}$, $P_{x}$-c.s., con $\theta_{t}$
definido como el operador shift.


Donde $\tau$ es un $\mathcal{F}_{t}$-tiempo de paro
\[\left(X\circ\theta_{\tau}\right)\left(w\right)=\left\{X\left(\tau\left(w\right)+t,w\right),t\geq0\right\},\]
y $f$ es una funci\'on de valores reales acotada y medible, ver \cite{Dai, KaspiMandelbaum}.\\

Sea $P^{t}\left(x,D\right)$, $D\in\mathcal{B}_{\mathbb{X}}$,
$t\geq0$ la probabilidad de transici\'on de $X$ queda definida
como:
\[P^{t}\left(x,D\right)=P_{x}\left(X\left(t\right)\in
D\right).\]


\begin{Def}
Una medida no cero $\pi$ en
$\left(\mathbb{X},\mathcal{B}_{\mathbb{X}}\right)$ es invariante
para $X$ si $\pi$ es $\sigma$-finita y
\[\pi\left(D\right)=\int_{\mathbb{X}}P^{t}\left(x,D\right)\pi\left(dx\right),\]
para todo $D\in \mathcal{B}_{\mathbb{X}}$, con $t\geq0$.
\end{Def}

\begin{Def}
El proceso de Markov $X$ es llamado Harris Recurrente si existe
una medida de probabilidad $\nu$ en
$\left(\mathbb{X},\mathcal{B}_{\mathbb{X}}\right)$, tal que si
$\nu\left(D\right)>0$ y $D\in\mathcal{B}_{\mathbb{X}}$
\[P_{x}\left\{\tau_{D}<\infty\right\}\equiv1,\] cuando
$\tau_{D}=inf\left\{t\geq0:X_{t}\in D\right\}$.
\end{Def}

\begin{Note}
\begin{itemize}
\item[i)] Si $X$ es Harris recurrente, entonces existe una \'unica
medida invariante $\pi$ (Getoor \cite{Getoor}).

\item[ii)] Si la medida invariante es finita, entonces puede
normalizarse a una medida de probabilidad, en este caso al proceso
$X$ se le llama Harris recurrente positivo.


\item[iii)] Cuando $X$ es Harris recurrente positivo se dice que
la disciplina de servicio es estable. En este caso $\pi$ denota la
distribuci\'on estacionaria y hacemos
\[P_{\pi}\left(\cdot\right)=\int_{\mathbf{X}}P_{x}\left(\cdot\right)\pi\left(dx\right),\]
y se utiliza $E_{\pi}$ para denotar el operador esperanza
correspondiente, ver \cite{DaiSean}.
\end{itemize}
\end{Note}

\begin{Def}\label{Def.Cto.Peq.}
Un conjunto $D\in\mathcal{B_{\mathbb{X}}}$ es llamado peque\~no si
existe un $t>0$, una medida de probabilidad $\nu$ en
$\mathcal{B_{\mathbb{X}}}$, y un $\delta>0$ tal que
\[P^{t}\left(x,A\right)\geq\delta\nu\left(A\right),\] para $x\in
D,A\in\mathcal{B_{\mathbb{X}}}$.
\end{Def}

La siguiente serie de resultados vienen enunciados y demostrados
en Dai \cite{Dai}:
\begin{Lema}[Lema 3.1, Dai \cite{Dai}]
Sea $B$ conjunto peque\~no cerrado, supongamos que
$P_{x}\left(\tau_{B}<\infty\right)\equiv1$ y que para alg\'un
$\delta>0$ se cumple que
\begin{equation}\label{Eq.3.1}
\sup\esp_{x}\left[\tau_{B}\left(\delta\right)\right]<\infty,
\end{equation}
donde
$\tau_{B}\left(\delta\right)=inf\left\{t\geq\delta:X\left(t\right)\in
B\right\}$. Entonces, $X$ es un proceso Harris recurrente
positivo.
\end{Lema}

\begin{Lema}[Lema 3.1, Dai \cite{Dai}]\label{Lema.3.}
Bajo el supuesto (A3), el conjunto
$B=\left\{x\in\mathbb{X}:|x|\leq k\right\}$ es un conjunto
peque\~no cerrado para cualquier $k>0$.
\end{Lema}

\begin{Teo}[Teorema 3.1, Dai \cite{Dai}]\label{Tma.3.1}
Si existe un $\delta>0$ tal que
\begin{equation}
lim_{|x|\rightarrow\infty}\frac{1}{|x|}\esp|X^{x}\left(|x|\delta\right)|=0,
\end{equation}
donde $X^{x}$ se utiliza para denotar que el proceso $X$ comienza
a partir de $x$, entonces la ecuaci\'on (\ref{Eq.3.1}) se cumple
para $B=\left\{x\in\mathbb{X}:|x|\leq k\right\}$ con alg\'un
$k>0$. En particular, $X$ es Harris recurrente positivo.
\end{Teo}

Entonces, tenemos que el proceso $X$ es un proceso de Markov que
cumple con los supuestos $A1)$-$A3)$, lo que falta de hacer es
construir el Modelo de Flujo bas\'andonos en lo hasta ahora
presentado.
%_______________________________________________________________________
\subsection{Modelo de Flujo}
%_______________________________________________________________________

Dada una condici\'on inicial $x\in\mathbb{X}$, sea

\begin{itemize}
\item $Q_{k}^{x}\left(t\right)$ la longitud de la cola al tiempo
$t$,

\item $T_{m,k}^{x}\left(t\right)$ el tiempo acumulado, al tiempo
$t$, que tarda el servidor $m$ en atender a los usuarios de la
cola $k$.

\item $T_{m,k}^{x,0}\left(t\right)$ el tiempo acumulado, al tiempo
$t$, que tarda el servidor $m$ en trasladarse a otra cola a partir de la $k$-\'esima.\\
\end{itemize}

Sup\'ongase que la funci\'on
$\left(\overline{Q}\left(\cdot\right),\overline{T}_{m}
\left(\cdot\right),\overline{T}_{m}^{0} \left(\cdot\right)\right)$
para $m=1,2,\ldots,M$ es un punto l\'imite de
\begin{equation}\label{Eq.Punto.Limite}
\left(\frac{1}{|x|}Q^{x}\left(|x|t\right),\frac{1}{|x|}T_{m}^{x}\left(|x|t\right),\frac{1}{|x|}T_{m}^{x,0}\left(|x|t\right)\right)
\end{equation}
para $m=1,2,\ldots,M$, cuando $x\rightarrow\infty$, ver
\cite{Down}. Entonces
$\left(\overline{Q}\left(t\right),\overline{T}_{m}
\left(t\right),\overline{T}_{m}^{0} \left(t\right)\right)$ es un
flujo l\'imite del sistema. Al conjunto de todos las posibles
flujos l\'imite se le llama {\emph{Modelo de Flujo}} y se le
denotar\'a por $\mathcal{Q}$, ver \cite{Down, Dai, DaiSean}.\\

El modelo de flujo satisface el siguiente conjunto de ecuaciones:

\begin{equation}\label{Eq.MF.1}
\overline{Q}_{k}\left(t\right)=\overline{Q}_{k}\left(0\right)+\lambda_{k}t-\sum_{m=1}^{M}\mu_{k}\overline{T}_{m,k}\left(t\right),\\
\end{equation}
para $k=1,2,\ldots,K$.\\
\begin{equation}\label{Eq.MF.2}
\overline{Q}_{k}\left(t\right)\geq0\textrm{ para
}k=1,2,\ldots,K.\\
\end{equation}

\begin{equation}\label{Eq.MF.3}
\overline{T}_{m,k}\left(0\right)=0,\textrm{ y }\overline{T}_{m,k}\left(\cdot\right)\textrm{ es no decreciente},\\
\end{equation}
para $k=1,2,\ldots,K$ y $m=1,2,\ldots,M$.\\
\begin{equation}\label{Eq.MF.4}
\sum_{k=1}^{K}\overline{T}_{m,k}^{0}\left(t\right)+\overline{T}_{m,k}\left(t\right)=t\textrm{
para }m=1,2,\ldots,M.\\
\end{equation}


\begin{Def}[Definici\'on 4.1, Dai \cite{Dai}]\label{Def.Modelo.Flujo}
Sea una disciplina de servicio espec\'ifica. Cualquier l\'imite
$\left(\overline{Q}\left(\cdot\right),\overline{T}\left(\cdot\right),\overline{T}^{0}\left(\cdot\right)\right)$
en (\ref{Eq.Punto.Limite}) es un {\em flujo l\'imite} de la
disciplina. Cualquier soluci\'on (\ref{Eq.MF.1})-(\ref{Eq.MF.4})
es llamado flujo soluci\'on de la disciplina.
\end{Def}

\begin{Def}
Se dice que el modelo de flujo l\'imite, modelo de flujo, de la
disciplina de la cola es estable si existe una constante
$\delta>0$ que depende de $\mu,\lambda$ y $P$ solamente, tal que
cualquier flujo l\'imite con
$|\overline{Q}\left(0\right)|+|\overline{U}|+|\overline{V}|=1$, se
tiene que $\overline{Q}\left(\cdot+\delta\right)\equiv0$.
\end{Def}

Si se hace $|x|\rightarrow\infty$ sin restringir ninguna de las
componentes, tambi\'en se obtienen un modelo de flujo, pero en
este caso el residual de los procesos de arribo y servicio
introducen un retraso:
\begin{Teo}[Teorema 4.2, Dai \cite{Dai}]\label{Tma.4.2.Dai}
Sea una disciplina fija para la cola, suponga que se cumplen las
condiciones (A1)-(A3). Si el modelo de flujo l\'imite de la
disciplina de la cola es estable, entonces la cadena de Markov $X$
que describe la din\'amica de la red bajo la disciplina es Harris
recurrente positiva.
\end{Teo}

Ahora se procede a escalar el espacio y el tiempo para reducir la
aparente fluctuaci\'on del modelo. Consid\'erese el proceso
\begin{equation}\label{Eq.3.7}
\overline{Q}^{x}\left(t\right)=\frac{1}{|x|}Q^{x}\left(|x|t\right).
\end{equation}
A este proceso se le conoce como el flujo escalado, y cualquier
l\'imite $\overline{Q}^{x}\left(t\right)$ es llamado flujo
l\'imite del proceso de longitud de la cola. Haciendo
$|q|\rightarrow\infty$ mientras se mantiene el resto de las
componentes fijas, cualquier punto l\'imite del proceso de
longitud de la cola normalizado $\overline{Q}^{x}$ es soluci\'on
del siguiente modelo de flujo.


\begin{Def}[Definici\'on 3.3, Dai y Meyn \cite{DaiSean}]
El modelo de flujo es estable si existe un tiempo fijo $t_{0}$ tal
que $\overline{Q}\left(t\right)=0$, con $t\geq t_{0}$, para
cualquier $\overline{Q}\left(\cdot\right)\in\mathcal{Q}$ que
cumple con $|\overline{Q}\left(0\right)|=1$.
\end{Def}

\begin{Lemma}[Lema 3.1, Dai y Meyn \cite{DaiSean}]
Si el modelo de flujo definido por (\ref{Eq.MF.1})-(\ref{Eq.MF.4})
es estable, entonces el modelo de flujo retrasado es tambi\'en
estable, es decir, existe $t_{0}>0$ tal que
$\overline{Q}\left(t\right)=0$ para cualquier $t\geq t_{0}$, para
cualquier soluci\'on del modelo de flujo retrasado cuya
condici\'on inicial $\overline{x}$ satisface que
$|\overline{x}|=|\overline{Q}\left(0\right)|+|\overline{A}\left(0\right)|+|\overline{B}\left(0\right)|\leq1$.
\end{Lemma}


Ahora ya estamos en condiciones de enunciar los resultados principales:


\begin{Teo}[Teorema 2.1, Down \cite{Down}]\label{Tma2.1.Down}
Suponga que el modelo de flujo es estable, y que se cumplen los supuestos (A1) y (A2), entonces
\begin{itemize}
\item[i)] Para alguna constante $\kappa_{p}$, y para cada
condici\'on inicial $x\in X$
\begin{equation}\label{Estability.Eq1}
\limsup_{t\rightarrow\infty}\frac{1}{t}\int_{0}^{t}\esp_{x}\left[|Q\left(s\right)|^{p}\right]ds\leq\kappa_{p},
\end{equation}
donde $p$ es el entero dado en (A2).
\end{itemize}
Si adem\'as se cumple la condici\'on (A3), entonces para cada
condici\'on inicial:
\begin{itemize}
\item[ii)] Los momentos transitorios convergen a su estado
estacionario:
 \begin{equation}\label{Estability.Eq2}
lim_{t\rightarrow\infty}\esp_{x}\left[Q_{k}\left(t\right)^{r}\right]=\esp_{\pi}\left[Q_{k}\left(0\right)^{r}\right]\leq\kappa_{r},
\end{equation}
para $r=1,2,\ldots,p$ y $k=1,2,\ldots,K$. Donde $\pi$ es la
probabilidad invariante para $X$.

\item[iii)]  El primer momento converge con raz\'on $t^{p-1}$:
\begin{equation}\label{Estability.Eq3}
lim_{t\rightarrow\infty}t^{p-1}|\esp_{x}\left[Q_{k}\left(t\right)\right]-\esp_{\pi}\left[Q_{k}\left(0\right)\right]|=0.
\end{equation}

\item[iv)] La {\em Ley Fuerte de los grandes n\'umeros} se cumple:
\begin{equation}\label{Estability.Eq4}
lim_{t\rightarrow\infty}\frac{1}{t}\int_{0}^{t}Q_{k}^{r}\left(s\right)ds=\esp_{\pi}\left[Q_{k}\left(0\right)^{r}\right],\textrm{
}\prob_{x}\textrm{-c.s.}
\end{equation}
para $r=1,2,\ldots,p$ y $k=1,2,\ldots,K$.
\end{itemize}
\end{Teo}

La contribuci\'on de Down a la teor\'ia de los {\emph {sistemas de
visitas c\'iclicas}}, es la relaci\'on que hay entre la
estabilidad del sistema con el comportamiento de las medidas de
desempe\~no, es decir, la condici\'on suficiente para poder
garantizar la convergencia del proceso de la longitud de la cola
as\'i como de por los menos los dos primeros momentos adem\'as de
una versi\'on de la Ley Fuerte de los Grandes N\'umeros para los
sistemas de visitas.


\begin{Teo}[Teorema 2.3, Down \cite{Down}]\label{Tma2.3.Down}
Considere el siguiente valor:
\begin{equation}\label{Eq.Rho.1serv}
\rho=\sum_{k=1}^{K}\rho_{k}+max_{1\leq j\leq K}\left(\frac{\lambda_{j}}{\sum_{s=1}^{S}p_{js}\overline{N}_{s}}\right)\delta^{*}
\end{equation}
\begin{itemize}
\item[i)] Si $\rho<1$ entonces la red es estable, es decir, se
cumple el Teorema \ref{Tma2.1.Down}.

\item[ii)] Si $\rho>1$ entonces la red es inestable, es decir, se
cumple el Teorema \ref{Tma2.2.Down}
\end{itemize}
\end{Teo}


%_________________________________________________________________________
\subsection{Supuestos}
%_________________________________________________________________________
Consideremos el caso en el que se tienen varias colas a las cuales
llegan uno o varios servidores para dar servicio a los usuarios
que se encuentran presentes en la cola, como ya se mencion\'o hay
varios tipos de pol\'iticas de servicio, incluso podr\'ia ocurrir
que la manera en que atiende al resto de las colas sea distinta a
como lo hizo en las anteriores.\\

Para ejemplificar los sistemas de visitas c\'iclicas se
considerar\'a el caso en que a las colas los usuarios son atendidos con
una s\'ola pol\'itica de servicio.\\



Si $\omega$ es el n\'umero de usuarios en la cola al comienzo del
periodo de servicio y $N\left(\omega\right)$ es el n\'umero de
usuarios que son atendidos con una pol\'itica en espec\'ifico
durante el periodo de servicio, entonces se asume que:
\begin{itemize}
\item[1)]\label{S1}$lim_{\omega\rightarrow\infty}\esp\left[N\left(\omega\right)\right]=\overline{N}>0$;
\item[2)]\label{S2}$\esp\left[N\left(\omega\right)\right]\leq\overline{N}$
para cualquier valor de $\omega$.
\end{itemize}
La manera en que atiende el servidor $m$-\'esimo, es la siguiente:
\begin{itemize}
\item Al t\'ermino de la visita a la cola $j$, el servidor cambia
a la cola $j^{'}$ con probabilidad $r_{j,j^{'}}^{m}$

\item La $n$-\'esima vez que el servidor cambia de la cola $j$ a
$j'$, va acompa\~nada con el tiempo de cambio de longitud
$\delta_{j,j^{'}}^{m}\left(n\right)$, con
$\delta_{j,j^{'}}^{m}\left(n\right)$, $n\geq1$, variables
aleatorias independientes e id\'enticamente distribuidas, tales
que $\esp\left[\delta_{j,j^{'}}^{m}\left(1\right)\right]\geq0$.

\item Sea $\left\{p_{j}^{m}\right\}$ la distribuci\'on invariante
estacionaria \'unica para la Cadena de Markov con matriz de
transici\'on $\left(r_{j,j^{'}}^{m}\right)$, se supone que \'esta
existe.

\item Finalmente, se define el tiempo promedio total de traslado
entre las colas como
\begin{equation}
\delta^{*}:=\sum_{j,j^{'}}p_{j}^{m}r_{j,j^{'}}^{m}\esp\left[\delta_{j,j^{'}}^{m}\left(i\right)\right].
\end{equation}
\end{itemize}

Consideremos el caso donde los tiempos entre arribo a cada una de
las colas, $\left\{\xi_{k}\left(n\right)\right\}_{n\geq1}$ son
variables aleatorias independientes a id\'enticamente
distribuidas, y los tiempos de servicio en cada una de las colas
se distribuyen de manera independiente e id\'enticamente
distribuidas $\left\{\eta_{k}\left(n\right)\right\}_{n\geq1}$;
adem\'as ambos procesos cumplen la condici\'on de ser
independientes entre s\'i. Para la $k$-\'esima cola se define la
tasa de arribo por
$\lambda_{k}=1/\esp\left[\xi_{k}\left(1\right)\right]$ y la tasa
de servicio como
$\mu_{k}=1/\esp\left[\eta_{k}\left(1\right)\right]$, finalmente se
define la carga de la cola como $\rho_{k}=\lambda_{k}/\mu_{k}$,
donde se pide que $\rho=\sum_{k=1}^{K}\rho_{k}<1$, para garantizar
la estabilidad del sistema, esto es cierto para las pol\'iticas de
servicio exhaustiva y cerrada, ver Geetor \cite{Getoor}.\\

Si denotamos por
\begin{itemize}
\item $Q_{k}\left(t\right)$ el n\'umero de usuarios presentes en
la cola $k$ al tiempo $t$; \item $A_{k}\left(t\right)$ los
residuales de los tiempos entre arribos a la cola $k$; para cada
servidor $m$; \item $B_{m}\left(t\right)$ denota a los residuales
de los tiempos de servicio al tiempo $t$; \item
$B_{m}^{0}\left(t\right)$ los residuales de los tiempos de
traslado de la cola $k$ a la pr\'oxima por atender al tiempo $t$,

\item sea
$C_{m}\left(t\right)$ el n\'umero de usuarios atendidos durante la
visita del servidor a la cola $k$ al tiempo $t$.
\end{itemize}


En este sentido, el proceso para el sistema de visitas se puede
definir como:

\begin{equation}\label{Esp.Edos.Down}
X\left(t\right)^{T}=\left(Q_{k}\left(t\right),A_{k}\left(t\right),B_{m}\left(t\right),B_{m}^{0}\left(t\right),C_{m}\left(t\right)\right),
\end{equation}
para $k=1,\ldots,K$ y $m=1,2,\ldots,M$, donde $T$ indica que es el
transpuesto del vector que se est\'a definiendo. El proceso $X$
evoluciona en el espacio de estados:
$\mathbb{X}=\ent_{+}^{K}\times\rea_{+}^{K}\times\left(\left\{1,2,\ldots,K\right\}\times\left\{1,2,\ldots,S\right\}\right)^{M}\times\rea_{+}^{K}\times\ent_{+}^{K}$.\\

El sistema aqu\'i descrito debe de cumplir con los siguientes supuestos b\'asicos de un sistema de visitas:
%__________________________________________________________________________
\subsubsection{Supuestos B\'asicos}
%__________________________________________________________________________
\begin{itemize}
\item[A1)] Los procesos
$\xi_{1},\ldots,\xi_{K},\eta_{1},\ldots,\eta_{K}$ son mutuamente
independientes y son sucesiones independientes e id\'enticamente
distribuidas.

\item[A2)] Para alg\'un entero $p\geq1$
\begin{eqnarray*}
\esp\left[\xi_{l}\left(1\right)^{p+1}\right]&<&\infty\textrm{ para }l=1,\ldots,K\textrm{ y }\\
\esp\left[\eta_{k}\left(1\right)^{p+1}\right]&<&\infty\textrm{
para }k=1,\ldots,K.
\end{eqnarray*}
donde $\mathcal{A}$ es la clase de posibles arribos.

\item[A3)] Para cada $k=1,2,\ldots,K$ existe una funci\'on
positiva $q_{k}\left(\cdot\right)$ definida en $\rea_{+}$, y un
entero $j_{k}$, tal que
\begin{eqnarray}
P\left(\xi_{k}\left(1\right)\geq x\right)&>&0\textrm{, para todo }x>0,\\
P\left\{a\leq\sum_{i=1}^{j_{k}}\xi_{k}\left(i\right)\leq
b\right\}&\geq&\int_{a}^{b}q_{k}\left(x\right)dx, \textrm{ }0\leq
a<b.
\end{eqnarray}
\end{itemize}

En lo que respecta al supuesto (A3), en Dai y Meyn \cite{DaiSean}
hacen ver que este se puede sustituir por

\begin{itemize}
\item[A3')] Para el Proceso de Markov $X$, cada subconjunto
compacto del espacio de estados de $X$ es un conjunto peque\~no,
ver definici\'on \ref{Def.Cto.Peq.}.
\end{itemize}

Es por esta raz\'on que con la finalidad de poder hacer uso de
$A3^{'})$ es necesario recurrir a los Procesos de Harris y en
particular a los Procesos Harris Recurrente, ver \cite{Dai,
DaiSean}.
%_______________________________________________________________________
\subsection{Procesos Harris Recurrente}
%_______________________________________________________________________

Por el supuesto (A1) conforme a Davis \cite{Davis}, se puede
definir el proceso de saltos correspondiente de manera tal que
satisfaga el supuesto (A3'), de hecho la demostraci\'on est\'a
basada en la l\'inea de argumentaci\'on de Davis, \cite{Davis},
p\'aginas 362-364.\\

Entonces se tiene un espacio de estados en el cual el proceso $X$
satisface la Propiedad Fuerte de Markov, ver Dai y Meyn
\cite{DaiSean}, dado por

\[\left(\Omega,\mathcal{F},\mathcal{F}_{t},X\left(t\right),\theta_{t},P_{x}\right),\]
adem\'as de ser un proceso de Borel Derecho (Sharpe \cite{Sharpe})
en el espacio de estados medible
$\left(\mathbb{X},\mathcal{B}_\mathbb{X}\right)$. El Proceso
$X=\left\{X\left(t\right),t\geq0\right\}$ tiene trayectorias
continuas por la derecha, est\'a definido en
$\left(\Omega,\mathcal{F}\right)$ y est\'a adaptado a
$\left\{\mathcal{F}_{t},t\geq0\right\}$; la colecci\'on
$\left\{P_{x},x\in \mathbb{X}\right\}$ son medidas de probabilidad
en $\left(\Omega,\mathcal{F}\right)$ tales que para todo $x\in
\mathbb{X}$
\[P_{x}\left\{X\left(0\right)=x\right\}=1,\] y
\[E_{x}\left\{f\left(X\circ\theta_{t}\right)|\mathcal{F}_{t}\right\}=E_{X}\left(\tau\right)f\left(X\right),\]
en $\left\{\tau<\infty\right\}$, $P_{x}$-c.s., con $\theta_{t}$
definido como el operador shift.


Donde $\tau$ es un $\mathcal{F}_{t}$-tiempo de paro
\[\left(X\circ\theta_{\tau}\right)\left(w\right)=\left\{X\left(\tau\left(w\right)+t,w\right),t\geq0\right\},\]
y $f$ es una funci\'on de valores reales acotada y medible, ver \cite{Dai, KaspiMandelbaum}.\\

Sea $P^{t}\left(x,D\right)$, $D\in\mathcal{B}_{\mathbb{X}}$,
$t\geq0$ la probabilidad de transici\'on de $X$ queda definida
como:
\[P^{t}\left(x,D\right)=P_{x}\left(X\left(t\right)\in
D\right).\]


\begin{Def}
Una medida no cero $\pi$ en
$\left(\mathbb{X},\mathcal{B}_{\mathbb{X}}\right)$ es invariante
para $X$ si $\pi$ es $\sigma$-finita y
\[\pi\left(D\right)=\int_{\mathbb{X}}P^{t}\left(x,D\right)\pi\left(dx\right),\]
para todo $D\in \mathcal{B}_{\mathbb{X}}$, con $t\geq0$.
\end{Def}

\begin{Def}
El proceso de Markov $X$ es llamado Harris Recurrente si existe
una medida de probabilidad $\nu$ en
$\left(\mathbb{X},\mathcal{B}_{\mathbb{X}}\right)$, tal que si
$\nu\left(D\right)>0$ y $D\in\mathcal{B}_{\mathbb{X}}$
\[P_{x}\left\{\tau_{D}<\infty\right\}\equiv1,\] cuando
$\tau_{D}=inf\left\{t\geq0:X_{t}\in D\right\}$.
\end{Def}

\begin{Note}
\begin{itemize}
\item[i)] Si $X$ es Harris recurrente, entonces existe una \'unica
medida invariante $\pi$ (Getoor \cite{Getoor}).

\item[ii)] Si la medida invariante es finita, entonces puede
normalizarse a una medida de probabilidad, en este caso al proceso
$X$ se le llama Harris recurrente positivo.


\item[iii)] Cuando $X$ es Harris recurrente positivo se dice que
la disciplina de servicio es estable. En este caso $\pi$ denota la
distribuci\'on estacionaria y hacemos
\[P_{\pi}\left(\cdot\right)=\int_{\mathbf{X}}P_{x}\left(\cdot\right)\pi\left(dx\right),\]
y se utiliza $E_{\pi}$ para denotar el operador esperanza
correspondiente, ver \cite{DaiSean}.
\end{itemize}
\end{Note}

\begin{Def}\label{Def.Cto.Peq.}
Un conjunto $D\in\mathcal{B_{\mathbb{X}}}$ es llamado peque\~no si
existe un $t>0$, una medida de probabilidad $\nu$ en
$\mathcal{B_{\mathbb{X}}}$, y un $\delta>0$ tal que
\[P^{t}\left(x,A\right)\geq\delta\nu\left(A\right),\] para $x\in
D,A\in\mathcal{B_{\mathbb{X}}}$.
\end{Def}

La siguiente serie de resultados vienen enunciados y demostrados
en Dai \cite{Dai}:
\begin{Lema}[Lema 3.1, Dai \cite{Dai}]
Sea $B$ conjunto peque\~no cerrado, supongamos que
$P_{x}\left(\tau_{B}<\infty\right)\equiv1$ y que para alg\'un
$\delta>0$ se cumple que
\begin{equation}\label{Eq.3.1}
\sup\esp_{x}\left[\tau_{B}\left(\delta\right)\right]<\infty,
\end{equation}
donde
$\tau_{B}\left(\delta\right)=inf\left\{t\geq\delta:X\left(t\right)\in
B\right\}$. Entonces, $X$ es un proceso Harris recurrente
positivo.
\end{Lema}

\begin{Lema}[Lema 3.1, Dai \cite{Dai}]\label{Lema.3.}
Bajo el supuesto (A3), el conjunto
$B=\left\{x\in\mathbb{X}:|x|\leq k\right\}$ es un conjunto
peque\~no cerrado para cualquier $k>0$.
\end{Lema}

\begin{Teo}[Teorema 3.1, Dai \cite{Dai}]\label{Tma.3.1}
Si existe un $\delta>0$ tal que
\begin{equation}
lim_{|x|\rightarrow\infty}\frac{1}{|x|}\esp|X^{x}\left(|x|\delta\right)|=0,
\end{equation}
donde $X^{x}$ se utiliza para denotar que el proceso $X$ comienza
a partir de $x$, entonces la ecuaci\'on (\ref{Eq.3.1}) se cumple
para $B=\left\{x\in\mathbb{X}:|x|\leq k\right\}$ con alg\'un
$k>0$. En particular, $X$ es Harris recurrente positivo.
\end{Teo}

Entonces, tenemos que el proceso $X$ es un proceso de Markov que
cumple con los supuestos $A1)$-$A3)$, lo que falta de hacer es
construir el Modelo de Flujo bas\'andonos en lo hasta ahora
presentado.
%_______________________________________________________________________
\subsection{Modelo de Flujo}
%_______________________________________________________________________

Dada una condici\'on inicial $x\in\mathbb{X}$, sea

\begin{itemize}
\item $Q_{k}^{x}\left(t\right)$ la longitud de la cola al tiempo
$t$,

\item $T_{m,k}^{x}\left(t\right)$ el tiempo acumulado, al tiempo
$t$, que tarda el servidor $m$ en atender a los usuarios de la
cola $k$.

\item $T_{m,k}^{x,0}\left(t\right)$ el tiempo acumulado, al tiempo
$t$, que tarda el servidor $m$ en trasladarse a otra cola a partir de la $k$-\'esima.\\
\end{itemize}

Sup\'ongase que la funci\'on
$\left(\overline{Q}\left(\cdot\right),\overline{T}_{m}
\left(\cdot\right),\overline{T}_{m}^{0} \left(\cdot\right)\right)$
para $m=1,2,\ldots,M$ es un punto l\'imite de
\begin{equation}\label{Eq.Punto.Limite}
\left(\frac{1}{|x|}Q^{x}\left(|x|t\right),\frac{1}{|x|}T_{m}^{x}\left(|x|t\right),\frac{1}{|x|}T_{m}^{x,0}\left(|x|t\right)\right)
\end{equation}
para $m=1,2,\ldots,M$, cuando $x\rightarrow\infty$, ver
\cite{Down}. Entonces
$\left(\overline{Q}\left(t\right),\overline{T}_{m}
\left(t\right),\overline{T}_{m}^{0} \left(t\right)\right)$ es un
flujo l\'imite del sistema. Al conjunto de todos las posibles
flujos l\'imite se le llama {\emph{Modelo de Flujo}} y se le
denotar\'a por $\mathcal{Q}$, ver \cite{Down, Dai, DaiSean}.\\

El modelo de flujo satisface el siguiente conjunto de ecuaciones:

\begin{equation}\label{Eq.MF.1}
\overline{Q}_{k}\left(t\right)=\overline{Q}_{k}\left(0\right)+\lambda_{k}t-\sum_{m=1}^{M}\mu_{k}\overline{T}_{m,k}\left(t\right),\\
\end{equation}
para $k=1,2,\ldots,K$.\\
\begin{equation}\label{Eq.MF.2}
\overline{Q}_{k}\left(t\right)\geq0\textrm{ para
}k=1,2,\ldots,K.\\
\end{equation}

\begin{equation}\label{Eq.MF.3}
\overline{T}_{m,k}\left(0\right)=0,\textrm{ y }\overline{T}_{m,k}\left(\cdot\right)\textrm{ es no decreciente},\\
\end{equation}
para $k=1,2,\ldots,K$ y $m=1,2,\ldots,M$.\\
\begin{equation}\label{Eq.MF.4}
\sum_{k=1}^{K}\overline{T}_{m,k}^{0}\left(t\right)+\overline{T}_{m,k}\left(t\right)=t\textrm{
para }m=1,2,\ldots,M.\\
\end{equation}


\begin{Def}[Definici\'on 4.1, Dai \cite{Dai}]\label{Def.Modelo.Flujo}
Sea una disciplina de servicio espec\'ifica. Cualquier l\'imite
$\left(\overline{Q}\left(\cdot\right),\overline{T}\left(\cdot\right),\overline{T}^{0}\left(\cdot\right)\right)$
en (\ref{Eq.Punto.Limite}) es un {\em flujo l\'imite} de la
disciplina. Cualquier soluci\'on (\ref{Eq.MF.1})-(\ref{Eq.MF.4})
es llamado flujo soluci\'on de la disciplina.
\end{Def}

\begin{Def}
Se dice que el modelo de flujo l\'imite, modelo de flujo, de la
disciplina de la cola es estable si existe una constante
$\delta>0$ que depende de $\mu,\lambda$ y $P$ solamente, tal que
cualquier flujo l\'imite con
$|\overline{Q}\left(0\right)|+|\overline{U}|+|\overline{V}|=1$, se
tiene que $\overline{Q}\left(\cdot+\delta\right)\equiv0$.
\end{Def}

Si se hace $|x|\rightarrow\infty$ sin restringir ninguna de las
componentes, tambi\'en se obtienen un modelo de flujo, pero en
este caso el residual de los procesos de arribo y servicio
introducen un retraso:
\begin{Teo}[Teorema 4.2, Dai \cite{Dai}]\label{Tma.4.2.Dai}
Sea una disciplina fija para la cola, suponga que se cumplen las
condiciones (A1)-(A3). Si el modelo de flujo l\'imite de la
disciplina de la cola es estable, entonces la cadena de Markov $X$
que describe la din\'amica de la red bajo la disciplina es Harris
recurrente positiva.
\end{Teo}

Ahora se procede a escalar el espacio y el tiempo para reducir la
aparente fluctuaci\'on del modelo. Consid\'erese el proceso
\begin{equation}\label{Eq.3.7}
\overline{Q}^{x}\left(t\right)=\frac{1}{|x|}Q^{x}\left(|x|t\right).
\end{equation}
A este proceso se le conoce como el flujo escalado, y cualquier
l\'imite $\overline{Q}^{x}\left(t\right)$ es llamado flujo
l\'imite del proceso de longitud de la cola. Haciendo
$|q|\rightarrow\infty$ mientras se mantiene el resto de las
componentes fijas, cualquier punto l\'imite del proceso de
longitud de la cola normalizado $\overline{Q}^{x}$ es soluci\'on
del siguiente modelo de flujo.


\begin{Def}[Definici\'on 3.3, Dai y Meyn \cite{DaiSean}]
El modelo de flujo es estable si existe un tiempo fijo $t_{0}$ tal
que $\overline{Q}\left(t\right)=0$, con $t\geq t_{0}$, para
cualquier $\overline{Q}\left(\cdot\right)\in\mathcal{Q}$ que
cumple con $|\overline{Q}\left(0\right)|=1$.
\end{Def}

\begin{Lemma}[Lema 3.1, Dai y Meyn \cite{DaiSean}]
Si el modelo de flujo definido por (\ref{Eq.MF.1})-(\ref{Eq.MF.4})
es estable, entonces el modelo de flujo retrasado es tambi\'en
estable, es decir, existe $t_{0}>0$ tal que
$\overline{Q}\left(t\right)=0$ para cualquier $t\geq t_{0}$, para
cualquier soluci\'on del modelo de flujo retrasado cuya
condici\'on inicial $\overline{x}$ satisface que
$|\overline{x}|=|\overline{Q}\left(0\right)|+|\overline{A}\left(0\right)|+|\overline{B}\left(0\right)|\leq1$.
\end{Lemma}


Ahora ya estamos en condiciones de enunciar los resultados principales:


\begin{Teo}[Teorema 2.1, Down \cite{Down}]\label{Tma2.1.Down}
Suponga que el modelo de flujo es estable, y que se cumplen los supuestos (A1) y (A2), entonces
\begin{itemize}
\item[i)] Para alguna constante $\kappa_{p}$, y para cada
condici\'on inicial $x\in X$
\begin{equation}\label{Estability.Eq1}
\limsup_{t\rightarrow\infty}\frac{1}{t}\int_{0}^{t}\esp_{x}\left[|Q\left(s\right)|^{p}\right]ds\leq\kappa_{p},
\end{equation}
donde $p$ es el entero dado en (A2).
\end{itemize}
Si adem\'as se cumple la condici\'on (A3), entonces para cada
condici\'on inicial:
\begin{itemize}
\item[ii)] Los momentos transitorios convergen a su estado
estacionario:
 \begin{equation}\label{Estability.Eq2}
lim_{t\rightarrow\infty}\esp_{x}\left[Q_{k}\left(t\right)^{r}\right]=\esp_{\pi}\left[Q_{k}\left(0\right)^{r}\right]\leq\kappa_{r},
\end{equation}
para $r=1,2,\ldots,p$ y $k=1,2,\ldots,K$. Donde $\pi$ es la
probabilidad invariante para $X$.

\item[iii)]  El primer momento converge con raz\'on $t^{p-1}$:
\begin{equation}\label{Estability.Eq3}
lim_{t\rightarrow\infty}t^{p-1}|\esp_{x}\left[Q_{k}\left(t\right)\right]-\esp_{\pi}\left[Q_{k}\left(0\right)\right]|=0.
\end{equation}

\item[iv)] La {\em Ley Fuerte de los grandes n\'umeros} se cumple:
\begin{equation}\label{Estability.Eq4}
lim_{t\rightarrow\infty}\frac{1}{t}\int_{0}^{t}Q_{k}^{r}\left(s\right)ds=\esp_{\pi}\left[Q_{k}\left(0\right)^{r}\right],\textrm{
}\prob_{x}\textrm{-c.s.}
\end{equation}
para $r=1,2,\ldots,p$ y $k=1,2,\ldots,K$.
\end{itemize}
\end{Teo}

La contribuci\'on de Down a la teor\'ia de los {\emph {sistemas de
visitas c\'iclicas}}, es la relaci\'on que hay entre la
estabilidad del sistema con el comportamiento de las medidas de
desempe\~no, es decir, la condici\'on suficiente para poder
garantizar la convergencia del proceso de la longitud de la cola
as\'i como de por los menos los dos primeros momentos adem\'as de
una versi\'on de la Ley Fuerte de los Grandes N\'umeros para los
sistemas de visitas.


\begin{Teo}[Teorema 2.3, Down \cite{Down}]\label{Tma2.3.Down}
Considere el siguiente valor:
\begin{equation}\label{Eq.Rho.1serv}
\rho=\sum_{k=1}^{K}\rho_{k}+max_{1\leq j\leq K}\left(\frac{\lambda_{j}}{\sum_{s=1}^{S}p_{js}\overline{N}_{s}}\right)\delta^{*}
\end{equation}
\begin{itemize}
\item[i)] Si $\rho<1$ entonces la red es estable, es decir, se
cumple el Teorema \ref{Tma2.1.Down}.

\item[ii)] Si $\rho>1$ entonces la red es inestable, es decir, se
cumple el Teorema \ref{Tma2.2.Down}
\end{itemize}
\end{Teo}




\part{APÉNDICES}
\chapter{IMPLEMENTACIONES NUM\'ERICAS}
\section{D\'ia 1: Regresión Logística}

\subsection*{Implementación Básica en R}

Para implementar una regresión logística en R, primero es necesario instalar y cargar los paquetes necesarios.

\subsection*{Instalación y Configuración de R y RStudio}
\begin{itemize}
    \item Descargue e instale R desde \texttt{https://cran.r-project.org/}. Siga las instrucciones para su sistema operativo (Windows, MacOS, Linux).
    \item Descargue e instale RStudio desde \texttt{https://rstudio.com/products/rstudio/download/}. 
\end{itemize}

\subsection{Ejemplo de Regresión Logística en R}

A continuación, se muestra un ejemplo de cómo ajustar un modelo de regresión logística en R utilizando un conjunto de datos simulado. El ejemplo incluye la instalación del paquete necesario, la carga de datos, el ajuste del modelo, y la interpretación de los resultados.

\begin{verbatim}
# Instalación del paquete necesario
install.packages("stats")

# Carga del paquete
library(stats)

# Ejemplo de conjunto de datos
data <- data.frame(
  outcome = c(1, 0, 1, 0, 1, 1, 0, 1, 0, 0),
  predictor = c(2.3, 1.9, 3.1, 2.8, 3.6, 2.4, 2.1, 3.3, 2.2, 1.7)
)

# Ajuste del modelo de regresión logística
model <- glm(outcome ~ predictor, data = data, family = binomial)

# Resumen del modelo
summary(model)
\end{verbatim}

En este ejemplo, se utiliza el conjunto de datos \textit{data} que contiene una variable de resultado binaria \textit{outcome} y una variable predictora continua \textit{predictor}. El modelo de regresión logística se ajusta utilizando la función \texttt{glm} con la familia binomial. La función \texttt{summary(model)} proporciona un resumen del modelo ajustado, incluyendo los coeficientes estimados, sus errores estándar, valores z, y p-valores.

\begin{itemize}
    \item \textbf{Coeficientes}: Los coeficientes estimados $\beta_0$ y $\beta_1$ indican la dirección y magnitud de la relación entre las variables predictoras y la probabilidad del resultado.
    \item \textbf{Errores Estándar}: Los errores estándar proporcionan una medida de la precisión de los coeficientes estimados.
    \item \textbf{Valores z y p-valores}: Los valores z y p-valores se utilizan para evaluar la significancia estadística de los coeficientes. Un p-valor pequeño (generalmente < 0.05) indica que el coeficiente es significativamente diferente de cero.
\end{itemize}

Este es solo un ejemplo básico, en aplicaciones reales, es posible que necesites realizar más análisis y validaciones, como la evaluación de la bondad de ajuste del modelo, el diagnóstico de posibles problemas de multicolinealidad, y la validación cruzada del modelo.

\begin{verbatim}
# Archivo: regresionlogistica.R

# Instalación del paquete necesario
#install.packages("stats")

# Carga del paquete
library(stats)

# Fijar la semilla para reproducibilidad
set.seed(123)

# Número de observaciones
n <- 100

# Generar las variables independientes X1, X2, ..., X15
# Creamos una matriz de tamaño n x 15 con valores generados aleatoriamente de una
 distribución normal
X <- as.data.frame(matrix(rnorm(n * 15), nrow = n, ncol = 15))
colnames(X) <- paste0("X", 1:15)  # Nombramos las columnas como X1, X2, ..., X15

# Coeficientes verdaderos para las variables independientes
# Generamos un vector de 16 coeficientes (incluyendo el intercepto) aleatorios entre -1 y 1
beta <- runif(16, -1, 1)  # 15 coeficientes más el intercepto

# Generar el término lineal
# Calculamos el término lineal utilizando los coeficientes y las variables independientes
linear_term <- beta[1] + as.matrix(X) %*% beta[-1]

# Generar la probabilidad utilizando la función logística
# Calculamos las probabilidades utilizando la función logística
p <- 1 / (1 + exp(-linear_term))

# Generar la variable dependiente binaria Y
# Generamos valores binarios (0 o 1) utilizando las probabilidades calculadas
Y <- rbinom(n, 1, p)

# Combinar las variables independientes y la variable dependiente en un data frame
data <- cbind(Y, X)

# Dividir el conjunto de datos en entrenamiento y prueba
set.seed(123)  # Fijar la semilla para reproducibilidad
train_indices <- sample(1:n, size = 0.7 * n)  # 70% de los datos para entrenamiento
train_set <- data[train_indices, ]  # Conjunto de entrenamiento
test_set <- data[-train_indices, ]  # Conjunto de prueba

# Ajuste del modelo de regresión logística en el conjunto de entrenamiento
# Ajustamos un modelo de regresión logística utilizando las variables independientes
para predecir Y
model <- glm(Y ~ ., data = train_set, family = binomial)

# Resumen del modelo
# Mostramos un resumen del modelo ajustado
summary(model)

# Guardar el modelo y los resultados en un archivo
# Guardamos el modelo ajustado en un archivo .RData
save(model, file = "regresion_logistica_modelo.RData")

# Guardar los datos simulados en archivos CSV
# Guardamos los conjuntos de datos de entrenamiento y prueba en archivos CSV
write.csv(train_set, "datos_entrenamiento_regresion_logistica.csv", row.names = FALSE)
write.csv(test_set, "datos_prueba_regresion_logistica.csv", row.names = FALSE)

# Hacer predicciones en el conjunto de prueba
# Utilizamos el modelo ajustado para hacer predicciones en el conjunto de prueba
test_set$prob_pred <- predict(model, newdata = test_set, type = "response")
test_set$Y_pred <- ifelse(test_set$prob_pred > 0.5, 1, 0)  
# Convertimos probabilidades a clases binarias

# Calcular la precisión de las predicciones
# Calculamos la precisión de las predicciones comparando con los valores reales de Y
accuracy <- mean(test_set$Y_pred == test_set$Y)
cat("La precisión del modelo en el conjunto de prueba es:", accuracy, "\n")

# Guardar las predicciones en un archivo CSV
# Guardamos las predicciones y las probabilidades predichas en un archivo CSV
write.csv(test_set, "predicciones_regresion_logistica.csv", row.names = FALSE)

# Graficar los coeficientes estimados
# Graficamos los coeficientes estimados del modelo ajustado
plot(coef(model), main = "Coeficientes Estimados del Modelo de Regresión Logística", 
     xlab = "Variables", ylab = "Coeficientes", type = "h", col = "blue")
abline(h = 0, col = "red", lwd = 2)

# Mostrar un mensaje indicando que el proceso ha finalizado
cat("El modelo de regresión logística se ha ajustado, se han hecho predicciones y los resultados se han guardado en 'regresion_logistica_modelo.RData'.\n")
\end{verbatim}

\subsection{Aplicación a Datos de Cáncer - Parte I}

A continuación, se muestra un ejemplo de cómo ajustar un modelo de regresión logística en R utilizando el conjunto de datos del cáncer de mama de Wisconsin.

\begin{verbatim}
# Archivo: regresionlogistica_cancer.R

# Instalación del paquete necesario
install.packages("mlbench")
install.packages("dplyr")

# Carga de los paquetes
library(mlbench)
library(dplyr)

# Cargar el conjunto de datos BreastCancer
data("BreastCancer")

# Ver las primeras filas del conjunto de datos
head(BreastCancer)

# Preprocesamiento de los datos
# Eliminar la columna de identificación y filas con valores faltantes
breast_cancer_clean <- BreastCancer %>%
  select(-Id) %>%
  na.omit()

# Convertir la variable 'Class' a factor binario
breast_cancer_clean$Class <- ifelse(breast_cancer_clean$Class == "malignant", 1, 0)
breast_cancer_clean$Class <- as.factor(breast_cancer_clean$Class)

# Convertir las demás columnas a numéricas
breast_cancer_clean[, 1:9] <- lapply(breast_cancer_clean[, 1:9], as.numeric)

# Dividir el conjunto de datos en entrenamiento (70%) y prueba (30%)
set.seed(123)
train_indices <- sample(1:nrow(breast_cancer_clean), size = 0.7 * nrow(breast_cancer_clean))
train_set <- breast_cancer_clean[train_indices, ]
test_set <- breast_cancer_clean[-train_indices, ]

# Ajuste del modelo de regresión logística en el conjunto de entrenamiento
model <- glm(Class ~ ., data = train_set, family = binomial)

# Resumen del modelo
summary(model)

# Guardar el modelo y los resultados en un archivo
save(model, file = "regresion_logistica_cancer_modelo.RData")

# Guardar los datos simulados en archivos CSV
write.csv(train_set, "datos_entrenamiento_cancer.csv", row.names = FALSE)
write.csv(test_set, "datos_prueba_cancer.csv", row.names = FALSE)

# Hacer predicciones en el conjunto de prueba
test_set$prob_pred <- predict(model, newdata = test_set, type = "response")
test_set$Class_pred <- ifelse(test_set$prob_pred > 0.5, 1, 0)

# Calcular la precisión de las predicciones
accuracy <- mean(test_set$Class_pred == test_set$Class)
cat("La precisión del modelo en el conjunto de prueba es:", accuracy, "\n")

# Guardar las predicciones en un archivo CSV
write.csv(test_set, "predicciones_cancer.csv", row.names = FALSE)

# Graficar los coeficientes estimados
plot(coef(model), main = "Coeficientes Estimados del Modelo de Regresión Logística", 
     xlab = "Variables", ylab = "Coeficientes", type = "h", col = "blue")
abline(h = 0, col = "red", lwd = 2)

# Mostrar un mensaje indicando que el proceso ha finalizado
cat("El modelo de regresión logística se ha ajustado, se han hecho predicciones y los resultados se han guardado en 'regresion_logistica_cancer_modelo.RData'.\n")
\end{verbatim}

\subsubsection*{Descripción del Código}

\textbf{Instalación y Carga de Paquetes:}

Instalamos y cargamos el paquete \texttt{stats} necesario para la regresión logística.

\textbf{Generación de Datos Simulados:}

\begin{itemize}
    \item Fijamos una semilla para la reproducibilidad.
    \item Generamos un conjunto de datos con 100 observaciones y 15 variables independientes (\texttt{X1, X2, ..., X15}) usando una distribución normal.
    \item Definimos los coeficientes verdaderos para las variables independientes y calculamos el término lineal.
    \item Calculamos las probabilidades usando la función logística y generamos una variable dependiente binaria \texttt{Y} basada en esas probabilidades.
    \item Combinamos las variables independientes y la variable dependiente en un \texttt{data frame}.
\end{itemize}

\textbf{División de Datos en Conjuntos de Entrenamiento y Prueba:}

\begin{itemize}
    \item Dividimos los datos en un conjunto de entrenamiento (70\%) y un conjunto de prueba (30\%).
\end{itemize}

\textbf{Ajuste del Modelo de Regresión Logística:}

\begin{itemize}
    \item Ajustamos un modelo de regresión logística en el conjunto de entrenamiento.
    \item Mostramos un resumen del modelo ajustado.
\end{itemize}

\textbf{Guardado de Datos y Modelo:}

\begin{itemize}
    \item Guardamos el modelo ajustado en un archivo \texttt{.RData}.
    \item Guardamos los conjuntos de datos de entrenamiento y prueba en archivos CSV.
\end{itemize}

\textbf{Predicciones y Evaluación del Modelo:}

\begin{itemize}
    \item Hacemos predicciones en el conjunto de prueba utilizando el modelo ajustado.
    \item Calculamos la precisión de las predicciones comparando con los valores reales de \texttt{Y}.
    \item Guardamos las predicciones y las probabilidades predichas en un archivo CSV.
\end{itemize}

\textbf{Visualización de los Coeficientes del Modelo:}

\begin{itemize}
    \item Graficamos los coeficientes estimados del modelo ajustado.
    \item Mostramos un mensaje indicando que el proceso ha finalizado.
\end{itemize}

Para ejecutar este script, guarda el código en un archivo llamado \textit{regresionlogistica.R}, abre R o RStudio, navega hasta el directorio donde guardaste el archivo y ejecuta el script usando \textit{source("regresionlogistica.R")}.

\subsubsection{Ejemplo Titanic}

Cuando realizas una regresión logística, obtienes coeficientes para cada variable independiente en tu modelo. Estos coeficientes indican la dirección y la magnitud de la relación entre cada variable independiente y la variable dependiente (en este caso, \textit{Survived}).

\subsubsection*{Interpretación de los Coeficientes}

\begin{itemize}
    \item \textbf{Intercepto} (\textit{(Intercept)}): Este coeficiente representa el logaritmo de las probabilidades (log-odds) de que \textit{Survived} sea 1 (supervivencia) cuando todas las variables independientes son cero.
    \item \textbf{Pclass}: El coeficiente asociado con \textit{Pclass} indica cómo cambia el log-odds de supervivencia con cada incremento en la clase del pasajero. Si el coeficiente es negativo, sugiere que una clase más alta (por ejemplo, de primera clase a tercera clase) reduce las probabilidades de supervivencia.
    \item \textbf{Sex}: Este coeficiente muestra el efecto de ser hombre o mujer en las probabilidades de supervivencia. Generalmente, se espera que el coeficiente sea positivo para \textit{female} indicando que las mujeres tenían mayores probabilidades de sobrevivir.
    \item \textbf{Age}: El coeficiente de \textit{Age} indica cómo cambia el log-odds de supervivencia con cada año de incremento en la edad. Un coeficiente negativo sugiere que la probabilidad de supervivencia disminuye con la edad.
    \item \textbf{SibSp} y \textbf{Parch}: Estos coeficientes indican el efecto del número de hermanos/cónyuges a bordo y padres/hijos a bordo en las probabilidades de supervivencia.
    \item \textbf{Fare}: Este coeficiente indica el efecto del precio del billete en las probabilidades de supervivencia. Un coeficiente positivo sugiere que pagar más por el billete se asocia con mayores probabilidades de supervivencia.
\end{itemize}

\subsubsection*{Estadísticas de Ajuste del Modelo}

El resumen del modelo (\textit{summary(model)}) incluye varias estadísticas importantes:

\begin{itemize}
    \item \textbf{Estadísticos z y p-valores}: Estas estadísticas indican la significancia de cada coeficiente. Un p-valor bajo (generalmente < 0.05) sugiere que la variable es un predictor significativo de la variable dependiente.
    \item \textbf{Desviación Residual}: La desviación residual mide la calidad del ajuste del modelo. Valores más bajos indican un mejor ajuste.
    \item \textbf{AIC (Akaike Information Criterion)}: El AIC es una medida de la calidad del modelo que toma en cuenta tanto la bondad del ajuste como la complejidad del modelo. Modelos con AIC más bajo son preferidos.
\end{itemize}

\subsubsection*{Precisión del Modelo}

La precisión del modelo en el conjunto de prueba es una métrica importante para evaluar el rendimiento del modelo. La precisión se calcula como el número de predicciones correctas dividido por el número total de predicciones.

\subsubsection*{Ejemplo de Resultados}

Supongamos que la precisión del modelo es 0.78 (78\%). Esto significa que el modelo correctamente predijo el estado de supervivencia del 78\% de los pasajeros en el conjunto de prueba.

\subsubsection*{Matriz de Confusión y Otras Métricas}

Además de la precisión, otras métricas como la matriz de confusión, la sensibilidad, la especificidad, y el área bajo la curva ROC (AUC-ROC) también pueden proporcionar una visión más completa del rendimiento del modelo.

\subsubsection*{Matriz de Confusión}

\begin{itemize}
    \item \textbf{Verdaderos Positivos (TP)}: Número de pasajeros que sobrevivieron y fueron predichos como sobrevivientes.
    \item \textbf{Verdaderos Negativos (TN)}: Número de pasajeros que no sobrevivieron y fueron predichos como no sobrevivientes.
    \item \textbf{Falsos Positivos (FP)}: Número de pasajeros que no sobrevivieron pero fueron predichos como sobrevivientes.
    \item \textbf{Falsos Negativos (FN)}: Número de pasajeros que sobrevivieron pero fueron predichos como no sobrevivientes.
\end{itemize}

\subsubsection*{Ejemplo de Cálculo de Métricas}

\begin{verbatim}
# Calcular la matriz de confusión
table(test_set$Survived, test_set$Survived_pred)

# Calcular sensibilidad y especificidad
sensitivity <- sum(test_set$Survived == 1 & test_set$Survived_pred == 1) / sum(test_set$Survived == 1)
specificity <- sum(test_set$Survived == 0 & test_set$Survived_pred == 0) / sum(test_set$Survived == 0)

# Calcular AUC-ROC
library(pROC)
roc_curve <- roc(test_set$Survived, test_set$prob_pred)
auc(roc_curve)
\end{verbatim}

\subsubsection*{Visualización de Resultados}

Graficar los coeficientes del modelo, la curva ROC y otras visualizaciones ayudan a entender mejor el rendimiento y la importancia de cada variable en el modelo.

\begin{verbatim}
# Graficar la curva ROC
plot(roc_curve, main = "Curva ROC para el Modelo de Regresión Logística")
\end{verbatim}

\subsubsection*{Resumen Final}

El modelo de regresión logística aplicado al conjunto de datos del Titanic proporciona una forma de entender cómo diferentes características de los pasajeros influyen en sus probabilidades de supervivencia. La interpretación de los coeficientes del modelo, las estadísticas de ajuste, y la precisión del modelo en el conjunto de prueba son fundamentales para evaluar el rendimiento y la utilidad del modelo en hacer predicciones sobre la supervivencia de los pasajeros del Titanic.

\subsection{Simulaci\'on de Datos de Cáncer - Parte II}

Aquí se presenta un ejemplo de cómo realizar una regresión logística utilizando datos simulados de pacientes con cáncer.

\begin{verbatim}
#---- Archivo: cancerLogRegSimulado.R ----

# Instalación del paquete necesario
if (!requireNamespace("dplyr", quietly = TRUE)) {
  install.packages("dplyr")
}

# Carga del paquete
library(dplyr)

# Fijar la semilla para reproducibilidad
set.seed(123)

# Número de observaciones
n <- 150

# Generar las variables independientes X1, X2, ..., X15
# Creamos una matriz de tamaño n x 15 con valores generados aleatoriamente de una 
distribución normal
X <- as.data.frame(matrix(rnorm(n * 15), nrow = n, ncol = 15))
colnames(X) <- paste0("X", 1:15)  # Nombramos las columnas como X1, X2, ..., X15

# Coeficientes verdaderos para las variables independientes
# Generamos un vector de 16 coeficientes (incluyendo el intercepto) aleatorios entre -1 y 1
beta <- runif(16, -1, 1)  # 15 coeficientes más el intercepto

# Generar el término lineal
# Calculamos el término lineal utilizando los coeficientes y las variables independientes
linear_term <- beta[1] + as.matrix(X) %*% beta[-1]

# Generar la probabilidad utilizando la función logística
# Calculamos las probabilidades utilizando la función logística
p <- 1 / (1 + exp(-linear_term))

# Generar la variable dependiente binaria Y
# Generamos valores binarios (0 o 1) utilizando las probabilidades calculadas
Y <- rbinom(n, 1, p)

# Combinar las variables independientes y la variable dependiente en un data frame
data <- cbind(Y, X)

# Dividir el conjunto de datos en entrenamiento y prueba
set.seed(123)  # Fijar la semilla para reproducibilidad
train_indices <- sample(1:n, size = 0.7 * n)  # 70% de los datos para entrenamiento
train_set <- data[train_indices, ]  # Conjunto de entrenamiento
test_set <- data[-train_indices, ]  # Conjunto de prueba

# Ajuste del modelo de regresión logística en el conjunto de entrenamiento
# Ajustamos un modelo de regresión logística utilizando las variables independientes 
para predecir Y
model <- glm(Y ~ ., data = train_set, family = binomial)

# Resumen del modelo
# Mostramos un resumen del modelo ajustado
summary(model)

# Guardar el modelo y los resultados en un archivo
# Guardamos el modelo ajustado en un archivo .RData
save(model, file = "regresion_logistica_cancer_modelo_simulado.RData")

# Guardar los datos simulados en archivos CSV
# Guardamos los conjuntos de datos de entrenamiento y prueba en archivos CSV
write.csv(train_set, "datos_entrenamiento_cancer_simulado.csv", row.names = FALSE)
write.csv(test_set, "datos_prueba_cancer_simulado.csv", row.names = FALSE)

# Hacer predicciones en el conjunto de prueba
# Utilizamos el modelo ajustado para hacer predicciones en el conjunto de prueba
test_set$prob_pred <- predict(model, newdata = test_set, type = "response")
test_set$Y_pred <- ifelse(test_set$prob_pred > 0.5, 1, 0)  
# Convertimos probabilidades a clases binarias

# Calcular la precisión de las predicciones
# Calculamos la precisión de las predicciones comparando con los valores reales de Y
accuracy <- mean(test_set$Y_pred == test_set$Y)
cat("La precisión del modelo en el conjunto de prueba es:", accuracy, "\n")

# Guardar las predicciones en un archivo CSV
# Guardamos las predicciones y las probabilidades predichas en un archivo CSV
write.csv(test_set, "predicciones_cancer_simulado.csv", row.names = FALSE)

# Graficar los coeficientes estimados
# Graficamos los coeficientes estimados del modelo ajustado
plot(coef(model), main = "Coeficientes Estimados del Modelo de Regresión Logística", 
     xlab = "Variables", ylab = "Coeficientes", type = "h", col = "blue")
abline(h = 0, col = "red", lwd = 2)

# Mostrar un mensaje indicando que el proceso ha finalizado
cat("El modelo de regresión logística se ha ajustado, se han hecho predicciones 
y los resultados se han guardado en 'regresion_logistica_cancer_modelo_simulado.RData'.\n")
\end{verbatim}

\subsection{Simulaci\'on de Datos de Cáncer - Parte III}

En un estudio sobre cáncer, especialmente en el contexto del cáncer de mama, las principales mediciones suelen incluir una variedad de características clínicas y patológicas. Aquí hay algunas de las principales mediciones que se tienen en cuenta:

\begin{itemize}
    \item \textbf{Tamaño del Tumor}: Medición del diámetro del tumor.
    \item \textbf{Estado de los Ganglios Linfáticos}: Número de ganglios linfáticos afectados.
    \item \textbf{Grado del Tumor}: Clasificación del tumor basada en la apariencia de las células cancerosas.
    \item \textbf{Receptores Hormonales}: Estado de los receptores de estrógeno y progesterona.
    \item \textbf{Estado HER2}: Expresión del receptor 2 del factor de crecimiento epidérmico humano.
    \item \textbf{Ki-67}: Índice de proliferación celular.
    \item \textbf{Edad del Paciente}: Edad en el momento del diagnóstico.
    \item \textbf{Histopatología}: Tipo y subtipo histológico del cáncer.
    \item \textbf{Márgenes Quirúrgicos}: Estado de los márgenes después de la cirugía (si están libres de cáncer o no).
    \item \textbf{Invasión Linfovascular}: Presencia de células cancerosas en los vasos linfáticos o sanguíneos.
    \item \textbf{Tratamientos Previos}: Tipos de tratamientos recibidos antes del diagnóstico (quimioterapia, radioterapia, etc.).
    \item \textbf{Tipo de Cirugía}: Tipo de procedimiento quirúrgico realizado (mastectomía, lumpectomía, etc.).
    \item \textbf{Metástasis}: Presencia de metástasis y ubicación de las mismas.
    \item \textbf{Índice de Masa Corporal (IMC)}: Relación entre el peso y la altura del paciente.
    \item \textbf{Marcadores Genéticos}: Presencia de mutaciones genéticas específicas (BRCA1, BRCA2, etc.).
\end{itemize}

Estas mediciones proporcionan una visión integral del estado del cáncer y se utilizan para planificar el tratamiento y predecir el pronóstico.

A continuación, se muestra un ejemplo de cómo ajustar un modelo de regresión logística en R utilizando un conjunto de datos simulado con estas mediciones.

\begin{verbatim}
# Archivo: simulcorrectedCancer.R

# Instalación del paquete necesario
if (!requireNamespace("dplyr", quietly = TRUE)) {
  install.packages("dplyr")
}

# Carga del paquete
library(dplyr)

# Fijar la semilla para reproducibilidad
set.seed(123)

# Número de observaciones
n <- 1500

# Simulación de las variables independientes
# Tamaño del Tumor (en cm)
Tumor_Size <- rnorm(n, mean = 3, sd = 1.5)

# Estado de los Ganglios Linfáticos (número de ganglios afectados)
Lymph_Nodes <- rpois(n, lambda = 3)

# Grado del Tumor (1 a 3)
Tumor_Grade <- sample(1:3, n, replace = TRUE)

# Receptores Hormonales (0: negativo, 1: positivo)
Estrogen_Receptor <- rbinom(n, 1, 0.7)
Progesterone_Receptor <- rbinom(n, 1, 0.7)

# Estado HER2 (0: negativo, 1: positivo)
HER2_Status <- rbinom(n, 1, 0.3)

# Ki-67 (% de células proliferativas)
Ki_67 <- rnorm(n, mean = 20, sd = 10)

# Edad del Paciente (años)
Age <- rnorm(n, mean = 50, sd = 10)

# Histopatología (1: ductal, 2: lobular, 3: otros)
Histopathology <- sample(1:3, n, replace = TRUE)

# Márgenes Quirúrgicos (0: positivo, 1: negativo)
Surgical_Margins <- rbinom(n, 1, 0.8)

# Invasión Linfovascular (0: no, 1: sí)
Lymphovascular_Invasion <- rbinom(n, 1, 0.4)

# Tratamientos Previos (0: no, 1: sí)
Prior_Treatments <- rbinom(n, 1, 0.5)

# Tipo de Cirugía (0: mastectomía, 1: lumpectomía)
Surgery_Type <- rbinom(n, 1, 0.5)

# Metástasis (0: no, 1: sí)
Metastasis <- rbinom(n, 1, 0.2)

# Índice de Masa Corporal (IMC)
BMI <- rnorm(n, mean = 25, sd = 5)

# Marcadores Genéticos (0: negativo, 1: positivo)
Genetic_Markers <- rbinom(n, 1, 0.1)

# Generar la variable dependiente binaria Y (sobrevivencia 0: no, 1: sí)
# Utilizaremos una combinación arbitraria de las variables para generar Y
linear_term <- -1 + 0.5 * Tumor_Size - 0.3 * Lymph_Nodes + 0.2 * Tumor_Grade + 
  0.4 * Estrogen_Receptor + 0.3 * Progesterone_Receptor - 0.2 * HER2_Status + 
  0.1 * Ki_67 - 0.05 * Age + 0.3 * Surgical_Margins - 0.4 * Lymphovascular_Invasion +
  0.2 * Prior_Treatments + 0.1 * Surgery_Type - 0.5 * Metastasis + 0.01 * BMI + 
  0.2 * Genetic_Markers
p <- 1 / (1 + exp(-linear_term))
Y <- rbinom(n, 1, p)

# Combinar las variables independientes y la variable dependiente en un data frame
data <- data.frame(Y, Tumor_Size, Lymph_Nodes, Tumor_Grade, Estrogen_Receptor, 
                   Progesterone_Receptor, HER2_Status, Ki_67, Age, Histopathology,
                   Surgical_Margins, Lymphovascular_Invasion, Prior_Treatments,
                   Surgery_Type, Metastasis, BMI, Genetic_Markers)

# Dividir el conjunto de datos en entrenamiento y prueba
set.seed(123)  # Fijar la semilla para reproducibilidad
train_indices <- sample(1:n, size = 0.7 * n)  # 70% de los datos para entrenamiento
train_set <- data[train_indices, ]  # Conjunto de entrenamiento
test_set <- data[-train_indices, ]  # Conjunto de prueba

# Ajuste del modelo de regresión logística en el conjunto de entrenamiento
# Ajustamos un modelo de regresión logística utilizando las variables independientes para
 predecir Y
model <- glm(Y ~ ., data = train_set, family = binomial)

# Resumen del modelo
# Mostramos un resumen del modelo ajustado
summary(model)

# Guardar el modelo y los resultados en un archivo
# Guardamos el modelo ajustado en un archivo .RData
save(model, file = "regresion_logistica_cancer_modelo_simulado.RData")

# Guardar los datos simulados en archivos CSV
# Guardamos los conjuntos de datos de entrenamiento y prueba en archivos CSV
write.csv(train_set, "datos_entrenamiento_cancer_simulado.csv", row.names = FALSE)
write.csv(test_set, "datos_prueba_cancer_simulado.csv", row.names = FALSE)

# Hacer predicciones en el conjunto de prueba
# Utilizamos el modelo ajustado para hacer predicciones en el conjunto de prueba
test_set$prob_pred <- predict(model, newdata = test_set, type = "response")
test_set$Y_pred <- ifelse(test_set$prob_pred > 0.5, 1, 0)  
# Convertimos probabilidades a clases binarias

# Calcular la precisión de las predicciones
# Calculamos la precisión de las predicciones comparando con los valores reales de Y
accuracy <- mean(test_set$Y_pred == test_set$Y)
cat("La precisión del modelo en el conjunto de prueba es:", accuracy, "\n")

# Guardar las predicciones en un archivo CSV
# Guardamos las predicciones y las probabilidades predichas en un archivo CSV
write.csv(test_set, "predicciones_cancer_simulado.csv", row.names = FALSE)

# Graficar los coeficientes estimados
# Graficamos los coeficientes estimados del modelo ajustado
plot(coef(model), main = "Coeficientes Estimados del Modelo de Regresión Logística", 
     xlab = "Variables", ylab = "Coeficientes", type = "h", col = "blue")
abline(h = 0, col = "red", lwd = 2)

# Mostrar un mensaje indicando que el proceso ha finalizado
cat("El modelo de regresión logística se ha ajustado, se han hecho predicciones 
y los resultados se han guardado en 'regresion_logistica_cancer_modelo_simulado.RData'.\n")
\end{verbatim}




\chapter{Bibliografía}
\begin{thebibliography}{99}

\bibitem{ISL}
James, G., Witten, D., Hastie, T., and Tibshirani, R. (2013). \textit{An Introduction to Statistical Learning: with Applications in R}. Springer.

\bibitem{Logistic}
Hosmer, D. W., Lemeshow, S., and Sturdivant, R. X. (2013). \textit{Applied Logistic Regression} (3rd ed.). Wiley.

\bibitem{PatternRecognition}
Bishop, C. M. (2006). \textit{Pattern Recognition and Machine Learning}. Springer.

\bibitem{Harrell}
Harrell, F. E. (2015). \textit{Regression Modeling Strategies: With Applications to Linear Models, Logistic and Ordinal Regression, and Survival Analysis}. Springer.

\bibitem{RDocumentation}
R Documentation and Tutorials: \url{https://cran.r-project.org/manuals.html}

\bibitem{RBlogger}
Tutorials on R-bloggers: \url{https://www.r-bloggers.com/}

\bibitem{CourseraML}
Coursera: \textit{Machine Learning} by Andrew Ng.

\bibitem{edXDS}
edX: \textit{Data Science and Machine Learning Essentials} by Microsoft.

% Libros adicionales
\bibitem{Ross}
Ross, S. M. (2014). \textit{Introduction to Probability and Statistics for Engineers and Scientists}. Academic Press.

\bibitem{DeGroot}
DeGroot, M. H., and Schervish, M. J. (2012). \textit{Probability and Statistics} (4th ed.). Pearson.

\bibitem{Hogg}
Hogg, R. V., McKean, J., and Craig, A. T. (2019). \textit{Introduction to Mathematical Statistics} (8th ed.). Pearson.

\bibitem{Kleinbaum}
Kleinbaum, D. G., and Klein, M. (2010). \textit{Logistic Regression: A Self-Learning Text} (3rd ed.). Springer.

% Artículos y tutoriales adicionales
\bibitem{Wasserman}
Wasserman, L. (2004). \textit{All of Statistics: A Concise Course in Statistical Inference}. Springer.

\bibitem{KhanAcademy}
Probability and Statistics Tutorials on Khan Academy: \url{https://www.khanacademy.org/math/statistics-probability}

\bibitem{OnlineStatBook}
Online Statistics Education: \url{http://onlinestatbook.com/}

\bibitem{Peng}
Peng, C. Y. J., Lee, K. L., and Ingersoll, G. M. (2002). \textit{An Introduction to Logistic Regression Analysis and Reporting}. The Journal of Educational Research.

\bibitem{Agresti}
Agresti, A. (2007). \textit{An Introduction to Categorical Data Analysis} (2nd ed.). Wiley.

\bibitem{Han}
Han, J., Pei, J., and Kamber, M. (2011). \textit{Data Mining: Concepts and Techniques}. Morgan Kaufmann.

\bibitem{TowardsDataScience}
Data Cleaning and Preprocessing on Towards Data Science: \url{https://towardsdatascience.com/data-cleaning-and-preprocessing}

\bibitem{Molinaro}
Molinaro, A. M., Simon, R., and Pfeiffer, R. M. (2005). \textit{Prediction error estimation: a comparison of resampling methods}. Bioinformatics.

\bibitem{EvaluatingModels}
Evaluating Machine Learning Models on Towards Data Science: \url{https://towardsdatascience.com/evaluating-machine-learning-models}

\bibitem{LogisticRegressionGuide}
Practical Guide to Logistic Regression in R on Towards Data Science: \url{https://towardsdatascience.com/practical-guide-to-logistic-regression-in-r}

% Cursos en línea adicionales
\bibitem{CourseraStatistics}
Coursera: \textit{Statistics with R} by Duke University.

\bibitem{edXProbability}
edX: \textit{Data Science: Probability} by Harvard University.

\bibitem{CourseraLogistic}
Coursera: \textit{Logistic Regression} by Stanford University.

\bibitem{edXInference}
edX: \textit{Data Science: Inference and Modeling} by Harvard University.

\bibitem{CourseraWrangling}
Coursera: \textit{Data Science: Wrangling and Cleaning} by Johns Hopkins University.

\bibitem{edXRBasics}
edX: \textit{Data Science: R Basics} by Harvard University.

\bibitem{CourseraRegression}
Coursera: \textit{Regression Models} by Johns Hopkins University.

\bibitem{edXStatInference}
edX: \textit{Data Science: Statistical Inference} by Harvard University.

\bibitem{SurvivalAnalysis}
An Introduction to Survival Analysis on Towards Data Science: \url{https://towardsdatascience.com/an-introduction-to-survival-analysis}

\bibitem{MultinomialLogistic}
Multinomial Logistic Regression on DataCamp: \url{https://www.datacamp.com/community/tutorials/multinomial-logistic-regression-R}

\bibitem{CourseraSurvival}
Coursera: \textit{Survival Analysis} by Johns Hopkins University.

\bibitem{edXHighthroughput}
edX: \textit{Data Science: Statistical Inference and Modeling for High-throughput Experiments} by Harvard University.

\end{thebibliography}


\end{document}
