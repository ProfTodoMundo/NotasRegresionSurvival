\documentclass{report}
\usepackage[utf8]{inputenc}
\usepackage{amsmath}
\usepackage{amssymb}
\usepackage{geometry}
\usepackage{hyperref}
\usepackage{fancyhdr}
\usepackage{titlesec} % Paquete para modificar títulos de secciones
\usepackage[spanish]{babel} % Paquete para definir el idioma
\usepackage{listings} % Para incluir código fuente


\title{Curso Elemental de Regresi\'on Logística y Análisis de Supervivencia}
\author{Carlos E Mart\'inez-Rodr\'iguez}
\date{Julio 2024}

\geometry{
  a4paper,
  left=25mm,
  right=25mm,
  top=30mm,
  bottom=30mm,
}

% Configuración de encabezados y pies de página
\pagestyle{fancy}
\fancyhf{}
\fancyhead[L]{\leftmark}
\fancyfoot[C]{\thepage}
\fancyfoot[R]{\rightmark}
\fancyfoot[L]{Carlos E Mart\'inez-Rodr\'iguez} % Nombre del autor en la parte inferior izquierda

% Redefinir el nombre de los capítulos
\titleformat{\chapter}[display]
  {\normalfont\huge\bfseries}
  {CAPÍTULO \thechapter}
  {20pt}
  {\Huge}

\newtheorem{Algthm}{Algoritmo}%[section]

% Configuración para la inclusión de código fuente en R
\lstset{
    language=R,
    basicstyle=\ttfamily\small,
    numbers=left,
    numberstyle=\tiny,
    stepnumber=1,
    numbersep=5pt,
    showspaces=false,
    showstringspaces=false,
    showtabs=false,
    frame=single,
    tabsize=2,
    captionpos=b,
    breaklines=true,
    breakatwhitespace=false,
    title=\lstname
}

\begin{document}

\maketitle

\tableofcontents

\part{PRIMERA PARTE}
\chapter{Introducci\'on a la Regresi\'on Logística}

\section{Conceptos Básicos}

La regresión logística es una técnica de modelado estadístico utilizada para predecir la probabilidad de un evento binario (es decir, un evento que tiene dos posibles resultados) en función de una o más variables independientes. Es ampliamente utilizada en diversas disciplinas, como medicina, economía, biología, y ciencias sociales, para analizar y predecir resultados binarios.

Un modelo de regresión logística tiene la forma de una ecuación que describe cómo una variable dependiente binaria $Y$ (que puede tomar los valores $0$ o $1$) está relacionada con una o más variables independientes $X_1, X_2, \ldots, X_n$. A diferencia de la regresión lineal, que predice un valor continuo, la regresión logística predice una probabilidad que puede ser interpretada como la probabilidad de que $Y=1$ dado un conjunto de valores para $X_1, X_2, \ldots, X_n$.

\section{Regresión Lineal}

La regresión lineal es una técnica de modelado estadístico utilizada para predecir el valor de una variable dependiente continua en función de una o más variables independientes.

\subsection*{Modelo Lineal}

El modelo de regresión lineal tiene la forma:
\begin{equation}
Y = \beta_0 + \beta_1 X_1 + \beta_2 X_2 + \ldots + \beta_n X_n + \epsilon
\end{equation}
donde:
\begin{itemize}
    \item $Y$ es la variable dependiente.
    \item $\beta_0$ es la intersección con el eje $Y$ o término constante.
    \item $\beta_1, \beta_2, \ldots, \beta_n$ son los coeficientes que representan la relación entre las variables independientes y la variable dependiente.
    \item $X_1, X_2, \ldots, X_n$ son las variables independientes.
    \item $\epsilon$ es el término de error, que representa la desviación de los datos observados de los valores predichos por el modelo.
\end{itemize}

\subsection*{Mínimos Cuadrados Ordinarios (OLS)}

El objetivo de la regresión lineal es encontrar los valores de los coeficientes $\beta_0, \beta_1, \ldots, \beta_n$ que minimicen la suma de los cuadrados de las diferencias entre los valores observados y los valores predichos. Este método se conoce como mínimos cuadrados ordinarios (OLS, por sus siglas en inglés).

La función de costo que se minimiza es:
\begin{equation}
J\left(\beta_0, \beta_1, \ldots, \beta_n\right) = \sum_{i=1}^{n}\left(y_i - \hat{y}_i\right)^2
\end{equation}
donde:
\begin{itemize}
    \item $y_i$ es el valor observado de la variable dependiente para la $i$-ésima observación.
    \item $\hat{y}_i$ es el valor predicho por el modelo para la $i$-ésima observación, dado por:
    \begin{equation}
    \hat{y}_i = \beta_0 + \beta_1 x_{i1} + \beta_2 x_{i2} + \ldots + \beta_n x_{in}
    \end{equation}
\end{itemize}

Para encontrar los valores óptimos de los coeficientes, se toman las derivadas parciales de la función de costo con respecto a cada coeficiente y se igualan a cero:
\begin{equation}
\frac{\partial J}{\partial \beta_j} = 0 \quad \text{para } j = 0, 1, \ldots, n
\end{equation}

Resolviendo este sistema de ecuaciones, se obtienen los valores de los coeficientes que minimizan la función de costo.

\section{Regresión Logística}

La deducción de la fórmula de la regresión logística comienza con la necesidad de modelar la probabilidad de un evento binario. Queremos encontrar una función que relacione las variables independientes con la probabilidad de que la variable dependiente tome el valor $1$.

\subsection*{Probabilidad y Odds}

La probabilidad de que el evento ocurra, $P(Y=1)$, se denota como $p$. La probabilidad de que el evento no ocurra, $P(Y=0)$, es $1-p$. Los \textit{odds} (chances) de que ocurra el evento se definen como:
\begin{equation}
\text{odds} = \frac{p}{1-p}
\end{equation}
Los \textit{odds} nos indican cuántas veces más probable es que ocurra el evento frente a que no ocurra.

\subsection*{Transformación Logit}

Para simplificar el modelado de los \textit{odds}, aplicamos el logaritmo natural, obteniendo la función logit:
\begin{equation}
\text{logit}(p) = \log\left(\frac{p}{1-p}\right)
\end{equation}
La transformación logit es útil porque convierte el rango de la probabilidad (0, 1) al rango de números reales $\left(-\infty, \infty\right)$.

\subsection*{Modelo Lineal en el Espacio Logit}

La idea clave de la regresión logística es modelar la transformación logit de la probabilidad como una combinación lineal de las variables independientes:
\begin{equation}
\text{logit}(p) = \log\left(\frac{p}{1-p}\right) = \beta_0 + \beta_1 X_1 + \beta_2 X_2 + \ldots + \beta_n X_n
\end{equation}
Aquí, $\beta_0$ es el intercepto y $\beta_1, \beta_2, \ldots, \beta_n$ son los coeficientes asociados con las variables independientes $X_1, X_2, \ldots, X_n$.

\subsection*{Invertir la Transformación Logit}

Para expresar $p$ en función de una combinación lineal de las variables independientes, invertimos la transformación logit. Partimos de la ecuación:
\begin{equation}
\log\left(\frac{p}{1-p}\right) = \beta_0 + \beta_1 X_1 + \beta_2 X_2 + \ldots + \beta_n X_n
\end{equation}
Aplicamos la exponenciación a ambos lados:
\begin{equation}
\frac{p}{1-p} = e^{\beta_0 + \beta_1 X_1 + \beta_2 X_2 + \ldots + \beta_n X_n}
\end{equation}
Despejamos $p$:
\begin{equation}
p = \frac{e^{\beta_0 + \beta_1 X_1 + \beta_2 X_2 + \ldots + \beta_n X_n}}{1 + e^{\beta_0 + \beta_1 X_1 + \beta_2 X_2 + \ldots + \beta_n X_n}}
\end{equation}

\subsection*{Función Logística}

La expresión final que obtenemos es conocida como la función logística:
\begin{equation}
p = \frac{1}{1 + e^{-\left(\beta_0 + \beta_1 X_1 + \beta_2 X_2 + \ldots + \beta_n X_n\right)}}
\end{equation}
Esta función describe cómo las variables independientes se relacionan con la probabilidad de que el evento de interés ocurra. Los coeficientes $\beta_0, \beta_1, \ldots, \beta_n$ se estiman a partir de los datos utilizando el método de máxima verosimilitud.

\section{Método de Máxima Verosimilitud}

En la regresión logística, los coeficientes del modelo se estiman utilizando el método de máxima verosimilitud. Este método busca encontrar los valores de los coeficientes que maximicen la probabilidad de observar los datos dados los valores de las variables independientes.

\subsection*{Función de Verosimilitud}

Para un conjunto de $n$ observaciones, la función de verosimilitud $L$ se define como el producto de las probabilidades individuales de observar cada dato:
\begin{equation}
L(\beta_0, \beta_1, \ldots, \beta_n) = \prod_{i=1}^{n} p_i^{y_i} (1 - p_i)^{1 - y_i}
\end{equation}
donde $y_i$ es el valor observado de la variable dependiente para la $i$-ésima observación y $p_i$ es la probabilidad predicha de que $Y_i = 1$. Aquí, $p_i$ es dado por la función logística:
\begin{equation}
p_i = \frac{1}{1 + e^{-(\beta_0 + \beta_1 X_{i1} + \beta_2 X_{i2} + \ldots + \beta_n X_{in})}}
\end{equation}

\subsection*{Función de Log-Verosimilitud}

Para simplificar los cálculos, trabajamos con el logaritmo de la función de verosimilitud, conocido como la función de log-verosimilitud. Tomar el logaritmo convierte el producto en una suma:
\begin{equation}
\log L(\beta_0, \beta_1, \ldots, \beta_n) = \sum_{i=1}^{n} \left[ y_i \log(p_i) + (1 - y_i) \log(1 - p_i) \right]
\end{equation}

Sustituyendo $p_i$:
\begin{equation}
\log L(\beta_0, \beta_1, \ldots, \beta_n) = \sum_{i=1}^{n} \left[ y_i (\beta_0 + \beta_1 X_{i1} + \beta_2 X_{i2} + \ldots + \beta_n X_{in}) - \log(1 + e^{\beta_0 + \beta_1 X_{i1} + \beta_2 X_{i2} + \ldots + \beta_n X_{in}}) \right]
\end{equation}

\subsection*{Maximización de la Log-Verosimilitud}

El objetivo es encontrar los valores de $\beta_0, \beta_1, \ldots, \beta_n$ que maximicen la función de log-verosimilitud. Esto se hace derivando la función de log-verosimilitud con respecto a cada uno de los coeficientes y encontrando los puntos críticos.

Para $\beta_j$, la derivada parcial de la función de log-verosimilitud es:
\begin{equation}
\frac{\partial \log L}{\partial \beta_j} = \sum_{i=1}^{n} \left[ y_i X_{ij} - \frac{X_{ij} e^{\beta_0 + \beta_1 X_{i1} + \beta_2 X_{i2} + \ldots + \beta_n X_{in}}}{1 + e^{\beta_0 + \beta_1 X_{i1} + \beta_2 X_{i2} + \ldots + \beta_n X_{in}}} \right]
\end{equation}

Esto se simplifica a:
\begin{equation}
\frac{\partial \log L}{\partial \beta_j} = \sum_{i=1}^{n} X_{ij} (y_i - p_i)
\end{equation}

Para maximizar la log-verosimilitud, resolvemos el sistema de ecuaciones $\frac{\partial \log L}{\partial \beta_j} = 0$ para todos los $j$ de 0 a $n$. Este sistema de ecuaciones no tiene una solución analítica cerrada, por lo que se resuelve numéricamente utilizando métodos iterativos como el algoritmo de Newton-Raphson.

\subsection*{Método de Newton-Raphson}

El método de Newton-Raphson es un algoritmo iterativo que se utiliza para encontrar las raíces de una función. En el contexto de la regresión logística, se utiliza para maximizar la función de log-verosimilitud encontrando los valores de los coeficientes $\beta_0, \beta_1, \ldots, \beta_n$. El método de Newton-Raphson se basa en una aproximación de segundo orden de la función objetivo. Dado un valor inicial de los coeficientes $\beta^{(0)}$, se iterativamente actualiza el valor de los coeficientes utilizando la fórmula:
\begin{equation}
\beta^{(k+1)} = \beta^{(k)} - \left[ \mathbf{H}(\beta^{(k)}) \right]^{-1} \mathbf{g}(\beta^{(k)})
\end{equation}
donde:
\begin{itemize}
    \item $\beta^{(k)}$ es el vector de coeficientes en la $k$-ésima iteración.
    \item $\mathbf{H}(\beta^{(k)})$ es la matriz Hessiana (matriz de segundas derivadas) evaluada en $\beta^{(k)}$.
    \item $\mathbf{g}(\beta^{(k)})$ es el gradiente (vector de primeras derivadas) evaluado en $\beta^{(k)}$.
\end{itemize}

\subsubsection*{Gradiente}

El gradiente de la función de log-verosimilitud con respecto a los coeficientes $\beta$ es:
\begin{equation}
\mathbf{g}(\beta) = \frac{\partial \log L}{\partial \beta} = \sum_{i=1}^{n} \mathbf{X}_i (y_i - p_i)
\end{equation}
donde $\mathbf{X}_i$ es el vector de valores de las variables independientes para la $i$-ésima observación.

\subsubsection*{Hessiana}

La matriz Hessiana de la función de log-verosimilitud con respecto a los coeficientes $\beta$ es:
\begin{equation}
\mathbf{H}(\beta) = \frac{\partial^2 \log L}{\partial \beta \partial \beta^T} = -\sum_{i=1}^{n} p_i (1 - p_i) \mathbf{X}_i \mathbf{X}_i^T
\end{equation}

\subsubsection*{Algoritmo Newton-Raphson}

El algoritmo Newton-Raphson para la regresión logística se puede resumir en los siguientes pasos:
\begin{Algthm}
\begin{enumerate}
    \item Inicializar el vector de coeficientes $\beta^{(0)}$ (por ejemplo, con ceros o valores pequeños aleatorios).
    \item Calcular el gradiente $\mathbf{g}(\beta^{(k)})$ y la matriz Hessiana $\mathbf{H}(\beta^{(k)})$ en la iteración $k$.
    \item Actualizar los coeficientes utilizando la fórmula:
    \begin{equation}
    \beta^{(k+1)} = \beta^{(k)} - \left[ \mathbf{H}(\beta^{(k)}) \right]^{-1} \mathbf{g}(\beta^{(k)})
    \end{equation}
    \item Repetir los pasos 2 y 3 hasta que la diferencia entre $\beta^{(k+1)}$ y $\beta^{(k)}$ sea menor que un umbral predefinido (criterio de convergencia).
\end{enumerate}
\end{Algthm}

\section{Representaci\'on vectorial}

En el contexto de la regresión logística, los vectores $X_1, X_2, \ldots, X_n$ representan las variables independientes. Cada $X_j$ es un vector columna que contiene los valores de la variable independiente $j$ para cada una de las $n$ observaciones. Es decir,

\begin{equation}
X_j = \begin{bmatrix}
x_{1j} \\
x_{2j} \\
\vdots \\
x_{nj}
\end{bmatrix}
\end{equation}

Para simplificar la notación y los cálculos, a menudo combinamos todos los vectores de variables independientes en una única matriz de diseño $\mathbf{X}$ de tamaño $n \times (k+1)$, donde $n$ es el número de observaciones y $k+1$ es el número de variables independientes más el término de intercepto. La primera columna de $\mathbf{X}$ corresponde a un vector de unos para el término de intercepto, y las demás columnas corresponden a los valores de las variables independientes:

\begin{equation}
\mathbf{X} = \begin{bmatrix}
1 & x_{11} & x_{12} & \ldots & x_{1k} \\
1 & x_{21} & x_{22} & \ldots & x_{2k} \\
\vdots & \vdots & \vdots & \ddots & \vdots \\
1 & x_{n1} & x_{n2} & \ldots & x_{nk}
\end{bmatrix}
\end{equation}

De esta forma, el modelo logit puede ser escrito de manera compacta utilizando la notación matricial:

\begin{equation}
\text{logit}(p) = \log\left(\frac{p}{1-p}\right) = \mathbf{X} \boldsymbol{\beta}
\end{equation}

donde $\boldsymbol{\beta}$ es el vector de coeficientes:

\begin{equation}
\boldsymbol{\beta} = \begin{bmatrix}
\beta_0 \\
\beta_1 \\
\beta_2 \\
\vdots \\
\beta_k
\end{bmatrix}
\end{equation}

Así, la probabilidad $p$ se puede expresar como:

\begin{equation}
p = \frac{1}{1 + e^{-\mathbf{X} \boldsymbol{\beta}}}
\end{equation}

Esta notación matricial simplifica la implementación y la derivación de los estimadores de los coeficientes en la regresión logística.

\subsection*{Estimación de los Coeficientes}

Para estimar los coeficientes $\boldsymbol{\beta}$ en la regresión logística, se utiliza el método de máxima verosimilitud. La función de verosimilitud $L(\boldsymbol{\beta})$ se define como el producto de las probabilidades de las observaciones dadas las variables independientes:

\begin{equation}
L(\boldsymbol{\beta}) = \prod_{i=1}^{n} p_i^{y_i} (1 - p_i)^{1 - y_i}
\end{equation}

donde $y_i$ es el valor observado de la variable dependiente para la $i$-ésima observación, y $p_i$ es la probabilidad predicha de que $Y_i = 1$.

La función de log-verosimilitud, que es más fácil de maximizar, se obtiene tomando el logaritmo natural de la función de verosimilitud:

\begin{equation}
\log L(\boldsymbol{\beta}) = \sum_{i=1}^{n} \left[ y_i \log(p_i) + (1 - y_i) \log(1 - p_i) \right]
\end{equation}

Sustituyendo $p_i = \frac{1}{1 + e^{-\mathbf{X}_i \boldsymbol{\beta}}}$, donde $\mathbf{X}_i$ es la $i$-ésima fila de la matriz de diseño $\mathbf{X}$, obtenemos:

\begin{equation}
\log L(\boldsymbol{\beta}) = \sum_{i=1}^{n} \left[ y_i (\mathbf{X}_i \boldsymbol{\beta}) - \log(1 + e^{\mathbf{X}_i \boldsymbol{\beta}}) \right]
\end{equation}

Para encontrar los valores de $\boldsymbol{\beta}$ que maximizan la función de log-verosimilitud, se utiliza un algoritmo iterativo como el método de Newton-Raphson. Este método requiere calcular el gradiente y la matriz Hessiana de la función de log-verosimilitud.

\subsubsection*{Gradiente}

El gradiente de la función de log-verosimilitud con respecto a $\boldsymbol{\beta}$ es:

\begin{equation}
\nabla \log L(\boldsymbol{\beta}) = \mathbf{X}^T (\mathbf{y} - \mathbf{p})
\end{equation}

donde $\mathbf{y}$ es el vector de valores observados y $\mathbf{p}$ es el vector de probabilidades predichas.

\subsubsection*{Matriz Hessiana}

La matriz Hessiana de la función de log-verosimilitud es:

\begin{equation}
\mathbf{H}(\boldsymbol{\beta}) = -\mathbf{X}^T \mathbf{W} \mathbf{X}
\end{equation}

donde $\mathbf{W}$ es una matriz diagonal de pesos con elementos $w_i = p_i (1 - p_i)$.

\subsubsection*{Método de Newton-Raphson}

El método de Newton-Raphson actualiza los coeficientes $\boldsymbol{\beta}$ de la siguiente manera:

\begin{equation}
\boldsymbol{\beta}^{(t+1)} = \boldsymbol{\beta}^{(t)} - [\mathbf{H}(\boldsymbol{\beta}^{(t)})]^{-1} \nabla \log L(\boldsymbol{\beta}^{(t)})
\end{equation}

Iterando este proceso hasta que la diferencia entre $\boldsymbol{\beta}^{(t+1)}$ y $\boldsymbol{\beta}^{(t)}$ sea menor que un umbral predefinido, se obtienen los estimadores de máxima verosimilitud para los coeficientes de la regresión logística.



\chapter{Elementos de Probabilidad}
\section{Introducci\'on}

Los fundamentos de probabilidad y estad\'istica son esenciales para comprender y aplicar t\'ecnicas de an\'alisis de datos y modelado estad\'istico, incluyendo la regresi\'on lineal y log\'istica. Este cap\'itulo proporciona una revisi\'on de los conceptos clave en probabilidad y estad\'istica que son relevantes para estos m\'etodos.

\section{Probabilidad}

La probabilidad es una medida de la incertidumbre o el grado de creencia en la ocurrencia de un evento. Los conceptos fundamentales incluyen:

\subsection{Espacio Muestral y Eventos}

El espacio muestral, denotado como $S$, es el conjunto de todos los posibles resultados de un experimento aleatorio. Un evento es un subconjunto del espacio muestral. Por ejemplo, si lanzamos un dado, el espacio muestral es:
\begin{eqnarray*}
S = \{1, 2, 3, 4, 5, 6\}
\end{eqnarray*}
Un evento podr\'ia ser obtener un n\'umero par:
\begin{eqnarray*}
E = \{2, 4, 6\}
\end{eqnarray*}

\subsection{Definiciones de Probabilidad}

Existen varias definiciones de probabilidad, incluyendo la probabilidad cl\'asica, la probabilidad frecuentista y la probabilidad bayesiana.

\subsubsection{Probabilidad Cl\'asica}

La probabilidad cl\'asica se define como el n\'umero de resultados favorables dividido por el n\'umero total de resultados posibles:
\begin{eqnarray*}
P(E) = \frac{|E|}{|S|}
\end{eqnarray*}
donde $|E|$ es el n\'umero de elementos en el evento $E$ y $|S|$ es el n\'umero de elementos en el espacio muestral $S$.

\subsubsection{Probabilidad Frecuentista}

La probabilidad frecuentista se basa en la frecuencia relativa de ocurrencia de un evento en un gran n\'umero de repeticiones del experimento:
\begin{eqnarray*}
P(E) = \lim_{n \to \infty} \frac{n_E}{n}
\end{eqnarray*}
donde $n_E$ es el n\'umero de veces que ocurre el evento $E$ y $n$ es el n\'umero total de repeticiones del experimento.

\subsubsection{Probabilidad Bayesiana}

La probabilidad bayesiana se interpreta como un grado de creencia actualizado a medida que se dispone de nueva informaci\'on. Se basa en el teorema de Bayes:
\begin{eqnarray*}
P(A|B) = \frac{P(B|A)P(A)}{P(B)}
\end{eqnarray*}
donde $P(A|B)$ es la probabilidad de $A$ dado $B$, $P(B|A)$ es la probabilidad de $B$ dado $A$, $P(A)$ y $P(B)$ son las probabilidades de $A$ y $B$ respectivamente.

\section{Estad\'istica Bayesiana}

La estad\'istica bayesiana proporciona un enfoque coherente para el an\'alisis de datos basado en el teorema de Bayes. Los conceptos fundamentales incluyen:

\subsection{Prior y Posterior}

\subsubsection{Distribuci\'on Prior}

La distribuci\'on prior (apriori) representa nuestra creencia sobre los par\'ametros antes de observar los datos. Es una distribuci\'on de probabilidad que refleja nuestra incertidumbre inicial sobre los par\'ametros. Por ejemplo, si creemos que un par\'ametro $\theta$ sigue una distribuci\'on normal con media $\mu_0$ y varianza $\sigma_0^2$, nuestra prior ser\'ia:
\begin{eqnarray*}
P(\theta) = \frac{1}{\sqrt{2\pi\sigma_0^2}} e^{-\frac{(\theta-\mu_0)^2}{2\sigma_0^2}}
\end{eqnarray*}

\subsubsection{Verosimilitud}

La verosimilitud (likelihood) es la probabilidad de observar los datos dados los par\'ametros. Es una funci\'on de los par\'ametros $\theta$ dada una muestra de datos $X$:
\begin{eqnarray*}
L(\theta; X) = P(X|\theta)
\end{eqnarray*}
donde $X$ son los datos observados y $\theta$ son los par\'ametros del modelo.

\subsubsection{Distribuci\'on Posterior}

La distribuci\'on posterior (a posteriori) combina la informaci\'on de la prior y la verosimilitud utilizando el teorema de Bayes. Representa nuestra creencia sobre los par\'ametros despu\'es de observar los datos:
\begin{eqnarray*}
P(\theta|X) = \frac{P(X|\theta)P(\theta)}{P(X)}
\end{eqnarray*}
donde $P(\theta|X)$ es la distribuci\'on posterior, $P(X|\theta)$ es la verosimilitud, $P(\theta)$ es la prior y $P(X)$ es la probabilidad marginal de los datos.

La probabilidad marginal de los datos $P(X)$ se puede calcular como:
\begin{eqnarray*}
P(X) = \int_{\Theta} P(X|\theta)P(\theta) d\theta
\end{eqnarray*}
donde $\Theta$ es el espacio de todos los posibles valores del par\'ametro $\theta$.

\section{Distribuciones de Probabilidad}

Las distribuciones de probabilidad describen c\'omo se distribuyen los valores de una variable aleatoria. Existen distribuciones de probabilidad discretas y continuas.

\subsection{Distribuciones Discretas}

Una variable aleatoria discreta toma un n\'umero finito o contable de valores. Algunas distribuciones discretas comunes incluyen:

\subsubsection{Distribuci\'on Binomial}

La distribuci\'on binomial describe el n\'umero de \'exitos en una serie de ensayos de Bernoulli independientes y con la misma probabilidad de \'exito. La funci\'on de probabilidad es:
\begin{eqnarray*}
P(X = k) = \binom{n}{k} p^k (1-p)^{n-k}
\end{eqnarray*}
donde $X$ es el n\'umero de \'exitos, $n$ es el n\'umero de ensayos, $p$ es la probabilidad de \'exito en cada ensayo, y $\binom{n}{k}$ es el coeficiente binomial.

La funci\'on generadora de momentos (MGF) para la distribuci\'on binomial es:
\begin{eqnarray*}
M_X(t) = \left( 1 - p + pe^t \right)^n
\end{eqnarray*}

El valor esperado y la varianza de una variable aleatoria binomial son:
\begin{eqnarray*}
E(X) &=& np \\
\text{Var}(X) &=& np(1-p)
\end{eqnarray*}

\subsubsection{Distribuci\'on de Poisson}

La distribuci\'on de Poisson describe el n\'umero de eventos que ocurren en un intervalo de tiempo fijo o en un \'area fija. La funci\'on de probabilidad es:
\begin{eqnarray*}
P(X = k) = \frac{\lambda^k e^{-\lambda}}{k!}
\end{eqnarray*}
donde $X$ es el n\'umero de eventos, $\lambda$ es la tasa media de eventos por intervalo, y $k$ es el n\'umero de eventos observados.

La funci\'on generadora de momentos (MGF) para la distribuci\'on de Poisson es:
\begin{eqnarray*}
M_X(t) = e^{\lambda (e^t - 1)}
\end{eqnarray*}

El valor esperado y la varianza de una variable aleatoria de Poisson son:
\begin{eqnarray*}
E(X) &=& \lambda \\
\text{Var}(X) &=& \lambda
\end{eqnarray*}

\subsection{Distribuciones Continuas}

Una variable aleatoria continua toma un n\'umero infinito de valores en un intervalo continuo. Algunas distribuciones continuas comunes incluyen:

\subsubsection{Distribuci\'on Normal}

La distribuci\'on normal, tambi\'en conocida como distribuci\'on gaussiana, es una de las distribuciones m\'as importantes en estad\'istica. La funci\'on de densidad de probabilidad es:
\begin{eqnarray*}
f(x) = \frac{1}{\sqrt{2\pi\sigma^2}} e^{-\frac{(x-\mu)^2}{2\sigma^2}}
\end{eqnarray*}
donde $x$ es un valor de la variable aleatoria, $\mu$ es la media, y $\sigma$ es la desviaci\'on est\'andar.

La funci\'on generadora de momentos (MGF) para la distribuci\'on normal es:
\begin{eqnarray*}
M_X(t) = e^{\mu t + \frac{1}{2} \sigma^2 t^2}
\end{eqnarray*}

El valor esperado y la varianza de una variable aleatoria normal son:
\begin{eqnarray*}
E(X) &=& \mu \\
\text{Var}(X) &=& \sigma^2
\end{eqnarray*}

\subsubsection{Distribuci\'on Exponencial}

La distribuci\'on exponencial describe el tiempo entre eventos en un proceso de Poisson. La funci\'on de densidad de probabilidad es:
\begin{eqnarray*}
f(x) = \lambda e^{-\lambda x}
\end{eqnarray*}
donde $x$ es el tiempo entre eventos y $\lambda$ es la tasa media de eventos.

La funci\'on generadora de momentos (MGF) para la distribuci\'on exponencial es:
\begin{eqnarray*}
M_X(t) = \frac{\lambda}{\lambda - t}, \quad \text{para } t < \lambda
\end{eqnarray*}

El valor esperado y la varianza de una variable aleatoria exponencial son:
\begin{eqnarray*}
E(X) &=& \frac{1}{\lambda} \\
\text{Var}(X) &=& \frac{1}{\lambda^2}
\end{eqnarray*}

\section{Estad\'istica Descriptiva}

La estad\'istica descriptiva resume y describe las caracter\'isticas de un conjunto de datos. Incluye medidas de tendencia central, medidas de dispersi\'on y medidas de forma.

\subsection{Medidas de Tendencia Central}

Las medidas de tendencia central incluyen la media, la mediana y la moda.

\subsubsection{Media}

La media aritm\'etica es la suma de los valores dividida por el n\'umero de valores:
\begin{eqnarray*}
\bar{x} = \frac{1}{n} \sum_{i=1}^{n} x_i
\end{eqnarray*}
donde $x_i$ son los valores de la muestra y $n$ es el tama\~no de la muestra.

\subsubsection{Mediana}

La mediana es el valor medio cuando los datos est\'an ordenados. Si el n\'umero de valores es impar, la mediana es el valor central. Si es par, es el promedio de los dos valores centrales.

\subsubsection{Moda}

La moda es el valor que ocurre con mayor frecuencia en un conjunto de datos.

\subsection{Medidas de Dispersi\'on}

Las medidas de dispersi\'on incluyen el rango, la varianza y la desviaci\'on est\'andar.

\subsubsection{Rango}

El rango es la diferencia entre el valor m\'aximo y el valor m\'inimo de los datos:
\begin{eqnarray*}
Rango = x_{\text{max}} - x_{\text{min}}
\end{eqnarray*}

\subsubsection{Varianza}

La varianza es la media de los cuadrados de las diferencias entre los valores y la media:
\begin{eqnarray*}
\sigma^2 = \frac{1}{n} \sum_{i=1}^{n} (x_i - \bar{x})^2
\end{eqnarray*}

\subsubsection{Desviaci\'on Est\'andar}

La desviaci\'on est\'andar es la ra\'iz cuadrada de la varianza:
\begin{eqnarray*}
\sigma = \sqrt{\frac{1}{n} \sum_{i=1}^{n} (x_i - \bar{x})^2}
\end{eqnarray*}

\section{Inferencia Estad\'istica}

La inferencia estad\'istica es el proceso de sacar conclusiones sobre una poblaci\'on a partir de una muestra. Incluye la estimaci\'on de par\'ametros y la prueba de hip\'otesis.

\subsection{Estimaci\'on de Par\'ametros}

La estimaci\'on de par\'ametros implica el uso de datos muestrales para estimar los par\'ametros de una poblaci\'on.

\subsubsection{Estimador Puntual}

Un estimador puntual proporciona un \'unico valor como estimaci\'on de un par\'ametro de la poblaci\'on. Por ejemplo, la media muestral $\bar{x}$ es un estimador puntual de la media poblacional $\mu$. Otros ejemplos de estimadores puntuales son:

\begin{itemize}
    \item \textbf{Mediana muestral ($\tilde{x}$)}: Estimador de la mediana poblacional.
    \item \textbf{Varianza muestral ($s^2$)}: Estimador de la varianza poblacional $\sigma^2$, definido como:
    \begin{eqnarray*}
    s^2 = \frac{1}{n-1} \sum_{i=1}^{n} (x_i - \bar{x})^2
    \end{eqnarray*}
    \item \textbf{Desviaci\'on est\'andar muestral ($s$)}: Estimador de la desviaci\'on est\'andar poblacional $\sigma$, definido como:
    \begin{eqnarray*}
    s = \sqrt{s^2}
    \end{eqnarray*}
\end{itemize}

\subsubsection{Propiedades de los Estimadores Puntuales}

Los estimadores puntuales deben cumplir ciertas propiedades deseables, como:

\begin{itemize}
    \item \textbf{Insesgadez}: Un estimador es insesgado si su valor esperado es igual al valor del par\'ametro que estima.
    \begin{eqnarray*}
    E(\hat{\theta}) = \theta
    \end{eqnarray*}
    \item \textbf{Consistencia}: Un estimador es consistente si converge en probabilidad al valor del par\'ametro a medida que el tama\~no de la muestra tiende a infinito.
    \item \textbf{Eficiencia}: Un estimador es eficiente si tiene la varianza m\'as baja entre todos los estimadores insesgados.
\end{itemize}

\subsubsection{Estimador por Intervalo}

Un estimador por intervalo proporciona un rango de valores dentro del cual se espera que se encuentre el par\'ametro poblacional con un cierto nivel de confianza. Por ejemplo, un intervalo de confianza para la media es:
\begin{eqnarray*}
\left( \bar{x} - z \frac{\sigma}{\sqrt{n}}, \bar{x} + z \frac{\sigma}{\sqrt{n}} \right)
\end{eqnarray*}
donde $z$ es el valor cr\'itico correspondiente al nivel de confianza deseado, $\sigma$ es la desviaci\'on est\'andar poblacional y $n$ es el tama\~no de la muestra.

\subsection{Prueba de Hip\'otesis}

La prueba de hip\'otesis es un procedimiento para decidir si una afirmaci\'on sobre un par\'ametro poblacional es consistente con los datos muestrales.

\subsubsection{Hip\'otesis Nula y Alternativa}

La hip\'otesis nula ($H_0$) es la afirmaci\'on que se somete a prueba, y la hip\'otesis alternativa ($H_a$) es la afirmaci\'on que se acepta si se rechaza la hip\'otesis nula.

\subsubsection{Nivel de Significancia}

El nivel de significancia ($\alpha$) es la probabilidad de rechazar la hip\'otesis nula cuando es verdadera. Un valor com\'unmente utilizado es $\alpha = 0.05$.

\subsubsection{Estad\'istico de Prueba}

El estad\'istico de prueba es una medida calculada a partir de los datos muestrales que se utiliza para decidir si se rechaza la hip\'otesis nula. Por ejemplo, en una prueba $t$ para la media:
\begin{eqnarray*}
t = \frac{\bar{x} - \mu_0}{s / \sqrt{n}}
\end{eqnarray*}
donde $\bar{x}$ es la media muestral, $\mu_0$ es la media poblacional bajo la hip\'otesis nula, $s$ es la desviaci\'on est\'andar muestral y $n$ es el tama\~no de la muestra.

\subsubsection{P-valor}

El p-valor es la probabilidad de obtener un valor del estad\'istico de prueba al menos tan extremo como el observado, bajo la suposici\'on de que la hip\'otesis nula es verdadera. Si el p-valor es menor que el nivel de significancia $\alpha$, se rechaza la hip\'otesis nula. El p-valor se interpreta de la siguiente manera:

\begin{itemize}
    \item \textbf{P-valor bajo (p < 0.05)}: Evidencia suficiente para rechazar la hip\'otesis nula.
    \item \textbf{P-valor alto (p > 0.05)}: No hay suficiente evidencia para rechazar la hip\'otesis nula.
\end{itemize}

\subsubsection{Tipos de Errores}

En la prueba de hip\'otesis, se pueden cometer dos tipos de errores:

\begin{itemize}
    \item \textbf{Error Tipo I ($\alpha$)}: Rechazar la hip\'otesis nula cuando es verdadera.
    \item \textbf{Error Tipo II ($\beta$)}: No rechazar la hip\'otesis nula cuando es falsa.
\end{itemize}

\subsubsection{Tabla de Errores en la Prueba de Hip\'otesis}

A continuaci\'on se presenta una tabla que muestra los posibles resultados en una prueba de hip\'otesis, incluyendo los falsos positivos (error tipo I) y los falsos negativos (error tipo II):

\begin{table}[h]
\centering
\begin{tabular}{|c|c|c|}
\hline
 & \textbf{Hip\'otesis Nula Verdadera} & \textbf{Hip\'otesis Nula Falsa} \\
\hline
\textbf{Rechazar $H_0$} & Error Tipo I ($\alpha$) & Aceptar $H_a$ \\
\hline
\textbf{No Rechazar $H_0$} & Aceptar $H_0$ & Error Tipo II ($\beta$) \\
\hline
\end{tabular}
\caption{Resultados de la Prueba de Hip\'otesis}
\label{tab:hypothesis_testing}
\end{table}



\chapter{Matem\'aticas Detr\'as de la Regresi\'on Logística}
\section{Introducci\'on}

La regresi\'on log\'istica es una t\'ecnica de modelado estad\'istico utilizada para predecir la probabilidad de un evento binario en funci\'on de una o m\'as variables independientes. Este cap\'itulo profundiza en las matem\'aticas subyacentes a la regresi\'on log\'istica, incluyendo la funci\'on log\'istica, la funci\'on de verosimilitud, y los m\'etodos para estimar los coeficientes del modelo.

\section{Funci\'on Log\'istica}

La funci\'on log\'istica es la base de la regresi\'on log\'istica. Esta funci\'on transforma una combinaci\'on lineal de variables independientes en una probabilidad.

\subsection{Definici\'on}

La funci\'on log\'istica se define como:
\begin{eqnarray*}
p = \frac{1}{1 + e^{-(\beta_0 + \beta_1 X_1 + \beta_2 X_2 + \ldots + \beta_n X_n)}}
\end{eqnarray*}
donde $p$ es la probabilidad de que el evento ocurra, $\beta_0, \beta_1, \ldots, \beta_n$ son los coeficientes del modelo, y $X_1, X_2, \ldots, X_n$ son las variables independientes.

\subsection{Propiedades}

La funci\'on log\'istica tiene varias propiedades importantes:
\begin{itemize}
    \item \textbf{Rango}: La funci\'on log\'istica siempre produce un valor entre 0 y 1, lo que la hace adecuada para modelar probabilidades.
    \item \textbf{Monoton\'ia}: La funci\'on es mon\'otona creciente, lo que significa que a medida que la combinaci\'on lineal de variables independientes aumenta, la probabilidad tambi\'en aumenta.
    \item \textbf{Simetr\'ia}: La funci\'on log\'istica es sim\'etrica en torno a $p = 0.5$.
\end{itemize}

\section{Funci\'on de Verosimilitud}

La funci\'on de verosimilitud se utiliza para estimar los coeficientes del modelo de regresi\'on log\'istica. Esta funci\'on mide la probabilidad de observar los datos dados los coeficientes del modelo.

\subsection{Definici\'on}

Para un conjunto de $n$ observaciones, la funci\'on de verosimilitud $L$ se define como el producto de las probabilidades individuales de observar cada dato:
\begin{eqnarray*}
L(\beta_0, \beta_1, \ldots, \beta_n) = \prod_{i=1}^{n} p_i^{y_i} (1 - p_i)^{1 - y_i}
\end{eqnarray*}
donde $y_i$ es el valor observado de la variable dependiente para la $i$-\'esima observaci\'on y $p_i$ es la probabilidad predicha de que $Y_i = 1$.

\subsection{Funci\'on de Log-Verosimilitud}

Para simplificar los c\'alculos, trabajamos con el logaritmo de la funci\'on de verosimilitud, conocido como la funci\'on de log-verosimilitud. Tomar el logaritmo convierte el producto en una suma:
\begin{eqnarray*}
\log L(\beta_0, \beta_1, \ldots, \beta_n) = \sum_{i=1}^{n} \left[ y_i \log(p_i) + (1 - y_i) \log(1 - p_i) \right]
\end{eqnarray*}

Sustituyendo $p_i$:
\begin{eqnarray*}
\log L(\beta_0, \beta_1, \ldots, \beta_n) = \sum_{i=1}^{n} \left[ y_i (\beta_0 + \beta_1 X_{i1} + \beta_2 X_{i2} + \ldots + \beta_n X_{in}) - \log(1 + e^{\beta_0 + \beta_1 X_{i1} + \beta_2 X_{i2} + \ldots + \beta_n X_{in}}) \right]
\end{eqnarray*}

\section{Estimaci\'on de Coeficientes}

Los coeficientes del modelo de regresi\'on log\'istica se estiman maximizando la funci\'on de log-verosimilitud. Este proceso generalmente se realiza mediante m\'etodos iterativos como el algoritmo de Newton-Raphson.

\subsection{Gradiente y Hessiana}

Para maximizar la funci\'on de log-verosimilitud, necesitamos calcular su gradiente y su matriz Hessiana.

\subsubsection{Gradiente}

El gradiente de la funci\'on de log-verosimilitud con respecto a los coeficientes $\beta$ es:
\begin{eqnarray*}
\mathbf{g}(\beta) = \frac{\partial \log L}{\partial \beta} = \sum_{i=1}^{n} \mathbf{X}_i (y_i - p_i)
\end{eqnarray*}
donde $\mathbf{X}_i$ es el vector de valores de las variables independientes para la $i$-\'esima observaci\'on.

\subsubsection{Hessiana}

La matriz Hessiana de la funci\'on de log-verosimilitud con respecto a los coeficientes $\beta$ es:
\begin{eqnarray*}
\mathbf{H}(\beta) = \frac{\partial^2 \log L}{\partial \beta \partial \beta^T} = -\sum_{i=1}^{n} p_i (1 - p_i) \mathbf{X}_i \mathbf{X}_i^T
\end{eqnarray*}

\subsection{Algoritmo Newton-Raphson}

El algoritmo Newton-Raphson se utiliza para encontrar los valores de los coeficientes que maximizan la funci\'on de log-verosimilitud. El algoritmo se puede resumir en los siguientes pasos:
\begin{enumerate}
    \item Inicializar el vector de coeficientes $\beta^{(0)}$ (por ejemplo, con ceros o valores peque\~nos aleatorios).
    \item Calcular el gradiente $\mathbf{g}(\beta^{(k)})$ y la matriz Hessiana $\mathbf{H}(\beta^{(k)})$ en la iteraci\'on $k$.
    \item Actualizar los coeficientes utilizando la f\'ormula:
    \begin{eqnarray*}
    \beta^{(k+1)} = \beta^{(k)} - \left[ \mathbf{H}(\beta^{(k)}) \right]^{-1} \mathbf{g}(\beta^{(k)})
    \end{eqnarray*}
    \item Repetir los pasos 2 y 3 hasta que la diferencia entre $\beta^{(k+1)}$ y $\beta^{(k)}$ sea menor que un umbral predefinido (criterio de convergencia).
\end{enumerate}

\section{Validaci\'on del Modelo}

Una vez que se han estimado los coeficientes del modelo de regresi\'on log\'istica, es importante validar el modelo para asegurarse de que proporciona predicciones precisas.

\subsection{Curva ROC y AUC}

La curva ROC (Receiver Operating Characteristic) es una herramienta gr\'afica utilizada para evaluar el rendimiento de un modelo de clasificaci\'on binaria. El \'area bajo la curva (AUC) mide la capacidad del modelo para distinguir entre las clases.

\subsection{Matriz de Confusi\'on}

La matriz de confusi\'on es una tabla que resume el rendimiento de un modelo de clasificaci\'on al comparar las predicciones del modelo con los valores reales. Los t\'erminos en la matriz de confusi\'on incluyen verdaderos positivos, falsos positivos, verdaderos negativos y falsos negativos.



\chapter{Preparaci\'on de Datos y Selecci\'on de Variables}


\section{Introducci\'on}

La preparaci\'on de datos y la selecci\'on de variables son pasos cruciales en el proceso de modelado estad\'istico. Un modelo bien preparado y con las variables adecuadas puede mejorar significativamente la precisi\'on y la interpretabilidad del modelo. Este cap\'itulo proporciona una revisi\'on detallada de las t\'ecnicas de limpieza de datos, tratamiento de datos faltantes, codificaci\'on de variables categ\'oricas y selecci\'on de variables.

\section{Importancia de la Preparaci\'on de Datos}

La calidad de los datos es fundamental para el \'exito de cualquier an\'alisis estad\'istico. Los datos sin limpiar pueden llevar a modelos inexactos y conclusiones err\'oneas. La preparaci\'on de datos incluye varias etapas:
\begin{itemize}
    \item Limpieza de datos
    \item Tratamiento de datos faltantes
    \item Codificaci\'on de variables categ\'oricas
    \item Selecci\'on y transformaci\'on de variables
\end{itemize}

\section{Limpieza de Datos}

La limpieza de datos es el proceso de detectar y corregir (o eliminar) los datos incorrectos, incompletos o irrelevantes. Este proceso incluye:
\begin{itemize}
    \item Eliminaci\'on de duplicados
    \item Correcci\'on de errores tipogr\'aficos
    \item Consistencia de formato
    \item Tratamiento de valores extremos (outliers)
\end{itemize}

\section{Tratamiento de Datos Faltantes}

Los datos faltantes son un problema com\'un en los conjuntos de datos y pueden afectar la calidad de los modelos. Hay varias estrategias para manejar los datos faltantes:
\begin{itemize}
    \item \textbf{Eliminaci\'on de Datos Faltantes}: Se eliminan las filas o columnas con datos faltantes.
    \item \textbf{Imputaci\'on}: Se reemplazan los valores faltantes con estimaciones, como la media, la mediana o la moda.
    \item \textbf{Modelos Predictivos}: Se utilizan modelos predictivos para estimar los valores faltantes.
\end{itemize}

\subsection{Imputaci\'on de la Media}

Una t\'ecnica com\'un es reemplazar los valores faltantes con la media de la variable. Esto se puede hacer de la siguiente manera:
\begin{eqnarray*}
x_i = \begin{cases} 
      x_i & \text{si } x_i \text{ no es faltante} \\
      \bar{x} & \text{si } x_i \text{ es faltante}
   \end{cases}
\end{eqnarray*}
donde $\bar{x}$ es la media de la variable.

\section{Codificaci\'on de Variables Categ\'oricas}

Las variables categ\'oricas deben ser convertidas a un formato num\'erico antes de ser usadas en un modelo de regresi\'on log\'istica. Hay varias t\'ecnicas para codificar variables categ\'oricas:

\subsection{Codificaci\'on One-Hot}

La codificaci\'on one-hot crea una columna binaria para cada categor\'ia. Por ejemplo, si tenemos una variable categ\'orica con tres categor\'ias (A, B, C), se crean tres columnas:
\begin{eqnarray*}
\text{A} &=& [1, 0, 0] \\
\text{B} &=& [0, 1, 0] \\
\text{C} &=& [0, 0, 1]
\end{eqnarray*}

\subsection{Codificaci\'on Ordinal}

La codificaci\'on ordinal asigna un valor entero \'unico a cada categor\'ia, preservando el orden natural de las categor\'ias. Por ejemplo:
\begin{eqnarray*}
\text{Bajo} &=& 1 \\
\text{Medio} &=& 2 \\
\text{Alto} &=& 3
\end{eqnarray*}

\section{Selecci\'on de Variables}

La selecci\'on de variables es el proceso de elegir las variables m\'as relevantes para el modelo. Existen varias t\'ecnicas para la selecci\'on de variables:

\subsection{M\'etodos de Filtrado}

Los m\'etodos de filtrado seleccionan variables basadas en criterios estad\'isticos, como la correlaci\'on o la chi-cuadrado. Algunas t\'ecnicas comunes incluyen:
\begin{itemize}
    \item \textbf{An\'alisis de Correlaci\'on}: Se seleccionan variables con alta correlaci\'on con la variable dependiente y baja correlaci\'on entre ellas.
    \item \textbf{Pruebas de Chi-cuadrado}: Se utilizan para variables categ\'oricas para determinar la asociaci\'on entre la variable independiente y la variable dependiente.
\end{itemize}

\subsection{M\'etodos de Wrapper}

Los m\'etodos de wrapper eval\'uan m\'ultiples combinaciones de variables y seleccionan la combinaci\'on que optimiza el rendimiento del modelo. Ejemplos incluyen:
\begin{itemize}
    \item \textbf{Selecci\'on hacia Adelante}: Comienza con un modelo vac\'io y agrega variables una por una, seleccionando la variable que mejora m\'as el modelo en cada paso.
    \item \textbf{Selecci\'on hacia Atr\'as}: Comienza con todas las variables y elimina una por una, removiendo la variable que tiene el menor impacto en el modelo en cada paso.
    \item \textbf{Selecci\'on Paso a Paso}: Combina la selecci\'on hacia adelante y hacia atr\'as, agregando y eliminando variables seg\'un sea necesario.
\end{itemize}

\subsection{M\'etodos Basados en Modelos}

Los m\'etodos basados en modelos utilizan t\'ecnicas de regularizaci\'on como Lasso y Ridge para seleccionar variables. Estas t\'ecnicas a\~naden un t\'ermino de penalizaci\'on a la funci\'on de costo para evitar el sobreajuste.

\subsubsection{Regresi\'on Lasso}

La regresi\'on Lasso (Least Absolute Shrinkage and Selection Operator) a\~nade una penalizaci\'on $L_1$ a la funci\'on de costo:
\begin{eqnarray*}
J(\beta) = \sum_{i=1}^{n} (y_i - \hat{y}_i)^2 + \lambda \sum_{j=1}^{p} |\beta_j|
\end{eqnarray*}
donde $\lambda$ es el par\'ametro de regularizaci\'on que controla la cantidad de penalizaci\'on.

\subsubsection{Regresi\'on Ridge}

La regresi\'on Ridge a\~nade una penalizaci\'on $L_2$ a la funci\'on de costo:
\begin{eqnarray*}
J(\beta) = \sum_{i=1}^{n} (y_i - \hat{y}_i)^2 + \lambda \sum_{j=1}^{p} \beta_j^2
\end{eqnarray*}
donde $\lambda$ es el par\'ametro de regularizaci\'on.

\section{Implementaci\'on en R}

\subsection{Limpieza de Datos}

Para ilustrar la limpieza de datos en R, considere el siguiente conjunto de datos:
\begin{verbatim}
data <- data.frame(
  var1 = c(1, 2, 3, NA, 5),
  var2 = c("A", "B", "A", "B", "A"),
  var3 = c(10, 15, 10, 20, 25)
)

# Eliminaci\'on de filas con datos faltantes
data_clean <- na.omit(data)

# Imputaci\'on de la media
data$var1[is.na(data$var1)] <- mean(data$var1, na.rm = TRUE)
\end{verbatim}

\subsection{Codificaci\'on de Variables Categ\'oricas}

Para codificar variables categ\'oricas, utilice la funci\'on `model.matrix`:
\begin{verbatim}
data <- data.frame(
  var1 = c(1, 2, 3, 4, 5),
  var2 = c("A", "B", "A", "B", "A")
)

# Codificaci\'on one-hot
data_onehot <- model.matrix(~ var2 - 1, data = data)
\end{verbatim}

\subsection{Selecci\'on de Variables}

Para la selecci\'on de variables, utilice el paquete `caret`:
\begin{verbatim}
library(caret)

# Dividir los datos en conjuntos de entrenamiento y prueba
set.seed(123)
trainIndex <- createDataPartition(data$var1, p = .8, 
                                  list = FALSE, 
                                  times = 1)
dataTrain <- data[trainIndex,]
dataTest <- data[-trainIndex,]

# Modelo de regresi\'on log\'istica
model <- train(var1 ~ ., data = dataTrain, method = "glm", family = "binomial")

# Selecci\'on de variables
model <- step(model, direction = "both")
summary(model)
\end{verbatim}



\chapter{Evaluaci\'on del Modelo y Validaci\'on Cruzada}


\section{Introducción}

Evaluar la calidad y el rendimiento de un modelo de regresión logística es crucial para asegurar que las predicciones sean precisas y útiles. Este capítulo se centra en las técnicas y métricas utilizadas para evaluar modelos de clasificación binaria, así como en la validación cruzada, una técnica para evaluar la generalización del modelo.

\section{Métricas de Evaluación del Modelo}

Las métricas de evaluación permiten cuantificar la precisión y el rendimiento de un modelo. Algunas de las métricas más comunes incluyen:

\subsection{Curva ROC y AUC}

La curva ROC (Receiver Operating Characteristic) es una representación gráfica de la sensibilidad (verdaderos positivos) frente a 1 - especificidad (falsos positivos). El área bajo la curva (AUC) mide la capacidad del modelo para distinguir entre las clases.

\begin{eqnarray*}
\text{Sensibilidad} &=& \frac{\text{TP}}{\text{TP} + \text{FN}} \\
\text{Especificidad} &=& \frac{\text{TN}}{\text{TN} + \text{FP}}
\end{eqnarray*}

\subsection{Matriz de Confusión}

La matriz de confusión es una tabla que muestra el rendimiento del modelo comparando las predicciones con los valores reales. Los términos incluyen:
\begin{itemize}
    \item \textbf{Verdaderos Positivos (TP)}: Predicciones correctas de la clase positiva.
    \item \textbf{Falsos Positivos (FP)}: Predicciones incorrectas de la clase positiva.
    \item \textbf{Verdaderos Negativos (TN)}: Predicciones correctas de la clase negativa.
    \item \textbf{Falsos Negativos (FN)}: Predicciones incorrectas de la clase negativa.
\end{itemize}

\begin{table}[h]
\centering
\begin{tabular}{|c|c|c|}
\hline
 & \textbf{Predicción Positiva} & \textbf{Predicción Negativa} \\
\hline
\textbf{Real Positiva} & TP & FN \\
\hline
\textbf{Real Negativa} & FP & TN \\
\hline
\end{tabular}
\caption{Matriz de Confusión}
\label{tab:confusion_matrix}
\end{table}

\subsection{Precisión, Recall y F1-Score}

\begin{eqnarray*}
\text{Precisión} &=& \frac{\text{TP}}{\text{TP} + \text{FP}} \\
\text{Recall} &=& \frac{\text{TP}}{\text{TP} + \text{FN}} \\
\text{F1-Score} &=& 2 \cdot \frac{\text{Precisión} \cdot \text{Recall}}{\text{Precisión} + \text{Recall}}
\end{eqnarray*}

\subsection{Log-Loss}

La pérdida logarítmica (Log-Loss) mide la precisión de las probabilidades predichas. La fórmula es:
\begin{eqnarray*}
\text{Log-Loss} = -\frac{1}{n} \sum_{i=1}^{n} \left[ y_i \log(p_i) + (1 - y_i) \log(1 - p_i) \right]
\end{eqnarray*}
donde $y_i$ son los valores reales y $p_i$ son las probabilidades predichas.

\section{Validación Cruzada}

La validación cruzada es una técnica para evaluar la capacidad de generalización de un modelo. Existen varios tipos de validación cruzada:

\subsection{K-Fold Cross-Validation}

En K-Fold Cross-Validation, los datos se dividen en K subconjuntos. El modelo se entrena K veces, cada vez utilizando K-1 subconjuntos para el entrenamiento y el subconjunto restante para la validación.

\begin{eqnarray*}
\text{Error Medio} = \frac{1}{K} \sum_{k=1}^{K} \text{Error}_k
\end{eqnarray*}

\subsection{Leave-One-Out Cross-Validation (LOOCV)}

En LOOCV, cada observación se usa una vez como conjunto de validación y las restantes como conjunto de entrenamiento. Este método es computacionalmente costoso pero útil para conjuntos de datos pequeños.

\section{Ajuste y Sobreajuste del Modelo}

El ajuste adecuado del modelo es crucial para evitar el sobreajuste (overfitting) y el subajuste (underfitting).

\subsection{Sobreajuste}

El sobreajuste ocurre cuando un modelo se ajusta demasiado bien a los datos de entrenamiento, capturando ruido y patrones irrelevantes. Los síntomas incluyen una alta precisión en el entrenamiento y baja precisión en la validación.

\subsection{Subajuste}

El subajuste ocurre cuando un modelo no captura los patrones subyacentes de los datos. Los síntomas incluyen baja precisión tanto en el entrenamiento como en la validación.

\subsection{Regularización}

La regularización es una técnica para prevenir el sobreajuste añadiendo un término de penalización a la función de costo. Las técnicas comunes incluyen:
\begin{itemize}
    \item \textbf{Regresión Lasso (L1)}
    \item \textbf{Regresión Ridge (L2)}
\end{itemize}

\section{Implementación en R}

\subsection{Evaluación del Modelo}

\begin{verbatim}
# Cargar el paquete necesario
library(caret)

# Dividir los datos en conjuntos de entrenamiento y prueba
set.seed(123)
trainIndex <- createDataPartition(data$var1, p = .8, 
                                  list = FALSE, 
                                  times = 1)
dataTrain <- data[trainIndex,]
dataTest <- data[-trainIndex,]

# Entrenar el modelo de regresión logística
model <- train(var1 ~ ., data = dataTrain, method = "glm", family = "binomial")

# Predicciones en el conjunto de prueba
predictions <- predict(model, dataTest)

# Matriz de confusión
confusionMatrix(predictions, dataTest$var1)
\end{verbatim}

\subsection{Validación Cruzada}

\begin{verbatim}
# K-Fold Cross-Validation
control <- trainControl(method = "cv", number = 10)
model_cv <- train(var1 ~ ., data = dataTrain, method = "glm", 
                  family = "binomial", trControl = control)

# Evaluación del modelo con validación cruzada
print(model_cv)
\end{verbatim}



\chapter{Diagn\'ostico del Modelo y Ajuste de Par\'ametros}


\section{Introducci\'on}

El diagn\'ostico del modelo y el ajuste de par\'ametros son pasos esenciales para mejorar la precisi\'on y la robustez de los modelos de regresi\'on log\'istica. Este cap\'itulo se enfoca en las t\'ecnicas para diagnosticar problemas en los modelos y en m\'etodos para ajustar los par\'ametros de manera \'optima.

\section{Diagn\'ostico del Modelo}

El diagn\'ostico del modelo implica evaluar el rendimiento del modelo y detectar posibles problemas, como el sobreajuste, la multicolinealidad y la influencia de puntos de datos individuales.

\subsection{Residuos}

Los residuos son las diferencias entre los valores observados y los valores predichos por el modelo. El an\'alisis de residuos puede revelar patrones que indican problemas con el modelo.

\begin{eqnarray*}
\text{Residuo}_i = y_i - \hat{y}_i
\end{eqnarray*}

\subsubsection{Residuos Estudiantizados}

Los residuos estudiantizados se ajustan por la variabilidad del residuo y se utilizan para detectar outliers.

\begin{eqnarray*}
r_i = \frac{\text{Residuo}_i}{\hat{\sigma} \sqrt{1 - h_i}}
\end{eqnarray*}
donde $h_i$ es el leverage del punto de datos.

\subsection{Influencia}

La influencia mide el impacto de un punto de datos en los coeficientes del modelo. Los puntos con alta influencia pueden distorsionar el modelo.

\subsubsection{Distancia de Cook}

La distancia de Cook es una medida de la influencia de un punto de datos en los coeficientes del modelo.

\begin{eqnarray*}
D_i = \frac{r_i^2}{p} \cdot \frac{h_i}{1 - h_i}
\end{eqnarray*}
donde $p$ es el n\'umero de par\'ametros en el modelo.

\subsection{Multicolinealidad}

La multicolinealidad ocurre cuando dos o m\'as variables independientes est\'an altamente correlacionadas. Esto puede inflar las varianzas de los coeficientes y hacer que el modelo sea inestable.

\subsubsection{Factor de Inflaci\'on de la Varianza (VIF)}

El VIF mide cu\'anto se inflan las varianzas de los coeficientes debido a la multicolinealidad.

\begin{eqnarray*}
\text{VIF}_j = \frac{1}{1 - R_j^2}
\end{eqnarray*}
donde $R_j^2$ es el coeficiente de determinaci\'on de la regresi\'on de la variable $j$ contra todas las dem\'as variables.

\section{Ajuste de Par\'ametros}

El ajuste de par\'ametros implica seleccionar los valores \'optimos para los hiperpar\'ametros del modelo. Esto puede mejorar el rendimiento y prevenir el sobreajuste.

\subsection{Grid Search}

El grid search es un m\'etodo exhaustivo para ajustar los par\'ametros. Se define una rejilla de posibles valores de par\'ametros y se eval\'ua el rendimiento del modelo para cada combinaci\'on.

\subsection{Random Search}

El random search selecciona aleatoriamente combinaciones de valores de par\'ametros dentro de un rango especificado. Es menos exhaustivo que el grid search, pero puede ser m\'as eficiente.

\subsection{Bayesian Optimization}

La optimizaci\'on bayesiana utiliza modelos probabil\'isticos para seleccionar iterativamente los valores de par\'ametros m\'as prometedores.

\section{Implementaci\'on en R}

\subsection{Diagn\'ostico del Modelo}

\begin{verbatim}
# Cargar el paquete necesario
library(car)

# Residuos estudentizados
dataTrain$resid <- rstudent(model)
hist(dataTrain$resid, breaks = 20, main = "Residuos Estudentizados")

# Distancia de Cook
dataTrain$cook <- cooks.distance(model)
plot(dataTrain$cook, type = "h", main = "Distancia de Cook")

# Factor de Inflaci\'on de la Varianza
vif_values <- vif(model)
print(vif_values)
\end{verbatim}

\subsection{Ajuste de Par\'ametros}

\begin{verbatim}
# Grid Search con caret
control <- trainControl(method = "cv", number = 10)
tune_grid <- expand.grid(.alpha = c(0, 0.5, 1), .lambda = seq(0.01, 0.1, by = 0.01))

model_tune <- train(var1 ~ ., data = dataTrain, method = "glmnet", 
                    trControl = control, tuneGrid = tune_grid)

print(model_tune)
\end{verbatim}



\chapter{Interpretaci\'on de los Resultados}

\section{Introducci\'on}

Interpretar correctamente los resultados de un modelo de regresi\'on log\'istica es esencial para tomar decisiones informadas. Este cap\'itulo se centra en la interpretaci\'on de los coeficientes del modelo, las odds ratios, los intervalos de confianza y la significancia estad\'istica.

\section{Coeficientes de Regresi\'on Log\'istica}

Los coeficientes de regresi\'on log\'istica representan la relaci\'on entre las variables independientes y la variable dependiente en t\'erminos de log-odds. 

\subsection{Interpretaci\'on de los Coeficientes}

Cada coeficiente $\beta_j$ en el modelo de regresi\'on log\'istica se interpreta como el cambio en el log-odds de la variable dependiente por unidad de cambio en la variable independiente $X_j$.

\begin{eqnarray*}
\log\left(\frac{p}{1-p}\right) = \beta_0 + \beta_1 X_1 + \beta_2 X_2 + \ldots + \beta_n X_n
\end{eqnarray*}

\subsection{Signo de los Coeficientes}

\begin{itemize}
    \item \textbf{Coeficiente Positivo}: Un coeficiente positivo indica que un aumento en la variable independiente est\'a asociado con un aumento en el log-odds de la variable dependiente.
    \item \textbf{Coeficiente Negativo}: Un coeficiente negativo indica que un aumento en la variable independiente est\'a asociado con una disminuci\'on en el log-odds de la variable dependiente.
\end{itemize}

\section{Odds Ratios}

Las odds ratios proporcionan una interpretaci\'on m\'as intuitiva de los coeficientes de regresi\'on log\'istica. La odds ratio para una variable independiente $X_j$ se calcula como $e^{\beta_j}$.

\subsection{C\'alculo de las Odds Ratios}

\begin{eqnarray*}
\text{OR}_j = e^{\beta_j}
\end{eqnarray*}

\subsection{Interpretaci\'on de las Odds Ratios}

\begin{itemize}
    \item \textbf{OR > 1}: Un OR mayor que 1 indica que un aumento en la variable independiente est\'a asociado con un aumento en las odds de la variable dependiente.
    \item \textbf{OR < 1}: Un OR menor que 1 indica que un aumento en la variable independiente est\'a asociado con una disminuci\'on en las odds de la variable dependiente.
    \item \textbf{OR = 1}: Un OR igual a 1 indica que la variable independiente no tiene efecto sobre las odds de la variable dependiente.
\end{itemize}

\section{Intervalos de Confianza}

Los intervalos de confianza proporcionan una medida de la incertidumbre asociada con los estimadores de los coeficientes. Un intervalo de confianza del 95\% para un coeficiente $\beta_j$ indica que, en el 95\% de las muestras, el intervalo contendr\'a el valor verdadero de $\beta_j$.

\subsection{C\'alculo de los Intervalos de Confianza}

Para calcular un intervalo de confianza del 95\% para un coeficiente $\beta_j$, utilizamos la f\'ormula:
\begin{eqnarray*}
\beta_j \pm 1.96 \cdot \text{SE}(\beta_j)
\end{eqnarray*}
donde $\text{SE}(\beta_j)$ es el error est\'andar de $\beta_j$.

\section{Significancia Estad\'istica}

La significancia estad\'istica se utiliza para determinar si los coeficientes del modelo son significativamente diferentes de cero. Esto se eval\'ua mediante pruebas de hip\'otesis.

\subsection{Prueba de Hip\'otesis}

Para cada coeficiente $\beta_j$, la hip\'otesis nula $H_0$ es que $\beta_j = 0$. La hip\'otesis alternativa $H_a$ es que $\beta_j \neq 0$.

\subsection{P-valor}

El p-valor indica la probabilidad de obtener un coeficiente tan extremo como el observado, asumiendo que la hip\'otesis nula es verdadera. Un p-valor menor que el nivel de significancia $\alpha$ (t\'ipicamente 0.05) indica que podemos rechazar la hip\'otesis nula.

\section{Implementaci\'on en R}

\subsection{C\'alculo de Coeficientes y Odds Ratios}

\begin{verbatim}
# Cargar el paquete necesario
library(broom)

# Entrenar el modelo de regresi\'on log\'istica
model <- glm(var1 ~ ., data = dataTrain, family = "binomial")

# Coeficientes del modelo
coef(model)

# Odds ratios
exp(coef(model))
\end{verbatim}

\subsection{Intervalos de Confianza}

\begin{verbatim}
# Intervalos de confianza para los coeficientes
confint(model)

# Intervalos de confianza para las odds ratios
exp(confint(model))
\end{verbatim}

\subsection{P-valores y Significancia Estad\'istica}

\begin{verbatim}
# Resumen del modelo con p-valores
summary(model)
\end{verbatim}



\chapter{Regresi\'on Logística Multinomial y An\'alisis de Supervivencia}

\section{Introducci\'on}

La regresi\'on log\'istica multinomial y el an\'alisis de supervivencia son extensiones de la regresi\'on log\'istica binaria. Este cap\'itulo se enfoca en las t\'ecnicas y aplicaciones de estos m\'etodos avanzados.

\section{Regresi\'on Log\'istica Multinomial}

La regresi\'on log\'istica multinomial se utiliza cuando la variable dependiente tiene m\'as de dos categor\'ias.

\subsection{Modelo Multinomial}

El modelo de regresi\'on log\'istica multinomial generaliza el modelo binario para manejar m\'ultiples categor\'ias. La probabilidad de que una observaci\'on pertenezca a la categor\'ia $k$ se expresa como:

\begin{eqnarray*}
P(Y = k) = \frac{e^{\beta_{0k} + \beta_{1k} X_1 + \ldots + \beta_{nk} X_n}}{\sum_{j=1}^{K} e^{\beta_{0j} + \beta_{1j} X_1 + \ldots + \beta_{nj} X_n}}
\end{eqnarray*}

\subsection{Estimaci\'on de Par\'ametros}

Los coeficientes del modelo multinomial se estiman utilizando m\'axima verosimilitud, similar a la regresi\'on log\'istica binaria.

\section{An\'alisis de Supervivencia}

El an\'alisis de supervivencia se utiliza para modelar el tiempo hasta que ocurre un evento de inter\'es, como la muerte o la falla de un componente.

\subsection{Funci\'on de Supervivencia}

La funci\'on de supervivencia $S(t)$ describe la probabilidad de que una observaci\'on sobreviva m\'as all\'a del tiempo $t$:

\begin{eqnarray*}
S(t) = P(T > t)
\end{eqnarray*}

\subsection{Modelo de Riesgos Proporcionales de Cox}

El modelo de Cox es un modelo de regresi\'on semiparam\'etrico utilizado para analizar datos de supervivencia:

\begin{eqnarray*}
h(t|X) = h_0(t) e^{\beta_1 X_1 + \ldots + \beta_p X_p}
\end{eqnarray*}
donde $h(t|X)$ es la tasa de riesgo en el tiempo $t$ dado el vector de covariables $X$ y $h_0(t)$ es la tasa de riesgo basal.

\section{Implementaci\'on en R}

\subsection{Regresi\'on Log\'istica Multinomial}

\begin{verbatim}
# Cargar el paquete necesario
library(nnet)

# Entrenar el modelo de regresi\'on log\'istica multinomial
model_multinom <- multinom(var1 ~ ., data = dataTrain)

# Resumen del modelo
summary(model_multinom)
\end{verbatim}

\subsection{An\'alisis de Supervivencia}

\begin{verbatim}
# Cargar el paquete necesario
library(survival)

# Crear el objeto de supervivencia
surv_object <- Surv(time = data$time, event = data$status)

# Ajustar el modelo de Cox
model_cox <- coxph(surv_object ~ var1 + var2, data = data)

# Resumen del modelo
summary(model_cox)
\end{verbatim}



\chapter{Implementaci\'on de Regresi\'on Logística en Datos Reales}
\section{Introducci\'on}

Implementar un modelo de regresi\'on log\'istica en datos reales implica varias etapas, desde la limpieza de datos hasta la evaluaci\'on y validaci\'on del modelo. Este cap\'itulo presenta un ejemplo pr\'actico de la implementaci\'on de un modelo de regresi\'on log\'istica utilizando un conjunto de datos real.

\section{Conjunto de Datos}

Para este ejemplo, utilizaremos un conjunto de datos disponible p\'ublicamente que contiene informaci\'on sobre clientes bancarios. El objetivo es predecir si un cliente suscribir\'a un dep\'osito a plazo fijo.

\section{Preparaci\'on de Datos}

\subsection{Carga y Exploraci\'on de Datos}

Primero, cargamos y exploramos el conjunto de datos para entender su estructura y contenido.

\begin{verbatim}
# Cargar el paquete necesario
library(dplyr)

# Cargar el conjunto de datos
data <- read.csv("bank.csv")

# Explorar los datos
str(data)
summary(data)
\end{verbatim}

\subsection{Limpieza de Datos}

El siguiente paso es limpiar los datos, lo que incluye tratar los valores faltantes y eliminar las duplicidades.

\begin{verbatim}
# Eliminar duplicados
data <- data %>% distinct()

# Imputar valores faltantes (si existen)
data <- data %>% mutate_if(is.numeric, ~ifelse(is.na(.), mean(., na.rm = TRUE), .))
\end{verbatim}

\subsection{Codificaci\'on de Variables Categ\'oricas}

Convertimos las variables categ\'oricas en variables num\'ericas utilizando la codificaci\'on one-hot.

\begin{verbatim}
# Codificaci\'on one-hot de variables categ\'oricas
data <- data %>% mutate(across(where(is.factor), ~ as.numeric(as.factor(.))))
\end{verbatim}

\section{Divisi\'on de Datos}

Dividimos los datos en conjuntos de entrenamiento y prueba.

\begin{verbatim}
# Dividir los datos en conjuntos de entrenamiento y prueba
set.seed(123)
trainIndex <- createDataPartition(data$y, p = .8, list = FALSE, times = 1)
dataTrain <- data[trainIndex,]
dataTest <- data[-trainIndex,]
\end{verbatim}

\section{Entrenamiento del Modelo}

Entrenamos un modelo de regresi\'on log\'istica utilizando el conjunto de entrenamiento.

\begin{verbatim}
# Entrenar el modelo de regresi\'on log\'istica
model <- glm(y ~ ., data = dataTrain, family = "binomial")

# Resumen del modelo
summary(model)
\end{verbatim}

\section{Evaluaci\'on del Modelo}

Evaluamos el rendimiento del modelo utilizando el conjunto de prueba.

\begin{verbatim}
# Predicciones en el conjunto de prueba
predictions <- predict(model, dataTest, type = "response")

# Convertir probabilidades a etiquetas
predicted_labels <- ifelse(predictions > 0.5, 1, 0)

# Matriz de confusi\'on
confusionMatrix(predicted_labels, dataTest$y)
\end{verbatim}

\section{Interpretaci\'on de los Resultados}

Interpretamos los coeficientes del modelo y las odds ratios.

\begin{verbatim}
# Coeficientes del modelo
coef(model)

# Odds ratios
exp(coef(model))
\end{verbatim}



\chapter{Resumen y Proyecto Final}
\section{Resumen de Conceptos Clave}

En este curso, hemos cubierto una variedad de conceptos y t\'ecnicas esenciales para la regresi\'on log\'istica. Los conceptos clave incluyen:

\begin{itemize}
    \item \textbf{Fundamentos de Probabilidad y Estad\'istica}: Comprensi\'on de distribuciones de probabilidad, medidas de tendencia central y dispersi\'on, inferencia estad\'istica y pruebas de hip\'otesis.
    \item \textbf{Regresi\'on Log\'istica}: Modelo de regresi\'on log\'istica binaria y multinomial, interpretaci\'on de coeficientes y odds ratios, m\'etodos de estimaci\'on y validaci\'on.
    \item \textbf{Preparaci\'on de Datos}: Limpieza de datos, tratamiento de valores faltantes, codificaci\'on de variables categ\'oricas y selecci\'on de variables.
    \item \textbf{Evaluaci\'on del Modelo}: Curva ROC, AUC, matriz de confusi\'on, precisi\'on, recall, F1-score y validaci\'on cruzada.
    \item \textbf{Diagn\'ostico del Modelo}: An\'alisis de residuos, influencia, multicolinealidad y ajuste de par\'ametros.
    \item \textbf{An\'alisis de Supervivencia}: Modelos de supervivencia, funci\'on de supervivencia y modelos de riesgos proporcionales de Cox.
\end{itemize}

\section{Buenas Pr\'acticas}

Al implementar modelos de regresi\'on log\'istica, es importante seguir buenas pr\'acticas para garantizar la precisi\'on y la robustez de los modelos. Algunas buenas pr\'acticas incluyen:

\begin{itemize}
    \item \textbf{Exploraci\'on y Preparaci\'on de Datos}: Realizar un an\'alisis exploratorio exhaustivo y preparar los datos adecuadamente antes de construir el modelo.
    \item \textbf{Evaluaci\'on y Validaci\'on del Modelo}: Utilizar m\'etricas adecuadas para evaluar el rendimiento del modelo y validar el modelo utilizando t\'ecnicas como la validaci\'on cruzada.
    \item \textbf{Interpretaci\'on de Resultados}: Interpretar correctamente los coeficientes del modelo y las odds ratios, y comunicar los resultados de manera clara y concisa.
    \item \textbf{Revisi\'on y Ajuste del Modelo}: Diagnosticar problemas en el modelo y ajustar los par\'ametros para mejorar el rendimiento.
\end{itemize}

\section{Proyecto Final}

Para aplicar los conceptos y t\'ecnicas aprendidos en este curso, te proponemos realizar un proyecto final utilizando un conjunto de datos de tu elecci\'on. El proyecto debe incluir las siguientes etapas:

\subsection{Selecci\'on del Conjunto de Datos}

Elige un conjunto de datos relevante que contenga una variable dependiente binaria o multinomial y varias variables independientes.

\subsection{Exploraci\'on y Preparaci\'on de Datos}

Realiza un an\'alisis exploratorio de los datos y prepara los datos para el modelado. Esto incluye la limpieza de datos, el tratamiento de valores faltantes y la codificaci\'on de variables categ\'oricas.

\subsection{Entrenamiento y Evaluaci\'on del Modelo}

Entrena un modelo de regresi\'on log\'istica utilizando el conjunto de datos preparado y eval\'ua su rendimiento utilizando m\'etricas apropiadas.

\subsection{Interpretaci\'on de Resultados}

Interpreta los coeficientes del modelo y las odds ratios, y proporciona una explicaci\'on clara de los resultados.

\subsection{Presentaci\'on del Proyecto}

Presenta tu proyecto en un informe detallado que incluya la descripci\'on del conjunto de datos, los pasos de preparaci\'on y modelado, los resultados del modelo y las conclusiones.




\part{SEGUNDA PARTE}
\chapter{Introducci\'on al An\'alisis de Supervivencia}

\section{Conceptos Básicos}
El análisis de supervivencia es una rama de la estad\'istica que se ocupa del análisis del tiempo que transcurre hasta que ocurre un evento de inter\'es, com\'unmente referido como "tiempo de falla". Este campo es ampliamente utilizado en medicina, biolog\'ia, ingenier\'ia, ciencias sociales, y otros campos.

\section{Definici\'on de Eventos y Tiempos}
En el análisis de supervivencia, un "evento" se refiere a la ocurrencia de un evento espec\'ifico, como la muerte, la falla de un componente, la reca\'ida de una enfermedad, etc. El "tiempo de supervivencia" es el tiempo que transcurre desde un punto de inicio definido hasta la ocurrencia del evento.

\section{Censura}
La censura ocurre cuando la informaci\'on completa sobre el tiempo hasta el evento no está disponible para todos los individuos en el estudio. Hay tres tipos principales de censura:
\begin{itemize}
    \item \textbf{Censura a la derecha:} Ocurre cuando el evento de inter\'es no se ha observado para algunos sujetos antes del final del estudio.
    \item \textbf{Censura a la izquierda:} Ocurre cuando el evento de inter\'es ocurri\'o antes del inicio del periodo de observaci\'on.
    \item \textbf{Censura por intervalo:} Ocurre cuando el evento de inter\'es se sabe que ocurri\'o en un intervalo de tiempo, pero no se conoce el momento exacto.
\end{itemize}

\section{Funci\'on de Supervivencia}
La funci\'on de supervivencia, $S(t)$, se define como la probabilidad de que un individuo sobreviva más allá de un tiempo $t$. Matemáticamente, se expresa como:
\begin{eqnarray*}
S(t) = P(T > t)
\end{eqnarray*}
donde $T$ es una variable aleatoria que representa el tiempo hasta el evento. La funci\'on de supervivencia tiene las siguientes propiedades:
\begin{itemize}
    \item $S(0) = 1$: Esto indica que al inicio (tiempo $t=0$), la probabilidad de haber experimentado el evento es cero, por lo tanto, la supervivencia es del 100%.
    \item $\lim_{t \to \infty} S(t) = 0$: A medida que el tiempo tiende al infinito, la probabilidad de que cualquier individuo a\'un no haya experimentado el evento tiende a cero.
    \item $S(t)$ es una funci\'on no creciente: Esto significa que a medida que el tiempo avanza, la probabilidad de supervivencia no aumenta.
\end{itemize}

\section{Funci\'on de Densidad de Probabilidad}
La funci\'on de densidad de probabilidad $f(t)$ describe la probabilidad de que el evento ocurra en un instante de tiempo espec\'ifico. Se define como:
\begin{eqnarray*}
f(t) = \frac{dF(t)}{dt}
\end{eqnarray*}
donde $F(t)$ es la funci\'on de distribuci\'on acumulada, $F(t) = P(T \leq t)$. La relaci\'on entre $S(t)$ y $f(t)$ es:
\begin{eqnarray*}
f(t) = -\frac{dS(t)}{dt}
\end{eqnarray*}

\section{Funci\'on de Riesgo}
La funci\'on de riesgo, $\lambda(t)$, tambi\'en conocida como funci\'on de tasa de fallas o hazard rate, se define como la tasa instant\'anea de ocurrencia del evento en el tiempo $t$, dado que el individuo ha sobrevivido hasta el tiempo $t$. Matem\'aticamente, se expresa como:
\begin{eqnarray*}
\lambda(t) = \lim_{\Delta t \to 0} \frac{P(t \leq T < t + \Delta t \mid T \geq t)}{\Delta t}
\end{eqnarray*}
Esto se puede reescribir usando $f(t)$ y $S(t)$ como:
\begin{eqnarray*}
\lambda(t) = \frac{f(t)}{S(t)}
\end{eqnarray*}

\section{Relaci\'on entre Funci\'on de Supervivencia y Funci\'on de Riesgo}
La funci\'on de supervivencia y la funci\'on de riesgo est\'an relacionadas a trav\'es de la siguiente ecuaci\'on:
\begin{eqnarray*}
S(t) = \exp\left(-\int_0^t \lambda(u) \, du\right)
\end{eqnarray*}
Esta f\'ormula se deriva del hecho de que la funci\'on de supervivencia es la probabilidad acumulativa de no haber experimentado el evento hasta el tiempo $t$, y $\lambda(t)$ es la tasa instant\'anea de ocurrencia del evento.

La funci\'on de riesgo tambi\'en puede ser expresada como:
\begin{eqnarray*}
\lambda(t) = -\frac{d}{dt} \log S(t)
\end{eqnarray*}

\section{Deducci\'on de la Funci\'on de Supervivencia}
La relaci\'on entre la funci\'on de supervivencia y la funci\'on de riesgo se puede deducir integrando la funci\'on de riesgo:
\begin{eqnarray*}
S(t) &=& \exp\left(-\int_0^t \lambda(u) \, du\right) \\
\log S(t) &=& -\int_0^t \lambda(u) \, du \\
\frac{d}{dt} \log S(t) &=& -\lambda(t) \\
\lambda(t) &=& -\frac{d}{dt} \log S(t)
\end{eqnarray*}

\section{Ejemplo de C\'alculo}
Supongamos que tenemos una muestra de tiempos de supervivencia $T_1, T_2, \ldots, T_n$. Podemos estimar la funci\'on de supervivencia emp\'irica como:
\begin{eqnarray*}
\hat{S}(t) = \frac{\text{N\'umero de individuos que sobreviven m\'as all\'a de } t}{\text{N\'umero total de individuos en riesgo en } t}
\end{eqnarray*}
y la funci\'on de riesgo emp\'irica como:
\begin{eqnarray*}
\hat{\lambda}(t) = \frac{\text{N\'umero de eventos en } t}{\text{N\'umero de individuos en riesgo en } t}
\end{eqnarray*}

\section{Conclusi\'on}
El an\'alisis de supervivencia es una herramienta poderosa para analizar datos de tiempo hasta evento. Entender los conceptos b\'asicos como la funci\'on de supervivencia y la funci\'on de riesgo es fundamental para el an\'alisis m\'as avanzado.


\chapter{Funci\'on de Supervivencia y Funci\'on de Riesgo}
\section{Introducci\'on}
Este cap\'itulo profundiza en la definici\'on y propiedades de la funci\'on de supervivencia y la funci\'on de riesgo, dos conceptos fundamentales en el análisis de supervivencia. Entender estas funciones y su relaci\'on es crucial para modelar y analizar datos de tiempo hasta evento.

\section{Funci\'on de Supervivencia}
La funci\'on de supervivencia, $S(t)$, describe la probabilidad de que un individuo sobreviva más allá de un tiempo $t$. Formalmente, se define como:
\begin{eqnarray*}
S(t) = P(T > t)
\end{eqnarray*}
donde $T$ es una variable aleatoria que representa el tiempo hasta el evento.

\subsection{Propiedades de la Funci\'on de Supervivencia}
La funci\'on de supervivencia tiene varias propiedades importantes:
\begin{itemize}
    \item $S(0) = 1$: Indica que la probabilidad de haber experimentado el evento en el tiempo 0 es cero.
    \item $\lim_{t \to \infty} S(t) = 0$: A medida que el tiempo tiende al infinito, la probabilidad de supervivencia tiende a cero.
    \item $S(t)$ es una funci\'on no creciente: A medida que el tiempo avanza, la probabilidad de supervivencia no aumenta.
\end{itemize}

\subsection{Derivaci\'on de $S(t)$}
Si la funci\'on de densidad de probabilidad $f(t)$ del tiempo de supervivencia $T$ es conocida, la funci\'on de supervivencia puede derivarse como:
\begin{eqnarray*}
S(t) &=& P(T > t) \\
     &=& 1 - P(T \leq t) \\
     &=& 1 - F(t) \\
     &=& 1 - \int_0^t f(u) \, du
\end{eqnarray*}
donde $F(t)$ es la funci\'on de distribuci\'on acumulada.

\subsection{Ejemplo de Cálculo de $S(t)$}
Consideremos un ejemplo donde el tiempo de supervivencia $T$ sigue una distribuci\'on exponencial con tasa $\lambda$. La funci\'on de densidad de probabilidad $f(t)$ es:
\begin{eqnarray*}
f(t) = \lambda e^{-\lambda t}, \quad t \geq 0
\end{eqnarray*}
La funci\'on de distribuci\'on acumulada $F(t)$ es:
\begin{eqnarray*}
F(t) = \int_0^t \lambda e^{-\lambda u} \, du = 1 - e^{-\lambda t}
\end{eqnarray*}
Por lo tanto, la funci\'on de supervivencia $S(t)$ es:
\begin{eqnarray*}
S(t) = 1 - F(t) = e^{-\lambda t}
\end{eqnarray*}

\section{Funci\'on de Riesgo}
La funci\'on de riesgo, $\lambda(t)$, proporciona la tasa instant\'anea de ocurrencia del evento en el tiempo $t$, dado que el individuo ha sobrevivido hasta el tiempo $t$. Matem\'aticamente, se define como:
\begin{eqnarray*}
\lambda(t) = \lim_{\Delta t \to 0} \frac{P(t \leq T < t + \Delta t \mid T \geq t)}{\Delta t}
\end{eqnarray*}

\subsection{Relaci\'on entre $\lambda(t)$ y $f(t)$}
La funci\'on de riesgo se puede relacionar con la funci\'on de densidad de probabilidad $f(t)$ y la funci\'on de supervivencia $S(t)$ de la siguiente manera:
\begin{eqnarray*}
\lambda(t) &=& \frac{f(t)}{S(t)}
\end{eqnarray*}

\subsection{Derivaci\'on de $\lambda(t)$}
La derivaci\'on de $\lambda(t)$ se basa en la definici\'on condicional de la probabilidad:
\begin{eqnarray*}
\lambda(t) &=& \lim_{\Delta t \to 0} \frac{P(t \leq T < t + \Delta t \mid T \geq t)}{\Delta t} \\
           &=& \lim_{\Delta t \to 0} \frac{\frac{P(t \leq T < t + \Delta t \text{ y } T \geq t)}{P(T \geq t)}}{\Delta t} \\
           &=& \lim_{\Delta t \to 0} \frac{\frac{P(t \leq T < t + \Delta t)}{P(T \geq t)}}{\Delta t} \\
           &=& \frac{f(t)}{S(t)}
\end{eqnarray*}

\section{Relaci\'on entre Funci\'on de Supervivencia y Funci\'on de Riesgo}
La funci\'on de supervivencia y la funci\'on de riesgo est\'an estrechamente relacionadas. La relaci\'on se expresa mediante la siguiente ecuaci\'on:
\begin{eqnarray*}
S(t) = \exp\left(-\int_0^t \lambda(u) \, du\right)
\end{eqnarray*}

\subsection{Deducci\'on de la Relaci\'on}
Para deducir esta relaci\'on, consideramos la derivada logar\'itmica de la funci\'on de supervivencia:
\begin{eqnarray*}
S(t) &=& \exp\left(-\int_0^t \lambda(u) \, du\right) \\
\log S(t) &=& -\int_0^t \lambda(u) \, du \\
\frac{d}{dt} \log S(t) &=& -\lambda(t) \\
\lambda(t) &=& -\frac{d}{dt} \log S(t)
\end{eqnarray*}

\section{Interpretaci\'on de la Funci\'on de Riesgo}
La funci\'on de riesgo, $\lambda(t)$, se interpreta como la tasa instant\'anea de ocurrencia del evento por unidad de tiempo, dado que el individuo ha sobrevivido hasta el tiempo $t$. Es una medida local del riesgo de falla en un instante espec\'ifico.

\subsection{Ejemplo de C\'alculo de $\lambda(t)$}
Consideremos nuevamente el caso donde el tiempo de supervivencia $T$ sigue una distribuci\'on exponencial con tasa $\lambda$. La funci\'on de densidad de probabilidad $f(t)$ es:
\begin{eqnarray*}
f(t) = \lambda e^{-\lambda t}
\end{eqnarray*}
La funci\'on de supervivencia $S(t)$ es:
\begin{eqnarray*}
S(t) = e^{-\lambda t}
\end{eqnarray*}
La funci\'on de riesgo $\lambda(t)$ se calcula como:
\begin{eqnarray*}
\lambda(t) &=& \frac{f(t)}{S(t)} \\
           &=& \frac{\lambda e^{-\lambda t}}{e^{-\lambda t}} \\
           &=& \lambda
\end{eqnarray*}
En este caso, $\lambda(t)$ es constante y igual a $\lambda$, lo que es una caracter\'istica de la distribuci\'on exponencial.

\section{Funciones de Riesgo Acumulada y Media Residual}
La funci\'on de riesgo acumulada $H(t)$ se define como:
\begin{eqnarray*}
H(t) = \int_0^t \lambda(u) \, du
\end{eqnarray*}
Esta funci\'on proporciona la suma acumulada de la tasa de riesgo hasta el tiempo $t$.

La funci\'on de vida media residual $e(t)$ se define como la esperanza del tiempo de vida restante dado que el individuo ha sobrevivido hasta el tiempo $t$:
\begin{eqnarray*}
e(t) = \mathbb{E}[T - t \mid T > t] = \int_t^\infty S(u) \, du
\end{eqnarray*}

\section{Ejemplo de C\'alculo de Funci\'on de Riesgo Acumulada y Vida Media Residual}
Consideremos nuevamente la distribuci\'on exponencial con tasa $\lambda$. La funci\'on de riesgo acumulada $H(t)$ es:
\begin{eqnarray*}
H(t) &=& \int_0^t \lambda \, du \\
     &=& \lambda t
\end{eqnarray*}

La funci\'on de vida media residual $e(t)$ es:
\begin{eqnarray*}
e(t) &=& \int_t^\infty e^{-\lambda u} \, du \\
     &=& \left[ \frac{-1}{\lambda} e^{-\lambda u} \right]_t^\infty \\
     &=& \frac{1}{\lambda} e^{-\lambda t} \\
     &=& \frac{1}{\lambda}
\end{eqnarray*}
En este caso, la vida media residual es constante e igual a $\frac{1}{\lambda}$, otra caracter\'istica de la distribuci\'on exponencial.

\section{Conclusi\'on}
La funci\'on de supervivencia y la funci\'on de riesgo son herramientas fundamentales en el an\'alisis de supervivencia. Entender su definici\'on, propiedades, y la relaci\'on entre ellas es esencial para modelar y analizar correctamente los datos de tiempo hasta evento. Las funciones de riesgo acumulada y vida media residual proporcionan informaci\'on adicional sobre la din\'amica del riesgo a lo largo del tiempo.



\chapter{Estimador de Kaplan-Meier}

\section{Introducci\'on}
El estimador de Kaplan-Meier, tambi\'en conocido como la funci\'on de supervivencia emp\'irica, es una herramienta no param\'etrica para estimar la funci\'on de supervivencia a partir de datos censurados. Este m\'etodo es especialmente \'util cuando los tiempos de evento están censurados a la derecha.

\section{Definici\'on del Estimador de Kaplan-Meier}
El estimador de Kaplan-Meier se define como:
\begin{eqnarray*}
\hat{S}(t) = \prod_{t_i \leq t} \left(1 - \frac{d_i}{n_i}\right)
\end{eqnarray*}
donde:
\begin{itemize}
    \item $t_i$ es el tiempo del $i$-\'esimo evento,
    \item $d_i$ es el n\'umero de eventos que ocurren en $t_i$,
    \item $n_i$ es el n\'umero de individuos en riesgo justo antes de $t_i$.
\end{itemize}

\section{Propiedades del Estimador de Kaplan-Meier}
El estimador de Kaplan-Meier tiene las siguientes propiedades:
\begin{itemize}
    \item Es una funci\'on escalonada que disminuye en los tiempos de los eventos observados.
    \item Puede manejar datos censurados a la derecha.
    \item Proporciona una estimaci\'on no param\'etrica de la funci\'on de supervivencia.
\end{itemize}

\subsection{Funci\'on Escalonada}
La funci\'on escalonada del estimador de Kaplan-Meier significa que $\hat{S}(t)$ permanece constante entre los tiempos de los eventos y disminuye en los tiempos de los eventos. Matem\'aticamente, si $t_i$ es el tiempo del $i$-\'esimo evento, entonces:
\begin{eqnarray*}
\hat{S}(t) = \hat{S}(t_i) \quad \text{para} \ t_i \leq t < t_{i+1}
\end{eqnarray*}

\subsection{Manejo de Datos Censurados}
El estimador de Kaplan-Meier maneja datos censurados a la derecha al ajustar la estimaci\'on de la funci\'on de supervivencia s\'olo en los tiempos en que ocurren eventos. Si un individuo es censurado antes de experimentar el evento, no contribuye a la disminuci\'on de $\hat{S}(t)$ en el tiempo de censura. Esto asegura que la censura no sesga la estimaci\'on de la supervivencia.

\subsection{Estimaci\'on No Param\'etrica}
El estimador de Kaplan-Meier es no param\'etrico porque no asume ninguna forma espec\'ifica para la distribuci\'on de los tiempos de supervivencia. En cambio, utiliza la informaci\'on emp\'irica disponible para estimar la funci\'on de supervivencia.

\section{Deducci\'on del Estimador de Kaplan-Meier}
La deducci\'on del estimador de Kaplan-Meier se basa en el principio de probabilidad condicional. Consideremos un conjunto de tiempos de supervivencia observados $t_1, t_2, \ldots, t_k$ con eventos en cada uno de estos tiempos. El estimador de la probabilidad de supervivencia m\'as all\'a del tiempo $t$ es el producto de las probabilidades de sobrevivir m\'as all\'a de cada uno de los tiempos de evento observados hasta $t$.

\subsection{Probabilidad Condicional}
La probabilidad de sobrevivir m\'as all\'a de $t_i$, dado que el individuo ha sobrevivido justo antes de $t_i$, es:
\begin{eqnarray*}
P(T > t_i \mid T \geq t_i) = 1 - \frac{d_i}{n_i}
\end{eqnarray*}
donde $d_i$ es el n\'umero de eventos en $t_i$ y $n_i$ es el n\'umero de individuos en riesgo justo antes de $t_i$.

\subsection{Producto de Probabilidades Condicionales}
La probabilidad de sobrevivir m\'as all\'a de un tiempo $t$ cualquiera, dada la secuencia de tiempos de evento, es el producto de las probabilidades condicionales de sobrevivir m\'as all\'a de cada uno de los tiempos de evento observados hasta $t$. As\'i, el estimador de Kaplan-Meier se obtiene como:
\begin{eqnarray*}
\hat{S}(t) = \prod_{t_i \leq t} \left(1 - \frac{d_i}{n_i}\right)
\end{eqnarray*}

\section{Ejemplo de C\'alculo}
Supongamos que tenemos los siguientes tiempos de supervivencia observados para cinco individuos: 2, 3, 5, 7, 8. Supongamos adem\'as que tenemos censura a la derecha en el tiempo 10. Los tiempos de evento y el n\'umero de individuos en riesgo justo antes de cada evento son:

\begin{table}[h]
\centering
\begin{tabular}{|c|c|c|}
\hline
Tiempo ($t_i$) & Eventos ($d_i$) & En Riesgo ($n_i$) \\
\hline
2 & 1 & 5 \\
3 & 1 & 4 \\
5 & 1 & 3 \\
7 & 1 & 2 \\
8 & 1 & 1 \\
\hline
\end{tabular}
\caption{Ejemplo de c\'alculo del estimador de Kaplan-Meier}
\end{table}

Usando estos datos, el estimador de Kaplan-Meier se calcula como:
\begin{eqnarray*}
\hat{S}(2) &=& 1 - \frac{1}{5} = 0.8 \\
\hat{S}(3) &=& 0.8 \times \left(1 - \frac{1}{4}\right) = 0.8 \times 0.75 = 0.6 \\
\hat{S}(5) &=& 0.6 \times \left(1 - \frac{1}{3}\right) = 0.6 \times 0.6667 = 0.4 \\
\hat{S}(7) &=& 0.4 \times \left(1 - \frac{1}{2}\right) = 0.4 \times 0.5 = 0.2 \\
\hat{S}(8) &=& 0.2 \times \left(1 - \frac{1}{1}\right) = 0.2 \times 0 = 0 \\
\end{eqnarray*}

\section{Intervalos de Confianza para el Estimador de Kaplan-Meier}
Para calcular intervalos de confianza para el estimador de Kaplan-Meier, se puede usar la transformaci\'on logar\'itmica y la aproximaci\'on normal. Un intervalo de confianza aproximado para $\log(-\log(\hat{S}(t)))$ se obtiene como:
\begin{eqnarray*}
\log(-\log(\hat{S}(t))) \pm z_{\alpha/2} \sqrt{\frac{1}{d_i(n_i - d_i)}}
\end{eqnarray*}
donde $z_{\alpha/2}$ es el percentil correspondiente de la distribuci\'on normal est\'andar.

\section{Transformaci\'on Logar\'itmica Inversa}
La transformaci\'on logar\'itmica inversa se utiliza para obtener los l\'imites del intervalo de confianza para $S(t)$:
\begin{eqnarray*}
\hat{S}(t) = \exp\left(-\exp\left(\log(-\log(\hat{S}(t))) \pm z_{\alpha/2} \sqrt{\frac{1}{d_i(n_i - d_i)}}\right)\right)
\end{eqnarray*}

\section{C\'alculo Detallado de Intervalos de Confianza}
Para un c\'alculo m\'as detallado de los intervalos de confianza, consideremos un tiempo espec\'ifico $t_j$. La varianza del estimador de Kaplan-Meier en $t_j$ se puede estimar usando Greenwood's formula:
\begin{eqnarray*}
\text{Var}(\hat{S}(t_j)) = \hat{S}(t_j)^2 \sum_{t_i \leq t_j} \frac{d_i}{n_i(n_i - d_i)}
\end{eqnarray*}
El intervalo de confianza aproximado para $\hat{S}(t_j)$ es entonces:
\begin{eqnarray*}
\hat{S}(t_j) \pm z_{\alpha/2} \sqrt{\text{Var}(\hat{S}(t_j))}
\end{eqnarray*}

\section{Ejemplo de Intervalo de Confianza}
Supongamos que en el ejemplo anterior queremos calcular el intervalo de confianza para $\hat{S}(3)$. Primero, calculamos la varianza:
\begin{eqnarray*}
\text{Var}(\hat{S}(3)) &=& \hat{S}(3)^2 \left( \frac{1}{5 \times 4} + \frac{1}{4 \times 3} \right) \\
                       &=& 0.6^2 \left( \frac{1}{20} + \frac{1}{12} \right) \\
                       &=& 0.36 \left( 0.05 + 0.0833 \right) \\
                       &=& 0.36 \times 0.1333 \\
                       &=& 0.048
\end{eqnarray*}
El intervalo de confianza es entonces:
\begin{eqnarray*}
0.6 \pm 1.96 \sqrt{0.048} = 0.6 \pm 1.96 \times 0.219 = 0.6 \pm 0.429
\end{eqnarray*}
Por lo tanto, el intervalo de confianza para $\hat{S}(3)$ es aproximadamente $(0.171, 1.029)$. Dado que una probabilidad no puede exceder 1, ajustamos el intervalo a $(0.171, 1.0)$.

\section{Interpretaci\'on del Estimador de Kaplan-Meier}
El estimador de Kaplan-Meier proporciona una estimaci\'on emp\'irica de la funci\'on de supervivencia que es f\'acil de interpretar y calcular. Su capacidad para manejar datos censurados lo hace especialmente \'util en estudios de supervivencia.

\section{Conclusi\'on}
El estimador de Kaplan-Meier es una herramienta poderosa para estimar la funci\'on de supervivencia en presencia de datos censurados. Su c\'alculo es relativamente sencillo y proporciona una estimaci\'on no param\'etrica robusta de la supervivencia a lo largo del tiempo. La interpretaci\'on adecuada de este estimador y su intervalo de confianza asociado es fundamental para el an\'alisis de datos de supervivencia.



\chapter{Comparaci\'on de Curvas de Supervivencia}

\section{Introducci\'on}
Comparar curvas de supervivencia es crucial para determinar si existen diferencias significativas en las tasas de supervivencia entre diferentes grupos. Las pruebas de hip\'otesis, como el test de log-rank, son herramientas comunes para esta comparaci\'on.

\section{Test de Log-rank}
El test de log-rank se utiliza para comparar las curvas de supervivencia de dos o más grupos. La hip\'otesis nula es que no hay diferencia en las funciones de riesgo entre los grupos.

\subsection{F\'ormula del Test de Log-rank}
El estad\'istico del test de log-rank se define como:
\begin{eqnarray*}
\chi^2 = \frac{\left(\sum_{i=1}^k (O_i - E_i)\right)^2}{\sum_{i=1}^k V_i}
\end{eqnarray*}
donde:
\begin{itemize}
    \item $O_i$ es el n\'umero observado de eventos en el grupo $i$.
    \item $E_i$ es el n\'umero esperado de eventos en el grupo $i$.
    \item $V_i$ es la varianza del n\'umero de eventos en el grupo $i$.
\end{itemize}

\subsection{Cálculo de $E_i$ y $V_i$}
El n\'umero esperado de eventos $E_i$ y la varianza $V_i$ se calculan como:
\begin{eqnarray*}
E_i &=& \frac{d_i \cdot n_i}{n} \\
V_i &=& \frac{d_i \cdot (n - d_i) \cdot n_i \cdot (n - n_i)}{n^2 \cdot (n - 1)}
\end{eqnarray*}
donde:
\begin{itemize}
    \item $d_i$ es el n\'umero total de eventos en el grupo $i$.
    \item $n_i$ es el n\'umero de individuos en riesgo en el grupo $i$.
    \item $n$ es el n\'umero total de individuos en todos los grupos.
\end{itemize}

\section{Ejemplo de C\'alculo del Test de Log-rank}
Supongamos que tenemos dos grupos con los siguientes datos de eventos:

\begin{table}[h]
\centering
\begin{tabular}{|c|c|c|c|}
\hline
Grupo & Tiempo ($t_i$) & Eventos ($O_i$) & En Riesgo ($n_i$) \\
\hline
1 & 2 & 1 & 5 \\
1 & 4 & 1 & 4 \\
2 & 3 & 1 & 4 \\
2 & 5 & 1 & 3 \\
\hline
\end{tabular}
\caption{Ejemplo de datos para el test de log-rank}
\end{table}

Calculemos $E_i$ y $V_i$ para cada grupo:

\begin{eqnarray*}
E_1 &=& \frac{2 \cdot 5}{9} + \frac{2 \cdot 4}{8} = \frac{10}{9} + \frac{8}{8} = 1.11 + 1 = 2.11 \\
V_1 &=& \frac{2 \cdot 7 \cdot 5 \cdot 4}{81 \cdot 8} = \frac{2 \cdot 7 \cdot 5 \cdot 4}{648} = \frac{280}{648} = 0.432 \\
E_2 &=& \frac{2 \cdot 4}{9} + \frac{2 \cdot 3}{8} = \frac{8}{9} + \frac{6}{8} = 0.89 + 0.75 = 1.64 \\
V_2 &=& \frac{2 \cdot 7 \cdot 4 \cdot 4}{81 \cdot 8} = \frac{2 \cdot 7 \cdot 4 \cdot 4}{648} = \frac{224}{648} = 0.346 \\
\end{eqnarray*}

El estad\'istico de log-rank se calcula como:
\begin{eqnarray*}
\chi^2 &=& \frac{\left((1 - 2.11) + (1 - 1.64)\right)^2}{0.432 + 0.346} \\
       &=& \frac{\left(-1.11 - 0.64\right)^2}{0.778} \\
       &=& \frac{3.04}{0.778} \\
       &=& 3.91
\end{eqnarray*}

El valor p se puede obtener comparando $\chi^2$ con una distribuci\'on $\chi^2$ con un grado de libertad (dado que estamos comparando dos grupos).

\section{Interpretaci\'on del Test de Log-rank}
Un valor p peque\~no (generalmente menos de 0.05) indica que hay una diferencia significativa en las curvas de supervivencia entre los grupos. Un valor p grande sugiere que no hay suficiente evidencia para rechazar la hip\'otesis nula de que las curvas de supervivencia son iguales.

\section{Pruebas Alternativas}
Adem\'as del test de log-rank, existen otras pruebas para comparar curvas de supervivencia, como el test de Wilcoxon (Breslow), que da m\'as peso a los eventos en tiempos tempranos.

\section{Conclusi\'on}
El test de log-rank es una herramienta esencial para comparar curvas de supervivencia entre diferentes grupos. Su c\'alculo se basa en la diferencia entre los eventos observados y esperados en cada grupo, y su interpretaci\'on puede ayudar a identificar diferencias significativas en la supervivencia.



\chapter{Modelos de Riesgos Proporcionales de Cox}

\section{Introducci\'on}
El modelo de riesgos proporcionales de Cox, propuesto por David Cox en 1972, es una de las herramientas más utilizadas en el análisis de supervivencia. Este modelo permite evaluar el efecto de varias covariables en el tiempo hasta el evento, sin asumir una forma espec\'ifica para la distribuci\'on de los tiempos de supervivencia.

\section{Definici\'on del Modelo de Cox}
El modelo de Cox se define como:
\begin{eqnarray*}
\lambda(t \mid X) = \lambda_0(t) \exp(\beta^T X)
\end{eqnarray*}
donde:
\begin{itemize}
    \item $\lambda(t \mid X)$ es la funci\'on de riesgo en el tiempo $t$ dado el vector de covariables $X$.
    \item $\lambda_0(t)$ es la funci\'on de riesgo basal en el tiempo $t$.
    \item $\beta$ es el vector de coeficientes del modelo.
    \item $X$ es el vector de covariables.
\end{itemize}

\section{Supuesto de Proporcionalidad de Riesgos}
El modelo de Cox asume que las razones de riesgo entre dos individuos son constantes a lo largo del tiempo. Matemáticamente, si $X_i$ y $X_j$ son las covariables de dos individuos, la raz\'on de riesgos se expresa como:
\begin{eqnarray*}
\frac{\lambda(t \mid X_i)}{\lambda(t \mid X_j)} = \frac{\lambda_0(t) \exp(\beta^T X_i)}{\lambda_0(t) \exp(\beta^T X_j)} = \exp(\beta^T (X_i - X_j))
\end{eqnarray*}

\section{Estimaci\'on de los Par\'ametros}
Los par\'ametros $\beta$ se estiman utilizando el m\'etodo de m\'axima verosimilitud parcial. La funci\'on de verosimilitud parcial se define como:
\begin{eqnarray*}
L(\beta) = \prod_{i=1}^k \frac{\exp(\beta^T X_i)}{\sum_{j \in R(t_i)} \exp(\beta^T X_j)}
\end{eqnarray*}
donde $R(t_i)$ es el conjunto de individuos en riesgo en el tiempo $t_i$.

\subsection{Funci\'on de Log-Verosimilitud Parcial}
La funci\'on de log-verosimilitud parcial es:
\begin{eqnarray*}
\log L(\beta) = \sum_{i=1}^k \left(\beta^T X_i - \log \sum_{j \in R(t_i)} \exp(\beta^T X_j)\right)
\end{eqnarray*}

\subsection{Derivadas Parciales y Maximizaci\'on}
Para encontrar los estimadores de m\'axima verosimilitud, resolvemos el sistema de ecuaciones obtenido al igualar a cero las derivadas parciales de $\log L(\beta)$ con respecto a $\beta$:
\begin{eqnarray*}
\frac{\partial \log L(\beta)}{\partial \beta} = \sum_{i=1}^k \left(X_i - \frac{\sum_{j \in R(t_i)} X_j \exp(\beta^T X_j)}{\sum_{j \in R(t_i)} \exp(\beta^T X_j)}\right) = 0
\end{eqnarray*}

\section{Interpretaci\'on de los Coeficientes}
Cada coeficiente $\beta_i$ representa el logaritmo de la raz\'on de riesgos asociado con un incremento unitario en la covariable $X_i$. Un valor positivo de $\beta_i$ indica que un aumento en $X_i$ incrementa el riesgo del evento, mientras que un valor negativo indica una reducci\'on del riesgo.

\section{Evaluaci\'on del Modelo}
El modelo de Cox se eval\'ua utilizando varias t\'ecnicas, como el an\'alisis de residuos de Schoenfeld para verificar el supuesto de proporcionalidad de riesgos, y el uso de curvas de supervivencia estimadas para evaluar la bondad de ajuste.

\subsection{Residuos de Schoenfeld}
Los residuos de Schoenfeld se utilizan para evaluar la proporcionalidad de riesgos. Para cada evento en el tiempo $t_i$, el residuo de Schoenfeld para la covariable $X_j$ se define como:
\begin{eqnarray*}
r_{ij} = X_{ij} - \hat{X}_{ij}
\end{eqnarray*}
donde $\hat{X}_{ij}$ es la covariable ajustada.

\subsection{Curvas de Supervivencia Ajustadas}
Las curvas de supervivencia ajustadas se obtienen utilizando la funci\'on de riesgo basal estimada y los coeficientes del modelo. La funci\'on de supervivencia ajustada se define como:
\begin{eqnarray*}
\hat{S}(t \mid X) = \hat{S}_0(t)^{\exp(\beta^T X)}
\end{eqnarray*}
donde $\hat{S}_0(t)$ es la funci\'on de supervivencia basal estimada.

\section{Ejemplo de Aplicaci\'on del Modelo de Cox}
Consideremos un ejemplo con tres covariables: edad, sexo y tratamiento. Supongamos que los datos se ajustan a un modelo de Cox y obtenemos los siguientes coeficientes:
\begin{eqnarray*}
\hat{\beta}_{edad} = 0.02, \quad \hat{\beta}_{sexo} = -0.5, \quad \hat{\beta}_{tratamiento} = 1.2
\end{eqnarray*}

La funci\'on de riesgo ajustada se expresa como:
\begin{eqnarray*}
\lambda(t \mid X) = \lambda_0(t) \exp(0.02 \cdot \text{edad} - 0.5 \cdot \text{sexo} + 1.2 \cdot \text{tratamiento})
\end{eqnarray*}

\section{Conclusi\'on}
El modelo de riesgos proporcionales de Cox es una herramienta poderosa para analizar datos de supervivencia con m\'ultiples covariables. Su flexibilidad y la falta de suposiciones fuertes sobre la distribuci\'on de los tiempos de supervivencia lo hacen ampliamente aplicable en diversas disciplinas.



\chapter{Diagn\'ostico y Validaci\'on de Modelos de Cox}

\section{Introducci\'on}
Una vez ajustado un modelo de Cox, es crucial realizar diagn\'osticos y validaciones para asegurar que el modelo es apropiado y que los supuestos subyacentes son válidos. Esto incluye la verificaci\'on del supuesto de proporcionalidad de riesgos y la evaluaci\'on del ajuste del modelo.

\section{Supuesto de Proporcionalidad de Riesgos}
El supuesto de proporcionalidad de riesgos implica que la raz\'on de riesgos entre dos individuos es constante a lo largo del tiempo. Si este supuesto no se cumple, las inferencias hechas a partir del modelo pueden ser incorrectas.

\subsection{Residuos de Schoenfeld}
Los residuos de Schoenfeld se utilizan para evaluar la proporcionalidad de riesgos. Para cada evento en el tiempo $t_i$, el residuo de Schoenfeld para la covariable $X_j$ se define como:
\begin{eqnarray*}
r_{ij} = X_{ij} - \hat{X}_{ij}
\end{eqnarray*}
donde $\hat{X}_{ij}$ es la covariable ajustada. Si los residuos de Schoenfeld no muestran una tendencia sistemática cuando se trazan contra el tiempo, el supuesto de proporcionalidad de riesgos es razonable.

\section{Bondad de Ajuste}
La bondad de ajuste del modelo de Cox se eval\'ua comparando las curvas de supervivencia observadas y ajustadas, y utilizando estad\'isticas de ajuste global.

\subsection{Curvas de Supervivencia Ajustadas}
Las curvas de supervivencia aaustadas se obtienen utilizando la funci\'on de riesgo basal estimada y los coeficientes del modelo. La funci\'on de supervivencia ajustada se define como:
\begin{eqnarray*}
\hat{S}(t \mid X) = \hat{S}_0(t)^{\exp(\beta^T X)}
\end{eqnarray*}
donde $\hat{S}_0(t)$ es la funci\'on de supervivencia basal estimada. Comparar estas curvas con las curvas de Kaplan-Meier para diferentes niveles de las covariables puede proporcionar una validaci\'on visual del ajuste del modelo.

\subsection{Estad\'isticas de Ajuste Global}
Las estad\'isticas de ajuste global, como el test de la desviaci\'on y el test de la bondad de ajuste de Grambsch y Therneau, se utilizan para evaluar el ajuste global del modelo de Cox.

\section{Diagn\'ostico de Influencia}
El diagn\'ostico de influencia identifica observaciones individuales que tienen un gran impacto en los estimados del modelo. Los residuos de devianza y los residuos de martingala se utilizan com\'unmente para este prop\'osito.

\subsection{Residuos de Deviance}
Los residuos de deviance se definen como:
\begin{eqnarray*}
D_i = \text{sign}(O_i - E_i) \sqrt{-2 \left(O_i \log \frac{O_i}{E_i} - (O_i - E_i)\right)}
\end{eqnarray*}
donde $O_i$ es el n\'umero observado de eventos y $E_i$ es el n\'umero esperado de eventos. Observaciones con residuos de deviance grandes en valor absoluto pueden ser influyentes.

\subsection{Residuos de Martingala}
Los residuos de martingala se definen como:
\begin{eqnarray*}
M_i = O_i - E_i
\end{eqnarray*}
donde $O_i$ es el n\'umero observado de eventos y $E_i$ es el n\'umero esperado de eventos. Los residuos de martingala se utilizan para detectar observaciones que no se ajustan bien al modelo.

\section{Ejemplo de Diagn\'ostico}
Consideremos un modelo de Cox ajustado con las covariables edad, sexo y tratamiento. Para diagnosticar la influencia de observaciones individuales, calculamos los residuos de deviance y martingala para cada observaci\'on.

\begin{table}[h]
\centering
\begin{tabular}{|c|c|c|c|c|}
\hline
Observaci\'on & Edad & Sexo & Tratamiento & Residuo de Deviance \\
\hline
1 & 50 & 0 & 1 & 1.2 \\
2 & 60 & 1 & 0 & -0.5 \\
3 & 45 & 0 & 1 & -1.8 \\
4 & 70 & 1 & 0 & 0.3 \\
\hline
\end{tabular}
\caption{Residuos de deviance para observaciones individuales}
\end{table}

Observaciones con residuos de deviance grandes en valor absoluto (como la observaci\'on 3) pueden ser influyentes y requieren una revisi\'on adicional.

\section{Conclusi\'on}
El diagn\'ostico y la validaci\'on son pasos cr\'iticos en el an\'slisis de modelos de Cox. Evaluar el supuesto de proporcionalidad de riesgos, la bondad de ajuste y la influencia de observaciones individuales asegura que las inferencias y conclusiones derivadas del modelo sean v\'slidas y fiables.



\chapter{Modelos Acelerados de Fallos}
\section{Introducci\'on}
Los modelos acelerados de fallos (AFT) son una alternativa a los modelos de riesgos proporcionales de Cox. En lugar de asumir que las covariables afectan la tasa de riesgo, los modelos AFT asumen que las covariables multiplican el tiempo de supervivencia por una constante.

\section{Definici\'on del Modelo AFT}
Un modelo AFT se expresa como:
\begin{eqnarray*}
T = T_0 \exp(\beta^T X)
\end{eqnarray*}
donde:
\begin{itemize}
    \item $T$ es el tiempo de supervivencia observado.
    \item $T_0$ es el tiempo de supervivencia bajo condiciones basales.
    \item $\beta$ es el vector de coeficientes del modelo.
    \item $X$ es el vector de covariables.
\end{itemize}

\subsection{Transformaci\'on Logar\'itmica}
El modelo AFT se puede transformar logar\'itmicamente para obtener una forma lineal:
\begin{eqnarray*}
\log(T) = \log(T_0) + \beta^T X
\end{eqnarray*}

\section{Estimaci\'on de los Parámetros}
Los parámetros del modelo AFT se estiman utilizando el m\'etodo de máxima verosimilitud. La funci\'on de verosimilitud se define como:
\begin{eqnarray*}
L(\beta) = \prod_{i=1}^n f(t_i \mid X_i; \beta)
\end{eqnarray*}
donde $f(t_i \mid X_i; \beta)$ es la funci\'on de densidad de probabilidad del tiempo de supervivencia $t_i$ dado el vector de covariables $X_i$ y los par\'ametros $\beta$.

\subsection{Funci\'on de Log-Verosimilitud}
La funci\'on de log-verosimilitud es:
\begin{eqnarray*}
\log L(\beta) = \sum_{i=1}^n \log f(t_i \mid X_i; \beta)
\end{eqnarray*}

\subsection{Maximizaci\'on de la Verosimilitud}
Los estimadores de m\'axima verosimilitud se obtienen resolviendo el sistema de ecuaciones obtenido al igualar a cero las derivadas parciales de $\log L(\beta)$ con respecto a $\beta$:
\begin{eqnarray*}
\frac{\partial \log L(\beta)}{\partial \beta} = 0
\end{eqnarray*}

\section{Distribuciones Comunes en Modelos AFT}
En los modelos AFT, el tiempo de supervivencia $T$ puede seguir varias distribuciones comunes, como la exponencial, Weibull, log-normal y log-log\'istica. Cada una de estas distribuciones tiene diferentes propiedades y aplicaciones.

\subsection{Modelo Exponencial AFT}
En un modelo exponencial AFT, el tiempo de supervivencia $T$ sigue una distribuci\'on exponencial con par\'ametro $\lambda$:
\begin{eqnarray*}
f(t) = \lambda \exp(-\lambda t)
\end{eqnarray*}
La funci\'on de supervivencia es:
\begin{eqnarray*}
S(t) = \exp(-\lambda t)
\end{eqnarray*}
La transformaci\'on logar\'itmica del tiempo de supervivencia es:
\begin{eqnarray*}
\log(T) = \log\left(\frac{1}{\lambda}\right) + \beta^T X
\end{eqnarray*}

\subsection{Modelo Weibull AFT}
En un modelo Weibull AFT, el tiempo de supervivencia $T$ sigue una distribuci\'on Weibull con par\'ametros $\lambda$ y $k$:
\begin{eqnarray*}
f(t) = \lambda k t^{k-1} \exp(-\lambda t^k)
\end{eqnarray*}
La funci\'on de supervivencia es:
\begin{eqnarray*}
S(t) = \exp(-\lambda t^k)
\end{eqnarray*}
La transformaci\'on logar\'itmica del tiempo de supervivencia es:
\begin{eqnarray*}
\log(T) = \log\left(\left(\frac{1}{\lambda}\right)^{1/k}\right) + \frac{\beta^T X}{k}
\end{eqnarray*}

\section{Interpretaci\'on de los Coeficientes}
En los modelos AFT, los coeficientes $\beta_i$ se interpretan como factores multiplicativos del tiempo de supervivencia. Un valor positivo de $\beta_i$ indica que un aumento en la covariable $X_i$ incrementa el tiempo de supervivencia, mientras que un valor negativo indica una reducci\'on del tiempo de supervivencia.

\section{Ejemplo de Aplicaci\'on del Modelo AFT}
Consideremos un ejemplo con tres covariables: edad, sexo y tratamiento. Supongamos que los datos se ajustan a un modelo Weibull AFT y obtenemos los siguientes coeficientes:
\begin{eqnarray*}
\hat{\beta}_{edad} = -0.02, \quad \hat{\beta}_{sexo} = 0.5, \quad \hat{\beta}_{tratamiento} = -1.2
\end{eqnarray*}

La funci\'on de supervivencia ajustada se expresa como:
\begin{eqnarray*}
S(t \mid X) = \exp\left(-\left(\frac{t \exp(-0.02 \cdot \text{edad} + 0.5 \cdot \text{sexo} - 1.2 \cdot \text{tratamiento})}{\lambda}\right)^k\right)
\end{eqnarray*}

\section{Conclusi\'on}
Los modelos AFT proporcionan una alternativa flexible a los modelos de riesgos proporcionales de Cox. Su enfoque en la multiplicaci\'on del tiempo de supervivencia por una constante permite una interpretaci\'on intuitiva y aplicaciones en diversas \'areas.



\chapter{An\'alisis Multivariado de Supervivencia}

\section{Introducci\'on}
El análisis multivariado de supervivencia extiende los modelos de supervivencia para incluir m\'ultiples covariables, permitiendo evaluar su efecto simultáneo sobre el tiempo hasta el evento. Los modelos de Cox y AFT son com\'unmente utilizados en este contexto.

\section{Modelo de Cox Multivariado}
El modelo de Cox multivariado se define como:
\begin{eqnarray*}
\lambda(t \mid X) = \lambda_0(t) \exp(\beta^T X)
\end{eqnarray*}
donde $X$ es un vector de covariables.

\subsection{Estimaci\'on de los Parámetros}
Los parámetros $\beta$ se estiman utilizando el m\'etodo de máxima verosimilitud parcial, como se discuti\'o anteriormente. La funci\'on de verosimilitud parcial se maximiza para obtener los estimadores de los coeficientes.

\section{Modelo AFT Multivariado}
El modelo AFT multivariado se expresa como:
\begin{eqnarray*}
T = T_0 \exp(\beta^T X)
\end{eqnarray*}

\subsection{Estimaci\'on de los Par\'ametros}
Los par\'ametros $\beta$ se estiman utilizando el m\'etodo de m\'axima verosimilitud, similar al caso univariado. La funci\'on de verosimilitud se maximiza para obtener los estimadores de los coeficientes.

\section{Interacci\'on y Efectos No Lineales}
En el an\'alisis multivariado, es importante considerar la posibilidad de interacciones entre covariables y efectos no lineales. Estos se pueden incluir en los modelos extendiendo las funciones de riesgo o supervivencia.

\subsection{Interacciones}
Las interacciones entre covariables se pueden modelar a\~nadiendo t\'erminos de interacci\'on en el modelo:
\begin{eqnarray*}
\lambda(t \mid X) = \lambda_0(t) \exp(\beta_1 X_1 + \beta_2 X_2 + \beta_3 X_1 X_2)
\end{eqnarray*}
donde $X_1 X_2$ es el t\'ermino de interacci\'on.

\subsection{Efectos No Lineales}
Los efectos no lineales se pueden modelar utilizando funciones no lineales de las covariables, como polinomios o splines:
\begin{eqnarray*}
\lambda(t \mid X) = \lambda_0(t) \exp(\beta_1 X + \beta_2 X^2)
\end{eqnarray*}

\section{Selecci\'on de Variables}
La selecci\'on de variables es crucial en el an\'alisis multivariado para evitar el sobreajuste y mejorar la interpretabilidad del modelo. M\'etodos como la regresi\'on hacia atr\'as, la regresi\'on hacia adelante y la selecci\'on por criterios de informaci\'on (AIC, BIC) son com\'unmente utilizados.

\subsection{Regresi\'on Hacia Atr\'as}
La regresi\'on hacia atr\'as comienza con todas las covariables en el modelo y elimina iterativamente la covariable menos significativa hasta que todas las covariables restantes sean significativas.

\subsection{Regresi\'on Hacia Adelante}
La regresi\'on hacia adelante comienza con un modelo vac\'io y a\~nade iterativamente la covariable m\'as significativa hasta que no se pueda a\~nadir ninguna covariable adicional significativa.

\subsection{Criterios de Informaci\'on}
Los criterios de informaci\'on, como el AIC (Akaike Information Criterion) y el BIC (Bayesian Information Criterion), se utilizan para seleccionar el modelo que mejor se ajusta a los datos con la menor complejidad posible:
\begin{eqnarray*}
AIC &=& -2 \log L + 2k \\
BIC &=& -2 \log L + k \log n
\end{eqnarray*}
donde $L$ es la funci\'on de verosimilitud del modelo, $k$ es el n\'umero de par\'ametros en el modelo y $n$ es el tama\~no de la muestra.

\section{Ejemplo de An\'alisis Multivariado}
Consideremos un ejemplo con tres covariables: edad, sexo y tratamiento. Ajustamos un modelo de Cox multivariado y obtenemos los siguientes coeficientes:
\begin{eqnarray*}
\hat{\beta}_{edad} = 0.03, \quad \hat{\beta}_{sexo} = -0.6, \quad \hat{\beta}_{tratamiento} = 1.5
\end{eqnarray*}

La funci\'on de riesgo ajustada se expresa como:
\begin{eqnarray*}
\lambda(t \mid X) = \lambda_0(t) \exp(0.03 \cdot \text{edad} - 0.6 \cdot \text{sexo} + 1.5 \cdot \text{tratamiento})
\end{eqnarray*}

\section{Conclusi\'on}
El an\'alisis multivariado de supervivencia permite evaluar el efecto conjunto de m\'ultiples covariables sobre el tiempo hasta el evento. La inclusi\'on de interacciones y efectos no lineales, junto con la selecci\'on adecuada de variables, mejora la precisi\'on y la interpretabilidad de los modelos de supervivencia.



\chapter{Supervivencia en Datos Complicados}

\section{Introducci\'on}
El análisis de supervivencia en datos complicados se refiere a la evaluaci\'on de datos de supervivencia que presentan desaf\'ios adicionales, como la censura por intervalo, datos truncados y datos con m\'ultiples tipos de eventos. Estos escenarios requieren m\'etodos avanzados para un análisis adecuado.

\section{Censura por Intervalo}
La censura por intervalo ocurre cuando el evento de inter\'es se sabe que ocurri\'o dentro de un intervalo de tiempo, pero no se conoce el momento exacto. Esto es com\'un en estudios donde las observaciones se realizan en puntos de tiempo discretos.

\subsection{Modelo para Datos Censurados por Intervalo}
Para datos censurados por intervalo, la funci\'on de verosimilitud se modifica para incluir la probabilidad de que el evento ocurra dentro de un intervalo:
\begin{eqnarray*}
L(\beta) = \prod_{i=1}^n P(T_i \in [L_i, U_i] \mid X_i; \beta)
\end{eqnarray*}
donde $[L_i, U_i]$ es el intervalo de tiempo durante el cual se sabe que ocurri\'o el evento para el individuo $i$.

\section{Datos Truncados}
Los datos truncados ocurren cuando los tiempos de supervivencia est\'an sujetos a un umbral, y solo se observan los individuos cuyos tiempos de supervivencia superan (o est\'an por debajo de) ese umbral. Existen dos tipos principales de truncamiento: truncamiento a la izquierda y truncamiento a la derecha.

\subsection{Modelo para Datos Truncados}
Para datos truncados a la izquierda, la funci\'on de verosimilitud se ajusta para considerar solo los individuos que superan el umbral de truncamiento:
\begin{eqnarray*}
L(\beta) = \prod_{i=1}^n \frac{f(t_i \mid X_i; \beta)}{1 - F(L_i \mid X_i; \beta)}
\end{eqnarray*}
donde $L_i$ es el umbral de truncamiento para el individuo $i$.

\section{An\'alisis de Competing Risks}
En estudios donde pueden ocurrir m\'ultiples tipos de eventos (competing risks), es crucial modelar adecuadamente el riesgo asociado con cada tipo de evento. La probabilidad de ocurrencia de cada evento compite con las probabilidades de ocurrencia de otros eventos.

\subsection{Modelo de Competing Risks}
Para un an\'alisis de competing risks, la funci\'on de riesgo se descompone en funciones de riesgo espec\'ificas para cada tipo de evento:
\begin{eqnarray*}
\lambda(t) = \sum_{j=1}^m \lambda_j(t)
\end{eqnarray*}
donde $\lambda_j(t)$ es la funci\'on de riesgo para el evento $j$.

\section{M\'etodos de Imputaci\'on}
Los m\'etodos de imputaci\'on se utilizan para manejar datos faltantes o censurados en estudios de supervivencia. La imputaci\'on m\'ultiple es un enfoque com\'un que crea m\'ultiples conjuntos de datos completos imputando valores faltantes varias veces y luego combina los resultados.

\subsection{Imputaci\'on M\'ultiple}
La imputaci\'on m\'ultiple para datos de supervivencia se realiza en tres pasos:
\begin{enumerate}
    \item Imputar los valores faltantes m\'ultiples veces para crear varios conjuntos de datos completos.
    \item Analizar cada conjunto de datos completo por separado utilizando m\'etodos de supervivencia est\'andar.
    \item Combinar los resultados de los an\'alisis separados para obtener estimaciones y varianzas combinadas.
\end{enumerate}

\section{Ejemplo de An\'alisis con Datos Complicados}
Consideremos un estudio con datos censurados por intervalo y competing risks. Ajustamos un modelo para los datos censurados por intervalo y obtenemos los siguientes coeficientes para las covariables edad y tratamiento:
\begin{eqnarray*}
\hat{\beta}_{edad} = 0.04, \quad \hat{\beta}_{tratamiento} = -0.8
\end{eqnarray*}

La funci\'on de supervivencia ajustada se expresa como:
\begin{eqnarray*}
S(t \mid X) = \exp\left(-\left(\frac{t \exp(0.04 \cdot \text{edad} - 0.8 \cdot \text{tratamiento})}{\lambda}\right)^k\right)
\end{eqnarray*}

\section{Conclusi\'on}
El an\'alisis de supervivencia en datos complicados requiere m\'etodos avanzados para manejar censura por intervalo, datos truncados y competing risks. La aplicaci\'on de modelos adecuados y m\'etodos de imputaci\'on asegura un an\'alisis preciso y completo de estos datos complejos.



\chapter{Proyecto Final y Revisi\'on}

\section{Introducci\'on}
El proyecto final proporciona una oportunidad para aplicar los conceptos y t\'ecnicas aprendidas en el curso de análisis de supervivencia. Este cap\'itulo incluye una gu\'ia para desarrollar un proyecto de análisis de supervivencia y una revisi\'on de los conceptos clave.

\section{Desarrollo del Proyecto}
El proyecto final debe incluir los siguientes componentes:
\begin{enumerate}
    \item Definici\'on del problema: Identificar la pregunta de investigaci\'on y los objetivos del análisis de supervivencia.
    \item Descripci\'on de los datos: Presentar los datos utilizados, incluyendo las covariables y la estructura de los datos.
    \item Análisis exploratorio: Realizar un análisis descriptivo de los datos, incluyendo la censura y la distribuci\'on de los tiempos de supervivencia.
    \item Ajuste del modelo: Ajustar modelos de supervivencia adecuados (Kaplan-Meier, Cox, AFT) y evaluar su bondad de ajuste.
    \item Diagn\'ostico del modelo: Realizar diagn\'osticos para evaluar los supuestos del modelo y la influencia de observaciones individuales.
    \item Interpretaci\'on de resultados: Interpretar los coeficientes del modelo y las curvas de supervivencia ajustadas.
    \item Conclusiones: Resumir los hallazgos del análisis y proporcionar recomendaciones basadas en los resultados.
\end{enumerate}

\section{Revisi\'on de Conceptos Clave}
Una revisi\'on de los conceptos clave del an\'alisis de supervivencia incluye:
\begin{itemize}
    \item \textbf{Funci\'on de Supervivencia:} Define la probabilidad de sobrevivir m\'as all\'a de un tiempo espec\'ifico.
    \item \textbf{Funci\'on de Riesgo:} Define la tasa instant\'anea de ocurrencia del evento.
    \item \textbf{Estimador de Kaplan-Meier:} Proporciona una estimaci\'on no param\'etrica de la funci\'on de supervivencia.
    \item \textbf{Test de Log-rank:} Compara curvas de supervivencia entre diferentes grupos.
    \item \textbf{Modelo de Cox:} Eval\'ua el efecto de m\'ultiples covariables sobre el tiempo hasta el evento, asumiendo proporcionalidad de riesgos.
    \item \textbf{Modelos AFT:} Modelan el efecto de las covariables multiplicando el tiempo de supervivencia por una constante.
    \item \textbf{An\'alisis Multivariado:} Considera interacciones y efectos no lineales entre m\'ultiples covariables.
    \item \textbf{Supervivencia en Datos Complicados:} Maneja censura por intervalo, datos truncados y competing risks.
\end{itemize}

\section{Ejemplo de Proyecto Final}
A continuaci\'on se presenta un ejemplo de estructura de un proyecto final de an\'alisis de supervivencia:

\subsection{Definici\'on del Problema}
Analizar el efecto del tratamiento y la edad sobre la supervivencia de pacientes con una enfermedad espec\'ifica.

\subsection{Descripci\'on de los Datos}
Datos de supervivencia de 100 pacientes, con covariables: edad, sexo y tipo de tratamiento. Los tiempos de supervivencia est\'an censurados a la derecha.

\subsection{An\'alisis Exploratorio}
Realizar histogramas y curvas de Kaplan-Meier para explorar la distribuci\'on de los tiempos de supervivencia y la censura.

\subsection{Ajuste del Modelo}
Ajustar un modelo de Cox y un modelo AFT con las covariables edad y tratamiento.

\subsection{Diagn\'ostico del Modelo}
Evaluar la proporcionalidad de riesgos y realizar an\'alisis de residuos para identificar observaciones influyentes.

\subsection{Interpretaci\'on de Resultados}
Interpretar los coeficientes del modelo y las curvas de supervivencia ajustadas para diferentes niveles de las covariables.

\subsection{Conclusiones}
Resumir los hallazgos y proporcionar recomendaciones sobre el efecto del tratamiento y la edad en la supervivencia de los pacientes.

\section{Conclusi\'on}
El proyecto final es una oportunidad para aplicar los conocimientos adquiridos en un contexto pr\'actico. La revisi\'on de los conceptos clave y la aplicaci\'on de t\'ecnicas adecuadas de an\'alisis de supervivencia aseguran un an\'alisis riguroso y significativo.



\part{APÉNDICES}
\chapter{IMPLEMENTACIONES NUM\'ERICAS}
\section{D\'ia 1: Regresión Logística}

\subsection*{Implementación Básica en R}

Para implementar una regresión logística en R, primero es necesario instalar y cargar los paquetes necesarios.

\subsection*{Instalación y Configuración de R y RStudio}
\begin{itemize}
    \item Descargue e instale R desde \texttt{https://cran.r-project.org/}. Siga las instrucciones para su sistema operativo (Windows, MacOS, Linux).
    \item Descargue e instale RStudio desde \texttt{https://rstudio.com/products/rstudio/download/}. 
\end{itemize}

\subsection{Ejemplo de Regresión Logística en R}

A continuación, se muestra un ejemplo de cómo ajustar un modelo de regresión logística en R utilizando un conjunto de datos simulado. El ejemplo incluye la instalación del paquete necesario, la carga de datos, el ajuste del modelo, y la interpretación de los resultados.

\begin{verbatim}
# Instalación del paquete necesario
install.packages("stats")

# Carga del paquete
library(stats)

# Ejemplo de conjunto de datos
data <- data.frame(
  outcome = c(1, 0, 1, 0, 1, 1, 0, 1, 0, 0),
  predictor = c(2.3, 1.9, 3.1, 2.8, 3.6, 2.4, 2.1, 3.3, 2.2, 1.7)
)

# Ajuste del modelo de regresión logística
model <- glm(outcome ~ predictor, data = data, family = binomial)

# Resumen del modelo
summary(model)
\end{verbatim}

En este ejemplo, se utiliza el conjunto de datos \textit{data} que contiene una variable de resultado binaria \textit{outcome} y una variable predictora continua \textit{predictor}. El modelo de regresión logística se ajusta utilizando la función \texttt{glm} con la familia binomial. La función \texttt{summary(model)} proporciona un resumen del modelo ajustado, incluyendo los coeficientes estimados, sus errores estándar, valores z, y p-valores.

\begin{itemize}
    \item \textbf{Coeficientes}: Los coeficientes estimados $\beta_0$ y $\beta_1$ indican la dirección y magnitud de la relación entre las variables predictoras y la probabilidad del resultado.
    \item \textbf{Errores Estándar}: Los errores estándar proporcionan una medida de la precisión de los coeficientes estimados.
    \item \textbf{Valores z y p-valores}: Los valores z y p-valores se utilizan para evaluar la significancia estadística de los coeficientes. Un p-valor pequeño (generalmente < 0.05) indica que el coeficiente es significativamente diferente de cero.
\end{itemize}

Este es solo un ejemplo básico, en aplicaciones reales, es posible que necesites realizar más análisis y validaciones, como la evaluación de la bondad de ajuste del modelo, el diagnóstico de posibles problemas de multicolinealidad, y la validación cruzada del modelo.

\begin{verbatim}
# Archivo: regresionlogistica.R

# Instalación del paquete necesario
#install.packages("stats")

# Carga del paquete
library(stats)

# Fijar la semilla para reproducibilidad
set.seed(123)

# Número de observaciones
n <- 100

# Generar las variables independientes X1, X2, ..., X15
# Creamos una matriz de tamaño n x 15 con valores generados aleatoriamente de una
 distribución normal
X <- as.data.frame(matrix(rnorm(n * 15), nrow = n, ncol = 15))
colnames(X) <- paste0("X", 1:15)  # Nombramos las columnas como X1, X2, ..., X15

# Coeficientes verdaderos para las variables independientes
# Generamos un vector de 16 coeficientes (incluyendo el intercepto) aleatorios entre -1 y 1
beta <- runif(16, -1, 1)  # 15 coeficientes más el intercepto

# Generar el término lineal
# Calculamos el término lineal utilizando los coeficientes y las variables independientes
linear_term <- beta[1] + as.matrix(X) %*% beta[-1]

# Generar la probabilidad utilizando la función logística
# Calculamos las probabilidades utilizando la función logística
p <- 1 / (1 + exp(-linear_term))

# Generar la variable dependiente binaria Y
# Generamos valores binarios (0 o 1) utilizando las probabilidades calculadas
Y <- rbinom(n, 1, p)

# Combinar las variables independientes y la variable dependiente en un data frame
data <- cbind(Y, X)

# Dividir el conjunto de datos en entrenamiento y prueba
set.seed(123)  # Fijar la semilla para reproducibilidad
train_indices <- sample(1:n, size = 0.7 * n)  # 70% de los datos para entrenamiento
train_set <- data[train_indices, ]  # Conjunto de entrenamiento
test_set <- data[-train_indices, ]  # Conjunto de prueba

# Ajuste del modelo de regresión logística en el conjunto de entrenamiento
# Ajustamos un modelo de regresión logística utilizando las variables independientes
para predecir Y
model <- glm(Y ~ ., data = train_set, family = binomial)

# Resumen del modelo
# Mostramos un resumen del modelo ajustado
summary(model)

# Guardar el modelo y los resultados en un archivo
# Guardamos el modelo ajustado en un archivo .RData
save(model, file = "regresion_logistica_modelo.RData")

# Guardar los datos simulados en archivos CSV
# Guardamos los conjuntos de datos de entrenamiento y prueba en archivos CSV
write.csv(train_set, "datos_entrenamiento_regresion_logistica.csv", row.names = FALSE)
write.csv(test_set, "datos_prueba_regresion_logistica.csv", row.names = FALSE)

# Hacer predicciones en el conjunto de prueba
# Utilizamos el modelo ajustado para hacer predicciones en el conjunto de prueba
test_set$prob_pred <- predict(model, newdata = test_set, type = "response")
test_set$Y_pred <- ifelse(test_set$prob_pred > 0.5, 1, 0)  
# Convertimos probabilidades a clases binarias

# Calcular la precisión de las predicciones
# Calculamos la precisión de las predicciones comparando con los valores reales de Y
accuracy <- mean(test_set$Y_pred == test_set$Y)
cat("La precisión del modelo en el conjunto de prueba es:", accuracy, "\n")

# Guardar las predicciones en un archivo CSV
# Guardamos las predicciones y las probabilidades predichas en un archivo CSV
write.csv(test_set, "predicciones_regresion_logistica.csv", row.names = FALSE)

# Graficar los coeficientes estimados
# Graficamos los coeficientes estimados del modelo ajustado
plot(coef(model), main = "Coeficientes Estimados del Modelo de Regresión Logística", 
     xlab = "Variables", ylab = "Coeficientes", type = "h", col = "blue")
abline(h = 0, col = "red", lwd = 2)

# Mostrar un mensaje indicando que el proceso ha finalizado
cat("El modelo de regresión logística se ha ajustado, se han hecho predicciones y los resultados se han guardado en 'regresion_logistica_modelo.RData'.\n")
\end{verbatim}

\subsection{Aplicación a Datos de Cáncer - Parte I}

A continuación, se muestra un ejemplo de cómo ajustar un modelo de regresión logística en R utilizando el conjunto de datos del cáncer de mama de Wisconsin.

\begin{verbatim}
# Archivo: regresionlogistica_cancer.R

# Instalación del paquete necesario
install.packages("mlbench")
install.packages("dplyr")

# Carga de los paquetes
library(mlbench)
library(dplyr)

# Cargar el conjunto de datos BreastCancer
data("BreastCancer")

# Ver las primeras filas del conjunto de datos
head(BreastCancer)

# Preprocesamiento de los datos
# Eliminar la columna de identificación y filas con valores faltantes
breast_cancer_clean <- BreastCancer %>%
  select(-Id) %>%
  na.omit()

# Convertir la variable 'Class' a factor binario
breast_cancer_clean$Class <- ifelse(breast_cancer_clean$Class == "malignant", 1, 0)
breast_cancer_clean$Class <- as.factor(breast_cancer_clean$Class)

# Convertir las demás columnas a numéricas
breast_cancer_clean[, 1:9] <- lapply(breast_cancer_clean[, 1:9], as.numeric)

# Dividir el conjunto de datos en entrenamiento (70%) y prueba (30%)
set.seed(123)
train_indices <- sample(1:nrow(breast_cancer_clean), size = 0.7 * nrow(breast_cancer_clean))
train_set <- breast_cancer_clean[train_indices, ]
test_set <- breast_cancer_clean[-train_indices, ]

# Ajuste del modelo de regresión logística en el conjunto de entrenamiento
model <- glm(Class ~ ., data = train_set, family = binomial)

# Resumen del modelo
summary(model)

# Guardar el modelo y los resultados en un archivo
save(model, file = "regresion_logistica_cancer_modelo.RData")

# Guardar los datos simulados en archivos CSV
write.csv(train_set, "datos_entrenamiento_cancer.csv", row.names = FALSE)
write.csv(test_set, "datos_prueba_cancer.csv", row.names = FALSE)

# Hacer predicciones en el conjunto de prueba
test_set$prob_pred <- predict(model, newdata = test_set, type = "response")
test_set$Class_pred <- ifelse(test_set$prob_pred > 0.5, 1, 0)

# Calcular la precisión de las predicciones
accuracy <- mean(test_set$Class_pred == test_set$Class)
cat("La precisión del modelo en el conjunto de prueba es:", accuracy, "\n")

# Guardar las predicciones en un archivo CSV
write.csv(test_set, "predicciones_cancer.csv", row.names = FALSE)

# Graficar los coeficientes estimados
plot(coef(model), main = "Coeficientes Estimados del Modelo de Regresión Logística", 
     xlab = "Variables", ylab = "Coeficientes", type = "h", col = "blue")
abline(h = 0, col = "red", lwd = 2)

# Mostrar un mensaje indicando que el proceso ha finalizado
cat("El modelo de regresión logística se ha ajustado, se han hecho predicciones y los resultados se han guardado en 'regresion_logistica_cancer_modelo.RData'.\n")
\end{verbatim}

\subsubsection*{Descripción del Código}

\textbf{Instalación y Carga de Paquetes:}

Instalamos y cargamos el paquete \texttt{stats} necesario para la regresión logística.

\textbf{Generación de Datos Simulados:}

\begin{itemize}
    \item Fijamos una semilla para la reproducibilidad.
    \item Generamos un conjunto de datos con 100 observaciones y 15 variables independientes (\texttt{X1, X2, ..., X15}) usando una distribución normal.
    \item Definimos los coeficientes verdaderos para las variables independientes y calculamos el término lineal.
    \item Calculamos las probabilidades usando la función logística y generamos una variable dependiente binaria \texttt{Y} basada en esas probabilidades.
    \item Combinamos las variables independientes y la variable dependiente en un \texttt{data frame}.
\end{itemize}

\textbf{División de Datos en Conjuntos de Entrenamiento y Prueba:}

\begin{itemize}
    \item Dividimos los datos en un conjunto de entrenamiento (70\%) y un conjunto de prueba (30\%).
\end{itemize}

\textbf{Ajuste del Modelo de Regresión Logística:}

\begin{itemize}
    \item Ajustamos un modelo de regresión logística en el conjunto de entrenamiento.
    \item Mostramos un resumen del modelo ajustado.
\end{itemize}

\textbf{Guardado de Datos y Modelo:}

\begin{itemize}
    \item Guardamos el modelo ajustado en un archivo \texttt{.RData}.
    \item Guardamos los conjuntos de datos de entrenamiento y prueba en archivos CSV.
\end{itemize}

\textbf{Predicciones y Evaluación del Modelo:}

\begin{itemize}
    \item Hacemos predicciones en el conjunto de prueba utilizando el modelo ajustado.
    \item Calculamos la precisión de las predicciones comparando con los valores reales de \texttt{Y}.
    \item Guardamos las predicciones y las probabilidades predichas en un archivo CSV.
\end{itemize}

\textbf{Visualización de los Coeficientes del Modelo:}

\begin{itemize}
    \item Graficamos los coeficientes estimados del modelo ajustado.
    \item Mostramos un mensaje indicando que el proceso ha finalizado.
\end{itemize}

Para ejecutar este script, guarda el código en un archivo llamado \textit{regresionlogistica.R}, abre R o RStudio, navega hasta el directorio donde guardaste el archivo y ejecuta el script usando \textit{source("regresionlogistica.R")}.

\subsubsection{Ejemplo Titanic}

Cuando realizas una regresión logística, obtienes coeficientes para cada variable independiente en tu modelo. Estos coeficientes indican la dirección y la magnitud de la relación entre cada variable independiente y la variable dependiente (en este caso, \textit{Survived}).

\subsubsection*{Interpretación de los Coeficientes}

\begin{itemize}
    \item \textbf{Intercepto} (\textit{(Intercept)}): Este coeficiente representa el logaritmo de las probabilidades (log-odds) de que \textit{Survived} sea 1 (supervivencia) cuando todas las variables independientes son cero.
    \item \textbf{Pclass}: El coeficiente asociado con \textit{Pclass} indica cómo cambia el log-odds de supervivencia con cada incremento en la clase del pasajero. Si el coeficiente es negativo, sugiere que una clase más alta (por ejemplo, de primera clase a tercera clase) reduce las probabilidades de supervivencia.
    \item \textbf{Sex}: Este coeficiente muestra el efecto de ser hombre o mujer en las probabilidades de supervivencia. Generalmente, se espera que el coeficiente sea positivo para \textit{female} indicando que las mujeres tenían mayores probabilidades de sobrevivir.
    \item \textbf{Age}: El coeficiente de \textit{Age} indica cómo cambia el log-odds de supervivencia con cada año de incremento en la edad. Un coeficiente negativo sugiere que la probabilidad de supervivencia disminuye con la edad.
    \item \textbf{SibSp} y \textbf{Parch}: Estos coeficientes indican el efecto del número de hermanos/cónyuges a bordo y padres/hijos a bordo en las probabilidades de supervivencia.
    \item \textbf{Fare}: Este coeficiente indica el efecto del precio del billete en las probabilidades de supervivencia. Un coeficiente positivo sugiere que pagar más por el billete se asocia con mayores probabilidades de supervivencia.
\end{itemize}

\subsubsection*{Estadísticas de Ajuste del Modelo}

El resumen del modelo (\textit{summary(model)}) incluye varias estadísticas importantes:

\begin{itemize}
    \item \textbf{Estadísticos z y p-valores}: Estas estadísticas indican la significancia de cada coeficiente. Un p-valor bajo (generalmente < 0.05) sugiere que la variable es un predictor significativo de la variable dependiente.
    \item \textbf{Desviación Residual}: La desviación residual mide la calidad del ajuste del modelo. Valores más bajos indican un mejor ajuste.
    \item \textbf{AIC (Akaike Information Criterion)}: El AIC es una medida de la calidad del modelo que toma en cuenta tanto la bondad del ajuste como la complejidad del modelo. Modelos con AIC más bajo son preferidos.
\end{itemize}

\subsubsection*{Precisión del Modelo}

La precisión del modelo en el conjunto de prueba es una métrica importante para evaluar el rendimiento del modelo. La precisión se calcula como el número de predicciones correctas dividido por el número total de predicciones.

\subsubsection*{Ejemplo de Resultados}

Supongamos que la precisión del modelo es 0.78 (78\%). Esto significa que el modelo correctamente predijo el estado de supervivencia del 78\% de los pasajeros en el conjunto de prueba.

\subsubsection*{Matriz de Confusión y Otras Métricas}

Además de la precisión, otras métricas como la matriz de confusión, la sensibilidad, la especificidad, y el área bajo la curva ROC (AUC-ROC) también pueden proporcionar una visión más completa del rendimiento del modelo.

\subsubsection*{Matriz de Confusión}

\begin{itemize}
    \item \textbf{Verdaderos Positivos (TP)}: Número de pasajeros que sobrevivieron y fueron predichos como sobrevivientes.
    \item \textbf{Verdaderos Negativos (TN)}: Número de pasajeros que no sobrevivieron y fueron predichos como no sobrevivientes.
    \item \textbf{Falsos Positivos (FP)}: Número de pasajeros que no sobrevivieron pero fueron predichos como sobrevivientes.
    \item \textbf{Falsos Negativos (FN)}: Número de pasajeros que sobrevivieron pero fueron predichos como no sobrevivientes.
\end{itemize}

\subsubsection*{Ejemplo de Cálculo de Métricas}

\begin{verbatim}
# Calcular la matriz de confusión
table(test_set$Survived, test_set$Survived_pred)

# Calcular sensibilidad y especificidad
sensitivity <- sum(test_set$Survived == 1 & test_set$Survived_pred == 1) / sum(test_set$Survived == 1)
specificity <- sum(test_set$Survived == 0 & test_set$Survived_pred == 0) / sum(test_set$Survived == 0)

# Calcular AUC-ROC
library(pROC)
roc_curve <- roc(test_set$Survived, test_set$prob_pred)
auc(roc_curve)
\end{verbatim}

\subsubsection*{Visualización de Resultados}

Graficar los coeficientes del modelo, la curva ROC y otras visualizaciones ayudan a entender mejor el rendimiento y la importancia de cada variable en el modelo.

\begin{verbatim}
# Graficar la curva ROC
plot(roc_curve, main = "Curva ROC para el Modelo de Regresión Logística")
\end{verbatim}

\subsubsection*{Resumen Final}

El modelo de regresión logística aplicado al conjunto de datos del Titanic proporciona una forma de entender cómo diferentes características de los pasajeros influyen en sus probabilidades de supervivencia. La interpretación de los coeficientes del modelo, las estadísticas de ajuste, y la precisión del modelo en el conjunto de prueba son fundamentales para evaluar el rendimiento y la utilidad del modelo en hacer predicciones sobre la supervivencia de los pasajeros del Titanic.

\subsection{Simulaci\'on de Datos de Cáncer - Parte II}

Aquí se presenta un ejemplo de cómo realizar una regresión logística utilizando datos simulados de pacientes con cáncer.

\begin{verbatim}
#---- Archivo: cancerLogRegSimulado.R ----

# Instalación del paquete necesario
if (!requireNamespace("dplyr", quietly = TRUE)) {
  install.packages("dplyr")
}

# Carga del paquete
library(dplyr)

# Fijar la semilla para reproducibilidad
set.seed(123)

# Número de observaciones
n <- 150

# Generar las variables independientes X1, X2, ..., X15
# Creamos una matriz de tamaño n x 15 con valores generados aleatoriamente de una 
distribución normal
X <- as.data.frame(matrix(rnorm(n * 15), nrow = n, ncol = 15))
colnames(X) <- paste0("X", 1:15)  # Nombramos las columnas como X1, X2, ..., X15

# Coeficientes verdaderos para las variables independientes
# Generamos un vector de 16 coeficientes (incluyendo el intercepto) aleatorios entre -1 y 1
beta <- runif(16, -1, 1)  # 15 coeficientes más el intercepto

# Generar el término lineal
# Calculamos el término lineal utilizando los coeficientes y las variables independientes
linear_term <- beta[1] + as.matrix(X) %*% beta[-1]

# Generar la probabilidad utilizando la función logística
# Calculamos las probabilidades utilizando la función logística
p <- 1 / (1 + exp(-linear_term))

# Generar la variable dependiente binaria Y
# Generamos valores binarios (0 o 1) utilizando las probabilidades calculadas
Y <- rbinom(n, 1, p)

# Combinar las variables independientes y la variable dependiente en un data frame
data <- cbind(Y, X)

# Dividir el conjunto de datos en entrenamiento y prueba
set.seed(123)  # Fijar la semilla para reproducibilidad
train_indices <- sample(1:n, size = 0.7 * n)  # 70% de los datos para entrenamiento
train_set <- data[train_indices, ]  # Conjunto de entrenamiento
test_set <- data[-train_indices, ]  # Conjunto de prueba

# Ajuste del modelo de regresión logística en el conjunto de entrenamiento
# Ajustamos un modelo de regresión logística utilizando las variables independientes 
para predecir Y
model <- glm(Y ~ ., data = train_set, family = binomial)

# Resumen del modelo
# Mostramos un resumen del modelo ajustado
summary(model)

# Guardar el modelo y los resultados en un archivo
# Guardamos el modelo ajustado en un archivo .RData
save(model, file = "regresion_logistica_cancer_modelo_simulado.RData")

# Guardar los datos simulados en archivos CSV
# Guardamos los conjuntos de datos de entrenamiento y prueba en archivos CSV
write.csv(train_set, "datos_entrenamiento_cancer_simulado.csv", row.names = FALSE)
write.csv(test_set, "datos_prueba_cancer_simulado.csv", row.names = FALSE)

# Hacer predicciones en el conjunto de prueba
# Utilizamos el modelo ajustado para hacer predicciones en el conjunto de prueba
test_set$prob_pred <- predict(model, newdata = test_set, type = "response")
test_set$Y_pred <- ifelse(test_set$prob_pred > 0.5, 1, 0)  
# Convertimos probabilidades a clases binarias

# Calcular la precisión de las predicciones
# Calculamos la precisión de las predicciones comparando con los valores reales de Y
accuracy <- mean(test_set$Y_pred == test_set$Y)
cat("La precisión del modelo en el conjunto de prueba es:", accuracy, "\n")

# Guardar las predicciones en un archivo CSV
# Guardamos las predicciones y las probabilidades predichas en un archivo CSV
write.csv(test_set, "predicciones_cancer_simulado.csv", row.names = FALSE)

# Graficar los coeficientes estimados
# Graficamos los coeficientes estimados del modelo ajustado
plot(coef(model), main = "Coeficientes Estimados del Modelo de Regresión Logística", 
     xlab = "Variables", ylab = "Coeficientes", type = "h", col = "blue")
abline(h = 0, col = "red", lwd = 2)

# Mostrar un mensaje indicando que el proceso ha finalizado
cat("El modelo de regresión logística se ha ajustado, se han hecho predicciones 
y los resultados se han guardado en 'regresion_logistica_cancer_modelo_simulado.RData'.\n")
\end{verbatim}

\subsection{Simulaci\'on de Datos de Cáncer - Parte III}

En un estudio sobre cáncer, especialmente en el contexto del cáncer de mama, las principales mediciones suelen incluir una variedad de características clínicas y patológicas. Aquí hay algunas de las principales mediciones que se tienen en cuenta:

\begin{itemize}
    \item \textbf{Tamaño del Tumor}: Medición del diámetro del tumor.
    \item \textbf{Estado de los Ganglios Linfáticos}: Número de ganglios linfáticos afectados.
    \item \textbf{Grado del Tumor}: Clasificación del tumor basada en la apariencia de las células cancerosas.
    \item \textbf{Receptores Hormonales}: Estado de los receptores de estrógeno y progesterona.
    \item \textbf{Estado HER2}: Expresión del receptor 2 del factor de crecimiento epidérmico humano.
    \item \textbf{Ki-67}: Índice de proliferación celular.
    \item \textbf{Edad del Paciente}: Edad en el momento del diagnóstico.
    \item \textbf{Histopatología}: Tipo y subtipo histológico del cáncer.
    \item \textbf{Márgenes Quirúrgicos}: Estado de los márgenes después de la cirugía (si están libres de cáncer o no).
    \item \textbf{Invasión Linfovascular}: Presencia de células cancerosas en los vasos linfáticos o sanguíneos.
    \item \textbf{Tratamientos Previos}: Tipos de tratamientos recibidos antes del diagnóstico (quimioterapia, radioterapia, etc.).
    \item \textbf{Tipo de Cirugía}: Tipo de procedimiento quirúrgico realizado (mastectomía, lumpectomía, etc.).
    \item \textbf{Metástasis}: Presencia de metástasis y ubicación de las mismas.
    \item \textbf{Índice de Masa Corporal (IMC)}: Relación entre el peso y la altura del paciente.
    \item \textbf{Marcadores Genéticos}: Presencia de mutaciones genéticas específicas (BRCA1, BRCA2, etc.).
\end{itemize}

Estas mediciones proporcionan una visión integral del estado del cáncer y se utilizan para planificar el tratamiento y predecir el pronóstico.

A continuación, se muestra un ejemplo de cómo ajustar un modelo de regresión logística en R utilizando un conjunto de datos simulado con estas mediciones.

\begin{verbatim}
# Archivo: simulcorrectedCancer.R

# Instalación del paquete necesario
if (!requireNamespace("dplyr", quietly = TRUE)) {
  install.packages("dplyr")
}

# Carga del paquete
library(dplyr)

# Fijar la semilla para reproducibilidad
set.seed(123)

# Número de observaciones
n <- 1500

# Simulación de las variables independientes
# Tamaño del Tumor (en cm)
Tumor_Size <- rnorm(n, mean = 3, sd = 1.5)

# Estado de los Ganglios Linfáticos (número de ganglios afectados)
Lymph_Nodes <- rpois(n, lambda = 3)

# Grado del Tumor (1 a 3)
Tumor_Grade <- sample(1:3, n, replace = TRUE)

# Receptores Hormonales (0: negativo, 1: positivo)
Estrogen_Receptor <- rbinom(n, 1, 0.7)
Progesterone_Receptor <- rbinom(n, 1, 0.7)

# Estado HER2 (0: negativo, 1: positivo)
HER2_Status <- rbinom(n, 1, 0.3)

# Ki-67 (% de células proliferativas)
Ki_67 <- rnorm(n, mean = 20, sd = 10)

# Edad del Paciente (años)
Age <- rnorm(n, mean = 50, sd = 10)

# Histopatología (1: ductal, 2: lobular, 3: otros)
Histopathology <- sample(1:3, n, replace = TRUE)

# Márgenes Quirúrgicos (0: positivo, 1: negativo)
Surgical_Margins <- rbinom(n, 1, 0.8)

# Invasión Linfovascular (0: no, 1: sí)
Lymphovascular_Invasion <- rbinom(n, 1, 0.4)

# Tratamientos Previos (0: no, 1: sí)
Prior_Treatments <- rbinom(n, 1, 0.5)

# Tipo de Cirugía (0: mastectomía, 1: lumpectomía)
Surgery_Type <- rbinom(n, 1, 0.5)

# Metástasis (0: no, 1: sí)
Metastasis <- rbinom(n, 1, 0.2)

# Índice de Masa Corporal (IMC)
BMI <- rnorm(n, mean = 25, sd = 5)

# Marcadores Genéticos (0: negativo, 1: positivo)
Genetic_Markers <- rbinom(n, 1, 0.1)

# Generar la variable dependiente binaria Y (sobrevivencia 0: no, 1: sí)
# Utilizaremos una combinación arbitraria de las variables para generar Y
linear_term <- -1 + 0.5 * Tumor_Size - 0.3 * Lymph_Nodes + 0.2 * Tumor_Grade + 
  0.4 * Estrogen_Receptor + 0.3 * Progesterone_Receptor - 0.2 * HER2_Status + 
  0.1 * Ki_67 - 0.05 * Age + 0.3 * Surgical_Margins - 0.4 * Lymphovascular_Invasion +
  0.2 * Prior_Treatments + 0.1 * Surgery_Type - 0.5 * Metastasis + 0.01 * BMI + 
  0.2 * Genetic_Markers
p <- 1 / (1 + exp(-linear_term))
Y <- rbinom(n, 1, p)

# Combinar las variables independientes y la variable dependiente en un data frame
data <- data.frame(Y, Tumor_Size, Lymph_Nodes, Tumor_Grade, Estrogen_Receptor, 
                   Progesterone_Receptor, HER2_Status, Ki_67, Age, Histopathology,
                   Surgical_Margins, Lymphovascular_Invasion, Prior_Treatments,
                   Surgery_Type, Metastasis, BMI, Genetic_Markers)

# Dividir el conjunto de datos en entrenamiento y prueba
set.seed(123)  # Fijar la semilla para reproducibilidad
train_indices <- sample(1:n, size = 0.7 * n)  # 70% de los datos para entrenamiento
train_set <- data[train_indices, ]  # Conjunto de entrenamiento
test_set <- data[-train_indices, ]  # Conjunto de prueba

# Ajuste del modelo de regresión logística en el conjunto de entrenamiento
# Ajustamos un modelo de regresión logística utilizando las variables independientes para
 predecir Y
model <- glm(Y ~ ., data = train_set, family = binomial)

# Resumen del modelo
# Mostramos un resumen del modelo ajustado
summary(model)

# Guardar el modelo y los resultados en un archivo
# Guardamos el modelo ajustado en un archivo .RData
save(model, file = "regresion_logistica_cancer_modelo_simulado.RData")

# Guardar los datos simulados en archivos CSV
# Guardamos los conjuntos de datos de entrenamiento y prueba en archivos CSV
write.csv(train_set, "datos_entrenamiento_cancer_simulado.csv", row.names = FALSE)
write.csv(test_set, "datos_prueba_cancer_simulado.csv", row.names = FALSE)

# Hacer predicciones en el conjunto de prueba
# Utilizamos el modelo ajustado para hacer predicciones en el conjunto de prueba
test_set$prob_pred <- predict(model, newdata = test_set, type = "response")
test_set$Y_pred <- ifelse(test_set$prob_pred > 0.5, 1, 0)  
# Convertimos probabilidades a clases binarias

# Calcular la precisión de las predicciones
# Calculamos la precisión de las predicciones comparando con los valores reales de Y
accuracy <- mean(test_set$Y_pred == test_set$Y)
cat("La precisión del modelo en el conjunto de prueba es:", accuracy, "\n")

# Guardar las predicciones en un archivo CSV
# Guardamos las predicciones y las probabilidades predichas en un archivo CSV
write.csv(test_set, "predicciones_cancer_simulado.csv", row.names = FALSE)

# Graficar los coeficientes estimados
# Graficamos los coeficientes estimados del modelo ajustado
plot(coef(model), main = "Coeficientes Estimados del Modelo de Regresión Logística", 
     xlab = "Variables", ylab = "Coeficientes", type = "h", col = "blue")
abline(h = 0, col = "red", lwd = 2)

# Mostrar un mensaje indicando que el proceso ha finalizado
cat("El modelo de regresión logística se ha ajustado, se han hecho predicciones 
y los resultados se han guardado en 'regresion_logistica_cancer_modelo_simulado.RData'.\n")
\end{verbatim}

\chapter{Bibliografía}
\begin{thebibliography}{99}

\bibitem{ISL}
James, G., Witten, D., Hastie, T., and Tibshirani, R. (2013). \textit{An Introduction to Statistical Learning: with Applications in R}. Springer.

\bibitem{Logistic}
Hosmer, D. W., Lemeshow, S., and Sturdivant, R. X. (2013). \textit{Applied Logistic Regression} (3rd ed.). Wiley.

\bibitem{PatternRecognition}
Bishop, C. M. (2006). \textit{Pattern Recognition and Machine Learning}. Springer.

\bibitem{Harrell}
Harrell, F. E. (2015). \textit{Regression Modeling Strategies: With Applications to Linear Models, Logistic and Ordinal Regression, and Survival Analysis}. Springer.

\bibitem{RDocumentation}
R Documentation and Tutorials: \url{https://cran.r-project.org/manuals.html}

\bibitem{RBlogger}
Tutorials on R-bloggers: \url{https://www.r-bloggers.com/}

\bibitem{CourseraML}
Coursera: \textit{Machine Learning} by Andrew Ng.

\bibitem{edXDS}
edX: \textit{Data Science and Machine Learning Essentials} by Microsoft.

% Libros adicionales
\bibitem{Ross}
Ross, S. M. (2014). \textit{Introduction to Probability and Statistics for Engineers and Scientists}. Academic Press.

\bibitem{DeGroot}
DeGroot, M. H., and Schervish, M. J. (2012). \textit{Probability and Statistics} (4th ed.). Pearson.

\bibitem{Hogg}
Hogg, R. V., McKean, J., and Craig, A. T. (2019). \textit{Introduction to Mathematical Statistics} (8th ed.). Pearson.

\bibitem{Kleinbaum}
Kleinbaum, D. G., and Klein, M. (2010). \textit{Logistic Regression: A Self-Learning Text} (3rd ed.). Springer.

% Artículos y tutoriales adicionales
\bibitem{Wasserman}
Wasserman, L. (2004). \textit{All of Statistics: A Concise Course in Statistical Inference}. Springer.

\bibitem{KhanAcademy}
Probability and Statistics Tutorials on Khan Academy: \url{https://www.khanacademy.org/math/statistics-probability}

\bibitem{OnlineStatBook}
Online Statistics Education: \url{http://onlinestatbook.com/}

\bibitem{Peng}
Peng, C. Y. J., Lee, K. L., and Ingersoll, G. M. (2002). \textit{An Introduction to Logistic Regression Analysis and Reporting}. The Journal of Educational Research.

\bibitem{Agresti}
Agresti, A. (2007). \textit{An Introduction to Categorical Data Analysis} (2nd ed.). Wiley.

\bibitem{Han}
Han, J., Pei, J., and Kamber, M. (2011). \textit{Data Mining: Concepts and Techniques}. Morgan Kaufmann.

\bibitem{TowardsDataScience}
Data Cleaning and Preprocessing on Towards Data Science: \url{https://towardsdatascience.com/data-cleaning-and-preprocessing}

\bibitem{Molinaro}
Molinaro, A. M., Simon, R., and Pfeiffer, R. M. (2005). \textit{Prediction error estimation: a comparison of resampling methods}. Bioinformatics.

\bibitem{EvaluatingModels}
Evaluating Machine Learning Models on Towards Data Science: \url{https://towardsdatascience.com/evaluating-machine-learning-models}

\bibitem{LogisticRegressionGuide}
Practical Guide to Logistic Regression in R on Towards Data Science: \url{https://towardsdatascience.com/practical-guide-to-logistic-regression-in-r}

% Cursos en línea adicionales
\bibitem{CourseraStatistics}
Coursera: \textit{Statistics with R} by Duke University.

\bibitem{edXProbability}
edX: \textit{Data Science: Probability} by Harvard University.

\bibitem{CourseraLogistic}
Coursera: \textit{Logistic Regression} by Stanford University.

\bibitem{edXInference}
edX: \textit{Data Science: Inference and Modeling} by Harvard University.

\bibitem{CourseraWrangling}
Coursera: \textit{Data Science: Wrangling and Cleaning} by Johns Hopkins University.

\bibitem{edXRBasics}
edX: \textit{Data Science: R Basics} by Harvard University.

\bibitem{CourseraRegression}
Coursera: \textit{Regression Models} by Johns Hopkins University.

\bibitem{edXStatInference}
edX: \textit{Data Science: Statistical Inference} by Harvard University.

\bibitem{SurvivalAnalysis}
An Introduction to Survival Analysis on Towards Data Science: \url{https://towardsdatascience.com/an-introduction-to-survival-analysis}

\bibitem{MultinomialLogistic}
Multinomial Logistic Regression on DataCamp: \url{https://www.datacamp.com/community/tutorials/multinomial-logistic-regression-R}

\bibitem{CourseraSurvival}
Coursera: \textit{Survival Analysis} by Johns Hopkins University.

\bibitem{edXHighthroughput}
edX: \textit{Data Science: Statistical Inference and Modeling for High-throughput Experiments} by Harvard University.

\end{thebibliography}


\end{document}
