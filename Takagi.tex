

%______________________________________________________________________
\section{Ecuaciones Centrales}
%______________________________________________________________________



\begin{Prop}
Supongamos

\begin{equation}\label{Eq.1}
f_{i}\left(i\right)-f_{j}\left(i\right)=\mu_{i}\left[\sum_{k=j}^{i-1}r_{k}+\sum_{k=j}^{i-1}\frac{f_{k}\left(k\right)}{1-\mu_{k}}\right]
\end{equation}

\begin{equation}\label{Eq.2}
f_{i+1}\left(i\right)=r_{i}\mu_{i},
\end{equation}

Demostrar que

\begin{eqnarray*}
f_{i}\left(i\right)&=&\mu_{i}\left[\sum_{k=1}^{N}r_{k}+\sum_{k=1,k\neq
i}^{N}\frac{f_{k}\left(k\right)}{1-\mu_{k}}\right].
\end{eqnarray*}


En la Ecuaci\'on (\ref{Eq.2}) hagamos $j=i+1$, entonces se tiene
$f_{j}=r_{i}\mu_{i}$, lo mismo para (\ref{Eq.1})

\begin{eqnarray*}
f_{i}\left(i\right)&=&r_{i}\mu_{i}+\mu_{i}\left[\sum_{k=j}^{i-1}r_{k}+\sum_{k=j}^{i-1}\frac{f_{k}\left(k\right)}{1-\mu_{k}}\right]\\
&=&\mu_{i}\left[\sum_{k=j}^{i}r_{k}+\sum_{k=j}^{i-1}\frac{f_{k}\left(k\right)}{1-\mu_{k}}\right]\\
\end{eqnarray*}

entonces, tomando sobre todo valor de $1,\ldots,N$, tanto para
antes de $i$ como para despu\'es de $i$, entonces

\begin{eqnarray*}
f_{i}\left(i\right)&=&\mu_{i}\left[\sum_{k=1}^{N}r_{k}+\sum_{k=1,k\neq
i}^{N}\frac{f_{k}\left(k\right)}{1-\mu_{k}}\right].
\end{eqnarray*}
\end{Prop}


Ahora, supongamos nuevamente la ecuaci\'on (\ref{Eq.1})

\begin{eqnarray*}
f_{i}\left(i\right)-f_{j}\left(i\right)&=&\mu_{i}\left[\sum_{k=j}^{i-1}r_{k}+\sum_{k=j}^{i-1}\frac{f_{k}\left(k\right)}{1-\mu_{k}}\right]\\
&\Leftrightarrow&\\
f_{j}\left(j\right)-f_{i}\left(j\right)&=&\mu_{j}\left[\sum_{k=i}^{j-1}r_{k}+\sum_{k=i}^{j-1}\frac{f_{k}\left(k\right)}{1-\mu_{k}}\right]\\
f_{i}\left(j\right)&=&f_{j}\left(j\right)-\mu_{j}\left[\sum_{k=i}^{j-1}r_{k}+\sum_{k=i}^{j-1}\frac{f_{k}\left(k\right)}{1-\mu_{k}}\right]\\
&=&\mu_{j}\left(1-\mu_{j}\right)\frac{r}{1-\mu}-\mu_{j}\left[\sum_{k=i}^{j-1}r_{k}+\sum_{k=i}^{j-1}\frac{f_{k}\left(k\right)}{1-\mu_{k}}\right]\\
&=&\mu_{j}\left[\left(1-\mu_{j}\right)\frac{r}{1-\mu}-\sum_{k=i}^{j-1}r_{k}-\sum_{k=i}^{j-1}\frac{f_{k}\left(k\right)}{1-\mu_{k}}\right]\\
&=&\mu_{j}\left[\left(1-\mu_{j}\right)\frac{r}{1-\mu}-\sum_{k=i}^{j-1}r_{k}-\frac{r}{1-\mu}\sum_{k=i}^{j-1}\mu_{k}\right]\\
&=&\mu_{j}\left[\frac{r}{1-\mu}\left(1-\mu_{j}-\sum_{k=i}^{j-1}\mu_{k}\right)-\sum_{k=i}^{j-1}r_{k}\right]\\
&=&\mu_{j}\left[\frac{r}{1-\mu}\left(1-\sum_{k=i}^{j}\mu_{k}\right)-\sum_{k=i}^{j-1}r_{k}\right].\\
\end{eqnarray*}

Ahora,

\begin{eqnarray*}
1-\sum_{k=i}^{j}\mu_{k}&=&1-\sum_{k=1}^{N}\mu_{k}+\sum_{k=j+1}^{i-1}\mu_{k}\\
&\Leftrightarrow&\\
\sum_{k=i}^{j}\mu_{k}&=&\sum_{k=1}^{N}\mu_{k}-\sum_{k=j+1}^{i-1}\mu_{k}\\
&\Leftrightarrow&\\
\sum_{k=1}^{N}\mu_{k}&=&\sum_{k=i}^{j}\mu_{k}+\sum_{k=j+1}^{i-1}\mu_{k}\\
\end{eqnarray*}

Por tanto
\begin{eqnarray*}
f_{i}\left(j\right)&=&\mu_{j}\left[\frac{r}{1-\mu}\sum_{k=j+1}^{i-1}\mu_{k}+\sum_{k=j}^{i-1}r_{k}\right].
\end{eqnarray*}





\begin{Teo}[Teorema de Continuidad]
Sup\'ongase que $\left\{X_{n},n=1,2,3,\ldots\right\}$ son
variables aleatorias finitas, no negativas con valores enteros
tales que $P\left(X_{n}=k\right)=p_{k}^{(n)}$, para
$n=1,2,3,\ldots$, $k=0,1,2,\ldots$, con
$\sum_{k=0}^{\infty}p_{k}^{(n)}=1$, para $n=1,2,3,\ldots$. Sea
$g_{n}$ la PGF para la variable aleatoria $X_{n}$. Entonces existe
una sucesi\'on $\left\{p_{k}\right\}$ tal que \begin{eqnarray*}
lim_{n\rightarrow\infty}p_{k}^{(n)}=p_{k}\textrm{ para }0<s<1.
\end{eqnarray*}
En este caso, $g\left(s\right)=\sum_{k=0}^{\infty}s^{k}p_{k}$.
Adem\'as
\begin{eqnarray*}
\sum_{k=0}^{\infty}p_{k}=1\textrm{ si y s\'olo si
}lim_{s\uparrow1}g\left(s\right)=1
\end{eqnarray*}
\end{Teo}

\begin{Teo}
Sea $N$ una variable aleatoria con valores enteros no negativos
finita tal que $P\left(N=k\right)=p_{k}$, para $k=0,1,2,\ldots$, y
$\sum_{k=0}^{\infty}p_{k}=P\left(N<\infty\right)=1$. Sea $\Phi$ la
PGF de $N$ tal que
$g\left(s\right)=\esp\left[s^{N}\right]=\sum_{k=0}^{\infty}s^{k}p_{k}$
con $g\left(1\right)=1$. Si $0\leq p_{1}\leq1$ y
$\esp\left[N\right]=g^{'}\left(1\right)\leq1$, entonces no existe
soluci\'on  de la ecuaci\'on $g\left(s\right)=s$ en el intervalo
$\left[0,1\right)$. Si $\esp\left[N\right]=g^{'}\left(1\right)>1$,
lo cual implica que $0\leq p_{1}<1$, entonces existe una \'unica
soluci\'on de la ecuaci\'on $g\left(s\right)=s$ en el intervalo
$\left[0,1\right)$.
\end{Teo}


\begin{Teo}
Si $X$ y $Y$ tienen PGF $G_{X}$ y $G_{Y}$ respectivamente,
entonces,\[G_{X}\left(s\right)=G_{Y}\left(s\right)\] para toda
$s$, si y s\'olo si \[P\left(X=k\right))=P\left(Y=k\right)\] para
toda $k=0,1,\ldots,$., es decir, si y s\'olo si $X$ y $Y$ tienen
la misma distribuci\'on de probabilidad.
\end{Teo}


\begin{Teo}
Para cada $n$ fijo, sea la sucesi\'oin de probabilidades
$\left\{a_{0,n},a_{1,n},\ldots,\right\}$, tales que $a_{k,n}\geq0$
para toda $k=0,1,2,\ldots,$ y $\sum_{k\geq0}a_{k,n}=1$, y sea
$G_{n}\left(s\right)$ la correspondiente funci\'on generadora,
$G_{n}\left(s\right)=\sum_{k\geq0}a_{k,n}s^{k}$. De modo que para
cada valor fijo de $k$
\begin{eqnarray*}
lim_{n\rightarrow\infty}a_{k,n}=a_{k},
\end{eqnarray*}
es decir converge en distribuci\'on, es necesario y suficiente que
para cada valor fijo $s\in\left[0,\right)$,
\begin{eqnarray*}
lim_{n\rightarrow\infty}G_{n}\left(s\right)=G\left(s\right),
\end{eqnarray*}
donde $G\left(s\right)=\sum_{k\geq0}p_{k}s^{k}$, para cualquier

la funci\'on generadora del l\'imite de la sucesi\'on.
\end{Teo}

\begin{Teo}[Teorema de Abel]
Sea $G\left(s\right)=\sum_{k\geq0}a_{k}s^{k}$ para cualquier
$\left\{p_{0},p_{1},\ldots,\right\}$, tales que $p_{k}\geq0$ para
toda $k=0,1,2,\ldots,$. Entonces $G\left(s\right)$ es continua por
la derecha en $s=1$, es decir
\begin{eqnarray*}
lim_{s\uparrow1}G\left(s\right)=\sum_{k\geq0}p_{k}=G\left(\right),
\end{eqnarray*}
sin importar si la suma es finita o no.
\end{Teo}
\begin{Note}
El radio de Convergencia para cualquier PGF es $R\geq1$, entonces,
el Teorema de Abel nos dice que a\'un en el peor escenario, cuando
$R=1$, a\'un se puede confiar en que la PGF ser\'a continua en
$s=1$, en contraste, no se puede asegurar que la PGF ser\'a
continua en el l\'imite inferior $-R$, puesto que la PGF es
sim\'etrica alrededor del cero: la PGF converge para todo
$s\in\left(-R,R\right)$, y no lo hace para $s<-R$ o $s>R$.
Adem\'as nos dice que podemos escribir $G_{X}\left(1\right)$ como
una abreviaci\'on de $lim_{s\uparrow1}G_{X}\left(s\right)$.
\end{Note}

Entonces si suponemos que la diferenciaci\'on t\'ermino a
t\'ermino est\'a permitida, entonces

\begin{eqnarray*}
G_{X}^{'}\left(s\right)&=&\sum_{x=1}^{\infty}xs^{x-1}p_{x}
\end{eqnarray*}

el Teorema de Abel nos dice que
\begin{eqnarray*}
\esp\left(X\right]&=&\lim_{s\uparrow1}G_{X}^{'}\left(s\right):\\
\esp\left[X\right]&=&=\sum_{x=1}^{\infty}xp_{x}=G_{X}^{'}\left(1\right)\\
&=&\lim_{s\uparrow1}G_{X}^{'}\left(s\right),
\end{eqnarray*}
dado que el Teorema de Abel se aplica a
\begin{eqnarray*}
G_{X}^{'}\left(s\right)&=&\sum_{x=1}^{\infty}xs^{x-1}p_{x},
\end{eqnarray*}
estableciendo as\'i que $G_{X}^{'}\left(s\right)$ es continua en
$s=1$. Sin el Teorema de Abel no se podr\'ia asegurar que el
l\'imite de $G_{X}^{'}\left(s\right)$ conforme $s\uparrow1$ sea la
respuesta correcta para $\esp\left[X\right]$.

\begin{Note}
La PGF converge para todo $|s|<R$, para alg\'un $R$. De hecho la
PGF converge absolutamente si $|s|<R$. La PGF adem\'as converge
uniformemente en conjuntos de la forma
$\left\{s:|s|<R^{'}\right\}$, donde $R^{'}<R$, es decir,
$\forall\epsilon>0, \exists n_{0}\in\ent$ tal que $\forall s$, con
$|s|<R^{'}$, y $\forall n\geq n_{0}$,
\begin{eqnarray*}
|\sum_{x=0}^{n}s^{x}\prob\left(X=x\right)-G_{X}\left(s\right)|<\epsilon.
\end{eqnarray*}
De hecho, la convergencia uniforme es la que nos permite
diferenciar t\'ermino a t\'ermino:
\begin{eqnarray*}
G_{X}\left(s\right)=\esp\left[s^{X}\right]=\sum_{x=0}^{\infty}s^{x}\prob\left(X=x\right),
\end{eqnarray*}
y sea $s<R$.
\begin{enumerate}
\item
\begin{eqnarray*}
G_{X}^{'}\left(s\right)&=&\frac{d}{ds}\left(\sum_{x=0}^{\infty}s^{x}\prob\left(X=x\right)\right)=\sum_{x=0}^{\infty}\frac{d}{ds}\left(s^{x}\prob\left(X=x\right)\right)\\
&=&\sum_{x=0}^{n}xs^{x-1}\prob\left(X=x\right).
\end{eqnarray*}

\item\begin{eqnarray*}
\int_{a}^{b}G_{X}\left(s\right)ds&=&\int_{a}^{b}\left(\sum_{x=0}^{\infty}s^{x}\prob\left(X=x\right)\right)ds=\sum_{x=0}^{\infty}\left(\int_{a}^{b}s^{x}\prob\left(X=x\right)ds\right)\\
&=&\sum_{x=0}^{\infty}\frac{s^{x+1}}{x+1}\prob\left(X=x\right),
\end{eqnarray*}
para $-R<a<b<R$.
\end{enumerate}
\end{Note}

\begin{Teo}[Teorema de Convergencia Mon\'otona para PGF]
Sean $X$ y $X_{n}$ variables aleatorias no negativas, con valores
en los enteros, finitas, tales que
\begin{eqnarray*}
lim_{n\rightarrow\infty}G_{X_{n}}\left(s\right)&=&G_{X}\left(s\right)
\end{eqnarray*}
para $0\leq s\leq1$, entonces
\begin{eqnarray*}
lim_{n\rightarrow\infty}P\left(X_{n}=k\right)=P\left(X=k\right),
\end{eqnarray*}
para $k=0,1,2,\ldots.$
\end{Teo}

El teorema anterior requiere del siguiente lema

\begin{Lemma}
Sean $a_{n,k}\in\ent^{+}$, $n\in\nat$ constantes no negativas con
$\sum_{k\geq0}a_{k,n}\leq1$. Sup\'ongase que para $0\leq s\leq1$,
se tiene

\begin{eqnarray*}
a_{n}\left(s\right)&=&\sum_{k=0}^{\infty}a_{k,n}s^{k}\rightarrow
a\left(s\right)=\sum_{k=0}^{\infty}a_{k}s^{k}.
\end{eqnarray*}
Entonces
\begin{eqnarray*}
a_{0,n}\rightarrow a_{0}.
\end{eqnarray*}
\end{Lemma}


%________________________________________________________
\section{Funciones Generadoras de Probabilidad Conjunta}
%________________________________________________________


De lo desarrollado hasta ahora se tiene lo siguiente

\begin{eqnarray*}
&&\esp\left[z_{1}^{L_{1}\left(\overline{\tau}_{1}\right)}z_{2}^{L_{2}\left(\overline{\tau}_{1}\right)}\right]=\esp\left[z_{2}^{L_{2}\left(\overline{\tau}_{1}\right)}\right]=\esp\left[z_{2}^{L_{2}\left(\tau_{1}\right)+X_{2}\left(\overline{\tau}_{1}-\tau_{1}\right)}\right]\\
&=&\esp\left[\left\{z_{2}^{L_{2}\left(\tau_{1}\right)}\right\}\left\{z_{2}^{X_{2}\left(\overline{\tau}_{1}-\tau_{1}\right)}\right\}\right]=\esp\left[\left\{z_{2}^{L_{2}\left(\tau_{1}\right)}\right\}\left\{P_{2}\left(z_{2}\right)\right\}^{\overline{\tau}_{1}-\tau_{1}}\right]\\
&=&\esp\left[\left\{z_{2}^{L_{2}\left(\tau_{1}\right)}\right\}\left\{\theta_{1}\left(P_{2}\left(z_{2}\right)\right)\right\}^{L_{1}\left(\tau_{1}\right)}\right]=F_{1}\left(\theta_{1}\left(P_{2}\left(z_{2}\right)\right),z_{2}\right)
\end{eqnarray*}

es decir %{{\tiny
\begin{equation}\label{Eq.base.F1}
\esp\left[z_{1}^{L_{1}\left(\overline{\tau}_{1}\right)}z_{2}^{L_{2}\left(\overline{\tau}_{1}\right)}\right]=F_{1}\left(\theta_{1}\left(P_{2}\left(z_{2}\right)\right),z_{2}\right).
\end{equation}

Procediendo de manera an\'aloga para $\overline{\tau}_{2}$:

\begin{eqnarray*}
\esp\left[z_{1}^{L_{1}\left(\overline{\tau}_{2}\right)}z_{2}^{L_{2}\left(\overline{\tau}_{2}\right)}\right]&=&\esp\left[z_{1}^{L_{1}\left(\overline{\tau}_{2}\right)}\right]=\esp\left[z_{1}^{L_{1}\left(\tau_{2}\right)+X_{1}\left(\overline{\tau}_{2}-\tau_{2}\right)}\right]=\esp\left[\left\{z_{1}^{L_{1}\left(\tau_{2}\right)}\right\}\left\{z_{1}^{X_{1}\left(\overline{\tau}_{2}-\tau_{2}\right)}\right\}\right]\\
&=&\esp\left[\left\{z_{1}^{L_{1}\left(\tau_{2}\right)}\right\}\left\{P_{1}\left(z_{1}\right)\right\}^{\overline{\tau}_{2}-\tau_{2}}\right]=\esp\left[\left\{z_{1}^{L_{1}\left(\tau_{2}\right)}\right\}\left\{\theta_{2}\left(P_{1}\left(z_{1}\right)\right)\right\}^{L_{2}\left(\tau_{2}\right)}\right]\\
&=&F_{2}\left(z_{1},\theta_{2}\left(P_{1}\left(z_{1}\right)\right)\right)
\end{eqnarray*}%}}


\begin{equation}\label{Eq.PGF.Conjunta.Tau2}
\esp\left[z_{1}^{L_{1}\left(\overline{\tau}_{2}\right)}z_{2}^{L_{2}\left(\overline{\tau}_{2}\right)}\right]=F_{2}\left(z_{1},\theta_{2}\left(P_{1}\left(z_{1}\right)\right)\right)
\end{equation}%}

Ahora, para el intervalo de tiempo
$\left[\overline{\tau}_{1},\tau_{2}\right]$ y
$\left[\overline{\tau}_{2},\tau_{1}\right]$, los arribos de los
usuarios modifican el n\'umero de usuarios que llegan a las colas,
es decir, los procesos
$L_{1}\left(t\right)$
y $L_{2}\left(t\right)$. La PGF para el n\'umero de arribos
a todas las estaciones durante el intervalo
$\left[\overline{\tau}_{1},\tau_{2}\right]$  cuya distribuci\'on
est\'a especificada por la distribuci\'on compuesta
$R_{1}\left(\mathbf{z}\right),R_{2}\left(\mathbf{z}\right)$:

\begin{eqnarray*}
R_{1}\left(\mathbf{z}\right)=R_{1}\left(\prod_{i=1}^{2}P\left(z_{i}\right)\right)=\esp\left[\left\{\prod_{i=1}^{2}P\left(z_{i}\right)\right\}^{\tau_{2}-\overline{\tau}_{1}}\right]\\
R_{2}\left(\mathbf{z}\right)=R_{2}\left(\prod_{i=1}^{2}P\left(z_{i}\right)\right)=\esp\left[\left\{\prod_{i=1}^{2}P\left(z_{i}\right)\right\}^{\tau_{1}-\overline{\tau}_{2}}\right]\\
\end{eqnarray*}


Dado que los eventos en
$\left[\tau_{1},\overline{\tau}_{1}\right]$ y
$\left[\overline{\tau}_{1},\tau_{2}\right]$ son independientes, la
PGF conjunta para el n\'umero de usuarios en el sistema al tiempo
$t=\tau_{2}$ la PGF conjunta para el n\'umero de usuarios en el sistema est\'an dadas por

{\footnotesize{
\begin{eqnarray*}
F_{1}\left(\mathbf{z}\right)&=&R_{2}\left(\prod_{i=1}^{2}P\left(z_{i}\right)\right)F_{2}\left(z_{1},\theta_{2}\left(P_{1}\left(z_{1}\right)\right)\right)\\
F_{2}\left(\mathbf{z}\right)&=&R_{1}\left(\prod_{i=1}^{2}P\left(z_{i}\right)\right)F_{1}\left(\theta_{1}\left(P_{2}\left(z_{2}\right)\right),z_{2}\right)\\
\end{eqnarray*}}}


Entonces debemos de determinar las siguientes expresiones:


\begin{eqnarray*}
\begin{array}{cc}
f_{1}\left(1\right)=\frac{\partial F_{1}\left(\mathbf{z}\right)}{\partial z_{1}}|_{\mathbf{z}=1}, & f_{1}\left(2\right)=\frac{\partial F_{1}\left(\mathbf{z}\right)}{\partial z_{2}}|_{\mathbf{z}=1},\\
f_{2}\left(1\right)=\frac{\partial F_{2}\left(\mathbf{z}\right)}{\partial z_{1}}|_{\mathbf{z}=1}, & f_{2}\left(2\right)=\frac{\partial F_{2}\left(\mathbf{z}\right)}{\partial z_{2}}|_{\mathbf{z}=1},\\
\end{array}
\end{eqnarray*}


\begin{eqnarray*}
\frac{\partial R_{1}\left(\mathbf{z}\right)}{\partial
z_{1}}|_{\mathbf{z}=1}&=&R_{1}^{(1)}\left(1\right)P_{1}^{(1)}\left(1\right)\\
\frac{\partial R_{1}\left(\mathbf{z}\right)}{\partial
z_{2}}|_{\mathbf{z}=1}&=&R_{1}^{(1)}\left(1\right)P_{2}^{(1)}\left(1\right)\\
\frac{\partial R_{2}\left(\mathbf{z}\right)}{\partial
z_{1}}|_{\mathbf{z}=1}&=&R_{2}^{(1)}\left(1\right)P_{1}^{(1)}\left(1\right)\\
\frac{\partial R_{2}\left(\mathbf{z}\right)}{\partial
z_{2}}|_{\mathbf{z}=1}&=&R_{2}^{(1)}\left(1\right)P_{2}^{(1)}\left(1\right)\\
\end{eqnarray*}



\begin{eqnarray*}
\frac{\partial}{\partial
z_{1}}F_{1}\left(\theta_{1}\left(P_{2}\left(z_{2}\right)\right),z_{2}\right)&=&0\\
\frac{\partial}{\partial
z_{2}}F_{1}\left(\theta_{1}\left(P_{2}\left(z_{2}\right)\right),z_{2}\right)&=&\frac{\partial
F_{1}}{\partial z_{2}}+\frac{\partial F_{1}}{\partial
z_{1}}\theta_{1}^{(1)}P_{2}^{(1)}\left(1\right)\\
\frac{\partial}{\partial
z_{1}}F_{2}\left(z_{1},\theta_{2}\left(P_{1}\left(z_{1}\right)\right)\right)&=&\frac{\partial
F_{2}}{\partial z_{1}}+\frac{\partial F_{2}}{\partial
z_{2}}\theta_{2}^{(1)}P_{1}^{(1)}\left(1\right)\\
\frac{\partial}{\partial
z_{2}}F_{2}\left(z_{1},\theta_{2}\left(P_{1}\left(z_{1}\right)\right)\right)&=&0\\
\end{eqnarray*}


Por lo tanto de las dos secciones anteriores se tiene que:


\begin{eqnarray*}
\frac{\partial F_{1}}{\partial z_{1}}&=&\frac{\partial
R_{2}}{\partial z_{1}}|_{\mathbf{z}=1}+\frac{\partial F_{2}}{\partial z_{1}}|_{\mathbf{z}=1}=R_{2}^{(1)}\left(1\right)P_{1}^{(1)}\left(1\right)+f_{2}\left(1\right)+f_{2}\left(2\right)\theta_{2}^{(1)}\left(1\right)P_{1}^{(1)}\left(1\right)\\
\frac{\partial F_{1}}{\partial z_{2}}&=&\frac{\partial
R_{2}}{\partial z_{2}}|_{\mathbf{z}=1}+\frac{\partial F_{2}}{\partial z_{2}}|_{\mathbf{z}=1}=R_{2}^{(1)}\left(1\right)P_{2}^{(1)}\left(1\right)\\
\frac{\partial F_{2}}{\partial z_{1}}&=&\frac{\partial
R_{1}}{\partial z_{1}}|_{\mathbf{z}=1}+\frac{\partial F_{1}}{\partial z_{1}}|_{\mathbf{z}=1}=R_{1}^{(1)}\left(1\right)P_{1}^{(1)}\left(1\right)\\
\frac{\partial F_{2}}{\partial z_{2}}&=&\frac{\partial
R_{1}}{\partial z_{2}}|_{\mathbf{z}=1}+\frac{\partial F_{1}}{\partial z_{2}}|_{\mathbf{z}=1}
=R_{1}^{(1)}\left(1\right)P_{2}^{(1)}\left(1\right)+f_{1}\left(1\right)\theta_{1}^{(1)}\left(1\right)P_{2}^{(1)}\left(1\right)\\
\end{eqnarray*}


El cual se puede escribir en forma equivalente:
\begin{eqnarray*}
f_{1}\left(1\right)&=&r_{2}\mu_{1}+f_{2}\left(1\right)+f_{2}\left(2\right)\frac{\mu_{1}}{1-\mu_{2}}\\
f_{1}\left(2\right)&=&r_{2}\mu_{2}\\
f_{2}\left(1\right)&=&r_{1}\mu_{1}\\
f_{2}\left(2\right)&=&r_{1}\mu_{2}+f_{1}\left(2\right)+f_{1}\left(1\right)\frac{\mu_{2}}{1-\mu_{1}}\\
\end{eqnarray*}

De donde:
\begin{eqnarray*}
f_{1}\left(1\right)&=&\mu_{1}\left[r_{2}+\frac{f_{2}\left(2\right)}{1-\mu_{2}}\right]+f_{2}\left(1\right)\\
f_{2}\left(2\right)&=&\mu_{2}\left[r_{1}+\frac{f_{1}\left(1\right)}{1-\mu_{1}}\right]+f_{1}\left(2\right)\\
\end{eqnarray*}

Resolviendo para $f_{1}\left(1\right)$:
\begin{eqnarray*}
f_{1}\left(1\right)&=&r_{2}\mu_{1}+f_{2}\left(1\right)+f_{2}\left(2\right)\frac{\mu_{1}}{1-\mu_{2}}=r_{2}\mu_{1}+r_{1}\mu_{1}+f_{2}\left(2\right)\frac{\mu_{1}}{1-\mu_{2}}\\
&=&\mu_{1}\left(r_{2}+r_{1}\right)+f_{2}\left(2\right)\frac{\mu_{1}}{1-\mu_{2}}=\mu_{1}\left(r+\frac{f_{2}\left(2\right)}{1-\mu_{2}}\right),\\
\end{eqnarray*}

entonces

\begin{eqnarray*}
f_{2}\left(2\right)&=&\mu_{2}\left(r_{1}+\frac{f_{1}\left(1\right)}{1-\mu_{1}}\right)+f_{1}\left(2\right)=\mu_{2}\left(r_{1}+\frac{f_{1}\left(1\right)}{1-\mu_{1}}\right)+r_{2}\mu_{2}\\
&=&\mu_{2}\left[r_{1}+r_{2}+\frac{f_{1}\left(1\right)}{1-\mu_{1}}\right]=\mu_{2}\left[r+\frac{f_{1}\left(1\right)}{1-\mu_{1}}\right]\\
&=&\mu_{2}r+\mu_{1}\left(r+\frac{f_{2}\left(2\right)}{1-\mu_{2}}\right)\frac{\mu_{2}}{1-\mu_{1}}\\
&=&\mu_{2}r+\mu_{2}\frac{r\mu_{1}}{1-\mu_{1}}+f_{2}\left(2\right)\frac{\mu_{1}\mu_{2}}{\left(1-\mu_{1}\right)\left(1-\mu_{2}\right)}\\
&=&\mu_{2}\left(r+\frac{r\mu_{1}}{1-\mu_{1}}\right)+f_{2}\left(2\right)\frac{\mu_{1}\mu_{2}}{\left(1-\mu_{1}\right)\left(1-\mu_{2}\right)}\\
&=&\mu_{2}\left(\frac{r}{1-\mu_{1}}\right)+f_{2}\left(2\right)\frac{\mu_{1}\mu_{2}}{\left(1-\mu_{1}\right)\left(1-\mu_{2}\right)}\\
\end{eqnarray*}
entonces
\begin{eqnarray*}
f_{2}\left(2\right)-f_{2}\left(2\right)\frac{\mu_{1}\mu_{2}}{\left(1-\mu_{1}\right)\left(1-\mu_{2}\right)}&=&\mu_{2}\left(\frac{r}{1-\mu_{1}}\right)\\
f_{2}\left(2\right)\left(1-\frac{\mu_{1}\mu_{2}}{\left(1-\mu_{1}\right)\left(1-\mu_{2}\right)}\right)&=&\mu_{2}\left(\frac{r}{1-\mu_{1}}\right)\\
f_{2}\left(2\right)\left(\frac{1-\mu_{1}-\mu_{2}+\mu_{1}\mu_{2}-\mu_{1}\mu_{2}}{\left(1-\mu_{1}\right)\left(1-\mu_{2}\right)}\right)&=&\mu_{2}\left(\frac{r}{1-\mu_{1}}\right)\\
f_{2}\left(2\right)\left(\frac{1-\mu}{\left(1-\mu_{1}\right)\left(1-\mu_{2}\right)}\right)&=&\mu_{2}\left(\frac{r}{1-\mu_{1}}\right)\\
\end{eqnarray*}
por tanto
\begin{eqnarray*}
f_{2}\left(2\right)&=&\frac{r\frac{\mu_{2}}{1-\mu_{1}}}{\frac{1-\mu}{\left(1-\mu_{1}\right)\left(1-\mu_{2}\right)}}=\frac{r\mu_{2}\left(1-\mu_{1}\right)\left(1-\mu_{2}\right)}{\left(1-\mu_{1}\right)\left(1-\mu\right)}\\
&=&\frac{\mu_{2}\left(1-\mu_{2}\right)}{1-\mu}r=r\mu_{2}\frac{1-\mu_{2}}{1-\mu}.
\end{eqnarray*}
es decir

\begin{eqnarray}
f_{2}\left(2\right)&=&r\mu_{2}\frac{1-\mu_{2}}{1-\mu}.
\end{eqnarray}

Entonces

\begin{eqnarray*}
f_{1}\left(1\right)&=&\mu_{1}r+f_{2}\left(2\right)\frac{\mu_{1}}{1-\mu_{2}}=\mu_{1}r+\left(\frac{\mu_{2}\left(1-\mu_{2}\right)}{1-\mu}r\right)\frac{\mu_{1}}{1-\mu_{2}}\\
&=&\mu_{1}r+\mu_{1}r\left(\frac{\mu_{2}}{1-\mu}\right)=\mu_{1}r\left[1+\frac{\mu_{2}}{1-\mu}\right]\\
&=&r\mu_{1}\frac{1-\mu_{1}}{1-\mu}\\
\end{eqnarray*}


