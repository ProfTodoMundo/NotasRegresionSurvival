\section{Introducci\'on}
Este cap\'itulo profundiza en la definici\'on y propiedades de la funci\'on de supervivencia y la funci\'on de riesgo, dos conceptos fundamentales en el análisis de supervivencia. Entender estas funciones y su relaci\'on es crucial para modelar y analizar datos de tiempo hasta evento.

\section{Funci\'on de Supervivencia}
La funci\'on de supervivencia, $S(t)$, describe la probabilidad de que un individuo sobreviva más allá de un tiempo $t$. Formalmente, se define como:
\begin{eqnarray*}
S(t) = P(T > t)
\end{eqnarray*}
donde $T$ es una variable aleatoria que representa el tiempo hasta el evento.

\subsection{Propiedades de la Funci\'on de Supervivencia}
La funci\'on de supervivencia tiene varias propiedades importantes:
\begin{itemize}
    \item $S(0) = 1$: Indica que la probabilidad de haber experimentado el evento en el tiempo 0 es cero.
    \item $\lim_{t \to \infty} S(t) = 0$: A medida que el tiempo tiende al infinito, la probabilidad de supervivencia tiende a cero.
    \item $S(t)$ es una funci\'on no creciente: A medida que el tiempo avanza, la probabilidad de supervivencia no aumenta.
\end{itemize}

\subsection{Derivaci\'on de $S(t)$}
Si la funci\'on de densidad de probabilidad $f(t)$ del tiempo de supervivencia $T$ es conocida, la funci\'on de supervivencia puede derivarse como:
\begin{eqnarray*}
S(t) &=& P(T > t) \\
     &=& 1 - P(T \leq t) \\
     &=& 1 - F(t) \\
     &=& 1 - \int_0^t f(u) \, du
\end{eqnarray*}
donde $F(t)$ es la funci\'on de distribuci\'on acumulada.

\subsection{Ejemplo de Cálculo de $S(t)$}
Consideremos un ejemplo donde el tiempo de supervivencia $T$ sigue una distribuci\'on exponencial con tasa $\lambda$. La funci\'on de densidad de probabilidad $f(t)$ es:
\begin{eqnarray*}
f(t) = \lambda e^{-\lambda t}, \quad t \geq 0
\end{eqnarray*}
La funci\'on de distribuci\'on acumulada $F(t)$ es:
\begin{eqnarray*}
F(t) = \int_0^t \lambda e^{-\lambda u} \, du = 1 - e^{-\lambda t}
\end{eqnarray*}
Por lo tanto, la funci\'on de supervivencia $S(t)$ es:
\begin{eqnarray*}
S(t) = 1 - F(t) = e^{-\lambda t}
\end{eqnarray*}

\section{Funci\'on de Riesgo}
La funci\'on de riesgo, $\lambda(t)$, proporciona la tasa instant\'anea de ocurrencia del evento en el tiempo $t$, dado que el individuo ha sobrevivido hasta el tiempo $t$. Matem\'aticamente, se define como:
\begin{eqnarray*}
\lambda(t) = \lim_{\Delta t \to 0} \frac{P(t \leq T < t + \Delta t \mid T \geq t)}{\Delta t}
\end{eqnarray*}

\subsection{Relaci\'on entre $\lambda(t)$ y $f(t)$}
La funci\'on de riesgo se puede relacionar con la funci\'on de densidad de probabilidad $f(t)$ y la funci\'on de supervivencia $S(t)$ de la siguiente manera:
\begin{eqnarray*}
\lambda(t) &=& \frac{f(t)}{S(t)}
\end{eqnarray*}

\subsection{Derivaci\'on de $\lambda(t)$}
La derivaci\'on de $\lambda(t)$ se basa en la definici\'on condicional de la probabilidad:
\begin{eqnarray*}
\lambda(t) &=& \lim_{\Delta t \to 0} \frac{P(t \leq T < t + \Delta t \mid T \geq t)}{\Delta t} \\
           &=& \lim_{\Delta t \to 0} \frac{\frac{P(t \leq T < t + \Delta t \text{ y } T \geq t)}{P(T \geq t)}}{\Delta t} \\
           &=& \lim_{\Delta t \to 0} \frac{\frac{P(t \leq T < t + \Delta t)}{P(T \geq t)}}{\Delta t} \\
           &=& \frac{f(t)}{S(t)}
\end{eqnarray*}

\section{Relaci\'on entre Funci\'on de Supervivencia y Funci\'on de Riesgo}
La funci\'on de supervivencia y la funci\'on de riesgo est\'an estrechamente relacionadas. La relaci\'on se expresa mediante la siguiente ecuaci\'on:
\begin{eqnarray*}
S(t) = \exp\left(-\int_0^t \lambda(u) \, du\right)
\end{eqnarray*}

\subsection{Deducci\'on de la Relaci\'on}
Para deducir esta relaci\'on, consideramos la derivada logar\'itmica de la funci\'on de supervivencia:
\begin{eqnarray*}
S(t) &=& \exp\left(-\int_0^t \lambda(u) \, du\right) \\
\log S(t) &=& -\int_0^t \lambda(u) \, du \\
\frac{d}{dt} \log S(t) &=& -\lambda(t) \\
\lambda(t) &=& -\frac{d}{dt} \log S(t)
\end{eqnarray*}

\section{Interpretaci\'on de la Funci\'on de Riesgo}
La funci\'on de riesgo, $\lambda(t)$, se interpreta como la tasa instant\'anea de ocurrencia del evento por unidad de tiempo, dado que el individuo ha sobrevivido hasta el tiempo $t$. Es una medida local del riesgo de falla en un instante espec\'ifico.

\subsection{Ejemplo de C\'alculo de $\lambda(t)$}
Consideremos nuevamente el caso donde el tiempo de supervivencia $T$ sigue una distribuci\'on exponencial con tasa $\lambda$. La funci\'on de densidad de probabilidad $f(t)$ es:
\begin{eqnarray*}
f(t) = \lambda e^{-\lambda t}
\end{eqnarray*}
La funci\'on de supervivencia $S(t)$ es:
\begin{eqnarray*}
S(t) = e^{-\lambda t}
\end{eqnarray*}
La funci\'on de riesgo $\lambda(t)$ se calcula como:
\begin{eqnarray*}
\lambda(t) &=& \frac{f(t)}{S(t)} \\
           &=& \frac{\lambda e^{-\lambda t}}{e^{-\lambda t}} \\
           &=& \lambda
\end{eqnarray*}
En este caso, $\lambda(t)$ es constante y igual a $\lambda$, lo que es una caracter\'istica de la distribuci\'on exponencial.

\section{Funciones de Riesgo Acumulada y Media Residual}
La funci\'on de riesgo acumulada $H(t)$ se define como:
\begin{eqnarray*}
H(t) = \int_0^t \lambda(u) \, du
\end{eqnarray*}
Esta funci\'on proporciona la suma acumulada de la tasa de riesgo hasta el tiempo $t$.

La funci\'on de vida media residual $e(t)$ se define como la esperanza del tiempo de vida restante dado que el individuo ha sobrevivido hasta el tiempo $t$:
\begin{eqnarray*}
e(t) = \mathbb{E}[T - t \mid T > t] = \int_t^\infty S(u) \, du
\end{eqnarray*}

\section{Ejemplo de C\'alculo de Funci\'on de Riesgo Acumulada y Vida Media Residual}
Consideremos nuevamente la distribuci\'on exponencial con tasa $\lambda$. La funci\'on de riesgo acumulada $H(t)$ es:
\begin{eqnarray*}
H(t) &=& \int_0^t \lambda \, du \\
     &=& \lambda t
\end{eqnarray*}

La funci\'on de vida media residual $e(t)$ es:
\begin{eqnarray*}
e(t) &=& \int_t^\infty e^{-\lambda u} \, du \\
     &=& \left[ \frac{-1}{\lambda} e^{-\lambda u} \right]_t^\infty \\
     &=& \frac{1}{\lambda} e^{-\lambda t} \\
     &=& \frac{1}{\lambda}
\end{eqnarray*}
En este caso, la vida media residual es constante e igual a $\frac{1}{\lambda}$, otra caracter\'istica de la distribuci\'on exponencial.

\section{Conclusi\'on}
La funci\'on de supervivencia y la funci\'on de riesgo son herramientas fundamentales en el an\'alisis de supervivencia. Entender su definici\'on, propiedades, y la relaci\'on entre ellas es esencial para modelar y analizar correctamente los datos de tiempo hasta evento. Las funciones de riesgo acumulada y vida media residual proporcionan informaci\'on adicional sobre la din\'amica del riesgo a lo largo del tiempo.

