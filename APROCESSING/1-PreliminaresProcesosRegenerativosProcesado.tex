

\subsection*{Procesos Regenerativos: Sigman\cite{Sigman1}}
\begin{Def}[Definici\'on Cl\'asica]
Un proceso estoc\'astico $X=\left\{X\left(t\right):t\geq0\right\}$ es llamado regenerativo is existe una variable aleatoria $R_{1}>0$ tal que
\begin{itemize}
\item[i)] $\left\{X\left(t+R_{1}\right):t\geq0\right\}$ es independiente de $\left\{\left\{X\left(t\right):t<R_{1}\right\},\right\}$
\item[ii)] $\left\{X\left(t+R_{1}\right):t\geq0\right\}$ es estoc\'asticamente equivalente a $\left\{X\left(t\right):t>0\right\}$
\end{itemize}

Llamamos a $R_{1}$ tiempo de regeneraci\'on, y decimos que $X$ se regenera en este punto.
\end{Def}

$\left\{X\left(t+R_{1}\right)\right\}$ es regenerativo con tiempo de regeneraci\'on $R_{2}$, independiente de $R_{1}$ pero con la misma distribuci\'on que $R_{1}$. Procediendo de esta manera se obtiene una secuencia de variables aleatorias independientes e id\'enticamente distribuidas $\left\{R_{n}\right\}$ llamados longitudes de ciclo. Si definimos a $Z_{k}\equiv R_{1}+R_{2}+\cdots+R_{k}$, se tiene un proceso de renovaci\'on llamado proceso de renovaci\'on encajado para $X$.




\begin{Def}
Para $x$ fijo y para cada $t\geq0$, sea $I_{x}\left(t\right)=1$ si $X\left(t\right)\leq x$,  $I_{x}\left(t\right)=0$ en caso contrario, y def\'inanse los tiempos promedio
\begin{eqnarray*}
\overline{X}&=&lim_{t\rightarrow\infty}\frac{1}{t}\int_{0}^{\infty}X\left(u\right)du\\
\prob\left(X_{\infty}\leq x\right)&=&lim_{t\rightarrow\infty}\frac{1}{t}\int_{0}^{\infty}I_{x}\left(u\right)du,
\end{eqnarray*}
cuando estos l\'imites existan.
\end{Def}

Como consecuencia del teorema de Renovaci\'on-Recompensa, se tiene que el primer l\'imite  existe y es igual a la constante
\begin{eqnarray*}
\overline{X}&=&\frac{\esp\left[\int_{0}^{R_{1}}X\left(t\right)dt\right]}{\esp\left[R_{1}\right]},
\end{eqnarray*}
suponiendo que ambas esperanzas son finitas.

\begin{Note}
\begin{itemize}
\item[a)] Si el proceso regenerativo $X$ es positivo recurrente y tiene trayectorias muestrales no negativas, entonces la ecuaci\'on anterior es v\'alida.
\item[b)] Si $X$ es positivo recurrente regenerativo, podemos construir una \'unica versi\'on estacionaria de este proceso, $X_{e}=\left\{X_{e}\left(t\right)\right\}$, donde $X_{e}$ es un proceso estoc\'astico regenerativo y estrictamente estacionario, con distribuci\'on marginal distribuida como $X_{\infty}$
\end{itemize}
\end{Note}



%______________________________________________________________________
\subsubsection{Procesos de Renovaci\'on}
%______________________________________________________________________

\begin{Def}%\label{Def.Tn}
Sean $0\leq T_{1}\leq T_{2}\leq \ldots$ son tiempos aleatorios infinitos en los cuales ocurren ciertos eventos. El n\'umero de tiempos $T_{n}$ en el intervalo $\left[0,t\right)$ es

\begin{eqnarray}
N\left(t\right)=\sum_{n=1}^{\infty}\indora\left(T_{n}\leq t\right),
\end{eqnarray}
para $t\geq0$.
\end{Def}

Si se consideran los puntos $T_{n}$ como elementos de $\rea_{+}$, y $N\left(t\right)$ es el n\'umero de puntos en $\rea$. El proceso denotado por $\left\{N\left(t\right):t\geq0\right\}$, denotado por $N\left(t\right)$, es un proceso puntual en $\rea_{+}$. Los $T_{n}$ son los tiempos de ocurrencia, el proceso puntual $N\left(t\right)$ es simple si su n\'umero de ocurrencias son distintas: $0<T_{1}<T_{2}<\ldots$ casi seguramente.

\begin{Def}
Un proceso puntual $N\left(t\right)$ es un proceso de renovaci\'on si los tiempos de interocurrencia $\xi_{n}=T_{n}-T_{n-1}$, para $n\geq1$, son independientes e identicamente distribuidos con distribuci\'on $F$, donde $F\left(0\right)=0$ y $T_{0}=0$. Los $T_{n}$ son llamados tiempos de renovaci\'on, referente a la independencia o renovaci\'on de la informaci\'on estoc\'astica en estos tiempos. Los $\xi_{n}$ son los tiempos de inter-renovaci\'on, y $N\left(t\right)$ es el n\'umero de renovaciones en el intervalo $\left[0,t\right)$
\end{Def}


\begin{Note}
Para definir un proceso de renovaci\'on para cualquier contexto, solamente hay que especificar una distribuci\'on $F$, con $F\left(0\right)=0$, para los tiempos de inter-renovaci\'on. La funci\'on $F$ en turno degune las otra variables aleatorias. De manera formal, existe un espacio de probabilidad y una sucesi\'on de variables aleatorias $\xi_{1},\xi_{2},\ldots$ definidas en este con distribuci\'on $F$. Entonces las otras cantidades son $T_{n}=\sum_{k=1}^{n}\xi_{k}$ y $N\left(t\right)=\sum_{n=1}^{\infty}\indora\left(T_{n}\leq t\right)$, donde $T_{n}\rightarrow\infty$ casi seguramente por la Ley Fuerte de los Grandes Números.
\end{Note}

%___________________________________________________________________________________________
%
\subsubsection{Teorema Principal de Renovaci\'on}
%___________________________________________________________________________________________
%

\begin{Note} Una funci\'on $h:\rea_{+}\rightarrow\rea$ es Directamente Riemann Integrable en los siguientes casos:
\begin{itemize}
\item[a)] $h\left(t\right)\geq0$ es decreciente y Riemann Integrable.
\item[b)] $h$ es continua excepto posiblemente en un conjunto de Lebesgue de medida 0, y $|h\left(t\right)|\leq b\left(t\right)$, donde $b$ es DRI.
\end{itemize}
\end{Note}

\begin{Teo}[Teorema Principal de Renovaci\'on]
Si $F$ es no aritm\'etica y $h\left(t\right)$ es Directamente Riemann Integrable (DRI), entonces

\begin{eqnarray*}
lim_{t\rightarrow\infty}U\star h=\frac{1}{\mu}\int_{\rea_{+}}h\left(s\right)ds.
\end{eqnarray*}
\end{Teo}

\begin{Prop}
Cualquier funci\'on $H\left(t\right)$ acotada en intervalos finitos y que es 0 para $t<0$ puede expresarse como
\begin{eqnarray*}
H\left(t\right)=U\star h\left(t\right)\textrm{,  donde }h\left(t\right)=H\left(t\right)-F\star H\left(t\right)
\end{eqnarray*}
\end{Prop}

\begin{Def}
Un proceso estoc\'astico $X\left(t\right)$ es crudamente regenerativo en un tiempo aleatorio positivo $T$ si
\begin{eqnarray*}
\esp\left[X\left(T+t\right)|T\right]=\esp\left[X\left(t\right)\right]\textrm{, para }t\geq0,\end{eqnarray*}
y con las esperanzas anteriores finitas.
\end{Def}

\begin{Prop}
Sup\'ongase que $X\left(t\right)$ es un proceso crudamente regenerativo en $T$, que tiene distribuci\'on $F$. Si $\esp\left[X\left(t\right)\right]$ es acotado en intervalos finitos, entonces
\begin{eqnarray*}
\esp\left[X\left(t\right)\right]=U\star h\left(t\right)\textrm{,  donde }h\left(t\right)=\esp\left[X\left(t\right)\indora\left(T>t\right)\right].
\end{eqnarray*}
\end{Prop}

\begin{Teo}[Regeneraci\'on Cruda]
Sup\'ongase que $X\left(t\right)$ es un proceso con valores positivo crudamente regenerativo en $T$, y def\'inase $M=\sup\left\{|X\left(t\right)|:t\leq T\right\}$. Si $T$ es no aritm\'etico y $M$ y $MT$ tienen media finita, entonces
\begin{eqnarray*}
lim_{t\rightarrow\infty}\esp\left[X\left(t\right)\right]=\frac{1}{\mu}\int_{\rea_{+}}h\left(s\right)ds,
\end{eqnarray*}
donde $h\left(t\right)=\esp\left[X\left(t\right)\indora\left(T>t\right)\right]$.
\end{Teo}

%___________________________________________________________________________________________
%
\subsubsection{Propiedades de los Procesos de Renovaci\'on}
%___________________________________________________________________________________________
%

Los tiempos $T_{n}$ est\'an relacionados con los conteos de $N\left(t\right)$ por

\begin{eqnarray*}
\left\{N\left(t\right)\geq n\right\}&=&\left\{T_{n}\leq t\right\}\\
T_{N\left(t\right)}\leq &t&<T_{N\left(t\right)+1},
\end{eqnarray*}

adem\'as $N\left(T_{n}\right)=n$, y 

\begin{eqnarray*}
N\left(t\right)=\max\left\{n:T_{n}\leq t\right\}=\min\left\{n:T_{n+1}>t\right\}
\end{eqnarray*}

Por propiedades de la convoluci\'on se sabe que

\begin{eqnarray*}
P\left\{T_{n}\leq t\right\}=F^{n\star}\left(t\right)
\end{eqnarray*}
que es la $n$-\'esima convoluci\'on de $F$. Entonces 

\begin{eqnarray*}
\left\{N\left(t\right)\geq n\right\}&=&\left\{T_{n}\leq t\right\}\\
P\left\{N\left(t\right)\leq n\right\}&=&1-F^{\left(n+1\right)\star}\left(t\right)
\end{eqnarray*}

Adem\'as usando el hecho de que $\esp\left[N\left(t\right)\right]=\sum_{n=1}^{\infty}P\left\{N\left(t\right)\geq n\right\}$
se tiene que

\begin{eqnarray*}
\esp\left[N\left(t\right)\right]=\sum_{n=1}^{\infty}F^{n\star}\left(t\right)
\end{eqnarray*}

\begin{Prop}
Para cada $t\geq0$, la funci\'on generadora de momentos $\esp\left[e^{\alpha N\left(t\right)}\right]$ existe para alguna $\alpha$ en una vecindad del 0, y de aqu\'i que $\esp\left[N\left(t\right)^{m}\right]<\infty$, para $m\geq1$.
\end{Prop}


\begin{Note}
Si el primer tiempo de renovaci\'on $\xi_{1}$ no tiene la misma distribuci\'on que el resto de las $\xi_{n}$, para $n\geq2$, a $N\left(t\right)$ se le llama Proceso de Renovaci\'on retardado, donde si $\xi$ tiene distribuci\'on $G$, entonces el tiempo $T_{n}$ de la $n$-\'esima renovaci\'on tiene distribuci\'on $G\star F^{\left(n-1\right)\star}\left(t\right)$
\end{Note}


\begin{Teo}
Para una constante $\mu\leq\infty$ ( o variable aleatoria), las siguientes expresiones son equivalentes:

\begin{eqnarray}
lim_{n\rightarrow\infty}n^{-1}T_{n}&=&\mu,\textrm{ c.s.}\\
lim_{t\rightarrow\infty}t^{-1}N\left(t\right)&=&1/\mu,\textrm{ c.s.}
\end{eqnarray}
\end{Teo}


Es decir, $T_{n}$ satisface la Ley Fuerte de los Grandes N\'umeros s\'i y s\'olo s\'i $N\left/t\right)$ la cumple.


\begin{Coro}[Ley Fuerte de los Grandes N\'umeros para Procesos de Renovaci\'on]
Si $N\left(t\right)$ es un proceso de renovaci\'on cuyos tiempos de inter-renovaci\'on tienen media $\mu\leq\infty$, entonces
\begin{eqnarray}
t^{-1}N\left(t\right)\rightarrow 1/\mu,\textrm{ c.s. cuando }t\rightarrow\infty.
\end{eqnarray}

\end{Coro}


Considerar el proceso estoc\'astico de valores reales $\left\{Z\left(t\right):t\geq0\right\}$ en el mismo espacio de probabilidad que $N\left(t\right)$

\begin{Def}
Para el proceso $\left\{Z\left(t\right):t\geq0\right\}$ se define la fluctuaci\'on m\'axima de $Z\left(t\right)$ en el intervalo $\left(T_{n-1},T_{n}\right]$:
\begin{eqnarray*}
M_{n}=\sup_{T_{n-1}<t\leq T_{n}}|Z\left(t\right)-Z\left(T_{n-1}\right)|
\end{eqnarray*}
\end{Def}

\begin{Teo}
Sup\'ongase que $n^{-1}T_{n}\rightarrow\mu$ c.s. cuando $n\rightarrow\infty$, donde $\mu\leq\infty$ es una constante o variable aleatoria. Sea $a$ una constante o variable aleatoria que puede ser infinita cuando $\mu$ es finita, y considere las expresiones l\'imite:
\begin{eqnarray}
lim_{n\rightarrow\infty}n^{-1}Z\left(T_{n}\right)&=&a,\textrm{ c.s.}\\
lim_{t\rightarrow\infty}t^{-1}Z\left(t\right)&=&a/\mu,\textrm{ c.s.}
\end{eqnarray}
La segunda expresi\'on implica la primera. Conversamente, la primera implica la segunda si el proceso $Z\left(t\right)$ es creciente, o si $lim_{n\rightarrow\infty}n^{-1}M_{n}=0$ c.s.
\end{Teo}

\begin{Coro}
Si $N\left(t\right)$ es un proceso de renovaci\'on, y $\left(Z\left(T_{n}\right)-Z\left(T_{n-1}\right),M_{n}\right)$, para $n\geq1$, son variables aleatorias independientes e id\'enticamente distribuidas con media finita, entonces,
\begin{eqnarray}
lim_{t\rightarrow\infty}t^{-1}Z\left(t\right)\rightarrow\frac{\esp\left[Z\left(T_{1}\right)-Z\left(T_{0}\right)\right]}{\esp\left[T_{1}\right]},\textrm{ c.s. cuando  }t\rightarrow\infty.
\end{eqnarray}
\end{Coro}



%___________________________________________________________________________________________
%
\subsection{Propiedades de los Procesos de Renovaci\'on}
%___________________________________________________________________________________________
%

Los tiempos $T_{n}$ est\'an relacionados con los conteos de $N\left(t\right)$ por

\begin{eqnarray*}
\left\{N\left(t\right)\geq n\right\}&=&\left\{T_{n}\leq t\right\}\\
T_{N\left(t\right)}\leq &t&<T_{N\left(t\right)+1},
\end{eqnarray*}

adem\'as $N\left(T_{n}\right)=n$, y 

\begin{eqnarray*}
N\left(t\right)=\max\left\{n:T_{n}\leq t\right\}=\min\left\{n:T_{n+1}>t\right\}
\end{eqnarray*}

Por propiedades de la convoluci\'on se sabe que

\begin{eqnarray*}
P\left\{T_{n}\leq t\right\}=F^{n\star}\left(t\right)
\end{eqnarray*}
que es la $n$-\'esima convoluci\'on de $F$. Entonces 

\begin{eqnarray*}
\left\{N\left(t\right)\geq n\right\}&=&\left\{T_{n}\leq t\right\}\\
P\left\{N\left(t\right)\leq n\right\}&=&1-F^{\left(n+1\right)\star}\left(t\right)
\end{eqnarray*}

Adem\'as usando el hecho de que $\esp\left[N\left(t\right)\right]=\sum_{n=1}^{\infty}P\left\{N\left(t\right)\geq n\right\}$
se tiene que

\begin{eqnarray*}
\esp\left[N\left(t\right)\right]=\sum_{n=1}^{\infty}F^{n\star}\left(t\right)
\end{eqnarray*}

\begin{Prop}
Para cada $t\geq0$, la funci\'on generadora de momentos $\esp\left[e^{\alpha N\left(t\right)}\right]$ existe para alguna $\alpha$ en una vecindad del 0, y de aqu\'i que $\esp\left[N\left(t\right)^{m}\right]<\infty$, para $m\geq1$.
\end{Prop}


\begin{Note}
Si el primer tiempo de renovaci\'on $\xi_{1}$ no tiene la misma distribuci\'on que el resto de las $\xi_{n}$, para $n\geq2$, a $N\left(t\right)$ se le llama Proceso de Renovaci\'on retardado, donde si $\xi$ tiene distribuci\'on $G$, entonces el tiempo $T_{n}$ de la $n$-\'esima renovaci\'on tiene distribuci\'on $G\star F^{\left(n-1\right)\star}\left(t\right)$
\end{Note}


\begin{Teo}
Para una constante $\mu\leq\infty$ ( o variable aleatoria), las siguientes expresiones son equivalentes:

\begin{eqnarray}
lim_{n\rightarrow\infty}n^{-1}T_{n}&=&\mu,\textrm{ c.s.}\\
lim_{t\rightarrow\infty}t^{-1}N\left(t\right)&=&1/\mu,\textrm{ c.s.}
\end{eqnarray}
\end{Teo}


Es decir, $T_{n}$ satisface la Ley Fuerte de los Grandes N\'umeros s\'i y s\'olo s\'i $N\left/t\right)$ la cumple.


\begin{Coro}[Ley Fuerte de los Grandes N\'umeros para Procesos de Renovaci\'on]
Si $N\left(t\right)$ es un proceso de renovaci\'on cuyos tiempos de inter-renovaci\'on tienen media $\mu\leq\infty$, entonces
\begin{eqnarray}
t^{-1}N\left(t\right)\rightarrow 1/\mu,\textrm{ c.s. cuando }t\rightarrow\infty.
\end{eqnarray}

\end{Coro}


Considerar el proceso estoc\'astico de valores reales $\left\{Z\left(t\right):t\geq0\right\}$ en el mismo espacio de probabilidad que $N\left(t\right)$

\begin{Def}
Para el proceso $\left\{Z\left(t\right):t\geq0\right\}$ se define la fluctuaci\'on m\'axima de $Z\left(t\right)$ en el intervalo $\left(T_{n-1},T_{n}\right]$:
\begin{eqnarray*}
M_{n}=\sup_{T_{n-1}<t\leq T_{n}}|Z\left(t\right)-Z\left(T_{n-1}\right)|
\end{eqnarray*}
\end{Def}

\begin{Teo}
Sup\'ongase que $n^{-1}T_{n}\rightarrow\mu$ c.s. cuando $n\rightarrow\infty$, donde $\mu\leq\infty$ es una constante o variable aleatoria. Sea $a$ una constante o variable aleatoria que puede ser infinita cuando $\mu$ es finita, y considere las expresiones l\'imite:
\begin{eqnarray}
lim_{n\rightarrow\infty}n^{-1}Z\left(T_{n}\right)&=&a,\textrm{ c.s.}\\
lim_{t\rightarrow\infty}t^{-1}Z\left(t\right)&=&a/\mu,\textrm{ c.s.}
\end{eqnarray}
La segunda expresi\'on implica la primera. Conversamente, la primera implica la segunda si el proceso $Z\left(t\right)$ es creciente, o si $lim_{n\rightarrow\infty}n^{-1}M_{n}=0$ c.s.
\end{Teo}

\begin{Coro}
Si $N\left(t\right)$ es un proceso de renovaci\'on, y $\left(Z\left(T_{n}\right)-Z\left(T_{n-1}\right),M_{n}\right)$, para $n\geq1$, son variables aleatorias independientes e id\'enticamente distribuidas con media finita, entonces,
\begin{eqnarray}
lim_{t\rightarrow\infty}t^{-1}Z\left(t\right)\rightarrow\frac{\esp\left[Z\left(T_{1}\right)-Z\left(T_{0}\right)\right]}{\esp\left[T_{1}\right]},\textrm{ c.s. cuando  }t\rightarrow\infty.
\end{eqnarray}
\end{Coro}


%___________________________________________________________________________________________
%
\subsection{Propiedades de los Procesos de Renovaci\'on}
%___________________________________________________________________________________________
%

Los tiempos $T_{n}$ est\'an relacionados con los conteos de $N\left(t\right)$ por

\begin{eqnarray*}
\left\{N\left(t\right)\geq n\right\}&=&\left\{T_{n}\leq t\right\}\\
T_{N\left(t\right)}\leq &t&<T_{N\left(t\right)+1},
\end{eqnarray*}

adem\'as $N\left(T_{n}\right)=n$, y 

\begin{eqnarray*}
N\left(t\right)=\max\left\{n:T_{n}\leq t\right\}=\min\left\{n:T_{n+1}>t\right\}
\end{eqnarray*}

Por propiedades de la convoluci\'on se sabe que

\begin{eqnarray*}
P\left\{T_{n}\leq t\right\}=F^{n\star}\left(t\right)
\end{eqnarray*}
que es la $n$-\'esima convoluci\'on de $F$. Entonces 

\begin{eqnarray*}
\left\{N\left(t\right)\geq n\right\}&=&\left\{T_{n}\leq t\right\}\\
P\left\{N\left(t\right)\leq n\right\}&=&1-F^{\left(n+1\right)\star}\left(t\right)
\end{eqnarray*}

Adem\'as usando el hecho de que $\esp\left[N\left(t\right)\right]=\sum_{n=1}^{\infty}P\left\{N\left(t\right)\geq n\right\}$
se tiene que

\begin{eqnarray*}
\esp\left[N\left(t\right)\right]=\sum_{n=1}^{\infty}F^{n\star}\left(t\right)
\end{eqnarray*}

\begin{Prop}
Para cada $t\geq0$, la funci\'on generadora de momentos $\esp\left[e^{\alpha N\left(t\right)}\right]$ existe para alguna $\alpha$ en una vecindad del 0, y de aqu\'i que $\esp\left[N\left(t\right)^{m}\right]<\infty$, para $m\geq1$.
\end{Prop}


\begin{Note}
Si el primer tiempo de renovaci\'on $\xi_{1}$ no tiene la misma distribuci\'on que el resto de las $\xi_{n}$, para $n\geq2$, a $N\left(t\right)$ se le llama Proceso de Renovaci\'on retardado, donde si $\xi$ tiene distribuci\'on $G$, entonces el tiempo $T_{n}$ de la $n$-\'esima renovaci\'on tiene distribuci\'on $G\star F^{\left(n-1\right)\star}\left(t\right)$
\end{Note}


\begin{Teo}
Para una constante $\mu\leq\infty$ ( o variable aleatoria), las siguientes expresiones son equivalentes:

\begin{eqnarray}
lim_{n\rightarrow\infty}n^{-1}T_{n}&=&\mu,\textrm{ c.s.}\\
lim_{t\rightarrow\infty}t^{-1}N\left(t\right)&=&1/\mu,\textrm{ c.s.}
\end{eqnarray}
\end{Teo}


Es decir, $T_{n}$ satisface la Ley Fuerte de los Grandes N\'umeros s\'i y s\'olo s\'i $N\left/t\right)$ la cumple.


\begin{Coro}[Ley Fuerte de los Grandes N\'umeros para Procesos de Renovaci\'on]
Si $N\left(t\right)$ es un proceso de renovaci\'on cuyos tiempos de inter-renovaci\'on tienen media $\mu\leq\infty$, entonces
\begin{eqnarray}
t^{-1}N\left(t\right)\rightarrow 1/\mu,\textrm{ c.s. cuando }t\rightarrow\infty.
\end{eqnarray}

\end{Coro}


Considerar el proceso estoc\'astico de valores reales $\left\{Z\left(t\right):t\geq0\right\}$ en el mismo espacio de probabilidad que $N\left(t\right)$

\begin{Def}
Para el proceso $\left\{Z\left(t\right):t\geq0\right\}$ se define la fluctuaci\'on m\'axima de $Z\left(t\right)$ en el intervalo $\left(T_{n-1},T_{n}\right]$:
\begin{eqnarray*}
M_{n}=\sup_{T_{n-1}<t\leq T_{n}}|Z\left(t\right)-Z\left(T_{n-1}\right)|
\end{eqnarray*}
\end{Def}

\begin{Teo}
Sup\'ongase que $n^{-1}T_{n}\rightarrow\mu$ c.s. cuando $n\rightarrow\infty$, donde $\mu\leq\infty$ es una constante o variable aleatoria. Sea $a$ una constante o variable aleatoria que puede ser infinita cuando $\mu$ es finita, y considere las expresiones l\'imite:
\begin{eqnarray}
lim_{n\rightarrow\infty}n^{-1}Z\left(T_{n}\right)&=&a,\textrm{ c.s.}\\
lim_{t\rightarrow\infty}t^{-1}Z\left(t\right)&=&a/\mu,\textrm{ c.s.}
\end{eqnarray}
La segunda expresi\'on implica la primera. Conversamente, la primera implica la segunda si el proceso $Z\left(t\right)$ es creciente, o si $lim_{n\rightarrow\infty}n^{-1}M_{n}=0$ c.s.
\end{Teo}

\begin{Coro}
Si $N\left(t\right)$ es un proceso de renovaci\'on, y $\left(Z\left(T_{n}\right)-Z\left(T_{n-1}\right),M_{n}\right)$, para $n\geq1$, son variables aleatorias independientes e id\'enticamente distribuidas con media finita, entonces,
\begin{eqnarray}
lim_{t\rightarrow\infty}t^{-1}Z\left(t\right)\rightarrow\frac{\esp\left[Z\left(T_{1}\right)-Z\left(T_{0}\right)\right]}{\esp\left[T_{1}\right]},\textrm{ c.s. cuando  }t\rightarrow\infty.
\end{eqnarray}
\end{Coro}

%___________________________________________________________________________________________
%
\subsection{Propiedades de los Procesos de Renovaci\'on}
%___________________________________________________________________________________________
%

Los tiempos $T_{n}$ est\'an relacionados con los conteos de $N\left(t\right)$ por

\begin{eqnarray*}
\left\{N\left(t\right)\geq n\right\}&=&\left\{T_{n}\leq t\right\}\\
T_{N\left(t\right)}\leq &t&<T_{N\left(t\right)+1},
\end{eqnarray*}

adem\'as $N\left(T_{n}\right)=n$, y 

\begin{eqnarray*}
N\left(t\right)=\max\left\{n:T_{n}\leq t\right\}=\min\left\{n:T_{n+1}>t\right\}
\end{eqnarray*}

Por propiedades de la convoluci\'on se sabe que

\begin{eqnarray*}
P\left\{T_{n}\leq t\right\}=F^{n\star}\left(t\right)
\end{eqnarray*}
que es la $n$-\'esima convoluci\'on de $F$. Entonces 

\begin{eqnarray*}
\left\{N\left(t\right)\geq n\right\}&=&\left\{T_{n}\leq t\right\}\\
P\left\{N\left(t\right)\leq n\right\}&=&1-F^{\left(n+1\right)\star}\left(t\right)
\end{eqnarray*}

Adem\'as usando el hecho de que $\esp\left[N\left(t\right)\right]=\sum_{n=1}^{\infty}P\left\{N\left(t\right)\geq n\right\}$
se tiene que

\begin{eqnarray*}
\esp\left[N\left(t\right)\right]=\sum_{n=1}^{\infty}F^{n\star}\left(t\right)
\end{eqnarray*}

\begin{Prop}
Para cada $t\geq0$, la funci\'on generadora de momentos $\esp\left[e^{\alpha N\left(t\right)}\right]$ existe para alguna $\alpha$ en una vecindad del 0, y de aqu\'i que $\esp\left[N\left(t\right)^{m}\right]<\infty$, para $m\geq1$.
\end{Prop}


\begin{Note}
Si el primer tiempo de renovaci\'on $\xi_{1}$ no tiene la misma distribuci\'on que el resto de las $\xi_{n}$, para $n\geq2$, a $N\left(t\right)$ se le llama Proceso de Renovaci\'on retardado, donde si $\xi$ tiene distribuci\'on $G$, entonces el tiempo $T_{n}$ de la $n$-\'esima renovaci\'on tiene distribuci\'on $G\star F^{\left(n-1\right)\star}\left(t\right)$
\end{Note}


\begin{Teo}
Para una constante $\mu\leq\infty$ ( o variable aleatoria), las siguientes expresiones son equivalentes:

\begin{eqnarray}
lim_{n\rightarrow\infty}n^{-1}T_{n}&=&\mu,\textrm{ c.s.}\\
lim_{t\rightarrow\infty}t^{-1}N\left(t\right)&=&1/\mu,\textrm{ c.s.}
\end{eqnarray}
\end{Teo}


Es decir, $T_{n}$ satisface la Ley Fuerte de los Grandes N\'umeros s\'i y s\'olo s\'i $N\left/t\right)$ la cumple.


\begin{Coro}[Ley Fuerte de los Grandes N\'umeros para Procesos de Renovaci\'on]
Si $N\left(t\right)$ es un proceso de renovaci\'on cuyos tiempos de inter-renovaci\'on tienen media $\mu\leq\infty$, entonces
\begin{eqnarray}
t^{-1}N\left(t\right)\rightarrow 1/\mu,\textrm{ c.s. cuando }t\rightarrow\infty.
\end{eqnarray}

\end{Coro}


Considerar el proceso estoc\'astico de valores reales $\left\{Z\left(t\right):t\geq0\right\}$ en el mismo espacio de probabilidad que $N\left(t\right)$

\begin{Def}
Para el proceso $\left\{Z\left(t\right):t\geq0\right\}$ se define la fluctuaci\'on m\'axima de $Z\left(t\right)$ en el intervalo $\left(T_{n-1},T_{n}\right]$:
\begin{eqnarray*}
M_{n}=\sup_{T_{n-1}<t\leq T_{n}}|Z\left(t\right)-Z\left(T_{n-1}\right)|
\end{eqnarray*}
\end{Def}

\begin{Teo}
Sup\'ongase que $n^{-1}T_{n}\rightarrow\mu$ c.s. cuando $n\rightarrow\infty$, donde $\mu\leq\infty$ es una constante o variable aleatoria. Sea $a$ una constante o variable aleatoria que puede ser infinita cuando $\mu$ es finita, y considere las expresiones l\'imite:
\begin{eqnarray}
lim_{n\rightarrow\infty}n^{-1}Z\left(T_{n}\right)&=&a,\textrm{ c.s.}\\
lim_{t\rightarrow\infty}t^{-1}Z\left(t\right)&=&a/\mu,\textrm{ c.s.}
\end{eqnarray}
La segunda expresi\'on implica la primera. Conversamente, la primera implica la segunda si el proceso $Z\left(t\right)$ es creciente, o si $lim_{n\rightarrow\infty}n^{-1}M_{n}=0$ c.s.
\end{Teo}

\begin{Coro}
Si $N\left(t\right)$ es un proceso de renovaci\'on, y $\left(Z\left(T_{n}\right)-Z\left(T_{n-1}\right),M_{n}\right)$, para $n\geq1$, son variables aleatorias independientes e id\'enticamente distribuidas con media finita, entonces,
\begin{eqnarray}
lim_{t\rightarrow\infty}t^{-1}Z\left(t\right)\rightarrow\frac{\esp\left[Z\left(T_{1}\right)-Z\left(T_{0}\right)\right]}{\esp\left[T_{1}\right]},\textrm{ c.s. cuando  }t\rightarrow\infty.
\end{eqnarray}
\end{Coro}
%___________________________________________________________________________________________
%
\subsection{Propiedades de los Procesos de Renovaci\'on}
%___________________________________________________________________________________________
%

Los tiempos $T_{n}$ est\'an relacionados con los conteos de $N\left(t\right)$ por

\begin{eqnarray*}
\left\{N\left(t\right)\geq n\right\}&=&\left\{T_{n}\leq t\right\}\\
T_{N\left(t\right)}\leq &t&<T_{N\left(t\right)+1},
\end{eqnarray*}

adem\'as $N\left(T_{n}\right)=n$, y 

\begin{eqnarray*}
N\left(t\right)=\max\left\{n:T_{n}\leq t\right\}=\min\left\{n:T_{n+1}>t\right\}
\end{eqnarray*}

Por propiedades de la convoluci\'on se sabe que

\begin{eqnarray*}
P\left\{T_{n}\leq t\right\}=F^{n\star}\left(t\right)
\end{eqnarray*}
que es la $n$-\'esima convoluci\'on de $F$. Entonces 

\begin{eqnarray*}
\left\{N\left(t\right)\geq n\right\}&=&\left\{T_{n}\leq t\right\}\\
P\left\{N\left(t\right)\leq n\right\}&=&1-F^{\left(n+1\right)\star}\left(t\right)
\end{eqnarray*}

Adem\'as usando el hecho de que $\esp\left[N\left(t\right)\right]=\sum_{n=1}^{\infty}P\left\{N\left(t\right)\geq n\right\}$
se tiene que

\begin{eqnarray*}
\esp\left[N\left(t\right)\right]=\sum_{n=1}^{\infty}F^{n\star}\left(t\right)
\end{eqnarray*}

\begin{Prop}
Para cada $t\geq0$, la funci\'on generadora de momentos $\esp\left[e^{\alpha N\left(t\right)}\right]$ existe para alguna $\alpha$ en una vecindad del 0, y de aqu\'i que $\esp\left[N\left(t\right)^{m}\right]<\infty$, para $m\geq1$.
\end{Prop}


\begin{Note}
Si el primer tiempo de renovaci\'on $\xi_{1}$ no tiene la misma distribuci\'on que el resto de las $\xi_{n}$, para $n\geq2$, a $N\left(t\right)$ se le llama Proceso de Renovaci\'on retardado, donde si $\xi$ tiene distribuci\'on $G$, entonces el tiempo $T_{n}$ de la $n$-\'esima renovaci\'on tiene distribuci\'on $G\star F^{\left(n-1\right)\star}\left(t\right)$
\end{Note}


\begin{Teo}
Para una constante $\mu\leq\infty$ ( o variable aleatoria), las siguientes expresiones son equivalentes:

\begin{eqnarray}
lim_{n\rightarrow\infty}n^{-1}T_{n}&=&\mu,\textrm{ c.s.}\\
lim_{t\rightarrow\infty}t^{-1}N\left(t\right)&=&1/\mu,\textrm{ c.s.}
\end{eqnarray}
\end{Teo}


Es decir, $T_{n}$ satisface la Ley Fuerte de los Grandes N\'umeros s\'i y s\'olo s\'i $N\left/t\right)$ la cumple.


\begin{Coro}[Ley Fuerte de los Grandes N\'umeros para Procesos de Renovaci\'on]
Si $N\left(t\right)$ es un proceso de renovaci\'on cuyos tiempos de inter-renovaci\'on tienen media $\mu\leq\infty$, entonces
\begin{eqnarray}
t^{-1}N\left(t\right)\rightarrow 1/\mu,\textrm{ c.s. cuando }t\rightarrow\infty.
\end{eqnarray}

\end{Coro}


Considerar el proceso estoc\'astico de valores reales $\left\{Z\left(t\right):t\geq0\right\}$ en el mismo espacio de probabilidad que $N\left(t\right)$

\begin{Def}
Para el proceso $\left\{Z\left(t\right):t\geq0\right\}$ se define la fluctuaci\'on m\'axima de $Z\left(t\right)$ en el intervalo $\left(T_{n-1},T_{n}\right]$:
\begin{eqnarray*}
M_{n}=\sup_{T_{n-1}<t\leq T_{n}}|Z\left(t\right)-Z\left(T_{n-1}\right)|
\end{eqnarray*}
\end{Def}

\begin{Teo}
Sup\'ongase que $n^{-1}T_{n}\rightarrow\mu$ c.s. cuando $n\rightarrow\infty$, donde $\mu\leq\infty$ es una constante o variable aleatoria. Sea $a$ una constante o variable aleatoria que puede ser infinita cuando $\mu$ es finita, y considere las expresiones l\'imite:
\begin{eqnarray}
lim_{n\rightarrow\infty}n^{-1}Z\left(T_{n}\right)&=&a,\textrm{ c.s.}\\
lim_{t\rightarrow\infty}t^{-1}Z\left(t\right)&=&a/\mu,\textrm{ c.s.}
\end{eqnarray}
La segunda expresi\'on implica la primera. Conversamente, la primera implica la segunda si el proceso $Z\left(t\right)$ es creciente, o si $lim_{n\rightarrow\infty}n^{-1}M_{n}=0$ c.s.
\end{Teo}

\begin{Coro}
Si $N\left(t\right)$ es un proceso de renovaci\'on, y $\left(Z\left(T_{n}\right)-Z\left(T_{n-1}\right),M_{n}\right)$, para $n\geq1$, son variables aleatorias independientes e id\'enticamente distribuidas con media finita, entonces,
\begin{eqnarray}
lim_{t\rightarrow\infty}t^{-1}Z\left(t\right)\rightarrow\frac{\esp\left[Z\left(T_{1}\right)-Z\left(T_{0}\right)\right]}{\esp\left[T_{1}\right]},\textrm{ c.s. cuando  }t\rightarrow\infty.
\end{eqnarray}
\end{Coro}


%___________________________________________________________________________________________
%
\subsubsection{Funci\'on de Renovaci\'on}
%___________________________________________________________________________________________
%


\begin{Def}
Sea $h\left(t\right)$ funci\'on de valores reales en $\rea$ acotada en intervalos finitos e igual a cero para $t<0$ La ecuaci\'on de renovaci\'on para $h\left(t\right)$ y la distribuci\'on $F$ es

\begin{eqnarray}%\label{Ec.Renovacion}
H\left(t\right)=h\left(t\right)+\int_{\left[0,t\right]}H\left(t-s\right)dF\left(s\right)\textrm{,    }t\geq0,
\end{eqnarray}
donde $H\left(t\right)$ es una funci\'on de valores reales. Esto es $H=h+F\star H$. Decimos que $H\left(t\right)$ es soluci\'on de esta ecuaci\'on si satisface la ecuaci\'on, y es acotada en intervalos finitos e iguales a cero para $t<0$.
\end{Def}

\begin{Prop}
La funci\'on $U\star h\left(t\right)$ es la \'unica soluci\'on de la ecuaci\'on de renovaci\'on (\ref{Ec.Renovacion}).
\end{Prop}

\begin{Teo}[Teorema Renovaci\'on Elemental]
\begin{eqnarray*}
t^{-1}U\left(t\right)\rightarrow 1/\mu\textrm{,    cuando }t\rightarrow\infty.
\end{eqnarray*}
\end{Teo}

%___________________________________________________________________________________________
%
\subsection{Funci\'on de Renovaci\'on}
%___________________________________________________________________________________________
%


Sup\'ongase que $N\left(t\right)$ es un proceso de renovaci\'on con distribuci\'on $F$ con media finita $\mu$.

\begin{Def}
La funci\'on de renovaci\'on asociada con la distribuci\'on $F$, del proceso $N\left(t\right)$, es
\begin{eqnarray*}
U\left(t\right)=\sum_{n=1}^{\infty}F^{n\star}\left(t\right),\textrm{   }t\geq0,
\end{eqnarray*}
donde $F^{0\star}\left(t\right)=\indora\left(t\geq0\right)$.
\end{Def}


\begin{Prop}
Sup\'ongase que la distribuci\'on de inter-renovaci\'on $F$ tiene densidad $f$. Entonces $U\left(t\right)$ tambi\'en tiene densidad, para $t>0$, y es $U^{'}\left(t\right)=\sum_{n=0}^{\infty}f^{n\star}\left(t\right)$. Adem\'as
\begin{eqnarray*}
\prob\left\{N\left(t\right)>N\left(t-\right)\right\}=0\textrm{,   }t\geq0.
\end{eqnarray*}
\end{Prop}

\begin{Def}
La Transformada de Laplace-Stieljes de $F$ est\'a dada por

\begin{eqnarray*}
\hat{F}\left(\alpha\right)=\int_{\rea_{+}}e^{-\alpha t}dF\left(t\right)\textrm{,  }\alpha\geq0.
\end{eqnarray*}
\end{Def}

Entonces

\begin{eqnarray*}
\hat{U}\left(\alpha\right)=\sum_{n=0}^{\infty}\hat{F^{n\star}}\left(\alpha\right)=\sum_{n=0}^{\infty}\hat{F}\left(\alpha\right)^{n}=\frac{1}{1-\hat{F}\left(\alpha\right)}.
\end{eqnarray*}


\begin{Prop}
La Transformada de Laplace $\hat{U}\left(\alpha\right)$ y $\hat{F}\left(\alpha\right)$ determina una a la otra de manera \'unica por la relaci\'on $\hat{U}\left(\alpha\right)=\frac{1}{1-\hat{F}\left(\alpha\right)}$.
\end{Prop}


\begin{Note}
Un proceso de renovaci\'on $N\left(t\right)$ cuyos tiempos de inter-renovaci\'on tienen media finita, es un proceso Poisson con tasa $\lambda$ si y s\'olo s\'i $\esp\left[U\left(t\right)\right]=\lambda t$, para $t\geq0$.
\end{Note}


\begin{Teo}
Sea $N\left(t\right)$ un proceso puntual simple con puntos de localizaci\'on $T_{n}$ tal que $\eta\left(t\right)=\esp\left[N\left(\right)\right]$ es finita para cada $t$. Entonces para cualquier funci\'on $f:\rea_{+}\rightarrow\rea$,
\begin{eqnarray*}
\esp\left[\sum_{n=1}^{N\left(\right)}f\left(T_{n}\right)\right]=\int_{\left(0,t\right]}f\left(s\right)d\eta\left(s\right)\textrm{,  }t\geq0,
\end{eqnarray*}
suponiendo que la integral exista. Adem\'as si $X_{1},X_{2},\ldots$ son variables aleatorias definidas en el mismo espacio de probabilidad que el proceso $N\left(t\right)$ tal que $\esp\left[X_{n}|T_{n}=s\right]=f\left(s\right)$, independiente de $n$. Entonces
\begin{eqnarray*}
\esp\left[\sum_{n=1}^{N\left(t\right)}X_{n}\right]=\int_{\left(0,t\right]}f\left(s\right)d\eta\left(s\right)\textrm{,  }t\geq0,
\end{eqnarray*} 
suponiendo que la integral exista. 
\end{Teo}

\begin{Coro}[Identidad de Wald para Renovaciones]
Para el proceso de renovaci\'on $N\left(t\right)$,
\begin{eqnarray*}
\esp\left[T_{N\left(t\right)+1}\right]=\mu\esp\left[N\left(t\right)+1\right]\textrm{,  }t\geq0,
\end{eqnarray*}  
\end{Coro}

%______________________________________________________________________
\subsection{Procesos de Renovaci\'on}
%______________________________________________________________________

\begin{Def}%\label{Def.Tn}
Sean $0\leq T_{1}\leq T_{2}\leq \ldots$ son tiempos aleatorios infinitos en los cuales ocurren ciertos eventos. El n\'umero de tiempos $T_{n}$ en el intervalo $\left[0,t\right)$ es

\begin{eqnarray}
N\left(t\right)=\sum_{n=1}^{\infty}\indora\left(T_{n}\leq t\right),
\end{eqnarray}
para $t\geq0$.
\end{Def}

Si se consideran los puntos $T_{n}$ como elementos de $\rea_{+}$, y $N\left(t\right)$ es el n\'umero de puntos en $\rea$. El proceso denotado por $\left\{N\left(t\right):t\geq0\right\}$, denotado por $N\left(t\right)$, es un proceso puntual en $\rea_{+}$. Los $T_{n}$ son los tiempos de ocurrencia, el proceso puntual $N\left(t\right)$ es simple si su n\'umero de ocurrencias son distintas: $0<T_{1}<T_{2}<\ldots$ casi seguramente.

\begin{Def}
Un proceso puntual $N\left(t\right)$ es un proceso de renovaci\'on si los tiempos de interocurrencia $\xi_{n}=T_{n}-T_{n-1}$, para $n\geq1$, son independientes e identicamente distribuidos con distribuci\'on $F$, donde $F\left(0\right)=0$ y $T_{0}=0$. Los $T_{n}$ son llamados tiempos de renovaci\'on, referente a la independencia o renovaci\'on de la informaci\'on estoc\'astica en estos tiempos. Los $\xi_{n}$ son los tiempos de inter-renovaci\'on, y $N\left(t\right)$ es el n\'umero de renovaciones en el intervalo $\left[0,t\right)$
\end{Def}


\begin{Note}
Para definir un proceso de renovaci\'on para cualquier contexto, solamente hay que especificar una distribuci\'on $F$, con $F\left(0\right)=0$, para los tiempos de inter-renovaci\'on. La funci\'on $F$ en turno degune las otra variables aleatorias. De manera formal, existe un espacio de probabilidad y una sucesi\'on de variables aleatorias $\xi_{1},\xi_{2},\ldots$ definidas en este con distribuci\'on $F$. Entonces las otras cantidades son $T_{n}=\sum_{k=1}^{n}\xi_{k}$ y $N\left(t\right)=\sum_{n=1}^{\infty}\indora\left(T_{n}\leq t\right)$, donde $T_{n}\rightarrow\infty$ casi seguramente por la Ley Fuerte de los Grandes Números.
\end{Note}

%___________________________________________________________________________________________
%
\subsection{Renewal and Regenerative Processes: Serfozo\cite{Serfozo}}
%___________________________________________________________________________________________
%
\begin{Def}%\label{Def.Tn}
Sean $0\leq T_{1}\leq T_{2}\leq \ldots$ son tiempos aleatorios infinitos en los cuales ocurren ciertos eventos. El n\'umero de tiempos $T_{n}$ en el intervalo $\left[0,t\right)$ es

\begin{eqnarray}
N\left(t\right)=\sum_{n=1}^{\infty}\indora\left(T_{n}\leq t\right),
\end{eqnarray}
para $t\geq0$.
\end{Def}

Si se consideran los puntos $T_{n}$ como elementos de $\rea_{+}$, y $N\left(t\right)$ es el n\'umero de puntos en $\rea$. El proceso denotado por $\left\{N\left(t\right):t\geq0\right\}$, denotado por $N\left(t\right)$, es un proceso puntual en $\rea_{+}$. Los $T_{n}$ son los tiempos de ocurrencia, el proceso puntual $N\left(t\right)$ es simple si su n\'umero de ocurrencias son distintas: $0<T_{1}<T_{2}<\ldots$ casi seguramente.

\begin{Def}
Un proceso puntual $N\left(t\right)$ es un proceso de renovaci\'on si los tiempos de interocurrencia $\xi_{n}=T_{n}-T_{n-1}$, para $n\geq1$, son independientes e identicamente distribuidos con distribuci\'on $F$, donde $F\left(0\right)=0$ y $T_{0}=0$. Los $T_{n}$ son llamados tiempos de renovaci\'on, referente a la independencia o renovaci\'on de la informaci\'on estoc\'astica en estos tiempos. Los $\xi_{n}$ son los tiempos de inter-renovaci\'on, y $N\left(t\right)$ es el n\'umero de renovaciones en el intervalo $\left[0,t\right)$
\end{Def}


\begin{Note}
Para definir un proceso de renovaci\'on para cualquier contexto, solamente hay que especificar una distribuci\'on $F$, con $F\left(0\right)=0$, para los tiempos de inter-renovaci\'on. La funci\'on $F$ en turno degune las otra variables aleatorias. De manera formal, existe un espacio de probabilidad y una sucesi\'on de variables aleatorias $\xi_{1},\xi_{2},\ldots$ definidas en este con distribuci\'on $F$. Entonces las otras cantidades son $T_{n}=\sum_{k=1}^{n}\xi_{k}$ y $N\left(t\right)=\sum_{n=1}^{\infty}\indora\left(T_{n}\leq t\right)$, donde $T_{n}\rightarrow\infty$ casi seguramente por la Ley Fuerte de los Grandes N\'umeros.
\end{Note}







Los tiempos $T_{n}$ est\'an relacionados con los conteos de $N\left(t\right)$ por

\begin{eqnarray*}
\left\{N\left(t\right)\geq n\right\}&=&\left\{T_{n}\leq t\right\}\\
T_{N\left(t\right)}\leq &t&<T_{N\left(t\right)+1},
\end{eqnarray*}

adem\'as $N\left(T_{n}\right)=n$, y 

\begin{eqnarray*}
N\left(t\right)=\max\left\{n:T_{n}\leq t\right\}=\min\left\{n:T_{n+1}>t\right\}
\end{eqnarray*}

Por propiedades de la convoluci\'on se sabe que

\begin{eqnarray*}
P\left\{T_{n}\leq t\right\}=F^{n\star}\left(t\right)
\end{eqnarray*}
que es la $n$-\'esima convoluci\'on de $F$. Entonces 

\begin{eqnarray*}
\left\{N\left(t\right)\geq n\right\}&=&\left\{T_{n}\leq t\right\}\\
P\left\{N\left(t\right)\leq n\right\}&=&1-F^{\left(n+1\right)\star}\left(t\right)
\end{eqnarray*}

Adem\'as usando el hecho de que $\esp\left[N\left(t\right)\right]=\sum_{n=1}^{\infty}P\left\{N\left(t\right)\geq n\right\}$
se tiene que

\begin{eqnarray*}
\esp\left[N\left(t\right)\right]=\sum_{n=1}^{\infty}F^{n\star}\left(t\right)
\end{eqnarray*}

\begin{Prop}
Para cada $t\geq0$, la funci\'on generadora de momentos $\esp\left[e^{\alpha N\left(t\right)}\right]$ existe para alguna $\alpha$ en una vecindad del 0, y de aqu\'i que $\esp\left[N\left(t\right)^{m}\right]<\infty$, para $m\geq1$.
\end{Prop}

\begin{Ejem}[\textbf{Proceso Poisson}]

Suponga que se tienen tiempos de inter-renovaci\'on \textit{i.i.d.} del proceso de renovaci\'on $N\left(t\right)$ tienen distribuci\'on exponencial $F\left(t\right)=q-e^{-\lambda t}$ con tasa $\lambda$. Entonces $N\left(t\right)$ es un proceso Poisson con tasa $\lambda$.

\end{Ejem}


\begin{Note}
Si el primer tiempo de renovaci\'on $\xi_{1}$ no tiene la misma distribuci\'on que el resto de las $\xi_{n}$, para $n\geq2$, a $N\left(t\right)$ se le llama Proceso de Renovaci\'on retardado, donde si $\xi$ tiene distribuci\'on $G$, entonces el tiempo $T_{n}$ de la $n$-\'esima renovaci\'on tiene distribuci\'on $G\star F^{\left(n-1\right)\star}\left(t\right)$
\end{Note}


\begin{Teo}
Para una constante $\mu\leq\infty$ ( o variable aleatoria), las siguientes expresiones son equivalentes:

\begin{eqnarray}
lim_{n\rightarrow\infty}n^{-1}T_{n}&=&\mu,\textrm{ c.s.}\\
lim_{t\rightarrow\infty}t^{-1}N\left(t\right)&=&1/\mu,\textrm{ c.s.}
\end{eqnarray}
\end{Teo}


Es decir, $T_{n}$ satisface la Ley Fuerte de los Grandes N\'umeros s\'i y s\'olo s\'i $N\left/t\right)$ la cumple.


\begin{Coro}[Ley Fuerte de los Grandes N\'umeros para Procesos de Renovaci\'on]
Si $N\left(t\right)$ es un proceso de renovaci\'on cuyos tiempos de inter-renovaci\'on tienen media $\mu\leq\infty$, entonces
\begin{eqnarray}
t^{-1}N\left(t\right)\rightarrow 1/\mu,\textrm{ c.s. cuando }t\rightarrow\infty.
\end{eqnarray}

\end{Coro}


Considerar el proceso estoc\'astico de valores reales $\left\{Z\left(t\right):t\geq0\right\}$ en el mismo espacio de probabilidad que $N\left(t\right)$

\begin{Def}
Para el proceso $\left\{Z\left(t\right):t\geq0\right\}$ se define la fluctuaci\'on m\'axima de $Z\left(t\right)$ en el intervalo $\left(T_{n-1},T_{n}\right]$:
\begin{eqnarray*}
M_{n}=\sup_{T_{n-1}<t\leq T_{n}}|Z\left(t\right)-Z\left(T_{n-1}\right)|
\end{eqnarray*}
\end{Def}

\begin{Teo}
Sup\'ongase que $n^{-1}T_{n}\rightarrow\mu$ c.s. cuando $n\rightarrow\infty$, donde $\mu\leq\infty$ es una constante o variable aleatoria. Sea $a$ una constante o variable aleatoria que puede ser infinita cuando $\mu$ es finita, y considere las expresiones l\'imite:
\begin{eqnarray}
lim_{n\rightarrow\infty}n^{-1}Z\left(T_{n}\right)&=&a,\textrm{ c.s.}\\
lim_{t\rightarrow\infty}t^{-1}Z\left(t\right)&=&a/\mu,\textrm{ c.s.}
\end{eqnarray}
La segunda expresi\'on implica la primera. Conversamente, la primera implica la segunda si el proceso $Z\left(t\right)$ es creciente, o si $lim_{n\rightarrow\infty}n^{-1}M_{n}=0$ c.s.
\end{Teo}

\begin{Coro}
Si $N\left(t\right)$ es un proceso de renovaci\'on, y $\left(Z\left(T_{n}\right)-Z\left(T_{n-1}\right),M_{n}\right)$, para $n\geq1$, son variables aleatorias independientes e id\'enticamente distribuidas con media finita, entonces,
\begin{eqnarray}
lim_{t\rightarrow\infty}t^{-1}Z\left(t\right)\rightarrow\frac{\esp\left[Z\left(T_{1}\right)-Z\left(T_{0}\right)\right]}{\esp\left[T_{1}\right]},\textrm{ c.s. cuando  }t\rightarrow\infty.
\end{eqnarray}
\end{Coro}


Sup\'ongase que $N\left(t\right)$ es un proceso de renovaci\'on con distribuci\'on $F$ con media finita $\mu$.

\begin{Def}
La funci\'on de renovaci\'on asociada con la distribuci\'on $F$, del proceso $N\left(t\right)$, es
\begin{eqnarray*}
U\left(t\right)=\sum_{n=1}^{\infty}F^{n\star}\left(t\right),\textrm{   }t\geq0,
\end{eqnarray*}
donde $F^{0\star}\left(t\right)=\indora\left(t\geq0\right)$.
\end{Def}


\begin{Prop}
Sup\'ongase que la distribuci\'on de inter-renovaci\'on $F$ tiene densidad $f$. Entonces $U\left(t\right)$ tambi\'en tiene densidad, para $t>0$, y es $U^{'}\left(t\right)=\sum_{n=0}^{\infty}f^{n\star}\left(t\right)$. Adem\'as
\begin{eqnarray*}
\prob\left\{N\left(t\right)>N\left(t-\right)\right\}=0\textrm{,   }t\geq0.
\end{eqnarray*}
\end{Prop}

\begin{Def}
La Transformada de Laplace-Stieljes de $F$ est\'a dada por

\begin{eqnarray*}
\hat{F}\left(\alpha\right)=\int_{\rea_{+}}e^{-\alpha t}dF\left(t\right)\textrm{,  }\alpha\geq0.
\end{eqnarray*}
\end{Def}

Entonces

\begin{eqnarray*}
\hat{U}\left(\alpha\right)=\sum_{n=0}^{\infty}\hat{F^{n\star}}\left(\alpha\right)=\sum_{n=0}^{\infty}\hat{F}\left(\alpha\right)^{n}=\frac{1}{1-\hat{F}\left(\alpha\right)}.
\end{eqnarray*}


\begin{Prop}
La Transformada de Laplace $\hat{U}\left(\alpha\right)$ y $\hat{F}\left(\alpha\right)$ determina una a la otra de manera \'unica por la relaci\'on $\hat{U}\left(\alpha\right)=\frac{1}{1-\hat{F}\left(\alpha\right)}$.
\end{Prop}


\begin{Note}
Un proceso de renovaci\'on $N\left(t\right)$ cuyos tiempos de inter-renovaci\'on tienen media finita, es un proceso Poisson con tasa $\lambda$ si y s\'olo s\'i $\esp\left[U\left(t\right)\right]=\lambda t$, para $t\geq0$.
\end{Note}


\begin{Teo}
Sea $N\left(t\right)$ un proceso puntual simple con puntos de localizaci\'on $T_{n}$ tal que $\eta\left(t\right)=\esp\left[N\left(\right)\right]$ es finita para cada $t$. Entonces para cualquier funci\'on $f:\rea_{+}\rightarrow\rea$,
\begin{eqnarray*}
\esp\left[\sum_{n=1}^{N\left(\right)}f\left(T_{n}\right)\right]=\int_{\left(0,t\right]}f\left(s\right)d\eta\left(s\right)\textrm{,  }t\geq0,
\end{eqnarray*}
suponiendo que la integral exista. Adem\'as si $X_{1},X_{2},\ldots$ son variables aleatorias definidas en el mismo espacio de probabilidad que el proceso $N\left(t\right)$ tal que $\esp\left[X_{n}|T_{n}=s\right]=f\left(s\right)$, independiente de $n$. Entonces
\begin{eqnarray*}
\esp\left[\sum_{n=1}^{N\left(t\right)}X_{n}\right]=\int_{\left(0,t\right]}f\left(s\right)d\eta\left(s\right)\textrm{,  }t\geq0,
\end{eqnarray*} 
suponiendo que la integral exista. 
\end{Teo}

\begin{Coro}[Identidad de Wald para Renovaciones]
Para el proceso de renovaci\'on $N\left(t\right)$,
\begin{eqnarray*}
\esp\left[T_{N\left(t\right)+1}\right]=\mu\esp\left[N\left(t\right)+1\right]\textrm{,  }t\geq0,
\end{eqnarray*}  
\end{Coro}


\begin{Def}
Sea $h\left(t\right)$ funci\'on de valores reales en $\rea$ acotada en intervalos finitos e igual a cero para $t<0$ La ecuaci\'on de renovaci\'on para $h\left(t\right)$ y la distribuci\'on $F$ es

\begin{eqnarray}%\label{Ec.Renovacion}
H\left(t\right)=h\left(t\right)+\int_{\left[0,t\right]}H\left(t-s\right)dF\left(s\right)\textrm{,    }t\geq0,
\end{eqnarray}
donde $H\left(t\right)$ es una funci\'on de valores reales. Esto es $H=h+F\star H$. Decimos que $H\left(t\right)$ es soluci\'on de esta ecuaci\'on si satisface la ecuaci\'on, y es acotada en intervalos finitos e iguales a cero para $t<0$.
\end{Def}

\begin{Prop}
La funci\'on $U\star h\left(t\right)$ es la \'unica soluci\'on de la ecuaci\'on de renovaci\'on (\ref{Ec.Renovacion}).
\end{Prop}

\begin{Teo}[Teorema Renovaci\'on Elemental]
\begin{eqnarray*}
t^{-1}U\left(t\right)\rightarrow 1/\mu\textrm{,    cuando }t\rightarrow\infty.
\end{eqnarray*}
\end{Teo}



Sup\'ongase que $N\left(t\right)$ es un proceso de renovaci\'on con distribuci\'on $F$ con media finita $\mu$.

\begin{Def}
La funci\'on de renovaci\'on asociada con la distribuci\'on $F$, del proceso $N\left(t\right)$, es
\begin{eqnarray*}
U\left(t\right)=\sum_{n=1}^{\infty}F^{n\star}\left(t\right),\textrm{   }t\geq0,
\end{eqnarray*}
donde $F^{0\star}\left(t\right)=\indora\left(t\geq0\right)$.
\end{Def}


\begin{Prop}
Sup\'ongase que la distribuci\'on de inter-renovaci\'on $F$ tiene densidad $f$. Entonces $U\left(t\right)$ tambi\'en tiene densidad, para $t>0$, y es $U^{'}\left(t\right)=\sum_{n=0}^{\infty}f^{n\star}\left(t\right)$. Adem\'as
\begin{eqnarray*}
\prob\left\{N\left(t\right)>N\left(t-\right)\right\}=0\textrm{,   }t\geq0.
\end{eqnarray*}
\end{Prop}

\begin{Def}
La Transformada de Laplace-Stieljes de $F$ est\'a dada por

\begin{eqnarray*}
\hat{F}\left(\alpha\right)=\int_{\rea_{+}}e^{-\alpha t}dF\left(t\right)\textrm{,  }\alpha\geq0.
\end{eqnarray*}
\end{Def}

Entonces

\begin{eqnarray*}
\hat{U}\left(\alpha\right)=\sum_{n=0}^{\infty}\hat{F^{n\star}}\left(\alpha\right)=\sum_{n=0}^{\infty}\hat{F}\left(\alpha\right)^{n}=\frac{1}{1-\hat{F}\left(\alpha\right)}.
\end{eqnarray*}


\begin{Prop}
La Transformada de Laplace $\hat{U}\left(\alpha\right)$ y $\hat{F}\left(\alpha\right)$ determina una a la otra de manera \'unica por la relaci\'on $\hat{U}\left(\alpha\right)=\frac{1}{1-\hat{F}\left(\alpha\right)}$.
\end{Prop}


\begin{Note}
Un proceso de renovaci\'on $N\left(t\right)$ cuyos tiempos de inter-renovaci\'on tienen media finita, es un proceso Poisson con tasa $\lambda$ si y s\'olo s\'i $\esp\left[U\left(t\right)\right]=\lambda t$, para $t\geq0$.
\end{Note}


\begin{Teo}
Sea $N\left(t\right)$ un proceso puntual simple con puntos de localizaci\'on $T_{n}$ tal que $\eta\left(t\right)=\esp\left[N\left(\right)\right]$ es finita para cada $t$. Entonces para cualquier funci\'on $f:\rea_{+}\rightarrow\rea$,
\begin{eqnarray*}
\esp\left[\sum_{n=1}^{N\left(\right)}f\left(T_{n}\right)\right]=\int_{\left(0,t\right]}f\left(s\right)d\eta\left(s\right)\textrm{,  }t\geq0,
\end{eqnarray*}
suponiendo que la integral exista. Adem\'as si $X_{1},X_{2},\ldots$ son variables aleatorias definidas en el mismo espacio de probabilidad que el proceso $N\left(t\right)$ tal que $\esp\left[X_{n}|T_{n}=s\right]=f\left(s\right)$, independiente de $n$. Entonces
\begin{eqnarray*}
\esp\left[\sum_{n=1}^{N\left(t\right)}X_{n}\right]=\int_{\left(0,t\right]}f\left(s\right)d\eta\left(s\right)\textrm{,  }t\geq0,
\end{eqnarray*} 
suponiendo que la integral exista. 
\end{Teo}

\begin{Coro}[Identidad de Wald para Renovaciones]
Para el proceso de renovaci\'on $N\left(t\right)$,
\begin{eqnarray*}
\esp\left[T_{N\left(t\right)+1}\right]=\mu\esp\left[N\left(t\right)+1\right]\textrm{,  }t\geq0,
\end{eqnarray*}  
\end{Coro}


\begin{Def}
Sea $h\left(t\right)$ funci\'on de valores reales en $\rea$ acotada en intervalos finitos e igual a cero para $t<0$ La ecuaci\'on de renovaci\'on para $h\left(t\right)$ y la distribuci\'on $F$ es

\begin{eqnarray}%\label{Ec.Renovacion}
H\left(t\right)=h\left(t\right)+\int_{\left[0,t\right]}H\left(t-s\right)dF\left(s\right)\textrm{,    }t\geq0,
\end{eqnarray}
donde $H\left(t\right)$ es una funci\'on de valores reales. Esto es $H=h+F\star H$. Decimos que $H\left(t\right)$ es soluci\'on de esta ecuaci\'on si satisface la ecuaci\'on, y es acotada en intervalos finitos e iguales a cero para $t<0$.
\end{Def}

\begin{Prop}
La funci\'on $U\star h\left(t\right)$ es la \'unica soluci\'on de la ecuaci\'on de renovaci\'on (\ref{Ec.Renovacion}).
\end{Prop}

\begin{Teo}[Teorema Renovaci\'on Elemental]
\begin{eqnarray*}
t^{-1}U\left(t\right)\rightarrow 1/\mu\textrm{,    cuando }t\rightarrow\infty.
\end{eqnarray*}
\end{Teo}


\begin{Note} Una funci\'on $h:\rea_{+}\rightarrow\rea$ es Directamente Riemann Integrable en los siguientes casos:
\begin{itemize}
\item[a)] $h\left(t\right)\geq0$ es decreciente y Riemann Integrable.
\item[b)] $h$ es continua excepto posiblemente en un conjunto de Lebesgue de medida 0, y $|h\left(t\right)|\leq b\left(t\right)$, donde $b$ es DRI.
\end{itemize}
\end{Note}

\begin{Teo}[Teorema Principal de Renovaci\'on]
Si $F$ es no aritm\'etica y $h\left(t\right)$ es Directamente Riemann Integrable (DRI), entonces

\begin{eqnarray*}
lim_{t\rightarrow\infty}U\star h=\frac{1}{\mu}\int_{\rea_{+}}h\left(s\right)ds.
\end{eqnarray*}
\end{Teo}

\begin{Prop}
Cualquier funci\'on $H\left(t\right)$ acotada en intervalos finitos y que es 0 para $t<0$ puede expresarse como
\begin{eqnarray*}
H\left(t\right)=U\star h\left(t\right)\textrm{,  donde }h\left(t\right)=H\left(t\right)-F\star H\left(t\right)
\end{eqnarray*}
\end{Prop}

\begin{Def}
Un proceso estoc\'astico $X\left(t\right)$ es crudamente regenerativo en un tiempo aleatorio positivo $T$ si
\begin{eqnarray*}
\esp\left[X\left(T+t\right)|T\right]=\esp\left[X\left(t\right)\right]\textrm{, para }t\geq0,\end{eqnarray*}
y con las esperanzas anteriores finitas.
\end{Def}

\begin{Prop}
Sup\'ongase que $X\left(t\right)$ es un proceso crudamente regenerativo en $T$, que tiene distribuci\'on $F$. Si $\esp\left[X\left(t\right)\right]$ es acotado en intervalos finitos, entonces
\begin{eqnarray*}
\esp\left[X\left(t\right)\right]=U\star h\left(t\right)\textrm{,  donde }h\left(t\right)=\esp\left[X\left(t\right)\indora\left(T>t\right)\right].
\end{eqnarray*}
\end{Prop}

\begin{Teo}[Regeneraci\'on Cruda]
Sup\'ongase que $X\left(t\right)$ es un proceso con valores positivo crudamente regenerativo en $T$, y def\'inase $M=\sup\left\{|X\left(t\right)|:t\leq T\right\}$. Si $T$ es no aritm\'etico y $M$ y $MT$ tienen media finita, entonces
\begin{eqnarray*}
lim_{t\rightarrow\infty}\esp\left[X\left(t\right)\right]=\frac{1}{\mu}\int_{\rea_{+}}h\left(s\right)ds,
\end{eqnarray*}
donde $h\left(t\right)=\esp\left[X\left(t\right)\indora\left(T>t\right)\right]$.
\end{Teo}


\begin{Note} Una funci\'on $h:\rea_{+}\rightarrow\rea$ es Directamente Riemann Integrable en los siguientes casos:
\begin{itemize}
\item[a)] $h\left(t\right)\geq0$ es decreciente y Riemann Integrable.
\item[b)] $h$ es continua excepto posiblemente en un conjunto de Lebesgue de medida 0, y $|h\left(t\right)|\leq b\left(t\right)$, donde $b$ es DRI.
\end{itemize}
\end{Note}

\begin{Teo}[Teorema Principal de Renovaci\'on]
Si $F$ es no aritm\'etica y $h\left(t\right)$ es Directamente Riemann Integrable (DRI), entonces

\begin{eqnarray*}
lim_{t\rightarrow\infty}U\star h=\frac{1}{\mu}\int_{\rea_{+}}h\left(s\right)ds.
\end{eqnarray*}
\end{Teo}

\begin{Prop}
Cualquier funci\'on $H\left(t\right)$ acotada en intervalos finitos y que es 0 para $t<0$ puede expresarse como
\begin{eqnarray*}
H\left(t\right)=U\star h\left(t\right)\textrm{,  donde }h\left(t\right)=H\left(t\right)-F\star H\left(t\right)
\end{eqnarray*}
\end{Prop}

\begin{Def}
Un proceso estoc\'astico $X\left(t\right)$ es crudamente regenerativo en un tiempo aleatorio positivo $T$ si
\begin{eqnarray*}
\esp\left[X\left(T+t\right)|T\right]=\esp\left[X\left(t\right)\right]\textrm{, para }t\geq0,\end{eqnarray*}
y con las esperanzas anteriores finitas.
\end{Def}

\begin{Prop}
Sup\'ongase que $X\left(t\right)$ es un proceso crudamente regenerativo en $T$, que tiene distribuci\'on $F$. Si $\esp\left[X\left(t\right)\right]$ es acotado en intervalos finitos, entonces
\begin{eqnarray*}
\esp\left[X\left(t\right)\right]=U\star h\left(t\right)\textrm{,  donde }h\left(t\right)=\esp\left[X\left(t\right)\indora\left(T>t\right)\right].
\end{eqnarray*}
\end{Prop}

\begin{Teo}[Regeneraci\'on Cruda]
Sup\'ongase que $X\left(t\right)$ es un proceso con valores positivo crudamente regenerativo en $T$, y def\'inase $M=\sup\left\{|X\left(t\right)|:t\leq T\right\}$. Si $T$ es no aritm\'etico y $M$ y $MT$ tienen media finita, entonces
\begin{eqnarray*}
lim_{t\rightarrow\infty}\esp\left[X\left(t\right)\right]=\frac{1}{\mu}\int_{\rea_{+}}h\left(s\right)ds,
\end{eqnarray*}
donde $h\left(t\right)=\esp\left[X\left(t\right)\indora\left(T>t\right)\right]$.
\end{Teo}

\begin{Def}
Para el proceso $\left\{\left(N\left(t\right),X\left(t\right)\right):t\geq0\right\}$, sus trayectoria muestrales en el intervalo de tiempo $\left[T_{n-1},T_{n}\right)$ est\'an descritas por
\begin{eqnarray*}
\zeta_{n}=\left(\xi_{n},\left\{X\left(T_{n-1}+t\right):0\leq t<\xi_{n}\right\}\right)
\end{eqnarray*}
Este $\zeta_{n}$ es el $n$-\'esimo segmento del proceso. El proceso es regenerativo sobre los tiempos $T_{n}$ si sus segmentos $\zeta_{n}$ son independientes e id\'enticamennte distribuidos.
\end{Def}


\begin{Note}
Si $\tilde{X}\left(t\right)$ con espacio de estados $\tilde{S}$ es regenerativo sobre $T_{n}$, entonces $X\left(t\right)=f\left(\tilde{X}\left(t\right)\right)$ tambi\'en es regenerativo sobre $T_{n}$, para cualquier funci\'on $f:\tilde{S}\rightarrow S$.
\end{Note}

\begin{Note}
Los procesos regenerativos son crudamente regenerativos, pero no al rev\'es.
\end{Note}


\begin{Note}
Un proceso estoc\'astico a tiempo continuo o discreto es regenerativo si existe un proceso de renovaci\'on  tal que los segmentos del proceso entre tiempos de renovaci\'on sucesivos son i.i.d., es decir, para $\left\{X\left(t\right):t\geq0\right\}$ proceso estoc\'astico a tiempo continuo con espacio de estados $S$, espacio m\'etrico.
\end{Note}

Para $\left\{X\left(t\right):t\geq0\right\}$ Proceso Estoc\'astico a tiempo continuo con estado de espacios $S$, que es un espacio m\'etrico, con trayectorias continuas por la derecha y con l\'imites por la izquierda c.s. Sea $N\left(t\right)$ un proceso de renovaci\'on en $\rea_{+}$ definido en el mismo espacio de probabilidad que $X\left(t\right)$, con tiempos de renovaci\'on $T$ y tiempos de inter-renovaci\'on $\xi_{n}=T_{n}-T_{n-1}$, con misma distribuci\'on $F$ de media finita $\mu$.



\begin{Def}
Para el proceso $\left\{\left(N\left(t\right),X\left(t\right)\right):t\geq0\right\}$, sus trayectoria muestrales en el intervalo de tiempo $\left[T_{n-1},T_{n}\right)$ est\'an descritas por
\begin{eqnarray*}
\zeta_{n}=\left(\xi_{n},\left\{X\left(T_{n-1}+t\right):0\leq t<\xi_{n}\right\}\right)
\end{eqnarray*}
Este $\zeta_{n}$ es el $n$-\'esimo segmento del proceso. El proceso es regenerativo sobre los tiempos $T_{n}$ si sus segmentos $\zeta_{n}$ son independientes e id\'enticamennte distribuidos.
\end{Def}

\begin{Note}
Un proceso regenerativo con media de la longitud de ciclo finita es llamado positivo recurrente.
\end{Note}

\begin{Teo}[Procesos Regenerativos]
Suponga que el proceso
\end{Teo}


\begin{Def}[Renewal Process Trinity]
Para un proceso de renovaci\'on $N\left(t\right)$, los siguientes procesos proveen de informaci\'on sobre los tiempos de renovaci\'on.
\begin{itemize}
\item $A\left(t\right)=t-T_{N\left(t\right)}$, el tiempo de recurrencia hacia atr\'as al tiempo $t$, que es el tiempo desde la \'ultima renovaci\'on para $t$.

\item $B\left(t\right)=T_{N\left(t\right)+1}-t$, el tiempo de recurrencia hacia adelante al tiempo $t$, residual del tiempo de renovaci\'on, que es el tiempo para la pr\'oxima renovaci\'on despu\'es de $t$.

\item $L\left(t\right)=\xi_{N\left(t\right)+1}=A\left(t\right)+B\left(t\right)$, la longitud del intervalo de renovaci\'on que contiene a $t$.
\end{itemize}
\end{Def}

\begin{Note}
El proceso tridimensional $\left(A\left(t\right),B\left(t\right),L\left(t\right)\right)$ es regenerativo sobre $T_{n}$, y por ende cada proceso lo es. Cada proceso $A\left(t\right)$ y $B\left(t\right)$ son procesos de MArkov a tiempo continuo con trayectorias continuas por partes en el espacio de estados $\rea_{+}$. Una expresi\'on conveniente para su distribuci\'on conjunta es, para $0\leq x<t,y\geq0$
\begin{equation}\label{NoRenovacion}
P\left\{A\left(t\right)>x,B\left(t\right)>y\right\}=
P\left\{N\left(t+y\right)-N\left((t-x)\right)=0\right\}
\end{equation}
\end{Note}

\begin{Ejem}[Tiempos de recurrencia Poisson]
Si $N\left(t\right)$ es un proceso Poisson con tasa $\lambda$, entonces de la expresi\'on (\ref{NoRenovacion}) se tiene que

\begin{eqnarray*}
\begin{array}{lc}
P\left\{A\left(t\right)>x,B\left(t\right)>y\right\}=e^{-\lambda\left(x+y\right)},&0\leq x<t,y\geq0,
\end{array}
\end{eqnarray*}
que es la probabilidad Poisson de no renovaciones en un intervalo de longitud $x+y$.

\end{Ejem}

%\begin{Note}
Una cadena de Markov erg\'odica tiene la propiedad de ser estacionaria si la distribuci\'on de su estado al tiempo $0$ es su distribuci\'on estacionaria.
%\end{Note}


\begin{Def}
Un proceso estoc\'astico a tiempo continuo $\left\{X\left(t\right):t\geq0\right\}$ en un espacio general es estacionario si sus distribuciones finito dimensionales son invariantes bajo cualquier  traslado: para cada $0\leq s_{1}<s_{2}<\cdots<s_{k}$ y $t\geq0$,
\begin{eqnarray*}
\left(X\left(s_{1}+t\right),\ldots,X\left(s_{k}+t\right)\right)=_{d}\left(X\left(s_{1}\right),\ldots,X\left(s_{k}\right)\right).
\end{eqnarray*}
\end{Def}

\begin{Note}
Un proceso de Markov es estacionario si $X\left(t\right)=_{d}X\left(0\right)$, $t\geq0$.
\end{Note}

Considerese el proceso $N\left(t\right)=\sum_{n}\indora\left(\tau_{n}\leq t\right)$ en $\rea_{+}$, con puntos $0<\tau_{1}<\tau_{2}<\cdots$.

\begin{Prop}
Si $N$ es un proceso puntual estacionario y $\esp\left[N\left(1\right)\right]<\infty$, entonces $\esp\left[N\left(t\right)\right]=t\esp\left[N\left(1\right)\right]$, $t\geq0$

\end{Prop}

\begin{Teo}
Los siguientes enunciados son equivalentes
\begin{itemize}
\item[i)] El proceso retardado de renovaci\'on $N$ es estacionario.

\item[ii)] EL proceso de tiempos de recurrencia hacia adelante $B\left(t\right)$ es estacionario.


\item[iii)] $\esp\left[N\left(t\right)\right]=t/\mu$,


\item[iv)] $G\left(t\right)=F_{e}\left(t\right)=\frac{1}{\mu}\int_{0}^{t}\left[1-F\left(s\right)\right]ds$
\end{itemize}
Cuando estos enunciados son ciertos, $P\left\{B\left(t\right)\leq x\right\}=F_{e}\left(x\right)$, para $t,x\geq0$.

\end{Teo}

\begin{Note}
Una consecuencia del teorema anterior es que el Proceso Poisson es el \'unico proceso sin retardo que es estacionario.
\end{Note}

\begin{Coro}
El proceso de renovaci\'on $N\left(t\right)$ sin retardo, y cuyos tiempos de inter renonaci\'on tienen media finita, es estacionario si y s\'olo si es un proceso Poisson.

\end{Coro}



%________________________________________________________________________
\subsubsection{Procesos Regenerativos}
%________________________________________________________________________



\begin{Note}
Si $\tilde{X}\left(t\right)$ con espacio de estados $\tilde{S}$ es regenerativo sobre $T_{n}$, entonces $X\left(t\right)=f\left(\tilde{X}\left(t\right)\right)$ tambi\'en es regenerativo sobre $T_{n}$, para cualquier funci\'on $f:\tilde{S}\rightarrow S$.
\end{Note}

\begin{Note}
Los procesos regenerativos son crudamente regenerativos, pero no al rev\'es.
\end{Note}
\subsection*{Procesos Regenerativos: Sigman\cite{Sigman1}}
\begin{Def}[Definici\'on Cl\'asica]
Un proceso estoc\'astico $X=\left\{X\left(t\right):t\geq0\right\}$ es llamado regenerativo is existe una variable aleatoria $R_{1}>0$ tal que
\begin{itemize}
\item[i)] $\left\{X\left(t+R_{1}\right):t\geq0\right\}$ es independiente de $\left\{\left\{X\left(t\right):t<R_{1}\right\},\right\}$
\item[ii)] $\left\{X\left(t+R_{1}\right):t\geq0\right\}$ es estoc\'asticamente equivalente a $\left\{X\left(t\right):t>0\right\}$
\end{itemize}

Llamamos a $R_{1}$ tiempo de regeneraci\'on, y decimos que $X$ se regenera en este punto.
\end{Def}

$\left\{X\left(t+R_{1}\right)\right\}$ es regenerativo con tiempo de regeneraci\'on $R_{2}$, independiente de $R_{1}$ pero con la misma distribuci\'on que $R_{1}$. Procediendo de esta manera se obtiene una secuencia de variables aleatorias independientes e id\'enticamente distribuidas $\left\{R_{n}\right\}$ llamados longitudes de ciclo. Si definimos a $Z_{k}\equiv R_{1}+R_{2}+\cdots+R_{k}$, se tiene un proceso de renovaci\'on llamado proceso de renovaci\'on encajado para $X$.




\begin{Def}
Para $x$ fijo y para cada $t\geq0$, sea $I_{x}\left(t\right)=1$ si $X\left(t\right)\leq x$,  $I_{x}\left(t\right)=0$ en caso contrario, y def\'inanse los tiempos promedio
\begin{eqnarray*}
\overline{X}&=&lim_{t\rightarrow\infty}\frac{1}{t}\int_{0}^{\infty}X\left(u\right)du\\
\prob\left(X_{\infty}\leq x\right)&=&lim_{t\rightarrow\infty}\frac{1}{t}\int_{0}^{\infty}I_{x}\left(u\right)du,
\end{eqnarray*}
cuando estos l\'imites existan.
\end{Def}

Como consecuencia del teorema de Renovaci\'on-Recompensa, se tiene que el primer l\'imite  existe y es igual a la constante
\begin{eqnarray*}
\overline{X}&=&\frac{\esp\left[\int_{0}^{R_{1}}X\left(t\right)dt\right]}{\esp\left[R_{1}\right]},
\end{eqnarray*}
suponiendo que ambas esperanzas son finitas.

\begin{Note}
\begin{itemize}
\item[a)] Si el proceso regenerativo $X$ es positivo recurrente y tiene trayectorias muestrales no negativas, entonces la ecuaci\'on anterior es v\'alida.
\item[b)] Si $X$ es positivo recurrente regenerativo, podemos construir una \'unica versi\'on estacionaria de este proceso, $X_{e}=\left\{X_{e}\left(t\right)\right\}$, donde $X_{e}$ es un proceso estoc\'astico regenerativo y estrictamente estacionario, con distribuci\'on marginal distribuida como $X_{\infty}$
\end{itemize}
\end{Note}

%________________________________________________________________________
\subsection{Procesos Regenerativos}
%________________________________________________________________________

Para $\left\{X\left(t\right):t\geq0\right\}$ Proceso Estoc\'astico a tiempo continuo con estado de espacios $S$, que es un espacio m\'etrico, con trayectorias continuas por la derecha y con l\'imites por la izquierda c.s. Sea $N\left(t\right)$ un proceso de renovaci\'on en $\rea_{+}$ definido en el mismo espacio de probabilidad que $X\left(t\right)$, con tiempos de renovaci\'on $T$ y tiempos de inter-renovaci\'on $\xi_{n}=T_{n}-T_{n-1}$, con misma distribuci\'on $F$ de media finita $\mu$.



\begin{Def}
Para el proceso $\left\{\left(N\left(t\right),X\left(t\right)\right):t\geq0\right\}$, sus trayectoria muestrales en el intervalo de tiempo $\left[T_{n-1},T_{n}\right)$ est\'an descritas por
\begin{eqnarray*}
\zeta_{n}=\left(\xi_{n},\left\{X\left(T_{n-1}+t\right):0\leq t<\xi_{n}\right\}\right)
\end{eqnarray*}
Este $\zeta_{n}$ es el $n$-\'esimo segmento del proceso. El proceso es regenerativo sobre los tiempos $T_{n}$ si sus segmentos $\zeta_{n}$ son independientes e id\'enticamennte distribuidos.
\end{Def}


\begin{Note}
Si $\tilde{X}\left(t\right)$ con espacio de estados $\tilde{S}$ es regenerativo sobre $T_{n}$, entonces $X\left(t\right)=f\left(\tilde{X}\left(t\right)\right)$ tambi\'en es regenerativo sobre $T_{n}$, para cualquier funci\'on $f:\tilde{S}\rightarrow S$.
\end{Note}

\begin{Note}
Los procesos regenerativos son crudamente regenerativos, pero no al rev\'es.
\end{Note}

\begin{Def}[Definici\'on Cl\'asica]
Un proceso estoc\'astico $X=\left\{X\left(t\right):t\geq0\right\}$ es llamado regenerativo is existe una variable aleatoria $R_{1}>0$ tal que
\begin{itemize}
\item[i)] $\left\{X\left(t+R_{1}\right):t\geq0\right\}$ es independiente de $\left\{\left\{X\left(t\right):t<R_{1}\right\},\right\}$
\item[ii)] $\left\{X\left(t+R_{1}\right):t\geq0\right\}$ es estoc\'asticamente equivalente a $\left\{X\left(t\right):t>0\right\}$
\end{itemize}

Llamamos a $R_{1}$ tiempo de regeneraci\'on, y decimos que $X$ se regenera en este punto.
\end{Def}

$\left\{X\left(t+R_{1}\right)\right\}$ es regenerativo con tiempo de regeneraci\'on $R_{2}$, independiente de $R_{1}$ pero con la misma distribuci\'on que $R_{1}$. Procediendo de esta manera se obtiene una secuencia de variables aleatorias independientes e id\'enticamente distribuidas $\left\{R_{n}\right\}$ llamados longitudes de ciclo. Si definimos a $Z_{k}\equiv R_{1}+R_{2}+\cdots+R_{k}$, se tiene un proceso de renovaci\'on llamado proceso de renovaci\'on encajado para $X$.

\begin{Note}
Un proceso regenerativo con media de la longitud de ciclo finita es llamado positivo recurrente.
\end{Note}


\begin{Def}
Para $x$ fijo y para cada $t\geq0$, sea $I_{x}\left(t\right)=1$ si $X\left(t\right)\leq x$,  $I_{x}\left(t\right)=0$ en caso contrario, y def\'inanse los tiempos promedio
\begin{eqnarray*}
\overline{X}&=&lim_{t\rightarrow\infty}\frac{1}{t}\int_{0}^{\infty}X\left(u\right)du\\
\prob\left(X_{\infty}\leq x\right)&=&lim_{t\rightarrow\infty}\frac{1}{t}\int_{0}^{\infty}I_{x}\left(u\right)du,
\end{eqnarray*}
cuando estos l\'imites existan.
\end{Def}

Como consecuencia del teorema de Renovaci\'on-Recompensa, se tiene que el primer l\'imite  existe y es igual a la constante
\begin{eqnarray*}
\overline{X}&=&\frac{\esp\left[\int_{0}^{R_{1}}X\left(t\right)dt\right]}{\esp\left[R_{1}\right]},
\end{eqnarray*}
suponiendo que ambas esperanzas son finitas.

\begin{Note}
\begin{itemize}
\item[a)] Si el proceso regenerativo $X$ es positivo recurrente y tiene trayectorias muestrales no negativas, entonces la ecuaci\'on anterior es v\'alida.
\item[b)] Si $X$ es positivo recurrente regenerativo, podemos construir una \'unica versi\'on estacionaria de este proceso, $X_{e}=\left\{X_{e}\left(t\right)\right\}$, donde $X_{e}$ es un proceso estoc\'astico regenerativo y estrictamente estacionario, con distribuci\'on marginal distribuida como $X_{\infty}$
\end{itemize}
\end{Note}

%__________________________________________________________________________________________
\subsection{Procesos Regenerativos Estacionarios - Stidham \cite{Stidham}}
%__________________________________________________________________________________________


Un proceso estoc\'astico a tiempo continuo $\left\{V\left(t\right),t\geq0\right\}$ es un proceso regenerativo si existe una sucesi\'on de variables aleatorias independientes e id\'enticamente distribuidas $\left\{X_{1},X_{2},\ldots\right\}$, sucesi\'on de renovaci\'on, tal que para cualquier conjunto de Borel $A$, 

\begin{eqnarray*}
\prob\left\{V\left(t\right)\in A|X_{1}+X_{2}+\cdots+X_{R\left(t\right)}=s,\left\{V\left(\tau\right),\tau<s\right\}\right\}=\prob\left\{V\left(t-s\right)\in A|X_{1}>t-s\right\},
\end{eqnarray*}
para todo $0\leq s\leq t$, donde $R\left(t\right)=\max\left\{X_{1}+X_{2}+\cdots+X_{j}\leq t\right\}=$n\'umero de renovaciones ({\emph{puntos de regeneraci\'on}}) que ocurren en $\left[0,t\right]$. El intervalo $\left[0,X_{1}\right)$ es llamado {\emph{primer ciclo de regeneraci\'on}} de $\left\{V\left(t \right),t\geq0\right\}$, $\left[X_{1},X_{1}+X_{2}\right)$ el {\emph{segundo ciclo de regeneraci\'on}}, y as\'i sucesivamente.

Sea $X=X_{1}$ y sea $F$ la funci\'on de distrbuci\'on de $X$


\begin{Def}
Se define el proceso estacionario, $\left\{V^{*}\left(t\right),t\geq0\right\}$, para $\left\{V\left(t\right),t\geq0\right\}$ por

\begin{eqnarray*}
\prob\left\{V\left(t\right)\in A\right\}=\frac{1}{\esp\left[X\right]}\int_{0}^{\infty}\prob\left\{V\left(t+x\right)\in A|X>x\right\}\left(1-F\left(x\right)\right)dx,
\end{eqnarray*} 
para todo $t\geq0$ y todo conjunto de Borel $A$.
\end{Def}

\begin{Def}
Una distribuci\'on se dice que es {\emph{aritm\'etica}} si todos sus puntos de incremento son m\'ultiplos de la forma $0,\lambda, 2\lambda,\ldots$ para alguna $\lambda>0$ entera.
\end{Def}


\begin{Def}
Una modificaci\'on medible de un proceso $\left\{V\left(t\right),t\geq0\right\}$, es una versi\'on de este, $\left\{V\left(t,w\right)\right\}$ conjuntamente medible para $t\geq0$ y para $w\in S$, $S$ espacio de estados para $\left\{V\left(t\right),t\geq0\right\}$.
\end{Def}

\begin{Teo}
Sea $\left\{V\left(t\right),t\geq\right\}$ un proceso regenerativo no negativo con modificaci\'on medible. Sea $\esp\left[X\right]<\infty$. Entonces el proceso estacionario dado por la ecuaci\'on anterior est\'a bien definido y tiene funci\'on de distribuci\'on independiente de $t$, adem\'as
\begin{itemize}
\item[i)] \begin{eqnarray*}
\esp\left[V^{*}\left(0\right)\right]&=&\frac{\esp\left[\int_{0}^{X}V\left(s\right)ds\right]}{\esp\left[X\right]}\end{eqnarray*}
\item[ii)] Si $\esp\left[V^{*}\left(0\right)\right]<\infty$, equivalentemente, si $\esp\left[\int_{0}^{X}V\left(s\right)ds\right]<\infty$,entonces
\begin{eqnarray*}
\frac{\int_{0}^{t}V\left(s\right)ds}{t}\rightarrow\frac{\esp\left[\int_{0}^{X}V\left(s\right)ds\right]}{\esp\left[X\right]}
\end{eqnarray*}
con probabilidad 1 y en media, cuando $t\rightarrow\infty$.
\end{itemize}
\end{Teo}
%
%___________________________________________________________________________________________
%\vspace{5.5cm}
\section{Cadenas de Markov estacionarias}
%\vspace{-1.0cm}


%__________________________________________________________________________________________
\subsection{Procesos Regenerativos Estacionarios - Stidham \cite{Stidham}}
%__________________________________________________________________________________________


Un proceso estoc\'astico a tiempo continuo $\left\{V\left(t\right),t\geq0\right\}$ es un proceso regenerativo si existe una sucesi\'on de variables aleatorias independientes e id\'enticamente distribuidas $\left\{X_{1},X_{2},\ldots\right\}$, sucesi\'on de renovaci\'on, tal que para cualquier conjunto de Borel $A$, 

\begin{eqnarray*}
\prob\left\{V\left(t\right)\in A|X_{1}+X_{2}+\cdots+X_{R\left(t\right)}=s,\left\{V\left(\tau\right),\tau<s\right\}\right\}=\prob\left\{V\left(t-s\right)\in A|X_{1}>t-s\right\},
\end{eqnarray*}
para todo $0\leq s\leq t$, donde $R\left(t\right)=\max\left\{X_{1}+X_{2}+\cdots+X_{j}\leq t\right\}=$n\'umero de renovaciones ({\emph{puntos de regeneraci\'on}}) que ocurren en $\left[0,t\right]$. El intervalo $\left[0,X_{1}\right)$ es llamado {\emph{primer ciclo de regeneraci\'on}} de $\left\{V\left(t \right),t\geq0\right\}$, $\left[X_{1},X_{1}+X_{2}\right)$ el {\emph{segundo ciclo de regeneraci\'on}}, y as\'i sucesivamente.

Sea $X=X_{1}$ y sea $F$ la funci\'on de distrbuci\'on de $X$


\begin{Def}
Se define el proceso estacionario, $\left\{V^{*}\left(t\right),t\geq0\right\}$, para $\left\{V\left(t\right),t\geq0\right\}$ por

\begin{eqnarray*}
\prob\left\{V\left(t\right)\in A\right\}=\frac{1}{\esp\left[X\right]}\int_{0}^{\infty}\prob\left\{V\left(t+x\right)\in A|X>x\right\}\left(1-F\left(x\right)\right)dx,
\end{eqnarray*} 
para todo $t\geq0$ y todo conjunto de Borel $A$.
\end{Def}

\begin{Def}
Una distribuci\'on se dice que es {\emph{aritm\'etica}} si todos sus puntos de incremento son m\'ultiplos de la forma $0,\lambda, 2\lambda,\ldots$ para alguna $\lambda>0$ entera.
\end{Def}


\begin{Def}
Una modificaci\'on medible de un proceso $\left\{V\left(t\right),t\geq0\right\}$, es una versi\'on de este, $\left\{V\left(t,w\right)\right\}$ conjuntamente medible para $t\geq0$ y para $w\in S$, $S$ espacio de estados para $\left\{V\left(t\right),t\geq0\right\}$.
\end{Def}

\begin{Teo}
Sea $\left\{V\left(t\right),t\geq\right\}$ un proceso regenerativo no negativo con modificaci\'on medible. Sea $\esp\left[X\right]<\infty$. Entonces el proceso estacionario dado por la ecuaci\'on anterior est\'a bien definido y tiene funci\'on de distribuci\'on independiente de $t$, adem\'as
\begin{itemize}
\item[i)] \begin{eqnarray*}
\esp\left[V^{*}\left(0\right)\right]&=&\frac{\esp\left[\int_{0}^{X}V\left(s\right)ds\right]}{\esp\left[X\right]}\end{eqnarray*}
\item[ii)] Si $\esp\left[V^{*}\left(0\right)\right]<\infty$, equivalentemente, si $\esp\left[\int_{0}^{X}V\left(s\right)ds\right]<\infty$,entonces
\begin{eqnarray*}
\frac{\int_{0}^{t}V\left(s\right)ds}{t}\rightarrow\frac{\esp\left[\int_{0}^{X}V\left(s\right)ds\right]}{\esp\left[X\right]}
\end{eqnarray*}
con probabilidad 1 y en media, cuando $t\rightarrow\infty$.
\end{itemize}
\end{Teo}

Para $\left\{X\left(t\right):t\geq0\right\}$ Proceso Estoc\'astico a tiempo continuo con estado de espacios $S$, que es un espacio m\'etrico, con trayectorias continuas por la derecha y con l\'imites por la izquierda c.s. Sea $N\left(t\right)$ un proceso de renovaci\'on en $\rea_{+}$ definido en el mismo espacio de probabilidad que $X\left(t\right)$, con tiempos de renovaci\'on $T$ y tiempos de inter-renovaci\'on $\xi_{n}=T_{n}-T_{n-1}$, con misma distribuci\'on $F$ de media finita $\mu$.

%________________________________________________________________________
\section{Procesos Regenerativos}
%________________________________________________________________________

Para $\left\{X\left(t\right):t\geq0\right\}$ Proceso Estoc\'astico a tiempo continuo con estado de espacios $S$, que es un espacio m\'etrico, con trayectorias continuas por la derecha y con l\'imites por la izquierda c.s. Sea $N\left(t\right)$ un proceso de renovaci\'on en $\rea_{+}$ definido en el mismo espacio de probabilidad que $X\left(t\right)$, con tiempos de renovaci\'on $T$ y tiempos de inter-renovaci\'on $\xi_{n}=T_{n}-T_{n-1}$, con misma distribuci\'on $F$ de media finita $\mu$.



\begin{Def}
Para el proceso $\left\{\left(N\left(t\right),X\left(t\right)\right):t\geq0\right\}$, sus trayectoria muestrales en el intervalo de tiempo $\left[T_{n-1},T_{n}\right)$ est\'an descritas por
\begin{eqnarray*}
\zeta_{n}=\left(\xi_{n},\left\{X\left(T_{n-1}+t\right):0\leq t<\xi_{n}\right\}\right)
\end{eqnarray*}
Este $\zeta_{n}$ es el $n$-\'esimo segmento del proceso. El proceso es regenerativo sobre los tiempos $T_{n}$ si sus segmentos $\zeta_{n}$ son independientes e id\'enticamennte distribuidos.
\end{Def}


\begin{Obs}
Si $\tilde{X}\left(t\right)$ con espacio de estados $\tilde{S}$ es regenerativo sobre $T_{n}$, entonces $X\left(t\right)=f\left(\tilde{X}\left(t\right)\right)$ tambi\'en es regenerativo sobre $T_{n}$, para cualquier funci\'on $f:\tilde{S}\rightarrow S$.
\end{Obs}

\begin{Obs}
Los procesos regenerativos son crudamente regenerativos, pero no al rev\'es.
\end{Obs}

\begin{Def}[Definici\'on Cl\'asica]
Un proceso estoc\'astico $X=\left\{X\left(t\right):t\geq0\right\}$ es llamado regenerativo is existe una variable aleatoria $R_{1}>0$ tal que
\begin{itemize}
\item[i)] $\left\{X\left(t+R_{1}\right):t\geq0\right\}$ es independiente de $\left\{\left\{X\left(t\right):t<R_{1}\right\},\right\}$
\item[ii)] $\left\{X\left(t+R_{1}\right):t\geq0\right\}$ es estoc\'asticamente equivalente a $\left\{X\left(t\right):t>0\right\}$
\end{itemize}

Llamamos a $R_{1}$ tiempo de regeneraci\'on, y decimos que $X$ se regenera en este punto.
\end{Def}

$\left\{X\left(t+R_{1}\right)\right\}$ es regenerativo con tiempo de regeneraci\'on $R_{2}$, independiente de $R_{1}$ pero con la misma distribuci\'on que $R_{1}$. Procediendo de esta manera se obtiene una secuencia de variables aleatorias independientes e id\'enticamente distribuidas $\left\{R_{n}\right\}$ llamados longitudes de ciclo. Si definimos a $Z_{k}\equiv R_{1}+R_{2}+\cdots+R_{k}$, se tiene un proceso de renovaci\'on llamado proceso de renovaci\'on encajado para $X$.

\begin{Note}
Un proceso regenerativo con media de la longitud de ciclo finita es llamado positivo recurrente.
\end{Note}


\begin{Def}
Para $x$ fijo y para cada $t\geq0$, sea $I_{x}\left(t\right)=1$ si $X\left(t\right)\leq x$,  $I_{x}\left(t\right)=0$ en caso contrario, y def\'inanse los tiempos promedio
\begin{eqnarray*}
\overline{X}&=&lim_{t\rightarrow\infty}\frac{1}{t}\int_{0}^{\infty}X\left(u\right)du\\
\prob\left(X_{\infty}\leq x\right)&=&lim_{t\rightarrow\infty}\frac{1}{t}\int_{0}^{\infty}I_{x}\left(u\right)du,
\end{eqnarray*}
cuando estos l\'imites existan.
\end{Def}

Como consecuencia del teorema de Renovaci\'on-Recompensa, se tiene que el primer l\'imite  existe y es igual a la constante
\begin{eqnarray*}
\overline{X}&=&\frac{\esp\left[\int_{0}^{R_{1}}X\left(t\right)dt\right]}{\esp\left[R_{1}\right]},
\end{eqnarray*}
suponiendo que ambas esperanzas son finitas.

\begin{Note}
\begin{itemize}
\item[a)] Si el proceso regenerativo $X$ es positivo recurrente y tiene trayectorias muestrales no negativas, entonces la ecuaci\'on anterior es v\'alida.
\item[b)] Si $X$ es positivo recurrente regenerativo, podemos construir una \'unica versi\'on estacionaria de este proceso, $X_{e}=\left\{X_{e}\left(t\right)\right\}$, donde $X_{e}$ es un proceso estoc\'astico regenerativo y estrictamente estacionario, con distribuci\'on marginal distribuida como $X_{\infty}$
\end{itemize}
\end{Note}

\section{Renewal and Regenerative Processes: Serfozo\cite{Serfozo}}
\begin{Def}\label{Def.Tn}
Sean $0\leq T_{1}\leq T_{2}\leq \ldots$ son tiempos aleatorios infinitos en los cuales ocurren ciertos eventos. El n\'umero de tiempos $T_{n}$ en el intervalo $\left[0,t\right)$ es

\begin{eqnarray}
N\left(t\right)=\sum_{n=1}^{\infty}\indora\left(T_{n}\leq t\right),
\end{eqnarray}
para $t\geq0$.
\end{Def}

Si se consideran los puntos $T_{n}$ como elementos de $\rea_{+}$, y $N\left(t\right)$ es el n\'umero de puntos en $\rea$. El proceso denotado por $\left\{N\left(t\right):t\geq0\right\}$, denotado por $N\left(t\right)$, es un proceso puntual en $\rea_{+}$. Los $T_{n}$ son los tiempos de ocurrencia, el proceso puntual $N\left(t\right)$ es simple si su n\'umero de ocurrencias son distintas: $0<T_{1}<T_{2}<\ldots$ casi seguramente.

\begin{Def}
Un proceso puntual $N\left(t\right)$ es un proceso de renovaci\'on si los tiempos de interocurrencia $\xi_{n}=T_{n}-T_{n-1}$, para $n\geq1$, son independientes e identicamente distribuidos con distribuci\'on $F$, donde $F\left(0\right)=0$ y $T_{0}=0$. Los $T_{n}$ son llamados tiempos de renovaci\'on, referente a la independencia o renovaci\'on de la informaci\'on estoc\'astica en estos tiempos. Los $\xi_{n}$ son los tiempos de inter-renovaci\'on, y $N\left(t\right)$ es el n\'umero de renovaciones en el intervalo $\left[0,t\right)$
\end{Def}


\begin{Note}
Para definir un proceso de renovaci\'on para cualquier contexto, solamente hay que especificar una distribuci\'on $F$, con $F\left(0\right)=0$, para los tiempos de inter-renovaci\'on. La funci\'on $F$ en turno degune las otra variables aleatorias. De manera formal, existe un espacio de probabilidad y una sucesi\'on de variables aleatorias $\xi_{1},\xi_{2},\ldots$ definidas en este con distribuci\'on $F$. Entonces las otras cantidades son $T_{n}=\sum_{k=1}^{n}\xi_{k}$ y $N\left(t\right)=\sum_{n=1}^{\infty}\indora\left(T_{n}\leq t\right)$, donde $T_{n}\rightarrow\infty$ casi seguramente por la Ley Fuerte de los Grandes N\'umeros.
\end{Note}







Los tiempos $T_{n}$ est\'an relacionados con los conteos de $N\left(t\right)$ por

\begin{eqnarray*}
\left\{N\left(t\right)\geq n\right\}&=&\left\{T_{n}\leq t\right\}\\
T_{N\left(t\right)}\leq &t&<T_{N\left(t\right)+1},
\end{eqnarray*}

adem\'as $N\left(T_{n}\right)=n$, y 

\begin{eqnarray*}
N\left(t\right)=\max\left\{n:T_{n}\leq t\right\}=\min\left\{n:T_{n+1}>t\right\}
\end{eqnarray*}

Por propiedades de la convoluci\'on se sabe que

\begin{eqnarray*}
P\left\{T_{n}\leq t\right\}=F^{n\star}\left(t\right)
\end{eqnarray*}
que es la $n$-\'esima convoluci\'on de $F$. Entonces 

\begin{eqnarray*}
\left\{N\left(t\right)\geq n\right\}&=&\left\{T_{n}\leq t\right\}\\
P\left\{N\left(t\right)\leq n\right\}&=&1-F^{\left(n+1\right)\star}\left(t\right)
\end{eqnarray*}

Adem\'as usando el hecho de que $\esp\left[N\left(t\right)\right]=\sum_{n=1}^{\infty}P\left\{N\left(t\right)\geq n\right\}$
se tiene que

\begin{eqnarray*}
\esp\left[N\left(t\right)\right]=\sum_{n=1}^{\infty}F^{n\star}\left(t\right)
\end{eqnarray*}

\begin{Prop}
Para cada $t\geq0$, la funci\'on generadora de momentos $\esp\left[e^{\alpha N\left(t\right)}\right]$ existe para alguna $\alpha$ en una vecindad del 0, y de aqu\'i que $\esp\left[N\left(t\right)^{m}\right]<\infty$, para $m\geq1$.
\end{Prop}


\begin{Note}
Si el primer tiempo de renovaci\'on $\xi_{1}$ no tiene la misma distribuci\'on que el resto de las $\xi_{n}$, para $n\geq2$, a $N\left(t\right)$ se le llama Proceso de Renovaci\'on retardado, donde si $\xi$ tiene distribuci\'on $G$, entonces el tiempo $T_{n}$ de la $n$-\'esima renovaci\'on tiene distribuci\'on $G\star F^{\left(n-1\right)\star}\left(t\right)$
\end{Note}


\begin{Teo}
Para una constante $\mu\leq\infty$ ( o variable aleatoria), las siguientes expresiones son equivalentes:

\begin{eqnarray}
lim_{n\rightarrow\infty}n^{-1}T_{n}&=&\mu,\textrm{ c.s.}\\
lim_{t\rightarrow\infty}t^{-1}N\left(t\right)&=&1/\mu,\textrm{ c.s.}
\end{eqnarray}
\end{Teo}


Es decir, $T_{n}$ satisface la Ley Fuerte de los Grandes N\'umeros s\'i y s\'olo s\'i $N\left/t\right)$ la cumple.


\begin{Coro}[Ley Fuerte de los Grandes N\'umeros para Procesos de Renovaci\'on]
Si $N\left(t\right)$ es un proceso de renovaci\'on cuyos tiempos de inter-renovaci\'on tienen media $\mu\leq\infty$, entonces
\begin{eqnarray}
t^{-1}N\left(t\right)\rightarrow 1/\mu,\textrm{ c.s. cuando }t\rightarrow\infty.
\end{eqnarray}

\end{Coro}


Considerar el proceso estoc\'astico de valores reales $\left\{Z\left(t\right):t\geq0\right\}$ en el mismo espacio de probabilidad que $N\left(t\right)$

\begin{Def}
Para el proceso $\left\{Z\left(t\right):t\geq0\right\}$ se define la fluctuaci\'on m\'axima de $Z\left(t\right)$ en el intervalo $\left(T_{n-1},T_{n}\right]$:
\begin{eqnarray*}
M_{n}=\sup_{T_{n-1}<t\leq T_{n}}|Z\left(t\right)-Z\left(T_{n-1}\right)|
\end{eqnarray*}
\end{Def}

\begin{Teo}
Sup\'ongase que $n^{-1}T_{n}\rightarrow\mu$ c.s. cuando $n\rightarrow\infty$, donde $\mu\leq\infty$ es una constante o variable aleatoria. Sea $a$ una constante o variable aleatoria que puede ser infinita cuando $\mu$ es finita, y considere las expresiones l\'imite:
\begin{eqnarray}
lim_{n\rightarrow\infty}n^{-1}Z\left(T_{n}\right)&=&a,\textrm{ c.s.}\\
lim_{t\rightarrow\infty}t^{-1}Z\left(t\right)&=&a/\mu,\textrm{ c.s.}
\end{eqnarray}
La segunda expresi\'on implica la primera. Conversamente, la primera implica la segunda si el proceso $Z\left(t\right)$ es creciente, o si $lim_{n\rightarrow\infty}n^{-1}M_{n}=0$ c.s.
\end{Teo}

\begin{Coro}
Si $N\left(t\right)$ es un proceso de renovaci\'on, y $\left(Z\left(T_{n}\right)-Z\left(T_{n-1}\right),M_{n}\right)$, para $n\geq1$, son variables aleatorias independientes e id\'enticamente distribuidas con media finita, entonces,
\begin{eqnarray}
lim_{t\rightarrow\infty}t^{-1}Z\left(t\right)\rightarrow\frac{\esp\left[Z\left(T_{1}\right)-Z\left(T_{0}\right)\right]}{\esp\left[T_{1}\right]},\textrm{ c.s. cuando  }t\rightarrow\infty.
\end{eqnarray}
\end{Coro}


Sup\'ongase que $N\left(t\right)$ es un proceso de renovaci\'on con distribuci\'on $F$ con media finita $\mu$.

\begin{Def}
La funci\'on de renovaci\'on asociada con la distribuci\'on $F$, del proceso $N\left(t\right)$, es
\begin{eqnarray*}
U\left(t\right)=\sum_{n=1}^{\infty}F^{n\star}\left(t\right),\textrm{   }t\geq0,
\end{eqnarray*}
donde $F^{0\star}\left(t\right)=\indora\left(t\geq0\right)$.
\end{Def}


\begin{Prop}
Sup\'ongase que la distribuci\'on de inter-renovaci\'on $F$ tiene densidad $f$. Entonces $U\left(t\right)$ tambi\'en tiene densidad, para $t>0$, y es $U^{'}\left(t\right)=\sum_{n=0}^{\infty}f^{n\star}\left(t\right)$. Adem\'as
\begin{eqnarray*}
\prob\left\{N\left(t\right)>N\left(t-\right)\right\}=0\textrm{,   }t\geq0.
\end{eqnarray*}
\end{Prop}

\begin{Def}
La Transformada de Laplace-Stieljes de $F$ est\'a dada por

\begin{eqnarray*}
\hat{F}\left(\alpha\right)=\int_{\rea_{+}}e^{-\alpha t}dF\left(t\right)\textrm{,  }\alpha\geq0.
\end{eqnarray*}
\end{Def}

Entonces

\begin{eqnarray*}
\hat{U}\left(\alpha\right)=\sum_{n=0}^{\infty}\hat{F^{n\star}}\left(\alpha\right)=\sum_{n=0}^{\infty}\hat{F}\left(\alpha\right)^{n}=\frac{1}{1-\hat{F}\left(\alpha\right)}.
\end{eqnarray*}


\begin{Prop}
La Transformada de Laplace $\hat{U}\left(\alpha\right)$ y $\hat{F}\left(\alpha\right)$ determina una a la otra de manera \'unica por la relaci\'on $\hat{U}\left(\alpha\right)=\frac{1}{1-\hat{F}\left(\alpha\right)}$.
\end{Prop}


\begin{Note}
Un proceso de renovaci\'on $N\left(t\right)$ cuyos tiempos de inter-renovaci\'on tienen media finita, es un proceso Poisson con tasa $\lambda$ si y s\'olo s\'i $\esp\left[U\left(t\right)\right]=\lambda t$, para $t\geq0$.
\end{Note}


\begin{Teo}
Sea $N\left(t\right)$ un proceso puntual simple con puntos de localizaci\'on $T_{n}$ tal que $\eta\left(t\right)=\esp\left[N\left(\right)\right]$ es finita para cada $t$. Entonces para cualquier funci\'on $f:\rea_{+}\rightarrow\rea$,
\begin{eqnarray*}
\esp\left[\sum_{n=1}^{N\left(\right)}f\left(T_{n}\right)\right]=\int_{\left(0,t\right]}f\left(s\right)d\eta\left(s\right)\textrm{,  }t\geq0,
\end{eqnarray*}
suponiendo que la integral exista. Adem\'as si $X_{1},X_{2},\ldots$ son variables aleatorias definidas en el mismo espacio de probabilidad que el proceso $N\left(t\right)$ tal que $\esp\left[X_{n}|T_{n}=s\right]=f\left(s\right)$, independiente de $n$. Entonces
\begin{eqnarray*}
\esp\left[\sum_{n=1}^{N\left(t\right)}X_{n}\right]=\int_{\left(0,t\right]}f\left(s\right)d\eta\left(s\right)\textrm{,  }t\geq0,
\end{eqnarray*} 
suponiendo que la integral exista. 
\end{Teo}

\begin{Coro}[Identidad de Wald para Renovaciones]
Para el proceso de renovaci\'on $N\left(t\right)$,
\begin{eqnarray*}
\esp\left[T_{N\left(t\right)+1}\right]=\mu\esp\left[N\left(t\right)+1\right]\textrm{,  }t\geq0,
\end{eqnarray*}  
\end{Coro}


\begin{Def}
Sea $h\left(t\right)$ funci\'on de valores reales en $\rea$ acotada en intervalos finitos e igual a cero para $t<0$ La ecuaci\'on de renovaci\'on para $h\left(t\right)$ y la distribuci\'on $F$ es

\begin{eqnarray}\label{Ec.Renovacion}
H\left(t\right)=h\left(t\right)+\int_{\left[0,t\right]}H\left(t-s\right)dF\left(s\right)\textrm{,    }t\geq0,
\end{eqnarray}
donde $H\left(t\right)$ es una funci\'on de valores reales. Esto es $H=h+F\star H$. Decimos que $H\left(t\right)$ es soluci\'on de esta ecuaci\'on si satisface la ecuaci\'on, y es acotada en intervalos finitos e iguales a cero para $t<0$.
\end{Def}

\begin{Prop}
La funci\'on $U\star h\left(t\right)$ es la \'unica soluci\'on de la ecuaci\'on de renovaci\'on (\ref{Ec.Renovacion}).
\end{Prop}

\begin{Teo}[Teorema Renovaci\'on Elemental]
\begin{eqnarray*}
t^{-1}U\left(t\right)\rightarrow 1/\mu\textrm{,    cuando }t\rightarrow\infty.
\end{eqnarray*}
\end{Teo}



Sup\'ongase que $N\left(t\right)$ es un proceso de renovaci\'on con distribuci\'on $F$ con media finita $\mu$.

\begin{Def}
La funci\'on de renovaci\'on asociada con la distribuci\'on $F$, del proceso $N\left(t\right)$, es
\begin{eqnarray*}
U\left(t\right)=\sum_{n=1}^{\infty}F^{n\star}\left(t\right),\textrm{   }t\geq0,
\end{eqnarray*}
donde $F^{0\star}\left(t\right)=\indora\left(t\geq0\right)$.
\end{Def}


\begin{Prop}
Sup\'ongase que la distribuci\'on de inter-renovaci\'on $F$ tiene densidad $f$. Entonces $U\left(t\right)$ tambi\'en tiene densidad, para $t>0$, y es $U^{'}\left(t\right)=\sum_{n=0}^{\infty}f^{n\star}\left(t\right)$. Adem\'as
\begin{eqnarray*}
\prob\left\{N\left(t\right)>N\left(t-\right)\right\}=0\textrm{,   }t\geq0.
\end{eqnarray*}
\end{Prop}

\begin{Def}
La Transformada de Laplace-Stieljes de $F$ est\'a dada por

\begin{eqnarray*}
\hat{F}\left(\alpha\right)=\int_{\rea_{+}}e^{-\alpha t}dF\left(t\right)\textrm{,  }\alpha\geq0.
\end{eqnarray*}
\end{Def}

Entonces

\begin{eqnarray*}
\hat{U}\left(\alpha\right)=\sum_{n=0}^{\infty}\hat{F^{n\star}}\left(\alpha\right)=\sum_{n=0}^{\infty}\hat{F}\left(\alpha\right)^{n}=\frac{1}{1-\hat{F}\left(\alpha\right)}.
\end{eqnarray*}


\begin{Prop}
La Transformada de Laplace $\hat{U}\left(\alpha\right)$ y $\hat{F}\left(\alpha\right)$ determina una a la otra de manera \'unica por la relaci\'on $\hat{U}\left(\alpha\right)=\frac{1}{1-\hat{F}\left(\alpha\right)}$.
\end{Prop}


\begin{Note}
Un proceso de renovaci\'on $N\left(t\right)$ cuyos tiempos de inter-renovaci\'on tienen media finita, es un proceso Poisson con tasa $\lambda$ si y s\'olo s\'i $\esp\left[U\left(t\right)\right]=\lambda t$, para $t\geq0$.
\end{Note}


\begin{Teo}
Sea $N\left(t\right)$ un proceso puntual simple con puntos de localizaci\'on $T_{n}$ tal que $\eta\left(t\right)=\esp\left[N\left(\right)\right]$ es finita para cada $t$. Entonces para cualquier funci\'on $f:\rea_{+}\rightarrow\rea$,
\begin{eqnarray*}
\esp\left[\sum_{n=1}^{N\left(\right)}f\left(T_{n}\right)\right]=\int_{\left(0,t\right]}f\left(s\right)d\eta\left(s\right)\textrm{,  }t\geq0,
\end{eqnarray*}
suponiendo que la integral exista. Adem\'as si $X_{1},X_{2},\ldots$ son variables aleatorias definidas en el mismo espacio de probabilidad que el proceso $N\left(t\right)$ tal que $\esp\left[X_{n}|T_{n}=s\right]=f\left(s\right)$, independiente de $n$. Entonces
\begin{eqnarray*}
\esp\left[\sum_{n=1}^{N\left(t\right)}X_{n}\right]=\int_{\left(0,t\right]}f\left(s\right)d\eta\left(s\right)\textrm{,  }t\geq0,
\end{eqnarray*} 
suponiendo que la integral exista. 
\end{Teo}

\begin{Coro}[Identidad de Wald para Renovaciones]
Para el proceso de renovaci\'on $N\left(t\right)$,
\begin{eqnarray*}
\esp\left[T_{N\left(t\right)+1}\right]=\mu\esp\left[N\left(t\right)+1\right]\textrm{,  }t\geq0,
\end{eqnarray*}  
\end{Coro}


\begin{Def}
Sea $h\left(t\right)$ funci\'on de valores reales en $\rea$ acotada en intervalos finitos e igual a cero para $t<0$ La ecuaci\'on de renovaci\'on para $h\left(t\right)$ y la distribuci\'on $F$ es

\begin{eqnarray}\label{Ec.Renovacion}
H\left(t\right)=h\left(t\right)+\int_{\left[0,t\right]}H\left(t-s\right)dF\left(s\right)\textrm{,    }t\geq0,
\end{eqnarray}
donde $H\left(t\right)$ es una funci\'on de valores reales. Esto es $H=h+F\star H$. Decimos que $H\left(t\right)$ es soluci\'on de esta ecuaci\'on si satisface la ecuaci\'on, y es acotada en intervalos finitos e iguales a cero para $t<0$.
\end{Def}

\begin{Prop}
La funci\'on $U\star h\left(t\right)$ es la \'unica soluci\'on de la ecuaci\'on de renovaci\'on (\ref{Ec.Renovacion}).
\end{Prop}

\begin{Teo}[Teorema Renovaci\'on Elemental]
\begin{eqnarray*}
t^{-1}U\left(t\right)\rightarrow 1/\mu\textrm{,    cuando }t\rightarrow\infty.
\end{eqnarray*}
\end{Teo}


\begin{Note} Una funci\'on $h:\rea_{+}\rightarrow\rea$ es Directamente Riemann Integrable en los siguientes casos:
\begin{itemize}
\item[a)] $h\left(t\right)\geq0$ es decreciente y Riemann Integrable.
\item[b)] $h$ es continua excepto posiblemente en un conjunto de Lebesgue de medida 0, y $|h\left(t\right)|\leq b\left(t\right)$, donde $b$ es DRI.
\end{itemize}
\end{Note}

\begin{Teo}[Teorema Principal de Renovaci\'on]
Si $F$ es no aritm\'etica y $h\left(t\right)$ es Directamente Riemann Integrable (DRI), entonces

\begin{eqnarray*}
lim_{t\rightarrow\infty}U\star h=\frac{1}{\mu}\int_{\rea_{+}}h\left(s\right)ds.
\end{eqnarray*}
\end{Teo}

\begin{Prop}
Cualquier funci\'on $H\left(t\right)$ acotada en intervalos finitos y que es 0 para $t<0$ puede expresarse como
\begin{eqnarray*}
H\left(t\right)=U\star h\left(t\right)\textrm{,  donde }h\left(t\right)=H\left(t\right)-F\star H\left(t\right)
\end{eqnarray*}
\end{Prop}

\begin{Def}
Un proceso estoc\'astico $X\left(t\right)$ es crudamente regenerativo en un tiempo aleatorio positivo $T$ si
\begin{eqnarray*}
\esp\left[X\left(T+t\right)|T\right]=\esp\left[X\left(t\right)\right]\textrm{, para }t\geq0,\end{eqnarray*}
y con las esperanzas anteriores finitas.
\end{Def}

\begin{Prop}
Sup\'ongase que $X\left(t\right)$ es un proceso crudamente regenerativo en $T$, que tiene distribuci\'on $F$. Si $\esp\left[X\left(t\right)\right]$ es acotado en intervalos finitos, entonces
\begin{eqnarray*}
\esp\left[X\left(t\right)\right]=U\star h\left(t\right)\textrm{,  donde }h\left(t\right)=\esp\left[X\left(t\right)\indora\left(T>t\right)\right].
\end{eqnarray*}
\end{Prop}

\begin{Teo}[Regeneraci\'on Cruda]
Sup\'ongase que $X\left(t\right)$ es un proceso con valores positivo crudamente regenerativo en $T$, y def\'inase $M=\sup\left\{|X\left(t\right)|:t\leq T\right\}$. Si $T$ es no aritm\'etico y $M$ y $MT$ tienen media finita, entonces
\begin{eqnarray*}
lim_{t\rightarrow\infty}\esp\left[X\left(t\right)\right]=\frac{1}{\mu}\int_{\rea_{+}}h\left(s\right)ds,
\end{eqnarray*}
donde $h\left(t\right)=\esp\left[X\left(t\right)\indora\left(T>t\right)\right]$.
\end{Teo}


\begin{Note} Una funci\'on $h:\rea_{+}\rightarrow\rea$ es Directamente Riemann Integrable en los siguientes casos:
\begin{itemize}
\item[a)] $h\left(t\right)\geq0$ es decreciente y Riemann Integrable.
\item[b)] $h$ es continua excepto posiblemente en un conjunto de Lebesgue de medida 0, y $|h\left(t\right)|\leq b\left(t\right)$, donde $b$ es DRI.
\end{itemize}
\end{Note}

\begin{Teo}[Teorema Principal de Renovaci\'on]
Si $F$ es no aritm\'etica y $h\left(t\right)$ es Directamente Riemann Integrable (DRI), entonces

\begin{eqnarray*}
lim_{t\rightarrow\infty}U\star h=\frac{1}{\mu}\int_{\rea_{+}}h\left(s\right)ds.
\end{eqnarray*}
\end{Teo}

\begin{Prop}
Cualquier funci\'on $H\left(t\right)$ acotada en intervalos finitos y que es 0 para $t<0$ puede expresarse como
\begin{eqnarray*}
H\left(t\right)=U\star h\left(t\right)\textrm{,  donde }h\left(t\right)=H\left(t\right)-F\star H\left(t\right)
\end{eqnarray*}
\end{Prop}

\begin{Def}
Un proceso estoc\'astico $X\left(t\right)$ es crudamente regenerativo en un tiempo aleatorio positivo $T$ si
\begin{eqnarray*}
\esp\left[X\left(T+t\right)|T\right]=\esp\left[X\left(t\right)\right]\textrm{, para }t\geq0,\end{eqnarray*}
y con las esperanzas anteriores finitas.
\end{Def}

\begin{Prop}
Sup\'ongase que $X\left(t\right)$ es un proceso crudamente regenerativo en $T$, que tiene distribuci\'on $F$. Si $\esp\left[X\left(t\right)\right]$ es acotado en intervalos finitos, entonces
\begin{eqnarray*}
\esp\left[X\left(t\right)\right]=U\star h\left(t\right)\textrm{,  donde }h\left(t\right)=\esp\left[X\left(t\right)\indora\left(T>t\right)\right].
\end{eqnarray*}
\end{Prop}

\begin{Teo}[Regeneraci\'on Cruda]
Sup\'ongase que $X\left(t\right)$ es un proceso con valores positivo crudamente regenerativo en $T$, y def\'inase $M=\sup\left\{|X\left(t\right)|:t\leq T\right\}$. Si $T$ es no aritm\'etico y $M$ y $MT$ tienen media finita, entonces
\begin{eqnarray*}
lim_{t\rightarrow\infty}\esp\left[X\left(t\right)\right]=\frac{1}{\mu}\int_{\rea_{+}}h\left(s\right)ds,
\end{eqnarray*}
donde $h\left(t\right)=\esp\left[X\left(t\right)\indora\left(T>t\right)\right]$.
\end{Teo}

%________________________________________________________________________
\section{Procesos Regenerativos}
%________________________________________________________________________

Para $\left\{X\left(t\right):t\geq0\right\}$ Proceso Estoc\'astico a tiempo continuo con estado de espacios $S$, que es un espacio m\'etrico, con trayectorias continuas por la derecha y con l\'imites por la izquierda c.s. Sea $N\left(t\right)$ un proceso de renovaci\'on en $\rea_{+}$ definido en el mismo espacio de probabilidad que $X\left(t\right)$, con tiempos de renovaci\'on $T$ y tiempos de inter-renovaci\'on $\xi_{n}=T_{n}-T_{n-1}$, con misma distribuci\'on $F$ de media finita $\mu$.



\begin{Def}
Para el proceso $\left\{\left(N\left(t\right),X\left(t\right)\right):t\geq0\right\}$, sus trayectoria muestrales en el intervalo de tiempo $\left[T_{n-1},T_{n}\right)$ est\'an descritas por
\begin{eqnarray*}
\zeta_{n}=\left(\xi_{n},\left\{X\left(T_{n-1}+t\right):0\leq t<\xi_{n}\right\}\right)
\end{eqnarray*}
Este $\zeta_{n}$ es el $n$-\'esimo segmento del proceso. El proceso es regenerativo sobre los tiempos $T_{n}$ si sus segmentos $\zeta_{n}$ son independientes e id\'enticamennte distribuidos.
\end{Def}


\begin{Obs}
Si $\tilde{X}\left(t\right)$ con espacio de estados $\tilde{S}$ es regenerativo sobre $T_{n}$, entonces $X\left(t\right)=f\left(\tilde{X}\left(t\right)\right)$ tambi\'en es regenerativo sobre $T_{n}$, para cualquier funci\'on $f:\tilde{S}\rightarrow S$.
\end{Obs}

\begin{Obs}
Los procesos regenerativos son crudamente regenerativos, pero no al rev\'es.
\end{Obs}

\begin{Def}[Definici\'on Cl\'asica]
Un proceso estoc\'astico $X=\left\{X\left(t\right):t\geq0\right\}$ es llamado regenerativo is existe una variable aleatoria $R_{1}>0$ tal que
\begin{itemize}
\item[i)] $\left\{X\left(t+R_{1}\right):t\geq0\right\}$ es independiente de $\left\{\left\{X\left(t\right):t<R_{1}\right\},\right\}$
\item[ii)] $\left\{X\left(t+R_{1}\right):t\geq0\right\}$ es estoc\'asticamente equivalente a $\left\{X\left(t\right):t>0\right\}$
\end{itemize}

Llamamos a $R_{1}$ tiempo de regeneraci\'on, y decimos que $X$ se regenera en este punto.
\end{Def}

$\left\{X\left(t+R_{1}\right)\right\}$ es regenerativo con tiempo de regeneraci\'on $R_{2}$, independiente de $R_{1}$ pero con la misma distribuci\'on que $R_{1}$. Procediendo de esta manera se obtiene una secuencia de variables aleatorias independientes e id\'enticamente distribuidas $\left\{R_{n}\right\}$ llamados longitudes de ciclo. Si definimos a $Z_{k}\equiv R_{1}+R_{2}+\cdots+R_{k}$, se tiene un proceso de renovaci\'on llamado proceso de renovaci\'on encajado para $X$.

\begin{Note}
Un proceso regenerativo con media de la longitud de ciclo finita es llamado positivo recurrente.
\end{Note}


\begin{Def}
Para $x$ fijo y para cada $t\geq0$, sea $I_{x}\left(t\right)=1$ si $X\left(t\right)\leq x$,  $I_{x}\left(t\right)=0$ en caso contrario, y def\'inanse los tiempos promedio
\begin{eqnarray*}
\overline{X}&=&lim_{t\rightarrow\infty}\frac{1}{t}\int_{0}^{\infty}X\left(u\right)du\\
\prob\left(X_{\infty}\leq x\right)&=&lim_{t\rightarrow\infty}\frac{1}{t}\int_{0}^{\infty}I_{x}\left(u\right)du,
\end{eqnarray*}
cuando estos l\'imites existan.
\end{Def}

Como consecuencia del teorema de Renovaci\'on-Recompensa, se tiene que el primer l\'imite  existe y es igual a la constante
\begin{eqnarray*}
\overline{X}&=&\frac{\esp\left[\int_{0}^{R_{1}}X\left(t\right)dt\right]}{\esp\left[R_{1}\right]},
\end{eqnarray*}
suponiendo que ambas esperanzas son finitas.

\begin{Note}
\begin{itemize}
\item[a)] Si el proceso regenerativo $X$ es positivo recurrente y tiene trayectorias muestrales no negativas, entonces la ecuaci\'on anterior es v\'alida.
\item[b)] Si $X$ es positivo recurrente regenerativo, podemos construir una \'unica versi\'on estacionaria de este proceso, $X_{e}=\left\{X_{e}\left(t\right)\right\}$, donde $X_{e}$ es un proceso estoc\'astico regenerativo y estrictamente estacionario, con distribuci\'on marginal distribuida como $X_{\infty}$
\end{itemize}
\end{Note}

%________________________________________________________________________
\subsection{Procesos Regenerativos}
%________________________________________________________________________

Para $\left\{X\left(t\right):t\geq0\right\}$ Proceso Estoc\'astico a tiempo continuo con estado de espacios $S$, que es un espacio m\'etrico, con trayectorias continuas por la derecha y con l\'imites por la izquierda c.s. Sea $N\left(t\right)$ un proceso de renovaci\'on en $\rea_{+}$ definido en el mismo espacio de probabilidad que $X\left(t\right)$, con tiempos de renovaci\'on $T$ y tiempos de inter-renovaci\'on $\xi_{n}=T_{n}-T_{n-1}$, con misma distribuci\'on $F$ de media finita $\mu$.



\begin{Def}
Para el proceso $\left\{\left(N\left(t\right),X\left(t\right)\right):t\geq0\right\}$, sus trayectoria muestrales en el intervalo de tiempo $\left[T_{n-1},T_{n}\right)$ est\'an descritas por
\begin{eqnarray*}
\zeta_{n}=\left(\xi_{n},\left\{X\left(T_{n-1}+t\right):0\leq t<\xi_{n}\right\}\right)
\end{eqnarray*}
Este $\zeta_{n}$ es el $n$-\'esimo segmento del proceso. El proceso es regenerativo sobre los tiempos $T_{n}$ si sus segmentos $\zeta_{n}$ son independientes e id\'enticamennte distribuidos.
\end{Def}


\begin{Obs}
Si $\tilde{X}\left(t\right)$ con espacio de estados $\tilde{S}$ es regenerativo sobre $T_{n}$, entonces $X\left(t\right)=f\left(\tilde{X}\left(t\right)\right)$ tambi\'en es regenerativo sobre $T_{n}$, para cualquier funci\'on $f:\tilde{S}\rightarrow S$.
\end{Obs}

\begin{Obs}
Los procesos regenerativos son crudamente regenerativos, pero no al rev\'es.
\end{Obs}

\begin{Def}[Definici\'on Cl\'asica]
Un proceso estoc\'astico $X=\left\{X\left(t\right):t\geq0\right\}$ es llamado regenerativo is existe una variable aleatoria $R_{1}>0$ tal que
\begin{itemize}
\item[i)] $\left\{X\left(t+R_{1}\right):t\geq0\right\}$ es independiente de $\left\{\left\{X\left(t\right):t<R_{1}\right\},\right\}$
\item[ii)] $\left\{X\left(t+R_{1}\right):t\geq0\right\}$ es estoc\'asticamente equivalente a $\left\{X\left(t\right):t>0\right\}$
\end{itemize}

Llamamos a $R_{1}$ tiempo de regeneraci\'on, y decimos que $X$ se regenera en este punto.
\end{Def}

$\left\{X\left(t+R_{1}\right)\right\}$ es regenerativo con tiempo de regeneraci\'on $R_{2}$, independiente de $R_{1}$ pero con la misma distribuci\'on que $R_{1}$. Procediendo de esta manera se obtiene una secuencia de variables aleatorias independientes e id\'enticamente distribuidas $\left\{R_{n}\right\}$ llamados longitudes de ciclo. Si definimos a $Z_{k}\equiv R_{1}+R_{2}+\cdots+R_{k}$, se tiene un proceso de renovaci\'on llamado proceso de renovaci\'on encajado para $X$.

\begin{Note}
Un proceso regenerativo con media de la longitud de ciclo finita es llamado positivo recurrente.
\end{Note}


\begin{Def}
Para $x$ fijo y para cada $t\geq0$, sea $I_{x}\left(t\right)=1$ si $X\left(t\right)\leq x$,  $I_{x}\left(t\right)=0$ en caso contrario, y def\'inanse los tiempos promedio
\begin{eqnarray*}
\overline{X}&=&lim_{t\rightarrow\infty}\frac{1}{t}\int_{0}^{\infty}X\left(u\right)du\\
\prob\left(X_{\infty}\leq x\right)&=&lim_{t\rightarrow\infty}\frac{1}{t}\int_{0}^{\infty}I_{x}\left(u\right)du,
\end{eqnarray*}
cuando estos l\'imites existan.
\end{Def}

Como consecuencia del teorema de Renovaci\'on-Recompensa, se tiene que el primer l\'imite  existe y es igual a la constante
\begin{eqnarray*}
\overline{X}&=&\frac{\esp\left[\int_{0}^{R_{1}}X\left(t\right)dt\right]}{\esp\left[R_{1}\right]},
\end{eqnarray*}
suponiendo que ambas esperanzas son finitas.

\begin{Note}
\begin{itemize}
\item[a)] Si el proceso regenerativo $X$ es positivo recurrente y tiene trayectorias muestrales no negativas, entonces la ecuaci\'on anterior es v\'alida.
\item[b)] Si $X$ es positivo recurrente regenerativo, podemos construir una \'unica versi\'on estacionaria de este proceso, $X_{e}=\left\{X_{e}\left(t\right)\right\}$, donde $X_{e}$ es un proceso estoc\'astico regenerativo y estrictamente estacionario, con distribuci\'on marginal distribuida como $X_{\infty}$
\end{itemize}
\end{Note}
%__________________________________________________________________________________________
\section{Procesos Regenerativos Estacionarios - Stidham \cite{Stidham}}
%__________________________________________________________________________________________


Un proceso estoc\'astico a tiempo continuo $\left\{V\left(t\right),t\geq0\right\}$ es un proceso regenerativo si existe una sucesi\'on de variables aleatorias independientes e id\'enticamente distribuidas $\left\{X_{1},X_{2},\ldots\right\}$, sucesi\'on de renovaci\'on, tal que para cualquier conjunto de Borel $A$, 

\begin{eqnarray*}
\prob\left\{V\left(t\right)\in A|X_{1}+X_{2}+\cdots+X_{R\left(t\right)}=s,\left\{V\left(\tau\right),\tau<s\right\}\right\}=\prob\left\{V\left(t-s\right)\in A|X_{1}>t-s\right\},
\end{eqnarray*}
para todo $0\leq s\leq t$, donde $R\left(t\right)=\max\left\{X_{1}+X_{2}+\cdots+X_{j}\leq t\right\}=$n\'umero de renovaciones ({\emph{puntos de regeneraci\'on}}) que ocurren en $\left[0,t\right]$. El intervalo $\left[0,X_{1}\right)$ es llamado {\emph{primer ciclo de regeneraci\'on}} de $\left\{V\left(t \right),t\geq0\right\}$, $\left[X_{1},X_{1}+X_{2}\right)$ el {\emph{segundo ciclo de regeneraci\'on}}, y as\'i sucesivamente.

Sea $X=X_{1}$ y sea $F$ la funci\'on de distrbuci\'on de $X$


\begin{Def}
Se define el proceso estacionario, $\left\{V^{*}\left(t\right),t\geq0\right\}$, para $\left\{V\left(t\right),t\geq0\right\}$ por

\begin{eqnarray*}
\prob\left\{V\left(t\right)\in A\right\}=\frac{1}{\esp\left[X\right]}\int_{0}^{\infty}\prob\left\{V\left(t+x\right)\in A|X>x\right\}\left(1-F\left(x\right)\right)dx,
\end{eqnarray*} 
para todo $t\geq0$ y todo conjunto de Borel $A$.
\end{Def}

\begin{Def}
Una distribuci\'on se dice que es {\emph{aritm\'etica}} si todos sus puntos de incremento son m\'ultiplos de la forma $0,\lambda, 2\lambda,\ldots$ para alguna $\lambda>0$ entera.
\end{Def}


\begin{Def}
Una modificaci\'on medible de un proceso $\left\{V\left(t\right),t\geq0\right\}$, es una versi\'on de este, $\left\{V\left(t,w\right)\right\}$ conjuntamente medible para $t\geq0$ y para $w\in S$, $S$ espacio de estados para $\left\{V\left(t\right),t\geq0\right\}$.
\end{Def}

\begin{Teo}
Sea $\left\{V\left(t\right),t\geq\right\}$ un proceso regenerativo no negativo con modificaci\'on medible. Sea $\esp\left[X\right]<\infty$. Entonces el proceso estacionario dado por la ecuaci\'on anterior est\'a bien definido y tiene funci\'on de distribuci\'on independiente de $t$, adem\'as
\begin{itemize}
\item[i)] \begin{eqnarray*}
\esp\left[V^{*}\left(0\right)\right]&=&\frac{\esp\left[\int_{0}^{X}V\left(s\right)ds\right]}{\esp\left[X\right]}\end{eqnarray*}
\item[ii)] Si $\esp\left[V^{*}\left(0\right)\right]<\infty$, equivalentemente, si $\esp\left[\int_{0}^{X}V\left(s\right)ds\right]<\infty$,entonces
\begin{eqnarray*}
\frac{\int_{0}^{t}V\left(s\right)ds}{t}\rightarrow\frac{\esp\left[\int_{0}^{X}V\left(s\right)ds\right]}{\esp\left[X\right]}
\end{eqnarray*}
con probabilidad 1 y en media, cuando $t\rightarrow\infty$.
\end{itemize}
\end{Teo}


%__________________________________________________________________________________________
\subsection{Procesos Regenerativos Estacionarios - Stidham \cite{Stidham}}
%__________________________________________________________________________________________


Un proceso estoc\'astico a tiempo continuo $\left\{V\left(t\right),t\geq0\right\}$ es un proceso regenerativo si existe una sucesi\'on de variables aleatorias independientes e id\'enticamente distribuidas $\left\{X_{1},X_{2},\ldots\right\}$, sucesi\'on de renovaci\'on, tal que para cualquier conjunto de Borel $A$, 

\begin{eqnarray*}
\prob\left\{V\left(t\right)\in A|X_{1}+X_{2}+\cdots+X_{R\left(t\right)}=s,\left\{V\left(\tau\right),\tau<s\right\}\right\}=\prob\left\{V\left(t-s\right)\in A|X_{1}>t-s\right\},
\end{eqnarray*}
para todo $0\leq s\leq t$, donde $R\left(t\right)=\max\left\{X_{1}+X_{2}+\cdots+X_{j}\leq t\right\}=$n\'umero de renovaciones ({\emph{puntos de regeneraci\'on}}) que ocurren en $\left[0,t\right]$. El intervalo $\left[0,X_{1}\right)$ es llamado {\emph{primer ciclo de regeneraci\'on}} de $\left\{V\left(t \right),t\geq0\right\}$, $\left[X_{1},X_{1}+X_{2}\right)$ el {\emph{segundo ciclo de regeneraci\'on}}, y as\'i sucesivamente.

Sea $X=X_{1}$ y sea $F$ la funci\'on de distrbuci\'on de $X$


\begin{Def}
Se define el proceso estacionario, $\left\{V^{*}\left(t\right),t\geq0\right\}$, para $\left\{V\left(t\right),t\geq0\right\}$ por

\begin{eqnarray*}
\prob\left\{V\left(t\right)\in A\right\}=\frac{1}{\esp\left[X\right]}\int_{0}^{\infty}\prob\left\{V\left(t+x\right)\in A|X>x\right\}\left(1-F\left(x\right)\right)dx,
\end{eqnarray*} 
para todo $t\geq0$ y todo conjunto de Borel $A$.
\end{Def}

\begin{Def}
Una distribuci\'on se dice que es {\emph{aritm\'etica}} si todos sus puntos de incremento son m\'ultiplos de la forma $0,\lambda, 2\lambda,\ldots$ para alguna $\lambda>0$ entera.
\end{Def}


\begin{Def}
Una modificaci\'on medible de un proceso $\left\{V\left(t\right),t\geq0\right\}$, es una versi\'on de este, $\left\{V\left(t,w\right)\right\}$ conjuntamente medible para $t\geq0$ y para $w\in S$, $S$ espacio de estados para $\left\{V\left(t\right),t\geq0\right\}$.
\end{Def}

\begin{Teo}
Sea $\left\{V\left(t\right),t\geq\right\}$ un proceso regenerativo no negativo con modificaci\'on medible. Sea $\esp\left[X\right]<\infty$. Entonces el proceso estacionario dado por la ecuaci\'on anterior est\'a bien definido y tiene funci\'on de distribuci\'on independiente de $t$, adem\'as
\begin{itemize}
\item[i)] \begin{eqnarray*}
\esp\left[V^{*}\left(0\right)\right]&=&\frac{\esp\left[\int_{0}^{X}V\left(s\right)ds\right]}{\esp\left[X\right]}\end{eqnarray*}
\item[ii)] Si $\esp\left[V^{*}\left(0\right)\right]<\infty$, equivalentemente, si $\esp\left[\int_{0}^{X}V\left(s\right)ds\right]<\infty$,entonces
\begin{eqnarray*}
\frac{\int_{0}^{t}V\left(s\right)ds}{t}\rightarrow\frac{\esp\left[\int_{0}^{X}V\left(s\right)ds\right]}{\esp\left[X\right]}
\end{eqnarray*}
con probabilidad 1 y en media, cuando $t\rightarrow\infty$.
\end{itemize}
\end{Teo}
%
%___________________________________________________________________________________________
%\vspace{5.5cm}
%\chapter{Cadenas de Markov estacionarias}
%\vspace{-1.0cm}
%___________________________________________________________________________________________
%
\subsection{Propiedades de los Procesos de Renovaci\'on}
%___________________________________________________________________________________________
%

Los tiempos $T_{n}$ est\'an relacionados con los conteos de $N\left(t\right)$ por

\begin{eqnarray*}
\left\{N\left(t\right)\geq n\right\}&=&\left\{T_{n}\leq t\right\}\\
T_{N\left(t\right)}\leq &t&<T_{N\left(t\right)+1},
\end{eqnarray*}

adem\'as $N\left(T_{n}\right)=n$, y 

\begin{eqnarray*}
N\left(t\right)=\max\left\{n:T_{n}\leq t\right\}=\min\left\{n:T_{n+1}>t\right\}
\end{eqnarray*}

Por propiedades de la convoluci\'on se sabe que

\begin{eqnarray*}
P\left\{T_{n}\leq t\right\}=F^{n\star}\left(t\right)
\end{eqnarray*}
que es la $n$-\'esima convoluci\'on de $F$. Entonces 

\begin{eqnarray*}
\left\{N\left(t\right)\geq n\right\}&=&\left\{T_{n}\leq t\right\}\\
P\left\{N\left(t\right)\leq n\right\}&=&1-F^{\left(n+1\right)\star}\left(t\right)
\end{eqnarray*}

Adem\'as usando el hecho de que $\esp\left[N\left(t\right)\right]=\sum_{n=1}^{\infty}P\left\{N\left(t\right)\geq n\right\}$
se tiene que

\begin{eqnarray*}
\esp\left[N\left(t\right)\right]=\sum_{n=1}^{\infty}F^{n\star}\left(t\right)
\end{eqnarray*}

\begin{Prop}
Para cada $t\geq0$, la funci\'on generadora de momentos $\esp\left[e^{\alpha N\left(t\right)}\right]$ existe para alguna $\alpha$ en una vecindad del 0, y de aqu\'i que $\esp\left[N\left(t\right)^{m}\right]<\infty$, para $m\geq1$.
\end{Prop}


\begin{Note}
Si el primer tiempo de renovaci\'on $\xi_{1}$ no tiene la misma distribuci\'on que el resto de las $\xi_{n}$, para $n\geq2$, a $N\left(t\right)$ se le llama Proceso de Renovaci\'on retardado, donde si $\xi$ tiene distribuci\'on $G$, entonces el tiempo $T_{n}$ de la $n$-\'esima renovaci\'on tiene distribuci\'on $G\star F^{\left(n-1\right)\star}\left(t\right)$
\end{Note}


\begin{Teo}
Para una constante $\mu\leq\infty$ ( o variable aleatoria), las siguientes expresiones son equivalentes:

\begin{eqnarray}
lim_{n\rightarrow\infty}n^{-1}T_{n}&=&\mu,\textrm{ c.s.}\\
lim_{t\rightarrow\infty}t^{-1}N\left(t\right)&=&1/\mu,\textrm{ c.s.}
\end{eqnarray}
\end{Teo}


Es decir, $T_{n}$ satisface la Ley Fuerte de los Grandes N\'umeros s\'i y s\'olo s\'i $N\left/t\right)$ la cumple.


\begin{Coro}[Ley Fuerte de los Grandes N\'umeros para Procesos de Renovaci\'on]
Si $N\left(t\right)$ es un proceso de renovaci\'on cuyos tiempos de inter-renovaci\'on tienen media $\mu\leq\infty$, entonces
\begin{eqnarray}
t^{-1}N\left(t\right)\rightarrow 1/\mu,\textrm{ c.s. cuando }t\rightarrow\infty.
\end{eqnarray}

\end{Coro}


Considerar el proceso estoc\'astico de valores reales $\left\{Z\left(t\right):t\geq0\right\}$ en el mismo espacio de probabilidad que $N\left(t\right)$

\begin{Def}
Para el proceso $\left\{Z\left(t\right):t\geq0\right\}$ se define la fluctuaci\'on m\'axima de $Z\left(t\right)$ en el intervalo $\left(T_{n-1},T_{n}\right]$:
\begin{eqnarray*}
M_{n}=\sup_{T_{n-1}<t\leq T_{n}}|Z\left(t\right)-Z\left(T_{n-1}\right)|
\end{eqnarray*}
\end{Def}

\begin{Teo}
Sup\'ongase que $n^{-1}T_{n}\rightarrow\mu$ c.s. cuando $n\rightarrow\infty$, donde $\mu\leq\infty$ es una constante o variable aleatoria. Sea $a$ una constante o variable aleatoria que puede ser infinita cuando $\mu$ es finita, y considere las expresiones l\'imite:
\begin{eqnarray}
lim_{n\rightarrow\infty}n^{-1}Z\left(T_{n}\right)&=&a,\textrm{ c.s.}\\
lim_{t\rightarrow\infty}t^{-1}Z\left(t\right)&=&a/\mu,\textrm{ c.s.}
\end{eqnarray}
La segunda expresi\'on implica la primera. Conversamente, la primera implica la segunda si el proceso $Z\left(t\right)$ es creciente, o si $lim_{n\rightarrow\infty}n^{-1}M_{n}=0$ c.s.
\end{Teo}

\begin{Coro}
Si $N\left(t\right)$ es un proceso de renovaci\'on, y $\left(Z\left(T_{n}\right)-Z\left(T_{n-1}\right),M_{n}\right)$, para $n\geq1$, son variables aleatorias independientes e id\'enticamente distribuidas con media finita, entonces,
\begin{eqnarray}
lim_{t\rightarrow\infty}t^{-1}Z\left(t\right)\rightarrow\frac{\esp\left[Z\left(T_{1}\right)-Z\left(T_{0}\right)\right]}{\esp\left[T_{1}\right]},\textrm{ c.s. cuando  }t\rightarrow\infty.
\end{eqnarray}
\end{Coro}


%___________________________________________________________________________________________
%
\subsection{Propiedades de los Procesos de Renovaci\'on}
%___________________________________________________________________________________________
%

Los tiempos $T_{n}$ est\'an relacionados con los conteos de $N\left(t\right)$ por

\begin{eqnarray*}
\left\{N\left(t\right)\geq n\right\}&=&\left\{T_{n}\leq t\right\}\\
T_{N\left(t\right)}\leq &t&<T_{N\left(t\right)+1},
\end{eqnarray*}

adem\'as $N\left(T_{n}\right)=n$, y 

\begin{eqnarray*}
N\left(t\right)=\max\left\{n:T_{n}\leq t\right\}=\min\left\{n:T_{n+1}>t\right\}
\end{eqnarray*}

Por propiedades de la convoluci\'on se sabe que

\begin{eqnarray*}
P\left\{T_{n}\leq t\right\}=F^{n\star}\left(t\right)
\end{eqnarray*}
que es la $n$-\'esima convoluci\'on de $F$. Entonces 

\begin{eqnarray*}
\left\{N\left(t\right)\geq n\right\}&=&\left\{T_{n}\leq t\right\}\\
P\left\{N\left(t\right)\leq n\right\}&=&1-F^{\left(n+1\right)\star}\left(t\right)
\end{eqnarray*}

Adem\'as usando el hecho de que $\esp\left[N\left(t\right)\right]=\sum_{n=1}^{\infty}P\left\{N\left(t\right)\geq n\right\}$
se tiene que

\begin{eqnarray*}
\esp\left[N\left(t\right)\right]=\sum_{n=1}^{\infty}F^{n\star}\left(t\right)
\end{eqnarray*}

\begin{Prop}
Para cada $t\geq0$, la funci\'on generadora de momentos $\esp\left[e^{\alpha N\left(t\right)}\right]$ existe para alguna $\alpha$ en una vecindad del 0, y de aqu\'i que $\esp\left[N\left(t\right)^{m}\right]<\infty$, para $m\geq1$.
\end{Prop}


\begin{Note}
Si el primer tiempo de renovaci\'on $\xi_{1}$ no tiene la misma distribuci\'on que el resto de las $\xi_{n}$, para $n\geq2$, a $N\left(t\right)$ se le llama Proceso de Renovaci\'on retardado, donde si $\xi$ tiene distribuci\'on $G$, entonces el tiempo $T_{n}$ de la $n$-\'esima renovaci\'on tiene distribuci\'on $G\star F^{\left(n-1\right)\star}\left(t\right)$
\end{Note}


\begin{Teo}
Para una constante $\mu\leq\infty$ ( o variable aleatoria), las siguientes expresiones son equivalentes:

\begin{eqnarray}
lim_{n\rightarrow\infty}n^{-1}T_{n}&=&\mu,\textrm{ c.s.}\\
lim_{t\rightarrow\infty}t^{-1}N\left(t\right)&=&1/\mu,\textrm{ c.s.}
\end{eqnarray}
\end{Teo}


Es decir, $T_{n}$ satisface la Ley Fuerte de los Grandes N\'umeros s\'i y s\'olo s\'i $N\left/t\right)$ la cumple.


\begin{Coro}[Ley Fuerte de los Grandes N\'umeros para Procesos de Renovaci\'on]
Si $N\left(t\right)$ es un proceso de renovaci\'on cuyos tiempos de inter-renovaci\'on tienen media $\mu\leq\infty$, entonces
\begin{eqnarray}
t^{-1}N\left(t\right)\rightarrow 1/\mu,\textrm{ c.s. cuando }t\rightarrow\infty.
\end{eqnarray}

\end{Coro}


Considerar el proceso estoc\'astico de valores reales $\left\{Z\left(t\right):t\geq0\right\}$ en el mismo espacio de probabilidad que $N\left(t\right)$

\begin{Def}
Para el proceso $\left\{Z\left(t\right):t\geq0\right\}$ se define la fluctuaci\'on m\'axima de $Z\left(t\right)$ en el intervalo $\left(T_{n-1},T_{n}\right]$:
\begin{eqnarray*}
M_{n}=\sup_{T_{n-1}<t\leq T_{n}}|Z\left(t\right)-Z\left(T_{n-1}\right)|
\end{eqnarray*}
\end{Def}

\begin{Teo}
Sup\'ongase que $n^{-1}T_{n}\rightarrow\mu$ c.s. cuando $n\rightarrow\infty$, donde $\mu\leq\infty$ es una constante o variable aleatoria. Sea $a$ una constante o variable aleatoria que puede ser infinita cuando $\mu$ es finita, y considere las expresiones l\'imite:
\begin{eqnarray}
lim_{n\rightarrow\infty}n^{-1}Z\left(T_{n}\right)&=&a,\textrm{ c.s.}\\
lim_{t\rightarrow\infty}t^{-1}Z\left(t\right)&=&a/\mu,\textrm{ c.s.}
\end{eqnarray}
La segunda expresi\'on implica la primera. Conversamente, la primera implica la segunda si el proceso $Z\left(t\right)$ es creciente, o si $lim_{n\rightarrow\infty}n^{-1}M_{n}=0$ c.s.
\end{Teo}

\begin{Coro}
Si $N\left(t\right)$ es un proceso de renovaci\'on, y $\left(Z\left(T_{n}\right)-Z\left(T_{n-1}\right),M_{n}\right)$, para $n\geq1$, son variables aleatorias independientes e id\'enticamente distribuidas con media finita, entonces,
\begin{eqnarray}
lim_{t\rightarrow\infty}t^{-1}Z\left(t\right)\rightarrow\frac{\esp\left[Z\left(T_{1}\right)-Z\left(T_{0}\right)\right]}{\esp\left[T_{1}\right]},\textrm{ c.s. cuando  }t\rightarrow\infty.
\end{eqnarray}
\end{Coro}

%___________________________________________________________________________________________
%
\subsection{Propiedades de los Procesos de Renovaci\'on}
%___________________________________________________________________________________________
%

Los tiempos $T_{n}$ est\'an relacionados con los conteos de $N\left(t\right)$ por

\begin{eqnarray*}
\left\{N\left(t\right)\geq n\right\}&=&\left\{T_{n}\leq t\right\}\\
T_{N\left(t\right)}\leq &t&<T_{N\left(t\right)+1},
\end{eqnarray*}

adem\'as $N\left(T_{n}\right)=n$, y 

\begin{eqnarray*}
N\left(t\right)=\max\left\{n:T_{n}\leq t\right\}=\min\left\{n:T_{n+1}>t\right\}
\end{eqnarray*}

Por propiedades de la convoluci\'on se sabe que

\begin{eqnarray*}
P\left\{T_{n}\leq t\right\}=F^{n\star}\left(t\right)
\end{eqnarray*}
que es la $n$-\'esima convoluci\'on de $F$. Entonces 

\begin{eqnarray*}
\left\{N\left(t\right)\geq n\right\}&=&\left\{T_{n}\leq t\right\}\\
P\left\{N\left(t\right)\leq n\right\}&=&1-F^{\left(n+1\right)\star}\left(t\right)
\end{eqnarray*}

Adem\'as usando el hecho de que $\esp\left[N\left(t\right)\right]=\sum_{n=1}^{\infty}P\left\{N\left(t\right)\geq n\right\}$
se tiene que

\begin{eqnarray*}
\esp\left[N\left(t\right)\right]=\sum_{n=1}^{\infty}F^{n\star}\left(t\right)
\end{eqnarray*}

\begin{Prop}
Para cada $t\geq0$, la funci\'on generadora de momentos $\esp\left[e^{\alpha N\left(t\right)}\right]$ existe para alguna $\alpha$ en una vecindad del 0, y de aqu\'i que $\esp\left[N\left(t\right)^{m}\right]<\infty$, para $m\geq1$.
\end{Prop}


\begin{Note}
Si el primer tiempo de renovaci\'on $\xi_{1}$ no tiene la misma distribuci\'on que el resto de las $\xi_{n}$, para $n\geq2$, a $N\left(t\right)$ se le llama Proceso de Renovaci\'on retardado, donde si $\xi$ tiene distribuci\'on $G$, entonces el tiempo $T_{n}$ de la $n$-\'esima renovaci\'on tiene distribuci\'on $G\star F^{\left(n-1\right)\star}\left(t\right)$
\end{Note}


\begin{Teo}
Para una constante $\mu\leq\infty$ ( o variable aleatoria), las siguientes expresiones son equivalentes:

\begin{eqnarray}
lim_{n\rightarrow\infty}n^{-1}T_{n}&=&\mu,\textrm{ c.s.}\\
lim_{t\rightarrow\infty}t^{-1}N\left(t\right)&=&1/\mu,\textrm{ c.s.}
\end{eqnarray}
\end{Teo}


Es decir, $T_{n}$ satisface la Ley Fuerte de los Grandes N\'umeros s\'i y s\'olo s\'i $N\left/t\right)$ la cumple.


\begin{Coro}[Ley Fuerte de los Grandes N\'umeros para Procesos de Renovaci\'on]
Si $N\left(t\right)$ es un proceso de renovaci\'on cuyos tiempos de inter-renovaci\'on tienen media $\mu\leq\infty$, entonces
\begin{eqnarray}
t^{-1}N\left(t\right)\rightarrow 1/\mu,\textrm{ c.s. cuando }t\rightarrow\infty.
\end{eqnarray}

\end{Coro}


Considerar el proceso estoc\'astico de valores reales $\left\{Z\left(t\right):t\geq0\right\}$ en el mismo espacio de probabilidad que $N\left(t\right)$

\begin{Def}
Para el proceso $\left\{Z\left(t\right):t\geq0\right\}$ se define la fluctuaci\'on m\'axima de $Z\left(t\right)$ en el intervalo $\left(T_{n-1},T_{n}\right]$:
\begin{eqnarray*}
M_{n}=\sup_{T_{n-1}<t\leq T_{n}}|Z\left(t\right)-Z\left(T_{n-1}\right)|
\end{eqnarray*}
\end{Def}

\begin{Teo}
Sup\'ongase que $n^{-1}T_{n}\rightarrow\mu$ c.s. cuando $n\rightarrow\infty$, donde $\mu\leq\infty$ es una constante o variable aleatoria. Sea $a$ una constante o variable aleatoria que puede ser infinita cuando $\mu$ es finita, y considere las expresiones l\'imite:
\begin{eqnarray}
lim_{n\rightarrow\infty}n^{-1}Z\left(T_{n}\right)&=&a,\textrm{ c.s.}\\
lim_{t\rightarrow\infty}t^{-1}Z\left(t\right)&=&a/\mu,\textrm{ c.s.}
\end{eqnarray}
La segunda expresi\'on implica la primera. Conversamente, la primera implica la segunda si el proceso $Z\left(t\right)$ es creciente, o si $lim_{n\rightarrow\infty}n^{-1}M_{n}=0$ c.s.
\end{Teo}

\begin{Coro}
Si $N\left(t\right)$ es un proceso de renovaci\'on, y $\left(Z\left(T_{n}\right)-Z\left(T_{n-1}\right),M_{n}\right)$, para $n\geq1$, son variables aleatorias independientes e id\'enticamente distribuidas con media finita, entonces,
\begin{eqnarray}
lim_{t\rightarrow\infty}t^{-1}Z\left(t\right)\rightarrow\frac{\esp\left[Z\left(T_{1}\right)-Z\left(T_{0}\right)\right]}{\esp\left[T_{1}\right]},\textrm{ c.s. cuando  }t\rightarrow\infty.
\end{eqnarray}
\end{Coro}



%___________________________________________________________________________________________
%
\section{Propiedades de los Procesos de Renovaci\'on}
%___________________________________________________________________________________________
%

Los tiempos $T_{n}$ est\'an relacionados con los conteos de $N\left(t\right)$ por

\begin{eqnarray*}
\left\{N\left(t\right)\geq n\right\}&=&\left\{T_{n}\leq t\right\}\\
T_{N\left(t\right)}\leq &t&<T_{N\left(t\right)+1},
\end{eqnarray*}

adem\'as $N\left(T_{n}\right)=n$, y 

\begin{eqnarray*}
N\left(t\right)=\max\left\{n:T_{n}\leq t\right\}=\min\left\{n:T_{n+1}>t\right\}
\end{eqnarray*}

Por propiedades de la convoluci\'on se sabe que

\begin{eqnarray*}
P\left\{T_{n}\leq t\right\}=F^{n\star}\left(t\right)
\end{eqnarray*}
que es la $n$-\'esima convoluci\'on de $F$. Entonces 

\begin{eqnarray*}
\left\{N\left(t\right)\geq n\right\}&=&\left\{T_{n}\leq t\right\}\\
P\left\{N\left(t\right)\leq n\right\}&=&1-F^{\left(n+1\right)\star}\left(t\right)
\end{eqnarray*}

Adem\'as usando el hecho de que $\esp\left[N\left(t\right)\right]=\sum_{n=1}^{\infty}P\left\{N\left(t\right)\geq n\right\}$
se tiene que

\begin{eqnarray*}
\esp\left[N\left(t\right)\right]=\sum_{n=1}^{\infty}F^{n\star}\left(t\right)
\end{eqnarray*}

\begin{Prop}
Para cada $t\geq0$, la funci\'on generadora de momentos $\esp\left[e^{\alpha N\left(t\right)}\right]$ existe para alguna $\alpha$ en una vecindad del 0, y de aqu\'i que $\esp\left[N\left(t\right)^{m}\right]<\infty$, para $m\geq1$.
\end{Prop}


\begin{Note}
Si el primer tiempo de renovaci\'on $\xi_{1}$ no tiene la misma distribuci\'on que el resto de las $\xi_{n}$, para $n\geq2$, a $N\left(t\right)$ se le llama Proceso de Renovaci\'on retardado, donde si $\xi$ tiene distribuci\'on $G$, entonces el tiempo $T_{n}$ de la $n$-\'esima renovaci\'on tiene distribuci\'on $G\star F^{\left(n-1\right)\star}\left(t\right)$
\end{Note}


\begin{Teo}
Para una constante $\mu\leq\infty$ ( o variable aleatoria), las siguientes expresiones son equivalentes:

\begin{eqnarray}
lim_{n\rightarrow\infty}n^{-1}T_{n}&=&\mu,\textrm{ c.s.}\\
lim_{t\rightarrow\infty}t^{-1}N\left(t\right)&=&1/\mu,\textrm{ c.s.}
\end{eqnarray}
\end{Teo}


Es decir, $T_{n}$ satisface la Ley Fuerte de los Grandes N\'umeros s\'i y s\'olo s\'i $N\left/t\right)$ la cumple.


\begin{Coro}[Ley Fuerte de los Grandes N\'umeros para Procesos de Renovaci\'on]
Si $N\left(t\right)$ es un proceso de renovaci\'on cuyos tiempos de inter-renovaci\'on tienen media $\mu\leq\infty$, entonces
\begin{eqnarray}
t^{-1}N\left(t\right)\rightarrow 1/\mu,\textrm{ c.s. cuando }t\rightarrow\infty.
\end{eqnarray}

\end{Coro}


Considerar el proceso estoc\'astico de valores reales $\left\{Z\left(t\right):t\geq0\right\}$ en el mismo espacio de probabilidad que $N\left(t\right)$

\begin{Def}
Para el proceso $\left\{Z\left(t\right):t\geq0\right\}$ se define la fluctuaci\'on m\'axima de $Z\left(t\right)$ en el intervalo $\left(T_{n-1},T_{n}\right]$:
\begin{eqnarray*}
M_{n}=\sup_{T_{n-1}<t\leq T_{n}}|Z\left(t\right)-Z\left(T_{n-1}\right)|
\end{eqnarray*}
\end{Def}

\begin{Teo}
Sup\'ongase que $n^{-1}T_{n}\rightarrow\mu$ c.s. cuando $n\rightarrow\infty$, donde $\mu\leq\infty$ es una constante o variable aleatoria. Sea $a$ una constante o variable aleatoria que puede ser infinita cuando $\mu$ es finita, y considere las expresiones l\'imite:
\begin{eqnarray}
lim_{n\rightarrow\infty}n^{-1}Z\left(T_{n}\right)&=&a,\textrm{ c.s.}\\
lim_{t\rightarrow\infty}t^{-1}Z\left(t\right)&=&a/\mu,\textrm{ c.s.}
\end{eqnarray}
La segunda expresi\'on implica la primera. Conversamente, la primera implica la segunda si el proceso $Z\left(t\right)$ es creciente, o si $lim_{n\rightarrow\infty}n^{-1}M_{n}=0$ c.s.
\end{Teo}

\begin{Coro}
Si $N\left(t\right)$ es un proceso de renovaci\'on, y $\left(Z\left(T_{n}\right)-Z\left(T_{n-1}\right),M_{n}\right)$, para $n\geq1$, son variables aleatorias independientes e id\'enticamente distribuidas con media finita, entonces,
\begin{eqnarray}
lim_{t\rightarrow\infty}t^{-1}Z\left(t\right)\rightarrow\frac{\esp\left[Z\left(T_{1}\right)-Z\left(T_{0}\right)\right]}{\esp\left[T_{1}\right]},\textrm{ c.s. cuando  }t\rightarrow\infty.
\end{eqnarray}
\end{Coro}




%__________________________________________________________________________________________
\section{Procesos Regenerativos Estacionarios - Stidham \cite{Stidham}}
%__________________________________________________________________________________________


Un proceso estoc\'astico a tiempo continuo $\left\{V\left(t\right),t\geq0\right\}$ es un proceso regenerativo si existe una sucesi\'on de variables aleatorias independientes e id\'enticamente distribuidas $\left\{X_{1},X_{2},\ldots\right\}$, sucesi\'on de renovaci\'on, tal que para cualquier conjunto de Borel $A$, 

\begin{eqnarray*}
\prob\left\{V\left(t\right)\in A|X_{1}+X_{2}+\cdots+X_{R\left(t\right)}=s,\left\{V\left(\tau\right),\tau<s\right\}\right\}=\prob\left\{V\left(t-s\right)\in A|X_{1}>t-s\right\},
\end{eqnarray*}
para todo $0\leq s\leq t$, donde $R\left(t\right)=\max\left\{X_{1}+X_{2}+\cdots+X_{j}\leq t\right\}=$n\'umero de renovaciones ({\emph{puntos de regeneraci\'on}}) que ocurren en $\left[0,t\right]$. El intervalo $\left[0,X_{1}\right)$ es llamado {\emph{primer ciclo de regeneraci\'on}} de $\left\{V\left(t \right),t\geq0\right\}$, $\left[X_{1},X_{1}+X_{2}\right)$ el {\emph{segundo ciclo de regeneraci\'on}}, y as\'i sucesivamente.

Sea $X=X_{1}$ y sea $F$ la funci\'on de distrbuci\'on de $X$


\begin{Def}
Se define el proceso estacionario, $\left\{V^{*}\left(t\right),t\geq0\right\}$, para $\left\{V\left(t\right),t\geq0\right\}$ por

\begin{eqnarray*}
\prob\left\{V\left(t\right)\in A\right\}=\frac{1}{\esp\left[X\right]}\int_{0}^{\infty}\prob\left\{V\left(t+x\right)\in A|X>x\right\}\left(1-F\left(x\right)\right)dx,
\end{eqnarray*} 
para todo $t\geq0$ y todo conjunto de Borel $A$.
\end{Def}

\begin{Def}
Una distribuci\'on se dice que es {\emph{aritm\'etica}} si todos sus puntos de incremento son m\'ultiplos de la forma $0,\lambda, 2\lambda,\ldots$ para alguna $\lambda>0$ entera.
\end{Def}


\begin{Def}
Una modificaci\'on medible de un proceso $\left\{V\left(t\right),t\geq0\right\}$, es una versi\'on de este, $\left\{V\left(t,w\right)\right\}$ conjuntamente medible para $t\geq0$ y para $w\in S$, $S$ espacio de estados para $\left\{V\left(t\right),t\geq0\right\}$.
\end{Def}

\begin{Teo}
Sea $\left\{V\left(t\right),t\geq\right\}$ un proceso regenerativo no negativo con modificaci\'on medible. Sea $\esp\left[X\right]<\infty$. Entonces el proceso estacionario dado por la ecuaci\'on anterior est\'a bien definido y tiene funci\'on de distribuci\'on independiente de $t$, adem\'as
\begin{itemize}
\item[i)] \begin{eqnarray*}
\esp\left[V^{*}\left(0\right)\right]&=&\frac{\esp\left[\int_{0}^{X}V\left(s\right)ds\right]}{\esp\left[X\right]}\end{eqnarray*}
\item[ii)] Si $\esp\left[V^{*}\left(0\right)\right]<\infty$, equivalentemente, si $\esp\left[\int_{0}^{X}V\left(s\right)ds\right]<\infty$,entonces
\begin{eqnarray*}
\frac{\int_{0}^{t}V\left(s\right)ds}{t}\rightarrow\frac{\esp\left[\int_{0}^{X}V\left(s\right)ds\right]}{\esp\left[X\right]}
\end{eqnarray*}
con probabilidad 1 y en media, cuando $t\rightarrow\infty$.
\end{itemize}
\end{Teo}

%______________________________________________________________________
\subsection{Procesos de Renovaci\'on}
%______________________________________________________________________

\begin{Def}\label{Def.Tn}
Sean $0\leq T_{1}\leq T_{2}\leq \ldots$ son tiempos aleatorios infinitos en los cuales ocurren ciertos eventos. El n\'umero de tiempos $T_{n}$ en el intervalo $\left[0,t\right)$ es

\begin{eqnarray}
N\left(t\right)=\sum_{n=1}^{\infty}\indora\left(T_{n}\leq t\right),
\end{eqnarray}
para $t\geq0$.
\end{Def}

Si se consideran los puntos $T_{n}$ como elementos de $\rea_{+}$, y $N\left(t\right)$ es el n\'umero de puntos en $\rea$. El proceso denotado por $\left\{N\left(t\right):t\geq0\right\}$, denotado por $N\left(t\right)$, es un proceso puntual en $\rea_{+}$. Los $T_{n}$ son los tiempos de ocurrencia, el proceso puntual $N\left(t\right)$ es simple si su n\'umero de ocurrencias son distintas: $0<T_{1}<T_{2}<\ldots$ casi seguramente.

\begin{Def}
Un proceso puntual $N\left(t\right)$ es un proceso de renovaci\'on si los tiempos de interocurrencia $\xi_{n}=T_{n}-T_{n-1}$, para $n\geq1$, son independientes e identicamente distribuidos con distribuci\'on $F$, donde $F\left(0\right)=0$ y $T_{0}=0$. Los $T_{n}$ son llamados tiempos de renovaci\'on, referente a la independencia o renovaci\'on de la informaci\'on estoc\'astica en estos tiempos. Los $\xi_{n}$ son los tiempos de inter-renovaci\'on, y $N\left(t\right)$ es el n\'umero de renovaciones en el intervalo $\left[0,t\right)$
\end{Def}


\begin{Note}
Para definir un proceso de renovaci\'on para cualquier contexto, solamente hay que especificar una distribuci\'on $F$, con $F\left(0\right)=0$, para los tiempos de inter-renovaci\'on. La funci\'on $F$ en turno degune las otra variables aleatorias. De manera formal, existe un espacio de probabilidad y una sucesi\'on de variables aleatorias $\xi_{1},\xi_{2},\ldots$ definidas en este con distribuci\'on $F$. Entonces las otras cantidades son $T_{n}=\sum_{k=1}^{n}\xi_{k}$ y $N\left(t\right)=\sum_{n=1}^{\infty}\indora\left(T_{n}\leq t\right)$, donde $T_{n}\rightarrow\infty$ casi seguramente por la Ley Fuerte de los Grandes Números.
\end{Note}

%___________________________________________________________________________________________
%
\subsection{Teorema Principal de Renovaci\'on}
%___________________________________________________________________________________________
%

\begin{Note} Una funci\'on $h:\rea_{+}\rightarrow\rea$ es Directamente Riemann Integrable en los siguientes casos:
\begin{itemize}
\item[a)] $h\left(t\right)\geq0$ es decreciente y Riemann Integrable.
\item[b)] $h$ es continua excepto posiblemente en un conjunto de Lebesgue de medida 0, y $|h\left(t\right)|\leq b\left(t\right)$, donde $b$ es DRI.
\end{itemize}
\end{Note}

\begin{Teo}[Teorema Principal de Renovaci\'on]
Si $F$ es no aritm\'etica y $h\left(t\right)$ es Directamente Riemann Integrable (DRI), entonces

\begin{eqnarray*}
lim_{t\rightarrow\infty}U\star h=\frac{1}{\mu}\int_{\rea_{+}}h\left(s\right)ds.
\end{eqnarray*}
\end{Teo}

\begin{Prop}
Cualquier funci\'on $H\left(t\right)$ acotada en intervalos finitos y que es 0 para $t<0$ puede expresarse como
\begin{eqnarray*}
H\left(t\right)=U\star h\left(t\right)\textrm{,  donde }h\left(t\right)=H\left(t\right)-F\star H\left(t\right)
\end{eqnarray*}
\end{Prop}

\begin{Def}
Un proceso estoc\'astico $X\left(t\right)$ es crudamente regenerativo en un tiempo aleatorio positivo $T$ si
\begin{eqnarray*}
\esp\left[X\left(T+t\right)|T\right]=\esp\left[X\left(t\right)\right]\textrm{, para }t\geq0,\end{eqnarray*}
y con las esperanzas anteriores finitas.
\end{Def}

\begin{Prop}
Sup\'ongase que $X\left(t\right)$ es un proceso crudamente regenerativo en $T$, que tiene distribuci\'on $F$. Si $\esp\left[X\left(t\right)\right]$ es acotado en intervalos finitos, entonces
\begin{eqnarray*}
\esp\left[X\left(t\right)\right]=U\star h\left(t\right)\textrm{,  donde }h\left(t\right)=\esp\left[X\left(t\right)\indora\left(T>t\right)\right].
\end{eqnarray*}
\end{Prop}

\begin{Teo}[Regeneraci\'on Cruda]
Sup\'ongase que $X\left(t\right)$ es un proceso con valores positivo crudamente regenerativo en $T$, y def\'inase $M=\sup\left\{|X\left(t\right)|:t\leq T\right\}$. Si $T$ es no aritm\'etico y $M$ y $MT$ tienen media finita, entonces
\begin{eqnarray*}
lim_{t\rightarrow\infty}\esp\left[X\left(t\right)\right]=\frac{1}{\mu}\int_{\rea_{+}}h\left(s\right)ds,
\end{eqnarray*}
donde $h\left(t\right)=\esp\left[X\left(t\right)\indora\left(T>t\right)\right]$.
\end{Teo}



%___________________________________________________________________________________________
%
\subsection{Funci\'on de Renovaci\'on}
%___________________________________________________________________________________________
%


\begin{Def}
Sea $h\left(t\right)$ funci\'on de valores reales en $\rea$ acotada en intervalos finitos e igual a cero para $t<0$ La ecuaci\'on de renovaci\'on para $h\left(t\right)$ y la distribuci\'on $F$ es

\begin{eqnarray}\label{Ec.Renovacion}
H\left(t\right)=h\left(t\right)+\int_{\left[0,t\right]}H\left(t-s\right)dF\left(s\right)\textrm{,    }t\geq0,
\end{eqnarray}
donde $H\left(t\right)$ es una funci\'on de valores reales. Esto es $H=h+F\star H$. Decimos que $H\left(t\right)$ es soluci\'on de esta ecuaci\'on si satisface la ecuaci\'on, y es acotada en intervalos finitos e iguales a cero para $t<0$.
\end{Def}

\begin{Prop}
La funci\'on $U\star h\left(t\right)$ es la \'unica soluci\'on de la ecuaci\'on de renovaci\'on (\ref{Ec.Renovacion}).
\end{Prop}

\begin{Teo}[Teorema Renovaci\'on Elemental]
\begin{eqnarray*}
t^{-1}U\left(t\right)\rightarrow 1/\mu\textrm{,    cuando }t\rightarrow\infty.
\end{eqnarray*}
\end{Teo}

%___________________________________________________________________________________________
%
\subsection{Propiedades de los Procesos de Renovaci\'on}
%___________________________________________________________________________________________
%

Los tiempos $T_{n}$ est\'an relacionados con los conteos de $N\left(t\right)$ por

\begin{eqnarray*}
\left\{N\left(t\right)\geq n\right\}&=&\left\{T_{n}\leq t\right\}\\
T_{N\left(t\right)}\leq &t&<T_{N\left(t\right)+1},
\end{eqnarray*}

adem\'as $N\left(T_{n}\right)=n$, y 

\begin{eqnarray*}
N\left(t\right)=\max\left\{n:T_{n}\leq t\right\}=\min\left\{n:T_{n+1}>t\right\}
\end{eqnarray*}

Por propiedades de la convoluci\'on se sabe que

\begin{eqnarray*}
P\left\{T_{n}\leq t\right\}=F^{n\star}\left(t\right)
\end{eqnarray*}
que es la $n$-\'esima convoluci\'on de $F$. Entonces 

\begin{eqnarray*}
\left\{N\left(t\right)\geq n\right\}&=&\left\{T_{n}\leq t\right\}\\
P\left\{N\left(t\right)\leq n\right\}&=&1-F^{\left(n+1\right)\star}\left(t\right)
\end{eqnarray*}

Adem\'as usando el hecho de que $\esp\left[N\left(t\right)\right]=\sum_{n=1}^{\infty}P\left\{N\left(t\right)\geq n\right\}$
se tiene que

\begin{eqnarray*}
\esp\left[N\left(t\right)\right]=\sum_{n=1}^{\infty}F^{n\star}\left(t\right)
\end{eqnarray*}

\begin{Prop}
Para cada $t\geq0$, la funci\'on generadora de momentos $\esp\left[e^{\alpha N\left(t\right)}\right]$ existe para alguna $\alpha$ en una vecindad del 0, y de aqu\'i que $\esp\left[N\left(t\right)^{m}\right]<\infty$, para $m\geq1$.
\end{Prop}


\begin{Note}
Si el primer tiempo de renovaci\'on $\xi_{1}$ no tiene la misma distribuci\'on que el resto de las $\xi_{n}$, para $n\geq2$, a $N\left(t\right)$ se le llama Proceso de Renovaci\'on retardado, donde si $\xi$ tiene distribuci\'on $G$, entonces el tiempo $T_{n}$ de la $n$-\'esima renovaci\'on tiene distribuci\'on $G\star F^{\left(n-1\right)\star}\left(t\right)$
\end{Note}


\begin{Teo}
Para una constante $\mu\leq\infty$ ( o variable aleatoria), las siguientes expresiones son equivalentes:

\begin{eqnarray}
lim_{n\rightarrow\infty}n^{-1}T_{n}&=&\mu,\textrm{ c.s.}\\
lim_{t\rightarrow\infty}t^{-1}N\left(t\right)&=&1/\mu,\textrm{ c.s.}
\end{eqnarray}
\end{Teo}


Es decir, $T_{n}$ satisface la Ley Fuerte de los Grandes N\'umeros s\'i y s\'olo s\'i $N\left/t\right)$ la cumple.


\begin{Coro}[Ley Fuerte de los Grandes N\'umeros para Procesos de Renovaci\'on]
Si $N\left(t\right)$ es un proceso de renovaci\'on cuyos tiempos de inter-renovaci\'on tienen media $\mu\leq\infty$, entonces
\begin{eqnarray}
t^{-1}N\left(t\right)\rightarrow 1/\mu,\textrm{ c.s. cuando }t\rightarrow\infty.
\end{eqnarray}

\end{Coro}


Considerar el proceso estoc\'astico de valores reales $\left\{Z\left(t\right):t\geq0\right\}$ en el mismo espacio de probabilidad que $N\left(t\right)$

\begin{Def}
Para el proceso $\left\{Z\left(t\right):t\geq0\right\}$ se define la fluctuaci\'on m\'axima de $Z\left(t\right)$ en el intervalo $\left(T_{n-1},T_{n}\right]$:
\begin{eqnarray*}
M_{n}=\sup_{T_{n-1}<t\leq T_{n}}|Z\left(t\right)-Z\left(T_{n-1}\right)|
\end{eqnarray*}
\end{Def}

\begin{Teo}
Sup\'ongase que $n^{-1}T_{n}\rightarrow\mu$ c.s. cuando $n\rightarrow\infty$, donde $\mu\leq\infty$ es una constante o variable aleatoria. Sea $a$ una constante o variable aleatoria que puede ser infinita cuando $\mu$ es finita, y considere las expresiones l\'imite:
\begin{eqnarray}
lim_{n\rightarrow\infty}n^{-1}Z\left(T_{n}\right)&=&a,\textrm{ c.s.}\\
lim_{t\rightarrow\infty}t^{-1}Z\left(t\right)&=&a/\mu,\textrm{ c.s.}
\end{eqnarray}
La segunda expresi\'on implica la primera. Conversamente, la primera implica la segunda si el proceso $Z\left(t\right)$ es creciente, o si $lim_{n\rightarrow\infty}n^{-1}M_{n}=0$ c.s.
\end{Teo}

\begin{Coro}
Si $N\left(t\right)$ es un proceso de renovaci\'on, y $\left(Z\left(T_{n}\right)-Z\left(T_{n-1}\right),M_{n}\right)$, para $n\geq1$, son variables aleatorias independientes e id\'enticamente distribuidas con media finita, entonces,
\begin{eqnarray}
lim_{t\rightarrow\infty}t^{-1}Z\left(t\right)\rightarrow\frac{\esp\left[Z\left(T_{1}\right)-Z\left(T_{0}\right)\right]}{\esp\left[T_{1}\right]},\textrm{ c.s. cuando  }t\rightarrow\infty.
\end{eqnarray}
\end{Coro}

%___________________________________________________________________________________________
%
\subsection{Funci\'on de Renovaci\'on}
%___________________________________________________________________________________________
%


Sup\'ongase que $N\left(t\right)$ es un proceso de renovaci\'on con distribuci\'on $F$ con media finita $\mu$.

\begin{Def}
La funci\'on de renovaci\'on asociada con la distribuci\'on $F$, del proceso $N\left(t\right)$, es
\begin{eqnarray*}
U\left(t\right)=\sum_{n=1}^{\infty}F^{n\star}\left(t\right),\textrm{   }t\geq0,
\end{eqnarray*}
donde $F^{0\star}\left(t\right)=\indora\left(t\geq0\right)$.
\end{Def}


\begin{Prop}
Sup\'ongase que la distribuci\'on de inter-renovaci\'on $F$ tiene densidad $f$. Entonces $U\left(t\right)$ tambi\'en tiene densidad, para $t>0$, y es $U^{'}\left(t\right)=\sum_{n=0}^{\infty}f^{n\star}\left(t\right)$. Adem\'as
\begin{eqnarray*}
\prob\left\{N\left(t\right)>N\left(t-\right)\right\}=0\textrm{,   }t\geq0.
\end{eqnarray*}
\end{Prop}

\begin{Def}
La Transformada de Laplace-Stieljes de $F$ est\'a dada por

\begin{eqnarray*}
\hat{F}\left(\alpha\right)=\int_{\rea_{+}}e^{-\alpha t}dF\left(t\right)\textrm{,  }\alpha\geq0.
\end{eqnarray*}
\end{Def}

Entonces

\begin{eqnarray*}
\hat{U}\left(\alpha\right)=\sum_{n=0}^{\infty}\hat{F^{n\star}}\left(\alpha\right)=\sum_{n=0}^{\infty}\hat{F}\left(\alpha\right)^{n}=\frac{1}{1-\hat{F}\left(\alpha\right)}.
\end{eqnarray*}


\begin{Prop}
La Transformada de Laplace $\hat{U}\left(\alpha\right)$ y $\hat{F}\left(\alpha\right)$ determina una a la otra de manera \'unica por la relaci\'on $\hat{U}\left(\alpha\right)=\frac{1}{1-\hat{F}\left(\alpha\right)}$.
\end{Prop}


\begin{Note}
Un proceso de renovaci\'on $N\left(t\right)$ cuyos tiempos de inter-renovaci\'on tienen media finita, es un proceso Poisson con tasa $\lambda$ si y s\'olo s\'i $\esp\left[U\left(t\right)\right]=\lambda t$, para $t\geq0$.
\end{Note}


\begin{Teo}
Sea $N\left(t\right)$ un proceso puntual simple con puntos de localizaci\'on $T_{n}$ tal que $\eta\left(t\right)=\esp\left[N\left(\right)\right]$ es finita para cada $t$. Entonces para cualquier funci\'on $f:\rea_{+}\rightarrow\rea$,
\begin{eqnarray*}
\esp\left[\sum_{n=1}^{N\left(\right)}f\left(T_{n}\right)\right]=\int_{\left(0,t\right]}f\left(s\right)d\eta\left(s\right)\textrm{,  }t\geq0,
\end{eqnarray*}
suponiendo que la integral exista. Adem\'as si $X_{1},X_{2},\ldots$ son variables aleatorias definidas en el mismo espacio de probabilidad que el proceso $N\left(t\right)$ tal que $\esp\left[X_{n}|T_{n}=s\right]=f\left(s\right)$, independiente de $n$. Entonces
\begin{eqnarray*}
\esp\left[\sum_{n=1}^{N\left(t\right)}X_{n}\right]=\int_{\left(0,t\right]}f\left(s\right)d\eta\left(s\right)\textrm{,  }t\geq0,
\end{eqnarray*} 
suponiendo que la integral exista. 
\end{Teo}

\begin{Coro}[Identidad de Wald para Renovaciones]
Para el proceso de renovaci\'on $N\left(t\right)$,
\begin{eqnarray*}
\esp\left[T_{N\left(t\right)+1}\right]=\mu\esp\left[N\left(t\right)+1\right]\textrm{,  }t\geq0,
\end{eqnarray*}  
\end{Coro}

%______________________________________________________________________
\section{Procesos de Renovaci\'on}
%______________________________________________________________________

\begin{Def}\label{Def.Tn}
Sean $0\leq T_{1}\leq T_{2}\leq \ldots$ son tiempos aleatorios infinitos en los cuales ocurren ciertos eventos. El n\'umero de tiempos $T_{n}$ en el intervalo $\left[0,t\right)$ es

\begin{eqnarray}
N\left(t\right)=\sum_{n=1}^{\infty}\indora\left(T_{n}\leq t\right),
\end{eqnarray}
para $t\geq0$.
\end{Def}

Si se consideran los puntos $T_{n}$ como elementos de $\rea_{+}$, y $N\left(t\right)$ es el n\'umero de puntos en $\rea$. El proceso denotado por $\left\{N\left(t\right):t\geq0\right\}$, denotado por $N\left(t\right)$, es un proceso puntual en $\rea_{+}$. Los $T_{n}$ son los tiempos de ocurrencia, el proceso puntual $N\left(t\right)$ es simple si su n\'umero de ocurrencias son distintas: $0<T_{1}<T_{2}<\ldots$ casi seguramente.

\begin{Def}
Un proceso puntual $N\left(t\right)$ es un proceso de renovaci\'on si los tiempos de interocurrencia $\xi_{n}=T_{n}-T_{n-1}$, para $n\geq1$, son independientes e identicamente distribuidos con distribuci\'on $F$, donde $F\left(0\right)=0$ y $T_{0}=0$. Los $T_{n}$ son llamados tiempos de renovaci\'on, referente a la independencia o renovaci\'on de la informaci\'on estoc\'astica en estos tiempos. Los $\xi_{n}$ son los tiempos de inter-renovaci\'on, y $N\left(t\right)$ es el n\'umero de renovaciones en el intervalo $\left[0,t\right)$
\end{Def}


\begin{Note}
Para definir un proceso de renovaci\'on para cualquier contexto, solamente hay que especificar una distribuci\'on $F$, con $F\left(0\right)=0$, para los tiempos de inter-renovaci\'on. La funci\'on $F$ en turno degune las otra variables aleatorias. De manera formal, existe un espacio de probabilidad y una sucesi\'on de variables aleatorias $\xi_{1},\xi_{2},\ldots$ definidas en este con distribuci\'on $F$. Entonces las otras cantidades son $T_{n}=\sum_{k=1}^{n}\xi_{k}$ y $N\left(t\right)=\sum_{n=1}^{\infty}\indora\left(T_{n}\leq t\right)$, donde $T_{n}\rightarrow\infty$ casi seguramente por la Ley Fuerte de los Grandes Números.
\end{Note}
%_____________________________________________________
\section{Puntos de Renovaci\'on}
%_____________________________________________________

Para cada cola $Q_{i}$ se tienen los procesos de arribo a la cola, para estas, los tiempos de arribo est\'an dados por $$\left\{T_{1}^{i},T_{2}^{i},\ldots,T_{k}^{i},\ldots\right\},$$ entonces, consideremos solamente los primeros tiempos de arribo a cada una de las colas, es decir, $$\left\{T_{1}^{1},T_{1}^{2},T_{1}^{3},T_{1}^{4}\right\},$$ se sabe que cada uno de estos tiempos se distribuye de manera exponencial con par\'ametro $1/mu_{i}$. Adem\'as se sabe que para $$T^{*}=\min\left\{T_{1}^{1},T_{1}^{2},T_{1}^{3},T_{1}^{4}\right\},$$ $T^{*}$ se distribuye de manera exponencial con par\'ametro $$\mu^{*}=\sum_{i=1}^{4}\mu_{i}.$$ Ahora, dado que 
\begin{center}
\begin{tabular}{lcl}
$\tilde{r}=r_{1}+r_{2}$ & y &$\hat{r}=r_{3}+r_{4}.$
\end{tabular}
\end{center}


Supongamos que $$\tilde{r},\hat{r}<\mu^{*},$$ entonces si tomamos $$r^{*}=\min\left\{\tilde{r},\hat{r}\right\},$$ se tiene que para  $$t^{*}\in\left(0,r^{*}\right)$$ se cumple que 
\begin{center}
\begin{tabular}{lcl}
$\tau_{1}\left(1\right)=0$ & y por tanto & $\overline{\tau}_{1}=0,$
\end{tabular}
\end{center}
entonces para la segunda cola en este primer ciclo se cumple que $$\tau_{2}=\overline{\tau}_{1}+r_{1}=r_{1}<\mu^{*},$$ y por tanto se tiene que  $$\overline{\tau}_{2}=\tau_{2}.$$ Por lo tanto, nuevamente para la primer cola en el segundo ciclo $$\tau_{1}\left(2\right)=\tau_{2}\left(1\right)+r_{2}=\tilde{r}<\mu^{*}.$$ An\'alogamente para el segundo sistema se tiene que ambas colas est\'an vac\'ias, es decir, existe un valor $t^{*}$ tal que en el intervalo $\left(0,t^{*}\right)$ no ha llegado ning\'un usuario, es decir, $$L_{i}\left(t^{*}\right)=0$$ para $i=1,2,3,4$.


%___________________________________________________________
%
\section{Existencia de Tiempos de Regeneraci\'on}
%___________________________________________________________
%
\begin{Def}
Un elemento aleatorio en un espacio medible $\left(E,\mathcal{E}\right)$ en un espacio de probabilidad $\left(\Omega,\mathcal{F},\prob\right)$ a $\left(E,\mathcal{E}\right)$, es decir,
para $A\in \mathcal{E}$,  se tiene que $\left\{Y\in A\right\}\in\mathcal{F}$, donde $\left\{Y\in A\right\}:=\left\{w\in\Omega:Y\left(w\right)\in A\right\}=:Y^{-1}A$.
\end{Def}

\begin{Note}
Tambi\'en se dice que $Y$ est\'a soportado por el espacio de probabilidad $\left(\Omega,\mathcal{F},\prob\right)$ y que $Y$ es un mapeo medible de $\Omega$ en $E$, es decir, es $\mathcal{F}/\mathcal{E}$ medible.
\end{Note}

\begin{Def}
Para cada $i\in \mathbb{I}$ sea $P_{i}$ una medida de probabilidad en un espacio medible $\left(E_{i},\mathcal{E}_{i}\right)$. Se define el espacio producto
$\otimes_{i\in\mathbb{I}}\left(E_{i},\mathcal{E}_{i}\right):=\left(\prod_{i\in\mathbb{I}}E_{i},\otimes_{i\in\mathbb{I}}\mathcal{E}_{i}\right)$, donde $\prod_{i\in\mathbb{I}}E_{i}$ es el producto cartesiano de los $E_{i}$'s, y $\otimes_{i\in\mathbb{I}}\mathcal{E}_{i}$ es la $\sigma$-\'algebra producto, es decir, es la $\sigma$-\'algebra m\'as peque\~na en $\prod_{i\in\mathbb{I}}E_{i}$ que hace al $i$-\'esimo mapeo proyecci\'on en $E_{i}$ medible para toda $i\in\mathbb{I}$ es la $\sigma$-\'algebra inducida por los mapeos proyecci\'on. $$\otimes_{i\in\mathbb{I}}\mathcal{E}_{i}:=\sigma\left\{\left\{y:y_{i}\in A\right\}:i\in\mathbb{I}\textrm{ y }A\in\mathcal{E}_{i}\right\}.$$
\end{Def}

\begin{Def}
Un espacio de probabilidad $\left(\tilde{\Omega},\tilde{\mathcal{F}},\tilde{\prob}\right)$ es una extensi\'on de otro espacio de probabilidad $\left(\Omega,\mathcal{F},\prob\right)$ si $\left(\tilde{\Omega},\tilde{\mathcal{F}},\tilde{\prob}\right)$ soporta un elemento aleatorio $\xi\in\left(\Omega,\mathcal{F}\right)$ que tienen a $\prob$ como distribuci\'on.
\end{Def}

\begin{Teo}
Sea $\mathbb{I}$ un conjunto de \'indices arbitrario. Para cada $i\in\mathbb{I}$ sea $P_{i}$ una medida de probabilidad en un espacio medible $\left(E_{i},\mathcal{E}_{i}\right)$. Entonces existe una \'unica medida de probabilidad $\otimes_{i\in\mathbb{I}}P_{i}$ en $\otimes_{i\in\mathbb{I}}\left(E_{i},\mathcal{E}_{i}\right)$ tal que 

\begin{eqnarray*}
\otimes_{i\in\mathbb{I}}P_{i}\left(y\in\prod_{i\in\mathbb{I}}E_{i}:y_{i}\in A_{i_{1}},\ldots,y_{n}\in A_{i_{n}}\right)=P_{i_{1}}\left(A_{i_{n}}\right)\cdots P_{i_{n}}\left(A_{i_{n}}\right)
\end{eqnarray*}
para todos los enteros $n>0$, toda $i_{1},\ldots,i_{n}\in\mathbb{I}$ y todo $A_{i_{1}}\in\mathcal{E}_{i_{1}},\ldots,A_{i_{n}}\in\mathcal{E}_{i_{n}}$
\end{Teo}

La medida $\otimes_{i\in\mathbb{I}}P_{i}$ es llamada la medida producto y $\otimes_{i\in\mathbb{I}}\left(E_{i},\mathcal{E}_{i},P_{i}\right):=\left(\prod_{i\in\mathbb{I}},E_{i},\otimes_{i\in\mathbb{I}}\mathcal{E}_{i},\otimes_{i\in\mathbb{I}}P_{i}\right)$, es llamado espacio de probabilidad producto.


\begin{Def}
Un espacio medible $\left(E,\mathcal{E}\right)$ es \textit{Polaco} si existe una m\'etrica en $E$ tal que $E$ es completo, es decir cada sucesi\'on de Cauchy converge a un l\'imite en $E$, y \textit{separable}, $E$ tienen un subconjunto denso numerable, y tal que $\mathcal{E}$ es generado por conjuntos abiertos.
\end{Def}


\begin{Def}
Dos espacios medibles $\left(E,\mathcal{E}\right)$ y $\left(G,\mathcal{G}\right)$ son Borel equivalentes \textit{isomorfos} si existe una biyecci\'on $f:E\rightarrow G$ tal que $f$ es $\mathcal{E}/\mathcal{G}$ medible y su inversa $f^{-1}$ es $\mathcal{G}/\mathcal{E}$ medible. La biyecci\'on es una equivalencia de Borel.
\end{Def}

\begin{Def}
Un espacio medible  $\left(E,\mathcal{E}\right)$ es un \textit{espacio est\'andar} si es Borel equivalente a $\left(G,\mathcal{G}\right)$, donde $G$ es un subconjunto de Borel de $\left[0,1\right]$ y $\mathcal{G}$ son los subconjuntos de Borel de $G$.
\end{Def}

\begin{Note}
Cualquier espacio Polaco es un espacio est\'andar.
\end{Note}


\begin{Def}
Un proceso estoc\'astico con conjunto de \'indices $\mathbb{I}$ y espacio de estados $\left(E,\mathcal{E}\right)$ es una familia $Z=\left(\mathbb{Z}_{s}\right)_{s\in\mathbb{I}}$ donde $\mathbb{Z}_{s}$ son elementos aleatorios definidos en un espacio de probabilidad com\'un $\left(\Omega,\mathcal{F},\prob\right)$ y todos toman valores en $\left(E,\mathcal{E}\right)$.
\end{Def}

\begin{Def}
Un proceso estoc\'astico \textit{one-sided contiuous time} (\textbf{PEOSCT}) es un proceso estoc\'astico con conjunto de \'indices $\mathbb{I}=\left[0,\infty\right)$.
\end{Def}


Sea $\left(E^{\mathbb{I}},\mathcal{E}^{\mathbb{I}}\right)$ denota el espacio producto $\left(E^{\mathbb{I}},\mathcal{E}^{\mathbb{I}}\right):=\otimes_{s\in\mathbb{I}}\left(E,\mathcal{E}\right)$. Vamos a considerar $\mathbb{Z}$ como un mapeo aleatorio, es decir, como un elemento aleatorio en $\left(E^{\mathbb{I}},\mathcal{E}^{\mathbb{I}}\right)$ definido por $Z\left(w\right)=\left(Z_{s}\left(w\right)\right)_{s\in\mathbb{I}}$ y $w\in\Omega$.

\begin{Note}
La distribuci\'on de un proceso estoc\'astico $Z$ es la distribuci\'on de $Z$ como un elemento aleatorio en $\left(E^{\mathbb{I}},\mathcal{E}^{\mathbb{I}}\right)$. La distribuci\'on de $Z$ esta determinada de manera \'unica por las distribuciones finito dimensionales.
\end{Note}

\begin{Note}
En particular cuando $Z$ toma valores reales, es decir, $\left(E,\mathcal{E}\right)=\left(\mathbb{R},\mathcal{B}\right)$ las distribuciones finito dimensionales est\'an determinadas por las funciones de distribuci\'on finito dimensionales

\begin{eqnarray}
\prob\left(Z_{t_{1}}\leq x_{1},\ldots,Z_{t_{n}}\leq x_{n}\right),x_{1},\ldots,x_{n}\in\mathbb{R},t_{1},\ldots,t_{n}\in\mathbb{I},n\geq1.
\end{eqnarray}
\end{Note}

\begin{Note}
Para espacios polacos $\left(E,\mathcal{E}\right)$ el Teorema de Consistencia de Kolmogorov asegura que dada una colecci\'on de distribuciones finito dimensionales consistentes, siempre existe un proceso estoc\'astico que posee tales distribuciones finito dimensionales.
\end{Note}


\begin{Def}
Las trayectorias de $Z$ son las realizaciones $Z\left(w\right)$ para $w\in\Omega$ del mapeo aleatorio $Z$.
\end{Def}

\begin{Note}
Algunas restricciones se imponen sobre las trayectorias, por ejemplo que sean continuas por la derecha, o continuas por la derecha con l\'imites por la izquierda, o de manera m\'as general, se pedir\'a que caigan en alg\'un subconjunto $H$ de $E^{\mathbb{I}}$. En este caso es natural considerar a $Z$ como un elemento aleatorio que no est\'a en $\left(E^{\mathbb{I}},\mathcal{E}^{\mathbb{I}}\right)$ sino en $\left(H,\mathcal{H}\right)$, donde $\mathcal{H}$ es la $\sigma$-\'algebra generada por los mapeos proyecci\'on que toman a $z\in H$ a $z_{t}\in E$ para $t\in\mathbb{I}$. A $\mathcal{H}$ se le conoce como la traza de $H$ en $E^{\mathbb{I}}$, es decir,
\begin{eqnarray}
\mathcal{H}:=E^{\mathbb{I}}\cap H:=\left\{A\cap H:A\in E^{\mathbb{I}}\right\}.
\end{eqnarray}
\end{Note}


\begin{Note}
$Z$ tiene trayectorias con valores en $H$ y cada $Z_{t}$ es un mapeo medible de $\left(\Omega,\mathcal{F}\right)$ a $\left(H,\mathcal{H}\right)$. Cuando se considera un espacio de trayectorias en particular $H$, al espacio $\left(H,\mathcal{H}\right)$ se le llama el espacio de trayectorias de $Z$.
\end{Note}

\begin{Note}
La distribuci\'on del proceso estoc\'astico $Z$ con espacio de trayectorias $\left(H,\mathcal{H}\right)$ es la distribuci\'on de $Z$ como  un elemento aleatorio en $\left(H,\mathcal{H}\right)$. La distribuci\'on, nuevemente, est\'a determinada de manera \'unica por las distribuciones finito dimensionales.
\end{Note}


\begin{Def}
Sea $Z$ un PEOSCT  con espacio de estados $\left(E,\mathcal{E}\right)$ y sea $T$ un tiempo aleatorio en $\left[0,\infty\right)$. Por $Z_{T}$ se entiende el mapeo con valores en $E$ definido en $\Omega$ en la manera obvia:
\begin{eqnarray*}
Z_{T}\left(w\right):=Z_{T\left(w\right)}\left(w\right). w\in\Omega.
\end{eqnarray*}
\end{Def}

\begin{Def}
Un PEOSCT $Z$ es conjuntamente medible (\textbf{CM}) si el mapeo que toma $\left(w,t\right)\in\Omega\times\left[0,\infty\right)$ a $Z_{t}\left(w\right)\in E$ es $\mathcal{F}\otimes\mathcal{B}\left[0,\infty\right)/\mathcal{E}$ medible.
\end{Def}

\begin{Note}
Un PEOSCT-CM implica que el proceso es medible, dado que $Z_{T}$ es una composici\'on  de dos mapeos continuos: el primero que toma $w$ en $\left(w,T\left(w\right)\right)$ es $\mathcal{F}/\mathcal{F}\otimes\mathcal{B}\left[0,\infty\right)$ medible, mientras que el segundo toma $\left(w,T\left(w\right)\right)$ en $Z_{T\left(w\right)}\left(w\right)$ es $\mathcal{F}\otimes\mathcal{B}\left[0,\infty\right)/\mathcal{E}$ medible.
\end{Note}


\begin{Def}
Un PEOSCT con espacio de estados $\left(H,\mathcal{H}\right)$ es can\'onicamente conjuntamente medible (\textbf{CCM}) si el mapeo $\left(z,t\right)\in H\times\left[0,\infty\right)$ en $Z_{t}\in E$ es $\mathcal{H}\otimes\mathcal{B}\left[0,\infty\right)/\mathcal{E}$ medible.
\end{Def}

\begin{Note}
Un PEOSCTCCM implica que el proceso es CM, dado que un PECCM $Z$ es un mapeo de $\Omega\times\left[0,\infty\right)$ a $E$, es la composici\'on de dos mapeos medibles: el primero, toma $\left(w,t\right)$ en $\left(Z\left(w\right),t\right)$ es $\mathcal{F}\otimes\mathcal{B}\left[0,\infty\right)/\mathcal{H}\otimes\mathcal{B}\left[0,\infty\right)$ medible, y el segundo que toma $\left(Z\left(w\right),t\right)$  en $Z_{t}\left(w\right)$ es $\mathcal{H}\otimes\mathcal{B}\left[0,\infty\right)/\mathcal{E}$ medible. Por tanto CCM es una condici\'on m\'as fuerte que CM.
\end{Note}

\begin{Def}
Un conjunto de trayectorias $H$ de un PEOSCT $Z$, es internamente shift-invariante (\textbf{ISI}) si 
\begin{eqnarray*}
\left\{\left(z_{t+s}\right)_{s\in\left[0,\infty\right)}:z\in H\right\}=H\textrm{, }t\in\left[0,\infty\right).
\end{eqnarray*}
\end{Def}


\begin{Def}
Dado un PEOSCTISI, se define el mapeo-shift $\theta_{t}$, $t\in\left[0,\infty\right)$, de $H$ a $H$ por 
\begin{eqnarray*}
\theta_{t}z=\left(z_{t+s}\right)_{s\in\left[0,\infty\right)}\textrm{, }z\in H.
\end{eqnarray*}
\end{Def}

\begin{Def}
Se dice que un proceso $Z$ es shift-medible (\textbf{SM}) si $Z$ tiene un conjunto de trayectorias $H$ que es ISI y adem\'as el mapeo que toma $\left(z,t\right)\in H\times\left[0,\infty\right)$ en $\theta_{t}z\in H$ es $\mathcal{H}\otimes\mathcal{B}\left[0,\infty\right)/\mathcal{H}$ medible.
\end{Def}

\begin{Note}
Un proceso estoc\'astico con conjunto de trayectorias $H$ ISI es shift-medible si y s\'olo si es CCM
\end{Note}

\begin{Note}
\begin{itemize}
\item Dado el espacio polaco $\left(E,\mathcal{E}\right)$ se tiene el  conjunto de trayectorias $D_{E}\left[0,\infty\right)$ que es ISI, entonces cumpe con ser CCM.

\item Si $G$ es abierto, podemos cubrirlo por bolas abiertas cuay cerradura este contenida en $G$, y como $G$ es segundo numerable como subespacio de $E$, lo podemos cubrir por una cantidad numerable de bolas abiertas.

\end{itemize}
\end{Note}


\begin{Note}
Los procesos estoc\'asticos $Z$ a tiempo discreto con espacio de estados polaco, tambi\'en tiene un espacio de trayectorias polaco y por tanto tiene distribuciones condicionales regulares.
\end{Note}

\begin{Teo}
El producto numerable de espacios polacos es polaco.
\end{Teo}


\begin{Def}
Sea $\left(\Omega,\mathcal{F},\prob\right)$ espacio de probabilidad que soporta al proceso $Z=\left(Z_{s}\right)_{s\in\left[0,\infty\right)}$ y $S=\left(S_{k}\right)_{0}^{\infty}$ donde $Z$ es un PEOSCTM con espacio de estados $\left(E,\mathcal{E}\right)$  y espacio de trayectorias $\left(H,\mathcal{H}\right)$  y adem\'as $S$ es una sucesi\'on de tiempos aleatorios one-sided que satisfacen la condici\'on $0\leq S_{0}<S_{1}<\cdots\rightarrow\infty$. Considerando $S$ como un mapeo medible de $\left(\Omega,\mathcal{F}\right)$ al espacio sucesi\'on $\left(L,\mathcal{L}\right)$, donde 
\begin{eqnarray*}
L=\left\{\left(s_{k}\right)_{0}^{\infty}\in\left[0,\infty\right)^{\left\{0,1,\ldots\right\}}:s_{0}<s_{1}<\cdots\rightarrow\infty\right\},
\end{eqnarray*}
donde $\mathcal{L}$ son los subconjuntos de Borel de $L$, es decir, $\mathcal{L}=L\cap\mathcal{B}^{\left\{0,1,\ldots\right\}}$.

As\'i el par $\left(Z,S\right)$ es un mapeo medible de  $\left(\Omega,\mathcal{F}\right)$ en $\left(H\times L,\mathcal{H}\otimes\mathcal{L}\right)$. El par $\mathcal{H}\otimes\mathcal{L}^{+}$ denotar\'a la clase de todas las funciones medibles de $\left(H\times L,\mathcal{H}\otimes\mathcal{L}\right)$ en $\left(\left[0,\infty\right),\mathcal{B}\left[0,\infty\right)\right)$.
\end{Def}


\begin{Def}
Sea $\theta_{t}$ el mapeo-shift conjunto de $H\times L$ en $H\times L$ dado por
\begin{eqnarray*}
\theta_{t}\left(z,\left(s_{k}\right)_{0}^{\infty}\right)=\theta_{t}\left(z,\left(s_{n_{t-}+k}-t\right)_{0}^{\infty}\right)
\end{eqnarray*}
donde 
$n_{t-}=inf\left\{n\geq1:s_{n}\geq t\right\}$.
\end{Def}

\begin{Note}
Con la finalidad de poder realizar los shift's sin complicaciones de medibilidad, se supondr\'a que $Z$ es shit-medible, es decir, el conjunto de trayectorias $H$ es invariante bajo shifts del tiempo y el mapeo que toma $\left(z,t\right)\in H\times\left[0,\infty\right)$ en $z_{t}\in E$ es $\mathcal{H}\otimes\mathcal{B}\left[0,\infty\right)/\mathcal{E}$ medible.
\end{Note}

\begin{Def}
Dado un proceso \textbf{PEOSSM} (Proceso Estoc\'astico One Side Shift Medible) $Z$, se dice regenerativo cl\'asico con tiempos de regeneraci\'on $S$ si 

\begin{eqnarray*}
\theta_{S_{n}}\left(Z,S\right)=\left(Z^{0},S^{0}\right),n\geq0
\end{eqnarray*}
y adem\'as $\theta_{S_{n}}\left(Z,S\right)$ es independiente de $\left(\left(Z_{s}\right)s\in\left[0,S_{n}\right),S_{0},\ldots,S_{n}\right)$
Si lo anterior se cumple, al par $\left(Z,S\right)$ se le llama regenerativo cl\'asico.
\end{Def}

\begin{Note}
Si el par $\left(Z,S\right)$ es regenerativo cl\'asico, entonces las longitudes de los ciclos $X_{1},X_{2},\ldots,$ son i.i.d. e independientes de la longitud del retraso $S_{0}$, es decir, $S$ es un proceso de renovaci\'on. Las longitudes de los ciclos tambi\'en son llamados tiempos de inter-regeneraci\'on y tiempos de ocurrencia.

\end{Note}

\begin{Teo}
Sup\'ongase que el par $\left(Z,S\right)$ es regenerativo cl\'asico con $\esp\left[X_{1}\right]<\infty$. Entonces $\left(Z^{*},S^{*}\right)$ en el teorema 2.1 es una versi\'on estacionaria de $\left(Z,S\right)$. Adem\'as, si $X_{1}$ es lattice con span $d$, entonces $\left(Z^{**},S^{**}\right)$ en el teorema 2.2 es una versi\'on periodicamente estacionaria de $\left(Z,S\right)$ con periodo $d$.

\end{Teo}

\begin{Def}
Una variable aleatoria $X_{1}$ es \textit{spread out} si existe una $n\geq1$ y una  funci\'on $f\in\mathcal{B}^{+}$ tal que $\int_{\rea}f\left(x\right)dx>0$ con $X_{2},X_{3},\ldots,X_{n}$ copias i.i.d  de $X_{1}$, $$\prob\left(X_{1}+\cdots+X_{n}\in B\right)\geq\int_{B}f\left(x\right)dx$$ para $B\in\mathcal{B}$.

\end{Def}



\begin{Def}
Dado un proceso estoc\'astico $Z$ se le llama \textit{wide-sense regenerative} (\textbf{WSR}) con tiempos de regeneraci\'on $S$ si $\theta_{S_{n}}\left(Z,S\right)=\left(Z^{0},S^{0}\right)$ para $n\geq0$ en distribuci\'on y $\theta_{S_{n}}\left(Z,S\right)$ es independiente de $\left(S_{0},S_{1},\ldots,S_{n}\right)$ para $n\geq0$.
Se dice que el par $\left(Z,S\right)$ es WSR si lo anterior se cumple.
\end{Def}


\begin{Note}
\begin{itemize}
\item El proceso de trayectorias $\left(\theta_{s}Z\right)_{s\in\left[0,\infty\right)}$ es WSR con tiempos de regeneraci\'on $S$ pero no es regenerativo cl\'asico.

\item Si $Z$ es cualquier proceso estacionario y $S$ es un proceso de renovaci\'on que es independiente de $Z$, entonces $\left(Z,S\right)$ es WSR pero en general no es regenerativo cl\'asico

\end{itemize}

\end{Note}


\begin{Note}
Para cualquier proceso estoc\'astico $Z$, el proceso de trayectorias $\left(\theta_{s}Z\right)_{s\in\left[0,\infty\right)}$ es siempre un proceso de Markov.
\end{Note}



\begin{Teo}
Supongase que el par $\left(Z,S\right)$ es WSR con $\esp\left[X_{1}\right]<\infty$. Entonces $\left(Z^{*},S^{*}\right)$ en el teorema 2.1 es una versi\'on estacionaria de 
$\left(Z,S\right)$.
\end{Teo}


\begin{Teo}
Supongase que $\left(Z,S\right)$ es cycle-stationary con $\esp\left[X_{1}\right]<\infty$. Sea $U$ distribuida uniformemente en $\left[0,1\right)$ e independiente de $\left(Z^{0},S^{0}\right)$ y sea $\prob^{*}$ la medida de probabilidad en $\left(\Omega,\prob\right)$ definida por $$d\prob^{*}=\frac{X_{1}}{\esp\left[X_{1}\right]}d\prob$$. Sea $\left(Z^{*},S^{*}\right)$ con distribuci\'on $\prob^{*}\left(\theta_{UX_{1}}\left(Z^{0},S^{0}\right)\in\cdot\right)$. Entonces $\left(Z^{}*,S^{*}\right)$ es estacionario,
\begin{eqnarray*}
\esp\left[f\left(Z^{*},S^{*}\right)\right]=\esp\left[\int_{0}^{X_{1}}f\left(\theta_{s}\left(Z^{0},S^{0}\right)\right)ds\right]/\esp\left[X_{1}\right]
\end{eqnarray*}
$f\in\mathcal{H}\otimes\mathcal{L}^{+}$, and $S_{0}^{*}$ es continuo con funci\'on distribuci\'on $G_{\infty}$ definida por $$G_{\infty}\left(x\right):=\frac{\esp\left[X_{1}\right]\wedge x}{\esp\left[X_{1}\right]}$$ para $x\geq0$ y densidad $\prob\left[X_{1}>x\right]/\esp\left[X_{1}\right]$, con $x\geq0$.

\end{Teo}


\begin{Teo}
Sea $Z$ un Proceso Estoc\'astico un lado shift-medible \textit{one-sided shift-measurable stochastic process}, (PEOSSM),
y $S_{0}$ y $S_{1}$ tiempos aleatorios tales que $0\leq S_{0}<S_{1}$ y
\begin{equation}
\theta_{S_{1}}Z=\theta_{S_{0}}Z\textrm{ en distribuci\'on}.
\end{equation}

Entonces el espacio de probabilidad subyacente $\left(\Omega,\mathcal{F},\prob\right)$ puede extenderse para soportar una sucesi\'on de tiempos aleatorios $S$ tales que

\begin{eqnarray}
\theta_{S_{n}}\left(Z,S\right)=\left(Z^{0},S^{0}\right),n\geq0,\textrm{ en distribuci\'on},\\
\left(Z,S_{0},S_{1}\right)\textrm{ depende de }\left(X_{2},X_{3},\ldots\right)\textrm{ solamente a traves de }\theta_{S_{1}}Z.
\end{eqnarray}
\end{Teo}


%___________________________________________________________
%

\subsection{Material por agregar}

%___________________________________________________________
%
\section{Existencia de Tiempos de Regeneraci\'on}
%___________________________________________________________
%
\begin{Def}
Un elemento aleatorio en un espacio medible $\left(E,\mathcal{E}\right)$ en un espacio de probabilidad $\left(\Omega,\mathcal{F},\prob\right)$ a $\left(E,\mathcal{E}\right)$, es decir,
para $A\in \mathcal{E}$,  se tiene que $\left\{Y\in A\right\}\in\mathcal{F}$, donde $\left\{Y\in A\right\}:=\left\{w\in\Omega:Y\left(w\right)\in A\right\}=:Y^{-1}A$.
\end{Def}

\begin{Note}
Tambi\'en se dice que $Y$ est\'a soportado por el espacio de probabilidad $\left(\Omega,\mathcal{F},\prob\right)$ y que $Y$ es un mapeo medible de $\Omega$ en $E$, es decir, es $\mathcal{F}/\mathcal{E}$ medible.
\end{Note}

\begin{Def}
Para cada $i\in \mathbb{I}$ sea $P_{i}$ una medida de probabilidad en un espacio medible $\left(E_{i},\mathcal{E}_{i}\right)$. Se define el espacio producto
$\otimes_{i\in\mathbb{I}}\left(E_{i},\mathcal{E}_{i}\right):=\left(\prod_{i\in\mathbb{I}}E_{i},\otimes_{i\in\mathbb{I}}\mathcal{E}_{i}\right)$, donde $\prod_{i\in\mathbb{I}}E_{i}$ es el producto cartesiano de los $E_{i}$'s, y $\otimes_{i\in\mathbb{I}}\mathcal{E}_{i}$ es la $\sigma$-\'algebra producto, es decir, es la $\sigma$-\'algebra m\'as peque\~na en $\prod_{i\in\mathbb{I}}E_{i}$ que hace al $i$-\'esimo mapeo proyecci\'on en $E_{i}$ medible para toda $i\in\mathbb{I}$ es la $\sigma$-\'algebra inducida por los mapeos proyecci\'on. $$\otimes_{i\in\mathbb{I}}\mathcal{E}_{i}:=\sigma\left\{\left\{y:y_{i}\in A\right\}:i\in\mathbb{I}\textrm{ y }A\in\mathcal{E}_{i}\right\}.$$
\end{Def}

\begin{Def}
Un espacio de probabilidad $\left(\tilde{\Omega},\tilde{\mathcal{F}},\tilde{\prob}\right)$ es una extensi\'on de otro espacio de probabilidad $\left(\Omega,\mathcal{F},\prob\right)$ si $\left(\tilde{\Omega},\tilde{\mathcal{F}},\tilde{\prob}\right)$ soporta un elemento aleatorio $\xi\in\left(\Omega,\mathcal{F}\right)$ que tienen a $\prob$ como distribuci\'on.
\end{Def}

\begin{Teo}
Sea $\mathbb{I}$ un conjunto de \'indices arbitrario. Para cada $i\in\mathbb{I}$ sea $P_{i}$ una medida de probabilidad en un espacio medible $\left(E_{i},\mathcal{E}_{i}\right)$. Entonces existe una \'unica medida de probabilidad $\otimes_{i\in\mathbb{I}}P_{i}$ en $\otimes_{i\in\mathbb{I}}\left(E_{i},\mathcal{E}_{i}\right)$ tal que 

\begin{eqnarray*}
\otimes_{i\in\mathbb{I}}P_{i}\left(y\in\prod_{i\in\mathbb{I}}E_{i}:y_{i}\in A_{i_{1}},\ldots,y_{n}\in A_{i_{n}}\right)=P_{i_{1}}\left(A_{i_{n}}\right)\cdots P_{i_{n}}\left(A_{i_{n}}\right)
\end{eqnarray*}
para todos los enteros $n>0$, toda $i_{1},\ldots,i_{n}\in\mathbb{I}$ y todo $A_{i_{1}}\in\mathcal{E}_{i_{1}},\ldots,A_{i_{n}}\in\mathcal{E}_{i_{n}}$
\end{Teo}

La medida $\otimes_{i\in\mathbb{I}}P_{i}$ es llamada la medida producto y $\otimes_{i\in\mathbb{I}}\left(E_{i},\mathcal{E}_{i},P_{i}\right):=\left(\prod_{i\in\mathbb{I}},E_{i},\otimes_{i\in\mathbb{I}}\mathcal{E}_{i},\otimes_{i\in\mathbb{I}}P_{i}\right)$, es llamado espacio de probabilidad producto.


\begin{Def}
Un espacio medible $\left(E,\mathcal{E}\right)$ es \textit{Polaco} si existe una m\'etrica en $E$ tal que $E$ es completo, es decir cada sucesi\'on de Cauchy converge a un l\'imite en $E$, y \textit{separable}, $E$ tienen un subconjunto denso numerable, y tal que $\mathcal{E}$ es generado por conjuntos abiertos.
\end{Def}


\begin{Def}
Dos espacios medibles $\left(E,\mathcal{E}\right)$ y $\left(G,\mathcal{G}\right)$ son Borel equivalentes \textit{isomorfos} si existe una biyecci\'on $f:E\rightarrow G$ tal que $f$ es $\mathcal{E}/\mathcal{G}$ medible y su inversa $f^{-1}$ es $\mathcal{G}/\mathcal{E}$ medible. La biyecci\'on es una equivalencia de Borel.
\end{Def}

\begin{Def}
Un espacio medible  $\left(E,\mathcal{E}\right)$ es un \textit{espacio est\'andar} si es Borel equivalente a $\left(G,\mathcal{G}\right)$, donde $G$ es un subconjunto de Borel de $\left[0,1\right]$ y $\mathcal{G}$ son los subconjuntos de Borel de $G$.
\end{Def}

\begin{Note}
Cualquier espacio Polaco es un espacio est\'andar.
\end{Note}


\begin{Def}
Un proceso estoc\'astico con conjunto de \'indices $\mathbb{I}$ y espacio de estados $\left(E,\mathcal{E}\right)$ es una familia $Z=\left(\mathbb{Z}_{s}\right)_{s\in\mathbb{I}}$ donde $\mathbb{Z}_{s}$ son elementos aleatorios definidos en un espacio de probabilidad com\'un $\left(\Omega,\mathcal{F},\prob\right)$ y todos toman valores en $\left(E,\mathcal{E}\right)$.
\end{Def}

\begin{Def}
Un proceso estoc\'astico \textit{one-sided contiuous time} (\textbf{PEOSCT}) es un proceso estoc\'astico con conjunto de \'indices $\mathbb{I}=\left[0,\infty\right)$.
\end{Def}


Sea $\left(E^{\mathbb{I}},\mathcal{E}^{\mathbb{I}}\right)$ denota el espacio producto $\left(E^{\mathbb{I}},\mathcal{E}^{\mathbb{I}}\right):=\otimes_{s\in\mathbb{I}}\left(E,\mathcal{E}\right)$. Vamos a considerar $\mathbb{Z}$ como un mapeo aleatorio, es decir, como un elemento aleatorio en $\left(E^{\mathbb{I}},\mathcal{E}^{\mathbb{I}}\right)$ definido por $Z\left(w\right)=\left(Z_{s}\left(w\right)\right)_{s\in\mathbb{I}}$ y $w\in\Omega$.

\begin{Note}
La distribuci\'on de un proceso estoc\'astico $Z$ es la distribuci\'on de $Z$ como un elemento aleatorio en $\left(E^{\mathbb{I}},\mathcal{E}^{\mathbb{I}}\right)$. La distribuci\'on de $Z$ esta determinada de manera \'unica por las distribuciones finito dimensionales.
\end{Note}

\begin{Note}
En particular cuando $Z$ toma valores reales, es decir, $\left(E,\mathcal{E}\right)=\left(\mathbb{R},\mathcal{B}\right)$ las distribuciones finito dimensionales est\'an determinadas por las funciones de distribuci\'on finito dimensionales

\begin{eqnarray}
\prob\left(Z_{t_{1}}\leq x_{1},\ldots,Z_{t_{n}}\leq x_{n}\right),x_{1},\ldots,x_{n}\in\mathbb{R},t_{1},\ldots,t_{n}\in\mathbb{I},n\geq1.
\end{eqnarray}
\end{Note}

\begin{Note}
Para espacios polacos $\left(E,\mathcal{E}\right)$ el Teorema de Consistencia de Kolmogorov asegura que dada una colecci\'on de distribuciones finito dimensionales consistentes, siempre existe un proceso estoc\'astico que posee tales distribuciones finito dimensionales.
\end{Note}


\begin{Def}
Las trayectorias de $Z$ son las realizaciones $Z\left(w\right)$ para $w\in\Omega$ del mapeo aleatorio $Z$.
\end{Def}

\begin{Note}
Algunas restricciones se imponen sobre las trayectorias, por ejemplo que sean continuas por la derecha, o continuas por la derecha con l\'imites por la izquierda, o de manera m\'as general, se pedir\'a que caigan en alg\'un subconjunto $H$ de $E^{\mathbb{I}}$. En este caso es natural considerar a $Z$ como un elemento aleatorio que no est\'a en $\left(E^{\mathbb{I}},\mathcal{E}^{\mathbb{I}}\right)$ sino en $\left(H,\mathcal{H}\right)$, donde $\mathcal{H}$ es la $\sigma$-\'algebra generada por los mapeos proyecci\'on que toman a $z\in H$ a $z_{t}\in E$ para $t\in\mathbb{I}$. A $\mathcal{H}$ se le conoce como la traza de $H$ en $E^{\mathbb{I}}$, es decir,
\begin{eqnarray}
\mathcal{H}:=E^{\mathbb{I}}\cap H:=\left\{A\cap H:A\in E^{\mathbb{I}}\right\}.
\end{eqnarray}
\end{Note}


\begin{Note}
$Z$ tiene trayectorias con valores en $H$ y cada $Z_{t}$ es un mapeo medible de $\left(\Omega,\mathcal{F}\right)$ a $\left(H,\mathcal{H}\right)$. Cuando se considera un espacio de trayectorias en particular $H$, al espacio $\left(H,\mathcal{H}\right)$ se le llama el espacio de trayectorias de $Z$.
\end{Note}

\begin{Note}
La distribuci\'on del proceso estoc\'astico $Z$ con espacio de trayectorias $\left(H,\mathcal{H}\right)$ es la distribuci\'on de $Z$ como  un elemento aleatorio en $\left(H,\mathcal{H}\right)$. La distribuci\'on, nuevemente, est\'a determinada de manera \'unica por las distribuciones finito dimensionales.
\end{Note}


\begin{Def}
Sea $Z$ un PEOSCT  con espacio de estados $\left(E,\mathcal{E}\right)$ y sea $T$ un tiempo aleatorio en $\left[0,\infty\right)$. Por $Z_{T}$ se entiende el mapeo con valores en $E$ definido en $\Omega$ en la manera obvia:
\begin{eqnarray*}
Z_{T}\left(w\right):=Z_{T\left(w\right)}\left(w\right). w\in\Omega.
\end{eqnarray*}
\end{Def}

\begin{Def}
Un PEOSCT $Z$ es conjuntamente medible (\textbf{CM}) si el mapeo que toma $\left(w,t\right)\in\Omega\times\left[0,\infty\right)$ a $Z_{t}\left(w\right)\in E$ es $\mathcal{F}\otimes\mathcal{B}\left[0,\infty\right)/\mathcal{E}$ medible.
\end{Def}

\begin{Note}
Un PEOSCT-CM implica que el proceso es medible, dado que $Z_{T}$ es una composici\'on  de dos mapeos continuos: el primero que toma $w$ en $\left(w,T\left(w\right)\right)$ es $\mathcal{F}/\mathcal{F}\otimes\mathcal{B}\left[0,\infty\right)$ medible, mientras que el segundo toma $\left(w,T\left(w\right)\right)$ en $Z_{T\left(w\right)}\left(w\right)$ es $\mathcal{F}\otimes\mathcal{B}\left[0,\infty\right)/\mathcal{E}$ medible.
\end{Note}


\begin{Def}
Un PEOSCT con espacio de estados $\left(H,\mathcal{H}\right)$ es can\'onicamente conjuntamente medible (\textbf{CCM}) si el mapeo $\left(z,t\right)\in H\times\left[0,\infty\right)$ en $Z_{t}\in E$ es $\mathcal{H}\otimes\mathcal{B}\left[0,\infty\right)/\mathcal{E}$ medible.
\end{Def}

\begin{Note}
Un PEOSCTCCM implica que el proceso es CM, dado que un PECCM $Z$ es un mapeo de $\Omega\times\left[0,\infty\right)$ a $E$, es la composici\'on de dos mapeos medibles: el primero, toma $\left(w,t\right)$ en $\left(Z\left(w\right),t\right)$ es $\mathcal{F}\otimes\mathcal{B}\left[0,\infty\right)/\mathcal{H}\otimes\mathcal{B}\left[0,\infty\right)$ medible, y el segundo que toma $\left(Z\left(w\right),t\right)$  en $Z_{t}\left(w\right)$ es $\mathcal{H}\otimes\mathcal{B}\left[0,\infty\right)/\mathcal{E}$ medible. Por tanto CCM es una condici\'on m\'as fuerte que CM.
\end{Note}

\begin{Def}
Un conjunto de trayectorias $H$ de un PEOSCT $Z$, es internamente shift-invariante (\textbf{ISI}) si 
\begin{eqnarray*}
\left\{\left(z_{t+s}\right)_{s\in\left[0,\infty\right)}:z\in H\right\}=H\textrm{, }t\in\left[0,\infty\right).
\end{eqnarray*}
\end{Def}


\begin{Def}
Dado un PEOSCTISI, se define el mapeo-shift $\theta_{t}$, $t\in\left[0,\infty\right)$, de $H$ a $H$ por 
\begin{eqnarray*}
\theta_{t}z=\left(z_{t+s}\right)_{s\in\left[0,\infty\right)}\textrm{, }z\in H.
\end{eqnarray*}
\end{Def}

\begin{Def}
Se dice que un proceso $Z$ es shift-medible (\textbf{SM}) si $Z$ tiene un conjunto de trayectorias $H$ que es ISI y adem\'as el mapeo que toma $\left(z,t\right)\in H\times\left[0,\infty\right)$ en $\theta_{t}z\in H$ es $\mathcal{H}\otimes\mathcal{B}\left[0,\infty\right)/\mathcal{H}$ medible.
\end{Def}

\begin{Note}
Un proceso estoc\'astico con conjunto de trayectorias $H$ ISI es shift-medible si y s\'olo si es CCM
\end{Note}

\begin{Note}
\begin{itemize}
\item Dado el espacio polaco $\left(E,\mathcal{E}\right)$ se tiene el  conjunto de trayectorias $D_{E}\left[0,\infty\right)$ que es ISI, entonces cumpe con ser CCM.

\item Si $G$ es abierto, podemos cubrirlo por bolas abiertas cuay cerradura este contenida en $G$, y como $G$ es segundo numerable como subespacio de $E$, lo podemos cubrir por una cantidad numerable de bolas abiertas.

\end{itemize}
\end{Note}


\begin{Note}
Los procesos estoc\'asticos $Z$ a tiempo discreto con espacio de estados polaco, tambi\'en tiene un espacio de trayectorias polaco y por tanto tiene distribuciones condicionales regulares.
\end{Note}

\begin{Teo}
El producto numerable de espacios polacos es polaco.
\end{Teo}


\begin{Def}
Sea $\left(\Omega,\mathcal{F},\prob\right)$ espacio de probabilidad que soporta al proceso $Z=\left(Z_{s}\right)_{s\in\left[0,\infty\right)}$ y $S=\left(S_{k}\right)_{0}^{\infty}$ donde $Z$ es un PEOSCTM con espacio de estados $\left(E,\mathcal{E}\right)$  y espacio de trayectorias $\left(H,\mathcal{H}\right)$  y adem\'as $S$ es una sucesi\'on de tiempos aleatorios one-sided que satisfacen la condici\'on $0\leq S_{0}<S_{1}<\cdots\rightarrow\infty$. Considerando $S$ como un mapeo medible de $\left(\Omega,\mathcal{F}\right)$ al espacio sucesi\'on $\left(L,\mathcal{L}\right)$, donde 
\begin{eqnarray*}
L=\left\{\left(s_{k}\right)_{0}^{\infty}\in\left[0,\infty\right)^{\left\{0,1,\ldots\right\}}:s_{0}<s_{1}<\cdots\rightarrow\infty\right\},
\end{eqnarray*}
donde $\mathcal{L}$ son los subconjuntos de Borel de $L$, es decir, $\mathcal{L}=L\cap\mathcal{B}^{\left\{0,1,\ldots\right\}}$.

As\'i el par $\left(Z,S\right)$ es un mapeo medible de  $\left(\Omega,\mathcal{F}\right)$ en $\left(H\times L,\mathcal{H}\otimes\mathcal{L}\right)$. El par $\mathcal{H}\otimes\mathcal{L}^{+}$ denotar\'a la clase de todas las funciones medibles de $\left(H\times L,\mathcal{H}\otimes\mathcal{L}\right)$ en $\left(\left[0,\infty\right),\mathcal{B}\left[0,\infty\right)\right)$.
\end{Def}


\begin{Def}
Sea $\theta_{t}$ el mapeo-shift conjunto de $H\times L$ en $H\times L$ dado por
\begin{eqnarray*}
\theta_{t}\left(z,\left(s_{k}\right)_{0}^{\infty}\right)=\theta_{t}\left(z,\left(s_{n_{t-}+k}-t\right)_{0}^{\infty}\right)
\end{eqnarray*}
donde 
$n_{t-}=inf\left\{n\geq1:s_{n}\geq t\right\}$.
\end{Def}

\begin{Note}
Con la finalidad de poder realizar los shift's sin complicaciones de medibilidad, se supondr\'a que $Z$ es shit-medible, es decir, el conjunto de trayectorias $H$ es invariante bajo shifts del tiempo y el mapeo que toma $\left(z,t\right)\in H\times\left[0,\infty\right)$ en $z_{t}\in E$ es $\mathcal{H}\otimes\mathcal{B}\left[0,\infty\right)/\mathcal{E}$ medible.
\end{Note}

\begin{Def}
Dado un proceso \textbf{PEOSSM} (Proceso Estoc\'astico One Side Shift Medible) $Z$, se dice regenerativo cl\'asico con tiempos de regeneraci\'on $S$ si 

\begin{eqnarray*}
\theta_{S_{n}}\left(Z,S\right)=\left(Z^{0},S^{0}\right),n\geq0
\end{eqnarray*}
y adem\'as $\theta_{S_{n}}\left(Z,S\right)$ es independiente de $\left(\left(Z_{s}\right)s\in\left[0,S_{n}\right),S_{0},\ldots,S_{n}\right)$
Si lo anterior se cumple, al par $\left(Z,S\right)$ se le llama regenerativo cl\'asico.
\end{Def}

\begin{Note}
Si el par $\left(Z,S\right)$ es regenerativo cl\'asico, entonces las longitudes de los ciclos $X_{1},X_{2},\ldots,$ son i.i.d. e independientes de la longitud del retraso $S_{0}$, es decir, $S$ es un proceso de renovaci\'on. Las longitudes de los ciclos tambi\'en son llamados tiempos de inter-regeneraci\'on y tiempos de ocurrencia.

\end{Note}

\begin{Teo}
Sup\'ongase que el par $\left(Z,S\right)$ es regenerativo cl\'asico con $\esp\left[X_{1}\right]<\infty$. Entonces $\left(Z^{*},S^{*}\right)$ en el teorema 2.1 es una versi\'on estacionaria de $\left(Z,S\right)$. Adem\'as, si $X_{1}$ es lattice con span $d$, entonces $\left(Z^{**},S^{**}\right)$ en el teorema 2.2 es una versi\'on periodicamente estacionaria de $\left(Z,S\right)$ con periodo $d$.

\end{Teo}

\begin{Def}
Una variable aleatoria $X_{1}$ es \textit{spread out} si existe una $n\geq1$ y una  funci\'on $f\in\mathcal{B}^{+}$ tal que $\int_{\rea}f\left(x\right)dx>0$ con $X_{2},X_{3},\ldots,X_{n}$ copias i.i.d  de $X_{1}$, $$\prob\left(X_{1}+\cdots+X_{n}\in B\right)\geq\int_{B}f\left(x\right)dx$$ para $B\in\mathcal{B}$.

\end{Def}



\begin{Def}
Dado un proceso estoc\'astico $Z$ se le llama \textit{wide-sense regenerative} (\textbf{WSR}) con tiempos de regeneraci\'on $S$ si $\theta_{S_{n}}\left(Z,S\right)=\left(Z^{0},S^{0}\right)$ para $n\geq0$ en distribuci\'on y $\theta_{S_{n}}\left(Z,S\right)$ es independiente de $\left(S_{0},S_{1},\ldots,S_{n}\right)$ para $n\geq0$.
Se dice que el par $\left(Z,S\right)$ es WSR si lo anterior se cumple.
\end{Def}


\begin{Note}
\begin{itemize}
\item El proceso de trayectorias $\left(\theta_{s}Z\right)_{s\in\left[0,\infty\right)}$ es WSR con tiempos de regeneraci\'on $S$ pero no es regenerativo cl\'asico.

\item Si $Z$ es cualquier proceso estacionario y $S$ es un proceso de renovaci\'on que es independiente de $Z$, entonces $\left(Z,S\right)$ es WSR pero en general no es regenerativo cl\'asico

\end{itemize}

\end{Note}


\begin{Note}
Para cualquier proceso estoc\'astico $Z$, el proceso de trayectorias $\left(\theta_{s}Z\right)_{s\in\left[0,\infty\right)}$ es siempre un proceso de Markov.
\end{Note}



\begin{Teo}
Supongase que el par $\left(Z,S\right)$ es WSR con $\esp\left[X_{1}\right]<\infty$. Entonces $\left(Z^{*},S^{*}\right)$ en el teorema 2.1 es una versi\'on estacionaria de 
$\left(Z,S\right)$.
\end{Teo}


\begin{Teo}
Supongase que $\left(Z,S\right)$ es cycle-stationary con $\esp\left[X_{1}\right]<\infty$. Sea $U$ distribuida uniformemente en $\left[0,1\right)$ e independiente de $\left(Z^{0},S^{0}\right)$ y sea $\prob^{*}$ la medida de probabilidad en $\left(\Omega,\prob\right)$ definida por $$d\prob^{*}=\frac{X_{1}}{\esp\left[X_{1}\right]}d\prob$$. Sea $\left(Z^{*},S^{*}\right)$ con distribuci\'on $\prob^{*}\left(\theta_{UX_{1}}\left(Z^{0},S^{0}\right)\in\cdot\right)$. Entonces $\left(Z^{}*,S^{*}\right)$ es estacionario,
\begin{eqnarray*}
\esp\left[f\left(Z^{*},S^{*}\right)\right]=\esp\left[\int_{0}^{X_{1}}f\left(\theta_{s}\left(Z^{0},S^{0}\right)\right)ds\right]/\esp\left[X_{1}\right]
\end{eqnarray*}
$f\in\mathcal{H}\otimes\mathcal{L}^{+}$, and $S_{0}^{*}$ es continuo con funci\'on distribuci\'on $G_{\infty}$ definida por $$G_{\infty}\left(x\right):=\frac{\esp\left[X_{1}\right]\wedge x}{\esp\left[X_{1}\right]}$$ para $x\geq0$ y densidad $\prob\left[X_{1}>x\right]/\esp\left[X_{1}\right]$, con $x\geq0$.

\end{Teo}


\begin{Teo}
Sea $Z$ un Proceso Estoc\'astico un lado shift-medible \textit{one-sided shift-measurable stochastic process}, (PEOSSM),
y $S_{0}$ y $S_{1}$ tiempos aleatorios tales que $0\leq S_{0}<S_{1}$ y
\begin{equation}
\theta_{S_{1}}Z=\theta_{S_{0}}Z\textrm{ en distribuci\'on}.
\end{equation}

Entonces el espacio de probabilidad subyacente $\left(\Omega,\mathcal{F},\prob\right)$ puede extenderse para soportar una sucesi\'on de tiempos aleatorios $S$ tales que

\begin{eqnarray}
\theta_{S_{n}}\left(Z,S\right)=\left(Z^{0},S^{0}\right),n\geq0,\textrm{ en distribuci\'on},\\
\left(Z,S_{0},S_{1}\right)\textrm{ depende de }\left(X_{2},X_{3},\ldots\right)\textrm{ solamente a traves de }\theta_{S_{1}}Z.
\end{eqnarray}
\end{Teo}

%________________________________________________________________________
\section{Procesos Regenerativos}
%________________________________________________________________________

%________________________________________________________________________
\subsection*{Procesos Regenerativos Sigman, Thorisson y Wolff \cite{Sigman1}}
%________________________________________________________________________


\begin{Def}[Definici\'on Cl\'asica]
Un proceso estoc\'astico $X=\left\{X\left(t\right):t\geq0\right\}$ es llamado regenerativo is existe una variable aleatoria $R_{1}>0$ tal que
\begin{itemize}
\item[i)] $\left\{X\left(t+R_{1}\right):t\geq0\right\}$ es independiente de $\left\{\left\{X\left(t\right):t<R_{1}\right\},\right\}$
\item[ii)] $\left\{X\left(t+R_{1}\right):t\geq0\right\}$ es estoc\'asticamente equivalente a $\left\{X\left(t\right):t>0\right\}$
\end{itemize}

Llamamos a $R_{1}$ tiempo de regeneraci\'on, y decimos que $X$ se regenera en este punto.
\end{Def}

$\left\{X\left(t+R_{1}\right)\right\}$ es regenerativo con tiempo de regeneraci\'on $R_{2}$, independiente de $R_{1}$ pero con la misma distribuci\'on que $R_{1}$. Procediendo de esta manera se obtiene una secuencia de variables aleatorias independientes e id\'enticamente distribuidas $\left\{R_{n}\right\}$ llamados longitudes de ciclo. Si definimos a $Z_{k}\equiv R_{1}+R_{2}+\cdots+R_{k}$, se tiene un proceso de renovaci\'on llamado proceso de renovaci\'on encajado para $X$.


\begin{Note}
La existencia de un primer tiempo de regeneraci\'on, $R_{1}$, implica la existencia de una sucesi\'on completa de estos tiempos $R_{1},R_{2}\ldots,$ que satisfacen la propiedad deseada \cite{Sigman2}.
\end{Note}


\begin{Note} Para la cola $GI/GI/1$ los usuarios arriban con tiempos $t_{n}$ y son atendidos con tiempos de servicio $S_{n}$, los tiempos de arribo forman un proceso de renovaci\'on  con tiempos entre arribos independientes e identicamente distribuidos (\texttt{i.i.d.})$T_{n}=t_{n}-t_{n-1}$, adem\'as los tiempos de servicio son \texttt{i.i.d.} e independientes de los procesos de arribo. Por \textit{estable} se entiende que $\esp S_{n}<\esp T_{n}<\infty$.
\end{Note}
 


\begin{Def}
Para $x$ fijo y para cada $t\geq0$, sea $I_{x}\left(t\right)=1$ si $X\left(t\right)\leq x$,  $I_{x}\left(t\right)=0$ en caso contrario, y def\'inanse los tiempos promedio
\begin{eqnarray*}
\overline{X}&=&lim_{t\rightarrow\infty}\frac{1}{t}\int_{0}^{\infty}X\left(u\right)du\\
\prob\left(X_{\infty}\leq x\right)&=&lim_{t\rightarrow\infty}\frac{1}{t}\int_{0}^{\infty}I_{x}\left(u\right)du,
\end{eqnarray*}
cuando estos l\'imites existan.
\end{Def}

Como consecuencia del teorema de Renovaci\'on-Recompensa, se tiene que el primer l\'imite  existe y es igual a la constante
\begin{eqnarray*}
\overline{X}&=&\frac{\esp\left[\int_{0}^{R_{1}}X\left(t\right)dt\right]}{\esp\left[R_{1}\right]},
\end{eqnarray*}
suponiendo que ambas esperanzas son finitas.
 
\begin{Note}
Funciones de procesos regenerativos son regenerativas, es decir, si $X\left(t\right)$ es regenerativo y se define el proceso $Y\left(t\right)$ por $Y\left(t\right)=f\left(X\left(t\right)\right)$ para alguna funci\'on Borel medible $f\left(\cdot\right)$. Adem\'as $Y$ es regenerativo con los mismos tiempos de renovaci\'on que $X$. 

En general, los tiempos de renovaci\'on, $Z_{k}$ de un proceso regenerativo no requieren ser tiempos de paro con respecto a la evoluci\'on de $X\left(t\right)$.
\end{Note} 

\begin{Note}
Una funci\'on de un proceso de Markov, usualmente no ser\'a un proceso de Markov, sin embargo ser\'a regenerativo si el proceso de Markov lo es.
\end{Note}

 
\begin{Note}
Un proceso regenerativo con media de la longitud de ciclo finita es llamado positivo recurrente.
\end{Note}


\begin{Note}
\begin{itemize}
\item[a)] Si el proceso regenerativo $X$ es positivo recurrente y tiene trayectorias muestrales no negativas, entonces la ecuaci\'on anterior es v\'alida.
\item[b)] Si $X$ es positivo recurrente regenerativo, podemos construir una \'unica versi\'on estacionaria de este proceso, $X_{e}=\left\{X_{e}\left(t\right)\right\}$, donde $X_{e}$ es un proceso estoc\'astico regenerativo y estrictamente estacionario, con distribuci\'on marginal distribuida como $X_{\infty}$
\end{itemize}
\end{Note}


%__________________________________________________________________________________________
\subsection*{Procesos Regenerativos Estacionarios - Stidham \cite{Stidham}}
%__________________________________________________________________________________________


Un proceso estoc\'astico a tiempo continuo $\left\{V\left(t\right),t\geq0\right\}$ es un proceso regenerativo si existe una sucesi\'on de variables aleatorias independientes e id\'enticamente distribuidas $\left\{X_{1},X_{2},\ldots\right\}$, sucesi\'on de renovaci\'on, tal que para cualquier conjunto de Borel $A$, 

\begin{eqnarray*}
\prob\left\{V\left(t\right)\in A|X_{1}+X_{2}+\cdots+X_{R\left(t\right)}=s,\left\{V\left(\tau\right),\tau<s\right\}\right\}=\prob\left\{V\left(t-s\right)\in A|X_{1}>t-s\right\},
\end{eqnarray*}
para todo $0\leq s\leq t$, donde $R\left(t\right)=\max\left\{X_{1}+X_{2}+\cdots+X_{j}\leq t\right\}=$n\'umero de renovaciones ({\emph{puntos de regeneraci\'on}}) que ocurren en $\left[0,t\right]$. El intervalo $\left[0,X_{1}\right)$ es llamado {\emph{primer ciclo de regeneraci\'on}} de $\left\{V\left(t \right),t\geq0\right\}$, $\left[X_{1},X_{1}+X_{2}\right)$ el {\emph{segundo ciclo de regeneraci\'on}}, y as\'i sucesivamente.

Sea $X=X_{1}$ y sea $F$ la funci\'on de distrbuci\'on de $X$


\begin{Def}
Se define el proceso estacionario, $\left\{V^{*}\left(t\right),t\geq0\right\}$, para $\left\{V\left(t\right),t\geq0\right\}$ por

\begin{eqnarray*}
\prob\left\{V\left(t\right)\in A\right\}=\frac{1}{\esp\left[X\right]}\int_{0}^{\infty}\prob\left\{V\left(t+x\right)\in A|X>x\right\}\left(1-F\left(x\right)\right)dx,
\end{eqnarray*} 
para todo $t\geq0$ y todo conjunto de Borel $A$.
\end{Def}

\begin{Def}
Una distribuci\'on se dice que es {\emph{aritm\'etica}} si todos sus puntos de incremento son m\'ultiplos de la forma $0,\lambda, 2\lambda,\ldots$ para alguna $\lambda>0$ entera.
\end{Def}


\begin{Def}
Una modificaci\'on medible de un proceso $\left\{V\left(t\right),t\geq0\right\}$, es una versi\'on de este, $\left\{V\left(t,w\right)\right\}$ conjuntamente medible para $t\geq0$ y para $w\in S$, $S$ espacio de estados para $\left\{V\left(t\right),t\geq0\right\}$.
\end{Def}

\begin{Teo}
Sea $\left\{V\left(t\right),t\geq\right\}$ un proceso regenerativo no negativo con modificaci\'on medible. Sea $\esp\left[X\right]<\infty$. Entonces el proceso estacionario dado por la ecuaci\'on anterior est\'a bien definido y tiene funci\'on de distribuci\'on independiente de $t$, adem\'as
\begin{itemize}
\item[i)] \begin{eqnarray*}
\esp\left[V^{*}\left(0\right)\right]&=&\frac{\esp\left[\int_{0}^{X}V\left(s\right)ds\right]}{\esp\left[X\right]}\end{eqnarray*}
\item[ii)] Si $\esp\left[V^{*}\left(0\right)\right]<\infty$, equivalentemente, si $\esp\left[\int_{0}^{X}V\left(s\right)ds\right]<\infty$,entonces
\begin{eqnarray*}
\frac{\int_{0}^{t}V\left(s\right)ds}{t}\rightarrow\frac{\esp\left[\int_{0}^{X}V\left(s\right)ds\right]}{\esp\left[X\right]}
\end{eqnarray*}
con probabilidad 1 y en media, cuando $t\rightarrow\infty$.
\end{itemize}
\end{Teo}

\begin{Coro}
Sea $\left\{V\left(t\right),t\geq0\right\}$ un proceso regenerativo no negativo, con modificaci\'on medible. Si $\esp <\infty$, $F$ es no-aritm\'etica, y para todo $x\geq0$, $P\left\{V\left(t\right)\leq x,C>x\right\}$ es de variaci\'on acotada como funci\'on de $t$ en cada intervalo finito $\left[0,\tau\right]$, entonces $V\left(t\right)$ converge en distribuci\'on  cuando $t\rightarrow\infty$ y $$\esp V=\frac{\esp \int_{0}^{X}V\left(s\right)ds}{\esp X}$$
Donde $V$ tiene la distribuci\'on l\'imite de $V\left(t\right)$ cuando $t\rightarrow\infty$.

\end{Coro}

Para el caso discreto se tienen resultados similares.



%______________________________________________________________________
\section{Procesos de Renovaci\'on}
%______________________________________________________________________

\begin{Def}\label{Def.Tn}
Sean $0\leq T_{1}\leq T_{2}\leq \ldots$ son tiempos aleatorios infinitos en los cuales ocurren ciertos eventos. El n\'umero de tiempos $T_{n}$ en el intervalo $\left[0,t\right)$ es

\begin{eqnarray}
N\left(t\right)=\sum_{n=1}^{\infty}\indora\left(T_{n}\leq t\right),
\end{eqnarray}
para $t\geq0$.
\end{Def}

Si se consideran los puntos $T_{n}$ como elementos de $\rea_{+}$, y $N\left(t\right)$ es el n\'umero de puntos en $\rea$. El proceso denotado por $\left\{N\left(t\right):t\geq0\right\}$, denotado por $N\left(t\right)$, es un proceso puntual en $\rea_{+}$. Los $T_{n}$ son los tiempos de ocurrencia, el proceso puntual $N\left(t\right)$ es simple si su n\'umero de ocurrencias son distintas: $0<T_{1}<T_{2}<\ldots$ casi seguramente.

\begin{Def}
Un proceso puntual $N\left(t\right)$ es un proceso de renovaci\'on si los tiempos de interocurrencia $\xi_{n}=T_{n}-T_{n-1}$, para $n\geq1$, son independientes e identicamente distribuidos con distribuci\'on $F$, donde $F\left(0\right)=0$ y $T_{0}=0$. Los $T_{n}$ son llamados tiempos de renovaci\'on, referente a la independencia o renovaci\'on de la informaci\'on estoc\'astica en estos tiempos. Los $\xi_{n}$ son los tiempos de inter-renovaci\'on, y $N\left(t\right)$ es el n\'umero de renovaciones en el intervalo $\left[0,t\right)$
\end{Def}


\begin{Note}
Para definir un proceso de renovaci\'on para cualquier contexto, solamente hay que especificar una distribuci\'on $F$, con $F\left(0\right)=0$, para los tiempos de inter-renovaci\'on. La funci\'on $F$ en turno degune las otra variables aleatorias. De manera formal, existe un espacio de probabilidad y una sucesi\'on de variables aleatorias $\xi_{1},\xi_{2},\ldots$ definidas en este con distribuci\'on $F$. Entonces las otras cantidades son $T_{n}=\sum_{k=1}^{n}\xi_{k}$ y $N\left(t\right)=\sum_{n=1}^{\infty}\indora\left(T_{n}\leq t\right)$, donde $T_{n}\rightarrow\infty$ casi seguramente por la Ley Fuerte de los Grandes Números.
\end{Note}

%___________________________________________________________________________________________
%
\subsection{Teorema Principal de Renovaci\'on}
%___________________________________________________________________________________________
%

\begin{Note} Una funci\'on $h:\rea_{+}\rightarrow\rea$ es Directamente Riemann Integrable en los siguientes casos:
\begin{itemize}
\item[a)] $h\left(t\right)\geq0$ es decreciente y Riemann Integrable.
\item[b)] $h$ es continua excepto posiblemente en un conjunto de Lebesgue de medida 0, y $|h\left(t\right)|\leq b\left(t\right)$, donde $b$ es DRI.
\end{itemize}
\end{Note}

\begin{Teo}[Teorema Principal de Renovaci\'on]
Si $F$ es no aritm\'etica y $h\left(t\right)$ es Directamente Riemann Integrable (DRI), entonces

\begin{eqnarray*}
lim_{t\rightarrow\infty}U\star h=\frac{1}{\mu}\int_{\rea_{+}}h\left(s\right)ds.
\end{eqnarray*}
\end{Teo}

\begin{Prop}
Cualquier funci\'on $H\left(t\right)$ acotada en intervalos finitos y que es 0 para $t<0$ puede expresarse como
\begin{eqnarray*}
H\left(t\right)=U\star h\left(t\right)\textrm{,  donde }h\left(t\right)=H\left(t\right)-F\star H\left(t\right)
\end{eqnarray*}
\end{Prop}

\begin{Def}
Un proceso estoc\'astico $X\left(t\right)$ es crudamente regenerativo en un tiempo aleatorio positivo $T$ si
\begin{eqnarray*}
\esp\left[X\left(T+t\right)|T\right]=\esp\left[X\left(t\right)\right]\textrm{, para }t\geq0,\end{eqnarray*}
y con las esperanzas anteriores finitas.
\end{Def}

\begin{Prop}
Sup\'ongase que $X\left(t\right)$ es un proceso crudamente regenerativo en $T$, que tiene distribuci\'on $F$. Si $\esp\left[X\left(t\right)\right]$ es acotado en intervalos finitos, entonces
\begin{eqnarray*}
\esp\left[X\left(t\right)\right]=U\star h\left(t\right)\textrm{,  donde }h\left(t\right)=\esp\left[X\left(t\right)\indora\left(T>t\right)\right].
\end{eqnarray*}
\end{Prop}

\begin{Teo}[Regeneraci\'on Cruda]
Sup\'ongase que $X\left(t\right)$ es un proceso con valores positivo crudamente regenerativo en $T$, y def\'inase $M=\sup\left\{|X\left(t\right)|:t\leq T\right\}$. Si $T$ es no aritm\'etico y $M$ y $MT$ tienen media finita, entonces
\begin{eqnarray*}
lim_{t\rightarrow\infty}\esp\left[X\left(t\right)\right]=\frac{1}{\mu}\int_{\rea_{+}}h\left(s\right)ds,
\end{eqnarray*}
donde $h\left(t\right)=\esp\left[X\left(t\right)\indora\left(T>t\right)\right]$.
\end{Teo}

%___________________________________________________________________________________________
%
\subsection{Propiedades de los Procesos de Renovaci\'on}
%___________________________________________________________________________________________
%

Los tiempos $T_{n}$ est\'an relacionados con los conteos de $N\left(t\right)$ por

\begin{eqnarray*}
\left\{N\left(t\right)\geq n\right\}&=&\left\{T_{n}\leq t\right\}\\
T_{N\left(t\right)}\leq &t&<T_{N\left(t\right)+1},
\end{eqnarray*}

adem\'as $N\left(T_{n}\right)=n$, y 

\begin{eqnarray*}
N\left(t\right)=\max\left\{n:T_{n}\leq t\right\}=\min\left\{n:T_{n+1}>t\right\}
\end{eqnarray*}

Por propiedades de la convoluci\'on se sabe que

\begin{eqnarray*}
P\left\{T_{n}\leq t\right\}=F^{n\star}\left(t\right)
\end{eqnarray*}
que es la $n$-\'esima convoluci\'on de $F$. Entonces 

\begin{eqnarray*}
\left\{N\left(t\right)\geq n\right\}&=&\left\{T_{n}\leq t\right\}\\
P\left\{N\left(t\right)\leq n\right\}&=&1-F^{\left(n+1\right)\star}\left(t\right)
\end{eqnarray*}

Adem\'as usando el hecho de que $\esp\left[N\left(t\right)\right]=\sum_{n=1}^{\infty}P\left\{N\left(t\right)\geq n\right\}$
se tiene que

\begin{eqnarray*}
\esp\left[N\left(t\right)\right]=\sum_{n=1}^{\infty}F^{n\star}\left(t\right)
\end{eqnarray*}

\begin{Prop}
Para cada $t\geq0$, la funci\'on generadora de momentos $\esp\left[e^{\alpha N\left(t\right)}\right]$ existe para alguna $\alpha$ en una vecindad del 0, y de aqu\'i que $\esp\left[N\left(t\right)^{m}\right]<\infty$, para $m\geq1$.
\end{Prop}


\begin{Note}
Si el primer tiempo de renovaci\'on $\xi_{1}$ no tiene la misma distribuci\'on que el resto de las $\xi_{n}$, para $n\geq2$, a $N\left(t\right)$ se le llama Proceso de Renovaci\'on retardado, donde si $\xi$ tiene distribuci\'on $G$, entonces el tiempo $T_{n}$ de la $n$-\'esima renovaci\'on tiene distribuci\'on $G\star F^{\left(n-1\right)\star}\left(t\right)$
\end{Note}


\begin{Teo}
Para una constante $\mu\leq\infty$ ( o variable aleatoria), las siguientes expresiones son equivalentes:

\begin{eqnarray}
lim_{n\rightarrow\infty}n^{-1}T_{n}&=&\mu,\textrm{ c.s.}\\
lim_{t\rightarrow\infty}t^{-1}N\left(t\right)&=&1/\mu,\textrm{ c.s.}
\end{eqnarray}
\end{Teo}


Es decir, $T_{n}$ satisface la Ley Fuerte de los Grandes N\'umeros s\'i y s\'olo s\'i $N\left/t\right)$ la cumple.


\begin{Coro}[Ley Fuerte de los Grandes N\'umeros para Procesos de Renovaci\'on]
Si $N\left(t\right)$ es un proceso de renovaci\'on cuyos tiempos de inter-renovaci\'on tienen media $\mu\leq\infty$, entonces
\begin{eqnarray}
t^{-1}N\left(t\right)\rightarrow 1/\mu,\textrm{ c.s. cuando }t\rightarrow\infty.
\end{eqnarray}

\end{Coro}


Considerar el proceso estoc\'astico de valores reales $\left\{Z\left(t\right):t\geq0\right\}$ en el mismo espacio de probabilidad que $N\left(t\right)$

\begin{Def}
Para el proceso $\left\{Z\left(t\right):t\geq0\right\}$ se define la fluctuaci\'on m\'axima de $Z\left(t\right)$ en el intervalo $\left(T_{n-1},T_{n}\right]$:
\begin{eqnarray*}
M_{n}=\sup_{T_{n-1}<t\leq T_{n}}|Z\left(t\right)-Z\left(T_{n-1}\right)|
\end{eqnarray*}
\end{Def}

\begin{Teo}
Sup\'ongase que $n^{-1}T_{n}\rightarrow\mu$ c.s. cuando $n\rightarrow\infty$, donde $\mu\leq\infty$ es una constante o variable aleatoria. Sea $a$ una constante o variable aleatoria que puede ser infinita cuando $\mu$ es finita, y considere las expresiones l\'imite:
\begin{eqnarray}
lim_{n\rightarrow\infty}n^{-1}Z\left(T_{n}\right)&=&a,\textrm{ c.s.}\\
lim_{t\rightarrow\infty}t^{-1}Z\left(t\right)&=&a/\mu,\textrm{ c.s.}
\end{eqnarray}
La segunda expresi\'on implica la primera. Conversamente, la primera implica la segunda si el proceso $Z\left(t\right)$ es creciente, o si $lim_{n\rightarrow\infty}n^{-1}M_{n}=0$ c.s.
\end{Teo}

\begin{Coro}
Si $N\left(t\right)$ es un proceso de renovaci\'on, y $\left(Z\left(T_{n}\right)-Z\left(T_{n-1}\right),M_{n}\right)$, para $n\geq1$, son variables aleatorias independientes e id\'enticamente distribuidas con media finita, entonces,
\begin{eqnarray}
lim_{t\rightarrow\infty}t^{-1}Z\left(t\right)\rightarrow\frac{\esp\left[Z\left(T_{1}\right)-Z\left(T_{0}\right)\right]}{\esp\left[T_{1}\right]},\textrm{ c.s. cuando  }t\rightarrow\infty.
\end{eqnarray}
\end{Coro}



%___________________________________________________________________________________________
%
\subsection{Propiedades de los Procesos de Renovaci\'on}
%___________________________________________________________________________________________
%

Los tiempos $T_{n}$ est\'an relacionados con los conteos de $N\left(t\right)$ por

\begin{eqnarray*}
\left\{N\left(t\right)\geq n\right\}&=&\left\{T_{n}\leq t\right\}\\
T_{N\left(t\right)}\leq &t&<T_{N\left(t\right)+1},
\end{eqnarray*}

adem\'as $N\left(T_{n}\right)=n$, y 

\begin{eqnarray*}
N\left(t\right)=\max\left\{n:T_{n}\leq t\right\}=\min\left\{n:T_{n+1}>t\right\}
\end{eqnarray*}

Por propiedades de la convoluci\'on se sabe que

\begin{eqnarray*}
P\left\{T_{n}\leq t\right\}=F^{n\star}\left(t\right)
\end{eqnarray*}
que es la $n$-\'esima convoluci\'on de $F$. Entonces 

\begin{eqnarray*}
\left\{N\left(t\right)\geq n\right\}&=&\left\{T_{n}\leq t\right\}\\
P\left\{N\left(t\right)\leq n\right\}&=&1-F^{\left(n+1\right)\star}\left(t\right)
\end{eqnarray*}

Adem\'as usando el hecho de que $\esp\left[N\left(t\right)\right]=\sum_{n=1}^{\infty}P\left\{N\left(t\right)\geq n\right\}$
se tiene que

\begin{eqnarray*}
\esp\left[N\left(t\right)\right]=\sum_{n=1}^{\infty}F^{n\star}\left(t\right)
\end{eqnarray*}

\begin{Prop}
Para cada $t\geq0$, la funci\'on generadora de momentos $\esp\left[e^{\alpha N\left(t\right)}\right]$ existe para alguna $\alpha$ en una vecindad del 0, y de aqu\'i que $\esp\left[N\left(t\right)^{m}\right]<\infty$, para $m\geq1$.
\end{Prop}


\begin{Note}
Si el primer tiempo de renovaci\'on $\xi_{1}$ no tiene la misma distribuci\'on que el resto de las $\xi_{n}$, para $n\geq2$, a $N\left(t\right)$ se le llama Proceso de Renovaci\'on retardado, donde si $\xi$ tiene distribuci\'on $G$, entonces el tiempo $T_{n}$ de la $n$-\'esima renovaci\'on tiene distribuci\'on $G\star F^{\left(n-1\right)\star}\left(t\right)$
\end{Note}


\begin{Teo}
Para una constante $\mu\leq\infty$ ( o variable aleatoria), las siguientes expresiones son equivalentes:

\begin{eqnarray}
lim_{n\rightarrow\infty}n^{-1}T_{n}&=&\mu,\textrm{ c.s.}\\
lim_{t\rightarrow\infty}t^{-1}N\left(t\right)&=&1/\mu,\textrm{ c.s.}
\end{eqnarray}
\end{Teo}


Es decir, $T_{n}$ satisface la Ley Fuerte de los Grandes N\'umeros s\'i y s\'olo s\'i $N\left/t\right)$ la cumple.


\begin{Coro}[Ley Fuerte de los Grandes N\'umeros para Procesos de Renovaci\'on]
Si $N\left(t\right)$ es un proceso de renovaci\'on cuyos tiempos de inter-renovaci\'on tienen media $\mu\leq\infty$, entonces
\begin{eqnarray}
t^{-1}N\left(t\right)\rightarrow 1/\mu,\textrm{ c.s. cuando }t\rightarrow\infty.
\end{eqnarray}

\end{Coro}


Considerar el proceso estoc\'astico de valores reales $\left\{Z\left(t\right):t\geq0\right\}$ en el mismo espacio de probabilidad que $N\left(t\right)$

\begin{Def}
Para el proceso $\left\{Z\left(t\right):t\geq0\right\}$ se define la fluctuaci\'on m\'axima de $Z\left(t\right)$ en el intervalo $\left(T_{n-1},T_{n}\right]$:
\begin{eqnarray*}
M_{n}=\sup_{T_{n-1}<t\leq T_{n}}|Z\left(t\right)-Z\left(T_{n-1}\right)|
\end{eqnarray*}
\end{Def}

\begin{Teo}
Sup\'ongase que $n^{-1}T_{n}\rightarrow\mu$ c.s. cuando $n\rightarrow\infty$, donde $\mu\leq\infty$ es una constante o variable aleatoria. Sea $a$ una constante o variable aleatoria que puede ser infinita cuando $\mu$ es finita, y considere las expresiones l\'imite:
\begin{eqnarray}
lim_{n\rightarrow\infty}n^{-1}Z\left(T_{n}\right)&=&a,\textrm{ c.s.}\\
lim_{t\rightarrow\infty}t^{-1}Z\left(t\right)&=&a/\mu,\textrm{ c.s.}
\end{eqnarray}
La segunda expresi\'on implica la primera. Conversamente, la primera implica la segunda si el proceso $Z\left(t\right)$ es creciente, o si $lim_{n\rightarrow\infty}n^{-1}M_{n}=0$ c.s.
\end{Teo}

\begin{Coro}
Si $N\left(t\right)$ es un proceso de renovaci\'on, y $\left(Z\left(T_{n}\right)-Z\left(T_{n-1}\right),M_{n}\right)$, para $n\geq1$, son variables aleatorias independientes e id\'enticamente distribuidas con media finita, entonces,
\begin{eqnarray}
lim_{t\rightarrow\infty}t^{-1}Z\left(t\right)\rightarrow\frac{\esp\left[Z\left(T_{1}\right)-Z\left(T_{0}\right)\right]}{\esp\left[T_{1}\right]},\textrm{ c.s. cuando  }t\rightarrow\infty.
\end{eqnarray}
\end{Coro}


%___________________________________________________________________________________________
%
\subsection{Propiedades de los Procesos de Renovaci\'on}
%___________________________________________________________________________________________
%

Los tiempos $T_{n}$ est\'an relacionados con los conteos de $N\left(t\right)$ por

\begin{eqnarray*}
\left\{N\left(t\right)\geq n\right\}&=&\left\{T_{n}\leq t\right\}\\
T_{N\left(t\right)}\leq &t&<T_{N\left(t\right)+1},
\end{eqnarray*}

adem\'as $N\left(T_{n}\right)=n$, y 

\begin{eqnarray*}
N\left(t\right)=\max\left\{n:T_{n}\leq t\right\}=\min\left\{n:T_{n+1}>t\right\}
\end{eqnarray*}

Por propiedades de la convoluci\'on se sabe que

\begin{eqnarray*}
P\left\{T_{n}\leq t\right\}=F^{n\star}\left(t\right)
\end{eqnarray*}
que es la $n$-\'esima convoluci\'on de $F$. Entonces 

\begin{eqnarray*}
\left\{N\left(t\right)\geq n\right\}&=&\left\{T_{n}\leq t\right\}\\
P\left\{N\left(t\right)\leq n\right\}&=&1-F^{\left(n+1\right)\star}\left(t\right)
\end{eqnarray*}

Adem\'as usando el hecho de que $\esp\left[N\left(t\right)\right]=\sum_{n=1}^{\infty}P\left\{N\left(t\right)\geq n\right\}$
se tiene que

\begin{eqnarray*}
\esp\left[N\left(t\right)\right]=\sum_{n=1}^{\infty}F^{n\star}\left(t\right)
\end{eqnarray*}

\begin{Prop}
Para cada $t\geq0$, la funci\'on generadora de momentos $\esp\left[e^{\alpha N\left(t\right)}\right]$ existe para alguna $\alpha$ en una vecindad del 0, y de aqu\'i que $\esp\left[N\left(t\right)^{m}\right]<\infty$, para $m\geq1$.
\end{Prop}


\begin{Note}
Si el primer tiempo de renovaci\'on $\xi_{1}$ no tiene la misma distribuci\'on que el resto de las $\xi_{n}$, para $n\geq2$, a $N\left(t\right)$ se le llama Proceso de Renovaci\'on retardado, donde si $\xi$ tiene distribuci\'on $G$, entonces el tiempo $T_{n}$ de la $n$-\'esima renovaci\'on tiene distribuci\'on $G\star F^{\left(n-1\right)\star}\left(t\right)$
\end{Note}


\begin{Teo}
Para una constante $\mu\leq\infty$ ( o variable aleatoria), las siguientes expresiones son equivalentes:

\begin{eqnarray}
lim_{n\rightarrow\infty}n^{-1}T_{n}&=&\mu,\textrm{ c.s.}\\
lim_{t\rightarrow\infty}t^{-1}N\left(t\right)&=&1/\mu,\textrm{ c.s.}
\end{eqnarray}
\end{Teo}


Es decir, $T_{n}$ satisface la Ley Fuerte de los Grandes N\'umeros s\'i y s\'olo s\'i $N\left/t\right)$ la cumple.


\begin{Coro}[Ley Fuerte de los Grandes N\'umeros para Procesos de Renovaci\'on]
Si $N\left(t\right)$ es un proceso de renovaci\'on cuyos tiempos de inter-renovaci\'on tienen media $\mu\leq\infty$, entonces
\begin{eqnarray}
t^{-1}N\left(t\right)\rightarrow 1/\mu,\textrm{ c.s. cuando }t\rightarrow\infty.
\end{eqnarray}

\end{Coro}


Considerar el proceso estoc\'astico de valores reales $\left\{Z\left(t\right):t\geq0\right\}$ en el mismo espacio de probabilidad que $N\left(t\right)$

\begin{Def}
Para el proceso $\left\{Z\left(t\right):t\geq0\right\}$ se define la fluctuaci\'on m\'axima de $Z\left(t\right)$ en el intervalo $\left(T_{n-1},T_{n}\right]$:
\begin{eqnarray*}
M_{n}=\sup_{T_{n-1}<t\leq T_{n}}|Z\left(t\right)-Z\left(T_{n-1}\right)|
\end{eqnarray*}
\end{Def}

\begin{Teo}
Sup\'ongase que $n^{-1}T_{n}\rightarrow\mu$ c.s. cuando $n\rightarrow\infty$, donde $\mu\leq\infty$ es una constante o variable aleatoria. Sea $a$ una constante o variable aleatoria que puede ser infinita cuando $\mu$ es finita, y considere las expresiones l\'imite:
\begin{eqnarray}
lim_{n\rightarrow\infty}n^{-1}Z\left(T_{n}\right)&=&a,\textrm{ c.s.}\\
lim_{t\rightarrow\infty}t^{-1}Z\left(t\right)&=&a/\mu,\textrm{ c.s.}
\end{eqnarray}
La segunda expresi\'on implica la primera. Conversamente, la primera implica la segunda si el proceso $Z\left(t\right)$ es creciente, o si $lim_{n\rightarrow\infty}n^{-1}M_{n}=0$ c.s.
\end{Teo}

\begin{Coro}
Si $N\left(t\right)$ es un proceso de renovaci\'on, y $\left(Z\left(T_{n}\right)-Z\left(T_{n-1}\right),M_{n}\right)$, para $n\geq1$, son variables aleatorias independientes e id\'enticamente distribuidas con media finita, entonces,
\begin{eqnarray}
lim_{t\rightarrow\infty}t^{-1}Z\left(t\right)\rightarrow\frac{\esp\left[Z\left(T_{1}\right)-Z\left(T_{0}\right)\right]}{\esp\left[T_{1}\right]},\textrm{ c.s. cuando  }t\rightarrow\infty.
\end{eqnarray}
\end{Coro}

%___________________________________________________________________________________________
%
\subsection{Propiedades de los Procesos de Renovaci\'on}
%___________________________________________________________________________________________
%

Los tiempos $T_{n}$ est\'an relacionados con los conteos de $N\left(t\right)$ por

\begin{eqnarray*}
\left\{N\left(t\right)\geq n\right\}&=&\left\{T_{n}\leq t\right\}\\
T_{N\left(t\right)}\leq &t&<T_{N\left(t\right)+1},
\end{eqnarray*}

adem\'as $N\left(T_{n}\right)=n$, y 

\begin{eqnarray*}
N\left(t\right)=\max\left\{n:T_{n}\leq t\right\}=\min\left\{n:T_{n+1}>t\right\}
\end{eqnarray*}

Por propiedades de la convoluci\'on se sabe que

\begin{eqnarray*}
P\left\{T_{n}\leq t\right\}=F^{n\star}\left(t\right)
\end{eqnarray*}
que es la $n$-\'esima convoluci\'on de $F$. Entonces 

\begin{eqnarray*}
\left\{N\left(t\right)\geq n\right\}&=&\left\{T_{n}\leq t\right\}\\
P\left\{N\left(t\right)\leq n\right\}&=&1-F^{\left(n+1\right)\star}\left(t\right)
\end{eqnarray*}

Adem\'as usando el hecho de que $\esp\left[N\left(t\right)\right]=\sum_{n=1}^{\infty}P\left\{N\left(t\right)\geq n\right\}$
se tiene que

\begin{eqnarray*}
\esp\left[N\left(t\right)\right]=\sum_{n=1}^{\infty}F^{n\star}\left(t\right)
\end{eqnarray*}

\begin{Prop}
Para cada $t\geq0$, la funci\'on generadora de momentos $\esp\left[e^{\alpha N\left(t\right)}\right]$ existe para alguna $\alpha$ en una vecindad del 0, y de aqu\'i que $\esp\left[N\left(t\right)^{m}\right]<\infty$, para $m\geq1$.
\end{Prop}


\begin{Note}
Si el primer tiempo de renovaci\'on $\xi_{1}$ no tiene la misma distribuci\'on que el resto de las $\xi_{n}$, para $n\geq2$, a $N\left(t\right)$ se le llama Proceso de Renovaci\'on retardado, donde si $\xi$ tiene distribuci\'on $G$, entonces el tiempo $T_{n}$ de la $n$-\'esima renovaci\'on tiene distribuci\'on $G\star F^{\left(n-1\right)\star}\left(t\right)$
\end{Note}


\begin{Teo}
Para una constante $\mu\leq\infty$ ( o variable aleatoria), las siguientes expresiones son equivalentes:

\begin{eqnarray}
lim_{n\rightarrow\infty}n^{-1}T_{n}&=&\mu,\textrm{ c.s.}\\
lim_{t\rightarrow\infty}t^{-1}N\left(t\right)&=&1/\mu,\textrm{ c.s.}
\end{eqnarray}
\end{Teo}


Es decir, $T_{n}$ satisface la Ley Fuerte de los Grandes N\'umeros s\'i y s\'olo s\'i $N\left/t\right)$ la cumple.


\begin{Coro}[Ley Fuerte de los Grandes N\'umeros para Procesos de Renovaci\'on]
Si $N\left(t\right)$ es un proceso de renovaci\'on cuyos tiempos de inter-renovaci\'on tienen media $\mu\leq\infty$, entonces
\begin{eqnarray}
t^{-1}N\left(t\right)\rightarrow 1/\mu,\textrm{ c.s. cuando }t\rightarrow\infty.
\end{eqnarray}

\end{Coro}


Considerar el proceso estoc\'astico de valores reales $\left\{Z\left(t\right):t\geq0\right\}$ en el mismo espacio de probabilidad que $N\left(t\right)$

\begin{Def}
Para el proceso $\left\{Z\left(t\right):t\geq0\right\}$ se define la fluctuaci\'on m\'axima de $Z\left(t\right)$ en el intervalo $\left(T_{n-1},T_{n}\right]$:
\begin{eqnarray*}
M_{n}=\sup_{T_{n-1}<t\leq T_{n}}|Z\left(t\right)-Z\left(T_{n-1}\right)|
\end{eqnarray*}
\end{Def}

\begin{Teo}
Sup\'ongase que $n^{-1}T_{n}\rightarrow\mu$ c.s. cuando $n\rightarrow\infty$, donde $\mu\leq\infty$ es una constante o variable aleatoria. Sea $a$ una constante o variable aleatoria que puede ser infinita cuando $\mu$ es finita, y considere las expresiones l\'imite:
\begin{eqnarray}
lim_{n\rightarrow\infty}n^{-1}Z\left(T_{n}\right)&=&a,\textrm{ c.s.}\\
lim_{t\rightarrow\infty}t^{-1}Z\left(t\right)&=&a/\mu,\textrm{ c.s.}
\end{eqnarray}
La segunda expresi\'on implica la primera. Conversamente, la primera implica la segunda si el proceso $Z\left(t\right)$ es creciente, o si $lim_{n\rightarrow\infty}n^{-1}M_{n}=0$ c.s.
\end{Teo}

\begin{Coro}
Si $N\left(t\right)$ es un proceso de renovaci\'on, y $\left(Z\left(T_{n}\right)-Z\left(T_{n-1}\right),M_{n}\right)$, para $n\geq1$, son variables aleatorias independientes e id\'enticamente distribuidas con media finita, entonces,
\begin{eqnarray}
lim_{t\rightarrow\infty}t^{-1}Z\left(t\right)\rightarrow\frac{\esp\left[Z\left(T_{1}\right)-Z\left(T_{0}\right)\right]}{\esp\left[T_{1}\right]},\textrm{ c.s. cuando  }t\rightarrow\infty.
\end{eqnarray}
\end{Coro}
%___________________________________________________________________________________________
%
\subsection{Propiedades de los Procesos de Renovaci\'on}
%___________________________________________________________________________________________
%

Los tiempos $T_{n}$ est\'an relacionados con los conteos de $N\left(t\right)$ por

\begin{eqnarray*}
\left\{N\left(t\right)\geq n\right\}&=&\left\{T_{n}\leq t\right\}\\
T_{N\left(t\right)}\leq &t&<T_{N\left(t\right)+1},
\end{eqnarray*}

adem\'as $N\left(T_{n}\right)=n$, y 

\begin{eqnarray*}
N\left(t\right)=\max\left\{n:T_{n}\leq t\right\}=\min\left\{n:T_{n+1}>t\right\}
\end{eqnarray*}

Por propiedades de la convoluci\'on se sabe que

\begin{eqnarray*}
P\left\{T_{n}\leq t\right\}=F^{n\star}\left(t\right)
\end{eqnarray*}
que es la $n$-\'esima convoluci\'on de $F$. Entonces 

\begin{eqnarray*}
\left\{N\left(t\right)\geq n\right\}&=&\left\{T_{n}\leq t\right\}\\
P\left\{N\left(t\right)\leq n\right\}&=&1-F^{\left(n+1\right)\star}\left(t\right)
\end{eqnarray*}

Adem\'as usando el hecho de que $\esp\left[N\left(t\right)\right]=\sum_{n=1}^{\infty}P\left\{N\left(t\right)\geq n\right\}$
se tiene que

\begin{eqnarray*}
\esp\left[N\left(t\right)\right]=\sum_{n=1}^{\infty}F^{n\star}\left(t\right)
\end{eqnarray*}

\begin{Prop}
Para cada $t\geq0$, la funci\'on generadora de momentos $\esp\left[e^{\alpha N\left(t\right)}\right]$ existe para alguna $\alpha$ en una vecindad del 0, y de aqu\'i que $\esp\left[N\left(t\right)^{m}\right]<\infty$, para $m\geq1$.
\end{Prop}


\begin{Note}
Si el primer tiempo de renovaci\'on $\xi_{1}$ no tiene la misma distribuci\'on que el resto de las $\xi_{n}$, para $n\geq2$, a $N\left(t\right)$ se le llama Proceso de Renovaci\'on retardado, donde si $\xi$ tiene distribuci\'on $G$, entonces el tiempo $T_{n}$ de la $n$-\'esima renovaci\'on tiene distribuci\'on $G\star F^{\left(n-1\right)\star}\left(t\right)$
\end{Note}


\begin{Teo}
Para una constante $\mu\leq\infty$ ( o variable aleatoria), las siguientes expresiones son equivalentes:

\begin{eqnarray}
lim_{n\rightarrow\infty}n^{-1}T_{n}&=&\mu,\textrm{ c.s.}\\
lim_{t\rightarrow\infty}t^{-1}N\left(t\right)&=&1/\mu,\textrm{ c.s.}
\end{eqnarray}
\end{Teo}


Es decir, $T_{n}$ satisface la Ley Fuerte de los Grandes N\'umeros s\'i y s\'olo s\'i $N\left/t\right)$ la cumple.


\begin{Coro}[Ley Fuerte de los Grandes N\'umeros para Procesos de Renovaci\'on]
Si $N\left(t\right)$ es un proceso de renovaci\'on cuyos tiempos de inter-renovaci\'on tienen media $\mu\leq\infty$, entonces
\begin{eqnarray}
t^{-1}N\left(t\right)\rightarrow 1/\mu,\textrm{ c.s. cuando }t\rightarrow\infty.
\end{eqnarray}

\end{Coro}


Considerar el proceso estoc\'astico de valores reales $\left\{Z\left(t\right):t\geq0\right\}$ en el mismo espacio de probabilidad que $N\left(t\right)$

\begin{Def}
Para el proceso $\left\{Z\left(t\right):t\geq0\right\}$ se define la fluctuaci\'on m\'axima de $Z\left(t\right)$ en el intervalo $\left(T_{n-1},T_{n}\right]$:
\begin{eqnarray*}
M_{n}=\sup_{T_{n-1}<t\leq T_{n}}|Z\left(t\right)-Z\left(T_{n-1}\right)|
\end{eqnarray*}
\end{Def}

\begin{Teo}
Sup\'ongase que $n^{-1}T_{n}\rightarrow\mu$ c.s. cuando $n\rightarrow\infty$, donde $\mu\leq\infty$ es una constante o variable aleatoria. Sea $a$ una constante o variable aleatoria que puede ser infinita cuando $\mu$ es finita, y considere las expresiones l\'imite:
\begin{eqnarray}
lim_{n\rightarrow\infty}n^{-1}Z\left(T_{n}\right)&=&a,\textrm{ c.s.}\\
lim_{t\rightarrow\infty}t^{-1}Z\left(t\right)&=&a/\mu,\textrm{ c.s.}
\end{eqnarray}
La segunda expresi\'on implica la primera. Conversamente, la primera implica la segunda si el proceso $Z\left(t\right)$ es creciente, o si $lim_{n\rightarrow\infty}n^{-1}M_{n}=0$ c.s.
\end{Teo}

\begin{Coro}
Si $N\left(t\right)$ es un proceso de renovaci\'on, y $\left(Z\left(T_{n}\right)-Z\left(T_{n-1}\right),M_{n}\right)$, para $n\geq1$, son variables aleatorias independientes e id\'enticamente distribuidas con media finita, entonces,
\begin{eqnarray}
lim_{t\rightarrow\infty}t^{-1}Z\left(t\right)\rightarrow\frac{\esp\left[Z\left(T_{1}\right)-Z\left(T_{0}\right)\right]}{\esp\left[T_{1}\right]},\textrm{ c.s. cuando  }t\rightarrow\infty.
\end{eqnarray}
\end{Coro}


%___________________________________________________________________________________________
%
\subsection{Funci\'on de Renovaci\'on}
%___________________________________________________________________________________________
%


\begin{Def}
Sea $h\left(t\right)$ funci\'on de valores reales en $\rea$ acotada en intervalos finitos e igual a cero para $t<0$ La ecuaci\'on de renovaci\'on para $h\left(t\right)$ y la distribuci\'on $F$ es

\begin{eqnarray}\label{Ec.Renovacion}
H\left(t\right)=h\left(t\right)+\int_{\left[0,t\right]}H\left(t-s\right)dF\left(s\right)\textrm{,    }t\geq0,
\end{eqnarray}
donde $H\left(t\right)$ es una funci\'on de valores reales. Esto es $H=h+F\star H$. Decimos que $H\left(t\right)$ es soluci\'on de esta ecuaci\'on si satisface la ecuaci\'on, y es acotada en intervalos finitos e iguales a cero para $t<0$.
\end{Def}

\begin{Prop}
La funci\'on $U\star h\left(t\right)$ es la \'unica soluci\'on de la ecuaci\'on de renovaci\'on (\ref{Ec.Renovacion}).
\end{Prop}

\begin{Teo}[Teorema Renovaci\'on Elemental]
\begin{eqnarray*}
t^{-1}U\left(t\right)\rightarrow 1/\mu\textrm{,    cuando }t\rightarrow\infty.
\end{eqnarray*}
\end{Teo}

%___________________________________________________________________________________________
%
\subsection{Funci\'on de Renovaci\'on}
%___________________________________________________________________________________________
%


Sup\'ongase que $N\left(t\right)$ es un proceso de renovaci\'on con distribuci\'on $F$ con media finita $\mu$.

\begin{Def}
La funci\'on de renovaci\'on asociada con la distribuci\'on $F$, del proceso $N\left(t\right)$, es
\begin{eqnarray*}
U\left(t\right)=\sum_{n=1}^{\infty}F^{n\star}\left(t\right),\textrm{   }t\geq0,
\end{eqnarray*}
donde $F^{0\star}\left(t\right)=\indora\left(t\geq0\right)$.
\end{Def}


\begin{Prop}
Sup\'ongase que la distribuci\'on de inter-renovaci\'on $F$ tiene densidad $f$. Entonces $U\left(t\right)$ tambi\'en tiene densidad, para $t>0$, y es $U^{'}\left(t\right)=\sum_{n=0}^{\infty}f^{n\star}\left(t\right)$. Adem\'as
\begin{eqnarray*}
\prob\left\{N\left(t\right)>N\left(t-\right)\right\}=0\textrm{,   }t\geq0.
\end{eqnarray*}
\end{Prop}

\begin{Def}
La Transformada de Laplace-Stieljes de $F$ est\'a dada por

\begin{eqnarray*}
\hat{F}\left(\alpha\right)=\int_{\rea_{+}}e^{-\alpha t}dF\left(t\right)\textrm{,  }\alpha\geq0.
\end{eqnarray*}
\end{Def}

Entonces

\begin{eqnarray*}
\hat{U}\left(\alpha\right)=\sum_{n=0}^{\infty}\hat{F^{n\star}}\left(\alpha\right)=\sum_{n=0}^{\infty}\hat{F}\left(\alpha\right)^{n}=\frac{1}{1-\hat{F}\left(\alpha\right)}.
\end{eqnarray*}


\begin{Prop}
La Transformada de Laplace $\hat{U}\left(\alpha\right)$ y $\hat{F}\left(\alpha\right)$ determina una a la otra de manera \'unica por la relaci\'on $\hat{U}\left(\alpha\right)=\frac{1}{1-\hat{F}\left(\alpha\right)}$.
\end{Prop}


\begin{Note}
Un proceso de renovaci\'on $N\left(t\right)$ cuyos tiempos de inter-renovaci\'on tienen media finita, es un proceso Poisson con tasa $\lambda$ si y s\'olo s\'i $\esp\left[U\left(t\right)\right]=\lambda t$, para $t\geq0$.
\end{Note}


\begin{Teo}
Sea $N\left(t\right)$ un proceso puntual simple con puntos de localizaci\'on $T_{n}$ tal que $\eta\left(t\right)=\esp\left[N\left(\right)\right]$ es finita para cada $t$. Entonces para cualquier funci\'on $f:\rea_{+}\rightarrow\rea$,
\begin{eqnarray*}
\esp\left[\sum_{n=1}^{N\left(\right)}f\left(T_{n}\right)\right]=\int_{\left(0,t\right]}f\left(s\right)d\eta\left(s\right)\textrm{,  }t\geq0,
\end{eqnarray*}
suponiendo que la integral exista. Adem\'as si $X_{1},X_{2},\ldots$ son variables aleatorias definidas en el mismo espacio de probabilidad que el proceso $N\left(t\right)$ tal que $\esp\left[X_{n}|T_{n}=s\right]=f\left(s\right)$, independiente de $n$. Entonces
\begin{eqnarray*}
\esp\left[\sum_{n=1}^{N\left(t\right)}X_{n}\right]=\int_{\left(0,t\right]}f\left(s\right)d\eta\left(s\right)\textrm{,  }t\geq0,
\end{eqnarray*} 
suponiendo que la integral exista. 
\end{Teo}

\begin{Coro}[Identidad de Wald para Renovaciones]
Para el proceso de renovaci\'on $N\left(t\right)$,
\begin{eqnarray*}
\esp\left[T_{N\left(t\right)+1}\right]=\mu\esp\left[N\left(t\right)+1\right]\textrm{,  }t\geq0,
\end{eqnarray*}  
\end{Coro}

%______________________________________________________________________
\subsection{Procesos de Renovaci\'on}
%______________________________________________________________________

\begin{Def}\label{Def.Tn}
Sean $0\leq T_{1}\leq T_{2}\leq \ldots$ son tiempos aleatorios infinitos en los cuales ocurren ciertos eventos. El n\'umero de tiempos $T_{n}$ en el intervalo $\left[0,t\right)$ es

\begin{eqnarray}
N\left(t\right)=\sum_{n=1}^{\infty}\indora\left(T_{n}\leq t\right),
\end{eqnarray}
para $t\geq0$.
\end{Def}

Si se consideran los puntos $T_{n}$ como elementos de $\rea_{+}$, y $N\left(t\right)$ es el n\'umero de puntos en $\rea$. El proceso denotado por $\left\{N\left(t\right):t\geq0\right\}$, denotado por $N\left(t\right)$, es un proceso puntual en $\rea_{+}$. Los $T_{n}$ son los tiempos de ocurrencia, el proceso puntual $N\left(t\right)$ es simple si su n\'umero de ocurrencias son distintas: $0<T_{1}<T_{2}<\ldots$ casi seguramente.

\begin{Def}
Un proceso puntual $N\left(t\right)$ es un proceso de renovaci\'on si los tiempos de interocurrencia $\xi_{n}=T_{n}-T_{n-1}$, para $n\geq1$, son independientes e identicamente distribuidos con distribuci\'on $F$, donde $F\left(0\right)=0$ y $T_{0}=0$. Los $T_{n}$ son llamados tiempos de renovaci\'on, referente a la independencia o renovaci\'on de la informaci\'on estoc\'astica en estos tiempos. Los $\xi_{n}$ son los tiempos de inter-renovaci\'on, y $N\left(t\right)$ es el n\'umero de renovaciones en el intervalo $\left[0,t\right)$
\end{Def}


\begin{Note}
Para definir un proceso de renovaci\'on para cualquier contexto, solamente hay que especificar una distribuci\'on $F$, con $F\left(0\right)=0$, para los tiempos de inter-renovaci\'on. La funci\'on $F$ en turno degune las otra variables aleatorias. De manera formal, existe un espacio de probabilidad y una sucesi\'on de variables aleatorias $\xi_{1},\xi_{2},\ldots$ definidas en este con distribuci\'on $F$. Entonces las otras cantidades son $T_{n}=\sum_{k=1}^{n}\xi_{k}$ y $N\left(t\right)=\sum_{n=1}^{\infty}\indora\left(T_{n}\leq t\right)$, donde $T_{n}\rightarrow\infty$ casi seguramente por la Ley Fuerte de los Grandes Números.
\end{Note}

\begin{Def}\label{Def.Tn}
Sean $0\leq T_{1}\leq T_{2}\leq \ldots$ son tiempos aleatorios infinitos en los cuales ocurren ciertos eventos. El n\'umero de tiempos $T_{n}$ en el intervalo $\left[0,t\right)$ es

\begin{eqnarray}
N\left(t\right)=\sum_{n=1}^{\infty}\indora\left(T_{n}\leq t\right),
\end{eqnarray}
para $t\geq0$.
\end{Def}

Si se consideran los puntos $T_{n}$ como elementos de $\rea_{+}$, y $N\left(t\right)$ es el n\'umero de puntos en $\rea$. El proceso denotado por $\left\{N\left(t\right):t\geq0\right\}$, denotado por $N\left(t\right)$, es un proceso puntual en $\rea_{+}$. Los $T_{n}$ son los tiempos de ocurrencia, el proceso puntual $N\left(t\right)$ es simple si su n\'umero de ocurrencias son distintas: $0<T_{1}<T_{2}<\ldots$ casi seguramente.

\begin{Def}
Un proceso puntual $N\left(t\right)$ es un proceso de renovaci\'on si los tiempos de interocurrencia $\xi_{n}=T_{n}-T_{n-1}$, para $n\geq1$, son independientes e identicamente distribuidos con distribuci\'on $F$, donde $F\left(0\right)=0$ y $T_{0}=0$. Los $T_{n}$ son llamados tiempos de renovaci\'on, referente a la independencia o renovaci\'on de la informaci\'on estoc\'astica en estos tiempos. Los $\xi_{n}$ son los tiempos de inter-renovaci\'on, y $N\left(t\right)$ es el n\'umero de renovaciones en el intervalo $\left[0,t\right)$
\end{Def}


\begin{Note}
Para definir un proceso de renovaci\'on para cualquier contexto, solamente hay que especificar una distribuci\'on $F$, con $F\left(0\right)=0$, para los tiempos de inter-renovaci\'on. La funci\'on $F$ en turno degune las otra variables aleatorias. De manera formal, existe un espacio de probabilidad y una sucesi\'on de variables aleatorias $\xi_{1},\xi_{2},\ldots$ definidas en este con distribuci\'on $F$. Entonces las otras cantidades son $T_{n}=\sum_{k=1}^{n}\xi_{k}$ y $N\left(t\right)=\sum_{n=1}^{\infty}\indora\left(T_{n}\leq t\right)$, donde $T_{n}\rightarrow\infty$ casi seguramente por la Ley Fuerte de los Grandes N\'umeros.
\end{Note}







Los tiempos $T_{n}$ est\'an relacionados con los conteos de $N\left(t\right)$ por

\begin{eqnarray*}
\left\{N\left(t\right)\geq n\right\}&=&\left\{T_{n}\leq t\right\}\\
T_{N\left(t\right)}\leq &t&<T_{N\left(t\right)+1},
\end{eqnarray*}

adem\'as $N\left(T_{n}\right)=n$, y 

\begin{eqnarray*}
N\left(t\right)=\max\left\{n:T_{n}\leq t\right\}=\min\left\{n:T_{n+1}>t\right\}
\end{eqnarray*}

Por propiedades de la convoluci\'on se sabe que

\begin{eqnarray*}
P\left\{T_{n}\leq t\right\}=F^{n\star}\left(t\right)
\end{eqnarray*}
que es la $n$-\'esima convoluci\'on de $F$. Entonces 

\begin{eqnarray*}
\left\{N\left(t\right)\geq n\right\}&=&\left\{T_{n}\leq t\right\}\\
P\left\{N\left(t\right)\leq n\right\}&=&1-F^{\left(n+1\right)\star}\left(t\right)
\end{eqnarray*}

Adem\'as usando el hecho de que $\esp\left[N\left(t\right)\right]=\sum_{n=1}^{\infty}P\left\{N\left(t\right)\geq n\right\}$
se tiene que

\begin{eqnarray*}
\esp\left[N\left(t\right)\right]=\sum_{n=1}^{\infty}F^{n\star}\left(t\right)
\end{eqnarray*}

\begin{Prop}
Para cada $t\geq0$, la funci\'on generadora de momentos $\esp\left[e^{\alpha N\left(t\right)}\right]$ existe para alguna $\alpha$ en una vecindad del 0, y de aqu\'i que $\esp\left[N\left(t\right)^{m}\right]<\infty$, para $m\geq1$.
\end{Prop}

\begin{Ejem}[\textbf{Proceso Poisson}]

Suponga que se tienen tiempos de inter-renovaci\'on \textit{i.i.d.} del proceso de renovaci\'on $N\left(t\right)$ tienen distribuci\'on exponencial $F\left(t\right)=q-e^{-\lambda t}$ con tasa $\lambda$. Entonces $N\left(t\right)$ es un proceso Poisson con tasa $\lambda$.

\end{Ejem}


\begin{Note}
Si el primer tiempo de renovaci\'on $\xi_{1}$ no tiene la misma distribuci\'on que el resto de las $\xi_{n}$, para $n\geq2$, a $N\left(t\right)$ se le llama Proceso de Renovaci\'on retardado, donde si $\xi$ tiene distribuci\'on $G$, entonces el tiempo $T_{n}$ de la $n$-\'esima renovaci\'on tiene distribuci\'on $G\star F^{\left(n-1\right)\star}\left(t\right)$
\end{Note}


\begin{Teo}
Para una constante $\mu\leq\infty$ ( o variable aleatoria), las siguientes expresiones son equivalentes:

\begin{eqnarray}
lim_{n\rightarrow\infty}n^{-1}T_{n}&=&\mu,\textrm{ c.s.}\\
lim_{t\rightarrow\infty}t^{-1}N\left(t\right)&=&1/\mu,\textrm{ c.s.}
\end{eqnarray}
\end{Teo}


Es decir, $T_{n}$ satisface la Ley Fuerte de los Grandes N\'umeros s\'i y s\'olo s\'i $N\left/t\right)$ la cumple.


\begin{Coro}[Ley Fuerte de los Grandes N\'umeros para Procesos de Renovaci\'on]
Si $N\left(t\right)$ es un proceso de renovaci\'on cuyos tiempos de inter-renovaci\'on tienen media $\mu\leq\infty$, entonces
\begin{eqnarray}
t^{-1}N\left(t\right)\rightarrow 1/\mu,\textrm{ c.s. cuando }t\rightarrow\infty.
\end{eqnarray}

\end{Coro}


Considerar el proceso estoc\'astico de valores reales $\left\{Z\left(t\right):t\geq0\right\}$ en el mismo espacio de probabilidad que $N\left(t\right)$

\begin{Def}
Para el proceso $\left\{Z\left(t\right):t\geq0\right\}$ se define la fluctuaci\'on m\'axima de $Z\left(t\right)$ en el intervalo $\left(T_{n-1},T_{n}\right]$:
\begin{eqnarray*}
M_{n}=\sup_{T_{n-1}<t\leq T_{n}}|Z\left(t\right)-Z\left(T_{n-1}\right)|
\end{eqnarray*}
\end{Def}

\begin{Teo}
Sup\'ongase que $n^{-1}T_{n}\rightarrow\mu$ c.s. cuando $n\rightarrow\infty$, donde $\mu\leq\infty$ es una constante o variable aleatoria. Sea $a$ una constante o variable aleatoria que puede ser infinita cuando $\mu$ es finita, y considere las expresiones l\'imite:
\begin{eqnarray}
lim_{n\rightarrow\infty}n^{-1}Z\left(T_{n}\right)&=&a,\textrm{ c.s.}\\
lim_{t\rightarrow\infty}t^{-1}Z\left(t\right)&=&a/\mu,\textrm{ c.s.}
\end{eqnarray}
La segunda expresi\'on implica la primera. Conversamente, la primera implica la segunda si el proceso $Z\left(t\right)$ es creciente, o si $lim_{n\rightarrow\infty}n^{-1}M_{n}=0$ c.s.
\end{Teo}

\begin{Coro}
Si $N\left(t\right)$ es un proceso de renovaci\'on, y $\left(Z\left(T_{n}\right)-Z\left(T_{n-1}\right),M_{n}\right)$, para $n\geq1$, son variables aleatorias independientes e id\'enticamente distribuidas con media finita, entonces,
\begin{eqnarray}
lim_{t\rightarrow\infty}t^{-1}Z\left(t\right)\rightarrow\frac{\esp\left[Z\left(T_{1}\right)-Z\left(T_{0}\right)\right]}{\esp\left[T_{1}\right]},\textrm{ c.s. cuando  }t\rightarrow\infty.
\end{eqnarray}
\end{Coro}


Sup\'ongase que $N\left(t\right)$ es un proceso de renovaci\'on con distribuci\'on $F$ con media finita $\mu$.

\begin{Def}
La funci\'on de renovaci\'on asociada con la distribuci\'on $F$, del proceso $N\left(t\right)$, es
\begin{eqnarray*}
U\left(t\right)=\sum_{n=1}^{\infty}F^{n\star}\left(t\right),\textrm{   }t\geq0,
\end{eqnarray*}
donde $F^{0\star}\left(t\right)=\indora\left(t\geq0\right)$.
\end{Def}


\begin{Prop}
Sup\'ongase que la distribuci\'on de inter-renovaci\'on $F$ tiene densidad $f$. Entonces $U\left(t\right)$ tambi\'en tiene densidad, para $t>0$, y es $U^{'}\left(t\right)=\sum_{n=0}^{\infty}f^{n\star}\left(t\right)$. Adem\'as
\begin{eqnarray*}
\prob\left\{N\left(t\right)>N\left(t-\right)\right\}=0\textrm{,   }t\geq0.
\end{eqnarray*}
\end{Prop}

\begin{Def}
La Transformada de Laplace-Stieljes de $F$ est\'a dada por

\begin{eqnarray*}
\hat{F}\left(\alpha\right)=\int_{\rea_{+}}e^{-\alpha t}dF\left(t\right)\textrm{,  }\alpha\geq0.
\end{eqnarray*}
\end{Def}

Entonces

\begin{eqnarray*}
\hat{U}\left(\alpha\right)=\sum_{n=0}^{\infty}\hat{F^{n\star}}\left(\alpha\right)=\sum_{n=0}^{\infty}\hat{F}\left(\alpha\right)^{n}=\frac{1}{1-\hat{F}\left(\alpha\right)}.
\end{eqnarray*}


\begin{Prop}
La Transformada de Laplace $\hat{U}\left(\alpha\right)$ y $\hat{F}\left(\alpha\right)$ determina una a la otra de manera \'unica por la relaci\'on $\hat{U}\left(\alpha\right)=\frac{1}{1-\hat{F}\left(\alpha\right)}$.
\end{Prop}


\begin{Note}
Un proceso de renovaci\'on $N\left(t\right)$ cuyos tiempos de inter-renovaci\'on tienen media finita, es un proceso Poisson con tasa $\lambda$ si y s\'olo s\'i $\esp\left[U\left(t\right)\right]=\lambda t$, para $t\geq0$.
\end{Note}


\begin{Teo}
Sea $N\left(t\right)$ un proceso puntual simple con puntos de localizaci\'on $T_{n}$ tal que $\eta\left(t\right)=\esp\left[N\left(\right)\right]$ es finita para cada $t$. Entonces para cualquier funci\'on $f:\rea_{+}\rightarrow\rea$,
\begin{eqnarray*}
\esp\left[\sum_{n=1}^{N\left(\right)}f\left(T_{n}\right)\right]=\int_{\left(0,t\right]}f\left(s\right)d\eta\left(s\right)\textrm{,  }t\geq0,
\end{eqnarray*}
suponiendo que la integral exista. Adem\'as si $X_{1},X_{2},\ldots$ son variables aleatorias definidas en el mismo espacio de probabilidad que el proceso $N\left(t\right)$ tal que $\esp\left[X_{n}|T_{n}=s\right]=f\left(s\right)$, independiente de $n$. Entonces
\begin{eqnarray*}
\esp\left[\sum_{n=1}^{N\left(t\right)}X_{n}\right]=\int_{\left(0,t\right]}f\left(s\right)d\eta\left(s\right)\textrm{,  }t\geq0,
\end{eqnarray*} 
suponiendo que la integral exista. 
\end{Teo}

\begin{Coro}[Identidad de Wald para Renovaciones]
Para el proceso de renovaci\'on $N\left(t\right)$,
\begin{eqnarray*}
\esp\left[T_{N\left(t\right)+1}\right]=\mu\esp\left[N\left(t\right)+1\right]\textrm{,  }t\geq0,
\end{eqnarray*}  
\end{Coro}


\begin{Def}
Sea $h\left(t\right)$ funci\'on de valores reales en $\rea$ acotada en intervalos finitos e igual a cero para $t<0$ La ecuaci\'on de renovaci\'on para $h\left(t\right)$ y la distribuci\'on $F$ es

\begin{eqnarray}\label{Ec.Renovacion}
H\left(t\right)=h\left(t\right)+\int_{\left[0,t\right]}H\left(t-s\right)dF\left(s\right)\textrm{,    }t\geq0,
\end{eqnarray}
donde $H\left(t\right)$ es una funci\'on de valores reales. Esto es $H=h+F\star H$. Decimos que $H\left(t\right)$ es soluci\'on de esta ecuaci\'on si satisface la ecuaci\'on, y es acotada en intervalos finitos e iguales a cero para $t<0$.
\end{Def}

\begin{Prop}
La funci\'on $U\star h\left(t\right)$ es la \'unica soluci\'on de la ecuaci\'on de renovaci\'on (\ref{Ec.Renovacion}).
\end{Prop}

\begin{Teo}[Teorema Renovaci\'on Elemental]
\begin{eqnarray*}
t^{-1}U\left(t\right)\rightarrow 1/\mu\textrm{,    cuando }t\rightarrow\infty.
\end{eqnarray*}
\end{Teo}



Sup\'ongase que $N\left(t\right)$ es un proceso de renovaci\'on con distribuci\'on $F$ con media finita $\mu$.

\begin{Def}
La funci\'on de renovaci\'on asociada con la distribuci\'on $F$, del proceso $N\left(t\right)$, es
\begin{eqnarray*}
U\left(t\right)=\sum_{n=1}^{\infty}F^{n\star}\left(t\right),\textrm{   }t\geq0,
\end{eqnarray*}
donde $F^{0\star}\left(t\right)=\indora\left(t\geq0\right)$.
\end{Def}


\begin{Prop}
Sup\'ongase que la distribuci\'on de inter-renovaci\'on $F$ tiene densidad $f$. Entonces $U\left(t\right)$ tambi\'en tiene densidad, para $t>0$, y es $U^{'}\left(t\right)=\sum_{n=0}^{\infty}f^{n\star}\left(t\right)$. Adem\'as
\begin{eqnarray*}
\prob\left\{N\left(t\right)>N\left(t-\right)\right\}=0\textrm{,   }t\geq0.
\end{eqnarray*}
\end{Prop}

\begin{Def}
La Transformada de Laplace-Stieljes de $F$ est\'a dada por

\begin{eqnarray*}
\hat{F}\left(\alpha\right)=\int_{\rea_{+}}e^{-\alpha t}dF\left(t\right)\textrm{,  }\alpha\geq0.
\end{eqnarray*}
\end{Def}

Entonces

\begin{eqnarray*}
\hat{U}\left(\alpha\right)=\sum_{n=0}^{\infty}\hat{F^{n\star}}\left(\alpha\right)=\sum_{n=0}^{\infty}\hat{F}\left(\alpha\right)^{n}=\frac{1}{1-\hat{F}\left(\alpha\right)}.
\end{eqnarray*}


\begin{Prop}
La Transformada de Laplace $\hat{U}\left(\alpha\right)$ y $\hat{F}\left(\alpha\right)$ determina una a la otra de manera \'unica por la relaci\'on $\hat{U}\left(\alpha\right)=\frac{1}{1-\hat{F}\left(\alpha\right)}$.
\end{Prop}


\begin{Note}
Un proceso de renovaci\'on $N\left(t\right)$ cuyos tiempos de inter-renovaci\'on tienen media finita, es un proceso Poisson con tasa $\lambda$ si y s\'olo s\'i $\esp\left[U\left(t\right)\right]=\lambda t$, para $t\geq0$.
\end{Note}


\begin{Teo}
Sea $N\left(t\right)$ un proceso puntual simple con puntos de localizaci\'on $T_{n}$ tal que $\eta\left(t\right)=\esp\left[N\left(\right)\right]$ es finita para cada $t$. Entonces para cualquier funci\'on $f:\rea_{+}\rightarrow\rea$,
\begin{eqnarray*}
\esp\left[\sum_{n=1}^{N\left(\right)}f\left(T_{n}\right)\right]=\int_{\left(0,t\right]}f\left(s\right)d\eta\left(s\right)\textrm{,  }t\geq0,
\end{eqnarray*}
suponiendo que la integral exista. Adem\'as si $X_{1},X_{2},\ldots$ son variables aleatorias definidas en el mismo espacio de probabilidad que el proceso $N\left(t\right)$ tal que $\esp\left[X_{n}|T_{n}=s\right]=f\left(s\right)$, independiente de $n$. Entonces
\begin{eqnarray*}
\esp\left[\sum_{n=1}^{N\left(t\right)}X_{n}\right]=\int_{\left(0,t\right]}f\left(s\right)d\eta\left(s\right)\textrm{,  }t\geq0,
\end{eqnarray*} 
suponiendo que la integral exista. 
\end{Teo}

\begin{Coro}[Identidad de Wald para Renovaciones]
Para el proceso de renovaci\'on $N\left(t\right)$,
\begin{eqnarray*}
\esp\left[T_{N\left(t\right)+1}\right]=\mu\esp\left[N\left(t\right)+1\right]\textrm{,  }t\geq0,
\end{eqnarray*}  
\end{Coro}


\begin{Def}
Sea $h\left(t\right)$ funci\'on de valores reales en $\rea$ acotada en intervalos finitos e igual a cero para $t<0$ La ecuaci\'on de renovaci\'on para $h\left(t\right)$ y la distribuci\'on $F$ es

\begin{eqnarray}\label{Ec.Renovacion}
H\left(t\right)=h\left(t\right)+\int_{\left[0,t\right]}H\left(t-s\right)dF\left(s\right)\textrm{,    }t\geq0,
\end{eqnarray}
donde $H\left(t\right)$ es una funci\'on de valores reales. Esto es $H=h+F\star H$. Decimos que $H\left(t\right)$ es soluci\'on de esta ecuaci\'on si satisface la ecuaci\'on, y es acotada en intervalos finitos e iguales a cero para $t<0$.
\end{Def}

\begin{Prop}
La funci\'on $U\star h\left(t\right)$ es la \'unica soluci\'on de la ecuaci\'on de renovaci\'on (\ref{Ec.Renovacion}).
\end{Prop}

\begin{Teo}[Teorema Renovaci\'on Elemental]
\begin{eqnarray*}
t^{-1}U\left(t\right)\rightarrow 1/\mu\textrm{,    cuando }t\rightarrow\infty.
\end{eqnarray*}
\end{Teo}


\begin{Note} Una funci\'on $h:\rea_{+}\rightarrow\rea$ es Directamente Riemann Integrable en los siguientes casos:
\begin{itemize}
\item[a)] $h\left(t\right)\geq0$ es decreciente y Riemann Integrable.
\item[b)] $h$ es continua excepto posiblemente en un conjunto de Lebesgue de medida 0, y $|h\left(t\right)|\leq b\left(t\right)$, donde $b$ es DRI.
\end{itemize}
\end{Note}

\begin{Teo}[Teorema Principal de Renovaci\'on]
Si $F$ es no aritm\'etica y $h\left(t\right)$ es Directamente Riemann Integrable (DRI), entonces

\begin{eqnarray*}
lim_{t\rightarrow\infty}U\star h=\frac{1}{\mu}\int_{\rea_{+}}h\left(s\right)ds.
\end{eqnarray*}
\end{Teo}

\begin{Prop}
Cualquier funci\'on $H\left(t\right)$ acotada en intervalos finitos y que es 0 para $t<0$ puede expresarse como
\begin{eqnarray*}
H\left(t\right)=U\star h\left(t\right)\textrm{,  donde }h\left(t\right)=H\left(t\right)-F\star H\left(t\right)
\end{eqnarray*}
\end{Prop}

\begin{Def}
Un proceso estoc\'astico $X\left(t\right)$ es crudamente regenerativo en un tiempo aleatorio positivo $T$ si
\begin{eqnarray*}
\esp\left[X\left(T+t\right)|T\right]=\esp\left[X\left(t\right)\right]\textrm{, para }t\geq0,\end{eqnarray*}
y con las esperanzas anteriores finitas.
\end{Def}

\begin{Prop}
Sup\'ongase que $X\left(t\right)$ es un proceso crudamente regenerativo en $T$, que tiene distribuci\'on $F$. Si $\esp\left[X\left(t\right)\right]$ es acotado en intervalos finitos, entonces
\begin{eqnarray*}
\esp\left[X\left(t\right)\right]=U\star h\left(t\right)\textrm{,  donde }h\left(t\right)=\esp\left[X\left(t\right)\indora\left(T>t\right)\right].
\end{eqnarray*}
\end{Prop}

\begin{Teo}[Regeneraci\'on Cruda]
Sup\'ongase que $X\left(t\right)$ es un proceso con valores positivo crudamente regenerativo en $T$, y def\'inase $M=\sup\left\{|X\left(t\right)|:t\leq T\right\}$. Si $T$ es no aritm\'etico y $M$ y $MT$ tienen media finita, entonces
\begin{eqnarray*}
lim_{t\rightarrow\infty}\esp\left[X\left(t\right)\right]=\frac{1}{\mu}\int_{\rea_{+}}h\left(s\right)ds,
\end{eqnarray*}
donde $h\left(t\right)=\esp\left[X\left(t\right)\indora\left(T>t\right)\right]$.
\end{Teo}


\begin{Note} Una funci\'on $h:\rea_{+}\rightarrow\rea$ es Directamente Riemann Integrable en los siguientes casos:
\begin{itemize}
\item[a)] $h\left(t\right)\geq0$ es decreciente y Riemann Integrable.
\item[b)] $h$ es continua excepto posiblemente en un conjunto de Lebesgue de medida 0, y $|h\left(t\right)|\leq b\left(t\right)$, donde $b$ es DRI.
\end{itemize}
\end{Note}

\begin{Teo}[Teorema Principal de Renovaci\'on]
Si $F$ es no aritm\'etica y $h\left(t\right)$ es Directamente Riemann Integrable (DRI), entonces

\begin{eqnarray*}
lim_{t\rightarrow\infty}U\star h=\frac{1}{\mu}\int_{\rea_{+}}h\left(s\right)ds.
\end{eqnarray*}
\end{Teo}

\begin{Prop}
Cualquier funci\'on $H\left(t\right)$ acotada en intervalos finitos y que es 0 para $t<0$ puede expresarse como
\begin{eqnarray*}
H\left(t\right)=U\star h\left(t\right)\textrm{,  donde }h\left(t\right)=H\left(t\right)-F\star H\left(t\right)
\end{eqnarray*}
\end{Prop}

\begin{Def}
Un proceso estoc\'astico $X\left(t\right)$ es crudamente regenerativo en un tiempo aleatorio positivo $T$ si
\begin{eqnarray*}
\esp\left[X\left(T+t\right)|T\right]=\esp\left[X\left(t\right)\right]\textrm{, para }t\geq0,\end{eqnarray*}
y con las esperanzas anteriores finitas.
\end{Def}

\begin{Prop}
Sup\'ongase que $X\left(t\right)$ es un proceso crudamente regenerativo en $T$, que tiene distribuci\'on $F$. Si $\esp\left[X\left(t\right)\right]$ es acotado en intervalos finitos, entonces
\begin{eqnarray*}
\esp\left[X\left(t\right)\right]=U\star h\left(t\right)\textrm{,  donde }h\left(t\right)=\esp\left[X\left(t\right)\indora\left(T>t\right)\right].
\end{eqnarray*}
\end{Prop}

\begin{Teo}[Regeneraci\'on Cruda]
Sup\'ongase que $X\left(t\right)$ es un proceso con valores positivo crudamente regenerativo en $T$, y def\'inase $M=\sup\left\{|X\left(t\right)|:t\leq T\right\}$. Si $T$ es no aritm\'etico y $M$ y $MT$ tienen media finita, entonces
\begin{eqnarray*}
lim_{t\rightarrow\infty}\esp\left[X\left(t\right)\right]=\frac{1}{\mu}\int_{\rea_{+}}h\left(s\right)ds,
\end{eqnarray*}
donde $h\left(t\right)=\esp\left[X\left(t\right)\indora\left(T>t\right)\right]$.
\end{Teo}

\begin{Def}
Para el proceso $\left\{\left(N\left(t\right),X\left(t\right)\right):t\geq0\right\}$, sus trayectoria muestrales en el intervalo de tiempo $\left[T_{n-1},T_{n}\right)$ est\'an descritas por
\begin{eqnarray*}
\zeta_{n}=\left(\xi_{n},\left\{X\left(T_{n-1}+t\right):0\leq t<\xi_{n}\right\}\right)
\end{eqnarray*}
Este $\zeta_{n}$ es el $n$-\'esimo segmento del proceso. El proceso es regenerativo sobre los tiempos $T_{n}$ si sus segmentos $\zeta_{n}$ son independientes e id\'enticamennte distribuidos.
\end{Def}


\begin{Note}
Si $\tilde{X}\left(t\right)$ con espacio de estados $\tilde{S}$ es regenerativo sobre $T_{n}$, entonces $X\left(t\right)=f\left(\tilde{X}\left(t\right)\right)$ tambi\'en es regenerativo sobre $T_{n}$, para cualquier funci\'on $f:\tilde{S}\rightarrow S$.
\end{Note}

\begin{Note}
Los procesos regenerativos son crudamente regenerativos, pero no al rev\'es.
\end{Note}


\begin{Note}
Un proceso estoc\'astico a tiempo continuo o discreto es regenerativo si existe un proceso de renovaci\'on  tal que los segmentos del proceso entre tiempos de renovaci\'on sucesivos son i.i.d., es decir, para $\left\{X\left(t\right):t\geq0\right\}$ proceso estoc\'astico a tiempo continuo con espacio de estados $S$, espacio m\'etrico.
\end{Note}

Para $\left\{X\left(t\right):t\geq0\right\}$ Proceso Estoc\'astico a tiempo continuo con estado de espacios $S$, que es un espacio m\'etrico, con trayectorias continuas por la derecha y con l\'imites por la izquierda c.s. Sea $N\left(t\right)$ un proceso de renovaci\'on en $\rea_{+}$ definido en el mismo espacio de probabilidad que $X\left(t\right)$, con tiempos de renovaci\'on $T$ y tiempos de inter-renovaci\'on $\xi_{n}=T_{n}-T_{n-1}$, con misma distribuci\'on $F$ de media finita $\mu$.



\begin{Def}
Para el proceso $\left\{\left(N\left(t\right),X\left(t\right)\right):t\geq0\right\}$, sus trayectoria muestrales en el intervalo de tiempo $\left[T_{n-1},T_{n}\right)$ est\'an descritas por
\begin{eqnarray*}
\zeta_{n}=\left(\xi_{n},\left\{X\left(T_{n-1}+t\right):0\leq t<\xi_{n}\right\}\right)
\end{eqnarray*}
Este $\zeta_{n}$ es el $n$-\'esimo segmento del proceso. El proceso es regenerativo sobre los tiempos $T_{n}$ si sus segmentos $\zeta_{n}$ son independientes e id\'enticamennte distribuidos.
\end{Def}

\begin{Note}
Un proceso regenerativo con media de la longitud de ciclo finita es llamado positivo recurrente.
\end{Note}

\begin{Teo}[Procesos Regenerativos]
Suponga que el proceso
\end{Teo}


\begin{Def}[Renewal Process Trinity]
Para un proceso de renovaci\'on $N\left(t\right)$, los siguientes procesos proveen de informaci\'on sobre los tiempos de renovaci\'on.
\begin{itemize}
\item $A\left(t\right)=t-T_{N\left(t\right)}$, el tiempo de recurrencia hacia atr\'as al tiempo $t$, que es el tiempo desde la \'ultima renovaci\'on para $t$.

\item $B\left(t\right)=T_{N\left(t\right)+1}-t$, el tiempo de recurrencia hacia adelante al tiempo $t$, residual del tiempo de renovaci\'on, que es el tiempo para la pr\'oxima renovaci\'on despu\'es de $t$.

\item $L\left(t\right)=\xi_{N\left(t\right)+1}=A\left(t\right)+B\left(t\right)$, la longitud del intervalo de renovaci\'on que contiene a $t$.
\end{itemize}
\end{Def}

\begin{Note}
El proceso tridimensional $\left(A\left(t\right),B\left(t\right),L\left(t\right)\right)$ es regenerativo sobre $T_{n}$, y por ende cada proceso lo es. Cada proceso $A\left(t\right)$ y $B\left(t\right)$ son procesos de MArkov a tiempo continuo con trayectorias continuas por partes en el espacio de estados $\rea_{+}$. Una expresi\'on conveniente para su distribuci\'on conjunta es, para $0\leq x<t,y\geq0$
\begin{equation}\label{NoRenovacion}
P\left\{A\left(t\right)>x,B\left(t\right)>y\right\}=
P\left\{N\left(t+y\right)-N\left((t-x)\right)=0\right\}
\end{equation}
\end{Note}

\begin{Ejem}[Tiempos de recurrencia Poisson]
Si $N\left(t\right)$ es un proceso Poisson con tasa $\lambda$, entonces de la expresi\'on (\ref{NoRenovacion}) se tiene que

\begin{eqnarray*}
\begin{array}{lc}
P\left\{A\left(t\right)>x,B\left(t\right)>y\right\}=e^{-\lambda\left(x+y\right)},&0\leq x<t,y\geq0,
\end{array}
\end{eqnarray*}
que es la probabilidad Poisson de no renovaciones en un intervalo de longitud $x+y$.

\end{Ejem}

\begin{Note}
Una cadena de Markov erg\'odica tiene la propiedad de ser estacionaria si la distribuci\'on de su estado al tiempo $0$ es su distribuci\'on estacionaria.
\end{Note}


\begin{Def}
Un proceso estoc\'astico a tiempo continuo $\left\{X\left(t\right):t\geq0\right\}$ en un espacio general es estacionario si sus distribuciones finito dimensionales son invariantes bajo cualquier  traslado: para cada $0\leq s_{1}<s_{2}<\cdots<s_{k}$ y $t\geq0$,
\begin{eqnarray*}
\left(X\left(s_{1}+t\right),\ldots,X\left(s_{k}+t\right)\right)=_{d}\left(X\left(s_{1}\right),\ldots,X\left(s_{k}\right)\right).
\end{eqnarray*}
\end{Def}

\begin{Note}
Un proceso de Markov es estacionario si $X\left(t\right)=_{d}X\left(0\right)$, $t\geq0$.
\end{Note}

Considerese el proceso $N\left(t\right)=\sum_{n}\indora\left(\tau_{n}\leq t\right)$ en $\rea_{+}$, con puntos $0<\tau_{1}<\tau_{2}<\cdots$.

\begin{Prop}
Si $N$ es un proceso puntual estacionario y $\esp\left[N\left(1\right)\right]<\infty$, entonces $\esp\left[N\left(t\right)\right]=t\esp\left[N\left(1\right)\right]$, $t\geq0$

\end{Prop}

\begin{Teo}
Los siguientes enunciados son equivalentes
\begin{itemize}
\item[i)] El proceso retardado de renovaci\'on $N$ es estacionario.

\item[ii)] EL proceso de tiempos de recurrencia hacia adelante $B\left(t\right)$ es estacionario.


\item[iii)] $\esp\left[N\left(t\right)\right]=t/\mu$,


\item[iv)] $G\left(t\right)=F_{e}\left(t\right)=\frac{1}{\mu}\int_{0}^{t}\left[1-F\left(s\right)\right]ds$
\end{itemize}
Cuando estos enunciados son ciertos, $P\left\{B\left(t\right)\leq x\right\}=F_{e}\left(x\right)$, para $t,x\geq0$.

\end{Teo}

\begin{Note}
Una consecuencia del teorema anterior es que el Proceso Poisson es el \'unico proceso sin retardo que es estacionario.
\end{Note}

\begin{Coro}
El proceso de renovaci\'on $N\left(t\right)$ sin retardo, y cuyos tiempos de inter renonaci\'on tienen media finita, es estacionario si y s\'olo si es un proceso Poisson.

\end{Coro}

%________________________________________________________________________
\subsection*{Procesos Regenerativos}
%________________________________________________________________________



\begin{Note}
Si $\tilde{X}\left(t\right)$ con espacio de estados $\tilde{S}$ es regenerativo sobre $T_{n}$, entonces $X\left(t\right)=f\left(\tilde{X}\left(t\right)\right)$ tambi\'en es regenerativo sobre $T_{n}$, para cualquier funci\'on $f:\tilde{S}\rightarrow S$.
\end{Note}

\begin{Note}
Los procesos regenerativos son crudamente regenerativos, pero no al rev\'es.
\end{Note}
%\subsection*{Procesos Regenerativos: Sigman\cite{Sigman1}}
\begin{Def}[Definici\'on Cl\'asica]
Un proceso estoc\'astico $X=\left\{X\left(t\right):t\geq0\right\}$ es llamado regenerativo is existe una variable aleatoria $R_{1}>0$ tal que
\begin{itemize}
\item[i)] $\left\{X\left(t+R_{1}\right):t\geq0\right\}$ es independiente de $\left\{\left\{X\left(t\right):t<R_{1}\right\},\right\}$
\item[ii)] $\left\{X\left(t+R_{1}\right):t\geq0\right\}$ es estoc\'asticamente equivalente a $\left\{X\left(t\right):t>0\right\}$
\end{itemize}

Llamamos a $R_{1}$ tiempo de regeneraci\'on, y decimos que $X$ se regenera en este punto.
\end{Def}

$\left\{X\left(t+R_{1}\right)\right\}$ es regenerativo con tiempo de regeneraci\'on $R_{2}$, independiente de $R_{1}$ pero con la misma distribuci\'on que $R_{1}$. Procediendo de esta manera se obtiene una secuencia de variables aleatorias independientes e id\'enticamente distribuidas $\left\{R_{n}\right\}$ llamados longitudes de ciclo. Si definimos a $Z_{k}\equiv R_{1}+R_{2}+\cdots+R_{k}$, se tiene un proceso de renovaci\'on llamado proceso de renovaci\'on encajado para $X$.




\begin{Def}
Para $x$ fijo y para cada $t\geq0$, sea $I_{x}\left(t\right)=1$ si $X\left(t\right)\leq x$,  $I_{x}\left(t\right)=0$ en caso contrario, y def\'inanse los tiempos promedio
\begin{eqnarray*}
\overline{X}&=&lim_{t\rightarrow\infty}\frac{1}{t}\int_{0}^{\infty}X\left(u\right)du\\
\prob\left(X_{\infty}\leq x\right)&=&lim_{t\rightarrow\infty}\frac{1}{t}\int_{0}^{\infty}I_{x}\left(u\right)du,
\end{eqnarray*}
cuando estos l\'imites existan.
\end{Def}

Como consecuencia del teorema de Renovaci\'on-Recompensa, se tiene que el primer l\'imite  existe y es igual a la constante
\begin{eqnarray*}
\overline{X}&=&\frac{\esp\left[\int_{0}^{R_{1}}X\left(t\right)dt\right]}{\esp\left[R_{1}\right]},
\end{eqnarray*}
suponiendo que ambas esperanzas son finitas.

\begin{Note}
\begin{itemize}
\item[a)] Si el proceso regenerativo $X$ es positivo recurrente y tiene trayectorias muestrales no negativas, entonces la ecuaci\'on anterior es v\'alida.
\item[b)] Si $X$ es positivo recurrente regenerativo, podemos construir una \'unica versi\'on estacionaria de este proceso, $X_{e}=\left\{X_{e}\left(t\right)\right\}$, donde $X_{e}$ es un proceso estoc\'astico regenerativo y estrictamente estacionario, con distribuci\'on marginal distribuida como $X_{\infty}$
\end{itemize}
\end{Note}

Para $\left\{X\left(t\right):t\geq0\right\}$ Proceso Estoc\'astico a tiempo continuo con estado de espacios $S$, que es un espacio m\'etrico, con trayectorias continuas por la derecha y con l\'imites por la izquierda c.s. Sea $N\left(t\right)$ un proceso de renovaci\'on en $\rea_{+}$ definido en el mismo espacio de probabilidad que $X\left(t\right)$, con tiempos de renovaci\'on $T$ y tiempos de inter-renovaci\'on $\xi_{n}=T_{n}-T_{n-1}$, con misma distribuci\'on $F$ de media finita $\mu$.


\begin{Def}
Para el proceso $\left\{\left(N\left(t\right),X\left(t\right)\right):t\geq0\right\}$, sus trayectoria muestrales en el intervalo de tiempo $\left[T_{n-1},T_{n}\right)$ est\'an descritas por
\begin{eqnarray*}
\zeta_{n}=\left(\xi_{n},\left\{X\left(T_{n-1}+t\right):0\leq t<\xi_{n}\right\}\right)
\end{eqnarray*}
Este $\zeta_{n}$ es el $n$-\'esimo segmento del proceso. El proceso es regenerativo sobre los tiempos $T_{n}$ si sus segmentos $\zeta_{n}$ son independientes e id\'enticamennte distribuidos.
\end{Def}


\begin{Note}
Si $\tilde{X}\left(t\right)$ con espacio de estados $\tilde{S}$ es regenerativo sobre $T_{n}$, entonces $X\left(t\right)=f\left(\tilde{X}\left(t\right)\right)$ tambi\'en es regenerativo sobre $T_{n}$, para cualquier funci\'on $f:\tilde{S}\rightarrow S$.
\end{Note}

\begin{Note}
Los procesos regenerativos son crudamente regenerativos, pero no al rev\'es.
\end{Note}

\begin{Def}[Definici\'on Cl\'asica]
Un proceso estoc\'astico $X=\left\{X\left(t\right):t\geq0\right\}$ es llamado regenerativo is existe una variable aleatoria $R_{1}>0$ tal que
\begin{itemize}
\item[i)] $\left\{X\left(t+R_{1}\right):t\geq0\right\}$ es independiente de $\left\{\left\{X\left(t\right):t<R_{1}\right\},\right\}$
\item[ii)] $\left\{X\left(t+R_{1}\right):t\geq0\right\}$ es estoc\'asticamente equivalente a $\left\{X\left(t\right):t>0\right\}$
\end{itemize}

Llamamos a $R_{1}$ tiempo de regeneraci\'on, y decimos que $X$ se regenera en este punto.
\end{Def}

$\left\{X\left(t+R_{1}\right)\right\}$ es regenerativo con tiempo de regeneraci\'on $R_{2}$, independiente de $R_{1}$ pero con la misma distribuci\'on que $R_{1}$. Procediendo de esta manera se obtiene una secuencia de variables aleatorias independientes e id\'enticamente distribuidas $\left\{R_{n}\right\}$ llamados longitudes de ciclo. Si definimos a $Z_{k}\equiv R_{1}+R_{2}+\cdots+R_{k}$, se tiene un proceso de renovaci\'on llamado proceso de renovaci\'on encajado para $X$.

\begin{Note}
Un proceso regenerativo con media de la longitud de ciclo finita es llamado positivo recurrente.
\end{Note}


\begin{Def}
Para $x$ fijo y para cada $t\geq0$, sea $I_{x}\left(t\right)=1$ si $X\left(t\right)\leq x$,  $I_{x}\left(t\right)=0$ en caso contrario, y def\'inanse los tiempos promedio
\begin{eqnarray*}
\overline{X}&=&lim_{t\rightarrow\infty}\frac{1}{t}\int_{0}^{\infty}X\left(u\right)du\\
\prob\left(X_{\infty}\leq x\right)&=&lim_{t\rightarrow\infty}\frac{1}{t}\int_{0}^{\infty}I_{x}\left(u\right)du,
\end{eqnarray*}
cuando estos l\'imites existan.
\end{Def}

Como consecuencia del teorema de Renovaci\'on-Recompensa, se tiene que el primer l\'imite  existe y es igual a la constante
\begin{eqnarray*}
\overline{X}&=&\frac{\esp\left[\int_{0}^{R_{1}}X\left(t\right)dt\right]}{\esp\left[R_{1}\right]},
\end{eqnarray*}
suponiendo que ambas esperanzas son finitas.

\begin{Note}
\begin{itemize}
\item[a)] Si el proceso regenerativo $X$ es positivo recurrente y tiene trayectorias muestrales no negativas, entonces la ecuaci\'on anterior es v\'alida.
\item[b)] Si $X$ es positivo recurrente regenerativo, podemos construir una \'unica versi\'on estacionaria de este proceso, $X_{e}=\left\{X_{e}\left(t\right)\right\}$, donde $X_{e}$ es un proceso estoc\'astico regenerativo y estrictamente estacionario, con distribuci\'on marginal distribuida como $X_{\infty}$
\end{itemize}
\end{Note}

%__________________________________________________________________________________________
\subsection{Procesos Regenerativos Estacionarios - Stidham \cite{Stidham}}
%__________________________________________________________________________________________


Un proceso estoc\'astico a tiempo continuo $\left\{V\left(t\right),t\geq0\right\}$ es un proceso regenerativo si existe una sucesi\'on de variables aleatorias independientes e id\'enticamente distribuidas $\left\{X_{1},X_{2},\ldots\right\}$, sucesi\'on de renovaci\'on, tal que para cualquier conjunto de Borel $A$, 

\begin{eqnarray*}
\prob\left\{V\left(t\right)\in A|X_{1}+X_{2}+\cdots+X_{R\left(t\right)}=s,\left\{V\left(\tau\right),\tau<s\right\}\right\}=\prob\left\{V\left(t-s\right)\in A|X_{1}>t-s\right\},
\end{eqnarray*}
para todo $0\leq s\leq t$, donde $R\left(t\right)=\max\left\{X_{1}+X_{2}+\cdots+X_{j}\leq t\right\}=$n\'umero de renovaciones ({\emph{puntos de regeneraci\'on}}) que ocurren en $\left[0,t\right]$. El intervalo $\left[0,X_{1}\right)$ es llamado {\emph{primer ciclo de regeneraci\'on}} de $\left\{V\left(t \right),t\geq0\right\}$, $\left[X_{1},X_{1}+X_{2}\right)$ el {\emph{segundo ciclo de regeneraci\'on}}, y as\'i sucesivamente.

Sea $X=X_{1}$ y sea $F$ la funci\'on de distrbuci\'on de $X$


\begin{Def}
Se define el proceso estacionario, $\left\{V^{*}\left(t\right),t\geq0\right\}$, para $\left\{V\left(t\right),t\geq0\right\}$ por

\begin{eqnarray*}
\prob\left\{V\left(t\right)\in A\right\}=\frac{1}{\esp\left[X\right]}\int_{0}^{\infty}\prob\left\{V\left(t+x\right)\in A|X>x\right\}\left(1-F\left(x\right)\right)dx,
\end{eqnarray*} 
para todo $t\geq0$ y todo conjunto de Borel $A$.
\end{Def}

\begin{Def}
Una distribuci\'on se dice que es {\emph{aritm\'etica}} si todos sus puntos de incremento son m\'ultiplos de la forma $0,\lambda, 2\lambda,\ldots$ para alguna $\lambda>0$ entera.
\end{Def}


\begin{Def}
Una modificaci\'on medible de un proceso $\left\{V\left(t\right),t\geq0\right\}$, es una versi\'on de este, $\left\{V\left(t,w\right)\right\}$ conjuntamente medible para $t\geq0$ y para $w\in S$, $S$ espacio de estados para $\left\{V\left(t\right),t\geq0\right\}$.
\end{Def}

\begin{Teo}
Sea $\left\{V\left(t\right),t\geq\right\}$ un proceso regenerativo no negativo con modificaci\'on medible. Sea $\esp\left[X\right]<\infty$. Entonces el proceso estacionario dado por la ecuaci\'on anterior est\'a bien definido y tiene funci\'on de distribuci\'on independiente de $t$, adem\'as
\begin{itemize}
\item[i)] \begin{eqnarray*}
\esp\left[V^{*}\left(0\right)\right]&=&\frac{\esp\left[\int_{0}^{X}V\left(s\right)ds\right]}{\esp\left[X\right]}\end{eqnarray*}
\item[ii)] Si $\esp\left[V^{*}\left(0\right)\right]<\infty$, equivalentemente, si $\esp\left[\int_{0}^{X}V\left(s\right)ds\right]<\infty$,entonces
\begin{eqnarray*}
\frac{\int_{0}^{t}V\left(s\right)ds}{t}\rightarrow\frac{\esp\left[\int_{0}^{X}V\left(s\right)ds\right]}{\esp\left[X\right]}
\end{eqnarray*}
con probabilidad 1 y en media, cuando $t\rightarrow\infty$.
\end{itemize}
\end{Teo}
%
%___________________________________________________________________________________________
%\vspace{5.5cm}
%\chapter{Cadenas de Markov estacionarias}
%\vspace{-1.0cm}


%__________________________________________________________________________________________
\subsection{Procesos Regenerativos Estacionarios - Stidham \cite{Stidham}}
%__________________________________________________________________________________________


Un proceso estoc\'astico a tiempo continuo $\left\{V\left(t\right),t\geq0\right\}$ es un proceso regenerativo si existe una sucesi\'on de variables aleatorias independientes e id\'enticamente distribuidas $\left\{X_{1},X_{2},\ldots\right\}$, sucesi\'on de renovaci\'on, tal que para cualquier conjunto de Borel $A$, 

\begin{eqnarray*}
\prob\left\{V\left(t\right)\in A|X_{1}+X_{2}+\cdots+X_{R\left(t\right)}=s,\left\{V\left(\tau\right),\tau<s\right\}\right\}=\prob\left\{V\left(t-s\right)\in A|X_{1}>t-s\right\},
\end{eqnarray*}
para todo $0\leq s\leq t$, donde $R\left(t\right)=\max\left\{X_{1}+X_{2}+\cdots+X_{j}\leq t\right\}=$n\'umero de renovaciones ({\emph{puntos de regeneraci\'on}}) que ocurren en $\left[0,t\right]$. El intervalo $\left[0,X_{1}\right)$ es llamado {\emph{primer ciclo de regeneraci\'on}} de $\left\{V\left(t \right),t\geq0\right\}$, $\left[X_{1},X_{1}+X_{2}\right)$ el {\emph{segundo ciclo de regeneraci\'on}}, y as\'i sucesivamente.

Sea $X=X_{1}$ y sea $F$ la funci\'on de distrbuci\'on de $X$


\begin{Def}
Se define el proceso estacionario, $\left\{V^{*}\left(t\right),t\geq0\right\}$, para $\left\{V\left(t\right),t\geq0\right\}$ por

\begin{eqnarray*}
\prob\left\{V\left(t\right)\in A\right\}=\frac{1}{\esp\left[X\right]}\int_{0}^{\infty}\prob\left\{V\left(t+x\right)\in A|X>x\right\}\left(1-F\left(x\right)\right)dx,
\end{eqnarray*} 
para todo $t\geq0$ y todo conjunto de Borel $A$.
\end{Def}

\begin{Def}
Una distribuci\'on se dice que es {\emph{aritm\'etica}} si todos sus puntos de incremento son m\'ultiplos de la forma $0,\lambda, 2\lambda,\ldots$ para alguna $\lambda>0$ entera.
\end{Def}


\begin{Def}
Una modificaci\'on medible de un proceso $\left\{V\left(t\right),t\geq0\right\}$, es una versi\'on de este, $\left\{V\left(t,w\right)\right\}$ conjuntamente medible para $t\geq0$ y para $w\in S$, $S$ espacio de estados para $\left\{V\left(t\right),t\geq0\right\}$.
\end{Def}

\begin{Teo}
Sea $\left\{V\left(t\right),t\geq\right\}$ un proceso regenerativo no negativo con modificaci\'on medible. Sea $\esp\left[X\right]<\infty$. Entonces el proceso estacionario dado por la ecuaci\'on anterior est\'a bien definido y tiene funci\'on de distribuci\'on independiente de $t$, adem\'as
\begin{itemize}
\item[i)] \begin{eqnarray*}
\esp\left[V^{*}\left(0\right)\right]&=&\frac{\esp\left[\int_{0}^{X}V\left(s\right)ds\right]}{\esp\left[X\right]}\end{eqnarray*}
\item[ii)] Si $\esp\left[V^{*}\left(0\right)\right]<\infty$, equivalentemente, si $\esp\left[\int_{0}^{X}V\left(s\right)ds\right]<\infty$,entonces
\begin{eqnarray*}
\frac{\int_{0}^{t}V\left(s\right)ds}{t}\rightarrow\frac{\esp\left[\int_{0}^{X}V\left(s\right)ds\right]}{\esp\left[X\right]}
\end{eqnarray*}
con probabilidad 1 y en media, cuando $t\rightarrow\infty$.
\end{itemize}
\end{Teo}



Para $\left\{X\left(t\right):t\geq0\right\}$ Proceso Estoc\'astico a tiempo continuo con estado de espacios $S$, que es un espacio m\'etrico, con trayectorias continuas por la derecha y con l\'imites por la izquierda c.s. Sea $N\left(t\right)$ un proceso de renovaci\'on en $\rea_{+}$ definido en el mismo espacio de probabilidad que $X\left(t\right)$, con tiempos de renovaci\'on $T$ y tiempos de inter-renovaci\'on $\xi_{n}=T_{n}-T_{n-1}$, con misma distribuci\'on $F$ de media finita $\mu$.

\begin{Def}
Un elemento aleatorio en un espacio medible $\left(E,\mathcal{E}\right)$ en un espacio de probabilidad $\left(\Omega,\mathcal{F},\prob\right)$ a $\left(E,\mathcal{E}\right)$, es decir,
para $A\in \mathcal{E}$,  se tiene que $\left\{Y\in A\right\}\in\mathcal{F}$, donde $\left\{Y\in A\right\}:=\left\{w\in\Omega:Y\left(w\right)\in A\right\}=:Y^{-1}A$.
\end{Def}

\begin{Note}
Tambi\'en se dice que $Y$ est\'a soportado por el espacio de probabilidad $\left(\Omega,\mathcal{F},\prob\right)$ y que $Y$ es un mapeo medible de $\Omega$ en $E$, es decir, es $\mathcal{F}/\mathcal{E}$ medible.
\end{Note}

\begin{Def}
Para cada $i\in \mathbb{I}$ sea $P_{i}$ una medida de probabilidad en un espacio medible $\left(E_{i},\mathcal{E}_{i}\right)$. Se define el espacio producto
$\otimes_{i\in\mathbb{I}}\left(E_{i},\mathcal{E}_{i}\right):=\left(\prod_{i\in\mathbb{I}}E_{i},\otimes_{i\in\mathbb{I}}\mathcal{E}_{i}\right)$, donde $\prod_{i\in\mathbb{I}}E_{i}$ es el producto cartesiano de los $E_{i}$'s, y $\otimes_{i\in\mathbb{I}}\mathcal{E}_{i}$ es la $\sigma$-\'algebra producto, es decir, es la $\sigma$-\'algebra m\'as peque\~na en $\prod_{i\in\mathbb{I}}E_{i}$ que hace al $i$-\'esimo mapeo proyecci\'on en $E_{i}$ medible para toda $i\in\mathbb{I}$ es la $\sigma$-\'algebra inducida por los mapeos proyecci\'on. $$\otimes_{i\in\mathbb{I}}\mathcal{E}_{i}:=\sigma\left\{\left\{y:y_{i}\in A\right\}:i\in\mathbb{I}\textrm{ y }A\in\mathcal{E}_{i}\right\}.$$
\end{Def}

\begin{Def}
Un espacio de probabilidad $\left(\tilde{\Omega},\tilde{\mathcal{F}},\tilde{\prob}\right)$ es una extensi\'on de otro espacio de probabilidad $\left(\Omega,\mathcal{F},\prob\right)$ si $\left(\tilde{\Omega},\tilde{\mathcal{F}},\tilde{\prob}\right)$ soporta un elemento aleatorio $\xi\in\left(\Omega,\mathcal{F}\right)$ que tienen a $\prob$ como distribuci\'on.
\end{Def}

\begin{Teo}
Sea $\mathbb{I}$ un conjunto de \'indices arbitrario. Para cada $i\in\mathbb{I}$ sea $P_{i}$ una medida de probabilidad en un espacio medible $\left(E_{i},\mathcal{E}_{i}\right)$. Entonces existe una \'unica medida de probabilidad $\otimes_{i\in\mathbb{I}}P_{i}$ en $\otimes_{i\in\mathbb{I}}\left(E_{i},\mathcal{E}_{i}\right)$ tal que 

\begin{eqnarray*}
\otimes_{i\in\mathbb{I}}P_{i}\left(y\in\prod_{i\in\mathbb{I}}E_{i}:y_{i}\in A_{i_{1}},\ldots,y_{n}\in A_{i_{n}}\right)=P_{i_{1}}\left(A_{i_{n}}\right)\cdots P_{i_{n}}\left(A_{i_{n}}\right)
\end{eqnarray*}
para todos los enteros $n>0$, toda $i_{1},\ldots,i_{n}\in\mathbb{I}$ y todo $A_{i_{1}}\in\mathcal{E}_{i_{1}},\ldots,A_{i_{n}}\in\mathcal{E}_{i_{n}}$
\end{Teo}

La medida $\otimes_{i\in\mathbb{I}}P_{i}$ es llamada la medida producto y $\otimes_{i\in\mathbb{I}}\left(E_{i},\mathcal{E}_{i},P_{i}\right):=\left(\prod_{i\in\mathbb{I}},E_{i},\otimes_{i\in\mathbb{I}}\mathcal{E}_{i},\otimes_{i\in\mathbb{I}}P_{i}\right)$, es llamado espacio de probabilidad producto.


\begin{Def}
Un espacio medible $\left(E,\mathcal{E}\right)$ es \textit{Polaco} si existe una m\'etrica en $E$ tal que $E$ es completo, es decir cada sucesi\'on de Cauchy converge a un l\'imite en $E$, y \textit{separable}, $E$ tienen un subconjunto denso numerable, y tal que $\mathcal{E}$ es generado por conjuntos abiertos.
\end{Def}


\begin{Def}
Dos espacios medibles $\left(E,\mathcal{E}\right)$ y $\left(G,\mathcal{G}\right)$ son Borel equivalentes \textit{isomorfos} si existe una biyecci\'on $f:E\rightarrow G$ tal que $f$ es $\mathcal{E}/\mathcal{G}$ medible y su inversa $f^{-1}$ es $\mathcal{G}/\mathcal{E}$ medible. La biyecci\'on es una equivalencia de Borel.
\end{Def}

\begin{Def}
Un espacio medible  $\left(E,\mathcal{E}\right)$ es un \textit{espacio est\'andar} si es Borel equivalente a $\left(G,\mathcal{G}\right)$, donde $G$ es un subconjunto de Borel de $\left[0,1\right]$ y $\mathcal{G}$ son los subconjuntos de Borel de $G$.
\end{Def}

\begin{Note}
Cualquier espacio Polaco es un espacio est\'andar.
\end{Note}


\begin{Def}
Un proceso estoc\'astico con conjunto de \'indices $\mathbb{I}$ y espacio de estados $\left(E,\mathcal{E}\right)$ es una familia $Z=\left(\mathbb{Z}_{s}\right)_{s\in\mathbb{I}}$ donde $\mathbb{Z}_{s}$ son elementos aleatorios definidos en un espacio de probabilidad com\'un $\left(\Omega,\mathcal{F},\prob\right)$ y todos toman valores en $\left(E,\mathcal{E}\right)$.
\end{Def}

\begin{Def}
Un proceso estoc\'astico \textit{one-sided contiuous time} (\textbf{PEOSCT}) es un proceso estoc\'astico con conjunto de \'indices $\mathbb{I}=\left[0,\infty\right)$.
\end{Def}


Sea $\left(E^{\mathbb{I}},\mathcal{E}^{\mathbb{I}}\right)$ denota el espacio producto $\left(E^{\mathbb{I}},\mathcal{E}^{\mathbb{I}}\right):=\otimes_{s\in\mathbb{I}}\left(E,\mathcal{E}\right)$. Vamos a considerar $\mathbb{Z}$ como un mapeo aleatorio, es decir, como un elemento aleatorio en $\left(E^{\mathbb{I}},\mathcal{E}^{\mathbb{I}}\right)$ definido por $Z\left(w\right)=\left(Z_{s}\left(w\right)\right)_{s\in\mathbb{I}}$ y $w\in\Omega$.

\begin{Note}
La distribuci\'on de un proceso estoc\'astico $Z$ es la distribuci\'on de $Z$ como un elemento aleatorio en $\left(E^{\mathbb{I}},\mathcal{E}^{\mathbb{I}}\right)$. La distribuci\'on de $Z$ esta determinada de manera \'unica por las distribuciones finito dimensionales.
\end{Note}

\begin{Note}
En particular cuando $Z$ toma valores reales, es decir, $\left(E,\mathcal{E}\right)=\left(\mathbb{R},\mathcal{B}\right)$ las distribuciones finito dimensionales est\'an determinadas por las funciones de distribuci\'on finito dimensionales

\begin{eqnarray}
\prob\left(Z_{t_{1}}\leq x_{1},\ldots,Z_{t_{n}}\leq x_{n}\right),x_{1},\ldots,x_{n}\in\mathbb{R},t_{1},\ldots,t_{n}\in\mathbb{I},n\geq1.
\end{eqnarray}
\end{Note}

\begin{Note}
Para espacios polacos $\left(E,\mathcal{E}\right)$ el Teorema de Consistencia de Kolmogorov asegura que dada una colecci\'on de distribuciones finito dimensionales consistentes, siempre existe un proceso estoc\'astico que posee tales distribuciones finito dimensionales.
\end{Note}


\begin{Def}
Las trayectorias de $Z$ son las realizaciones $Z\left(w\right)$ para $w\in\Omega$ del mapeo aleatorio $Z$.
\end{Def}

\begin{Note}
Algunas restricciones se imponen sobre las trayectorias, por ejemplo que sean continuas por la derecha, o continuas por la derecha con l\'imites por la izquierda, o de manera m\'as general, se pedir\'a que caigan en alg\'un subconjunto $H$ de $E^{\mathbb{I}}$. En este caso es natural considerar a $Z$ como un elemento aleatorio que no est\'a en $\left(E^{\mathbb{I}},\mathcal{E}^{\mathbb{I}}\right)$ sino en $\left(H,\mathcal{H}\right)$, donde $\mathcal{H}$ es la $\sigma$-\'algebra generada por los mapeos proyecci\'on que toman a $z\in H$ a $z_{t}\in E$ para $t\in\mathbb{I}$. A $\mathcal{H}$ se le conoce como la traza de $H$ en $E^{\mathbb{I}}$, es decir,
\begin{eqnarray}
\mathcal{H}:=E^{\mathbb{I}}\cap H:=\left\{A\cap H:A\in E^{\mathbb{I}}\right\}.
\end{eqnarray}
\end{Note}


\begin{Note}
$Z$ tiene trayectorias con valores en $H$ y cada $Z_{t}$ es un mapeo medible de $\left(\Omega,\mathcal{F}\right)$ a $\left(H,\mathcal{H}\right)$. Cuando se considera un espacio de trayectorias en particular $H$, al espacio $\left(H,\mathcal{H}\right)$ se le llama el espacio de trayectorias de $Z$.
\end{Note}

\begin{Note}
La distribuci\'on del proceso estoc\'astico $Z$ con espacio de trayectorias $\left(H,\mathcal{H}\right)$ es la distribuci\'on de $Z$ como  un elemento aleatorio en $\left(H,\mathcal{H}\right)$. La distribuci\'on, nuevemente, est\'a determinada de manera \'unica por las distribuciones finito dimensionales.
\end{Note}


\begin{Def}
Sea $Z$ un PEOSCT  con espacio de estados $\left(E,\mathcal{E}\right)$ y sea $T$ un tiempo aleatorio en $\left[0,\infty\right)$. Por $Z_{T}$ se entiende el mapeo con valores en $E$ definido en $\Omega$ en la manera obvia:
\begin{eqnarray*}
Z_{T}\left(w\right):=Z_{T\left(w\right)}\left(w\right). w\in\Omega.
\end{eqnarray*}
\end{Def}

\begin{Def}
Un PEOSCT $Z$ es conjuntamente medible (\textbf{CM}) si el mapeo que toma $\left(w,t\right)\in\Omega\times\left[0,\infty\right)$ a $Z_{t}\left(w\right)\in E$ es $\mathcal{F}\otimes\mathcal{B}\left[0,\infty\right)/\mathcal{E}$ medible.
\end{Def}

\begin{Note}
Un PEOSCT-CM implica que el proceso es medible, dado que $Z_{T}$ es una composici\'on  de dos mapeos continuos: el primero que toma $w$ en $\left(w,T\left(w\right)\right)$ es $\mathcal{F}/\mathcal{F}\otimes\mathcal{B}\left[0,\infty\right)$ medible, mientras que el segundo toma $\left(w,T\left(w\right)\right)$ en $Z_{T\left(w\right)}\left(w\right)$ es $\mathcal{F}\otimes\mathcal{B}\left[0,\infty\right)/\mathcal{E}$ medible.
\end{Note}


\begin{Def}
Un PEOSCT con espacio de estados $\left(H,\mathcal{H}\right)$ es can\'onicamente conjuntamente medible (\textbf{CCM}) si el mapeo $\left(z,t\right)\in H\times\left[0,\infty\right)$ en $Z_{t}\in E$ es $\mathcal{H}\otimes\mathcal{B}\left[0,\infty\right)/\mathcal{E}$ medible.
\end{Def}

\begin{Note}
Un PEOSCTCCM implica que el proceso es CM, dado que un PECCM $Z$ es un mapeo de $\Omega\times\left[0,\infty\right)$ a $E$, es la composici\'on de dos mapeos medibles: el primero, toma $\left(w,t\right)$ en $\left(Z\left(w\right),t\right)$ es $\mathcal{F}\otimes\mathcal{B}\left[0,\infty\right)/\mathcal{H}\otimes\mathcal{B}\left[0,\infty\right)$ medible, y el segundo que toma $\left(Z\left(w\right),t\right)$  en $Z_{t}\left(w\right)$ es $\mathcal{H}\otimes\mathcal{B}\left[0,\infty\right)/\mathcal{E}$ medible. Por tanto CCM es una condici\'on m\'as fuerte que CM.
\end{Note}

\begin{Def}
Un conjunto de trayectorias $H$ de un PEOSCT $Z$, es internamente shift-invariante (\textbf{ISI}) si 
\begin{eqnarray*}
\left\{\left(z_{t+s}\right)_{s\in\left[0,\infty\right)}:z\in H\right\}=H\textrm{, }t\in\left[0,\infty\right).
\end{eqnarray*}
\end{Def}


\begin{Def}
Dado un PEOSCTISI, se define el mapeo-shift $\theta_{t}$, $t\in\left[0,\infty\right)$, de $H$ a $H$ por 
\begin{eqnarray*}
\theta_{t}z=\left(z_{t+s}\right)_{s\in\left[0,\infty\right)}\textrm{, }z\in H.
\end{eqnarray*}
\end{Def}

\begin{Def}
Se dice que un proceso $Z$ es shift-medible (\textbf{SM}) si $Z$ tiene un conjunto de trayectorias $H$ que es ISI y adem\'as el mapeo que toma $\left(z,t\right)\in H\times\left[0,\infty\right)$ en $\theta_{t}z\in H$ es $\mathcal{H}\otimes\mathcal{B}\left[0,\infty\right)/\mathcal{H}$ medible.
\end{Def}

\begin{Note}
Un proceso estoc\'astico con conjunto de trayectorias $H$ ISI es shift-medible si y s\'olo si es CCM
\end{Note}

\begin{Note}
\begin{itemize}
\item Dado el espacio polaco $\left(E,\mathcal{E}\right)$ se tiene el  conjunto de trayectorias $D_{E}\left[0,\infty\right)$ que es ISI, entonces cumpe con ser CCM.

\item Si $G$ es abierto, podemos cubrirlo por bolas abiertas cuay cerradura este contenida en $G$, y como $G$ es segundo numerable como subespacio de $E$, lo podemos cubrir por una cantidad numerable de bolas abiertas.

\end{itemize}
\end{Note}


\begin{Note}
Los procesos estoc\'asticos $Z$ a tiempo discreto con espacio de estados polaco, tambi\'en tiene un espacio de trayectorias polaco y por tanto tiene distribuciones condicionales regulares.
\end{Note}

\begin{Teo}
El producto numerable de espacios polacos es polaco.
\end{Teo}


\begin{Def}
Sea $\left(\Omega,\mathcal{F},\prob\right)$ espacio de probabilidad que soporta al proceso $Z=\left(Z_{s}\right)_{s\in\left[0,\infty\right)}$ y $S=\left(S_{k}\right)_{0}^{\infty}$ donde $Z$ es un PEOSCTM con espacio de estados $\left(E,\mathcal{E}\right)$  y espacio de trayectorias $\left(H,\mathcal{H}\right)$  y adem\'as $S$ es una sucesi\'on de tiempos aleatorios one-sided que satisfacen la condici\'on $0\leq S_{0}<S_{1}<\cdots\rightarrow\infty$. Considerando $S$ como un mapeo medible de $\left(\Omega,\mathcal{F}\right)$ al espacio sucesi\'on $\left(L,\mathcal{L}\right)$, donde 
\begin{eqnarray*}
L=\left\{\left(s_{k}\right)_{0}^{\infty}\in\left[0,\infty\right)^{\left\{0,1,\ldots\right\}}:s_{0}<s_{1}<\cdots\rightarrow\infty\right\},
\end{eqnarray*}
donde $\mathcal{L}$ son los subconjuntos de Borel de $L$, es decir, $\mathcal{L}=L\cap\mathcal{B}^{\left\{0,1,\ldots\right\}}$.

As\'i el par $\left(Z,S\right)$ es un mapeo medible de  $\left(\Omega,\mathcal{F}\right)$ en $\left(H\times L,\mathcal{H}\otimes\mathcal{L}\right)$. El par $\mathcal{H}\otimes\mathcal{L}^{+}$ denotar\'a la clase de todas las funciones medibles de $\left(H\times L,\mathcal{H}\otimes\mathcal{L}\right)$ en $\left(\left[0,\infty\right),\mathcal{B}\left[0,\infty\right)\right)$.
\end{Def}


\begin{Def}
Sea $\theta_{t}$ el mapeo-shift conjunto de $H\times L$ en $H\times L$ dado por
\begin{eqnarray*}
\theta_{t}\left(z,\left(s_{k}\right)_{0}^{\infty}\right)=\theta_{t}\left(z,\left(s_{n_{t-}+k}-t\right)_{0}^{\infty}\right)
\end{eqnarray*}
donde 
$n_{t-}=inf\left\{n\geq1:s_{n}\geq t\right\}$.
\end{Def}

\begin{Note}
Con la finalidad de poder realizar los shift's sin complicaciones de medibilidad, se supondr\'a que $Z$ es shit-medible, es decir, el conjunto de trayectorias $H$ es invariante bajo shifts del tiempo y el mapeo que toma $\left(z,t\right)\in H\times\left[0,\infty\right)$ en $z_{t}\in E$ es $\mathcal{H}\otimes\mathcal{B}\left[0,\infty\right)/\mathcal{E}$ medible.
\end{Note}

\begin{Def}
Dado un proceso \textbf{PEOSSM} (Proceso Estoc\'astico One Side Shift Medible) $Z$, se dice regenerativo cl\'asico con tiempos de regeneraci\'on $S$ si 

\begin{eqnarray*}
\theta_{S_{n}}\left(Z,S\right)=\left(Z^{0},S^{0}\right),n\geq0
\end{eqnarray*}
y adem\'as $\theta_{S_{n}}\left(Z,S\right)$ es independiente de $\left(\left(Z_{s}\right)s\in\left[0,S_{n}\right),S_{0},\ldots,S_{n}\right)$
Si lo anterior se cumple, al par $\left(Z,S\right)$ se le llama regenerativo cl\'asico.
\end{Def}

\begin{Note}
Si el par $\left(Z,S\right)$ es regenerativo cl\'asico, entonces las longitudes de los ciclos $X_{1},X_{2},\ldots,$ son i.i.d. e independientes de la longitud del retraso $S_{0}$, es decir, $S$ es un proceso de renovaci\'on. Las longitudes de los ciclos tambi\'en son llamados tiempos de inter-regeneraci\'on y tiempos de ocurrencia.

\end{Note}

\begin{Teo}
Sup\'ongase que el par $\left(Z,S\right)$ es regenerativo cl\'asico con $\esp\left[X_{1}\right]<\infty$. Entonces $\left(Z^{*},S^{*}\right)$ en el teorema 2.1 es una versi\'on estacionaria de $\left(Z,S\right)$. Adem\'as, si $X_{1}$ es lattice con span $d$, entonces $\left(Z^{**},S^{**}\right)$ en el teorema 2.2 es una versi\'on periodicamente estacionaria de $\left(Z,S\right)$ con periodo $d$.

\end{Teo}

\begin{Def}
Una variable aleatoria $X_{1}$ es \textit{spread out} si existe una $n\geq1$ y una  funci\'on $f\in\mathcal{B}^{+}$ tal que $\int_{\rea}f\left(x\right)dx>0$ con $X_{2},X_{3},\ldots,X_{n}$ copias i.i.d  de $X_{1}$, $$\prob\left(X_{1}+\cdots+X_{n}\in B\right)\geq\int_{B}f\left(x\right)dx$$ para $B\in\mathcal{B}$.

\end{Def}



\begin{Def}
Dado un proceso estoc\'astico $Z$ se le llama \textit{wide-sense regenerative} (\textbf{WSR}) con tiempos de regeneraci\'on $S$ si $\theta_{S_{n}}\left(Z,S\right)=\left(Z^{0},S^{0}\right)$ para $n\geq0$ en distribuci\'on y $\theta_{S_{n}}\left(Z,S\right)$ es independiente de $\left(S_{0},S_{1},\ldots,S_{n}\right)$ para $n\geq0$.
Se dice que el par $\left(Z,S\right)$ es WSR si lo anterior se cumple.
\end{Def}


\begin{Note}
\begin{itemize}
\item El proceso de trayectorias $\left(\theta_{s}Z\right)_{s\in\left[0,\infty\right)}$ es WSR con tiempos de regeneraci\'on $S$ pero no es regenerativo cl\'asico.

\item Si $Z$ es cualquier proceso estacionario y $S$ es un proceso de renovaci\'on que es independiente de $Z$, entonces $\left(Z,S\right)$ es WSR pero en general no es regenerativo cl\'asico

\end{itemize}

\end{Note}


\begin{Note}
Para cualquier proceso estoc\'astico $Z$, el proceso de trayectorias $\left(\theta_{s}Z\right)_{s\in\left[0,\infty\right)}$ es siempre un proceso de Markov.
\end{Note}



\begin{Teo}
Supongase que el par $\left(Z,S\right)$ es WSR con $\esp\left[X_{1}\right]<\infty$. Entonces $\left(Z^{*},S^{*}\right)$ en el teorema 2.1 es una versi\'on estacionaria de 
$\left(Z,S\right)$.
\end{Teo}


\begin{Teo}
Supongase que $\left(Z,S\right)$ es cycle-stationary con $\esp\left[X_{1}\right]<\infty$. Sea $U$ distribuida uniformemente en $\left[0,1\right)$ e independiente de $\left(Z^{0},S^{0}\right)$ y sea $\prob^{*}$ la medida de probabilidad en $\left(\Omega,\prob\right)$ definida por $$d\prob^{*}=\frac{X_{1}}{\esp\left[X_{1}\right]}d\prob$$. Sea $\left(Z^{*},S^{*}\right)$ con distribuci\'on $\prob^{*}\left(\theta_{UX_{1}}\left(Z^{0},S^{0}\right)\in\cdot\right)$. Entonces $\left(Z^{}*,S^{*}\right)$ es estacionario,
\begin{eqnarray*}
\esp\left[f\left(Z^{*},S^{*}\right)\right]=\esp\left[\int_{0}^{X_{1}}f\left(\theta_{s}\left(Z^{0},S^{0}\right)\right)ds\right]/\esp\left[X_{1}\right]
\end{eqnarray*}
$f\in\mathcal{H}\otimes\mathcal{L}^{+}$, and $S_{0}^{*}$ es continuo con funci\'on distribuci\'on $G_{\infty}$ definida por $$G_{\infty}\left(x\right):=\frac{\esp\left[X_{1}\right]\wedge x}{\esp\left[X_{1}\right]}$$ para $x\geq0$ y densidad $\prob\left[X_{1}>x\right]/\esp\left[X_{1}\right]$, con $x\geq0$.

\end{Teo}


\begin{Teo}
Sea $Z$ un Proceso Estoc\'astico un lado shift-medible \textit{one-sided shift-measurable stochastic process}, (PEOSSM),
y $S_{0}$ y $S_{1}$ tiempos aleatorios tales que $0\leq S_{0}<S_{1}$ y
\begin{equation}
\theta_{S_{1}}Z=\theta_{S_{0}}Z\textrm{ en distribuci\'on}.
\end{equation}

Entonces el espacio de probabilidad subyacente $\left(\Omega,\mathcal{F},\prob\right)$ puede extenderse para soportar una sucesi\'on de tiempos aleatorios $S$ tales que

\begin{eqnarray}
\theta_{S_{n}}\left(Z,S\right)=\left(Z^{0},S^{0}\right),n\geq0,\textrm{ en distribuci\'on},\\
\left(Z,S_{0},S_{1}\right)\textrm{ depende de }\left(X_{2},X_{3},\ldots\right)\textrm{ solamente a traves de }\theta_{S_{1}}Z.
\end{eqnarray}
\end{Teo}


%___________________________________________________________
%
\section{Existencia de Tiempos de Regeneraci\'on}
%___________________________________________________________
%




\begin{Def}
Un elemento aleatorio en un espacio medible $\left(E,\mathcal{E}\right)$ en un espacio de probabilidad $\left(\Omega,\mathcal{F},\prob\right)$ a $\left(E,\mathcal{E}\right)$, es decir,
para $A\in \mathcal{E}$,  se tiene que $\left\{Y\in A\right\}\in\mathcal{F}$, donde $\left\{Y\in A\right\}:=\left\{w\in\Omega:Y\left(w\right)\in A\right\}=:Y^{-1}A$.
\end{Def}

\begin{Note}
Tambi\'en se dice que $Y$ est\'a soportado por el espacio de probabilidad $\left(\Omega,\mathcal{F},\prob\right)$ y que $Y$ es un mapeo medible de $\Omega$ en $E$, es decir, es $\mathcal{F}/\mathcal{E}$ medible.
\end{Note}

\begin{Def}
Para cada $i\in \mathbb{I}$ sea $P_{i}$ una medida de probabilidad en un espacio medible $\left(E_{i},\mathcal{E}_{i}\right)$. Se define el espacio producto
$\otimes_{i\in\mathbb{I}}\left(E_{i},\mathcal{E}_{i}\right):=\left(\prod_{i\in\mathbb{I}}E_{i},\otimes_{i\in\mathbb{I}}\mathcal{E}_{i}\right)$, donde $\prod_{i\in\mathbb{I}}E_{i}$ es el producto cartesiano de los $E_{i}$'s, y $\otimes_{i\in\mathbb{I}}\mathcal{E}_{i}$ es la $\sigma$-\'algebra producto, es decir, es la $\sigma$-\'algebra m\'as peque\~na en $\prod_{i\in\mathbb{I}}E_{i}$ que hace al $i$-\'esimo mapeo proyecci\'on en $E_{i}$ medible para toda $i\in\mathbb{I}$ es la $\sigma$-\'algebra inducida por los mapeos proyecci\'on. $$\otimes_{i\in\mathbb{I}}\mathcal{E}_{i}:=\sigma\left\{\left\{y:y_{i}\in A\right\}:i\in\mathbb{I}\textrm{ y }A\in\mathcal{E}_{i}\right\}.$$
\end{Def}

\begin{Def}
Un espacio de probabilidad $\left(\tilde{\Omega},\tilde{\mathcal{F}},\tilde{\prob}\right)$ es una extensi\'on de otro espacio de probabilidad $\left(\Omega,\mathcal{F},\prob\right)$ si $\left(\tilde{\Omega},\tilde{\mathcal{F}},\tilde{\prob}\right)$ soporta un elemento aleatorio $\xi\in\left(\Omega,\mathcal{F}\right)$ que tienen a $\prob$ como distribuci\'on.
\end{Def}

\begin{Teo}
Sea $\mathbb{I}$ un conjunto de \'indices arbitrario. Para cada $i\in\mathbb{I}$ sea $P_{i}$ una medida de probabilidad en un espacio medible $\left(E_{i},\mathcal{E}_{i}\right)$. Entonces existe una \'unica medida de probabilidad $\otimes_{i\in\mathbb{I}}P_{i}$ en $\otimes_{i\in\mathbb{I}}\left(E_{i},\mathcal{E}_{i}\right)$ tal que 

\begin{eqnarray*}
\otimes_{i\in\mathbb{I}}P_{i}\left(y\in\prod_{i\in\mathbb{I}}E_{i}:y_{i}\in A_{i_{1}},\ldots,y_{n}\in A_{i_{n}}\right)=P_{i_{1}}\left(A_{i_{n}}\right)\cdots P_{i_{n}}\left(A_{i_{n}}\right)
\end{eqnarray*}
para todos los enteros $n>0$, toda $i_{1},\ldots,i_{n}\in\mathbb{I}$ y todo $A_{i_{1}}\in\mathcal{E}_{i_{1}},\ldots,A_{i_{n}}\in\mathcal{E}_{i_{n}}$
\end{Teo}

La medida $\otimes_{i\in\mathbb{I}}P_{i}$ es llamada la medida producto y $\otimes_{i\in\mathbb{I}}\left(E_{i},\mathcal{E}_{i},P_{i}\right):=\left(\prod_{i\in\mathbb{I}},E_{i},\otimes_{i\in\mathbb{I}}\mathcal{E}_{i},\otimes_{i\in\mathbb{I}}P_{i}\right)$, es llamado espacio de probabilidad producto.


\begin{Def}
Un espacio medible $\left(E,\mathcal{E}\right)$ es \textit{Polaco} si existe una m\'etrica en $E$ tal que $E$ es completo, es decir cada sucesi\'on de Cauchy converge a un l\'imite en $E$, y \textit{separable}, $E$ tienen un subconjunto denso numerable, y tal que $\mathcal{E}$ es generado por conjuntos abiertos.
\end{Def}


\begin{Def}
Dos espacios medibles $\left(E,\mathcal{E}\right)$ y $\left(G,\mathcal{G}\right)$ son Borel equivalentes \textit{isomorfos} si existe una biyecci\'on $f:E\rightarrow G$ tal que $f$ es $\mathcal{E}/\mathcal{G}$ medible y su inversa $f^{-1}$ es $\mathcal{G}/\mathcal{E}$ medible. La biyecci\'on es una equivalencia de Borel.
\end{Def}

\begin{Def}
Un espacio medible  $\left(E,\mathcal{E}\right)$ es un \textit{espacio est\'andar} si es Borel equivalente a $\left(G,\mathcal{G}\right)$, donde $G$ es un subconjunto de Borel de $\left[0,1\right]$ y $\mathcal{G}$ son los subconjuntos de Borel de $G$.
\end{Def}

\begin{Note}
Cualquier espacio Polaco es un espacio est\'andar.
\end{Note}


\begin{Def}
Un proceso estoc\'astico con conjunto de \'indices $\mathbb{I}$ y espacio de estados $\left(E,\mathcal{E}\right)$ es una familia $Z=\left(\mathbb{Z}_{s}\right)_{s\in\mathbb{I}}$ donde $\mathbb{Z}_{s}$ son elementos aleatorios definidos en un espacio de probabilidad com\'un $\left(\Omega,\mathcal{F},\prob\right)$ y todos toman valores en $\left(E,\mathcal{E}\right)$.
\end{Def}

\begin{Def}
Un proceso estoc\'astico \textit{one-sided contiuous time} (\textbf{PEOSCT}) es un proceso estoc\'astico con conjunto de \'indices $\mathbb{I}=\left[0,\infty\right)$.
\end{Def}


Sea $\left(E^{\mathbb{I}},\mathcal{E}^{\mathbb{I}}\right)$ denota el espacio producto $\left(E^{\mathbb{I}},\mathcal{E}^{\mathbb{I}}\right):=\otimes_{s\in\mathbb{I}}\left(E,\mathcal{E}\right)$. Vamos a considerar $\mathbb{Z}$ como un mapeo aleatorio, es decir, como un elemento aleatorio en $\left(E^{\mathbb{I}},\mathcal{E}^{\mathbb{I}}\right)$ definido por $Z\left(w\right)=\left(Z_{s}\left(w\right)\right)_{s\in\mathbb{I}}$ y $w\in\Omega$.

\begin{Note}
La distribuci\'on de un proceso estoc\'astico $Z$ es la distribuci\'on de $Z$ como un elemento aleatorio en $\left(E^{\mathbb{I}},\mathcal{E}^{\mathbb{I}}\right)$. La distribuci\'on de $Z$ esta determinada de manera \'unica por las distribuciones finito dimensionales.
\end{Note}

\begin{Note}
En particular cuando $Z$ toma valores reales, es decir, $\left(E,\mathcal{E}\right)=\left(\mathbb{R},\mathcal{B}\right)$ las distribuciones finito dimensionales est\'an determinadas por las funciones de distribuci\'on finito dimensionales

\begin{eqnarray}
\prob\left(Z_{t_{1}}\leq x_{1},\ldots,Z_{t_{n}}\leq x_{n}\right),x_{1},\ldots,x_{n}\in\mathbb{R},t_{1},\ldots,t_{n}\in\mathbb{I},n\geq1.
\end{eqnarray}
\end{Note}

\begin{Note}
Para espacios polacos $\left(E,\mathcal{E}\right)$ el Teorema de Consistencia de Kolmogorov asegura que dada una colecci\'on de distribuciones finito dimensionales consistentes, siempre existe un proceso estoc\'astico que posee tales distribuciones finito dimensionales.
\end{Note}


\begin{Def}
Las trayectorias de $Z$ son las realizaciones $Z\left(w\right)$ para $w\in\Omega$ del mapeo aleatorio $Z$.
\end{Def}

\begin{Note}
Algunas restricciones se imponen sobre las trayectorias, por ejemplo que sean continuas por la derecha, o continuas por la derecha con l\'imites por la izquierda, o de manera m\'as general, se pedir\'a que caigan en alg\'un subconjunto $H$ de $E^{\mathbb{I}}$. En este caso es natural considerar a $Z$ como un elemento aleatorio que no est\'a en $\left(E^{\mathbb{I}},\mathcal{E}^{\mathbb{I}}\right)$ sino en $\left(H,\mathcal{H}\right)$, donde $\mathcal{H}$ es la $\sigma$-\'algebra generada por los mapeos proyecci\'on que toman a $z\in H$ a $z_{t}\in E$ para $t\in\mathbb{I}$. A $\mathcal{H}$ se le conoce como la traza de $H$ en $E^{\mathbb{I}}$, es decir,
\begin{eqnarray}
\mathcal{H}:=E^{\mathbb{I}}\cap H:=\left\{A\cap H:A\in E^{\mathbb{I}}\right\}.
\end{eqnarray}
\end{Note}


\begin{Note}
$Z$ tiene trayectorias con valores en $H$ y cada $Z_{t}$ es un mapeo medible de $\left(\Omega,\mathcal{F}\right)$ a $\left(H,\mathcal{H}\right)$. Cuando se considera un espacio de trayectorias en particular $H$, al espacio $\left(H,\mathcal{H}\right)$ se le llama el espacio de trayectorias de $Z$.
\end{Note}

\begin{Note}
La distribuci\'on del proceso estoc\'astico $Z$ con espacio de trayectorias $\left(H,\mathcal{H}\right)$ es la distribuci\'on de $Z$ como  un elemento aleatorio en $\left(H,\mathcal{H}\right)$. La distribuci\'on, nuevemente, est\'a determinada de manera \'unica por las distribuciones finito dimensionales.
\end{Note}


\begin{Def}
Sea $Z$ un PEOSCT  con espacio de estados $\left(E,\mathcal{E}\right)$ y sea $T$ un tiempo aleatorio en $\left[0,\infty\right)$. Por $Z_{T}$ se entiende el mapeo con valores en $E$ definido en $\Omega$ en la manera obvia:
\begin{eqnarray*}
Z_{T}\left(w\right):=Z_{T\left(w\right)}\left(w\right). w\in\Omega.
\end{eqnarray*}
\end{Def}

\begin{Def}
Un PEOSCT $Z$ es conjuntamente medible (\textbf{CM}) si el mapeo que toma $\left(w,t\right)\in\Omega\times\left[0,\infty\right)$ a $Z_{t}\left(w\right)\in E$ es $\mathcal{F}\otimes\mathcal{B}\left[0,\infty\right)/\mathcal{E}$ medible.
\end{Def}

\begin{Note}
Un PEOSCT-CM implica que el proceso es medible, dado que $Z_{T}$ es una composici\'on  de dos mapeos continuos: el primero que toma $w$ en $\left(w,T\left(w\right)\right)$ es $\mathcal{F}/\mathcal{F}\otimes\mathcal{B}\left[0,\infty\right)$ medible, mientras que el segundo toma $\left(w,T\left(w\right)\right)$ en $Z_{T\left(w\right)}\left(w\right)$ es $\mathcal{F}\otimes\mathcal{B}\left[0,\infty\right)/\mathcal{E}$ medible.
\end{Note}


\begin{Def}
Un PEOSCT con espacio de estados $\left(H,\mathcal{H}\right)$ es can\'onicamente conjuntamente medible (\textbf{CCM}) si el mapeo $\left(z,t\right)\in H\times\left[0,\infty\right)$ en $Z_{t}\in E$ es $\mathcal{H}\otimes\mathcal{B}\left[0,\infty\right)/\mathcal{E}$ medible.
\end{Def}

\begin{Note}
Un PEOSCTCCM implica que el proceso es CM, dado que un PECCM $Z$ es un mapeo de $\Omega\times\left[0,\infty\right)$ a $E$, es la composici\'on de dos mapeos medibles: el primero, toma $\left(w,t\right)$ en $\left(Z\left(w\right),t\right)$ es $\mathcal{F}\otimes\mathcal{B}\left[0,\infty\right)/\mathcal{H}\otimes\mathcal{B}\left[0,\infty\right)$ medible, y el segundo que toma $\left(Z\left(w\right),t\right)$  en $Z_{t}\left(w\right)$ es $\mathcal{H}\otimes\mathcal{B}\left[0,\infty\right)/\mathcal{E}$ medible. Por tanto CCM es una condici\'on m\'as fuerte que CM.
\end{Note}

\begin{Def}
Un conjunto de trayectorias $H$ de un PEOSCT $Z$, es internamente shift-invariante (\textbf{ISI}) si 
\begin{eqnarray*}
\left\{\left(z_{t+s}\right)_{s\in\left[0,\infty\right)}:z\in H\right\}=H\textrm{, }t\in\left[0,\infty\right).
\end{eqnarray*}
\end{Def}


\begin{Def}
Dado un PEOSCTISI, se define el mapeo-shift $\theta_{t}$, $t\in\left[0,\infty\right)$, de $H$ a $H$ por 
\begin{eqnarray*}
\theta_{t}z=\left(z_{t+s}\right)_{s\in\left[0,\infty\right)}\textrm{, }z\in H.
\end{eqnarray*}
\end{Def}

\begin{Def}
Se dice que un proceso $Z$ es shift-medible (\textbf{SM}) si $Z$ tiene un conjunto de trayectorias $H$ que es ISI y adem\'as el mapeo que toma $\left(z,t\right)\in H\times\left[0,\infty\right)$ en $\theta_{t}z\in H$ es $\mathcal{H}\otimes\mathcal{B}\left[0,\infty\right)/\mathcal{H}$ medible.
\end{Def}

\begin{Note}
Un proceso estoc\'astico con conjunto de trayectorias $H$ ISI es shift-medible si y s\'olo si es CCM
\end{Note}

\begin{Note}
\begin{itemize}
\item Dado el espacio polaco $\left(E,\mathcal{E}\right)$ se tiene el  conjunto de trayectorias $D_{E}\left[0,\infty\right)$ que es ISI, entonces cumpe con ser CCM.

\item Si $G$ es abierto, podemos cubrirlo por bolas abiertas cuay cerradura este contenida en $G$, y como $G$ es segundo numerable como subespacio de $E$, lo podemos cubrir por una cantidad numerable de bolas abiertas.

\end{itemize}
\end{Note}


\begin{Note}
Los procesos estoc\'asticos $Z$ a tiempo discreto con espacio de estados polaco, tambi\'en tiene un espacio de trayectorias polaco y por tanto tiene distribuciones condicionales regulares.
\end{Note}

\begin{Teo}
El producto numerable de espacios polacos es polaco.
\end{Teo}


\begin{Def}
Sea $\left(\Omega,\mathcal{F},\prob\right)$ espacio de probabilidad que soporta al proceso $Z=\left(Z_{s}\right)_{s\in\left[0,\infty\right)}$ y $S=\left(S_{k}\right)_{0}^{\infty}$ donde $Z$ es un PEOSCTM con espacio de estados $\left(E,\mathcal{E}\right)$  y espacio de trayectorias $\left(H,\mathcal{H}\right)$  y adem\'as $S$ es una sucesi\'on de tiempos aleatorios one-sided que satisfacen la condici\'on $0\leq S_{0}<S_{1}<\cdots\rightarrow\infty$. Considerando $S$ como un mapeo medible de $\left(\Omega,\mathcal{F}\right)$ al espacio sucesi\'on $\left(L,\mathcal{L}\right)$, donde 
\begin{eqnarray*}
L=\left\{\left(s_{k}\right)_{0}^{\infty}\in\left[0,\infty\right)^{\left\{0,1,\ldots\right\}}:s_{0}<s_{1}<\cdots\rightarrow\infty\right\},
\end{eqnarray*}
donde $\mathcal{L}$ son los subconjuntos de Borel de $L$, es decir, $\mathcal{L}=L\cap\mathcal{B}^{\left\{0,1,\ldots\right\}}$.

As\'i el par $\left(Z,S\right)$ es un mapeo medible de  $\left(\Omega,\mathcal{F}\right)$ en $\left(H\times L,\mathcal{H}\otimes\mathcal{L}\right)$. El par $\mathcal{H}\otimes\mathcal{L}^{+}$ denotar\'a la clase de todas las funciones medibles de $\left(H\times L,\mathcal{H}\otimes\mathcal{L}\right)$ en $\left(\left[0,\infty\right),\mathcal{B}\left[0,\infty\right)\right)$.
\end{Def}


\begin{Def}
Sea $\theta_{t}$ el mapeo-shift conjunto de $H\times L$ en $H\times L$ dado por
\begin{eqnarray*}
\theta_{t}\left(z,\left(s_{k}\right)_{0}^{\infty}\right)=\theta_{t}\left(z,\left(s_{n_{t-}+k}-t\right)_{0}^{\infty}\right)
\end{eqnarray*}
donde 
$n_{t-}=inf\left\{n\geq1:s_{n}\geq t\right\}$.
\end{Def}

\begin{Note}
Con la finalidad de poder realizar los shift's sin complicaciones de medibilidad, se supondr\'a que $Z$ es shit-medible, es decir, el conjunto de trayectorias $H$ es invariante bajo shifts del tiempo y el mapeo que toma $\left(z,t\right)\in H\times\left[0,\infty\right)$ en $z_{t}\in E$ es $\mathcal{H}\otimes\mathcal{B}\left[0,\infty\right)/\mathcal{E}$ medible.
\end{Note}

\begin{Def}
Dado un proceso \textbf{PEOSSM} (Proceso Estoc\'astico One Side Shift Medible) $Z$, se dice regenerativo cl\'asico con tiempos de regeneraci\'on $S$ si 

\begin{eqnarray*}
\theta_{S_{n}}\left(Z,S\right)=\left(Z^{0},S^{0}\right),n\geq0
\end{eqnarray*}
y adem\'as $\theta_{S_{n}}\left(Z,S\right)$ es independiente de $\left(\left(Z_{s}\right)s\in\left[0,S_{n}\right),S_{0},\ldots,S_{n}\right)$
Si lo anterior se cumple, al par $\left(Z,S\right)$ se le llama regenerativo cl\'asico.
\end{Def}

\begin{Note}
Si el par $\left(Z,S\right)$ es regenerativo cl\'asico, entonces las longitudes de los ciclos $X_{1},X_{2},\ldots,$ son i.i.d. e independientes de la longitud del retraso $S_{0}$, es decir, $S$ es un proceso de renovaci\'on. Las longitudes de los ciclos tambi\'en son llamados tiempos de inter-regeneraci\'on y tiempos de ocurrencia.

\end{Note}

\begin{Teo}
Sup\'ongase que el par $\left(Z,S\right)$ es regenerativo cl\'asico con $\esp\left[X_{1}\right]<\infty$. Entonces $\left(Z^{*},S^{*}\right)$ en el teorema 2.1 es una versi\'on estacionaria de $\left(Z,S\right)$. Adem\'as, si $X_{1}$ es lattice con span $d$, entonces $\left(Z^{**},S^{**}\right)$ en el teorema 2.2 es una versi\'on periodicamente estacionaria de $\left(Z,S\right)$ con periodo $d$.

\end{Teo}

\begin{Def}
Una variable aleatoria $X_{1}$ es \textit{spread out} si existe una $n\geq1$ y una  funci\'on $f\in\mathcal{B}^{+}$ tal que $\int_{\rea}f\left(x\right)dx>0$ con $X_{2},X_{3},\ldots,X_{n}$ copias i.i.d  de $X_{1}$, $$\prob\left(X_{1}+\cdots+X_{n}\in B\right)\geq\int_{B}f\left(x\right)dx$$ para $B\in\mathcal{B}$.

\end{Def}



\begin{Def}
Dado un proceso estoc\'astico $Z$ se le llama \textit{wide-sense regenerative} (\textbf{WSR}) con tiempos de regeneraci\'on $S$ si $\theta_{S_{n}}\left(Z,S\right)=\left(Z^{0},S^{0}\right)$ para $n\geq0$ en distribuci\'on y $\theta_{S_{n}}\left(Z,S\right)$ es independiente de $\left(S_{0},S_{1},\ldots,S_{n}\right)$ para $n\geq0$.
Se dice que el par $\left(Z,S\right)$ es WSR si lo anterior se cumple.
\end{Def}


\begin{Note}
\begin{itemize}
\item El proceso de trayectorias $\left(\theta_{s}Z\right)_{s\in\left[0,\infty\right)}$ es WSR con tiempos de regeneraci\'on $S$ pero no es regenerativo cl\'asico.

\item Si $Z$ es cualquier proceso estacionario y $S$ es un proceso de renovaci\'on que es independiente de $Z$, entonces $\left(Z,S\right)$ es WSR pero en general no es regenerativo cl\'asico

\end{itemize}

\end{Note}


\begin{Note}
Para cualquier proceso estoc\'astico $Z$, el proceso de trayectorias $\left(\theta_{s}Z\right)_{s\in\left[0,\infty\right)}$ es siempre un proceso de Markov.
\end{Note}



\begin{Teo}
Supongase que el par $\left(Z,S\right)$ es WSR con $\esp\left[X_{1}\right]<\infty$. Entonces $\left(Z^{*},S^{*}\right)$ en el teorema 2.1 es una versi\'on estacionaria de 
$\left(Z,S\right)$.
\end{Teo}


\begin{Teo}
Supongase que $\left(Z,S\right)$ es cycle-stationary con $\esp\left[X_{1}\right]<\infty$. Sea $U$ distribuida uniformemente en $\left[0,1\right)$ e independiente de $\left(Z^{0},S^{0}\right)$ y sea $\prob^{*}$ la medida de probabilidad en $\left(\Omega,\prob\right)$ definida por $$d\prob^{*}=\frac{X_{1}}{\esp\left[X_{1}\right]}d\prob$$. Sea $\left(Z^{*},S^{*}\right)$ con distribuci\'on $\prob^{*}\left(\theta_{UX_{1}}\left(Z^{0},S^{0}\right)\in\cdot\right)$. Entonces $\left(Z^{}*,S^{*}\right)$ es estacionario,
\begin{eqnarray*}
\esp\left[f\left(Z^{*},S^{*}\right)\right]=\esp\left[\int_{0}^{X_{1}}f\left(\theta_{s}\left(Z^{0},S^{0}\right)\right)ds\right]/\esp\left[X_{1}\right]
\end{eqnarray*}
$f\in\mathcal{H}\otimes\mathcal{L}^{+}$, and $S_{0}^{*}$ es continuo con funci\'on distribuci\'on $G_{\infty}$ definida por $$G_{\infty}\left(x\right):=\frac{\esp\left[X_{1}\right]\wedge x}{\esp\left[X_{1}\right]}$$ para $x\geq0$ y densidad $\prob\left[X_{1}>x\right]/\esp\left[X_{1}\right]$, con $x\geq0$.

\end{Teo}


\begin{Teo}
Sea $Z$ un Proceso Estoc\'astico un lado shift-medible \textit{one-sided shift-measurable stochastic process}, (PEOSSM),
y $S_{0}$ y $S_{1}$ tiempos aleatorios tales que $0\leq S_{0}<S_{1}$ y
\begin{equation}
\theta_{S_{1}}Z=\theta_{S_{0}}Z\textrm{ en distribuci\'on}.
\end{equation}

Entonces el espacio de probabilidad subyacente $\left(\Omega,\mathcal{F},\prob\right)$ puede extenderse para soportar una sucesi\'on de tiempos aleatorios $S$ tales que

\begin{eqnarray}
\theta_{S_{n}}\left(Z,S\right)=\left(Z^{0},S^{0}\right),n\geq0,\textrm{ en distribuci\'on},\\
\left(Z,S_{0},S_{1}\right)\textrm{ depende de }\left(X_{2},X_{3},\ldots\right)\textrm{ solamente a traves de }\theta_{S_{1}}Z.
\end{eqnarray}
\end{Teo}


%________________________________________________________________________
\subsection{Procesos Regenerativos Sigman, Thorisson y Wolff \cite{Sigman1}}
%________________________________________________________________________


\begin{Def}[Definici\'on Cl\'asica]
Un proceso estoc\'astico $X=\left\{X\left(t\right):t\geq0\right\}$ es llamado regenerativo is existe una variable aleatoria $R_{1}>0$ tal que
\begin{itemize}
\item[i)] $\left\{X\left(t+R_{1}\right):t\geq0\right\}$ es independiente de $\left\{\left\{X\left(t\right):t<R_{1}\right\},\right\}$
\item[ii)] $\left\{X\left(t+R_{1}\right):t\geq0\right\}$ es estoc\'asticamente equivalente a $\left\{X\left(t\right):t>0\right\}$
\end{itemize}

Llamamos a $R_{1}$ tiempo de regeneraci\'on, y decimos que $X$ se regenera en este punto.
\end{Def}

$\left\{X\left(t+R_{1}\right)\right\}$ es regenerativo con tiempo de regeneraci\'on $R_{2}$, independiente de $R_{1}$ pero con la misma distribuci\'on que $R_{1}$. Procediendo de esta manera se obtiene una secuencia de variables aleatorias independientes e id\'enticamente distribuidas $\left\{R_{n}\right\}$ llamados longitudes de ciclo. Si definimos a $Z_{k}\equiv R_{1}+R_{2}+\cdots+R_{k}$, se tiene un proceso de renovaci\'on llamado proceso de renovaci\'on encajado para $X$.


\begin{Note}
La existencia de un primer tiempo de regeneraci\'on, $R_{1}$, implica la existencia de una sucesi\'on completa de estos tiempos $R_{1},R_{2}\ldots,$ que satisfacen la propiedad deseada \cite{Sigman2}.
\end{Note}


\begin{Note} Para la cola $GI/GI/1$ los usuarios arriban con tiempos $t_{n}$ y son atendidos con tiempos de servicio $S_{n}$, los tiempos de arribo forman un proceso de renovaci\'on  con tiempos entre arribos independientes e identicamente distribuidos (\texttt{i.i.d.})$T_{n}=t_{n}-t_{n-1}$, adem\'as los tiempos de servicio son \texttt{i.i.d.} e independientes de los procesos de arribo. Por \textit{estable} se entiende que $\esp S_{n}<\esp T_{n}<\infty$.
\end{Note}
 


\begin{Def}
Para $x$ fijo y para cada $t\geq0$, sea $I_{x}\left(t\right)=1$ si $X\left(t\right)\leq x$,  $I_{x}\left(t\right)=0$ en caso contrario, y def\'inanse los tiempos promedio
\begin{eqnarray*}
\overline{X}&=&lim_{t\rightarrow\infty}\frac{1}{t}\int_{0}^{\infty}X\left(u\right)du\\
\prob\left(X_{\infty}\leq x\right)&=&lim_{t\rightarrow\infty}\frac{1}{t}\int_{0}^{\infty}I_{x}\left(u\right)du,
\end{eqnarray*}
cuando estos l\'imites existan.
\end{Def}

Como consecuencia del teorema de Renovaci\'on-Recompensa, se tiene que el primer l\'imite  existe y es igual a la constante
\begin{eqnarray*}
\overline{X}&=&\frac{\esp\left[\int_{0}^{R_{1}}X\left(t\right)dt\right]}{\esp\left[R_{1}\right]},
\end{eqnarray*}
suponiendo que ambas esperanzas son finitas.
 
\begin{Note}
Funciones de procesos regenerativos son regenerativas, es decir, si $X\left(t\right)$ es regenerativo y se define el proceso $Y\left(t\right)$ por $Y\left(t\right)=f\left(X\left(t\right)\right)$ para alguna funci\'on Borel medible $f\left(\cdot\right)$. Adem\'as $Y$ es regenerativo con los mismos tiempos de renovaci\'on que $X$. 

En general, los tiempos de renovaci\'on, $Z_{k}$ de un proceso regenerativo no requieren ser tiempos de paro con respecto a la evoluci\'on de $X\left(t\right)$.
\end{Note} 

\begin{Note}
Una funci\'on de un proceso de Markov, usualmente no ser\'a un proceso de Markov, sin embargo ser\'a regenerativo si el proceso de Markov lo es.
\end{Note}

 
\begin{Note}
Un proceso regenerativo con media de la longitud de ciclo finita es llamado positivo recurrente.
\end{Note}


\begin{Note}
\begin{itemize}
\item[a)] Si el proceso regenerativo $X$ es positivo recurrente y tiene trayectorias muestrales no negativas, entonces la ecuaci\'on anterior es v\'alida.
\item[b)] Si $X$ es positivo recurrente regenerativo, podemos construir una \'unica versi\'on estacionaria de este proceso, $X_{e}=\left\{X_{e}\left(t\right)\right\}$, donde $X_{e}$ es un proceso estoc\'astico regenerativo y estrictamente estacionario, con distribuci\'on marginal distribuida como $X_{\infty}$
\end{itemize}
\end{Note}


%__________________________________________________________________________________________
%\subsection{Procesos Regenerativos Estacionarios - Stidham \cite{Stidham}}
%__________________________________________________________________________________________


Un proceso estoc\'astico a tiempo continuo $\left\{V\left(t\right),t\geq0\right\}$ es un proceso regenerativo si existe una sucesi\'on de variables aleatorias independientes e id\'enticamente distribuidas $\left\{X_{1},X_{2},\ldots\right\}$, sucesi\'on de renovaci\'on, tal que para cualquier conjunto de Borel $A$, 

\begin{eqnarray*}
\prob\left\{V\left(t\right)\in A|X_{1}+X_{2}+\cdots+X_{R\left(t\right)}=s,\left\{V\left(\tau\right),\tau<s\right\}\right\}=\prob\left\{V\left(t-s\right)\in A|X_{1}>t-s\right\},
\end{eqnarray*}
para todo $0\leq s\leq t$, donde $R\left(t\right)=\max\left\{X_{1}+X_{2}+\cdots+X_{j}\leq t\right\}=$n\'umero de renovaciones ({\emph{puntos de regeneraci\'on}}) que ocurren en $\left[0,t\right]$. El intervalo $\left[0,X_{1}\right)$ es llamado {\emph{primer ciclo de regeneraci\'on}} de $\left\{V\left(t \right),t\geq0\right\}$, $\left[X_{1},X_{1}+X_{2}\right)$ el {\emph{segundo ciclo de regeneraci\'on}}, y as\'i sucesivamente.

Sea $X=X_{1}$ y sea $F$ la funci\'on de distrbuci\'on de $X$


\begin{Def}
Se define el proceso estacionario, $\left\{V^{*}\left(t\right),t\geq0\right\}$, para $\left\{V\left(t\right),t\geq0\right\}$ por

\begin{eqnarray*}
\prob\left\{V\left(t\right)\in A\right\}=\frac{1}{\esp\left[X\right]}\int_{0}^{\infty}\prob\left\{V\left(t+x\right)\in A|X>x\right\}\left(1-F\left(x\right)\right)dx,
\end{eqnarray*} 
para todo $t\geq0$ y todo conjunto de Borel $A$.
\end{Def}

\begin{Def}
Una distribuci\'on se dice que es {\emph{aritm\'etica}} si todos sus puntos de incremento son m\'ultiplos de la forma $0,\lambda, 2\lambda,\ldots$ para alguna $\lambda>0$ entera.
\end{Def}


\begin{Def}
Una modificaci\'on medible de un proceso $\left\{V\left(t\right),t\geq0\right\}$, es una versi\'on de este, $\left\{V\left(t,w\right)\right\}$ conjuntamente medible para $t\geq0$ y para $w\in S$, $S$ espacio de estados para $\left\{V\left(t\right),t\geq0\right\}$.
\end{Def}

\begin{Teo}
Sea $\left\{V\left(t\right),t\geq\right\}$ un proceso regenerativo no negativo con modificaci\'on medible. Sea $\esp\left[X\right]<\infty$. Entonces el proceso estacionario dado por la ecuaci\'on anterior est\'a bien definido y tiene funci\'on de distribuci\'on independiente de $t$, adem\'as
\begin{itemize}
\item[i)] \begin{eqnarray*}
\esp\left[V^{*}\left(0\right)\right]&=&\frac{\esp\left[\int_{0}^{X}V\left(s\right)ds\right]}{\esp\left[X\right]}\end{eqnarray*}
\item[ii)] Si $\esp\left[V^{*}\left(0\right)\right]<\infty$, equivalentemente, si $\esp\left[\int_{0}^{X}V\left(s\right)ds\right]<\infty$,entonces
\begin{eqnarray*}
\frac{\int_{0}^{t}V\left(s\right)ds}{t}\rightarrow\frac{\esp\left[\int_{0}^{X}V\left(s\right)ds\right]}{\esp\left[X\right]}
\end{eqnarray*}
con probabilidad 1 y en media, cuando $t\rightarrow\infty$.
\end{itemize}
\end{Teo}

\begin{Coro}
Sea $\left\{V\left(t\right),t\geq0\right\}$ un proceso regenerativo no negativo, con modificaci\'on medible. Si $\esp <\infty$, $F$ es no-aritm\'etica, y para todo $x\geq0$, $P\left\{V\left(t\right)\leq x,C>x\right\}$ es de variaci\'on acotada como funci\'on de $t$ en cada intervalo finito $\left[0,\tau\right]$, entonces $V\left(t\right)$ converge en distribuci\'on  cuando $t\rightarrow\infty$ y $$\esp V=\frac{\esp \int_{0}^{X}V\left(s\right)ds}{\esp X}$$
Donde $V$ tiene la distribuci\'on l\'imite de $V\left(t\right)$ cuando $t\rightarrow\infty$.

\end{Coro}

Para el caso discreto se tienen resultados similares.



%______________________________________________________________________
\subsection{Procesos de Renovaci\'on}
%______________________________________________________________________

\begin{Def}%\label{Def.Tn}
Sean $0\leq T_{1}\leq T_{2}\leq \ldots$ son tiempos aleatorios infinitos en los cuales ocurren ciertos eventos. El n\'umero de tiempos $T_{n}$ en el intervalo $\left[0,t\right)$ es

\begin{eqnarray}
N\left(t\right)=\sum_{n=1}^{\infty}\indora\left(T_{n}\leq t\right),
\end{eqnarray}
para $t\geq0$.
\end{Def}

Si se consideran los puntos $T_{n}$ como elementos de $\rea_{+}$, y $N\left(t\right)$ es el n\'umero de puntos en $\rea$. El proceso denotado por $\left\{N\left(t\right):t\geq0\right\}$, denotado por $N\left(t\right)$, es un proceso puntual en $\rea_{+}$. Los $T_{n}$ son los tiempos de ocurrencia, el proceso puntual $N\left(t\right)$ es simple si su n\'umero de ocurrencias son distintas: $0<T_{1}<T_{2}<\ldots$ casi seguramente.

\begin{Def}
Un proceso puntual $N\left(t\right)$ es un proceso de renovaci\'on si los tiempos de interocurrencia $\xi_{n}=T_{n}-T_{n-1}$, para $n\geq1$, son independientes e identicamente distribuidos con distribuci\'on $F$, donde $F\left(0\right)=0$ y $T_{0}=0$. Los $T_{n}$ son llamados tiempos de renovaci\'on, referente a la independencia o renovaci\'on de la informaci\'on estoc\'astica en estos tiempos. Los $\xi_{n}$ son los tiempos de inter-renovaci\'on, y $N\left(t\right)$ es el n\'umero de renovaciones en el intervalo $\left[0,t\right)$
\end{Def}


\begin{Note}
Para definir un proceso de renovaci\'on para cualquier contexto, solamente hay que especificar una distribuci\'on $F$, con $F\left(0\right)=0$, para los tiempos de inter-renovaci\'on. La funci\'on $F$ en turno degune las otra variables aleatorias. De manera formal, existe un espacio de probabilidad y una sucesi\'on de variables aleatorias $\xi_{1},\xi_{2},\ldots$ definidas en este con distribuci\'on $F$. Entonces las otras cantidades son $T_{n}=\sum_{k=1}^{n}\xi_{k}$ y $N\left(t\right)=\sum_{n=1}^{\infty}\indora\left(T_{n}\leq t\right)$, donde $T_{n}\rightarrow\infty$ casi seguramente por la Ley Fuerte de los Grandes Números.
\end{Note}

%___________________________________________________________________________________________
%
\subsection{Teorema Principal de Renovaci\'on}
%___________________________________________________________________________________________
%

\begin{Note} Una funci\'on $h:\rea_{+}\rightarrow\rea$ es Directamente Riemann Integrable en los siguientes casos:
\begin{itemize}
\item[a)] $h\left(t\right)\geq0$ es decreciente y Riemann Integrable.
\item[b)] $h$ es continua excepto posiblemente en un conjunto de Lebesgue de medida 0, y $|h\left(t\right)|\leq b\left(t\right)$, donde $b$ es DRI.
\end{itemize}
\end{Note}

\begin{Teo}[Teorema Principal de Renovaci\'on]
Si $F$ es no aritm\'etica y $h\left(t\right)$ es Directamente Riemann Integrable (DRI), entonces

\begin{eqnarray*}
lim_{t\rightarrow\infty}U\star h=\frac{1}{\mu}\int_{\rea_{+}}h\left(s\right)ds.
\end{eqnarray*}
\end{Teo}

\begin{Prop}
Cualquier funci\'on $H\left(t\right)$ acotada en intervalos finitos y que es 0 para $t<0$ puede expresarse como
\begin{eqnarray*}
H\left(t\right)=U\star h\left(t\right)\textrm{,  donde }h\left(t\right)=H\left(t\right)-F\star H\left(t\right)
\end{eqnarray*}
\end{Prop}

\begin{Def}
Un proceso estoc\'astico $X\left(t\right)$ es crudamente regenerativo en un tiempo aleatorio positivo $T$ si
\begin{eqnarray*}
\esp\left[X\left(T+t\right)|T\right]=\esp\left[X\left(t\right)\right]\textrm{, para }t\geq0,\end{eqnarray*}
y con las esperanzas anteriores finitas.
\end{Def}

\begin{Prop}
Sup\'ongase que $X\left(t\right)$ es un proceso crudamente regenerativo en $T$, que tiene distribuci\'on $F$. Si $\esp\left[X\left(t\right)\right]$ es acotado en intervalos finitos, entonces
\begin{eqnarray*}
\esp\left[X\left(t\right)\right]=U\star h\left(t\right)\textrm{,  donde }h\left(t\right)=\esp\left[X\left(t\right)\indora\left(T>t\right)\right].
\end{eqnarray*}
\end{Prop}

\begin{Teo}[Regeneraci\'on Cruda]
Sup\'ongase que $X\left(t\right)$ es un proceso con valores positivo crudamente regenerativo en $T$, y def\'inase $M=\sup\left\{|X\left(t\right)|:t\leq T\right\}$. Si $T$ es no aritm\'etico y $M$ y $MT$ tienen media finita, entonces
\begin{eqnarray*}
lim_{t\rightarrow\infty}\esp\left[X\left(t\right)\right]=\frac{1}{\mu}\int_{\rea_{+}}h\left(s\right)ds,
\end{eqnarray*}
donde $h\left(t\right)=\esp\left[X\left(t\right)\indora\left(T>t\right)\right]$.
\end{Teo}

%___________________________________________________________________________________________
%
\subsection{Propiedades de los Procesos de Renovaci\'on}
%___________________________________________________________________________________________
%

Los tiempos $T_{n}$ est\'an relacionados con los conteos de $N\left(t\right)$ por

\begin{eqnarray*}
\left\{N\left(t\right)\geq n\right\}&=&\left\{T_{n}\leq t\right\}\\
T_{N\left(t\right)}\leq &t&<T_{N\left(t\right)+1},
\end{eqnarray*}

adem\'as $N\left(T_{n}\right)=n$, y 

\begin{eqnarray*}
N\left(t\right)=\max\left\{n:T_{n}\leq t\right\}=\min\left\{n:T_{n+1}>t\right\}
\end{eqnarray*}

Por propiedades de la convoluci\'on se sabe que

\begin{eqnarray*}
P\left\{T_{n}\leq t\right\}=F^{n\star}\left(t\right)
\end{eqnarray*}
que es la $n$-\'esima convoluci\'on de $F$. Entonces 

\begin{eqnarray*}
\left\{N\left(t\right)\geq n\right\}&=&\left\{T_{n}\leq t\right\}\\
P\left\{N\left(t\right)\leq n\right\}&=&1-F^{\left(n+1\right)\star}\left(t\right)
\end{eqnarray*}

Adem\'as usando el hecho de que $\esp\left[N\left(t\right)\right]=\sum_{n=1}^{\infty}P\left\{N\left(t\right)\geq n\right\}$
se tiene que

\begin{eqnarray*}
\esp\left[N\left(t\right)\right]=\sum_{n=1}^{\infty}F^{n\star}\left(t\right)
\end{eqnarray*}

\begin{Prop}
Para cada $t\geq0$, la funci\'on generadora de momentos $\esp\left[e^{\alpha N\left(t\right)}\right]$ existe para alguna $\alpha$ en una vecindad del 0, y de aqu\'i que $\esp\left[N\left(t\right)^{m}\right]<\infty$, para $m\geq1$.
\end{Prop}


\begin{Note}
Si el primer tiempo de renovaci\'on $\xi_{1}$ no tiene la misma distribuci\'on que el resto de las $\xi_{n}$, para $n\geq2$, a $N\left(t\right)$ se le llama Proceso de Renovaci\'on retardado, donde si $\xi$ tiene distribuci\'on $G$, entonces el tiempo $T_{n}$ de la $n$-\'esima renovaci\'on tiene distribuci\'on $G\star F^{\left(n-1\right)\star}\left(t\right)$
\end{Note}


\begin{Teo}
Para una constante $\mu\leq\infty$ ( o variable aleatoria), las siguientes expresiones son equivalentes:

\begin{eqnarray}
lim_{n\rightarrow\infty}n^{-1}T_{n}&=&\mu,\textrm{ c.s.}\\
lim_{t\rightarrow\infty}t^{-1}N\left(t\right)&=&1/\mu,\textrm{ c.s.}
\end{eqnarray}
\end{Teo}


Es decir, $T_{n}$ satisface la Ley Fuerte de los Grandes N\'umeros s\'i y s\'olo s\'i $N\left/t\right)$ la cumple.


\begin{Coro}[Ley Fuerte de los Grandes N\'umeros para Procesos de Renovaci\'on]
Si $N\left(t\right)$ es un proceso de renovaci\'on cuyos tiempos de inter-renovaci\'on tienen media $\mu\leq\infty$, entonces
\begin{eqnarray}
t^{-1}N\left(t\right)\rightarrow 1/\mu,\textrm{ c.s. cuando }t\rightarrow\infty.
\end{eqnarray}

\end{Coro}


Considerar el proceso estoc\'astico de valores reales $\left\{Z\left(t\right):t\geq0\right\}$ en el mismo espacio de probabilidad que $N\left(t\right)$

\begin{Def}
Para el proceso $\left\{Z\left(t\right):t\geq0\right\}$ se define la fluctuaci\'on m\'axima de $Z\left(t\right)$ en el intervalo $\left(T_{n-1},T_{n}\right]$:
\begin{eqnarray*}
M_{n}=\sup_{T_{n-1}<t\leq T_{n}}|Z\left(t\right)-Z\left(T_{n-1}\right)|
\end{eqnarray*}
\end{Def}

\begin{Teo}
Sup\'ongase que $n^{-1}T_{n}\rightarrow\mu$ c.s. cuando $n\rightarrow\infty$, donde $\mu\leq\infty$ es una constante o variable aleatoria. Sea $a$ una constante o variable aleatoria que puede ser infinita cuando $\mu$ es finita, y considere las expresiones l\'imite:
\begin{eqnarray}
lim_{n\rightarrow\infty}n^{-1}Z\left(T_{n}\right)&=&a,\textrm{ c.s.}\\
lim_{t\rightarrow\infty}t^{-1}Z\left(t\right)&=&a/\mu,\textrm{ c.s.}
\end{eqnarray}
La segunda expresi\'on implica la primera. Conversamente, la primera implica la segunda si el proceso $Z\left(t\right)$ es creciente, o si $lim_{n\rightarrow\infty}n^{-1}M_{n}=0$ c.s.
\end{Teo}

\begin{Coro}
Si $N\left(t\right)$ es un proceso de renovaci\'on, y $\left(Z\left(T_{n}\right)-Z\left(T_{n-1}\right),M_{n}\right)$, para $n\geq1$, son variables aleatorias independientes e id\'enticamente distribuidas con media finita, entonces,
\begin{eqnarray}
lim_{t\rightarrow\infty}t^{-1}Z\left(t\right)\rightarrow\frac{\esp\left[Z\left(T_{1}\right)-Z\left(T_{0}\right)\right]}{\esp\left[T_{1}\right]},\textrm{ c.s. cuando  }t\rightarrow\infty.
\end{eqnarray}
\end{Coro}



%___________________________________________________________________________________________
%
\subsection{Propiedades de los Procesos de Renovaci\'on}
%___________________________________________________________________________________________
%

Los tiempos $T_{n}$ est\'an relacionados con los conteos de $N\left(t\right)$ por

\begin{eqnarray*}
\left\{N\left(t\right)\geq n\right\}&=&\left\{T_{n}\leq t\right\}\\
T_{N\left(t\right)}\leq &t&<T_{N\left(t\right)+1},
\end{eqnarray*}

adem\'as $N\left(T_{n}\right)=n$, y 

\begin{eqnarray*}
N\left(t\right)=\max\left\{n:T_{n}\leq t\right\}=\min\left\{n:T_{n+1}>t\right\}
\end{eqnarray*}

Por propiedades de la convoluci\'on se sabe que

\begin{eqnarray*}
P\left\{T_{n}\leq t\right\}=F^{n\star}\left(t\right)
\end{eqnarray*}
que es la $n$-\'esima convoluci\'on de $F$. Entonces 

\begin{eqnarray*}
\left\{N\left(t\right)\geq n\right\}&=&\left\{T_{n}\leq t\right\}\\
P\left\{N\left(t\right)\leq n\right\}&=&1-F^{\left(n+1\right)\star}\left(t\right)
\end{eqnarray*}

Adem\'as usando el hecho de que $\esp\left[N\left(t\right)\right]=\sum_{n=1}^{\infty}P\left\{N\left(t\right)\geq n\right\}$
se tiene que

\begin{eqnarray*}
\esp\left[N\left(t\right)\right]=\sum_{n=1}^{\infty}F^{n\star}\left(t\right)
\end{eqnarray*}

\begin{Prop}
Para cada $t\geq0$, la funci\'on generadora de momentos $\esp\left[e^{\alpha N\left(t\right)}\right]$ existe para alguna $\alpha$ en una vecindad del 0, y de aqu\'i que $\esp\left[N\left(t\right)^{m}\right]<\infty$, para $m\geq1$.
\end{Prop}


\begin{Note}
Si el primer tiempo de renovaci\'on $\xi_{1}$ no tiene la misma distribuci\'on que el resto de las $\xi_{n}$, para $n\geq2$, a $N\left(t\right)$ se le llama Proceso de Renovaci\'on retardado, donde si $\xi$ tiene distribuci\'on $G$, entonces el tiempo $T_{n}$ de la $n$-\'esima renovaci\'on tiene distribuci\'on $G\star F^{\left(n-1\right)\star}\left(t\right)$
\end{Note}


\begin{Teo}
Para una constante $\mu\leq\infty$ ( o variable aleatoria), las siguientes expresiones son equivalentes:

\begin{eqnarray}
lim_{n\rightarrow\infty}n^{-1}T_{n}&=&\mu,\textrm{ c.s.}\\
lim_{t\rightarrow\infty}t^{-1}N\left(t\right)&=&1/\mu,\textrm{ c.s.}
\end{eqnarray}
\end{Teo}


Es decir, $T_{n}$ satisface la Ley Fuerte de los Grandes N\'umeros s\'i y s\'olo s\'i $N\left/t\right)$ la cumple.


\begin{Coro}[Ley Fuerte de los Grandes N\'umeros para Procesos de Renovaci\'on]
Si $N\left(t\right)$ es un proceso de renovaci\'on cuyos tiempos de inter-renovaci\'on tienen media $\mu\leq\infty$, entonces
\begin{eqnarray}
t^{-1}N\left(t\right)\rightarrow 1/\mu,\textrm{ c.s. cuando }t\rightarrow\infty.
\end{eqnarray}

\end{Coro}


Considerar el proceso estoc\'astico de valores reales $\left\{Z\left(t\right):t\geq0\right\}$ en el mismo espacio de probabilidad que $N\left(t\right)$

\begin{Def}
Para el proceso $\left\{Z\left(t\right):t\geq0\right\}$ se define la fluctuaci\'on m\'axima de $Z\left(t\right)$ en el intervalo $\left(T_{n-1},T_{n}\right]$:
\begin{eqnarray*}
M_{n}=\sup_{T_{n-1}<t\leq T_{n}}|Z\left(t\right)-Z\left(T_{n-1}\right)|
\end{eqnarray*}
\end{Def}

\begin{Teo}
Sup\'ongase que $n^{-1}T_{n}\rightarrow\mu$ c.s. cuando $n\rightarrow\infty$, donde $\mu\leq\infty$ es una constante o variable aleatoria. Sea $a$ una constante o variable aleatoria que puede ser infinita cuando $\mu$ es finita, y considere las expresiones l\'imite:
\begin{eqnarray}
lim_{n\rightarrow\infty}n^{-1}Z\left(T_{n}\right)&=&a,\textrm{ c.s.}\\
lim_{t\rightarrow\infty}t^{-1}Z\left(t\right)&=&a/\mu,\textrm{ c.s.}
\end{eqnarray}
La segunda expresi\'on implica la primera. Conversamente, la primera implica la segunda si el proceso $Z\left(t\right)$ es creciente, o si $lim_{n\rightarrow\infty}n^{-1}M_{n}=0$ c.s.
\end{Teo}

\begin{Coro}
Si $N\left(t\right)$ es un proceso de renovaci\'on, y $\left(Z\left(T_{n}\right)-Z\left(T_{n-1}\right),M_{n}\right)$, para $n\geq1$, son variables aleatorias independientes e id\'enticamente distribuidas con media finita, entonces,
\begin{eqnarray}
lim_{t\rightarrow\infty}t^{-1}Z\left(t\right)\rightarrow\frac{\esp\left[Z\left(T_{1}\right)-Z\left(T_{0}\right)\right]}{\esp\left[T_{1}\right]},\textrm{ c.s. cuando  }t\rightarrow\infty.
\end{eqnarray}
\end{Coro}


%___________________________________________________________________________________________
%
\subsection{Propiedades de los Procesos de Renovaci\'on}
%___________________________________________________________________________________________
%

Los tiempos $T_{n}$ est\'an relacionados con los conteos de $N\left(t\right)$ por

\begin{eqnarray*}
\left\{N\left(t\right)\geq n\right\}&=&\left\{T_{n}\leq t\right\}\\
T_{N\left(t\right)}\leq &t&<T_{N\left(t\right)+1},
\end{eqnarray*}

adem\'as $N\left(T_{n}\right)=n$, y 

\begin{eqnarray*}
N\left(t\right)=\max\left\{n:T_{n}\leq t\right\}=\min\left\{n:T_{n+1}>t\right\}
\end{eqnarray*}

Por propiedades de la convoluci\'on se sabe que

\begin{eqnarray*}
P\left\{T_{n}\leq t\right\}=F^{n\star}\left(t\right)
\end{eqnarray*}
que es la $n$-\'esima convoluci\'on de $F$. Entonces 

\begin{eqnarray*}
\left\{N\left(t\right)\geq n\right\}&=&\left\{T_{n}\leq t\right\}\\
P\left\{N\left(t\right)\leq n\right\}&=&1-F^{\left(n+1\right)\star}\left(t\right)
\end{eqnarray*}

Adem\'as usando el hecho de que $\esp\left[N\left(t\right)\right]=\sum_{n=1}^{\infty}P\left\{N\left(t\right)\geq n\right\}$
se tiene que

\begin{eqnarray*}
\esp\left[N\left(t\right)\right]=\sum_{n=1}^{\infty}F^{n\star}\left(t\right)
\end{eqnarray*}

\begin{Prop}
Para cada $t\geq0$, la funci\'on generadora de momentos $\esp\left[e^{\alpha N\left(t\right)}\right]$ existe para alguna $\alpha$ en una vecindad del 0, y de aqu\'i que $\esp\left[N\left(t\right)^{m}\right]<\infty$, para $m\geq1$.
\end{Prop}


\begin{Note}
Si el primer tiempo de renovaci\'on $\xi_{1}$ no tiene la misma distribuci\'on que el resto de las $\xi_{n}$, para $n\geq2$, a $N\left(t\right)$ se le llama Proceso de Renovaci\'on retardado, donde si $\xi$ tiene distribuci\'on $G$, entonces el tiempo $T_{n}$ de la $n$-\'esima renovaci\'on tiene distribuci\'on $G\star F^{\left(n-1\right)\star}\left(t\right)$
\end{Note}


\begin{Teo}
Para una constante $\mu\leq\infty$ ( o variable aleatoria), las siguientes expresiones son equivalentes:

\begin{eqnarray}
lim_{n\rightarrow\infty}n^{-1}T_{n}&=&\mu,\textrm{ c.s.}\\
lim_{t\rightarrow\infty}t^{-1}N\left(t\right)&=&1/\mu,\textrm{ c.s.}
\end{eqnarray}
\end{Teo}


Es decir, $T_{n}$ satisface la Ley Fuerte de los Grandes N\'umeros s\'i y s\'olo s\'i $N\left/t\right)$ la cumple.


\begin{Coro}[Ley Fuerte de los Grandes N\'umeros para Procesos de Renovaci\'on]
Si $N\left(t\right)$ es un proceso de renovaci\'on cuyos tiempos de inter-renovaci\'on tienen media $\mu\leq\infty$, entonces
\begin{eqnarray}
t^{-1}N\left(t\right)\rightarrow 1/\mu,\textrm{ c.s. cuando }t\rightarrow\infty.
\end{eqnarray}

\end{Coro}


Considerar el proceso estoc\'astico de valores reales $\left\{Z\left(t\right):t\geq0\right\}$ en el mismo espacio de probabilidad que $N\left(t\right)$

\begin{Def}
Para el proceso $\left\{Z\left(t\right):t\geq0\right\}$ se define la fluctuaci\'on m\'axima de $Z\left(t\right)$ en el intervalo $\left(T_{n-1},T_{n}\right]$:
\begin{eqnarray*}
M_{n}=\sup_{T_{n-1}<t\leq T_{n}}|Z\left(t\right)-Z\left(T_{n-1}\right)|
\end{eqnarray*}
\end{Def}

\begin{Teo}
Sup\'ongase que $n^{-1}T_{n}\rightarrow\mu$ c.s. cuando $n\rightarrow\infty$, donde $\mu\leq\infty$ es una constante o variable aleatoria. Sea $a$ una constante o variable aleatoria que puede ser infinita cuando $\mu$ es finita, y considere las expresiones l\'imite:
\begin{eqnarray}
lim_{n\rightarrow\infty}n^{-1}Z\left(T_{n}\right)&=&a,\textrm{ c.s.}\\
lim_{t\rightarrow\infty}t^{-1}Z\left(t\right)&=&a/\mu,\textrm{ c.s.}
\end{eqnarray}
La segunda expresi\'on implica la primera. Conversamente, la primera implica la segunda si el proceso $Z\left(t\right)$ es creciente, o si $lim_{n\rightarrow\infty}n^{-1}M_{n}=0$ c.s.
\end{Teo}

\begin{Coro}
Si $N\left(t\right)$ es un proceso de renovaci\'on, y $\left(Z\left(T_{n}\right)-Z\left(T_{n-1}\right),M_{n}\right)$, para $n\geq1$, son variables aleatorias independientes e id\'enticamente distribuidas con media finita, entonces,
\begin{eqnarray}
lim_{t\rightarrow\infty}t^{-1}Z\left(t\right)\rightarrow\frac{\esp\left[Z\left(T_{1}\right)-Z\left(T_{0}\right)\right]}{\esp\left[T_{1}\right]},\textrm{ c.s. cuando  }t\rightarrow\infty.
\end{eqnarray}
\end{Coro}

%___________________________________________________________________________________________
%
\subsection{Propiedades de los Procesos de Renovaci\'on}
%___________________________________________________________________________________________
%

Los tiempos $T_{n}$ est\'an relacionados con los conteos de $N\left(t\right)$ por

\begin{eqnarray*}
\left\{N\left(t\right)\geq n\right\}&=&\left\{T_{n}\leq t\right\}\\
T_{N\left(t\right)}\leq &t&<T_{N\left(t\right)+1},
\end{eqnarray*}

adem\'as $N\left(T_{n}\right)=n$, y 

\begin{eqnarray*}
N\left(t\right)=\max\left\{n:T_{n}\leq t\right\}=\min\left\{n:T_{n+1}>t\right\}
\end{eqnarray*}

Por propiedades de la convoluci\'on se sabe que

\begin{eqnarray*}
P\left\{T_{n}\leq t\right\}=F^{n\star}\left(t\right)
\end{eqnarray*}
que es la $n$-\'esima convoluci\'on de $F$. Entonces 

\begin{eqnarray*}
\left\{N\left(t\right)\geq n\right\}&=&\left\{T_{n}\leq t\right\}\\
P\left\{N\left(t\right)\leq n\right\}&=&1-F^{\left(n+1\right)\star}\left(t\right)
\end{eqnarray*}

Adem\'as usando el hecho de que $\esp\left[N\left(t\right)\right]=\sum_{n=1}^{\infty}P\left\{N\left(t\right)\geq n\right\}$
se tiene que

\begin{eqnarray*}
\esp\left[N\left(t\right)\right]=\sum_{n=1}^{\infty}F^{n\star}\left(t\right)
\end{eqnarray*}

\begin{Prop}
Para cada $t\geq0$, la funci\'on generadora de momentos $\esp\left[e^{\alpha N\left(t\right)}\right]$ existe para alguna $\alpha$ en una vecindad del 0, y de aqu\'i que $\esp\left[N\left(t\right)^{m}\right]<\infty$, para $m\geq1$.
\end{Prop}


\begin{Note}
Si el primer tiempo de renovaci\'on $\xi_{1}$ no tiene la misma distribuci\'on que el resto de las $\xi_{n}$, para $n\geq2$, a $N\left(t\right)$ se le llama Proceso de Renovaci\'on retardado, donde si $\xi$ tiene distribuci\'on $G$, entonces el tiempo $T_{n}$ de la $n$-\'esima renovaci\'on tiene distribuci\'on $G\star F^{\left(n-1\right)\star}\left(t\right)$
\end{Note}


\begin{Teo}
Para una constante $\mu\leq\infty$ ( o variable aleatoria), las siguientes expresiones son equivalentes:

\begin{eqnarray}
lim_{n\rightarrow\infty}n^{-1}T_{n}&=&\mu,\textrm{ c.s.}\\
lim_{t\rightarrow\infty}t^{-1}N\left(t\right)&=&1/\mu,\textrm{ c.s.}
\end{eqnarray}
\end{Teo}


Es decir, $T_{n}$ satisface la Ley Fuerte de los Grandes N\'umeros s\'i y s\'olo s\'i $N\left/t\right)$ la cumple.


\begin{Coro}[Ley Fuerte de los Grandes N\'umeros para Procesos de Renovaci\'on]
Si $N\left(t\right)$ es un proceso de renovaci\'on cuyos tiempos de inter-renovaci\'on tienen media $\mu\leq\infty$, entonces
\begin{eqnarray}
t^{-1}N\left(t\right)\rightarrow 1/\mu,\textrm{ c.s. cuando }t\rightarrow\infty.
\end{eqnarray}

\end{Coro}


Considerar el proceso estoc\'astico de valores reales $\left\{Z\left(t\right):t\geq0\right\}$ en el mismo espacio de probabilidad que $N\left(t\right)$

\begin{Def}
Para el proceso $\left\{Z\left(t\right):t\geq0\right\}$ se define la fluctuaci\'on m\'axima de $Z\left(t\right)$ en el intervalo $\left(T_{n-1},T_{n}\right]$:
\begin{eqnarray*}
M_{n}=\sup_{T_{n-1}<t\leq T_{n}}|Z\left(t\right)-Z\left(T_{n-1}\right)|
\end{eqnarray*}
\end{Def}

\begin{Teo}
Sup\'ongase que $n^{-1}T_{n}\rightarrow\mu$ c.s. cuando $n\rightarrow\infty$, donde $\mu\leq\infty$ es una constante o variable aleatoria. Sea $a$ una constante o variable aleatoria que puede ser infinita cuando $\mu$ es finita, y considere las expresiones l\'imite:
\begin{eqnarray}
lim_{n\rightarrow\infty}n^{-1}Z\left(T_{n}\right)&=&a,\textrm{ c.s.}\\
lim_{t\rightarrow\infty}t^{-1}Z\left(t\right)&=&a/\mu,\textrm{ c.s.}
\end{eqnarray}
La segunda expresi\'on implica la primera. Conversamente, la primera implica la segunda si el proceso $Z\left(t\right)$ es creciente, o si $lim_{n\rightarrow\infty}n^{-1}M_{n}=0$ c.s.
\end{Teo}

\begin{Coro}
Si $N\left(t\right)$ es un proceso de renovaci\'on, y $\left(Z\left(T_{n}\right)-Z\left(T_{n-1}\right),M_{n}\right)$, para $n\geq1$, son variables aleatorias independientes e id\'enticamente distribuidas con media finita, entonces,
\begin{eqnarray}
lim_{t\rightarrow\infty}t^{-1}Z\left(t\right)\rightarrow\frac{\esp\left[Z\left(T_{1}\right)-Z\left(T_{0}\right)\right]}{\esp\left[T_{1}\right]},\textrm{ c.s. cuando  }t\rightarrow\infty.
\end{eqnarray}
\end{Coro}
%___________________________________________________________________________________________
%
\subsection{Propiedades de los Procesos de Renovaci\'on}
%___________________________________________________________________________________________
%

Los tiempos $T_{n}$ est\'an relacionados con los conteos de $N\left(t\right)$ por

\begin{eqnarray*}
\left\{N\left(t\right)\geq n\right\}&=&\left\{T_{n}\leq t\right\}\\
T_{N\left(t\right)}\leq &t&<T_{N\left(t\right)+1},
\end{eqnarray*}

adem\'as $N\left(T_{n}\right)=n$, y 

\begin{eqnarray*}
N\left(t\right)=\max\left\{n:T_{n}\leq t\right\}=\min\left\{n:T_{n+1}>t\right\}
\end{eqnarray*}

Por propiedades de la convoluci\'on se sabe que

\begin{eqnarray*}
P\left\{T_{n}\leq t\right\}=F^{n\star}\left(t\right)
\end{eqnarray*}
que es la $n$-\'esima convoluci\'on de $F$. Entonces 

\begin{eqnarray*}
\left\{N\left(t\right)\geq n\right\}&=&\left\{T_{n}\leq t\right\}\\
P\left\{N\left(t\right)\leq n\right\}&=&1-F^{\left(n+1\right)\star}\left(t\right)
\end{eqnarray*}

Adem\'as usando el hecho de que $\esp\left[N\left(t\right)\right]=\sum_{n=1}^{\infty}P\left\{N\left(t\right)\geq n\right\}$
se tiene que

\begin{eqnarray*}
\esp\left[N\left(t\right)\right]=\sum_{n=1}^{\infty}F^{n\star}\left(t\right)
\end{eqnarray*}

\begin{Prop}
Para cada $t\geq0$, la funci\'on generadora de momentos $\esp\left[e^{\alpha N\left(t\right)}\right]$ existe para alguna $\alpha$ en una vecindad del 0, y de aqu\'i que $\esp\left[N\left(t\right)^{m}\right]<\infty$, para $m\geq1$.
\end{Prop}


\begin{Note}
Si el primer tiempo de renovaci\'on $\xi_{1}$ no tiene la misma distribuci\'on que el resto de las $\xi_{n}$, para $n\geq2$, a $N\left(t\right)$ se le llama Proceso de Renovaci\'on retardado, donde si $\xi$ tiene distribuci\'on $G$, entonces el tiempo $T_{n}$ de la $n$-\'esima renovaci\'on tiene distribuci\'on $G\star F^{\left(n-1\right)\star}\left(t\right)$
\end{Note}


\begin{Teo}
Para una constante $\mu\leq\infty$ ( o variable aleatoria), las siguientes expresiones son equivalentes:

\begin{eqnarray}
lim_{n\rightarrow\infty}n^{-1}T_{n}&=&\mu,\textrm{ c.s.}\\
lim_{t\rightarrow\infty}t^{-1}N\left(t\right)&=&1/\mu,\textrm{ c.s.}
\end{eqnarray}
\end{Teo}


Es decir, $T_{n}$ satisface la Ley Fuerte de los Grandes N\'umeros s\'i y s\'olo s\'i $N\left/t\right)$ la cumple.


\begin{Coro}[Ley Fuerte de los Grandes N\'umeros para Procesos de Renovaci\'on]
Si $N\left(t\right)$ es un proceso de renovaci\'on cuyos tiempos de inter-renovaci\'on tienen media $\mu\leq\infty$, entonces
\begin{eqnarray}
t^{-1}N\left(t\right)\rightarrow 1/\mu,\textrm{ c.s. cuando }t\rightarrow\infty.
\end{eqnarray}

\end{Coro}


Considerar el proceso estoc\'astico de valores reales $\left\{Z\left(t\right):t\geq0\right\}$ en el mismo espacio de probabilidad que $N\left(t\right)$

\begin{Def}
Para el proceso $\left\{Z\left(t\right):t\geq0\right\}$ se define la fluctuaci\'on m\'axima de $Z\left(t\right)$ en el intervalo $\left(T_{n-1},T_{n}\right]$:
\begin{eqnarray*}
M_{n}=\sup_{T_{n-1}<t\leq T_{n}}|Z\left(t\right)-Z\left(T_{n-1}\right)|
\end{eqnarray*}
\end{Def}

\begin{Teo}
Sup\'ongase que $n^{-1}T_{n}\rightarrow\mu$ c.s. cuando $n\rightarrow\infty$, donde $\mu\leq\infty$ es una constante o variable aleatoria. Sea $a$ una constante o variable aleatoria que puede ser infinita cuando $\mu$ es finita, y considere las expresiones l\'imite:
\begin{eqnarray}
lim_{n\rightarrow\infty}n^{-1}Z\left(T_{n}\right)&=&a,\textrm{ c.s.}\\
lim_{t\rightarrow\infty}t^{-1}Z\left(t\right)&=&a/\mu,\textrm{ c.s.}
\end{eqnarray}
La segunda expresi\'on implica la primera. Conversamente, la primera implica la segunda si el proceso $Z\left(t\right)$ es creciente, o si $lim_{n\rightarrow\infty}n^{-1}M_{n}=0$ c.s.
\end{Teo}

\begin{Coro}
Si $N\left(t\right)$ es un proceso de renovaci\'on, y $\left(Z\left(T_{n}\right)-Z\left(T_{n-1}\right),M_{n}\right)$, para $n\geq1$, son variables aleatorias independientes e id\'enticamente distribuidas con media finita, entonces,
\begin{eqnarray}
lim_{t\rightarrow\infty}t^{-1}Z\left(t\right)\rightarrow\frac{\esp\left[Z\left(T_{1}\right)-Z\left(T_{0}\right)\right]}{\esp\left[T_{1}\right]},\textrm{ c.s. cuando  }t\rightarrow\infty.
\end{eqnarray}
\end{Coro}


%___________________________________________________________________________________________
%
\subsection{Funci\'on de Renovaci\'on}
%___________________________________________________________________________________________
%


\begin{Def}
Sea $h\left(t\right)$ funci\'on de valores reales en $\rea$ acotada en intervalos finitos e igual a cero para $t<0$ La ecuaci\'on de renovaci\'on para $h\left(t\right)$ y la distribuci\'on $F$ es

\begin{eqnarray}%\label{Ec.Renovacion}
H\left(t\right)=h\left(t\right)+\int_{\left[0,t\right]}H\left(t-s\right)dF\left(s\right)\textrm{,    }t\geq0,
\end{eqnarray}
donde $H\left(t\right)$ es una funci\'on de valores reales. Esto es $H=h+F\star H$. Decimos que $H\left(t\right)$ es soluci\'on de esta ecuaci\'on si satisface la ecuaci\'on, y es acotada en intervalos finitos e iguales a cero para $t<0$.
\end{Def}

\begin{Prop}
La funci\'on $U\star h\left(t\right)$ es la \'unica soluci\'on de la ecuaci\'on de renovaci\'on (\ref{Ec.Renovacion}).
\end{Prop}

\begin{Teo}[Teorema Renovaci\'on Elemental]
\begin{eqnarray*}
t^{-1}U\left(t\right)\rightarrow 1/\mu\textrm{,    cuando }t\rightarrow\infty.
\end{eqnarray*}
\end{Teo}

%___________________________________________________________________________________________
%
\subsection{Funci\'on de Renovaci\'on}
%___________________________________________________________________________________________
%


Sup\'ongase que $N\left(t\right)$ es un proceso de renovaci\'on con distribuci\'on $F$ con media finita $\mu$.

\begin{Def}
La funci\'on de renovaci\'on asociada con la distribuci\'on $F$, del proceso $N\left(t\right)$, es
\begin{eqnarray*}
U\left(t\right)=\sum_{n=1}^{\infty}F^{n\star}\left(t\right),\textrm{   }t\geq0,
\end{eqnarray*}
donde $F^{0\star}\left(t\right)=\indora\left(t\geq0\right)$.
\end{Def}


\begin{Prop}
Sup\'ongase que la distribuci\'on de inter-renovaci\'on $F$ tiene densidad $f$. Entonces $U\left(t\right)$ tambi\'en tiene densidad, para $t>0$, y es $U^{'}\left(t\right)=\sum_{n=0}^{\infty}f^{n\star}\left(t\right)$. Adem\'as
\begin{eqnarray*}
\prob\left\{N\left(t\right)>N\left(t-\right)\right\}=0\textrm{,   }t\geq0.
\end{eqnarray*}
\end{Prop}

\begin{Def}
La Transformada de Laplace-Stieljes de $F$ est\'a dada por

\begin{eqnarray*}
\hat{F}\left(\alpha\right)=\int_{\rea_{+}}e^{-\alpha t}dF\left(t\right)\textrm{,  }\alpha\geq0.
\end{eqnarray*}
\end{Def}

Entonces

\begin{eqnarray*}
\hat{U}\left(\alpha\right)=\sum_{n=0}^{\infty}\hat{F^{n\star}}\left(\alpha\right)=\sum_{n=0}^{\infty}\hat{F}\left(\alpha\right)^{n}=\frac{1}{1-\hat{F}\left(\alpha\right)}.
\end{eqnarray*}


\begin{Prop}
La Transformada de Laplace $\hat{U}\left(\alpha\right)$ y $\hat{F}\left(\alpha\right)$ determina una a la otra de manera \'unica por la relaci\'on $\hat{U}\left(\alpha\right)=\frac{1}{1-\hat{F}\left(\alpha\right)}$.
\end{Prop}


\begin{Note}
Un proceso de renovaci\'on $N\left(t\right)$ cuyos tiempos de inter-renovaci\'on tienen media finita, es un proceso Poisson con tasa $\lambda$ si y s\'olo s\'i $\esp\left[U\left(t\right)\right]=\lambda t$, para $t\geq0$.
\end{Note}


\begin{Teo}
Sea $N\left(t\right)$ un proceso puntual simple con puntos de localizaci\'on $T_{n}$ tal que $\eta\left(t\right)=\esp\left[N\left(\right)\right]$ es finita para cada $t$. Entonces para cualquier funci\'on $f:\rea_{+}\rightarrow\rea$,
\begin{eqnarray*}
\esp\left[\sum_{n=1}^{N\left(\right)}f\left(T_{n}\right)\right]=\int_{\left(0,t\right]}f\left(s\right)d\eta\left(s\right)\textrm{,  }t\geq0,
\end{eqnarray*}
suponiendo que la integral exista. Adem\'as si $X_{1},X_{2},\ldots$ son variables aleatorias definidas en el mismo espacio de probabilidad que el proceso $N\left(t\right)$ tal que $\esp\left[X_{n}|T_{n}=s\right]=f\left(s\right)$, independiente de $n$. Entonces
\begin{eqnarray*}
\esp\left[\sum_{n=1}^{N\left(t\right)}X_{n}\right]=\int_{\left(0,t\right]}f\left(s\right)d\eta\left(s\right)\textrm{,  }t\geq0,
\end{eqnarray*} 
suponiendo que la integral exista. 
\end{Teo}

\begin{Coro}[Identidad de Wald para Renovaciones]
Para el proceso de renovaci\'on $N\left(t\right)$,
\begin{eqnarray*}
\esp\left[T_{N\left(t\right)+1}\right]=\mu\esp\left[N\left(t\right)+1\right]\textrm{,  }t\geq0,
\end{eqnarray*}  
\end{Coro}

%______________________________________________________________________
%\subsection{Procesos de Renovaci\'on}
%______________________________________________________________________

\begin{Def}%\label{Def.Tn}
Sean $0\leq T_{1}\leq T_{2}\leq \ldots$ son tiempos aleatorios infinitos en los cuales ocurren ciertos eventos. El n\'umero de tiempos $T_{n}$ en el intervalo $\left[0,t\right)$ es

\begin{eqnarray}
N\left(t\right)=\sum_{n=1}^{\infty}\indora\left(T_{n}\leq t\right),
\end{eqnarray}
para $t\geq0$.
\end{Def}

Si se consideran los puntos $T_{n}$ como elementos de $\rea_{+}$, y $N\left(t\right)$ es el n\'umero de puntos en $\rea$. El proceso denotado por $\left\{N\left(t\right):t\geq0\right\}$, denotado por $N\left(t\right)$, es un proceso puntual en $\rea_{+}$. Los $T_{n}$ son los tiempos de ocurrencia, el proceso puntual $N\left(t\right)$ es simple si su n\'umero de ocurrencias son distintas: $0<T_{1}<T_{2}<\ldots$ casi seguramente.

\begin{Def}
Un proceso puntual $N\left(t\right)$ es un proceso de renovaci\'on si los tiempos de interocurrencia $\xi_{n}=T_{n}-T_{n-1}$, para $n\geq1$, son independientes e identicamente distribuidos con distribuci\'on $F$, donde $F\left(0\right)=0$ y $T_{0}=0$. Los $T_{n}$ son llamados tiempos de renovaci\'on, referente a la independencia o renovaci\'on de la informaci\'on estoc\'astica en estos tiempos. Los $\xi_{n}$ son los tiempos de inter-renovaci\'on, y $N\left(t\right)$ es el n\'umero de renovaciones en el intervalo $\left[0,t\right)$
\end{Def}


\begin{Note}
Para definir un proceso de renovaci\'on para cualquier contexto, solamente hay que especificar una distribuci\'on $F$, con $F\left(0\right)=0$, para los tiempos de inter-renovaci\'on. La funci\'on $F$ en turno degune las otra variables aleatorias. De manera formal, existe un espacio de probabilidad y una sucesi\'on de variables aleatorias $\xi_{1},\xi_{2},\ldots$ definidas en este con distribuci\'on $F$. Entonces las otras cantidades son $T_{n}=\sum_{k=1}^{n}\xi_{k}$ y $N\left(t\right)=\sum_{n=1}^{\infty}\indora\left(T_{n}\leq t\right)$, donde $T_{n}\rightarrow\infty$ casi seguramente por la Ley Fuerte de los Grandes Números.
\end{Note}

%___________________________________________________________________________________________
%
\subsection{Renewal and Regenerative Processes: Serfozo\cite{Serfozo}}
%___________________________________________________________________________________________
%
\begin{Def}%\label{Def.Tn}
Sean $0\leq T_{1}\leq T_{2}\leq \ldots$ son tiempos aleatorios infinitos en los cuales ocurren ciertos eventos. El n\'umero de tiempos $T_{n}$ en el intervalo $\left[0,t\right)$ es

\begin{eqnarray}
N\left(t\right)=\sum_{n=1}^{\infty}\indora\left(T_{n}\leq t\right),
\end{eqnarray}
para $t\geq0$.
\end{Def}

Si se consideran los puntos $T_{n}$ como elementos de $\rea_{+}$, y $N\left(t\right)$ es el n\'umero de puntos en $\rea$. El proceso denotado por $\left\{N\left(t\right):t\geq0\right\}$, denotado por $N\left(t\right)$, es un proceso puntual en $\rea_{+}$. Los $T_{n}$ son los tiempos de ocurrencia, el proceso puntual $N\left(t\right)$ es simple si su n\'umero de ocurrencias son distintas: $0<T_{1}<T_{2}<\ldots$ casi seguramente.

\begin{Def}
Un proceso puntual $N\left(t\right)$ es un proceso de renovaci\'on si los tiempos de interocurrencia $\xi_{n}=T_{n}-T_{n-1}$, para $n\geq1$, son independientes e identicamente distribuidos con distribuci\'on $F$, donde $F\left(0\right)=0$ y $T_{0}=0$. Los $T_{n}$ son llamados tiempos de renovaci\'on, referente a la independencia o renovaci\'on de la informaci\'on estoc\'astica en estos tiempos. Los $\xi_{n}$ son los tiempos de inter-renovaci\'on, y $N\left(t\right)$ es el n\'umero de renovaciones en el intervalo $\left[0,t\right)$
\end{Def}


\begin{Note}
Para definir un proceso de renovaci\'on para cualquier contexto, solamente hay que especificar una distribuci\'on $F$, con $F\left(0\right)=0$, para los tiempos de inter-renovaci\'on. La funci\'on $F$ en turno degune las otra variables aleatorias. De manera formal, existe un espacio de probabilidad y una sucesi\'on de variables aleatorias $\xi_{1},\xi_{2},\ldots$ definidas en este con distribuci\'on $F$. Entonces las otras cantidades son $T_{n}=\sum_{k=1}^{n}\xi_{k}$ y $N\left(t\right)=\sum_{n=1}^{\infty}\indora\left(T_{n}\leq t\right)$, donde $T_{n}\rightarrow\infty$ casi seguramente por la Ley Fuerte de los Grandes N\'umeros.
\end{Note}







Los tiempos $T_{n}$ est\'an relacionados con los conteos de $N\left(t\right)$ por

\begin{eqnarray*}
\left\{N\left(t\right)\geq n\right\}&=&\left\{T_{n}\leq t\right\}\\
T_{N\left(t\right)}\leq &t&<T_{N\left(t\right)+1},
\end{eqnarray*}

adem\'as $N\left(T_{n}\right)=n$, y 

\begin{eqnarray*}
N\left(t\right)=\max\left\{n:T_{n}\leq t\right\}=\min\left\{n:T_{n+1}>t\right\}
\end{eqnarray*}

Por propiedades de la convoluci\'on se sabe que

\begin{eqnarray*}
P\left\{T_{n}\leq t\right\}=F^{n\star}\left(t\right)
\end{eqnarray*}
que es la $n$-\'esima convoluci\'on de $F$. Entonces 

\begin{eqnarray*}
\left\{N\left(t\right)\geq n\right\}&=&\left\{T_{n}\leq t\right\}\\
P\left\{N\left(t\right)\leq n\right\}&=&1-F^{\left(n+1\right)\star}\left(t\right)
\end{eqnarray*}

Adem\'as usando el hecho de que $\esp\left[N\left(t\right)\right]=\sum_{n=1}^{\infty}P\left\{N\left(t\right)\geq n\right\}$
se tiene que

\begin{eqnarray*}
\esp\left[N\left(t\right)\right]=\sum_{n=1}^{\infty}F^{n\star}\left(t\right)
\end{eqnarray*}

\begin{Prop}
Para cada $t\geq0$, la funci\'on generadora de momentos $\esp\left[e^{\alpha N\left(t\right)}\right]$ existe para alguna $\alpha$ en una vecindad del 0, y de aqu\'i que $\esp\left[N\left(t\right)^{m}\right]<\infty$, para $m\geq1$.
\end{Prop}

\begin{Ejem}[\textbf{Proceso Poisson}]

Suponga que se tienen tiempos de inter-renovaci\'on \textit{i.i.d.} del proceso de renovaci\'on $N\left(t\right)$ tienen distribuci\'on exponencial $F\left(t\right)=q-e^{-\lambda t}$ con tasa $\lambda$. Entonces $N\left(t\right)$ es un proceso Poisson con tasa $\lambda$.

\end{Ejem}


\begin{Note}
Si el primer tiempo de renovaci\'on $\xi_{1}$ no tiene la misma distribuci\'on que el resto de las $\xi_{n}$, para $n\geq2$, a $N\left(t\right)$ se le llama Proceso de Renovaci\'on retardado, donde si $\xi$ tiene distribuci\'on $G$, entonces el tiempo $T_{n}$ de la $n$-\'esima renovaci\'on tiene distribuci\'on $G\star F^{\left(n-1\right)\star}\left(t\right)$
\end{Note}


\begin{Teo}
Para una constante $\mu\leq\infty$ ( o variable aleatoria), las siguientes expresiones son equivalentes:

\begin{eqnarray}
lim_{n\rightarrow\infty}n^{-1}T_{n}&=&\mu,\textrm{ c.s.}\\
lim_{t\rightarrow\infty}t^{-1}N\left(t\right)&=&1/\mu,\textrm{ c.s.}
\end{eqnarray}
\end{Teo}


Es decir, $T_{n}$ satisface la Ley Fuerte de los Grandes N\'umeros s\'i y s\'olo s\'i $N\left/t\right)$ la cumple.


\begin{Coro}[Ley Fuerte de los Grandes N\'umeros para Procesos de Renovaci\'on]
Si $N\left(t\right)$ es un proceso de renovaci\'on cuyos tiempos de inter-renovaci\'on tienen media $\mu\leq\infty$, entonces
\begin{eqnarray}
t^{-1}N\left(t\right)\rightarrow 1/\mu,\textrm{ c.s. cuando }t\rightarrow\infty.
\end{eqnarray}

\end{Coro}


Considerar el proceso estoc\'astico de valores reales $\left\{Z\left(t\right):t\geq0\right\}$ en el mismo espacio de probabilidad que $N\left(t\right)$

\begin{Def}
Para el proceso $\left\{Z\left(t\right):t\geq0\right\}$ se define la fluctuaci\'on m\'axima de $Z\left(t\right)$ en el intervalo $\left(T_{n-1},T_{n}\right]$:
\begin{eqnarray*}
M_{n}=\sup_{T_{n-1}<t\leq T_{n}}|Z\left(t\right)-Z\left(T_{n-1}\right)|
\end{eqnarray*}
\end{Def}

\begin{Teo}
Sup\'ongase que $n^{-1}T_{n}\rightarrow\mu$ c.s. cuando $n\rightarrow\infty$, donde $\mu\leq\infty$ es una constante o variable aleatoria. Sea $a$ una constante o variable aleatoria que puede ser infinita cuando $\mu$ es finita, y considere las expresiones l\'imite:
\begin{eqnarray}
lim_{n\rightarrow\infty}n^{-1}Z\left(T_{n}\right)&=&a,\textrm{ c.s.}\\
lim_{t\rightarrow\infty}t^{-1}Z\left(t\right)&=&a/\mu,\textrm{ c.s.}
\end{eqnarray}
La segunda expresi\'on implica la primera. Conversamente, la primera implica la segunda si el proceso $Z\left(t\right)$ es creciente, o si $lim_{n\rightarrow\infty}n^{-1}M_{n}=0$ c.s.
\end{Teo}

\begin{Coro}
Si $N\left(t\right)$ es un proceso de renovaci\'on, y $\left(Z\left(T_{n}\right)-Z\left(T_{n-1}\right),M_{n}\right)$, para $n\geq1$, son variables aleatorias independientes e id\'enticamente distribuidas con media finita, entonces,
\begin{eqnarray}
lim_{t\rightarrow\infty}t^{-1}Z\left(t\right)\rightarrow\frac{\esp\left[Z\left(T_{1}\right)-Z\left(T_{0}\right)\right]}{\esp\left[T_{1}\right]},\textrm{ c.s. cuando  }t\rightarrow\infty.
\end{eqnarray}
\end{Coro}


Sup\'ongase que $N\left(t\right)$ es un proceso de renovaci\'on con distribuci\'on $F$ con media finita $\mu$.

\begin{Def}
La funci\'on de renovaci\'on asociada con la distribuci\'on $F$, del proceso $N\left(t\right)$, es
\begin{eqnarray*}
U\left(t\right)=\sum_{n=1}^{\infty}F^{n\star}\left(t\right),\textrm{   }t\geq0,
\end{eqnarray*}
donde $F^{0\star}\left(t\right)=\indora\left(t\geq0\right)$.
\end{Def}


\begin{Prop}
Sup\'ongase que la distribuci\'on de inter-renovaci\'on $F$ tiene densidad $f$. Entonces $U\left(t\right)$ tambi\'en tiene densidad, para $t>0$, y es $U^{'}\left(t\right)=\sum_{n=0}^{\infty}f^{n\star}\left(t\right)$. Adem\'as
\begin{eqnarray*}
\prob\left\{N\left(t\right)>N\left(t-\right)\right\}=0\textrm{,   }t\geq0.
\end{eqnarray*}
\end{Prop}

\begin{Def}
La Transformada de Laplace-Stieljes de $F$ est\'a dada por

\begin{eqnarray*}
\hat{F}\left(\alpha\right)=\int_{\rea_{+}}e^{-\alpha t}dF\left(t\right)\textrm{,  }\alpha\geq0.
\end{eqnarray*}
\end{Def}

Entonces

\begin{eqnarray*}
\hat{U}\left(\alpha\right)=\sum_{n=0}^{\infty}\hat{F^{n\star}}\left(\alpha\right)=\sum_{n=0}^{\infty}\hat{F}\left(\alpha\right)^{n}=\frac{1}{1-\hat{F}\left(\alpha\right)}.
\end{eqnarray*}


\begin{Prop}
La Transformada de Laplace $\hat{U}\left(\alpha\right)$ y $\hat{F}\left(\alpha\right)$ determina una a la otra de manera \'unica por la relaci\'on $\hat{U}\left(\alpha\right)=\frac{1}{1-\hat{F}\left(\alpha\right)}$.
\end{Prop}


\begin{Note}
Un proceso de renovaci\'on $N\left(t\right)$ cuyos tiempos de inter-renovaci\'on tienen media finita, es un proceso Poisson con tasa $\lambda$ si y s\'olo s\'i $\esp\left[U\left(t\right)\right]=\lambda t$, para $t\geq0$.
\end{Note}


\begin{Teo}
Sea $N\left(t\right)$ un proceso puntual simple con puntos de localizaci\'on $T_{n}$ tal que $\eta\left(t\right)=\esp\left[N\left(\right)\right]$ es finita para cada $t$. Entonces para cualquier funci\'on $f:\rea_{+}\rightarrow\rea$,
\begin{eqnarray*}
\esp\left[\sum_{n=1}^{N\left(\right)}f\left(T_{n}\right)\right]=\int_{\left(0,t\right]}f\left(s\right)d\eta\left(s\right)\textrm{,  }t\geq0,
\end{eqnarray*}
suponiendo que la integral exista. Adem\'as si $X_{1},X_{2},\ldots$ son variables aleatorias definidas en el mismo espacio de probabilidad que el proceso $N\left(t\right)$ tal que $\esp\left[X_{n}|T_{n}=s\right]=f\left(s\right)$, independiente de $n$. Entonces
\begin{eqnarray*}
\esp\left[\sum_{n=1}^{N\left(t\right)}X_{n}\right]=\int_{\left(0,t\right]}f\left(s\right)d\eta\left(s\right)\textrm{,  }t\geq0,
\end{eqnarray*} 
suponiendo que la integral exista. 
\end{Teo}

\begin{Coro}[Identidad de Wald para Renovaciones]
Para el proceso de renovaci\'on $N\left(t\right)$,
\begin{eqnarray*}
\esp\left[T_{N\left(t\right)+1}\right]=\mu\esp\left[N\left(t\right)+1\right]\textrm{,  }t\geq0,
\end{eqnarray*}  
\end{Coro}


\begin{Def}
Sea $h\left(t\right)$ funci\'on de valores reales en $\rea$ acotada en intervalos finitos e igual a cero para $t<0$ La ecuaci\'on de renovaci\'on para $h\left(t\right)$ y la distribuci\'on $F$ es

\begin{eqnarray}%\label{Ec.Renovacion}
H\left(t\right)=h\left(t\right)+\int_{\left[0,t\right]}H\left(t-s\right)dF\left(s\right)\textrm{,    }t\geq0,
\end{eqnarray}
donde $H\left(t\right)$ es una funci\'on de valores reales. Esto es $H=h+F\star H$. Decimos que $H\left(t\right)$ es soluci\'on de esta ecuaci\'on si satisface la ecuaci\'on, y es acotada en intervalos finitos e iguales a cero para $t<0$.
\end{Def}

\begin{Prop}
La funci\'on $U\star h\left(t\right)$ es la \'unica soluci\'on de la ecuaci\'on de renovaci\'on (\ref{Ec.Renovacion}).
\end{Prop}

\begin{Teo}[Teorema Renovaci\'on Elemental]
\begin{eqnarray*}
t^{-1}U\left(t\right)\rightarrow 1/\mu\textrm{,    cuando }t\rightarrow\infty.
\end{eqnarray*}
\end{Teo}



Sup\'ongase que $N\left(t\right)$ es un proceso de renovaci\'on con distribuci\'on $F$ con media finita $\mu$.

\begin{Def}
La funci\'on de renovaci\'on asociada con la distribuci\'on $F$, del proceso $N\left(t\right)$, es
\begin{eqnarray*}
U\left(t\right)=\sum_{n=1}^{\infty}F^{n\star}\left(t\right),\textrm{   }t\geq0,
\end{eqnarray*}
donde $F^{0\star}\left(t\right)=\indora\left(t\geq0\right)$.
\end{Def}


\begin{Prop}
Sup\'ongase que la distribuci\'on de inter-renovaci\'on $F$ tiene densidad $f$. Entonces $U\left(t\right)$ tambi\'en tiene densidad, para $t>0$, y es $U^{'}\left(t\right)=\sum_{n=0}^{\infty}f^{n\star}\left(t\right)$. Adem\'as
\begin{eqnarray*}
\prob\left\{N\left(t\right)>N\left(t-\right)\right\}=0\textrm{,   }t\geq0.
\end{eqnarray*}
\end{Prop}

\begin{Def}
La Transformada de Laplace-Stieljes de $F$ est\'a dada por

\begin{eqnarray*}
\hat{F}\left(\alpha\right)=\int_{\rea_{+}}e^{-\alpha t}dF\left(t\right)\textrm{,  }\alpha\geq0.
\end{eqnarray*}
\end{Def}

Entonces

\begin{eqnarray*}
\hat{U}\left(\alpha\right)=\sum_{n=0}^{\infty}\hat{F^{n\star}}\left(\alpha\right)=\sum_{n=0}^{\infty}\hat{F}\left(\alpha\right)^{n}=\frac{1}{1-\hat{F}\left(\alpha\right)}.
\end{eqnarray*}


\begin{Prop}
La Transformada de Laplace $\hat{U}\left(\alpha\right)$ y $\hat{F}\left(\alpha\right)$ determina una a la otra de manera \'unica por la relaci\'on $\hat{U}\left(\alpha\right)=\frac{1}{1-\hat{F}\left(\alpha\right)}$.
\end{Prop}


\begin{Note}
Un proceso de renovaci\'on $N\left(t\right)$ cuyos tiempos de inter-renovaci\'on tienen media finita, es un proceso Poisson con tasa $\lambda$ si y s\'olo s\'i $\esp\left[U\left(t\right)\right]=\lambda t$, para $t\geq0$.
\end{Note}


\begin{Teo}
Sea $N\left(t\right)$ un proceso puntual simple con puntos de localizaci\'on $T_{n}$ tal que $\eta\left(t\right)=\esp\left[N\left(\right)\right]$ es finita para cada $t$. Entonces para cualquier funci\'on $f:\rea_{+}\rightarrow\rea$,
\begin{eqnarray*}
\esp\left[\sum_{n=1}^{N\left(\right)}f\left(T_{n}\right)\right]=\int_{\left(0,t\right]}f\left(s\right)d\eta\left(s\right)\textrm{,  }t\geq0,
\end{eqnarray*}
suponiendo que la integral exista. Adem\'as si $X_{1},X_{2},\ldots$ son variables aleatorias definidas en el mismo espacio de probabilidad que el proceso $N\left(t\right)$ tal que $\esp\left[X_{n}|T_{n}=s\right]=f\left(s\right)$, independiente de $n$. Entonces
\begin{eqnarray*}
\esp\left[\sum_{n=1}^{N\left(t\right)}X_{n}\right]=\int_{\left(0,t\right]}f\left(s\right)d\eta\left(s\right)\textrm{,  }t\geq0,
\end{eqnarray*} 
suponiendo que la integral exista. 
\end{Teo}

\begin{Coro}[Identidad de Wald para Renovaciones]
Para el proceso de renovaci\'on $N\left(t\right)$,
\begin{eqnarray*}
\esp\left[T_{N\left(t\right)+1}\right]=\mu\esp\left[N\left(t\right)+1\right]\textrm{,  }t\geq0,
\end{eqnarray*}  
\end{Coro}


\begin{Def}
Sea $h\left(t\right)$ funci\'on de valores reales en $\rea$ acotada en intervalos finitos e igual a cero para $t<0$ La ecuaci\'on de renovaci\'on para $h\left(t\right)$ y la distribuci\'on $F$ es

\begin{eqnarray}%\label{Ec.Renovacion}
H\left(t\right)=h\left(t\right)+\int_{\left[0,t\right]}H\left(t-s\right)dF\left(s\right)\textrm{,    }t\geq0,
\end{eqnarray}
donde $H\left(t\right)$ es una funci\'on de valores reales. Esto es $H=h+F\star H$. Decimos que $H\left(t\right)$ es soluci\'on de esta ecuaci\'on si satisface la ecuaci\'on, y es acotada en intervalos finitos e iguales a cero para $t<0$.
\end{Def}

\begin{Prop}
La funci\'on $U\star h\left(t\right)$ es la \'unica soluci\'on de la ecuaci\'on de renovaci\'on (\ref{Ec.Renovacion}).
\end{Prop}

\begin{Teo}[Teorema Renovaci\'on Elemental]
\begin{eqnarray*}
t^{-1}U\left(t\right)\rightarrow 1/\mu\textrm{,    cuando }t\rightarrow\infty.
\end{eqnarray*}
\end{Teo}


\begin{Note} Una funci\'on $h:\rea_{+}\rightarrow\rea$ es Directamente Riemann Integrable en los siguientes casos:
\begin{itemize}
\item[a)] $h\left(t\right)\geq0$ es decreciente y Riemann Integrable.
\item[b)] $h$ es continua excepto posiblemente en un conjunto de Lebesgue de medida 0, y $|h\left(t\right)|\leq b\left(t\right)$, donde $b$ es DRI.
\end{itemize}
\end{Note}

\begin{Teo}[Teorema Principal de Renovaci\'on]
Si $F$ es no aritm\'etica y $h\left(t\right)$ es Directamente Riemann Integrable (DRI), entonces

\begin{eqnarray*}
lim_{t\rightarrow\infty}U\star h=\frac{1}{\mu}\int_{\rea_{+}}h\left(s\right)ds.
\end{eqnarray*}
\end{Teo}

\begin{Prop}
Cualquier funci\'on $H\left(t\right)$ acotada en intervalos finitos y que es 0 para $t<0$ puede expresarse como
\begin{eqnarray*}
H\left(t\right)=U\star h\left(t\right)\textrm{,  donde }h\left(t\right)=H\left(t\right)-F\star H\left(t\right)
\end{eqnarray*}
\end{Prop}

\begin{Def}
Un proceso estoc\'astico $X\left(t\right)$ es crudamente regenerativo en un tiempo aleatorio positivo $T$ si
\begin{eqnarray*}
\esp\left[X\left(T+t\right)|T\right]=\esp\left[X\left(t\right)\right]\textrm{, para }t\geq0,\end{eqnarray*}
y con las esperanzas anteriores finitas.
\end{Def}

\begin{Prop}
Sup\'ongase que $X\left(t\right)$ es un proceso crudamente regenerativo en $T$, que tiene distribuci\'on $F$. Si $\esp\left[X\left(t\right)\right]$ es acotado en intervalos finitos, entonces
\begin{eqnarray*}
\esp\left[X\left(t\right)\right]=U\star h\left(t\right)\textrm{,  donde }h\left(t\right)=\esp\left[X\left(t\right)\indora\left(T>t\right)\right].
\end{eqnarray*}
\end{Prop}

\begin{Teo}[Regeneraci\'on Cruda]
Sup\'ongase que $X\left(t\right)$ es un proceso con valores positivo crudamente regenerativo en $T$, y def\'inase $M=\sup\left\{|X\left(t\right)|:t\leq T\right\}$. Si $T$ es no aritm\'etico y $M$ y $MT$ tienen media finita, entonces
\begin{eqnarray*}
lim_{t\rightarrow\infty}\esp\left[X\left(t\right)\right]=\frac{1}{\mu}\int_{\rea_{+}}h\left(s\right)ds,
\end{eqnarray*}
donde $h\left(t\right)=\esp\left[X\left(t\right)\indora\left(T>t\right)\right]$.
\end{Teo}


\begin{Note} Una funci\'on $h:\rea_{+}\rightarrow\rea$ es Directamente Riemann Integrable en los siguientes casos:
\begin{itemize}
\item[a)] $h\left(t\right)\geq0$ es decreciente y Riemann Integrable.
\item[b)] $h$ es continua excepto posiblemente en un conjunto de Lebesgue de medida 0, y $|h\left(t\right)|\leq b\left(t\right)$, donde $b$ es DRI.
\end{itemize}
\end{Note}

\begin{Teo}[Teorema Principal de Renovaci\'on]
Si $F$ es no aritm\'etica y $h\left(t\right)$ es Directamente Riemann Integrable (DRI), entonces

\begin{eqnarray*}
lim_{t\rightarrow\infty}U\star h=\frac{1}{\mu}\int_{\rea_{+}}h\left(s\right)ds.
\end{eqnarray*}
\end{Teo}

\begin{Prop}
Cualquier funci\'on $H\left(t\right)$ acotada en intervalos finitos y que es 0 para $t<0$ puede expresarse como
\begin{eqnarray*}
H\left(t\right)=U\star h\left(t\right)\textrm{,  donde }h\left(t\right)=H\left(t\right)-F\star H\left(t\right)
\end{eqnarray*}
\end{Prop}

\begin{Def}
Un proceso estoc\'astico $X\left(t\right)$ es crudamente regenerativo en un tiempo aleatorio positivo $T$ si
\begin{eqnarray*}
\esp\left[X\left(T+t\right)|T\right]=\esp\left[X\left(t\right)\right]\textrm{, para }t\geq0,\end{eqnarray*}
y con las esperanzas anteriores finitas.
\end{Def}

\begin{Prop}
Sup\'ongase que $X\left(t\right)$ es un proceso crudamente regenerativo en $T$, que tiene distribuci\'on $F$. Si $\esp\left[X\left(t\right)\right]$ es acotado en intervalos finitos, entonces
\begin{eqnarray*}
\esp\left[X\left(t\right)\right]=U\star h\left(t\right)\textrm{,  donde }h\left(t\right)=\esp\left[X\left(t\right)\indora\left(T>t\right)\right].
\end{eqnarray*}
\end{Prop}

\begin{Teo}[Regeneraci\'on Cruda]
Sup\'ongase que $X\left(t\right)$ es un proceso con valores positivo crudamente regenerativo en $T$, y def\'inase $M=\sup\left\{|X\left(t\right)|:t\leq T\right\}$. Si $T$ es no aritm\'etico y $M$ y $MT$ tienen media finita, entonces
\begin{eqnarray*}
lim_{t\rightarrow\infty}\esp\left[X\left(t\right)\right]=\frac{1}{\mu}\int_{\rea_{+}}h\left(s\right)ds,
\end{eqnarray*}
donde $h\left(t\right)=\esp\left[X\left(t\right)\indora\left(T>t\right)\right]$.
\end{Teo}

\begin{Def}
Para el proceso $\left\{\left(N\left(t\right),X\left(t\right)\right):t\geq0\right\}$, sus trayectoria muestrales en el intervalo de tiempo $\left[T_{n-1},T_{n}\right)$ est\'an descritas por
\begin{eqnarray*}
\zeta_{n}=\left(\xi_{n},\left\{X\left(T_{n-1}+t\right):0\leq t<\xi_{n}\right\}\right)
\end{eqnarray*}
Este $\zeta_{n}$ es el $n$-\'esimo segmento del proceso. El proceso es regenerativo sobre los tiempos $T_{n}$ si sus segmentos $\zeta_{n}$ son independientes e id\'enticamennte distribuidos.
\end{Def}


\begin{Note}
Si $\tilde{X}\left(t\right)$ con espacio de estados $\tilde{S}$ es regenerativo sobre $T_{n}$, entonces $X\left(t\right)=f\left(\tilde{X}\left(t\right)\right)$ tambi\'en es regenerativo sobre $T_{n}$, para cualquier funci\'on $f:\tilde{S}\rightarrow S$.
\end{Note}

\begin{Note}
Los procesos regenerativos son crudamente regenerativos, pero no al rev\'es.
\end{Note}


\begin{Note}
Un proceso estoc\'astico a tiempo continuo o discreto es regenerativo si existe un proceso de renovaci\'on  tal que los segmentos del proceso entre tiempos de renovaci\'on sucesivos son i.i.d., es decir, para $\left\{X\left(t\right):t\geq0\right\}$ proceso estoc\'astico a tiempo continuo con espacio de estados $S$, espacio m\'etrico.
\end{Note}

Para $\left\{X\left(t\right):t\geq0\right\}$ Proceso Estoc\'astico a tiempo continuo con estado de espacios $S$, que es un espacio m\'etrico, con trayectorias continuas por la derecha y con l\'imites por la izquierda c.s. Sea $N\left(t\right)$ un proceso de renovaci\'on en $\rea_{+}$ definido en el mismo espacio de probabilidad que $X\left(t\right)$, con tiempos de renovaci\'on $T$ y tiempos de inter-renovaci\'on $\xi_{n}=T_{n}-T_{n-1}$, con misma distribuci\'on $F$ de media finita $\mu$.



\begin{Def}
Para el proceso $\left\{\left(N\left(t\right),X\left(t\right)\right):t\geq0\right\}$, sus trayectoria muestrales en el intervalo de tiempo $\left[T_{n-1},T_{n}\right)$ est\'an descritas por
\begin{eqnarray*}
\zeta_{n}=\left(\xi_{n},\left\{X\left(T_{n-1}+t\right):0\leq t<\xi_{n}\right\}\right)
\end{eqnarray*}
Este $\zeta_{n}$ es el $n$-\'esimo segmento del proceso. El proceso es regenerativo sobre los tiempos $T_{n}$ si sus segmentos $\zeta_{n}$ son independientes e id\'enticamennte distribuidos.
\end{Def}

\begin{Note}
Un proceso regenerativo con media de la longitud de ciclo finita es llamado positivo recurrente.
\end{Note}

\begin{Teo}[Procesos Regenerativos]
Suponga que el proceso
\end{Teo}


\begin{Def}[Renewal Process Trinity]
Para un proceso de renovaci\'on $N\left(t\right)$, los siguientes procesos proveen de informaci\'on sobre los tiempos de renovaci\'on.
\begin{itemize}
\item $A\left(t\right)=t-T_{N\left(t\right)}$, el tiempo de recurrencia hacia atr\'as al tiempo $t$, que es el tiempo desde la \'ultima renovaci\'on para $t$.

\item $B\left(t\right)=T_{N\left(t\right)+1}-t$, el tiempo de recurrencia hacia adelante al tiempo $t$, residual del tiempo de renovaci\'on, que es el tiempo para la pr\'oxima renovaci\'on despu\'es de $t$.

\item $L\left(t\right)=\xi_{N\left(t\right)+1}=A\left(t\right)+B\left(t\right)$, la longitud del intervalo de renovaci\'on que contiene a $t$.
\end{itemize}
\end{Def}

\begin{Note}
El proceso tridimensional $\left(A\left(t\right),B\left(t\right),L\left(t\right)\right)$ es regenerativo sobre $T_{n}$, y por ende cada proceso lo es. Cada proceso $A\left(t\right)$ y $B\left(t\right)$ son procesos de MArkov a tiempo continuo con trayectorias continuas por partes en el espacio de estados $\rea_{+}$. Una expresi\'on conveniente para su distribuci\'on conjunta es, para $0\leq x<t,y\geq0$
\begin{equation}\label{NoRenovacion}
P\left\{A\left(t\right)>x,B\left(t\right)>y\right\}=
P\left\{N\left(t+y\right)-N\left((t-x)\right)=0\right\}
\end{equation}
\end{Note}

\begin{Ejem}[Tiempos de recurrencia Poisson]
Si $N\left(t\right)$ es un proceso Poisson con tasa $\lambda$, entonces de la expresi\'on (\ref{NoRenovacion}) se tiene que

\begin{eqnarray*}
\begin{array}{lc}
P\left\{A\left(t\right)>x,B\left(t\right)>y\right\}=e^{-\lambda\left(x+y\right)},&0\leq x<t,y\geq0,
\end{array}
\end{eqnarray*}
que es la probabilidad Poisson de no renovaciones en un intervalo de longitud $x+y$.

\end{Ejem}

\begin{Note}
Una cadena de Markov erg\'odica tiene la propiedad de ser estacionaria si la distribuci\'on de su estado al tiempo $0$ es su distribuci\'on estacionaria.
\end{Note}


\begin{Def}
Un proceso estoc\'astico a tiempo continuo $\left\{X\left(t\right):t\geq0\right\}$ en un espacio general es estacionario si sus distribuciones finito dimensionales son invariantes bajo cualquier  traslado: para cada $0\leq s_{1}<s_{2}<\cdots<s_{k}$ y $t\geq0$,
\begin{eqnarray*}
\left(X\left(s_{1}+t\right),\ldots,X\left(s_{k}+t\right)\right)=_{d}\left(X\left(s_{1}\right),\ldots,X\left(s_{k}\right)\right).
\end{eqnarray*}
\end{Def}

\begin{Note}
Un proceso de Markov es estacionario si $X\left(t\right)=_{d}X\left(0\right)$, $t\geq0$.
\end{Note}

Considerese el proceso $N\left(t\right)=\sum_{n}\indora\left(\tau_{n}\leq t\right)$ en $\rea_{+}$, con puntos $0<\tau_{1}<\tau_{2}<\cdots$.

\begin{Prop}
Si $N$ es un proceso puntual estacionario y $\esp\left[N\left(1\right)\right]<\infty$, entonces $\esp\left[N\left(t\right)\right]=t\esp\left[N\left(1\right)\right]$, $t\geq0$

\end{Prop}

\begin{Teo}
Los siguientes enunciados son equivalentes
\begin{itemize}
\item[i)] El proceso retardado de renovaci\'on $N$ es estacionario.

\item[ii)] EL proceso de tiempos de recurrencia hacia adelante $B\left(t\right)$ es estacionario.


\item[iii)] $\esp\left[N\left(t\right)\right]=t/\mu$,


\item[iv)] $G\left(t\right)=F_{e}\left(t\right)=\frac{1}{\mu}\int_{0}^{t}\left[1-F\left(s\right)\right]ds$
\end{itemize}
Cuando estos enunciados son ciertos, $P\left\{B\left(t\right)\leq x\right\}=F_{e}\left(x\right)$, para $t,x\geq0$.

\end{Teo}

\begin{Note}
Una consecuencia del teorema anterior es que el Proceso Poisson es el \'unico proceso sin retardo que es estacionario.
\end{Note}

\begin{Coro}
El proceso de renovaci\'on $N\left(t\right)$ sin retardo, y cuyos tiempos de inter renonaci\'on tienen media finita, es estacionario si y s\'olo si es un proceso Poisson.

\end{Coro}


%________________________________________________________________________
\subsection{Procesos Regenerativos}
%________________________________________________________________________



\begin{Note}
Si $\tilde{X}\left(t\right)$ con espacio de estados $\tilde{S}$ es regenerativo sobre $T_{n}$, entonces $X\left(t\right)=f\left(\tilde{X}\left(t\right)\right)$ tambi\'en es regenerativo sobre $T_{n}$, para cualquier funci\'on $f:\tilde{S}\rightarrow S$.
\end{Note}

\begin{Note}
Los procesos regenerativos son crudamente regenerativos, pero no al rev\'es.
\end{Note}
%\subsection*{Procesos Regenerativos: Sigman\cite{Sigman1}}
\begin{Def}[Definici\'on Cl\'asica]
Un proceso estoc\'astico $X=\left\{X\left(t\right):t\geq0\right\}$ es llamado regenerativo is existe una variable aleatoria $R_{1}>0$ tal que
\begin{itemize}
\item[i)] $\left\{X\left(t+R_{1}\right):t\geq0\right\}$ es independiente de $\left\{\left\{X\left(t\right):t<R_{1}\right\},\right\}$
\item[ii)] $\left\{X\left(t+R_{1}\right):t\geq0\right\}$ es estoc\'asticamente equivalente a $\left\{X\left(t\right):t>0\right\}$
\end{itemize}

Llamamos a $R_{1}$ tiempo de regeneraci\'on, y decimos que $X$ se regenera en este punto.
\end{Def}

$\left\{X\left(t+R_{1}\right)\right\}$ es regenerativo con tiempo de regeneraci\'on $R_{2}$, independiente de $R_{1}$ pero con la misma distribuci\'on que $R_{1}$. Procediendo de esta manera se obtiene una secuencia de variables aleatorias independientes e id\'enticamente distribuidas $\left\{R_{n}\right\}$ llamados longitudes de ciclo. Si definimos a $Z_{k}\equiv R_{1}+R_{2}+\cdots+R_{k}$, se tiene un proceso de renovaci\'on llamado proceso de renovaci\'on encajado para $X$.




\begin{Def}
Para $x$ fijo y para cada $t\geq0$, sea $I_{x}\left(t\right)=1$ si $X\left(t\right)\leq x$,  $I_{x}\left(t\right)=0$ en caso contrario, y def\'inanse los tiempos promedio
\begin{eqnarray*}
\overline{X}&=&lim_{t\rightarrow\infty}\frac{1}{t}\int_{0}^{\infty}X\left(u\right)du\\
\prob\left(X_{\infty}\leq x\right)&=&lim_{t\rightarrow\infty}\frac{1}{t}\int_{0}^{\infty}I_{x}\left(u\right)du,
\end{eqnarray*}
cuando estos l\'imites existan.
\end{Def}

Como consecuencia del teorema de Renovaci\'on-Recompensa, se tiene que el primer l\'imite  existe y es igual a la constante
\begin{eqnarray*}
\overline{X}&=&\frac{\esp\left[\int_{0}^{R_{1}}X\left(t\right)dt\right]}{\esp\left[R_{1}\right]},
\end{eqnarray*}
suponiendo que ambas esperanzas son finitas.

\begin{Note}
\begin{itemize}
\item[a)] Si el proceso regenerativo $X$ es positivo recurrente y tiene trayectorias muestrales no negativas, entonces la ecuaci\'on anterior es v\'alida.
\item[b)] Si $X$ es positivo recurrente regenerativo, podemos construir una \'unica versi\'on estacionaria de este proceso, $X_{e}=\left\{X_{e}\left(t\right)\right\}$, donde $X_{e}$ es un proceso estoc\'astico regenerativo y estrictamente estacionario, con distribuci\'on marginal distribuida como $X_{\infty}$
\end{itemize}
\end{Note}

%________________________________________________________________________
\subsection{Procesos Regenerativos}
%________________________________________________________________________

Para $\left\{X\left(t\right):t\geq0\right\}$ Proceso Estoc\'astico a tiempo continuo con estado de espacios $S$, que es un espacio m\'etrico, con trayectorias continuas por la derecha y con l\'imites por la izquierda c.s. Sea $N\left(t\right)$ un proceso de renovaci\'on en $\rea_{+}$ definido en el mismo espacio de probabilidad que $X\left(t\right)$, con tiempos de renovaci\'on $T$ y tiempos de inter-renovaci\'on $\xi_{n}=T_{n}-T_{n-1}$, con misma distribuci\'on $F$ de media finita $\mu$.



\begin{Def}
Para el proceso $\left\{\left(N\left(t\right),X\left(t\right)\right):t\geq0\right\}$, sus trayectoria muestrales en el intervalo de tiempo $\left[T_{n-1},T_{n}\right)$ est\'an descritas por
\begin{eqnarray*}
\zeta_{n}=\left(\xi_{n},\left\{X\left(T_{n-1}+t\right):0\leq t<\xi_{n}\right\}\right)
\end{eqnarray*}
Este $\zeta_{n}$ es el $n$-\'esimo segmento del proceso. El proceso es regenerativo sobre los tiempos $T_{n}$ si sus segmentos $\zeta_{n}$ son independientes e id\'enticamennte distribuidos.
\end{Def}


\begin{Note}
Si $\tilde{X}\left(t\right)$ con espacio de estados $\tilde{S}$ es regenerativo sobre $T_{n}$, entonces $X\left(t\right)=f\left(\tilde{X}\left(t\right)\right)$ tambi\'en es regenerativo sobre $T_{n}$, para cualquier funci\'on $f:\tilde{S}\rightarrow S$.
\end{Note}

\begin{Note}
Los procesos regenerativos son crudamente regenerativos, pero no al rev\'es.
\end{Note}

\begin{Def}[Definici\'on Cl\'asica]
Un proceso estoc\'astico $X=\left\{X\left(t\right):t\geq0\right\}$ es llamado regenerativo is existe una variable aleatoria $R_{1}>0$ tal que
\begin{itemize}
\item[i)] $\left\{X\left(t+R_{1}\right):t\geq0\right\}$ es independiente de $\left\{\left\{X\left(t\right):t<R_{1}\right\},\right\}$
\item[ii)] $\left\{X\left(t+R_{1}\right):t\geq0\right\}$ es estoc\'asticamente equivalente a $\left\{X\left(t\right):t>0\right\}$
\end{itemize}

Llamamos a $R_{1}$ tiempo de regeneraci\'on, y decimos que $X$ se regenera en este punto.
\end{Def}

$\left\{X\left(t+R_{1}\right)\right\}$ es regenerativo con tiempo de regeneraci\'on $R_{2}$, independiente de $R_{1}$ pero con la misma distribuci\'on que $R_{1}$. Procediendo de esta manera se obtiene una secuencia de variables aleatorias independientes e id\'enticamente distribuidas $\left\{R_{n}\right\}$ llamados longitudes de ciclo. Si definimos a $Z_{k}\equiv R_{1}+R_{2}+\cdots+R_{k}$, se tiene un proceso de renovaci\'on llamado proceso de renovaci\'on encajado para $X$.

\begin{Note}
Un proceso regenerativo con media de la longitud de ciclo finita es llamado positivo recurrente.
\end{Note}


\begin{Def}
Para $x$ fijo y para cada $t\geq0$, sea $I_{x}\left(t\right)=1$ si $X\left(t\right)\leq x$,  $I_{x}\left(t\right)=0$ en caso contrario, y def\'inanse los tiempos promedio
\begin{eqnarray*}
\overline{X}&=&lim_{t\rightarrow\infty}\frac{1}{t}\int_{0}^{\infty}X\left(u\right)du\\
\prob\left(X_{\infty}\leq x\right)&=&lim_{t\rightarrow\infty}\frac{1}{t}\int_{0}^{\infty}I_{x}\left(u\right)du,
\end{eqnarray*}
cuando estos l\'imites existan.
\end{Def}

Como consecuencia del teorema de Renovaci\'on-Recompensa, se tiene que el primer l\'imite  existe y es igual a la constante
\begin{eqnarray*}
\overline{X}&=&\frac{\esp\left[\int_{0}^{R_{1}}X\left(t\right)dt\right]}{\esp\left[R_{1}\right]},
\end{eqnarray*}
suponiendo que ambas esperanzas son finitas.

\begin{Note}
\begin{itemize}
\item[a)] Si el proceso regenerativo $X$ es positivo recurrente y tiene trayectorias muestrales no negativas, entonces la ecuaci\'on anterior es v\'alida.
\item[b)] Si $X$ es positivo recurrente regenerativo, podemos construir una \'unica versi\'on estacionaria de este proceso, $X_{e}=\left\{X_{e}\left(t\right)\right\}$, donde $X_{e}$ es un proceso estoc\'astico regenerativo y estrictamente estacionario, con distribuci\'on marginal distribuida como $X_{\infty}$
\end{itemize}
\end{Note}

%__________________________________________________________________________________________
\subsection{Procesos Regenerativos Estacionarios - Stidham \cite{Stidham}}
%__________________________________________________________________________________________


Un proceso estoc\'astico a tiempo continuo $\left\{V\left(t\right),t\geq0\right\}$ es un proceso regenerativo si existe una sucesi\'on de variables aleatorias independientes e id\'enticamente distribuidas $\left\{X_{1},X_{2},\ldots\right\}$, sucesi\'on de renovaci\'on, tal que para cualquier conjunto de Borel $A$, 

\begin{eqnarray*}
\prob\left\{V\left(t\right)\in A|X_{1}+X_{2}+\cdots+X_{R\left(t\right)}=s,\left\{V\left(\tau\right),\tau<s\right\}\right\}=\prob\left\{V\left(t-s\right)\in A|X_{1}>t-s\right\},
\end{eqnarray*}
para todo $0\leq s\leq t$, donde $R\left(t\right)=\max\left\{X_{1}+X_{2}+\cdots+X_{j}\leq t\right\}=$n\'umero de renovaciones ({\emph{puntos de regeneraci\'on}}) que ocurren en $\left[0,t\right]$. El intervalo $\left[0,X_{1}\right)$ es llamado {\emph{primer ciclo de regeneraci\'on}} de $\left\{V\left(t \right),t\geq0\right\}$, $\left[X_{1},X_{1}+X_{2}\right)$ el {\emph{segundo ciclo de regeneraci\'on}}, y as\'i sucesivamente.

Sea $X=X_{1}$ y sea $F$ la funci\'on de distrbuci\'on de $X$


\begin{Def}
Se define el proceso estacionario, $\left\{V^{*}\left(t\right),t\geq0\right\}$, para $\left\{V\left(t\right),t\geq0\right\}$ por

\begin{eqnarray*}
\prob\left\{V\left(t\right)\in A\right\}=\frac{1}{\esp\left[X\right]}\int_{0}^{\infty}\prob\left\{V\left(t+x\right)\in A|X>x\right\}\left(1-F\left(x\right)\right)dx,
\end{eqnarray*} 
para todo $t\geq0$ y todo conjunto de Borel $A$.
\end{Def}

\begin{Def}
Una distribuci\'on se dice que es {\emph{aritm\'etica}} si todos sus puntos de incremento son m\'ultiplos de la forma $0,\lambda, 2\lambda,\ldots$ para alguna $\lambda>0$ entera.
\end{Def}


\begin{Def}
Una modificaci\'on medible de un proceso $\left\{V\left(t\right),t\geq0\right\}$, es una versi\'on de este, $\left\{V\left(t,w\right)\right\}$ conjuntamente medible para $t\geq0$ y para $w\in S$, $S$ espacio de estados para $\left\{V\left(t\right),t\geq0\right\}$.
\end{Def}

\begin{Teo}
Sea $\left\{V\left(t\right),t\geq\right\}$ un proceso regenerativo no negativo con modificaci\'on medible. Sea $\esp\left[X\right]<\infty$. Entonces el proceso estacionario dado por la ecuaci\'on anterior est\'a bien definido y tiene funci\'on de distribuci\'on independiente de $t$, adem\'as
\begin{itemize}
\item[i)] \begin{eqnarray*}
\esp\left[V^{*}\left(0\right)\right]&=&\frac{\esp\left[\int_{0}^{X}V\left(s\right)ds\right]}{\esp\left[X\right]}\end{eqnarray*}
\item[ii)] Si $\esp\left[V^{*}\left(0\right)\right]<\infty$, equivalentemente, si $\esp\left[\int_{0}^{X}V\left(s\right)ds\right]<\infty$,entonces
\begin{eqnarray*}
\frac{\int_{0}^{t}V\left(s\right)ds}{t}\rightarrow\frac{\esp\left[\int_{0}^{X}V\left(s\right)ds\right]}{\esp\left[X\right]}
\end{eqnarray*}
con probabilidad 1 y en media, cuando $t\rightarrow\infty$.
\end{itemize}
\end{Teo}
%
%___________________________________________________________________________________________
%\vspace{5.5cm}
%\chapter{Cadenas de Markov estacionarias}
%\vspace{-1.0cm}


%__________________________________________________________________________________________
\subsection{Procesos Regenerativos Estacionarios - Stidham \cite{Stidham}}
%__________________________________________________________________________________________


Un proceso estoc\'astico a tiempo continuo $\left\{V\left(t\right),t\geq0\right\}$ es un proceso regenerativo si existe una sucesi\'on de variables aleatorias independientes e id\'enticamente distribuidas $\left\{X_{1},X_{2},\ldots\right\}$, sucesi\'on de renovaci\'on, tal que para cualquier conjunto de Borel $A$, 

\begin{eqnarray*}
\prob\left\{V\left(t\right)\in A|X_{1}+X_{2}+\cdots+X_{R\left(t\right)}=s,\left\{V\left(\tau\right),\tau<s\right\}\right\}=\prob\left\{V\left(t-s\right)\in A|X_{1}>t-s\right\},
\end{eqnarray*}
para todo $0\leq s\leq t$, donde $R\left(t\right)=\max\left\{X_{1}+X_{2}+\cdots+X_{j}\leq t\right\}=$n\'umero de renovaciones ({\emph{puntos de regeneraci\'on}}) que ocurren en $\left[0,t\right]$. El intervalo $\left[0,X_{1}\right)$ es llamado {\emph{primer ciclo de regeneraci\'on}} de $\left\{V\left(t \right),t\geq0\right\}$, $\left[X_{1},X_{1}+X_{2}\right)$ el {\emph{segundo ciclo de regeneraci\'on}}, y as\'i sucesivamente.

Sea $X=X_{1}$ y sea $F$ la funci\'on de distrbuci\'on de $X$


\begin{Def}
Se define el proceso estacionario, $\left\{V^{*}\left(t\right),t\geq0\right\}$, para $\left\{V\left(t\right),t\geq0\right\}$ por

\begin{eqnarray*}
\prob\left\{V\left(t\right)\in A\right\}=\frac{1}{\esp\left[X\right]}\int_{0}^{\infty}\prob\left\{V\left(t+x\right)\in A|X>x\right\}\left(1-F\left(x\right)\right)dx,
\end{eqnarray*} 
para todo $t\geq0$ y todo conjunto de Borel $A$.
\end{Def}

\begin{Def}
Una distribuci\'on se dice que es {\emph{aritm\'etica}} si todos sus puntos de incremento son m\'ultiplos de la forma $0,\lambda, 2\lambda,\ldots$ para alguna $\lambda>0$ entera.
\end{Def}


\begin{Def}
Una modificaci\'on medible de un proceso $\left\{V\left(t\right),t\geq0\right\}$, es una versi\'on de este, $\left\{V\left(t,w\right)\right\}$ conjuntamente medible para $t\geq0$ y para $w\in S$, $S$ espacio de estados para $\left\{V\left(t\right),t\geq0\right\}$.
\end{Def}

\begin{Teo}
Sea $\left\{V\left(t\right),t\geq\right\}$ un proceso regenerativo no negativo con modificaci\'on medible. Sea $\esp\left[X\right]<\infty$. Entonces el proceso estacionario dado por la ecuaci\'on anterior est\'a bien definido y tiene funci\'on de distribuci\'on independiente de $t$, adem\'as
\begin{itemize}
\item[i)] \begin{eqnarray*}
\esp\left[V^{*}\left(0\right)\right]&=&\frac{\esp\left[\int_{0}^{X}V\left(s\right)ds\right]}{\esp\left[X\right]}\end{eqnarray*}
\item[ii)] Si $\esp\left[V^{*}\left(0\right)\right]<\infty$, equivalentemente, si $\esp\left[\int_{0}^{X}V\left(s\right)ds\right]<\infty$,entonces
\begin{eqnarray*}
\frac{\int_{0}^{t}V\left(s\right)ds}{t}\rightarrow\frac{\esp\left[\int_{0}^{X}V\left(s\right)ds\right]}{\esp\left[X\right]}
\end{eqnarray*}
con probabilidad 1 y en media, cuando $t\rightarrow\infty$.
\end{itemize}
\end{Teo}

Para $\left\{X\left(t\right):t\geq0\right\}$ Proceso Estoc\'astico a tiempo continuo con estado de espacios $S$, que es un espacio m\'etrico, con trayectorias continuas por la derecha y con l\'imites por la izquierda c.s. Sea $N\left(t\right)$ un proceso de renovaci\'on en $\rea_{+}$ definido en el mismo espacio de probabilidad que $X\left(t\right)$, con tiempos de renovaci\'on $T$ y tiempos de inter-renovaci\'on $\xi_{n}=T_{n}-T_{n-1}$, con misma distribuci\'on $F$ de media finita $\mu$.



%________________________________________________________________________
\section{Procesos Regenerativos}
%________________________________________________________________________

%________________________________________________________________________
\subsection{Procesos Regenerativos Sigman, Thorisson y Wolff \cite{Sigman1}}
%________________________________________________________________________


\begin{Def}[Definici\'on Cl\'asica]
Un proceso estoc\'astico $X=\left\{X\left(t\right):t\geq0\right\}$ es llamado regenerativo is existe una variable aleatoria $R_{1}>0$ tal que
\begin{itemize}
\item[i)] $\left\{X\left(t+R_{1}\right):t\geq0\right\}$ es independiente de $\left\{\left\{X\left(t\right):t<R_{1}\right\},\right\}$
\item[ii)] $\left\{X\left(t+R_{1}\right):t\geq0\right\}$ es estoc\'asticamente equivalente a $\left\{X\left(t\right):t>0\right\}$
\end{itemize}

Llamamos a $R_{1}$ tiempo de regeneraci\'on, y decimos que $X$ se regenera en este punto.
\end{Def}

$\left\{X\left(t+R_{1}\right)\right\}$ es regenerativo con tiempo de regeneraci\'on $R_{2}$, independiente de $R_{1}$ pero con la misma distribuci\'on que $R_{1}$. Procediendo de esta manera se obtiene una secuencia de variables aleatorias independientes e id\'enticamente distribuidas $\left\{R_{n}\right\}$ llamados longitudes de ciclo. Si definimos a $Z_{k}\equiv R_{1}+R_{2}+\cdots+R_{k}$, se tiene un proceso de renovaci\'on llamado proceso de renovaci\'on encajado para $X$.


\begin{Note}
La existencia de un primer tiempo de regeneraci\'on, $R_{1}$, implica la existencia de una sucesi\'on completa de estos tiempos $R_{1},R_{2}\ldots,$ que satisfacen la propiedad deseada \cite{Sigman2}.
\end{Note}


\begin{Note} Para la cola $GI/GI/1$ los usuarios arriban con tiempos $t_{n}$ y son atendidos con tiempos de servicio $S_{n}$, los tiempos de arribo forman un proceso de renovaci\'on  con tiempos entre arribos independientes e identicamente distribuidos (\texttt{i.i.d.})$T_{n}=t_{n}-t_{n-1}$, adem\'as los tiempos de servicio son \texttt{i.i.d.} e independientes de los procesos de arribo. Por \textit{estable} se entiende que $\esp S_{n}<\esp T_{n}<\infty$.
\end{Note}
 


\begin{Def}
Para $x$ fijo y para cada $t\geq0$, sea $I_{x}\left(t\right)=1$ si $X\left(t\right)\leq x$,  $I_{x}\left(t\right)=0$ en caso contrario, y def\'inanse los tiempos promedio
\begin{eqnarray*}
\overline{X}&=&lim_{t\rightarrow\infty}\frac{1}{t}\int_{0}^{\infty}X\left(u\right)du\\
\prob\left(X_{\infty}\leq x\right)&=&lim_{t\rightarrow\infty}\frac{1}{t}\int_{0}^{\infty}I_{x}\left(u\right)du,
\end{eqnarray*}
cuando estos l\'imites existan.
\end{Def}

Como consecuencia del teorema de Renovaci\'on-Recompensa, se tiene que el primer l\'imite  existe y es igual a la constante
\begin{eqnarray*}
\overline{X}&=&\frac{\esp\left[\int_{0}^{R_{1}}X\left(t\right)dt\right]}{\esp\left[R_{1}\right]},
\end{eqnarray*}
suponiendo que ambas esperanzas son finitas.
 
\begin{Note}
Funciones de procesos regenerativos son regenerativas, es decir, si $X\left(t\right)$ es regenerativo y se define el proceso $Y\left(t\right)$ por $Y\left(t\right)=f\left(X\left(t\right)\right)$ para alguna funci\'on Borel medible $f\left(\cdot\right)$. Adem\'as $Y$ es regenerativo con los mismos tiempos de renovaci\'on que $X$. 

En general, los tiempos de renovaci\'on, $Z_{k}$ de un proceso regenerativo no requieren ser tiempos de paro con respecto a la evoluci\'on de $X\left(t\right)$.
\end{Note} 

\begin{Note}
Una funci\'on de un proceso de Markov, usualmente no ser\'a un proceso de Markov, sin embargo ser\'a regenerativo si el proceso de Markov lo es.
\end{Note}

 
\begin{Note}
Un proceso regenerativo con media de la longitud de ciclo finita es llamado positivo recurrente.
\end{Note}


\begin{Note}
\begin{itemize}
\item[a)] Si el proceso regenerativo $X$ es positivo recurrente y tiene trayectorias muestrales no negativas, entonces la ecuaci\'on anterior es v\'alida.
\item[b)] Si $X$ es positivo recurrente regenerativo, podemos construir una \'unica versi\'on estacionaria de este proceso, $X_{e}=\left\{X_{e}\left(t\right)\right\}$, donde $X_{e}$ es un proceso estoc\'astico regenerativo y estrictamente estacionario, con distribuci\'on marginal distribuida como $X_{\infty}$
\end{itemize}
\end{Note}


%__________________________________________________________________________________________
\subsection{Procesos Regenerativos Estacionarios - Stidham \cite{Stidham}}
%__________________________________________________________________________________________


Un proceso estoc\'astico a tiempo continuo $\left\{V\left(t\right),t\geq0\right\}$ es un proceso regenerativo si existe una sucesi\'on de variables aleatorias independientes e id\'enticamente distribuidas $\left\{X_{1},X_{2},\ldots\right\}$, sucesi\'on de renovaci\'on, tal que para cualquier conjunto de Borel $A$, 

\begin{eqnarray*}
\prob\left\{V\left(t\right)\in A|X_{1}+X_{2}+\cdots+X_{R\left(t\right)}=s,\left\{V\left(\tau\right),\tau<s\right\}\right\}=\prob\left\{V\left(t-s\right)\in A|X_{1}>t-s\right\},
\end{eqnarray*}
para todo $0\leq s\leq t$, donde $R\left(t\right)=\max\left\{X_{1}+X_{2}+\cdots+X_{j}\leq t\right\}=$n\'umero de renovaciones ({\emph{puntos de regeneraci\'on}}) que ocurren en $\left[0,t\right]$. El intervalo $\left[0,X_{1}\right)$ es llamado {\emph{primer ciclo de regeneraci\'on}} de $\left\{V\left(t \right),t\geq0\right\}$, $\left[X_{1},X_{1}+X_{2}\right)$ el {\emph{segundo ciclo de regeneraci\'on}}, y as\'i sucesivamente.

Sea $X=X_{1}$ y sea $F$ la funci\'on de distrbuci\'on de $X$


\begin{Def}
Se define el proceso estacionario, $\left\{V^{*}\left(t\right),t\geq0\right\}$, para $\left\{V\left(t\right),t\geq0\right\}$ por

\begin{eqnarray*}
\prob\left\{V\left(t\right)\in A\right\}=\frac{1}{\esp\left[X\right]}\int_{0}^{\infty}\prob\left\{V\left(t+x\right)\in A|X>x\right\}\left(1-F\left(x\right)\right)dx,
\end{eqnarray*} 
para todo $t\geq0$ y todo conjunto de Borel $A$.
\end{Def}

\begin{Def}
Una distribuci\'on se dice que es {\emph{aritm\'etica}} si todos sus puntos de incremento son m\'ultiplos de la forma $0,\lambda, 2\lambda,\ldots$ para alguna $\lambda>0$ entera.
\end{Def}


\begin{Def}
Una modificaci\'on medible de un proceso $\left\{V\left(t\right),t\geq0\right\}$, es una versi\'on de este, $\left\{V\left(t,w\right)\right\}$ conjuntamente medible para $t\geq0$ y para $w\in S$, $S$ espacio de estados para $\left\{V\left(t\right),t\geq0\right\}$.
\end{Def}

\begin{Teo}
Sea $\left\{V\left(t\right),t\geq\right\}$ un proceso regenerativo no negativo con modificaci\'on medible. Sea $\esp\left[X\right]<\infty$. Entonces el proceso estacionario dado por la ecuaci\'on anterior est\'a bien definido y tiene funci\'on de distribuci\'on independiente de $t$, adem\'as
\begin{itemize}
\item[i)] \begin{eqnarray*}
\esp\left[V^{*}\left(0\right)\right]&=&\frac{\esp\left[\int_{0}^{X}V\left(s\right)ds\right]}{\esp\left[X\right]}\end{eqnarray*}
\item[ii)] Si $\esp\left[V^{*}\left(0\right)\right]<\infty$, equivalentemente, si $\esp\left[\int_{0}^{X}V\left(s\right)ds\right]<\infty$,entonces
\begin{eqnarray*}
\frac{\int_{0}^{t}V\left(s\right)ds}{t}\rightarrow\frac{\esp\left[\int_{0}^{X}V\left(s\right)ds\right]}{\esp\left[X\right]}
\end{eqnarray*}
con probabilidad 1 y en media, cuando $t\rightarrow\infty$.
\end{itemize}
\end{Teo}

\begin{Coro}
Sea $\left\{V\left(t\right),t\geq0\right\}$ un proceso regenerativo no negativo, con modificaci\'on medible. Si $\esp <\infty$, $F$ es no-aritm\'etica, y para todo $x\geq0$, $P\left\{V\left(t\right)\leq x,C>x\right\}$ es de variaci\'on acotada como funci\'on de $t$ en cada intervalo finito $\left[0,\tau\right]$, entonces $V\left(t\right)$ converge en distribuci\'on  cuando $t\rightarrow\infty$ y $$\esp V=\frac{\esp \int_{0}^{X}V\left(s\right)ds}{\esp X}$$
Donde $V$ tiene la distribuci\'on l\'imite de $V\left(t\right)$ cuando $t\rightarrow\infty$.

\end{Coro}

Para el caso discreto se tienen resultados similares.



%______________________________________________________________________
\section{Procesos de Renovaci\'on}
%______________________________________________________________________

\begin{Def}\label{Def.Tn}
Sean $0\leq T_{1}\leq T_{2}\leq \ldots$ son tiempos aleatorios infinitos en los cuales ocurren ciertos eventos. El n\'umero de tiempos $T_{n}$ en el intervalo $\left[0,t\right)$ es

\begin{eqnarray}
N\left(t\right)=\sum_{n=1}^{\infty}\indora\left(T_{n}\leq t\right),
\end{eqnarray}
para $t\geq0$.
\end{Def}

Si se consideran los puntos $T_{n}$ como elementos de $\rea_{+}$, y $N\left(t\right)$ es el n\'umero de puntos en $\rea$. El proceso denotado por $\left\{N\left(t\right):t\geq0\right\}$, denotado por $N\left(t\right)$, es un proceso puntual en $\rea_{+}$. Los $T_{n}$ son los tiempos de ocurrencia, el proceso puntual $N\left(t\right)$ es simple si su n\'umero de ocurrencias son distintas: $0<T_{1}<T_{2}<\ldots$ casi seguramente.

\begin{Def}
Un proceso puntual $N\left(t\right)$ es un proceso de renovaci\'on si los tiempos de interocurrencia $\xi_{n}=T_{n}-T_{n-1}$, para $n\geq1$, son independientes e identicamente distribuidos con distribuci\'on $F$, donde $F\left(0\right)=0$ y $T_{0}=0$. Los $T_{n}$ son llamados tiempos de renovaci\'on, referente a la independencia o renovaci\'on de la informaci\'on estoc\'astica en estos tiempos. Los $\xi_{n}$ son los tiempos de inter-renovaci\'on, y $N\left(t\right)$ es el n\'umero de renovaciones en el intervalo $\left[0,t\right)$
\end{Def}


\begin{Note}
Para definir un proceso de renovaci\'on para cualquier contexto, solamente hay que especificar una distribuci\'on $F$, con $F\left(0\right)=0$, para los tiempos de inter-renovaci\'on. La funci\'on $F$ en turno degune las otra variables aleatorias. De manera formal, existe un espacio de probabilidad y una sucesi\'on de variables aleatorias $\xi_{1},\xi_{2},\ldots$ definidas en este con distribuci\'on $F$. Entonces las otras cantidades son $T_{n}=\sum_{k=1}^{n}\xi_{k}$ y $N\left(t\right)=\sum_{n=1}^{\infty}\indora\left(T_{n}\leq t\right)$, donde $T_{n}\rightarrow\infty$ casi seguramente por la Ley Fuerte de los Grandes Números.
\end{Note}

%___________________________________________________________________________________________
%
\subsection{Teorema Principal de Renovaci\'on}
%___________________________________________________________________________________________
%

\begin{Note} Una funci\'on $h:\rea_{+}\rightarrow\rea$ es Directamente Riemann Integrable en los siguientes casos:
\begin{itemize}
\item[a)] $h\left(t\right)\geq0$ es decreciente y Riemann Integrable.
\item[b)] $h$ es continua excepto posiblemente en un conjunto de Lebesgue de medida 0, y $|h\left(t\right)|\leq b\left(t\right)$, donde $b$ es DRI.
\end{itemize}
\end{Note}

\begin{Teo}[Teorema Principal de Renovaci\'on]
Si $F$ es no aritm\'etica y $h\left(t\right)$ es Directamente Riemann Integrable (DRI), entonces

\begin{eqnarray*}
lim_{t\rightarrow\infty}U\star h=\frac{1}{\mu}\int_{\rea_{+}}h\left(s\right)ds.
\end{eqnarray*}
\end{Teo}

\begin{Prop}
Cualquier funci\'on $H\left(t\right)$ acotada en intervalos finitos y que es 0 para $t<0$ puede expresarse como
\begin{eqnarray*}
H\left(t\right)=U\star h\left(t\right)\textrm{,  donde }h\left(t\right)=H\left(t\right)-F\star H\left(t\right)
\end{eqnarray*}
\end{Prop}

\begin{Def}
Un proceso estoc\'astico $X\left(t\right)$ es crudamente regenerativo en un tiempo aleatorio positivo $T$ si
\begin{eqnarray*}
\esp\left[X\left(T+t\right)|T\right]=\esp\left[X\left(t\right)\right]\textrm{, para }t\geq0,\end{eqnarray*}
y con las esperanzas anteriores finitas.
\end{Def}

\begin{Prop}
Sup\'ongase que $X\left(t\right)$ es un proceso crudamente regenerativo en $T$, que tiene distribuci\'on $F$. Si $\esp\left[X\left(t\right)\right]$ es acotado en intervalos finitos, entonces
\begin{eqnarray*}
\esp\left[X\left(t\right)\right]=U\star h\left(t\right)\textrm{,  donde }h\left(t\right)=\esp\left[X\left(t\right)\indora\left(T>t\right)\right].
\end{eqnarray*}
\end{Prop}

\begin{Teo}[Regeneraci\'on Cruda]
Sup\'ongase que $X\left(t\right)$ es un proceso con valores positivo crudamente regenerativo en $T$, y def\'inase $M=\sup\left\{|X\left(t\right)|:t\leq T\right\}$. Si $T$ es no aritm\'etico y $M$ y $MT$ tienen media finita, entonces
\begin{eqnarray*}
lim_{t\rightarrow\infty}\esp\left[X\left(t\right)\right]=\frac{1}{\mu}\int_{\rea_{+}}h\left(s\right)ds,
\end{eqnarray*}
donde $h\left(t\right)=\esp\left[X\left(t\right)\indora\left(T>t\right)\right]$.
\end{Teo}

%___________________________________________________________________________________________
%
\subsection{Propiedades de los Procesos de Renovaci\'on}
%___________________________________________________________________________________________
%

Los tiempos $T_{n}$ est\'an relacionados con los conteos de $N\left(t\right)$ por

\begin{eqnarray*}
\left\{N\left(t\right)\geq n\right\}&=&\left\{T_{n}\leq t\right\}\\
T_{N\left(t\right)}\leq &t&<T_{N\left(t\right)+1},
\end{eqnarray*}

adem\'as $N\left(T_{n}\right)=n$, y 

\begin{eqnarray*}
N\left(t\right)=\max\left\{n:T_{n}\leq t\right\}=\min\left\{n:T_{n+1}>t\right\}
\end{eqnarray*}

Por propiedades de la convoluci\'on se sabe que

\begin{eqnarray*}
P\left\{T_{n}\leq t\right\}=F^{n\star}\left(t\right)
\end{eqnarray*}
que es la $n$-\'esima convoluci\'on de $F$. Entonces 

\begin{eqnarray*}
\left\{N\left(t\right)\geq n\right\}&=&\left\{T_{n}\leq t\right\}\\
P\left\{N\left(t\right)\leq n\right\}&=&1-F^{\left(n+1\right)\star}\left(t\right)
\end{eqnarray*}

Adem\'as usando el hecho de que $\esp\left[N\left(t\right)\right]=\sum_{n=1}^{\infty}P\left\{N\left(t\right)\geq n\right\}$
se tiene que

\begin{eqnarray*}
\esp\left[N\left(t\right)\right]=\sum_{n=1}^{\infty}F^{n\star}\left(t\right)
\end{eqnarray*}

\begin{Prop}
Para cada $t\geq0$, la funci\'on generadora de momentos $\esp\left[e^{\alpha N\left(t\right)}\right]$ existe para alguna $\alpha$ en una vecindad del 0, y de aqu\'i que $\esp\left[N\left(t\right)^{m}\right]<\infty$, para $m\geq1$.
\end{Prop}


\begin{Note}
Si el primer tiempo de renovaci\'on $\xi_{1}$ no tiene la misma distribuci\'on que el resto de las $\xi_{n}$, para $n\geq2$, a $N\left(t\right)$ se le llama Proceso de Renovaci\'on retardado, donde si $\xi$ tiene distribuci\'on $G$, entonces el tiempo $T_{n}$ de la $n$-\'esima renovaci\'on tiene distribuci\'on $G\star F^{\left(n-1\right)\star}\left(t\right)$
\end{Note}


\begin{Teo}
Para una constante $\mu\leq\infty$ ( o variable aleatoria), las siguientes expresiones son equivalentes:

\begin{eqnarray}
lim_{n\rightarrow\infty}n^{-1}T_{n}&=&\mu,\textrm{ c.s.}\\
lim_{t\rightarrow\infty}t^{-1}N\left(t\right)&=&1/\mu,\textrm{ c.s.}
\end{eqnarray}
\end{Teo}


Es decir, $T_{n}$ satisface la Ley Fuerte de los Grandes N\'umeros s\'i y s\'olo s\'i $N\left/t\right)$ la cumple.


\begin{Coro}[Ley Fuerte de los Grandes N\'umeros para Procesos de Renovaci\'on]
Si $N\left(t\right)$ es un proceso de renovaci\'on cuyos tiempos de inter-renovaci\'on tienen media $\mu\leq\infty$, entonces
\begin{eqnarray}
t^{-1}N\left(t\right)\rightarrow 1/\mu,\textrm{ c.s. cuando }t\rightarrow\infty.
\end{eqnarray}

\end{Coro}


Considerar el proceso estoc\'astico de valores reales $\left\{Z\left(t\right):t\geq0\right\}$ en el mismo espacio de probabilidad que $N\left(t\right)$

\begin{Def}
Para el proceso $\left\{Z\left(t\right):t\geq0\right\}$ se define la fluctuaci\'on m\'axima de $Z\left(t\right)$ en el intervalo $\left(T_{n-1},T_{n}\right]$:
\begin{eqnarray*}
M_{n}=\sup_{T_{n-1}<t\leq T_{n}}|Z\left(t\right)-Z\left(T_{n-1}\right)|
\end{eqnarray*}
\end{Def}

\begin{Teo}
Sup\'ongase que $n^{-1}T_{n}\rightarrow\mu$ c.s. cuando $n\rightarrow\infty$, donde $\mu\leq\infty$ es una constante o variable aleatoria. Sea $a$ una constante o variable aleatoria que puede ser infinita cuando $\mu$ es finita, y considere las expresiones l\'imite:
\begin{eqnarray}
lim_{n\rightarrow\infty}n^{-1}Z\left(T_{n}\right)&=&a,\textrm{ c.s.}\\
lim_{t\rightarrow\infty}t^{-1}Z\left(t\right)&=&a/\mu,\textrm{ c.s.}
\end{eqnarray}
La segunda expresi\'on implica la primera. Conversamente, la primera implica la segunda si el proceso $Z\left(t\right)$ es creciente, o si $lim_{n\rightarrow\infty}n^{-1}M_{n}=0$ c.s.
\end{Teo}

\begin{Coro}
Si $N\left(t\right)$ es un proceso de renovaci\'on, y $\left(Z\left(T_{n}\right)-Z\left(T_{n-1}\right),M_{n}\right)$, para $n\geq1$, son variables aleatorias independientes e id\'enticamente distribuidas con media finita, entonces,
\begin{eqnarray}
lim_{t\rightarrow\infty}t^{-1}Z\left(t\right)\rightarrow\frac{\esp\left[Z\left(T_{1}\right)-Z\left(T_{0}\right)\right]}{\esp\left[T_{1}\right]},\textrm{ c.s. cuando  }t\rightarrow\infty.
\end{eqnarray}
\end{Coro}



%___________________________________________________________________________________________
%
\subsection{Propiedades de los Procesos de Renovaci\'on}
%___________________________________________________________________________________________
%

Los tiempos $T_{n}$ est\'an relacionados con los conteos de $N\left(t\right)$ por

\begin{eqnarray*}
\left\{N\left(t\right)\geq n\right\}&=&\left\{T_{n}\leq t\right\}\\
T_{N\left(t\right)}\leq &t&<T_{N\left(t\right)+1},
\end{eqnarray*}

adem\'as $N\left(T_{n}\right)=n$, y 

\begin{eqnarray*}
N\left(t\right)=\max\left\{n:T_{n}\leq t\right\}=\min\left\{n:T_{n+1}>t\right\}
\end{eqnarray*}

Por propiedades de la convoluci\'on se sabe que

\begin{eqnarray*}
P\left\{T_{n}\leq t\right\}=F^{n\star}\left(t\right)
\end{eqnarray*}
que es la $n$-\'esima convoluci\'on de $F$. Entonces 

\begin{eqnarray*}
\left\{N\left(t\right)\geq n\right\}&=&\left\{T_{n}\leq t\right\}\\
P\left\{N\left(t\right)\leq n\right\}&=&1-F^{\left(n+1\right)\star}\left(t\right)
\end{eqnarray*}

Adem\'as usando el hecho de que $\esp\left[N\left(t\right)\right]=\sum_{n=1}^{\infty}P\left\{N\left(t\right)\geq n\right\}$
se tiene que

\begin{eqnarray*}
\esp\left[N\left(t\right)\right]=\sum_{n=1}^{\infty}F^{n\star}\left(t\right)
\end{eqnarray*}

\begin{Prop}
Para cada $t\geq0$, la funci\'on generadora de momentos $\esp\left[e^{\alpha N\left(t\right)}\right]$ existe para alguna $\alpha$ en una vecindad del 0, y de aqu\'i que $\esp\left[N\left(t\right)^{m}\right]<\infty$, para $m\geq1$.
\end{Prop}


\begin{Note}
Si el primer tiempo de renovaci\'on $\xi_{1}$ no tiene la misma distribuci\'on que el resto de las $\xi_{n}$, para $n\geq2$, a $N\left(t\right)$ se le llama Proceso de Renovaci\'on retardado, donde si $\xi$ tiene distribuci\'on $G$, entonces el tiempo $T_{n}$ de la $n$-\'esima renovaci\'on tiene distribuci\'on $G\star F^{\left(n-1\right)\star}\left(t\right)$
\end{Note}


\begin{Teo}
Para una constante $\mu\leq\infty$ ( o variable aleatoria), las siguientes expresiones son equivalentes:

\begin{eqnarray}
lim_{n\rightarrow\infty}n^{-1}T_{n}&=&\mu,\textrm{ c.s.}\\
lim_{t\rightarrow\infty}t^{-1}N\left(t\right)&=&1/\mu,\textrm{ c.s.}
\end{eqnarray}
\end{Teo}


Es decir, $T_{n}$ satisface la Ley Fuerte de los Grandes N\'umeros s\'i y s\'olo s\'i $N\left/t\right)$ la cumple.


\begin{Coro}[Ley Fuerte de los Grandes N\'umeros para Procesos de Renovaci\'on]
Si $N\left(t\right)$ es un proceso de renovaci\'on cuyos tiempos de inter-renovaci\'on tienen media $\mu\leq\infty$, entonces
\begin{eqnarray}
t^{-1}N\left(t\right)\rightarrow 1/\mu,\textrm{ c.s. cuando }t\rightarrow\infty.
\end{eqnarray}

\end{Coro}


Considerar el proceso estoc\'astico de valores reales $\left\{Z\left(t\right):t\geq0\right\}$ en el mismo espacio de probabilidad que $N\left(t\right)$

\begin{Def}
Para el proceso $\left\{Z\left(t\right):t\geq0\right\}$ se define la fluctuaci\'on m\'axima de $Z\left(t\right)$ en el intervalo $\left(T_{n-1},T_{n}\right]$:
\begin{eqnarray*}
M_{n}=\sup_{T_{n-1}<t\leq T_{n}}|Z\left(t\right)-Z\left(T_{n-1}\right)|
\end{eqnarray*}
\end{Def}

\begin{Teo}
Sup\'ongase que $n^{-1}T_{n}\rightarrow\mu$ c.s. cuando $n\rightarrow\infty$, donde $\mu\leq\infty$ es una constante o variable aleatoria. Sea $a$ una constante o variable aleatoria que puede ser infinita cuando $\mu$ es finita, y considere las expresiones l\'imite:
\begin{eqnarray}
lim_{n\rightarrow\infty}n^{-1}Z\left(T_{n}\right)&=&a,\textrm{ c.s.}\\
lim_{t\rightarrow\infty}t^{-1}Z\left(t\right)&=&a/\mu,\textrm{ c.s.}
\end{eqnarray}
La segunda expresi\'on implica la primera. Conversamente, la primera implica la segunda si el proceso $Z\left(t\right)$ es creciente, o si $lim_{n\rightarrow\infty}n^{-1}M_{n}=0$ c.s.
\end{Teo}

\begin{Coro}
Si $N\left(t\right)$ es un proceso de renovaci\'on, y $\left(Z\left(T_{n}\right)-Z\left(T_{n-1}\right),M_{n}\right)$, para $n\geq1$, son variables aleatorias independientes e id\'enticamente distribuidas con media finita, entonces,
\begin{eqnarray}
lim_{t\rightarrow\infty}t^{-1}Z\left(t\right)\rightarrow\frac{\esp\left[Z\left(T_{1}\right)-Z\left(T_{0}\right)\right]}{\esp\left[T_{1}\right]},\textrm{ c.s. cuando  }t\rightarrow\infty.
\end{eqnarray}
\end{Coro}


%___________________________________________________________________________________________
%
\subsection{Propiedades de los Procesos de Renovaci\'on}
%___________________________________________________________________________________________
%

Los tiempos $T_{n}$ est\'an relacionados con los conteos de $N\left(t\right)$ por

\begin{eqnarray*}
\left\{N\left(t\right)\geq n\right\}&=&\left\{T_{n}\leq t\right\}\\
T_{N\left(t\right)}\leq &t&<T_{N\left(t\right)+1},
\end{eqnarray*}

adem\'as $N\left(T_{n}\right)=n$, y 

\begin{eqnarray*}
N\left(t\right)=\max\left\{n:T_{n}\leq t\right\}=\min\left\{n:T_{n+1}>t\right\}
\end{eqnarray*}

Por propiedades de la convoluci\'on se sabe que

\begin{eqnarray*}
P\left\{T_{n}\leq t\right\}=F^{n\star}\left(t\right)
\end{eqnarray*}
que es la $n$-\'esima convoluci\'on de $F$. Entonces 

\begin{eqnarray*}
\left\{N\left(t\right)\geq n\right\}&=&\left\{T_{n}\leq t\right\}\\
P\left\{N\left(t\right)\leq n\right\}&=&1-F^{\left(n+1\right)\star}\left(t\right)
\end{eqnarray*}

Adem\'as usando el hecho de que $\esp\left[N\left(t\right)\right]=\sum_{n=1}^{\infty}P\left\{N\left(t\right)\geq n\right\}$
se tiene que

\begin{eqnarray*}
\esp\left[N\left(t\right)\right]=\sum_{n=1}^{\infty}F^{n\star}\left(t\right)
\end{eqnarray*}

\begin{Prop}
Para cada $t\geq0$, la funci\'on generadora de momentos $\esp\left[e^{\alpha N\left(t\right)}\right]$ existe para alguna $\alpha$ en una vecindad del 0, y de aqu\'i que $\esp\left[N\left(t\right)^{m}\right]<\infty$, para $m\geq1$.
\end{Prop}


\begin{Note}
Si el primer tiempo de renovaci\'on $\xi_{1}$ no tiene la misma distribuci\'on que el resto de las $\xi_{n}$, para $n\geq2$, a $N\left(t\right)$ se le llama Proceso de Renovaci\'on retardado, donde si $\xi$ tiene distribuci\'on $G$, entonces el tiempo $T_{n}$ de la $n$-\'esima renovaci\'on tiene distribuci\'on $G\star F^{\left(n-1\right)\star}\left(t\right)$
\end{Note}


\begin{Teo}
Para una constante $\mu\leq\infty$ ( o variable aleatoria), las siguientes expresiones son equivalentes:

\begin{eqnarray}
lim_{n\rightarrow\infty}n^{-1}T_{n}&=&\mu,\textrm{ c.s.}\\
lim_{t\rightarrow\infty}t^{-1}N\left(t\right)&=&1/\mu,\textrm{ c.s.}
\end{eqnarray}
\end{Teo}


Es decir, $T_{n}$ satisface la Ley Fuerte de los Grandes N\'umeros s\'i y s\'olo s\'i $N\left/t\right)$ la cumple.


\begin{Coro}[Ley Fuerte de los Grandes N\'umeros para Procesos de Renovaci\'on]
Si $N\left(t\right)$ es un proceso de renovaci\'on cuyos tiempos de inter-renovaci\'on tienen media $\mu\leq\infty$, entonces
\begin{eqnarray}
t^{-1}N\left(t\right)\rightarrow 1/\mu,\textrm{ c.s. cuando }t\rightarrow\infty.
\end{eqnarray}

\end{Coro}


Considerar el proceso estoc\'astico de valores reales $\left\{Z\left(t\right):t\geq0\right\}$ en el mismo espacio de probabilidad que $N\left(t\right)$

\begin{Def}
Para el proceso $\left\{Z\left(t\right):t\geq0\right\}$ se define la fluctuaci\'on m\'axima de $Z\left(t\right)$ en el intervalo $\left(T_{n-1},T_{n}\right]$:
\begin{eqnarray*}
M_{n}=\sup_{T_{n-1}<t\leq T_{n}}|Z\left(t\right)-Z\left(T_{n-1}\right)|
\end{eqnarray*}
\end{Def}

\begin{Teo}
Sup\'ongase que $n^{-1}T_{n}\rightarrow\mu$ c.s. cuando $n\rightarrow\infty$, donde $\mu\leq\infty$ es una constante o variable aleatoria. Sea $a$ una constante o variable aleatoria que puede ser infinita cuando $\mu$ es finita, y considere las expresiones l\'imite:
\begin{eqnarray}
lim_{n\rightarrow\infty}n^{-1}Z\left(T_{n}\right)&=&a,\textrm{ c.s.}\\
lim_{t\rightarrow\infty}t^{-1}Z\left(t\right)&=&a/\mu,\textrm{ c.s.}
\end{eqnarray}
La segunda expresi\'on implica la primera. Conversamente, la primera implica la segunda si el proceso $Z\left(t\right)$ es creciente, o si $lim_{n\rightarrow\infty}n^{-1}M_{n}=0$ c.s.
\end{Teo}

\begin{Coro}
Si $N\left(t\right)$ es un proceso de renovaci\'on, y $\left(Z\left(T_{n}\right)-Z\left(T_{n-1}\right),M_{n}\right)$, para $n\geq1$, son variables aleatorias independientes e id\'enticamente distribuidas con media finita, entonces,
\begin{eqnarray}
lim_{t\rightarrow\infty}t^{-1}Z\left(t\right)\rightarrow\frac{\esp\left[Z\left(T_{1}\right)-Z\left(T_{0}\right)\right]}{\esp\left[T_{1}\right]},\textrm{ c.s. cuando  }t\rightarrow\infty.
\end{eqnarray}
\end{Coro}

%___________________________________________________________________________________________
%
\subsection{Propiedades de los Procesos de Renovaci\'on}
%___________________________________________________________________________________________
%

Los tiempos $T_{n}$ est\'an relacionados con los conteos de $N\left(t\right)$ por

\begin{eqnarray*}
\left\{N\left(t\right)\geq n\right\}&=&\left\{T_{n}\leq t\right\}\\
T_{N\left(t\right)}\leq &t&<T_{N\left(t\right)+1},
\end{eqnarray*}

adem\'as $N\left(T_{n}\right)=n$, y 

\begin{eqnarray*}
N\left(t\right)=\max\left\{n:T_{n}\leq t\right\}=\min\left\{n:T_{n+1}>t\right\}
\end{eqnarray*}

Por propiedades de la convoluci\'on se sabe que

\begin{eqnarray*}
P\left\{T_{n}\leq t\right\}=F^{n\star}\left(t\right)
\end{eqnarray*}
que es la $n$-\'esima convoluci\'on de $F$. Entonces 

\begin{eqnarray*}
\left\{N\left(t\right)\geq n\right\}&=&\left\{T_{n}\leq t\right\}\\
P\left\{N\left(t\right)\leq n\right\}&=&1-F^{\left(n+1\right)\star}\left(t\right)
\end{eqnarray*}

Adem\'as usando el hecho de que $\esp\left[N\left(t\right)\right]=\sum_{n=1}^{\infty}P\left\{N\left(t\right)\geq n\right\}$
se tiene que

\begin{eqnarray*}
\esp\left[N\left(t\right)\right]=\sum_{n=1}^{\infty}F^{n\star}\left(t\right)
\end{eqnarray*}

\begin{Prop}
Para cada $t\geq0$, la funci\'on generadora de momentos $\esp\left[e^{\alpha N\left(t\right)}\right]$ existe para alguna $\alpha$ en una vecindad del 0, y de aqu\'i que $\esp\left[N\left(t\right)^{m}\right]<\infty$, para $m\geq1$.
\end{Prop}


\begin{Note}
Si el primer tiempo de renovaci\'on $\xi_{1}$ no tiene la misma distribuci\'on que el resto de las $\xi_{n}$, para $n\geq2$, a $N\left(t\right)$ se le llama Proceso de Renovaci\'on retardado, donde si $\xi$ tiene distribuci\'on $G$, entonces el tiempo $T_{n}$ de la $n$-\'esima renovaci\'on tiene distribuci\'on $G\star F^{\left(n-1\right)\star}\left(t\right)$
\end{Note}


\begin{Teo}
Para una constante $\mu\leq\infty$ ( o variable aleatoria), las siguientes expresiones son equivalentes:

\begin{eqnarray}
lim_{n\rightarrow\infty}n^{-1}T_{n}&=&\mu,\textrm{ c.s.}\\
lim_{t\rightarrow\infty}t^{-1}N\left(t\right)&=&1/\mu,\textrm{ c.s.}
\end{eqnarray}
\end{Teo}


Es decir, $T_{n}$ satisface la Ley Fuerte de los Grandes N\'umeros s\'i y s\'olo s\'i $N\left/t\right)$ la cumple.


\begin{Coro}[Ley Fuerte de los Grandes N\'umeros para Procesos de Renovaci\'on]
Si $N\left(t\right)$ es un proceso de renovaci\'on cuyos tiempos de inter-renovaci\'on tienen media $\mu\leq\infty$, entonces
\begin{eqnarray}
t^{-1}N\left(t\right)\rightarrow 1/\mu,\textrm{ c.s. cuando }t\rightarrow\infty.
\end{eqnarray}

\end{Coro}


Considerar el proceso estoc\'astico de valores reales $\left\{Z\left(t\right):t\geq0\right\}$ en el mismo espacio de probabilidad que $N\left(t\right)$

\begin{Def}
Para el proceso $\left\{Z\left(t\right):t\geq0\right\}$ se define la fluctuaci\'on m\'axima de $Z\left(t\right)$ en el intervalo $\left(T_{n-1},T_{n}\right]$:
\begin{eqnarray*}
M_{n}=\sup_{T_{n-1}<t\leq T_{n}}|Z\left(t\right)-Z\left(T_{n-1}\right)|
\end{eqnarray*}
\end{Def}

\begin{Teo}
Sup\'ongase que $n^{-1}T_{n}\rightarrow\mu$ c.s. cuando $n\rightarrow\infty$, donde $\mu\leq\infty$ es una constante o variable aleatoria. Sea $a$ una constante o variable aleatoria que puede ser infinita cuando $\mu$ es finita, y considere las expresiones l\'imite:
\begin{eqnarray}
lim_{n\rightarrow\infty}n^{-1}Z\left(T_{n}\right)&=&a,\textrm{ c.s.}\\
lim_{t\rightarrow\infty}t^{-1}Z\left(t\right)&=&a/\mu,\textrm{ c.s.}
\end{eqnarray}
La segunda expresi\'on implica la primera. Conversamente, la primera implica la segunda si el proceso $Z\left(t\right)$ es creciente, o si $lim_{n\rightarrow\infty}n^{-1}M_{n}=0$ c.s.
\end{Teo}

\begin{Coro}
Si $N\left(t\right)$ es un proceso de renovaci\'on, y $\left(Z\left(T_{n}\right)-Z\left(T_{n-1}\right),M_{n}\right)$, para $n\geq1$, son variables aleatorias independientes e id\'enticamente distribuidas con media finita, entonces,
\begin{eqnarray}
lim_{t\rightarrow\infty}t^{-1}Z\left(t\right)\rightarrow\frac{\esp\left[Z\left(T_{1}\right)-Z\left(T_{0}\right)\right]}{\esp\left[T_{1}\right]},\textrm{ c.s. cuando  }t\rightarrow\infty.
\end{eqnarray}
\end{Coro}
%___________________________________________________________________________________________
%
\subsection{Propiedades de los Procesos de Renovaci\'on}
%___________________________________________________________________________________________
%

Los tiempos $T_{n}$ est\'an relacionados con los conteos de $N\left(t\right)$ por

\begin{eqnarray*}
\left\{N\left(t\right)\geq n\right\}&=&\left\{T_{n}\leq t\right\}\\
T_{N\left(t\right)}\leq &t&<T_{N\left(t\right)+1},
\end{eqnarray*}

adem\'as $N\left(T_{n}\right)=n$, y 

\begin{eqnarray*}
N\left(t\right)=\max\left\{n:T_{n}\leq t\right\}=\min\left\{n:T_{n+1}>t\right\}
\end{eqnarray*}

Por propiedades de la convoluci\'on se sabe que

\begin{eqnarray*}
P\left\{T_{n}\leq t\right\}=F^{n\star}\left(t\right)
\end{eqnarray*}
que es la $n$-\'esima convoluci\'on de $F$. Entonces 

\begin{eqnarray*}
\left\{N\left(t\right)\geq n\right\}&=&\left\{T_{n}\leq t\right\}\\
P\left\{N\left(t\right)\leq n\right\}&=&1-F^{\left(n+1\right)\star}\left(t\right)
\end{eqnarray*}

Adem\'as usando el hecho de que $\esp\left[N\left(t\right)\right]=\sum_{n=1}^{\infty}P\left\{N\left(t\right)\geq n\right\}$
se tiene que

\begin{eqnarray*}
\esp\left[N\left(t\right)\right]=\sum_{n=1}^{\infty}F^{n\star}\left(t\right)
\end{eqnarray*}

\begin{Prop}
Para cada $t\geq0$, la funci\'on generadora de momentos $\esp\left[e^{\alpha N\left(t\right)}\right]$ existe para alguna $\alpha$ en una vecindad del 0, y de aqu\'i que $\esp\left[N\left(t\right)^{m}\right]<\infty$, para $m\geq1$.
\end{Prop}


\begin{Note}
Si el primer tiempo de renovaci\'on $\xi_{1}$ no tiene la misma distribuci\'on que el resto de las $\xi_{n}$, para $n\geq2$, a $N\left(t\right)$ se le llama Proceso de Renovaci\'on retardado, donde si $\xi$ tiene distribuci\'on $G$, entonces el tiempo $T_{n}$ de la $n$-\'esima renovaci\'on tiene distribuci\'on $G\star F^{\left(n-1\right)\star}\left(t\right)$
\end{Note}


\begin{Teo}
Para una constante $\mu\leq\infty$ ( o variable aleatoria), las siguientes expresiones son equivalentes:

\begin{eqnarray}
lim_{n\rightarrow\infty}n^{-1}T_{n}&=&\mu,\textrm{ c.s.}\\
lim_{t\rightarrow\infty}t^{-1}N\left(t\right)&=&1/\mu,\textrm{ c.s.}
\end{eqnarray}
\end{Teo}


Es decir, $T_{n}$ satisface la Ley Fuerte de los Grandes N\'umeros s\'i y s\'olo s\'i $N\left/t\right)$ la cumple.


\begin{Coro}[Ley Fuerte de los Grandes N\'umeros para Procesos de Renovaci\'on]
Si $N\left(t\right)$ es un proceso de renovaci\'on cuyos tiempos de inter-renovaci\'on tienen media $\mu\leq\infty$, entonces
\begin{eqnarray}
t^{-1}N\left(t\right)\rightarrow 1/\mu,\textrm{ c.s. cuando }t\rightarrow\infty.
\end{eqnarray}

\end{Coro}


Considerar el proceso estoc\'astico de valores reales $\left\{Z\left(t\right):t\geq0\right\}$ en el mismo espacio de probabilidad que $N\left(t\right)$

\begin{Def}
Para el proceso $\left\{Z\left(t\right):t\geq0\right\}$ se define la fluctuaci\'on m\'axima de $Z\left(t\right)$ en el intervalo $\left(T_{n-1},T_{n}\right]$:
\begin{eqnarray*}
M_{n}=\sup_{T_{n-1}<t\leq T_{n}}|Z\left(t\right)-Z\left(T_{n-1}\right)|
\end{eqnarray*}
\end{Def}

\begin{Teo}
Sup\'ongase que $n^{-1}T_{n}\rightarrow\mu$ c.s. cuando $n\rightarrow\infty$, donde $\mu\leq\infty$ es una constante o variable aleatoria. Sea $a$ una constante o variable aleatoria que puede ser infinita cuando $\mu$ es finita, y considere las expresiones l\'imite:
\begin{eqnarray}
lim_{n\rightarrow\infty}n^{-1}Z\left(T_{n}\right)&=&a,\textrm{ c.s.}\\
lim_{t\rightarrow\infty}t^{-1}Z\left(t\right)&=&a/\mu,\textrm{ c.s.}
\end{eqnarray}
La segunda expresi\'on implica la primera. Conversamente, la primera implica la segunda si el proceso $Z\left(t\right)$ es creciente, o si $lim_{n\rightarrow\infty}n^{-1}M_{n}=0$ c.s.
\end{Teo}

\begin{Coro}
Si $N\left(t\right)$ es un proceso de renovaci\'on, y $\left(Z\left(T_{n}\right)-Z\left(T_{n-1}\right),M_{n}\right)$, para $n\geq1$, son variables aleatorias independientes e id\'enticamente distribuidas con media finita, entonces,
\begin{eqnarray}
lim_{t\rightarrow\infty}t^{-1}Z\left(t\right)\rightarrow\frac{\esp\left[Z\left(T_{1}\right)-Z\left(T_{0}\right)\right]}{\esp\left[T_{1}\right]},\textrm{ c.s. cuando  }t\rightarrow\infty.
\end{eqnarray}
\end{Coro}


%___________________________________________________________________________________________
%
\subsection{Funci\'on de Renovaci\'on}
%___________________________________________________________________________________________
%


\begin{Def}
Sea $h\left(t\right)$ funci\'on de valores reales en $\rea$ acotada en intervalos finitos e igual a cero para $t<0$ La ecuaci\'on de renovaci\'on para $h\left(t\right)$ y la distribuci\'on $F$ es

\begin{eqnarray}\label{Ec.Renovacion}
H\left(t\right)=h\left(t\right)+\int_{\left[0,t\right]}H\left(t-s\right)dF\left(s\right)\textrm{,    }t\geq0,
\end{eqnarray}
donde $H\left(t\right)$ es una funci\'on de valores reales. Esto es $H=h+F\star H$. Decimos que $H\left(t\right)$ es soluci\'on de esta ecuaci\'on si satisface la ecuaci\'on, y es acotada en intervalos finitos e iguales a cero para $t<0$.
\end{Def}

\begin{Prop}
La funci\'on $U\star h\left(t\right)$ es la \'unica soluci\'on de la ecuaci\'on de renovaci\'on (\ref{Ec.Renovacion}).
\end{Prop}

\begin{Teo}[Teorema Renovaci\'on Elemental]
\begin{eqnarray*}
t^{-1}U\left(t\right)\rightarrow 1/\mu\textrm{,    cuando }t\rightarrow\infty.
\end{eqnarray*}
\end{Teo}

%___________________________________________________________________________________________
%
\subsection{Funci\'on de Renovaci\'on}
%___________________________________________________________________________________________
%


Sup\'ongase que $N\left(t\right)$ es un proceso de renovaci\'on con distribuci\'on $F$ con media finita $\mu$.

\begin{Def}
La funci\'on de renovaci\'on asociada con la distribuci\'on $F$, del proceso $N\left(t\right)$, es
\begin{eqnarray*}
U\left(t\right)=\sum_{n=1}^{\infty}F^{n\star}\left(t\right),\textrm{   }t\geq0,
\end{eqnarray*}
donde $F^{0\star}\left(t\right)=\indora\left(t\geq0\right)$.
\end{Def}


\begin{Prop}
Sup\'ongase que la distribuci\'on de inter-renovaci\'on $F$ tiene densidad $f$. Entonces $U\left(t\right)$ tambi\'en tiene densidad, para $t>0$, y es $U^{'}\left(t\right)=\sum_{n=0}^{\infty}f^{n\star}\left(t\right)$. Adem\'as
\begin{eqnarray*}
\prob\left\{N\left(t\right)>N\left(t-\right)\right\}=0\textrm{,   }t\geq0.
\end{eqnarray*}
\end{Prop}

\begin{Def}
La Transformada de Laplace-Stieljes de $F$ est\'a dada por

\begin{eqnarray*}
\hat{F}\left(\alpha\right)=\int_{\rea_{+}}e^{-\alpha t}dF\left(t\right)\textrm{,  }\alpha\geq0.
\end{eqnarray*}
\end{Def}

Entonces

\begin{eqnarray*}
\hat{U}\left(\alpha\right)=\sum_{n=0}^{\infty}\hat{F^{n\star}}\left(\alpha\right)=\sum_{n=0}^{\infty}\hat{F}\left(\alpha\right)^{n}=\frac{1}{1-\hat{F}\left(\alpha\right)}.
\end{eqnarray*}


\begin{Prop}
La Transformada de Laplace $\hat{U}\left(\alpha\right)$ y $\hat{F}\left(\alpha\right)$ determina una a la otra de manera \'unica por la relaci\'on $\hat{U}\left(\alpha\right)=\frac{1}{1-\hat{F}\left(\alpha\right)}$.
\end{Prop}


\begin{Note}
Un proceso de renovaci\'on $N\left(t\right)$ cuyos tiempos de inter-renovaci\'on tienen media finita, es un proceso Poisson con tasa $\lambda$ si y s\'olo s\'i $\esp\left[U\left(t\right)\right]=\lambda t$, para $t\geq0$.
\end{Note}


\begin{Teo}
Sea $N\left(t\right)$ un proceso puntual simple con puntos de localizaci\'on $T_{n}$ tal que $\eta\left(t\right)=\esp\left[N\left(\right)\right]$ es finita para cada $t$. Entonces para cualquier funci\'on $f:\rea_{+}\rightarrow\rea$,
\begin{eqnarray*}
\esp\left[\sum_{n=1}^{N\left(\right)}f\left(T_{n}\right)\right]=\int_{\left(0,t\right]}f\left(s\right)d\eta\left(s\right)\textrm{,  }t\geq0,
\end{eqnarray*}
suponiendo que la integral exista. Adem\'as si $X_{1},X_{2},\ldots$ son variables aleatorias definidas en el mismo espacio de probabilidad que el proceso $N\left(t\right)$ tal que $\esp\left[X_{n}|T_{n}=s\right]=f\left(s\right)$, independiente de $n$. Entonces
\begin{eqnarray*}
\esp\left[\sum_{n=1}^{N\left(t\right)}X_{n}\right]=\int_{\left(0,t\right]}f\left(s\right)d\eta\left(s\right)\textrm{,  }t\geq0,
\end{eqnarray*} 
suponiendo que la integral exista. 
\end{Teo}

\begin{Coro}[Identidad de Wald para Renovaciones]
Para el proceso de renovaci\'on $N\left(t\right)$,
\begin{eqnarray*}
\esp\left[T_{N\left(t\right)+1}\right]=\mu\esp\left[N\left(t\right)+1\right]\textrm{,  }t\geq0,
\end{eqnarray*}  
\end{Coro}

%______________________________________________________________________
\subsection{Procesos de Renovaci\'on}
%______________________________________________________________________

\begin{Def}\label{Def.Tn}
Sean $0\leq T_{1}\leq T_{2}\leq \ldots$ son tiempos aleatorios infinitos en los cuales ocurren ciertos eventos. El n\'umero de tiempos $T_{n}$ en el intervalo $\left[0,t\right)$ es

\begin{eqnarray}
N\left(t\right)=\sum_{n=1}^{\infty}\indora\left(T_{n}\leq t\right),
\end{eqnarray}
para $t\geq0$.
\end{Def}

Si se consideran los puntos $T_{n}$ como elementos de $\rea_{+}$, y $N\left(t\right)$ es el n\'umero de puntos en $\rea$. El proceso denotado por $\left\{N\left(t\right):t\geq0\right\}$, denotado por $N\left(t\right)$, es un proceso puntual en $\rea_{+}$. Los $T_{n}$ son los tiempos de ocurrencia, el proceso puntual $N\left(t\right)$ es simple si su n\'umero de ocurrencias son distintas: $0<T_{1}<T_{2}<\ldots$ casi seguramente.

\begin{Def}
Un proceso puntual $N\left(t\right)$ es un proceso de renovaci\'on si los tiempos de interocurrencia $\xi_{n}=T_{n}-T_{n-1}$, para $n\geq1$, son independientes e identicamente distribuidos con distribuci\'on $F$, donde $F\left(0\right)=0$ y $T_{0}=0$. Los $T_{n}$ son llamados tiempos de renovaci\'on, referente a la independencia o renovaci\'on de la informaci\'on estoc\'astica en estos tiempos. Los $\xi_{n}$ son los tiempos de inter-renovaci\'on, y $N\left(t\right)$ es el n\'umero de renovaciones en el intervalo $\left[0,t\right)$
\end{Def}


\begin{Note}
Para definir un proceso de renovaci\'on para cualquier contexto, solamente hay que especificar una distribuci\'on $F$, con $F\left(0\right)=0$, para los tiempos de inter-renovaci\'on. La funci\'on $F$ en turno degune las otra variables aleatorias. De manera formal, existe un espacio de probabilidad y una sucesi\'on de variables aleatorias $\xi_{1},\xi_{2},\ldots$ definidas en este con distribuci\'on $F$. Entonces las otras cantidades son $T_{n}=\sum_{k=1}^{n}\xi_{k}$ y $N\left(t\right)=\sum_{n=1}^{\infty}\indora\left(T_{n}\leq t\right)$, donde $T_{n}\rightarrow\infty$ casi seguramente por la Ley Fuerte de los Grandes Números.
\end{Note}

%___________________________________________________________________________________________
%
\section{Renewal and Regenerative Processes: Serfozo\cite{Serfozo}}
%___________________________________________________________________________________________
%
\begin{Def}\label{Def.Tn}
Sean $0\leq T_{1}\leq T_{2}\leq \ldots$ son tiempos aleatorios infinitos en los cuales ocurren ciertos eventos. El n\'umero de tiempos $T_{n}$ en el intervalo $\left[0,t\right)$ es

\begin{eqnarray}
N\left(t\right)=\sum_{n=1}^{\infty}\indora\left(T_{n}\leq t\right),
\end{eqnarray}
para $t\geq0$.
\end{Def}

Si se consideran los puntos $T_{n}$ como elementos de $\rea_{+}$, y $N\left(t\right)$ es el n\'umero de puntos en $\rea$. El proceso denotado por $\left\{N\left(t\right):t\geq0\right\}$, denotado por $N\left(t\right)$, es un proceso puntual en $\rea_{+}$. Los $T_{n}$ son los tiempos de ocurrencia, el proceso puntual $N\left(t\right)$ es simple si su n\'umero de ocurrencias son distintas: $0<T_{1}<T_{2}<\ldots$ casi seguramente.

\begin{Def}
Un proceso puntual $N\left(t\right)$ es un proceso de renovaci\'on si los tiempos de interocurrencia $\xi_{n}=T_{n}-T_{n-1}$, para $n\geq1$, son independientes e identicamente distribuidos con distribuci\'on $F$, donde $F\left(0\right)=0$ y $T_{0}=0$. Los $T_{n}$ son llamados tiempos de renovaci\'on, referente a la independencia o renovaci\'on de la informaci\'on estoc\'astica en estos tiempos. Los $\xi_{n}$ son los tiempos de inter-renovaci\'on, y $N\left(t\right)$ es el n\'umero de renovaciones en el intervalo $\left[0,t\right)$
\end{Def}


\begin{Note}
Para definir un proceso de renovaci\'on para cualquier contexto, solamente hay que especificar una distribuci\'on $F$, con $F\left(0\right)=0$, para los tiempos de inter-renovaci\'on. La funci\'on $F$ en turno degune las otra variables aleatorias. De manera formal, existe un espacio de probabilidad y una sucesi\'on de variables aleatorias $\xi_{1},\xi_{2},\ldots$ definidas en este con distribuci\'on $F$. Entonces las otras cantidades son $T_{n}=\sum_{k=1}^{n}\xi_{k}$ y $N\left(t\right)=\sum_{n=1}^{\infty}\indora\left(T_{n}\leq t\right)$, donde $T_{n}\rightarrow\infty$ casi seguramente por la Ley Fuerte de los Grandes N\'umeros.
\end{Note}







Los tiempos $T_{n}$ est\'an relacionados con los conteos de $N\left(t\right)$ por

\begin{eqnarray*}
\left\{N\left(t\right)\geq n\right\}&=&\left\{T_{n}\leq t\right\}\\
T_{N\left(t\right)}\leq &t&<T_{N\left(t\right)+1},
\end{eqnarray*}

adem\'as $N\left(T_{n}\right)=n$, y 

\begin{eqnarray*}
N\left(t\right)=\max\left\{n:T_{n}\leq t\right\}=\min\left\{n:T_{n+1}>t\right\}
\end{eqnarray*}

Por propiedades de la convoluci\'on se sabe que

\begin{eqnarray*}
P\left\{T_{n}\leq t\right\}=F^{n\star}\left(t\right)
\end{eqnarray*}
que es la $n$-\'esima convoluci\'on de $F$. Entonces 

\begin{eqnarray*}
\left\{N\left(t\right)\geq n\right\}&=&\left\{T_{n}\leq t\right\}\\
P\left\{N\left(t\right)\leq n\right\}&=&1-F^{\left(n+1\right)\star}\left(t\right)
\end{eqnarray*}

Adem\'as usando el hecho de que $\esp\left[N\left(t\right)\right]=\sum_{n=1}^{\infty}P\left\{N\left(t\right)\geq n\right\}$
se tiene que

\begin{eqnarray*}
\esp\left[N\left(t\right)\right]=\sum_{n=1}^{\infty}F^{n\star}\left(t\right)
\end{eqnarray*}

\begin{Prop}
Para cada $t\geq0$, la funci\'on generadora de momentos $\esp\left[e^{\alpha N\left(t\right)}\right]$ existe para alguna $\alpha$ en una vecindad del 0, y de aqu\'i que $\esp\left[N\left(t\right)^{m}\right]<\infty$, para $m\geq1$.
\end{Prop}

\begin{Ejem}[\textbf{Proceso Poisson}]

Suponga que se tienen tiempos de inter-renovaci\'on \textit{i.i.d.} del proceso de renovaci\'on $N\left(t\right)$ tienen distribuci\'on exponencial $F\left(t\right)=q-e^{-\lambda t}$ con tasa $\lambda$. Entonces $N\left(t\right)$ es un proceso Poisson con tasa $\lambda$.

\end{Ejem}


\begin{Note}
Si el primer tiempo de renovaci\'on $\xi_{1}$ no tiene la misma distribuci\'on que el resto de las $\xi_{n}$, para $n\geq2$, a $N\left(t\right)$ se le llama Proceso de Renovaci\'on retardado, donde si $\xi$ tiene distribuci\'on $G$, entonces el tiempo $T_{n}$ de la $n$-\'esima renovaci\'on tiene distribuci\'on $G\star F^{\left(n-1\right)\star}\left(t\right)$
\end{Note}


\begin{Teo}
Para una constante $\mu\leq\infty$ ( o variable aleatoria), las siguientes expresiones son equivalentes:

\begin{eqnarray}
lim_{n\rightarrow\infty}n^{-1}T_{n}&=&\mu,\textrm{ c.s.}\\
lim_{t\rightarrow\infty}t^{-1}N\left(t\right)&=&1/\mu,\textrm{ c.s.}
\end{eqnarray}
\end{Teo}


Es decir, $T_{n}$ satisface la Ley Fuerte de los Grandes N\'umeros s\'i y s\'olo s\'i $N\left/t\right)$ la cumple.


\begin{Coro}[Ley Fuerte de los Grandes N\'umeros para Procesos de Renovaci\'on]
Si $N\left(t\right)$ es un proceso de renovaci\'on cuyos tiempos de inter-renovaci\'on tienen media $\mu\leq\infty$, entonces
\begin{eqnarray}
t^{-1}N\left(t\right)\rightarrow 1/\mu,\textrm{ c.s. cuando }t\rightarrow\infty.
\end{eqnarray}

\end{Coro}


Considerar el proceso estoc\'astico de valores reales $\left\{Z\left(t\right):t\geq0\right\}$ en el mismo espacio de probabilidad que $N\left(t\right)$

\begin{Def}
Para el proceso $\left\{Z\left(t\right):t\geq0\right\}$ se define la fluctuaci\'on m\'axima de $Z\left(t\right)$ en el intervalo $\left(T_{n-1},T_{n}\right]$:
\begin{eqnarray*}
M_{n}=\sup_{T_{n-1}<t\leq T_{n}}|Z\left(t\right)-Z\left(T_{n-1}\right)|
\end{eqnarray*}
\end{Def}

\begin{Teo}
Sup\'ongase que $n^{-1}T_{n}\rightarrow\mu$ c.s. cuando $n\rightarrow\infty$, donde $\mu\leq\infty$ es una constante o variable aleatoria. Sea $a$ una constante o variable aleatoria que puede ser infinita cuando $\mu$ es finita, y considere las expresiones l\'imite:
\begin{eqnarray}
lim_{n\rightarrow\infty}n^{-1}Z\left(T_{n}\right)&=&a,\textrm{ c.s.}\\
lim_{t\rightarrow\infty}t^{-1}Z\left(t\right)&=&a/\mu,\textrm{ c.s.}
\end{eqnarray}
La segunda expresi\'on implica la primera. Conversamente, la primera implica la segunda si el proceso $Z\left(t\right)$ es creciente, o si $lim_{n\rightarrow\infty}n^{-1}M_{n}=0$ c.s.
\end{Teo}

\begin{Coro}
Si $N\left(t\right)$ es un proceso de renovaci\'on, y $\left(Z\left(T_{n}\right)-Z\left(T_{n-1}\right),M_{n}\right)$, para $n\geq1$, son variables aleatorias independientes e id\'enticamente distribuidas con media finita, entonces,
\begin{eqnarray}
lim_{t\rightarrow\infty}t^{-1}Z\left(t\right)\rightarrow\frac{\esp\left[Z\left(T_{1}\right)-Z\left(T_{0}\right)\right]}{\esp\left[T_{1}\right]},\textrm{ c.s. cuando  }t\rightarrow\infty.
\end{eqnarray}
\end{Coro}


Sup\'ongase que $N\left(t\right)$ es un proceso de renovaci\'on con distribuci\'on $F$ con media finita $\mu$.

\begin{Def}
La funci\'on de renovaci\'on asociada con la distribuci\'on $F$, del proceso $N\left(t\right)$, es
\begin{eqnarray*}
U\left(t\right)=\sum_{n=1}^{\infty}F^{n\star}\left(t\right),\textrm{   }t\geq0,
\end{eqnarray*}
donde $F^{0\star}\left(t\right)=\indora\left(t\geq0\right)$.
\end{Def}


\begin{Prop}
Sup\'ongase que la distribuci\'on de inter-renovaci\'on $F$ tiene densidad $f$. Entonces $U\left(t\right)$ tambi\'en tiene densidad, para $t>0$, y es $U^{'}\left(t\right)=\sum_{n=0}^{\infty}f^{n\star}\left(t\right)$. Adem\'as
\begin{eqnarray*}
\prob\left\{N\left(t\right)>N\left(t-\right)\right\}=0\textrm{,   }t\geq0.
\end{eqnarray*}
\end{Prop}

\begin{Def}
La Transformada de Laplace-Stieljes de $F$ est\'a dada por

\begin{eqnarray*}
\hat{F}\left(\alpha\right)=\int_{\rea_{+}}e^{-\alpha t}dF\left(t\right)\textrm{,  }\alpha\geq0.
\end{eqnarray*}
\end{Def}

Entonces

\begin{eqnarray*}
\hat{U}\left(\alpha\right)=\sum_{n=0}^{\infty}\hat{F^{n\star}}\left(\alpha\right)=\sum_{n=0}^{\infty}\hat{F}\left(\alpha\right)^{n}=\frac{1}{1-\hat{F}\left(\alpha\right)}.
\end{eqnarray*}


\begin{Prop}
La Transformada de Laplace $\hat{U}\left(\alpha\right)$ y $\hat{F}\left(\alpha\right)$ determina una a la otra de manera \'unica por la relaci\'on $\hat{U}\left(\alpha\right)=\frac{1}{1-\hat{F}\left(\alpha\right)}$.
\end{Prop}


\begin{Note}
Un proceso de renovaci\'on $N\left(t\right)$ cuyos tiempos de inter-renovaci\'on tienen media finita, es un proceso Poisson con tasa $\lambda$ si y s\'olo s\'i $\esp\left[U\left(t\right)\right]=\lambda t$, para $t\geq0$.
\end{Note}


\begin{Teo}
Sea $N\left(t\right)$ un proceso puntual simple con puntos de localizaci\'on $T_{n}$ tal que $\eta\left(t\right)=\esp\left[N\left(\right)\right]$ es finita para cada $t$. Entonces para cualquier funci\'on $f:\rea_{+}\rightarrow\rea$,
\begin{eqnarray*}
\esp\left[\sum_{n=1}^{N\left(\right)}f\left(T_{n}\right)\right]=\int_{\left(0,t\right]}f\left(s\right)d\eta\left(s\right)\textrm{,  }t\geq0,
\end{eqnarray*}
suponiendo que la integral exista. Adem\'as si $X_{1},X_{2},\ldots$ son variables aleatorias definidas en el mismo espacio de probabilidad que el proceso $N\left(t\right)$ tal que $\esp\left[X_{n}|T_{n}=s\right]=f\left(s\right)$, independiente de $n$. Entonces
\begin{eqnarray*}
\esp\left[\sum_{n=1}^{N\left(t\right)}X_{n}\right]=\int_{\left(0,t\right]}f\left(s\right)d\eta\left(s\right)\textrm{,  }t\geq0,
\end{eqnarray*} 
suponiendo que la integral exista. 
\end{Teo}

\begin{Coro}[Identidad de Wald para Renovaciones]
Para el proceso de renovaci\'on $N\left(t\right)$,
\begin{eqnarray*}
\esp\left[T_{N\left(t\right)+1}\right]=\mu\esp\left[N\left(t\right)+1\right]\textrm{,  }t\geq0,
\end{eqnarray*}  
\end{Coro}


\begin{Def}
Sea $h\left(t\right)$ funci\'on de valores reales en $\rea$ acotada en intervalos finitos e igual a cero para $t<0$ La ecuaci\'on de renovaci\'on para $h\left(t\right)$ y la distribuci\'on $F$ es

\begin{eqnarray}\label{Ec.Renovacion}
H\left(t\right)=h\left(t\right)+\int_{\left[0,t\right]}H\left(t-s\right)dF\left(s\right)\textrm{,    }t\geq0,
\end{eqnarray}
donde $H\left(t\right)$ es una funci\'on de valores reales. Esto es $H=h+F\star H$. Decimos que $H\left(t\right)$ es soluci\'on de esta ecuaci\'on si satisface la ecuaci\'on, y es acotada en intervalos finitos e iguales a cero para $t<0$.
\end{Def}

\begin{Prop}
La funci\'on $U\star h\left(t\right)$ es la \'unica soluci\'on de la ecuaci\'on de renovaci\'on (\ref{Ec.Renovacion}).
\end{Prop}

\begin{Teo}[Teorema Renovaci\'on Elemental]
\begin{eqnarray*}
t^{-1}U\left(t\right)\rightarrow 1/\mu\textrm{,    cuando }t\rightarrow\infty.
\end{eqnarray*}
\end{Teo}



Sup\'ongase que $N\left(t\right)$ es un proceso de renovaci\'on con distribuci\'on $F$ con media finita $\mu$.

\begin{Def}
La funci\'on de renovaci\'on asociada con la distribuci\'on $F$, del proceso $N\left(t\right)$, es
\begin{eqnarray*}
U\left(t\right)=\sum_{n=1}^{\infty}F^{n\star}\left(t\right),\textrm{   }t\geq0,
\end{eqnarray*}
donde $F^{0\star}\left(t\right)=\indora\left(t\geq0\right)$.
\end{Def}


\begin{Prop}
Sup\'ongase que la distribuci\'on de inter-renovaci\'on $F$ tiene densidad $f$. Entonces $U\left(t\right)$ tambi\'en tiene densidad, para $t>0$, y es $U^{'}\left(t\right)=\sum_{n=0}^{\infty}f^{n\star}\left(t\right)$. Adem\'as
\begin{eqnarray*}
\prob\left\{N\left(t\right)>N\left(t-\right)\right\}=0\textrm{,   }t\geq0.
\end{eqnarray*}
\end{Prop}

\begin{Def}
La Transformada de Laplace-Stieljes de $F$ est\'a dada por

\begin{eqnarray*}
\hat{F}\left(\alpha\right)=\int_{\rea_{+}}e^{-\alpha t}dF\left(t\right)\textrm{,  }\alpha\geq0.
\end{eqnarray*}
\end{Def}

Entonces

\begin{eqnarray*}
\hat{U}\left(\alpha\right)=\sum_{n=0}^{\infty}\hat{F^{n\star}}\left(\alpha\right)=\sum_{n=0}^{\infty}\hat{F}\left(\alpha\right)^{n}=\frac{1}{1-\hat{F}\left(\alpha\right)}.
\end{eqnarray*}


\begin{Prop}
La Transformada de Laplace $\hat{U}\left(\alpha\right)$ y $\hat{F}\left(\alpha\right)$ determina una a la otra de manera \'unica por la relaci\'on $\hat{U}\left(\alpha\right)=\frac{1}{1-\hat{F}\left(\alpha\right)}$.
\end{Prop}


\begin{Note}
Un proceso de renovaci\'on $N\left(t\right)$ cuyos tiempos de inter-renovaci\'on tienen media finita, es un proceso Poisson con tasa $\lambda$ si y s\'olo s\'i $\esp\left[U\left(t\right)\right]=\lambda t$, para $t\geq0$.
\end{Note}


\begin{Teo}
Sea $N\left(t\right)$ un proceso puntual simple con puntos de localizaci\'on $T_{n}$ tal que $\eta\left(t\right)=\esp\left[N\left(\right)\right]$ es finita para cada $t$. Entonces para cualquier funci\'on $f:\rea_{+}\rightarrow\rea$,
\begin{eqnarray*}
\esp\left[\sum_{n=1}^{N\left(\right)}f\left(T_{n}\right)\right]=\int_{\left(0,t\right]}f\left(s\right)d\eta\left(s\right)\textrm{,  }t\geq0,
\end{eqnarray*}
suponiendo que la integral exista. Adem\'as si $X_{1},X_{2},\ldots$ son variables aleatorias definidas en el mismo espacio de probabilidad que el proceso $N\left(t\right)$ tal que $\esp\left[X_{n}|T_{n}=s\right]=f\left(s\right)$, independiente de $n$. Entonces
\begin{eqnarray*}
\esp\left[\sum_{n=1}^{N\left(t\right)}X_{n}\right]=\int_{\left(0,t\right]}f\left(s\right)d\eta\left(s\right)\textrm{,  }t\geq0,
\end{eqnarray*} 
suponiendo que la integral exista. 
\end{Teo}

\begin{Coro}[Identidad de Wald para Renovaciones]
Para el proceso de renovaci\'on $N\left(t\right)$,
\begin{eqnarray*}
\esp\left[T_{N\left(t\right)+1}\right]=\mu\esp\left[N\left(t\right)+1\right]\textrm{,  }t\geq0,
\end{eqnarray*}  
\end{Coro}


\begin{Def}
Sea $h\left(t\right)$ funci\'on de valores reales en $\rea$ acotada en intervalos finitos e igual a cero para $t<0$ La ecuaci\'on de renovaci\'on para $h\left(t\right)$ y la distribuci\'on $F$ es

\begin{eqnarray}\label{Ec.Renovacion}
H\left(t\right)=h\left(t\right)+\int_{\left[0,t\right]}H\left(t-s\right)dF\left(s\right)\textrm{,    }t\geq0,
\end{eqnarray}
donde $H\left(t\right)$ es una funci\'on de valores reales. Esto es $H=h+F\star H$. Decimos que $H\left(t\right)$ es soluci\'on de esta ecuaci\'on si satisface la ecuaci\'on, y es acotada en intervalos finitos e iguales a cero para $t<0$.
\end{Def}

\begin{Prop}
La funci\'on $U\star h\left(t\right)$ es la \'unica soluci\'on de la ecuaci\'on de renovaci\'on (\ref{Ec.Renovacion}).
\end{Prop}

\begin{Teo}[Teorema Renovaci\'on Elemental]
\begin{eqnarray*}
t^{-1}U\left(t\right)\rightarrow 1/\mu\textrm{,    cuando }t\rightarrow\infty.
\end{eqnarray*}
\end{Teo}


\begin{Note} Una funci\'on $h:\rea_{+}\rightarrow\rea$ es Directamente Riemann Integrable en los siguientes casos:
\begin{itemize}
\item[a)] $h\left(t\right)\geq0$ es decreciente y Riemann Integrable.
\item[b)] $h$ es continua excepto posiblemente en un conjunto de Lebesgue de medida 0, y $|h\left(t\right)|\leq b\left(t\right)$, donde $b$ es DRI.
\end{itemize}
\end{Note}

\begin{Teo}[Teorema Principal de Renovaci\'on]
Si $F$ es no aritm\'etica y $h\left(t\right)$ es Directamente Riemann Integrable (DRI), entonces

\begin{eqnarray*}
lim_{t\rightarrow\infty}U\star h=\frac{1}{\mu}\int_{\rea_{+}}h\left(s\right)ds.
\end{eqnarray*}
\end{Teo}

\begin{Prop}
Cualquier funci\'on $H\left(t\right)$ acotada en intervalos finitos y que es 0 para $t<0$ puede expresarse como
\begin{eqnarray*}
H\left(t\right)=U\star h\left(t\right)\textrm{,  donde }h\left(t\right)=H\left(t\right)-F\star H\left(t\right)
\end{eqnarray*}
\end{Prop}

\begin{Def}
Un proceso estoc\'astico $X\left(t\right)$ es crudamente regenerativo en un tiempo aleatorio positivo $T$ si
\begin{eqnarray*}
\esp\left[X\left(T+t\right)|T\right]=\esp\left[X\left(t\right)\right]\textrm{, para }t\geq0,\end{eqnarray*}
y con las esperanzas anteriores finitas.
\end{Def}

\begin{Prop}
Sup\'ongase que $X\left(t\right)$ es un proceso crudamente regenerativo en $T$, que tiene distribuci\'on $F$. Si $\esp\left[X\left(t\right)\right]$ es acotado en intervalos finitos, entonces
\begin{eqnarray*}
\esp\left[X\left(t\right)\right]=U\star h\left(t\right)\textrm{,  donde }h\left(t\right)=\esp\left[X\left(t\right)\indora\left(T>t\right)\right].
\end{eqnarray*}
\end{Prop}

\begin{Teo}[Regeneraci\'on Cruda]
Sup\'ongase que $X\left(t\right)$ es un proceso con valores positivo crudamente regenerativo en $T$, y def\'inase $M=\sup\left\{|X\left(t\right)|:t\leq T\right\}$. Si $T$ es no aritm\'etico y $M$ y $MT$ tienen media finita, entonces
\begin{eqnarray*}
lim_{t\rightarrow\infty}\esp\left[X\left(t\right)\right]=\frac{1}{\mu}\int_{\rea_{+}}h\left(s\right)ds,
\end{eqnarray*}
donde $h\left(t\right)=\esp\left[X\left(t\right)\indora\left(T>t\right)\right]$.
\end{Teo}


\begin{Note} Una funci\'on $h:\rea_{+}\rightarrow\rea$ es Directamente Riemann Integrable en los siguientes casos:
\begin{itemize}
\item[a)] $h\left(t\right)\geq0$ es decreciente y Riemann Integrable.
\item[b)] $h$ es continua excepto posiblemente en un conjunto de Lebesgue de medida 0, y $|h\left(t\right)|\leq b\left(t\right)$, donde $b$ es DRI.
\end{itemize}
\end{Note}

\begin{Teo}[Teorema Principal de Renovaci\'on]
Si $F$ es no aritm\'etica y $h\left(t\right)$ es Directamente Riemann Integrable (DRI), entonces

\begin{eqnarray*}
lim_{t\rightarrow\infty}U\star h=\frac{1}{\mu}\int_{\rea_{+}}h\left(s\right)ds.
\end{eqnarray*}
\end{Teo}

\begin{Prop}
Cualquier funci\'on $H\left(t\right)$ acotada en intervalos finitos y que es 0 para $t<0$ puede expresarse como
\begin{eqnarray*}
H\left(t\right)=U\star h\left(t\right)\textrm{,  donde }h\left(t\right)=H\left(t\right)-F\star H\left(t\right)
\end{eqnarray*}
\end{Prop}

\begin{Def}
Un proceso estoc\'astico $X\left(t\right)$ es crudamente regenerativo en un tiempo aleatorio positivo $T$ si
\begin{eqnarray*}
\esp\left[X\left(T+t\right)|T\right]=\esp\left[X\left(t\right)\right]\textrm{, para }t\geq0,\end{eqnarray*}
y con las esperanzas anteriores finitas.
\end{Def}

\begin{Prop}
Sup\'ongase que $X\left(t\right)$ es un proceso crudamente regenerativo en $T$, que tiene distribuci\'on $F$. Si $\esp\left[X\left(t\right)\right]$ es acotado en intervalos finitos, entonces
\begin{eqnarray*}
\esp\left[X\left(t\right)\right]=U\star h\left(t\right)\textrm{,  donde }h\left(t\right)=\esp\left[X\left(t\right)\indora\left(T>t\right)\right].
\end{eqnarray*}
\end{Prop}

\begin{Teo}[Regeneraci\'on Cruda]
Sup\'ongase que $X\left(t\right)$ es un proceso con valores positivo crudamente regenerativo en $T$, y def\'inase $M=\sup\left\{|X\left(t\right)|:t\leq T\right\}$. Si $T$ es no aritm\'etico y $M$ y $MT$ tienen media finita, entonces
\begin{eqnarray*}
lim_{t\rightarrow\infty}\esp\left[X\left(t\right)\right]=\frac{1}{\mu}\int_{\rea_{+}}h\left(s\right)ds,
\end{eqnarray*}
donde $h\left(t\right)=\esp\left[X\left(t\right)\indora\left(T>t\right)\right]$.
\end{Teo}

\begin{Def}
Para el proceso $\left\{\left(N\left(t\right),X\left(t\right)\right):t\geq0\right\}$, sus trayectoria muestrales en el intervalo de tiempo $\left[T_{n-1},T_{n}\right)$ est\'an descritas por
\begin{eqnarray*}
\zeta_{n}=\left(\xi_{n},\left\{X\left(T_{n-1}+t\right):0\leq t<\xi_{n}\right\}\right)
\end{eqnarray*}
Este $\zeta_{n}$ es el $n$-\'esimo segmento del proceso. El proceso es regenerativo sobre los tiempos $T_{n}$ si sus segmentos $\zeta_{n}$ son independientes e id\'enticamennte distribuidos.
\end{Def}


\begin{Note}
Si $\tilde{X}\left(t\right)$ con espacio de estados $\tilde{S}$ es regenerativo sobre $T_{n}$, entonces $X\left(t\right)=f\left(\tilde{X}\left(t\right)\right)$ tambi\'en es regenerativo sobre $T_{n}$, para cualquier funci\'on $f:\tilde{S}\rightarrow S$.
\end{Note}

\begin{Note}
Los procesos regenerativos son crudamente regenerativos, pero no al rev\'es.
\end{Note}


\begin{Note}
Un proceso estoc\'astico a tiempo continuo o discreto es regenerativo si existe un proceso de renovaci\'on  tal que los segmentos del proceso entre tiempos de renovaci\'on sucesivos son i.i.d., es decir, para $\left\{X\left(t\right):t\geq0\right\}$ proceso estoc\'astico a tiempo continuo con espacio de estados $S$, espacio m\'etrico.
\end{Note}

Para $\left\{X\left(t\right):t\geq0\right\}$ Proceso Estoc\'astico a tiempo continuo con estado de espacios $S$, que es un espacio m\'etrico, con trayectorias continuas por la derecha y con l\'imites por la izquierda c.s. Sea $N\left(t\right)$ un proceso de renovaci\'on en $\rea_{+}$ definido en el mismo espacio de probabilidad que $X\left(t\right)$, con tiempos de renovaci\'on $T$ y tiempos de inter-renovaci\'on $\xi_{n}=T_{n}-T_{n-1}$, con misma distribuci\'on $F$ de media finita $\mu$.



\begin{Def}
Para el proceso $\left\{\left(N\left(t\right),X\left(t\right)\right):t\geq0\right\}$, sus trayectoria muestrales en el intervalo de tiempo $\left[T_{n-1},T_{n}\right)$ est\'an descritas por
\begin{eqnarray*}
\zeta_{n}=\left(\xi_{n},\left\{X\left(T_{n-1}+t\right):0\leq t<\xi_{n}\right\}\right)
\end{eqnarray*}
Este $\zeta_{n}$ es el $n$-\'esimo segmento del proceso. El proceso es regenerativo sobre los tiempos $T_{n}$ si sus segmentos $\zeta_{n}$ son independientes e id\'enticamennte distribuidos.
\end{Def}

\begin{Note}
Un proceso regenerativo con media de la longitud de ciclo finita es llamado positivo recurrente.
\end{Note}

\begin{Teo}[Procesos Regenerativos]
Suponga que el proceso
\end{Teo}


\begin{Def}[Renewal Process Trinity]
Para un proceso de renovaci\'on $N\left(t\right)$, los siguientes procesos proveen de informaci\'on sobre los tiempos de renovaci\'on.
\begin{itemize}
\item $A\left(t\right)=t-T_{N\left(t\right)}$, el tiempo de recurrencia hacia atr\'as al tiempo $t$, que es el tiempo desde la \'ultima renovaci\'on para $t$.

\item $B\left(t\right)=T_{N\left(t\right)+1}-t$, el tiempo de recurrencia hacia adelante al tiempo $t$, residual del tiempo de renovaci\'on, que es el tiempo para la pr\'oxima renovaci\'on despu\'es de $t$.

\item $L\left(t\right)=\xi_{N\left(t\right)+1}=A\left(t\right)+B\left(t\right)$, la longitud del intervalo de renovaci\'on que contiene a $t$.
\end{itemize}
\end{Def}

\begin{Note}
El proceso tridimensional $\left(A\left(t\right),B\left(t\right),L\left(t\right)\right)$ es regenerativo sobre $T_{n}$, y por ende cada proceso lo es. Cada proceso $A\left(t\right)$ y $B\left(t\right)$ son procesos de MArkov a tiempo continuo con trayectorias continuas por partes en el espacio de estados $\rea_{+}$. Una expresi\'on conveniente para su distribuci\'on conjunta es, para $0\leq x<t,y\geq0$
\begin{equation}\label{NoRenovacion}
P\left\{A\left(t\right)>x,B\left(t\right)>y\right\}=
P\left\{N\left(t+y\right)-N\left((t-x)\right)=0\right\}
\end{equation}
\end{Note}

\begin{Ejem}[Tiempos de recurrencia Poisson]
Si $N\left(t\right)$ es un proceso Poisson con tasa $\lambda$, entonces de la expresi\'on (\ref{NoRenovacion}) se tiene que

\begin{eqnarray*}
\begin{array}{lc}
P\left\{A\left(t\right)>x,B\left(t\right)>y\right\}=e^{-\lambda\left(x+y\right)},&0\leq x<t,y\geq0,
\end{array}
\end{eqnarray*}
que es la probabilidad Poisson de no renovaciones en un intervalo de longitud $x+y$.

\end{Ejem}

\begin{Note}
Una cadena de Markov erg\'odica tiene la propiedad de ser estacionaria si la distribuci\'on de su estado al tiempo $0$ es su distribuci\'on estacionaria.
\end{Note}


\begin{Def}
Un proceso estoc\'astico a tiempo continuo $\left\{X\left(t\right):t\geq0\right\}$ en un espacio general es estacionario si sus distribuciones finito dimensionales son invariantes bajo cualquier  traslado: para cada $0\leq s_{1}<s_{2}<\cdots<s_{k}$ y $t\geq0$,
\begin{eqnarray*}
\left(X\left(s_{1}+t\right),\ldots,X\left(s_{k}+t\right)\right)=_{d}\left(X\left(s_{1}\right),\ldots,X\left(s_{k}\right)\right).
\end{eqnarray*}
\end{Def}

\begin{Note}
Un proceso de Markov es estacionario si $X\left(t\right)=_{d}X\left(0\right)$, $t\geq0$.
\end{Note}

Considerese el proceso $N\left(t\right)=\sum_{n}\indora\left(\tau_{n}\leq t\right)$ en $\rea_{+}$, con puntos $0<\tau_{1}<\tau_{2}<\cdots$.

\begin{Prop}
Si $N$ es un proceso puntual estacionario y $\esp\left[N\left(1\right)\right]<\infty$, entonces $\esp\left[N\left(t\right)\right]=t\esp\left[N\left(1\right)\right]$, $t\geq0$

\end{Prop}

\begin{Teo}
Los siguientes enunciados son equivalentes
\begin{itemize}
\item[i)] El proceso retardado de renovaci\'on $N$ es estacionario.

\item[ii)] EL proceso de tiempos de recurrencia hacia adelante $B\left(t\right)$ es estacionario.


\item[iii)] $\esp\left[N\left(t\right)\right]=t/\mu$,


\item[iv)] $G\left(t\right)=F_{e}\left(t\right)=\frac{1}{\mu}\int_{0}^{t}\left[1-F\left(s\right)\right]ds$
\end{itemize}
Cuando estos enunciados son ciertos, $P\left\{B\left(t\right)\leq x\right\}=F_{e}\left(x\right)$, para $t,x\geq0$.

\end{Teo}

\begin{Note}
Una consecuencia del teorema anterior es que el Proceso Poisson es el \'unico proceso sin retardo que es estacionario.
\end{Note}

\begin{Coro}
El proceso de renovaci\'on $N\left(t\right)$ sin retardo, y cuyos tiempos de inter renonaci\'on tienen media finita, es estacionario si y s\'olo si es un proceso Poisson.

\end{Coro}

%______________________________________________________________________

%\section{Ejemplos, Notas importantes}
%______________________________________________________________________
%\section*{Ap\'endice A}
%__________________________________________________________________

%________________________________________________________________________
%\subsection*{Procesos Regenerativos}
%________________________________________________________________________



\begin{Note}
Si $\tilde{X}\left(t\right)$ con espacio de estados $\tilde{S}$ es regenerativo sobre $T_{n}$, entonces $X\left(t\right)=f\left(\tilde{X}\left(t\right)\right)$ tambi\'en es regenerativo sobre $T_{n}$, para cualquier funci\'on $f:\tilde{S}\rightarrow S$.
\end{Note}

\begin{Note}
Los procesos regenerativos son crudamente regenerativos, pero no al rev\'es.
\end{Note}
%\subsection*{Procesos Regenerativos: Sigman\cite{Sigman1}}
\begin{Def}[Definici\'on Cl\'asica]
Un proceso estoc\'astico $X=\left\{X\left(t\right):t\geq0\right\}$ es llamado regenerativo is existe una variable aleatoria $R_{1}>0$ tal que
\begin{itemize}
\item[i)] $\left\{X\left(t+R_{1}\right):t\geq0\right\}$ es independiente de $\left\{\left\{X\left(t\right):t<R_{1}\right\},\right\}$
\item[ii)] $\left\{X\left(t+R_{1}\right):t\geq0\right\}$ es estoc\'asticamente equivalente a $\left\{X\left(t\right):t>0\right\}$
\end{itemize}

Llamamos a $R_{1}$ tiempo de regeneraci\'on, y decimos que $X$ se regenera en este punto.
\end{Def}

$\left\{X\left(t+R_{1}\right)\right\}$ es regenerativo con tiempo de regeneraci\'on $R_{2}$, independiente de $R_{1}$ pero con la misma distribuci\'on que $R_{1}$. Procediendo de esta manera se obtiene una secuencia de variables aleatorias independientes e id\'enticamente distribuidas $\left\{R_{n}\right\}$ llamados longitudes de ciclo. Si definimos a $Z_{k}\equiv R_{1}+R_{2}+\cdots+R_{k}$, se tiene un proceso de renovaci\'on llamado proceso de renovaci\'on encajado para $X$.




\begin{Def}
Para $x$ fijo y para cada $t\geq0$, sea $I_{x}\left(t\right)=1$ si $X\left(t\right)\leq x$,  $I_{x}\left(t\right)=0$ en caso contrario, y def\'inanse los tiempos promedio
\begin{eqnarray*}
\overline{X}&=&lim_{t\rightarrow\infty}\frac{1}{t}\int_{0}^{\infty}X\left(u\right)du\\
\prob\left(X_{\infty}\leq x\right)&=&lim_{t\rightarrow\infty}\frac{1}{t}\int_{0}^{\infty}I_{x}\left(u\right)du,
\end{eqnarray*}
cuando estos l\'imites existan.
\end{Def}

Como consecuencia del teorema de Renovaci\'on-Recompensa, se tiene que el primer l\'imite  existe y es igual a la constante
\begin{eqnarray*}
\overline{X}&=&\frac{\esp\left[\int_{0}^{R_{1}}X\left(t\right)dt\right]}{\esp\left[R_{1}\right]},
\end{eqnarray*}
suponiendo que ambas esperanzas son finitas.

\begin{Note}
\begin{itemize}
\item[a)] Si el proceso regenerativo $X$ es positivo recurrente y tiene trayectorias muestrales no negativas, entonces la ecuaci\'on anterior es v\'alida.
\item[b)] Si $X$ es positivo recurrente regenerativo, podemos construir una \'unica versi\'on estacionaria de este proceso, $X_{e}=\left\{X_{e}\left(t\right)\right\}$, donde $X_{e}$ es un proceso estoc\'astico regenerativo y estrictamente estacionario, con distribuci\'on marginal distribuida como $X_{\infty}$
\end{itemize}
\end{Note}

Para $\left\{X\left(t\right):t\geq0\right\}$ Proceso Estoc\'astico a tiempo continuo con estado de espacios $S$, que es un espacio m\'etrico, con trayectorias continuas por la derecha y con l\'imites por la izquierda c.s. Sea $N\left(t\right)$ un proceso de renovaci\'on en $\rea_{+}$ definido en el mismo espacio de probabilidad que $X\left(t\right)$, con tiempos de renovaci\'on $T$ y tiempos de inter-renovaci\'on $\xi_{n}=T_{n}-T_{n-1}$, con misma distribuci\'on $F$ de media finita $\mu$.


\begin{Def}
Para el proceso $\left\{\left(N\left(t\right),X\left(t\right)\right):t\geq0\right\}$, sus trayectoria muestrales en el intervalo de tiempo $\left[T_{n-1},T_{n}\right)$ est\'an descritas por
\begin{eqnarray*}
\zeta_{n}=\left(\xi_{n},\left\{X\left(T_{n-1}+t\right):0\leq t<\xi_{n}\right\}\right)
\end{eqnarray*}
Este $\zeta_{n}$ es el $n$-\'esimo segmento del proceso. El proceso es regenerativo sobre los tiempos $T_{n}$ si sus segmentos $\zeta_{n}$ son independientes e id\'enticamennte distribuidos.
\end{Def}


\begin{Note}
Si $\tilde{X}\left(t\right)$ con espacio de estados $\tilde{S}$ es regenerativo sobre $T_{n}$, entonces $X\left(t\right)=f\left(\tilde{X}\left(t\right)\right)$ tambi\'en es regenerativo sobre $T_{n}$, para cualquier funci\'on $f:\tilde{S}\rightarrow S$.
\end{Note}

\begin{Note}
Los procesos regenerativos son crudamente regenerativos, pero no al rev\'es.
\end{Note}

\begin{Def}[Definici\'on Cl\'asica]
Un proceso estoc\'astico $X=\left\{X\left(t\right):t\geq0\right\}$ es llamado regenerativo is existe una variable aleatoria $R_{1}>0$ tal que
\begin{itemize}
\item[i)] $\left\{X\left(t+R_{1}\right):t\geq0\right\}$ es independiente de $\left\{\left\{X\left(t\right):t<R_{1}\right\},\right\}$
\item[ii)] $\left\{X\left(t+R_{1}\right):t\geq0\right\}$ es estoc\'asticamente equivalente a $\left\{X\left(t\right):t>0\right\}$
\end{itemize}

Llamamos a $R_{1}$ tiempo de regeneraci\'on, y decimos que $X$ se regenera en este punto.
\end{Def}

$\left\{X\left(t+R_{1}\right)\right\}$ es regenerativo con tiempo de regeneraci\'on $R_{2}$, independiente de $R_{1}$ pero con la misma distribuci\'on que $R_{1}$. Procediendo de esta manera se obtiene una secuencia de variables aleatorias independientes e id\'enticamente distribuidas $\left\{R_{n}\right\}$ llamados longitudes de ciclo. Si definimos a $Z_{k}\equiv R_{1}+R_{2}+\cdots+R_{k}$, se tiene un proceso de renovaci\'on llamado proceso de renovaci\'on encajado para $X$.

\begin{Note}
Un proceso regenerativo con media de la longitud de ciclo finita es llamado positivo recurrente.
\end{Note}


\begin{Def}
Para $x$ fijo y para cada $t\geq0$, sea $I_{x}\left(t\right)=1$ si $X\left(t\right)\leq x$,  $I_{x}\left(t\right)=0$ en caso contrario, y def\'inanse los tiempos promedio
\begin{eqnarray*}
\overline{X}&=&lim_{t\rightarrow\infty}\frac{1}{t}\int_{0}^{\infty}X\left(u\right)du\\
\prob\left(X_{\infty}\leq x\right)&=&lim_{t\rightarrow\infty}\frac{1}{t}\int_{0}^{\infty}I_{x}\left(u\right)du,
\end{eqnarray*}
cuando estos l\'imites existan.
\end{Def}

Como consecuencia del teorema de Renovaci\'on-Recompensa, se tiene que el primer l\'imite  existe y es igual a la constante
\begin{eqnarray*}
\overline{X}&=&\frac{\esp\left[\int_{0}^{R_{1}}X\left(t\right)dt\right]}{\esp\left[R_{1}\right]},
\end{eqnarray*}
suponiendo que ambas esperanzas son finitas.

\begin{Note}
\begin{itemize}
\item[a)] Si el proceso regenerativo $X$ es positivo recurrente y tiene trayectorias muestrales no negativas, entonces la ecuaci\'on anterior es v\'alida.
\item[b)] Si $X$ es positivo recurrente regenerativo, podemos construir una \'unica versi\'on estacionaria de este proceso, $X_{e}=\left\{X_{e}\left(t\right)\right\}$, donde $X_{e}$ es un proceso estoc\'astico regenerativo y estrictamente estacionario, con distribuci\'on marginal distribuida como $X_{\infty}$
\end{itemize}
\end{Note}

%__________________________________________________________________________________________
%\subsection{Procesos Regenerativos Estacionarios - Stidham \cite{Stidham}}
%__________________________________________________________________________________________


Un proceso estoc\'astico a tiempo continuo $\left\{V\left(t\right),t\geq0\right\}$ es un proceso regenerativo si existe una sucesi\'on de variables aleatorias independientes e id\'enticamente distribuidas $\left\{X_{1},X_{2},\ldots\right\}$, sucesi\'on de renovaci\'on, tal que para cualquier conjunto de Borel $A$, 

\begin{eqnarray*}
\prob\left\{V\left(t\right)\in A|X_{1}+X_{2}+\cdots+X_{R\left(t\right)}=s,\left\{V\left(\tau\right),\tau<s\right\}\right\}=\prob\left\{V\left(t-s\right)\in A|X_{1}>t-s\right\},
\end{eqnarray*}
para todo $0\leq s\leq t$, donde $R\left(t\right)=\max\left\{X_{1}+X_{2}+\cdots+X_{j}\leq t\right\}=$n\'umero de renovaciones ({\emph{puntos de regeneraci\'on}}) que ocurren en $\left[0,t\right]$. El intervalo $\left[0,X_{1}\right)$ es llamado {\emph{primer ciclo de regeneraci\'on}} de $\left\{V\left(t \right),t\geq0\right\}$, $\left[X_{1},X_{1}+X_{2}\right)$ el {\emph{segundo ciclo de regeneraci\'on}}, y as\'i sucesivamente.

Sea $X=X_{1}$ y sea $F$ la funci\'on de distrbuci\'on de $X$


\begin{Def}
Se define el proceso estacionario, $\left\{V^{*}\left(t\right),t\geq0\right\}$, para $\left\{V\left(t\right),t\geq0\right\}$ por

\begin{eqnarray*}
\prob\left\{V\left(t\right)\in A\right\}=\frac{1}{\esp\left[X\right]}\int_{0}^{\infty}\prob\left\{V\left(t+x\right)\in A|X>x\right\}\left(1-F\left(x\right)\right)dx,
\end{eqnarray*} 
para todo $t\geq0$ y todo conjunto de Borel $A$.
\end{Def}

\begin{Def}
Una distribuci\'on se dice que es {\emph{aritm\'etica}} si todos sus puntos de incremento son m\'ultiplos de la forma $0,\lambda, 2\lambda,\ldots$ para alguna $\lambda>0$ entera.
\end{Def}


\begin{Def}
Una modificaci\'on medible de un proceso $\left\{V\left(t\right),t\geq0\right\}$, es una versi\'on de este, $\left\{V\left(t,w\right)\right\}$ conjuntamente medible para $t\geq0$ y para $w\in S$, $S$ espacio de estados para $\left\{V\left(t\right),t\geq0\right\}$.
\end{Def}

\begin{Teo}
Sea $\left\{V\left(t\right),t\geq\right\}$ un proceso regenerativo no negativo con modificaci\'on medible. Sea $\esp\left[X\right]<\infty$. Entonces el proceso estacionario dado por la ecuaci\'on anterior est\'a bien definido y tiene funci\'on de distribuci\'on independiente de $t$, adem\'as
\begin{itemize}
\item[i)] \begin{eqnarray*}
\esp\left[V^{*}\left(0\right)\right]&=&\frac{\esp\left[\int_{0}^{X}V\left(s\right)ds\right]}{\esp\left[X\right]}\end{eqnarray*}
\item[ii)] Si $\esp\left[V^{*}\left(0\right)\right]<\infty$, equivalentemente, si $\esp\left[\int_{0}^{X}V\left(s\right)ds\right]<\infty$,entonces
\begin{eqnarray*}
\frac{\int_{0}^{t}V\left(s\right)ds}{t}\rightarrow\frac{\esp\left[\int_{0}^{X}V\left(s\right)ds\right]}{\esp\left[X\right]}
\end{eqnarray*}
con probabilidad 1 y en media, cuando $t\rightarrow\infty$.
\end{itemize}
\end{Teo}
%
%___________________________________________________________________________________________
%\vspace{5.5cm}
%\chapter{Cadenas de Markov estacionarias}
%\vspace{-1.0cm}


%__________________________________________________________________________________________
%\subsection{Procesos Regenerativos Estacionarios - Stidham \cite{Stidham}}
%__________________________________________________________________________________________


Un proceso estoc\'astico a tiempo continuo $\left\{V\left(t\right),t\geq0\right\}$ es un proceso regenerativo si existe una sucesi\'on de variables aleatorias independientes e id\'enticamente distribuidas $\left\{X_{1},X_{2},\ldots\right\}$, sucesi\'on de renovaci\'on, tal que para cualquier conjunto de Borel $A$, 

\begin{eqnarray*}
\prob\left\{V\left(t\right)\in A|X_{1}+X_{2}+\cdots+X_{R\left(t\right)}=s,\left\{V\left(\tau\right),\tau<s\right\}\right\}=\prob\left\{V\left(t-s\right)\in A|X_{1}>t-s\right\},
\end{eqnarray*}
para todo $0\leq s\leq t$, donde $R\left(t\right)=\max\left\{X_{1}+X_{2}+\cdots+X_{j}\leq t\right\}=$n\'umero de renovaciones ({\emph{puntos de regeneraci\'on}}) que ocurren en $\left[0,t\right]$. El intervalo $\left[0,X_{1}\right)$ es llamado {\emph{primer ciclo de regeneraci\'on}} de $\left\{V\left(t \right),t\geq0\right\}$, $\left[X_{1},X_{1}+X_{2}\right)$ el {\emph{segundo ciclo de regeneraci\'on}}, y as\'i sucesivamente.

Sea $X=X_{1}$ y sea $F$ la funci\'on de distrbuci\'on de $X$


\begin{Def}
Se define el proceso estacionario, $\left\{V^{*}\left(t\right),t\geq0\right\}$, para $\left\{V\left(t\right),t\geq0\right\}$ por

\begin{eqnarray*}
\prob\left\{V\left(t\right)\in A\right\}=\frac{1}{\esp\left[X\right]}\int_{0}^{\infty}\prob\left\{V\left(t+x\right)\in A|X>x\right\}\left(1-F\left(x\right)\right)dx,
\end{eqnarray*} 
para todo $t\geq0$ y todo conjunto de Borel $A$.
\end{Def}

\begin{Def}
Una distribuci\'on se dice que es {\emph{aritm\'etica}} si todos sus puntos de incremento son m\'ultiplos de la forma $0,\lambda, 2\lambda,\ldots$ para alguna $\lambda>0$ entera.
\end{Def}


\begin{Def}
Una modificaci\'on medible de un proceso $\left\{V\left(t\right),t\geq0\right\}$, es una versi\'on de este, $\left\{V\left(t,w\right)\right\}$ conjuntamente medible para $t\geq0$ y para $w\in S$, $S$ espacio de estados para $\left\{V\left(t\right),t\geq0\right\}$.
\end{Def}

\begin{Teo}
Sea $\left\{V\left(t\right),t\geq\right\}$ un proceso regenerativo no negativo con modificaci\'on medible. Sea $\esp\left[X\right]<\infty$. Entonces el proceso estacionario dado por la ecuaci\'on anterior est\'a bien definido y tiene funci\'on de distribuci\'on independiente de $t$, adem\'as
\begin{itemize}
\item[i)] \begin{eqnarray*}
\esp\left[V^{*}\left(0\right)\right]&=&\frac{\esp\left[\int_{0}^{X}V\left(s\right)ds\right]}{\esp\left[X\right]}\end{eqnarray*}
\item[ii)] Si $\esp\left[V^{*}\left(0\right)\right]<\infty$, equivalentemente, si $\esp\left[\int_{0}^{X}V\left(s\right)ds\right]<\infty$,entonces
\begin{eqnarray*}
\frac{\int_{0}^{t}V\left(s\right)ds}{t}\rightarrow\frac{\esp\left[\int_{0}^{X}V\left(s\right)ds\right]}{\esp\left[X\right]}
\end{eqnarray*}
con probabilidad 1 y en media, cuando $t\rightarrow\infty$.
\end{itemize}
\end{Teo}

Sea la funci\'on generadora de momentos para $L_{i}$, el n\'umero de usuarios en la cola $Q_{i}\left(z\right)$ en cualquier momento, est\'a dada por el tiempo promedio de $z^{L_{i}\left(t\right)}$ sobre el ciclo regenerativo definido anteriormente. Entonces 



Es decir, es posible determinar las longitudes de las colas a cualquier tiempo $t$. Entonces, determinando el primer momento es posible ver que


\begin{Def}
El tiempo de Ciclo $C_{i}$ es el periodo de tiempo que comienza cuando la cola $i$ es visitada por primera vez en un ciclo, y termina cuando es visitado nuevamente en el pr\'oximo ciclo. La duraci\'on del mismo est\'a dada por $\tau_{i}\left(m+1\right)-\tau_{i}\left(m\right)$, o equivalentemente $\overline{\tau}_{i}\left(m+1\right)-\overline{\tau}_{i}\left(m\right)$ bajo condiciones de estabilidad.
\end{Def}


\begin{Def}
El tiempo de intervisita $I_{i}$ es el periodo de tiempo que comienza cuando se ha completado el servicio en un ciclo y termina cuando es visitada nuevamente en el pr\'oximo ciclo. Su  duraci\'on del mismo est\'a dada por $\tau_{i}\left(m+1\right)-\overline{\tau}_{i}\left(m\right)$.
\end{Def}

La duraci\'on del tiempo de intervisita es $\tau_{i}\left(m+1\right)-\overline{\tau}\left(m\right)$. Dado que el n\'umero de usuarios presentes en $Q_{i}$ al tiempo $t=\tau_{i}\left(m+1\right)$ es igual al n\'umero de arribos durante el intervalo de tiempo $\left[\overline{\tau}\left(m\right),\tau_{i}\left(m+1\right)\right]$ se tiene que


\begin{eqnarray*}
\esp\left[z_{i}^{L_{i}\left(\tau_{i}\left(m+1\right)\right)}\right]=\esp\left[\left\{P_{i}\left(z_{i}\right)\right\}^{\tau_{i}\left(m+1\right)-\overline{\tau}\left(m\right)}\right]
\end{eqnarray*}

entonces, si $I_{i}\left(z\right)=\esp\left[z^{\tau_{i}\left(m+1\right)-\overline{\tau}\left(m\right)}\right]$
se tiene que $F_{i}\left(z\right)=I_{i}\left[P_{i}\left(z\right)\right]$
para $i=1,2$.

Conforme a la definici\'on dada al principio del cap\'itulo, definici\'on (\ref{Def.Tn}), sean $T_{1},T_{2},\ldots$ los puntos donde las longitudes de las colas de la red de sistemas de visitas c\'iclicas son cero simult\'aneamente, cuando la cola $Q_{j}$ es visitada por el servidor para dar servicio, es decir, $L_{1}\left(T_{i}\right)=0,L_{2}\left(T_{i}\right)=0,\hat{L}_{1}\left(T_{i}\right)=0$ y $\hat{L}_{2}\left(T_{i}\right)=0$, a estos puntos se les denominar\'a puntos regenerativos. Entonces, 

\begin{Def}
Al intervalo de tiempo entre dos puntos regenerativos se le llamar\'a ciclo regenerativo.
\end{Def}

\begin{Def}
Para $T_{i}$ se define, $M_{i}$, el n\'umero de ciclos de visita a la cola $Q_{l}$, durante el ciclo regenerativo, es decir, $M_{i}$ es un proceso de renovaci\'on.
\end{Def}

\begin{Def}
Para cada uno de los $M_{i}$'s, se definen a su vez la duraci\'on de cada uno de estos ciclos de visita en el ciclo regenerativo, $C_{i}^{(m)}$, para $m=1,2,\ldots,M_{i}$, que a su vez, tambi\'en es n proceso de renovaci\'on.
\end{Def}

\footnote{In Stidham and  Heyman \cite{Stidham} shows that is sufficient for the regenerative process to be stationary that the mean regenerative cycle time is finite: $\esp\left[\sum_{m=1}^{M_{i}}C_{i}^{(m)}\right]<\infty$, 


 como cada $C_{i}^{(m)}$ contiene intervalos de r\'eplica positivos, se tiene que $\esp\left[M_{i}\right]<\infty$, adem\'as, como $M_{i}>0$, se tiene que la condici\'on anterior es equivalente a tener que $\esp\left[C_{i}\right]<\infty$,
por lo tanto una condici\'on suficiente para la existencia del proceso regenerativo est\'a dada por $\sum_{k=1}^{N}\mu_{k}<1.$}
%________________________________________________________________________
\subsection{Procesos Regenerativos: Thorisson}
%________________________________________________________________________

Para $\left\{X\left(t\right):t\geq0\right\}$ Proceso Estoc\'astico a tiempo continuo con estado de espacios $S$, que es un espacio m\'etrico, con trayectorias continuas por la derecha y con l\'imites por la izquierda c.s. Sea $N\left(t\right)$ un proceso de renovaci\'on en $\rea_{+}$ definido en el mismo espacio de probabilidad que $X\left(t\right)$, con tiempos de renovaci\'on $T$ y tiempos de inter-renovaci\'on $\xi_{n}=T_{n}-T_{n-1}$, con misma distribuci\'on $F$ de media finita $\mu$.

\begin{Def}
Un elemento aleatorio en un espacio medible $\left(E,\mathcal{E}\right)$ en un espacio de probabilidad $\left(\Omega,\mathcal{F},\prob\right)$ a $\left(E,\mathcal{E}\right)$, es decir,
para $A\in \mathcal{E}$,  se tiene que $\left\{Y\in A\right\}\in\mathcal{F}$, donde $\left\{Y\in A\right\}:=\left\{w\in\Omega:Y\left(w\right)\in A\right\}=:Y^{-1}A$.
\end{Def}

\begin{Note}
Tambi\'en se dice que $Y$ est\'a soportado por el espacio de probabilidad $\left(\Omega,\mathcal{F},\prob\right)$ y que $Y$ es un mapeo medible de $\Omega$ en $E$, es decir, es $\mathcal{F}/\mathcal{E}$ medible.
\end{Note}

\begin{Def}
Para cada $i\in \mathbb{I}$ sea $P_{i}$ una medida de probabilidad en un espacio medible $\left(E_{i},\mathcal{E}_{i}\right)$. Se define el espacio producto
$\otimes_{i\in\mathbb{I}}\left(E_{i},\mathcal{E}_{i}\right):=\left(\prod_{i\in\mathbb{I}}E_{i},\otimes_{i\in\mathbb{I}}\mathcal{E}_{i}\right)$, donde $\prod_{i\in\mathbb{I}}E_{i}$ es el producto cartesiano de los $E_{i}$'s, y $\otimes_{i\in\mathbb{I}}\mathcal{E}_{i}$ es la $\sigma$-\'algebra producto, es decir, es la $\sigma$-\'algebra m\'as peque\~na en $\prod_{i\in\mathbb{I}}E_{i}$ que hace al $i$-\'esimo mapeo proyecci\'on en $E_{i}$ medible para toda $i\in\mathbb{I}$ es la $\sigma$-\'algebra inducida por los mapeos proyecci\'on. $$\otimes_{i\in\mathbb{I}}\mathcal{E}_{i}:=\sigma\left\{\left\{y:y_{i}\in A\right\}:i\in\mathbb{I}\textrm{ y }A\in\mathcal{E}_{i}\right\}.$$
\end{Def}

\begin{Def}
Un espacio de probabilidad $\left(\tilde{\Omega},\tilde{\mathcal{F}},\tilde{\prob}\right)$ es una extensi\'on de otro espacio de probabilidad $\left(\Omega,\mathcal{F},\prob\right)$ si $\left(\tilde{\Omega},\tilde{\mathcal{F}},\tilde{\prob}\right)$ soporta un elemento aleatorio $\xi\in\left(\Omega,\mathcal{F}\right)$ que tienen a $\prob$ como distribuci\'on.
\end{Def}

\begin{Teo}
Sea $\mathbb{I}$ un conjunto de \'indices arbitrario. Para cada $i\in\mathbb{I}$ sea $P_{i}$ una medida de probabilidad en un espacio medible $\left(E_{i},\mathcal{E}_{i}\right)$. Entonces existe una \'unica medida de probabilidad $\otimes_{i\in\mathbb{I}}P_{i}$ en $\otimes_{i\in\mathbb{I}}\left(E_{i},\mathcal{E}_{i}\right)$ tal que 

\begin{eqnarray*}
\otimes_{i\in\mathbb{I}}P_{i}\left(y\in\prod_{i\in\mathbb{I}}E_{i}:y_{i}\in A_{i_{1}},\ldots,y_{n}\in A_{i_{n}}\right)=P_{i_{1}}\left(A_{i_{n}}\right)\cdots P_{i_{n}}\left(A_{i_{n}}\right)
\end{eqnarray*}
para todos los enteros $n>0$, toda $i_{1},\ldots,i_{n}\in\mathbb{I}$ y todo $A_{i_{1}}\in\mathcal{E}_{i_{1}},\ldots,A_{i_{n}}\in\mathcal{E}_{i_{n}}$
\end{Teo}

La medida $\otimes_{i\in\mathbb{I}}P_{i}$ es llamada la medida producto y $\otimes_{i\in\mathbb{I}}\left(E_{i},\mathcal{E}_{i},P_{i}\right):=\left(\prod_{i\in\mathbb{I}},E_{i},\otimes_{i\in\mathbb{I}}\mathcal{E}_{i},\otimes_{i\in\mathbb{I}}P_{i}\right)$, es llamado espacio de probabilidad producto.


\begin{Def}
Un espacio medible $\left(E,\mathcal{E}\right)$ es \textit{Polaco} si existe una m\'etrica en $E$ tal que $E$ es completo, es decir cada sucesi\'on de Cauchy converge a un l\'imite en $E$, y \textit{separable}, $E$ tienen un subconjunto denso numerable, y tal que $\mathcal{E}$ es generado por conjuntos abiertos.
\end{Def}


\begin{Def}
Dos espacios medibles $\left(E,\mathcal{E}\right)$ y $\left(G,\mathcal{G}\right)$ son Borel equivalentes \textit{isomorfos} si existe una biyecci\'on $f:E\rightarrow G$ tal que $f$ es $\mathcal{E}/\mathcal{G}$ medible y su inversa $f^{-1}$ es $\mathcal{G}/\mathcal{E}$ medible. La biyecci\'on es una equivalencia de Borel.
\end{Def}

\begin{Def}
Un espacio medible  $\left(E,\mathcal{E}\right)$ es un \textit{espacio est\'andar} si es Borel equivalente a $\left(G,\mathcal{G}\right)$, donde $G$ es un subconjunto de Borel de $\left[0,1\right]$ y $\mathcal{G}$ son los subconjuntos de Borel de $G$.
\end{Def}

\begin{Note}
Cualquier espacio Polaco es un espacio est\'andar.
\end{Note}


\begin{Def}
Un proceso estoc\'astico con conjunto de \'indices $\mathbb{I}$ y espacio de estados $\left(E,\mathcal{E}\right)$ es una familia $Z=\left(\mathbb{Z}_{s}\right)_{s\in\mathbb{I}}$ donde $\mathbb{Z}_{s}$ son elementos aleatorios definidos en un espacio de probabilidad com\'un $\left(\Omega,\mathcal{F},\prob\right)$ y todos toman valores en $\left(E,\mathcal{E}\right)$.
\end{Def}

\begin{Def}
Un proceso estoc\'astico \textit{one-sided contiuous time} (\textbf{PEOSCT}) es un proceso estoc\'astico con conjunto de \'indices $\mathbb{I}=\left[0,\infty\right)$.
\end{Def}


Sea $\left(E^{\mathbb{I}},\mathcal{E}^{\mathbb{I}}\right)$ denota el espacio producto $\left(E^{\mathbb{I}},\mathcal{E}^{\mathbb{I}}\right):=\otimes_{s\in\mathbb{I}}\left(E,\mathcal{E}\right)$. Vamos a considerar $\mathbb{Z}$ como un mapeo aleatorio, es decir, como un elemento aleatorio en $\left(E^{\mathbb{I}},\mathcal{E}^{\mathbb{I}}\right)$ definido por $Z\left(w\right)=\left(Z_{s}\left(w\right)\right)_{s\in\mathbb{I}}$ y $w\in\Omega$.

\begin{Note}
La distribuci\'on de un proceso estoc\'astico $Z$ es la distribuci\'on de $Z$ como un elemento aleatorio en $\left(E^{\mathbb{I}},\mathcal{E}^{\mathbb{I}}\right)$. La distribuci\'on de $Z$ esta determinada de manera \'unica por las distribuciones finito dimensionales.
\end{Note}

\begin{Note}
En particular cuando $Z$ toma valores reales, es decir, $\left(E,\mathcal{E}\right)=\left(\mathbb{R},\mathcal{B}\right)$ las distribuciones finito dimensionales est\'an determinadas por las funciones de distribuci\'on finito dimensionales

\begin{eqnarray}
\prob\left(Z_{t_{1}}\leq x_{1},\ldots,Z_{t_{n}}\leq x_{n}\right),x_{1},\ldots,x_{n}\in\mathbb{R},t_{1},\ldots,t_{n}\in\mathbb{I},n\geq1.
\end{eqnarray}
\end{Note}

\begin{Note}
Para espacios polacos $\left(E,\mathcal{E}\right)$ el Teorema de Consistencia de Kolmogorov asegura que dada una colecci\'on de distribuciones finito dimensionales consistentes, siempre existe un proceso estoc\'astico que posee tales distribuciones finito dimensionales.
\end{Note}


\begin{Def}
Las trayectorias de $Z$ son las realizaciones $Z\left(w\right)$ para $w\in\Omega$ del mapeo aleatorio $Z$.
\end{Def}

\begin{Note}
Algunas restricciones se imponen sobre las trayectorias, por ejemplo que sean continuas por la derecha, o continuas por la derecha con l\'imites por la izquierda, o de manera m\'as general, se pedir\'a que caigan en alg\'un subconjunto $H$ de $E^{\mathbb{I}}$. En este caso es natural considerar a $Z$ como un elemento aleatorio que no est\'a en $\left(E^{\mathbb{I}},\mathcal{E}^{\mathbb{I}}\right)$ sino en $\left(H,\mathcal{H}\right)$, donde $\mathcal{H}$ es la $\sigma$-\'algebra generada por los mapeos proyecci\'on que toman a $z\in H$ a $z_{t}\in E$ para $t\in\mathbb{I}$. A $\mathcal{H}$ se le conoce como la traza de $H$ en $E^{\mathbb{I}}$, es decir,
\begin{eqnarray}
\mathcal{H}:=E^{\mathbb{I}}\cap H:=\left\{A\cap H:A\in E^{\mathbb{I}}\right\}.
\end{eqnarray}
\end{Note}


\begin{Note}
$Z$ tiene trayectorias con valores en $H$ y cada $Z_{t}$ es un mapeo medible de $\left(\Omega,\mathcal{F}\right)$ a $\left(H,\mathcal{H}\right)$. Cuando se considera un espacio de trayectorias en particular $H$, al espacio $\left(H,\mathcal{H}\right)$ se le llama el espacio de trayectorias de $Z$.
\end{Note}

\begin{Note}
La distribuci\'on del proceso estoc\'astico $Z$ con espacio de trayectorias $\left(H,\mathcal{H}\right)$ es la distribuci\'on de $Z$ como  un elemento aleatorio en $\left(H,\mathcal{H}\right)$. La distribuci\'on, nuevemente, est\'a determinada de manera \'unica por las distribuciones finito dimensionales.
\end{Note}


\begin{Def}
Sea $Z$ un PEOSCT  con espacio de estados $\left(E,\mathcal{E}\right)$ y sea $T$ un tiempo aleatorio en $\left[0,\infty\right)$. Por $Z_{T}$ se entiende el mapeo con valores en $E$ definido en $\Omega$ en la manera obvia:
\begin{eqnarray*}
Z_{T}\left(w\right):=Z_{T\left(w\right)}\left(w\right). w\in\Omega.
\end{eqnarray*}
\end{Def}

\begin{Def}
Un PEOSCT $Z$ es conjuntamente medible (\textbf{CM}) si el mapeo que toma $\left(w,t\right)\in\Omega\times\left[0,\infty\right)$ a $Z_{t}\left(w\right)\in E$ es $\mathcal{F}\otimes\mathcal{B}\left[0,\infty\right)/\mathcal{E}$ medible.
\end{Def}

\begin{Note}
Un PEOSCT-CM implica que el proceso es medible, dado que $Z_{T}$ es una composici\'on  de dos mapeos continuos: el primero que toma $w$ en $\left(w,T\left(w\right)\right)$ es $\mathcal{F}/\mathcal{F}\otimes\mathcal{B}\left[0,\infty\right)$ medible, mientras que el segundo toma $\left(w,T\left(w\right)\right)$ en $Z_{T\left(w\right)}\left(w\right)$ es $\mathcal{F}\otimes\mathcal{B}\left[0,\infty\right)/\mathcal{E}$ medible.
\end{Note}


\begin{Def}
Un PEOSCT con espacio de estados $\left(H,\mathcal{H}\right)$ es can\'onicamente conjuntamente medible (\textbf{CCM}) si el mapeo $\left(z,t\right)\in H\times\left[0,\infty\right)$ en $Z_{t}\in E$ es $\mathcal{H}\otimes\mathcal{B}\left[0,\infty\right)/\mathcal{E}$ medible.
\end{Def}

\begin{Note}
Un PEOSCTCCM implica que el proceso es CM, dado que un PECCM $Z$ es un mapeo de $\Omega\times\left[0,\infty\right)$ a $E$, es la composici\'on de dos mapeos medibles: el primero, toma $\left(w,t\right)$ en $\left(Z\left(w\right),t\right)$ es $\mathcal{F}\otimes\mathcal{B}\left[0,\infty\right)/\mathcal{H}\otimes\mathcal{B}\left[0,\infty\right)$ medible, y el segundo que toma $\left(Z\left(w\right),t\right)$  en $Z_{t}\left(w\right)$ es $\mathcal{H}\otimes\mathcal{B}\left[0,\infty\right)/\mathcal{E}$ medible. Por tanto CCM es una condici\'on m\'as fuerte que CM.
\end{Note}

\begin{Def}
Un conjunto de trayectorias $H$ de un PEOSCT $Z$, es internamente shift-invariante (\textbf{ISI}) si 
\begin{eqnarray*}
\left\{\left(z_{t+s}\right)_{s\in\left[0,\infty\right)}:z\in H\right\}=H\textrm{, }t\in\left[0,\infty\right).
\end{eqnarray*}
\end{Def}


\begin{Def}
Dado un PEOSCTISI, se define el mapeo-shift $\theta_{t}$, $t\in\left[0,\infty\right)$, de $H$ a $H$ por 
\begin{eqnarray*}
\theta_{t}z=\left(z_{t+s}\right)_{s\in\left[0,\infty\right)}\textrm{, }z\in H.
\end{eqnarray*}
\end{Def}

\begin{Def}
Se dice que un proceso $Z$ es shift-medible (\textbf{SM}) si $Z$ tiene un conjunto de trayectorias $H$ que es ISI y adem\'as el mapeo que toma $\left(z,t\right)\in H\times\left[0,\infty\right)$ en $\theta_{t}z\in H$ es $\mathcal{H}\otimes\mathcal{B}\left[0,\infty\right)/\mathcal{H}$ medible.
\end{Def}

\begin{Note}
Un proceso estoc\'astico con conjunto de trayectorias $H$ ISI es shift-medible si y s\'olo si es CCM
\end{Note}

\begin{Note}
\begin{itemize}
\item Dado el espacio polaco $\left(E,\mathcal{E}\right)$ se tiene el  conjunto de trayectorias $D_{E}\left[0,\infty\right)$ que es ISI, entonces cumpe con ser CCM.

\item Si $G$ es abierto, podemos cubrirlo por bolas abiertas cuay cerradura este contenida en $G$, y como $G$ es segundo numerable como subespacio de $E$, lo podemos cubrir por una cantidad numerable de bolas abiertas.

\end{itemize}
\end{Note}


\begin{Note}
Los procesos estoc\'asticos $Z$ a tiempo discreto con espacio de estados polaco, tambi\'en tiene un espacio de trayectorias polaco y por tanto tiene distribuciones condicionales regulares.
\end{Note}

\begin{Teo}
El producto numerable de espacios polacos es polaco.
\end{Teo}


\begin{Def}
Sea $\left(\Omega,\mathcal{F},\prob\right)$ espacio de probabilidad que soporta al proceso $Z=\left(Z_{s}\right)_{s\in\left[0,\infty\right)}$ y $S=\left(S_{k}\right)_{0}^{\infty}$ donde $Z$ es un PEOSCTM con espacio de estados $\left(E,\mathcal{E}\right)$  y espacio de trayectorias $\left(H,\mathcal{H}\right)$  y adem\'as $S$ es una sucesi\'on de tiempos aleatorios one-sided que satisfacen la condici\'on $0\leq S_{0}<S_{1}<\cdots\rightarrow\infty$. Considerando $S$ como un mapeo medible de $\left(\Omega,\mathcal{F}\right)$ al espacio sucesi\'on $\left(L,\mathcal{L}\right)$, donde 
\begin{eqnarray*}
L=\left\{\left(s_{k}\right)_{0}^{\infty}\in\left[0,\infty\right)^{\left\{0,1,\ldots\right\}}:s_{0}<s_{1}<\cdots\rightarrow\infty\right\},
\end{eqnarray*}
donde $\mathcal{L}$ son los subconjuntos de Borel de $L$, es decir, $\mathcal{L}=L\cap\mathcal{B}^{\left\{0,1,\ldots\right\}}$.

As\'i el par $\left(Z,S\right)$ es un mapeo medible de  $\left(\Omega,\mathcal{F}\right)$ en $\left(H\times L,\mathcal{H}\otimes\mathcal{L}\right)$. El par $\mathcal{H}\otimes\mathcal{L}^{+}$ denotar\'a la clase de todas las funciones medibles de $\left(H\times L,\mathcal{H}\otimes\mathcal{L}\right)$ en $\left(\left[0,\infty\right),\mathcal{B}\left[0,\infty\right)\right)$.
\end{Def}


\begin{Def}
Sea $\theta_{t}$ el mapeo-shift conjunto de $H\times L$ en $H\times L$ dado por
\begin{eqnarray*}
\theta_{t}\left(z,\left(s_{k}\right)_{0}^{\infty}\right)=\theta_{t}\left(z,\left(s_{n_{t-}+k}-t\right)_{0}^{\infty}\right)
\end{eqnarray*}
donde 
$n_{t-}=inf\left\{n\geq1:s_{n}\geq t\right\}$.
\end{Def}

\begin{Note}
Con la finalidad de poder realizar los shift's sin complicaciones de medibilidad, se supondr\'a que $Z$ es shit-medible, es decir, el conjunto de trayectorias $H$ es invariante bajo shifts del tiempo y el mapeo que toma $\left(z,t\right)\in H\times\left[0,\infty\right)$ en $z_{t}\in E$ es $\mathcal{H}\otimes\mathcal{B}\left[0,\infty\right)/\mathcal{E}$ medible.
\end{Note}

\begin{Def}
Dado un proceso \textbf{PEOSSM} (Proceso Estoc\'astico One Side Shift Medible) $Z$, se dice regenerativo cl\'asico con tiempos de regeneraci\'on $S$ si 

\begin{eqnarray*}
\theta_{S_{n}}\left(Z,S\right)=\left(Z^{0},S^{0}\right),n\geq0
\end{eqnarray*}
y adem\'as $\theta_{S_{n}}\left(Z,S\right)$ es independiente de $\left(\left(Z_{s}\right)s\in\left[0,S_{n}\right),S_{0},\ldots,S_{n}\right)$
Si lo anterior se cumple, al par $\left(Z,S\right)$ se le llama regenerativo cl\'asico.
\end{Def}

\begin{Note}
Si el par $\left(Z,S\right)$ es regenerativo cl\'asico, entonces las longitudes de los ciclos $X_{1},X_{2},\ldots,$ son i.i.d. e independientes de la longitud del retraso $S_{0}$, es decir, $S$ es un proceso de renovaci\'on. Las longitudes de los ciclos tambi\'en son llamados tiempos de inter-regeneraci\'on y tiempos de ocurrencia.

\end{Note}

\begin{Teo}
Sup\'ongase que el par $\left(Z,S\right)$ es regenerativo cl\'asico con $\esp\left[X_{1}\right]<\infty$. Entonces $\left(Z^{*},S^{*}\right)$ en el teorema 2.1 es una versi\'on estacionaria de $\left(Z,S\right)$. Adem\'as, si $X_{1}$ es lattice con span $d$, entonces $\left(Z^{**},S^{**}\right)$ en el teorema 2.2 es una versi\'on periodicamente estacionaria de $\left(Z,S\right)$ con periodo $d$.

\end{Teo}

\begin{Def}
Una variable aleatoria $X_{1}$ es \textit{spread out} si existe una $n\geq1$ y una  funci\'on $f\in\mathcal{B}^{+}$ tal que $\int_{\rea}f\left(x\right)dx>0$ con $X_{2},X_{3},\ldots,X_{n}$ copias i.i.d  de $X_{1}$, $$\prob\left(X_{1}+\cdots+X_{n}\in B\right)\geq\int_{B}f\left(x\right)dx$$ para $B\in\mathcal{B}$.

\end{Def}



\begin{Def}
Dado un proceso estoc\'astico $Z$ se le llama \textit{wide-sense regenerative} (\textbf{WSR}) con tiempos de regeneraci\'on $S$ si $\theta_{S_{n}}\left(Z,S\right)=\left(Z^{0},S^{0}\right)$ para $n\geq0$ en distribuci\'on y $\theta_{S_{n}}\left(Z,S\right)$ es independiente de $\left(S_{0},S_{1},\ldots,S_{n}\right)$ para $n\geq0$.
Se dice que el par $\left(Z,S\right)$ es WSR si lo anterior se cumple.
\end{Def}


\begin{Note}
\begin{itemize}
\item El proceso de trayectorias $\left(\theta_{s}Z\right)_{s\in\left[0,\infty\right)}$ es WSR con tiempos de regeneraci\'on $S$ pero no es regenerativo cl\'asico.

\item Si $Z$ es cualquier proceso estacionario y $S$ es un proceso de renovaci\'on que es independiente de $Z$, entonces $\left(Z,S\right)$ es WSR pero en general no es regenerativo cl\'asico

\end{itemize}

\end{Note}


\begin{Note}
Para cualquier proceso estoc\'astico $Z$, el proceso de trayectorias $\left(\theta_{s}Z\right)_{s\in\left[0,\infty\right)}$ es siempre un proceso de Markov.
\end{Note}



\begin{Teo}
Supongase que el par $\left(Z,S\right)$ es WSR con $\esp\left[X_{1}\right]<\infty$. Entonces $\left(Z^{*},S^{*}\right)$ en el teorema 2.1 es una versi\'on estacionaria de 
$\left(Z,S\right)$.
\end{Teo}


\begin{Teo}
Supongase que $\left(Z,S\right)$ es cycle-stationary con $\esp\left[X_{1}\right]<\infty$. Sea $U$ distribuida uniformemente en $\left[0,1\right)$ e independiente de $\left(Z^{0},S^{0}\right)$ y sea $\prob^{*}$ la medida de probabilidad en $\left(\Omega,\prob\right)$ definida por $$d\prob^{*}=\frac{X_{1}}{\esp\left[X_{1}\right]}d\prob$$. Sea $\left(Z^{*},S^{*}\right)$ con distribuci\'on $\prob^{*}\left(\theta_{UX_{1}}\left(Z^{0},S^{0}\right)\in\cdot\right)$. Entonces $\left(Z^{}*,S^{*}\right)$ es estacionario,
\begin{eqnarray*}
\esp\left[f\left(Z^{*},S^{*}\right)\right]=\esp\left[\int_{0}^{X_{1}}f\left(\theta_{s}\left(Z^{0},S^{0}\right)\right)ds\right]/\esp\left[X_{1}\right]
\end{eqnarray*}
$f\in\mathcal{H}\otimes\mathcal{L}^{+}$, and $S_{0}^{*}$ es continuo con funci\'on distribuci\'on $G_{\infty}$ definida por $$G_{\infty}\left(x\right):=\frac{\esp\left[X_{1}\right]\wedge x}{\esp\left[X_{1}\right]}$$ para $x\geq0$ y densidad $\prob\left[X_{1}>x\right]/\esp\left[X_{1}\right]$, con $x\geq0$.

\end{Teo}


\begin{Teo}
Sea $Z$ un Proceso Estoc\'astico un lado shift-medible \textit{one-sided shift-measurable stochastic process}, (PEOSSM),
y $S_{0}$ y $S_{1}$ tiempos aleatorios tales que $0\leq S_{0}<S_{1}$ y
\begin{equation}
\theta_{S_{1}}Z=\theta_{S_{0}}Z\textrm{ en distribuci\'on}.
\end{equation}

Entonces el espacio de probabilidad subyacente $\left(\Omega,\mathcal{F},\prob\right)$ puede extenderse para soportar una sucesi\'on de tiempos aleatorios $S$ tales que

\begin{eqnarray}
\theta_{S_{n}}\left(Z,S\right)=\left(Z^{0},S^{0}\right),n\geq0,\textrm{ en distribuci\'on},\\
\left(Z,S_{0},S_{1}\right)\textrm{ depende de }\left(X_{2},X_{3},\ldots\right)\textrm{ solamente a traves de }\theta_{S_{1}}Z.
\end{eqnarray}
\end{Teo}

%________________________________________________________________________
\subsection{Procesos Regenerativos: Thorisson}
%________________________________________________________________________

Para $\left\{X\left(t\right):t\geq0\right\}$ Proceso Estoc\'astico a tiempo continuo con estado de espacios $S$, que es un espacio m\'etrico, con trayectorias continuas por la derecha y con l\'imites por la izquierda c.s. Sea $N\left(t\right)$ un proceso de renovaci\'on en $\rea_{+}$ definido en el mismo espacio de probabilidad que $X\left(t\right)$, con tiempos de renovaci\'on $T$ y tiempos de inter-renovaci\'on $\xi_{n}=T_{n}-T_{n-1}$, con misma distribuci\'on $F$ de media finita $\mu$.

\begin{Def}
Un elemento aleatorio en un espacio medible $\left(E,\mathcal{E}\right)$ en un espacio de probabilidad $\left(\Omega,\mathcal{F},\prob\right)$ a $\left(E,\mathcal{E}\right)$, es decir,
para $A\in \mathcal{E}$,  se tiene que $\left\{Y\in A\right\}\in\mathcal{F}$, donde $\left\{Y\in A\right\}:=\left\{w\in\Omega:Y\left(w\right)\in A\right\}=:Y^{-1}A$.
\end{Def}

\begin{Note}
Tambi\'en se dice que $Y$ est\'a soportado por el espacio de probabilidad $\left(\Omega,\mathcal{F},\prob\right)$ y que $Y$ es un mapeo medible de $\Omega$ en $E$, es decir, es $\mathcal{F}/\mathcal{E}$ medible.
\end{Note}

\begin{Def}
Para cada $i\in \mathbb{I}$ sea $P_{i}$ una medida de probabilidad en un espacio medible $\left(E_{i},\mathcal{E}_{i}\right)$. Se define el espacio producto
$\otimes_{i\in\mathbb{I}}\left(E_{i},\mathcal{E}_{i}\right):=\left(\prod_{i\in\mathbb{I}}E_{i},\otimes_{i\in\mathbb{I}}\mathcal{E}_{i}\right)$, donde $\prod_{i\in\mathbb{I}}E_{i}$ es el producto cartesiano de los $E_{i}$'s, y $\otimes_{i\in\mathbb{I}}\mathcal{E}_{i}$ es la $\sigma$-\'algebra producto, es decir, es la $\sigma$-\'algebra m\'as peque\~na en $\prod_{i\in\mathbb{I}}E_{i}$ que hace al $i$-\'esimo mapeo proyecci\'on en $E_{i}$ medible para toda $i\in\mathbb{I}$ es la $\sigma$-\'algebra inducida por los mapeos proyecci\'on. $$\otimes_{i\in\mathbb{I}}\mathcal{E}_{i}:=\sigma\left\{\left\{y:y_{i}\in A\right\}:i\in\mathbb{I}\textrm{ y }A\in\mathcal{E}_{i}\right\}.$$
\end{Def}

\begin{Def}
Un espacio de probabilidad $\left(\tilde{\Omega},\tilde{\mathcal{F}},\tilde{\prob}\right)$ es una extensi\'on de otro espacio de probabilidad $\left(\Omega,\mathcal{F},\prob\right)$ si $\left(\tilde{\Omega},\tilde{\mathcal{F}},\tilde{\prob}\right)$ soporta un elemento aleatorio $\xi\in\left(\Omega,\mathcal{F}\right)$ que tienen a $\prob$ como distribuci\'on.
\end{Def}

\begin{Teo}
Sea $\mathbb{I}$ un conjunto de \'indices arbitrario. Para cada $i\in\mathbb{I}$ sea $P_{i}$ una medida de probabilidad en un espacio medible $\left(E_{i},\mathcal{E}_{i}\right)$. Entonces existe una \'unica medida de probabilidad $\otimes_{i\in\mathbb{I}}P_{i}$ en $\otimes_{i\in\mathbb{I}}\left(E_{i},\mathcal{E}_{i}\right)$ tal que 

\begin{eqnarray*}
\otimes_{i\in\mathbb{I}}P_{i}\left(y\in\prod_{i\in\mathbb{I}}E_{i}:y_{i}\in A_{i_{1}},\ldots,y_{n}\in A_{i_{n}}\right)=P_{i_{1}}\left(A_{i_{n}}\right)\cdots P_{i_{n}}\left(A_{i_{n}}\right)
\end{eqnarray*}
para todos los enteros $n>0$, toda $i_{1},\ldots,i_{n}\in\mathbb{I}$ y todo $A_{i_{1}}\in\mathcal{E}_{i_{1}},\ldots,A_{i_{n}}\in\mathcal{E}_{i_{n}}$
\end{Teo}

La medida $\otimes_{i\in\mathbb{I}}P_{i}$ es llamada la medida producto y $\otimes_{i\in\mathbb{I}}\left(E_{i},\mathcal{E}_{i},P_{i}\right):=\left(\prod_{i\in\mathbb{I}},E_{i},\otimes_{i\in\mathbb{I}}\mathcal{E}_{i},\otimes_{i\in\mathbb{I}}P_{i}\right)$, es llamado espacio de probabilidad producto.


\begin{Def}
Un espacio medible $\left(E,\mathcal{E}\right)$ es \textit{Polaco} si existe una m\'etrica en $E$ tal que $E$ es completo, es decir cada sucesi\'on de Cauchy converge a un l\'imite en $E$, y \textit{separable}, $E$ tienen un subconjunto denso numerable, y tal que $\mathcal{E}$ es generado por conjuntos abiertos.
\end{Def}


\begin{Def}
Dos espacios medibles $\left(E,\mathcal{E}\right)$ y $\left(G,\mathcal{G}\right)$ son Borel equivalentes \textit{isomorfos} si existe una biyecci\'on $f:E\rightarrow G$ tal que $f$ es $\mathcal{E}/\mathcal{G}$ medible y su inversa $f^{-1}$ es $\mathcal{G}/\mathcal{E}$ medible. La biyecci\'on es una equivalencia de Borel.
\end{Def}

\begin{Def}
Un espacio medible  $\left(E,\mathcal{E}\right)$ es un \textit{espacio est\'andar} si es Borel equivalente a $\left(G,\mathcal{G}\right)$, donde $G$ es un subconjunto de Borel de $\left[0,1\right]$ y $\mathcal{G}$ son los subconjuntos de Borel de $G$.
\end{Def}

\begin{Note}
Cualquier espacio Polaco es un espacio est\'andar.
\end{Note}


\begin{Def}
Un proceso estoc\'astico con conjunto de \'indices $\mathbb{I}$ y espacio de estados $\left(E,\mathcal{E}\right)$ es una familia $Z=\left(\mathbb{Z}_{s}\right)_{s\in\mathbb{I}}$ donde $\mathbb{Z}_{s}$ son elementos aleatorios definidos en un espacio de probabilidad com\'un $\left(\Omega,\mathcal{F},\prob\right)$ y todos toman valores en $\left(E,\mathcal{E}\right)$.
\end{Def}

\begin{Def}
Un proceso estoc\'astico \textit{one-sided contiuous time} (\textbf{PEOSCT}) es un proceso estoc\'astico con conjunto de \'indices $\mathbb{I}=\left[0,\infty\right)$.
\end{Def}


Sea $\left(E^{\mathbb{I}},\mathcal{E}^{\mathbb{I}}\right)$ denota el espacio producto $\left(E^{\mathbb{I}},\mathcal{E}^{\mathbb{I}}\right):=\otimes_{s\in\mathbb{I}}\left(E,\mathcal{E}\right)$. Vamos a considerar $\mathbb{Z}$ como un mapeo aleatorio, es decir, como un elemento aleatorio en $\left(E^{\mathbb{I}},\mathcal{E}^{\mathbb{I}}\right)$ definido por $Z\left(w\right)=\left(Z_{s}\left(w\right)\right)_{s\in\mathbb{I}}$ y $w\in\Omega$.

\begin{Note}
La distribuci\'on de un proceso estoc\'astico $Z$ es la distribuci\'on de $Z$ como un elemento aleatorio en $\left(E^{\mathbb{I}},\mathcal{E}^{\mathbb{I}}\right)$. La distribuci\'on de $Z$ esta determinada de manera \'unica por las distribuciones finito dimensionales.
\end{Note}

\begin{Note}
En particular cuando $Z$ toma valores reales, es decir, $\left(E,\mathcal{E}\right)=\left(\mathbb{R},\mathcal{B}\right)$ las distribuciones finito dimensionales est\'an determinadas por las funciones de distribuci\'on finito dimensionales

\begin{eqnarray}
\prob\left(Z_{t_{1}}\leq x_{1},\ldots,Z_{t_{n}}\leq x_{n}\right),x_{1},\ldots,x_{n}\in\mathbb{R},t_{1},\ldots,t_{n}\in\mathbb{I},n\geq1.
\end{eqnarray}
\end{Note}

\begin{Note}
Para espacios polacos $\left(E,\mathcal{E}\right)$ el Teorema de Consistencia de Kolmogorov asegura que dada una colecci\'on de distribuciones finito dimensionales consistentes, siempre existe un proceso estoc\'astico que posee tales distribuciones finito dimensionales.
\end{Note}


\begin{Def}
Las trayectorias de $Z$ son las realizaciones $Z\left(w\right)$ para $w\in\Omega$ del mapeo aleatorio $Z$.
\end{Def}

\begin{Note}
Algunas restricciones se imponen sobre las trayectorias, por ejemplo que sean continuas por la derecha, o continuas por la derecha con l\'imites por la izquierda, o de manera m\'as general, se pedir\'a que caigan en alg\'un subconjunto $H$ de $E^{\mathbb{I}}$. En este caso es natural considerar a $Z$ como un elemento aleatorio que no est\'a en $\left(E^{\mathbb{I}},\mathcal{E}^{\mathbb{I}}\right)$ sino en $\left(H,\mathcal{H}\right)$, donde $\mathcal{H}$ es la $\sigma$-\'algebra generada por los mapeos proyecci\'on que toman a $z\in H$ a $z_{t}\in E$ para $t\in\mathbb{I}$. A $\mathcal{H}$ se le conoce como la traza de $H$ en $E^{\mathbb{I}}$, es decir,
\begin{eqnarray}
\mathcal{H}:=E^{\mathbb{I}}\cap H:=\left\{A\cap H:A\in E^{\mathbb{I}}\right\}.
\end{eqnarray}
\end{Note}


\begin{Note}
$Z$ tiene trayectorias con valores en $H$ y cada $Z_{t}$ es un mapeo medible de $\left(\Omega,\mathcal{F}\right)$ a $\left(H,\mathcal{H}\right)$. Cuando se considera un espacio de trayectorias en particular $H$, al espacio $\left(H,\mathcal{H}\right)$ se le llama el espacio de trayectorias de $Z$.
\end{Note}

\begin{Note}
La distribuci\'on del proceso estoc\'astico $Z$ con espacio de trayectorias $\left(H,\mathcal{H}\right)$ es la distribuci\'on de $Z$ como  un elemento aleatorio en $\left(H,\mathcal{H}\right)$. La distribuci\'on, nuevemente, est\'a determinada de manera \'unica por las distribuciones finito dimensionales.
\end{Note}


\begin{Def}
Sea $Z$ un PEOSCT  con espacio de estados $\left(E,\mathcal{E}\right)$ y sea $T$ un tiempo aleatorio en $\left[0,\infty\right)$. Por $Z_{T}$ se entiende el mapeo con valores en $E$ definido en $\Omega$ en la manera obvia:
\begin{eqnarray*}
Z_{T}\left(w\right):=Z_{T\left(w\right)}\left(w\right). w\in\Omega.
\end{eqnarray*}
\end{Def}

\begin{Def}
Un PEOSCT $Z$ es conjuntamente medible (\textbf{CM}) si el mapeo que toma $\left(w,t\right)\in\Omega\times\left[0,\infty\right)$ a $Z_{t}\left(w\right)\in E$ es $\mathcal{F}\otimes\mathcal{B}\left[0,\infty\right)/\mathcal{E}$ medible.
\end{Def}

\begin{Note}
Un PEOSCT-CM implica que el proceso es medible, dado que $Z_{T}$ es una composici\'on  de dos mapeos continuos: el primero que toma $w$ en $\left(w,T\left(w\right)\right)$ es $\mathcal{F}/\mathcal{F}\otimes\mathcal{B}\left[0,\infty\right)$ medible, mientras que el segundo toma $\left(w,T\left(w\right)\right)$ en $Z_{T\left(w\right)}\left(w\right)$ es $\mathcal{F}\otimes\mathcal{B}\left[0,\infty\right)/\mathcal{E}$ medible.
\end{Note}


\begin{Def}
Un PEOSCT con espacio de estados $\left(H,\mathcal{H}\right)$ es can\'onicamente conjuntamente medible (\textbf{CCM}) si el mapeo $\left(z,t\right)\in H\times\left[0,\infty\right)$ en $Z_{t}\in E$ es $\mathcal{H}\otimes\mathcal{B}\left[0,\infty\right)/\mathcal{E}$ medible.
\end{Def}

\begin{Note}
Un PEOSCTCCM implica que el proceso es CM, dado que un PECCM $Z$ es un mapeo de $\Omega\times\left[0,\infty\right)$ a $E$, es la composici\'on de dos mapeos medibles: el primero, toma $\left(w,t\right)$ en $\left(Z\left(w\right),t\right)$ es $\mathcal{F}\otimes\mathcal{B}\left[0,\infty\right)/\mathcal{H}\otimes\mathcal{B}\left[0,\infty\right)$ medible, y el segundo que toma $\left(Z\left(w\right),t\right)$  en $Z_{t}\left(w\right)$ es $\mathcal{H}\otimes\mathcal{B}\left[0,\infty\right)/\mathcal{E}$ medible. Por tanto CCM es una condici\'on m\'as fuerte que CM.
\end{Note}

\begin{Def}
Un conjunto de trayectorias $H$ de un PEOSCT $Z$, es internamente shift-invariante (\textbf{ISI}) si 
\begin{eqnarray*}
\left\{\left(z_{t+s}\right)_{s\in\left[0,\infty\right)}:z\in H\right\}=H\textrm{, }t\in\left[0,\infty\right).
\end{eqnarray*}
\end{Def}


\begin{Def}
Dado un PEOSCTISI, se define el mapeo-shift $\theta_{t}$, $t\in\left[0,\infty\right)$, de $H$ a $H$ por 
\begin{eqnarray*}
\theta_{t}z=\left(z_{t+s}\right)_{s\in\left[0,\infty\right)}\textrm{, }z\in H.
\end{eqnarray*}
\end{Def}

\begin{Def}
Se dice que un proceso $Z$ es shift-medible (\textbf{SM}) si $Z$ tiene un conjunto de trayectorias $H$ que es ISI y adem\'as el mapeo que toma $\left(z,t\right)\in H\times\left[0,\infty\right)$ en $\theta_{t}z\in H$ es $\mathcal{H}\otimes\mathcal{B}\left[0,\infty\right)/\mathcal{H}$ medible.
\end{Def}

\begin{Note}
Un proceso estoc\'astico con conjunto de trayectorias $H$ ISI es shift-medible si y s\'olo si es CCM
\end{Note}

\begin{Note}
\begin{itemize}
\item Dado el espacio polaco $\left(E,\mathcal{E}\right)$ se tiene el  conjunto de trayectorias $D_{E}\left[0,\infty\right)$ que es ISI, entonces cumpe con ser CCM.

\item Si $G$ es abierto, podemos cubrirlo por bolas abiertas cuay cerradura este contenida en $G$, y como $G$ es segundo numerable como subespacio de $E$, lo podemos cubrir por una cantidad numerable de bolas abiertas.

\end{itemize}
\end{Note}


\begin{Note}
Los procesos estoc\'asticos $Z$ a tiempo discreto con espacio de estados polaco, tambi\'en tiene un espacio de trayectorias polaco y por tanto tiene distribuciones condicionales regulares.
\end{Note}

\begin{Teo}
El producto numerable de espacios polacos es polaco.
\end{Teo}


\begin{Def}
Sea $\left(\Omega,\mathcal{F},\prob\right)$ espacio de probabilidad que soporta al proceso $Z=\left(Z_{s}\right)_{s\in\left[0,\infty\right)}$ y $S=\left(S_{k}\right)_{0}^{\infty}$ donde $Z$ es un PEOSCTM con espacio de estados $\left(E,\mathcal{E}\right)$  y espacio de trayectorias $\left(H,\mathcal{H}\right)$  y adem\'as $S$ es una sucesi\'on de tiempos aleatorios one-sided que satisfacen la condici\'on $0\leq S_{0}<S_{1}<\cdots\rightarrow\infty$. Considerando $S$ como un mapeo medible de $\left(\Omega,\mathcal{F}\right)$ al espacio sucesi\'on $\left(L,\mathcal{L}\right)$, donde 
\begin{eqnarray*}
L=\left\{\left(s_{k}\right)_{0}^{\infty}\in\left[0,\infty\right)^{\left\{0,1,\ldots\right\}}:s_{0}<s_{1}<\cdots\rightarrow\infty\right\},
\end{eqnarray*}
donde $\mathcal{L}$ son los subconjuntos de Borel de $L$, es decir, $\mathcal{L}=L\cap\mathcal{B}^{\left\{0,1,\ldots\right\}}$.

As\'i el par $\left(Z,S\right)$ es un mapeo medible de  $\left(\Omega,\mathcal{F}\right)$ en $\left(H\times L,\mathcal{H}\otimes\mathcal{L}\right)$. El par $\mathcal{H}\otimes\mathcal{L}^{+}$ denotar\'a la clase de todas las funciones medibles de $\left(H\times L,\mathcal{H}\otimes\mathcal{L}\right)$ en $\left(\left[0,\infty\right),\mathcal{B}\left[0,\infty\right)\right)$.
\end{Def}


\begin{Def}
Sea $\theta_{t}$ el mapeo-shift conjunto de $H\times L$ en $H\times L$ dado por
\begin{eqnarray*}
\theta_{t}\left(z,\left(s_{k}\right)_{0}^{\infty}\right)=\theta_{t}\left(z,\left(s_{n_{t-}+k}-t\right)_{0}^{\infty}\right)
\end{eqnarray*}
donde 
$n_{t-}=inf\left\{n\geq1:s_{n}\geq t\right\}$.
\end{Def}

\begin{Note}
Con la finalidad de poder realizar los shift's sin complicaciones de medibilidad, se supondr\'a que $Z$ es shit-medible, es decir, el conjunto de trayectorias $H$ es invariante bajo shifts del tiempo y el mapeo que toma $\left(z,t\right)\in H\times\left[0,\infty\right)$ en $z_{t}\in E$ es $\mathcal{H}\otimes\mathcal{B}\left[0,\infty\right)/\mathcal{E}$ medible.
\end{Note}

\begin{Def}
Dado un proceso \textbf{PEOSSM} (Proceso Estoc\'astico One Side Shift Medible) $Z$, se dice regenerativo cl\'asico con tiempos de regeneraci\'on $S$ si 

\begin{eqnarray*}
\theta_{S_{n}}\left(Z,S\right)=\left(Z^{0},S^{0}\right),n\geq0
\end{eqnarray*}
y adem\'as $\theta_{S_{n}}\left(Z,S\right)$ es independiente de $\left(\left(Z_{s}\right)s\in\left[0,S_{n}\right),S_{0},\ldots,S_{n}\right)$
Si lo anterior se cumple, al par $\left(Z,S\right)$ se le llama regenerativo cl\'asico.
\end{Def}

\begin{Note}
Si el par $\left(Z,S\right)$ es regenerativo cl\'asico, entonces las longitudes de los ciclos $X_{1},X_{2},\ldots,$ son i.i.d. e independientes de la longitud del retraso $S_{0}$, es decir, $S$ es un proceso de renovaci\'on. Las longitudes de los ciclos tambi\'en son llamados tiempos de inter-regeneraci\'on y tiempos de ocurrencia.

\end{Note}

\begin{Teo}
Sup\'ongase que el par $\left(Z,S\right)$ es regenerativo cl\'asico con $\esp\left[X_{1}\right]<\infty$. Entonces $\left(Z^{*},S^{*}\right)$ en el teorema 2.1 es una versi\'on estacionaria de $\left(Z,S\right)$. Adem\'as, si $X_{1}$ es lattice con span $d$, entonces $\left(Z^{**},S^{**}\right)$ en el teorema 2.2 es una versi\'on periodicamente estacionaria de $\left(Z,S\right)$ con periodo $d$.

\end{Teo}

\begin{Def}
Una variable aleatoria $X_{1}$ es \textit{spread out} si existe una $n\geq1$ y una  funci\'on $f\in\mathcal{B}^{+}$ tal que $\int_{\rea}f\left(x\right)dx>0$ con $X_{2},X_{3},\ldots,X_{n}$ copias i.i.d  de $X_{1}$, $$\prob\left(X_{1}+\cdots+X_{n}\in B\right)\geq\int_{B}f\left(x\right)dx$$ para $B\in\mathcal{B}$.

\end{Def}



\begin{Def}
Dado un proceso estoc\'astico $Z$ se le llama \textit{wide-sense regenerative} (\textbf{WSR}) con tiempos de regeneraci\'on $S$ si $\theta_{S_{n}}\left(Z,S\right)=\left(Z^{0},S^{0}\right)$ para $n\geq0$ en distribuci\'on y $\theta_{S_{n}}\left(Z,S\right)$ es independiente de $\left(S_{0},S_{1},\ldots,S_{n}\right)$ para $n\geq0$.
Se dice que el par $\left(Z,S\right)$ es WSR si lo anterior se cumple.
\end{Def}


\begin{Note}
\begin{itemize}
\item El proceso de trayectorias $\left(\theta_{s}Z\right)_{s\in\left[0,\infty\right)}$ es WSR con tiempos de regeneraci\'on $S$ pero no es regenerativo cl\'asico.

\item Si $Z$ es cualquier proceso estacionario y $S$ es un proceso de renovaci\'on que es independiente de $Z$, entonces $\left(Z,S\right)$ es WSR pero en general no es regenerativo cl\'asico

\end{itemize}

\end{Note}


\begin{Note}
Para cualquier proceso estoc\'astico $Z$, el proceso de trayectorias $\left(\theta_{s}Z\right)_{s\in\left[0,\infty\right)}$ es siempre un proceso de Markov.
\end{Note}



\begin{Teo}
Supongase que el par $\left(Z,S\right)$ es WSR con $\esp\left[X_{1}\right]<\infty$. Entonces $\left(Z^{*},S^{*}\right)$ en el teorema 2.1 es una versi\'on estacionaria de 
$\left(Z,S\right)$.
\end{Teo}


\begin{Teo}
Supongase que $\left(Z,S\right)$ es cycle-stationary con $\esp\left[X_{1}\right]<\infty$. Sea $U$ distribuida uniformemente en $\left[0,1\right)$ e independiente de $\left(Z^{0},S^{0}\right)$ y sea $\prob^{*}$ la medida de probabilidad en $\left(\Omega,\prob\right)$ definida por $$d\prob^{*}=\frac{X_{1}}{\esp\left[X_{1}\right]}d\prob$$. Sea $\left(Z^{*},S^{*}\right)$ con distribuci\'on $\prob^{*}\left(\theta_{UX_{1}}\left(Z^{0},S^{0}\right)\in\cdot\right)$. Entonces $\left(Z^{}*,S^{*}\right)$ es estacionario,
\begin{eqnarray*}
\esp\left[f\left(Z^{*},S^{*}\right)\right]=\esp\left[\int_{0}^{X_{1}}f\left(\theta_{s}\left(Z^{0},S^{0}\right)\right)ds\right]/\esp\left[X_{1}\right]
\end{eqnarray*}
$f\in\mathcal{H}\otimes\mathcal{L}^{+}$, and $S_{0}^{*}$ es continuo con funci\'on distribuci\'on $G_{\infty}$ definida por $$G_{\infty}\left(x\right):=\frac{\esp\left[X_{1}\right]\wedge x}{\esp\left[X_{1}\right]}$$ para $x\geq0$ y densidad $\prob\left[X_{1}>x\right]/\esp\left[X_{1}\right]$, con $x\geq0$.

\end{Teo}


\begin{Teo}
Sea $Z$ un Proceso Estoc\'astico un lado shift-medible \textit{one-sided shift-measurable stochastic process}, (PEOSSM),
y $S_{0}$ y $S_{1}$ tiempos aleatorios tales que $0\leq S_{0}<S_{1}$ y
\begin{equation}
\theta_{S_{1}}Z=\theta_{S_{0}}Z\textrm{ en distribuci\'on}.
\end{equation}

Entonces el espacio de probabilidad subyacente $\left(\Omega,\mathcal{F},\prob\right)$ puede extenderse para soportar una sucesi\'on de tiempos aleatorios $S$ tales que

\begin{eqnarray}
\theta_{S_{n}}\left(Z,S\right)=\left(Z^{0},S^{0}\right),n\geq0,\textrm{ en distribuci\'on},\\
\left(Z,S_{0},S_{1}\right)\textrm{ depende de }\left(X_{2},X_{3},\ldots\right)\textrm{ solamente a traves de }\theta_{S_{1}}Z.
\end{eqnarray}
\end{Teo}


\begin{Def}
Un elemento aleatorio en un espacio medible $\left(E,\mathcal{E}\right)$ en un espacio de probabilidad $\left(\Omega,\mathcal{F},\prob\right)$ a $\left(E,\mathcal{E}\right)$, es decir,
para $A\in \mathcal{E}$,  se tiene que $\left\{Y\in A\right\}\in\mathcal{F}$, donde $\left\{Y\in A\right\}:=\left\{w\in\Omega:Y\left(w\right)\in A\right\}=:Y^{-1}A$.
\end{Def}

\begin{Note}
Tambi\'en se dice que $Y$ est\'a soportado por el espacio de probabilidad $\left(\Omega,\mathcal{F},\prob\right)$ y que $Y$ es un mapeo medible de $\Omega$ en $E$, es decir, es $\mathcal{F}/\mathcal{E}$ medible.
\end{Note}

\begin{Def}
Para cada $i\in \mathbb{I}$ sea $P_{i}$ una medida de probabilidad en un espacio medible $\left(E_{i},\mathcal{E}_{i}\right)$. Se define el espacio producto
$\otimes_{i\in\mathbb{I}}\left(E_{i},\mathcal{E}_{i}\right):=\left(\prod_{i\in\mathbb{I}}E_{i},\otimes_{i\in\mathbb{I}}\mathcal{E}_{i}\right)$, donde $\prod_{i\in\mathbb{I}}E_{i}$ es el producto cartesiano de los $E_{i}$'s, y $\otimes_{i\in\mathbb{I}}\mathcal{E}_{i}$ es la $\sigma$-\'algebra producto, es decir, es la $\sigma$-\'algebra m\'as peque\~na en $\prod_{i\in\mathbb{I}}E_{i}$ que hace al $i$-\'esimo mapeo proyecci\'on en $E_{i}$ medible para toda $i\in\mathbb{I}$ es la $\sigma$-\'algebra inducida por los mapeos proyecci\'on. $$\otimes_{i\in\mathbb{I}}\mathcal{E}_{i}:=\sigma\left\{\left\{y:y_{i}\in A\right\}:i\in\mathbb{I}\textrm{ y }A\in\mathcal{E}_{i}\right\}.$$
\end{Def}

\begin{Def}
Un espacio de probabilidad $\left(\tilde{\Omega},\tilde{\mathcal{F}},\tilde{\prob}\right)$ es una extensi\'on de otro espacio de probabilidad $\left(\Omega,\mathcal{F},\prob\right)$ si $\left(\tilde{\Omega},\tilde{\mathcal{F}},\tilde{\prob}\right)$ soporta un elemento aleatorio $\xi\in\left(\Omega,\mathcal{F}\right)$ que tienen a $\prob$ como distribuci\'on.
\end{Def}

\begin{Teo}
Sea $\mathbb{I}$ un conjunto de \'indices arbitrario. Para cada $i\in\mathbb{I}$ sea $P_{i}$ una medida de probabilidad en un espacio medible $\left(E_{i},\mathcal{E}_{i}\right)$. Entonces existe una \'unica medida de probabilidad $\otimes_{i\in\mathbb{I}}P_{i}$ en $\otimes_{i\in\mathbb{I}}\left(E_{i},\mathcal{E}_{i}\right)$ tal que 

\begin{eqnarray*}
\otimes_{i\in\mathbb{I}}P_{i}\left(y\in\prod_{i\in\mathbb{I}}E_{i}:y_{i}\in A_{i_{1}},\ldots,y_{n}\in A_{i_{n}}\right)=P_{i_{1}}\left(A_{i_{n}}\right)\cdots P_{i_{n}}\left(A_{i_{n}}\right)
\end{eqnarray*}
para todos los enteros $n>0$, toda $i_{1},\ldots,i_{n}\in\mathbb{I}$ y todo $A_{i_{1}}\in\mathcal{E}_{i_{1}},\ldots,A_{i_{n}}\in\mathcal{E}_{i_{n}}$
\end{Teo}

La medida $\otimes_{i\in\mathbb{I}}P_{i}$ es llamada la medida producto y $\otimes_{i\in\mathbb{I}}\left(E_{i},\mathcal{E}_{i},P_{i}\right):=\left(\prod_{i\in\mathbb{I}},E_{i},\otimes_{i\in\mathbb{I}}\mathcal{E}_{i},\otimes_{i\in\mathbb{I}}P_{i}\right)$, es llamado espacio de probabilidad producto.


\begin{Def}
Un espacio medible $\left(E,\mathcal{E}\right)$ es \textit{Polaco} si existe una m\'etrica en $E$ tal que $E$ es completo, es decir cada sucesi\'on de Cauchy converge a un l\'imite en $E$, y \textit{separable}, $E$ tienen un subconjunto denso numerable, y tal que $\mathcal{E}$ es generado por conjuntos abiertos.
\end{Def}


\begin{Def}
Dos espacios medibles $\left(E,\mathcal{E}\right)$ y $\left(G,\mathcal{G}\right)$ son Borel equivalentes \textit{isomorfos} si existe una biyecci\'on $f:E\rightarrow G$ tal que $f$ es $\mathcal{E}/\mathcal{G}$ medible y su inversa $f^{-1}$ es $\mathcal{G}/\mathcal{E}$ medible. La biyecci\'on es una equivalencia de Borel.
\end{Def}

\begin{Def}
Un espacio medible  $\left(E,\mathcal{E}\right)$ es un \textit{espacio est\'andar} si es Borel equivalente a $\left(G,\mathcal{G}\right)$, donde $G$ es un subconjunto de Borel de $\left[0,1\right]$ y $\mathcal{G}$ son los subconjuntos de Borel de $G$.
\end{Def}

\begin{Note}
Cualquier espacio Polaco es un espacio est\'andar.
\end{Note}


\begin{Def}
Un proceso estoc\'astico con conjunto de \'indices $\mathbb{I}$ y espacio de estados $\left(E,\mathcal{E}\right)$ es una familia $Z=\left(\mathbb{Z}_{s}\right)_{s\in\mathbb{I}}$ donde $\mathbb{Z}_{s}$ son elementos aleatorios definidos en un espacio de probabilidad com\'un $\left(\Omega,\mathcal{F},\prob\right)$ y todos toman valores en $\left(E,\mathcal{E}\right)$.
\end{Def}

\begin{Def}
Un proceso estoc\'astico \textit{one-sided contiuous time} (\textbf{PEOSCT}) es un proceso estoc\'astico con conjunto de \'indices $\mathbb{I}=\left[0,\infty\right)$.
\end{Def}


Sea $\left(E^{\mathbb{I}},\mathcal{E}^{\mathbb{I}}\right)$ denota el espacio producto $\left(E^{\mathbb{I}},\mathcal{E}^{\mathbb{I}}\right):=\otimes_{s\in\mathbb{I}}\left(E,\mathcal{E}\right)$. Vamos a considerar $\mathbb{Z}$ como un mapeo aleatorio, es decir, como un elemento aleatorio en $\left(E^{\mathbb{I}},\mathcal{E}^{\mathbb{I}}\right)$ definido por $Z\left(w\right)=\left(Z_{s}\left(w\right)\right)_{s\in\mathbb{I}}$ y $w\in\Omega$.

\begin{Note}
La distribuci\'on de un proceso estoc\'astico $Z$ es la distribuci\'on de $Z$ como un elemento aleatorio en $\left(E^{\mathbb{I}},\mathcal{E}^{\mathbb{I}}\right)$. La distribuci\'on de $Z$ esta determinada de manera \'unica por las distribuciones finito dimensionales.
\end{Note}

\begin{Note}
En particular cuando $Z$ toma valores reales, es decir, $\left(E,\mathcal{E}\right)=\left(\mathbb{R},\mathcal{B}\right)$ las distribuciones finito dimensionales est\'an determinadas por las funciones de distribuci\'on finito dimensionales

\begin{eqnarray}
\prob\left(Z_{t_{1}}\leq x_{1},\ldots,Z_{t_{n}}\leq x_{n}\right),x_{1},\ldots,x_{n}\in\mathbb{R},t_{1},\ldots,t_{n}\in\mathbb{I},n\geq1.
\end{eqnarray}
\end{Note}

\begin{Note}
Para espacios polacos $\left(E,\mathcal{E}\right)$ el Teorema de Consistencia de Kolmogorov asegura que dada una colecci\'on de distribuciones finito dimensionales consistentes, siempre existe un proceso estoc\'astico que posee tales distribuciones finito dimensionales.
\end{Note}


\begin{Def}
Las trayectorias de $Z$ son las realizaciones $Z\left(w\right)$ para $w\in\Omega$ del mapeo aleatorio $Z$.
\end{Def}

\begin{Note}
Algunas restricciones se imponen sobre las trayectorias, por ejemplo que sean continuas por la derecha, o continuas por la derecha con l\'imites por la izquierda, o de manera m\'as general, se pedir\'a que caigan en alg\'un subconjunto $H$ de $E^{\mathbb{I}}$. En este caso es natural considerar a $Z$ como un elemento aleatorio que no est\'a en $\left(E^{\mathbb{I}},\mathcal{E}^{\mathbb{I}}\right)$ sino en $\left(H,\mathcal{H}\right)$, donde $\mathcal{H}$ es la $\sigma$-\'algebra generada por los mapeos proyecci\'on que toman a $z\in H$ a $z_{t}\in E$ para $t\in\mathbb{I}$. A $\mathcal{H}$ se le conoce como la traza de $H$ en $E^{\mathbb{I}}$, es decir,
\begin{eqnarray}
\mathcal{H}:=E^{\mathbb{I}}\cap H:=\left\{A\cap H:A\in E^{\mathbb{I}}\right\}.
\end{eqnarray}
\end{Note}


\begin{Note}
$Z$ tiene trayectorias con valores en $H$ y cada $Z_{t}$ es un mapeo medible de $\left(\Omega,\mathcal{F}\right)$ a $\left(H,\mathcal{H}\right)$. Cuando se considera un espacio de trayectorias en particular $H$, al espacio $\left(H,\mathcal{H}\right)$ se le llama el espacio de trayectorias de $Z$.
\end{Note}

\begin{Note}
La distribuci\'on del proceso estoc\'astico $Z$ con espacio de trayectorias $\left(H,\mathcal{H}\right)$ es la distribuci\'on de $Z$ como  un elemento aleatorio en $\left(H,\mathcal{H}\right)$. La distribuci\'on, nuevemente, est\'a determinada de manera \'unica por las distribuciones finito dimensionales.
\end{Note}


\begin{Def}
Sea $Z$ un PEOSCT  con espacio de estados $\left(E,\mathcal{E}\right)$ y sea $T$ un tiempo aleatorio en $\left[0,\infty\right)$. Por $Z_{T}$ se entiende el mapeo con valores en $E$ definido en $\Omega$ en la manera obvia:
\begin{eqnarray*}
Z_{T}\left(w\right):=Z_{T\left(w\right)}\left(w\right). w\in\Omega.
\end{eqnarray*}
\end{Def}

\begin{Def}
Un PEOSCT $Z$ es conjuntamente medible (\textbf{CM}) si el mapeo que toma $\left(w,t\right)\in\Omega\times\left[0,\infty\right)$ a $Z_{t}\left(w\right)\in E$ es $\mathcal{F}\otimes\mathcal{B}\left[0,\infty\right)/\mathcal{E}$ medible.
\end{Def}

\begin{Note}
Un PEOSCT-CM implica que el proceso es medible, dado que $Z_{T}$ es una composici\'on  de dos mapeos continuos: el primero que toma $w$ en $\left(w,T\left(w\right)\right)$ es $\mathcal{F}/\mathcal{F}\otimes\mathcal{B}\left[0,\infty\right)$ medible, mientras que el segundo toma $\left(w,T\left(w\right)\right)$ en $Z_{T\left(w\right)}\left(w\right)$ es $\mathcal{F}\otimes\mathcal{B}\left[0,\infty\right)/\mathcal{E}$ medible.
\end{Note}


\begin{Def}
Un PEOSCT con espacio de estados $\left(H,\mathcal{H}\right)$ es can\'onicamente conjuntamente medible (\textbf{CCM}) si el mapeo $\left(z,t\right)\in H\times\left[0,\infty\right)$ en $Z_{t}\in E$ es $\mathcal{H}\otimes\mathcal{B}\left[0,\infty\right)/\mathcal{E}$ medible.
\end{Def}

\begin{Note}
Un PEOSCTCCM implica que el proceso es CM, dado que un PECCM $Z$ es un mapeo de $\Omega\times\left[0,\infty\right)$ a $E$, es la composici\'on de dos mapeos medibles: el primero, toma $\left(w,t\right)$ en $\left(Z\left(w\right),t\right)$ es $\mathcal{F}\otimes\mathcal{B}\left[0,\infty\right)/\mathcal{H}\otimes\mathcal{B}\left[0,\infty\right)$ medible, y el segundo que toma $\left(Z\left(w\right),t\right)$  en $Z_{t}\left(w\right)$ es $\mathcal{H}\otimes\mathcal{B}\left[0,\infty\right)/\mathcal{E}$ medible. Por tanto CCM es una condici\'on m\'as fuerte que CM.
\end{Note}

\begin{Def}
Un conjunto de trayectorias $H$ de un PEOSCT $Z$, es internamente shift-invariante (\textbf{ISI}) si 
\begin{eqnarray*}
\left\{\left(z_{t+s}\right)_{s\in\left[0,\infty\right)}:z\in H\right\}=H\textrm{, }t\in\left[0,\infty\right).
\end{eqnarray*}
\end{Def}


\begin{Def}
Dado un PEOSCTISI, se define el mapeo-shift $\theta_{t}$, $t\in\left[0,\infty\right)$, de $H$ a $H$ por 
\begin{eqnarray*}
\theta_{t}z=\left(z_{t+s}\right)_{s\in\left[0,\infty\right)}\textrm{, }z\in H.
\end{eqnarray*}
\end{Def}

\begin{Def}
Se dice que un proceso $Z$ es shift-medible (\textbf{SM}) si $Z$ tiene un conjunto de trayectorias $H$ que es ISI y adem\'as el mapeo que toma $\left(z,t\right)\in H\times\left[0,\infty\right)$ en $\theta_{t}z\in H$ es $\mathcal{H}\otimes\mathcal{B}\left[0,\infty\right)/\mathcal{H}$ medible.
\end{Def}

\begin{Note}
Un proceso estoc\'astico con conjunto de trayectorias $H$ ISI es shift-medible si y s\'olo si es CCM
\end{Note}

\begin{Note}
\begin{itemize}
\item Dado el espacio polaco $\left(E,\mathcal{E}\right)$ se tiene el  conjunto de trayectorias $D_{E}\left[0,\infty\right)$ que es ISI, entonces cumpe con ser CCM.

\item Si $G$ es abierto, podemos cubrirlo por bolas abiertas cuay cerradura este contenida en $G$, y como $G$ es segundo numerable como subespacio de $E$, lo podemos cubrir por una cantidad numerable de bolas abiertas.

\end{itemize}
\end{Note}


\begin{Note}
Los procesos estoc\'asticos $Z$ a tiempo discreto con espacio de estados polaco, tambi\'en tiene un espacio de trayectorias polaco y por tanto tiene distribuciones condicionales regulares.
\end{Note}

\begin{Teo}
El producto numerable de espacios polacos es polaco.
\end{Teo}


\begin{Def}
Sea $\left(\Omega,\mathcal{F},\prob\right)$ espacio de probabilidad que soporta al proceso $Z=\left(Z_{s}\right)_{s\in\left[0,\infty\right)}$ y $S=\left(S_{k}\right)_{0}^{\infty}$ donde $Z$ es un PEOSCTM con espacio de estados $\left(E,\mathcal{E}\right)$  y espacio de trayectorias $\left(H,\mathcal{H}\right)$  y adem\'as $S$ es una sucesi\'on de tiempos aleatorios one-sided que satisfacen la condici\'on $0\leq S_{0}<S_{1}<\cdots\rightarrow\infty$. Considerando $S$ como un mapeo medible de $\left(\Omega,\mathcal{F}\right)$ al espacio sucesi\'on $\left(L,\mathcal{L}\right)$, donde 
\begin{eqnarray*}
L=\left\{\left(s_{k}\right)_{0}^{\infty}\in\left[0,\infty\right)^{\left\{0,1,\ldots\right\}}:s_{0}<s_{1}<\cdots\rightarrow\infty\right\},
\end{eqnarray*}
donde $\mathcal{L}$ son los subconjuntos de Borel de $L$, es decir, $\mathcal{L}=L\cap\mathcal{B}^{\left\{0,1,\ldots\right\}}$.

As\'i el par $\left(Z,S\right)$ es un mapeo medible de  $\left(\Omega,\mathcal{F}\right)$ en $\left(H\times L,\mathcal{H}\otimes\mathcal{L}\right)$. El par $\mathcal{H}\otimes\mathcal{L}^{+}$ denotar\'a la clase de todas las funciones medibles de $\left(H\times L,\mathcal{H}\otimes\mathcal{L}\right)$ en $\left(\left[0,\infty\right),\mathcal{B}\left[0,\infty\right)\right)$.
\end{Def}


\begin{Def}
Sea $\theta_{t}$ el mapeo-shift conjunto de $H\times L$ en $H\times L$ dado por
\begin{eqnarray*}
\theta_{t}\left(z,\left(s_{k}\right)_{0}^{\infty}\right)=\theta_{t}\left(z,\left(s_{n_{t-}+k}-t\right)_{0}^{\infty}\right)
\end{eqnarray*}
donde 
$n_{t-}=inf\left\{n\geq1:s_{n}\geq t\right\}$.
\end{Def}

\begin{Note}
Con la finalidad de poder realizar los shift's sin complicaciones de medibilidad, se supondr\'a que $Z$ es shit-medible, es decir, el conjunto de trayectorias $H$ es invariante bajo shifts del tiempo y el mapeo que toma $\left(z,t\right)\in H\times\left[0,\infty\right)$ en $z_{t}\in E$ es $\mathcal{H}\otimes\mathcal{B}\left[0,\infty\right)/\mathcal{E}$ medible.
\end{Note}

\begin{Def}
Dado un proceso \textbf{PEOSSM} (Proceso Estoc\'astico One Side Shift Medible) $Z$, se dice regenerativo cl\'asico con tiempos de regeneraci\'on $S$ si 

\begin{eqnarray*}
\theta_{S_{n}}\left(Z,S\right)=\left(Z^{0},S^{0}\right),n\geq0
\end{eqnarray*}
y adem\'as $\theta_{S_{n}}\left(Z,S\right)$ es independiente de $\left(\left(Z_{s}\right)s\in\left[0,S_{n}\right),S_{0},\ldots,S_{n}\right)$
Si lo anterior se cumple, al par $\left(Z,S\right)$ se le llama regenerativo cl\'asico.
\end{Def}

\begin{Note}
Si el par $\left(Z,S\right)$ es regenerativo cl\'asico, entonces las longitudes de los ciclos $X_{1},X_{2},\ldots,$ son i.i.d. e independientes de la longitud del retraso $S_{0}$, es decir, $S$ es un proceso de renovaci\'on. Las longitudes de los ciclos tambi\'en son llamados tiempos de inter-regeneraci\'on y tiempos de ocurrencia.

\end{Note}

\begin{Teo}
Sup\'ongase que el par $\left(Z,S\right)$ es regenerativo cl\'asico con $\esp\left[X_{1}\right]<\infty$. Entonces $\left(Z^{*},S^{*}\right)$ en el teorema 2.1 es una versi\'on estacionaria de $\left(Z,S\right)$. Adem\'as, si $X_{1}$ es lattice con span $d$, entonces $\left(Z^{**},S^{**}\right)$ en el teorema 2.2 es una versi\'on periodicamente estacionaria de $\left(Z,S\right)$ con periodo $d$.

\end{Teo}

\begin{Def}
Una variable aleatoria $X_{1}$ es \textit{spread out} si existe una $n\geq1$ y una  funci\'on $f\in\mathcal{B}^{+}$ tal que $\int_{\rea}f\left(x\right)dx>0$ con $X_{2},X_{3},\ldots,X_{n}$ copias i.i.d  de $X_{1}$, $$\prob\left(X_{1}+\cdots+X_{n}\in B\right)\geq\int_{B}f\left(x\right)dx$$ para $B\in\mathcal{B}$.

\end{Def}



\begin{Def}
Dado un proceso estoc\'astico $Z$ se le llama \textit{wide-sense regenerative} (\textbf{WSR}) con tiempos de regeneraci\'on $S$ si $\theta_{S_{n}}\left(Z,S\right)=\left(Z^{0},S^{0}\right)$ para $n\geq0$ en distribuci\'on y $\theta_{S_{n}}\left(Z,S\right)$ es independiente de $\left(S_{0},S_{1},\ldots,S_{n}\right)$ para $n\geq0$.
Se dice que el par $\left(Z,S\right)$ es WSR si lo anterior se cumple.
\end{Def}


\begin{Note}
\begin{itemize}
\item El proceso de trayectorias $\left(\theta_{s}Z\right)_{s\in\left[0,\infty\right)}$ es WSR con tiempos de regeneraci\'on $S$ pero no es regenerativo cl\'asico.

\item Si $Z$ es cualquier proceso estacionario y $S$ es un proceso de renovaci\'on que es independiente de $Z$, entonces $\left(Z,S\right)$ es WSR pero en general no es regenerativo cl\'asico

\end{itemize}

\end{Note}


\begin{Note}
Para cualquier proceso estoc\'astico $Z$, el proceso de trayectorias $\left(\theta_{s}Z\right)_{s\in\left[0,\infty\right)}$ es siempre un proceso de Markov.
\end{Note}



\begin{Teo}
Supongase que el par $\left(Z,S\right)$ es WSR con $\esp\left[X_{1}\right]<\infty$. Entonces $\left(Z^{*},S^{*}\right)$ en el teorema 2.1 es una versi\'on estacionaria de 
$\left(Z,S\right)$.
\end{Teo}


\begin{Teo}
Supongase que $\left(Z,S\right)$ es cycle-stationary con $\esp\left[X_{1}\right]<\infty$. Sea $U$ distribuida uniformemente en $\left[0,1\right)$ e independiente de $\left(Z^{0},S^{0}\right)$ y sea $\prob^{*}$ la medida de probabilidad en $\left(\Omega,\prob\right)$ definida por $$d\prob^{*}=\frac{X_{1}}{\esp\left[X_{1}\right]}d\prob$$. Sea $\left(Z^{*},S^{*}\right)$ con distribuci\'on $\prob^{*}\left(\theta_{UX_{1}}\left(Z^{0},S^{0}\right)\in\cdot\right)$. Entonces $\left(Z^{}*,S^{*}\right)$ es estacionario,
\begin{eqnarray*}
\esp\left[f\left(Z^{*},S^{*}\right)\right]=\esp\left[\int_{0}^{X_{1}}f\left(\theta_{s}\left(Z^{0},S^{0}\right)\right)ds\right]/\esp\left[X_{1}\right]
\end{eqnarray*}
$f\in\mathcal{H}\otimes\mathcal{L}^{+}$, and $S_{0}^{*}$ es continuo con funci\'on distribuci\'on $G_{\infty}$ definida por $$G_{\infty}\left(x\right):=\frac{\esp\left[X_{1}\right]\wedge x}{\esp\left[X_{1}\right]}$$ para $x\geq0$ y densidad $\prob\left[X_{1}>x\right]/\esp\left[X_{1}\right]$, con $x\geq0$.

\end{Teo}


\begin{Teo}
Sea $Z$ un Proceso Estoc\'astico un lado shift-medible \textit{one-sided shift-measurable stochastic process}, (PEOSSM),
y $S_{0}$ y $S_{1}$ tiempos aleatorios tales que $0\leq S_{0}<S_{1}$ y
\begin{equation}
\theta_{S_{1}}Z=\theta_{S_{0}}Z\textrm{ en distribuci\'on}.
\end{equation}

Entonces el espacio de probabilidad subyacente $\left(\Omega,\mathcal{F},\prob\right)$ puede extenderse para soportar una sucesi\'on de tiempos aleatorios $S$ tales que

\begin{eqnarray}
\theta_{S_{n}}\left(Z,S\right)=\left(Z^{0},S^{0}\right),n\geq0,\textrm{ en distribuci\'on},\\
\left(Z,S_{0},S_{1}\right)\textrm{ depende de }\left(X_{2},X_{3},\ldots\right)\textrm{ solamente a traves de }\theta_{S_{1}}Z.
\end{eqnarray}
\end{Teo}

\begin{Def}
Un elemento aleatorio en un espacio medible $\left(E,\mathcal{E}\right)$ en un espacio de probabilidad $\left(\Omega,\mathcal{F},\prob\right)$ a $\left(E,\mathcal{E}\right)$, es decir,
para $A\in \mathcal{E}$,  se tiene que $\left\{Y\in A\right\}\in\mathcal{F}$, donde $\left\{Y\in A\right\}:=\left\{w\in\Omega:Y\left(w\right)\in A\right\}=:Y^{-1}A$.
\end{Def}

\begin{Note}
Tambi\'en se dice que $Y$ est\'a soportado por el espacio de probabilidad $\left(\Omega,\mathcal{F},\prob\right)$ y que $Y$ es un mapeo medible de $\Omega$ en $E$, es decir, es $\mathcal{F}/\mathcal{E}$ medible.
\end{Note}

\begin{Def}
Para cada $i\in \mathbb{I}$ sea $P_{i}$ una medida de probabilidad en un espacio medible $\left(E_{i},\mathcal{E}_{i}\right)$. Se define el espacio producto
$\otimes_{i\in\mathbb{I}}\left(E_{i},\mathcal{E}_{i}\right):=\left(\prod_{i\in\mathbb{I}}E_{i},\otimes_{i\in\mathbb{I}}\mathcal{E}_{i}\right)$, donde $\prod_{i\in\mathbb{I}}E_{i}$ es el producto cartesiano de los $E_{i}$'s, y $\otimes_{i\in\mathbb{I}}\mathcal{E}_{i}$ es la $\sigma$-\'algebra producto, es decir, es la $\sigma$-\'algebra m\'as peque\~na en $\prod_{i\in\mathbb{I}}E_{i}$ que hace al $i$-\'esimo mapeo proyecci\'on en $E_{i}$ medible para toda $i\in\mathbb{I}$ es la $\sigma$-\'algebra inducida por los mapeos proyecci\'on. $$\otimes_{i\in\mathbb{I}}\mathcal{E}_{i}:=\sigma\left\{\left\{y:y_{i}\in A\right\}:i\in\mathbb{I}\textrm{ y }A\in\mathcal{E}_{i}\right\}.$$
\end{Def}

\begin{Def}
Un espacio de probabilidad $\left(\tilde{\Omega},\tilde{\mathcal{F}},\tilde{\prob}\right)$ es una extensi\'on de otro espacio de probabilidad $\left(\Omega,\mathcal{F},\prob\right)$ si $\left(\tilde{\Omega},\tilde{\mathcal{F}},\tilde{\prob}\right)$ soporta un elemento aleatorio $\xi\in\left(\Omega,\mathcal{F}\right)$ que tienen a $\prob$ como distribuci\'on.
\end{Def}

\begin{Teo}
Sea $\mathbb{I}$ un conjunto de \'indices arbitrario. Para cada $i\in\mathbb{I}$ sea $P_{i}$ una medida de probabilidad en un espacio medible $\left(E_{i},\mathcal{E}_{i}\right)$. Entonces existe una \'unica medida de probabilidad $\otimes_{i\in\mathbb{I}}P_{i}$ en $\otimes_{i\in\mathbb{I}}\left(E_{i},\mathcal{E}_{i}\right)$ tal que 

\begin{eqnarray*}
\otimes_{i\in\mathbb{I}}P_{i}\left(y\in\prod_{i\in\mathbb{I}}E_{i}:y_{i}\in A_{i_{1}},\ldots,y_{n}\in A_{i_{n}}\right)=P_{i_{1}}\left(A_{i_{n}}\right)\cdots P_{i_{n}}\left(A_{i_{n}}\right)
\end{eqnarray*}
para todos los enteros $n>0$, toda $i_{1},\ldots,i_{n}\in\mathbb{I}$ y todo $A_{i_{1}}\in\mathcal{E}_{i_{1}},\ldots,A_{i_{n}}\in\mathcal{E}_{i_{n}}$
\end{Teo}

La medida $\otimes_{i\in\mathbb{I}}P_{i}$ es llamada la medida producto y $\otimes_{i\in\mathbb{I}}\left(E_{i},\mathcal{E}_{i},P_{i}\right):=\left(\prod_{i\in\mathbb{I}},E_{i},\otimes_{i\in\mathbb{I}}\mathcal{E}_{i},\otimes_{i\in\mathbb{I}}P_{i}\right)$, es llamado espacio de probabilidad producto.


\begin{Def}
Un espacio medible $\left(E,\mathcal{E}\right)$ es \textit{Polaco} si existe una m\'etrica en $E$ tal que $E$ es completo, es decir cada sucesi\'on de Cauchy converge a un l\'imite en $E$, y \textit{separable}, $E$ tienen un subconjunto denso numerable, y tal que $\mathcal{E}$ es generado por conjuntos abiertos.
\end{Def}


\begin{Def}
Dos espacios medibles $\left(E,\mathcal{E}\right)$ y $\left(G,\mathcal{G}\right)$ son Borel equivalentes \textit{isomorfos} si existe una biyecci\'on $f:E\rightarrow G$ tal que $f$ es $\mathcal{E}/\mathcal{G}$ medible y su inversa $f^{-1}$ es $\mathcal{G}/\mathcal{E}$ medible. La biyecci\'on es una equivalencia de Borel.
\end{Def}

\begin{Def}
Un espacio medible  $\left(E,\mathcal{E}\right)$ es un \textit{espacio est\'andar} si es Borel equivalente a $\left(G,\mathcal{G}\right)$, donde $G$ es un subconjunto de Borel de $\left[0,1\right]$ y $\mathcal{G}$ son los subconjuntos de Borel de $G$.
\end{Def}

\begin{Note}
Cualquier espacio Polaco es un espacio est\'andar.
\end{Note}


\begin{Def}
Un proceso estoc\'astico con conjunto de \'indices $\mathbb{I}$ y espacio de estados $\left(E,\mathcal{E}\right)$ es una familia $Z=\left(\mathbb{Z}_{s}\right)_{s\in\mathbb{I}}$ donde $\mathbb{Z}_{s}$ son elementos aleatorios definidos en un espacio de probabilidad com\'un $\left(\Omega,\mathcal{F},\prob\right)$ y todos toman valores en $\left(E,\mathcal{E}\right)$.
\end{Def}

\begin{Def}
Un proceso estoc\'astico \textit{one-sided contiuous time} (\textbf{PEOSCT}) es un proceso estoc\'astico con conjunto de \'indices $\mathbb{I}=\left[0,\infty\right)$.
\end{Def}


Sea $\left(E^{\mathbb{I}},\mathcal{E}^{\mathbb{I}}\right)$ denota el espacio producto $\left(E^{\mathbb{I}},\mathcal{E}^{\mathbb{I}}\right):=\otimes_{s\in\mathbb{I}}\left(E,\mathcal{E}\right)$. Vamos a considerar $\mathbb{Z}$ como un mapeo aleatorio, es decir, como un elemento aleatorio en $\left(E^{\mathbb{I}},\mathcal{E}^{\mathbb{I}}\right)$ definido por $Z\left(w\right)=\left(Z_{s}\left(w\right)\right)_{s\in\mathbb{I}}$ y $w\in\Omega$.

\begin{Note}
La distribuci\'on de un proceso estoc\'astico $Z$ es la distribuci\'on de $Z$ como un elemento aleatorio en $\left(E^{\mathbb{I}},\mathcal{E}^{\mathbb{I}}\right)$. La distribuci\'on de $Z$ esta determinada de manera \'unica por las distribuciones finito dimensionales.
\end{Note}

\begin{Note}
En particular cuando $Z$ toma valores reales, es decir, $\left(E,\mathcal{E}\right)=\left(\mathbb{R},\mathcal{B}\right)$ las distribuciones finito dimensionales est\'an determinadas por las funciones de distribuci\'on finito dimensionales

\begin{eqnarray}
\prob\left(Z_{t_{1}}\leq x_{1},\ldots,Z_{t_{n}}\leq x_{n}\right),x_{1},\ldots,x_{n}\in\mathbb{R},t_{1},\ldots,t_{n}\in\mathbb{I},n\geq1.
\end{eqnarray}
\end{Note}

\begin{Note}
Para espacios polacos $\left(E,\mathcal{E}\right)$ el Teorema de Consistencia de Kolmogorov asegura que dada una colecci\'on de distribuciones finito dimensionales consistentes, siempre existe un proceso estoc\'astico que posee tales distribuciones finito dimensionales.
\end{Note}


\begin{Def}
Las trayectorias de $Z$ son las realizaciones $Z\left(w\right)$ para $w\in\Omega$ del mapeo aleatorio $Z$.
\end{Def}

\begin{Note}
Algunas restricciones se imponen sobre las trayectorias, por ejemplo que sean continuas por la derecha, o continuas por la derecha con l\'imites por la izquierda, o de manera m\'as general, se pedir\'a que caigan en alg\'un subconjunto $H$ de $E^{\mathbb{I}}$. En este caso es natural considerar a $Z$ como un elemento aleatorio que no est\'a en $\left(E^{\mathbb{I}},\mathcal{E}^{\mathbb{I}}\right)$ sino en $\left(H,\mathcal{H}\right)$, donde $\mathcal{H}$ es la $\sigma$-\'algebra generada por los mapeos proyecci\'on que toman a $z\in H$ a $z_{t}\in E$ para $t\in\mathbb{I}$. A $\mathcal{H}$ se le conoce como la traza de $H$ en $E^{\mathbb{I}}$, es decir,
\begin{eqnarray}
\mathcal{H}:=E^{\mathbb{I}}\cap H:=\left\{A\cap H:A\in E^{\mathbb{I}}\right\}.
\end{eqnarray}
\end{Note}


\begin{Note}
$Z$ tiene trayectorias con valores en $H$ y cada $Z_{t}$ es un mapeo medible de $\left(\Omega,\mathcal{F}\right)$ a $\left(H,\mathcal{H}\right)$. Cuando se considera un espacio de trayectorias en particular $H$, al espacio $\left(H,\mathcal{H}\right)$ se le llama el espacio de trayectorias de $Z$.
\end{Note}

\begin{Note}
La distribuci\'on del proceso estoc\'astico $Z$ con espacio de trayectorias $\left(H,\mathcal{H}\right)$ es la distribuci\'on de $Z$ como  un elemento aleatorio en $\left(H,\mathcal{H}\right)$. La distribuci\'on, nuevemente, est\'a determinada de manera \'unica por las distribuciones finito dimensionales.
\end{Note}


\begin{Def}
Sea $Z$ un PEOSCT  con espacio de estados $\left(E,\mathcal{E}\right)$ y sea $T$ un tiempo aleatorio en $\left[0,\infty\right)$. Por $Z_{T}$ se entiende el mapeo con valores en $E$ definido en $\Omega$ en la manera obvia:
\begin{eqnarray*}
Z_{T}\left(w\right):=Z_{T\left(w\right)}\left(w\right). w\in\Omega.
\end{eqnarray*}
\end{Def}

\begin{Def}
Un PEOSCT $Z$ es conjuntamente medible (\textbf{CM}) si el mapeo que toma $\left(w,t\right)\in\Omega\times\left[0,\infty\right)$ a $Z_{t}\left(w\right)\in E$ es $\mathcal{F}\otimes\mathcal{B}\left[0,\infty\right)/\mathcal{E}$ medible.
\end{Def}

\begin{Note}
Un PEOSCT-CM implica que el proceso es medible, dado que $Z_{T}$ es una composici\'on  de dos mapeos continuos: el primero que toma $w$ en $\left(w,T\left(w\right)\right)$ es $\mathcal{F}/\mathcal{F}\otimes\mathcal{B}\left[0,\infty\right)$ medible, mientras que el segundo toma $\left(w,T\left(w\right)\right)$ en $Z_{T\left(w\right)}\left(w\right)$ es $\mathcal{F}\otimes\mathcal{B}\left[0,\infty\right)/\mathcal{E}$ medible.
\end{Note}


\begin{Def}
Un PEOSCT con espacio de estados $\left(H,\mathcal{H}\right)$ es can\'onicamente conjuntamente medible (\textbf{CCM}) si el mapeo $\left(z,t\right)\in H\times\left[0,\infty\right)$ en $Z_{t}\in E$ es $\mathcal{H}\otimes\mathcal{B}\left[0,\infty\right)/\mathcal{E}$ medible.
\end{Def}

\begin{Note}
Un PEOSCTCCM implica que el proceso es CM, dado que un PECCM $Z$ es un mapeo de $\Omega\times\left[0,\infty\right)$ a $E$, es la composici\'on de dos mapeos medibles: el primero, toma $\left(w,t\right)$ en $\left(Z\left(w\right),t\right)$ es $\mathcal{F}\otimes\mathcal{B}\left[0,\infty\right)/\mathcal{H}\otimes\mathcal{B}\left[0,\infty\right)$ medible, y el segundo que toma $\left(Z\left(w\right),t\right)$  en $Z_{t}\left(w\right)$ es $\mathcal{H}\otimes\mathcal{B}\left[0,\infty\right)/\mathcal{E}$ medible. Por tanto CCM es una condici\'on m\'as fuerte que CM.
\end{Note}

\begin{Def}
Un conjunto de trayectorias $H$ de un PEOSCT $Z$, es internamente shift-invariante (\textbf{ISI}) si 
\begin{eqnarray*}
\left\{\left(z_{t+s}\right)_{s\in\left[0,\infty\right)}:z\in H\right\}=H\textrm{, }t\in\left[0,\infty\right).
\end{eqnarray*}
\end{Def}


\begin{Def}
Dado un PEOSCTISI, se define el mapeo-shift $\theta_{t}$, $t\in\left[0,\infty\right)$, de $H$ a $H$ por 
\begin{eqnarray*}
\theta_{t}z=\left(z_{t+s}\right)_{s\in\left[0,\infty\right)}\textrm{, }z\in H.
\end{eqnarray*}
\end{Def}

\begin{Def}
Se dice que un proceso $Z$ es shift-medible (\textbf{SM}) si $Z$ tiene un conjunto de trayectorias $H$ que es ISI y adem\'as el mapeo que toma $\left(z,t\right)\in H\times\left[0,\infty\right)$ en $\theta_{t}z\in H$ es $\mathcal{H}\otimes\mathcal{B}\left[0,\infty\right)/\mathcal{H}$ medible.
\end{Def}

\begin{Note}
Un proceso estoc\'astico con conjunto de trayectorias $H$ ISI es shift-medible si y s\'olo si es CCM
\end{Note}

\begin{Note}
\begin{itemize}
\item Dado el espacio polaco $\left(E,\mathcal{E}\right)$ se tiene el  conjunto de trayectorias $D_{E}\left[0,\infty\right)$ que es ISI, entonces cumpe con ser CCM.

\item Si $G$ es abierto, podemos cubrirlo por bolas abiertas cuay cerradura este contenida en $G$, y como $G$ es segundo numerable como subespacio de $E$, lo podemos cubrir por una cantidad numerable de bolas abiertas.

\end{itemize}
\end{Note}


\begin{Note}
Los procesos estoc\'asticos $Z$ a tiempo discreto con espacio de estados polaco, tambi\'en tiene un espacio de trayectorias polaco y por tanto tiene distribuciones condicionales regulares.
\end{Note}

\begin{Teo}
El producto numerable de espacios polacos es polaco.
\end{Teo}


\begin{Def}
Sea $\left(\Omega,\mathcal{F},\prob\right)$ espacio de probabilidad que soporta al proceso $Z=\left(Z_{s}\right)_{s\in\left[0,\infty\right)}$ y $S=\left(S_{k}\right)_{0}^{\infty}$ donde $Z$ es un PEOSCTM con espacio de estados $\left(E,\mathcal{E}\right)$  y espacio de trayectorias $\left(H,\mathcal{H}\right)$  y adem\'as $S$ es una sucesi\'on de tiempos aleatorios one-sided que satisfacen la condici\'on $0\leq S_{0}<S_{1}<\cdots\rightarrow\infty$. Considerando $S$ como un mapeo medible de $\left(\Omega,\mathcal{F}\right)$ al espacio sucesi\'on $\left(L,\mathcal{L}\right)$, donde 
\begin{eqnarray*}
L=\left\{\left(s_{k}\right)_{0}^{\infty}\in\left[0,\infty\right)^{\left\{0,1,\ldots\right\}}:s_{0}<s_{1}<\cdots\rightarrow\infty\right\},
\end{eqnarray*}
donde $\mathcal{L}$ son los subconjuntos de Borel de $L$, es decir, $\mathcal{L}=L\cap\mathcal{B}^{\left\{0,1,\ldots\right\}}$.

As\'i el par $\left(Z,S\right)$ es un mapeo medible de  $\left(\Omega,\mathcal{F}\right)$ en $\left(H\times L,\mathcal{H}\otimes\mathcal{L}\right)$. El par $\mathcal{H}\otimes\mathcal{L}^{+}$ denotar\'a la clase de todas las funciones medibles de $\left(H\times L,\mathcal{H}\otimes\mathcal{L}\right)$ en $\left(\left[0,\infty\right),\mathcal{B}\left[0,\infty\right)\right)$.
\end{Def}


\begin{Def}
Sea $\theta_{t}$ el mapeo-shift conjunto de $H\times L$ en $H\times L$ dado por
\begin{eqnarray*}
\theta_{t}\left(z,\left(s_{k}\right)_{0}^{\infty}\right)=\theta_{t}\left(z,\left(s_{n_{t-}+k}-t\right)_{0}^{\infty}\right)
\end{eqnarray*}
donde 
$n_{t-}=inf\left\{n\geq1:s_{n}\geq t\right\}$.
\end{Def}

\begin{Note}
Con la finalidad de poder realizar los shift's sin complicaciones de medibilidad, se supondr\'a que $Z$ es shit-medible, es decir, el conjunto de trayectorias $H$ es invariante bajo shifts del tiempo y el mapeo que toma $\left(z,t\right)\in H\times\left[0,\infty\right)$ en $z_{t}\in E$ es $\mathcal{H}\otimes\mathcal{B}\left[0,\infty\right)/\mathcal{E}$ medible.
\end{Note}

\begin{Def}
Dado un proceso \textbf{PEOSSM} (Proceso Estoc\'astico One Side Shift Medible) $Z$, se dice regenerativo cl\'asico con tiempos de regeneraci\'on $S$ si 

\begin{eqnarray*}
\theta_{S_{n}}\left(Z,S\right)=\left(Z^{0},S^{0}\right),n\geq0
\end{eqnarray*}
y adem\'as $\theta_{S_{n}}\left(Z,S\right)$ es independiente de $\left(\left(Z_{s}\right)s\in\left[0,S_{n}\right),S_{0},\ldots,S_{n}\right)$
Si lo anterior se cumple, al par $\left(Z,S\right)$ se le llama regenerativo cl\'asico.
\end{Def}

\begin{Note}
Si el par $\left(Z,S\right)$ es regenerativo cl\'asico, entonces las longitudes de los ciclos $X_{1},X_{2},\ldots,$ son i.i.d. e independientes de la longitud del retraso $S_{0}$, es decir, $S$ es un proceso de renovaci\'on. Las longitudes de los ciclos tambi\'en son llamados tiempos de inter-regeneraci\'on y tiempos de ocurrencia.

\end{Note}

\begin{Teo}
Sup\'ongase que el par $\left(Z,S\right)$ es regenerativo cl\'asico con $\esp\left[X_{1}\right]<\infty$. Entonces $\left(Z^{*},S^{*}\right)$ en el teorema 2.1 es una versi\'on estacionaria de $\left(Z,S\right)$. Adem\'as, si $X_{1}$ es lattice con span $d$, entonces $\left(Z^{**},S^{**}\right)$ en el teorema 2.2 es una versi\'on periodicamente estacionaria de $\left(Z,S\right)$ con periodo $d$.

\end{Teo}

\begin{Def}
Una variable aleatoria $X_{1}$ es \textit{spread out} si existe una $n\geq1$ y una  funci\'on $f\in\mathcal{B}^{+}$ tal que $\int_{\rea}f\left(x\right)dx>0$ con $X_{2},X_{3},\ldots,X_{n}$ copias i.i.d  de $X_{1}$, $$\prob\left(X_{1}+\cdots+X_{n}\in B\right)\geq\int_{B}f\left(x\right)dx$$ para $B\in\mathcal{B}$.

\end{Def}



\begin{Def}
Dado un proceso estoc\'astico $Z$ se le llama \textit{wide-sense regenerative} (\textbf{WSR}) con tiempos de regeneraci\'on $S$ si $\theta_{S_{n}}\left(Z,S\right)=\left(Z^{0},S^{0}\right)$ para $n\geq0$ en distribuci\'on y $\theta_{S_{n}}\left(Z,S\right)$ es independiente de $\left(S_{0},S_{1},\ldots,S_{n}\right)$ para $n\geq0$.
Se dice que el par $\left(Z,S\right)$ es WSR si lo anterior se cumple.
\end{Def}


\begin{Note}
\begin{itemize}
\item El proceso de trayectorias $\left(\theta_{s}Z\right)_{s\in\left[0,\infty\right)}$ es WSR con tiempos de regeneraci\'on $S$ pero no es regenerativo cl\'asico.

\item Si $Z$ es cualquier proceso estacionario y $S$ es un proceso de renovaci\'on que es independiente de $Z$, entonces $\left(Z,S\right)$ es WSR pero en general no es regenerativo cl\'asico

\end{itemize}

\end{Note}


\begin{Note}
Para cualquier proceso estoc\'astico $Z$, el proceso de trayectorias $\left(\theta_{s}Z\right)_{s\in\left[0,\infty\right)}$ es siempre un proceso de Markov.
\end{Note}



\begin{Teo}
Supongase que el par $\left(Z,S\right)$ es WSR con $\esp\left[X_{1}\right]<\infty$. Entonces $\left(Z^{*},S^{*}\right)$ en el teorema 2.1 es una versi\'on estacionaria de 
$\left(Z,S\right)$.
\end{Teo}


\begin{Teo}
Supongase que $\left(Z,S\right)$ es cycle-stationary con $\esp\left[X_{1}\right]<\infty$. Sea $U$ distribuida uniformemente en $\left[0,1\right)$ e independiente de $\left(Z^{0},S^{0}\right)$ y sea $\prob^{*}$ la medida de probabilidad en $\left(\Omega,\prob\right)$ definida por $$d\prob^{*}=\frac{X_{1}}{\esp\left[X_{1}\right]}d\prob$$. Sea $\left(Z^{*},S^{*}\right)$ con distribuci\'on $\prob^{*}\left(\theta_{UX_{1}}\left(Z^{0},S^{0}\right)\in\cdot\right)$. Entonces $\left(Z^{}*,S^{*}\right)$ es estacionario,
\begin{eqnarray*}
\esp\left[f\left(Z^{*},S^{*}\right)\right]=\esp\left[\int_{0}^{X_{1}}f\left(\theta_{s}\left(Z^{0},S^{0}\right)\right)ds\right]/\esp\left[X_{1}\right]
\end{eqnarray*}
$f\in\mathcal{H}\otimes\mathcal{L}^{+}$, and $S_{0}^{*}$ es continuo con funci\'on distribuci\'on $G_{\infty}$ definida por $$G_{\infty}\left(x\right):=\frac{\esp\left[X_{1}\right]\wedge x}{\esp\left[X_{1}\right]}$$ para $x\geq0$ y densidad $\prob\left[X_{1}>x\right]/\esp\left[X_{1}\right]$, con $x\geq0$.

\end{Teo}


\begin{Teo}
Sea $Z$ un Proceso Estoc\'astico un lado shift-medible \textit{one-sided shift-measurable stochastic process}, (PEOSSM),
y $S_{0}$ y $S_{1}$ tiempos aleatorios tales que $0\leq S_{0}<S_{1}$ y
\begin{equation}
\theta_{S_{1}}Z=\theta_{S_{0}}Z\textrm{ en distribuci\'on}.
\end{equation}

Entonces el espacio de probabilidad subyacente $\left(\Omega,\mathcal{F},\prob\right)$ puede extenderse para soportar una sucesi\'on de tiempos aleatorios $S$ tales que

\begin{eqnarray}
\theta_{S_{n}}\left(Z,S\right)=\left(Z^{0},S^{0}\right),n\geq0,\textrm{ en distribuci\'on},\\
\left(Z,S_{0},S_{1}\right)\textrm{ depende de }\left(X_{2},X_{3},\ldots\right)\textrm{ solamente a traves de }\theta_{S_{1}}Z.
\end{eqnarray}
\end{Teo}




\begin{Def}
Un elemento aleatorio en un espacio medible $\left(E,\mathcal{E}\right)$ en un espacio de probabilidad $\left(\Omega,\mathcal{F},\prob\right)$ a $\left(E,\mathcal{E}\right)$, es decir,
para $A\in \mathcal{E}$,  se tiene que $\left\{Y\in A\right\}\in\mathcal{F}$, donde $\left\{Y\in A\right\}:=\left\{w\in\Omega:Y\left(w\right)\in A\right\}=:Y^{-1}A$.
\end{Def}

\begin{Note}
Tambi\'en se dice que $Y$ est\'a soportado por el espacio de probabilidad $\left(\Omega,\mathcal{F},\prob\right)$ y que $Y$ es un mapeo medible de $\Omega$ en $E$, es decir, es $\mathcal{F}/\mathcal{E}$ medible.
\end{Note}

\begin{Def}
Para cada $i\in \mathbb{I}$ sea $P_{i}$ una medida de probabilidad en un espacio medible $\left(E_{i},\mathcal{E}_{i}\right)$. Se define el espacio producto
$\otimes_{i\in\mathbb{I}}\left(E_{i},\mathcal{E}_{i}\right):=\left(\prod_{i\in\mathbb{I}}E_{i},\otimes_{i\in\mathbb{I}}\mathcal{E}_{i}\right)$, donde $\prod_{i\in\mathbb{I}}E_{i}$ es el producto cartesiano de los $E_{i}$'s, y $\otimes_{i\in\mathbb{I}}\mathcal{E}_{i}$ es la $\sigma$-\'algebra producto, es decir, es la $\sigma$-\'algebra m\'as peque\~na en $\prod_{i\in\mathbb{I}}E_{i}$ que hace al $i$-\'esimo mapeo proyecci\'on en $E_{i}$ medible para toda $i\in\mathbb{I}$ es la $\sigma$-\'algebra inducida por los mapeos proyecci\'on. $$\otimes_{i\in\mathbb{I}}\mathcal{E}_{i}:=\sigma\left\{\left\{y:y_{i}\in A\right\}:i\in\mathbb{I}\textrm{ y }A\in\mathcal{E}_{i}\right\}.$$
\end{Def}

\begin{Def}
Un espacio de probabilidad $\left(\tilde{\Omega},\tilde{\mathcal{F}},\tilde{\prob}\right)$ es una extensi\'on de otro espacio de probabilidad $\left(\Omega,\mathcal{F},\prob\right)$ si $\left(\tilde{\Omega},\tilde{\mathcal{F}},\tilde{\prob}\right)$ soporta un elemento aleatorio $\xi\in\left(\Omega,\mathcal{F}\right)$ que tienen a $\prob$ como distribuci\'on.
\end{Def}

\begin{Teo}
Sea $\mathbb{I}$ un conjunto de \'indices arbitrario. Para cada $i\in\mathbb{I}$ sea $P_{i}$ una medida de probabilidad en un espacio medible $\left(E_{i},\mathcal{E}_{i}\right)$. Entonces existe una \'unica medida de probabilidad $\otimes_{i\in\mathbb{I}}P_{i}$ en $\otimes_{i\in\mathbb{I}}\left(E_{i},\mathcal{E}_{i}\right)$ tal que 

\begin{eqnarray*}
\otimes_{i\in\mathbb{I}}P_{i}\left(y\in\prod_{i\in\mathbb{I}}E_{i}:y_{i}\in A_{i_{1}},\ldots,y_{n}\in A_{i_{n}}\right)=P_{i_{1}}\left(A_{i_{n}}\right)\cdots P_{i_{n}}\left(A_{i_{n}}\right)
\end{eqnarray*}
para todos los enteros $n>0$, toda $i_{1},\ldots,i_{n}\in\mathbb{I}$ y todo $A_{i_{1}}\in\mathcal{E}_{i_{1}},\ldots,A_{i_{n}}\in\mathcal{E}_{i_{n}}$
\end{Teo}

La medida $\otimes_{i\in\mathbb{I}}P_{i}$ es llamada la medida producto y $\otimes_{i\in\mathbb{I}}\left(E_{i},\mathcal{E}_{i},P_{i}\right):=\left(\prod_{i\in\mathbb{I}},E_{i},\otimes_{i\in\mathbb{I}}\mathcal{E}_{i},\otimes_{i\in\mathbb{I}}P_{i}\right)$, es llamado espacio de probabilidad producto.


\begin{Def}
Un espacio medible $\left(E,\mathcal{E}\right)$ es \textit{Polaco} si existe una m\'etrica en $E$ tal que $E$ es completo, es decir cada sucesi\'on de Cauchy converge a un l\'imite en $E$, y \textit{separable}, $E$ tienen un subconjunto denso numerable, y tal que $\mathcal{E}$ es generado por conjuntos abiertos.
\end{Def}


\begin{Def}
Dos espacios medibles $\left(E,\mathcal{E}\right)$ y $\left(G,\mathcal{G}\right)$ son Borel equivalentes \textit{isomorfos} si existe una biyecci\'on $f:E\rightarrow G$ tal que $f$ es $\mathcal{E}/\mathcal{G}$ medible y su inversa $f^{-1}$ es $\mathcal{G}/\mathcal{E}$ medible. La biyecci\'on es una equivalencia de Borel.
\end{Def}

\begin{Def}
Un espacio medible  $\left(E,\mathcal{E}\right)$ es un \textit{espacio est\'andar} si es Borel equivalente a $\left(G,\mathcal{G}\right)$, donde $G$ es un subconjunto de Borel de $\left[0,1\right]$ y $\mathcal{G}$ son los subconjuntos de Borel de $G$.
\end{Def}

\begin{Note}
Cualquier espacio Polaco es un espacio est\'andar.
\end{Note}


\begin{Def}
Un proceso estoc\'astico con conjunto de \'indices $\mathbb{I}$ y espacio de estados $\left(E,\mathcal{E}\right)$ es una familia $Z=\left(\mathbb{Z}_{s}\right)_{s\in\mathbb{I}}$ donde $\mathbb{Z}_{s}$ son elementos aleatorios definidos en un espacio de probabilidad com\'un $\left(\Omega,\mathcal{F},\prob\right)$ y todos toman valores en $\left(E,\mathcal{E}\right)$.
\end{Def}

\begin{Def}
Un proceso estoc\'astico \textit{one-sided contiuous time} (\textbf{PEOSCT}) es un proceso estoc\'astico con conjunto de \'indices $\mathbb{I}=\left[0,\infty\right)$.
\end{Def}


Sea $\left(E^{\mathbb{I}},\mathcal{E}^{\mathbb{I}}\right)$ denota el espacio producto $\left(E^{\mathbb{I}},\mathcal{E}^{\mathbb{I}}\right):=\otimes_{s\in\mathbb{I}}\left(E,\mathcal{E}\right)$. Vamos a considerar $\mathbb{Z}$ como un mapeo aleatorio, es decir, como un elemento aleatorio en $\left(E^{\mathbb{I}},\mathcal{E}^{\mathbb{I}}\right)$ definido por $Z\left(w\right)=\left(Z_{s}\left(w\right)\right)_{s\in\mathbb{I}}$ y $w\in\Omega$.

\begin{Note}
La distribuci\'on de un proceso estoc\'astico $Z$ es la distribuci\'on de $Z$ como un elemento aleatorio en $\left(E^{\mathbb{I}},\mathcal{E}^{\mathbb{I}}\right)$. La distribuci\'on de $Z$ esta determinada de manera \'unica por las distribuciones finito dimensionales.
\end{Note}

\begin{Note}
En particular cuando $Z$ toma valores reales, es decir, $\left(E,\mathcal{E}\right)=\left(\mathbb{R},\mathcal{B}\right)$ las distribuciones finito dimensionales est\'an determinadas por las funciones de distribuci\'on finito dimensionales

\begin{eqnarray}
\prob\left(Z_{t_{1}}\leq x_{1},\ldots,Z_{t_{n}}\leq x_{n}\right),x_{1},\ldots,x_{n}\in\mathbb{R},t_{1},\ldots,t_{n}\in\mathbb{I},n\geq1.
\end{eqnarray}
\end{Note}

\begin{Note}
Para espacios polacos $\left(E,\mathcal{E}\right)$ el Teorema de Consistencia de Kolmogorov asegura que dada una colecci\'on de distribuciones finito dimensionales consistentes, siempre existe un proceso estoc\'astico que posee tales distribuciones finito dimensionales.
\end{Note}


\begin{Def}
Las trayectorias de $Z$ son las realizaciones $Z\left(w\right)$ para $w\in\Omega$ del mapeo aleatorio $Z$.
\end{Def}

\begin{Note}
Algunas restricciones se imponen sobre las trayectorias, por ejemplo que sean continuas por la derecha, o continuas por la derecha con l\'imites por la izquierda, o de manera m\'as general, se pedir\'a que caigan en alg\'un subconjunto $H$ de $E^{\mathbb{I}}$. En este caso es natural considerar a $Z$ como un elemento aleatorio que no est\'a en $\left(E^{\mathbb{I}},\mathcal{E}^{\mathbb{I}}\right)$ sino en $\left(H,\mathcal{H}\right)$, donde $\mathcal{H}$ es la $\sigma$-\'algebra generada por los mapeos proyecci\'on que toman a $z\in H$ a $z_{t}\in E$ para $t\in\mathbb{I}$. A $\mathcal{H}$ se le conoce como la traza de $H$ en $E^{\mathbb{I}}$, es decir,
\begin{eqnarray}
\mathcal{H}:=E^{\mathbb{I}}\cap H:=\left\{A\cap H:A\in E^{\mathbb{I}}\right\}.
\end{eqnarray}
\end{Note}


\begin{Note}
$Z$ tiene trayectorias con valores en $H$ y cada $Z_{t}$ es un mapeo medible de $\left(\Omega,\mathcal{F}\right)$ a $\left(H,\mathcal{H}\right)$. Cuando se considera un espacio de trayectorias en particular $H$, al espacio $\left(H,\mathcal{H}\right)$ se le llama el espacio de trayectorias de $Z$.
\end{Note}

\begin{Note}
La distribuci\'on del proceso estoc\'astico $Z$ con espacio de trayectorias $\left(H,\mathcal{H}\right)$ es la distribuci\'on de $Z$ como  un elemento aleatorio en $\left(H,\mathcal{H}\right)$. La distribuci\'on, nuevemente, est\'a determinada de manera \'unica por las distribuciones finito dimensionales.
\end{Note}


\begin{Def}
Sea $Z$ un PEOSCT  con espacio de estados $\left(E,\mathcal{E}\right)$ y sea $T$ un tiempo aleatorio en $\left[0,\infty\right)$. Por $Z_{T}$ se entiende el mapeo con valores en $E$ definido en $\Omega$ en la manera obvia:
\begin{eqnarray*}
Z_{T}\left(w\right):=Z_{T\left(w\right)}\left(w\right). w\in\Omega.
\end{eqnarray*}
\end{Def}

\begin{Def}
Un PEOSCT $Z$ es conjuntamente medible (\textbf{CM}) si el mapeo que toma $\left(w,t\right)\in\Omega\times\left[0,\infty\right)$ a $Z_{t}\left(w\right)\in E$ es $\mathcal{F}\otimes\mathcal{B}\left[0,\infty\right)/\mathcal{E}$ medible.
\end{Def}

\begin{Note}
Un PEOSCT-CM implica que el proceso es medible, dado que $Z_{T}$ es una composici\'on  de dos mapeos continuos: el primero que toma $w$ en $\left(w,T\left(w\right)\right)$ es $\mathcal{F}/\mathcal{F}\otimes\mathcal{B}\left[0,\infty\right)$ medible, mientras que el segundo toma $\left(w,T\left(w\right)\right)$ en $Z_{T\left(w\right)}\left(w\right)$ es $\mathcal{F}\otimes\mathcal{B}\left[0,\infty\right)/\mathcal{E}$ medible.
\end{Note}


\begin{Def}
Un PEOSCT con espacio de estados $\left(H,\mathcal{H}\right)$ es can\'onicamente conjuntamente medible (\textbf{CCM}) si el mapeo $\left(z,t\right)\in H\times\left[0,\infty\right)$ en $Z_{t}\in E$ es $\mathcal{H}\otimes\mathcal{B}\left[0,\infty\right)/\mathcal{E}$ medible.
\end{Def}

\begin{Note}
Un PEOSCTCCM implica que el proceso es CM, dado que un PECCM $Z$ es un mapeo de $\Omega\times\left[0,\infty\right)$ a $E$, es la composici\'on de dos mapeos medibles: el primero, toma $\left(w,t\right)$ en $\left(Z\left(w\right),t\right)$ es $\mathcal{F}\otimes\mathcal{B}\left[0,\infty\right)/\mathcal{H}\otimes\mathcal{B}\left[0,\infty\right)$ medible, y el segundo que toma $\left(Z\left(w\right),t\right)$  en $Z_{t}\left(w\right)$ es $\mathcal{H}\otimes\mathcal{B}\left[0,\infty\right)/\mathcal{E}$ medible. Por tanto CCM es una condici\'on m\'as fuerte que CM.
\end{Note}

\begin{Def}
Un conjunto de trayectorias $H$ de un PEOSCT $Z$, es internamente shift-invariante (\textbf{ISI}) si 
\begin{eqnarray*}
\left\{\left(z_{t+s}\right)_{s\in\left[0,\infty\right)}:z\in H\right\}=H\textrm{, }t\in\left[0,\infty\right).
\end{eqnarray*}
\end{Def}


\begin{Def}
Dado un PEOSCTISI, se define el mapeo-shift $\theta_{t}$, $t\in\left[0,\infty\right)$, de $H$ a $H$ por 
\begin{eqnarray*}
\theta_{t}z=\left(z_{t+s}\right)_{s\in\left[0,\infty\right)}\textrm{, }z\in H.
\end{eqnarray*}
\end{Def}

\begin{Def}
Se dice que un proceso $Z$ es shift-medible (\textbf{SM}) si $Z$ tiene un conjunto de trayectorias $H$ que es ISI y adem\'as el mapeo que toma $\left(z,t\right)\in H\times\left[0,\infty\right)$ en $\theta_{t}z\in H$ es $\mathcal{H}\otimes\mathcal{B}\left[0,\infty\right)/\mathcal{H}$ medible.
\end{Def}

\begin{Note}
Un proceso estoc\'astico con conjunto de trayectorias $H$ ISI es shift-medible si y s\'olo si es CCM
\end{Note}

\begin{Note}
\begin{itemize}
\item Dado el espacio polaco $\left(E,\mathcal{E}\right)$ se tiene el  conjunto de trayectorias $D_{E}\left[0,\infty\right)$ que es ISI, entonces cumpe con ser CCM.

\item Si $G$ es abierto, podemos cubrirlo por bolas abiertas cuay cerradura este contenida en $G$, y como $G$ es segundo numerable como subespacio de $E$, lo podemos cubrir por una cantidad numerable de bolas abiertas.

\end{itemize}
\end{Note}


\begin{Note}
Los procesos estoc\'asticos $Z$ a tiempo discreto con espacio de estados polaco, tambi\'en tiene un espacio de trayectorias polaco y por tanto tiene distribuciones condicionales regulares.
\end{Note}

\begin{Teo}
El producto numerable de espacios polacos es polaco.
\end{Teo}


\begin{Def}
Sea $\left(\Omega,\mathcal{F},\prob\right)$ espacio de probabilidad que soporta al proceso $Z=\left(Z_{s}\right)_{s\in\left[0,\infty\right)}$ y $S=\left(S_{k}\right)_{0}^{\infty}$ donde $Z$ es un PEOSCTM con espacio de estados $\left(E,\mathcal{E}\right)$  y espacio de trayectorias $\left(H,\mathcal{H}\right)$  y adem\'as $S$ es una sucesi\'on de tiempos aleatorios one-sided que satisfacen la condici\'on $0\leq S_{0}<S_{1}<\cdots\rightarrow\infty$. Considerando $S$ como un mapeo medible de $\left(\Omega,\mathcal{F}\right)$ al espacio sucesi\'on $\left(L,\mathcal{L}\right)$, donde 
\begin{eqnarray*}
L=\left\{\left(s_{k}\right)_{0}^{\infty}\in\left[0,\infty\right)^{\left\{0,1,\ldots\right\}}:s_{0}<s_{1}<\cdots\rightarrow\infty\right\},
\end{eqnarray*}
donde $\mathcal{L}$ son los subconjuntos de Borel de $L$, es decir, $\mathcal{L}=L\cap\mathcal{B}^{\left\{0,1,\ldots\right\}}$.

As\'i el par $\left(Z,S\right)$ es un mapeo medible de  $\left(\Omega,\mathcal{F}\right)$ en $\left(H\times L,\mathcal{H}\otimes\mathcal{L}\right)$. El par $\mathcal{H}\otimes\mathcal{L}^{+}$ denotar\'a la clase de todas las funciones medibles de $\left(H\times L,\mathcal{H}\otimes\mathcal{L}\right)$ en $\left(\left[0,\infty\right),\mathcal{B}\left[0,\infty\right)\right)$.
\end{Def}


\begin{Def}
Sea $\theta_{t}$ el mapeo-shift conjunto de $H\times L$ en $H\times L$ dado por
\begin{eqnarray*}
\theta_{t}\left(z,\left(s_{k}\right)_{0}^{\infty}\right)=\theta_{t}\left(z,\left(s_{n_{t-}+k}-t\right)_{0}^{\infty}\right)
\end{eqnarray*}
donde 
$n_{t-}=inf\left\{n\geq1:s_{n}\geq t\right\}$.
\end{Def}

\begin{Note}
Con la finalidad de poder realizar los shift's sin complicaciones de medibilidad, se supondr\'a que $Z$ es shit-medible, es decir, el conjunto de trayectorias $H$ es invariante bajo shifts del tiempo y el mapeo que toma $\left(z,t\right)\in H\times\left[0,\infty\right)$ en $z_{t}\in E$ es $\mathcal{H}\otimes\mathcal{B}\left[0,\infty\right)/\mathcal{E}$ medible.
\end{Note}

\begin{Def}
Dado un proceso \textbf{PEOSSM} (Proceso Estoc\'astico One Side Shift Medible) $Z$, se dice regenerativo cl\'asico con tiempos de regeneraci\'on $S$ si 

\begin{eqnarray*}
\theta_{S_{n}}\left(Z,S\right)=\left(Z^{0},S^{0}\right),n\geq0
\end{eqnarray*}
y adem\'as $\theta_{S_{n}}\left(Z,S\right)$ es independiente de $\left(\left(Z_{s}\right)s\in\left[0,S_{n}\right),S_{0},\ldots,S_{n}\right)$
Si lo anterior se cumple, al par $\left(Z,S\right)$ se le llama regenerativo cl\'asico.
\end{Def}

\begin{Note}
Si el par $\left(Z,S\right)$ es regenerativo cl\'asico, entonces las longitudes de los ciclos $X_{1},X_{2},\ldots,$ son i.i.d. e independientes de la longitud del retraso $S_{0}$, es decir, $S$ es un proceso de renovaci\'on. Las longitudes de los ciclos tambi\'en son llamados tiempos de inter-regeneraci\'on y tiempos de ocurrencia.

\end{Note}

\begin{Teo}
Sup\'ongase que el par $\left(Z,S\right)$ es regenerativo cl\'asico con $\esp\left[X_{1}\right]<\infty$. Entonces $\left(Z^{*},S^{*}\right)$ en el teorema 2.1 es una versi\'on estacionaria de $\left(Z,S\right)$. Adem\'as, si $X_{1}$ es lattice con span $d$, entonces $\left(Z^{**},S^{**}\right)$ en el teorema 2.2 es una versi\'on periodicamente estacionaria de $\left(Z,S\right)$ con periodo $d$.

\end{Teo}

\begin{Def}
Una variable aleatoria $X_{1}$ es \textit{spread out} si existe una $n\geq1$ y una  funci\'on $f\in\mathcal{B}^{+}$ tal que $\int_{\rea}f\left(x\right)dx>0$ con $X_{2},X_{3},\ldots,X_{n}$ copias i.i.d  de $X_{1}$, $$\prob\left(X_{1}+\cdots+X_{n}\in B\right)\geq\int_{B}f\left(x\right)dx$$ para $B\in\mathcal{B}$.

\end{Def}



\begin{Def}
Dado un proceso estoc\'astico $Z$ se le llama \textit{wide-sense regenerative} (\textbf{WSR}) con tiempos de regeneraci\'on $S$ si $\theta_{S_{n}}\left(Z,S\right)=\left(Z^{0},S^{0}\right)$ para $n\geq0$ en distribuci\'on y $\theta_{S_{n}}\left(Z,S\right)$ es independiente de $\left(S_{0},S_{1},\ldots,S_{n}\right)$ para $n\geq0$.
Se dice que el par $\left(Z,S\right)$ es WSR si lo anterior se cumple.
\end{Def}


\begin{Note}
\begin{itemize}
\item El proceso de trayectorias $\left(\theta_{s}Z\right)_{s\in\left[0,\infty\right)}$ es WSR con tiempos de regeneraci\'on $S$ pero no es regenerativo cl\'asico.

\item Si $Z$ es cualquier proceso estacionario y $S$ es un proceso de renovaci\'on que es independiente de $Z$, entonces $\left(Z,S\right)$ es WSR pero en general no es regenerativo cl\'asico

\end{itemize}

\end{Note}


\begin{Note}
Para cualquier proceso estoc\'astico $Z$, el proceso de trayectorias $\left(\theta_{s}Z\right)_{s\in\left[0,\infty\right)}$ es siempre un proceso de Markov.
\end{Note}



\begin{Teo}
Supongase que el par $\left(Z,S\right)$ es WSR con $\esp\left[X_{1}\right]<\infty$. Entonces $\left(Z^{*},S^{*}\right)$ en el teorema 2.1 es una versi\'on estacionaria de 
$\left(Z,S\right)$.
\end{Teo}


\begin{Teo}
Supongase que $\left(Z,S\right)$ es cycle-stationary con $\esp\left[X_{1}\right]<\infty$. Sea $U$ distribuida uniformemente en $\left[0,1\right)$ e independiente de $\left(Z^{0},S^{0}\right)$ y sea $\prob^{*}$ la medida de probabilidad en $\left(\Omega,\prob\right)$ definida por $$d\prob^{*}=\frac{X_{1}}{\esp\left[X_{1}\right]}d\prob$$. Sea $\left(Z^{*},S^{*}\right)$ con distribuci\'on $\prob^{*}\left(\theta_{UX_{1}}\left(Z^{0},S^{0}\right)\in\cdot\right)$. Entonces $\left(Z^{}*,S^{*}\right)$ es estacionario,
\begin{eqnarray*}
\esp\left[f\left(Z^{*},S^{*}\right)\right]=\esp\left[\int_{0}^{X_{1}}f\left(\theta_{s}\left(Z^{0},S^{0}\right)\right)ds\right]/\esp\left[X_{1}\right]
\end{eqnarray*}
$f\in\mathcal{H}\otimes\mathcal{L}^{+}$, and $S_{0}^{*}$ es continuo con funci\'on distribuci\'on $G_{\infty}$ definida por $$G_{\infty}\left(x\right):=\frac{\esp\left[X_{1}\right]\wedge x}{\esp\left[X_{1}\right]}$$ para $x\geq0$ y densidad $\prob\left[X_{1}>x\right]/\esp\left[X_{1}\right]$, con $x\geq0$.

\end{Teo}


\begin{Teo}
Sea $Z$ un Proceso Estoc\'astico un lado shift-medible \textit{one-sided shift-measurable stochastic process}, (PEOSSM),
y $S_{0}$ y $S_{1}$ tiempos aleatorios tales que $0\leq S_{0}<S_{1}$ y
\begin{equation}
\theta_{S_{1}}Z=\theta_{S_{0}}Z\textrm{ en distribuci\'on}.
\end{equation}

Entonces el espacio de probabilidad subyacente $\left(\Omega,\mathcal{F},\prob\right)$ puede extenderse para soportar una sucesi\'on de tiempos aleatorios $S$ tales que

\begin{eqnarray}
\theta_{S_{n}}\left(Z,S\right)=\left(Z^{0},S^{0}\right),n\geq0,\textrm{ en distribuci\'on},\\
\left(Z,S_{0},S_{1}\right)\textrm{ depende de }\left(X_{2},X_{3},\ldots\right)\textrm{ solamente a traves de }\theta_{S_{1}}Z.
\end{eqnarray}
\end{Teo}
%______________________________________________________________________


\section{Procesos Regenerativos}
%________________________________________________________________________

%_________________________________________________________________________
%
\section{Appendix C: Output Process and Regenerative Processes}
%_________________________________________________________________________
%
En Sigman, Thorison y Wolff \cite{Sigman2} prueban que para la existencia de un una sucesi\'on infinita no decreciente de tiempos de regeneraci\'on $\tau_{1}\leq\tau_{2}\leq\cdots$ en los cuales el proceso se regenera, basta un tiempo de regeneraci\'on $R_{1}$, donde $R_{j}=\tau_{j}-\tau_{j-1}$. Para tal efecto se requiere la existencia de un espacio de probabilidad $\left(\Omega,\mathcal{F},\prob\right)$, y proceso estoc\'astico $\textit{X}=\left\{X\left(t\right):t\geq0\right\}$ con espacio de estados $\left(S,\mathcal{R}\right)$, con $\mathcal{R}$ $\sigma$-\'algebra.

\begin{Prop}
Si existe una variable aleatoria no negativa $R_{1}$ tal que $\theta_{R1}X=_{D}X$, entonces $\left(\Omega,\mathcal{F},\prob\right)$ puede extenderse para soportar una sucesi\'on estacionaria de variables aleatorias $R=\left\{R_{k}:k\geq1\right\}$, tal que para $k\geq1$,
\begin{eqnarray*}
\theta_{k}\left(X,R\right)=_{D}\left(X,R\right).
\end{eqnarray*}

Adem\'as, para $k\geq1$, $\theta_{k}R$ es condicionalmente independiente de $\left(X,R_{1},\ldots,R_{k}\right)$, dado $\theta_{\tau k}X$.

\end{Prop}


\begin{itemize}
\item Doob en 1953 demostr\'o que el estado estacionario de un proceso de partida en un sistema de espera $M/G/\infty$, es Poisson con la misma tasa que el proceso de arribos.

\item Burke en 1968, fue el primero en demostrar que el estado estacionario de un proceso de salida de una cola $M/M/s$ es un proceso Poisson.

\item Disney en 1973 obtuvo el siguiente resultado:

\begin{Teo}
Para el sistema de espera $M/G/1/L$ con disciplina FIFO, el proceso $\textbf{I}$ es un proceso de renovaci\'on si y s\'olo si el proceso denominado longitud de la cola es estacionario y se cumple cualquiera de los siguientes casos:

\begin{itemize}
\item[a)] Los tiempos de servicio son identicamente cero;
\item[b)] $L=0$, para cualquier proceso de servicio $S$;
\item[c)] $L=1$ y $G=D$;
\item[d)] $L=\infty$ y $G=M$.
\end{itemize}
En estos casos, respectivamente, las distribuciones de interpartida $P\left\{T_{n+1}-T_{n}\leq t\right\}$ son


\begin{itemize}
\item[a)] $1-e^{-\lambda t}$, $t\geq0$;
\item[b)] $1-e^{-\lambda t}*F\left(t\right)$, $t\geq0$;
\item[c)] $1-e^{-\lambda t}*\indora_{d}\left(t\right)$, $t\geq0$;
\item[d)] $1-e^{-\lambda t}*F\left(t\right)$, $t\geq0$.
\end{itemize}
\end{Teo}


\item Finch (1959) mostr\'o que para los sistemas $M/G/1/L$, con $1\leq L\leq \infty$ con distribuciones de servicio dos veces diferenciable, solamente el sistema $M/M/1/\infty$ tiene proceso de salida de renovaci\'on estacionario.

\item King (1971) demostro que un sistema de colas estacionario $M/G/1/1$ tiene sus tiempos de interpartida sucesivas $D_{n}$ y $D_{n+1}$ son independientes, si y s\'olo si, $G=D$, en cuyo caso le proceso de salida es de renovaci\'on.

\item Disney (1973) demostr\'o que el \'unico sistema estacionario $M/G/1/L$, que tiene proceso de salida de renovaci\'on  son los sistemas $M/M/1$ y $M/D/1/1$.



\item El siguiente resultado es de Disney y Koning (1985)
\begin{Teo}
En un sistema de espera $M/G/s$, el estado estacionario del proceso de salida es un proceso Poisson para cualquier distribuci\'on de los tiempos de servicio si el sistema tiene cualquiera de las siguientes cuatro propiedades.

\begin{itemize}
\item[a)] $s=\infty$
\item[b)] La disciplina de servicio es de procesador compartido.
\item[c)] La disciplina de servicio es LCFS y preemptive resume, esto se cumple para $L<\infty$
\item[d)] $G=M$.
\end{itemize}

\end{Teo}

\item El siguiente resultado es de Alamatsaz (1983)

\begin{Teo}
En cualquier sistema de colas $GI/G/1/L$ con $1\leq L<\infty$ y distribuci\'on de interarribos $A$ y distribuci\'on de los tiempos de servicio $B$, tal que $A\left(0\right)=0$, $A\left(t\right)\left(1-B\left(t\right)\right)>0$ para alguna $t>0$ y $B\left(t\right)$ para toda $t>0$, es imposible que el proceso de salida estacionario sea de renovaci\'on.
\end{Teo}

\end{itemize}



%________________________________________________________________________
%\subsection{Procesos Regenerativos Sigman, Thorisson y Wolff \cite{Sigman1}}
%________________________________________________________________________


\begin{Def}[Definici\'on Cl\'asica]
Un proceso estoc\'astico $X=\left\{X\left(t\right):t\geq0\right\}$ es llamado regenerativo is existe una variable aleatoria $R_{1}>0$ tal que
\begin{itemize}
\item[i)] $\left\{X\left(t+R_{1}\right):t\geq0\right\}$ es independiente de $\left\{\left\{X\left(t\right):t<R_{1}\right\},\right\}$
\item[ii)] $\left\{X\left(t+R_{1}\right):t\geq0\right\}$ es estoc\'asticamente equivalente a $\left\{X\left(t\right):t>0\right\}$
\end{itemize}

Llamamos a $R_{1}$ tiempo de regeneraci\'on, y decimos que $X$ se regenera en este punto.
\end{Def}

$\left\{X\left(t+R_{1}\right)\right\}$ es regenerativo con tiempo de regeneraci\'on $R_{2}$, independiente de $R_{1}$ pero con la misma distribuci\'on que $R_{1}$. Procediendo de esta manera se obtiene una secuencia de variables aleatorias independientes e id\'enticamente distribuidas $\left\{R_{n}\right\}$ llamados longitudes de ciclo. Si definimos a $Z_{k}\equiv R_{1}+R_{2}+\cdots+R_{k}$, se tiene un proceso de renovaci\'on llamado proceso de renovaci\'on encajado para $X$.


\begin{Note}
La existencia de un primer tiempo de regeneraci\'on, $R_{1}$, implica la existencia de una sucesi\'on completa de estos tiempos $R_{1},R_{2}\ldots,$ que satisfacen la propiedad deseada \cite{Sigman2}.
\end{Note}


\begin{Note} Para la cola $GI/GI/1$ los usuarios arriban con tiempos $t_{n}$ y son atendidos con tiempos de servicio $S_{n}$, los tiempos de arribo forman un proceso de renovaci\'on  con tiempos entre arribos independientes e identicamente distribuidos (\texttt{i.i.d.})$T_{n}=t_{n}-t_{n-1}$, adem\'as los tiempos de servicio son \texttt{i.i.d.} e independientes de los procesos de arribo. Por \textit{estable} se entiende que $\esp S_{n}<\esp T_{n}<\infty$.
\end{Note}
 


\begin{Def}
Para $x$ fijo y para cada $t\geq0$, sea $I_{x}\left(t\right)=1$ si $X\left(t\right)\leq x$,  $I_{x}\left(t\right)=0$ en caso contrario, y def\'inanse los tiempos promedio
\begin{eqnarray*}
\overline{X}&=&lim_{t\rightarrow\infty}\frac{1}{t}\int_{0}^{\infty}X\left(u\right)du\\
\prob\left(X_{\infty}\leq x\right)&=&lim_{t\rightarrow\infty}\frac{1}{t}\int_{0}^{\infty}I_{x}\left(u\right)du,
\end{eqnarray*}
cuando estos l\'imites existan.
\end{Def}

Como consecuencia del teorema de Renovaci\'on-Recompensa, se tiene que el primer l\'imite  existe y es igual a la constante
\begin{eqnarray*}
\overline{X}&=&\frac{\esp\left[\int_{0}^{R_{1}}X\left(t\right)dt\right]}{\esp\left[R_{1}\right]},
\end{eqnarray*}
suponiendo que ambas esperanzas son finitas.
 
\begin{Note}
Funciones de procesos regenerativos son regenerativas, es decir, si $X\left(t\right)$ es regenerativo y se define el proceso $Y\left(t\right)$ por $Y\left(t\right)=f\left(X\left(t\right)\right)$ para alguna funci\'on Borel medible $f\left(\cdot\right)$. Adem\'as $Y$ es regenerativo con los mismos tiempos de renovaci\'on que $X$. 

En general, los tiempos de renovaci\'on, $Z_{k}$ de un proceso regenerativo no requieren ser tiempos de paro con respecto a la evoluci\'on de $X\left(t\right)$.
\end{Note} 

\begin{Note}
Una funci\'on de un proceso de Markov, usualmente no ser\'a un proceso de Markov, sin embargo ser\'a regenerativo si el proceso de Markov lo es.
\end{Note}

 
\begin{Note}
Un proceso regenerativo con media de la longitud de ciclo finita es llamado positivo recurrente.
\end{Note}


\begin{Note}
\begin{itemize}
\item[a)] Si el proceso regenerativo $X$ es positivo recurrente y tiene trayectorias muestrales no negativas, entonces la ecuaci\'on anterior es v\'alida.
\item[b)] Si $X$ es positivo recurrente regenerativo, podemos construir una \'unica versi\'on estacionaria de este proceso, $X_{e}=\left\{X_{e}\left(t\right)\right\}$, donde $X_{e}$ es un proceso estoc\'astico regenerativo y estrictamente estacionario, con distribuci\'on marginal distribuida como $X_{\infty}$
\end{itemize}
\end{Note}


%__________________________________________________________________________________________
%\subsection{Procesos Regenerativos Estacionarios - Stidham \cite{Stidham}}
%__________________________________________________________________________________________


Un proceso estoc\'astico a tiempo continuo $\left\{V\left(t\right),t\geq0\right\}$ es un proceso regenerativo si existe una sucesi\'on de variables aleatorias independientes e id\'enticamente distribuidas $\left\{X_{1},X_{2},\ldots\right\}$, sucesi\'on de renovaci\'on, tal que para cualquier conjunto de Borel $A$, 

\begin{eqnarray*}
\prob\left\{V\left(t\right)\in A|X_{1}+X_{2}+\cdots+X_{R\left(t\right)}=s,\left\{V\left(\tau\right),\tau<s\right\}\right\}=\prob\left\{V\left(t-s\right)\in A|X_{1}>t-s\right\},
\end{eqnarray*}
para todo $0\leq s\leq t$, donde $R\left(t\right)=\max\left\{X_{1}+X_{2}+\cdots+X_{j}\leq t\right\}=$n\'umero de renovaciones ({\emph{puntos de regeneraci\'on}}) que ocurren en $\left[0,t\right]$. El intervalo $\left[0,X_{1}\right)$ es llamado {\emph{primer ciclo de regeneraci\'on}} de $\left\{V\left(t \right),t\geq0\right\}$, $\left[X_{1},X_{1}+X_{2}\right)$ el {\emph{segundo ciclo de regeneraci\'on}}, y as\'i sucesivamente.

Sea $X=X_{1}$ y sea $F$ la funci\'on de distrbuci\'on de $X$


\begin{Def}
Se define el proceso estacionario, $\left\{V^{*}\left(t\right),t\geq0\right\}$, para $\left\{V\left(t\right),t\geq0\right\}$ por

\begin{eqnarray*}
\prob\left\{V\left(t\right)\in A\right\}=\frac{1}{\esp\left[X\right]}\int_{0}^{\infty}\prob\left\{V\left(t+x\right)\in A|X>x\right\}\left(1-F\left(x\right)\right)dx,
\end{eqnarray*} 
para todo $t\geq0$ y todo conjunto de Borel $A$.
\end{Def}

\begin{Def}
Una distribuci\'on se dice que es {\emph{aritm\'etica}} si todos sus puntos de incremento son m\'ultiplos de la forma $0,\lambda, 2\lambda,\ldots$ para alguna $\lambda>0$ entera.
\end{Def}


\begin{Def}
Una modificaci\'on medible de un proceso $\left\{V\left(t\right),t\geq0\right\}$, es una versi\'on de este, $\left\{V\left(t,w\right)\right\}$ conjuntamente medible para $t\geq0$ y para $w\in S$, $S$ espacio de estados para $\left\{V\left(t\right),t\geq0\right\}$.
\end{Def}

\begin{Teo}
Sea $\left\{V\left(t\right),t\geq\right\}$ un proceso regenerativo no negativo con modificaci\'on medible. Sea $\esp\left[X\right]<\infty$. Entonces el proceso estacionario dado por la ecuaci\'on anterior est\'a bien definido y tiene funci\'on de distribuci\'on independiente de $t$, adem\'as
\begin{itemize}
\item[i)] \begin{eqnarray*}
\esp\left[V^{*}\left(0\right)\right]&=&\frac{\esp\left[\int_{0}^{X}V\left(s\right)ds\right]}{\esp\left[X\right]}\end{eqnarray*}
\item[ii)] Si $\esp\left[V^{*}\left(0\right)\right]<\infty$, equivalentemente, si $\esp\left[\int_{0}^{X}V\left(s\right)ds\right]<\infty$,entonces
\begin{eqnarray*}
\frac{\int_{0}^{t}V\left(s\right)ds}{t}\rightarrow\frac{\esp\left[\int_{0}^{X}V\left(s\right)ds\right]}{\esp\left[X\right]}
\end{eqnarray*}
con probabilidad 1 y en media, cuando $t\rightarrow\infty$.
\end{itemize}
\end{Teo}

\begin{Coro}
Sea $\left\{V\left(t\right),t\geq0\right\}$ un proceso regenerativo no negativo, con modificaci\'on medible. Si $\esp <\infty$, $F$ es no-aritm\'etica, y para todo $x\geq0$, $P\left\{V\left(t\right)\leq x,C>x\right\}$ es de variaci\'on acotada como funci\'on de $t$ en cada intervalo finito $\left[0,\tau\right]$, entonces $V\left(t\right)$ converge en distribuci\'on  cuando $t\rightarrow\infty$ y $$\esp V=\frac{\esp \int_{0}^{X}V\left(s\right)ds}{\esp X}$$
Donde $V$ tiene la distribuci\'on l\'imite de $V\left(t\right)$ cuando $t\rightarrow\infty$.

\end{Coro}

Para el caso discreto se tienen resultados similares.



%______________________________________________________________________
%\subsection{Procesos de Renovaci\'on}
%______________________________________________________________________

\begin{Def}%\label{Def.Tn}
Sean $0\leq T_{1}\leq T_{2}\leq \ldots$ son tiempos aleatorios infinitos en los cuales ocurren ciertos eventos. El n\'umero de tiempos $T_{n}$ en el intervalo $\left[0,t\right)$ es

\begin{eqnarray}
N\left(t\right)=\sum_{n=1}^{\infty}\indora\left(T_{n}\leq t\right),
\end{eqnarray}
para $t\geq0$.
\end{Def}

Si se consideran los puntos $T_{n}$ como elementos de $\rea_{+}$, y $N\left(t\right)$ es el n\'umero de puntos en $\rea$. El proceso denotado por $\left\{N\left(t\right):t\geq0\right\}$, denotado por $N\left(t\right)$, es un proceso puntual en $\rea_{+}$. Los $T_{n}$ son los tiempos de ocurrencia, el proceso puntual $N\left(t\right)$ es simple si su n\'umero de ocurrencias son distintas: $0<T_{1}<T_{2}<\ldots$ casi seguramente.

\begin{Def}
Un proceso puntual $N\left(t\right)$ es un proceso de renovaci\'on si los tiempos de interocurrencia $\xi_{n}=T_{n}-T_{n-1}$, para $n\geq1$, son independientes e identicamente distribuidos con distribuci\'on $F$, donde $F\left(0\right)=0$ y $T_{0}=0$. Los $T_{n}$ son llamados tiempos de renovaci\'on, referente a la independencia o renovaci\'on de la informaci\'on estoc\'astica en estos tiempos. Los $\xi_{n}$ son los tiempos de inter-renovaci\'on, y $N\left(t\right)$ es el n\'umero de renovaciones en el intervalo $\left[0,t\right)$
\end{Def}


\begin{Note}
Para definir un proceso de renovaci\'on para cualquier contexto, solamente hay que especificar una distribuci\'on $F$, con $F\left(0\right)=0$, para los tiempos de inter-renovaci\'on. La funci\'on $F$ en turno degune las otra variables aleatorias. De manera formal, existe un espacio de probabilidad y una sucesi\'on de variables aleatorias $\xi_{1},\xi_{2},\ldots$ definidas en este con distribuci\'on $F$. Entonces las otras cantidades son $T_{n}=\sum_{k=1}^{n}\xi_{k}$ y $N\left(t\right)=\sum_{n=1}^{\infty}\indora\left(T_{n}\leq t\right)$, donde $T_{n}\rightarrow\infty$ casi seguramente por la Ley Fuerte de los Grandes Números.
\end{Note}

%___________________________________________________________________________________________
%
%\subsection{Teorema Principal de Renovaci\'on}
%___________________________________________________________________________________________
%

\begin{Note} Una funci\'on $h:\rea_{+}\rightarrow\rea$ es Directamente Riemann Integrable en los siguientes casos:
\begin{itemize}
\item[a)] $h\left(t\right)\geq0$ es decreciente y Riemann Integrable.
\item[b)] $h$ es continua excepto posiblemente en un conjunto de Lebesgue de medida 0, y $|h\left(t\right)|\leq b\left(t\right)$, donde $b$ es DRI.
\end{itemize}
\end{Note}

\begin{Teo}[Teorema Principal de Renovaci\'on]
Si $F$ es no aritm\'etica y $h\left(t\right)$ es Directamente Riemann Integrable (DRI), entonces

\begin{eqnarray*}
lim_{t\rightarrow\infty}U\star h=\frac{1}{\mu}\int_{\rea_{+}}h\left(s\right)ds.
\end{eqnarray*}
\end{Teo}

\begin{Prop}
Cualquier funci\'on $H\left(t\right)$ acotada en intervalos finitos y que es 0 para $t<0$ puede expresarse como
\begin{eqnarray*}
H\left(t\right)=U\star h\left(t\right)\textrm{,  donde }h\left(t\right)=H\left(t\right)-F\star H\left(t\right)
\end{eqnarray*}
\end{Prop}

\begin{Def}
Un proceso estoc\'astico $X\left(t\right)$ es crudamente regenerativo en un tiempo aleatorio positivo $T$ si
\begin{eqnarray*}
\esp\left[X\left(T+t\right)|T\right]=\esp\left[X\left(t\right)\right]\textrm{, para }t\geq0,\end{eqnarray*}
y con las esperanzas anteriores finitas.
\end{Def}

\begin{Prop}
Sup\'ongase que $X\left(t\right)$ es un proceso crudamente regenerativo en $T$, que tiene distribuci\'on $F$. Si $\esp\left[X\left(t\right)\right]$ es acotado en intervalos finitos, entonces
\begin{eqnarray*}
\esp\left[X\left(t\right)\right]=U\star h\left(t\right)\textrm{,  donde }h\left(t\right)=\esp\left[X\left(t\right)\indora\left(T>t\right)\right].
\end{eqnarray*}
\end{Prop}

\begin{Teo}[Regeneraci\'on Cruda]
Sup\'ongase que $X\left(t\right)$ es un proceso con valores positivo crudamente regenerativo en $T$, y def\'inase $M=\sup\left\{|X\left(t\right)|:t\leq T\right\}$. Si $T$ es no aritm\'etico y $M$ y $MT$ tienen media finita, entonces
\begin{eqnarray*}
lim_{t\rightarrow\infty}\esp\left[X\left(t\right)\right]=\frac{1}{\mu}\int_{\rea_{+}}h\left(s\right)ds,
\end{eqnarray*}
donde $h\left(t\right)=\esp\left[X\left(t\right)\indora\left(T>t\right)\right]$.
\end{Teo}

%___________________________________________________________________________________________
%
%\subsection{Propiedades de los Procesos de Renovaci\'on}
%___________________________________________________________________________________________
%

Los tiempos $T_{n}$ est\'an relacionados con los conteos de $N\left(t\right)$ por

\begin{eqnarray*}
\left\{N\left(t\right)\geq n\right\}&=&\left\{T_{n}\leq t\right\}\\
T_{N\left(t\right)}\leq &t&<T_{N\left(t\right)+1},
\end{eqnarray*}

adem\'as $N\left(T_{n}\right)=n$, y 

\begin{eqnarray*}
N\left(t\right)=\max\left\{n:T_{n}\leq t\right\}=\min\left\{n:T_{n+1}>t\right\}
\end{eqnarray*}

Por propiedades de la convoluci\'on se sabe que

\begin{eqnarray*}
P\left\{T_{n}\leq t\right\}=F^{n\star}\left(t\right)
\end{eqnarray*}
que es la $n$-\'esima convoluci\'on de $F$. Entonces 

\begin{eqnarray*}
\left\{N\left(t\right)\geq n\right\}&=&\left\{T_{n}\leq t\right\}\\
P\left\{N\left(t\right)\leq n\right\}&=&1-F^{\left(n+1\right)\star}\left(t\right)
\end{eqnarray*}

Adem\'as usando el hecho de que $\esp\left[N\left(t\right)\right]=\sum_{n=1}^{\infty}P\left\{N\left(t\right)\geq n\right\}$
se tiene que

\begin{eqnarray*}
\esp\left[N\left(t\right)\right]=\sum_{n=1}^{\infty}F^{n\star}\left(t\right)
\end{eqnarray*}

\begin{Prop}
Para cada $t\geq0$, la funci\'on generadora de momentos $\esp\left[e^{\alpha N\left(t\right)}\right]$ existe para alguna $\alpha$ en una vecindad del 0, y de aqu\'i que $\esp\left[N\left(t\right)^{m}\right]<\infty$, para $m\geq1$.
\end{Prop}


\begin{Note}
Si el primer tiempo de renovaci\'on $\xi_{1}$ no tiene la misma distribuci\'on que el resto de las $\xi_{n}$, para $n\geq2$, a $N\left(t\right)$ se le llama Proceso de Renovaci\'on retardado, donde si $\xi$ tiene distribuci\'on $G$, entonces el tiempo $T_{n}$ de la $n$-\'esima renovaci\'on tiene distribuci\'on $G\star F^{\left(n-1\right)\star}\left(t\right)$
\end{Note}


\begin{Teo}
Para una constante $\mu\leq\infty$ ( o variable aleatoria), las siguientes expresiones son equivalentes:

\begin{eqnarray}
lim_{n\rightarrow\infty}n^{-1}T_{n}&=&\mu,\textrm{ c.s.}\\
lim_{t\rightarrow\infty}t^{-1}N\left(t\right)&=&1/\mu,\textrm{ c.s.}
\end{eqnarray}
\end{Teo}


Es decir, $T_{n}$ satisface la Ley Fuerte de los Grandes N\'umeros s\'i y s\'olo s\'i $N\left/t\right)$ la cumple.


\begin{Coro}[Ley Fuerte de los Grandes N\'umeros para Procesos de Renovaci\'on]
Si $N\left(t\right)$ es un proceso de renovaci\'on cuyos tiempos de inter-renovaci\'on tienen media $\mu\leq\infty$, entonces
\begin{eqnarray}
t^{-1}N\left(t\right)\rightarrow 1/\mu,\textrm{ c.s. cuando }t\rightarrow\infty.
\end{eqnarray}

\end{Coro}


Considerar el proceso estoc\'astico de valores reales $\left\{Z\left(t\right):t\geq0\right\}$ en el mismo espacio de probabilidad que $N\left(t\right)$

\begin{Def}
Para el proceso $\left\{Z\left(t\right):t\geq0\right\}$ se define la fluctuaci\'on m\'axima de $Z\left(t\right)$ en el intervalo $\left(T_{n-1},T_{n}\right]$:
\begin{eqnarray*}
M_{n}=\sup_{T_{n-1}<t\leq T_{n}}|Z\left(t\right)-Z\left(T_{n-1}\right)|
\end{eqnarray*}
\end{Def}

\begin{Teo}
Sup\'ongase que $n^{-1}T_{n}\rightarrow\mu$ c.s. cuando $n\rightarrow\infty$, donde $\mu\leq\infty$ es una constante o variable aleatoria. Sea $a$ una constante o variable aleatoria que puede ser infinita cuando $\mu$ es finita, y considere las expresiones l\'imite:
\begin{eqnarray}
lim_{n\rightarrow\infty}n^{-1}Z\left(T_{n}\right)&=&a,\textrm{ c.s.}\\
lim_{t\rightarrow\infty}t^{-1}Z\left(t\right)&=&a/\mu,\textrm{ c.s.}
\end{eqnarray}
La segunda expresi\'on implica la primera. Conversamente, la primera implica la segunda si el proceso $Z\left(t\right)$ es creciente, o si $lim_{n\rightarrow\infty}n^{-1}M_{n}=0$ c.s.
\end{Teo}

\begin{Coro}
Si $N\left(t\right)$ es un proceso de renovaci\'on, y $\left(Z\left(T_{n}\right)-Z\left(T_{n-1}\right),M_{n}\right)$, para $n\geq1$, son variables aleatorias independientes e id\'enticamente distribuidas con media finita, entonces,
\begin{eqnarray}
lim_{t\rightarrow\infty}t^{-1}Z\left(t\right)\rightarrow\frac{\esp\left[Z\left(T_{1}\right)-Z\left(T_{0}\right)\right]}{\esp\left[T_{1}\right]},\textrm{ c.s. cuando  }t\rightarrow\infty.
\end{eqnarray}
\end{Coro}



%___________________________________________________________________________________________
%
%\subsection{Propiedades de los Procesos de Renovaci\'on}
%___________________________________________________________________________________________
%

Los tiempos $T_{n}$ est\'an relacionados con los conteos de $N\left(t\right)$ por

\begin{eqnarray*}
\left\{N\left(t\right)\geq n\right\}&=&\left\{T_{n}\leq t\right\}\\
T_{N\left(t\right)}\leq &t&<T_{N\left(t\right)+1},
\end{eqnarray*}

adem\'as $N\left(T_{n}\right)=n$, y 

\begin{eqnarray*}
N\left(t\right)=\max\left\{n:T_{n}\leq t\right\}=\min\left\{n:T_{n+1}>t\right\}
\end{eqnarray*}

Por propiedades de la convoluci\'on se sabe que

\begin{eqnarray*}
P\left\{T_{n}\leq t\right\}=F^{n\star}\left(t\right)
\end{eqnarray*}
que es la $n$-\'esima convoluci\'on de $F$. Entonces 

\begin{eqnarray*}
\left\{N\left(t\right)\geq n\right\}&=&\left\{T_{n}\leq t\right\}\\
P\left\{N\left(t\right)\leq n\right\}&=&1-F^{\left(n+1\right)\star}\left(t\right)
\end{eqnarray*}

Adem\'as usando el hecho de que $\esp\left[N\left(t\right)\right]=\sum_{n=1}^{\infty}P\left\{N\left(t\right)\geq n\right\}$
se tiene que

\begin{eqnarray*}
\esp\left[N\left(t\right)\right]=\sum_{n=1}^{\infty}F^{n\star}\left(t\right)
\end{eqnarray*}

\begin{Prop}
Para cada $t\geq0$, la funci\'on generadora de momentos $\esp\left[e^{\alpha N\left(t\right)}\right]$ existe para alguna $\alpha$ en una vecindad del 0, y de aqu\'i que $\esp\left[N\left(t\right)^{m}\right]<\infty$, para $m\geq1$.
\end{Prop}


\begin{Note}
Si el primer tiempo de renovaci\'on $\xi_{1}$ no tiene la misma distribuci\'on que el resto de las $\xi_{n}$, para $n\geq2$, a $N\left(t\right)$ se le llama Proceso de Renovaci\'on retardado, donde si $\xi$ tiene distribuci\'on $G$, entonces el tiempo $T_{n}$ de la $n$-\'esima renovaci\'on tiene distribuci\'on $G\star F^{\left(n-1\right)\star}\left(t\right)$
\end{Note}


\begin{Teo}
Para una constante $\mu\leq\infty$ ( o variable aleatoria), las siguientes expresiones son equivalentes:

\begin{eqnarray}
lim_{n\rightarrow\infty}n^{-1}T_{n}&=&\mu,\textrm{ c.s.}\\
lim_{t\rightarrow\infty}t^{-1}N\left(t\right)&=&1/\mu,\textrm{ c.s.}
\end{eqnarray}
\end{Teo}


Es decir, $T_{n}$ satisface la Ley Fuerte de los Grandes N\'umeros s\'i y s\'olo s\'i $N\left/t\right)$ la cumple.


\begin{Coro}[Ley Fuerte de los Grandes N\'umeros para Procesos de Renovaci\'on]
Si $N\left(t\right)$ es un proceso de renovaci\'on cuyos tiempos de inter-renovaci\'on tienen media $\mu\leq\infty$, entonces
\begin{eqnarray}
t^{-1}N\left(t\right)\rightarrow 1/\mu,\textrm{ c.s. cuando }t\rightarrow\infty.
\end{eqnarray}

\end{Coro}


Considerar el proceso estoc\'astico de valores reales $\left\{Z\left(t\right):t\geq0\right\}$ en el mismo espacio de probabilidad que $N\left(t\right)$

\begin{Def}
Para el proceso $\left\{Z\left(t\right):t\geq0\right\}$ se define la fluctuaci\'on m\'axima de $Z\left(t\right)$ en el intervalo $\left(T_{n-1},T_{n}\right]$:
\begin{eqnarray*}
M_{n}=\sup_{T_{n-1}<t\leq T_{n}}|Z\left(t\right)-Z\left(T_{n-1}\right)|
\end{eqnarray*}
\end{Def}

\begin{Teo}
Sup\'ongase que $n^{-1}T_{n}\rightarrow\mu$ c.s. cuando $n\rightarrow\infty$, donde $\mu\leq\infty$ es una constante o variable aleatoria. Sea $a$ una constante o variable aleatoria que puede ser infinita cuando $\mu$ es finita, y considere las expresiones l\'imite:
\begin{eqnarray}
lim_{n\rightarrow\infty}n^{-1}Z\left(T_{n}\right)&=&a,\textrm{ c.s.}\\
lim_{t\rightarrow\infty}t^{-1}Z\left(t\right)&=&a/\mu,\textrm{ c.s.}
\end{eqnarray}
La segunda expresi\'on implica la primera. Conversamente, la primera implica la segunda si el proceso $Z\left(t\right)$ es creciente, o si $lim_{n\rightarrow\infty}n^{-1}M_{n}=0$ c.s.
\end{Teo}

\begin{Coro}
Si $N\left(t\right)$ es un proceso de renovaci\'on, y $\left(Z\left(T_{n}\right)-Z\left(T_{n-1}\right),M_{n}\right)$, para $n\geq1$, son variables aleatorias independientes e id\'enticamente distribuidas con media finita, entonces,
\begin{eqnarray}
lim_{t\rightarrow\infty}t^{-1}Z\left(t\right)\rightarrow\frac{\esp\left[Z\left(T_{1}\right)-Z\left(T_{0}\right)\right]}{\esp\left[T_{1}\right]},\textrm{ c.s. cuando  }t\rightarrow\infty.
\end{eqnarray}
\end{Coro}


%___________________________________________________________________________________________
%
%\subsection{Propiedades de los Procesos de Renovaci\'on}
%___________________________________________________________________________________________
%

Los tiempos $T_{n}$ est\'an relacionados con los conteos de $N\left(t\right)$ por

\begin{eqnarray*}
\left\{N\left(t\right)\geq n\right\}&=&\left\{T_{n}\leq t\right\}\\
T_{N\left(t\right)}\leq &t&<T_{N\left(t\right)+1},
\end{eqnarray*}

adem\'as $N\left(T_{n}\right)=n$, y 

\begin{eqnarray*}
N\left(t\right)=\max\left\{n:T_{n}\leq t\right\}=\min\left\{n:T_{n+1}>t\right\}
\end{eqnarray*}

Por propiedades de la convoluci\'on se sabe que

\begin{eqnarray*}
P\left\{T_{n}\leq t\right\}=F^{n\star}\left(t\right)
\end{eqnarray*}
que es la $n$-\'esima convoluci\'on de $F$. Entonces 

\begin{eqnarray*}
\left\{N\left(t\right)\geq n\right\}&=&\left\{T_{n}\leq t\right\}\\
P\left\{N\left(t\right)\leq n\right\}&=&1-F^{\left(n+1\right)\star}\left(t\right)
\end{eqnarray*}

Adem\'as usando el hecho de que $\esp\left[N\left(t\right)\right]=\sum_{n=1}^{\infty}P\left\{N\left(t\right)\geq n\right\}$
se tiene que

\begin{eqnarray*}
\esp\left[N\left(t\right)\right]=\sum_{n=1}^{\infty}F^{n\star}\left(t\right)
\end{eqnarray*}

\begin{Prop}
Para cada $t\geq0$, la funci\'on generadora de momentos $\esp\left[e^{\alpha N\left(t\right)}\right]$ existe para alguna $\alpha$ en una vecindad del 0, y de aqu\'i que $\esp\left[N\left(t\right)^{m}\right]<\infty$, para $m\geq1$.
\end{Prop}


\begin{Note}
Si el primer tiempo de renovaci\'on $\xi_{1}$ no tiene la misma distribuci\'on que el resto de las $\xi_{n}$, para $n\geq2$, a $N\left(t\right)$ se le llama Proceso de Renovaci\'on retardado, donde si $\xi$ tiene distribuci\'on $G$, entonces el tiempo $T_{n}$ de la $n$-\'esima renovaci\'on tiene distribuci\'on $G\star F^{\left(n-1\right)\star}\left(t\right)$
\end{Note}


\begin{Teo}
Para una constante $\mu\leq\infty$ ( o variable aleatoria), las siguientes expresiones son equivalentes:

\begin{eqnarray}
lim_{n\rightarrow\infty}n^{-1}T_{n}&=&\mu,\textrm{ c.s.}\\
lim_{t\rightarrow\infty}t^{-1}N\left(t\right)&=&1/\mu,\textrm{ c.s.}
\end{eqnarray}
\end{Teo}


Es decir, $T_{n}$ satisface la Ley Fuerte de los Grandes N\'umeros s\'i y s\'olo s\'i $N\left/t\right)$ la cumple.


\begin{Coro}[Ley Fuerte de los Grandes N\'umeros para Procesos de Renovaci\'on]
Si $N\left(t\right)$ es un proceso de renovaci\'on cuyos tiempos de inter-renovaci\'on tienen media $\mu\leq\infty$, entonces
\begin{eqnarray}
t^{-1}N\left(t\right)\rightarrow 1/\mu,\textrm{ c.s. cuando }t\rightarrow\infty.
\end{eqnarray}

\end{Coro}


Considerar el proceso estoc\'astico de valores reales $\left\{Z\left(t\right):t\geq0\right\}$ en el mismo espacio de probabilidad que $N\left(t\right)$

\begin{Def}
Para el proceso $\left\{Z\left(t\right):t\geq0\right\}$ se define la fluctuaci\'on m\'axima de $Z\left(t\right)$ en el intervalo $\left(T_{n-1},T_{n}\right]$:
\begin{eqnarray*}
M_{n}=\sup_{T_{n-1}<t\leq T_{n}}|Z\left(t\right)-Z\left(T_{n-1}\right)|
\end{eqnarray*}
\end{Def}

\begin{Teo}
Sup\'ongase que $n^{-1}T_{n}\rightarrow\mu$ c.s. cuando $n\rightarrow\infty$, donde $\mu\leq\infty$ es una constante o variable aleatoria. Sea $a$ una constante o variable aleatoria que puede ser infinita cuando $\mu$ es finita, y considere las expresiones l\'imite:
\begin{eqnarray}
lim_{n\rightarrow\infty}n^{-1}Z\left(T_{n}\right)&=&a,\textrm{ c.s.}\\
lim_{t\rightarrow\infty}t^{-1}Z\left(t\right)&=&a/\mu,\textrm{ c.s.}
\end{eqnarray}
La segunda expresi\'on implica la primera. Conversamente, la primera implica la segunda si el proceso $Z\left(t\right)$ es creciente, o si $lim_{n\rightarrow\infty}n^{-1}M_{n}=0$ c.s.
\end{Teo}

\begin{Coro}
Si $N\left(t\right)$ es un proceso de renovaci\'on, y $\left(Z\left(T_{n}\right)-Z\left(T_{n-1}\right),M_{n}\right)$, para $n\geq1$, son variables aleatorias independientes e id\'enticamente distribuidas con media finita, entonces,
\begin{eqnarray}
lim_{t\rightarrow\infty}t^{-1}Z\left(t\right)\rightarrow\frac{\esp\left[Z\left(T_{1}\right)-Z\left(T_{0}\right)\right]}{\esp\left[T_{1}\right]},\textrm{ c.s. cuando  }t\rightarrow\infty.
\end{eqnarray}
\end{Coro}

%___________________________________________________________________________________________
%
%\subsection{Propiedades de los Procesos de Renovaci\'on}
%___________________________________________________________________________________________
%

Los tiempos $T_{n}$ est\'an relacionados con los conteos de $N\left(t\right)$ por

\begin{eqnarray*}
\left\{N\left(t\right)\geq n\right\}&=&\left\{T_{n}\leq t\right\}\\
T_{N\left(t\right)}\leq &t&<T_{N\left(t\right)+1},
\end{eqnarray*}

adem\'as $N\left(T_{n}\right)=n$, y 

\begin{eqnarray*}
N\left(t\right)=\max\left\{n:T_{n}\leq t\right\}=\min\left\{n:T_{n+1}>t\right\}
\end{eqnarray*}

Por propiedades de la convoluci\'on se sabe que

\begin{eqnarray*}
P\left\{T_{n}\leq t\right\}=F^{n\star}\left(t\right)
\end{eqnarray*}
que es la $n$-\'esima convoluci\'on de $F$. Entonces 

\begin{eqnarray*}
\left\{N\left(t\right)\geq n\right\}&=&\left\{T_{n}\leq t\right\}\\
P\left\{N\left(t\right)\leq n\right\}&=&1-F^{\left(n+1\right)\star}\left(t\right)
\end{eqnarray*}

Adem\'as usando el hecho de que $\esp\left[N\left(t\right)\right]=\sum_{n=1}^{\infty}P\left\{N\left(t\right)\geq n\right\}$
se tiene que

\begin{eqnarray*}
\esp\left[N\left(t\right)\right]=\sum_{n=1}^{\infty}F^{n\star}\left(t\right)
\end{eqnarray*}

\begin{Prop}
Para cada $t\geq0$, la funci\'on generadora de momentos $\esp\left[e^{\alpha N\left(t\right)}\right]$ existe para alguna $\alpha$ en una vecindad del 0, y de aqu\'i que $\esp\left[N\left(t\right)^{m}\right]<\infty$, para $m\geq1$.
\end{Prop}


\begin{Note}
Si el primer tiempo de renovaci\'on $\xi_{1}$ no tiene la misma distribuci\'on que el resto de las $\xi_{n}$, para $n\geq2$, a $N\left(t\right)$ se le llama Proceso de Renovaci\'on retardado, donde si $\xi$ tiene distribuci\'on $G$, entonces el tiempo $T_{n}$ de la $n$-\'esima renovaci\'on tiene distribuci\'on $G\star F^{\left(n-1\right)\star}\left(t\right)$
\end{Note}


\begin{Teo}
Para una constante $\mu\leq\infty$ ( o variable aleatoria), las siguientes expresiones son equivalentes:

\begin{eqnarray}
lim_{n\rightarrow\infty}n^{-1}T_{n}&=&\mu,\textrm{ c.s.}\\
lim_{t\rightarrow\infty}t^{-1}N\left(t\right)&=&1/\mu,\textrm{ c.s.}
\end{eqnarray}
\end{Teo}


Es decir, $T_{n}$ satisface la Ley Fuerte de los Grandes N\'umeros s\'i y s\'olo s\'i $N\left/t\right)$ la cumple.


\begin{Coro}[Ley Fuerte de los Grandes N\'umeros para Procesos de Renovaci\'on]
Si $N\left(t\right)$ es un proceso de renovaci\'on cuyos tiempos de inter-renovaci\'on tienen media $\mu\leq\infty$, entonces
\begin{eqnarray}
t^{-1}N\left(t\right)\rightarrow 1/\mu,\textrm{ c.s. cuando }t\rightarrow\infty.
\end{eqnarray}

\end{Coro}


Considerar el proceso estoc\'astico de valores reales $\left\{Z\left(t\right):t\geq0\right\}$ en el mismo espacio de probabilidad que $N\left(t\right)$

\begin{Def}
Para el proceso $\left\{Z\left(t\right):t\geq0\right\}$ se define la fluctuaci\'on m\'axima de $Z\left(t\right)$ en el intervalo $\left(T_{n-1},T_{n}\right]$:
\begin{eqnarray*}
M_{n}=\sup_{T_{n-1}<t\leq T_{n}}|Z\left(t\right)-Z\left(T_{n-1}\right)|
\end{eqnarray*}
\end{Def}

\begin{Teo}
Sup\'ongase que $n^{-1}T_{n}\rightarrow\mu$ c.s. cuando $n\rightarrow\infty$, donde $\mu\leq\infty$ es una constante o variable aleatoria. Sea $a$ una constante o variable aleatoria que puede ser infinita cuando $\mu$ es finita, y considere las expresiones l\'imite:
\begin{eqnarray}
lim_{n\rightarrow\infty}n^{-1}Z\left(T_{n}\right)&=&a,\textrm{ c.s.}\\
lim_{t\rightarrow\infty}t^{-1}Z\left(t\right)&=&a/\mu,\textrm{ c.s.}
\end{eqnarray}
La segunda expresi\'on implica la primera. Conversamente, la primera implica la segunda si el proceso $Z\left(t\right)$ es creciente, o si $lim_{n\rightarrow\infty}n^{-1}M_{n}=0$ c.s.
\end{Teo}

\begin{Coro}
Si $N\left(t\right)$ es un proceso de renovaci\'on, y $\left(Z\left(T_{n}\right)-Z\left(T_{n-1}\right),M_{n}\right)$, para $n\geq1$, son variables aleatorias independientes e id\'enticamente distribuidas con media finita, entonces,
\begin{eqnarray}
lim_{t\rightarrow\infty}t^{-1}Z\left(t\right)\rightarrow\frac{\esp\left[Z\left(T_{1}\right)-Z\left(T_{0}\right)\right]}{\esp\left[T_{1}\right]},\textrm{ c.s. cuando  }t\rightarrow\infty.
\end{eqnarray}
\end{Coro}
%___________________________________________________________________________________________
%
%\subsection{Propiedades de los Procesos de Renovaci\'on}
%___________________________________________________________________________________________
%

Los tiempos $T_{n}$ est\'an relacionados con los conteos de $N\left(t\right)$ por

\begin{eqnarray*}
\left\{N\left(t\right)\geq n\right\}&=&\left\{T_{n}\leq t\right\}\\
T_{N\left(t\right)}\leq &t&<T_{N\left(t\right)+1},
\end{eqnarray*}

adem\'as $N\left(T_{n}\right)=n$, y 

\begin{eqnarray*}
N\left(t\right)=\max\left\{n:T_{n}\leq t\right\}=\min\left\{n:T_{n+1}>t\right\}
\end{eqnarray*}

Por propiedades de la convoluci\'on se sabe que

\begin{eqnarray*}
P\left\{T_{n}\leq t\right\}=F^{n\star}\left(t\right)
\end{eqnarray*}
que es la $n$-\'esima convoluci\'on de $F$. Entonces 

\begin{eqnarray*}
\left\{N\left(t\right)\geq n\right\}&=&\left\{T_{n}\leq t\right\}\\
P\left\{N\left(t\right)\leq n\right\}&=&1-F^{\left(n+1\right)\star}\left(t\right)
\end{eqnarray*}

Adem\'as usando el hecho de que $\esp\left[N\left(t\right)\right]=\sum_{n=1}^{\infty}P\left\{N\left(t\right)\geq n\right\}$
se tiene que

\begin{eqnarray*}
\esp\left[N\left(t\right)\right]=\sum_{n=1}^{\infty}F^{n\star}\left(t\right)
\end{eqnarray*}

\begin{Prop}
Para cada $t\geq0$, la funci\'on generadora de momentos $\esp\left[e^{\alpha N\left(t\right)}\right]$ existe para alguna $\alpha$ en una vecindad del 0, y de aqu\'i que $\esp\left[N\left(t\right)^{m}\right]<\infty$, para $m\geq1$.
\end{Prop}


\begin{Note}
Si el primer tiempo de renovaci\'on $\xi_{1}$ no tiene la misma distribuci\'on que el resto de las $\xi_{n}$, para $n\geq2$, a $N\left(t\right)$ se le llama Proceso de Renovaci\'on retardado, donde si $\xi$ tiene distribuci\'on $G$, entonces el tiempo $T_{n}$ de la $n$-\'esima renovaci\'on tiene distribuci\'on $G\star F^{\left(n-1\right)\star}\left(t\right)$
\end{Note}


\begin{Teo}
Para una constante $\mu\leq\infty$ ( o variable aleatoria), las siguientes expresiones son equivalentes:

\begin{eqnarray}
lim_{n\rightarrow\infty}n^{-1}T_{n}&=&\mu,\textrm{ c.s.}\\
lim_{t\rightarrow\infty}t^{-1}N\left(t\right)&=&1/\mu,\textrm{ c.s.}
\end{eqnarray}
\end{Teo}


Es decir, $T_{n}$ satisface la Ley Fuerte de los Grandes N\'umeros s\'i y s\'olo s\'i $N\left/t\right)$ la cumple.


\begin{Coro}[Ley Fuerte de los Grandes N\'umeros para Procesos de Renovaci\'on]
Si $N\left(t\right)$ es un proceso de renovaci\'on cuyos tiempos de inter-renovaci\'on tienen media $\mu\leq\infty$, entonces
\begin{eqnarray}
t^{-1}N\left(t\right)\rightarrow 1/\mu,\textrm{ c.s. cuando }t\rightarrow\infty.
\end{eqnarray}

\end{Coro}


Considerar el proceso estoc\'astico de valores reales $\left\{Z\left(t\right):t\geq0\right\}$ en el mismo espacio de probabilidad que $N\left(t\right)$

\begin{Def}
Para el proceso $\left\{Z\left(t\right):t\geq0\right\}$ se define la fluctuaci\'on m\'axima de $Z\left(t\right)$ en el intervalo $\left(T_{n-1},T_{n}\right]$:
\begin{eqnarray*}
M_{n}=\sup_{T_{n-1}<t\leq T_{n}}|Z\left(t\right)-Z\left(T_{n-1}\right)|
\end{eqnarray*}
\end{Def}

\begin{Teo}
Sup\'ongase que $n^{-1}T_{n}\rightarrow\mu$ c.s. cuando $n\rightarrow\infty$, donde $\mu\leq\infty$ es una constante o variable aleatoria. Sea $a$ una constante o variable aleatoria que puede ser infinita cuando $\mu$ es finita, y considere las expresiones l\'imite:
\begin{eqnarray}
lim_{n\rightarrow\infty}n^{-1}Z\left(T_{n}\right)&=&a,\textrm{ c.s.}\\
lim_{t\rightarrow\infty}t^{-1}Z\left(t\right)&=&a/\mu,\textrm{ c.s.}
\end{eqnarray}
La segunda expresi\'on implica la primera. Conversamente, la primera implica la segunda si el proceso $Z\left(t\right)$ es creciente, o si $lim_{n\rightarrow\infty}n^{-1}M_{n}=0$ c.s.
\end{Teo}

\begin{Coro}
Si $N\left(t\right)$ es un proceso de renovaci\'on, y $\left(Z\left(T_{n}\right)-Z\left(T_{n-1}\right),M_{n}\right)$, para $n\geq1$, son variables aleatorias independientes e id\'enticamente distribuidas con media finita, entonces,
\begin{eqnarray}
lim_{t\rightarrow\infty}t^{-1}Z\left(t\right)\rightarrow\frac{\esp\left[Z\left(T_{1}\right)-Z\left(T_{0}\right)\right]}{\esp\left[T_{1}\right]},\textrm{ c.s. cuando  }t\rightarrow\infty.
\end{eqnarray}
\end{Coro}


%___________________________________________________________________________________________
%
%\subsection{Funci\'on de Renovaci\'on}
%___________________________________________________________________________________________
%


\begin{Def}
Sea $h\left(t\right)$ funci\'on de valores reales en $\rea$ acotada en intervalos finitos e igual a cero para $t<0$ La ecuaci\'on de renovaci\'on para $h\left(t\right)$ y la distribuci\'on $F$ es

\begin{eqnarray}%\label{Ec.Renovacion}
H\left(t\right)=h\left(t\right)+\int_{\left[0,t\right]}H\left(t-s\right)dF\left(s\right)\textrm{,    }t\geq0,
\end{eqnarray}
donde $H\left(t\right)$ es una funci\'on de valores reales. Esto es $H=h+F\star H$. Decimos que $H\left(t\right)$ es soluci\'on de esta ecuaci\'on si satisface la ecuaci\'on, y es acotada en intervalos finitos e iguales a cero para $t<0$.
\end{Def}

\begin{Prop}
La funci\'on $U\star h\left(t\right)$ es la \'unica soluci\'on de la ecuaci\'on de renovaci\'on (\ref{Ec.Renovacion}).
\end{Prop}

\begin{Teo}[Teorema Renovaci\'on Elemental]
\begin{eqnarray*}
t^{-1}U\left(t\right)\rightarrow 1/\mu\textrm{,    cuando }t\rightarrow\infty.
\end{eqnarray*}
\end{Teo}

%___________________________________________________________________________________________
%
%\subsection{Funci\'on de Renovaci\'on}
%___________________________________________________________________________________________
%


Sup\'ongase que $N\left(t\right)$ es un proceso de renovaci\'on con distribuci\'on $F$ con media finita $\mu$.

\begin{Def}
La funci\'on de renovaci\'on asociada con la distribuci\'on $F$, del proceso $N\left(t\right)$, es
\begin{eqnarray*}
U\left(t\right)=\sum_{n=1}^{\infty}F^{n\star}\left(t\right),\textrm{   }t\geq0,
\end{eqnarray*}
donde $F^{0\star}\left(t\right)=\indora\left(t\geq0\right)$.
\end{Def}


\begin{Prop}
Sup\'ongase que la distribuci\'on de inter-renovaci\'on $F$ tiene densidad $f$. Entonces $U\left(t\right)$ tambi\'en tiene densidad, para $t>0$, y es $U^{'}\left(t\right)=\sum_{n=0}^{\infty}f^{n\star}\left(t\right)$. Adem\'as
\begin{eqnarray*}
\prob\left\{N\left(t\right)>N\left(t-\right)\right\}=0\textrm{,   }t\geq0.
\end{eqnarray*}
\end{Prop}

\begin{Def}
La Transformada de Laplace-Stieljes de $F$ est\'a dada por

\begin{eqnarray*}
\hat{F}\left(\alpha\right)=\int_{\rea_{+}}e^{-\alpha t}dF\left(t\right)\textrm{,  }\alpha\geq0.
\end{eqnarray*}
\end{Def}

Entonces

\begin{eqnarray*}
\hat{U}\left(\alpha\right)=\sum_{n=0}^{\infty}\hat{F^{n\star}}\left(\alpha\right)=\sum_{n=0}^{\infty}\hat{F}\left(\alpha\right)^{n}=\frac{1}{1-\hat{F}\left(\alpha\right)}.
\end{eqnarray*}


\begin{Prop}
La Transformada de Laplace $\hat{U}\left(\alpha\right)$ y $\hat{F}\left(\alpha\right)$ determina una a la otra de manera \'unica por la relaci\'on $\hat{U}\left(\alpha\right)=\frac{1}{1-\hat{F}\left(\alpha\right)}$.
\end{Prop}


\begin{Note}
Un proceso de renovaci\'on $N\left(t\right)$ cuyos tiempos de inter-renovaci\'on tienen media finita, es un proceso Poisson con tasa $\lambda$ si y s\'olo s\'i $\esp\left[U\left(t\right)\right]=\lambda t$, para $t\geq0$.
\end{Note}


\begin{Teo}
Sea $N\left(t\right)$ un proceso puntual simple con puntos de localizaci\'on $T_{n}$ tal que $\eta\left(t\right)=\esp\left[N\left(\right)\right]$ es finita para cada $t$. Entonces para cualquier funci\'on $f:\rea_{+}\rightarrow\rea$,
\begin{eqnarray*}
\esp\left[\sum_{n=1}^{N\left(\right)}f\left(T_{n}\right)\right]=\int_{\left(0,t\right]}f\left(s\right)d\eta\left(s\right)\textrm{,  }t\geq0,
\end{eqnarray*}
suponiendo que la integral exista. Adem\'as si $X_{1},X_{2},\ldots$ son variables aleatorias definidas en el mismo espacio de probabilidad que el proceso $N\left(t\right)$ tal que $\esp\left[X_{n}|T_{n}=s\right]=f\left(s\right)$, independiente de $n$. Entonces
\begin{eqnarray*}
\esp\left[\sum_{n=1}^{N\left(t\right)}X_{n}\right]=\int_{\left(0,t\right]}f\left(s\right)d\eta\left(s\right)\textrm{,  }t\geq0,
\end{eqnarray*} 
suponiendo que la integral exista. 
\end{Teo}

\begin{Coro}[Identidad de Wald para Renovaciones]
Para el proceso de renovaci\'on $N\left(t\right)$,
\begin{eqnarray*}
\esp\left[T_{N\left(t\right)+1}\right]=\mu\esp\left[N\left(t\right)+1\right]\textrm{,  }t\geq0,
\end{eqnarray*}  
\end{Coro}

%______________________________________________________________________
%\subsection{Procesos de Renovaci\'on}
%______________________________________________________________________

\begin{Def}%\label{Def.Tn}
Sean $0\leq T_{1}\leq T_{2}\leq \ldots$ son tiempos aleatorios infinitos en los cuales ocurren ciertos eventos. El n\'umero de tiempos $T_{n}$ en el intervalo $\left[0,t\right)$ es

\begin{eqnarray}
N\left(t\right)=\sum_{n=1}^{\infty}\indora\left(T_{n}\leq t\right),
\end{eqnarray}
para $t\geq0$.
\end{Def}

Si se consideran los puntos $T_{n}$ como elementos de $\rea_{+}$, y $N\left(t\right)$ es el n\'umero de puntos en $\rea$. El proceso denotado por $\left\{N\left(t\right):t\geq0\right\}$, denotado por $N\left(t\right)$, es un proceso puntual en $\rea_{+}$. Los $T_{n}$ son los tiempos de ocurrencia, el proceso puntual $N\left(t\right)$ es simple si su n\'umero de ocurrencias son distintas: $0<T_{1}<T_{2}<\ldots$ casi seguramente.

\begin{Def}
Un proceso puntual $N\left(t\right)$ es un proceso de renovaci\'on si los tiempos de interocurrencia $\xi_{n}=T_{n}-T_{n-1}$, para $n\geq1$, son independientes e identicamente distribuidos con distribuci\'on $F$, donde $F\left(0\right)=0$ y $T_{0}=0$. Los $T_{n}$ son llamados tiempos de renovaci\'on, referente a la independencia o renovaci\'on de la informaci\'on estoc\'astica en estos tiempos. Los $\xi_{n}$ son los tiempos de inter-renovaci\'on, y $N\left(t\right)$ es el n\'umero de renovaciones en el intervalo $\left[0,t\right)$
\end{Def}


\begin{Note}
Para definir un proceso de renovaci\'on para cualquier contexto, solamente hay que especificar una distribuci\'on $F$, con $F\left(0\right)=0$, para los tiempos de inter-renovaci\'on. La funci\'on $F$ en turno degune las otra variables aleatorias. De manera formal, existe un espacio de probabilidad y una sucesi\'on de variables aleatorias $\xi_{1},\xi_{2},\ldots$ definidas en este con distribuci\'on $F$. Entonces las otras cantidades son $T_{n}=\sum_{k=1}^{n}\xi_{k}$ y $N\left(t\right)=\sum_{n=1}^{\infty}\indora\left(T_{n}\leq t\right)$, donde $T_{n}\rightarrow\infty$ casi seguramente por la Ley Fuerte de los Grandes Números.
\end{Note}

%___________________________________________________________________________________________
%
%\subsection{Renewal and Regenerative Processes: Serfozo\cite{Serfozo}}
%___________________________________________________________________________________________
%
\begin{Def}%\label{Def.Tn}
Sean $0\leq T_{1}\leq T_{2}\leq \ldots$ son tiempos aleatorios infinitos en los cuales ocurren ciertos eventos. El n\'umero de tiempos $T_{n}$ en el intervalo $\left[0,t\right)$ es

\begin{eqnarray}
N\left(t\right)=\sum_{n=1}^{\infty}\indora\left(T_{n}\leq t\right),
\end{eqnarray}
para $t\geq0$.
\end{Def}

Si se consideran los puntos $T_{n}$ como elementos de $\rea_{+}$, y $N\left(t\right)$ es el n\'umero de puntos en $\rea$. El proceso denotado por $\left\{N\left(t\right):t\geq0\right\}$, denotado por $N\left(t\right)$, es un proceso puntual en $\rea_{+}$. Los $T_{n}$ son los tiempos de ocurrencia, el proceso puntual $N\left(t\right)$ es simple si su n\'umero de ocurrencias son distintas: $0<T_{1}<T_{2}<\ldots$ casi seguramente.

\begin{Def}
Un proceso puntual $N\left(t\right)$ es un proceso de renovaci\'on si los tiempos de interocurrencia $\xi_{n}=T_{n}-T_{n-1}$, para $n\geq1$, son independientes e identicamente distribuidos con distribuci\'on $F$, donde $F\left(0\right)=0$ y $T_{0}=0$. Los $T_{n}$ son llamados tiempos de renovaci\'on, referente a la independencia o renovaci\'on de la informaci\'on estoc\'astica en estos tiempos. Los $\xi_{n}$ son los tiempos de inter-renovaci\'on, y $N\left(t\right)$ es el n\'umero de renovaciones en el intervalo $\left[0,t\right)$
\end{Def}


\begin{Note}
Para definir un proceso de renovaci\'on para cualquier contexto, solamente hay que especificar una distribuci\'on $F$, con $F\left(0\right)=0$, para los tiempos de inter-renovaci\'on. La funci\'on $F$ en turno degune las otra variables aleatorias. De manera formal, existe un espacio de probabilidad y una sucesi\'on de variables aleatorias $\xi_{1},\xi_{2},\ldots$ definidas en este con distribuci\'on $F$. Entonces las otras cantidades son $T_{n}=\sum_{k=1}^{n}\xi_{k}$ y $N\left(t\right)=\sum_{n=1}^{\infty}\indora\left(T_{n}\leq t\right)$, donde $T_{n}\rightarrow\infty$ casi seguramente por la Ley Fuerte de los Grandes N\'umeros.
\end{Note}







Los tiempos $T_{n}$ est\'an relacionados con los conteos de $N\left(t\right)$ por

\begin{eqnarray*}
\left\{N\left(t\right)\geq n\right\}&=&\left\{T_{n}\leq t\right\}\\
T_{N\left(t\right)}\leq &t&<T_{N\left(t\right)+1},
\end{eqnarray*}

adem\'as $N\left(T_{n}\right)=n$, y 

\begin{eqnarray*}
N\left(t\right)=\max\left\{n:T_{n}\leq t\right\}=\min\left\{n:T_{n+1}>t\right\}
\end{eqnarray*}

Por propiedades de la convoluci\'on se sabe que

\begin{eqnarray*}
P\left\{T_{n}\leq t\right\}=F^{n\star}\left(t\right)
\end{eqnarray*}
que es la $n$-\'esima convoluci\'on de $F$. Entonces 

\begin{eqnarray*}
\left\{N\left(t\right)\geq n\right\}&=&\left\{T_{n}\leq t\right\}\\
P\left\{N\left(t\right)\leq n\right\}&=&1-F^{\left(n+1\right)\star}\left(t\right)
\end{eqnarray*}

Adem\'as usando el hecho de que $\esp\left[N\left(t\right)\right]=\sum_{n=1}^{\infty}P\left\{N\left(t\right)\geq n\right\}$
se tiene que

\begin{eqnarray*}
\esp\left[N\left(t\right)\right]=\sum_{n=1}^{\infty}F^{n\star}\left(t\right)
\end{eqnarray*}

\begin{Prop}
Para cada $t\geq0$, la funci\'on generadora de momentos $\esp\left[e^{\alpha N\left(t\right)}\right]$ existe para alguna $\alpha$ en una vecindad del 0, y de aqu\'i que $\esp\left[N\left(t\right)^{m}\right]<\infty$, para $m\geq1$.
\end{Prop}

\begin{Ejem}[\textbf{Proceso Poisson}]

Suponga que se tienen tiempos de inter-renovaci\'on \textit{i.i.d.} del proceso de renovaci\'on $N\left(t\right)$ tienen distribuci\'on exponencial $F\left(t\right)=q-e^{-\lambda t}$ con tasa $\lambda$. Entonces $N\left(t\right)$ es un proceso Poisson con tasa $\lambda$.

\end{Ejem}


\begin{Note}
Si el primer tiempo de renovaci\'on $\xi_{1}$ no tiene la misma distribuci\'on que el resto de las $\xi_{n}$, para $n\geq2$, a $N\left(t\right)$ se le llama Proceso de Renovaci\'on retardado, donde si $\xi$ tiene distribuci\'on $G$, entonces el tiempo $T_{n}$ de la $n$-\'esima renovaci\'on tiene distribuci\'on $G\star F^{\left(n-1\right)\star}\left(t\right)$
\end{Note}


\begin{Teo}
Para una constante $\mu\leq\infty$ ( o variable aleatoria), las siguientes expresiones son equivalentes:

\begin{eqnarray}
lim_{n\rightarrow\infty}n^{-1}T_{n}&=&\mu,\textrm{ c.s.}\\
lim_{t\rightarrow\infty}t^{-1}N\left(t\right)&=&1/\mu,\textrm{ c.s.}
\end{eqnarray}
\end{Teo}


Es decir, $T_{n}$ satisface la Ley Fuerte de los Grandes N\'umeros s\'i y s\'olo s\'i $N\left/t\right)$ la cumple.


\begin{Coro}[Ley Fuerte de los Grandes N\'umeros para Procesos de Renovaci\'on]
Si $N\left(t\right)$ es un proceso de renovaci\'on cuyos tiempos de inter-renovaci\'on tienen media $\mu\leq\infty$, entonces
\begin{eqnarray}
t^{-1}N\left(t\right)\rightarrow 1/\mu,\textrm{ c.s. cuando }t\rightarrow\infty.
\end{eqnarray}

\end{Coro}


Considerar el proceso estoc\'astico de valores reales $\left\{Z\left(t\right):t\geq0\right\}$ en el mismo espacio de probabilidad que $N\left(t\right)$

\begin{Def}
Para el proceso $\left\{Z\left(t\right):t\geq0\right\}$ se define la fluctuaci\'on m\'axima de $Z\left(t\right)$ en el intervalo $\left(T_{n-1},T_{n}\right]$:
\begin{eqnarray*}
M_{n}=\sup_{T_{n-1}<t\leq T_{n}}|Z\left(t\right)-Z\left(T_{n-1}\right)|
\end{eqnarray*}
\end{Def}

\begin{Teo}
Sup\'ongase que $n^{-1}T_{n}\rightarrow\mu$ c.s. cuando $n\rightarrow\infty$, donde $\mu\leq\infty$ es una constante o variable aleatoria. Sea $a$ una constante o variable aleatoria que puede ser infinita cuando $\mu$ es finita, y considere las expresiones l\'imite:
\begin{eqnarray}
lim_{n\rightarrow\infty}n^{-1}Z\left(T_{n}\right)&=&a,\textrm{ c.s.}\\
lim_{t\rightarrow\infty}t^{-1}Z\left(t\right)&=&a/\mu,\textrm{ c.s.}
\end{eqnarray}
La segunda expresi\'on implica la primera. Conversamente, la primera implica la segunda si el proceso $Z\left(t\right)$ es creciente, o si $lim_{n\rightarrow\infty}n^{-1}M_{n}=0$ c.s.
\end{Teo}

\begin{Coro}
Si $N\left(t\right)$ es un proceso de renovaci\'on, y $\left(Z\left(T_{n}\right)-Z\left(T_{n-1}\right),M_{n}\right)$, para $n\geq1$, son variables aleatorias independientes e id\'enticamente distribuidas con media finita, entonces,
\begin{eqnarray}
lim_{t\rightarrow\infty}t^{-1}Z\left(t\right)\rightarrow\frac{\esp\left[Z\left(T_{1}\right)-Z\left(T_{0}\right)\right]}{\esp\left[T_{1}\right]},\textrm{ c.s. cuando  }t\rightarrow\infty.
\end{eqnarray}
\end{Coro}


Sup\'ongase que $N\left(t\right)$ es un proceso de renovaci\'on con distribuci\'on $F$ con media finita $\mu$.

\begin{Def}
La funci\'on de renovaci\'on asociada con la distribuci\'on $F$, del proceso $N\left(t\right)$, es
\begin{eqnarray*}
U\left(t\right)=\sum_{n=1}^{\infty}F^{n\star}\left(t\right),\textrm{   }t\geq0,
\end{eqnarray*}
donde $F^{0\star}\left(t\right)=\indora\left(t\geq0\right)$.
\end{Def}


\begin{Prop}
Sup\'ongase que la distribuci\'on de inter-renovaci\'on $F$ tiene densidad $f$. Entonces $U\left(t\right)$ tambi\'en tiene densidad, para $t>0$, y es $U^{'}\left(t\right)=\sum_{n=0}^{\infty}f^{n\star}\left(t\right)$. Adem\'as
\begin{eqnarray*}
\prob\left\{N\left(t\right)>N\left(t-\right)\right\}=0\textrm{,   }t\geq0.
\end{eqnarray*}
\end{Prop}

\begin{Def}
La Transformada de Laplace-Stieljes de $F$ est\'a dada por

\begin{eqnarray*}
\hat{F}\left(\alpha\right)=\int_{\rea_{+}}e^{-\alpha t}dF\left(t\right)\textrm{,  }\alpha\geq0.
\end{eqnarray*}
\end{Def}

Entonces

\begin{eqnarray*}
\hat{U}\left(\alpha\right)=\sum_{n=0}^{\infty}\hat{F^{n\star}}\left(\alpha\right)=\sum_{n=0}^{\infty}\hat{F}\left(\alpha\right)^{n}=\frac{1}{1-\hat{F}\left(\alpha\right)}.
\end{eqnarray*}


\begin{Prop}
La Transformada de Laplace $\hat{U}\left(\alpha\right)$ y $\hat{F}\left(\alpha\right)$ determina una a la otra de manera \'unica por la relaci\'on $\hat{U}\left(\alpha\right)=\frac{1}{1-\hat{F}\left(\alpha\right)}$.
\end{Prop}


\begin{Note}
Un proceso de renovaci\'on $N\left(t\right)$ cuyos tiempos de inter-renovaci\'on tienen media finita, es un proceso Poisson con tasa $\lambda$ si y s\'olo s\'i $\esp\left[U\left(t\right)\right]=\lambda t$, para $t\geq0$.
\end{Note}


\begin{Teo}
Sea $N\left(t\right)$ un proceso puntual simple con puntos de localizaci\'on $T_{n}$ tal que $\eta\left(t\right)=\esp\left[N\left(\right)\right]$ es finita para cada $t$. Entonces para cualquier funci\'on $f:\rea_{+}\rightarrow\rea$,
\begin{eqnarray*}
\esp\left[\sum_{n=1}^{N\left(\right)}f\left(T_{n}\right)\right]=\int_{\left(0,t\right]}f\left(s\right)d\eta\left(s\right)\textrm{,  }t\geq0,
\end{eqnarray*}
suponiendo que la integral exista. Adem\'as si $X_{1},X_{2},\ldots$ son variables aleatorias definidas en el mismo espacio de probabilidad que el proceso $N\left(t\right)$ tal que $\esp\left[X_{n}|T_{n}=s\right]=f\left(s\right)$, independiente de $n$. Entonces
\begin{eqnarray*}
\esp\left[\sum_{n=1}^{N\left(t\right)}X_{n}\right]=\int_{\left(0,t\right]}f\left(s\right)d\eta\left(s\right)\textrm{,  }t\geq0,
\end{eqnarray*} 
suponiendo que la integral exista. 
\end{Teo}

\begin{Coro}[Identidad de Wald para Renovaciones]
Para el proceso de renovaci\'on $N\left(t\right)$,
\begin{eqnarray*}
\esp\left[T_{N\left(t\right)+1}\right]=\mu\esp\left[N\left(t\right)+1\right]\textrm{,  }t\geq0,
\end{eqnarray*}  
\end{Coro}


\begin{Def}
Sea $h\left(t\right)$ funci\'on de valores reales en $\rea$ acotada en intervalos finitos e igual a cero para $t<0$ La ecuaci\'on de renovaci\'on para $h\left(t\right)$ y la distribuci\'on $F$ es

\begin{eqnarray}%\label{Ec.Renovacion}
H\left(t\right)=h\left(t\right)+\int_{\left[0,t\right]}H\left(t-s\right)dF\left(s\right)\textrm{,    }t\geq0,
\end{eqnarray}
donde $H\left(t\right)$ es una funci\'on de valores reales. Esto es $H=h+F\star H$. Decimos que $H\left(t\right)$ es soluci\'on de esta ecuaci\'on si satisface la ecuaci\'on, y es acotada en intervalos finitos e iguales a cero para $t<0$.
\end{Def}

\begin{Prop}
La funci\'on $U\star h\left(t\right)$ es la \'unica soluci\'on de la ecuaci\'on de renovaci\'on (\ref{Ec.Renovacion}).
\end{Prop}

\begin{Teo}[Teorema Renovaci\'on Elemental]
\begin{eqnarray*}
t^{-1}U\left(t\right)\rightarrow 1/\mu\textrm{,    cuando }t\rightarrow\infty.
\end{eqnarray*}
\end{Teo}



Sup\'ongase que $N\left(t\right)$ es un proceso de renovaci\'on con distribuci\'on $F$ con media finita $\mu$.

\begin{Def}
La funci\'on de renovaci\'on asociada con la distribuci\'on $F$, del proceso $N\left(t\right)$, es
\begin{eqnarray*}
U\left(t\right)=\sum_{n=1}^{\infty}F^{n\star}\left(t\right),\textrm{   }t\geq0,
\end{eqnarray*}
donde $F^{0\star}\left(t\right)=\indora\left(t\geq0\right)$.
\end{Def}


\begin{Prop}
Sup\'ongase que la distribuci\'on de inter-renovaci\'on $F$ tiene densidad $f$. Entonces $U\left(t\right)$ tambi\'en tiene densidad, para $t>0$, y es $U^{'}\left(t\right)=\sum_{n=0}^{\infty}f^{n\star}\left(t\right)$. Adem\'as
\begin{eqnarray*}
\prob\left\{N\left(t\right)>N\left(t-\right)\right\}=0\textrm{,   }t\geq0.
\end{eqnarray*}
\end{Prop}

\begin{Def}
La Transformada de Laplace-Stieljes de $F$ est\'a dada por

\begin{eqnarray*}
\hat{F}\left(\alpha\right)=\int_{\rea_{+}}e^{-\alpha t}dF\left(t\right)\textrm{,  }\alpha\geq0.
\end{eqnarray*}
\end{Def}

Entonces

\begin{eqnarray*}
\hat{U}\left(\alpha\right)=\sum_{n=0}^{\infty}\hat{F^{n\star}}\left(\alpha\right)=\sum_{n=0}^{\infty}\hat{F}\left(\alpha\right)^{n}=\frac{1}{1-\hat{F}\left(\alpha\right)}.
\end{eqnarray*}


\begin{Prop}
La Transformada de Laplace $\hat{U}\left(\alpha\right)$ y $\hat{F}\left(\alpha\right)$ determina una a la otra de manera \'unica por la relaci\'on $\hat{U}\left(\alpha\right)=\frac{1}{1-\hat{F}\left(\alpha\right)}$.
\end{Prop}


\begin{Note}
Un proceso de renovaci\'on $N\left(t\right)$ cuyos tiempos de inter-renovaci\'on tienen media finita, es un proceso Poisson con tasa $\lambda$ si y s\'olo s\'i $\esp\left[U\left(t\right)\right]=\lambda t$, para $t\geq0$.
\end{Note}


\begin{Teo}
Sea $N\left(t\right)$ un proceso puntual simple con puntos de localizaci\'on $T_{n}$ tal que $\eta\left(t\right)=\esp\left[N\left(\right)\right]$ es finita para cada $t$. Entonces para cualquier funci\'on $f:\rea_{+}\rightarrow\rea$,
\begin{eqnarray*}
\esp\left[\sum_{n=1}^{N\left(\right)}f\left(T_{n}\right)\right]=\int_{\left(0,t\right]}f\left(s\right)d\eta\left(s\right)\textrm{,  }t\geq0,
\end{eqnarray*}
suponiendo que la integral exista. Adem\'as si $X_{1},X_{2},\ldots$ son variables aleatorias definidas en el mismo espacio de probabilidad que el proceso $N\left(t\right)$ tal que $\esp\left[X_{n}|T_{n}=s\right]=f\left(s\right)$, independiente de $n$. Entonces
\begin{eqnarray*}
\esp\left[\sum_{n=1}^{N\left(t\right)}X_{n}\right]=\int_{\left(0,t\right]}f\left(s\right)d\eta\left(s\right)\textrm{,  }t\geq0,
\end{eqnarray*} 
suponiendo que la integral exista. 
\end{Teo}

\begin{Coro}[Identidad de Wald para Renovaciones]
Para el proceso de renovaci\'on $N\left(t\right)$,
\begin{eqnarray*}
\esp\left[T_{N\left(t\right)+1}\right]=\mu\esp\left[N\left(t\right)+1\right]\textrm{,  }t\geq0,
\end{eqnarray*}  
\end{Coro}


\begin{Def}
Sea $h\left(t\right)$ funci\'on de valores reales en $\rea$ acotada en intervalos finitos e igual a cero para $t<0$ La ecuaci\'on de renovaci\'on para $h\left(t\right)$ y la distribuci\'on $F$ es

\begin{eqnarray}%\label{Ec.Renovacion}
H\left(t\right)=h\left(t\right)+\int_{\left[0,t\right]}H\left(t-s\right)dF\left(s\right)\textrm{,    }t\geq0,
\end{eqnarray}
donde $H\left(t\right)$ es una funci\'on de valores reales. Esto es $H=h+F\star H$. Decimos que $H\left(t\right)$ es soluci\'on de esta ecuaci\'on si satisface la ecuaci\'on, y es acotada en intervalos finitos e iguales a cero para $t<0$.
\end{Def}

\begin{Prop}
La funci\'on $U\star h\left(t\right)$ es la \'unica soluci\'on de la ecuaci\'on de renovaci\'on (\ref{Ec.Renovacion}).
\end{Prop}

\begin{Teo}[Teorema Renovaci\'on Elemental]
\begin{eqnarray*}
t^{-1}U\left(t\right)\rightarrow 1/\mu\textrm{,    cuando }t\rightarrow\infty.
\end{eqnarray*}
\end{Teo}


\begin{Note} Una funci\'on $h:\rea_{+}\rightarrow\rea$ es Directamente Riemann Integrable en los siguientes casos:
\begin{itemize}
\item[a)] $h\left(t\right)\geq0$ es decreciente y Riemann Integrable.
\item[b)] $h$ es continua excepto posiblemente en un conjunto de Lebesgue de medida 0, y $|h\left(t\right)|\leq b\left(t\right)$, donde $b$ es DRI.
\end{itemize}
\end{Note}

\begin{Teo}[Teorema Principal de Renovaci\'on]
Si $F$ es no aritm\'etica y $h\left(t\right)$ es Directamente Riemann Integrable (DRI), entonces

\begin{eqnarray*}
lim_{t\rightarrow\infty}U\star h=\frac{1}{\mu}\int_{\rea_{+}}h\left(s\right)ds.
\end{eqnarray*}
\end{Teo}

\begin{Prop}
Cualquier funci\'on $H\left(t\right)$ acotada en intervalos finitos y que es 0 para $t<0$ puede expresarse como
\begin{eqnarray*}
H\left(t\right)=U\star h\left(t\right)\textrm{,  donde }h\left(t\right)=H\left(t\right)-F\star H\left(t\right)
\end{eqnarray*}
\end{Prop}

\begin{Def}
Un proceso estoc\'astico $X\left(t\right)$ es crudamente regenerativo en un tiempo aleatorio positivo $T$ si
\begin{eqnarray*}
\esp\left[X\left(T+t\right)|T\right]=\esp\left[X\left(t\right)\right]\textrm{, para }t\geq0,\end{eqnarray*}
y con las esperanzas anteriores finitas.
\end{Def}

\begin{Prop}
Sup\'ongase que $X\left(t\right)$ es un proceso crudamente regenerativo en $T$, que tiene distribuci\'on $F$. Si $\esp\left[X\left(t\right)\right]$ es acotado en intervalos finitos, entonces
\begin{eqnarray*}
\esp\left[X\left(t\right)\right]=U\star h\left(t\right)\textrm{,  donde }h\left(t\right)=\esp\left[X\left(t\right)\indora\left(T>t\right)\right].
\end{eqnarray*}
\end{Prop}

\begin{Teo}[Regeneraci\'on Cruda]
Sup\'ongase que $X\left(t\right)$ es un proceso con valores positivo crudamente regenerativo en $T$, y def\'inase $M=\sup\left\{|X\left(t\right)|:t\leq T\right\}$. Si $T$ es no aritm\'etico y $M$ y $MT$ tienen media finita, entonces
\begin{eqnarray*}
lim_{t\rightarrow\infty}\esp\left[X\left(t\right)\right]=\frac{1}{\mu}\int_{\rea_{+}}h\left(s\right)ds,
\end{eqnarray*}
donde $h\left(t\right)=\esp\left[X\left(t\right)\indora\left(T>t\right)\right]$.
\end{Teo}


\begin{Note} Una funci\'on $h:\rea_{+}\rightarrow\rea$ es Directamente Riemann Integrable en los siguientes casos:
\begin{itemize}
\item[a)] $h\left(t\right)\geq0$ es decreciente y Riemann Integrable.
\item[b)] $h$ es continua excepto posiblemente en un conjunto de Lebesgue de medida 0, y $|h\left(t\right)|\leq b\left(t\right)$, donde $b$ es DRI.
\end{itemize}
\end{Note}

\begin{Teo}[Teorema Principal de Renovaci\'on]
Si $F$ es no aritm\'etica y $h\left(t\right)$ es Directamente Riemann Integrable (DRI), entonces

\begin{eqnarray*}
lim_{t\rightarrow\infty}U\star h=\frac{1}{\mu}\int_{\rea_{+}}h\left(s\right)ds.
\end{eqnarray*}
\end{Teo}

\begin{Prop}
Cualquier funci\'on $H\left(t\right)$ acotada en intervalos finitos y que es 0 para $t<0$ puede expresarse como
\begin{eqnarray*}
H\left(t\right)=U\star h\left(t\right)\textrm{,  donde }h\left(t\right)=H\left(t\right)-F\star H\left(t\right)
\end{eqnarray*}
\end{Prop}

\begin{Def}
Un proceso estoc\'astico $X\left(t\right)$ es crudamente regenerativo en un tiempo aleatorio positivo $T$ si
\begin{eqnarray*}
\esp\left[X\left(T+t\right)|T\right]=\esp\left[X\left(t\right)\right]\textrm{, para }t\geq0,\end{eqnarray*}
y con las esperanzas anteriores finitas.
\end{Def}

\begin{Prop}
Sup\'ongase que $X\left(t\right)$ es un proceso crudamente regenerativo en $T$, que tiene distribuci\'on $F$. Si $\esp\left[X\left(t\right)\right]$ es acotado en intervalos finitos, entonces
\begin{eqnarray*}
\esp\left[X\left(t\right)\right]=U\star h\left(t\right)\textrm{,  donde }h\left(t\right)=\esp\left[X\left(t\right)\indora\left(T>t\right)\right].
\end{eqnarray*}
\end{Prop}

\begin{Teo}[Regeneraci\'on Cruda]
Sup\'ongase que $X\left(t\right)$ es un proceso con valores positivo crudamente regenerativo en $T$, y def\'inase $M=\sup\left\{|X\left(t\right)|:t\leq T\right\}$. Si $T$ es no aritm\'etico y $M$ y $MT$ tienen media finita, entonces
\begin{eqnarray*}
lim_{t\rightarrow\infty}\esp\left[X\left(t\right)\right]=\frac{1}{\mu}\int_{\rea_{+}}h\left(s\right)ds,
\end{eqnarray*}
donde $h\left(t\right)=\esp\left[X\left(t\right)\indora\left(T>t\right)\right]$.
\end{Teo}

\begin{Def}
Para el proceso $\left\{\left(N\left(t\right),X\left(t\right)\right):t\geq0\right\}$, sus trayectoria muestrales en el intervalo de tiempo $\left[T_{n-1},T_{n}\right)$ est\'an descritas por
\begin{eqnarray*}
\zeta_{n}=\left(\xi_{n},\left\{X\left(T_{n-1}+t\right):0\leq t<\xi_{n}\right\}\right)
\end{eqnarray*}
Este $\zeta_{n}$ es el $n$-\'esimo segmento del proceso. El proceso es regenerativo sobre los tiempos $T_{n}$ si sus segmentos $\zeta_{n}$ son independientes e id\'enticamennte distribuidos.
\end{Def}


\begin{Note}
Si $\tilde{X}\left(t\right)$ con espacio de estados $\tilde{S}$ es regenerativo sobre $T_{n}$, entonces $X\left(t\right)=f\left(\tilde{X}\left(t\right)\right)$ tambi\'en es regenerativo sobre $T_{n}$, para cualquier funci\'on $f:\tilde{S}\rightarrow S$.
\end{Note}

\begin{Note}
Los procesos regenerativos son crudamente regenerativos, pero no al rev\'es.
\end{Note}


\begin{Note}
Un proceso estoc\'astico a tiempo continuo o discreto es regenerativo si existe un proceso de renovaci\'on  tal que los segmentos del proceso entre tiempos de renovaci\'on sucesivos son i.i.d., es decir, para $\left\{X\left(t\right):t\geq0\right\}$ proceso estoc\'astico a tiempo continuo con espacio de estados $S$, espacio m\'etrico.
\end{Note}

Para $\left\{X\left(t\right):t\geq0\right\}$ Proceso Estoc\'astico a tiempo continuo con estado de espacios $S$, que es un espacio m\'etrico, con trayectorias continuas por la derecha y con l\'imites por la izquierda c.s. Sea $N\left(t\right)$ un proceso de renovaci\'on en $\rea_{+}$ definido en el mismo espacio de probabilidad que $X\left(t\right)$, con tiempos de renovaci\'on $T$ y tiempos de inter-renovaci\'on $\xi_{n}=T_{n}-T_{n-1}$, con misma distribuci\'on $F$ de media finita $\mu$.



\begin{Def}
Para el proceso $\left\{\left(N\left(t\right),X\left(t\right)\right):t\geq0\right\}$, sus trayectoria muestrales en el intervalo de tiempo $\left[T_{n-1},T_{n}\right)$ est\'an descritas por
\begin{eqnarray*}
\zeta_{n}=\left(\xi_{n},\left\{X\left(T_{n-1}+t\right):0\leq t<\xi_{n}\right\}\right)
\end{eqnarray*}
Este $\zeta_{n}$ es el $n$-\'esimo segmento del proceso. El proceso es regenerativo sobre los tiempos $T_{n}$ si sus segmentos $\zeta_{n}$ son independientes e id\'enticamennte distribuidos.
\end{Def}

\begin{Note}
Un proceso regenerativo con media de la longitud de ciclo finita es llamado positivo recurrente.
\end{Note}

\begin{Teo}[Procesos Regenerativos]
Suponga que el proceso
\end{Teo}


\begin{Def}[Renewal Process Trinity]
Para un proceso de renovaci\'on $N\left(t\right)$, los siguientes procesos proveen de informaci\'on sobre los tiempos de renovaci\'on.
\begin{itemize}
\item $A\left(t\right)=t-T_{N\left(t\right)}$, el tiempo de recurrencia hacia atr\'as al tiempo $t$, que es el tiempo desde la \'ultima renovaci\'on para $t$.

\item $B\left(t\right)=T_{N\left(t\right)+1}-t$, el tiempo de recurrencia hacia adelante al tiempo $t$, residual del tiempo de renovaci\'on, que es el tiempo para la pr\'oxima renovaci\'on despu\'es de $t$.

\item $L\left(t\right)=\xi_{N\left(t\right)+1}=A\left(t\right)+B\left(t\right)$, la longitud del intervalo de renovaci\'on que contiene a $t$.
\end{itemize}
\end{Def}

\begin{Note}
El proceso tridimensional $\left(A\left(t\right),B\left(t\right),L\left(t\right)\right)$ es regenerativo sobre $T_{n}$, y por ende cada proceso lo es. Cada proceso $A\left(t\right)$ y $B\left(t\right)$ son procesos de MArkov a tiempo continuo con trayectorias continuas por partes en el espacio de estados $\rea_{+}$. Una expresi\'on conveniente para su distribuci\'on conjunta es, para $0\leq x<t,y\geq0$
\begin{equation}\label{NoRenovacion}
P\left\{A\left(t\right)>x,B\left(t\right)>y\right\}=
P\left\{N\left(t+y\right)-N\left((t-x)\right)=0\right\}
\end{equation}
\end{Note}

\begin{Ejem}[Tiempos de recurrencia Poisson]
Si $N\left(t\right)$ es un proceso Poisson con tasa $\lambda$, entonces de la expresi\'on (\ref{NoRenovacion}) se tiene que

\begin{eqnarray*}
\begin{array}{lc}
P\left\{A\left(t\right)>x,B\left(t\right)>y\right\}=e^{-\lambda\left(x+y\right)},&0\leq x<t,y\geq0,
\end{array}
\end{eqnarray*}
que es la probabilidad Poisson de no renovaciones en un intervalo de longitud $x+y$.

\end{Ejem}

\begin{Note}
Una cadena de Markov erg\'odica tiene la propiedad de ser estacionaria si la distribuci\'on de su estado al tiempo $0$ es su distribuci\'on estacionaria.
\end{Note}


\begin{Def}
Un proceso estoc\'astico a tiempo continuo $\left\{X\left(t\right):t\geq0\right\}$ en un espacio general es estacionario si sus distribuciones finito dimensionales son invariantes bajo cualquier  traslado: para cada $0\leq s_{1}<s_{2}<\cdots<s_{k}$ y $t\geq0$,
\begin{eqnarray*}
\left(X\left(s_{1}+t\right),\ldots,X\left(s_{k}+t\right)\right)=_{d}\left(X\left(s_{1}\right),\ldots,X\left(s_{k}\right)\right).
\end{eqnarray*}
\end{Def}

\begin{Note}
Un proceso de Markov es estacionario si $X\left(t\right)=_{d}X\left(0\right)$, $t\geq0$.
\end{Note}

Considerese el proceso $N\left(t\right)=\sum_{n}\indora\left(\tau_{n}\leq t\right)$ en $\rea_{+}$, con puntos $0<\tau_{1}<\tau_{2}<\cdots$.

\begin{Prop}
Si $N$ es un proceso puntual estacionario y $\esp\left[N\left(1\right)\right]<\infty$, entonces $\esp\left[N\left(t\right)\right]=t\esp\left[N\left(1\right)\right]$, $t\geq0$

\end{Prop}

\begin{Teo}
Los siguientes enunciados son equivalentes
\begin{itemize}
\item[i)] El proceso retardado de renovaci\'on $N$ es estacionario.

\item[ii)] EL proceso de tiempos de recurrencia hacia adelante $B\left(t\right)$ es estacionario.


\item[iii)] $\esp\left[N\left(t\right)\right]=t/\mu$,


\item[iv)] $G\left(t\right)=F_{e}\left(t\right)=\frac{1}{\mu}\int_{0}^{t}\left[1-F\left(s\right)\right]ds$
\end{itemize}
Cuando estos enunciados son ciertos, $P\left\{B\left(t\right)\leq x\right\}=F_{e}\left(x\right)$, para $t,x\geq0$.

\end{Teo}

\begin{Note}
Una consecuencia del teorema anterior es que el Proceso Poisson es el \'unico proceso sin retardo que es estacionario.
\end{Note}

\begin{Coro}
El proceso de renovaci\'on $N\left(t\right)$ sin retardo, y cuyos tiempos de inter renonaci\'on tienen media finita, es estacionario si y s\'olo si es un proceso Poisson.

\end{Coro}


%________________________________________________________________________
%\subsection{Procesos Regenerativos}
%________________________________________________________________________



\begin{Note}
Si $\tilde{X}\left(t\right)$ con espacio de estados $\tilde{S}$ es regenerativo sobre $T_{n}$, entonces $X\left(t\right)=f\left(\tilde{X}\left(t\right)\right)$ tambi\'en es regenerativo sobre $T_{n}$, para cualquier funci\'on $f:\tilde{S}\rightarrow S$.
\end{Note}

\begin{Note}
Los procesos regenerativos son crudamente regenerativos, pero no al rev\'es.
\end{Note}
%\subsection*{Procesos Regenerativos: Sigman\cite{Sigman1}}
\begin{Def}[Definici\'on Cl\'asica]
Un proceso estoc\'astico $X=\left\{X\left(t\right):t\geq0\right\}$ es llamado regenerativo is existe una variable aleatoria $R_{1}>0$ tal que
\begin{itemize}
\item[i)] $\left\{X\left(t+R_{1}\right):t\geq0\right\}$ es independiente de $\left\{\left\{X\left(t\right):t<R_{1}\right\},\right\}$
\item[ii)] $\left\{X\left(t+R_{1}\right):t\geq0\right\}$ es estoc\'asticamente equivalente a $\left\{X\left(t\right):t>0\right\}$
\end{itemize}

Llamamos a $R_{1}$ tiempo de regeneraci\'on, y decimos que $X$ se regenera en este punto.
\end{Def}

$\left\{X\left(t+R_{1}\right)\right\}$ es regenerativo con tiempo de regeneraci\'on $R_{2}$, independiente de $R_{1}$ pero con la misma distribuci\'on que $R_{1}$. Procediendo de esta manera se obtiene una secuencia de variables aleatorias independientes e id\'enticamente distribuidas $\left\{R_{n}\right\}$ llamados longitudes de ciclo. Si definimos a $Z_{k}\equiv R_{1}+R_{2}+\cdots+R_{k}$, se tiene un proceso de renovaci\'on llamado proceso de renovaci\'on encajado para $X$.




\begin{Def}
Para $x$ fijo y para cada $t\geq0$, sea $I_{x}\left(t\right)=1$ si $X\left(t\right)\leq x$,  $I_{x}\left(t\right)=0$ en caso contrario, y def\'inanse los tiempos promedio
\begin{eqnarray*}
\overline{X}&=&lim_{t\rightarrow\infty}\frac{1}{t}\int_{0}^{\infty}X\left(u\right)du\\
\prob\left(X_{\infty}\leq x\right)&=&lim_{t\rightarrow\infty}\frac{1}{t}\int_{0}^{\infty}I_{x}\left(u\right)du,
\end{eqnarray*}
cuando estos l\'imites existan.
\end{Def}

Como consecuencia del teorema de Renovaci\'on-Recompensa, se tiene que el primer l\'imite  existe y es igual a la constante
\begin{eqnarray*}
\overline{X}&=&\frac{\esp\left[\int_{0}^{R_{1}}X\left(t\right)dt\right]}{\esp\left[R_{1}\right]},
\end{eqnarray*}
suponiendo que ambas esperanzas son finitas.

\begin{Note}
\begin{itemize}
\item[a)] Si el proceso regenerativo $X$ es positivo recurrente y tiene trayectorias muestrales no negativas, entonces la ecuaci\'on anterior es v\'alida.
\item[b)] Si $X$ es positivo recurrente regenerativo, podemos construir una \'unica versi\'on estacionaria de este proceso, $X_{e}=\left\{X_{e}\left(t\right)\right\}$, donde $X_{e}$ es un proceso estoc\'astico regenerativo y estrictamente estacionario, con distribuci\'on marginal distribuida como $X_{\infty}$
\end{itemize}
\end{Note}

%________________________________________________________________________
%\subsection{Procesos Regenerativos}
%________________________________________________________________________

Para $\left\{X\left(t\right):t\geq0\right\}$ Proceso Estoc\'astico a tiempo continuo con estado de espacios $S$, que es un espacio m\'etrico, con trayectorias continuas por la derecha y con l\'imites por la izquierda c.s. Sea $N\left(t\right)$ un proceso de renovaci\'on en $\rea_{+}$ definido en el mismo espacio de probabilidad que $X\left(t\right)$, con tiempos de renovaci\'on $T$ y tiempos de inter-renovaci\'on $\xi_{n}=T_{n}-T_{n-1}$, con misma distribuci\'on $F$ de media finita $\mu$.



\begin{Def}
Para el proceso $\left\{\left(N\left(t\right),X\left(t\right)\right):t\geq0\right\}$, sus trayectoria muestrales en el intervalo de tiempo $\left[T_{n-1},T_{n}\right)$ est\'an descritas por
\begin{eqnarray*}
\zeta_{n}=\left(\xi_{n},\left\{X\left(T_{n-1}+t\right):0\leq t<\xi_{n}\right\}\right)
\end{eqnarray*}
Este $\zeta_{n}$ es el $n$-\'esimo segmento del proceso. El proceso es regenerativo sobre los tiempos $T_{n}$ si sus segmentos $\zeta_{n}$ son independientes e id\'enticamennte distribuidos.
\end{Def}


\begin{Note}
Si $\tilde{X}\left(t\right)$ con espacio de estados $\tilde{S}$ es regenerativo sobre $T_{n}$, entonces $X\left(t\right)=f\left(\tilde{X}\left(t\right)\right)$ tambi\'en es regenerativo sobre $T_{n}$, para cualquier funci\'on $f:\tilde{S}\rightarrow S$.
\end{Note}

\begin{Note}
Los procesos regenerativos son crudamente regenerativos, pero no al rev\'es.
\end{Note}

\begin{Def}[Definici\'on Cl\'asica]
Un proceso estoc\'astico $X=\left\{X\left(t\right):t\geq0\right\}$ es llamado regenerativo is existe una variable aleatoria $R_{1}>0$ tal que
\begin{itemize}
\item[i)] $\left\{X\left(t+R_{1}\right):t\geq0\right\}$ es independiente de $\left\{\left\{X\left(t\right):t<R_{1}\right\},\right\}$
\item[ii)] $\left\{X\left(t+R_{1}\right):t\geq0\right\}$ es estoc\'asticamente equivalente a $\left\{X\left(t\right):t>0\right\}$
\end{itemize}

Llamamos a $R_{1}$ tiempo de regeneraci\'on, y decimos que $X$ se regenera en este punto.
\end{Def}

$\left\{X\left(t+R_{1}\right)\right\}$ es regenerativo con tiempo de regeneraci\'on $R_{2}$, independiente de $R_{1}$ pero con la misma distribuci\'on que $R_{1}$. Procediendo de esta manera se obtiene una secuencia de variables aleatorias independientes e id\'enticamente distribuidas $\left\{R_{n}\right\}$ llamados longitudes de ciclo. Si definimos a $Z_{k}\equiv R_{1}+R_{2}+\cdots+R_{k}$, se tiene un proceso de renovaci\'on llamado proceso de renovaci\'on encajado para $X$.

\begin{Note}
Un proceso regenerativo con media de la longitud de ciclo finita es llamado positivo recurrente.
\end{Note}


\begin{Def}
Para $x$ fijo y para cada $t\geq0$, sea $I_{x}\left(t\right)=1$ si $X\left(t\right)\leq x$,  $I_{x}\left(t\right)=0$ en caso contrario, y def\'inanse los tiempos promedio
\begin{eqnarray*}
\overline{X}&=&lim_{t\rightarrow\infty}\frac{1}{t}\int_{0}^{\infty}X\left(u\right)du\\
\prob\left(X_{\infty}\leq x\right)&=&lim_{t\rightarrow\infty}\frac{1}{t}\int_{0}^{\infty}I_{x}\left(u\right)du,
\end{eqnarray*}
cuando estos l\'imites existan.
\end{Def}

Como consecuencia del teorema de Renovaci\'on-Recompensa, se tiene que el primer l\'imite  existe y es igual a la constante
\begin{eqnarray*}
\overline{X}&=&\frac{\esp\left[\int_{0}^{R_{1}}X\left(t\right)dt\right]}{\esp\left[R_{1}\right]},
\end{eqnarray*}
suponiendo que ambas esperanzas son finitas.

\begin{Note}
\begin{itemize}
\item[a)] Si el proceso regenerativo $X$ es positivo recurrente y tiene trayectorias muestrales no negativas, entonces la ecuaci\'on anterior es v\'alida.
\item[b)] Si $X$ es positivo recurrente regenerativo, podemos construir una \'unica versi\'on estacionaria de este proceso, $X_{e}=\left\{X_{e}\left(t\right)\right\}$, donde $X_{e}$ es un proceso estoc\'astico regenerativo y estrictamente estacionario, con distribuci\'on marginal distribuida como $X_{\infty}$
\end{itemize}
\end{Note}

%__________________________________________________________________________________________
%\subsection{Procesos Regenerativos Estacionarios - Stidham \cite{Stidham}}
%__________________________________________________________________________________________


Un proceso estoc\'astico a tiempo continuo $\left\{V\left(t\right),t\geq0\right\}$ es un proceso regenerativo si existe una sucesi\'on de variables aleatorias independientes e id\'enticamente distribuidas $\left\{X_{1},X_{2},\ldots\right\}$, sucesi\'on de renovaci\'on, tal que para cualquier conjunto de Borel $A$, 

\begin{eqnarray*}
\prob\left\{V\left(t\right)\in A|X_{1}+X_{2}+\cdots+X_{R\left(t\right)}=s,\left\{V\left(\tau\right),\tau<s\right\}\right\}=\prob\left\{V\left(t-s\right)\in A|X_{1}>t-s\right\},
\end{eqnarray*}
para todo $0\leq s\leq t$, donde $R\left(t\right)=\max\left\{X_{1}+X_{2}+\cdots+X_{j}\leq t\right\}=$n\'umero de renovaciones ({\emph{puntos de regeneraci\'on}}) que ocurren en $\left[0,t\right]$. El intervalo $\left[0,X_{1}\right)$ es llamado {\emph{primer ciclo de regeneraci\'on}} de $\left\{V\left(t \right),t\geq0\right\}$, $\left[X_{1},X_{1}+X_{2}\right)$ el {\emph{segundo ciclo de regeneraci\'on}}, y as\'i sucesivamente.

Sea $X=X_{1}$ y sea $F$ la funci\'on de distrbuci\'on de $X$


\begin{Def}
Se define el proceso estacionario, $\left\{V^{*}\left(t\right),t\geq0\right\}$, para $\left\{V\left(t\right),t\geq0\right\}$ por

\begin{eqnarray*}
\prob\left\{V\left(t\right)\in A\right\}=\frac{1}{\esp\left[X\right]}\int_{0}^{\infty}\prob\left\{V\left(t+x\right)\in A|X>x\right\}\left(1-F\left(x\right)\right)dx,
\end{eqnarray*} 
para todo $t\geq0$ y todo conjunto de Borel $A$.
\end{Def}

\begin{Def}
Una distribuci\'on se dice que es {\emph{aritm\'etica}} si todos sus puntos de incremento son m\'ultiplos de la forma $0,\lambda, 2\lambda,\ldots$ para alguna $\lambda>0$ entera.
\end{Def}


\begin{Def}
Una modificaci\'on medible de un proceso $\left\{V\left(t\right),t\geq0\right\}$, es una versi\'on de este, $\left\{V\left(t,w\right)\right\}$ conjuntamente medible para $t\geq0$ y para $w\in S$, $S$ espacio de estados para $\left\{V\left(t\right),t\geq0\right\}$.
\end{Def}

\begin{Teo}
Sea $\left\{V\left(t\right),t\geq\right\}$ un proceso regenerativo no negativo con modificaci\'on medible. Sea $\esp\left[X\right]<\infty$. Entonces el proceso estacionario dado por la ecuaci\'on anterior est\'a bien definido y tiene funci\'on de distribuci\'on independiente de $t$, adem\'as
\begin{itemize}
\item[i)] \begin{eqnarray*}
\esp\left[V^{*}\left(0\right)\right]&=&\frac{\esp\left[\int_{0}^{X}V\left(s\right)ds\right]}{\esp\left[X\right]}\end{eqnarray*}
\item[ii)] Si $\esp\left[V^{*}\left(0\right)\right]<\infty$, equivalentemente, si $\esp\left[\int_{0}^{X}V\left(s\right)ds\right]<\infty$,entonces
\begin{eqnarray*}
\frac{\int_{0}^{t}V\left(s\right)ds}{t}\rightarrow\frac{\esp\left[\int_{0}^{X}V\left(s\right)ds\right]}{\esp\left[X\right]}
\end{eqnarray*}
con probabilidad 1 y en media, cuando $t\rightarrow\infty$.
\end{itemize}
\end{Teo}
%
%___________________________________________________________________________________________
%\vspace{5.5cm}
%\chapter{Cadenas de Markov estacionarias}
%\vspace{-1.0cm}


%__________________________________________________________________________________________
%\subsection{Procesos Regenerativos Estacionarios - Stidham \cite{Stidham}}
%__________________________________________________________________________________________


Un proceso estoc\'astico a tiempo continuo $\left\{V\left(t\right),t\geq0\right\}$ es un proceso regenerativo si existe una sucesi\'on de variables aleatorias independientes e id\'enticamente distribuidas $\left\{X_{1},X_{2},\ldots\right\}$, sucesi\'on de renovaci\'on, tal que para cualquier conjunto de Borel $A$, 

\begin{eqnarray*}
\prob\left\{V\left(t\right)\in A|X_{1}+X_{2}+\cdots+X_{R\left(t\right)}=s,\left\{V\left(\tau\right),\tau<s\right\}\right\}=\prob\left\{V\left(t-s\right)\in A|X_{1}>t-s\right\},
\end{eqnarray*}
para todo $0\leq s\leq t$, donde $R\left(t\right)=\max\left\{X_{1}+X_{2}+\cdots+X_{j}\leq t\right\}=$n\'umero de renovaciones ({\emph{puntos de regeneraci\'on}}) que ocurren en $\left[0,t\right]$. El intervalo $\left[0,X_{1}\right)$ es llamado {\emph{primer ciclo de regeneraci\'on}} de $\left\{V\left(t \right),t\geq0\right\}$, $\left[X_{1},X_{1}+X_{2}\right)$ el {\emph{segundo ciclo de regeneraci\'on}}, y as\'i sucesivamente.

Sea $X=X_{1}$ y sea $F$ la funci\'on de distrbuci\'on de $X$


\begin{Def}
Se define el proceso estacionario, $\left\{V^{*}\left(t\right),t\geq0\right\}$, para $\left\{V\left(t\right),t\geq0\right\}$ por

\begin{eqnarray*}
\prob\left\{V\left(t\right)\in A\right\}=\frac{1}{\esp\left[X\right]}\int_{0}^{\infty}\prob\left\{V\left(t+x\right)\in A|X>x\right\}\left(1-F\left(x\right)\right)dx,
\end{eqnarray*} 
para todo $t\geq0$ y todo conjunto de Borel $A$.
\end{Def}

\begin{Def}
Una distribuci\'on se dice que es {\emph{aritm\'etica}} si todos sus puntos de incremento son m\'ultiplos de la forma $0,\lambda, 2\lambda,\ldots$ para alguna $\lambda>0$ entera.
\end{Def}


\begin{Def}
Una modificaci\'on medible de un proceso $\left\{V\left(t\right),t\geq0\right\}$, es una versi\'on de este, $\left\{V\left(t,w\right)\right\}$ conjuntamente medible para $t\geq0$ y para $w\in S$, $S$ espacio de estados para $\left\{V\left(t\right),t\geq0\right\}$.
\end{Def}

\begin{Teo}
Sea $\left\{V\left(t\right),t\geq\right\}$ un proceso regenerativo no negativo con modificaci\'on medible. Sea $\esp\left[X\right]<\infty$. Entonces el proceso estacionario dado por la ecuaci\'on anterior est\'a bien definido y tiene funci\'on de distribuci\'on independiente de $t$, adem\'as
\begin{itemize}
\item[i)] \begin{eqnarray*}
\esp\left[V^{*}\left(0\right)\right]&=&\frac{\esp\left[\int_{0}^{X}V\left(s\right)ds\right]}{\esp\left[X\right]}\end{eqnarray*}
\item[ii)] Si $\esp\left[V^{*}\left(0\right)\right]<\infty$, equivalentemente, si $\esp\left[\int_{0}^{X}V\left(s\right)ds\right]<\infty$,entonces
\begin{eqnarray*}
\frac{\int_{0}^{t}V\left(s\right)ds}{t}\rightarrow\frac{\esp\left[\int_{0}^{X}V\left(s\right)ds\right]}{\esp\left[X\right]}
\end{eqnarray*}
con probabilidad 1 y en media, cuando $t\rightarrow\infty$.
\end{itemize}
\end{Teo}

Para $\left\{X\left(t\right):t\geq0\right\}$ Proceso Estoc\'astico a tiempo continuo con estado de espacios $S$, que es un espacio m\'etrico, con trayectorias continuas por la derecha y con l\'imites por la izquierda c.s. Sea $N\left(t\right)$ un proceso de renovaci\'on en $\rea_{+}$ definido en el mismo espacio de probabilidad que $X\left(t\right)$, con tiempos de renovaci\'on $T$ y tiempos de inter-renovaci\'on $\xi_{n}=T_{n}-T_{n-1}$, con misma distribuci\'on $F$ de media finita $\mu$.


%______________________________________________________________________
%\subsection{Ejemplos, Notas importantes}


Sean $T_{1},T_{2},\ldots$ los puntos donde las longitudes de las colas de la red de sistemas de visitas c\'iclicas son cero simult\'aneamente, cuando la cola $Q_{j}$ es visitada por el servidor para dar servicio, es decir, $L_{1}\left(T_{i}\right)=0,L_{2}\left(T_{i}\right)=0,\hat{L}_{1}\left(T_{i}\right)=0$ y $\hat{L}_{2}\left(T_{i}\right)=0$, a estos puntos se les denominar\'a puntos regenerativos. Sea la funci\'on generadora de momentos para $L_{i}$, el n\'umero de usuarios en la cola $Q_{i}\left(z\right)$ en cualquier momento, est\'a dada por el tiempo promedio de $z^{L_{i}\left(t\right)}$ sobre el ciclo regenerativo definido anteriormente:

\begin{eqnarray*}
Q_{i}\left(z\right)&=&\esp\left[z^{L_{i}\left(t\right)}\right]=\frac{\esp\left[\sum_{m=1}^{M_{i}}\sum_{t=\tau_{i}\left(m\right)}^{\tau_{i}\left(m+1\right)-1}z^{L_{i}\left(t\right)}\right]}{\esp\left[\sum_{m=1}^{M_{i}}\tau_{i}\left(m+1\right)-\tau_{i}\left(m\right)\right]}
\end{eqnarray*}

$M_{i}$ es un tiempo de paro en el proceso regenerativo con $\esp\left[M_{i}\right]<\infty$\footnote{En Stidham\cite{Stidham} y Heyman  se muestra que una condici\'on suficiente para que el proceso regenerativo 
estacionario sea un procesoo estacionario es que el valor esperado del tiempo del ciclo regenerativo sea finito, es decir: $\esp\left[\sum_{m=1}^{M_{i}}C_{i}^{(m)}\right]<\infty$, como cada $C_{i}^{(m)}$ contiene intervalos de r\'eplica positivos, se tiene que $\esp\left[M_{i}\right]<\infty$, adem\'as, como $M_{i}>0$, se tiene que la condici\'on anterior es equivalente a tener que $\esp\left[C_{i}\right]<\infty$,
por lo tanto una condici\'on suficiente para la existencia del proceso regenerativo est\'a dada por $\sum_{k=1}^{N}\mu_{k}<1.$}, se sigue del lema de Wald que:


\begin{eqnarray*}
\esp\left[\sum_{m=1}^{M_{i}}\sum_{t=\tau_{i}\left(m\right)}^{\tau_{i}\left(m+1\right)-1}z^{L_{i}\left(t\right)}\right]&=&\esp\left[M_{i}\right]\esp\left[\sum_{t=\tau_{i}\left(m\right)}^{\tau_{i}\left(m+1\right)-1}z^{L_{i}\left(t\right)}\right]\\
\esp\left[\sum_{m=1}^{M_{i}}\tau_{i}\left(m+1\right)-\tau_{i}\left(m\right)\right]&=&\esp\left[M_{i}\right]\esp\left[\tau_{i}\left(m+1\right)-\tau_{i}\left(m\right)\right]
\end{eqnarray*}

por tanto se tiene que


\begin{eqnarray*}
Q_{i}\left(z\right)&=&\frac{\esp\left[\sum_{t=\tau_{i}\left(m\right)}^{\tau_{i}\left(m+1\right)-1}z^{L_{i}\left(t\right)}\right]}{\esp\left[\tau_{i}\left(m+1\right)-\tau_{i}\left(m\right)\right]}
\end{eqnarray*}

observar que el denominador es simplemente la duraci\'on promedio del tiempo del ciclo.


Haciendo las siguientes sustituciones en la ecuaci\'on (\ref{Corolario2}): $n\rightarrow t-\tau_{i}\left(m\right)$, $T \rightarrow \overline{\tau}_{i}\left(m\right)-\tau_{i}\left(m\right)$, $L_{n}\rightarrow L_{i}\left(t\right)$ y $F\left(z\right)=\esp\left[z^{L_{0}}\right]\rightarrow F_{i}\left(z\right)=\esp\left[z^{L_{i}\tau_{i}\left(m\right)}\right]$, se puede ver que

\begin{eqnarray}\label{Eq.Arribos.Primera}
\esp\left[\sum_{n=0}^{T-1}z^{L_{n}}\right]=
\esp\left[\sum_{t=\tau_{i}\left(m\right)}^{\overline{\tau}_{i}\left(m\right)-1}z^{L_{i}\left(t\right)}\right]
=z\frac{F_{i}\left(z\right)-1}{z-P_{i}\left(z\right)}
\end{eqnarray}

Por otra parte durante el tiempo de intervisita para la cola $i$, $L_{i}\left(t\right)$ solamente se incrementa de manera que el incremento por intervalo de tiempo est\'a dado por la funci\'on generadora de probabilidades de $P_{i}\left(z\right)$, por tanto la suma sobre el tiempo de intervisita puede evaluarse como:

\begin{eqnarray*}
\esp\left[\sum_{t=\tau_{i}\left(m\right)}^{\tau_{i}\left(m+1\right)-1}z^{L_{i}\left(t\right)}\right]&=&\esp\left[\sum_{t=\tau_{i}\left(m\right)}^{\tau_{i}\left(m+1\right)-1}\left\{P_{i}\left(z\right)\right\}^{t-\overline{\tau}_{i}\left(m\right)}\right]=\frac{1-\esp\left[\left\{P_{i}\left(z\right)\right\}^{\tau_{i}\left(m+1\right)-\overline{\tau}_{i}\left(m\right)}\right]}{1-P_{i}\left(z\right)}\\
&=&\frac{1-I_{i}\left[P_{i}\left(z\right)\right]}{1-P_{i}\left(z\right)}
\end{eqnarray*}
por tanto

\begin{eqnarray*}
\esp\left[\sum_{t=\tau_{i}\left(m\right)}^{\tau_{i}\left(m+1\right)-1}z^{L_{i}\left(t\right)}\right]&=&
\frac{1-F_{i}\left(z\right)}{1-P_{i}\left(z\right)}
\end{eqnarray*}

Por lo tanto

\begin{eqnarray*}
Q_{i}\left(z\right)&=&\frac{\esp\left[\sum_{t=\tau_{i}\left(m\right)}^{\tau_{i}
\left(m+1\right)-1}z^{L_{i}\left(t\right)}\right]}{\esp\left[\tau_{i}\left(m+1\right)-\tau_{i}\left(m\right)\right]}\\
&=&\frac{1}{\esp\left[\tau_{i}\left(m+1\right)-\tau_{i}\left(m\right)\right]}
\left\{
\esp\left[\sum_{t=\tau_{i}\left(m\right)}^{\overline{\tau}_{i}\left(m\right)-1}
z^{L_{i}\left(t\right)}\right]
+\esp\left[\sum_{t=\overline{\tau}_{i}\left(m\right)}^{\tau_{i}\left(m+1\right)-1}
z^{L_{i}\left(t\right)}\right]\right\}\\
&=&\frac{1}{\esp\left[\tau_{i}\left(m+1\right)-\tau_{i}\left(m\right)\right]}
\left\{
z\frac{F_{i}\left(z\right)-1}{z-P_{i}\left(z\right)}+\frac{1-F_{i}\left(z\right)}
{1-P_{i}\left(z\right)}
\right\}
\end{eqnarray*}

es decir

\begin{equation}
Q_{i}\left(z\right)=\frac{1}{\esp\left[C_{i}\right]}\cdot\frac{1-F_{i}\left(z\right)}{P_{i}\left(z\right)-z}\cdot\frac{\left(1-z\right)P_{i}\left(z\right)}{1-P_{i}\left(z\right)}
\end{equation}

\begin{Teo}
Dada una Red de Sistemas de Visitas C\'iclicas (RSVC), conformada por dos Sistemas de Visitas C\'iclicas (SVC), donde cada uno de ellos consta de dos colas tipo $M/M/1$. Los dos sistemas est\'an comunicados entre s\'i por medio de la transferencia de usuarios entre las colas $Q_{1}\leftrightarrow Q_{3}$ y $Q_{2}\leftrightarrow Q_{4}$. Se definen los eventos para los procesos de arribos al tiempo $t$, $A_{j}\left(t\right)=\left\{0 \textrm{ arribos en }Q_{j}\textrm{ al tiempo }t\right\}$ para alg\'un tiempo $t\geq0$ y $Q_{j}$ la cola $j$-\'esima en la RSVC, para $j=1,2,3,4$.  Existe un intervalo $I\neq\emptyset$ tal que para $T^{*}\in I$, tal que $\prob\left\{A_{1}\left(T^{*}\right),A_{2}\left(Tt^{*}\right),
A_{3}\left(T^{*}\right),A_{4}\left(T^{*}\right)|T^{*}\in I\right\}>0$.
\end{Teo}

\begin{proof}
Sin p\'erdida de generalidad podemos considerar como base del an\'alisis a la cola $Q_{1}$ del primer sistema que conforma la RSVC.

Sea $n>0$, ciclo en el primer sistema en el que se sabe que $L_{j}\left(\overline{\tau}_{1}\left(n\right)\right)=0$, pues la pol\'itica de servicio con que atienden los servidores es la exhaustiva. Como es sabido, para trasladarse a la siguiente cola, el servidor incurre en un tiempo de traslado $r_{1}\left(n\right)>0$, entonces tenemos que $\tau_{2}\left(n\right)=\overline{\tau}_{1}\left(n\right)+r_{1}\left(n\right)$.


Definamos el intervalo $I_{1}\equiv\left[\overline{\tau}_{1}\left(n\right),\tau_{2}\left(n\right)\right]$ de longitud $\xi_{1}=r_{1}\left(n\right)$. Dado que los tiempos entre arribo son exponenciales con tasa $\tilde{\mu}_{1}=\mu_{1}+\hat{\mu}_{1}$ ($\mu_{1}$ son los arribos a $Q_{1}$ por primera vez al sistema, mientras que $\hat{\mu}_{1}$ son los arribos de traslado procedentes de $Q_{3}$) se tiene que la probabilidad del evento $A_{1}\left(t\right)$ est\'a dada por 

\begin{equation}
\prob\left\{A_{1}\left(t\right)|t\in I_{1}\left(n\right)\right\}=e^{-\tilde{\mu}_{1}\xi_{1}\left(n\right)}.
\end{equation} 

Por otra parte, para la cola $Q_{2}$, el tiempo $\overline{\tau}_{2}\left(n-1\right)$ es tal que $L_{2}\left(\overline{\tau}_{2}\left(n-1\right)\right)=0$, es decir, es el tiempo en que la cola queda totalmente vac\'ia en el ciclo anterior a $n$. Entonces tenemos un sgundo intervalo $I_{2}\equiv\left[\overline{\tau}_{2}\left(n-1\right),\tau_{2}\left(n\right)\right]$. Por lo tanto la probabilidad del evento $A_{2}\left(t\right)$ tiene probabilidad dada por

\begin{equation}
\prob\left\{A_{2}\left(t\right)|t\in I_{2}\left(n\right)\right\}=e^{-\tilde{\mu}_{2}\xi_{2}\left(n\right)},
\end{equation} 

donde $\xi_{2}\left(n\right)=\tau_{2}\left(n\right)-\overline{\tau}_{2}\left(n-1\right)$.



Entonces, se tiene que

\begin{eqnarray*}
\prob\left\{A_{1}\left(t\right),A_{2}\left(t\right)|t\in I_{1}\left(n\right)\right\}&=&
\prob\left\{A_{1}\left(t\right)|t\in I_{1}\left(n\right)\right\}
\prob\left\{A_{2}\left(t\right)|t\in I_{1}\left(n\right)\right\}\\
&\geq&
\prob\left\{A_{1}\left(t\right)|t\in I_{1}\left(n\right)\right\}
\prob\left\{A_{2}\left(t\right)|t\in I_{2}\left(n\right)\right\}\\
&=&e^{-\tilde{\mu}_{1}\xi_{1}\left(n\right)}e^{-\tilde{\mu}_{2}\xi_{2}\left(n\right)}
=e^{-\left[\tilde{\mu}_{1}\xi_{1}\left(n\right)+\tilde{\mu}_{2}\xi_{2}\left(n\right)\right]}.
\end{eqnarray*}


es decir, 

\begin{equation}
\prob\left\{A_{1}\left(t\right),A_{2}\left(t\right)|t\in I_{1}\left(n\right)\right\}
=e^{-\left[\tilde{\mu}_{1}\xi_{1}\left(n\right)+\tilde{\mu}_{2}\xi_{2}
\left(n\right)\right]}>0.
\end{equation}

En lo que respecta a la relaci\'on entre los dos SVC que conforman la RSVC, se afirma que existe $m>0$ tal que $\overline{\tau}_{3}\left(m\right)<\tau_{2}\left(n\right)<\tau_{4}\left(m\right)$.

Para $Q_{3}$ sea $I_{3}=\left[\overline{\tau}_{3}\left(m\right),\tau_{4}\left(m\right)\right]$ con longitud  $\xi_{3}\left(m\right)=r_{3}\left(m\right)$, entonces 

\begin{equation}
\prob\left\{A_{3}\left(t\right)|t\in I_{3}\left(n\right)\right\}=e^{-\tilde{\mu}_{3}\xi_{3}\left(n\right)}.
\end{equation} 

An\'alogamente que como se hizo para $Q_{2}$, tenemos que para $Q_{4}$ se tiene el intervalo $I_{4}=\left[\overline{\tau}_{4}\left(m-1\right),\tau_{4}\left(m\right)\right]$ con longitud $\xi_{4}\left(m\right)=\tau_{4}\left(m\right)-\overline{\tau}_{4}\left(m-1\right)$, entonces


\begin{equation}
\prob\left\{A_{4}\left(t\right)|t\in I_{4}\left(m\right)\right\}=e^{-\tilde{\mu}_{4}\xi_{4}\left(n\right)}.
\end{equation} 

Al igual que para el primer sistema, dado que $I_{3}\left(m\right)\subset I_{4}\left(m\right)$, se tiene que

\begin{eqnarray*}
\xi_{3}\left(m\right)\leq\xi_{4}\left(m\right)&\Leftrightarrow& -\xi_{3}\left(m\right)\geq-\xi_{4}\left(m\right)
\\
-\tilde{\mu}_{4}\xi_{3}\left(m\right)\geq-\tilde{\mu}_{4}\xi_{4}\left(m\right)&\Leftrightarrow&
e^{-\tilde{\mu}_{4}\xi_{3}\left(m\right)}\geq e^{-\tilde{\mu}_{4}\xi_{4}\left(m\right)}\\
\prob\left\{A_{4}\left(t\right)|t\in I_{3}\left(m\right)\right\}&\geq&
\prob\left\{A_{4}\left(t\right)|t\in I_{4}\left(m\right)\right\}
\end{eqnarray*}

Entonces, dado que los eventos $A_{3}$ y $A_{4}$ son independientes, se tiene que

\begin{eqnarray*}
\prob\left\{A_{3}\left(t\right),A_{4}\left(t\right)|t\in I_{3}\left(m\right)\right\}&=&
\prob\left\{A_{3}\left(t\right)|t\in I_{3}\left(m\right)\right\}
\prob\left\{A_{4}\left(t\right)|t\in I_{3}\left(m\right)\right\}\\
&\geq&
\prob\left\{A_{3}\left(t\right)|t\in I_{3}\left(n\right)\right\}
\prob\left\{A_{4}\left(t\right)|t\in I_{4}\left(n\right)\right\}\\
&=&e^{-\tilde{\mu}_{3}\xi_{3}\left(m\right)}e^{-\tilde{\mu}_{4}\xi_{4}
\left(m\right)}
=e^{-\left[\tilde{\mu}_{3}\xi_{3}\left(m\right)+\tilde{\mu}_{4}\xi_{4}
\left(m\right)\right]}.
\end{eqnarray*}


es decir, 

\begin{equation}
\prob\left\{A_{3}\left(t\right),A_{4}\left(t\right)|t\in I_{3}\left(m\right)\right\}
=e^{-\left[\tilde{\mu}_{3}\xi_{3}\left(m\right)+\tilde{\mu}_{4}\xi_{4}
\left(m\right)\right]}>0.
\end{equation}

Por construcci\'on se tiene que $I\left(n,m\right)\equiv I_{1}\left(n\right)\cap I_{3}\left(m\right)\neq\emptyset$,entonces en particular se tienen las contenciones $I\left(n,m\right)\subseteq I_{1}\left(n\right)$ y $I\left(n,m\right)\subseteq I_{3}\left(m\right)$, por lo tanto si definimos $\xi_{n,m}\equiv\ell\left(I\left(n,m\right)\right)$ tenemos que

\begin{eqnarray*}
\xi_{n,m}\leq\xi_{1}\left(n\right)\textrm{ y }\xi_{n,m}\leq\xi_{3}\left(m\right)\textrm{ entonces }
-\xi_{n,m}\geq-\xi_{1}\left(n\right)\textrm{ y }-\xi_{n,m}\leq-\xi_{3}\left(m\right)\\
\end{eqnarray*}
por lo tanto tenemos las desigualdades 



\begin{eqnarray*}
\begin{array}{ll}
-\tilde{\mu}_{1}\xi_{n,m}\geq-\tilde{\mu}_{1}\xi_{1}\left(n\right),&
-\tilde{\mu}_{2}\xi_{n,m}\geq-\tilde{\mu}_{2}\xi_{1}\left(n\right)
\geq-\tilde{\mu}_{2}\xi_{2}\left(n\right),\\
-\tilde{\mu}_{3}\xi_{n,m}\geq-\tilde{\mu}_{3}\xi_{3}\left(m\right),&
-\tilde{\mu}_{4}\xi_{n,m}\geq-\tilde{\mu}_{4}\xi_{3}\left(m\right)
\geq-\tilde{\mu}_{4}\xi_{4}\left(m\right).
\end{array}
\end{eqnarray*}

Sea $T^{*}\in I_{n,m}$, entonces dado que en particular $T^{*}\in I_{1}\left(n\right)$ se cumple con probabilidad positiva que no hay arribos a las colas $Q_{1}$ y $Q_{2}$, en consecuencia, tampoco hay usuarios de transferencia para $Q_{3}$ y $Q_{4}$, es decir, $\tilde{\mu}_{1}=\mu_{1}$, $\tilde{\mu}_{2}=\mu_{2}$, $\tilde{\mu}_{3}=\mu_{3}$, $\tilde{\mu}_{4}=\mu_{4}$, es decir, los eventos $Q_{1}$ y $Q_{3}$ son condicionalmente independientes en el intervalo $I_{n,m}$; lo mismo ocurre para las colas $Q_{2}$ y $Q_{4}$, por lo tanto tenemos que


\begin{eqnarray}
\begin{array}{l}
\prob\left\{A_{1}\left(T^{*}\right),A_{2}\left(T^{*}\right),
A_{3}\left(T^{*}\right),A_{4}\left(T^{*}\right)|T^{*}\in I_{n,m}\right\}
=\prod_{j=1}^{4}\prob\left\{A_{j}\left(T^{*}\right)|T^{*}\in I_{n,m}\right\}\\
\geq\prob\left\{A_{1}\left(T^{*}\right)|T^{*}\in I_{1}\left(n\right)\right\}
\prob\left\{A_{2}\left(T^{*}\right)|T^{*}\in I_{2}\left(n\right)\right\}
\prob\left\{A_{3}\left(T^{*}\right)|T^{*}\in I_{3}\left(m\right)\right\}
\prob\left\{A_{4}\left(T^{*}\right)|T^{*}\in I_{4}\left(m\right)\right\}\\
=e^{-\mu_{1}\xi_{1}\left(n\right)}
e^{-\mu_{2}\xi_{2}\left(n\right)}
e^{-\mu_{3}\xi_{3}\left(m\right)}
e^{-\mu_{4}\xi_{4}\left(m\right)}
=e^{-\left[\tilde{\mu}_{1}\xi_{1}\left(n\right)
+\tilde{\mu}_{2}\xi_{2}\left(n\right)
+\tilde{\mu}_{3}\xi_{3}\left(m\right)
+\tilde{\mu}_{4}\xi_{4}
\left(m\right)\right]}>0.
\end{array}
\end{eqnarray}
\end{proof}


Estos resultados aparecen en Daley (1968) \cite{Daley68} para $\left\{T_{n}\right\}$ intervalos de inter-arribo, $\left\{D_{n}\right\}$ intervalos de inter-salida y $\left\{S_{n}\right\}$ tiempos de servicio.

\begin{itemize}
\item Si el proceso $\left\{T_{n}\right\}$ es Poisson, el proceso $\left\{D_{n}\right\}$ es no correlacionado si y s\'olo si es un proceso Poisso, lo cual ocurre si y s\'olo si $\left\{S_{n}\right\}$ son exponenciales negativas.

\item Si $\left\{S_{n}\right\}$ son exponenciales negativas, $\left\{D_{n}\right\}$ es un proceso de renovaci\'on  si y s\'olo si es un proceso Poisson, lo cual ocurre si y s\'olo si $\left\{T_{n}\right\}$ es un proceso Poisson.

\item $\esp\left(D_{n}\right)=\esp\left(T_{n}\right)$.

\item Para un sistema de visitas $GI/M/1$ se tiene el siguiente teorema:

\begin{Teo}
En un sistema estacionario $GI/M/1$ los intervalos de interpartida tienen
\begin{eqnarray*}
\esp\left(e^{-\theta D_{n}}\right)&=&\mu\left(\mu+\theta\right)^{-1}\left[\delta\theta
-\mu\left(1-\delta\right)\alpha\left(\theta\right)\right]
\left[\theta-\mu\left(1-\delta\right)^{-1}\right]\\
\alpha\left(\theta\right)&=&\esp\left[e^{-\theta T_{0}}\right]\\
var\left(D_{n}\right)&=&var\left(T_{0}\right)-\left(\tau^{-1}-\delta^{-1}\right)
2\delta\left(\esp\left(S_{0}\right)\right)^{2}\left(1-\delta\right)^{-1}.
\end{eqnarray*}
\end{Teo}



\begin{Teo}
El proceso de salida de un sistema de colas estacionario $GI/M/1$ es un proceso de renovaci\'on si y s\'olo si el proceso de entrada es un proceso Poisson, en cuyo caso el proceso de salida es un proceso Poisson.
\end{Teo}


\begin{Teo}
Los intervalos de interpartida $\left\{D_{n}\right\}$ de un sistema $M/G/1$ estacionario son no correlacionados si y s\'olo si la distribuci\'on de los tiempos de servicio es exponencial negativa, es decir, el sistema es de tipo  $M/M/1$.

\end{Teo}



\end{itemize}


%\section{Resultados para Procesos de Salida}

En Sigman, Thorison y Wolff \cite{Sigman2} prueban que para la existencia de un una sucesi\'on infinita no decreciente de tiempos de regeneraci\'on $\tau_{1}\leq\tau_{2}\leq\cdots$ en los cuales el proceso se regenera, basta un tiempo de regeneraci\'on $R_{1}$, donde $R_{j}=\tau_{j}-\tau_{j-1}$. Para tal efecto se requiere la existencia de un espacio de probabilidad $\left(\Omega,\mathcal{F},\prob\right)$, y proceso estoc\'astico $\textit{X}=\left\{X\left(t\right):t\geq0\right\}$ con espacio de estados $\left(S,\mathcal{R}\right)$, con $\mathcal{R}$ $\sigma$-\'algebra.

\begin{Prop}
Si existe una variable aleatoria no negativa $R_{1}$ tal que $\theta_{R\footnotesize{1}}X=_{D}X$, entonces $\left(\Omega,\mathcal{F},\prob\right)$ puede extenderse para soportar una sucesi\'on estacionaria de variables aleatorias $R=\left\{R_{k}:k\geq1\right\}$, tal que para $k\geq1$,
\begin{eqnarray*}
\theta_{k}\left(X,R\right)=_{D}\left(X,R\right).
\end{eqnarray*}

Adem\'as, para $k\geq1$, $\theta_{k}R$ es condicionalmente independiente de $\left(X,R_{1},\ldots,R_{k}\right)$, dado $\theta_{\tau k}X$.

\end{Prop}


\begin{itemize}
\item Doob en 1953 demostr\'o que el estado estacionario de un proceso de partida en un sistema de espera $M/G/\infty$, es Poisson con la misma tasa que el proceso de arribos.

\item Burke en 1968, fue el primero en demostrar que el estado estacionario de un proceso de salida de una cola $M/M/s$ es un proceso Poisson.

\item Disney en 1973 obtuvo el siguiente resultado:

\begin{Teo}
Para el sistema de espera $M/G/1/L$ con disciplina FIFO, el proceso $\textbf{I}$ es un proceso de renovaci\'on si y s\'olo si el proceso denominado longitud de la cola es estacionario y se cumple cualquiera de los siguientes casos:

\begin{itemize}
\item[a)] Los tiempos de servicio son identicamente cero;
\item[b)] $L=0$, para cualquier proceso de servicio $S$;
\item[c)] $L=1$ y $G=D$;
\item[d)] $L=\infty$ y $G=M$.
\end{itemize}
En estos casos, respectivamente, las distribuciones de interpartida $P\left\{T_{n+1}-T_{n}\leq t\right\}$ son


\begin{itemize}
\item[a)] $1-e^{-\lambda t}$, $t\geq0$;
\item[b)] $1-e^{-\lambda t}*F\left(t\right)$, $t\geq0$;
\item[c)] $1-e^{-\lambda t}*\indora_{d}\left(t\right)$, $t\geq0$;
\item[d)] $1-e^{-\lambda t}*F\left(t\right)$, $t\geq0$.
\end{itemize}
\end{Teo}


\item Finch (1959) mostr\'o que para los sistemas $M/G/1/L$, con $1\leq L\leq \infty$ con distribuciones de servicio dos veces diferenciable, solamente el sistema $M/M/1/\infty$ tiene proceso de salida de renovaci\'on estacionario.

\item King (1971) demostro que un sistema de colas estacionario $M/G/1/1$ tiene sus tiempos de interpartida sucesivas $D_{n}$ y $D_{n+1}$ son independientes, si y s\'olo si, $G=D$, en cuyo caso le proceso de salida es de renovaci\'on.

\item Disney (1973) demostr\'o que el \'unico sistema estacionario $M/G/1/L$, que tiene proceso de salida de renovaci\'on  son los sistemas $M/M/1$ y $M/D/1/1$.



\item El siguiente resultado es de Disney y Koning (1985)
\begin{Teo}
En un sistema de espera $M/G/s$, el estado estacionario del proceso de salida es un proceso Poisson para cualquier distribuci\'on de los tiempos de servicio si el sistema tiene cualquiera de las siguientes cuatro propiedades.

\begin{itemize}
\item[a)] $s=\infty$
\item[b)] La disciplina de servicio es de procesador compartido.
\item[c)] La disciplina de servicio es LCFS y preemptive resume, esto se cumple para $L<\infty$
\item[d)] $G=M$.
\end{itemize}

\end{Teo}

\item El siguiente resultado es de Alamatsaz (1983)

\begin{Teo}
En cualquier sistema de colas $GI/G/1/L$ con $1\leq L<\infty$ y distribuci\'on de interarribos $A$ y distribuci\'on de los tiempos de servicio $B$, tal que $A\left(0\right)=0$, $A\left(t\right)\left(1-B\left(t\right)\right)>0$ para alguna $t>0$ y $B\left(t\right)$ para toda $t>0$, es imposible que el proceso de salida estacionario sea de renovaci\'on.
\end{Teo}

\end{itemize}

Estos resultados aparecen en Daley (1968) \cite{Daley68} para $\left\{T_{n}\right\}$ intervalos de inter-arribo, $\left\{D_{n}\right\}$ intervalos de inter-salida y $\left\{S_{n}\right\}$ tiempos de servicio.

\begin{itemize}
\item Si el proceso $\left\{T_{n}\right\}$ es Poisson, el proceso $\left\{D_{n}\right\}$ es no correlacionado si y s\'olo si es un proceso Poisso, lo cual ocurre si y s\'olo si $\left\{S_{n}\right\}$ son exponenciales negativas.

\item Si $\left\{S_{n}\right\}$ son exponenciales negativas, $\left\{D_{n}\right\}$ es un proceso de renovaci\'on  si y s\'olo si es un proceso Poisson, lo cual ocurre si y s\'olo si $\left\{T_{n}\right\}$ es un proceso Poisson.

\item $\esp\left(D_{n}\right)=\esp\left(T_{n}\right)$.

\item Para un sistema de visitas $GI/M/1$ se tiene el siguiente teorema:

\begin{Teo}
En un sistema estacionario $GI/M/1$ los intervalos de interpartida tienen
\begin{eqnarray*}
\esp\left(e^{-\theta D_{n}}\right)&=&\mu\left(\mu+\theta\right)^{-1}\left[\delta\theta
-\mu\left(1-\delta\right)\alpha\left(\theta\right)\right]
\left[\theta-\mu\left(1-\delta\right)^{-1}\right]\\
\alpha\left(\theta\right)&=&\esp\left[e^{-\theta T_{0}}\right]\\
var\left(D_{n}\right)&=&var\left(T_{0}\right)-\left(\tau^{-1}-\delta^{-1}\right)
2\delta\left(\esp\left(S_{0}\right)\right)^{2}\left(1-\delta\right)^{-1}.
\end{eqnarray*}
\end{Teo}



\begin{Teo}
El proceso de salida de un sistema de colas estacionario $GI/M/1$ es un proceso de renovaci\'on si y s\'olo si el proceso de entrada es un proceso Poisson, en cuyo caso el proceso de salida es un proceso Poisson.
\end{Teo}


\begin{Teo}
Los intervalos de interpartida $\left\{D_{n}\right\}$ de un sistema $M/G/1$ estacionario son no correlacionados si y s\'olo si la distribuci\'on de los tiempos de servicio es exponencial negativa, es decir, el sistema es de tipo  $M/M/1$.

\end{Teo}



\end{itemize}
%\newpage
%________________________________________________________________________
%\section{Redes de Sistemas de Visitas C\'iclicas}
%________________________________________________________________________

Sean $Q_{1},Q_{2},Q_{3}$ y $Q_{4}$ en una Red de Sistemas de Visitas C\'iclicas (RSVC). Supongamos que cada una de las colas es del tipo $M/M/1$ con tasa de arribo $\mu_{i}$ y que la transferencia de usuarios entre los dos sistemas ocurre entre $Q_{1}\leftrightarrow Q_{3}$ y $Q_{2}\leftrightarrow Q_{4}$ con respectiva tasa de arribo igual a la tasa de salida $\hat{\mu}_{i}=\mu_{i}$, esto se sabe por lo desarrollado en la secci\'on anterior.  

Consideremos, sin p\'erdida de generalidad como base del an\'alisis, la cola $Q_{1}$ adem\'as supongamos al servidor lo comenzamos a observar una vez que termina de atender a la misma para desplazarse y llegar a $Q_{2}$, es decir al tiempo $\tau_{2}$.

Sea $n\in\nat$, $n>0$, ciclo del servidor en que regresa a $Q_{1}$ para dar servicio y atender conforme a la pol\'itica exhaustiva, entonces se tiene que $\overline{\tau}_{1}\left(n\right)$ es el tiempo del servidor en el sistema 1 en que termina de dar servicio a todos los usuarios presentes en la cola, por lo tanto se cumple que $L_{1}\left(\overline{\tau}_{1}\left(n\right)\right)=0$, entonces el servidor para llegar a $Q_{2}$ incurre en un tiempo de traslado $r_{1}$ y por tanto se cumple que $\tau_{2}\left(n\right)=\overline{\tau}_{1}\left(n\right)+r_{1}$. Dado que los tiempos entre arribos son exponenciales se cumple que 

\begin{eqnarray*}
\prob\left\{0 \textrm{ arribos en }Q_{1}\textrm{ en el intervalo }\left[\overline{\tau}_{1}\left(n\right),\overline{\tau}_{1}\left(n\right)+r_{1}\right]\right\}=e^{-\tilde{\mu}_{1}r_{1}},\\
\prob\left\{0 \textrm{ arribos en }Q_{2}\textrm{ en el intervalo }\left[\overline{\tau}_{1}\left(n\right),\overline{\tau}_{1}\left(n\right)+r_{1}\right]\right\}=e^{-\tilde{\mu}_{2}r_{1}}.
\end{eqnarray*}

El evento que nos interesa consiste en que no haya arribos desde que el servidor abandon\'o $Q_{2}$ y regresa nuevamente para dar servicio, es decir en el intervalo de tiempo $\left[\overline{\tau}_{2}\left(n-1\right),\tau_{2}\left(n\right)\right]$. Entonces, si hacemos


\begin{eqnarray*}
\varphi_{1}\left(n\right)&\equiv&\overline{\tau}_{1}\left(n\right)+r_{1}=\overline{\tau}_{2}\left(n-1\right)+r_{1}+r_{2}+\overline{\tau}_{1}\left(n\right)-\tau_{1}\left(n\right)\\
&=&\overline{\tau}_{2}\left(n-1\right)+\overline{\tau}_{1}\left(n\right)-\tau_{1}\left(n\right)+r,,
\end{eqnarray*}

y longitud del intervalo

\begin{eqnarray*}
\xi&\equiv&\overline{\tau}_{1}\left(n\right)+r_{1}-\overline{\tau}_{2}\left(n-1\right)
=\overline{\tau}_{2}\left(n-1\right)+\overline{\tau}_{1}\left(n\right)-\tau_{1}\left(n\right)+r-\overline{\tau}_{2}\left(n-1\right)\\
&=&\overline{\tau}_{1}\left(n\right)-\tau_{1}\left(n\right)+r.
\end{eqnarray*}


Entonces, determinemos la probabilidad del evento no arribos a $Q_{2}$ en $\left[\overline{\tau}_{2}\left(n-1\right),\varphi_{1}\left(n\right)\right]$:

\begin{eqnarray}
\prob\left\{0 \textrm{ arribos en }Q_{2}\textrm{ en el intervalo }\left[\overline{\tau}_{2}\left(n-1\right),\varphi_{1}\left(n\right)\right]\right\}
=e^{-\tilde{\mu}_{2}\xi}.
\end{eqnarray}

De manera an\'aloga, tenemos que la probabilidad de no arribos a $Q_{1}$ en $\left[\overline{\tau}_{2}\left(n-1\right),\varphi_{1}\left(n\right)\right]$ esta dada por

\begin{eqnarray}
\prob\left\{0 \textrm{ arribos en }Q_{1}\textrm{ en el intervalo }\left[\overline{\tau}_{2}\left(n-1\right),\varphi_{1}\left(n\right)\right]\right\}
=e^{-\tilde{\mu}_{1}\xi},
\end{eqnarray}

\begin{eqnarray}
\prob\left\{0 \textrm{ arribos en }Q_{2}\textrm{ en el intervalo }\left[\overline{\tau}_{2}\left(n-1\right),\varphi_{1}\left(n\right)\right]\right\}
=e^{-\tilde{\mu}_{2}\xi}.
\end{eqnarray}

Por tanto 

\begin{eqnarray}
\begin{array}{l}
\prob\left\{0 \textrm{ arribos en }Q_{1}\textrm{ y }Q_{2}\textrm{ en el intervalo }\left[\overline{\tau}_{2}\left(n-1\right),\varphi_{1}\left(n\right)\right]\right\}\\
=\prob\left\{0 \textrm{ arribos en }Q_{1}\textrm{ en el intervalo }\left[\overline{\tau}_{2}\left(n-1\right),\varphi_{1}\left(n\right)\right]\right\}\\
\times
\prob\left\{0 \textrm{ arribos en }Q_{2}\textrm{ en el intervalo }\left[\overline{\tau}_{2}\left(n-1\right),\varphi_{1}\left(n\right)\right]\right\}=e^{-\tilde{\mu}_{1}\xi}e^{-\tilde{\mu}_{2}\xi}
=e^{-\tilde{\mu}\xi}.
\end{array}
\end{eqnarray}

Para el segundo sistema, consideremos nuevamente $\overline{\tau}_{1}\left(n\right)+r_{1}$, sin p\'erdida de generalidad podemos suponer que existe $m>0$ tal que $\overline{\tau}_{3}\left(m\right)<\overline{\tau}_{1}+r_{1}<\tau_{4}\left(m\right)$, entonces

\begin{eqnarray}
\prob\left\{0 \textrm{ arribos en }Q_{3}\textrm{ en el intervalo }\left[\overline{\tau}_{3}\left(m\right),\overline{\tau}_{1}\left(n\right)+r_{1}\right]\right\}
=e^{-\tilde{\mu}_{3}\xi_{3}},
\end{eqnarray}
donde 
\begin{eqnarray}
\xi_{3}=\overline{\tau}_{1}\left(n\right)+r_{1}-\overline{\tau}_{3}\left(m\right)=
\overline{\tau}_{1}\left(n\right)-\overline{\tau}_{3}\left(m\right)+r_{1},
\end{eqnarray}

mientras que para $Q_{4}$ al igual que con $Q_{2}$ escribiremos $\tau_{4}\left(m\right)$ en t\'erminos de $\overline{\tau}_{4}\left(m-1\right)$:

$\varphi_{2}\equiv\tau_{4}\left(m\right)=\overline{\tau}_{4}\left(m-1\right)+r_{4}+\overline{\tau}_{3}\left(m\right)
-\tau_{3}\left(m\right)+r_{3}=\overline{\tau}_{4}\left(m-1\right)+\overline{\tau}_{3}\left(m\right)
-\tau_{3}\left(m\right)+\hat{r}$, adem\'as,

$\xi_{2}\equiv\varphi_{2}\left(m\right)-\overline{\tau}_{1}-r_{1}=\overline{\tau}_{4}\left(m-1\right)+\overline{\tau}_{3}\left(m\right)
-\tau_{3}\left(m\right)-\overline{\tau}_{1}\left(n\right)+\hat{r}-r_{1}$. 

Entonces


\begin{eqnarray}
\prob\left\{0 \textrm{ arribos en }Q_{4}\textrm{ en el intervalo }\left[\overline{\tau}_{1}\left(n\right)+r_{1},\varphi_{2}\left(m\right)\right]\right\}
=e^{-\tilde{\mu}_{4}\xi_{2}},
\end{eqnarray}

mientras que para $Q_{3}$ se tiene que 

\begin{eqnarray}
\prob\left\{0 \textrm{ arribos en }Q_{3}\textrm{ en el intervalo }\left[\overline{\tau}_{1}\left(n\right)+r_{1},\varphi_{2}\left(m\right)\right]\right\}
=e^{-\tilde{\mu}_{3}\xi_{2}}
\end{eqnarray}

Por tanto

\begin{eqnarray}
\prob\left\{0 \textrm{ arribos en }Q_{3}\wedge Q_{4}\textrm{ en el intervalo }\left[\overline{\tau}_{1}\left(n\right)+r_{1},\varphi_{2}\left(m\right)\right]\right\}
=e^{-\hat{\mu}\xi_{2}}
\end{eqnarray}
donde $\hat{\mu}=\tilde{\mu}_{3}+\tilde{\mu}_{4}$.

Ahora, definamos los intervalos $\mathcal{I}_{1}=\left[\overline{\tau}_{1}\left(n\right)+r_{1},\varphi_{1}\left(n\right)\right]$  y $\mathcal{I}_{2}=\left[\overline{\tau}_{1}\left(n\right)+r_{1},\varphi_{2}\left(m\right)\right]$, entonces, sea $\mathcal{I}=\mathcal{I}_{1}\cap\mathcal{I}_{2}$ el intervalo donde cada una de las colas se encuentran vac\'ias, es decir, si tomamos $T^{*}\in\mathcal{I}$, entonces  $L_{1}\left(T^{*}\right)=L_{2}\left(T^{*}\right)=L_{3}\left(T^{*}\right)=L_{4}\left(T^{*}\right)=0$.

Ahora, dado que por construcci\'on $\mathcal{I}\neq\emptyset$ y que para $T^{*}\in\mathcal{I}$ en ninguna de las colas han llegado usuarios, se tiene que no hay transferencia entre las colas, por lo tanto, el sistema 1 y el sistema 2 son condicionalmente independientes en $\mathcal{I}$, es decir

\begin{eqnarray}
\prob\left\{L_{1}\left(T^{*}\right),L_{2}\left(T^{*}\right),L_{3}\left(T^{*}\right),L_{4}\left(T^{*}\right)|T^{*}\in\mathcal{I}\right\}=\prod_{j=1}^{4}\prob\left\{L_{j}\left(T^{*}\right)\right\},
\end{eqnarray}

para $T^{*}\in\mathcal{I}$. 

%\newpage























%________________________________________________________________________
%\section{Procesos Regenerativos}
%________________________________________________________________________

%________________________________________________________________________
%\subsection*{Procesos Regenerativos Sigman, Thorisson y Wolff \cite{Sigman1}}
%________________________________________________________________________


\begin{Def}[Definici\'on Cl\'asica]
Un proceso estoc\'astico $X=\left\{X\left(t\right):t\geq0\right\}$ es llamado regenerativo is existe una variable aleatoria $R_{1}>0$ tal que
\begin{itemize}
\item[i)] $\left\{X\left(t+R_{1}\right):t\geq0\right\}$ es independiente de $\left\{\left\{X\left(t\right):t<R_{1}\right\},\right\}$
\item[ii)] $\left\{X\left(t+R_{1}\right):t\geq0\right\}$ es estoc\'asticamente equivalente a $\left\{X\left(t\right):t>0\right\}$
\end{itemize}

Llamamos a $R_{1}$ tiempo de regeneraci\'on, y decimos que $X$ se regenera en este punto.
\end{Def}

$\left\{X\left(t+R_{1}\right)\right\}$ es regenerativo con tiempo de regeneraci\'on $R_{2}$, independiente de $R_{1}$ pero con la misma distribuci\'on que $R_{1}$. Procediendo de esta manera se obtiene una secuencia de variables aleatorias independientes e id\'enticamente distribuidas $\left\{R_{n}\right\}$ llamados longitudes de ciclo. Si definimos a $Z_{k}\equiv R_{1}+R_{2}+\cdots+R_{k}$, se tiene un proceso de renovaci\'on llamado proceso de renovaci\'on encajado para $X$.


\begin{Note}
La existencia de un primer tiempo de regeneraci\'on, $R_{1}$, implica la existencia de una sucesi\'on completa de estos tiempos $R_{1},R_{2}\ldots,$ que satisfacen la propiedad deseada \cite{Sigman2}.
\end{Note}


\begin{Note} Para la cola $GI/GI/1$ los usuarios arriban con tiempos $t_{n}$ y son atendidos con tiempos de servicio $S_{n}$, los tiempos de arribo forman un proceso de renovaci\'on  con tiempos entre arribos independientes e identicamente distribuidos (\texttt{i.i.d.})$T_{n}=t_{n}-t_{n-1}$, adem\'as los tiempos de servicio son \texttt{i.i.d.} e independientes de los procesos de arribo. Por \textit{estable} se entiende que $\esp S_{n}<\esp T_{n}<\infty$.
\end{Note}
 


\begin{Def}
Para $x$ fijo y para cada $t\geq0$, sea $I_{x}\left(t\right)=1$ si $X\left(t\right)\leq x$,  $I_{x}\left(t\right)=0$ en caso contrario, y def\'inanse los tiempos promedio
\begin{eqnarray*}
\overline{X}&=&lim_{t\rightarrow\infty}\frac{1}{t}\int_{0}^{\infty}X\left(u\right)du\\
\prob\left(X_{\infty}\leq x\right)&=&lim_{t\rightarrow\infty}\frac{1}{t}\int_{0}^{\infty}I_{x}\left(u\right)du,
\end{eqnarray*}
cuando estos l\'imites existan.
\end{Def}

Como consecuencia del teorema de Renovaci\'on-Recompensa, se tiene que el primer l\'imite  existe y es igual a la constante
\begin{eqnarray*}
\overline{X}&=&\frac{\esp\left[\int_{0}^{R_{1}}X\left(t\right)dt\right]}{\esp\left[R_{1}\right]},
\end{eqnarray*}
suponiendo que ambas esperanzas son finitas.
 
\begin{Note}
Funciones de procesos regenerativos son regenerativas, es decir, si $X\left(t\right)$ es regenerativo y se define el proceso $Y\left(t\right)$ por $Y\left(t\right)=f\left(X\left(t\right)\right)$ para alguna funci\'on Borel medible $f\left(\cdot\right)$. Adem\'as $Y$ es regenerativo con los mismos tiempos de renovaci\'on que $X$. 

En general, los tiempos de renovaci\'on, $Z_{k}$ de un proceso regenerativo no requieren ser tiempos de paro con respecto a la evoluci\'on de $X\left(t\right)$.
\end{Note} 

\begin{Note}
Una funci\'on de un proceso de Markov, usualmente no ser\'a un proceso de Markov, sin embargo ser\'a regenerativo si el proceso de Markov lo es.
\end{Note}

 
\begin{Note}
Un proceso regenerativo con media de la longitud de ciclo finita es llamado positivo recurrente.
\end{Note}


\begin{Note}
\begin{itemize}
\item[a)] Si el proceso regenerativo $X$ es positivo recurrente y tiene trayectorias muestrales no negativas, entonces la ecuaci\'on anterior es v\'alida.
\item[b)] Si $X$ es positivo recurrente regenerativo, podemos construir una \'unica versi\'on estacionaria de este proceso, $X_{e}=\left\{X_{e}\left(t\right)\right\}$, donde $X_{e}$ es un proceso estoc\'astico regenerativo y estrictamente estacionario, con distribuci\'on marginal distribuida como $X_{\infty}$
\end{itemize}
\end{Note}


%__________________________________________________________________________________________
%\subsection*{Procesos Regenerativos Estacionarios - Stidham \cite{Stidham}}
%__________________________________________________________________________________________


Un proceso estoc\'astico a tiempo continuo $\left\{V\left(t\right),t\geq0\right\}$ es un proceso regenerativo si existe una sucesi\'on de variables aleatorias independientes e id\'enticamente distribuidas $\left\{X_{1},X_{2},\ldots\right\}$, sucesi\'on de renovaci\'on, tal que para cualquier conjunto de Borel $A$, 

\begin{eqnarray*}
\prob\left\{V\left(t\right)\in A|X_{1}+X_{2}+\cdots+X_{R\left(t\right)}=s,\left\{V\left(\tau\right),\tau<s\right\}\right\}=\prob\left\{V\left(t-s\right)\in A|X_{1}>t-s\right\},
\end{eqnarray*}
para todo $0\leq s\leq t$, donde $R\left(t\right)=\max\left\{X_{1}+X_{2}+\cdots+X_{j}\leq t\right\}=$n\'umero de renovaciones ({\emph{puntos de regeneraci\'on}}) que ocurren en $\left[0,t\right]$. El intervalo $\left[0,X_{1}\right)$ es llamado {\emph{primer ciclo de regeneraci\'on}} de $\left\{V\left(t \right),t\geq0\right\}$, $\left[X_{1},X_{1}+X_{2}\right)$ el {\emph{segundo ciclo de regeneraci\'on}}, y as\'i sucesivamente.

Sea $X=X_{1}$ y sea $F$ la funci\'on de distrbuci\'on de $X$


\begin{Def}
Se define el proceso estacionario, $\left\{V^{*}\left(t\right),t\geq0\right\}$, para $\left\{V\left(t\right),t\geq0\right\}$ por

\begin{eqnarray*}
\prob\left\{V\left(t\right)\in A\right\}=\frac{1}{\esp\left[X\right]}\int_{0}^{\infty}\prob\left\{V\left(t+x\right)\in A|X>x\right\}\left(1-F\left(x\right)\right)dx,
\end{eqnarray*} 
para todo $t\geq0$ y todo conjunto de Borel $A$.
\end{Def}

\begin{Def}
Una distribuci\'on se dice que es {\emph{aritm\'etica}} si todos sus puntos de incremento son m\'ultiplos de la forma $0,\lambda, 2\lambda,\ldots$ para alguna $\lambda>0$ entera.
\end{Def}


\begin{Def}
Una modificaci\'on medible de un proceso $\left\{V\left(t\right),t\geq0\right\}$, es una versi\'on de este, $\left\{V\left(t,w\right)\right\}$ conjuntamente medible para $t\geq0$ y para $w\in S$, $S$ espacio de estados para $\left\{V\left(t\right),t\geq0\right\}$.
\end{Def}

\begin{Teo}
Sea $\left\{V\left(t\right),t\geq\right\}$ un proceso regenerativo no negativo con modificaci\'on medible. Sea $\esp\left[X\right]<\infty$. Entonces el proceso estacionario dado por la ecuaci\'on anterior est\'a bien definido y tiene funci\'on de distribuci\'on independiente de $t$, adem\'as
\begin{itemize}
\item[i)] \begin{eqnarray*}
\esp\left[V^{*}\left(0\right)\right]&=&\frac{\esp\left[\int_{0}^{X}V\left(s\right)ds\right]}{\esp\left[X\right]}\end{eqnarray*}
\item[ii)] Si $\esp\left[V^{*}\left(0\right)\right]<\infty$, equivalentemente, si $\esp\left[\int_{0}^{X}V\left(s\right)ds\right]<\infty$,entonces
\begin{eqnarray*}
\frac{\int_{0}^{t}V\left(s\right)ds}{t}\rightarrow\frac{\esp\left[\int_{0}^{X}V\left(s\right)ds\right]}{\esp\left[X\right]}
\end{eqnarray*}
con probabilidad 1 y en media, cuando $t\rightarrow\infty$.
\end{itemize}
\end{Teo}

\begin{Coro}
Sea $\left\{V\left(t\right),t\geq0\right\}$ un proceso regenerativo no negativo, con modificaci\'on medible. Si $\esp <\infty$, $F$ es no-aritm\'etica, y para todo $x\geq0$, $P\left\{V\left(t\right)\leq x,C>x\right\}$ es de variaci\'on acotada como funci\'on de $t$ en cada intervalo finito $\left[0,\tau\right]$, entonces $V\left(t\right)$ converge en distribuci\'on  cuando $t\rightarrow\infty$ y $$\esp V=\frac{\esp \int_{0}^{X}V\left(s\right)ds}{\esp X}$$
Donde $V$ tiene la distribuci\'on l\'imite de $V\left(t\right)$ cuando $t\rightarrow\infty$.

\end{Coro}

Para el caso discreto se tienen resultados similares.



%______________________________________________________________________
%\section{Procesos de Renovaci\'on}
%______________________________________________________________________

\begin{Def}\label{Def.Tn}
Sean $0\leq T_{1}\leq T_{2}\leq \ldots$ son tiempos aleatorios infinitos en los cuales ocurren ciertos eventos. El n\'umero de tiempos $T_{n}$ en el intervalo $\left[0,t\right)$ es

\begin{eqnarray}
N\left(t\right)=\sum_{n=1}^{\infty}\indora\left(T_{n}\leq t\right),
\end{eqnarray}
para $t\geq0$.
\end{Def}

Si se consideran los puntos $T_{n}$ como elementos de $\rea_{+}$, y $N\left(t\right)$ es el n\'umero de puntos en $\rea$. El proceso denotado por $\left\{N\left(t\right):t\geq0\right\}$, denotado por $N\left(t\right)$, es un proceso puntual en $\rea_{+}$. Los $T_{n}$ son los tiempos de ocurrencia, el proceso puntual $N\left(t\right)$ es simple si su n\'umero de ocurrencias son distintas: $0<T_{1}<T_{2}<\ldots$ casi seguramente.

\begin{Def}
Un proceso puntual $N\left(t\right)$ es un proceso de renovaci\'on si los tiempos de interocurrencia $\xi_{n}=T_{n}-T_{n-1}$, para $n\geq1$, son independientes e identicamente distribuidos con distribuci\'on $F$, donde $F\left(0\right)=0$ y $T_{0}=0$. Los $T_{n}$ son llamados tiempos de renovaci\'on, referente a la independencia o renovaci\'on de la informaci\'on estoc\'astica en estos tiempos. Los $\xi_{n}$ son los tiempos de inter-renovaci\'on, y $N\left(t\right)$ es el n\'umero de renovaciones en el intervalo $\left[0,t\right)$
\end{Def}


\begin{Note}
Para definir un proceso de renovaci\'on para cualquier contexto, solamente hay que especificar una distribuci\'on $F$, con $F\left(0\right)=0$, para los tiempos de inter-renovaci\'on. La funci\'on $F$ en turno degune las otra variables aleatorias. De manera formal, existe un espacio de probabilidad y una sucesi\'on de variables aleatorias $\xi_{1},\xi_{2},\ldots$ definidas en este con distribuci\'on $F$. Entonces las otras cantidades son $T_{n}=\sum_{k=1}^{n}\xi_{k}$ y $N\left(t\right)=\sum_{n=1}^{\infty}\indora\left(T_{n}\leq t\right)$, donde $T_{n}\rightarrow\infty$ casi seguramente por la Ley Fuerte de los Grandes Números.
\end{Note}

%___________________________________________________________________________________________
%
%\subsection*{Teorema Principal de Renovaci\'on}
%___________________________________________________________________________________________
%

\begin{Note} Una funci\'on $h:\rea_{+}\rightarrow\rea$ es Directamente Riemann Integrable en los siguientes casos:
\begin{itemize}
\item[a)] $h\left(t\right)\geq0$ es decreciente y Riemann Integrable.
\item[b)] $h$ es continua excepto posiblemente en un conjunto de Lebesgue de medida 0, y $|h\left(t\right)|\leq b\left(t\right)$, donde $b$ es DRI.
\end{itemize}
\end{Note}

\begin{Teo}[Teorema Principal de Renovaci\'on]
Si $F$ es no aritm\'etica y $h\left(t\right)$ es Directamente Riemann Integrable (DRI), entonces

\begin{eqnarray*}
lim_{t\rightarrow\infty}U\star h=\frac{1}{\mu}\int_{\rea_{+}}h\left(s\right)ds.
\end{eqnarray*}
\end{Teo}

\begin{Prop}
Cualquier funci\'on $H\left(t\right)$ acotada en intervalos finitos y que es 0 para $t<0$ puede expresarse como
\begin{eqnarray*}
H\left(t\right)=U\star h\left(t\right)\textrm{,  donde }h\left(t\right)=H\left(t\right)-F\star H\left(t\right)
\end{eqnarray*}
\end{Prop}

\begin{Def}
Un proceso estoc\'astico $X\left(t\right)$ es crudamente regenerativo en un tiempo aleatorio positivo $T$ si
\begin{eqnarray*}
\esp\left[X\left(T+t\right)|T\right]=\esp\left[X\left(t\right)\right]\textrm{, para }t\geq0,\end{eqnarray*}
y con las esperanzas anteriores finitas.
\end{Def}

\begin{Prop}
Sup\'ongase que $X\left(t\right)$ es un proceso crudamente regenerativo en $T$, que tiene distribuci\'on $F$. Si $\esp\left[X\left(t\right)\right]$ es acotado en intervalos finitos, entonces
\begin{eqnarray*}
\esp\left[X\left(t\right)\right]=U\star h\left(t\right)\textrm{,  donde }h\left(t\right)=\esp\left[X\left(t\right)\indora\left(T>t\right)\right].
\end{eqnarray*}
\end{Prop}

\begin{Teo}[Regeneraci\'on Cruda]
Sup\'ongase que $X\left(t\right)$ es un proceso con valores positivo crudamente regenerativo en $T$, y def\'inase $M=\sup\left\{|X\left(t\right)|:t\leq T\right\}$. Si $T$ es no aritm\'etico y $M$ y $MT$ tienen media finita, entonces
\begin{eqnarray*}
lim_{t\rightarrow\infty}\esp\left[X\left(t\right)\right]=\frac{1}{\mu}\int_{\rea_{+}}h\left(s\right)ds,
\end{eqnarray*}
donde $h\left(t\right)=\esp\left[X\left(t\right)\indora\left(T>t\right)\right]$.
\end{Teo}

%___________________________________________________________________________________________
%
%\subsection*{Propiedades de los Procesos de Renovaci\'on}
%___________________________________________________________________________________________
%

Los tiempos $T_{n}$ est\'an relacionados con los conteos de $N\left(t\right)$ por

\begin{eqnarray*}
\left\{N\left(t\right)\geq n\right\}&=&\left\{T_{n}\leq t\right\}\\
T_{N\left(t\right)}\leq &t&<T_{N\left(t\right)+1},
\end{eqnarray*}

adem\'as $N\left(T_{n}\right)=n$, y 

\begin{eqnarray*}
N\left(t\right)=\max\left\{n:T_{n}\leq t\right\}=\min\left\{n:T_{n+1}>t\right\}
\end{eqnarray*}

Por propiedades de la convoluci\'on se sabe que

\begin{eqnarray*}
P\left\{T_{n}\leq t\right\}=F^{n\star}\left(t\right)
\end{eqnarray*}
que es la $n$-\'esima convoluci\'on de $F$. Entonces 

\begin{eqnarray*}
\left\{N\left(t\right)\geq n\right\}&=&\left\{T_{n}\leq t\right\}\\
P\left\{N\left(t\right)\leq n\right\}&=&1-F^{\left(n+1\right)\star}\left(t\right)
\end{eqnarray*}

Adem\'as usando el hecho de que $\esp\left[N\left(t\right)\right]=\sum_{n=1}^{\infty}P\left\{N\left(t\right)\geq n\right\}$
se tiene que

\begin{eqnarray*}
\esp\left[N\left(t\right)\right]=\sum_{n=1}^{\infty}F^{n\star}\left(t\right)
\end{eqnarray*}

\begin{Prop}
Para cada $t\geq0$, la funci\'on generadora de momentos $\esp\left[e^{\alpha N\left(t\right)}\right]$ existe para alguna $\alpha$ en una vecindad del 0, y de aqu\'i que $\esp\left[N\left(t\right)^{m}\right]<\infty$, para $m\geq1$.
\end{Prop}


\begin{Note}
Si el primer tiempo de renovaci\'on $\xi_{1}$ no tiene la misma distribuci\'on que el resto de las $\xi_{n}$, para $n\geq2$, a $N\left(t\right)$ se le llama Proceso de Renovaci\'on retardado, donde si $\xi$ tiene distribuci\'on $G$, entonces el tiempo $T_{n}$ de la $n$-\'esima renovaci\'on tiene distribuci\'on $G\star F^{\left(n-1\right)\star}\left(t\right)$
\end{Note}


\begin{Teo}
Para una constante $\mu\leq\infty$ ( o variable aleatoria), las siguientes expresiones son equivalentes:

\begin{eqnarray}
lim_{n\rightarrow\infty}n^{-1}T_{n}&=&\mu,\textrm{ c.s.}\\
lim_{t\rightarrow\infty}t^{-1}N\left(t\right)&=&1/\mu,\textrm{ c.s.}
\end{eqnarray}
\end{Teo}


Es decir, $T_{n}$ satisface la Ley Fuerte de los Grandes N\'umeros s\'i y s\'olo s\'i $N\left/t\right)$ la cumple.


\begin{Coro}[Ley Fuerte de los Grandes N\'umeros para Procesos de Renovaci\'on]
Si $N\left(t\right)$ es un proceso de renovaci\'on cuyos tiempos de inter-renovaci\'on tienen media $\mu\leq\infty$, entonces
\begin{eqnarray}
t^{-1}N\left(t\right)\rightarrow 1/\mu,\textrm{ c.s. cuando }t\rightarrow\infty.
\end{eqnarray}

\end{Coro}


Considerar el proceso estoc\'astico de valores reales $\left\{Z\left(t\right):t\geq0\right\}$ en el mismo espacio de probabilidad que $N\left(t\right)$

\begin{Def}
Para el proceso $\left\{Z\left(t\right):t\geq0\right\}$ se define la fluctuaci\'on m\'axima de $Z\left(t\right)$ en el intervalo $\left(T_{n-1},T_{n}\right]$:
\begin{eqnarray*}
M_{n}=\sup_{T_{n-1}<t\leq T_{n}}|Z\left(t\right)-Z\left(T_{n-1}\right)|
\end{eqnarray*}
\end{Def}

\begin{Teo}
Sup\'ongase que $n^{-1}T_{n}\rightarrow\mu$ c.s. cuando $n\rightarrow\infty$, donde $\mu\leq\infty$ es una constante o variable aleatoria. Sea $a$ una constante o variable aleatoria que puede ser infinita cuando $\mu$ es finita, y considere las expresiones l\'imite:
\begin{eqnarray}
lim_{n\rightarrow\infty}n^{-1}Z\left(T_{n}\right)&=&a,\textrm{ c.s.}\\
lim_{t\rightarrow\infty}t^{-1}Z\left(t\right)&=&a/\mu,\textrm{ c.s.}
\end{eqnarray}
La segunda expresi\'on implica la primera. Conversamente, la primera implica la segunda si el proceso $Z\left(t\right)$ es creciente, o si $lim_{n\rightarrow\infty}n^{-1}M_{n}=0$ c.s.
\end{Teo}

\begin{Coro}
Si $N\left(t\right)$ es un proceso de renovaci\'on, y $\left(Z\left(T_{n}\right)-Z\left(T_{n-1}\right),M_{n}\right)$, para $n\geq1$, son variables aleatorias independientes e id\'enticamente distribuidas con media finita, entonces,
\begin{eqnarray}
lim_{t\rightarrow\infty}t^{-1}Z\left(t\right)\rightarrow\frac{\esp\left[Z\left(T_{1}\right)-Z\left(T_{0}\right)\right]}{\esp\left[T_{1}\right]},\textrm{ c.s. cuando  }t\rightarrow\infty.
\end{eqnarray}
\end{Coro}



%___________________________________________________________________________________________
%
%\subsection{Propiedades de los Procesos de Renovaci\'on}
%___________________________________________________________________________________________
%

Los tiempos $T_{n}$ est\'an relacionados con los conteos de $N\left(t\right)$ por

\begin{eqnarray*}
\left\{N\left(t\right)\geq n\right\}&=&\left\{T_{n}\leq t\right\}\\
T_{N\left(t\right)}\leq &t&<T_{N\left(t\right)+1},
\end{eqnarray*}

adem\'as $N\left(T_{n}\right)=n$, y 

\begin{eqnarray*}
N\left(t\right)=\max\left\{n:T_{n}\leq t\right\}=\min\left\{n:T_{n+1}>t\right\}
\end{eqnarray*}

Por propiedades de la convoluci\'on se sabe que

\begin{eqnarray*}
P\left\{T_{n}\leq t\right\}=F^{n\star}\left(t\right)
\end{eqnarray*}
que es la $n$-\'esima convoluci\'on de $F$. Entonces 

\begin{eqnarray*}
\left\{N\left(t\right)\geq n\right\}&=&\left\{T_{n}\leq t\right\}\\
P\left\{N\left(t\right)\leq n\right\}&=&1-F^{\left(n+1\right)\star}\left(t\right)
\end{eqnarray*}

Adem\'as usando el hecho de que $\esp\left[N\left(t\right)\right]=\sum_{n=1}^{\infty}P\left\{N\left(t\right)\geq n\right\}$
se tiene que

\begin{eqnarray*}
\esp\left[N\left(t\right)\right]=\sum_{n=1}^{\infty}F^{n\star}\left(t\right)
\end{eqnarray*}

\begin{Prop}
Para cada $t\geq0$, la funci\'on generadora de momentos $\esp\left[e^{\alpha N\left(t\right)}\right]$ existe para alguna $\alpha$ en una vecindad del 0, y de aqu\'i que $\esp\left[N\left(t\right)^{m}\right]<\infty$, para $m\geq1$.
\end{Prop}


\begin{Note}
Si el primer tiempo de renovaci\'on $\xi_{1}$ no tiene la misma distribuci\'on que el resto de las $\xi_{n}$, para $n\geq2$, a $N\left(t\right)$ se le llama Proceso de Renovaci\'on retardado, donde si $\xi$ tiene distribuci\'on $G$, entonces el tiempo $T_{n}$ de la $n$-\'esima renovaci\'on tiene distribuci\'on $G\star F^{\left(n-1\right)\star}\left(t\right)$
\end{Note}


\begin{Teo}
Para una constante $\mu\leq\infty$ ( o variable aleatoria), las siguientes expresiones son equivalentes:

\begin{eqnarray}
lim_{n\rightarrow\infty}n^{-1}T_{n}&=&\mu,\textrm{ c.s.}\\
lim_{t\rightarrow\infty}t^{-1}N\left(t\right)&=&1/\mu,\textrm{ c.s.}
\end{eqnarray}
\end{Teo}


Es decir, $T_{n}$ satisface la Ley Fuerte de los Grandes N\'umeros s\'i y s\'olo s\'i $N\left/t\right)$ la cumple.


\begin{Coro}[Ley Fuerte de los Grandes N\'umeros para Procesos de Renovaci\'on]
Si $N\left(t\right)$ es un proceso de renovaci\'on cuyos tiempos de inter-renovaci\'on tienen media $\mu\leq\infty$, entonces
\begin{eqnarray}
t^{-1}N\left(t\right)\rightarrow 1/\mu,\textrm{ c.s. cuando }t\rightarrow\infty.
\end{eqnarray}

\end{Coro}


Considerar el proceso estoc\'astico de valores reales $\left\{Z\left(t\right):t\geq0\right\}$ en el mismo espacio de probabilidad que $N\left(t\right)$

\begin{Def}
Para el proceso $\left\{Z\left(t\right):t\geq0\right\}$ se define la fluctuaci\'on m\'axima de $Z\left(t\right)$ en el intervalo $\left(T_{n-1},T_{n}\right]$:
\begin{eqnarray*}
M_{n}=\sup_{T_{n-1}<t\leq T_{n}}|Z\left(t\right)-Z\left(T_{n-1}\right)|
\end{eqnarray*}
\end{Def}

\begin{Teo}
Sup\'ongase que $n^{-1}T_{n}\rightarrow\mu$ c.s. cuando $n\rightarrow\infty$, donde $\mu\leq\infty$ es una constante o variable aleatoria. Sea $a$ una constante o variable aleatoria que puede ser infinita cuando $\mu$ es finita, y considere las expresiones l\'imite:
\begin{eqnarray}
lim_{n\rightarrow\infty}n^{-1}Z\left(T_{n}\right)&=&a,\textrm{ c.s.}\\
lim_{t\rightarrow\infty}t^{-1}Z\left(t\right)&=&a/\mu,\textrm{ c.s.}
\end{eqnarray}
La segunda expresi\'on implica la primera. Conversamente, la primera implica la segunda si el proceso $Z\left(t\right)$ es creciente, o si $lim_{n\rightarrow\infty}n^{-1}M_{n}=0$ c.s.
\end{Teo}

\begin{Coro}
Si $N\left(t\right)$ es un proceso de renovaci\'on, y $\left(Z\left(T_{n}\right)-Z\left(T_{n-1}\right),M_{n}\right)$, para $n\geq1$, son variables aleatorias independientes e id\'enticamente distribuidas con media finita, entonces,
\begin{eqnarray}
lim_{t\rightarrow\infty}t^{-1}Z\left(t\right)\rightarrow\frac{\esp\left[Z\left(T_{1}\right)-Z\left(T_{0}\right)\right]}{\esp\left[T_{1}\right]},\textrm{ c.s. cuando  }t\rightarrow\infty.
\end{eqnarray}
\end{Coro}


%___________________________________________________________________________________________
%
%\subsection{Propiedades de los Procesos de Renovaci\'on}
%___________________________________________________________________________________________
%

Los tiempos $T_{n}$ est\'an relacionados con los conteos de $N\left(t\right)$ por

\begin{eqnarray*}
\left\{N\left(t\right)\geq n\right\}&=&\left\{T_{n}\leq t\right\}\\
T_{N\left(t\right)}\leq &t&<T_{N\left(t\right)+1},
\end{eqnarray*}

adem\'as $N\left(T_{n}\right)=n$, y 

\begin{eqnarray*}
N\left(t\right)=\max\left\{n:T_{n}\leq t\right\}=\min\left\{n:T_{n+1}>t\right\}
\end{eqnarray*}

Por propiedades de la convoluci\'on se sabe que

\begin{eqnarray*}
P\left\{T_{n}\leq t\right\}=F^{n\star}\left(t\right)
\end{eqnarray*}
que es la $n$-\'esima convoluci\'on de $F$. Entonces 

\begin{eqnarray*}
\left\{N\left(t\right)\geq n\right\}&=&\left\{T_{n}\leq t\right\}\\
P\left\{N\left(t\right)\leq n\right\}&=&1-F^{\left(n+1\right)\star}\left(t\right)
\end{eqnarray*}

Adem\'as usando el hecho de que $\esp\left[N\left(t\right)\right]=\sum_{n=1}^{\infty}P\left\{N\left(t\right)\geq n\right\}$
se tiene que

\begin{eqnarray*}
\esp\left[N\left(t\right)\right]=\sum_{n=1}^{\infty}F^{n\star}\left(t\right)
\end{eqnarray*}

\begin{Prop}
Para cada $t\geq0$, la funci\'on generadora de momentos $\esp\left[e^{\alpha N\left(t\right)}\right]$ existe para alguna $\alpha$ en una vecindad del 0, y de aqu\'i que $\esp\left[N\left(t\right)^{m}\right]<\infty$, para $m\geq1$.
\end{Prop}


\begin{Note}
Si el primer tiempo de renovaci\'on $\xi_{1}$ no tiene la misma distribuci\'on que el resto de las $\xi_{n}$, para $n\geq2$, a $N\left(t\right)$ se le llama Proceso de Renovaci\'on retardado, donde si $\xi$ tiene distribuci\'on $G$, entonces el tiempo $T_{n}$ de la $n$-\'esima renovaci\'on tiene distribuci\'on $G\star F^{\left(n-1\right)\star}\left(t\right)$
\end{Note}


\begin{Teo}
Para una constante $\mu\leq\infty$ ( o variable aleatoria), las siguientes expresiones son equivalentes:

\begin{eqnarray}
lim_{n\rightarrow\infty}n^{-1}T_{n}&=&\mu,\textrm{ c.s.}\\
lim_{t\rightarrow\infty}t^{-1}N\left(t\right)&=&1/\mu,\textrm{ c.s.}
\end{eqnarray}
\end{Teo}


Es decir, $T_{n}$ satisface la Ley Fuerte de los Grandes N\'umeros s\'i y s\'olo s\'i $N\left/t\right)$ la cumple.


\begin{Coro}[Ley Fuerte de los Grandes N\'umeros para Procesos de Renovaci\'on]
Si $N\left(t\right)$ es un proceso de renovaci\'on cuyos tiempos de inter-renovaci\'on tienen media $\mu\leq\infty$, entonces
\begin{eqnarray}
t^{-1}N\left(t\right)\rightarrow 1/\mu,\textrm{ c.s. cuando }t\rightarrow\infty.
\end{eqnarray}

\end{Coro}


Considerar el proceso estoc\'astico de valores reales $\left\{Z\left(t\right):t\geq0\right\}$ en el mismo espacio de probabilidad que $N\left(t\right)$

\begin{Def}
Para el proceso $\left\{Z\left(t\right):t\geq0\right\}$ se define la fluctuaci\'on m\'axima de $Z\left(t\right)$ en el intervalo $\left(T_{n-1},T_{n}\right]$:
\begin{eqnarray*}
M_{n}=\sup_{T_{n-1}<t\leq T_{n}}|Z\left(t\right)-Z\left(T_{n-1}\right)|
\end{eqnarray*}
\end{Def}

\begin{Teo}
Sup\'ongase que $n^{-1}T_{n}\rightarrow\mu$ c.s. cuando $n\rightarrow\infty$, donde $\mu\leq\infty$ es una constante o variable aleatoria. Sea $a$ una constante o variable aleatoria que puede ser infinita cuando $\mu$ es finita, y considere las expresiones l\'imite:
\begin{eqnarray}
lim_{n\rightarrow\infty}n^{-1}Z\left(T_{n}\right)&=&a,\textrm{ c.s.}\\
lim_{t\rightarrow\infty}t^{-1}Z\left(t\right)&=&a/\mu,\textrm{ c.s.}
\end{eqnarray}
La segunda expresi\'on implica la primera. Conversamente, la primera implica la segunda si el proceso $Z\left(t\right)$ es creciente, o si $lim_{n\rightarrow\infty}n^{-1}M_{n}=0$ c.s.
\end{Teo}

\begin{Coro}
Si $N\left(t\right)$ es un proceso de renovaci\'on, y $\left(Z\left(T_{n}\right)-Z\left(T_{n-1}\right),M_{n}\right)$, para $n\geq1$, son variables aleatorias independientes e id\'enticamente distribuidas con media finita, entonces,
\begin{eqnarray}
lim_{t\rightarrow\infty}t^{-1}Z\left(t\right)\rightarrow\frac{\esp\left[Z\left(T_{1}\right)-Z\left(T_{0}\right)\right]}{\esp\left[T_{1}\right]},\textrm{ c.s. cuando  }t\rightarrow\infty.
\end{eqnarray}
\end{Coro}

%___________________________________________________________________________________________
%
%\subsection{Propiedades de los Procesos de Renovaci\'on}
%___________________________________________________________________________________________
%

Los tiempos $T_{n}$ est\'an relacionados con los conteos de $N\left(t\right)$ por

\begin{eqnarray*}
\left\{N\left(t\right)\geq n\right\}&=&\left\{T_{n}\leq t\right\}\\
T_{N\left(t\right)}\leq &t&<T_{N\left(t\right)+1},
\end{eqnarray*}

adem\'as $N\left(T_{n}\right)=n$, y 

\begin{eqnarray*}
N\left(t\right)=\max\left\{n:T_{n}\leq t\right\}=\min\left\{n:T_{n+1}>t\right\}
\end{eqnarray*}

Por propiedades de la convoluci\'on se sabe que

\begin{eqnarray*}
P\left\{T_{n}\leq t\right\}=F^{n\star}\left(t\right)
\end{eqnarray*}
que es la $n$-\'esima convoluci\'on de $F$. Entonces 

\begin{eqnarray*}
\left\{N\left(t\right)\geq n\right\}&=&\left\{T_{n}\leq t\right\}\\
P\left\{N\left(t\right)\leq n\right\}&=&1-F^{\left(n+1\right)\star}\left(t\right)
\end{eqnarray*}

Adem\'as usando el hecho de que $\esp\left[N\left(t\right)\right]=\sum_{n=1}^{\infty}P\left\{N\left(t\right)\geq n\right\}$
se tiene que

\begin{eqnarray*}
\esp\left[N\left(t\right)\right]=\sum_{n=1}^{\infty}F^{n\star}\left(t\right)
\end{eqnarray*}

\begin{Prop}
Para cada $t\geq0$, la funci\'on generadora de momentos $\esp\left[e^{\alpha N\left(t\right)}\right]$ existe para alguna $\alpha$ en una vecindad del 0, y de aqu\'i que $\esp\left[N\left(t\right)^{m}\right]<\infty$, para $m\geq1$.
\end{Prop}


\begin{Note}
Si el primer tiempo de renovaci\'on $\xi_{1}$ no tiene la misma distribuci\'on que el resto de las $\xi_{n}$, para $n\geq2$, a $N\left(t\right)$ se le llama Proceso de Renovaci\'on retardado, donde si $\xi$ tiene distribuci\'on $G$, entonces el tiempo $T_{n}$ de la $n$-\'esima renovaci\'on tiene distribuci\'on $G\star F^{\left(n-1\right)\star}\left(t\right)$
\end{Note}


\begin{Teo}
Para una constante $\mu\leq\infty$ ( o variable aleatoria), las siguientes expresiones son equivalentes:

\begin{eqnarray}
lim_{n\rightarrow\infty}n^{-1}T_{n}&=&\mu,\textrm{ c.s.}\\
lim_{t\rightarrow\infty}t^{-1}N\left(t\right)&=&1/\mu,\textrm{ c.s.}
\end{eqnarray}
\end{Teo}


Es decir, $T_{n}$ satisface la Ley Fuerte de los Grandes N\'umeros s\'i y s\'olo s\'i $N\left/t\right)$ la cumple.


\begin{Coro}[Ley Fuerte de los Grandes N\'umeros para Procesos de Renovaci\'on]
Si $N\left(t\right)$ es un proceso de renovaci\'on cuyos tiempos de inter-renovaci\'on tienen media $\mu\leq\infty$, entonces
\begin{eqnarray}
t^{-1}N\left(t\right)\rightarrow 1/\mu,\textrm{ c.s. cuando }t\rightarrow\infty.
\end{eqnarray}

\end{Coro}


Considerar el proceso estoc\'astico de valores reales $\left\{Z\left(t\right):t\geq0\right\}$ en el mismo espacio de probabilidad que $N\left(t\right)$

\begin{Def}
Para el proceso $\left\{Z\left(t\right):t\geq0\right\}$ se define la fluctuaci\'on m\'axima de $Z\left(t\right)$ en el intervalo $\left(T_{n-1},T_{n}\right]$:
\begin{eqnarray*}
M_{n}=\sup_{T_{n-1}<t\leq T_{n}}|Z\left(t\right)-Z\left(T_{n-1}\right)|
\end{eqnarray*}
\end{Def}

\begin{Teo}
Sup\'ongase que $n^{-1}T_{n}\rightarrow\mu$ c.s. cuando $n\rightarrow\infty$, donde $\mu\leq\infty$ es una constante o variable aleatoria. Sea $a$ una constante o variable aleatoria que puede ser infinita cuando $\mu$ es finita, y considere las expresiones l\'imite:
\begin{eqnarray}
lim_{n\rightarrow\infty}n^{-1}Z\left(T_{n}\right)&=&a,\textrm{ c.s.}\\
lim_{t\rightarrow\infty}t^{-1}Z\left(t\right)&=&a/\mu,\textrm{ c.s.}
\end{eqnarray}
La segunda expresi\'on implica la primera. Conversamente, la primera implica la segunda si el proceso $Z\left(t\right)$ es creciente, o si $lim_{n\rightarrow\infty}n^{-1}M_{n}=0$ c.s.
\end{Teo}

\begin{Coro}
Si $N\left(t\right)$ es un proceso de renovaci\'on, y $\left(Z\left(T_{n}\right)-Z\left(T_{n-1}\right),M_{n}\right)$, para $n\geq1$, son variables aleatorias independientes e id\'enticamente distribuidas con media finita, entonces,
\begin{eqnarray}
lim_{t\rightarrow\infty}t^{-1}Z\left(t\right)\rightarrow\frac{\esp\left[Z\left(T_{1}\right)-Z\left(T_{0}\right)\right]}{\esp\left[T_{1}\right]},\textrm{ c.s. cuando  }t\rightarrow\infty.
\end{eqnarray}
\end{Coro}
%___________________________________________________________________________________________
%
%\subsection{Propiedades de los Procesos de Renovaci\'on}
%___________________________________________________________________________________________
%

Los tiempos $T_{n}$ est\'an relacionados con los conteos de $N\left(t\right)$ por

\begin{eqnarray*}
\left\{N\left(t\right)\geq n\right\}&=&\left\{T_{n}\leq t\right\}\\
T_{N\left(t\right)}\leq &t&<T_{N\left(t\right)+1},
\end{eqnarray*}

adem\'as $N\left(T_{n}\right)=n$, y 

\begin{eqnarray*}
N\left(t\right)=\max\left\{n:T_{n}\leq t\right\}=\min\left\{n:T_{n+1}>t\right\}
\end{eqnarray*}

Por propiedades de la convoluci\'on se sabe que

\begin{eqnarray*}
P\left\{T_{n}\leq t\right\}=F^{n\star}\left(t\right)
\end{eqnarray*}
que es la $n$-\'esima convoluci\'on de $F$. Entonces 

\begin{eqnarray*}
\left\{N\left(t\right)\geq n\right\}&=&\left\{T_{n}\leq t\right\}\\
P\left\{N\left(t\right)\leq n\right\}&=&1-F^{\left(n+1\right)\star}\left(t\right)
\end{eqnarray*}

Adem\'as usando el hecho de que $\esp\left[N\left(t\right)\right]=\sum_{n=1}^{\infty}P\left\{N\left(t\right)\geq n\right\}$
se tiene que

\begin{eqnarray*}
\esp\left[N\left(t\right)\right]=\sum_{n=1}^{\infty}F^{n\star}\left(t\right)
\end{eqnarray*}

\begin{Prop}
Para cada $t\geq0$, la funci\'on generadora de momentos $\esp\left[e^{\alpha N\left(t\right)}\right]$ existe para alguna $\alpha$ en una vecindad del 0, y de aqu\'i que $\esp\left[N\left(t\right)^{m}\right]<\infty$, para $m\geq1$.
\end{Prop}


\begin{Note}
Si el primer tiempo de renovaci\'on $\xi_{1}$ no tiene la misma distribuci\'on que el resto de las $\xi_{n}$, para $n\geq2$, a $N\left(t\right)$ se le llama Proceso de Renovaci\'on retardado, donde si $\xi$ tiene distribuci\'on $G$, entonces el tiempo $T_{n}$ de la $n$-\'esima renovaci\'on tiene distribuci\'on $G\star F^{\left(n-1\right)\star}\left(t\right)$
\end{Note}


\begin{Teo}
Para una constante $\mu\leq\infty$ ( o variable aleatoria), las siguientes expresiones son equivalentes:

\begin{eqnarray}
lim_{n\rightarrow\infty}n^{-1}T_{n}&=&\mu,\textrm{ c.s.}\\
lim_{t\rightarrow\infty}t^{-1}N\left(t\right)&=&1/\mu,\textrm{ c.s.}
\end{eqnarray}
\end{Teo}


Es decir, $T_{n}$ satisface la Ley Fuerte de los Grandes N\'umeros s\'i y s\'olo s\'i $N\left/t\right)$ la cumple.


\begin{Coro}[Ley Fuerte de los Grandes N\'umeros para Procesos de Renovaci\'on]
Si $N\left(t\right)$ es un proceso de renovaci\'on cuyos tiempos de inter-renovaci\'on tienen media $\mu\leq\infty$, entonces
\begin{eqnarray}
t^{-1}N\left(t\right)\rightarrow 1/\mu,\textrm{ c.s. cuando }t\rightarrow\infty.
\end{eqnarray}

\end{Coro}


Considerar el proceso estoc\'astico de valores reales $\left\{Z\left(t\right):t\geq0\right\}$ en el mismo espacio de probabilidad que $N\left(t\right)$

\begin{Def}
Para el proceso $\left\{Z\left(t\right):t\geq0\right\}$ se define la fluctuaci\'on m\'axima de $Z\left(t\right)$ en el intervalo $\left(T_{n-1},T_{n}\right]$:
\begin{eqnarray*}
M_{n}=\sup_{T_{n-1}<t\leq T_{n}}|Z\left(t\right)-Z\left(T_{n-1}\right)|
\end{eqnarray*}
\end{Def}

\begin{Teo}
Sup\'ongase que $n^{-1}T_{n}\rightarrow\mu$ c.s. cuando $n\rightarrow\infty$, donde $\mu\leq\infty$ es una constante o variable aleatoria. Sea $a$ una constante o variable aleatoria que puede ser infinita cuando $\mu$ es finita, y considere las expresiones l\'imite:
\begin{eqnarray}
lim_{n\rightarrow\infty}n^{-1}Z\left(T_{n}\right)&=&a,\textrm{ c.s.}\\
lim_{t\rightarrow\infty}t^{-1}Z\left(t\right)&=&a/\mu,\textrm{ c.s.}
\end{eqnarray}
La segunda expresi\'on implica la primera. Conversamente, la primera implica la segunda si el proceso $Z\left(t\right)$ es creciente, o si $lim_{n\rightarrow\infty}n^{-1}M_{n}=0$ c.s.
\end{Teo}

\begin{Coro}
Si $N\left(t\right)$ es un proceso de renovaci\'on, y $\left(Z\left(T_{n}\right)-Z\left(T_{n-1}\right),M_{n}\right)$, para $n\geq1$, son variables aleatorias independientes e id\'enticamente distribuidas con media finita, entonces,
\begin{eqnarray}
lim_{t\rightarrow\infty}t^{-1}Z\left(t\right)\rightarrow\frac{\esp\left[Z\left(T_{1}\right)-Z\left(T_{0}\right)\right]}{\esp\left[T_{1}\right]},\textrm{ c.s. cuando  }t\rightarrow\infty.
\end{eqnarray}
\end{Coro}


%___________________________________________________________________________________________
%
%\subsection*{Funci\'on de Renovaci\'on}
%___________________________________________________________________________________________
%


\begin{Def}
Sea $h\left(t\right)$ funci\'on de valores reales en $\rea$ acotada en intervalos finitos e igual a cero para $t<0$ La ecuaci\'on de renovaci\'on para $h\left(t\right)$ y la distribuci\'on $F$ es

\begin{eqnarray}\label{Ec.Renovacion}
H\left(t\right)=h\left(t\right)+\int_{\left[0,t\right]}H\left(t-s\right)dF\left(s\right)\textrm{,    }t\geq0,
\end{eqnarray}
donde $H\left(t\right)$ es una funci\'on de valores reales. Esto es $H=h+F\star H$. Decimos que $H\left(t\right)$ es soluci\'on de esta ecuaci\'on si satisface la ecuaci\'on, y es acotada en intervalos finitos e iguales a cero para $t<0$.
\end{Def}

\begin{Prop}
La funci\'on $U\star h\left(t\right)$ es la \'unica soluci\'on de la ecuaci\'on de renovaci\'on (\ref{Ec.Renovacion}).
\end{Prop}

\begin{Teo}[Teorema Renovaci\'on Elemental]
\begin{eqnarray*}
t^{-1}U\left(t\right)\rightarrow 1/\mu\textrm{,    cuando }t\rightarrow\infty.
\end{eqnarray*}
\end{Teo}

%___________________________________________________________________________________________
%
%\subsection{Funci\'on de Renovaci\'on}
%___________________________________________________________________________________________
%


Sup\'ongase que $N\left(t\right)$ es un proceso de renovaci\'on con distribuci\'on $F$ con media finita $\mu$.

\begin{Def}
La funci\'on de renovaci\'on asociada con la distribuci\'on $F$, del proceso $N\left(t\right)$, es
\begin{eqnarray*}
U\left(t\right)=\sum_{n=1}^{\infty}F^{n\star}\left(t\right),\textrm{   }t\geq0,
\end{eqnarray*}
donde $F^{0\star}\left(t\right)=\indora\left(t\geq0\right)$.
\end{Def}


\begin{Prop}
Sup\'ongase que la distribuci\'on de inter-renovaci\'on $F$ tiene densidad $f$. Entonces $U\left(t\right)$ tambi\'en tiene densidad, para $t>0$, y es $U^{'}\left(t\right)=\sum_{n=0}^{\infty}f^{n\star}\left(t\right)$. Adem\'as
\begin{eqnarray*}
\prob\left\{N\left(t\right)>N\left(t-\right)\right\}=0\textrm{,   }t\geq0.
\end{eqnarray*}
\end{Prop}

\begin{Def}
La Transformada de Laplace-Stieljes de $F$ est\'a dada por

\begin{eqnarray*}
\hat{F}\left(\alpha\right)=\int_{\rea_{+}}e^{-\alpha t}dF\left(t\right)\textrm{,  }\alpha\geq0.
\end{eqnarray*}
\end{Def}

Entonces

\begin{eqnarray*}
\hat{U}\left(\alpha\right)=\sum_{n=0}^{\infty}\hat{F^{n\star}}\left(\alpha\right)=\sum_{n=0}^{\infty}\hat{F}\left(\alpha\right)^{n}=\frac{1}{1-\hat{F}\left(\alpha\right)}.
\end{eqnarray*}


\begin{Prop}
La Transformada de Laplace $\hat{U}\left(\alpha\right)$ y $\hat{F}\left(\alpha\right)$ determina una a la otra de manera \'unica por la relaci\'on $\hat{U}\left(\alpha\right)=\frac{1}{1-\hat{F}\left(\alpha\right)}$.
\end{Prop}


\begin{Note}
Un proceso de renovaci\'on $N\left(t\right)$ cuyos tiempos de inter-renovaci\'on tienen media finita, es un proceso Poisson con tasa $\lambda$ si y s\'olo s\'i $\esp\left[U\left(t\right)\right]=\lambda t$, para $t\geq0$.
\end{Note}


\begin{Teo}
Sea $N\left(t\right)$ un proceso puntual simple con puntos de localizaci\'on $T_{n}$ tal que $\eta\left(t\right)=\esp\left[N\left(\right)\right]$ es finita para cada $t$. Entonces para cualquier funci\'on $f:\rea_{+}\rightarrow\rea$,
\begin{eqnarray*}
\esp\left[\sum_{n=1}^{N\left(\right)}f\left(T_{n}\right)\right]=\int_{\left(0,t\right]}f\left(s\right)d\eta\left(s\right)\textrm{,  }t\geq0,
\end{eqnarray*}
suponiendo que la integral exista. Adem\'as si $X_{1},X_{2},\ldots$ son variables aleatorias definidas en el mismo espacio de probabilidad que el proceso $N\left(t\right)$ tal que $\esp\left[X_{n}|T_{n}=s\right]=f\left(s\right)$, independiente de $n$. Entonces
\begin{eqnarray*}
\esp\left[\sum_{n=1}^{N\left(t\right)}X_{n}\right]=\int_{\left(0,t\right]}f\left(s\right)d\eta\left(s\right)\textrm{,  }t\geq0,
\end{eqnarray*} 
suponiendo que la integral exista. 
\end{Teo}

\begin{Coro}[Identidad de Wald para Renovaciones]
Para el proceso de renovaci\'on $N\left(t\right)$,
\begin{eqnarray*}
\esp\left[T_{N\left(t\right)+1}\right]=\mu\esp\left[N\left(t\right)+1\right]\textrm{,  }t\geq0,
\end{eqnarray*}  
\end{Coro}

%______________________________________________________________________
%\subsection{Procesos de Renovaci\'on}
%______________________________________________________________________

\begin{Def}\label{Def.Tn}
Sean $0\leq T_{1}\leq T_{2}\leq \ldots$ son tiempos aleatorios infinitos en los cuales ocurren ciertos eventos. El n\'umero de tiempos $T_{n}$ en el intervalo $\left[0,t\right)$ es

\begin{eqnarray}
N\left(t\right)=\sum_{n=1}^{\infty}\indora\left(T_{n}\leq t\right),
\end{eqnarray}
para $t\geq0$.
\end{Def}

Si se consideran los puntos $T_{n}$ como elementos de $\rea_{+}$, y $N\left(t\right)$ es el n\'umero de puntos en $\rea$. El proceso denotado por $\left\{N\left(t\right):t\geq0\right\}$, denotado por $N\left(t\right)$, es un proceso puntual en $\rea_{+}$. Los $T_{n}$ son los tiempos de ocurrencia, el proceso puntual $N\left(t\right)$ es simple si su n\'umero de ocurrencias son distintas: $0<T_{1}<T_{2}<\ldots$ casi seguramente.

\begin{Def}
Un proceso puntual $N\left(t\right)$ es un proceso de renovaci\'on si los tiempos de interocurrencia $\xi_{n}=T_{n}-T_{n-1}$, para $n\geq1$, son independientes e identicamente distribuidos con distribuci\'on $F$, donde $F\left(0\right)=0$ y $T_{0}=0$. Los $T_{n}$ son llamados tiempos de renovaci\'on, referente a la independencia o renovaci\'on de la informaci\'on estoc\'astica en estos tiempos. Los $\xi_{n}$ son los tiempos de inter-renovaci\'on, y $N\left(t\right)$ es el n\'umero de renovaciones en el intervalo $\left[0,t\right)$
\end{Def}


\begin{Note}
Para definir un proceso de renovaci\'on para cualquier contexto, solamente hay que especificar una distribuci\'on $F$, con $F\left(0\right)=0$, para los tiempos de inter-renovaci\'on. La funci\'on $F$ en turno degune las otra variables aleatorias. De manera formal, existe un espacio de probabilidad y una sucesi\'on de variables aleatorias $\xi_{1},\xi_{2},\ldots$ definidas en este con distribuci\'on $F$. Entonces las otras cantidades son $T_{n}=\sum_{k=1}^{n}\xi_{k}$ y $N\left(t\right)=\sum_{n=1}^{\infty}\indora\left(T_{n}\leq t\right)$, donde $T_{n}\rightarrow\infty$ casi seguramente por la Ley Fuerte de los Grandes Números.
\end{Note}

%___________________________________________________________________________________________
%
%\section{Renewal and Regenerative Processes: Serfozo\cite{Serfozo}}
%___________________________________________________________________________________________
%
\begin{Def}\label{Def.Tn}
Sean $0\leq T_{1}\leq T_{2}\leq \ldots$ son tiempos aleatorios infinitos en los cuales ocurren ciertos eventos. El n\'umero de tiempos $T_{n}$ en el intervalo $\left[0,t\right)$ es

\begin{eqnarray}
N\left(t\right)=\sum_{n=1}^{\infty}\indora\left(T_{n}\leq t\right),
\end{eqnarray}
para $t\geq0$.
\end{Def}

Si se consideran los puntos $T_{n}$ como elementos de $\rea_{+}$, y $N\left(t\right)$ es el n\'umero de puntos en $\rea$. El proceso denotado por $\left\{N\left(t\right):t\geq0\right\}$, denotado por $N\left(t\right)$, es un proceso puntual en $\rea_{+}$. Los $T_{n}$ son los tiempos de ocurrencia, el proceso puntual $N\left(t\right)$ es simple si su n\'umero de ocurrencias son distintas: $0<T_{1}<T_{2}<\ldots$ casi seguramente.

\begin{Def}
Un proceso puntual $N\left(t\right)$ es un proceso de renovaci\'on si los tiempos de interocurrencia $\xi_{n}=T_{n}-T_{n-1}$, para $n\geq1$, son independientes e identicamente distribuidos con distribuci\'on $F$, donde $F\left(0\right)=0$ y $T_{0}=0$. Los $T_{n}$ son llamados tiempos de renovaci\'on, referente a la independencia o renovaci\'on de la informaci\'on estoc\'astica en estos tiempos. Los $\xi_{n}$ son los tiempos de inter-renovaci\'on, y $N\left(t\right)$ es el n\'umero de renovaciones en el intervalo $\left[0,t\right)$
\end{Def}


\begin{Note}
Para definir un proceso de renovaci\'on para cualquier contexto, solamente hay que especificar una distribuci\'on $F$, con $F\left(0\right)=0$, para los tiempos de inter-renovaci\'on. La funci\'on $F$ en turno degune las otra variables aleatorias. De manera formal, existe un espacio de probabilidad y una sucesi\'on de variables aleatorias $\xi_{1},\xi_{2},\ldots$ definidas en este con distribuci\'on $F$. Entonces las otras cantidades son $T_{n}=\sum_{k=1}^{n}\xi_{k}$ y $N\left(t\right)=\sum_{n=1}^{\infty}\indora\left(T_{n}\leq t\right)$, donde $T_{n}\rightarrow\infty$ casi seguramente por la Ley Fuerte de los Grandes N\'umeros.
\end{Note}







Los tiempos $T_{n}$ est\'an relacionados con los conteos de $N\left(t\right)$ por

\begin{eqnarray*}
\left\{N\left(t\right)\geq n\right\}&=&\left\{T_{n}\leq t\right\}\\
T_{N\left(t\right)}\leq &t&<T_{N\left(t\right)+1},
\end{eqnarray*}

adem\'as $N\left(T_{n}\right)=n$, y 

\begin{eqnarray*}
N\left(t\right)=\max\left\{n:T_{n}\leq t\right\}=\min\left\{n:T_{n+1}>t\right\}
\end{eqnarray*}

Por propiedades de la convoluci\'on se sabe que

\begin{eqnarray*}
P\left\{T_{n}\leq t\right\}=F^{n\star}\left(t\right)
\end{eqnarray*}
que es la $n$-\'esima convoluci\'on de $F$. Entonces 

\begin{eqnarray*}
\left\{N\left(t\right)\geq n\right\}&=&\left\{T_{n}\leq t\right\}\\
P\left\{N\left(t\right)\leq n\right\}&=&1-F^{\left(n+1\right)\star}\left(t\right)
\end{eqnarray*}

Adem\'as usando el hecho de que $\esp\left[N\left(t\right)\right]=\sum_{n=1}^{\infty}P\left\{N\left(t\right)\geq n\right\}$
se tiene que

\begin{eqnarray*}
\esp\left[N\left(t\right)\right]=\sum_{n=1}^{\infty}F^{n\star}\left(t\right)
\end{eqnarray*}

\begin{Prop}
Para cada $t\geq0$, la funci\'on generadora de momentos $\esp\left[e^{\alpha N\left(t\right)}\right]$ existe para alguna $\alpha$ en una vecindad del 0, y de aqu\'i que $\esp\left[N\left(t\right)^{m}\right]<\infty$, para $m\geq1$.
\end{Prop}

\begin{Ejem}[\textbf{Proceso Poisson}]

Suponga que se tienen tiempos de inter-renovaci\'on \textit{i.i.d.} del proceso de renovaci\'on $N\left(t\right)$ tienen distribuci\'on exponencial $F\left(t\right)=q-e^{-\lambda t}$ con tasa $\lambda$. Entonces $N\left(t\right)$ es un proceso Poisson con tasa $\lambda$.

\end{Ejem}


\begin{Note}
Si el primer tiempo de renovaci\'on $\xi_{1}$ no tiene la misma distribuci\'on que el resto de las $\xi_{n}$, para $n\geq2$, a $N\left(t\right)$ se le llama Proceso de Renovaci\'on retardado, donde si $\xi$ tiene distribuci\'on $G$, entonces el tiempo $T_{n}$ de la $n$-\'esima renovaci\'on tiene distribuci\'on $G\star F^{\left(n-1\right)\star}\left(t\right)$
\end{Note}


\begin{Teo}
Para una constante $\mu\leq\infty$ ( o variable aleatoria), las siguientes expresiones son equivalentes:

\begin{eqnarray}
lim_{n\rightarrow\infty}n^{-1}T_{n}&=&\mu,\textrm{ c.s.}\\
lim_{t\rightarrow\infty}t^{-1}N\left(t\right)&=&1/\mu,\textrm{ c.s.}
\end{eqnarray}
\end{Teo}


Es decir, $T_{n}$ satisface la Ley Fuerte de los Grandes N\'umeros s\'i y s\'olo s\'i $N\left/t\right)$ la cumple.


\begin{Coro}[Ley Fuerte de los Grandes N\'umeros para Procesos de Renovaci\'on]
Si $N\left(t\right)$ es un proceso de renovaci\'on cuyos tiempos de inter-renovaci\'on tienen media $\mu\leq\infty$, entonces
\begin{eqnarray}
t^{-1}N\left(t\right)\rightarrow 1/\mu,\textrm{ c.s. cuando }t\rightarrow\infty.
\end{eqnarray}

\end{Coro}


Considerar el proceso estoc\'astico de valores reales $\left\{Z\left(t\right):t\geq0\right\}$ en el mismo espacio de probabilidad que $N\left(t\right)$

\begin{Def}
Para el proceso $\left\{Z\left(t\right):t\geq0\right\}$ se define la fluctuaci\'on m\'axima de $Z\left(t\right)$ en el intervalo $\left(T_{n-1},T_{n}\right]$:
\begin{eqnarray*}
M_{n}=\sup_{T_{n-1}<t\leq T_{n}}|Z\left(t\right)-Z\left(T_{n-1}\right)|
\end{eqnarray*}
\end{Def}

\begin{Teo}
Sup\'ongase que $n^{-1}T_{n}\rightarrow\mu$ c.s. cuando $n\rightarrow\infty$, donde $\mu\leq\infty$ es una constante o variable aleatoria. Sea $a$ una constante o variable aleatoria que puede ser infinita cuando $\mu$ es finita, y considere las expresiones l\'imite:
\begin{eqnarray}
lim_{n\rightarrow\infty}n^{-1}Z\left(T_{n}\right)&=&a,\textrm{ c.s.}\\
lim_{t\rightarrow\infty}t^{-1}Z\left(t\right)&=&a/\mu,\textrm{ c.s.}
\end{eqnarray}
La segunda expresi\'on implica la primera. Conversamente, la primera implica la segunda si el proceso $Z\left(t\right)$ es creciente, o si $lim_{n\rightarrow\infty}n^{-1}M_{n}=0$ c.s.
\end{Teo}

\begin{Coro}
Si $N\left(t\right)$ es un proceso de renovaci\'on, y $\left(Z\left(T_{n}\right)-Z\left(T_{n-1}\right),M_{n}\right)$, para $n\geq1$, son variables aleatorias independientes e id\'enticamente distribuidas con media finita, entonces,
\begin{eqnarray}
lim_{t\rightarrow\infty}t^{-1}Z\left(t\right)\rightarrow\frac{\esp\left[Z\left(T_{1}\right)-Z\left(T_{0}\right)\right]}{\esp\left[T_{1}\right]},\textrm{ c.s. cuando  }t\rightarrow\infty.
\end{eqnarray}
\end{Coro}


Sup\'ongase que $N\left(t\right)$ es un proceso de renovaci\'on con distribuci\'on $F$ con media finita $\mu$.

\begin{Def}
La funci\'on de renovaci\'on asociada con la distribuci\'on $F$, del proceso $N\left(t\right)$, es
\begin{eqnarray*}
U\left(t\right)=\sum_{n=1}^{\infty}F^{n\star}\left(t\right),\textrm{   }t\geq0,
\end{eqnarray*}
donde $F^{0\star}\left(t\right)=\indora\left(t\geq0\right)$.
\end{Def}


\begin{Prop}
Sup\'ongase que la distribuci\'on de inter-renovaci\'on $F$ tiene densidad $f$. Entonces $U\left(t\right)$ tambi\'en tiene densidad, para $t>0$, y es $U^{'}\left(t\right)=\sum_{n=0}^{\infty}f^{n\star}\left(t\right)$. Adem\'as
\begin{eqnarray*}
\prob\left\{N\left(t\right)>N\left(t-\right)\right\}=0\textrm{,   }t\geq0.
\end{eqnarray*}
\end{Prop}

\begin{Def}
La Transformada de Laplace-Stieljes de $F$ est\'a dada por

\begin{eqnarray*}
\hat{F}\left(\alpha\right)=\int_{\rea_{+}}e^{-\alpha t}dF\left(t\right)\textrm{,  }\alpha\geq0.
\end{eqnarray*}
\end{Def}

Entonces

\begin{eqnarray*}
\hat{U}\left(\alpha\right)=\sum_{n=0}^{\infty}\hat{F^{n\star}}\left(\alpha\right)=\sum_{n=0}^{\infty}\hat{F}\left(\alpha\right)^{n}=\frac{1}{1-\hat{F}\left(\alpha\right)}.
\end{eqnarray*}


\begin{Prop}
La Transformada de Laplace $\hat{U}\left(\alpha\right)$ y $\hat{F}\left(\alpha\right)$ determina una a la otra de manera \'unica por la relaci\'on $\hat{U}\left(\alpha\right)=\frac{1}{1-\hat{F}\left(\alpha\right)}$.
\end{Prop}


\begin{Note}
Un proceso de renovaci\'on $N\left(t\right)$ cuyos tiempos de inter-renovaci\'on tienen media finita, es un proceso Poisson con tasa $\lambda$ si y s\'olo s\'i $\esp\left[U\left(t\right)\right]=\lambda t$, para $t\geq0$.
\end{Note}


\begin{Teo}
Sea $N\left(t\right)$ un proceso puntual simple con puntos de localizaci\'on $T_{n}$ tal que $\eta\left(t\right)=\esp\left[N\left(\right)\right]$ es finita para cada $t$. Entonces para cualquier funci\'on $f:\rea_{+}\rightarrow\rea$,
\begin{eqnarray*}
\esp\left[\sum_{n=1}^{N\left(\right)}f\left(T_{n}\right)\right]=\int_{\left(0,t\right]}f\left(s\right)d\eta\left(s\right)\textrm{,  }t\geq0,
\end{eqnarray*}
suponiendo que la integral exista. Adem\'as si $X_{1},X_{2},\ldots$ son variables aleatorias definidas en el mismo espacio de probabilidad que el proceso $N\left(t\right)$ tal que $\esp\left[X_{n}|T_{n}=s\right]=f\left(s\right)$, independiente de $n$. Entonces
\begin{eqnarray*}
\esp\left[\sum_{n=1}^{N\left(t\right)}X_{n}\right]=\int_{\left(0,t\right]}f\left(s\right)d\eta\left(s\right)\textrm{,  }t\geq0,
\end{eqnarray*} 
suponiendo que la integral exista. 
\end{Teo}

\begin{Coro}[Identidad de Wald para Renovaciones]
Para el proceso de renovaci\'on $N\left(t\right)$,
\begin{eqnarray*}
\esp\left[T_{N\left(t\right)+1}\right]=\mu\esp\left[N\left(t\right)+1\right]\textrm{,  }t\geq0,
\end{eqnarray*}  
\end{Coro}


\begin{Def}
Sea $h\left(t\right)$ funci\'on de valores reales en $\rea$ acotada en intervalos finitos e igual a cero para $t<0$ La ecuaci\'on de renovaci\'on para $h\left(t\right)$ y la distribuci\'on $F$ es

\begin{eqnarray}\label{Ec.Renovacion}
H\left(t\right)=h\left(t\right)+\int_{\left[0,t\right]}H\left(t-s\right)dF\left(s\right)\textrm{,    }t\geq0,
\end{eqnarray}
donde $H\left(t\right)$ es una funci\'on de valores reales. Esto es $H=h+F\star H$. Decimos que $H\left(t\right)$ es soluci\'on de esta ecuaci\'on si satisface la ecuaci\'on, y es acotada en intervalos finitos e iguales a cero para $t<0$.
\end{Def}

\begin{Prop}
La funci\'on $U\star h\left(t\right)$ es la \'unica soluci\'on de la ecuaci\'on de renovaci\'on (\ref{Ec.Renovacion}).
\end{Prop}

\begin{Teo}[Teorema Renovaci\'on Elemental]
\begin{eqnarray*}
t^{-1}U\left(t\right)\rightarrow 1/\mu\textrm{,    cuando }t\rightarrow\infty.
\end{eqnarray*}
\end{Teo}



Sup\'ongase que $N\left(t\right)$ es un proceso de renovaci\'on con distribuci\'on $F$ con media finita $\mu$.

\begin{Def}
La funci\'on de renovaci\'on asociada con la distribuci\'on $F$, del proceso $N\left(t\right)$, es
\begin{eqnarray*}
U\left(t\right)=\sum_{n=1}^{\infty}F^{n\star}\left(t\right),\textrm{   }t\geq0,
\end{eqnarray*}
donde $F^{0\star}\left(t\right)=\indora\left(t\geq0\right)$.
\end{Def}


\begin{Prop}
Sup\'ongase que la distribuci\'on de inter-renovaci\'on $F$ tiene densidad $f$. Entonces $U\left(t\right)$ tambi\'en tiene densidad, para $t>0$, y es $U^{'}\left(t\right)=\sum_{n=0}^{\infty}f^{n\star}\left(t\right)$. Adem\'as
\begin{eqnarray*}
\prob\left\{N\left(t\right)>N\left(t-\right)\right\}=0\textrm{,   }t\geq0.
\end{eqnarray*}
\end{Prop}

\begin{Def}
La Transformada de Laplace-Stieljes de $F$ est\'a dada por

\begin{eqnarray*}
\hat{F}\left(\alpha\right)=\int_{\rea_{+}}e^{-\alpha t}dF\left(t\right)\textrm{,  }\alpha\geq0.
\end{eqnarray*}
\end{Def}

Entonces

\begin{eqnarray*}
\hat{U}\left(\alpha\right)=\sum_{n=0}^{\infty}\hat{F^{n\star}}\left(\alpha\right)=\sum_{n=0}^{\infty}\hat{F}\left(\alpha\right)^{n}=\frac{1}{1-\hat{F}\left(\alpha\right)}.
\end{eqnarray*}


\begin{Prop}
La Transformada de Laplace $\hat{U}\left(\alpha\right)$ y $\hat{F}\left(\alpha\right)$ determina una a la otra de manera \'unica por la relaci\'on $\hat{U}\left(\alpha\right)=\frac{1}{1-\hat{F}\left(\alpha\right)}$.
\end{Prop}


\begin{Note}
Un proceso de renovaci\'on $N\left(t\right)$ cuyos tiempos de inter-renovaci\'on tienen media finita, es un proceso Poisson con tasa $\lambda$ si y s\'olo s\'i $\esp\left[U\left(t\right)\right]=\lambda t$, para $t\geq0$.
\end{Note}


\begin{Teo}
Sea $N\left(t\right)$ un proceso puntual simple con puntos de localizaci\'on $T_{n}$ tal que $\eta\left(t\right)=\esp\left[N\left(\right)\right]$ es finita para cada $t$. Entonces para cualquier funci\'on $f:\rea_{+}\rightarrow\rea$,
\begin{eqnarray*}
\esp\left[\sum_{n=1}^{N\left(\right)}f\left(T_{n}\right)\right]=\int_{\left(0,t\right]}f\left(s\right)d\eta\left(s\right)\textrm{,  }t\geq0,
\end{eqnarray*}
suponiendo que la integral exista. Adem\'as si $X_{1},X_{2},\ldots$ son variables aleatorias definidas en el mismo espacio de probabilidad que el proceso $N\left(t\right)$ tal que $\esp\left[X_{n}|T_{n}=s\right]=f\left(s\right)$, independiente de $n$. Entonces
\begin{eqnarray*}
\esp\left[\sum_{n=1}^{N\left(t\right)}X_{n}\right]=\int_{\left(0,t\right]}f\left(s\right)d\eta\left(s\right)\textrm{,  }t\geq0,
\end{eqnarray*} 
suponiendo que la integral exista. 
\end{Teo}

\begin{Coro}[Identidad de Wald para Renovaciones]
Para el proceso de renovaci\'on $N\left(t\right)$,
\begin{eqnarray*}
\esp\left[T_{N\left(t\right)+1}\right]=\mu\esp\left[N\left(t\right)+1\right]\textrm{,  }t\geq0,
\end{eqnarray*}  
\end{Coro}


\begin{Def}
Sea $h\left(t\right)$ funci\'on de valores reales en $\rea$ acotada en intervalos finitos e igual a cero para $t<0$ La ecuaci\'on de renovaci\'on para $h\left(t\right)$ y la distribuci\'on $F$ es

\begin{eqnarray}\label{Ec.Renovacion}
H\left(t\right)=h\left(t\right)+\int_{\left[0,t\right]}H\left(t-s\right)dF\left(s\right)\textrm{,    }t\geq0,
\end{eqnarray}
donde $H\left(t\right)$ es una funci\'on de valores reales. Esto es $H=h+F\star H$. Decimos que $H\left(t\right)$ es soluci\'on de esta ecuaci\'on si satisface la ecuaci\'on, y es acotada en intervalos finitos e iguales a cero para $t<0$.
\end{Def}

\begin{Prop}
La funci\'on $U\star h\left(t\right)$ es la \'unica soluci\'on de la ecuaci\'on de renovaci\'on (\ref{Ec.Renovacion}).
\end{Prop}

\begin{Teo}[Teorema Renovaci\'on Elemental]
\begin{eqnarray*}
t^{-1}U\left(t\right)\rightarrow 1/\mu\textrm{,    cuando }t\rightarrow\infty.
\end{eqnarray*}
\end{Teo}


\begin{Note} Una funci\'on $h:\rea_{+}\rightarrow\rea$ es Directamente Riemann Integrable en los siguientes casos:
\begin{itemize}
\item[a)] $h\left(t\right)\geq0$ es decreciente y Riemann Integrable.
\item[b)] $h$ es continua excepto posiblemente en un conjunto de Lebesgue de medida 0, y $|h\left(t\right)|\leq b\left(t\right)$, donde $b$ es DRI.
\end{itemize}
\end{Note}

\begin{Teo}[Teorema Principal de Renovaci\'on]
Si $F$ es no aritm\'etica y $h\left(t\right)$ es Directamente Riemann Integrable (DRI), entonces

\begin{eqnarray*}
lim_{t\rightarrow\infty}U\star h=\frac{1}{\mu}\int_{\rea_{+}}h\left(s\right)ds.
\end{eqnarray*}
\end{Teo}

\begin{Prop}
Cualquier funci\'on $H\left(t\right)$ acotada en intervalos finitos y que es 0 para $t<0$ puede expresarse como
\begin{eqnarray*}
H\left(t\right)=U\star h\left(t\right)\textrm{,  donde }h\left(t\right)=H\left(t\right)-F\star H\left(t\right)
\end{eqnarray*}
\end{Prop}

\begin{Def}
Un proceso estoc\'astico $X\left(t\right)$ es crudamente regenerativo en un tiempo aleatorio positivo $T$ si
\begin{eqnarray*}
\esp\left[X\left(T+t\right)|T\right]=\esp\left[X\left(t\right)\right]\textrm{, para }t\geq0,\end{eqnarray*}
y con las esperanzas anteriores finitas.
\end{Def}

\begin{Prop}
Sup\'ongase que $X\left(t\right)$ es un proceso crudamente regenerativo en $T$, que tiene distribuci\'on $F$. Si $\esp\left[X\left(t\right)\right]$ es acotado en intervalos finitos, entonces
\begin{eqnarray*}
\esp\left[X\left(t\right)\right]=U\star h\left(t\right)\textrm{,  donde }h\left(t\right)=\esp\left[X\left(t\right)\indora\left(T>t\right)\right].
\end{eqnarray*}
\end{Prop}

\begin{Teo}[Regeneraci\'on Cruda]
Sup\'ongase que $X\left(t\right)$ es un proceso con valores positivo crudamente regenerativo en $T$, y def\'inase $M=\sup\left\{|X\left(t\right)|:t\leq T\right\}$. Si $T$ es no aritm\'etico y $M$ y $MT$ tienen media finita, entonces
\begin{eqnarray*}
lim_{t\rightarrow\infty}\esp\left[X\left(t\right)\right]=\frac{1}{\mu}\int_{\rea_{+}}h\left(s\right)ds,
\end{eqnarray*}
donde $h\left(t\right)=\esp\left[X\left(t\right)\indora\left(T>t\right)\right]$.
\end{Teo}


\begin{Note} Una funci\'on $h:\rea_{+}\rightarrow\rea$ es Directamente Riemann Integrable en los siguientes casos:
\begin{itemize}
\item[a)] $h\left(t\right)\geq0$ es decreciente y Riemann Integrable.
\item[b)] $h$ es continua excepto posiblemente en un conjunto de Lebesgue de medida 0, y $|h\left(t\right)|\leq b\left(t\right)$, donde $b$ es DRI.
\end{itemize}
\end{Note}

\begin{Teo}[Teorema Principal de Renovaci\'on]
Si $F$ es no aritm\'etica y $h\left(t\right)$ es Directamente Riemann Integrable (DRI), entonces

\begin{eqnarray*}
lim_{t\rightarrow\infty}U\star h=\frac{1}{\mu}\int_{\rea_{+}}h\left(s\right)ds.
\end{eqnarray*}
\end{Teo}

\begin{Prop}
Cualquier funci\'on $H\left(t\right)$ acotada en intervalos finitos y que es 0 para $t<0$ puede expresarse como
\begin{eqnarray*}
H\left(t\right)=U\star h\left(t\right)\textrm{,  donde }h\left(t\right)=H\left(t\right)-F\star H\left(t\right)
\end{eqnarray*}
\end{Prop}

\begin{Def}
Un proceso estoc\'astico $X\left(t\right)$ es crudamente regenerativo en un tiempo aleatorio positivo $T$ si
\begin{eqnarray*}
\esp\left[X\left(T+t\right)|T\right]=\esp\left[X\left(t\right)\right]\textrm{, para }t\geq0,\end{eqnarray*}
y con las esperanzas anteriores finitas.
\end{Def}

\begin{Prop}
Sup\'ongase que $X\left(t\right)$ es un proceso crudamente regenerativo en $T$, que tiene distribuci\'on $F$. Si $\esp\left[X\left(t\right)\right]$ es acotado en intervalos finitos, entonces
\begin{eqnarray*}
\esp\left[X\left(t\right)\right]=U\star h\left(t\right)\textrm{,  donde }h\left(t\right)=\esp\left[X\left(t\right)\indora\left(T>t\right)\right].
\end{eqnarray*}
\end{Prop}

\begin{Teo}[Regeneraci\'on Cruda]
Sup\'ongase que $X\left(t\right)$ es un proceso con valores positivo crudamente regenerativo en $T$, y def\'inase $M=\sup\left\{|X\left(t\right)|:t\leq T\right\}$. Si $T$ es no aritm\'etico y $M$ y $MT$ tienen media finita, entonces
\begin{eqnarray*}
lim_{t\rightarrow\infty}\esp\left[X\left(t\right)\right]=\frac{1}{\mu}\int_{\rea_{+}}h\left(s\right)ds,
\end{eqnarray*}
donde $h\left(t\right)=\esp\left[X\left(t\right)\indora\left(T>t\right)\right]$.
\end{Teo}

\begin{Def}
Para el proceso $\left\{\left(N\left(t\right),X\left(t\right)\right):t\geq0\right\}$, sus trayectoria muestrales en el intervalo de tiempo $\left[T_{n-1},T_{n}\right)$ est\'an descritas por
\begin{eqnarray*}
\zeta_{n}=\left(\xi_{n},\left\{X\left(T_{n-1}+t\right):0\leq t<\xi_{n}\right\}\right)
\end{eqnarray*}
Este $\zeta_{n}$ es el $n$-\'esimo segmento del proceso. El proceso es regenerativo sobre los tiempos $T_{n}$ si sus segmentos $\zeta_{n}$ son independientes e id\'enticamennte distribuidos.
\end{Def}


\begin{Note}
Si $\tilde{X}\left(t\right)$ con espacio de estados $\tilde{S}$ es regenerativo sobre $T_{n}$, entonces $X\left(t\right)=f\left(\tilde{X}\left(t\right)\right)$ tambi\'en es regenerativo sobre $T_{n}$, para cualquier funci\'on $f:\tilde{S}\rightarrow S$.
\end{Note}

\begin{Note}
Los procesos regenerativos son crudamente regenerativos, pero no al rev\'es.
\end{Note}


\begin{Note}
Un proceso estoc\'astico a tiempo continuo o discreto es regenerativo si existe un proceso de renovaci\'on  tal que los segmentos del proceso entre tiempos de renovaci\'on sucesivos son i.i.d., es decir, para $\left\{X\left(t\right):t\geq0\right\}$ proceso estoc\'astico a tiempo continuo con espacio de estados $S$, espacio m\'etrico.
\end{Note}

Para $\left\{X\left(t\right):t\geq0\right\}$ Proceso Estoc\'astico a tiempo continuo con estado de espacios $S$, que es un espacio m\'etrico, con trayectorias continuas por la derecha y con l\'imites por la izquierda c.s. Sea $N\left(t\right)$ un proceso de renovaci\'on en $\rea_{+}$ definido en el mismo espacio de probabilidad que $X\left(t\right)$, con tiempos de renovaci\'on $T$ y tiempos de inter-renovaci\'on $\xi_{n}=T_{n}-T_{n-1}$, con misma distribuci\'on $F$ de media finita $\mu$.



\begin{Def}
Para el proceso $\left\{\left(N\left(t\right),X\left(t\right)\right):t\geq0\right\}$, sus trayectoria muestrales en el intervalo de tiempo $\left[T_{n-1},T_{n}\right)$ est\'an descritas por
\begin{eqnarray*}
\zeta_{n}=\left(\xi_{n},\left\{X\left(T_{n-1}+t\right):0\leq t<\xi_{n}\right\}\right)
\end{eqnarray*}
Este $\zeta_{n}$ es el $n$-\'esimo segmento del proceso. El proceso es regenerativo sobre los tiempos $T_{n}$ si sus segmentos $\zeta_{n}$ son independientes e id\'enticamennte distribuidos.
\end{Def}

\begin{Note}
Un proceso regenerativo con media de la longitud de ciclo finita es llamado positivo recurrente.
\end{Note}

\begin{Teo}[Procesos Regenerativos]
Suponga que el proceso
\end{Teo}


\begin{Def}[Renewal Process Trinity]
Para un proceso de renovaci\'on $N\left(t\right)$, los siguientes procesos proveen de informaci\'on sobre los tiempos de renovaci\'on.
\begin{itemize}
\item $A\left(t\right)=t-T_{N\left(t\right)}$, el tiempo de recurrencia hacia atr\'as al tiempo $t$, que es el tiempo desde la \'ultima renovaci\'on para $t$.

\item $B\left(t\right)=T_{N\left(t\right)+1}-t$, el tiempo de recurrencia hacia adelante al tiempo $t$, residual del tiempo de renovaci\'on, que es el tiempo para la pr\'oxima renovaci\'on despu\'es de $t$.

\item $L\left(t\right)=\xi_{N\left(t\right)+1}=A\left(t\right)+B\left(t\right)$, la longitud del intervalo de renovaci\'on que contiene a $t$.
\end{itemize}
\end{Def}

\begin{Note}
El proceso tridimensional $\left(A\left(t\right),B\left(t\right),L\left(t\right)\right)$ es regenerativo sobre $T_{n}$, y por ende cada proceso lo es. Cada proceso $A\left(t\right)$ y $B\left(t\right)$ son procesos de MArkov a tiempo continuo con trayectorias continuas por partes en el espacio de estados $\rea_{+}$. Una expresi\'on conveniente para su distribuci\'on conjunta es, para $0\leq x<t,y\geq0$
\begin{equation}\label{NoRenovacion}
P\left\{A\left(t\right)>x,B\left(t\right)>y\right\}=
P\left\{N\left(t+y\right)-N\left((t-x)\right)=0\right\}
\end{equation}
\end{Note}

\begin{Ejem}[Tiempos de recurrencia Poisson]
Si $N\left(t\right)$ es un proceso Poisson con tasa $\lambda$, entonces de la expresi\'on (\ref{NoRenovacion}) se tiene que

\begin{eqnarray*}
\begin{array}{lc}
P\left\{A\left(t\right)>x,B\left(t\right)>y\right\}=e^{-\lambda\left(x+y\right)},&0\leq x<t,y\geq0,
\end{array}
\end{eqnarray*}
que es la probabilidad Poisson de no renovaciones en un intervalo de longitud $x+y$.

\end{Ejem}

\begin{Note}
Una cadena de Markov erg\'odica tiene la propiedad de ser estacionaria si la distribuci\'on de su estado al tiempo $0$ es su distribuci\'on estacionaria.
\end{Note}


\begin{Def}
Un proceso estoc\'astico a tiempo continuo $\left\{X\left(t\right):t\geq0\right\}$ en un espacio general es estacionario si sus distribuciones finito dimensionales son invariantes bajo cualquier  traslado: para cada $0\leq s_{1}<s_{2}<\cdots<s_{k}$ y $t\geq0$,
\begin{eqnarray*}
\left(X\left(s_{1}+t\right),\ldots,X\left(s_{k}+t\right)\right)=_{d}\left(X\left(s_{1}\right),\ldots,X\left(s_{k}\right)\right).
\end{eqnarray*}
\end{Def}

\begin{Note}
Un proceso de Markov es estacionario si $X\left(t\right)=_{d}X\left(0\right)$, $t\geq0$.
\end{Note}

Considerese el proceso $N\left(t\right)=\sum_{n}\indora\left(\tau_{n}\leq t\right)$ en $\rea_{+}$, con puntos $0<\tau_{1}<\tau_{2}<\cdots$.

\begin{Prop}
Si $N$ es un proceso puntual estacionario y $\esp\left[N\left(1\right)\right]<\infty$, entonces $\esp\left[N\left(t\right)\right]=t\esp\left[N\left(1\right)\right]$, $t\geq0$

\end{Prop}

\begin{Teo}
Los siguientes enunciados son equivalentes
\begin{itemize}
\item[i)] El proceso retardado de renovaci\'on $N$ es estacionario.

\item[ii)] EL proceso de tiempos de recurrencia hacia adelante $B\left(t\right)$ es estacionario.


\item[iii)] $\esp\left[N\left(t\right)\right]=t/\mu$,


\item[iv)] $G\left(t\right)=F_{e}\left(t\right)=\frac{1}{\mu}\int_{0}^{t}\left[1-F\left(s\right)\right]ds$
\end{itemize}
Cuando estos enunciados son ciertos, $P\left\{B\left(t\right)\leq x\right\}=F_{e}\left(x\right)$, para $t,x\geq0$.

\end{Teo}

\begin{Note}
Una consecuencia del teorema anterior es que el Proceso Poisson es el \'unico proceso sin retardo que es estacionario.
\end{Note}

\begin{Coro}
El proceso de renovaci\'on $N\left(t\right)$ sin retardo, y cuyos tiempos de inter renonaci\'on tienen media finita, es estacionario si y s\'olo si es un proceso Poisson.

\end{Coro}

%______________________________________________________________________

%\section{Ejemplos, Notas importantes}
%______________________________________________________________________
%\section*{Ap\'endice A}
%__________________________________________________________________

%________________________________________________________________________
%\subsection*{Procesos Regenerativos}
%________________________________________________________________________



\begin{Note}
Si $\tilde{X}\left(t\right)$ con espacio de estados $\tilde{S}$ es regenerativo sobre $T_{n}$, entonces $X\left(t\right)=f\left(\tilde{X}\left(t\right)\right)$ tambi\'en es regenerativo sobre $T_{n}$, para cualquier funci\'on $f:\tilde{S}\rightarrow S$.
\end{Note}

\begin{Note}
Los procesos regenerativos son crudamente regenerativos, pero no al rev\'es.
\end{Note}
%\subsection*{Procesos Regenerativos: Sigman\cite{Sigman1}}
\begin{Def}[Definici\'on Cl\'asica]
Un proceso estoc\'astico $X=\left\{X\left(t\right):t\geq0\right\}$ es llamado regenerativo is existe una variable aleatoria $R_{1}>0$ tal que
\begin{itemize}
\item[i)] $\left\{X\left(t+R_{1}\right):t\geq0\right\}$ es independiente de $\left\{\left\{X\left(t\right):t<R_{1}\right\},\right\}$
\item[ii)] $\left\{X\left(t+R_{1}\right):t\geq0\right\}$ es estoc\'asticamente equivalente a $\left\{X\left(t\right):t>0\right\}$
\end{itemize}

Llamamos a $R_{1}$ tiempo de regeneraci\'on, y decimos que $X$ se regenera en este punto.
\end{Def}

$\left\{X\left(t+R_{1}\right)\right\}$ es regenerativo con tiempo de regeneraci\'on $R_{2}$, independiente de $R_{1}$ pero con la misma distribuci\'on que $R_{1}$. Procediendo de esta manera se obtiene una secuencia de variables aleatorias independientes e id\'enticamente distribuidas $\left\{R_{n}\right\}$ llamados longitudes de ciclo. Si definimos a $Z_{k}\equiv R_{1}+R_{2}+\cdots+R_{k}$, se tiene un proceso de renovaci\'on llamado proceso de renovaci\'on encajado para $X$.




\begin{Def}
Para $x$ fijo y para cada $t\geq0$, sea $I_{x}\left(t\right)=1$ si $X\left(t\right)\leq x$,  $I_{x}\left(t\right)=0$ en caso contrario, y def\'inanse los tiempos promedio
\begin{eqnarray*}
\overline{X}&=&lim_{t\rightarrow\infty}\frac{1}{t}\int_{0}^{\infty}X\left(u\right)du\\
\prob\left(X_{\infty}\leq x\right)&=&lim_{t\rightarrow\infty}\frac{1}{t}\int_{0}^{\infty}I_{x}\left(u\right)du,
\end{eqnarray*}
cuando estos l\'imites existan.
\end{Def}

Como consecuencia del teorema de Renovaci\'on-Recompensa, se tiene que el primer l\'imite  existe y es igual a la constante
\begin{eqnarray*}
\overline{X}&=&\frac{\esp\left[\int_{0}^{R_{1}}X\left(t\right)dt\right]}{\esp\left[R_{1}\right]},
\end{eqnarray*}
suponiendo que ambas esperanzas son finitas.

\begin{Note}
\begin{itemize}
\item[a)] Si el proceso regenerativo $X$ es positivo recurrente y tiene trayectorias muestrales no negativas, entonces la ecuaci\'on anterior es v\'alida.
\item[b)] Si $X$ es positivo recurrente regenerativo, podemos construir una \'unica versi\'on estacionaria de este proceso, $X_{e}=\left\{X_{e}\left(t\right)\right\}$, donde $X_{e}$ es un proceso estoc\'astico regenerativo y estrictamente estacionario, con distribuci\'on marginal distribuida como $X_{\infty}$
\end{itemize}
\end{Note}

Para $\left\{X\left(t\right):t\geq0\right\}$ Proceso Estoc\'astico a tiempo continuo con estado de espacios $S$, que es un espacio m\'etrico, con trayectorias continuas por la derecha y con l\'imites por la izquierda c.s. Sea $N\left(t\right)$ un proceso de renovaci\'on en $\rea_{+}$ definido en el mismo espacio de probabilidad que $X\left(t\right)$, con tiempos de renovaci\'on $T$ y tiempos de inter-renovaci\'on $\xi_{n}=T_{n}-T_{n-1}$, con misma distribuci\'on $F$ de media finita $\mu$.


\begin{Def}
Para el proceso $\left\{\left(N\left(t\right),X\left(t\right)\right):t\geq0\right\}$, sus trayectoria muestrales en el intervalo de tiempo $\left[T_{n-1},T_{n}\right)$ est\'an descritas por
\begin{eqnarray*}
\zeta_{n}=\left(\xi_{n},\left\{X\left(T_{n-1}+t\right):0\leq t<\xi_{n}\right\}\right)
\end{eqnarray*}
Este $\zeta_{n}$ es el $n$-\'esimo segmento del proceso. El proceso es regenerativo sobre los tiempos $T_{n}$ si sus segmentos $\zeta_{n}$ son independientes e id\'enticamennte distribuidos.
\end{Def}


\begin{Note}
Si $\tilde{X}\left(t\right)$ con espacio de estados $\tilde{S}$ es regenerativo sobre $T_{n}$, entonces $X\left(t\right)=f\left(\tilde{X}\left(t\right)\right)$ tambi\'en es regenerativo sobre $T_{n}$, para cualquier funci\'on $f:\tilde{S}\rightarrow S$.
\end{Note}

\begin{Note}
Los procesos regenerativos son crudamente regenerativos, pero no al rev\'es.
\end{Note}

\begin{Def}[Definici\'on Cl\'asica]
Un proceso estoc\'astico $X=\left\{X\left(t\right):t\geq0\right\}$ es llamado regenerativo is existe una variable aleatoria $R_{1}>0$ tal que
\begin{itemize}
\item[i)] $\left\{X\left(t+R_{1}\right):t\geq0\right\}$ es independiente de $\left\{\left\{X\left(t\right):t<R_{1}\right\},\right\}$
\item[ii)] $\left\{X\left(t+R_{1}\right):t\geq0\right\}$ es estoc\'asticamente equivalente a $\left\{X\left(t\right):t>0\right\}$
\end{itemize}

Llamamos a $R_{1}$ tiempo de regeneraci\'on, y decimos que $X$ se regenera en este punto.
\end{Def}

$\left\{X\left(t+R_{1}\right)\right\}$ es regenerativo con tiempo de regeneraci\'on $R_{2}$, independiente de $R_{1}$ pero con la misma distribuci\'on que $R_{1}$. Procediendo de esta manera se obtiene una secuencia de variables aleatorias independientes e id\'enticamente distribuidas $\left\{R_{n}\right\}$ llamados longitudes de ciclo. Si definimos a $Z_{k}\equiv R_{1}+R_{2}+\cdots+R_{k}$, se tiene un proceso de renovaci\'on llamado proceso de renovaci\'on encajado para $X$.

\begin{Note}
Un proceso regenerativo con media de la longitud de ciclo finita es llamado positivo recurrente.
\end{Note}


\begin{Def}
Para $x$ fijo y para cada $t\geq0$, sea $I_{x}\left(t\right)=1$ si $X\left(t\right)\leq x$,  $I_{x}\left(t\right)=0$ en caso contrario, y def\'inanse los tiempos promedio
\begin{eqnarray*}
\overline{X}&=&lim_{t\rightarrow\infty}\frac{1}{t}\int_{0}^{\infty}X\left(u\right)du\\
\prob\left(X_{\infty}\leq x\right)&=&lim_{t\rightarrow\infty}\frac{1}{t}\int_{0}^{\infty}I_{x}\left(u\right)du,
\end{eqnarray*}
cuando estos l\'imites existan.
\end{Def}

Como consecuencia del teorema de Renovaci\'on-Recompensa, se tiene que el primer l\'imite  existe y es igual a la constante
\begin{eqnarray*}
\overline{X}&=&\frac{\esp\left[\int_{0}^{R_{1}}X\left(t\right)dt\right]}{\esp\left[R_{1}\right]},
\end{eqnarray*}
suponiendo que ambas esperanzas son finitas.

\begin{Note}
\begin{itemize}
\item[a)] Si el proceso regenerativo $X$ es positivo recurrente y tiene trayectorias muestrales no negativas, entonces la ecuaci\'on anterior es v\'alida.
\item[b)] Si $X$ es positivo recurrente regenerativo, podemos construir una \'unica versi\'on estacionaria de este proceso, $X_{e}=\left\{X_{e}\left(t\right)\right\}$, donde $X_{e}$ es un proceso estoc\'astico regenerativo y estrictamente estacionario, con distribuci\'on marginal distribuida como $X_{\infty}$
\end{itemize}
\end{Note}

%__________________________________________________________________________________________
%\subsection{Procesos Regenerativos Estacionarios - Stidham \cite{Stidham}}
%__________________________________________________________________________________________


Un proceso estoc\'astico a tiempo continuo $\left\{V\left(t\right),t\geq0\right\}$ es un proceso regenerativo si existe una sucesi\'on de variables aleatorias independientes e id\'enticamente distribuidas $\left\{X_{1},X_{2},\ldots\right\}$, sucesi\'on de renovaci\'on, tal que para cualquier conjunto de Borel $A$, 

\begin{eqnarray*}
\prob\left\{V\left(t\right)\in A|X_{1}+X_{2}+\cdots+X_{R\left(t\right)}=s,\left\{V\left(\tau\right),\tau<s\right\}\right\}=\prob\left\{V\left(t-s\right)\in A|X_{1}>t-s\right\},
\end{eqnarray*}
para todo $0\leq s\leq t$, donde $R\left(t\right)=\max\left\{X_{1}+X_{2}+\cdots+X_{j}\leq t\right\}=$n\'umero de renovaciones ({\emph{puntos de regeneraci\'on}}) que ocurren en $\left[0,t\right]$. El intervalo $\left[0,X_{1}\right)$ es llamado {\emph{primer ciclo de regeneraci\'on}} de $\left\{V\left(t \right),t\geq0\right\}$, $\left[X_{1},X_{1}+X_{2}\right)$ el {\emph{segundo ciclo de regeneraci\'on}}, y as\'i sucesivamente.

Sea $X=X_{1}$ y sea $F$ la funci\'on de distrbuci\'on de $X$


\begin{Def}
Se define el proceso estacionario, $\left\{V^{*}\left(t\right),t\geq0\right\}$, para $\left\{V\left(t\right),t\geq0\right\}$ por

\begin{eqnarray*}
\prob\left\{V\left(t\right)\in A\right\}=\frac{1}{\esp\left[X\right]}\int_{0}^{\infty}\prob\left\{V\left(t+x\right)\in A|X>x\right\}\left(1-F\left(x\right)\right)dx,
\end{eqnarray*} 
para todo $t\geq0$ y todo conjunto de Borel $A$.
\end{Def}

\begin{Def}
Una distribuci\'on se dice que es {\emph{aritm\'etica}} si todos sus puntos de incremento son m\'ultiplos de la forma $0,\lambda, 2\lambda,\ldots$ para alguna $\lambda>0$ entera.
\end{Def}


\begin{Def}
Una modificaci\'on medible de un proceso $\left\{V\left(t\right),t\geq0\right\}$, es una versi\'on de este, $\left\{V\left(t,w\right)\right\}$ conjuntamente medible para $t\geq0$ y para $w\in S$, $S$ espacio de estados para $\left\{V\left(t\right),t\geq0\right\}$.
\end{Def}

\begin{Teo}
Sea $\left\{V\left(t\right),t\geq\right\}$ un proceso regenerativo no negativo con modificaci\'on medible. Sea $\esp\left[X\right]<\infty$. Entonces el proceso estacionario dado por la ecuaci\'on anterior est\'a bien definido y tiene funci\'on de distribuci\'on independiente de $t$, adem\'as
\begin{itemize}
\item[i)] \begin{eqnarray*}
\esp\left[V^{*}\left(0\right)\right]&=&\frac{\esp\left[\int_{0}^{X}V\left(s\right)ds\right]}{\esp\left[X\right]}\end{eqnarray*}
\item[ii)] Si $\esp\left[V^{*}\left(0\right)\right]<\infty$, equivalentemente, si $\esp\left[\int_{0}^{X}V\left(s\right)ds\right]<\infty$,entonces
\begin{eqnarray*}
\frac{\int_{0}^{t}V\left(s\right)ds}{t}\rightarrow\frac{\esp\left[\int_{0}^{X}V\left(s\right)ds\right]}{\esp\left[X\right]}
\end{eqnarray*}
con probabilidad 1 y en media, cuando $t\rightarrow\infty$.
\end{itemize}
\end{Teo}
%________________________________________________________________________
\subsection{Procesos Regenerativos Sigman, Thorisson y Wolff \cite{Sigman2}}
%________________________________________________________________________


\begin{Def}[Definici\'on Cl\'asica]
Un proceso estoc\'astico $X=\left\{X\left(t\right):t\geq0\right\}$ es llamado regenerativo is existe una variable aleatoria $R_{1}>0$ tal que
\begin{itemize}
\item[i)] $\left\{X\left(t+R_{1}\right):t\geq0\right\}$ es independiente de $\left\{\left\{X\left(t\right):t<R_{1}\right\},\right\}$
\item[ii)] $\left\{X\left(t+R_{1}\right):t\geq0\right\}$ es estoc\'asticamente equivalente a $\left\{X\left(t\right):t>0\right\}$
\end{itemize}

Llamamos a $R_{1}$ tiempo de regeneraci\'on, y decimos que $X$ se regenera en este punto.
\end{Def}

$\left\{X\left(t+R_{1}\right)\right\}$ es regenerativo con tiempo de regeneraci\'on $R_{2}$, independiente de $R_{1}$ pero con la misma distribuci\'on que $R_{1}$. Procediendo de esta manera se obtiene una secuencia de variables aleatorias independientes e id\'enticamente distribuidas $\left\{R_{n}\right\}$ llamados longitudes de ciclo. Si definimos a $Z_{k}\equiv R_{1}+R_{2}+\cdots+R_{k}$, se tiene un proceso de renovaci\'on llamado proceso de renovaci\'on encajado para $X$.


\begin{Note}
La existencia de un primer tiempo de regeneraci\'on, $R_{1}$, implica la existencia de una sucesi\'on completa de estos tiempos $R_{1},R_{2}\ldots,$ que satisfacen la propiedad deseada \cite{Sigman2}.
\end{Note}


\begin{Note} Para la cola $GI/GI/1$ los usuarios arriban con tiempos $t_{n}$ y son atendidos con tiempos de servicio $S_{n}$, los tiempos de arribo forman un proceso de renovaci\'on  con tiempos entre arribos independientes e identicamente distribuidos (\texttt{i.i.d.})$T_{n}=t_{n}-t_{n-1}$, adem\'as los tiempos de servicio son \texttt{i.i.d.} e independientes de los procesos de arribo. Por \textit{estable} se entiende que $\esp S_{n}<\esp T_{n}<\infty$.
\end{Note}
 


\begin{Def}
Para $x$ fijo y para cada $t\geq0$, sea $I_{x}\left(t\right)=1$ si $X\left(t\right)\leq x$,  $I_{x}\left(t\right)=0$ en caso contrario, y def\'inanse los tiempos promedio
\begin{eqnarray*}
\overline{X}&=&lim_{t\rightarrow\infty}\frac{1}{t}\int_{0}^{\infty}X\left(u\right)du\\
\prob\left(X_{\infty}\leq x\right)&=&lim_{t\rightarrow\infty}\frac{1}{t}\int_{0}^{\infty}I_{x}\left(u\right)du,
\end{eqnarray*}
cuando estos l\'imites existan.
\end{Def}

Como consecuencia del teorema de Renovaci\'on-Recompensa, se tiene que el primer l\'imite  existe y es igual a la constante
\begin{eqnarray*}
\overline{X}&=&\frac{\esp\left[\int_{0}^{R_{1}}X\left(t\right)dt\right]}{\esp\left[R_{1}\right]},
\end{eqnarray*}
suponiendo que ambas esperanzas son finitas.
 
\begin{Note}
Funciones de procesos regenerativos son regenerativas, es decir, si $X\left(t\right)$ es regenerativo y se define el proceso $Y\left(t\right)$ por $Y\left(t\right)=f\left(X\left(t\right)\right)$ para alguna funci\'on Borel medible $f\left(\cdot\right)$. Adem\'as $Y$ es regenerativo con los mismos tiempos de renovaci\'on que $X$. 

En general, los tiempos de renovaci\'on, $Z_{k}$ de un proceso regenerativo no requieren ser tiempos de paro con respecto a la evoluci\'on de $X\left(t\right)$.
\end{Note} 

\begin{Note}
Una funci\'on de un proceso de Markov, usualmente no ser\'a un proceso de Markov, sin embargo ser\'a regenerativo si el proceso de Markov lo es.
\end{Note}

 
\begin{Note}
Un proceso regenerativo con media de la longitud de ciclo finita es llamado positivo recurrente.
\end{Note}


\begin{Note}
\begin{itemize}
\item[a)] Si el proceso regenerativo $X$ es positivo recurrente y tiene trayectorias muestrales no negativas, entonces la ecuaci\'on anterior es v\'alida.
\item[b)] Si $X$ es positivo recurrente regenerativo, podemos construir una \'unica versi\'on estacionaria de este proceso, $X_{e}=\left\{X_{e}\left(t\right)\right\}$, donde $X_{e}$ es un proceso estoc\'astico regenerativo y estrictamente estacionario, con distribuci\'on marginal distribuida como $X_{\infty}$
\end{itemize}
\end{Note}


%__________________________________________________________________________________________
\subsection{Procesos Regenerativos Estacionarios - Stidham \cite{Stidham}}
%__________________________________________________________________________________________


Un proceso estoc\'astico a tiempo continuo $\left\{V\left(t\right),t\geq0\right\}$ es un proceso regenerativo si existe una sucesi\'on de variables aleatorias independientes e id\'enticamente distribuidas $\left\{X_{1},X_{2},\ldots\right\}$, sucesi\'on de renovaci\'on, tal que para cualquier conjunto de Borel $A$, 

\begin{eqnarray*}
\prob\left\{V\left(t\right)\in A|X_{1}+X_{2}+\cdots+X_{R\left(t\right)}=s,\left\{V\left(\tau\right),\tau<s\right\}\right\}=\prob\left\{V\left(t-s\right)\in A|X_{1}>t-s\right\},
\end{eqnarray*}
para todo $0\leq s\leq t$, donde $R\left(t\right)=\max\left\{X_{1}+X_{2}+\cdots+X_{j}\leq t\right\}=$n\'umero de renovaciones ({\emph{puntos de regeneraci\'on}}) que ocurren en $\left[0,t\right]$. El intervalo $\left[0,X_{1}\right)$ es llamado {\emph{primer ciclo de regeneraci\'on}} de $\left\{V\left(t \right),t\geq0\right\}$, $\left[X_{1},X_{1}+X_{2}\right)$ el {\emph{segundo ciclo de regeneraci\'on}}, y as\'i sucesivamente.

Sea $X=X_{1}$ y sea $F$ la funci\'on de distrbuci\'on de $X$


\begin{Def}
Se define el proceso estacionario, $\left\{V^{*}\left(t\right),t\geq0\right\}$, para $\left\{V\left(t\right),t\geq0\right\}$ por

\begin{eqnarray*}
\prob\left\{V\left(t\right)\in A\right\}=\frac{1}{\esp\left[X\right]}\int_{0}^{\infty}\prob\left\{V\left(t+x\right)\in A|X>x\right\}\left(1-F\left(x\right)\right)dx,
\end{eqnarray*} 
para todo $t\geq0$ y todo conjunto de Borel $A$.
\end{Def}

\begin{Def}
Una distribuci\'on se dice que es {\emph{aritm\'etica}} si todos sus puntos de incremento son m\'ultiplos de la forma $0,\lambda, 2\lambda,\ldots$ para alguna $\lambda>0$ entera.
\end{Def}


\begin{Def}
Una modificaci\'on medible de un proceso $\left\{V\left(t\right),t\geq0\right\}$, es una versi\'on de este, $\left\{V\left(t,w\right)\right\}$ conjuntamente medible para $t\geq0$ y para $w\in S$, $S$ espacio de estados para $\left\{V\left(t\right),t\geq0\right\}$.
\end{Def}

\begin{Teo}
Sea $\left\{V\left(t\right),t\geq\right\}$ un proceso regenerativo no negativo con modificaci\'on medible. Sea $\esp\left[X\right]<\infty$. Entonces el proceso estacionario dado por la ecuaci\'on anterior est\'a bien definido y tiene funci\'on de distribuci\'on independiente de $t$, adem\'as
\begin{itemize}
\item[i)] \begin{eqnarray*}
\esp\left[V^{*}\left(0\right)\right]&=&\frac{\esp\left[\int_{0}^{X}V\left(s\right)ds\right]}{\esp\left[X\right]}\end{eqnarray*}
\item[ii)] Si $\esp\left[V^{*}\left(0\right)\right]<\infty$, equivalentemente, si $\esp\left[\int_{0}^{X}V\left(s\right)ds\right]<\infty$,entonces
\begin{eqnarray*}
\frac{\int_{0}^{t}V\left(s\right)ds}{t}\rightarrow\frac{\esp\left[\int_{0}^{X}V\left(s\right)ds\right]}{\esp\left[X\right]}
\end{eqnarray*}
con probabilidad 1 y en media, cuando $t\rightarrow\infty$.
\end{itemize}
\end{Teo}

\begin{Coro}
Sea $\left\{V\left(t\right),t\geq0\right\}$ un proceso regenerativo no negativo, con modificaci\'on medible. Si $\esp <\infty$, $F$ es no-aritm\'etica, y para todo $x\geq0$, $P\left\{V\left(t\right)\leq x,C>x\right\}$ es de variaci\'on acotada como funci\'on de $t$ en cada intervalo finito $\left[0,\tau\right]$, entonces $V\left(t\right)$ converge en distribuci\'on  cuando $t\rightarrow\infty$ y $$\esp V=\frac{\esp \int_{0}^{X}V\left(s\right)ds}{\esp X}$$
Donde $V$ tiene la distribuci\'on l\'imite de $V\left(t\right)$ cuando $t\rightarrow\infty$.

\end{Coro}

Para el caso discreto se tienen resultados similares.



%______________________________________________________________________
%\subsubsection{Procesos de Renovaci\'on}
%______________________________________________________________________

\begin{Def}%\label{Def.Tn}
Sean $0\leq T_{1}\leq T_{2}\leq \ldots$ son tiempos aleatorios infinitos en los cuales ocurren ciertos eventos. El n\'umero de tiempos $T_{n}$ en el intervalo $\left[0,t\right)$ es

\begin{eqnarray}
N\left(t\right)=\sum_{n=1}^{\infty}\indora\left(T_{n}\leq t\right),
\end{eqnarray}
para $t\geq0$.
\end{Def}

Si se consideran los puntos $T_{n}$ como elementos de $\rea_{+}$, y $N\left(t\right)$ es el n\'umero de puntos en $\rea$. El proceso denotado por $\left\{N\left(t\right):t\geq0\right\}$, denotado por $N\left(t\right)$, es un proceso puntual en $\rea_{+}$. Los $T_{n}$ son los tiempos de ocurrencia, el proceso puntual $N\left(t\right)$ es simple si su n\'umero de ocurrencias son distintas: $0<T_{1}<T_{2}<\ldots$ casi seguramente.

\begin{Def}
Un proceso puntual $N\left(t\right)$ es un proceso de renovaci\'on si los tiempos de interocurrencia $\xi_{n}=T_{n}-T_{n-1}$, para $n\geq1$, son independientes e identicamente distribuidos con distribuci\'on $F$, donde $F\left(0\right)=0$ y $T_{0}=0$. Los $T_{n}$ son llamados tiempos de renovaci\'on, referente a la independencia o renovaci\'on de la informaci\'on estoc\'astica en estos tiempos. Los $\xi_{n}$ son los tiempos de inter-renovaci\'on, y $N\left(t\right)$ es el n\'umero de renovaciones en el intervalo $\left[0,t\right)$
\end{Def}


\begin{Note}
Para definir un proceso de renovaci\'on para cualquier contexto, solamente hay que especificar una distribuci\'on $F$, con $F\left(0\right)=0$, para los tiempos de inter-renovaci\'on. La funci\'on $F$ en turno degune las otra variables aleatorias. De manera formal, existe un espacio de probabilidad y una sucesi\'on de variables aleatorias $\xi_{1},\xi_{2},\ldots$ definidas en este con distribuci\'on $F$. Entonces las otras cantidades son $T_{n}=\sum_{k=1}^{n}\xi_{k}$ y $N\left(t\right)=\sum_{n=1}^{\infty}\indora\left(T_{n}\leq t\right)$, donde $T_{n}\rightarrow\infty$ casi seguramente por la Ley Fuerte de los Grandes Números.
\end{Note}

%___________________________________________________________________________________________
%
%\subsubsection{Teorema Principal de Renovaci\'on}
%___________________________________________________________________________________________
%

\begin{Note} Una funci\'on $h:\rea_{+}\rightarrow\rea$ es Directamente Riemann Integrable en los siguientes casos:
\begin{itemize}
\item[a)] $h\left(t\right)\geq0$ es decreciente y Riemann Integrable.
\item[b)] $h$ es continua excepto posiblemente en un conjunto de Lebesgue de medida 0, y $|h\left(t\right)|\leq b\left(t\right)$, donde $b$ es DRI.
\end{itemize}
\end{Note}

\begin{Teo}[Teorema Principal de Renovaci\'on]
Si $F$ es no aritm\'etica y $h\left(t\right)$ es Directamente Riemann Integrable (DRI), entonces

\begin{eqnarray*}
lim_{t\rightarrow\infty}U\star h=\frac{1}{\mu}\int_{\rea_{+}}h\left(s\right)ds.
\end{eqnarray*}
\end{Teo}

\begin{Prop}
Cualquier funci\'on $H\left(t\right)$ acotada en intervalos finitos y que es 0 para $t<0$ puede expresarse como
\begin{eqnarray*}
H\left(t\right)=U\star h\left(t\right)\textrm{,  donde }h\left(t\right)=H\left(t\right)-F\star H\left(t\right)
\end{eqnarray*}
\end{Prop}

\begin{Def}
Un proceso estoc\'astico $X\left(t\right)$ es crudamente regenerativo en un tiempo aleatorio positivo $T$ si
\begin{eqnarray*}
\esp\left[X\left(T+t\right)|T\right]=\esp\left[X\left(t\right)\right]\textrm{, para }t\geq0,\end{eqnarray*}
y con las esperanzas anteriores finitas.
\end{Def}

\begin{Prop}
Sup\'ongase que $X\left(t\right)$ es un proceso crudamente regenerativo en $T$, que tiene distribuci\'on $F$. Si $\esp\left[X\left(t\right)\right]$ es acotado en intervalos finitos, entonces
\begin{eqnarray*}
\esp\left[X\left(t\right)\right]=U\star h\left(t\right)\textrm{,  donde }h\left(t\right)=\esp\left[X\left(t\right)\indora\left(T>t\right)\right].
\end{eqnarray*}
\end{Prop}

\begin{Teo}[Regeneraci\'on Cruda]
Sup\'ongase que $X\left(t\right)$ es un proceso con valores positivo crudamente regenerativo en $T$, y def\'inase $M=\sup\left\{|X\left(t\right)|:t\leq T\right\}$. Si $T$ es no aritm\'etico y $M$ y $MT$ tienen media finita, entonces
\begin{eqnarray*}
lim_{t\rightarrow\infty}\esp\left[X\left(t\right)\right]=\frac{1}{\mu}\int_{\rea_{+}}h\left(s\right)ds,
\end{eqnarray*}
donde $h\left(t\right)=\esp\left[X\left(t\right)\indora\left(T>t\right)\right]$.
\end{Teo}

%___________________________________________________________________________________________
%
\subsection{Propiedades de los Procesos de Renovaci\'on}
%___________________________________________________________________________________________
%

Los tiempos $T_{n}$ est\'an relacionados con los conteos de $N\left(t\right)$ por

\begin{eqnarray*}
\left\{N\left(t\right)\geq n\right\}&=&\left\{T_{n}\leq t\right\}\\
T_{N\left(t\right)}\leq &t&<T_{N\left(t\right)+1},
\end{eqnarray*}

adem\'as $N\left(T_{n}\right)=n$, y 

\begin{eqnarray*}
N\left(t\right)=\max\left\{n:T_{n}\leq t\right\}=\min\left\{n:T_{n+1}>t\right\}
\end{eqnarray*}

Por propiedades de la convoluci\'on se sabe que

\begin{eqnarray*}
P\left\{T_{n}\leq t\right\}=F^{n\star}\left(t\right)
\end{eqnarray*}
que es la $n$-\'esima convoluci\'on de $F$. Entonces 

\begin{eqnarray*}
\left\{N\left(t\right)\geq n\right\}&=&\left\{T_{n}\leq t\right\}\\
P\left\{N\left(t\right)\leq n\right\}&=&1-F^{\left(n+1\right)\star}\left(t\right)
\end{eqnarray*}

Adem\'as usando el hecho de que $\esp\left[N\left(t\right)\right]=\sum_{n=1}^{\infty}P\left\{N\left(t\right)\geq n\right\}$
se tiene que

\begin{eqnarray*}
\esp\left[N\left(t\right)\right]=\sum_{n=1}^{\infty}F^{n\star}\left(t\right)
\end{eqnarray*}

\begin{Prop}
Para cada $t\geq0$, la funci\'on generadora de momentos $\esp\left[e^{\alpha N\left(t\right)}\right]$ existe para alguna $\alpha$ en una vecindad del 0, y de aqu\'i que $\esp\left[N\left(t\right)^{m}\right]<\infty$, para $m\geq1$.
\end{Prop}


\begin{Note}
Si el primer tiempo de renovaci\'on $\xi_{1}$ no tiene la misma distribuci\'on que el resto de las $\xi_{n}$, para $n\geq2$, a $N\left(t\right)$ se le llama Proceso de Renovaci\'on retardado, donde si $\xi$ tiene distribuci\'on $G$, entonces el tiempo $T_{n}$ de la $n$-\'esima renovaci\'on tiene distribuci\'on $G\star F^{\left(n-1\right)\star}\left(t\right)$
\end{Note}


\begin{Teo}
Para una constante $\mu\leq\infty$ ( o variable aleatoria), las siguientes expresiones son equivalentes:

\begin{eqnarray}
lim_{n\rightarrow\infty}n^{-1}T_{n}&=&\mu,\textrm{ c.s.}\\
lim_{t\rightarrow\infty}t^{-1}N\left(t\right)&=&1/\mu,\textrm{ c.s.}
\end{eqnarray}
\end{Teo}


Es decir, $T_{n}$ satisface la Ley Fuerte de los Grandes N\'umeros s\'i y s\'olo s\'i $N\left/t\right)$ la cumple.


\begin{Coro}[Ley Fuerte de los Grandes N\'umeros para Procesos de Renovaci\'on]
Si $N\left(t\right)$ es un proceso de renovaci\'on cuyos tiempos de inter-renovaci\'on tienen media $\mu\leq\infty$, entonces
\begin{eqnarray}
t^{-1}N\left(t\right)\rightarrow 1/\mu,\textrm{ c.s. cuando }t\rightarrow\infty.
\end{eqnarray}

\end{Coro}


Considerar el proceso estoc\'astico de valores reales $\left\{Z\left(t\right):t\geq0\right\}$ en el mismo espacio de probabilidad que $N\left(t\right)$

\begin{Def}
Para el proceso $\left\{Z\left(t\right):t\geq0\right\}$ se define la fluctuaci\'on m\'axima de $Z\left(t\right)$ en el intervalo $\left(T_{n-1},T_{n}\right]$:
\begin{eqnarray*}
M_{n}=\sup_{T_{n-1}<t\leq T_{n}}|Z\left(t\right)-Z\left(T_{n-1}\right)|
\end{eqnarray*}
\end{Def}

\begin{Teo}
Sup\'ongase que $n^{-1}T_{n}\rightarrow\mu$ c.s. cuando $n\rightarrow\infty$, donde $\mu\leq\infty$ es una constante o variable aleatoria. Sea $a$ una constante o variable aleatoria que puede ser infinita cuando $\mu$ es finita, y considere las expresiones l\'imite:
\begin{eqnarray}
lim_{n\rightarrow\infty}n^{-1}Z\left(T_{n}\right)&=&a,\textrm{ c.s.}\\
lim_{t\rightarrow\infty}t^{-1}Z\left(t\right)&=&a/\mu,\textrm{ c.s.}
\end{eqnarray}
La segunda expresi\'on implica la primera. Conversamente, la primera implica la segunda si el proceso $Z\left(t\right)$ es creciente, o si $lim_{n\rightarrow\infty}n^{-1}M_{n}=0$ c.s.
\end{Teo}

\begin{Coro}
Si $N\left(t\right)$ es un proceso de renovaci\'on, y $\left(Z\left(T_{n}\right)-Z\left(T_{n-1}\right),M_{n}\right)$, para $n\geq1$, son variables aleatorias independientes e id\'enticamente distribuidas con media finita, entonces,
\begin{eqnarray}
lim_{t\rightarrow\infty}t^{-1}Z\left(t\right)\rightarrow\frac{\esp\left[Z\left(T_{1}\right)-Z\left(T_{0}\right)\right]}{\esp\left[T_{1}\right]},\textrm{ c.s. cuando  }t\rightarrow\infty.
\end{eqnarray}
\end{Coro}



%___________________________________________________________________________________________
%
%\subsection{Propiedades de los Procesos de Renovaci\'on}
%___________________________________________________________________________________________
%

Los tiempos $T_{n}$ est\'an relacionados con los conteos de $N\left(t\right)$ por

\begin{eqnarray*}
\left\{N\left(t\right)\geq n\right\}&=&\left\{T_{n}\leq t\right\}\\
T_{N\left(t\right)}\leq &t&<T_{N\left(t\right)+1},
\end{eqnarray*}

adem\'as $N\left(T_{n}\right)=n$, y 

\begin{eqnarray*}
N\left(t\right)=\max\left\{n:T_{n}\leq t\right\}=\min\left\{n:T_{n+1}>t\right\}
\end{eqnarray*}

Por propiedades de la convoluci\'on se sabe que

\begin{eqnarray*}
P\left\{T_{n}\leq t\right\}=F^{n\star}\left(t\right)
\end{eqnarray*}
que es la $n$-\'esima convoluci\'on de $F$. Entonces 

\begin{eqnarray*}
\left\{N\left(t\right)\geq n\right\}&=&\left\{T_{n}\leq t\right\}\\
P\left\{N\left(t\right)\leq n\right\}&=&1-F^{\left(n+1\right)\star}\left(t\right)
\end{eqnarray*}

Adem\'as usando el hecho de que $\esp\left[N\left(t\right)\right]=\sum_{n=1}^{\infty}P\left\{N\left(t\right)\geq n\right\}$
se tiene que

\begin{eqnarray*}
\esp\left[N\left(t\right)\right]=\sum_{n=1}^{\infty}F^{n\star}\left(t\right)
\end{eqnarray*}

\begin{Prop}
Para cada $t\geq0$, la funci\'on generadora de momentos $\esp\left[e^{\alpha N\left(t\right)}\right]$ existe para alguna $\alpha$ en una vecindad del 0, y de aqu\'i que $\esp\left[N\left(t\right)^{m}\right]<\infty$, para $m\geq1$.
\end{Prop}


\begin{Note}
Si el primer tiempo de renovaci\'on $\xi_{1}$ no tiene la misma distribuci\'on que el resto de las $\xi_{n}$, para $n\geq2$, a $N\left(t\right)$ se le llama Proceso de Renovaci\'on retardado, donde si $\xi$ tiene distribuci\'on $G$, entonces el tiempo $T_{n}$ de la $n$-\'esima renovaci\'on tiene distribuci\'on $G\star F^{\left(n-1\right)\star}\left(t\right)$
\end{Note}


\begin{Teo}
Para una constante $\mu\leq\infty$ ( o variable aleatoria), las siguientes expresiones son equivalentes:

\begin{eqnarray}
lim_{n\rightarrow\infty}n^{-1}T_{n}&=&\mu,\textrm{ c.s.}\\
lim_{t\rightarrow\infty}t^{-1}N\left(t\right)&=&1/\mu,\textrm{ c.s.}
\end{eqnarray}
\end{Teo}


Es decir, $T_{n}$ satisface la Ley Fuerte de los Grandes N\'umeros s\'i y s\'olo s\'i $N\left/t\right)$ la cumple.


\begin{Coro}[Ley Fuerte de los Grandes N\'umeros para Procesos de Renovaci\'on]
Si $N\left(t\right)$ es un proceso de renovaci\'on cuyos tiempos de inter-renovaci\'on tienen media $\mu\leq\infty$, entonces
\begin{eqnarray}
t^{-1}N\left(t\right)\rightarrow 1/\mu,\textrm{ c.s. cuando }t\rightarrow\infty.
\end{eqnarray}

\end{Coro}


Considerar el proceso estoc\'astico de valores reales $\left\{Z\left(t\right):t\geq0\right\}$ en el mismo espacio de probabilidad que $N\left(t\right)$

\begin{Def}
Para el proceso $\left\{Z\left(t\right):t\geq0\right\}$ se define la fluctuaci\'on m\'axima de $Z\left(t\right)$ en el intervalo $\left(T_{n-1},T_{n}\right]$:
\begin{eqnarray*}
M_{n}=\sup_{T_{n-1}<t\leq T_{n}}|Z\left(t\right)-Z\left(T_{n-1}\right)|
\end{eqnarray*}
\end{Def}

\begin{Teo}
Sup\'ongase que $n^{-1}T_{n}\rightarrow\mu$ c.s. cuando $n\rightarrow\infty$, donde $\mu\leq\infty$ es una constante o variable aleatoria. Sea $a$ una constante o variable aleatoria que puede ser infinita cuando $\mu$ es finita, y considere las expresiones l\'imite:
\begin{eqnarray}
lim_{n\rightarrow\infty}n^{-1}Z\left(T_{n}\right)&=&a,\textrm{ c.s.}\\
lim_{t\rightarrow\infty}t^{-1}Z\left(t\right)&=&a/\mu,\textrm{ c.s.}
\end{eqnarray}
La segunda expresi\'on implica la primera. Conversamente, la primera implica la segunda si el proceso $Z\left(t\right)$ es creciente, o si $lim_{n\rightarrow\infty}n^{-1}M_{n}=0$ c.s.
\end{Teo}

\begin{Coro}
Si $N\left(t\right)$ es un proceso de renovaci\'on, y $\left(Z\left(T_{n}\right)-Z\left(T_{n-1}\right),M_{n}\right)$, para $n\geq1$, son variables aleatorias independientes e id\'enticamente distribuidas con media finita, entonces,
\begin{eqnarray}
lim_{t\rightarrow\infty}t^{-1}Z\left(t\right)\rightarrow\frac{\esp\left[Z\left(T_{1}\right)-Z\left(T_{0}\right)\right]}{\esp\left[T_{1}\right]},\textrm{ c.s. cuando  }t\rightarrow\infty.
\end{eqnarray}
\end{Coro}


%___________________________________________________________________________________________
%
%\subsection{Propiedades de los Procesos de Renovaci\'on}
%___________________________________________________________________________________________
%

Los tiempos $T_{n}$ est\'an relacionados con los conteos de $N\left(t\right)$ por

\begin{eqnarray*}
\left\{N\left(t\right)\geq n\right\}&=&\left\{T_{n}\leq t\right\}\\
T_{N\left(t\right)}\leq &t&<T_{N\left(t\right)+1},
\end{eqnarray*}

adem\'as $N\left(T_{n}\right)=n$, y 

\begin{eqnarray*}
N\left(t\right)=\max\left\{n:T_{n}\leq t\right\}=\min\left\{n:T_{n+1}>t\right\}
\end{eqnarray*}

Por propiedades de la convoluci\'on se sabe que

\begin{eqnarray*}
P\left\{T_{n}\leq t\right\}=F^{n\star}\left(t\right)
\end{eqnarray*}
que es la $n$-\'esima convoluci\'on de $F$. Entonces 

\begin{eqnarray*}
\left\{N\left(t\right)\geq n\right\}&=&\left\{T_{n}\leq t\right\}\\
P\left\{N\left(t\right)\leq n\right\}&=&1-F^{\left(n+1\right)\star}\left(t\right)
\end{eqnarray*}

Adem\'as usando el hecho de que $\esp\left[N\left(t\right)\right]=\sum_{n=1}^{\infty}P\left\{N\left(t\right)\geq n\right\}$
se tiene que

\begin{eqnarray*}
\esp\left[N\left(t\right)\right]=\sum_{n=1}^{\infty}F^{n\star}\left(t\right)
\end{eqnarray*}

\begin{Prop}
Para cada $t\geq0$, la funci\'on generadora de momentos $\esp\left[e^{\alpha N\left(t\right)}\right]$ existe para alguna $\alpha$ en una vecindad del 0, y de aqu\'i que $\esp\left[N\left(t\right)^{m}\right]<\infty$, para $m\geq1$.
\end{Prop}


\begin{Note}
Si el primer tiempo de renovaci\'on $\xi_{1}$ no tiene la misma distribuci\'on que el resto de las $\xi_{n}$, para $n\geq2$, a $N\left(t\right)$ se le llama Proceso de Renovaci\'on retardado, donde si $\xi$ tiene distribuci\'on $G$, entonces el tiempo $T_{n}$ de la $n$-\'esima renovaci\'on tiene distribuci\'on $G\star F^{\left(n-1\right)\star}\left(t\right)$
\end{Note}


\begin{Teo}
Para una constante $\mu\leq\infty$ ( o variable aleatoria), las siguientes expresiones son equivalentes:

\begin{eqnarray}
lim_{n\rightarrow\infty}n^{-1}T_{n}&=&\mu,\textrm{ c.s.}\\
lim_{t\rightarrow\infty}t^{-1}N\left(t\right)&=&1/\mu,\textrm{ c.s.}
\end{eqnarray}
\end{Teo}


Es decir, $T_{n}$ satisface la Ley Fuerte de los Grandes N\'umeros s\'i y s\'olo s\'i $N\left/t\right)$ la cumple.


\begin{Coro}[Ley Fuerte de los Grandes N\'umeros para Procesos de Renovaci\'on]
Si $N\left(t\right)$ es un proceso de renovaci\'on cuyos tiempos de inter-renovaci\'on tienen media $\mu\leq\infty$, entonces
\begin{eqnarray}
t^{-1}N\left(t\right)\rightarrow 1/\mu,\textrm{ c.s. cuando }t\rightarrow\infty.
\end{eqnarray}

\end{Coro}


Considerar el proceso estoc\'astico de valores reales $\left\{Z\left(t\right):t\geq0\right\}$ en el mismo espacio de probabilidad que $N\left(t\right)$

\begin{Def}
Para el proceso $\left\{Z\left(t\right):t\geq0\right\}$ se define la fluctuaci\'on m\'axima de $Z\left(t\right)$ en el intervalo $\left(T_{n-1},T_{n}\right]$:
\begin{eqnarray*}
M_{n}=\sup_{T_{n-1}<t\leq T_{n}}|Z\left(t\right)-Z\left(T_{n-1}\right)|
\end{eqnarray*}
\end{Def}

\begin{Teo}
Sup\'ongase que $n^{-1}T_{n}\rightarrow\mu$ c.s. cuando $n\rightarrow\infty$, donde $\mu\leq\infty$ es una constante o variable aleatoria. Sea $a$ una constante o variable aleatoria que puede ser infinita cuando $\mu$ es finita, y considere las expresiones l\'imite:
\begin{eqnarray}
lim_{n\rightarrow\infty}n^{-1}Z\left(T_{n}\right)&=&a,\textrm{ c.s.}\\
lim_{t\rightarrow\infty}t^{-1}Z\left(t\right)&=&a/\mu,\textrm{ c.s.}
\end{eqnarray}
La segunda expresi\'on implica la primera. Conversamente, la primera implica la segunda si el proceso $Z\left(t\right)$ es creciente, o si $lim_{n\rightarrow\infty}n^{-1}M_{n}=0$ c.s.
\end{Teo}

\begin{Coro}
Si $N\left(t\right)$ es un proceso de renovaci\'on, y $\left(Z\left(T_{n}\right)-Z\left(T_{n-1}\right),M_{n}\right)$, para $n\geq1$, son variables aleatorias independientes e id\'enticamente distribuidas con media finita, entonces,
\begin{eqnarray}
lim_{t\rightarrow\infty}t^{-1}Z\left(t\right)\rightarrow\frac{\esp\left[Z\left(T_{1}\right)-Z\left(T_{0}\right)\right]}{\esp\left[T_{1}\right]},\textrm{ c.s. cuando  }t\rightarrow\infty.
\end{eqnarray}
\end{Coro}

%___________________________________________________________________________________________
%
%\subsection{Propiedades de los Procesos de Renovaci\'on}
%___________________________________________________________________________________________
%

Los tiempos $T_{n}$ est\'an relacionados con los conteos de $N\left(t\right)$ por

\begin{eqnarray*}
\left\{N\left(t\right)\geq n\right\}&=&\left\{T_{n}\leq t\right\}\\
T_{N\left(t\right)}\leq &t&<T_{N\left(t\right)+1},
\end{eqnarray*}

adem\'as $N\left(T_{n}\right)=n$, y 

\begin{eqnarray*}
N\left(t\right)=\max\left\{n:T_{n}\leq t\right\}=\min\left\{n:T_{n+1}>t\right\}
\end{eqnarray*}

Por propiedades de la convoluci\'on se sabe que

\begin{eqnarray*}
P\left\{T_{n}\leq t\right\}=F^{n\star}\left(t\right)
\end{eqnarray*}
que es la $n$-\'esima convoluci\'on de $F$. Entonces 

\begin{eqnarray*}
\left\{N\left(t\right)\geq n\right\}&=&\left\{T_{n}\leq t\right\}\\
P\left\{N\left(t\right)\leq n\right\}&=&1-F^{\left(n+1\right)\star}\left(t\right)
\end{eqnarray*}

Adem\'as usando el hecho de que $\esp\left[N\left(t\right)\right]=\sum_{n=1}^{\infty}P\left\{N\left(t\right)\geq n\right\}$
se tiene que

\begin{eqnarray*}
\esp\left[N\left(t\right)\right]=\sum_{n=1}^{\infty}F^{n\star}\left(t\right)
\end{eqnarray*}

\begin{Prop}
Para cada $t\geq0$, la funci\'on generadora de momentos $\esp\left[e^{\alpha N\left(t\right)}\right]$ existe para alguna $\alpha$ en una vecindad del 0, y de aqu\'i que $\esp\left[N\left(t\right)^{m}\right]<\infty$, para $m\geq1$.
\end{Prop}


\begin{Note}
Si el primer tiempo de renovaci\'on $\xi_{1}$ no tiene la misma distribuci\'on que el resto de las $\xi_{n}$, para $n\geq2$, a $N\left(t\right)$ se le llama Proceso de Renovaci\'on retardado, donde si $\xi$ tiene distribuci\'on $G$, entonces el tiempo $T_{n}$ de la $n$-\'esima renovaci\'on tiene distribuci\'on $G\star F^{\left(n-1\right)\star}\left(t\right)$
\end{Note}


\begin{Teo}
Para una constante $\mu\leq\infty$ ( o variable aleatoria), las siguientes expresiones son equivalentes:

\begin{eqnarray}
lim_{n\rightarrow\infty}n^{-1}T_{n}&=&\mu,\textrm{ c.s.}\\
lim_{t\rightarrow\infty}t^{-1}N\left(t\right)&=&1/\mu,\textrm{ c.s.}
\end{eqnarray}
\end{Teo}


Es decir, $T_{n}$ satisface la Ley Fuerte de los Grandes N\'umeros s\'i y s\'olo s\'i $N\left/t\right)$ la cumple.


\begin{Coro}[Ley Fuerte de los Grandes N\'umeros para Procesos de Renovaci\'on]
Si $N\left(t\right)$ es un proceso de renovaci\'on cuyos tiempos de inter-renovaci\'on tienen media $\mu\leq\infty$, entonces
\begin{eqnarray}
t^{-1}N\left(t\right)\rightarrow 1/\mu,\textrm{ c.s. cuando }t\rightarrow\infty.
\end{eqnarray}

\end{Coro}


Considerar el proceso estoc\'astico de valores reales $\left\{Z\left(t\right):t\geq0\right\}$ en el mismo espacio de probabilidad que $N\left(t\right)$

\begin{Def}
Para el proceso $\left\{Z\left(t\right):t\geq0\right\}$ se define la fluctuaci\'on m\'axima de $Z\left(t\right)$ en el intervalo $\left(T_{n-1},T_{n}\right]$:
\begin{eqnarray*}
M_{n}=\sup_{T_{n-1}<t\leq T_{n}}|Z\left(t\right)-Z\left(T_{n-1}\right)|
\end{eqnarray*}
\end{Def}

\begin{Teo}
Sup\'ongase que $n^{-1}T_{n}\rightarrow\mu$ c.s. cuando $n\rightarrow\infty$, donde $\mu\leq\infty$ es una constante o variable aleatoria. Sea $a$ una constante o variable aleatoria que puede ser infinita cuando $\mu$ es finita, y considere las expresiones l\'imite:
\begin{eqnarray}
lim_{n\rightarrow\infty}n^{-1}Z\left(T_{n}\right)&=&a,\textrm{ c.s.}\\
lim_{t\rightarrow\infty}t^{-1}Z\left(t\right)&=&a/\mu,\textrm{ c.s.}
\end{eqnarray}
La segunda expresi\'on implica la primera. Conversamente, la primera implica la segunda si el proceso $Z\left(t\right)$ es creciente, o si $lim_{n\rightarrow\infty}n^{-1}M_{n}=0$ c.s.
\end{Teo}

\begin{Coro}
Si $N\left(t\right)$ es un proceso de renovaci\'on, y $\left(Z\left(T_{n}\right)-Z\left(T_{n-1}\right),M_{n}\right)$, para $n\geq1$, son variables aleatorias independientes e id\'enticamente distribuidas con media finita, entonces,
\begin{eqnarray}
lim_{t\rightarrow\infty}t^{-1}Z\left(t\right)\rightarrow\frac{\esp\left[Z\left(T_{1}\right)-Z\left(T_{0}\right)\right]}{\esp\left[T_{1}\right]},\textrm{ c.s. cuando  }t\rightarrow\infty.
\end{eqnarray}
\end{Coro}
%___________________________________________________________________________________________
%
%\subsection{Propiedades de los Procesos de Renovaci\'on}
%___________________________________________________________________________________________
%

Los tiempos $T_{n}$ est\'an relacionados con los conteos de $N\left(t\right)$ por

\begin{eqnarray*}
\left\{N\left(t\right)\geq n\right\}&=&\left\{T_{n}\leq t\right\}\\
T_{N\left(t\right)}\leq &t&<T_{N\left(t\right)+1},
\end{eqnarray*}

adem\'as $N\left(T_{n}\right)=n$, y 

\begin{eqnarray*}
N\left(t\right)=\max\left\{n:T_{n}\leq t\right\}=\min\left\{n:T_{n+1}>t\right\}
\end{eqnarray*}

Por propiedades de la convoluci\'on se sabe que

\begin{eqnarray*}
P\left\{T_{n}\leq t\right\}=F^{n\star}\left(t\right)
\end{eqnarray*}
que es la $n$-\'esima convoluci\'on de $F$. Entonces 

\begin{eqnarray*}
\left\{N\left(t\right)\geq n\right\}&=&\left\{T_{n}\leq t\right\}\\
P\left\{N\left(t\right)\leq n\right\}&=&1-F^{\left(n+1\right)\star}\left(t\right)
\end{eqnarray*}

Adem\'as usando el hecho de que $\esp\left[N\left(t\right)\right]=\sum_{n=1}^{\infty}P\left\{N\left(t\right)\geq n\right\}$
se tiene que

\begin{eqnarray*}
\esp\left[N\left(t\right)\right]=\sum_{n=1}^{\infty}F^{n\star}\left(t\right)
\end{eqnarray*}

\begin{Prop}
Para cada $t\geq0$, la funci\'on generadora de momentos $\esp\left[e^{\alpha N\left(t\right)}\right]$ existe para alguna $\alpha$ en una vecindad del 0, y de aqu\'i que $\esp\left[N\left(t\right)^{m}\right]<\infty$, para $m\geq1$.
\end{Prop}


\begin{Note}
Si el primer tiempo de renovaci\'on $\xi_{1}$ no tiene la misma distribuci\'on que el resto de las $\xi_{n}$, para $n\geq2$, a $N\left(t\right)$ se le llama Proceso de Renovaci\'on retardado, donde si $\xi$ tiene distribuci\'on $G$, entonces el tiempo $T_{n}$ de la $n$-\'esima renovaci\'on tiene distribuci\'on $G\star F^{\left(n-1\right)\star}\left(t\right)$
\end{Note}


\begin{Teo}
Para una constante $\mu\leq\infty$ ( o variable aleatoria), las siguientes expresiones son equivalentes:

\begin{eqnarray}
lim_{n\rightarrow\infty}n^{-1}T_{n}&=&\mu,\textrm{ c.s.}\\
lim_{t\rightarrow\infty}t^{-1}N\left(t\right)&=&1/\mu,\textrm{ c.s.}
\end{eqnarray}
\end{Teo}


Es decir, $T_{n}$ satisface la Ley Fuerte de los Grandes N\'umeros s\'i y s\'olo s\'i $N\left/t\right)$ la cumple.


\begin{Coro}[Ley Fuerte de los Grandes N\'umeros para Procesos de Renovaci\'on]
Si $N\left(t\right)$ es un proceso de renovaci\'on cuyos tiempos de inter-renovaci\'on tienen media $\mu\leq\infty$, entonces
\begin{eqnarray}
t^{-1}N\left(t\right)\rightarrow 1/\mu,\textrm{ c.s. cuando }t\rightarrow\infty.
\end{eqnarray}

\end{Coro}


Considerar el proceso estoc\'astico de valores reales $\left\{Z\left(t\right):t\geq0\right\}$ en el mismo espacio de probabilidad que $N\left(t\right)$

\begin{Def}
Para el proceso $\left\{Z\left(t\right):t\geq0\right\}$ se define la fluctuaci\'on m\'axima de $Z\left(t\right)$ en el intervalo $\left(T_{n-1},T_{n}\right]$:
\begin{eqnarray*}
M_{n}=\sup_{T_{n-1}<t\leq T_{n}}|Z\left(t\right)-Z\left(T_{n-1}\right)|
\end{eqnarray*}
\end{Def}

\begin{Teo}
Sup\'ongase que $n^{-1}T_{n}\rightarrow\mu$ c.s. cuando $n\rightarrow\infty$, donde $\mu\leq\infty$ es una constante o variable aleatoria. Sea $a$ una constante o variable aleatoria que puede ser infinita cuando $\mu$ es finita, y considere las expresiones l\'imite:
\begin{eqnarray}
lim_{n\rightarrow\infty}n^{-1}Z\left(T_{n}\right)&=&a,\textrm{ c.s.}\\
lim_{t\rightarrow\infty}t^{-1}Z\left(t\right)&=&a/\mu,\textrm{ c.s.}
\end{eqnarray}
La segunda expresi\'on implica la primera. Conversamente, la primera implica la segunda si el proceso $Z\left(t\right)$ es creciente, o si $lim_{n\rightarrow\infty}n^{-1}M_{n}=0$ c.s.
\end{Teo}

\begin{Coro}
Si $N\left(t\right)$ es un proceso de renovaci\'on, y $\left(Z\left(T_{n}\right)-Z\left(T_{n-1}\right),M_{n}\right)$, para $n\geq1$, son variables aleatorias independientes e id\'enticamente distribuidas con media finita, entonces,
\begin{eqnarray}
lim_{t\rightarrow\infty}t^{-1}Z\left(t\right)\rightarrow\frac{\esp\left[Z\left(T_{1}\right)-Z\left(T_{0}\right)\right]}{\esp\left[T_{1}\right]},\textrm{ c.s. cuando  }t\rightarrow\infty.
\end{eqnarray}
\end{Coro}


%___________________________________________________________________________________________
%
\subsection{Funci\'on de Renovaci\'on}
%___________________________________________________________________________________________
%


\begin{Def}
Sea $h\left(t\right)$ funci\'on de valores reales en $\rea$ acotada en intervalos finitos e igual a cero para $t<0$ La ecuaci\'on de renovaci\'on para $h\left(t\right)$ y la distribuci\'on $F$ es

\begin{eqnarray}%\label{Ec.Renovacion}
H\left(t\right)=h\left(t\right)+\int_{\left[0,t\right]}H\left(t-s\right)dF\left(s\right)\textrm{,    }t\geq0,
\end{eqnarray}
donde $H\left(t\right)$ es una funci\'on de valores reales. Esto es $H=h+F\star H$. Decimos que $H\left(t\right)$ es soluci\'on de esta ecuaci\'on si satisface la ecuaci\'on, y es acotada en intervalos finitos e iguales a cero para $t<0$.
\end{Def}

\begin{Prop}
La funci\'on $U\star h\left(t\right)$ es la \'unica soluci\'on de la ecuaci\'on de renovaci\'on (\ref{Ec.Renovacion}).
\end{Prop}

\begin{Teo}[Teorema Renovaci\'on Elemental]
\begin{eqnarray*}
t^{-1}U\left(t\right)\rightarrow 1/\mu\textrm{,    cuando }t\rightarrow\infty.
\end{eqnarray*}
\end{Teo}

%___________________________________________________________________________________________
%
%\subsection{Funci\'on de Renovaci\'on}
%___________________________________________________________________________________________
%


Sup\'ongase que $N\left(t\right)$ es un proceso de renovaci\'on con distribuci\'on $F$ con media finita $\mu$.

\begin{Def}
La funci\'on de renovaci\'on asociada con la distribuci\'on $F$, del proceso $N\left(t\right)$, es
\begin{eqnarray*}
U\left(t\right)=\sum_{n=1}^{\infty}F^{n\star}\left(t\right),\textrm{   }t\geq0,
\end{eqnarray*}
donde $F^{0\star}\left(t\right)=\indora\left(t\geq0\right)$.
\end{Def}


\begin{Prop}
Sup\'ongase que la distribuci\'on de inter-renovaci\'on $F$ tiene densidad $f$. Entonces $U\left(t\right)$ tambi\'en tiene densidad, para $t>0$, y es $U^{'}\left(t\right)=\sum_{n=0}^{\infty}f^{n\star}\left(t\right)$. Adem\'as
\begin{eqnarray*}
\prob\left\{N\left(t\right)>N\left(t-\right)\right\}=0\textrm{,   }t\geq0.
\end{eqnarray*}
\end{Prop}

\begin{Def}
La Transformada de Laplace-Stieljes de $F$ est\'a dada por

\begin{eqnarray*}
\hat{F}\left(\alpha\right)=\int_{\rea_{+}}e^{-\alpha t}dF\left(t\right)\textrm{,  }\alpha\geq0.
\end{eqnarray*}
\end{Def}

Entonces

\begin{eqnarray*}
\hat{U}\left(\alpha\right)=\sum_{n=0}^{\infty}\hat{F^{n\star}}\left(\alpha\right)=\sum_{n=0}^{\infty}\hat{F}\left(\alpha\right)^{n}=\frac{1}{1-\hat{F}\left(\alpha\right)}.
\end{eqnarray*}


\begin{Prop}
La Transformada de Laplace $\hat{U}\left(\alpha\right)$ y $\hat{F}\left(\alpha\right)$ determina una a la otra de manera \'unica por la relaci\'on $\hat{U}\left(\alpha\right)=\frac{1}{1-\hat{F}\left(\alpha\right)}$.
\end{Prop}


\begin{Note}
Un proceso de renovaci\'on $N\left(t\right)$ cuyos tiempos de inter-renovaci\'on tienen media finita, es un proceso Poisson con tasa $\lambda$ si y s\'olo s\'i $\esp\left[U\left(t\right)\right]=\lambda t$, para $t\geq0$.
\end{Note}


\begin{Teo}
Sea $N\left(t\right)$ un proceso puntual simple con puntos de localizaci\'on $T_{n}$ tal que $\eta\left(t\right)=\esp\left[N\left(\right)\right]$ es finita para cada $t$. Entonces para cualquier funci\'on $f:\rea_{+}\rightarrow\rea$,
\begin{eqnarray*}
\esp\left[\sum_{n=1}^{N\left(\right)}f\left(T_{n}\right)\right]=\int_{\left(0,t\right]}f\left(s\right)d\eta\left(s\right)\textrm{,  }t\geq0,
\end{eqnarray*}
suponiendo que la integral exista. Adem\'as si $X_{1},X_{2},\ldots$ son variables aleatorias definidas en el mismo espacio de probabilidad que el proceso $N\left(t\right)$ tal que $\esp\left[X_{n}|T_{n}=s\right]=f\left(s\right)$, independiente de $n$. Entonces
\begin{eqnarray*}
\esp\left[\sum_{n=1}^{N\left(t\right)}X_{n}\right]=\int_{\left(0,t\right]}f\left(s\right)d\eta\left(s\right)\textrm{,  }t\geq0,
\end{eqnarray*} 
suponiendo que la integral exista. 
\end{Teo}

\begin{Coro}[Identidad de Wald para Renovaciones]
Para el proceso de renovaci\'on $N\left(t\right)$,
\begin{eqnarray*}
\esp\left[T_{N\left(t\right)+1}\right]=\mu\esp\left[N\left(t\right)+1\right]\textrm{,  }t\geq0,
\end{eqnarray*}  
\end{Coro}

%______________________________________________________________________
%\subsection{Procesos de Renovaci\'on}
%______________________________________________________________________

\begin{Def}%\label{Def.Tn}
Sean $0\leq T_{1}\leq T_{2}\leq \ldots$ son tiempos aleatorios infinitos en los cuales ocurren ciertos eventos. El n\'umero de tiempos $T_{n}$ en el intervalo $\left[0,t\right)$ es

\begin{eqnarray}
N\left(t\right)=\sum_{n=1}^{\infty}\indora\left(T_{n}\leq t\right),
\end{eqnarray}
para $t\geq0$.
\end{Def}

Si se consideran los puntos $T_{n}$ como elementos de $\rea_{+}$, y $N\left(t\right)$ es el n\'umero de puntos en $\rea$. El proceso denotado por $\left\{N\left(t\right):t\geq0\right\}$, denotado por $N\left(t\right)$, es un proceso puntual en $\rea_{+}$. Los $T_{n}$ son los tiempos de ocurrencia, el proceso puntual $N\left(t\right)$ es simple si su n\'umero de ocurrencias son distintas: $0<T_{1}<T_{2}<\ldots$ casi seguramente.

\begin{Def}
Un proceso puntual $N\left(t\right)$ es un proceso de renovaci\'on si los tiempos de interocurrencia $\xi_{n}=T_{n}-T_{n-1}$, para $n\geq1$, son independientes e identicamente distribuidos con distribuci\'on $F$, donde $F\left(0\right)=0$ y $T_{0}=0$. Los $T_{n}$ son llamados tiempos de renovaci\'on, referente a la independencia o renovaci\'on de la informaci\'on estoc\'astica en estos tiempos. Los $\xi_{n}$ son los tiempos de inter-renovaci\'on, y $N\left(t\right)$ es el n\'umero de renovaciones en el intervalo $\left[0,t\right)$
\end{Def}


\begin{Note}
Para definir un proceso de renovaci\'on para cualquier contexto, solamente hay que especificar una distribuci\'on $F$, con $F\left(0\right)=0$, para los tiempos de inter-renovaci\'on. La funci\'on $F$ en turno degune las otra variables aleatorias. De manera formal, existe un espacio de probabilidad y una sucesi\'on de variables aleatorias $\xi_{1},\xi_{2},\ldots$ definidas en este con distribuci\'on $F$. Entonces las otras cantidades son $T_{n}=\sum_{k=1}^{n}\xi_{k}$ y $N\left(t\right)=\sum_{n=1}^{\infty}\indora\left(T_{n}\leq t\right)$, donde $T_{n}\rightarrow\infty$ casi seguramente por la Ley Fuerte de los Grandes Números.
\end{Note}

%___________________________________________________________________________________________
%
\subsection{Renewal and Regenerative Processes: Serfozo\cite{Serfozo}}
%___________________________________________________________________________________________
%
\begin{Def}%\label{Def.Tn}
Sean $0\leq T_{1}\leq T_{2}\leq \ldots$ son tiempos aleatorios infinitos en los cuales ocurren ciertos eventos. El n\'umero de tiempos $T_{n}$ en el intervalo $\left[0,t\right)$ es

\begin{eqnarray}
N\left(t\right)=\sum_{n=1}^{\infty}\indora\left(T_{n}\leq t\right),
\end{eqnarray}
para $t\geq0$.
\end{Def}

Si se consideran los puntos $T_{n}$ como elementos de $\rea_{+}$, y $N\left(t\right)$ es el n\'umero de puntos en $\rea$. El proceso denotado por $\left\{N\left(t\right):t\geq0\right\}$, denotado por $N\left(t\right)$, es un proceso puntual en $\rea_{+}$. Los $T_{n}$ son los tiempos de ocurrencia, el proceso puntual $N\left(t\right)$ es simple si su n\'umero de ocurrencias son distintas: $0<T_{1}<T_{2}<\ldots$ casi seguramente.

\begin{Def}
Un proceso puntual $N\left(t\right)$ es un proceso de renovaci\'on si los tiempos de interocurrencia $\xi_{n}=T_{n}-T_{n-1}$, para $n\geq1$, son independientes e identicamente distribuidos con distribuci\'on $F$, donde $F\left(0\right)=0$ y $T_{0}=0$. Los $T_{n}$ son llamados tiempos de renovaci\'on, referente a la independencia o renovaci\'on de la informaci\'on estoc\'astica en estos tiempos. Los $\xi_{n}$ son los tiempos de inter-renovaci\'on, y $N\left(t\right)$ es el n\'umero de renovaciones en el intervalo $\left[0,t\right)$
\end{Def}


\begin{Note}
Para definir un proceso de renovaci\'on para cualquier contexto, solamente hay que especificar una distribuci\'on $F$, con $F\left(0\right)=0$, para los tiempos de inter-renovaci\'on. La funci\'on $F$ en turno degune las otra variables aleatorias. De manera formal, existe un espacio de probabilidad y una sucesi\'on de variables aleatorias $\xi_{1},\xi_{2},\ldots$ definidas en este con distribuci\'on $F$. Entonces las otras cantidades son $T_{n}=\sum_{k=1}^{n}\xi_{k}$ y $N\left(t\right)=\sum_{n=1}^{\infty}\indora\left(T_{n}\leq t\right)$, donde $T_{n}\rightarrow\infty$ casi seguramente por la Ley Fuerte de los Grandes N\'umeros.
\end{Note}

Los tiempos $T_{n}$ est\'an relacionados con los conteos de $N\left(t\right)$ por

\begin{eqnarray*}
\left\{N\left(t\right)\geq n\right\}&=&\left\{T_{n}\leq t\right\}\\
T_{N\left(t\right)}\leq &t&<T_{N\left(t\right)+1},
\end{eqnarray*}

adem\'as $N\left(T_{n}\right)=n$, y 

\begin{eqnarray*}
N\left(t\right)=\max\left\{n:T_{n}\leq t\right\}=\min\left\{n:T_{n+1}>t\right\}
\end{eqnarray*}

Por propiedades de la convoluci\'on se sabe que

\begin{eqnarray*}
P\left\{T_{n}\leq t\right\}=F^{n\star}\left(t\right)
\end{eqnarray*}
que es la $n$-\'esima convoluci\'on de $F$. Entonces 

\begin{eqnarray*}
\left\{N\left(t\right)\geq n\right\}&=&\left\{T_{n}\leq t\right\}\\
P\left\{N\left(t\right)\leq n\right\}&=&1-F^{\left(n+1\right)\star}\left(t\right)
\end{eqnarray*}

Adem\'as usando el hecho de que $\esp\left[N\left(t\right)\right]=\sum_{n=1}^{\infty}P\left\{N\left(t\right)\geq n\right\}$
se tiene que

\begin{eqnarray*}
\esp\left[N\left(t\right)\right]=\sum_{n=1}^{\infty}F^{n\star}\left(t\right)
\end{eqnarray*}

\begin{Prop}
Para cada $t\geq0$, la funci\'on generadora de momentos $\esp\left[e^{\alpha N\left(t\right)}\right]$ existe para alguna $\alpha$ en una vecindad del 0, y de aqu\'i que $\esp\left[N\left(t\right)^{m}\right]<\infty$, para $m\geq1$.
\end{Prop}

\begin{Ejem}[\textbf{Proceso Poisson}]

Suponga que se tienen tiempos de inter-renovaci\'on \textit{i.i.d.} del proceso de renovaci\'on $N\left(t\right)$ tienen distribuci\'on exponencial $F\left(t\right)=q-e^{-\lambda t}$ con tasa $\lambda$. Entonces $N\left(t\right)$ es un proceso Poisson con tasa $\lambda$.

\end{Ejem}


\begin{Note}
Si el primer tiempo de renovaci\'on $\xi_{1}$ no tiene la misma distribuci\'on que el resto de las $\xi_{n}$, para $n\geq2$, a $N\left(t\right)$ se le llama Proceso de Renovaci\'on retardado, donde si $\xi$ tiene distribuci\'on $G$, entonces el tiempo $T_{n}$ de la $n$-\'esima renovaci\'on tiene distribuci\'on $G\star F^{\left(n-1\right)\star}\left(t\right)$
\end{Note}


\begin{Teo}
Para una constante $\mu\leq\infty$ ( o variable aleatoria), las siguientes expresiones son equivalentes:

\begin{eqnarray}
lim_{n\rightarrow\infty}n^{-1}T_{n}&=&\mu,\textrm{ c.s.}\\
lim_{t\rightarrow\infty}t^{-1}N\left(t\right)&=&1/\mu,\textrm{ c.s.}
\end{eqnarray}
\end{Teo}


Es decir, $T_{n}$ satisface la Ley Fuerte de los Grandes N\'umeros s\'i y s\'olo s\'i $N\left/t\right)$ la cumple.


\begin{Coro}[Ley Fuerte de los Grandes N\'umeros para Procesos de Renovaci\'on]
Si $N\left(t\right)$ es un proceso de renovaci\'on cuyos tiempos de inter-renovaci\'on tienen media $\mu\leq\infty$, entonces
\begin{eqnarray}
t^{-1}N\left(t\right)\rightarrow 1/\mu,\textrm{ c.s. cuando }t\rightarrow\infty.
\end{eqnarray}

\end{Coro}


Considerar el proceso estoc\'astico de valores reales $\left\{Z\left(t\right):t\geq0\right\}$ en el mismo espacio de probabilidad que $N\left(t\right)$

\begin{Def}
Para el proceso $\left\{Z\left(t\right):t\geq0\right\}$ se define la fluctuaci\'on m\'axima de $Z\left(t\right)$ en el intervalo $\left(T_{n-1},T_{n}\right]$:
\begin{eqnarray*}
M_{n}=\sup_{T_{n-1}<t\leq T_{n}}|Z\left(t\right)-Z\left(T_{n-1}\right)|
\end{eqnarray*}
\end{Def}

\begin{Teo}
Sup\'ongase que $n^{-1}T_{n}\rightarrow\mu$ c.s. cuando $n\rightarrow\infty$, donde $\mu\leq\infty$ es una constante o variable aleatoria. Sea $a$ una constante o variable aleatoria que puede ser infinita cuando $\mu$ es finita, y considere las expresiones l\'imite:
\begin{eqnarray}
lim_{n\rightarrow\infty}n^{-1}Z\left(T_{n}\right)&=&a,\textrm{ c.s.}\\
lim_{t\rightarrow\infty}t^{-1}Z\left(t\right)&=&a/\mu,\textrm{ c.s.}
\end{eqnarray}
La segunda expresi\'on implica la primera. Conversamente, la primera implica la segunda si el proceso $Z\left(t\right)$ es creciente, o si $lim_{n\rightarrow\infty}n^{-1}M_{n}=0$ c.s.
\end{Teo}

\begin{Coro}
Si $N\left(t\right)$ es un proceso de renovaci\'on, y $\left(Z\left(T_{n}\right)-Z\left(T_{n-1}\right),M_{n}\right)$, para $n\geq1$, son variables aleatorias independientes e id\'enticamente distribuidas con media finita, entonces,
\begin{eqnarray}
lim_{t\rightarrow\infty}t^{-1}Z\left(t\right)\rightarrow\frac{\esp\left[Z\left(T_{1}\right)-Z\left(T_{0}\right)\right]}{\esp\left[T_{1}\right]},\textrm{ c.s. cuando  }t\rightarrow\infty.
\end{eqnarray}
\end{Coro}


Sup\'ongase que $N\left(t\right)$ es un proceso de renovaci\'on con distribuci\'on $F$ con media finita $\mu$.

\begin{Def}
La funci\'on de renovaci\'on asociada con la distribuci\'on $F$, del proceso $N\left(t\right)$, es
\begin{eqnarray*}
U\left(t\right)=\sum_{n=1}^{\infty}F^{n\star}\left(t\right),\textrm{   }t\geq0,
\end{eqnarray*}
donde $F^{0\star}\left(t\right)=\indora\left(t\geq0\right)$.
\end{Def}


\begin{Prop}
Sup\'ongase que la distribuci\'on de inter-renovaci\'on $F$ tiene densidad $f$. Entonces $U\left(t\right)$ tambi\'en tiene densidad, para $t>0$, y es $U^{'}\left(t\right)=\sum_{n=0}^{\infty}f^{n\star}\left(t\right)$. Adem\'as
\begin{eqnarray*}
\prob\left\{N\left(t\right)>N\left(t-\right)\right\}=0\textrm{,   }t\geq0.
\end{eqnarray*}
\end{Prop}

\begin{Def}
La Transformada de Laplace-Stieljes de $F$ est\'a dada por

\begin{eqnarray*}
\hat{F}\left(\alpha\right)=\int_{\rea_{+}}e^{-\alpha t}dF\left(t\right)\textrm{,  }\alpha\geq0.
\end{eqnarray*}
\end{Def}

Entonces

\begin{eqnarray*}
\hat{U}\left(\alpha\right)=\sum_{n=0}^{\infty}\hat{F^{n\star}}\left(\alpha\right)=\sum_{n=0}^{\infty}\hat{F}\left(\alpha\right)^{n}=\frac{1}{1-\hat{F}\left(\alpha\right)}.
\end{eqnarray*}


\begin{Prop}
La Transformada de Laplace $\hat{U}\left(\alpha\right)$ y $\hat{F}\left(\alpha\right)$ determina una a la otra de manera \'unica por la relaci\'on $\hat{U}\left(\alpha\right)=\frac{1}{1-\hat{F}\left(\alpha\right)}$.
\end{Prop}


\begin{Note}
Un proceso de renovaci\'on $N\left(t\right)$ cuyos tiempos de inter-renovaci\'on tienen media finita, es un proceso Poisson con tasa $\lambda$ si y s\'olo s\'i $\esp\left[U\left(t\right)\right]=\lambda t$, para $t\geq0$.
\end{Note}


\begin{Teo}
Sea $N\left(t\right)$ un proceso puntual simple con puntos de localizaci\'on $T_{n}$ tal que $\eta\left(t\right)=\esp\left[N\left(\right)\right]$ es finita para cada $t$. Entonces para cualquier funci\'on $f:\rea_{+}\rightarrow\rea$,
\begin{eqnarray*}
\esp\left[\sum_{n=1}^{N\left(\right)}f\left(T_{n}\right)\right]=\int_{\left(0,t\right]}f\left(s\right)d\eta\left(s\right)\textrm{,  }t\geq0,
\end{eqnarray*}
suponiendo que la integral exista. Adem\'as si $X_{1},X_{2},\ldots$ son variables aleatorias definidas en el mismo espacio de probabilidad que el proceso $N\left(t\right)$ tal que $\esp\left[X_{n}|T_{n}=s\right]=f\left(s\right)$, independiente de $n$. Entonces
\begin{eqnarray*}
\esp\left[\sum_{n=1}^{N\left(t\right)}X_{n}\right]=\int_{\left(0,t\right]}f\left(s\right)d\eta\left(s\right)\textrm{,  }t\geq0,
\end{eqnarray*} 
suponiendo que la integral exista. 
\end{Teo}

\begin{Coro}[Identidad de Wald para Renovaciones]
Para el proceso de renovaci\'on $N\left(t\right)$,
\begin{eqnarray*}
\esp\left[T_{N\left(t\right)+1}\right]=\mu\esp\left[N\left(t\right)+1\right]\textrm{,  }t\geq0,
\end{eqnarray*}  
\end{Coro}


\begin{Def}
Sea $h\left(t\right)$ funci\'on de valores reales en $\rea$ acotada en intervalos finitos e igual a cero para $t<0$ La ecuaci\'on de renovaci\'on para $h\left(t\right)$ y la distribuci\'on $F$ es

\begin{eqnarray}%\label{Ec.Renovacion}
H\left(t\right)=h\left(t\right)+\int_{\left[0,t\right]}H\left(t-s\right)dF\left(s\right)\textrm{,    }t\geq0,
\end{eqnarray}
donde $H\left(t\right)$ es una funci\'on de valores reales. Esto es $H=h+F\star H$. Decimos que $H\left(t\right)$ es soluci\'on de esta ecuaci\'on si satisface la ecuaci\'on, y es acotada en intervalos finitos e iguales a cero para $t<0$.
\end{Def}

\begin{Prop}
La funci\'on $U\star h\left(t\right)$ es la \'unica soluci\'on de la ecuaci\'on de renovaci\'on (\ref{Ec.Renovacion}).
\end{Prop}

\begin{Teo}[Teorema Renovaci\'on Elemental]
\begin{eqnarray*}
t^{-1}U\left(t\right)\rightarrow 1/\mu\textrm{,    cuando }t\rightarrow\infty.
\end{eqnarray*}
\end{Teo}



Sup\'ongase que $N\left(t\right)$ es un proceso de renovaci\'on con distribuci\'on $F$ con media finita $\mu$.

\begin{Def}
La funci\'on de renovaci\'on asociada con la distribuci\'on $F$, del proceso $N\left(t\right)$, es
\begin{eqnarray*}
U\left(t\right)=\sum_{n=1}^{\infty}F^{n\star}\left(t\right),\textrm{   }t\geq0,
\end{eqnarray*}
donde $F^{0\star}\left(t\right)=\indora\left(t\geq0\right)$.
\end{Def}


\begin{Prop}
Sup\'ongase que la distribuci\'on de inter-renovaci\'on $F$ tiene densidad $f$. Entonces $U\left(t\right)$ tambi\'en tiene densidad, para $t>0$, y es $U^{'}\left(t\right)=\sum_{n=0}^{\infty}f^{n\star}\left(t\right)$. Adem\'as
\begin{eqnarray*}
\prob\left\{N\left(t\right)>N\left(t-\right)\right\}=0\textrm{,   }t\geq0.
\end{eqnarray*}
\end{Prop}

\begin{Def}
La Transformada de Laplace-Stieljes de $F$ est\'a dada por

\begin{eqnarray*}
\hat{F}\left(\alpha\right)=\int_{\rea_{+}}e^{-\alpha t}dF\left(t\right)\textrm{,  }\alpha\geq0.
\end{eqnarray*}
\end{Def}

Entonces

\begin{eqnarray*}
\hat{U}\left(\alpha\right)=\sum_{n=0}^{\infty}\hat{F^{n\star}}\left(\alpha\right)=\sum_{n=0}^{\infty}\hat{F}\left(\alpha\right)^{n}=\frac{1}{1-\hat{F}\left(\alpha\right)}.
\end{eqnarray*}


\begin{Prop}
La Transformada de Laplace $\hat{U}\left(\alpha\right)$ y $\hat{F}\left(\alpha\right)$ determina una a la otra de manera \'unica por la relaci\'on $\hat{U}\left(\alpha\right)=\frac{1}{1-\hat{F}\left(\alpha\right)}$.
\end{Prop}


\begin{Note}
Un proceso de renovaci\'on $N\left(t\right)$ cuyos tiempos de inter-renovaci\'on tienen media finita, es un proceso Poisson con tasa $\lambda$ si y s\'olo s\'i $\esp\left[U\left(t\right)\right]=\lambda t$, para $t\geq0$.
\end{Note}


\begin{Teo}
Sea $N\left(t\right)$ un proceso puntual simple con puntos de localizaci\'on $T_{n}$ tal que $\eta\left(t\right)=\esp\left[N\left(\right)\right]$ es finita para cada $t$. Entonces para cualquier funci\'on $f:\rea_{+}\rightarrow\rea$,
\begin{eqnarray*}
\esp\left[\sum_{n=1}^{N\left(\right)}f\left(T_{n}\right)\right]=\int_{\left(0,t\right]}f\left(s\right)d\eta\left(s\right)\textrm{,  }t\geq0,
\end{eqnarray*}
suponiendo que la integral exista. Adem\'as si $X_{1},X_{2},\ldots$ son variables aleatorias definidas en el mismo espacio de probabilidad que el proceso $N\left(t\right)$ tal que $\esp\left[X_{n}|T_{n}=s\right]=f\left(s\right)$, independiente de $n$. Entonces
\begin{eqnarray*}
\esp\left[\sum_{n=1}^{N\left(t\right)}X_{n}\right]=\int_{\left(0,t\right]}f\left(s\right)d\eta\left(s\right)\textrm{,  }t\geq0,
\end{eqnarray*} 
suponiendo que la integral exista. 
\end{Teo}

\begin{Coro}[Identidad de Wald para Renovaciones]
Para el proceso de renovaci\'on $N\left(t\right)$,
\begin{eqnarray*}
\esp\left[T_{N\left(t\right)+1}\right]=\mu\esp\left[N\left(t\right)+1\right]\textrm{,  }t\geq0,
\end{eqnarray*}  
\end{Coro}


\begin{Def}
Sea $h\left(t\right)$ funci\'on de valores reales en $\rea$ acotada en intervalos finitos e igual a cero para $t<0$ La ecuaci\'on de renovaci\'on para $h\left(t\right)$ y la distribuci\'on $F$ es

\begin{eqnarray}%\label{Ec.Renovacion}
H\left(t\right)=h\left(t\right)+\int_{\left[0,t\right]}H\left(t-s\right)dF\left(s\right)\textrm{,    }t\geq0,
\end{eqnarray}
donde $H\left(t\right)$ es una funci\'on de valores reales. Esto es $H=h+F\star H$. Decimos que $H\left(t\right)$ es soluci\'on de esta ecuaci\'on si satisface la ecuaci\'on, y es acotada en intervalos finitos e iguales a cero para $t<0$.
\end{Def}

\begin{Prop}
La funci\'on $U\star h\left(t\right)$ es la \'unica soluci\'on de la ecuaci\'on de renovaci\'on (\ref{Ec.Renovacion}).
\end{Prop}

\begin{Teo}[Teorema Renovaci\'on Elemental]
\begin{eqnarray*}
t^{-1}U\left(t\right)\rightarrow 1/\mu\textrm{,    cuando }t\rightarrow\infty.
\end{eqnarray*}
\end{Teo}


\begin{Note} Una funci\'on $h:\rea_{+}\rightarrow\rea$ es Directamente Riemann Integrable en los siguientes casos:
\begin{itemize}
\item[a)] $h\left(t\right)\geq0$ es decreciente y Riemann Integrable.
\item[b)] $h$ es continua excepto posiblemente en un conjunto de Lebesgue de medida 0, y $|h\left(t\right)|\leq b\left(t\right)$, donde $b$ es DRI.
\end{itemize}
\end{Note}

\begin{Teo}[Teorema Principal de Renovaci\'on]
Si $F$ es no aritm\'etica y $h\left(t\right)$ es Directamente Riemann Integrable (DRI), entonces

\begin{eqnarray*}
lim_{t\rightarrow\infty}U\star h=\frac{1}{\mu}\int_{\rea_{+}}h\left(s\right)ds.
\end{eqnarray*}
\end{Teo}

\begin{Prop}
Cualquier funci\'on $H\left(t\right)$ acotada en intervalos finitos y que es 0 para $t<0$ puede expresarse como
\begin{eqnarray*}
H\left(t\right)=U\star h\left(t\right)\textrm{,  donde }h\left(t\right)=H\left(t\right)-F\star H\left(t\right)
\end{eqnarray*}
\end{Prop}

\begin{Def}
Un proceso estoc\'astico $X\left(t\right)$ es crudamente regenerativo en un tiempo aleatorio positivo $T$ si
\begin{eqnarray*}
\esp\left[X\left(T+t\right)|T\right]=\esp\left[X\left(t\right)\right]\textrm{, para }t\geq0,\end{eqnarray*}
y con las esperanzas anteriores finitas.
\end{Def}

\begin{Prop}
Sup\'ongase que $X\left(t\right)$ es un proceso crudamente regenerativo en $T$, que tiene distribuci\'on $F$. Si $\esp\left[X\left(t\right)\right]$ es acotado en intervalos finitos, entonces
\begin{eqnarray*}
\esp\left[X\left(t\right)\right]=U\star h\left(t\right)\textrm{,  donde }h\left(t\right)=\esp\left[X\left(t\right)\indora\left(T>t\right)\right].
\end{eqnarray*}
\end{Prop}

\begin{Teo}[Regeneraci\'on Cruda]
Sup\'ongase que $X\left(t\right)$ es un proceso con valores positivo crudamente regenerativo en $T$, y def\'inase $M=\sup\left\{|X\left(t\right)|:t\leq T\right\}$. Si $T$ es no aritm\'etico y $M$ y $MT$ tienen media finita, entonces
\begin{eqnarray*}
lim_{t\rightarrow\infty}\esp\left[X\left(t\right)\right]=\frac{1}{\mu}\int_{\rea_{+}}h\left(s\right)ds,
\end{eqnarray*}
donde $h\left(t\right)=\esp\left[X\left(t\right)\indora\left(T>t\right)\right]$.
\end{Teo}


\begin{Note} Una funci\'on $h:\rea_{+}\rightarrow\rea$ es Directamente Riemann Integrable en los siguientes casos:
\begin{itemize}
\item[a)] $h\left(t\right)\geq0$ es decreciente y Riemann Integrable.
\item[b)] $h$ es continua excepto posiblemente en un conjunto de Lebesgue de medida 0, y $|h\left(t\right)|\leq b\left(t\right)$, donde $b$ es DRI.
\end{itemize}
\end{Note}

\begin{Teo}[Teorema Principal de Renovaci\'on]
Si $F$ es no aritm\'etica y $h\left(t\right)$ es Directamente Riemann Integrable (DRI), entonces

\begin{eqnarray*}
lim_{t\rightarrow\infty}U\star h=\frac{1}{\mu}\int_{\rea_{+}}h\left(s\right)ds.
\end{eqnarray*}
\end{Teo}

\begin{Prop}
Cualquier funci\'on $H\left(t\right)$ acotada en intervalos finitos y que es 0 para $t<0$ puede expresarse como
\begin{eqnarray*}
H\left(t\right)=U\star h\left(t\right)\textrm{,  donde }h\left(t\right)=H\left(t\right)-F\star H\left(t\right)
\end{eqnarray*}
\end{Prop}

\begin{Def}
Un proceso estoc\'astico $X\left(t\right)$ es crudamente regenerativo en un tiempo aleatorio positivo $T$ si
\begin{eqnarray*}
\esp\left[X\left(T+t\right)|T\right]=\esp\left[X\left(t\right)\right]\textrm{, para }t\geq0,\end{eqnarray*}
y con las esperanzas anteriores finitas.
\end{Def}

\begin{Prop}
Sup\'ongase que $X\left(t\right)$ es un proceso crudamente regenerativo en $T$, que tiene distribuci\'on $F$. Si $\esp\left[X\left(t\right)\right]$ es acotado en intervalos finitos, entonces
\begin{eqnarray*}
\esp\left[X\left(t\right)\right]=U\star h\left(t\right)\textrm{,  donde }h\left(t\right)=\esp\left[X\left(t\right)\indora\left(T>t\right)\right].
\end{eqnarray*}
\end{Prop}

\begin{Teo}[Regeneraci\'on Cruda]
Sup\'ongase que $X\left(t\right)$ es un proceso con valores positivo crudamente regenerativo en $T$, y def\'inase $M=\sup\left\{|X\left(t\right)|:t\leq T\right\}$. Si $T$ es no aritm\'etico y $M$ y $MT$ tienen media finita, entonces
\begin{eqnarray*}
lim_{t\rightarrow\infty}\esp\left[X\left(t\right)\right]=\frac{1}{\mu}\int_{\rea_{+}}h\left(s\right)ds,
\end{eqnarray*}
donde $h\left(t\right)=\esp\left[X\left(t\right)\indora\left(T>t\right)\right]$.
\end{Teo}

\begin{Def}
Para el proceso $\left\{\left(N\left(t\right),X\left(t\right)\right):t\geq0\right\}$, sus trayectoria muestrales en el intervalo de tiempo $\left[T_{n-1},T_{n}\right)$ est\'an descritas por
\begin{eqnarray*}
\zeta_{n}=\left(\xi_{n},\left\{X\left(T_{n-1}+t\right):0\leq t<\xi_{n}\right\}\right)
\end{eqnarray*}
Este $\zeta_{n}$ es el $n$-\'esimo segmento del proceso. El proceso es regenerativo sobre los tiempos $T_{n}$ si sus segmentos $\zeta_{n}$ son independientes e id\'enticamennte distribuidos.
\end{Def}


\begin{Note}
Si $\tilde{X}\left(t\right)$ con espacio de estados $\tilde{S}$ es regenerativo sobre $T_{n}$, entonces $X\left(t\right)=f\left(\tilde{X}\left(t\right)\right)$ tambi\'en es regenerativo sobre $T_{n}$, para cualquier funci\'on $f:\tilde{S}\rightarrow S$.
\end{Note}

\begin{Note}
Los procesos regenerativos son crudamente regenerativos, pero no al rev\'es.
\end{Note}


\begin{Note}
Un proceso estoc\'astico a tiempo continuo o discreto es regenerativo si existe un proceso de renovaci\'on  tal que los segmentos del proceso entre tiempos de renovaci\'on sucesivos son i.i.d., es decir, para $\left\{X\left(t\right):t\geq0\right\}$ proceso estoc\'astico a tiempo continuo con espacio de estados $S$, espacio m\'etrico.
\end{Note}

Para $\left\{X\left(t\right):t\geq0\right\}$ Proceso Estoc\'astico a tiempo continuo con estado de espacios $S$, que es un espacio m\'etrico, con trayectorias continuas por la derecha y con l\'imites por la izquierda c.s. Sea $N\left(t\right)$ un proceso de renovaci\'on en $\rea_{+}$ definido en el mismo espacio de probabilidad que $X\left(t\right)$, con tiempos de renovaci\'on $T$ y tiempos de inter-renovaci\'on $\xi_{n}=T_{n}-T_{n-1}$, con misma distribuci\'on $F$ de media finita $\mu$.



\begin{Def}
Para el proceso $\left\{\left(N\left(t\right),X\left(t\right)\right):t\geq0\right\}$, sus trayectoria muestrales en el intervalo de tiempo $\left[T_{n-1},T_{n}\right)$ est\'an descritas por
\begin{eqnarray*}
\zeta_{n}=\left(\xi_{n},\left\{X\left(T_{n-1}+t\right):0\leq t<\xi_{n}\right\}\right)
\end{eqnarray*}
Este $\zeta_{n}$ es el $n$-\'esimo segmento del proceso. El proceso es regenerativo sobre los tiempos $T_{n}$ si sus segmentos $\zeta_{n}$ son independientes e id\'enticamennte distribuidos.
\end{Def}

\begin{Note}
Un proceso regenerativo con media de la longitud de ciclo finita es llamado positivo recurrente.
\end{Note}

\begin{Teo}[Procesos Regenerativos]
Suponga que el proceso
\end{Teo}


\begin{Def}[Renewal Process Trinity]
Para un proceso de renovaci\'on $N\left(t\right)$, los siguientes procesos proveen de informaci\'on sobre los tiempos de renovaci\'on.
\begin{itemize}
\item $A\left(t\right)=t-T_{N\left(t\right)}$, el tiempo de recurrencia hacia atr\'as al tiempo $t$, que es el tiempo desde la \'ultima renovaci\'on para $t$.

\item $B\left(t\right)=T_{N\left(t\right)+1}-t$, el tiempo de recurrencia hacia adelante al tiempo $t$, residual del tiempo de renovaci\'on, que es el tiempo para la pr\'oxima renovaci\'on despu\'es de $t$.

\item $L\left(t\right)=\xi_{N\left(t\right)+1}=A\left(t\right)+B\left(t\right)$, la longitud del intervalo de renovaci\'on que contiene a $t$.
\end{itemize}
\end{Def}

\begin{Note}
El proceso tridimensional $\left(A\left(t\right),B\left(t\right),L\left(t\right)\right)$ es regenerativo sobre $T_{n}$, y por ende cada proceso lo es. Cada proceso $A\left(t\right)$ y $B\left(t\right)$ son procesos de MArkov a tiempo continuo con trayectorias continuas por partes en el espacio de estados $\rea_{+}$. Una expresi\'on conveniente para su distribuci\'on conjunta es, para $0\leq x<t,y\geq0$
\begin{equation}\label{NoRenovacion}
P\left\{A\left(t\right)>x,B\left(t\right)>y\right\}=
P\left\{N\left(t+y\right)-N\left((t-x)\right)=0\right\}
\end{equation}
\end{Note}

\begin{Ejem}[Tiempos de recurrencia Poisson]
Si $N\left(t\right)$ es un proceso Poisson con tasa $\lambda$, entonces de la expresi\'on (\ref{NoRenovacion}) se tiene que

\begin{eqnarray*}
\begin{array}{lc}
P\left\{A\left(t\right)>x,B\left(t\right)>y\right\}=e^{-\lambda\left(x+y\right)},&0\leq x<t,y\geq0,
\end{array}
\end{eqnarray*}
que es la probabilidad Poisson de no renovaciones en un intervalo de longitud $x+y$.

\end{Ejem}

%\begin{Note}
Una cadena de Markov erg\'odica tiene la propiedad de ser estacionaria si la distribuci\'on de su estado al tiempo $0$ es su distribuci\'on estacionaria.
%\end{Note}


\begin{Def}
Un proceso estoc\'astico a tiempo continuo $\left\{X\left(t\right):t\geq0\right\}$ en un espacio general es estacionario si sus distribuciones finito dimensionales son invariantes bajo cualquier  traslado: para cada $0\leq s_{1}<s_{2}<\cdots<s_{k}$ y $t\geq0$,
\begin{eqnarray*}
\left(X\left(s_{1}+t\right),\ldots,X\left(s_{k}+t\right)\right)=_{d}\left(X\left(s_{1}\right),\ldots,X\left(s_{k}\right)\right).
\end{eqnarray*}
\end{Def}

\begin{Note}
Un proceso de Markov es estacionario si $X\left(t\right)=_{d}X\left(0\right)$, $t\geq0$.
\end{Note}

Considerese el proceso $N\left(t\right)=\sum_{n}\indora\left(\tau_{n}\leq t\right)$ en $\rea_{+}$, con puntos $0<\tau_{1}<\tau_{2}<\cdots$.

\begin{Prop}
Si $N$ es un proceso puntual estacionario y $\esp\left[N\left(1\right)\right]<\infty$, entonces $\esp\left[N\left(t\right)\right]=t\esp\left[N\left(1\right)\right]$, $t\geq0$

\end{Prop}

\begin{Teo}
Los siguientes enunciados son equivalentes
\begin{itemize}
\item[i)] El proceso retardado de renovaci\'on $N$ es estacionario.

\item[ii)] EL proceso de tiempos de recurrencia hacia adelante $B\left(t\right)$ es estacionario.


\item[iii)] $\esp\left[N\left(t\right)\right]=t/\mu$,


\item[iv)] $G\left(t\right)=F_{e}\left(t\right)=\frac{1}{\mu}\int_{0}^{t}\left[1-F\left(s\right)\right]ds$
\end{itemize}
Cuando estos enunciados son ciertos, $P\left\{B\left(t\right)\leq x\right\}=F_{e}\left(x\right)$, para $t,x\geq0$.

\end{Teo}

\begin{Note}
Una consecuencia del teorema anterior es que el Proceso Poisson es el \'unico proceso sin retardo que es estacionario.
\end{Note}

\begin{Coro}
El proceso de renovaci\'on $N\left(t\right)$ sin retardo, y cuyos tiempos de inter renonaci\'on tienen media finita, es estacionario si y s\'olo si es un proceso Poisson.

\end{Coro}


%________________________________________________________________________
\subsection{Procesos Regenerativos}
%________________________________________________________________________



\begin{Note}
Si $\tilde{X}\left(t\right)$ con espacio de estados $\tilde{S}$ es regenerativo sobre $T_{n}$, entonces $X\left(t\right)=f\left(\tilde{X}\left(t\right)\right)$ tambi\'en es regenerativo sobre $T_{n}$, para cualquier funci\'on $f:\tilde{S}\rightarrow S$.
\end{Note}

\begin{Note}
Los procesos regenerativos son crudamente regenerativos, pero no al rev\'es.
\end{Note}
%\subsection*{Procesos Regenerativos: Sigman\cite{Sigman1}}
\begin{Def}[Definici\'on Cl\'asica]
Un proceso estoc\'astico $X=\left\{X\left(t\right):t\geq0\right\}$ es llamado regenerativo is existe una variable aleatoria $R_{1}>0$ tal que
\begin{itemize}
\item[i)] $\left\{X\left(t+R_{1}\right):t\geq0\right\}$ es independiente de $\left\{\left\{X\left(t\right):t<R_{1}\right\},\right\}$
\item[ii)] $\left\{X\left(t+R_{1}\right):t\geq0\right\}$ es estoc\'asticamente equivalente a $\left\{X\left(t\right):t>0\right\}$
\end{itemize}

Llamamos a $R_{1}$ tiempo de regeneraci\'on, y decimos que $X$ se regenera en este punto.
\end{Def}

$\left\{X\left(t+R_{1}\right)\right\}$ es regenerativo con tiempo de regeneraci\'on $R_{2}$, independiente de $R_{1}$ pero con la misma distribuci\'on que $R_{1}$. Procediendo de esta manera se obtiene una secuencia de variables aleatorias independientes e id\'enticamente distribuidas $\left\{R_{n}\right\}$ llamados longitudes de ciclo. Si definimos a $Z_{k}\equiv R_{1}+R_{2}+\cdots+R_{k}$, se tiene un proceso de renovaci\'on llamado proceso de renovaci\'on encajado para $X$.




\begin{Def}
Para $x$ fijo y para cada $t\geq0$, sea $I_{x}\left(t\right)=1$ si $X\left(t\right)\leq x$,  $I_{x}\left(t\right)=0$ en caso contrario, y def\'inanse los tiempos promedio
\begin{eqnarray*}
\overline{X}&=&lim_{t\rightarrow\infty}\frac{1}{t}\int_{0}^{\infty}X\left(u\right)du\\
\prob\left(X_{\infty}\leq x\right)&=&lim_{t\rightarrow\infty}\frac{1}{t}\int_{0}^{\infty}I_{x}\left(u\right)du,
\end{eqnarray*}
cuando estos l\'imites existan.
\end{Def}

Como consecuencia del teorema de Renovaci\'on-Recompensa, se tiene que el primer l\'imite  existe y es igual a la constante
\begin{eqnarray*}
\overline{X}&=&\frac{\esp\left[\int_{0}^{R_{1}}X\left(t\right)dt\right]}{\esp\left[R_{1}\right]},
\end{eqnarray*}
suponiendo que ambas esperanzas son finitas.

\begin{Note}
\begin{itemize}
\item[a)] Si el proceso regenerativo $X$ es positivo recurrente y tiene trayectorias muestrales no negativas, entonces la ecuaci\'on anterior es v\'alida.
\item[b)] Si $X$ es positivo recurrente regenerativo, podemos construir una \'unica versi\'on estacionaria de este proceso, $X_{e}=\left\{X_{e}\left(t\right)\right\}$, donde $X_{e}$ es un proceso estoc\'astico regenerativo y estrictamente estacionario, con distribuci\'on marginal distribuida como $X_{\infty}$
\end{itemize}
\end{Note}

%________________________________________________________________________
%\subsection{Procesos Regenerativos}
%________________________________________________________________________

Para $\left\{X\left(t\right):t\geq0\right\}$ Proceso Estoc\'astico a tiempo continuo con estado de espacios $S$, que es un espacio m\'etrico, con trayectorias continuas por la derecha y con l\'imites por la izquierda c.s. Sea $N\left(t\right)$ un proceso de renovaci\'on en $\rea_{+}$ definido en el mismo espacio de probabilidad que $X\left(t\right)$, con tiempos de renovaci\'on $T$ y tiempos de inter-renovaci\'on $\xi_{n}=T_{n}-T_{n-1}$, con misma distribuci\'on $F$ de media finita $\mu$.



\begin{Def}
Para el proceso $\left\{\left(N\left(t\right),X\left(t\right)\right):t\geq0\right\}$, sus trayectoria muestrales en el intervalo de tiempo $\left[T_{n-1},T_{n}\right)$ est\'an descritas por
\begin{eqnarray*}
\zeta_{n}=\left(\xi_{n},\left\{X\left(T_{n-1}+t\right):0\leq t<\xi_{n}\right\}\right)
\end{eqnarray*}
Este $\zeta_{n}$ es el $n$-\'esimo segmento del proceso. El proceso es regenerativo sobre los tiempos $T_{n}$ si sus segmentos $\zeta_{n}$ son independientes e id\'enticamennte distribuidos.
\end{Def}


\begin{Note}
Si $\tilde{X}\left(t\right)$ con espacio de estados $\tilde{S}$ es regenerativo sobre $T_{n}$, entonces $X\left(t\right)=f\left(\tilde{X}\left(t\right)\right)$ tambi\'en es regenerativo sobre $T_{n}$, para cualquier funci\'on $f:\tilde{S}\rightarrow S$.
\end{Note}

\begin{Note}
Los procesos regenerativos son crudamente regenerativos, pero no al rev\'es.
\end{Note}

\begin{Def}[Definici\'on Cl\'asica]
Un proceso estoc\'astico $X=\left\{X\left(t\right):t\geq0\right\}$ es llamado regenerativo is existe una variable aleatoria $R_{1}>0$ tal que
\begin{itemize}
\item[i)] $\left\{X\left(t+R_{1}\right):t\geq0\right\}$ es independiente de $\left\{\left\{X\left(t\right):t<R_{1}\right\},\right\}$
\item[ii)] $\left\{X\left(t+R_{1}\right):t\geq0\right\}$ es estoc\'asticamente equivalente a $\left\{X\left(t\right):t>0\right\}$
\end{itemize}

Llamamos a $R_{1}$ tiempo de regeneraci\'on, y decimos que $X$ se regenera en este punto.
\end{Def}

$\left\{X\left(t+R_{1}\right)\right\}$ es regenerativo con tiempo de regeneraci\'on $R_{2}$, independiente de $R_{1}$ pero con la misma distribuci\'on que $R_{1}$. Procediendo de esta manera se obtiene una secuencia de variables aleatorias independientes e id\'enticamente distribuidas $\left\{R_{n}\right\}$ llamados longitudes de ciclo. Si definimos a $Z_{k}\equiv R_{1}+R_{2}+\cdots+R_{k}$, se tiene un proceso de renovaci\'on llamado proceso de renovaci\'on encajado para $X$.

\begin{Note}
Un proceso regenerativo con media de la longitud de ciclo finita es llamado positivo recurrente.
\end{Note}


\begin{Def}
Para $x$ fijo y para cada $t\geq0$, sea $I_{x}\left(t\right)=1$ si $X\left(t\right)\leq x$,  $I_{x}\left(t\right)=0$ en caso contrario, y def\'inanse los tiempos promedio
\begin{eqnarray*}
\overline{X}&=&lim_{t\rightarrow\infty}\frac{1}{t}\int_{0}^{\infty}X\left(u\right)du\\
\prob\left(X_{\infty}\leq x\right)&=&lim_{t\rightarrow\infty}\frac{1}{t}\int_{0}^{\infty}I_{x}\left(u\right)du,
\end{eqnarray*}
cuando estos l\'imites existan.
\end{Def}

Como consecuencia del teorema de Renovaci\'on-Recompensa, se tiene que el primer l\'imite  existe y es igual a la constante
\begin{eqnarray*}
\overline{X}&=&\frac{\esp\left[\int_{0}^{R_{1}}X\left(t\right)dt\right]}{\esp\left[R_{1}\right]},
\end{eqnarray*}
suponiendo que ambas esperanzas son finitas.

\begin{Note}
\begin{itemize}
\item[a)] Si el proceso regenerativo $X$ es positivo recurrente y tiene trayectorias muestrales no negativas, entonces la ecuaci\'on anterior es v\'alida.
\item[b)] Si $X$ es positivo recurrente regenerativo, podemos construir una \'unica versi\'on estacionaria de este proceso, $X_{e}=\left\{X_{e}\left(t\right)\right\}$, donde $X_{e}$ es un proceso estoc\'astico regenerativo y estrictamente estacionario, con distribuci\'on marginal distribuida como $X_{\infty}$
\end{itemize}
\end{Note}

%__________________________________________________________________________________________
%\subsection{Procesos Regenerativos Estacionarios - Stidham \cite{Stidham}}
%__________________________________________________________________________________________


Un proceso estoc\'astico a tiempo continuo $\left\{V\left(t\right),t\geq0\right\}$ es un proceso regenerativo si existe una sucesi\'on de variables aleatorias independientes e id\'enticamente distribuidas $\left\{X_{1},X_{2},\ldots\right\}$, sucesi\'on de renovaci\'on, tal que para cualquier conjunto de Borel $A$, 

\begin{eqnarray*}
\prob\left\{V\left(t\right)\in A|X_{1}+X_{2}+\cdots+X_{R\left(t\right)}=s,\left\{V\left(\tau\right),\tau<s\right\}\right\}=\prob\left\{V\left(t-s\right)\in A|X_{1}>t-s\right\},
\end{eqnarray*}
para todo $0\leq s\leq t$, donde $R\left(t\right)=\max\left\{X_{1}+X_{2}+\cdots+X_{j}\leq t\right\}=$n\'umero de renovaciones ({\emph{puntos de regeneraci\'on}}) que ocurren en $\left[0,t\right]$. El intervalo $\left[0,X_{1}\right)$ es llamado {\emph{primer ciclo de regeneraci\'on}} de $\left\{V\left(t \right),t\geq0\right\}$, $\left[X_{1},X_{1}+X_{2}\right)$ el {\emph{segundo ciclo de regeneraci\'on}}, y as\'i sucesivamente.

Sea $X=X_{1}$ y sea $F$ la funci\'on de distrbuci\'on de $X$


\begin{Def}
Se define el proceso estacionario, $\left\{V^{*}\left(t\right),t\geq0\right\}$, para $\left\{V\left(t\right),t\geq0\right\}$ por

\begin{eqnarray*}
\prob\left\{V\left(t\right)\in A\right\}=\frac{1}{\esp\left[X\right]}\int_{0}^{\infty}\prob\left\{V\left(t+x\right)\in A|X>x\right\}\left(1-F\left(x\right)\right)dx,
\end{eqnarray*} 
para todo $t\geq0$ y todo conjunto de Borel $A$.
\end{Def}

\begin{Def}
Una distribuci\'on se dice que es {\emph{aritm\'etica}} si todos sus puntos de incremento son m\'ultiplos de la forma $0,\lambda, 2\lambda,\ldots$ para alguna $\lambda>0$ entera.
\end{Def}


\begin{Def}
Una modificaci\'on medible de un proceso $\left\{V\left(t\right),t\geq0\right\}$, es una versi\'on de este, $\left\{V\left(t,w\right)\right\}$ conjuntamente medible para $t\geq0$ y para $w\in S$, $S$ espacio de estados para $\left\{V\left(t\right),t\geq0\right\}$.
\end{Def}

\begin{Teo}
Sea $\left\{V\left(t\right),t\geq\right\}$ un proceso regenerativo no negativo con modificaci\'on medible. Sea $\esp\left[X\right]<\infty$. Entonces el proceso estacionario dado por la ecuaci\'on anterior est\'a bien definido y tiene funci\'on de distribuci\'on independiente de $t$, adem\'as
\begin{itemize}
\item[i)] \begin{eqnarray*}
\esp\left[V^{*}\left(0\right)\right]&=&\frac{\esp\left[\int_{0}^{X}V\left(s\right)ds\right]}{\esp\left[X\right]}\end{eqnarray*}
\item[ii)] Si $\esp\left[V^{*}\left(0\right)\right]<\infty$, equivalentemente, si $\esp\left[\int_{0}^{X}V\left(s\right)ds\right]<\infty$,entonces
\begin{eqnarray*}
\frac{\int_{0}^{t}V\left(s\right)ds}{t}\rightarrow\frac{\esp\left[\int_{0}^{X}V\left(s\right)ds\right]}{\esp\left[X\right]}
\end{eqnarray*}
con probabilidad 1 y en media, cuando $t\rightarrow\infty$.
\end{itemize}
\end{Teo}

%__________________________________________________________________________________________
%\subsection{Procesos Regenerativos Estacionarios - Stidham \cite{Stidham}}
%__________________________________________________________________________________________


Un proceso estoc\'astico a tiempo continuo $\left\{V\left(t\right),t\geq0\right\}$ es un proceso regenerativo si existe una sucesi\'on de variables aleatorias independientes e id\'enticamente distribuidas $\left\{X_{1},X_{2},\ldots\right\}$, sucesi\'on de renovaci\'on, tal que para cualquier conjunto de Borel $A$, 

\begin{eqnarray*}
\prob\left\{V\left(t\right)\in A|X_{1}+X_{2}+\cdots+X_{R\left(t\right)}=s,\left\{V\left(\tau\right),\tau<s\right\}\right\}=\prob\left\{V\left(t-s\right)\in A|X_{1}>t-s\right\},
\end{eqnarray*}
para todo $0\leq s\leq t$, donde $R\left(t\right)=\max\left\{X_{1}+X_{2}+\cdots+X_{j}\leq t\right\}=$n\'umero de renovaciones ({\emph{puntos de regeneraci\'on}}) que ocurren en $\left[0,t\right]$. El intervalo $\left[0,X_{1}\right)$ es llamado {\emph{primer ciclo de regeneraci\'on}} de $\left\{V\left(t \right),t\geq0\right\}$, $\left[X_{1},X_{1}+X_{2}\right)$ el {\emph{segundo ciclo de regeneraci\'on}}, y as\'i sucesivamente.

Sea $X=X_{1}$ y sea $F$ la funci\'on de distrbuci\'on de $X$


\begin{Def}
Se define el proceso estacionario, $\left\{V^{*}\left(t\right),t\geq0\right\}$, para $\left\{V\left(t\right),t\geq0\right\}$ por

\begin{eqnarray*}
\prob\left\{V\left(t\right)\in A\right\}=\frac{1}{\esp\left[X\right]}\int_{0}^{\infty}\prob\left\{V\left(t+x\right)\in A|X>x\right\}\left(1-F\left(x\right)\right)dx,
\end{eqnarray*} 
para todo $t\geq0$ y todo conjunto de Borel $A$.
\end{Def}

\begin{Def}
Una distribuci\'on se dice que es {\emph{aritm\'etica}} si todos sus puntos de incremento son m\'ultiplos de la forma $0,\lambda, 2\lambda,\ldots$ para alguna $\lambda>0$ entera.
\end{Def}


\begin{Def}
Una modificaci\'on medible de un proceso $\left\{V\left(t\right),t\geq0\right\}$, es una versi\'on de este, $\left\{V\left(t,w\right)\right\}$ conjuntamente medible para $t\geq0$ y para $w\in S$, $S$ espacio de estados para $\left\{V\left(t\right),t\geq0\right\}$.
\end{Def}

\begin{Teo}
Sea $\left\{V\left(t\right),t\geq\right\}$ un proceso regenerativo no negativo con modificaci\'on medible. Sea $\esp\left[X\right]<\infty$. Entonces el proceso estacionario dado por la ecuaci\'on anterior est\'a bien definido y tiene funci\'on de distribuci\'on independiente de $t$, adem\'as
\begin{itemize}
\item[i)] \begin{eqnarray*}
\esp\left[V^{*}\left(0\right)\right]&=&\frac{\esp\left[\int_{0}^{X}V\left(s\right)ds\right]}{\esp\left[X\right]}\end{eqnarray*}
\item[ii)] Si $\esp\left[V^{*}\left(0\right)\right]<\infty$, equivalentemente, si $\esp\left[\int_{0}^{X}V\left(s\right)ds\right]<\infty$,entonces
\begin{eqnarray*}
\frac{\int_{0}^{t}V\left(s\right)ds}{t}\rightarrow\frac{\esp\left[\int_{0}^{X}V\left(s\right)ds\right]}{\esp\left[X\right]}
\end{eqnarray*}
con probabilidad 1 y en media, cuando $t\rightarrow\infty$.
\end{itemize}
\end{Teo}

Para $\left\{X\left(t\right):t\geq0\right\}$ Proceso Estoc\'astico a tiempo continuo con estado de espacios $S$, que es un espacio m\'etrico, con trayectorias continuas por la derecha y con l\'imites por la izquierda c.s. Sea $N\left(t\right)$ un proceso de renovaci\'on en $\rea_{+}$ definido en el mismo espacio de probabilidad que $X\left(t\right)$, con tiempos de renovaci\'on $T$ y tiempos de inter-renovaci\'on $\xi_{n}=T_{n}-T_{n-1}$, con misma distribuci\'on $F$ de media finita $\mu$.
%_____________________________________________________
\subsection{Puntos de Renovaci\'on}
%_____________________________________________________

Para cada cola $Q_{i}$ se tienen los procesos de arribo a la cola, para estas, los tiempos de arribo est\'an dados por $$\left\{T_{1}^{i},T_{2}^{i},\ldots,T_{k}^{i},\ldots\right\},$$ entonces, consideremos solamente los primeros tiempos de arribo a cada una de las colas, es decir, $$\left\{T_{1}^{1},T_{1}^{2},T_{1}^{3},T_{1}^{4}\right\},$$ se sabe que cada uno de estos tiempos se distribuye de manera exponencial con par\'ametro $1/mu_{i}$. Adem\'as se sabe que para $$T^{*}=\min\left\{T_{1}^{1},T_{1}^{2},T_{1}^{3},T_{1}^{4}\right\},$$ $T^{*}$ se distribuye de manera exponencial con par\'ametro $$\mu^{*}=\sum_{i=1}^{4}\mu_{i}.$$ Ahora, dado que 
\begin{center}
\begin{tabular}{lcl}
$\tilde{r}=r_{1}+r_{2}$ & y &$\hat{r}=r_{3}+r_{4}.$
\end{tabular}
\end{center}


Supongamos que $$\tilde{r},\hat{r}<\mu^{*},$$ entonces si tomamos $$r^{*}=\min\left\{\tilde{r},\hat{r}\right\},$$ se tiene que para  $$t^{*}\in\left(0,r^{*}\right)$$ se cumple que 
\begin{center}
\begin{tabular}{lcl}
$\tau_{1}\left(1\right)=0$ & y por tanto & $\overline{\tau}_{1}=0,$
\end{tabular}
\end{center}
entonces para la segunda cola en este primer ciclo se cumple que $$\tau_{2}=\overline{\tau}_{1}+r_{1}=r_{1}<\mu^{*},$$ y por tanto se tiene que  $$\overline{\tau}_{2}=\tau_{2}.$$ Por lo tanto, nuevamente para la primer cola en el segundo ciclo $$\tau_{1}\left(2\right)=\tau_{2}\left(1\right)+r_{2}=\tilde{r}<\mu^{*}.$$ An\'alogamente para el segundo sistema se tiene que ambas colas est\'an vac\'ias, es decir, existe un valor $t^{*}$ tal que en el intervalo $\left(0,t^{*}\right)$ no ha llegado ning\'un usuario, es decir, $$L_{i}\left(t^{*}\right)=0$$ para $i=1,2,3,4$.

\subsection{Resultados para Procesos de Salida}




%________________________________________________________________________
\subsection{Procesos Regenerativos}
%________________________________________________________________________

Para $\left\{X\left(t\right):t\geq0\right\}$ Proceso Estoc\'astico a tiempo continuo con estado de espacios $S$, que es un espacio m\'etrico, con trayectorias continuas por la derecha y con l\'imites por la izquierda c.s. Sea $N\left(t\right)$ un proceso de renovaci\'on en $\rea_{+}$ definido en el mismo espacio de probabilidad que $X\left(t\right)$, con tiempos de renovaci\'on $T$ y tiempos de inter-renovaci\'on $\xi_{n}=T_{n}-T_{n-1}$, con misma distribuci\'on $F$ de media finita $\mu$.



\begin{Def}
Para el proceso $\left\{\left(N\left(t\right),X\left(t\right)\right):t\geq0\right\}$, sus trayectoria muestrales en el intervalo de tiempo $\left[T_{n-1},T_{n}\right)$ est\'an descritas por
\begin{eqnarray*}
\zeta_{n}=\left(\xi_{n},\left\{X\left(T_{n-1}+t\right):0\leq t<\xi_{n}\right\}\right)
\end{eqnarray*}
Este $\zeta_{n}$ es el $n$-\'esimo segmento del proceso. El proceso es regenerativo sobre los tiempos $T_{n}$ si sus segmentos $\zeta_{n}$ son independientes e id\'enticamennte distribuidos.
\end{Def}


\begin{Obs}
Si $\tilde{X}\left(t\right)$ con espacio de estados $\tilde{S}$ es regenerativo sobre $T_{n}$, entonces $X\left(t\right)=f\left(\tilde{X}\left(t\right)\right)$ tambi\'en es regenerativo sobre $T_{n}$, para cualquier funci\'on $f:\tilde{S}\rightarrow S$.
\end{Obs}

\begin{Obs}
Los procesos regenerativos son crudamente regenerativos, pero no al rev\'es.
\end{Obs}

\begin{Def}[Definici\'on Cl\'asica]
Un proceso estoc\'astico $X=\left\{X\left(t\right):t\geq0\right\}$ es llamado regenerativo is existe una variable aleatoria $R_{1}>0$ tal que
\begin{itemize}
\item[i)] $\left\{X\left(t+R_{1}\right):t\geq0\right\}$ es independiente de $\left\{\left\{X\left(t\right):t<R_{1}\right\},\right\}$
\item[ii)] $\left\{X\left(t+R_{1}\right):t\geq0\right\}$ es estoc\'asticamente equivalente a $\left\{X\left(t\right):t>0\right\}$
\end{itemize}

Llamamos a $R_{1}$ tiempo de regeneraci\'on, y decimos que $X$ se regenera en este punto.
\end{Def}

$\left\{X\left(t+R_{1}\right)\right\}$ es regenerativo con tiempo de regeneraci\'on $R_{2}$, independiente de $R_{1}$ pero con la misma distribuci\'on que $R_{1}$. Procediendo de esta manera se obtiene una secuencia de variables aleatorias independientes e id\'enticamente distribuidas $\left\{R_{n}\right\}$ llamados longitudes de ciclo. Si definimos a $Z_{k}\equiv R_{1}+R_{2}+\cdots+R_{k}$, se tiene un proceso de renovaci\'on llamado proceso de renovaci\'on encajado para $X$.

\begin{Note}
Un proceso regenerativo con media de la longitud de ciclo finita es llamado positivo recurrente.
\end{Note}


\begin{Def}
Para $x$ fijo y para cada $t\geq0$, sea $I_{x}\left(t\right)=1$ si $X\left(t\right)\leq x$,  $I_{x}\left(t\right)=0$ en caso contrario, y def\'inanse los tiempos promedio
\begin{eqnarray*}
\overline{X}&=&lim_{t\rightarrow\infty}\frac{1}{t}\int_{0}^{\infty}X\left(u\right)du\\
\prob\left(X_{\infty}\leq x\right)&=&lim_{t\rightarrow\infty}\frac{1}{t}\int_{0}^{\infty}I_{x}\left(u\right)du,
\end{eqnarray*}
cuando estos l\'imites existan.
\end{Def}

Como consecuencia del teorema de Renovaci\'on-Recompensa, se tiene que el primer l\'imite  existe y es igual a la constante
\begin{eqnarray*}
\overline{X}&=&\frac{\esp\left[\int_{0}^{R_{1}}X\left(t\right)dt\right]}{\esp\left[R_{1}\right]},
\end{eqnarray*}
suponiendo que ambas esperanzas son finitas.

\begin{Note}
\begin{itemize}
\item[a)] Si el proceso regenerativo $X$ es positivo recurrente y tiene trayectorias muestrales no negativas, entonces la ecuaci\'on anterior es v\'alida.
\item[b)] Si $X$ es positivo recurrente regenerativo, podemos construir una \'unica versi\'on estacionaria de este proceso, $X_{e}=\left\{X_{e}\left(t\right)\right\}$, donde $X_{e}$ es un proceso estoc\'astico regenerativo y estrictamente estacionario, con distribuci\'on marginal distribuida como $X_{\infty}$
\end{itemize}
\end{Note}

\subsection{Renewal and Regenerative Processes: Serfozo\cite{Serfozo}}
\begin{Def}\label{Def.Tn}
Sean $0\leq T_{1}\leq T_{2}\leq \ldots$ son tiempos aleatorios infinitos en los cuales ocurren ciertos eventos. El n\'umero de tiempos $T_{n}$ en el intervalo $\left[0,t\right)$ es

\begin{eqnarray}
N\left(t\right)=\sum_{n=1}^{\infty}\indora\left(T_{n}\leq t\right),
\end{eqnarray}
para $t\geq0$.
\end{Def}

Si se consideran los puntos $T_{n}$ como elementos de $\rea_{+}$, y $N\left(t\right)$ es el n\'umero de puntos en $\rea$. El proceso denotado por $\left\{N\left(t\right):t\geq0\right\}$, denotado por $N\left(t\right)$, es un proceso puntual en $\rea_{+}$. Los $T_{n}$ son los tiempos de ocurrencia, el proceso puntual $N\left(t\right)$ es simple si su n\'umero de ocurrencias son distintas: $0<T_{1}<T_{2}<\ldots$ casi seguramente.

\begin{Def}
Un proceso puntual $N\left(t\right)$ es un proceso de renovaci\'on si los tiempos de interocurrencia $\xi_{n}=T_{n}-T_{n-1}$, para $n\geq1$, son independientes e identicamente distribuidos con distribuci\'on $F$, donde $F\left(0\right)=0$ y $T_{0}=0$. Los $T_{n}$ son llamados tiempos de renovaci\'on, referente a la independencia o renovaci\'on de la informaci\'on estoc\'astica en estos tiempos. Los $\xi_{n}$ son los tiempos de inter-renovaci\'on, y $N\left(t\right)$ es el n\'umero de renovaciones en el intervalo $\left[0,t\right)$
\end{Def}


\begin{Note}
Para definir un proceso de renovaci\'on para cualquier contexto, solamente hay que especificar una distribuci\'on $F$, con $F\left(0\right)=0$, para los tiempos de inter-renovaci\'on. La funci\'on $F$ en turno degune las otra variables aleatorias. De manera formal, existe un espacio de probabilidad y una sucesi\'on de variables aleatorias $\xi_{1},\xi_{2},\ldots$ definidas en este con distribuci\'on $F$. Entonces las otras cantidades son $T_{n}=\sum_{k=1}^{n}\xi_{k}$ y $N\left(t\right)=\sum_{n=1}^{\infty}\indora\left(T_{n}\leq t\right)$, donde $T_{n}\rightarrow\infty$ casi seguramente por la Ley Fuerte de los Grandes N\'umeros.
\end{Note}


Los tiempos $T_{n}$ est\'an relacionados con los conteos de $N\left(t\right)$ por

\begin{eqnarray*}
\left\{N\left(t\right)\geq n\right\}&=&\left\{T_{n}\leq t\right\}\\
T_{N\left(t\right)}\leq &t&<T_{N\left(t\right)+1},
\end{eqnarray*}

adem\'as $N\left(T_{n}\right)=n$, y 

\begin{eqnarray*}
N\left(t\right)=\max\left\{n:T_{n}\leq t\right\}=\min\left\{n:T_{n+1}>t\right\}
\end{eqnarray*}

Por propiedades de la convoluci\'on se sabe que

\begin{eqnarray*}
P\left\{T_{n}\leq t\right\}=F^{n\star}\left(t\right)
\end{eqnarray*}
que es la $n$-\'esima convoluci\'on de $F$. Entonces 

\begin{eqnarray*}
\left\{N\left(t\right)\geq n\right\}&=&\left\{T_{n}\leq t\right\}\\
P\left\{N\left(t\right)\leq n\right\}&=&1-F^{\left(n+1\right)\star}\left(t\right)
\end{eqnarray*}

Adem\'as usando el hecho de que $\esp\left[N\left(t\right)\right]=\sum_{n=1}^{\infty}P\left\{N\left(t\right)\geq n\right\}$
se tiene que

\begin{eqnarray*}
\esp\left[N\left(t\right)\right]=\sum_{n=1}^{\infty}F^{n\star}\left(t\right)
\end{eqnarray*}

\begin{Prop}
Para cada $t\geq0$, la funci\'on generadora de momentos $\esp\left[e^{\alpha N\left(t\right)}\right]$ existe para alguna $\alpha$ en una vecindad del 0, y de aqu\'i que $\esp\left[N\left(t\right)^{m}\right]<\infty$, para $m\geq1$.
\end{Prop}


\begin{Note}
Si el primer tiempo de renovaci\'on $\xi_{1}$ no tiene la misma distribuci\'on que el resto de las $\xi_{n}$, para $n\geq2$, a $N\left(t\right)$ se le llama Proceso de Renovaci\'on retardado, donde si $\xi$ tiene distribuci\'on $G$, entonces el tiempo $T_{n}$ de la $n$-\'esima renovaci\'on tiene distribuci\'on $G\star F^{\left(n-1\right)\star}\left(t\right)$
\end{Note}


\begin{Teo}
Para una constante $\mu\leq\infty$ ( o variable aleatoria), las siguientes expresiones son equivalentes:

\begin{eqnarray}
lim_{n\rightarrow\infty}n^{-1}T_{n}&=&\mu,\textrm{ c.s.}\\
lim_{t\rightarrow\infty}t^{-1}N\left(t\right)&=&1/\mu,\textrm{ c.s.}
\end{eqnarray}
\end{Teo}


Es decir, $T_{n}$ satisface la Ley Fuerte de los Grandes N\'umeros s\'i y s\'olo s\'i $N\left/t\right)$ la cumple.


\begin{Coro}[Ley Fuerte de los Grandes N\'umeros para Procesos de Renovaci\'on]
Si $N\left(t\right)$ es un proceso de renovaci\'on cuyos tiempos de inter-renovaci\'on tienen media $\mu\leq\infty$, entonces
\begin{eqnarray}
t^{-1}N\left(t\right)\rightarrow 1/\mu,\textrm{ c.s. cuando }t\rightarrow\infty.
\end{eqnarray}

\end{Coro}


Considerar el proceso estoc\'astico de valores reales $\left\{Z\left(t\right):t\geq0\right\}$ en el mismo espacio de probabilidad que $N\left(t\right)$

\begin{Def}
Para el proceso $\left\{Z\left(t\right):t\geq0\right\}$ se define la fluctuaci\'on m\'axima de $Z\left(t\right)$ en el intervalo $\left(T_{n-1},T_{n}\right]$:
\begin{eqnarray*}
M_{n}=\sup_{T_{n-1}<t\leq T_{n}}|Z\left(t\right)-Z\left(T_{n-1}\right)|
\end{eqnarray*}
\end{Def}

\begin{Teo}
Sup\'ongase que $n^{-1}T_{n}\rightarrow\mu$ c.s. cuando $n\rightarrow\infty$, donde $\mu\leq\infty$ es una constante o variable aleatoria. Sea $a$ una constante o variable aleatoria que puede ser infinita cuando $\mu$ es finita, y considere las expresiones l\'imite:
\begin{eqnarray}
lim_{n\rightarrow\infty}n^{-1}Z\left(T_{n}\right)&=&a,\textrm{ c.s.}\\
lim_{t\rightarrow\infty}t^{-1}Z\left(t\right)&=&a/\mu,\textrm{ c.s.}
\end{eqnarray}
La segunda expresi\'on implica la primera. Conversamente, la primera implica la segunda si el proceso $Z\left(t\right)$ es creciente, o si $lim_{n\rightarrow\infty}n^{-1}M_{n}=0$ c.s.
\end{Teo}

\begin{Coro}
Si $N\left(t\right)$ es un proceso de renovaci\'on, y $\left(Z\left(T_{n}\right)-Z\left(T_{n-1}\right),M_{n}\right)$, para $n\geq1$, son variables aleatorias independientes e id\'enticamente distribuidas con media finita, entonces,
\begin{eqnarray}
lim_{t\rightarrow\infty}t^{-1}Z\left(t\right)\rightarrow\frac{\esp\left[Z\left(T_{1}\right)-Z\left(T_{0}\right)\right]}{\esp\left[T_{1}\right]},\textrm{ c.s. cuando  }t\rightarrow\infty.
\end{eqnarray}
\end{Coro}


Sup\'ongase que $N\left(t\right)$ es un proceso de renovaci\'on con distribuci\'on $F$ con media finita $\mu$.

\begin{Def}
La funci\'on de renovaci\'on asociada con la distribuci\'on $F$, del proceso $N\left(t\right)$, es
\begin{eqnarray*}
U\left(t\right)=\sum_{n=1}^{\infty}F^{n\star}\left(t\right),\textrm{   }t\geq0,
\end{eqnarray*}
donde $F^{0\star}\left(t\right)=\indora\left(t\geq0\right)$.
\end{Def}


\begin{Prop}
Sup\'ongase que la distribuci\'on de inter-renovaci\'on $F$ tiene densidad $f$. Entonces $U\left(t\right)$ tambi\'en tiene densidad, para $t>0$, y es $U^{'}\left(t\right)=\sum_{n=0}^{\infty}f^{n\star}\left(t\right)$. Adem\'as
\begin{eqnarray*}
\prob\left\{N\left(t\right)>N\left(t-\right)\right\}=0\textrm{,   }t\geq0.
\end{eqnarray*}
\end{Prop}

\begin{Def}
La Transformada de Laplace-Stieljes de $F$ est\'a dada por

\begin{eqnarray*}
\hat{F}\left(\alpha\right)=\int_{\rea_{+}}e^{-\alpha t}dF\left(t\right)\textrm{,  }\alpha\geq0.
\end{eqnarray*}
\end{Def}

Entonces

\begin{eqnarray*}
\hat{U}\left(\alpha\right)=\sum_{n=0}^{\infty}\hat{F^{n\star}}\left(\alpha\right)=\sum_{n=0}^{\infty}\hat{F}\left(\alpha\right)^{n}=\frac{1}{1-\hat{F}\left(\alpha\right)}.
\end{eqnarray*}


\begin{Prop}
La Transformada de Laplace $\hat{U}\left(\alpha\right)$ y $\hat{F}\left(\alpha\right)$ determina una a la otra de manera \'unica por la relaci\'on $\hat{U}\left(\alpha\right)=\frac{1}{1-\hat{F}\left(\alpha\right)}$.
\end{Prop}


\begin{Note}
Un proceso de renovaci\'on $N\left(t\right)$ cuyos tiempos de inter-renovaci\'on tienen media finita, es un proceso Poisson con tasa $\lambda$ si y s\'olo s\'i $\esp\left[U\left(t\right)\right]=\lambda t$, para $t\geq0$.
\end{Note}


\begin{Teo}
Sea $N\left(t\right)$ un proceso puntual simple con puntos de localizaci\'on $T_{n}$ tal que $\eta\left(t\right)=\esp\left[N\left(\right)\right]$ es finita para cada $t$. Entonces para cualquier funci\'on $f:\rea_{+}\rightarrow\rea$,
\begin{eqnarray*}
\esp\left[\sum_{n=1}^{N\left(\right)}f\left(T_{n}\right)\right]=\int_{\left(0,t\right]}f\left(s\right)d\eta\left(s\right)\textrm{,  }t\geq0,
\end{eqnarray*}
suponiendo que la integral exista. Adem\'as si $X_{1},X_{2},\ldots$ son variables aleatorias definidas en el mismo espacio de probabilidad que el proceso $N\left(t\right)$ tal que $\esp\left[X_{n}|T_{n}=s\right]=f\left(s\right)$, independiente de $n$. Entonces
\begin{eqnarray*}
\esp\left[\sum_{n=1}^{N\left(t\right)}X_{n}\right]=\int_{\left(0,t\right]}f\left(s\right)d\eta\left(s\right)\textrm{,  }t\geq0,
\end{eqnarray*} 
suponiendo que la integral exista. 
\end{Teo}

\begin{Coro}[Identidad de Wald para Renovaciones]
Para el proceso de renovaci\'on $N\left(t\right)$,
\begin{eqnarray*}
\esp\left[T_{N\left(t\right)+1}\right]=\mu\esp\left[N\left(t\right)+1\right]\textrm{,  }t\geq0,
\end{eqnarray*}  
\end{Coro}


\begin{Def}
Sea $h\left(t\right)$ funci\'on de valores reales en $\rea$ acotada en intervalos finitos e igual a cero para $t<0$ La ecuaci\'on de renovaci\'on para $h\left(t\right)$ y la distribuci\'on $F$ es

\begin{eqnarray}\label{Ec.Renovacion}
H\left(t\right)=h\left(t\right)+\int_{\left[0,t\right]}H\left(t-s\right)dF\left(s\right)\textrm{,    }t\geq0,
\end{eqnarray}
donde $H\left(t\right)$ es una funci\'on de valores reales. Esto es $H=h+F\star H$. Decimos que $H\left(t\right)$ es soluci\'on de esta ecuaci\'on si satisface la ecuaci\'on, y es acotada en intervalos finitos e iguales a cero para $t<0$.
\end{Def}

\begin{Prop}
La funci\'on $U\star h\left(t\right)$ es la \'unica soluci\'on de la ecuaci\'on de renovaci\'on (\ref{Ec.Renovacion}).
\end{Prop}

\begin{Teo}[Teorema Renovaci\'on Elemental]
\begin{eqnarray*}
t^{-1}U\left(t\right)\rightarrow 1/\mu\textrm{,    cuando }t\rightarrow\infty.
\end{eqnarray*}
\end{Teo}



Sup\'ongase que $N\left(t\right)$ es un proceso de renovaci\'on con distribuci\'on $F$ con media finita $\mu$.

\begin{Def}
La funci\'on de renovaci\'on asociada con la distribuci\'on $F$, del proceso $N\left(t\right)$, es
\begin{eqnarray*}
U\left(t\right)=\sum_{n=1}^{\infty}F^{n\star}\left(t\right),\textrm{   }t\geq0,
\end{eqnarray*}
donde $F^{0\star}\left(t\right)=\indora\left(t\geq0\right)$.
\end{Def}


\begin{Prop}
Sup\'ongase que la distribuci\'on de inter-renovaci\'on $F$ tiene densidad $f$. Entonces $U\left(t\right)$ tambi\'en tiene densidad, para $t>0$, y es $U^{'}\left(t\right)=\sum_{n=0}^{\infty}f^{n\star}\left(t\right)$. Adem\'as
\begin{eqnarray*}
\prob\left\{N\left(t\right)>N\left(t-\right)\right\}=0\textrm{,   }t\geq0.
\end{eqnarray*}
\end{Prop}

\begin{Def}
La Transformada de Laplace-Stieljes de $F$ est\'a dada por

\begin{eqnarray*}
\hat{F}\left(\alpha\right)=\int_{\rea_{+}}e^{-\alpha t}dF\left(t\right)\textrm{,  }\alpha\geq0.
\end{eqnarray*}
\end{Def}

Entonces

\begin{eqnarray*}
\hat{U}\left(\alpha\right)=\sum_{n=0}^{\infty}\hat{F^{n\star}}\left(\alpha\right)=\sum_{n=0}^{\infty}\hat{F}\left(\alpha\right)^{n}=\frac{1}{1-\hat{F}\left(\alpha\right)}.
\end{eqnarray*}


\begin{Prop}
La Transformada de Laplace $\hat{U}\left(\alpha\right)$ y $\hat{F}\left(\alpha\right)$ determina una a la otra de manera \'unica por la relaci\'on $\hat{U}\left(\alpha\right)=\frac{1}{1-\hat{F}\left(\alpha\right)}$.
\end{Prop}


\begin{Note}
Un proceso de renovaci\'on $N\left(t\right)$ cuyos tiempos de inter-renovaci\'on tienen media finita, es un proceso Poisson con tasa $\lambda$ si y s\'olo s\'i $\esp\left[U\left(t\right)\right]=\lambda t$, para $t\geq0$.
\end{Note}


\begin{Teo}
Sea $N\left(t\right)$ un proceso puntual simple con puntos de localizaci\'on $T_{n}$ tal que $\eta\left(t\right)=\esp\left[N\left(\right)\right]$ es finita para cada $t$. Entonces para cualquier funci\'on $f:\rea_{+}\rightarrow\rea$,
\begin{eqnarray*}
\esp\left[\sum_{n=1}^{N\left(\right)}f\left(T_{n}\right)\right]=\int_{\left(0,t\right]}f\left(s\right)d\eta\left(s\right)\textrm{,  }t\geq0,
\end{eqnarray*}
suponiendo que la integral exista. Adem\'as si $X_{1},X_{2},\ldots$ son variables aleatorias definidas en el mismo espacio de probabilidad que el proceso $N\left(t\right)$ tal que $\esp\left[X_{n}|T_{n}=s\right]=f\left(s\right)$, independiente de $n$. Entonces
\begin{eqnarray*}
\esp\left[\sum_{n=1}^{N\left(t\right)}X_{n}\right]=\int_{\left(0,t\right]}f\left(s\right)d\eta\left(s\right)\textrm{,  }t\geq0,
\end{eqnarray*} 
suponiendo que la integral exista. 
\end{Teo}

\begin{Coro}[Identidad de Wald para Renovaciones]
Para el proceso de renovaci\'on $N\left(t\right)$,
\begin{eqnarray*}
\esp\left[T_{N\left(t\right)+1}\right]=\mu\esp\left[N\left(t\right)+1\right]\textrm{,  }t\geq0,
\end{eqnarray*}  
\end{Coro}


\begin{Def}
Sea $h\left(t\right)$ funci\'on de valores reales en $\rea$ acotada en intervalos finitos e igual a cero para $t<0$ La ecuaci\'on de renovaci\'on para $h\left(t\right)$ y la distribuci\'on $F$ es

\begin{eqnarray}\label{Ec.Renovacion}
H\left(t\right)=h\left(t\right)+\int_{\left[0,t\right]}H\left(t-s\right)dF\left(s\right)\textrm{,    }t\geq0,
\end{eqnarray}
donde $H\left(t\right)$ es una funci\'on de valores reales. Esto es $H=h+F\star H$. Decimos que $H\left(t\right)$ es soluci\'on de esta ecuaci\'on si satisface la ecuaci\'on, y es acotada en intervalos finitos e iguales a cero para $t<0$.
\end{Def}

\begin{Prop}
La funci\'on $U\star h\left(t\right)$ es la \'unica soluci\'on de la ecuaci\'on de renovaci\'on (\ref{Ec.Renovacion}).
\end{Prop}

\begin{Teo}[Teorema Renovaci\'on Elemental]
\begin{eqnarray*}
t^{-1}U\left(t\right)\rightarrow 1/\mu\textrm{,    cuando }t\rightarrow\infty.
\end{eqnarray*}
\end{Teo}


\begin{Note} Una funci\'on $h:\rea_{+}\rightarrow\rea$ es Directamente Riemann Integrable en los siguientes casos:
\begin{itemize}
\item[a)] $h\left(t\right)\geq0$ es decreciente y Riemann Integrable.
\item[b)] $h$ es continua excepto posiblemente en un conjunto de Lebesgue de medida 0, y $|h\left(t\right)|\leq b\left(t\right)$, donde $b$ es DRI.
\end{itemize}
\end{Note}

\begin{Teo}[Teorema Principal de Renovaci\'on]
Si $F$ es no aritm\'etica y $h\left(t\right)$ es Directamente Riemann Integrable (DRI), entonces

\begin{eqnarray*}
lim_{t\rightarrow\infty}U\star h=\frac{1}{\mu}\int_{\rea_{+}}h\left(s\right)ds.
\end{eqnarray*}
\end{Teo}

\begin{Prop}
Cualquier funci\'on $H\left(t\right)$ acotada en intervalos finitos y que es 0 para $t<0$ puede expresarse como
\begin{eqnarray*}
H\left(t\right)=U\star h\left(t\right)\textrm{,  donde }h\left(t\right)=H\left(t\right)-F\star H\left(t\right)
\end{eqnarray*}
\end{Prop}

\begin{Def}
Un proceso estoc\'astico $X\left(t\right)$ es crudamente regenerativo en un tiempo aleatorio positivo $T$ si
\begin{eqnarray*}
\esp\left[X\left(T+t\right)|T\right]=\esp\left[X\left(t\right)\right]\textrm{, para }t\geq0,\end{eqnarray*}
y con las esperanzas anteriores finitas.
\end{Def}

\begin{Prop}
Sup\'ongase que $X\left(t\right)$ es un proceso crudamente regenerativo en $T$, que tiene distribuci\'on $F$. Si $\esp\left[X\left(t\right)\right]$ es acotado en intervalos finitos, entonces
\begin{eqnarray*}
\esp\left[X\left(t\right)\right]=U\star h\left(t\right)\textrm{,  donde }h\left(t\right)=\esp\left[X\left(t\right)\indora\left(T>t\right)\right].
\end{eqnarray*}
\end{Prop}

\begin{Teo}[Regeneraci\'on Cruda]
Sup\'ongase que $X\left(t\right)$ es un proceso con valores positivo crudamente regenerativo en $T$, y def\'inase $M=\sup\left\{|X\left(t\right)|:t\leq T\right\}$. Si $T$ es no aritm\'etico y $M$ y $MT$ tienen media finita, entonces
\begin{eqnarray*}
lim_{t\rightarrow\infty}\esp\left[X\left(t\right)\right]=\frac{1}{\mu}\int_{\rea_{+}}h\left(s\right)ds,
\end{eqnarray*}
donde $h\left(t\right)=\esp\left[X\left(t\right)\indora\left(T>t\right)\right]$.
\end{Teo}


\begin{Note} Una funci\'on $h:\rea_{+}\rightarrow\rea$ es Directamente Riemann Integrable en los siguientes casos:
\begin{itemize}
\item[a)] $h\left(t\right)\geq0$ es decreciente y Riemann Integrable.
\item[b)] $h$ es continua excepto posiblemente en un conjunto de Lebesgue de medida 0, y $|h\left(t\right)|\leq b\left(t\right)$, donde $b$ es DRI.
\end{itemize}
\end{Note}

\begin{Teo}[Teorema Principal de Renovaci\'on]
Si $F$ es no aritm\'etica y $h\left(t\right)$ es Directamente Riemann Integrable (DRI), entonces

\begin{eqnarray*}
lim_{t\rightarrow\infty}U\star h=\frac{1}{\mu}\int_{\rea_{+}}h\left(s\right)ds.
\end{eqnarray*}
\end{Teo}

\begin{Prop}
Cualquier funci\'on $H\left(t\right)$ acotada en intervalos finitos y que es 0 para $t<0$ puede expresarse como
\begin{eqnarray*}
H\left(t\right)=U\star h\left(t\right)\textrm{,  donde }h\left(t\right)=H\left(t\right)-F\star H\left(t\right)
\end{eqnarray*}
\end{Prop}

\begin{Def}
Un proceso estoc\'astico $X\left(t\right)$ es crudamente regenerativo en un tiempo aleatorio positivo $T$ si
\begin{eqnarray*}
\esp\left[X\left(T+t\right)|T\right]=\esp\left[X\left(t\right)\right]\textrm{, para }t\geq0,\end{eqnarray*}
y con las esperanzas anteriores finitas.
\end{Def}

\begin{Prop}
Sup\'ongase que $X\left(t\right)$ es un proceso crudamente regenerativo en $T$, que tiene distribuci\'on $F$. Si $\esp\left[X\left(t\right)\right]$ es acotado en intervalos finitos, entonces
\begin{eqnarray*}
\esp\left[X\left(t\right)\right]=U\star h\left(t\right)\textrm{,  donde }h\left(t\right)=\esp\left[X\left(t\right)\indora\left(T>t\right)\right].
\end{eqnarray*}
\end{Prop}

\begin{Teo}[Regeneraci\'on Cruda]
Sup\'ongase que $X\left(t\right)$ es un proceso con valores positivo crudamente regenerativo en $T$, y def\'inase $M=\sup\left\{|X\left(t\right)|:t\leq T\right\}$. Si $T$ es no aritm\'etico y $M$ y $MT$ tienen media finita, entonces
\begin{eqnarray*}
lim_{t\rightarrow\infty}\esp\left[X\left(t\right)\right]=\frac{1}{\mu}\int_{\rea_{+}}h\left(s\right)ds,
\end{eqnarray*}
donde $h\left(t\right)=\esp\left[X\left(t\right)\indora\left(T>t\right)\right]$.
\end{Teo}

%________________________________________________________________________
\subsection{Procesos Regenerativos}
%________________________________________________________________________

Para $\left\{X\left(t\right):t\geq0\right\}$ Proceso Estoc\'astico a tiempo continuo con estado de espacios $S$, que es un espacio m\'etrico, con trayectorias continuas por la derecha y con l\'imites por la izquierda c.s. Sea $N\left(t\right)$ un proceso de renovaci\'on en $\rea_{+}$ definido en el mismo espacio de probabilidad que $X\left(t\right)$, con tiempos de renovaci\'on $T$ y tiempos de inter-renovaci\'on $\xi_{n}=T_{n}-T_{n-1}$, con misma distribuci\'on $F$ de media finita $\mu$.



\begin{Def}
Para el proceso $\left\{\left(N\left(t\right),X\left(t\right)\right):t\geq0\right\}$, sus trayectoria muestrales en el intervalo de tiempo $\left[T_{n-1},T_{n}\right)$ est\'an descritas por
\begin{eqnarray*}
\zeta_{n}=\left(\xi_{n},\left\{X\left(T_{n-1}+t\right):0\leq t<\xi_{n}\right\}\right)
\end{eqnarray*}
Este $\zeta_{n}$ es el $n$-\'esimo segmento del proceso. El proceso es regenerativo sobre los tiempos $T_{n}$ si sus segmentos $\zeta_{n}$ son independientes e id\'enticamennte distribuidos.
\end{Def}


\begin{Obs}
Si $\tilde{X}\left(t\right)$ con espacio de estados $\tilde{S}$ es regenerativo sobre $T_{n}$, entonces $X\left(t\right)=f\left(\tilde{X}\left(t\right)\right)$ tambi\'en es regenerativo sobre $T_{n}$, para cualquier funci\'on $f:\tilde{S}\rightarrow S$.
\end{Obs}

\begin{Obs}
Los procesos regenerativos son crudamente regenerativos, pero no al rev\'es.
\end{Obs}

\begin{Def}[Definici\'on Cl\'asica]
Un proceso estoc\'astico $X=\left\{X\left(t\right):t\geq0\right\}$ es llamado regenerativo is existe una variable aleatoria $R_{1}>0$ tal que
\begin{itemize}
\item[i)] $\left\{X\left(t+R_{1}\right):t\geq0\right\}$ es independiente de $\left\{\left\{X\left(t\right):t<R_{1}\right\},\right\}$
\item[ii)] $\left\{X\left(t+R_{1}\right):t\geq0\right\}$ es estoc\'asticamente equivalente a $\left\{X\left(t\right):t>0\right\}$
\end{itemize}

Llamamos a $R_{1}$ tiempo de regeneraci\'on, y decimos que $X$ se regenera en este punto.
\end{Def}

$\left\{X\left(t+R_{1}\right)\right\}$ es regenerativo con tiempo de regeneraci\'on $R_{2}$, independiente de $R_{1}$ pero con la misma distribuci\'on que $R_{1}$. Procediendo de esta manera se obtiene una secuencia de variables aleatorias independientes e id\'enticamente distribuidas $\left\{R_{n}\right\}$ llamados longitudes de ciclo. Si definimos a $Z_{k}\equiv R_{1}+R_{2}+\cdots+R_{k}$, se tiene un proceso de renovaci\'on llamado proceso de renovaci\'on encajado para $X$.

\begin{Note}
Un proceso regenerativo con media de la longitud de ciclo finita es llamado positivo recurrente.
\end{Note}


\begin{Def}
Para $x$ fijo y para cada $t\geq0$, sea $I_{x}\left(t\right)=1$ si $X\left(t\right)\leq x$,  $I_{x}\left(t\right)=0$ en caso contrario, y def\'inanse los tiempos promedio
\begin{eqnarray*}
\overline{X}&=&lim_{t\rightarrow\infty}\frac{1}{t}\int_{0}^{\infty}X\left(u\right)du\\
\prob\left(X_{\infty}\leq x\right)&=&lim_{t\rightarrow\infty}\frac{1}{t}\int_{0}^{\infty}I_{x}\left(u\right)du,
\end{eqnarray*}
cuando estos l\'imites existan.
\end{Def}

Como consecuencia del teorema de Renovaci\'on-Recompensa, se tiene que el primer l\'imite  existe y es igual a la constante
\begin{eqnarray*}
\overline{X}&=&\frac{\esp\left[\int_{0}^{R_{1}}X\left(t\right)dt\right]}{\esp\left[R_{1}\right]},
\end{eqnarray*}
suponiendo que ambas esperanzas son finitas.

\begin{Note}
\begin{itemize}
\item[a)] Si el proceso regenerativo $X$ es positivo recurrente y tiene trayectorias muestrales no negativas, entonces la ecuaci\'on anterior es v\'alida.
\item[b)] Si $X$ es positivo recurrente regenerativo, podemos construir una \'unica versi\'on estacionaria de este proceso, $X_{e}=\left\{X_{e}\left(t\right)\right\}$, donde $X_{e}$ es un proceso estoc\'astico regenerativo y estrictamente estacionario, con distribuci\'on marginal distribuida como $X_{\infty}$
\end{itemize}
\end{Note}

%________________________________________________________________________
\subsection{Procesos Regenerativos}
%________________________________________________________________________

Para $\left\{X\left(t\right):t\geq0\right\}$ Proceso Estoc\'astico a tiempo continuo con estado de espacios $S$, que es un espacio m\'etrico, con trayectorias continuas por la derecha y con l\'imites por la izquierda c.s. Sea $N\left(t\right)$ un proceso de renovaci\'on en $\rea_{+}$ definido en el mismo espacio de probabilidad que $X\left(t\right)$, con tiempos de renovaci\'on $T$ y tiempos de inter-renovaci\'on $\xi_{n}=T_{n}-T_{n-1}$, con misma distribuci\'on $F$ de media finita $\mu$.



\begin{Def}
Para el proceso $\left\{\left(N\left(t\right),X\left(t\right)\right):t\geq0\right\}$, sus trayectoria muestrales en el intervalo de tiempo $\left[T_{n-1},T_{n}\right)$ est\'an descritas por
\begin{eqnarray*}
\zeta_{n}=\left(\xi_{n},\left\{X\left(T_{n-1}+t\right):0\leq t<\xi_{n}\right\}\right)
\end{eqnarray*}
Este $\zeta_{n}$ es el $n$-\'esimo segmento del proceso. El proceso es regenerativo sobre los tiempos $T_{n}$ si sus segmentos $\zeta_{n}$ son independientes e id\'enticamennte distribuidos.
\end{Def}


\begin{Obs}
Si $\tilde{X}\left(t\right)$ con espacio de estados $\tilde{S}$ es regenerativo sobre $T_{n}$, entonces $X\left(t\right)=f\left(\tilde{X}\left(t\right)\right)$ tambi\'en es regenerativo sobre $T_{n}$, para cualquier funci\'on $f:\tilde{S}\rightarrow S$.
\end{Obs}

\begin{Obs}
Los procesos regenerativos son crudamente regenerativos, pero no al rev\'es.
\end{Obs}

\begin{Def}[Definici\'on Cl\'asica]
Un proceso estoc\'astico $X=\left\{X\left(t\right):t\geq0\right\}$ es llamado regenerativo is existe una variable aleatoria $R_{1}>0$ tal que
\begin{itemize}
\item[i)] $\left\{X\left(t+R_{1}\right):t\geq0\right\}$ es independiente de $\left\{\left\{X\left(t\right):t<R_{1}\right\},\right\}$
\item[ii)] $\left\{X\left(t+R_{1}\right):t\geq0\right\}$ es estoc\'asticamente equivalente a $\left\{X\left(t\right):t>0\right\}$
\end{itemize}

Llamamos a $R_{1}$ tiempo de regeneraci\'on, y decimos que $X$ se regenera en este punto.
\end{Def}

$\left\{X\left(t+R_{1}\right)\right\}$ es regenerativo con tiempo de regeneraci\'on $R_{2}$, independiente de $R_{1}$ pero con la misma distribuci\'on que $R_{1}$. Procediendo de esta manera se obtiene una secuencia de variables aleatorias independientes e id\'enticamente distribuidas $\left\{R_{n}\right\}$ llamados longitudes de ciclo. Si definimos a $Z_{k}\equiv R_{1}+R_{2}+\cdots+R_{k}$, se tiene un proceso de renovaci\'on llamado proceso de renovaci\'on encajado para $X$.

\begin{Note}
Un proceso regenerativo con media de la longitud de ciclo finita es llamado positivo recurrente.
\end{Note}


\begin{Def}
Para $x$ fijo y para cada $t\geq0$, sea $I_{x}\left(t\right)=1$ si $X\left(t\right)\leq x$,  $I_{x}\left(t\right)=0$ en caso contrario, y def\'inanse los tiempos promedio
\begin{eqnarray*}
\overline{X}&=&lim_{t\rightarrow\infty}\frac{1}{t}\int_{0}^{\infty}X\left(u\right)du\\
\prob\left(X_{\infty}\leq x\right)&=&lim_{t\rightarrow\infty}\frac{1}{t}\int_{0}^{\infty}I_{x}\left(u\right)du,
\end{eqnarray*}
cuando estos l\'imites existan.
\end{Def}

Como consecuencia del teorema de Renovaci\'on-Recompensa, se tiene que el primer l\'imite  existe y es igual a la constante
\begin{eqnarray*}
\overline{X}&=&\frac{\esp\left[\int_{0}^{R_{1}}X\left(t\right)dt\right]}{\esp\left[R_{1}\right]},
\end{eqnarray*}
suponiendo que ambas esperanzas son finitas.

\begin{Note}
\begin{itemize}
\item[a)] Si el proceso regenerativo $X$ es positivo recurrente y tiene trayectorias muestrales no negativas, entonces la ecuaci\'on anterior es v\'alida.
\item[b)] Si $X$ es positivo recurrente regenerativo, podemos construir una \'unica versi\'on estacionaria de este proceso, $X_{e}=\left\{X_{e}\left(t\right)\right\}$, donde $X_{e}$ es un proceso estoc\'astico regenerativo y estrictamente estacionario, con distribuci\'on marginal distribuida como $X_{\infty}$
\end{itemize}
\end{Note}
%__________________________________________________________________________________________
\subsection{Procesos Regenerativos Estacionarios - Stidham \cite{Stidham}}
%__________________________________________________________________________________________


Un proceso estoc\'astico a tiempo continuo $\left\{V\left(t\right),t\geq0\right\}$ es un proceso regenerativo si existe una sucesi\'on de variables aleatorias independientes e id\'enticamente distribuidas $\left\{X_{1},X_{2},\ldots\right\}$, sucesi\'on de renovaci\'on, tal que para cualquier conjunto de Borel $A$, 

\begin{eqnarray*}
\prob\left\{V\left(t\right)\in A|X_{1}+X_{2}+\cdots+X_{R\left(t\right)}=s,\left\{V\left(\tau\right),\tau<s\right\}\right\}=\prob\left\{V\left(t-s\right)\in A|X_{1}>t-s\right\},
\end{eqnarray*}
para todo $0\leq s\leq t$, donde $R\left(t\right)=\max\left\{X_{1}+X_{2}+\cdots+X_{j}\leq t\right\}=$n\'umero de renovaciones ({\emph{puntos de regeneraci\'on}}) que ocurren en $\left[0,t\right]$. El intervalo $\left[0,X_{1}\right)$ es llamado {\emph{primer ciclo de regeneraci\'on}} de $\left\{V\left(t \right),t\geq0\right\}$, $\left[X_{1},X_{1}+X_{2}\right)$ el {\emph{segundo ciclo de regeneraci\'on}}, y as\'i sucesivamente.

Sea $X=X_{1}$ y sea $F$ la funci\'on de distrbuci\'on de $X$


\begin{Def}
Se define el proceso estacionario, $\left\{V^{*}\left(t\right),t\geq0\right\}$, para $\left\{V\left(t\right),t\geq0\right\}$ por

\begin{eqnarray*}
\prob\left\{V\left(t\right)\in A\right\}=\frac{1}{\esp\left[X\right]}\int_{0}^{\infty}\prob\left\{V\left(t+x\right)\in A|X>x\right\}\left(1-F\left(x\right)\right)dx,
\end{eqnarray*} 
para todo $t\geq0$ y todo conjunto de Borel $A$.
\end{Def}

\begin{Def}
Una distribuci\'on se dice que es {\emph{aritm\'etica}} si todos sus puntos de incremento son m\'ultiplos de la forma $0,\lambda, 2\lambda,\ldots$ para alguna $\lambda>0$ entera.
\end{Def}


\begin{Def}
Una modificaci\'on medible de un proceso $\left\{V\left(t\right),t\geq0\right\}$, es una versi\'on de este, $\left\{V\left(t,w\right)\right\}$ conjuntamente medible para $t\geq0$ y para $w\in S$, $S$ espacio de estados para $\left\{V\left(t\right),t\geq0\right\}$.
\end{Def}

\begin{Teo}
Sea $\left\{V\left(t\right),t\geq\right\}$ un proceso regenerativo no negativo con modificaci\'on medible. Sea $\esp\left[X\right]<\infty$. Entonces el proceso estacionario dado por la ecuaci\'on anterior est\'a bien definido y tiene funci\'on de distribuci\'on independiente de $t$, adem\'as
\begin{itemize}
\item[i)] \begin{eqnarray*}
\esp\left[V^{*}\left(0\right)\right]&=&\frac{\esp\left[\int_{0}^{X}V\left(s\right)ds\right]}{\esp\left[X\right]}\end{eqnarray*}
\item[ii)] Si $\esp\left[V^{*}\left(0\right)\right]<\infty$, equivalentemente, si $\esp\left[\int_{0}^{X}V\left(s\right)ds\right]<\infty$,entonces
\begin{eqnarray*}
\frac{\int_{0}^{t}V\left(s\right)ds}{t}\rightarrow\frac{\esp\left[\int_{0}^{X}V\left(s\right)ds\right]}{\esp\left[X\right]}
\end{eqnarray*}
con probabilidad 1 y en media, cuando $t\rightarrow\infty$.
\end{itemize}
\end{Teo}


%__________________________________________________________________________________________
\subsection{Procesos Regenerativos Estacionarios - Stidham \cite{Stidham}}
%__________________________________________________________________________________________


Un proceso estoc\'astico a tiempo continuo $\left\{V\left(t\right),t\geq0\right\}$ es un proceso regenerativo si existe una sucesi\'on de variables aleatorias independientes e id\'enticamente distribuidas $\left\{X_{1},X_{2},\ldots\right\}$, sucesi\'on de renovaci\'on, tal que para cualquier conjunto de Borel $A$, 

\begin{eqnarray*}
\prob\left\{V\left(t\right)\in A|X_{1}+X_{2}+\cdots+X_{R\left(t\right)}=s,\left\{V\left(\tau\right),\tau<s\right\}\right\}=\prob\left\{V\left(t-s\right)\in A|X_{1}>t-s\right\},
\end{eqnarray*}
para todo $0\leq s\leq t$, donde $R\left(t\right)=\max\left\{X_{1}+X_{2}+\cdots+X_{j}\leq t\right\}=$n\'umero de renovaciones ({\emph{puntos de regeneraci\'on}}) que ocurren en $\left[0,t\right]$. El intervalo $\left[0,X_{1}\right)$ es llamado {\emph{primer ciclo de regeneraci\'on}} de $\left\{V\left(t \right),t\geq0\right\}$, $\left[X_{1},X_{1}+X_{2}\right)$ el {\emph{segundo ciclo de regeneraci\'on}}, y as\'i sucesivamente.

Sea $X=X_{1}$ y sea $F$ la funci\'on de distrbuci\'on de $X$


\begin{Def}
Se define el proceso estacionario, $\left\{V^{*}\left(t\right),t\geq0\right\}$, para $\left\{V\left(t\right),t\geq0\right\}$ por

\begin{eqnarray*}
\prob\left\{V\left(t\right)\in A\right\}=\frac{1}{\esp\left[X\right]}\int_{0}^{\infty}\prob\left\{V\left(t+x\right)\in A|X>x\right\}\left(1-F\left(x\right)\right)dx,
\end{eqnarray*} 
para todo $t\geq0$ y todo conjunto de Borel $A$.
\end{Def}

\begin{Def}
Una distribuci\'on se dice que es {\emph{aritm\'etica}} si todos sus puntos de incremento son m\'ultiplos de la forma $0,\lambda, 2\lambda,\ldots$ para alguna $\lambda>0$ entera.
\end{Def}


\begin{Def}
Una modificaci\'on medible de un proceso $\left\{V\left(t\right),t\geq0\right\}$, es una versi\'on de este, $\left\{V\left(t,w\right)\right\}$ conjuntamente medible para $t\geq0$ y para $w\in S$, $S$ espacio de estados para $\left\{V\left(t\right),t\geq0\right\}$.
\end{Def}

\begin{Teo}
Sea $\left\{V\left(t\right),t\geq\right\}$ un proceso regenerativo no negativo con modificaci\'on medible. Sea $\esp\left[X\right]<\infty$. Entonces el proceso estacionario dado por la ecuaci\'on anterior est\'a bien definido y tiene funci\'on de distribuci\'on independiente de $t$, adem\'as
\begin{itemize}
\item[i)] \begin{eqnarray*}
\esp\left[V^{*}\left(0\right)\right]&=&\frac{\esp\left[\int_{0}^{X}V\left(s\right)ds\right]}{\esp\left[X\right]}\end{eqnarray*}
\item[ii)] Si $\esp\left[V^{*}\left(0\right)\right]<\infty$, equivalentemente, si $\esp\left[\int_{0}^{X}V\left(s\right)ds\right]<\infty$,entonces
\begin{eqnarray*}
\frac{\int_{0}^{t}V\left(s\right)ds}{t}\rightarrow\frac{\esp\left[\int_{0}^{X}V\left(s\right)ds\right]}{\esp\left[X\right]}
\end{eqnarray*}
con probabilidad 1 y en media, cuando $t\rightarrow\infty$.
\end{itemize}
\end{Teo}
%___________________________________________________________________________________________
%
\subsection{Propiedades de los Procesos de Renovaci\'on}
%___________________________________________________________________________________________
%

Los tiempos $T_{n}$ est\'an relacionados con los conteos de $N\left(t\right)$ por

\begin{eqnarray*}
\left\{N\left(t\right)\geq n\right\}&=&\left\{T_{n}\leq t\right\}\\
T_{N\left(t\right)}\leq &t&<T_{N\left(t\right)+1},
\end{eqnarray*}

adem\'as $N\left(T_{n}\right)=n$, y 

\begin{eqnarray*}
N\left(t\right)=\max\left\{n:T_{n}\leq t\right\}=\min\left\{n:T_{n+1}>t\right\}
\end{eqnarray*}

Por propiedades de la convoluci\'on se sabe que

\begin{eqnarray*}
P\left\{T_{n}\leq t\right\}=F^{n\star}\left(t\right)
\end{eqnarray*}
que es la $n$-\'esima convoluci\'on de $F$. Entonces 

\begin{eqnarray*}
\left\{N\left(t\right)\geq n\right\}&=&\left\{T_{n}\leq t\right\}\\
P\left\{N\left(t\right)\leq n\right\}&=&1-F^{\left(n+1\right)\star}\left(t\right)
\end{eqnarray*}

Adem\'as usando el hecho de que $\esp\left[N\left(t\right)\right]=\sum_{n=1}^{\infty}P\left\{N\left(t\right)\geq n\right\}$
se tiene que

\begin{eqnarray*}
\esp\left[N\left(t\right)\right]=\sum_{n=1}^{\infty}F^{n\star}\left(t\right)
\end{eqnarray*}

\begin{Prop}
Para cada $t\geq0$, la funci\'on generadora de momentos $\esp\left[e^{\alpha N\left(t\right)}\right]$ existe para alguna $\alpha$ en una vecindad del 0, y de aqu\'i que $\esp\left[N\left(t\right)^{m}\right]<\infty$, para $m\geq1$.
\end{Prop}


\begin{Note}
Si el primer tiempo de renovaci\'on $\xi_{1}$ no tiene la misma distribuci\'on que el resto de las $\xi_{n}$, para $n\geq2$, a $N\left(t\right)$ se le llama Proceso de Renovaci\'on retardado, donde si $\xi$ tiene distribuci\'on $G$, entonces el tiempo $T_{n}$ de la $n$-\'esima renovaci\'on tiene distribuci\'on $G\star F^{\left(n-1\right)\star}\left(t\right)$
\end{Note}


\begin{Teo}
Para una constante $\mu\leq\infty$ ( o variable aleatoria), las siguientes expresiones son equivalentes:

\begin{eqnarray}
lim_{n\rightarrow\infty}n^{-1}T_{n}&=&\mu,\textrm{ c.s.}\\
lim_{t\rightarrow\infty}t^{-1}N\left(t\right)&=&1/\mu,\textrm{ c.s.}
\end{eqnarray}
\end{Teo}


Es decir, $T_{n}$ satisface la Ley Fuerte de los Grandes N\'umeros s\'i y s\'olo s\'i $N\left/t\right)$ la cumple.


\begin{Coro}[Ley Fuerte de los Grandes N\'umeros para Procesos de Renovaci\'on]
Si $N\left(t\right)$ es un proceso de renovaci\'on cuyos tiempos de inter-renovaci\'on tienen media $\mu\leq\infty$, entonces
\begin{eqnarray}
t^{-1}N\left(t\right)\rightarrow 1/\mu,\textrm{ c.s. cuando }t\rightarrow\infty.
\end{eqnarray}

\end{Coro}


Considerar el proceso estoc\'astico de valores reales $\left\{Z\left(t\right):t\geq0\right\}$ en el mismo espacio de probabilidad que $N\left(t\right)$

\begin{Def}
Para el proceso $\left\{Z\left(t\right):t\geq0\right\}$ se define la fluctuaci\'on m\'axima de $Z\left(t\right)$ en el intervalo $\left(T_{n-1},T_{n}\right]$:
\begin{eqnarray*}
M_{n}=\sup_{T_{n-1}<t\leq T_{n}}|Z\left(t\right)-Z\left(T_{n-1}\right)|
\end{eqnarray*}
\end{Def}

\begin{Teo}
Sup\'ongase que $n^{-1}T_{n}\rightarrow\mu$ c.s. cuando $n\rightarrow\infty$, donde $\mu\leq\infty$ es una constante o variable aleatoria. Sea $a$ una constante o variable aleatoria que puede ser infinita cuando $\mu$ es finita, y considere las expresiones l\'imite:
\begin{eqnarray}
lim_{n\rightarrow\infty}n^{-1}Z\left(T_{n}\right)&=&a,\textrm{ c.s.}\\
lim_{t\rightarrow\infty}t^{-1}Z\left(t\right)&=&a/\mu,\textrm{ c.s.}
\end{eqnarray}
La segunda expresi\'on implica la primera. Conversamente, la primera implica la segunda si el proceso $Z\left(t\right)$ es creciente, o si $lim_{n\rightarrow\infty}n^{-1}M_{n}=0$ c.s.
\end{Teo}

\begin{Coro}
Si $N\left(t\right)$ es un proceso de renovaci\'on, y $\left(Z\left(T_{n}\right)-Z\left(T_{n-1}\right),M_{n}\right)$, para $n\geq1$, son variables aleatorias independientes e id\'enticamente distribuidas con media finita, entonces,
\begin{eqnarray}
lim_{t\rightarrow\infty}t^{-1}Z\left(t\right)\rightarrow\frac{\esp\left[Z\left(T_{1}\right)-Z\left(T_{0}\right)\right]}{\esp\left[T_{1}\right]},\textrm{ c.s. cuando  }t\rightarrow\infty.
\end{eqnarray}
\end{Coro}


%___________________________________________________________________________________________
%
\subsection{Propiedades de los Procesos de Renovaci\'on}
%___________________________________________________________________________________________
%

Los tiempos $T_{n}$ est\'an relacionados con los conteos de $N\left(t\right)$ por

\begin{eqnarray*}
\left\{N\left(t\right)\geq n\right\}&=&\left\{T_{n}\leq t\right\}\\
T_{N\left(t\right)}\leq &t&<T_{N\left(t\right)+1},
\end{eqnarray*}

adem\'as $N\left(T_{n}\right)=n$, y 

\begin{eqnarray*}
N\left(t\right)=\max\left\{n:T_{n}\leq t\right\}=\min\left\{n:T_{n+1}>t\right\}
\end{eqnarray*}

Por propiedades de la convoluci\'on se sabe que

\begin{eqnarray*}
P\left\{T_{n}\leq t\right\}=F^{n\star}\left(t\right)
\end{eqnarray*}
que es la $n$-\'esima convoluci\'on de $F$. Entonces 

\begin{eqnarray*}
\left\{N\left(t\right)\geq n\right\}&=&\left\{T_{n}\leq t\right\}\\
P\left\{N\left(t\right)\leq n\right\}&=&1-F^{\left(n+1\right)\star}\left(t\right)
\end{eqnarray*}

Adem\'as usando el hecho de que $\esp\left[N\left(t\right)\right]=\sum_{n=1}^{\infty}P\left\{N\left(t\right)\geq n\right\}$
se tiene que

\begin{eqnarray*}
\esp\left[N\left(t\right)\right]=\sum_{n=1}^{\infty}F^{n\star}\left(t\right)
\end{eqnarray*}

\begin{Prop}
Para cada $t\geq0$, la funci\'on generadora de momentos $\esp\left[e^{\alpha N\left(t\right)}\right]$ existe para alguna $\alpha$ en una vecindad del 0, y de aqu\'i que $\esp\left[N\left(t\right)^{m}\right]<\infty$, para $m\geq1$.
\end{Prop}


\begin{Note}
Si el primer tiempo de renovaci\'on $\xi_{1}$ no tiene la misma distribuci\'on que el resto de las $\xi_{n}$, para $n\geq2$, a $N\left(t\right)$ se le llama Proceso de Renovaci\'on retardado, donde si $\xi$ tiene distribuci\'on $G$, entonces el tiempo $T_{n}$ de la $n$-\'esima renovaci\'on tiene distribuci\'on $G\star F^{\left(n-1\right)\star}\left(t\right)$
\end{Note}


\begin{Teo}
Para una constante $\mu\leq\infty$ ( o variable aleatoria), las siguientes expresiones son equivalentes:

\begin{eqnarray}
lim_{n\rightarrow\infty}n^{-1}T_{n}&=&\mu,\textrm{ c.s.}\\
lim_{t\rightarrow\infty}t^{-1}N\left(t\right)&=&1/\mu,\textrm{ c.s.}
\end{eqnarray}
\end{Teo}


Es decir, $T_{n}$ satisface la Ley Fuerte de los Grandes N\'umeros s\'i y s\'olo s\'i $N\left/t\right)$ la cumple.


\begin{Coro}[Ley Fuerte de los Grandes N\'umeros para Procesos de Renovaci\'on]
Si $N\left(t\right)$ es un proceso de renovaci\'on cuyos tiempos de inter-renovaci\'on tienen media $\mu\leq\infty$, entonces
\begin{eqnarray}
t^{-1}N\left(t\right)\rightarrow 1/\mu,\textrm{ c.s. cuando }t\rightarrow\infty.
\end{eqnarray}

\end{Coro}


Considerar el proceso estoc\'astico de valores reales $\left\{Z\left(t\right):t\geq0\right\}$ en el mismo espacio de probabilidad que $N\left(t\right)$

\begin{Def}
Para el proceso $\left\{Z\left(t\right):t\geq0\right\}$ se define la fluctuaci\'on m\'axima de $Z\left(t\right)$ en el intervalo $\left(T_{n-1},T_{n}\right]$:
\begin{eqnarray*}
M_{n}=\sup_{T_{n-1}<t\leq T_{n}}|Z\left(t\right)-Z\left(T_{n-1}\right)|
\end{eqnarray*}
\end{Def}

\begin{Teo}
Sup\'ongase que $n^{-1}T_{n}\rightarrow\mu$ c.s. cuando $n\rightarrow\infty$, donde $\mu\leq\infty$ es una constante o variable aleatoria. Sea $a$ una constante o variable aleatoria que puede ser infinita cuando $\mu$ es finita, y considere las expresiones l\'imite:
\begin{eqnarray}
lim_{n\rightarrow\infty}n^{-1}Z\left(T_{n}\right)&=&a,\textrm{ c.s.}\\
lim_{t\rightarrow\infty}t^{-1}Z\left(t\right)&=&a/\mu,\textrm{ c.s.}
\end{eqnarray}
La segunda expresi\'on implica la primera. Conversamente, la primera implica la segunda si el proceso $Z\left(t\right)$ es creciente, o si $lim_{n\rightarrow\infty}n^{-1}M_{n}=0$ c.s.
\end{Teo}

\begin{Coro}
Si $N\left(t\right)$ es un proceso de renovaci\'on, y $\left(Z\left(T_{n}\right)-Z\left(T_{n-1}\right),M_{n}\right)$, para $n\geq1$, son variables aleatorias independientes e id\'enticamente distribuidas con media finita, entonces,
\begin{eqnarray}
lim_{t\rightarrow\infty}t^{-1}Z\left(t\right)\rightarrow\frac{\esp\left[Z\left(T_{1}\right)-Z\left(T_{0}\right)\right]}{\esp\left[T_{1}\right]},\textrm{ c.s. cuando  }t\rightarrow\infty.
\end{eqnarray}
\end{Coro}

%___________________________________________________________________________________________
%
\subsection{Propiedades de los Procesos de Renovaci\'on}
%___________________________________________________________________________________________
%

Los tiempos $T_{n}$ est\'an relacionados con los conteos de $N\left(t\right)$ por

\begin{eqnarray*}
\left\{N\left(t\right)\geq n\right\}&=&\left\{T_{n}\leq t\right\}\\
T_{N\left(t\right)}\leq &t&<T_{N\left(t\right)+1},
\end{eqnarray*}

adem\'as $N\left(T_{n}\right)=n$, y 

\begin{eqnarray*}
N\left(t\right)=\max\left\{n:T_{n}\leq t\right\}=\min\left\{n:T_{n+1}>t\right\}
\end{eqnarray*}

Por propiedades de la convoluci\'on se sabe que

\begin{eqnarray*}
P\left\{T_{n}\leq t\right\}=F^{n\star}\left(t\right)
\end{eqnarray*}
que es la $n$-\'esima convoluci\'on de $F$. Entonces 

\begin{eqnarray*}
\left\{N\left(t\right)\geq n\right\}&=&\left\{T_{n}\leq t\right\}\\
P\left\{N\left(t\right)\leq n\right\}&=&1-F^{\left(n+1\right)\star}\left(t\right)
\end{eqnarray*}

Adem\'as usando el hecho de que $\esp\left[N\left(t\right)\right]=\sum_{n=1}^{\infty}P\left\{N\left(t\right)\geq n\right\}$
se tiene que

\begin{eqnarray*}
\esp\left[N\left(t\right)\right]=\sum_{n=1}^{\infty}F^{n\star}\left(t\right)
\end{eqnarray*}

\begin{Prop}
Para cada $t\geq0$, la funci\'on generadora de momentos $\esp\left[e^{\alpha N\left(t\right)}\right]$ existe para alguna $\alpha$ en una vecindad del 0, y de aqu\'i que $\esp\left[N\left(t\right)^{m}\right]<\infty$, para $m\geq1$.
\end{Prop}


\begin{Note}
Si el primer tiempo de renovaci\'on $\xi_{1}$ no tiene la misma distribuci\'on que el resto de las $\xi_{n}$, para $n\geq2$, a $N\left(t\right)$ se le llama Proceso de Renovaci\'on retardado, donde si $\xi$ tiene distribuci\'on $G$, entonces el tiempo $T_{n}$ de la $n$-\'esima renovaci\'on tiene distribuci\'on $G\star F^{\left(n-1\right)\star}\left(t\right)$
\end{Note}


\begin{Teo}
Para una constante $\mu\leq\infty$ ( o variable aleatoria), las siguientes expresiones son equivalentes:

\begin{eqnarray}
lim_{n\rightarrow\infty}n^{-1}T_{n}&=&\mu,\textrm{ c.s.}\\
lim_{t\rightarrow\infty}t^{-1}N\left(t\right)&=&1/\mu,\textrm{ c.s.}
\end{eqnarray}
\end{Teo}


Es decir, $T_{n}$ satisface la Ley Fuerte de los Grandes N\'umeros s\'i y s\'olo s\'i $N\left/t\right)$ la cumple.


\begin{Coro}[Ley Fuerte de los Grandes N\'umeros para Procesos de Renovaci\'on]
Si $N\left(t\right)$ es un proceso de renovaci\'on cuyos tiempos de inter-renovaci\'on tienen media $\mu\leq\infty$, entonces
\begin{eqnarray}
t^{-1}N\left(t\right)\rightarrow 1/\mu,\textrm{ c.s. cuando }t\rightarrow\infty.
\end{eqnarray}

\end{Coro}


Considerar el proceso estoc\'astico de valores reales $\left\{Z\left(t\right):t\geq0\right\}$ en el mismo espacio de probabilidad que $N\left(t\right)$

\begin{Def}
Para el proceso $\left\{Z\left(t\right):t\geq0\right\}$ se define la fluctuaci\'on m\'axima de $Z\left(t\right)$ en el intervalo $\left(T_{n-1},T_{n}\right]$:
\begin{eqnarray*}
M_{n}=\sup_{T_{n-1}<t\leq T_{n}}|Z\left(t\right)-Z\left(T_{n-1}\right)|
\end{eqnarray*}
\end{Def}

\begin{Teo}
Sup\'ongase que $n^{-1}T_{n}\rightarrow\mu$ c.s. cuando $n\rightarrow\infty$, donde $\mu\leq\infty$ es una constante o variable aleatoria. Sea $a$ una constante o variable aleatoria que puede ser infinita cuando $\mu$ es finita, y considere las expresiones l\'imite:
\begin{eqnarray}
lim_{n\rightarrow\infty}n^{-1}Z\left(T_{n}\right)&=&a,\textrm{ c.s.}\\
lim_{t\rightarrow\infty}t^{-1}Z\left(t\right)&=&a/\mu,\textrm{ c.s.}
\end{eqnarray}
La segunda expresi\'on implica la primera. Conversamente, la primera implica la segunda si el proceso $Z\left(t\right)$ es creciente, o si $lim_{n\rightarrow\infty}n^{-1}M_{n}=0$ c.s.
\end{Teo}

\begin{Coro}
Si $N\left(t\right)$ es un proceso de renovaci\'on, y $\left(Z\left(T_{n}\right)-Z\left(T_{n-1}\right),M_{n}\right)$, para $n\geq1$, son variables aleatorias independientes e id\'enticamente distribuidas con media finita, entonces,
\begin{eqnarray}
lim_{t\rightarrow\infty}t^{-1}Z\left(t\right)\rightarrow\frac{\esp\left[Z\left(T_{1}\right)-Z\left(T_{0}\right)\right]}{\esp\left[T_{1}\right]},\textrm{ c.s. cuando  }t\rightarrow\infty.
\end{eqnarray}
\end{Coro}

%___________________________________________________________________________________________
%
\subsection{Propiedades de los Procesos de Renovaci\'on}
%___________________________________________________________________________________________
%

Los tiempos $T_{n}$ est\'an relacionados con los conteos de $N\left(t\right)$ por

\begin{eqnarray*}
\left\{N\left(t\right)\geq n\right\}&=&\left\{T_{n}\leq t\right\}\\
T_{N\left(t\right)}\leq &t&<T_{N\left(t\right)+1},
\end{eqnarray*}

adem\'as $N\left(T_{n}\right)=n$, y 

\begin{eqnarray*}
N\left(t\right)=\max\left\{n:T_{n}\leq t\right\}=\min\left\{n:T_{n+1}>t\right\}
\end{eqnarray*}

Por propiedades de la convoluci\'on se sabe que

\begin{eqnarray*}
P\left\{T_{n}\leq t\right\}=F^{n\star}\left(t\right)
\end{eqnarray*}
que es la $n$-\'esima convoluci\'on de $F$. Entonces 

\begin{eqnarray*}
\left\{N\left(t\right)\geq n\right\}&=&\left\{T_{n}\leq t\right\}\\
P\left\{N\left(t\right)\leq n\right\}&=&1-F^{\left(n+1\right)\star}\left(t\right)
\end{eqnarray*}

Adem\'as usando el hecho de que $\esp\left[N\left(t\right)\right]=\sum_{n=1}^{\infty}P\left\{N\left(t\right)\geq n\right\}$
se tiene que

\begin{eqnarray*}
\esp\left[N\left(t\right)\right]=\sum_{n=1}^{\infty}F^{n\star}\left(t\right)
\end{eqnarray*}

\begin{Prop}
Para cada $t\geq0$, la funci\'on generadora de momentos $\esp\left[e^{\alpha N\left(t\right)}\right]$ existe para alguna $\alpha$ en una vecindad del 0, y de aqu\'i que $\esp\left[N\left(t\right)^{m}\right]<\infty$, para $m\geq1$.
\end{Prop}


\begin{Note}
Si el primer tiempo de renovaci\'on $\xi_{1}$ no tiene la misma distribuci\'on que el resto de las $\xi_{n}$, para $n\geq2$, a $N\left(t\right)$ se le llama Proceso de Renovaci\'on retardado, donde si $\xi$ tiene distribuci\'on $G$, entonces el tiempo $T_{n}$ de la $n$-\'esima renovaci\'on tiene distribuci\'on $G\star F^{\left(n-1\right)\star}\left(t\right)$
\end{Note}


\begin{Teo}
Para una constante $\mu\leq\infty$ ( o variable aleatoria), las siguientes expresiones son equivalentes:

\begin{eqnarray}
lim_{n\rightarrow\infty}n^{-1}T_{n}&=&\mu,\textrm{ c.s.}\\
lim_{t\rightarrow\infty}t^{-1}N\left(t\right)&=&1/\mu,\textrm{ c.s.}
\end{eqnarray}
\end{Teo}


Es decir, $T_{n}$ satisface la Ley Fuerte de los Grandes N\'umeros s\'i y s\'olo s\'i $N\left/t\right)$ la cumple.


\begin{Coro}[Ley Fuerte de los Grandes N\'umeros para Procesos de Renovaci\'on]
Si $N\left(t\right)$ es un proceso de renovaci\'on cuyos tiempos de inter-renovaci\'on tienen media $\mu\leq\infty$, entonces
\begin{eqnarray}
t^{-1}N\left(t\right)\rightarrow 1/\mu,\textrm{ c.s. cuando }t\rightarrow\infty.
\end{eqnarray}

\end{Coro}


Considerar el proceso estoc\'astico de valores reales $\left\{Z\left(t\right):t\geq0\right\}$ en el mismo espacio de probabilidad que $N\left(t\right)$

\begin{Def}
Para el proceso $\left\{Z\left(t\right):t\geq0\right\}$ se define la fluctuaci\'on m\'axima de $Z\left(t\right)$ en el intervalo $\left(T_{n-1},T_{n}\right]$:
\begin{eqnarray*}
M_{n}=\sup_{T_{n-1}<t\leq T_{n}}|Z\left(t\right)-Z\left(T_{n-1}\right)|
\end{eqnarray*}
\end{Def}

\begin{Teo}
Sup\'ongase que $n^{-1}T_{n}\rightarrow\mu$ c.s. cuando $n\rightarrow\infty$, donde $\mu\leq\infty$ es una constante o variable aleatoria. Sea $a$ una constante o variable aleatoria que puede ser infinita cuando $\mu$ es finita, y considere las expresiones l\'imite:
\begin{eqnarray}
lim_{n\rightarrow\infty}n^{-1}Z\left(T_{n}\right)&=&a,\textrm{ c.s.}\\
lim_{t\rightarrow\infty}t^{-1}Z\left(t\right)&=&a/\mu,\textrm{ c.s.}
\end{eqnarray}
La segunda expresi\'on implica la primera. Conversamente, la primera implica la segunda si el proceso $Z\left(t\right)$ es creciente, o si $lim_{n\rightarrow\infty}n^{-1}M_{n}=0$ c.s.
\end{Teo}

\begin{Coro}
Si $N\left(t\right)$ es un proceso de renovaci\'on, y $\left(Z\left(T_{n}\right)-Z\left(T_{n-1}\right),M_{n}\right)$, para $n\geq1$, son variables aleatorias independientes e id\'enticamente distribuidas con media finita, entonces,
\begin{eqnarray}
lim_{t\rightarrow\infty}t^{-1}Z\left(t\right)\rightarrow\frac{\esp\left[Z\left(T_{1}\right)-Z\left(T_{0}\right)\right]}{\esp\left[T_{1}\right]},\textrm{ c.s. cuando  }t\rightarrow\infty.
\end{eqnarray}
\end{Coro}


%__________________________________________________________________________________________
\subsection{Procesos Regenerativos Estacionarios - Stidham \cite{Stidham}}
%__________________________________________________________________________________________


Un proceso estoc\'astico a tiempo continuo $\left\{V\left(t\right),t\geq0\right\}$ es un proceso regenerativo si existe una sucesi\'on de variables aleatorias independientes e id\'enticamente distribuidas $\left\{X_{1},X_{2},\ldots\right\}$, sucesi\'on de renovaci\'on, tal que para cualquier conjunto de Borel $A$, 

\begin{eqnarray*}
\prob\left\{V\left(t\right)\in A|X_{1}+X_{2}+\cdots+X_{R\left(t\right)}=s,\left\{V\left(\tau\right),\tau<s\right\}\right\}=\prob\left\{V\left(t-s\right)\in A|X_{1}>t-s\right\},
\end{eqnarray*}
para todo $0\leq s\leq t$, donde $R\left(t\right)=\max\left\{X_{1}+X_{2}+\cdots+X_{j}\leq t\right\}=$n\'umero de renovaciones ({\emph{puntos de regeneraci\'on}}) que ocurren en $\left[0,t\right]$. El intervalo $\left[0,X_{1}\right)$ es llamado {\emph{primer ciclo de regeneraci\'on}} de $\left\{V\left(t \right),t\geq0\right\}$, $\left[X_{1},X_{1}+X_{2}\right)$ el {\emph{segundo ciclo de regeneraci\'on}}, y as\'i sucesivamente.

Sea $X=X_{1}$ y sea $F$ la funci\'on de distrbuci\'on de $X$


\begin{Def}
Se define el proceso estacionario, $\left\{V^{*}\left(t\right),t\geq0\right\}$, para $\left\{V\left(t\right),t\geq0\right\}$ por

\begin{eqnarray*}
\prob\left\{V\left(t\right)\in A\right\}=\frac{1}{\esp\left[X\right]}\int_{0}^{\infty}\prob\left\{V\left(t+x\right)\in A|X>x\right\}\left(1-F\left(x\right)\right)dx,
\end{eqnarray*} 
para todo $t\geq0$ y todo conjunto de Borel $A$.
\end{Def}

\begin{Def}
Una distribuci\'on se dice que es {\emph{aritm\'etica}} si todos sus puntos de incremento son m\'ultiplos de la forma $0,\lambda, 2\lambda,\ldots$ para alguna $\lambda>0$ entera.
\end{Def}


\begin{Def}
Una modificaci\'on medible de un proceso $\left\{V\left(t\right),t\geq0\right\}$, es una versi\'on de este, $\left\{V\left(t,w\right)\right\}$ conjuntamente medible para $t\geq0$ y para $w\in S$, $S$ espacio de estados para $\left\{V\left(t\right),t\geq0\right\}$.
\end{Def}

\begin{Teo}
Sea $\left\{V\left(t\right),t\geq\right\}$ un proceso regenerativo no negativo con modificaci\'on medible. Sea $\esp\left[X\right]<\infty$. Entonces el proceso estacionario dado por la ecuaci\'on anterior est\'a bien definido y tiene funci\'on de distribuci\'on independiente de $t$, adem\'as
\begin{itemize}
\item[i)] \begin{eqnarray*}
\esp\left[V^{*}\left(0\right)\right]&=&\frac{\esp\left[\int_{0}^{X}V\left(s\right)ds\right]}{\esp\left[X\right]}\end{eqnarray*}
\item[ii)] Si $\esp\left[V^{*}\left(0\right)\right]<\infty$, equivalentemente, si $\esp\left[\int_{0}^{X}V\left(s\right)ds\right]<\infty$,entonces
\begin{eqnarray*}
\frac{\int_{0}^{t}V\left(s\right)ds}{t}\rightarrow\frac{\esp\left[\int_{0}^{X}V\left(s\right)ds\right]}{\esp\left[X\right]}
\end{eqnarray*}
con probabilidad 1 y en media, cuando $t\rightarrow\infty$.
\end{itemize}
\end{Teo}

%______________________________________________________________________
\subsection{Procesos de Renovaci\'on}
%______________________________________________________________________

\begin{Def}\label{Def.Tn}
Sean $0\leq T_{1}\leq T_{2}\leq \ldots$ son tiempos aleatorios infinitos en los cuales ocurren ciertos eventos. El n\'umero de tiempos $T_{n}$ en el intervalo $\left[0,t\right)$ es

\begin{eqnarray}
N\left(t\right)=\sum_{n=1}^{\infty}\indora\left(T_{n}\leq t\right),
\end{eqnarray}
para $t\geq0$.
\end{Def}

Si se consideran los puntos $T_{n}$ como elementos de $\rea_{+}$, y $N\left(t\right)$ es el n\'umero de puntos en $\rea$. El proceso denotado por $\left\{N\left(t\right):t\geq0\right\}$, denotado por $N\left(t\right)$, es un proceso puntual en $\rea_{+}$. Los $T_{n}$ son los tiempos de ocurrencia, el proceso puntual $N\left(t\right)$ es simple si su n\'umero de ocurrencias son distintas: $0<T_{1}<T_{2}<\ldots$ casi seguramente.

\begin{Def}
Un proceso puntual $N\left(t\right)$ es un proceso de renovaci\'on si los tiempos de interocurrencia $\xi_{n}=T_{n}-T_{n-1}$, para $n\geq1$, son independientes e identicamente distribuidos con distribuci\'on $F$, donde $F\left(0\right)=0$ y $T_{0}=0$. Los $T_{n}$ son llamados tiempos de renovaci\'on, referente a la independencia o renovaci\'on de la informaci\'on estoc\'astica en estos tiempos. Los $\xi_{n}$ son los tiempos de inter-renovaci\'on, y $N\left(t\right)$ es el n\'umero de renovaciones en el intervalo $\left[0,t\right)$
\end{Def}


\begin{Note}
Para definir un proceso de renovaci\'on para cualquier contexto, solamente hay que especificar una distribuci\'on $F$, con $F\left(0\right)=0$, para los tiempos de inter-renovaci\'on. La funci\'on $F$ en turno degune las otra variables aleatorias. De manera formal, existe un espacio de probabilidad y una sucesi\'on de variables aleatorias $\xi_{1},\xi_{2},\ldots$ definidas en este con distribuci\'on $F$. Entonces las otras cantidades son $T_{n}=\sum_{k=1}^{n}\xi_{k}$ y $N\left(t\right)=\sum_{n=1}^{\infty}\indora\left(T_{n}\leq t\right)$, donde $T_{n}\rightarrow\infty$ casi seguramente por la Ley Fuerte de los Grandes Números.
\end{Note}

%___________________________________________________________________________________________
%
\subsection{Teorema Principal de Renovaci\'on}
%___________________________________________________________________________________________
%

\begin{Note} Una funci\'on $h:\rea_{+}\rightarrow\rea$ es Directamente Riemann Integrable en los siguientes casos:
\begin{itemize}
\item[a)] $h\left(t\right)\geq0$ es decreciente y Riemann Integrable.
\item[b)] $h$ es continua excepto posiblemente en un conjunto de Lebesgue de medida 0, y $|h\left(t\right)|\leq b\left(t\right)$, donde $b$ es DRI.
\end{itemize}
\end{Note}

\begin{Teo}[Teorema Principal de Renovaci\'on]
Si $F$ es no aritm\'etica y $h\left(t\right)$ es Directamente Riemann Integrable (DRI), entonces

\begin{eqnarray*}
lim_{t\rightarrow\infty}U\star h=\frac{1}{\mu}\int_{\rea_{+}}h\left(s\right)ds.
\end{eqnarray*}
\end{Teo}

\begin{Prop}
Cualquier funci\'on $H\left(t\right)$ acotada en intervalos finitos y que es 0 para $t<0$ puede expresarse como
\begin{eqnarray*}
H\left(t\right)=U\star h\left(t\right)\textrm{,  donde }h\left(t\right)=H\left(t\right)-F\star H\left(t\right)
\end{eqnarray*}
\end{Prop}

\begin{Def}
Un proceso estoc\'astico $X\left(t\right)$ es crudamente regenerativo en un tiempo aleatorio positivo $T$ si
\begin{eqnarray*}
\esp\left[X\left(T+t\right)|T\right]=\esp\left[X\left(t\right)\right]\textrm{, para }t\geq0,\end{eqnarray*}
y con las esperanzas anteriores finitas.
\end{Def}

\begin{Prop}
Sup\'ongase que $X\left(t\right)$ es un proceso crudamente regenerativo en $T$, que tiene distribuci\'on $F$. Si $\esp\left[X\left(t\right)\right]$ es acotado en intervalos finitos, entonces
\begin{eqnarray*}
\esp\left[X\left(t\right)\right]=U\star h\left(t\right)\textrm{,  donde }h\left(t\right)=\esp\left[X\left(t\right)\indora\left(T>t\right)\right].
\end{eqnarray*}
\end{Prop}

\begin{Teo}[Regeneraci\'on Cruda]
Sup\'ongase que $X\left(t\right)$ es un proceso con valores positivo crudamente regenerativo en $T$, y def\'inase $M=\sup\left\{|X\left(t\right)|:t\leq T\right\}$. Si $T$ es no aritm\'etico y $M$ y $MT$ tienen media finita, entonces
\begin{eqnarray*}
lim_{t\rightarrow\infty}\esp\left[X\left(t\right)\right]=\frac{1}{\mu}\int_{\rea_{+}}h\left(s\right)ds,
\end{eqnarray*}
donde $h\left(t\right)=\esp\left[X\left(t\right)\indora\left(T>t\right)\right]$.
\end{Teo}



%___________________________________________________________________________________________
%
\subsection{Funci\'on de Renovaci\'on}
%___________________________________________________________________________________________
%


\begin{Def}
Sea $h\left(t\right)$ funci\'on de valores reales en $\rea$ acotada en intervalos finitos e igual a cero para $t<0$ La ecuaci\'on de renovaci\'on para $h\left(t\right)$ y la distribuci\'on $F$ es

\begin{eqnarray}\label{Ec.Renovacion}
H\left(t\right)=h\left(t\right)+\int_{\left[0,t\right]}H\left(t-s\right)dF\left(s\right)\textrm{,    }t\geq0,
\end{eqnarray}
donde $H\left(t\right)$ es una funci\'on de valores reales. Esto es $H=h+F\star H$. Decimos que $H\left(t\right)$ es soluci\'on de esta ecuaci\'on si satisface la ecuaci\'on, y es acotada en intervalos finitos e iguales a cero para $t<0$.
\end{Def}

\begin{Prop}
La funci\'on $U\star h\left(t\right)$ es la \'unica soluci\'on de la ecuaci\'on de renovaci\'on (\ref{Ec.Renovacion}).
\end{Prop}

\begin{Teo}[Teorema Renovaci\'on Elemental]
\begin{eqnarray*}
t^{-1}U\left(t\right)\rightarrow 1/\mu\textrm{,    cuando }t\rightarrow\infty.
\end{eqnarray*}
\end{Teo}

%___________________________________________________________________________________________
%
\subsection{Propiedades de los Procesos de Renovaci\'on}
%___________________________________________________________________________________________
%

Los tiempos $T_{n}$ est\'an relacionados con los conteos de $N\left(t\right)$ por

\begin{eqnarray*}
\left\{N\left(t\right)\geq n\right\}&=&\left\{T_{n}\leq t\right\}\\
T_{N\left(t\right)}\leq &t&<T_{N\left(t\right)+1},
\end{eqnarray*}

adem\'as $N\left(T_{n}\right)=n$, y 

\begin{eqnarray*}
N\left(t\right)=\max\left\{n:T_{n}\leq t\right\}=\min\left\{n:T_{n+1}>t\right\}
\end{eqnarray*}

Por propiedades de la convoluci\'on se sabe que

\begin{eqnarray*}
P\left\{T_{n}\leq t\right\}=F^{n\star}\left(t\right)
\end{eqnarray*}
que es la $n$-\'esima convoluci\'on de $F$. Entonces 

\begin{eqnarray*}
\left\{N\left(t\right)\geq n\right\}&=&\left\{T_{n}\leq t\right\}\\
P\left\{N\left(t\right)\leq n\right\}&=&1-F^{\left(n+1\right)\star}\left(t\right)
\end{eqnarray*}

Adem\'as usando el hecho de que $\esp\left[N\left(t\right)\right]=\sum_{n=1}^{\infty}P\left\{N\left(t\right)\geq n\right\}$
se tiene que

\begin{eqnarray*}
\esp\left[N\left(t\right)\right]=\sum_{n=1}^{\infty}F^{n\star}\left(t\right)
\end{eqnarray*}

\begin{Prop}
Para cada $t\geq0$, la funci\'on generadora de momentos $\esp\left[e^{\alpha N\left(t\right)}\right]$ existe para alguna $\alpha$ en una vecindad del 0, y de aqu\'i que $\esp\left[N\left(t\right)^{m}\right]<\infty$, para $m\geq1$.
\end{Prop}


\begin{Note}
Si el primer tiempo de renovaci\'on $\xi_{1}$ no tiene la misma distribuci\'on que el resto de las $\xi_{n}$, para $n\geq2$, a $N\left(t\right)$ se le llama Proceso de Renovaci\'on retardado, donde si $\xi$ tiene distribuci\'on $G$, entonces el tiempo $T_{n}$ de la $n$-\'esima renovaci\'on tiene distribuci\'on $G\star F^{\left(n-1\right)\star}\left(t\right)$
\end{Note}


\begin{Teo}
Para una constante $\mu\leq\infty$ ( o variable aleatoria), las siguientes expresiones son equivalentes:

\begin{eqnarray}
lim_{n\rightarrow\infty}n^{-1}T_{n}&=&\mu,\textrm{ c.s.}\\
lim_{t\rightarrow\infty}t^{-1}N\left(t\right)&=&1/\mu,\textrm{ c.s.}
\end{eqnarray}
\end{Teo}


Es decir, $T_{n}$ satisface la Ley Fuerte de los Grandes N\'umeros s\'i y s\'olo s\'i $N\left/t\right)$ la cumple.


\begin{Coro}[Ley Fuerte de los Grandes N\'umeros para Procesos de Renovaci\'on]
Si $N\left(t\right)$ es un proceso de renovaci\'on cuyos tiempos de inter-renovaci\'on tienen media $\mu\leq\infty$, entonces
\begin{eqnarray}
t^{-1}N\left(t\right)\rightarrow 1/\mu,\textrm{ c.s. cuando }t\rightarrow\infty.
\end{eqnarray}

\end{Coro}


Considerar el proceso estoc\'astico de valores reales $\left\{Z\left(t\right):t\geq0\right\}$ en el mismo espacio de probabilidad que $N\left(t\right)$

\begin{Def}
Para el proceso $\left\{Z\left(t\right):t\geq0\right\}$ se define la fluctuaci\'on m\'axima de $Z\left(t\right)$ en el intervalo $\left(T_{n-1},T_{n}\right]$:
\begin{eqnarray*}
M_{n}=\sup_{T_{n-1}<t\leq T_{n}}|Z\left(t\right)-Z\left(T_{n-1}\right)|
\end{eqnarray*}
\end{Def}

\begin{Teo}
Sup\'ongase que $n^{-1}T_{n}\rightarrow\mu$ c.s. cuando $n\rightarrow\infty$, donde $\mu\leq\infty$ es una constante o variable aleatoria. Sea $a$ una constante o variable aleatoria que puede ser infinita cuando $\mu$ es finita, y considere las expresiones l\'imite:
\begin{eqnarray}
lim_{n\rightarrow\infty}n^{-1}Z\left(T_{n}\right)&=&a,\textrm{ c.s.}\\
lim_{t\rightarrow\infty}t^{-1}Z\left(t\right)&=&a/\mu,\textrm{ c.s.}
\end{eqnarray}
La segunda expresi\'on implica la primera. Conversamente, la primera implica la segunda si el proceso $Z\left(t\right)$ es creciente, o si $lim_{n\rightarrow\infty}n^{-1}M_{n}=0$ c.s.
\end{Teo}

\begin{Coro}
Si $N\left(t\right)$ es un proceso de renovaci\'on, y $\left(Z\left(T_{n}\right)-Z\left(T_{n-1}\right),M_{n}\right)$, para $n\geq1$, son variables aleatorias independientes e id\'enticamente distribuidas con media finita, entonces,
\begin{eqnarray}
lim_{t\rightarrow\infty}t^{-1}Z\left(t\right)\rightarrow\frac{\esp\left[Z\left(T_{1}\right)-Z\left(T_{0}\right)\right]}{\esp\left[T_{1}\right]},\textrm{ c.s. cuando  }t\rightarrow\infty.
\end{eqnarray}
\end{Coro}

%___________________________________________________________________________________________
%
\subsection{Funci\'on de Renovaci\'on}
%___________________________________________________________________________________________
%


Sup\'ongase que $N\left(t\right)$ es un proceso de renovaci\'on con distribuci\'on $F$ con media finita $\mu$.

\begin{Def}
La funci\'on de renovaci\'on asociada con la distribuci\'on $F$, del proceso $N\left(t\right)$, es
\begin{eqnarray*}
U\left(t\right)=\sum_{n=1}^{\infty}F^{n\star}\left(t\right),\textrm{   }t\geq0,
\end{eqnarray*}
donde $F^{0\star}\left(t\right)=\indora\left(t\geq0\right)$.
\end{Def}


\begin{Prop}
Sup\'ongase que la distribuci\'on de inter-renovaci\'on $F$ tiene densidad $f$. Entonces $U\left(t\right)$ tambi\'en tiene densidad, para $t>0$, y es $U^{'}\left(t\right)=\sum_{n=0}^{\infty}f^{n\star}\left(t\right)$. Adem\'as
\begin{eqnarray*}
\prob\left\{N\left(t\right)>N\left(t-\right)\right\}=0\textrm{,   }t\geq0.
\end{eqnarray*}
\end{Prop}

\begin{Def}
La Transformada de Laplace-Stieljes de $F$ est\'a dada por

\begin{eqnarray*}
\hat{F}\left(\alpha\right)=\int_{\rea_{+}}e^{-\alpha t}dF\left(t\right)\textrm{,  }\alpha\geq0.
\end{eqnarray*}
\end{Def}

Entonces

\begin{eqnarray*}
\hat{U}\left(\alpha\right)=\sum_{n=0}^{\infty}\hat{F^{n\star}}\left(\alpha\right)=\sum_{n=0}^{\infty}\hat{F}\left(\alpha\right)^{n}=\frac{1}{1-\hat{F}\left(\alpha\right)}.
\end{eqnarray*}


\begin{Prop}
La Transformada de Laplace $\hat{U}\left(\alpha\right)$ y $\hat{F}\left(\alpha\right)$ determina una a la otra de manera \'unica por la relaci\'on $\hat{U}\left(\alpha\right)=\frac{1}{1-\hat{F}\left(\alpha\right)}$.
\end{Prop}


\begin{Note}
Un proceso de renovaci\'on $N\left(t\right)$ cuyos tiempos de inter-renovaci\'on tienen media finita, es un proceso Poisson con tasa $\lambda$ si y s\'olo s\'i $\esp\left[U\left(t\right)\right]=\lambda t$, para $t\geq0$.
\end{Note}


\begin{Teo}
Sea $N\left(t\right)$ un proceso puntual simple con puntos de localizaci\'on $T_{n}$ tal que $\eta\left(t\right)=\esp\left[N\left(\right)\right]$ es finita para cada $t$. Entonces para cualquier funci\'on $f:\rea_{+}\rightarrow\rea$,
\begin{eqnarray*}
\esp\left[\sum_{n=1}^{N\left(\right)}f\left(T_{n}\right)\right]=\int_{\left(0,t\right]}f\left(s\right)d\eta\left(s\right)\textrm{,  }t\geq0,
\end{eqnarray*}
suponiendo que la integral exista. Adem\'as si $X_{1},X_{2},\ldots$ son variables aleatorias definidas en el mismo espacio de probabilidad que el proceso $N\left(t\right)$ tal que $\esp\left[X_{n}|T_{n}=s\right]=f\left(s\right)$, independiente de $n$. Entonces
\begin{eqnarray*}
\esp\left[\sum_{n=1}^{N\left(t\right)}X_{n}\right]=\int_{\left(0,t\right]}f\left(s\right)d\eta\left(s\right)\textrm{,  }t\geq0,
\end{eqnarray*} 
suponiendo que la integral exista. 
\end{Teo}

\begin{Coro}[Identidad de Wald para Renovaciones]
Para el proceso de renovaci\'on $N\left(t\right)$,
\begin{eqnarray*}
\esp\left[T_{N\left(t\right)+1}\right]=\mu\esp\left[N\left(t\right)+1\right]\textrm{,  }t\geq0,
\end{eqnarray*}  
\end{Coro}

%______________________________________________________________________
\subsection{Procesos de Renovaci\'on}
%______________________________________________________________________

\begin{Def}\label{Def.Tn}
Sean $0\leq T_{1}\leq T_{2}\leq \ldots$ son tiempos aleatorios infinitos en los cuales ocurren ciertos eventos. El n\'umero de tiempos $T_{n}$ en el intervalo $\left[0,t\right)$ es

\begin{eqnarray}
N\left(t\right)=\sum_{n=1}^{\infty}\indora\left(T_{n}\leq t\right),
\end{eqnarray}
para $t\geq0$.
\end{Def}

Si se consideran los puntos $T_{n}$ como elementos de $\rea_{+}$, y $N\left(t\right)$ es el n\'umero de puntos en $\rea$. El proceso denotado por $\left\{N\left(t\right):t\geq0\right\}$, denotado por $N\left(t\right)$, es un proceso puntual en $\rea_{+}$. Los $T_{n}$ son los tiempos de ocurrencia, el proceso puntual $N\left(t\right)$ es simple si su n\'umero de ocurrencias son distintas: $0<T_{1}<T_{2}<\ldots$ casi seguramente.

\begin{Def}
Un proceso puntual $N\left(t\right)$ es un proceso de renovaci\'on si los tiempos de interocurrencia $\xi_{n}=T_{n}-T_{n-1}$, para $n\geq1$, son independientes e identicamente distribuidos con distribuci\'on $F$, donde $F\left(0\right)=0$ y $T_{0}=0$. Los $T_{n}$ son llamados tiempos de renovaci\'on, referente a la independencia o renovaci\'on de la informaci\'on estoc\'astica en estos tiempos. Los $\xi_{n}$ son los tiempos de inter-renovaci\'on, y $N\left(t\right)$ es el n\'umero de renovaciones en el intervalo $\left[0,t\right)$
\end{Def}


\begin{Note}
Para definir un proceso de renovaci\'on para cualquier contexto, solamente hay que especificar una distribuci\'on $F$, con $F\left(0\right)=0$, para los tiempos de inter-renovaci\'on. La funci\'on $F$ en turno degune las otra variables aleatorias. De manera formal, existe un espacio de probabilidad y una sucesi\'on de variables aleatorias $\xi_{1},\xi_{2},\ldots$ definidas en este con distribuci\'on $F$. Entonces las otras cantidades son $T_{n}=\sum_{k=1}^{n}\xi_{k}$ y $N\left(t\right)=\sum_{n=1}^{\infty}\indora\left(T_{n}\leq t\right)$, donde $T_{n}\rightarrow\infty$ casi seguramente por la Ley Fuerte de los Grandes Números.
\end{Note}
%_____________________________________________________
\subsection{Puntos de Renovaci\'on}
%_____________________________________________________

Para cada cola $Q_{i}$ se tienen los procesos de arribo a la cola, para estas, los tiempos de arribo est\'an dados por $$\left\{T_{1}^{i},T_{2}^{i},\ldots,T_{k}^{i},\ldots\right\},$$ entonces, consideremos solamente los primeros tiempos de arribo a cada una de las colas, es decir, $$\left\{T_{1}^{1},T_{1}^{2},T_{1}^{3},T_{1}^{4}\right\},$$ se sabe que cada uno de estos tiempos se distribuye de manera exponencial con par\'ametro $1/mu_{i}$. Adem\'as se sabe que para $$T^{*}=\min\left\{T_{1}^{1},T_{1}^{2},T_{1}^{3},T_{1}^{4}\right\},$$ $T^{*}$ se distribuye de manera exponencial con par\'ametro $$\mu^{*}=\sum_{i=1}^{4}\mu_{i}.$$ Ahora, dado que 
\begin{center}
\begin{tabular}{lcl}
$\tilde{r}=r_{1}+r_{2}$ & y &$\hat{r}=r_{3}+r_{4}.$
\end{tabular}
\end{center}


Supongamos que $$\tilde{r},\hat{r}<\mu^{*},$$ entonces si tomamos $$r^{*}=\min\left\{\tilde{r},\hat{r}\right\},$$ se tiene que para  $$t^{*}\in\left(0,r^{*}\right)$$ se cumple que 
\begin{center}
\begin{tabular}{lcl}
$\tau_{1}\left(1\right)=0$ & y por tanto & $\overline{\tau}_{1}=0,$
\end{tabular}
\end{center}
entonces para la segunda cola en este primer ciclo se cumple que $$\tau_{2}=\overline{\tau}_{1}+r_{1}=r_{1}<\mu^{*},$$ y por tanto se tiene que  $$\overline{\tau}_{2}=\tau_{2}.$$ Por lo tanto, nuevamente para la primer cola en el segundo ciclo $$\tau_{1}\left(2\right)=\tau_{2}\left(1\right)+r_{2}=\tilde{r}<\mu^{*}.$$ An\'alogamente para el segundo sistema se tiene que ambas colas est\'an vac\'ias, es decir, existe un valor $t^{*}$ tal que en el intervalo $\left(0,t^{*}\right)$ no ha llegado ning\'un usuario, es decir, $$L_{i}\left(t^{*}\right)=0$$ para $i=1,2,3,4$.

\subsection{Resultados para Procesos de Salida}

En \cite{Sigman2} prueban que para la existencia de un una sucesi\'on infinita no decreciente de tiempos de regeneraci\'on $\tau_{1}\leq\tau_{2}\leq\cdots$ en los cuales el proceso se regenera, basta un tiempo de regeneraci\'on $R_{1}$, donde $R_{j}=\tau_{j}-\tau_{j-1}$. Para tal efecto se requiere la existencia de un espacio de probabilidad $\left(\Omega,\mathcal{F},\prob\right)$, y proceso estoc\'astico $\textit{X}=\left\{X\left(t\right):t\geq0\right\}$ con espacio de estados $\left(S,\mathcal{R}\right)$, con $\mathcal{R}$ $\sigma$-\'algebra.

\begin{Prop}
Si existe una variable aleatoria no negativa $R_{1}$ tal que $\theta_{R\footnotesize{1}}X=_{D}X$, entonces $\left(\Omega,\mathcal{F},\prob\right)$ puede extenderse para soportar una sucesi\'on estacionaria de variables aleatorias $R=\left\{R_{k}:k\geq1\right\}$, tal que para $k\geq1$,
\begin{eqnarray*}
\theta_{k}\left(X,R\right)=_{D}\left(X,R\right).
\end{eqnarray*}

Adem\'as, para $k\geq1$, $\theta_{k}R$ es condicionalmente independiente de $\left(X,R_{1},\ldots,R_{k}\right)$, dado $\theta_{\tau k}X$.

\end{Prop}


\begin{itemize}
\item Doob en 1953 demostr\'o que el estado estacionario de un proceso de partida en un sistema de espera $M/G/\infty$, es Poisson con la misma tasa que el proceso de arribos.

\item Burke en 1968, fue el primero en demostrar que el estado estacionario de un proceso de salida de una cola $M/M/s$ es un proceso Poisson.

\item Disney en 1973 obtuvo el siguiente resultado:

\begin{Teo}
Para el sistema de espera $M/G/1/L$ con disciplina FIFO, el proceso $\textbf{I}$ es un proceso de renovaci\'on si y s\'olo si el proceso denominado longitud de la cola es estacionario y se cumple cualquiera de los siguientes casos:

\begin{itemize}
\item[a)] Los tiempos de servicio son identicamente cero;
\item[b)] $L=0$, para cualquier proceso de servicio $S$;
\item[c)] $L=1$ y $G=D$;
\item[d)] $L=\infty$ y $G=M$.
\end{itemize}
En estos casos, respectivamente, las distribuciones de interpartida $P\left\{T_{n+1}-T_{n}\leq t\right\}$ son


\begin{itemize}
\item[a)] $1-e^{-\lambda t}$, $t\geq0$;
\item[b)] $1-e^{-\lambda t}*F\left(t\right)$, $t\geq0$;
\item[c)] $1-e^{-\lambda t}*\indora_{d}\left(t\right)$, $t\geq0$;
\item[d)] $1-e^{-\lambda t}*F\left(t\right)$, $t\geq0$.
\end{itemize}
\end{Teo}


\item Finch (1959) mostr\'o que para los sistemas $M/G/1/L$, con $1\leq L\leq \infty$ con distribuciones de servicio dos veces diferenciable, solamente el sistema $M/M/1/\infty$ tiene proceso de salida de renovaci\'on estacionario.

\item King (1971) demostro que un sistema de colas estacionario $M/G/1/1$ tiene sus tiempos de interpartida sucesivas $D_{n}$ y $D_{n+1}$ son independientes, si y s\'olo si, $G=D$, en cuyo caso le proceso de salida es de renovaci\'on.

\item Disney (1973) demostr\'o que el \'unico sistema estacionario $M/G/1/L$, que tiene proceso de salida de renovaci\'on  son los sistemas $M/M/1$ y $M/D/1/1$.



\item El siguiente resultado es de Disney y Koning (1985)
\begin{Teo}
En un sistema de espera $M/G/s$, el estado estacionario del proceso de salida es un proceso Poisson para cualquier distribuci\'on de los tiempos de servicio si el sistema tiene cualquiera de las siguientes cuatro propiedades.

\begin{itemize}
\item[a)] $s=\infty$
\item[b)] La disciplina de servicio es de procesador compartido.
\item[c)] La disciplina de servicio es LCFS y preemptive resume, esto se cumple para $L<\infty$
\item[d)] $G=M$.
\end{itemize}

\end{Teo}

\item El siguiente resultado es de Alamatsaz (1983)

\begin{Teo}
En cualquier sistema de colas $GI/G/1/L$ con $1\leq L<\infty$ y distribuci\'on de interarribos $A$ y distribuci\'on de los tiempos de servicio $B$, tal que $A\left(0\right)=0$, $A\left(t\right)\left(1-B\left(t\right)\right)>0$ para alguna $t>0$ y $B\left(t\right)$ para toda $t>0$, es imposible que el proceso de salida estacionario sea de renovaci\'on.
\end{Teo}

\end{itemize}

Estos resultados aparecen en Daley (1968) \cite{Daley68} para $\left\{T_{n}\right\}$ intervalos de inter-arribo, $\left\{D_{n}\right\}$ intervalos de inter-salida y $\left\{S_{n}\right\}$ tiempos de servicio.

\begin{itemize}
\item Si el proceso $\left\{T_{n}\right\}$ es Poisson, el proceso $\left\{D_{n}\right\}$ es no correlacionado si y s\'olo si es un proceso Poisso, lo cual ocurre si y s\'olo si $\left\{S_{n}\right\}$ son exponenciales negativas.

\item Si $\left\{S_{n}\right\}$ son exponenciales negativas, $\left\{D_{n}\right\}$ es un proceso de renovaci\'on  si y s\'olo si es un proceso Poisson, lo cual ocurre si y s\'olo si $\left\{T_{n}\right\}$ es un proceso Poisson.

\item $\esp\left(D_{n}\right)=\esp\left(T_{n}\right)$.

\item Para un sistema de visitas $GI/M/1$ se tiene el siguiente teorema:

\begin{Teo}
En un sistema estacionario $GI/M/1$ los intervalos de interpartida tienen
\begin{eqnarray*}
\esp\left(e^{-\theta D_{n}}\right)&=&\mu\left(\mu+\theta\right)^{-1}\left[\delta\theta
-\mu\left(1-\delta\right)\alpha\left(\theta\right)\right]
\left[\theta-\mu\left(1-\delta\right)^{-1}\right]\\
\alpha\left(\theta\right)&=&\esp\left[e^{-\theta T_{0}}\right]\\
var\left(D_{n}\right)&=&var\left(T_{0}\right)-\left(\tau^{-1}-\delta^{-1}\right)
2\delta\left(\esp\left(S_{0}\right)\right)^{2}\left(1-\delta\right)^{-1}.
\end{eqnarray*}
\end{Teo}



\begin{Teo}
El proceso de salida de un sistema de colas estacionario $GI/M/1$ es un proceso de renovaci\'on si y s\'olo si el proceso de entrada es un proceso Poisson, en cuyo caso el proceso de salida es un proceso Poisson.
\end{Teo}


\begin{Teo}
Los intervalos de interpartida $\left\{D_{n}\right\}$ de un sistema $M/G/1$ estacionario son no correlacionados si y s\'olo si la distribuci\'on de los tiempos de servicio es exponencial negativa, es decir, el sistema es de tipo  $M/M/1$.

\end{Teo}



\end{itemize}


%________________________________________________________________________
\subsection{Procesos Regenerativos}
%________________________________________________________________________

Para $\left\{X\left(t\right):t\geq0\right\}$ Proceso Estoc\'astico a tiempo continuo con estado de espacios $S$, que es un espacio m\'etrico, con trayectorias continuas por la derecha y con l\'imites por la izquierda c.s. Sea $N\left(t\right)$ un proceso de renovaci\'on en $\rea_{+}$ definido en el mismo espacio de probabilidad que $X\left(t\right)$, con tiempos de renovaci\'on $T$ y tiempos de inter-renovaci\'on $\xi_{n}=T_{n}-T_{n-1}$, con misma distribuci\'on $F$ de media finita $\mu$.



\begin{Def}
Para el proceso $\left\{\left(N\left(t\right),X\left(t\right)\right):t\geq0\right\}$, sus trayectoria muestrales en el intervalo de tiempo $\left[T_{n-1},T_{n}\right)$ est\'an descritas por
\begin{eqnarray*}
\zeta_{n}=\left(\xi_{n},\left\{X\left(T_{n-1}+t\right):0\leq t<\xi_{n}\right\}\right)
\end{eqnarray*}
Este $\zeta_{n}$ es el $n$-\'esimo segmento del proceso. El proceso es regenerativo sobre los tiempos $T_{n}$ si sus segmentos $\zeta_{n}$ son independientes e id\'enticamennte distribuidos.
\end{Def}


\begin{Obs}
Si $\tilde{X}\left(t\right)$ con espacio de estados $\tilde{S}$ es regenerativo sobre $T_{n}$, entonces $X\left(t\right)=f\left(\tilde{X}\left(t\right)\right)$ tambi\'en es regenerativo sobre $T_{n}$, para cualquier funci\'on $f:\tilde{S}\rightarrow S$.
\end{Obs}

\begin{Obs}
Los procesos regenerativos son crudamente regenerativos, pero no al rev\'es.
\end{Obs}

\begin{Def}[Definici\'on Cl\'asica]
Un proceso estoc\'astico $X=\left\{X\left(t\right):t\geq0\right\}$ es llamado regenerativo is existe una variable aleatoria $R_{1}>0$ tal que
\begin{itemize}
\item[i)] $\left\{X\left(t+R_{1}\right):t\geq0\right\}$ es independiente de $\left\{\left\{X\left(t\right):t<R_{1}\right\},\right\}$
\item[ii)] $\left\{X\left(t+R_{1}\right):t\geq0\right\}$ es estoc\'asticamente equivalente a $\left\{X\left(t\right):t>0\right\}$
\end{itemize}

Llamamos a $R_{1}$ tiempo de regeneraci\'on, y decimos que $X$ se regenera en este punto.
\end{Def}

$\left\{X\left(t+R_{1}\right)\right\}$ es regenerativo con tiempo de regeneraci\'on $R_{2}$, independiente de $R_{1}$ pero con la misma distribuci\'on que $R_{1}$. Procediendo de esta manera se obtiene una secuencia de variables aleatorias independientes e id\'enticamente distribuidas $\left\{R_{n}\right\}$ llamados longitudes de ciclo. Si definimos a $Z_{k}\equiv R_{1}+R_{2}+\cdots+R_{k}$, se tiene un proceso de renovaci\'on llamado proceso de renovaci\'on encajado para $X$.

\begin{Note}
Un proceso regenerativo con media de la longitud de ciclo finita es llamado positivo recurrente.
\end{Note}


\begin{Def}
Para $x$ fijo y para cada $t\geq0$, sea $I_{x}\left(t\right)=1$ si $X\left(t\right)\leq x$,  $I_{x}\left(t\right)=0$ en caso contrario, y def\'inanse los tiempos promedio
\begin{eqnarray*}
\overline{X}&=&lim_{t\rightarrow\infty}\frac{1}{t}\int_{0}^{\infty}X\left(u\right)du\\
\prob\left(X_{\infty}\leq x\right)&=&lim_{t\rightarrow\infty}\frac{1}{t}\int_{0}^{\infty}I_{x}\left(u\right)du,
\end{eqnarray*}
cuando estos l\'imites existan.
\end{Def}

Como consecuencia del teorema de Renovaci\'on-Recompensa, se tiene que el primer l\'imite  existe y es igual a la constante
\begin{eqnarray*}
\overline{X}&=&\frac{\esp\left[\int_{0}^{R_{1}}X\left(t\right)dt\right]}{\esp\left[R_{1}\right]},
\end{eqnarray*}
suponiendo que ambas esperanzas son finitas.

\begin{Note}
\begin{itemize}
\item[a)] Si el proceso regenerativo $X$ es positivo recurrente y tiene trayectorias muestrales no negativas, entonces la ecuaci\'on anterior es v\'alida.
\item[b)] Si $X$ es positivo recurrente regenerativo, podemos construir una \'unica versi\'on estacionaria de este proceso, $X_{e}=\left\{X_{e}\left(t\right)\right\}$, donde $X_{e}$ es un proceso estoc\'astico regenerativo y estrictamente estacionario, con distribuci\'on marginal distribuida como $X_{\infty}$
\end{itemize}
\end{Note}

\subsection{Renewal and Regenerative Processes: Serfozo\cite{Serfozo}}
\begin{Def}\label{Def.Tn}
Sean $0\leq T_{1}\leq T_{2}\leq \ldots$ son tiempos aleatorios infinitos en los cuales ocurren ciertos eventos. El n\'umero de tiempos $T_{n}$ en el intervalo $\left[0,t\right)$ es

\begin{eqnarray}
N\left(t\right)=\sum_{n=1}^{\infty}\indora\left(T_{n}\leq t\right),
\end{eqnarray}
para $t\geq0$.
\end{Def}

Si se consideran los puntos $T_{n}$ como elementos de $\rea_{+}$, y $N\left(t\right)$ es el n\'umero de puntos en $\rea$. El proceso denotado por $\left\{N\left(t\right):t\geq0\right\}$, denotado por $N\left(t\right)$, es un proceso puntual en $\rea_{+}$. Los $T_{n}$ son los tiempos de ocurrencia, el proceso puntual $N\left(t\right)$ es simple si su n\'umero de ocurrencias son distintas: $0<T_{1}<T_{2}<\ldots$ casi seguramente.

\begin{Def}
Un proceso puntual $N\left(t\right)$ es un proceso de renovaci\'on si los tiempos de interocurrencia $\xi_{n}=T_{n}-T_{n-1}$, para $n\geq1$, son independientes e identicamente distribuidos con distribuci\'on $F$, donde $F\left(0\right)=0$ y $T_{0}=0$. Los $T_{n}$ son llamados tiempos de renovaci\'on, referente a la independencia o renovaci\'on de la informaci\'on estoc\'astica en estos tiempos. Los $\xi_{n}$ son los tiempos de inter-renovaci\'on, y $N\left(t\right)$ es el n\'umero de renovaciones en el intervalo $\left[0,t\right)$
\end{Def}


\begin{Note}
Para definir un proceso de renovaci\'on para cualquier contexto, solamente hay que especificar una distribuci\'on $F$, con $F\left(0\right)=0$, para los tiempos de inter-renovaci\'on. La funci\'on $F$ en turno degune las otra variables aleatorias. De manera formal, existe un espacio de probabilidad y una sucesi\'on de variables aleatorias $\xi_{1},\xi_{2},\ldots$ definidas en este con distribuci\'on $F$. Entonces las otras cantidades son $T_{n}=\sum_{k=1}^{n}\xi_{k}$ y $N\left(t\right)=\sum_{n=1}^{\infty}\indora\left(T_{n}\leq t\right)$, donde $T_{n}\rightarrow\infty$ casi seguramente por la Ley Fuerte de los Grandes N\'umeros.
\end{Note}







Los tiempos $T_{n}$ est\'an relacionados con los conteos de $N\left(t\right)$ por

\begin{eqnarray*}
\left\{N\left(t\right)\geq n\right\}&=&\left\{T_{n}\leq t\right\}\\
T_{N\left(t\right)}\leq &t&<T_{N\left(t\right)+1},
\end{eqnarray*}

adem\'as $N\left(T_{n}\right)=n$, y 

\begin{eqnarray*}
N\left(t\right)=\max\left\{n:T_{n}\leq t\right\}=\min\left\{n:T_{n+1}>t\right\}
\end{eqnarray*}

Por propiedades de la convoluci\'on se sabe que

\begin{eqnarray*}
P\left\{T_{n}\leq t\right\}=F^{n\star}\left(t\right)
\end{eqnarray*}
que es la $n$-\'esima convoluci\'on de $F$. Entonces 

\begin{eqnarray*}
\left\{N\left(t\right)\geq n\right\}&=&\left\{T_{n}\leq t\right\}\\
P\left\{N\left(t\right)\leq n\right\}&=&1-F^{\left(n+1\right)\star}\left(t\right)
\end{eqnarray*}

Adem\'as usando el hecho de que $\esp\left[N\left(t\right)\right]=\sum_{n=1}^{\infty}P\left\{N\left(t\right)\geq n\right\}$
se tiene que

\begin{eqnarray*}
\esp\left[N\left(t\right)\right]=\sum_{n=1}^{\infty}F^{n\star}\left(t\right)
\end{eqnarray*}

\begin{Prop}
Para cada $t\geq0$, la funci\'on generadora de momentos $\esp\left[e^{\alpha N\left(t\right)}\right]$ existe para alguna $\alpha$ en una vecindad del 0, y de aqu\'i que $\esp\left[N\left(t\right)^{m}\right]<\infty$, para $m\geq1$.
\end{Prop}


\begin{Note}
Si el primer tiempo de renovaci\'on $\xi_{1}$ no tiene la misma distribuci\'on que el resto de las $\xi_{n}$, para $n\geq2$, a $N\left(t\right)$ se le llama Proceso de Renovaci\'on retardado, donde si $\xi$ tiene distribuci\'on $G$, entonces el tiempo $T_{n}$ de la $n$-\'esima renovaci\'on tiene distribuci\'on $G\star F^{\left(n-1\right)\star}\left(t\right)$
\end{Note}


\begin{Teo}
Para una constante $\mu\leq\infty$ ( o variable aleatoria), las siguientes expresiones son equivalentes:

\begin{eqnarray}
lim_{n\rightarrow\infty}n^{-1}T_{n}&=&\mu,\textrm{ c.s.}\\
lim_{t\rightarrow\infty}t^{-1}N\left(t\right)&=&1/\mu,\textrm{ c.s.}
\end{eqnarray}
\end{Teo}


Es decir, $T_{n}$ satisface la Ley Fuerte de los Grandes N\'umeros s\'i y s\'olo s\'i $N\left/t\right)$ la cumple.


\begin{Coro}[Ley Fuerte de los Grandes N\'umeros para Procesos de Renovaci\'on]
Si $N\left(t\right)$ es un proceso de renovaci\'on cuyos tiempos de inter-renovaci\'on tienen media $\mu\leq\infty$, entonces
\begin{eqnarray}
t^{-1}N\left(t\right)\rightarrow 1/\mu,\textrm{ c.s. cuando }t\rightarrow\infty.
\end{eqnarray}

\end{Coro}


Considerar el proceso estoc\'astico de valores reales $\left\{Z\left(t\right):t\geq0\right\}$ en el mismo espacio de probabilidad que $N\left(t\right)$

\begin{Def}
Para el proceso $\left\{Z\left(t\right):t\geq0\right\}$ se define la fluctuaci\'on m\'axima de $Z\left(t\right)$ en el intervalo $\left(T_{n-1},T_{n}\right]$:
\begin{eqnarray*}
M_{n}=\sup_{T_{n-1}<t\leq T_{n}}|Z\left(t\right)-Z\left(T_{n-1}\right)|
\end{eqnarray*}
\end{Def}

\begin{Teo}
Sup\'ongase que $n^{-1}T_{n}\rightarrow\mu$ c.s. cuando $n\rightarrow\infty$, donde $\mu\leq\infty$ es una constante o variable aleatoria. Sea $a$ una constante o variable aleatoria que puede ser infinita cuando $\mu$ es finita, y considere las expresiones l\'imite:
\begin{eqnarray}
lim_{n\rightarrow\infty}n^{-1}Z\left(T_{n}\right)&=&a,\textrm{ c.s.}\\
lim_{t\rightarrow\infty}t^{-1}Z\left(t\right)&=&a/\mu,\textrm{ c.s.}
\end{eqnarray}
La segunda expresi\'on implica la primera. Conversamente, la primera implica la segunda si el proceso $Z\left(t\right)$ es creciente, o si $lim_{n\rightarrow\infty}n^{-1}M_{n}=0$ c.s.
\end{Teo}

\begin{Coro}
Si $N\left(t\right)$ es un proceso de renovaci\'on, y $\left(Z\left(T_{n}\right)-Z\left(T_{n-1}\right),M_{n}\right)$, para $n\geq1$, son variables aleatorias independientes e id\'enticamente distribuidas con media finita, entonces,
\begin{eqnarray}
lim_{t\rightarrow\infty}t^{-1}Z\left(t\right)\rightarrow\frac{\esp\left[Z\left(T_{1}\right)-Z\left(T_{0}\right)\right]}{\esp\left[T_{1}\right]},\textrm{ c.s. cuando  }t\rightarrow\infty.
\end{eqnarray}
\end{Coro}


Sup\'ongase que $N\left(t\right)$ es un proceso de renovaci\'on con distribuci\'on $F$ con media finita $\mu$.

\begin{Def}
La funci\'on de renovaci\'on asociada con la distribuci\'on $F$, del proceso $N\left(t\right)$, es
\begin{eqnarray*}
U\left(t\right)=\sum_{n=1}^{\infty}F^{n\star}\left(t\right),\textrm{   }t\geq0,
\end{eqnarray*}
donde $F^{0\star}\left(t\right)=\indora\left(t\geq0\right)$.
\end{Def}


\begin{Prop}
Sup\'ongase que la distribuci\'on de inter-renovaci\'on $F$ tiene densidad $f$. Entonces $U\left(t\right)$ tambi\'en tiene densidad, para $t>0$, y es $U^{'}\left(t\right)=\sum_{n=0}^{\infty}f^{n\star}\left(t\right)$. Adem\'as
\begin{eqnarray*}
\prob\left\{N\left(t\right)>N\left(t-\right)\right\}=0\textrm{,   }t\geq0.
\end{eqnarray*}
\end{Prop}

\begin{Def}
La Transformada de Laplace-Stieljes de $F$ est\'a dada por

\begin{eqnarray*}
\hat{F}\left(\alpha\right)=\int_{\rea_{+}}e^{-\alpha t}dF\left(t\right)\textrm{,  }\alpha\geq0.
\end{eqnarray*}
\end{Def}

Entonces

\begin{eqnarray*}
\hat{U}\left(\alpha\right)=\sum_{n=0}^{\infty}\hat{F^{n\star}}\left(\alpha\right)=\sum_{n=0}^{\infty}\hat{F}\left(\alpha\right)^{n}=\frac{1}{1-\hat{F}\left(\alpha\right)}.
\end{eqnarray*}


\begin{Prop}
La Transformada de Laplace $\hat{U}\left(\alpha\right)$ y $\hat{F}\left(\alpha\right)$ determina una a la otra de manera \'unica por la relaci\'on $\hat{U}\left(\alpha\right)=\frac{1}{1-\hat{F}\left(\alpha\right)}$.
\end{Prop}


\begin{Note}
Un proceso de renovaci\'on $N\left(t\right)$ cuyos tiempos de inter-renovaci\'on tienen media finita, es un proceso Poisson con tasa $\lambda$ si y s\'olo s\'i $\esp\left[U\left(t\right)\right]=\lambda t$, para $t\geq0$.
\end{Note}


\begin{Teo}
Sea $N\left(t\right)$ un proceso puntual simple con puntos de localizaci\'on $T_{n}$ tal que $\eta\left(t\right)=\esp\left[N\left(\right)\right]$ es finita para cada $t$. Entonces para cualquier funci\'on $f:\rea_{+}\rightarrow\rea$,
\begin{eqnarray*}
\esp\left[\sum_{n=1}^{N\left(\right)}f\left(T_{n}\right)\right]=\int_{\left(0,t\right]}f\left(s\right)d\eta\left(s\right)\textrm{,  }t\geq0,
\end{eqnarray*}
suponiendo que la integral exista. Adem\'as si $X_{1},X_{2},\ldots$ son variables aleatorias definidas en el mismo espacio de probabilidad que el proceso $N\left(t\right)$ tal que $\esp\left[X_{n}|T_{n}=s\right]=f\left(s\right)$, independiente de $n$. Entonces
\begin{eqnarray*}
\esp\left[\sum_{n=1}^{N\left(t\right)}X_{n}\right]=\int_{\left(0,t\right]}f\left(s\right)d\eta\left(s\right)\textrm{,  }t\geq0,
\end{eqnarray*} 
suponiendo que la integral exista. 
\end{Teo}

\begin{Coro}[Identidad de Wald para Renovaciones]
Para el proceso de renovaci\'on $N\left(t\right)$,
\begin{eqnarray*}
\esp\left[T_{N\left(t\right)+1}\right]=\mu\esp\left[N\left(t\right)+1\right]\textrm{,  }t\geq0,
\end{eqnarray*}  
\end{Coro}


\begin{Def}
Sea $h\left(t\right)$ funci\'on de valores reales en $\rea$ acotada en intervalos finitos e igual a cero para $t<0$ La ecuaci\'on de renovaci\'on para $h\left(t\right)$ y la distribuci\'on $F$ es

\begin{eqnarray}\label{Ec.Renovacion}
H\left(t\right)=h\left(t\right)+\int_{\left[0,t\right]}H\left(t-s\right)dF\left(s\right)\textrm{,    }t\geq0,
\end{eqnarray}
donde $H\left(t\right)$ es una funci\'on de valores reales. Esto es $H=h+F\star H$. Decimos que $H\left(t\right)$ es soluci\'on de esta ecuaci\'on si satisface la ecuaci\'on, y es acotada en intervalos finitos e iguales a cero para $t<0$.
\end{Def}

\begin{Prop}
La funci\'on $U\star h\left(t\right)$ es la \'unica soluci\'on de la ecuaci\'on de renovaci\'on (\ref{Ec.Renovacion}).
\end{Prop}

\begin{Teo}[Teorema Renovaci\'on Elemental]
\begin{eqnarray*}
t^{-1}U\left(t\right)\rightarrow 1/\mu\textrm{,    cuando }t\rightarrow\infty.
\end{eqnarray*}
\end{Teo}



Sup\'ongase que $N\left(t\right)$ es un proceso de renovaci\'on con distribuci\'on $F$ con media finita $\mu$.

\begin{Def}
La funci\'on de renovaci\'on asociada con la distribuci\'on $F$, del proceso $N\left(t\right)$, es
\begin{eqnarray*}
U\left(t\right)=\sum_{n=1}^{\infty}F^{n\star}\left(t\right),\textrm{   }t\geq0,
\end{eqnarray*}
donde $F^{0\star}\left(t\right)=\indora\left(t\geq0\right)$.
\end{Def}


\begin{Prop}
Sup\'ongase que la distribuci\'on de inter-renovaci\'on $F$ tiene densidad $f$. Entonces $U\left(t\right)$ tambi\'en tiene densidad, para $t>0$, y es $U^{'}\left(t\right)=\sum_{n=0}^{\infty}f^{n\star}\left(t\right)$. Adem\'as
\begin{eqnarray*}
\prob\left\{N\left(t\right)>N\left(t-\right)\right\}=0\textrm{,   }t\geq0.
\end{eqnarray*}
\end{Prop}

\begin{Def}
La Transformada de Laplace-Stieljes de $F$ est\'a dada por

\begin{eqnarray*}
\hat{F}\left(\alpha\right)=\int_{\rea_{+}}e^{-\alpha t}dF\left(t\right)\textrm{,  }\alpha\geq0.
\end{eqnarray*}
\end{Def}

Entonces

\begin{eqnarray*}
\hat{U}\left(\alpha\right)=\sum_{n=0}^{\infty}\hat{F^{n\star}}\left(\alpha\right)=\sum_{n=0}^{\infty}\hat{F}\left(\alpha\right)^{n}=\frac{1}{1-\hat{F}\left(\alpha\right)}.
\end{eqnarray*}


\begin{Prop}
La Transformada de Laplace $\hat{U}\left(\alpha\right)$ y $\hat{F}\left(\alpha\right)$ determina una a la otra de manera \'unica por la relaci\'on $\hat{U}\left(\alpha\right)=\frac{1}{1-\hat{F}\left(\alpha\right)}$.
\end{Prop}


\begin{Note}
Un proceso de renovaci\'on $N\left(t\right)$ cuyos tiempos de inter-renovaci\'on tienen media finita, es un proceso Poisson con tasa $\lambda$ si y s\'olo s\'i $\esp\left[U\left(t\right)\right]=\lambda t$, para $t\geq0$.
\end{Note}


\begin{Teo}
Sea $N\left(t\right)$ un proceso puntual simple con puntos de localizaci\'on $T_{n}$ tal que $\eta\left(t\right)=\esp\left[N\left(\right)\right]$ es finita para cada $t$. Entonces para cualquier funci\'on $f:\rea_{+}\rightarrow\rea$,
\begin{eqnarray*}
\esp\left[\sum_{n=1}^{N\left(\right)}f\left(T_{n}\right)\right]=\int_{\left(0,t\right]}f\left(s\right)d\eta\left(s\right)\textrm{,  }t\geq0,
\end{eqnarray*}
suponiendo que la integral exista. Adem\'as si $X_{1},X_{2},\ldots$ son variables aleatorias definidas en el mismo espacio de probabilidad que el proceso $N\left(t\right)$ tal que $\esp\left[X_{n}|T_{n}=s\right]=f\left(s\right)$, independiente de $n$. Entonces
\begin{eqnarray*}
\esp\left[\sum_{n=1}^{N\left(t\right)}X_{n}\right]=\int_{\left(0,t\right]}f\left(s\right)d\eta\left(s\right)\textrm{,  }t\geq0,
\end{eqnarray*} 
suponiendo que la integral exista. 
\end{Teo}

\begin{Coro}[Identidad de Wald para Renovaciones]
Para el proceso de renovaci\'on $N\left(t\right)$,
\begin{eqnarray*}
\esp\left[T_{N\left(t\right)+1}\right]=\mu\esp\left[N\left(t\right)+1\right]\textrm{,  }t\geq0,
\end{eqnarray*}  
\end{Coro}


\begin{Def}
Sea $h\left(t\right)$ funci\'on de valores reales en $\rea$ acotada en intervalos finitos e igual a cero para $t<0$ La ecuaci\'on de renovaci\'on para $h\left(t\right)$ y la distribuci\'on $F$ es

\begin{eqnarray}\label{Ec.Renovacion}
H\left(t\right)=h\left(t\right)+\int_{\left[0,t\right]}H\left(t-s\right)dF\left(s\right)\textrm{,    }t\geq0,
\end{eqnarray}
donde $H\left(t\right)$ es una funci\'on de valores reales. Esto es $H=h+F\star H$. Decimos que $H\left(t\right)$ es soluci\'on de esta ecuaci\'on si satisface la ecuaci\'on, y es acotada en intervalos finitos e iguales a cero para $t<0$.
\end{Def}

\begin{Prop}
La funci\'on $U\star h\left(t\right)$ es la \'unica soluci\'on de la ecuaci\'on de renovaci\'on (\ref{Ec.Renovacion}).
\end{Prop}

\begin{Teo}[Teorema Renovaci\'on Elemental]
\begin{eqnarray*}
t^{-1}U\left(t\right)\rightarrow 1/\mu\textrm{,    cuando }t\rightarrow\infty.
\end{eqnarray*}
\end{Teo}


\begin{Note} Una funci\'on $h:\rea_{+}\rightarrow\rea$ es Directamente Riemann Integrable en los siguientes casos:
\begin{itemize}
\item[a)] $h\left(t\right)\geq0$ es decreciente y Riemann Integrable.
\item[b)] $h$ es continua excepto posiblemente en un conjunto de Lebesgue de medida 0, y $|h\left(t\right)|\leq b\left(t\right)$, donde $b$ es DRI.
\end{itemize}
\end{Note}

\begin{Teo}[Teorema Principal de Renovaci\'on]
Si $F$ es no aritm\'etica y $h\left(t\right)$ es Directamente Riemann Integrable (DRI), entonces

\begin{eqnarray*}
lim_{t\rightarrow\infty}U\star h=\frac{1}{\mu}\int_{\rea_{+}}h\left(s\right)ds.
\end{eqnarray*}
\end{Teo}

\begin{Prop}
Cualquier funci\'on $H\left(t\right)$ acotada en intervalos finitos y que es 0 para $t<0$ puede expresarse como
\begin{eqnarray*}
H\left(t\right)=U\star h\left(t\right)\textrm{,  donde }h\left(t\right)=H\left(t\right)-F\star H\left(t\right)
\end{eqnarray*}
\end{Prop}

\begin{Def}
Un proceso estoc\'astico $X\left(t\right)$ es crudamente regenerativo en un tiempo aleatorio positivo $T$ si
\begin{eqnarray*}
\esp\left[X\left(T+t\right)|T\right]=\esp\left[X\left(t\right)\right]\textrm{, para }t\geq0,\end{eqnarray*}
y con las esperanzas anteriores finitas.
\end{Def}

\begin{Prop}
Sup\'ongase que $X\left(t\right)$ es un proceso crudamente regenerativo en $T$, que tiene distribuci\'on $F$. Si $\esp\left[X\left(t\right)\right]$ es acotado en intervalos finitos, entonces
\begin{eqnarray*}
\esp\left[X\left(t\right)\right]=U\star h\left(t\right)\textrm{,  donde }h\left(t\right)=\esp\left[X\left(t\right)\indora\left(T>t\right)\right].
\end{eqnarray*}
\end{Prop}

\begin{Teo}[Regeneraci\'on Cruda]
Sup\'ongase que $X\left(t\right)$ es un proceso con valores positivo crudamente regenerativo en $T$, y def\'inase $M=\sup\left\{|X\left(t\right)|:t\leq T\right\}$. Si $T$ es no aritm\'etico y $M$ y $MT$ tienen media finita, entonces
\begin{eqnarray*}
lim_{t\rightarrow\infty}\esp\left[X\left(t\right)\right]=\frac{1}{\mu}\int_{\rea_{+}}h\left(s\right)ds,
\end{eqnarray*}
donde $h\left(t\right)=\esp\left[X\left(t\right)\indora\left(T>t\right)\right]$.
\end{Teo}


\begin{Note} Una funci\'on $h:\rea_{+}\rightarrow\rea$ es Directamente Riemann Integrable en los siguientes casos:
\begin{itemize}
\item[a)] $h\left(t\right)\geq0$ es decreciente y Riemann Integrable.
\item[b)] $h$ es continua excepto posiblemente en un conjunto de Lebesgue de medida 0, y $|h\left(t\right)|\leq b\left(t\right)$, donde $b$ es DRI.
\end{itemize}
\end{Note}

\begin{Teo}[Teorema Principal de Renovaci\'on]
Si $F$ es no aritm\'etica y $h\left(t\right)$ es Directamente Riemann Integrable (DRI), entonces

\begin{eqnarray*}
lim_{t\rightarrow\infty}U\star h=\frac{1}{\mu}\int_{\rea_{+}}h\left(s\right)ds.
\end{eqnarray*}
\end{Teo}

\begin{Prop}
Cualquier funci\'on $H\left(t\right)$ acotada en intervalos finitos y que es 0 para $t<0$ puede expresarse como
\begin{eqnarray*}
H\left(t\right)=U\star h\left(t\right)\textrm{,  donde }h\left(t\right)=H\left(t\right)-F\star H\left(t\right)
\end{eqnarray*}
\end{Prop}

\begin{Def}
Un proceso estoc\'astico $X\left(t\right)$ es crudamente regenerativo en un tiempo aleatorio positivo $T$ si
\begin{eqnarray*}
\esp\left[X\left(T+t\right)|T\right]=\esp\left[X\left(t\right)\right]\textrm{, para }t\geq0,\end{eqnarray*}
y con las esperanzas anteriores finitas.
\end{Def}

\begin{Prop}
Sup\'ongase que $X\left(t\right)$ es un proceso crudamente regenerativo en $T$, que tiene distribuci\'on $F$. Si $\esp\left[X\left(t\right)\right]$ es acotado en intervalos finitos, entonces
\begin{eqnarray*}
\esp\left[X\left(t\right)\right]=U\star h\left(t\right)\textrm{,  donde }h\left(t\right)=\esp\left[X\left(t\right)\indora\left(T>t\right)\right].
\end{eqnarray*}
\end{Prop}

\begin{Teo}[Regeneraci\'on Cruda]
Sup\'ongase que $X\left(t\right)$ es un proceso con valores positivo crudamente regenerativo en $T$, y def\'inase $M=\sup\left\{|X\left(t\right)|:t\leq T\right\}$. Si $T$ es no aritm\'etico y $M$ y $MT$ tienen media finita, entonces
\begin{eqnarray*}
lim_{t\rightarrow\infty}\esp\left[X\left(t\right)\right]=\frac{1}{\mu}\int_{\rea_{+}}h\left(s\right)ds,
\end{eqnarray*}
donde $h\left(t\right)=\esp\left[X\left(t\right)\indora\left(T>t\right)\right]$.
\end{Teo}

%________________________________________________________________________
\subsection{Procesos Regenerativos}
%________________________________________________________________________

Para $\left\{X\left(t\right):t\geq0\right\}$ Proceso Estoc\'astico a tiempo continuo con estado de espacios $S$, que es un espacio m\'etrico, con trayectorias continuas por la derecha y con l\'imites por la izquierda c.s. Sea $N\left(t\right)$ un proceso de renovaci\'on en $\rea_{+}$ definido en el mismo espacio de probabilidad que $X\left(t\right)$, con tiempos de renovaci\'on $T$ y tiempos de inter-renovaci\'on $\xi_{n}=T_{n}-T_{n-1}$, con misma distribuci\'on $F$ de media finita $\mu$.



\begin{Def}
Para el proceso $\left\{\left(N\left(t\right),X\left(t\right)\right):t\geq0\right\}$, sus trayectoria muestrales en el intervalo de tiempo $\left[T_{n-1},T_{n}\right)$ est\'an descritas por
\begin{eqnarray*}
\zeta_{n}=\left(\xi_{n},\left\{X\left(T_{n-1}+t\right):0\leq t<\xi_{n}\right\}\right)
\end{eqnarray*}
Este $\zeta_{n}$ es el $n$-\'esimo segmento del proceso. El proceso es regenerativo sobre los tiempos $T_{n}$ si sus segmentos $\zeta_{n}$ son independientes e id\'enticamennte distribuidos.
\end{Def}


\begin{Obs}
Si $\tilde{X}\left(t\right)$ con espacio de estados $\tilde{S}$ es regenerativo sobre $T_{n}$, entonces $X\left(t\right)=f\left(\tilde{X}\left(t\right)\right)$ tambi\'en es regenerativo sobre $T_{n}$, para cualquier funci\'on $f:\tilde{S}\rightarrow S$.
\end{Obs}

\begin{Obs}
Los procesos regenerativos son crudamente regenerativos, pero no al rev\'es.
\end{Obs}

\begin{Def}[Definici\'on Cl\'asica]
Un proceso estoc\'astico $X=\left\{X\left(t\right):t\geq0\right\}$ es llamado regenerativo is existe una variable aleatoria $R_{1}>0$ tal que
\begin{itemize}
\item[i)] $\left\{X\left(t+R_{1}\right):t\geq0\right\}$ es independiente de $\left\{\left\{X\left(t\right):t<R_{1}\right\},\right\}$
\item[ii)] $\left\{X\left(t+R_{1}\right):t\geq0\right\}$ es estoc\'asticamente equivalente a $\left\{X\left(t\right):t>0\right\}$
\end{itemize}

Llamamos a $R_{1}$ tiempo de regeneraci\'on, y decimos que $X$ se regenera en este punto.
\end{Def}

$\left\{X\left(t+R_{1}\right)\right\}$ es regenerativo con tiempo de regeneraci\'on $R_{2}$, independiente de $R_{1}$ pero con la misma distribuci\'on que $R_{1}$. Procediendo de esta manera se obtiene una secuencia de variables aleatorias independientes e id\'enticamente distribuidas $\left\{R_{n}\right\}$ llamados longitudes de ciclo. Si definimos a $Z_{k}\equiv R_{1}+R_{2}+\cdots+R_{k}$, se tiene un proceso de renovaci\'on llamado proceso de renovaci\'on encajado para $X$.

\begin{Note}
Un proceso regenerativo con media de la longitud de ciclo finita es llamado positivo recurrente.
\end{Note}


\begin{Def}
Para $x$ fijo y para cada $t\geq0$, sea $I_{x}\left(t\right)=1$ si $X\left(t\right)\leq x$,  $I_{x}\left(t\right)=0$ en caso contrario, y def\'inanse los tiempos promedio
\begin{eqnarray*}
\overline{X}&=&lim_{t\rightarrow\infty}\frac{1}{t}\int_{0}^{\infty}X\left(u\right)du\\
\prob\left(X_{\infty}\leq x\right)&=&lim_{t\rightarrow\infty}\frac{1}{t}\int_{0}^{\infty}I_{x}\left(u\right)du,
\end{eqnarray*}
cuando estos l\'imites existan.
\end{Def}

Como consecuencia del teorema de Renovaci\'on-Recompensa, se tiene que el primer l\'imite  existe y es igual a la constante
\begin{eqnarray*}
\overline{X}&=&\frac{\esp\left[\int_{0}^{R_{1}}X\left(t\right)dt\right]}{\esp\left[R_{1}\right]},
\end{eqnarray*}
suponiendo que ambas esperanzas son finitas.

\begin{Note}
\begin{itemize}
\item[a)] Si el proceso regenerativo $X$ es positivo recurrente y tiene trayectorias muestrales no negativas, entonces la ecuaci\'on anterior es v\'alida.
\item[b)] Si $X$ es positivo recurrente regenerativo, podemos construir una \'unica versi\'on estacionaria de este proceso, $X_{e}=\left\{X_{e}\left(t\right)\right\}$, donde $X_{e}$ es un proceso estoc\'astico regenerativo y estrictamente estacionario, con distribuci\'on marginal distribuida como $X_{\infty}$
\end{itemize}
\end{Note}

%________________________________________________________________________
\subsection{Procesos Regenerativos}
%________________________________________________________________________

Para $\left\{X\left(t\right):t\geq0\right\}$ Proceso Estoc\'astico a tiempo continuo con estado de espacios $S$, que es un espacio m\'etrico, con trayectorias continuas por la derecha y con l\'imites por la izquierda c.s. Sea $N\left(t\right)$ un proceso de renovaci\'on en $\rea_{+}$ definido en el mismo espacio de probabilidad que $X\left(t\right)$, con tiempos de renovaci\'on $T$ y tiempos de inter-renovaci\'on $\xi_{n}=T_{n}-T_{n-1}$, con misma distribuci\'on $F$ de media finita $\mu$.



\begin{Def}
Para el proceso $\left\{\left(N\left(t\right),X\left(t\right)\right):t\geq0\right\}$, sus trayectoria muestrales en el intervalo de tiempo $\left[T_{n-1},T_{n}\right)$ est\'an descritas por
\begin{eqnarray*}
\zeta_{n}=\left(\xi_{n},\left\{X\left(T_{n-1}+t\right):0\leq t<\xi_{n}\right\}\right)
\end{eqnarray*}
Este $\zeta_{n}$ es el $n$-\'esimo segmento del proceso. El proceso es regenerativo sobre los tiempos $T_{n}$ si sus segmentos $\zeta_{n}$ son independientes e id\'enticamennte distribuidos.
\end{Def}


\begin{Obs}
Si $\tilde{X}\left(t\right)$ con espacio de estados $\tilde{S}$ es regenerativo sobre $T_{n}$, entonces $X\left(t\right)=f\left(\tilde{X}\left(t\right)\right)$ tambi\'en es regenerativo sobre $T_{n}$, para cualquier funci\'on $f:\tilde{S}\rightarrow S$.
\end{Obs}

\begin{Obs}
Los procesos regenerativos son crudamente regenerativos, pero no al rev\'es.
\end{Obs}

\begin{Def}[Definici\'on Cl\'asica]
Un proceso estoc\'astico $X=\left\{X\left(t\right):t\geq0\right\}$ es llamado regenerativo is existe una variable aleatoria $R_{1}>0$ tal que
\begin{itemize}
\item[i)] $\left\{X\left(t+R_{1}\right):t\geq0\right\}$ es independiente de $\left\{\left\{X\left(t\right):t<R_{1}\right\},\right\}$
\item[ii)] $\left\{X\left(t+R_{1}\right):t\geq0\right\}$ es estoc\'asticamente equivalente a $\left\{X\left(t\right):t>0\right\}$
\end{itemize}

Llamamos a $R_{1}$ tiempo de regeneraci\'on, y decimos que $X$ se regenera en este punto.
\end{Def}

$\left\{X\left(t+R_{1}\right)\right\}$ es regenerativo con tiempo de regeneraci\'on $R_{2}$, independiente de $R_{1}$ pero con la misma distribuci\'on que $R_{1}$. Procediendo de esta manera se obtiene una secuencia de variables aleatorias independientes e id\'enticamente distribuidas $\left\{R_{n}\right\}$ llamados longitudes de ciclo. Si definimos a $Z_{k}\equiv R_{1}+R_{2}+\cdots+R_{k}$, se tiene un proceso de renovaci\'on llamado proceso de renovaci\'on encajado para $X$.

\begin{Note}
Un proceso regenerativo con media de la longitud de ciclo finita es llamado positivo recurrente.
\end{Note}


\begin{Def}
Para $x$ fijo y para cada $t\geq0$, sea $I_{x}\left(t\right)=1$ si $X\left(t\right)\leq x$,  $I_{x}\left(t\right)=0$ en caso contrario, y def\'inanse los tiempos promedio
\begin{eqnarray*}
\overline{X}&=&lim_{t\rightarrow\infty}\frac{1}{t}\int_{0}^{\infty}X\left(u\right)du\\
\prob\left(X_{\infty}\leq x\right)&=&lim_{t\rightarrow\infty}\frac{1}{t}\int_{0}^{\infty}I_{x}\left(u\right)du,
\end{eqnarray*}
cuando estos l\'imites existan.
\end{Def}

Como consecuencia del teorema de Renovaci\'on-Recompensa, se tiene que el primer l\'imite  existe y es igual a la constante
\begin{eqnarray*}
\overline{X}&=&\frac{\esp\left[\int_{0}^{R_{1}}X\left(t\right)dt\right]}{\esp\left[R_{1}\right]},
\end{eqnarray*}
suponiendo que ambas esperanzas son finitas.

\begin{Note}
\begin{itemize}
\item[a)] Si el proceso regenerativo $X$ es positivo recurrente y tiene trayectorias muestrales no negativas, entonces la ecuaci\'on anterior es v\'alida.
\item[b)] Si $X$ es positivo recurrente regenerativo, podemos construir una \'unica versi\'on estacionaria de este proceso, $X_{e}=\left\{X_{e}\left(t\right)\right\}$, donde $X_{e}$ es un proceso estoc\'astico regenerativo y estrictamente estacionario, con distribuci\'on marginal distribuida como $X_{\infty}$
\end{itemize}
\end{Note}
%__________________________________________________________________________________________
\subsection{Procesos Regenerativos Estacionarios - Stidham \cite{Stidham}}
%__________________________________________________________________________________________


Un proceso estoc\'astico a tiempo continuo $\left\{V\left(t\right),t\geq0\right\}$ es un proceso regenerativo si existe una sucesi\'on de variables aleatorias independientes e id\'enticamente distribuidas $\left\{X_{1},X_{2},\ldots\right\}$, sucesi\'on de renovaci\'on, tal que para cualquier conjunto de Borel $A$, 

\begin{eqnarray*}
\prob\left\{V\left(t\right)\in A|X_{1}+X_{2}+\cdots+X_{R\left(t\right)}=s,\left\{V\left(\tau\right),\tau<s\right\}\right\}=\prob\left\{V\left(t-s\right)\in A|X_{1}>t-s\right\},
\end{eqnarray*}
para todo $0\leq s\leq t$, donde $R\left(t\right)=\max\left\{X_{1}+X_{2}+\cdots+X_{j}\leq t\right\}=$n\'umero de renovaciones ({\emph{puntos de regeneraci\'on}}) que ocurren en $\left[0,t\right]$. El intervalo $\left[0,X_{1}\right)$ es llamado {\emph{primer ciclo de regeneraci\'on}} de $\left\{V\left(t \right),t\geq0\right\}$, $\left[X_{1},X_{1}+X_{2}\right)$ el {\emph{segundo ciclo de regeneraci\'on}}, y as\'i sucesivamente.

Sea $X=X_{1}$ y sea $F$ la funci\'on de distrbuci\'on de $X$


\begin{Def}
Se define el proceso estacionario, $\left\{V^{*}\left(t\right),t\geq0\right\}$, para $\left\{V\left(t\right),t\geq0\right\}$ por

\begin{eqnarray*}
\prob\left\{V\left(t\right)\in A\right\}=\frac{1}{\esp\left[X\right]}\int_{0}^{\infty}\prob\left\{V\left(t+x\right)\in A|X>x\right\}\left(1-F\left(x\right)\right)dx,
\end{eqnarray*} 
para todo $t\geq0$ y todo conjunto de Borel $A$.
\end{Def}

\begin{Def}
Una distribuci\'on se dice que es {\emph{aritm\'etica}} si todos sus puntos de incremento son m\'ultiplos de la forma $0,\lambda, 2\lambda,\ldots$ para alguna $\lambda>0$ entera.
\end{Def}


\begin{Def}
Una modificaci\'on medible de un proceso $\left\{V\left(t\right),t\geq0\right\}$, es una versi\'on de este, $\left\{V\left(t,w\right)\right\}$ conjuntamente medible para $t\geq0$ y para $w\in S$, $S$ espacio de estados para $\left\{V\left(t\right),t\geq0\right\}$.
\end{Def}

\begin{Teo}
Sea $\left\{V\left(t\right),t\geq\right\}$ un proceso regenerativo no negativo con modificaci\'on medible. Sea $\esp\left[X\right]<\infty$. Entonces el proceso estacionario dado por la ecuaci\'on anterior est\'a bien definido y tiene funci\'on de distribuci\'on independiente de $t$, adem\'as
\begin{itemize}
\item[i)] \begin{eqnarray*}
\esp\left[V^{*}\left(0\right)\right]&=&\frac{\esp\left[\int_{0}^{X}V\left(s\right)ds\right]}{\esp\left[X\right]}\end{eqnarray*}
\item[ii)] Si $\esp\left[V^{*}\left(0\right)\right]<\infty$, equivalentemente, si $\esp\left[\int_{0}^{X}V\left(s\right)ds\right]<\infty$,entonces
\begin{eqnarray*}
\frac{\int_{0}^{t}V\left(s\right)ds}{t}\rightarrow\frac{\esp\left[\int_{0}^{X}V\left(s\right)ds\right]}{\esp\left[X\right]}
\end{eqnarray*}
con probabilidad 1 y en media, cuando $t\rightarrow\infty$.
\end{itemize}
\end{Teo}


%__________________________________________________________________________________________
\subsection{Procesos Regenerativos Estacionarios - Stidham \cite{Stidham}}
%__________________________________________________________________________________________


Un proceso estoc\'astico a tiempo continuo $\left\{V\left(t\right),t\geq0\right\}$ es un proceso regenerativo si existe una sucesi\'on de variables aleatorias independientes e id\'enticamente distribuidas $\left\{X_{1},X_{2},\ldots\right\}$, sucesi\'on de renovaci\'on, tal que para cualquier conjunto de Borel $A$, 

\begin{eqnarray*}
\prob\left\{V\left(t\right)\in A|X_{1}+X_{2}+\cdots+X_{R\left(t\right)}=s,\left\{V\left(\tau\right),\tau<s\right\}\right\}=\prob\left\{V\left(t-s\right)\in A|X_{1}>t-s\right\},
\end{eqnarray*}
para todo $0\leq s\leq t$, donde $R\left(t\right)=\max\left\{X_{1}+X_{2}+\cdots+X_{j}\leq t\right\}=$n\'umero de renovaciones ({\emph{puntos de regeneraci\'on}}) que ocurren en $\left[0,t\right]$. El intervalo $\left[0,X_{1}\right)$ es llamado {\emph{primer ciclo de regeneraci\'on}} de $\left\{V\left(t \right),t\geq0\right\}$, $\left[X_{1},X_{1}+X_{2}\right)$ el {\emph{segundo ciclo de regeneraci\'on}}, y as\'i sucesivamente.

Sea $X=X_{1}$ y sea $F$ la funci\'on de distrbuci\'on de $X$


\begin{Def}
Se define el proceso estacionario, $\left\{V^{*}\left(t\right),t\geq0\right\}$, para $\left\{V\left(t\right),t\geq0\right\}$ por

\begin{eqnarray*}
\prob\left\{V\left(t\right)\in A\right\}=\frac{1}{\esp\left[X\right]}\int_{0}^{\infty}\prob\left\{V\left(t+x\right)\in A|X>x\right\}\left(1-F\left(x\right)\right)dx,
\end{eqnarray*} 
para todo $t\geq0$ y todo conjunto de Borel $A$.
\end{Def}

\begin{Def}
Una distribuci\'on se dice que es {\emph{aritm\'etica}} si todos sus puntos de incremento son m\'ultiplos de la forma $0,\lambda, 2\lambda,\ldots$ para alguna $\lambda>0$ entera.
\end{Def}


\begin{Def}
Una modificaci\'on medible de un proceso $\left\{V\left(t\right),t\geq0\right\}$, es una versi\'on de este, $\left\{V\left(t,w\right)\right\}$ conjuntamente medible para $t\geq0$ y para $w\in S$, $S$ espacio de estados para $\left\{V\left(t\right),t\geq0\right\}$.
\end{Def}

\begin{Teo}
Sea $\left\{V\left(t\right),t\geq\right\}$ un proceso regenerativo no negativo con modificaci\'on medible. Sea $\esp\left[X\right]<\infty$. Entonces el proceso estacionario dado por la ecuaci\'on anterior est\'a bien definido y tiene funci\'on de distribuci\'on independiente de $t$, adem\'as
\begin{itemize}
\item[i)] \begin{eqnarray*}
\esp\left[V^{*}\left(0\right)\right]&=&\frac{\esp\left[\int_{0}^{X}V\left(s\right)ds\right]}{\esp\left[X\right]}\end{eqnarray*}
\item[ii)] Si $\esp\left[V^{*}\left(0\right)\right]<\infty$, equivalentemente, si $\esp\left[\int_{0}^{X}V\left(s\right)ds\right]<\infty$,entonces
\begin{eqnarray*}
\frac{\int_{0}^{t}V\left(s\right)ds}{t}\rightarrow\frac{\esp\left[\int_{0}^{X}V\left(s\right)ds\right]}{\esp\left[X\right]}
\end{eqnarray*}
con probabilidad 1 y en media, cuando $t\rightarrow\infty$.
\end{itemize}
\end{Teo}
%
%___________________________________________________________________________________________
%\vspace{5.5cm}
%\chapter{Cadenas de Markov estacionarias}
%\vspace{-1.0cm}
%___________________________________________________________________________________________
%
\subsection{Propiedades de los Procesos de Renovaci\'on}
%___________________________________________________________________________________________
%

Los tiempos $T_{n}$ est\'an relacionados con los conteos de $N\left(t\right)$ por

\begin{eqnarray*}
\left\{N\left(t\right)\geq n\right\}&=&\left\{T_{n}\leq t\right\}\\
T_{N\left(t\right)}\leq &t&<T_{N\left(t\right)+1},
\end{eqnarray*}

adem\'as $N\left(T_{n}\right)=n$, y 

\begin{eqnarray*}
N\left(t\right)=\max\left\{n:T_{n}\leq t\right\}=\min\left\{n:T_{n+1}>t\right\}
\end{eqnarray*}

Por propiedades de la convoluci\'on se sabe que

\begin{eqnarray*}
P\left\{T_{n}\leq t\right\}=F^{n\star}\left(t\right)
\end{eqnarray*}
que es la $n$-\'esima convoluci\'on de $F$. Entonces 

\begin{eqnarray*}
\left\{N\left(t\right)\geq n\right\}&=&\left\{T_{n}\leq t\right\}\\
P\left\{N\left(t\right)\leq n\right\}&=&1-F^{\left(n+1\right)\star}\left(t\right)
\end{eqnarray*}

Adem\'as usando el hecho de que $\esp\left[N\left(t\right)\right]=\sum_{n=1}^{\infty}P\left\{N\left(t\right)\geq n\right\}$
se tiene que

\begin{eqnarray*}
\esp\left[N\left(t\right)\right]=\sum_{n=1}^{\infty}F^{n\star}\left(t\right)
\end{eqnarray*}

\begin{Prop}
Para cada $t\geq0$, la funci\'on generadora de momentos $\esp\left[e^{\alpha N\left(t\right)}\right]$ existe para alguna $\alpha$ en una vecindad del 0, y de aqu\'i que $\esp\left[N\left(t\right)^{m}\right]<\infty$, para $m\geq1$.
\end{Prop}


\begin{Note}
Si el primer tiempo de renovaci\'on $\xi_{1}$ no tiene la misma distribuci\'on que el resto de las $\xi_{n}$, para $n\geq2$, a $N\left(t\right)$ se le llama Proceso de Renovaci\'on retardado, donde si $\xi$ tiene distribuci\'on $G$, entonces el tiempo $T_{n}$ de la $n$-\'esima renovaci\'on tiene distribuci\'on $G\star F^{\left(n-1\right)\star}\left(t\right)$
\end{Note}


\begin{Teo}
Para una constante $\mu\leq\infty$ ( o variable aleatoria), las siguientes expresiones son equivalentes:

\begin{eqnarray}
lim_{n\rightarrow\infty}n^{-1}T_{n}&=&\mu,\textrm{ c.s.}\\
lim_{t\rightarrow\infty}t^{-1}N\left(t\right)&=&1/\mu,\textrm{ c.s.}
\end{eqnarray}
\end{Teo}


Es decir, $T_{n}$ satisface la Ley Fuerte de los Grandes N\'umeros s\'i y s\'olo s\'i $N\left/t\right)$ la cumple.


\begin{Coro}[Ley Fuerte de los Grandes N\'umeros para Procesos de Renovaci\'on]
Si $N\left(t\right)$ es un proceso de renovaci\'on cuyos tiempos de inter-renovaci\'on tienen media $\mu\leq\infty$, entonces
\begin{eqnarray}
t^{-1}N\left(t\right)\rightarrow 1/\mu,\textrm{ c.s. cuando }t\rightarrow\infty.
\end{eqnarray}

\end{Coro}


Considerar el proceso estoc\'astico de valores reales $\left\{Z\left(t\right):t\geq0\right\}$ en el mismo espacio de probabilidad que $N\left(t\right)$

\begin{Def}
Para el proceso $\left\{Z\left(t\right):t\geq0\right\}$ se define la fluctuaci\'on m\'axima de $Z\left(t\right)$ en el intervalo $\left(T_{n-1},T_{n}\right]$:
\begin{eqnarray*}
M_{n}=\sup_{T_{n-1}<t\leq T_{n}}|Z\left(t\right)-Z\left(T_{n-1}\right)|
\end{eqnarray*}
\end{Def}

\begin{Teo}
Sup\'ongase que $n^{-1}T_{n}\rightarrow\mu$ c.s. cuando $n\rightarrow\infty$, donde $\mu\leq\infty$ es una constante o variable aleatoria. Sea $a$ una constante o variable aleatoria que puede ser infinita cuando $\mu$ es finita, y considere las expresiones l\'imite:
\begin{eqnarray}
lim_{n\rightarrow\infty}n^{-1}Z\left(T_{n}\right)&=&a,\textrm{ c.s.}\\
lim_{t\rightarrow\infty}t^{-1}Z\left(t\right)&=&a/\mu,\textrm{ c.s.}
\end{eqnarray}
La segunda expresi\'on implica la primera. Conversamente, la primera implica la segunda si el proceso $Z\left(t\right)$ es creciente, o si $lim_{n\rightarrow\infty}n^{-1}M_{n}=0$ c.s.
\end{Teo}

\begin{Coro}
Si $N\left(t\right)$ es un proceso de renovaci\'on, y $\left(Z\left(T_{n}\right)-Z\left(T_{n-1}\right),M_{n}\right)$, para $n\geq1$, son variables aleatorias independientes e id\'enticamente distribuidas con media finita, entonces,
\begin{eqnarray}
lim_{t\rightarrow\infty}t^{-1}Z\left(t\right)\rightarrow\frac{\esp\left[Z\left(T_{1}\right)-Z\left(T_{0}\right)\right]}{\esp\left[T_{1}\right]},\textrm{ c.s. cuando  }t\rightarrow\infty.
\end{eqnarray}
\end{Coro}


%___________________________________________________________________________________________
%
%\subsection{Propiedades de los Procesos de Renovaci\'on}
%___________________________________________________________________________________________
%

Los tiempos $T_{n}$ est\'an relacionados con los conteos de $N\left(t\right)$ por

\begin{eqnarray*}
\left\{N\left(t\right)\geq n\right\}&=&\left\{T_{n}\leq t\right\}\\
T_{N\left(t\right)}\leq &t&<T_{N\left(t\right)+1},
\end{eqnarray*}

adem\'as $N\left(T_{n}\right)=n$, y 

\begin{eqnarray*}
N\left(t\right)=\max\left\{n:T_{n}\leq t\right\}=\min\left\{n:T_{n+1}>t\right\}
\end{eqnarray*}

Por propiedades de la convoluci\'on se sabe que

\begin{eqnarray*}
P\left\{T_{n}\leq t\right\}=F^{n\star}\left(t\right)
\end{eqnarray*}
que es la $n$-\'esima convoluci\'on de $F$. Entonces 

\begin{eqnarray*}
\left\{N\left(t\right)\geq n\right\}&=&\left\{T_{n}\leq t\right\}\\
P\left\{N\left(t\right)\leq n\right\}&=&1-F^{\left(n+1\right)\star}\left(t\right)
\end{eqnarray*}

Adem\'as usando el hecho de que $\esp\left[N\left(t\right)\right]=\sum_{n=1}^{\infty}P\left\{N\left(t\right)\geq n\right\}$
se tiene que

\begin{eqnarray*}
\esp\left[N\left(t\right)\right]=\sum_{n=1}^{\infty}F^{n\star}\left(t\right)
\end{eqnarray*}

\begin{Prop}
Para cada $t\geq0$, la funci\'on generadora de momentos $\esp\left[e^{\alpha N\left(t\right)}\right]$ existe para alguna $\alpha$ en una vecindad del 0, y de aqu\'i que $\esp\left[N\left(t\right)^{m}\right]<\infty$, para $m\geq1$.
\end{Prop}


\begin{Note}
Si el primer tiempo de renovaci\'on $\xi_{1}$ no tiene la misma distribuci\'on que el resto de las $\xi_{n}$, para $n\geq2$, a $N\left(t\right)$ se le llama Proceso de Renovaci\'on retardado, donde si $\xi$ tiene distribuci\'on $G$, entonces el tiempo $T_{n}$ de la $n$-\'esima renovaci\'on tiene distribuci\'on $G\star F^{\left(n-1\right)\star}\left(t\right)$
\end{Note}


\begin{Teo}
Para una constante $\mu\leq\infty$ ( o variable aleatoria), las siguientes expresiones son equivalentes:

\begin{eqnarray}
lim_{n\rightarrow\infty}n^{-1}T_{n}&=&\mu,\textrm{ c.s.}\\
lim_{t\rightarrow\infty}t^{-1}N\left(t\right)&=&1/\mu,\textrm{ c.s.}
\end{eqnarray}
\end{Teo}


Es decir, $T_{n}$ satisface la Ley Fuerte de los Grandes N\'umeros s\'i y s\'olo s\'i $N\left/t\right)$ la cumple.


\begin{Coro}[Ley Fuerte de los Grandes N\'umeros para Procesos de Renovaci\'on]
Si $N\left(t\right)$ es un proceso de renovaci\'on cuyos tiempos de inter-renovaci\'on tienen media $\mu\leq\infty$, entonces
\begin{eqnarray}
t^{-1}N\left(t\right)\rightarrow 1/\mu,\textrm{ c.s. cuando }t\rightarrow\infty.
\end{eqnarray}

\end{Coro}


Considerar el proceso estoc\'astico de valores reales $\left\{Z\left(t\right):t\geq0\right\}$ en el mismo espacio de probabilidad que $N\left(t\right)$

\begin{Def}
Para el proceso $\left\{Z\left(t\right):t\geq0\right\}$ se define la fluctuaci\'on m\'axima de $Z\left(t\right)$ en el intervalo $\left(T_{n-1},T_{n}\right]$:
\begin{eqnarray*}
M_{n}=\sup_{T_{n-1}<t\leq T_{n}}|Z\left(t\right)-Z\left(T_{n-1}\right)|
\end{eqnarray*}
\end{Def}

\begin{Teo}
Sup\'ongase que $n^{-1}T_{n}\rightarrow\mu$ c.s. cuando $n\rightarrow\infty$, donde $\mu\leq\infty$ es una constante o variable aleatoria. Sea $a$ una constante o variable aleatoria que puede ser infinita cuando $\mu$ es finita, y considere las expresiones l\'imite:
\begin{eqnarray}
lim_{n\rightarrow\infty}n^{-1}Z\left(T_{n}\right)&=&a,\textrm{ c.s.}\\
lim_{t\rightarrow\infty}t^{-1}Z\left(t\right)&=&a/\mu,\textrm{ c.s.}
\end{eqnarray}
La segunda expresi\'on implica la primera. Conversamente, la primera implica la segunda si el proceso $Z\left(t\right)$ es creciente, o si $lim_{n\rightarrow\infty}n^{-1}M_{n}=0$ c.s.
\end{Teo}

\begin{Coro}
Si $N\left(t\right)$ es un proceso de renovaci\'on, y $\left(Z\left(T_{n}\right)-Z\left(T_{n-1}\right),M_{n}\right)$, para $n\geq1$, son variables aleatorias independientes e id\'enticamente distribuidas con media finita, entonces,
\begin{eqnarray}
lim_{t\rightarrow\infty}t^{-1}Z\left(t\right)\rightarrow\frac{\esp\left[Z\left(T_{1}\right)-Z\left(T_{0}\right)\right]}{\esp\left[T_{1}\right]},\textrm{ c.s. cuando  }t\rightarrow\infty.
\end{eqnarray}
\end{Coro}

%___________________________________________________________________________________________
%
%\subsection{Propiedades de los Procesos de Renovaci\'on}
%___________________________________________________________________________________________
%

Los tiempos $T_{n}$ est\'an relacionados con los conteos de $N\left(t\right)$ por

\begin{eqnarray*}
\left\{N\left(t\right)\geq n\right\}&=&\left\{T_{n}\leq t\right\}\\
T_{N\left(t\right)}\leq &t&<T_{N\left(t\right)+1},
\end{eqnarray*}

adem\'as $N\left(T_{n}\right)=n$, y 

\begin{eqnarray*}
N\left(t\right)=\max\left\{n:T_{n}\leq t\right\}=\min\left\{n:T_{n+1}>t\right\}
\end{eqnarray*}

Por propiedades de la convoluci\'on se sabe que

\begin{eqnarray*}
P\left\{T_{n}\leq t\right\}=F^{n\star}\left(t\right)
\end{eqnarray*}
que es la $n$-\'esima convoluci\'on de $F$. Entonces 

\begin{eqnarray*}
\left\{N\left(t\right)\geq n\right\}&=&\left\{T_{n}\leq t\right\}\\
P\left\{N\left(t\right)\leq n\right\}&=&1-F^{\left(n+1\right)\star}\left(t\right)
\end{eqnarray*}

Adem\'as usando el hecho de que $\esp\left[N\left(t\right)\right]=\sum_{n=1}^{\infty}P\left\{N\left(t\right)\geq n\right\}$
se tiene que

\begin{eqnarray*}
\esp\left[N\left(t\right)\right]=\sum_{n=1}^{\infty}F^{n\star}\left(t\right)
\end{eqnarray*}

\begin{Prop}
Para cada $t\geq0$, la funci\'on generadora de momentos $\esp\left[e^{\alpha N\left(t\right)}\right]$ existe para alguna $\alpha$ en una vecindad del 0, y de aqu\'i que $\esp\left[N\left(t\right)^{m}\right]<\infty$, para $m\geq1$.
\end{Prop}


\begin{Note}
Si el primer tiempo de renovaci\'on $\xi_{1}$ no tiene la misma distribuci\'on que el resto de las $\xi_{n}$, para $n\geq2$, a $N\left(t\right)$ se le llama Proceso de Renovaci\'on retardado, donde si $\xi$ tiene distribuci\'on $G$, entonces el tiempo $T_{n}$ de la $n$-\'esima renovaci\'on tiene distribuci\'on $G\star F^{\left(n-1\right)\star}\left(t\right)$
\end{Note}


\begin{Teo}
Para una constante $\mu\leq\infty$ ( o variable aleatoria), las siguientes expresiones son equivalentes:

\begin{eqnarray}
lim_{n\rightarrow\infty}n^{-1}T_{n}&=&\mu,\textrm{ c.s.}\\
lim_{t\rightarrow\infty}t^{-1}N\left(t\right)&=&1/\mu,\textrm{ c.s.}
\end{eqnarray}
\end{Teo}


Es decir, $T_{n}$ satisface la Ley Fuerte de los Grandes N\'umeros s\'i y s\'olo s\'i $N\left/t\right)$ la cumple.


\begin{Coro}[Ley Fuerte de los Grandes N\'umeros para Procesos de Renovaci\'on]
Si $N\left(t\right)$ es un proceso de renovaci\'on cuyos tiempos de inter-renovaci\'on tienen media $\mu\leq\infty$, entonces
\begin{eqnarray}
t^{-1}N\left(t\right)\rightarrow 1/\mu,\textrm{ c.s. cuando }t\rightarrow\infty.
\end{eqnarray}

\end{Coro}


Considerar el proceso estoc\'astico de valores reales $\left\{Z\left(t\right):t\geq0\right\}$ en el mismo espacio de probabilidad que $N\left(t\right)$

\begin{Def}
Para el proceso $\left\{Z\left(t\right):t\geq0\right\}$ se define la fluctuaci\'on m\'axima de $Z\left(t\right)$ en el intervalo $\left(T_{n-1},T_{n}\right]$:
\begin{eqnarray*}
M_{n}=\sup_{T_{n-1}<t\leq T_{n}}|Z\left(t\right)-Z\left(T_{n-1}\right)|
\end{eqnarray*}
\end{Def}

\begin{Teo}
Sup\'ongase que $n^{-1}T_{n}\rightarrow\mu$ c.s. cuando $n\rightarrow\infty$, donde $\mu\leq\infty$ es una constante o variable aleatoria. Sea $a$ una constante o variable aleatoria que puede ser infinita cuando $\mu$ es finita, y considere las expresiones l\'imite:
\begin{eqnarray}
lim_{n\rightarrow\infty}n^{-1}Z\left(T_{n}\right)&=&a,\textrm{ c.s.}\\
lim_{t\rightarrow\infty}t^{-1}Z\left(t\right)&=&a/\mu,\textrm{ c.s.}
\end{eqnarray}
La segunda expresi\'on implica la primera. Conversamente, la primera implica la segunda si el proceso $Z\left(t\right)$ es creciente, o si $lim_{n\rightarrow\infty}n^{-1}M_{n}=0$ c.s.
\end{Teo}

\begin{Coro}
Si $N\left(t\right)$ es un proceso de renovaci\'on, y $\left(Z\left(T_{n}\right)-Z\left(T_{n-1}\right),M_{n}\right)$, para $n\geq1$, son variables aleatorias independientes e id\'enticamente distribuidas con media finita, entonces,
\begin{eqnarray}
lim_{t\rightarrow\infty}t^{-1}Z\left(t\right)\rightarrow\frac{\esp\left[Z\left(T_{1}\right)-Z\left(T_{0}\right)\right]}{\esp\left[T_{1}\right]},\textrm{ c.s. cuando  }t\rightarrow\infty.
\end{eqnarray}
\end{Coro}



%___________________________________________________________________________________________
%
\subsection{Propiedades de los Procesos de Renovaci\'on}
%___________________________________________________________________________________________
%

Los tiempos $T_{n}$ est\'an relacionados con los conteos de $N\left(t\right)$ por

\begin{eqnarray*}
\left\{N\left(t\right)\geq n\right\}&=&\left\{T_{n}\leq t\right\}\\
T_{N\left(t\right)}\leq &t&<T_{N\left(t\right)+1},
\end{eqnarray*}

adem\'as $N\left(T_{n}\right)=n$, y 

\begin{eqnarray*}
N\left(t\right)=\max\left\{n:T_{n}\leq t\right\}=\min\left\{n:T_{n+1}>t\right\}
\end{eqnarray*}

Por propiedades de la convoluci\'on se sabe que

\begin{eqnarray*}
P\left\{T_{n}\leq t\right\}=F^{n\star}\left(t\right)
\end{eqnarray*}
que es la $n$-\'esima convoluci\'on de $F$. Entonces 

\begin{eqnarray*}
\left\{N\left(t\right)\geq n\right\}&=&\left\{T_{n}\leq t\right\}\\
P\left\{N\left(t\right)\leq n\right\}&=&1-F^{\left(n+1\right)\star}\left(t\right)
\end{eqnarray*}

Adem\'as usando el hecho de que $\esp\left[N\left(t\right)\right]=\sum_{n=1}^{\infty}P\left\{N\left(t\right)\geq n\right\}$
se tiene que

\begin{eqnarray*}
\esp\left[N\left(t\right)\right]=\sum_{n=1}^{\infty}F^{n\star}\left(t\right)
\end{eqnarray*}

\begin{Prop}
Para cada $t\geq0$, la funci\'on generadora de momentos $\esp\left[e^{\alpha N\left(t\right)}\right]$ existe para alguna $\alpha$ en una vecindad del 0, y de aqu\'i que $\esp\left[N\left(t\right)^{m}\right]<\infty$, para $m\geq1$.
\end{Prop}


\begin{Note}
Si el primer tiempo de renovaci\'on $\xi_{1}$ no tiene la misma distribuci\'on que el resto de las $\xi_{n}$, para $n\geq2$, a $N\left(t\right)$ se le llama Proceso de Renovaci\'on retardado, donde si $\xi$ tiene distribuci\'on $G$, entonces el tiempo $T_{n}$ de la $n$-\'esima renovaci\'on tiene distribuci\'on $G\star F^{\left(n-1\right)\star}\left(t\right)$
\end{Note}


\begin{Teo}
Para una constante $\mu\leq\infty$ ( o variable aleatoria), las siguientes expresiones son equivalentes:

\begin{eqnarray}
lim_{n\rightarrow\infty}n^{-1}T_{n}&=&\mu,\textrm{ c.s.}\\
lim_{t\rightarrow\infty}t^{-1}N\left(t\right)&=&1/\mu,\textrm{ c.s.}
\end{eqnarray}
\end{Teo}


Es decir, $T_{n}$ satisface la Ley Fuerte de los Grandes N\'umeros s\'i y s\'olo s\'i $N\left/t\right)$ la cumple.


\begin{Coro}[Ley Fuerte de los Grandes N\'umeros para Procesos de Renovaci\'on]
Si $N\left(t\right)$ es un proceso de renovaci\'on cuyos tiempos de inter-renovaci\'on tienen media $\mu\leq\infty$, entonces
\begin{eqnarray}
t^{-1}N\left(t\right)\rightarrow 1/\mu,\textrm{ c.s. cuando }t\rightarrow\infty.
\end{eqnarray}

\end{Coro}


Considerar el proceso estoc\'astico de valores reales $\left\{Z\left(t\right):t\geq0\right\}$ en el mismo espacio de probabilidad que $N\left(t\right)$

\begin{Def}
Para el proceso $\left\{Z\left(t\right):t\geq0\right\}$ se define la fluctuaci\'on m\'axima de $Z\left(t\right)$ en el intervalo $\left(T_{n-1},T_{n}\right]$:
\begin{eqnarray*}
M_{n}=\sup_{T_{n-1}<t\leq T_{n}}|Z\left(t\right)-Z\left(T_{n-1}\right)|
\end{eqnarray*}
\end{Def}

\begin{Teo}
Sup\'ongase que $n^{-1}T_{n}\rightarrow\mu$ c.s. cuando $n\rightarrow\infty$, donde $\mu\leq\infty$ es una constante o variable aleatoria. Sea $a$ una constante o variable aleatoria que puede ser infinita cuando $\mu$ es finita, y considere las expresiones l\'imite:
\begin{eqnarray}
lim_{n\rightarrow\infty}n^{-1}Z\left(T_{n}\right)&=&a,\textrm{ c.s.}\\
lim_{t\rightarrow\infty}t^{-1}Z\left(t\right)&=&a/\mu,\textrm{ c.s.}
\end{eqnarray}
La segunda expresi\'on implica la primera. Conversamente, la primera implica la segunda si el proceso $Z\left(t\right)$ es creciente, o si $lim_{n\rightarrow\infty}n^{-1}M_{n}=0$ c.s.
\end{Teo}

\begin{Coro}
Si $N\left(t\right)$ es un proceso de renovaci\'on, y $\left(Z\left(T_{n}\right)-Z\left(T_{n-1}\right),M_{n}\right)$, para $n\geq1$, son variables aleatorias independientes e id\'enticamente distribuidas con media finita, entonces,
\begin{eqnarray}
lim_{t\rightarrow\infty}t^{-1}Z\left(t\right)\rightarrow\frac{\esp\left[Z\left(T_{1}\right)-Z\left(T_{0}\right)\right]}{\esp\left[T_{1}\right]},\textrm{ c.s. cuando  }t\rightarrow\infty.
\end{eqnarray}
\end{Coro}




%__________________________________________________________________________________________
\subsection{Procesos Regenerativos Estacionarios - Stidham \cite{Stidham}}
%__________________________________________________________________________________________


Un proceso estoc\'astico a tiempo continuo $\left\{V\left(t\right),t\geq0\right\}$ es un proceso regenerativo si existe una sucesi\'on de variables aleatorias independientes e id\'enticamente distribuidas $\left\{X_{1},X_{2},\ldots\right\}$, sucesi\'on de renovaci\'on, tal que para cualquier conjunto de Borel $A$, 

\begin{eqnarray*}
\prob\left\{V\left(t\right)\in A|X_{1}+X_{2}+\cdots+X_{R\left(t\right)}=s,\left\{V\left(\tau\right),\tau<s\right\}\right\}=\prob\left\{V\left(t-s\right)\in A|X_{1}>t-s\right\},
\end{eqnarray*}
para todo $0\leq s\leq t$, donde $R\left(t\right)=\max\left\{X_{1}+X_{2}+\cdots+X_{j}\leq t\right\}=$n\'umero de renovaciones ({\emph{puntos de regeneraci\'on}}) que ocurren en $\left[0,t\right]$. El intervalo $\left[0,X_{1}\right)$ es llamado {\emph{primer ciclo de regeneraci\'on}} de $\left\{V\left(t \right),t\geq0\right\}$, $\left[X_{1},X_{1}+X_{2}\right)$ el {\emph{segundo ciclo de regeneraci\'on}}, y as\'i sucesivamente.

Sea $X=X_{1}$ y sea $F$ la funci\'on de distrbuci\'on de $X$


\begin{Def}
Se define el proceso estacionario, $\left\{V^{*}\left(t\right),t\geq0\right\}$, para $\left\{V\left(t\right),t\geq0\right\}$ por

\begin{eqnarray*}
\prob\left\{V\left(t\right)\in A\right\}=\frac{1}{\esp\left[X\right]}\int_{0}^{\infty}\prob\left\{V\left(t+x\right)\in A|X>x\right\}\left(1-F\left(x\right)\right)dx,
\end{eqnarray*} 
para todo $t\geq0$ y todo conjunto de Borel $A$.
\end{Def}

\begin{Def}
Una distribuci\'on se dice que es {\emph{aritm\'etica}} si todos sus puntos de incremento son m\'ultiplos de la forma $0,\lambda, 2\lambda,\ldots$ para alguna $\lambda>0$ entera.
\end{Def}


\begin{Def}
Una modificaci\'on medible de un proceso $\left\{V\left(t\right),t\geq0\right\}$, es una versi\'on de este, $\left\{V\left(t,w\right)\right\}$ conjuntamente medible para $t\geq0$ y para $w\in S$, $S$ espacio de estados para $\left\{V\left(t\right),t\geq0\right\}$.
\end{Def}

\begin{Teo}
Sea $\left\{V\left(t\right),t\geq\right\}$ un proceso regenerativo no negativo con modificaci\'on medible. Sea $\esp\left[X\right]<\infty$. Entonces el proceso estacionario dado por la ecuaci\'on anterior est\'a bien definido y tiene funci\'on de distribuci\'on independiente de $t$, adem\'as
\begin{itemize}
\item[i)] \begin{eqnarray*}
\esp\left[V^{*}\left(0\right)\right]&=&\frac{\esp\left[\int_{0}^{X}V\left(s\right)ds\right]}{\esp\left[X\right]}\end{eqnarray*}
\item[ii)] Si $\esp\left[V^{*}\left(0\right)\right]<\infty$, equivalentemente, si $\esp\left[\int_{0}^{X}V\left(s\right)ds\right]<\infty$,entonces
\begin{eqnarray*}
\frac{\int_{0}^{t}V\left(s\right)ds}{t}\rightarrow\frac{\esp\left[\int_{0}^{X}V\left(s\right)ds\right]}{\esp\left[X\right]}
\end{eqnarray*}
con probabilidad 1 y en media, cuando $t\rightarrow\infty$.
\end{itemize}
\end{Teo}

%______________________________________________________________________
\subsection{Procesos de Renovaci\'on}
%______________________________________________________________________

\begin{Def}\label{Def.Tn}
Sean $0\leq T_{1}\leq T_{2}\leq \ldots$ son tiempos aleatorios infinitos en los cuales ocurren ciertos eventos. El n\'umero de tiempos $T_{n}$ en el intervalo $\left[0,t\right)$ es

\begin{eqnarray}
N\left(t\right)=\sum_{n=1}^{\infty}\indora\left(T_{n}\leq t\right),
\end{eqnarray}
para $t\geq0$.
\end{Def}

Si se consideran los puntos $T_{n}$ como elementos de $\rea_{+}$, y $N\left(t\right)$ es el n\'umero de puntos en $\rea$. El proceso denotado por $\left\{N\left(t\right):t\geq0\right\}$, denotado por $N\left(t\right)$, es un proceso puntual en $\rea_{+}$. Los $T_{n}$ son los tiempos de ocurrencia, el proceso puntual $N\left(t\right)$ es simple si su n\'umero de ocurrencias son distintas: $0<T_{1}<T_{2}<\ldots$ casi seguramente.

\begin{Def}
Un proceso puntual $N\left(t\right)$ es un proceso de renovaci\'on si los tiempos de interocurrencia $\xi_{n}=T_{n}-T_{n-1}$, para $n\geq1$, son independientes e identicamente distribuidos con distribuci\'on $F$, donde $F\left(0\right)=0$ y $T_{0}=0$. Los $T_{n}$ son llamados tiempos de renovaci\'on, referente a la independencia o renovaci\'on de la informaci\'on estoc\'astica en estos tiempos. Los $\xi_{n}$ son los tiempos de inter-renovaci\'on, y $N\left(t\right)$ es el n\'umero de renovaciones en el intervalo $\left[0,t\right)$
\end{Def}


\begin{Note}
Para definir un proceso de renovaci\'on para cualquier contexto, solamente hay que especificar una distribuci\'on $F$, con $F\left(0\right)=0$, para los tiempos de inter-renovaci\'on. La funci\'on $F$ en turno degune las otra variables aleatorias. De manera formal, existe un espacio de probabilidad y una sucesi\'on de variables aleatorias $\xi_{1},\xi_{2},\ldots$ definidas en este con distribuci\'on $F$. Entonces las otras cantidades son $T_{n}=\sum_{k=1}^{n}\xi_{k}$ y $N\left(t\right)=\sum_{n=1}^{\infty}\indora\left(T_{n}\leq t\right)$, donde $T_{n}\rightarrow\infty$ casi seguramente por la Ley Fuerte de los Grandes Números.
\end{Note}

%___________________________________________________________________________________________
%
\subsection{Teorema Principal de Renovaci\'on}
%___________________________________________________________________________________________
%

\begin{Note} Una funci\'on $h:\rea_{+}\rightarrow\rea$ es Directamente Riemann Integrable en los siguientes casos:
\begin{itemize}
\item[a)] $h\left(t\right)\geq0$ es decreciente y Riemann Integrable.
\item[b)] $h$ es continua excepto posiblemente en un conjunto de Lebesgue de medida 0, y $|h\left(t\right)|\leq b\left(t\right)$, donde $b$ es DRI.
\end{itemize}
\end{Note}

\begin{Teo}[Teorema Principal de Renovaci\'on]
Si $F$ es no aritm\'etica y $h\left(t\right)$ es Directamente Riemann Integrable (DRI), entonces

\begin{eqnarray*}
lim_{t\rightarrow\infty}U\star h=\frac{1}{\mu}\int_{\rea_{+}}h\left(s\right)ds.
\end{eqnarray*}
\end{Teo}

\begin{Prop}
Cualquier funci\'on $H\left(t\right)$ acotada en intervalos finitos y que es 0 para $t<0$ puede expresarse como
\begin{eqnarray*}
H\left(t\right)=U\star h\left(t\right)\textrm{,  donde }h\left(t\right)=H\left(t\right)-F\star H\left(t\right)
\end{eqnarray*}
\end{Prop}

\begin{Def}
Un proceso estoc\'astico $X\left(t\right)$ es crudamente regenerativo en un tiempo aleatorio positivo $T$ si
\begin{eqnarray*}
\esp\left[X\left(T+t\right)|T\right]=\esp\left[X\left(t\right)\right]\textrm{, para }t\geq0,\end{eqnarray*}
y con las esperanzas anteriores finitas.
\end{Def}

\begin{Prop}
Sup\'ongase que $X\left(t\right)$ es un proceso crudamente regenerativo en $T$, que tiene distribuci\'on $F$. Si $\esp\left[X\left(t\right)\right]$ es acotado en intervalos finitos, entonces
\begin{eqnarray*}
\esp\left[X\left(t\right)\right]=U\star h\left(t\right)\textrm{,  donde }h\left(t\right)=\esp\left[X\left(t\right)\indora\left(T>t\right)\right].
\end{eqnarray*}
\end{Prop}

\begin{Teo}[Regeneraci\'on Cruda]
Sup\'ongase que $X\left(t\right)$ es un proceso con valores positivo crudamente regenerativo en $T$, y def\'inase $M=\sup\left\{|X\left(t\right)|:t\leq T\right\}$. Si $T$ es no aritm\'etico y $M$ y $MT$ tienen media finita, entonces
\begin{eqnarray*}
lim_{t\rightarrow\infty}\esp\left[X\left(t\right)\right]=\frac{1}{\mu}\int_{\rea_{+}}h\left(s\right)ds,
\end{eqnarray*}
donde $h\left(t\right)=\esp\left[X\left(t\right)\indora\left(T>t\right)\right]$.
\end{Teo}



%___________________________________________________________________________________________
%
\subsection{Funci\'on de Renovaci\'on}
%___________________________________________________________________________________________
%


\begin{Def}
Sea $h\left(t\right)$ funci\'on de valores reales en $\rea$ acotada en intervalos finitos e igual a cero para $t<0$ La ecuaci\'on de renovaci\'on para $h\left(t\right)$ y la distribuci\'on $F$ es

\begin{eqnarray}\label{Ec.Renovacion}
H\left(t\right)=h\left(t\right)+\int_{\left[0,t\right]}H\left(t-s\right)dF\left(s\right)\textrm{,    }t\geq0,
\end{eqnarray}
donde $H\left(t\right)$ es una funci\'on de valores reales. Esto es $H=h+F\star H$. Decimos que $H\left(t\right)$ es soluci\'on de esta ecuaci\'on si satisface la ecuaci\'on, y es acotada en intervalos finitos e iguales a cero para $t<0$.
\end{Def}

\begin{Prop}
La funci\'on $U\star h\left(t\right)$ es la \'unica soluci\'on de la ecuaci\'on de renovaci\'on (\ref{Ec.Renovacion}).
\end{Prop}

\begin{Teo}[Teorema Renovaci\'on Elemental]
\begin{eqnarray*}
t^{-1}U\left(t\right)\rightarrow 1/\mu\textrm{,    cuando }t\rightarrow\infty.
\end{eqnarray*}
\end{Teo}

%___________________________________________________________________________________________
%
\subsection{Propiedades de los Procesos de Renovaci\'on}
%___________________________________________________________________________________________
%

Los tiempos $T_{n}$ est\'an relacionados con los conteos de $N\left(t\right)$ por

\begin{eqnarray*}
\left\{N\left(t\right)\geq n\right\}&=&\left\{T_{n}\leq t\right\}\\
T_{N\left(t\right)}\leq &t&<T_{N\left(t\right)+1},
\end{eqnarray*}

adem\'as $N\left(T_{n}\right)=n$, y 

\begin{eqnarray*}
N\left(t\right)=\max\left\{n:T_{n}\leq t\right\}=\min\left\{n:T_{n+1}>t\right\}
\end{eqnarray*}

Por propiedades de la convoluci\'on se sabe que

\begin{eqnarray*}
P\left\{T_{n}\leq t\right\}=F^{n\star}\left(t\right)
\end{eqnarray*}
que es la $n$-\'esima convoluci\'on de $F$. Entonces 

\begin{eqnarray*}
\left\{N\left(t\right)\geq n\right\}&=&\left\{T_{n}\leq t\right\}\\
P\left\{N\left(t\right)\leq n\right\}&=&1-F^{\left(n+1\right)\star}\left(t\right)
\end{eqnarray*}

Adem\'as usando el hecho de que $\esp\left[N\left(t\right)\right]=\sum_{n=1}^{\infty}P\left\{N\left(t\right)\geq n\right\}$
se tiene que

\begin{eqnarray*}
\esp\left[N\left(t\right)\right]=\sum_{n=1}^{\infty}F^{n\star}\left(t\right)
\end{eqnarray*}

\begin{Prop}
Para cada $t\geq0$, la funci\'on generadora de momentos $\esp\left[e^{\alpha N\left(t\right)}\right]$ existe para alguna $\alpha$ en una vecindad del 0, y de aqu\'i que $\esp\left[N\left(t\right)^{m}\right]<\infty$, para $m\geq1$.
\end{Prop}


\begin{Note}
Si el primer tiempo de renovaci\'on $\xi_{1}$ no tiene la misma distribuci\'on que el resto de las $\xi_{n}$, para $n\geq2$, a $N\left(t\right)$ se le llama Proceso de Renovaci\'on retardado, donde si $\xi$ tiene distribuci\'on $G$, entonces el tiempo $T_{n}$ de la $n$-\'esima renovaci\'on tiene distribuci\'on $G\star F^{\left(n-1\right)\star}\left(t\right)$
\end{Note}


\begin{Teo}
Para una constante $\mu\leq\infty$ ( o variable aleatoria), las siguientes expresiones son equivalentes:

\begin{eqnarray}
lim_{n\rightarrow\infty}n^{-1}T_{n}&=&\mu,\textrm{ c.s.}\\
lim_{t\rightarrow\infty}t^{-1}N\left(t\right)&=&1/\mu,\textrm{ c.s.}
\end{eqnarray}
\end{Teo}


Es decir, $T_{n}$ satisface la Ley Fuerte de los Grandes N\'umeros s\'i y s\'olo s\'i $N\left/t\right)$ la cumple.


\begin{Coro}[Ley Fuerte de los Grandes N\'umeros para Procesos de Renovaci\'on]
Si $N\left(t\right)$ es un proceso de renovaci\'on cuyos tiempos de inter-renovaci\'on tienen media $\mu\leq\infty$, entonces
\begin{eqnarray}
t^{-1}N\left(t\right)\rightarrow 1/\mu,\textrm{ c.s. cuando }t\rightarrow\infty.
\end{eqnarray}

\end{Coro}


Considerar el proceso estoc\'astico de valores reales $\left\{Z\left(t\right):t\geq0\right\}$ en el mismo espacio de probabilidad que $N\left(t\right)$

\begin{Def}
Para el proceso $\left\{Z\left(t\right):t\geq0\right\}$ se define la fluctuaci\'on m\'axima de $Z\left(t\right)$ en el intervalo $\left(T_{n-1},T_{n}\right]$:
\begin{eqnarray*}
M_{n}=\sup_{T_{n-1}<t\leq T_{n}}|Z\left(t\right)-Z\left(T_{n-1}\right)|
\end{eqnarray*}
\end{Def}

\begin{Teo}
Sup\'ongase que $n^{-1}T_{n}\rightarrow\mu$ c.s. cuando $n\rightarrow\infty$, donde $\mu\leq\infty$ es una constante o variable aleatoria. Sea $a$ una constante o variable aleatoria que puede ser infinita cuando $\mu$ es finita, y considere las expresiones l\'imite:
\begin{eqnarray}
lim_{n\rightarrow\infty}n^{-1}Z\left(T_{n}\right)&=&a,\textrm{ c.s.}\\
lim_{t\rightarrow\infty}t^{-1}Z\left(t\right)&=&a/\mu,\textrm{ c.s.}
\end{eqnarray}
La segunda expresi\'on implica la primera. Conversamente, la primera implica la segunda si el proceso $Z\left(t\right)$ es creciente, o si $lim_{n\rightarrow\infty}n^{-1}M_{n}=0$ c.s.
\end{Teo}

\begin{Coro}
Si $N\left(t\right)$ es un proceso de renovaci\'on, y $\left(Z\left(T_{n}\right)-Z\left(T_{n-1}\right),M_{n}\right)$, para $n\geq1$, son variables aleatorias independientes e id\'enticamente distribuidas con media finita, entonces,
\begin{eqnarray}
lim_{t\rightarrow\infty}t^{-1}Z\left(t\right)\rightarrow\frac{\esp\left[Z\left(T_{1}\right)-Z\left(T_{0}\right)\right]}{\esp\left[T_{1}\right]},\textrm{ c.s. cuando  }t\rightarrow\infty.
\end{eqnarray}
\end{Coro}

%___________________________________________________________________________________________
%
\subsection{Funci\'on de Renovaci\'on}
%___________________________________________________________________________________________
%


Sup\'ongase que $N\left(t\right)$ es un proceso de renovaci\'on con distribuci\'on $F$ con media finita $\mu$.

\begin{Def}
La funci\'on de renovaci\'on asociada con la distribuci\'on $F$, del proceso $N\left(t\right)$, es
\begin{eqnarray*}
U\left(t\right)=\sum_{n=1}^{\infty}F^{n\star}\left(t\right),\textrm{   }t\geq0,
\end{eqnarray*}
donde $F^{0\star}\left(t\right)=\indora\left(t\geq0\right)$.
\end{Def}


\begin{Prop}
Sup\'ongase que la distribuci\'on de inter-renovaci\'on $F$ tiene densidad $f$. Entonces $U\left(t\right)$ tambi\'en tiene densidad, para $t>0$, y es $U^{'}\left(t\right)=\sum_{n=0}^{\infty}f^{n\star}\left(t\right)$. Adem\'as
\begin{eqnarray*}
\prob\left\{N\left(t\right)>N\left(t-\right)\right\}=0\textrm{,   }t\geq0.
\end{eqnarray*}
\end{Prop}

\begin{Def}
La Transformada de Laplace-Stieljes de $F$ est\'a dada por

\begin{eqnarray*}
\hat{F}\left(\alpha\right)=\int_{\rea_{+}}e^{-\alpha t}dF\left(t\right)\textrm{,  }\alpha\geq0.
\end{eqnarray*}
\end{Def}

Entonces

\begin{eqnarray*}
\hat{U}\left(\alpha\right)=\sum_{n=0}^{\infty}\hat{F^{n\star}}\left(\alpha\right)=\sum_{n=0}^{\infty}\hat{F}\left(\alpha\right)^{n}=\frac{1}{1-\hat{F}\left(\alpha\right)}.
\end{eqnarray*}


\begin{Prop}
La Transformada de Laplace $\hat{U}\left(\alpha\right)$ y $\hat{F}\left(\alpha\right)$ determina una a la otra de manera \'unica por la relaci\'on $\hat{U}\left(\alpha\right)=\frac{1}{1-\hat{F}\left(\alpha\right)}$.
\end{Prop}


\begin{Note}
Un proceso de renovaci\'on $N\left(t\right)$ cuyos tiempos de inter-renovaci\'on tienen media finita, es un proceso Poisson con tasa $\lambda$ si y s\'olo s\'i $\esp\left[U\left(t\right)\right]=\lambda t$, para $t\geq0$.
\end{Note}


\begin{Teo}
Sea $N\left(t\right)$ un proceso puntual simple con puntos de localizaci\'on $T_{n}$ tal que $\eta\left(t\right)=\esp\left[N\left(\right)\right]$ es finita para cada $t$. Entonces para cualquier funci\'on $f:\rea_{+}\rightarrow\rea$,
\begin{eqnarray*}
\esp\left[\sum_{n=1}^{N\left(\right)}f\left(T_{n}\right)\right]=\int_{\left(0,t\right]}f\left(s\right)d\eta\left(s\right)\textrm{,  }t\geq0,
\end{eqnarray*}
suponiendo que la integral exista. Adem\'as si $X_{1},X_{2},\ldots$ son variables aleatorias definidas en el mismo espacio de probabilidad que el proceso $N\left(t\right)$ tal que $\esp\left[X_{n}|T_{n}=s\right]=f\left(s\right)$, independiente de $n$. Entonces
\begin{eqnarray*}
\esp\left[\sum_{n=1}^{N\left(t\right)}X_{n}\right]=\int_{\left(0,t\right]}f\left(s\right)d\eta\left(s\right)\textrm{,  }t\geq0,
\end{eqnarray*} 
suponiendo que la integral exista. 
\end{Teo}

\begin{Coro}[Identidad de Wald para Renovaciones]
Para el proceso de renovaci\'on $N\left(t\right)$,
\begin{eqnarray*}
\esp\left[T_{N\left(t\right)+1}\right]=\mu\esp\left[N\left(t\right)+1\right]\textrm{,  }t\geq0,
\end{eqnarray*}  
\end{Coro}

%______________________________________________________________________
\subsection{Procesos de Renovaci\'on}
%______________________________________________________________________

\begin{Def}\label{Def.Tn}
Sean $0\leq T_{1}\leq T_{2}\leq \ldots$ son tiempos aleatorios infinitos en los cuales ocurren ciertos eventos. El n\'umero de tiempos $T_{n}$ en el intervalo $\left[0,t\right)$ es

\begin{eqnarray}
N\left(t\right)=\sum_{n=1}^{\infty}\indora\left(T_{n}\leq t\right),
\end{eqnarray}
para $t\geq0$.
\end{Def}

Si se consideran los puntos $T_{n}$ como elementos de $\rea_{+}$, y $N\left(t\right)$ es el n\'umero de puntos en $\rea$. El proceso denotado por $\left\{N\left(t\right):t\geq0\right\}$, denotado por $N\left(t\right)$, es un proceso puntual en $\rea_{+}$. Los $T_{n}$ son los tiempos de ocurrencia, el proceso puntual $N\left(t\right)$ es simple si su n\'umero de ocurrencias son distintas: $0<T_{1}<T_{2}<\ldots$ casi seguramente.

\begin{Def}
Un proceso puntual $N\left(t\right)$ es un proceso de renovaci\'on si los tiempos de interocurrencia $\xi_{n}=T_{n}-T_{n-1}$, para $n\geq1$, son independientes e identicamente distribuidos con distribuci\'on $F$, donde $F\left(0\right)=0$ y $T_{0}=0$. Los $T_{n}$ son llamados tiempos de renovaci\'on, referente a la independencia o renovaci\'on de la informaci\'on estoc\'astica en estos tiempos. Los $\xi_{n}$ son los tiempos de inter-renovaci\'on, y $N\left(t\right)$ es el n\'umero de renovaciones en el intervalo $\left[0,t\right)$
\end{Def}


\begin{Note}
Para definir un proceso de renovaci\'on para cualquier contexto, solamente hay que especificar una distribuci\'on $F$, con $F\left(0\right)=0$, para los tiempos de inter-renovaci\'on. La funci\'on $F$ en turno degune las otra variables aleatorias. De manera formal, existe un espacio de probabilidad y una sucesi\'on de variables aleatorias $\xi_{1},\xi_{2},\ldots$ definidas en este con distribuci\'on $F$. Entonces las otras cantidades son $T_{n}=\sum_{k=1}^{n}\xi_{k}$ y $N\left(t\right)=\sum_{n=1}^{\infty}\indora\left(T_{n}\leq t\right)$, donde $T_{n}\rightarrow\infty$ casi seguramente por la Ley Fuerte de los Grandes Números.
\end{Note}
%_____________________________________________________
\subsection{Puntos de Renovaci\'on}
%_____________________________________________________

Para cada cola $Q_{i}$ se tienen los procesos de arribo a la cola, para estas, los tiempos de arribo est\'an dados por $$\left\{T_{1}^{i},T_{2}^{i},\ldots,T_{k}^{i},\ldots\right\},$$ entonces, consideremos solamente los primeros tiempos de arribo a cada una de las colas, es decir, $$\left\{T_{1}^{1},T_{1}^{2},T_{1}^{3},T_{1}^{4}\right\},$$ se sabe que cada uno de estos tiempos se distribuye de manera exponencial con par\'ametro $1/mu_{i}$. Adem\'as se sabe que para $$T^{*}=\min\left\{T_{1}^{1},T_{1}^{2},T_{1}^{3},T_{1}^{4}\right\},$$ $T^{*}$ se distribuye de manera exponencial con par\'ametro $$\mu^{*}=\sum_{i=1}^{4}\mu_{i}.$$ Ahora, dado que 
\begin{center}
\begin{tabular}{lcl}
$\tilde{r}=r_{1}+r_{2}$ & y &$\hat{r}=r_{3}+r_{4}.$
\end{tabular}
\end{center}


Supongamos que $$\tilde{r},\hat{r}<\mu^{*},$$ entonces si tomamos $$r^{*}=\min\left\{\tilde{r},\hat{r}\right\},$$ se tiene que para  $$t^{*}\in\left(0,r^{*}\right)$$ se cumple que 
\begin{center}
\begin{tabular}{lcl}
$\tau_{1}\left(1\right)=0$ & y por tanto & $\overline{\tau}_{1}=0,$
\end{tabular}
\end{center}
entonces para la segunda cola en este primer ciclo se cumple que $$\tau_{2}=\overline{\tau}_{1}+r_{1}=r_{1}<\mu^{*},$$ y por tanto se tiene que  $$\overline{\tau}_{2}=\tau_{2}.$$ Por lo tanto, nuevamente para la primer cola en el segundo ciclo $$\tau_{1}\left(2\right)=\tau_{2}\left(1\right)+r_{2}=\tilde{r}<\mu^{*}.$$ An\'alogamente para el segundo sistema se tiene que ambas colas est\'an vac\'ias, es decir, existe un valor $t^{*}$ tal que en el intervalo $\left(0,t^{*}\right)$ no ha llegado ning\'un usuario, es decir, $$L_{i}\left(t^{*}\right)=0$$ para $i=1,2,3,4$.

\subsection{Resultados para Procesos de Salida}

En \cite{Sigman2} prueban que para la existencia de un una sucesi\'on infinita no decreciente de tiempos de regeneraci\'on $\tau_{1}\leq\tau_{2}\leq\cdots$ en los cuales el proceso se regenera, basta un tiempo de regeneraci\'on $R_{1}$, donde $R_{j}=\tau_{j}-\tau_{j-1}$. Para tal efecto se requiere la existencia de un espacio de probabilidad $\left(\Omega,\mathcal{F},\prob\right)$, y proceso estoc\'astico $\textit{X}=\left\{X\left(t\right):t\geq0\right\}$ con espacio de estados $\left(S,\mathcal{R}\right)$, con $\mathcal{R}$ $\sigma$-\'algebra.

\begin{Prop}
Si existe una variable aleatoria no negativa $R_{1}$ tal que $\theta_{R\footnotesize{1}}X=_{D}X$, entonces $\left(\Omega,\mathcal{F},\prob\right)$ puede extenderse para soportar una sucesi\'on estacionaria de variables aleatorias $R=\left\{R_{k}:k\geq1\right\}$, tal que para $k\geq1$,
\begin{eqnarray*}
\theta_{k}\left(X,R\right)=_{D}\left(X,R\right).
\end{eqnarray*}

Adem\'as, para $k\geq1$, $\theta_{k}R$ es condicionalmente independiente de $\left(X,R_{1},\ldots,R_{k}\right)$, dado $\theta_{\tau k}X$.

\end{Prop}


\begin{itemize}
\item Doob en 1953 demostr\'o que el estado estacionario de un proceso de partida en un sistema de espera $M/G/\infty$, es Poisson con la misma tasa que el proceso de arribos.

\item Burke en 1968, fue el primero en demostrar que el estado estacionario de un proceso de salida de una cola $M/M/s$ es un proceso Poisson.

\item Disney en 1973 obtuvo el siguiente resultado:

\begin{Teo}
Para el sistema de espera $M/G/1/L$ con disciplina FIFO, el proceso $\textbf{I}$ es un proceso de renovaci\'on si y s\'olo si el proceso denominado longitud de la cola es estacionario y se cumple cualquiera de los siguientes casos:

\begin{itemize}
\item[a)] Los tiempos de servicio son identicamente cero;
\item[b)] $L=0$, para cualquier proceso de servicio $S$;
\item[c)] $L=1$ y $G=D$;
\item[d)] $L=\infty$ y $G=M$.
\end{itemize}
En estos casos, respectivamente, las distribuciones de interpartida $P\left\{T_{n+1}-T_{n}\leq t\right\}$ son


\begin{itemize}
\item[a)] $1-e^{-\lambda t}$, $t\geq0$;
\item[b)] $1-e^{-\lambda t}*F\left(t\right)$, $t\geq0$;
\item[c)] $1-e^{-\lambda t}*\indora_{d}\left(t\right)$, $t\geq0$;
\item[d)] $1-e^{-\lambda t}*F\left(t\right)$, $t\geq0$.
\end{itemize}
\end{Teo}


\item Finch (1959) mostr\'o que para los sistemas $M/G/1/L$, con $1\leq L\leq \infty$ con distribuciones de servicio dos veces diferenciable, solamente el sistema $M/M/1/\infty$ tiene proceso de salida de renovaci\'on estacionario.

\item King (1971) demostro que un sistema de colas estacionario $M/G/1/1$ tiene sus tiempos de interpartida sucesivas $D_{n}$ y $D_{n+1}$ son independientes, si y s\'olo si, $G=D$, en cuyo caso le proceso de salida es de renovaci\'on.

\item Disney (1973) demostr\'o que el \'unico sistema estacionario $M/G/1/L$, que tiene proceso de salida de renovaci\'on  son los sistemas $M/M/1$ y $M/D/1/1$.



\item El siguiente resultado es de Disney y Koning (1985)
\begin{Teo}
En un sistema de espera $M/G/s$, el estado estacionario del proceso de salida es un proceso Poisson para cualquier distribuci\'on de los tiempos de servicio si el sistema tiene cualquiera de las siguientes cuatro propiedades.

\begin{itemize}
\item[a)] $s=\infty$
\item[b)] La disciplina de servicio es de procesador compartido.
\item[c)] La disciplina de servicio es LCFS y preemptive resume, esto se cumple para $L<\infty$
\item[d)] $G=M$.
\end{itemize}

\end{Teo}

\item El siguiente resultado es de Alamatsaz (1983)

\begin{Teo}
En cualquier sistema de colas $GI/G/1/L$ con $1\leq L<\infty$ y distribuci\'on de interarribos $A$ y distribuci\'on de los tiempos de servicio $B$, tal que $A\left(0\right)=0$, $A\left(t\right)\left(1-B\left(t\right)\right)>0$ para alguna $t>0$ y $B\left(t\right)$ para toda $t>0$, es imposible que el proceso de salida estacionario sea de renovaci\'on.
\end{Teo}

\end{itemize}

Estos resultados aparecen en Daley (1968) \cite{Daley68} para $\left\{T_{n}\right\}$ intervalos de inter-arribo, $\left\{D_{n}\right\}$ intervalos de inter-salida y $\left\{S_{n}\right\}$ tiempos de servicio.

\begin{itemize}
\item Si el proceso $\left\{T_{n}\right\}$ es Poisson, el proceso $\left\{D_{n}\right\}$ es no correlacionado si y s\'olo si es un proceso Poisso, lo cual ocurre si y s\'olo si $\left\{S_{n}\right\}$ son exponenciales negativas.

\item Si $\left\{S_{n}\right\}$ son exponenciales negativas, $\left\{D_{n}\right\}$ es un proceso de renovaci\'on  si y s\'olo si es un proceso Poisson, lo cual ocurre si y s\'olo si $\left\{T_{n}\right\}$ es un proceso Poisson.

\item $\esp\left(D_{n}\right)=\esp\left(T_{n}\right)$.

\item Para un sistema de visitas $GI/M/1$ se tiene el siguiente teorema:

\begin{Teo}
En un sistema estacionario $GI/M/1$ los intervalos de interpartida tienen
\begin{eqnarray*}
\esp\left(e^{-\theta D_{n}}\right)&=&\mu\left(\mu+\theta\right)^{-1}\left[\delta\theta
-\mu\left(1-\delta\right)\alpha\left(\theta\right)\right]
\left[\theta-\mu\left(1-\delta\right)^{-1}\right]\\
\alpha\left(\theta\right)&=&\esp\left[e^{-\theta T_{0}}\right]\\
var\left(D_{n}\right)&=&var\left(T_{0}\right)-\left(\tau^{-1}-\delta^{-1}\right)
2\delta\left(\esp\left(S_{0}\right)\right)^{2}\left(1-\delta\right)^{-1}.
\end{eqnarray*}
\end{Teo}



\begin{Teo}
El proceso de salida de un sistema de colas estacionario $GI/M/1$ es un proceso de renovaci\'on si y s\'olo si el proceso de entrada es un proceso Poisson, en cuyo caso el proceso de salida es un proceso Poisson.
\end{Teo}


\begin{Teo}
Los intervalos de interpartida $\left\{D_{n}\right\}$ de un sistema $M/G/1$ estacionario son no correlacionados si y s\'olo si la distribuci\'on de los tiempos de servicio es exponencial negativa, es decir, el sistema es de tipo  $M/M/1$.

\end{Teo}



\end{itemize}



%________________________________________________________________________
%\subsection{Procesos Regenerativos Sigman, Thorisson y Wolff \cite{Sigman1}}
%________________________________________________________________________


\begin{Def}[Definici\'on Cl\'asica]
Un proceso estoc\'astico $X=\left\{X\left(t\right):t\geq0\right\}$ es llamado regenerativo is existe una variable aleatoria $R_{1}>0$ tal que
\begin{itemize}
\item[i)] $\left\{X\left(t+R_{1}\right):t\geq0\right\}$ es independiente de $\left\{\left\{X\left(t\right):t<R_{1}\right\},\right\}$
\item[ii)] $\left\{X\left(t+R_{1}\right):t\geq0\right\}$ es estoc\'asticamente equivalente a $\left\{X\left(t\right):t>0\right\}$
\end{itemize}

Llamamos a $R_{1}$ tiempo de regeneraci\'on, y decimos que $X$ se regenera en este punto.
\end{Def}

$\left\{X\left(t+R_{1}\right)\right\}$ es regenerativo con tiempo de regeneraci\'on $R_{2}$, independiente de $R_{1}$ pero con la misma distribuci\'on que $R_{1}$. Procediendo de esta manera se obtiene una secuencia de variables aleatorias independientes e id\'enticamente distribuidas $\left\{R_{n}\right\}$ llamados longitudes de ciclo. Si definimos a $Z_{k}\equiv R_{1}+R_{2}+\cdots+R_{k}$, se tiene un proceso de renovaci\'on llamado proceso de renovaci\'on encajado para $X$.


\begin{Note}
La existencia de un primer tiempo de regeneraci\'on, $R_{1}$, implica la existencia de una sucesi\'on completa de estos tiempos $R_{1},R_{2}\ldots,$ que satisfacen la propiedad deseada \cite{Sigman2}.
\end{Note}


\begin{Note} Para la cola $GI/GI/1$ los usuarios arriban con tiempos $t_{n}$ y son atendidos con tiempos de servicio $S_{n}$, los tiempos de arribo forman un proceso de renovaci\'on  con tiempos entre arribos independientes e identicamente distribuidos (\texttt{i.i.d.})$T_{n}=t_{n}-t_{n-1}$, adem\'as los tiempos de servicio son \texttt{i.i.d.} e independientes de los procesos de arribo. Por \textit{estable} se entiende que $\esp S_{n}<\esp T_{n}<\infty$.
\end{Note}
 


\begin{Def}
Para $x$ fijo y para cada $t\geq0$, sea $I_{x}\left(t\right)=1$ si $X\left(t\right)\leq x$,  $I_{x}\left(t\right)=0$ en caso contrario, y def\'inanse los tiempos promedio
\begin{eqnarray*}
\overline{X}&=&lim_{t\rightarrow\infty}\frac{1}{t}\int_{0}^{\infty}X\left(u\right)du\\
\prob\left(X_{\infty}\leq x\right)&=&lim_{t\rightarrow\infty}\frac{1}{t}\int_{0}^{\infty}I_{x}\left(u\right)du,
\end{eqnarray*}
cuando estos l\'imites existan.
\end{Def}

Como consecuencia del teorema de Renovaci\'on-Recompensa, se tiene que el primer l\'imite  existe y es igual a la constante
\begin{eqnarray*}
\overline{X}&=&\frac{\esp\left[\int_{0}^{R_{1}}X\left(t\right)dt\right]}{\esp\left[R_{1}\right]},
\end{eqnarray*}
suponiendo que ambas esperanzas son finitas.
 
\begin{Note}
Funciones de procesos regenerativos son regenerativas, es decir, si $X\left(t\right)$ es regenerativo y se define el proceso $Y\left(t\right)$ por $Y\left(t\right)=f\left(X\left(t\right)\right)$ para alguna funci\'on Borel medible $f\left(\cdot\right)$. Adem\'as $Y$ es regenerativo con los mismos tiempos de renovaci\'on que $X$. 

En general, los tiempos de renovaci\'on, $Z_{k}$ de un proceso regenerativo no requieren ser tiempos de paro con respecto a la evoluci\'on de $X\left(t\right)$.
\end{Note} 

\begin{Note}
Una funci\'on de un proceso de Markov, usualmente no ser\'a un proceso de Markov, sin embargo ser\'a regenerativo si el proceso de Markov lo es.
\end{Note}

 
\begin{Note}
Un proceso regenerativo con media de la longitud de ciclo finita es llamado positivo recurrente.
\end{Note}


\begin{Note}
\begin{itemize}
\item[a)] Si el proceso regenerativo $X$ es positivo recurrente y tiene trayectorias muestrales no negativas, entonces la ecuaci\'on anterior es v\'alida.
\item[b)] Si $X$ es positivo recurrente regenerativo, podemos construir una \'unica versi\'on estacionaria de este proceso, $X_{e}=\left\{X_{e}\left(t\right)\right\}$, donde $X_{e}$ es un proceso estoc\'astico regenerativo y estrictamente estacionario, con distribuci\'on marginal distribuida como $X_{\infty}$
\end{itemize}
\end{Note}


%__________________________________________________________________________________________
%\subsection{Procesos Regenerativos Estacionarios - Stidham \cite{Stidham}}
%__________________________________________________________________________________________


Un proceso estoc\'astico a tiempo continuo $\left\{V\left(t\right),t\geq0\right\}$ es un proceso regenerativo si existe una sucesi\'on de variables aleatorias independientes e id\'enticamente distribuidas $\left\{X_{1},X_{2},\ldots\right\}$, sucesi\'on de renovaci\'on, tal que para cualquier conjunto de Borel $A$, 

\begin{eqnarray*}
\prob\left\{V\left(t\right)\in A|X_{1}+X_{2}+\cdots+X_{R\left(t\right)}=s,\left\{V\left(\tau\right),\tau<s\right\}\right\}=\prob\left\{V\left(t-s\right)\in A|X_{1}>t-s\right\},
\end{eqnarray*}
para todo $0\leq s\leq t$, donde $R\left(t\right)=\max\left\{X_{1}+X_{2}+\cdots+X_{j}\leq t\right\}=$n\'umero de renovaciones ({\emph{puntos de regeneraci\'on}}) que ocurren en $\left[0,t\right]$. El intervalo $\left[0,X_{1}\right)$ es llamado {\emph{primer ciclo de regeneraci\'on}} de $\left\{V\left(t \right),t\geq0\right\}$, $\left[X_{1},X_{1}+X_{2}\right)$ el {\emph{segundo ciclo de regeneraci\'on}}, y as\'i sucesivamente.

Sea $X=X_{1}$ y sea $F$ la funci\'on de distrbuci\'on de $X$


\begin{Def}
Se define el proceso estacionario, $\left\{V^{*}\left(t\right),t\geq0\right\}$, para $\left\{V\left(t\right),t\geq0\right\}$ por

\begin{eqnarray*}
\prob\left\{V\left(t\right)\in A\right\}=\frac{1}{\esp\left[X\right]}\int_{0}^{\infty}\prob\left\{V\left(t+x\right)\in A|X>x\right\}\left(1-F\left(x\right)\right)dx,
\end{eqnarray*} 
para todo $t\geq0$ y todo conjunto de Borel $A$.
\end{Def}

\begin{Def}
Una distribuci\'on se dice que es {\emph{aritm\'etica}} si todos sus puntos de incremento son m\'ultiplos de la forma $0,\lambda, 2\lambda,\ldots$ para alguna $\lambda>0$ entera.
\end{Def}


\begin{Def}
Una modificaci\'on medible de un proceso $\left\{V\left(t\right),t\geq0\right\}$, es una versi\'on de este, $\left\{V\left(t,w\right)\right\}$ conjuntamente medible para $t\geq0$ y para $w\in S$, $S$ espacio de estados para $\left\{V\left(t\right),t\geq0\right\}$.
\end{Def}

\begin{Teo}
Sea $\left\{V\left(t\right),t\geq\right\}$ un proceso regenerativo no negativo con modificaci\'on medible. Sea $\esp\left[X\right]<\infty$. Entonces el proceso estacionario dado por la ecuaci\'on anterior est\'a bien definido y tiene funci\'on de distribuci\'on independiente de $t$, adem\'as
\begin{itemize}
\item[i)] \begin{eqnarray*}
\esp\left[V^{*}\left(0\right)\right]&=&\frac{\esp\left[\int_{0}^{X}V\left(s\right)ds\right]}{\esp\left[X\right]}\end{eqnarray*}
\item[ii)] Si $\esp\left[V^{*}\left(0\right)\right]<\infty$, equivalentemente, si $\esp\left[\int_{0}^{X}V\left(s\right)ds\right]<\infty$,entonces
\begin{eqnarray*}
\frac{\int_{0}^{t}V\left(s\right)ds}{t}\rightarrow\frac{\esp\left[\int_{0}^{X}V\left(s\right)ds\right]}{\esp\left[X\right]}
\end{eqnarray*}
con probabilidad 1 y en media, cuando $t\rightarrow\infty$.
\end{itemize}
\end{Teo}

\begin{Coro}
Sea $\left\{V\left(t\right),t\geq0\right\}$ un proceso regenerativo no negativo, con modificaci\'on medible. Si $\esp <\infty$, $F$ es no-aritm\'etica, y para todo $x\geq0$, $P\left\{V\left(t\right)\leq x,C>x\right\}$ es de variaci\'on acotada como funci\'on de $t$ en cada intervalo finito $\left[0,\tau\right]$, entonces $V\left(t\right)$ converge en distribuci\'on  cuando $t\rightarrow\infty$ y $$\esp V=\frac{\esp \int_{0}^{X}V\left(s\right)ds}{\esp X}$$
Donde $V$ tiene la distribuci\'on l\'imite de $V\left(t\right)$ cuando $t\rightarrow\infty$.

\end{Coro}

Para el caso discreto se tienen resultados similares.



%______________________________________________________________________
%\subsection{Procesos de Renovaci\'on}
%______________________________________________________________________

\begin{Def}%\label{Def.Tn}
Sean $0\leq T_{1}\leq T_{2}\leq \ldots$ son tiempos aleatorios infinitos en los cuales ocurren ciertos eventos. El n\'umero de tiempos $T_{n}$ en el intervalo $\left[0,t\right)$ es

\begin{eqnarray}
N\left(t\right)=\sum_{n=1}^{\infty}\indora\left(T_{n}\leq t\right),
\end{eqnarray}
para $t\geq0$.
\end{Def}

Si se consideran los puntos $T_{n}$ como elementos de $\rea_{+}$, y $N\left(t\right)$ es el n\'umero de puntos en $\rea$. El proceso denotado por $\left\{N\left(t\right):t\geq0\right\}$, denotado por $N\left(t\right)$, es un proceso puntual en $\rea_{+}$. Los $T_{n}$ son los tiempos de ocurrencia, el proceso puntual $N\left(t\right)$ es simple si su n\'umero de ocurrencias son distintas: $0<T_{1}<T_{2}<\ldots$ casi seguramente.

\begin{Def}
Un proceso puntual $N\left(t\right)$ es un proceso de renovaci\'on si los tiempos de interocurrencia $\xi_{n}=T_{n}-T_{n-1}$, para $n\geq1$, son independientes e identicamente distribuidos con distribuci\'on $F$, donde $F\left(0\right)=0$ y $T_{0}=0$. Los $T_{n}$ son llamados tiempos de renovaci\'on, referente a la independencia o renovaci\'on de la informaci\'on estoc\'astica en estos tiempos. Los $\xi_{n}$ son los tiempos de inter-renovaci\'on, y $N\left(t\right)$ es el n\'umero de renovaciones en el intervalo $\left[0,t\right)$
\end{Def}


\begin{Note}
Para definir un proceso de renovaci\'on para cualquier contexto, solamente hay que especificar una distribuci\'on $F$, con $F\left(0\right)=0$, para los tiempos de inter-renovaci\'on. La funci\'on $F$ en turno degune las otra variables aleatorias. De manera formal, existe un espacio de probabilidad y una sucesi\'on de variables aleatorias $\xi_{1},\xi_{2},\ldots$ definidas en este con distribuci\'on $F$. Entonces las otras cantidades son $T_{n}=\sum_{k=1}^{n}\xi_{k}$ y $N\left(t\right)=\sum_{n=1}^{\infty}\indora\left(T_{n}\leq t\right)$, donde $T_{n}\rightarrow\infty$ casi seguramente por la Ley Fuerte de los Grandes Números.
\end{Note}

%___________________________________________________________________________________________
%
%\subsection{Teorema Principal de Renovaci\'on}
%___________________________________________________________________________________________
%

\begin{Note} Una funci\'on $h:\rea_{+}\rightarrow\rea$ es Directamente Riemann Integrable en los siguientes casos:
\begin{itemize}
\item[a)] $h\left(t\right)\geq0$ es decreciente y Riemann Integrable.
\item[b)] $h$ es continua excepto posiblemente en un conjunto de Lebesgue de medida 0, y $|h\left(t\right)|\leq b\left(t\right)$, donde $b$ es DRI.
\end{itemize}
\end{Note}

\begin{Teo}[Teorema Principal de Renovaci\'on]
Si $F$ es no aritm\'etica y $h\left(t\right)$ es Directamente Riemann Integrable (DRI), entonces

\begin{eqnarray*}
lim_{t\rightarrow\infty}U\star h=\frac{1}{\mu}\int_{\rea_{+}}h\left(s\right)ds.
\end{eqnarray*}
\end{Teo}

\begin{Prop}
Cualquier funci\'on $H\left(t\right)$ acotada en intervalos finitos y que es 0 para $t<0$ puede expresarse como
\begin{eqnarray*}
H\left(t\right)=U\star h\left(t\right)\textrm{,  donde }h\left(t\right)=H\left(t\right)-F\star H\left(t\right)
\end{eqnarray*}
\end{Prop}

\begin{Def}
Un proceso estoc\'astico $X\left(t\right)$ es crudamente regenerativo en un tiempo aleatorio positivo $T$ si
\begin{eqnarray*}
\esp\left[X\left(T+t\right)|T\right]=\esp\left[X\left(t\right)\right]\textrm{, para }t\geq0,\end{eqnarray*}
y con las esperanzas anteriores finitas.
\end{Def}

\begin{Prop}
Sup\'ongase que $X\left(t\right)$ es un proceso crudamente regenerativo en $T$, que tiene distribuci\'on $F$. Si $\esp\left[X\left(t\right)\right]$ es acotado en intervalos finitos, entonces
\begin{eqnarray*}
\esp\left[X\left(t\right)\right]=U\star h\left(t\right)\textrm{,  donde }h\left(t\right)=\esp\left[X\left(t\right)\indora\left(T>t\right)\right].
\end{eqnarray*}
\end{Prop}

\begin{Teo}[Regeneraci\'on Cruda]
Sup\'ongase que $X\left(t\right)$ es un proceso con valores positivo crudamente regenerativo en $T$, y def\'inase $M=\sup\left\{|X\left(t\right)|:t\leq T\right\}$. Si $T$ es no aritm\'etico y $M$ y $MT$ tienen media finita, entonces
\begin{eqnarray*}
lim_{t\rightarrow\infty}\esp\left[X\left(t\right)\right]=\frac{1}{\mu}\int_{\rea_{+}}h\left(s\right)ds,
\end{eqnarray*}
donde $h\left(t\right)=\esp\left[X\left(t\right)\indora\left(T>t\right)\right]$.
\end{Teo}

%___________________________________________________________________________________________
%
%\subsection{Propiedades de los Procesos de Renovaci\'on}
%___________________________________________________________________________________________
%

Los tiempos $T_{n}$ est\'an relacionados con los conteos de $N\left(t\right)$ por

\begin{eqnarray*}
\left\{N\left(t\right)\geq n\right\}&=&\left\{T_{n}\leq t\right\}\\
T_{N\left(t\right)}\leq &t&<T_{N\left(t\right)+1},
\end{eqnarray*}

adem\'as $N\left(T_{n}\right)=n$, y 

\begin{eqnarray*}
N\left(t\right)=\max\left\{n:T_{n}\leq t\right\}=\min\left\{n:T_{n+1}>t\right\}
\end{eqnarray*}

Por propiedades de la convoluci\'on se sabe que

\begin{eqnarray*}
P\left\{T_{n}\leq t\right\}=F^{n\star}\left(t\right)
\end{eqnarray*}
que es la $n$-\'esima convoluci\'on de $F$. Entonces 

\begin{eqnarray*}
\left\{N\left(t\right)\geq n\right\}&=&\left\{T_{n}\leq t\right\}\\
P\left\{N\left(t\right)\leq n\right\}&=&1-F^{\left(n+1\right)\star}\left(t\right)
\end{eqnarray*}

Adem\'as usando el hecho de que $\esp\left[N\left(t\right)\right]=\sum_{n=1}^{\infty}P\left\{N\left(t\right)\geq n\right\}$
se tiene que

\begin{eqnarray*}
\esp\left[N\left(t\right)\right]=\sum_{n=1}^{\infty}F^{n\star}\left(t\right)
\end{eqnarray*}

\begin{Prop}
Para cada $t\geq0$, la funci\'on generadora de momentos $\esp\left[e^{\alpha N\left(t\right)}\right]$ existe para alguna $\alpha$ en una vecindad del 0, y de aqu\'i que $\esp\left[N\left(t\right)^{m}\right]<\infty$, para $m\geq1$.
\end{Prop}


\begin{Note}
Si el primer tiempo de renovaci\'on $\xi_{1}$ no tiene la misma distribuci\'on que el resto de las $\xi_{n}$, para $n\geq2$, a $N\left(t\right)$ se le llama Proceso de Renovaci\'on retardado, donde si $\xi$ tiene distribuci\'on $G$, entonces el tiempo $T_{n}$ de la $n$-\'esima renovaci\'on tiene distribuci\'on $G\star F^{\left(n-1\right)\star}\left(t\right)$
\end{Note}


\begin{Teo}
Para una constante $\mu\leq\infty$ ( o variable aleatoria), las siguientes expresiones son equivalentes:

\begin{eqnarray}
lim_{n\rightarrow\infty}n^{-1}T_{n}&=&\mu,\textrm{ c.s.}\\
lim_{t\rightarrow\infty}t^{-1}N\left(t\right)&=&1/\mu,\textrm{ c.s.}
\end{eqnarray}
\end{Teo}


Es decir, $T_{n}$ satisface la Ley Fuerte de los Grandes N\'umeros s\'i y s\'olo s\'i $N\left/t\right)$ la cumple.


\begin{Coro}[Ley Fuerte de los Grandes N\'umeros para Procesos de Renovaci\'on]
Si $N\left(t\right)$ es un proceso de renovaci\'on cuyos tiempos de inter-renovaci\'on tienen media $\mu\leq\infty$, entonces
\begin{eqnarray}
t^{-1}N\left(t\right)\rightarrow 1/\mu,\textrm{ c.s. cuando }t\rightarrow\infty.
\end{eqnarray}

\end{Coro}


Considerar el proceso estoc\'astico de valores reales $\left\{Z\left(t\right):t\geq0\right\}$ en el mismo espacio de probabilidad que $N\left(t\right)$

\begin{Def}
Para el proceso $\left\{Z\left(t\right):t\geq0\right\}$ se define la fluctuaci\'on m\'axima de $Z\left(t\right)$ en el intervalo $\left(T_{n-1},T_{n}\right]$:
\begin{eqnarray*}
M_{n}=\sup_{T_{n-1}<t\leq T_{n}}|Z\left(t\right)-Z\left(T_{n-1}\right)|
\end{eqnarray*}
\end{Def}

\begin{Teo}
Sup\'ongase que $n^{-1}T_{n}\rightarrow\mu$ c.s. cuando $n\rightarrow\infty$, donde $\mu\leq\infty$ es una constante o variable aleatoria. Sea $a$ una constante o variable aleatoria que puede ser infinita cuando $\mu$ es finita, y considere las expresiones l\'imite:
\begin{eqnarray}
lim_{n\rightarrow\infty}n^{-1}Z\left(T_{n}\right)&=&a,\textrm{ c.s.}\\
lim_{t\rightarrow\infty}t^{-1}Z\left(t\right)&=&a/\mu,\textrm{ c.s.}
\end{eqnarray}
La segunda expresi\'on implica la primera. Conversamente, la primera implica la segunda si el proceso $Z\left(t\right)$ es creciente, o si $lim_{n\rightarrow\infty}n^{-1}M_{n}=0$ c.s.
\end{Teo}

\begin{Coro}
Si $N\left(t\right)$ es un proceso de renovaci\'on, y $\left(Z\left(T_{n}\right)-Z\left(T_{n-1}\right),M_{n}\right)$, para $n\geq1$, son variables aleatorias independientes e id\'enticamente distribuidas con media finita, entonces,
\begin{eqnarray}
lim_{t\rightarrow\infty}t^{-1}Z\left(t\right)\rightarrow\frac{\esp\left[Z\left(T_{1}\right)-Z\left(T_{0}\right)\right]}{\esp\left[T_{1}\right]},\textrm{ c.s. cuando  }t\rightarrow\infty.
\end{eqnarray}
\end{Coro}



%___________________________________________________________________________________________
%
%\subsection{Propiedades de los Procesos de Renovaci\'on}
%___________________________________________________________________________________________
%

Los tiempos $T_{n}$ est\'an relacionados con los conteos de $N\left(t\right)$ por

\begin{eqnarray*}
\left\{N\left(t\right)\geq n\right\}&=&\left\{T_{n}\leq t\right\}\\
T_{N\left(t\right)}\leq &t&<T_{N\left(t\right)+1},
\end{eqnarray*}

adem\'as $N\left(T_{n}\right)=n$, y 

\begin{eqnarray*}
N\left(t\right)=\max\left\{n:T_{n}\leq t\right\}=\min\left\{n:T_{n+1}>t\right\}
\end{eqnarray*}

Por propiedades de la convoluci\'on se sabe que

\begin{eqnarray*}
P\left\{T_{n}\leq t\right\}=F^{n\star}\left(t\right)
\end{eqnarray*}
que es la $n$-\'esima convoluci\'on de $F$. Entonces 

\begin{eqnarray*}
\left\{N\left(t\right)\geq n\right\}&=&\left\{T_{n}\leq t\right\}\\
P\left\{N\left(t\right)\leq n\right\}&=&1-F^{\left(n+1\right)\star}\left(t\right)
\end{eqnarray*}

Adem\'as usando el hecho de que $\esp\left[N\left(t\right)\right]=\sum_{n=1}^{\infty}P\left\{N\left(t\right)\geq n\right\}$
se tiene que

\begin{eqnarray*}
\esp\left[N\left(t\right)\right]=\sum_{n=1}^{\infty}F^{n\star}\left(t\right)
\end{eqnarray*}

\begin{Prop}
Para cada $t\geq0$, la funci\'on generadora de momentos $\esp\left[e^{\alpha N\left(t\right)}\right]$ existe para alguna $\alpha$ en una vecindad del 0, y de aqu\'i que $\esp\left[N\left(t\right)^{m}\right]<\infty$, para $m\geq1$.
\end{Prop}


\begin{Note}
Si el primer tiempo de renovaci\'on $\xi_{1}$ no tiene la misma distribuci\'on que el resto de las $\xi_{n}$, para $n\geq2$, a $N\left(t\right)$ se le llama Proceso de Renovaci\'on retardado, donde si $\xi$ tiene distribuci\'on $G$, entonces el tiempo $T_{n}$ de la $n$-\'esima renovaci\'on tiene distribuci\'on $G\star F^{\left(n-1\right)\star}\left(t\right)$
\end{Note}


\begin{Teo}
Para una constante $\mu\leq\infty$ ( o variable aleatoria), las siguientes expresiones son equivalentes:

\begin{eqnarray}
lim_{n\rightarrow\infty}n^{-1}T_{n}&=&\mu,\textrm{ c.s.}\\
lim_{t\rightarrow\infty}t^{-1}N\left(t\right)&=&1/\mu,\textrm{ c.s.}
\end{eqnarray}
\end{Teo}


Es decir, $T_{n}$ satisface la Ley Fuerte de los Grandes N\'umeros s\'i y s\'olo s\'i $N\left/t\right)$ la cumple.


\begin{Coro}[Ley Fuerte de los Grandes N\'umeros para Procesos de Renovaci\'on]
Si $N\left(t\right)$ es un proceso de renovaci\'on cuyos tiempos de inter-renovaci\'on tienen media $\mu\leq\infty$, entonces
\begin{eqnarray}
t^{-1}N\left(t\right)\rightarrow 1/\mu,\textrm{ c.s. cuando }t\rightarrow\infty.
\end{eqnarray}

\end{Coro}


Considerar el proceso estoc\'astico de valores reales $\left\{Z\left(t\right):t\geq0\right\}$ en el mismo espacio de probabilidad que $N\left(t\right)$

\begin{Def}
Para el proceso $\left\{Z\left(t\right):t\geq0\right\}$ se define la fluctuaci\'on m\'axima de $Z\left(t\right)$ en el intervalo $\left(T_{n-1},T_{n}\right]$:
\begin{eqnarray*}
M_{n}=\sup_{T_{n-1}<t\leq T_{n}}|Z\left(t\right)-Z\left(T_{n-1}\right)|
\end{eqnarray*}
\end{Def}

\begin{Teo}
Sup\'ongase que $n^{-1}T_{n}\rightarrow\mu$ c.s. cuando $n\rightarrow\infty$, donde $\mu\leq\infty$ es una constante o variable aleatoria. Sea $a$ una constante o variable aleatoria que puede ser infinita cuando $\mu$ es finita, y considere las expresiones l\'imite:
\begin{eqnarray}
lim_{n\rightarrow\infty}n^{-1}Z\left(T_{n}\right)&=&a,\textrm{ c.s.}\\
lim_{t\rightarrow\infty}t^{-1}Z\left(t\right)&=&a/\mu,\textrm{ c.s.}
\end{eqnarray}
La segunda expresi\'on implica la primera. Conversamente, la primera implica la segunda si el proceso $Z\left(t\right)$ es creciente, o si $lim_{n\rightarrow\infty}n^{-1}M_{n}=0$ c.s.
\end{Teo}

\begin{Coro}
Si $N\left(t\right)$ es un proceso de renovaci\'on, y $\left(Z\left(T_{n}\right)-Z\left(T_{n-1}\right),M_{n}\right)$, para $n\geq1$, son variables aleatorias independientes e id\'enticamente distribuidas con media finita, entonces,
\begin{eqnarray}
lim_{t\rightarrow\infty}t^{-1}Z\left(t\right)\rightarrow\frac{\esp\left[Z\left(T_{1}\right)-Z\left(T_{0}\right)\right]}{\esp\left[T_{1}\right]},\textrm{ c.s. cuando  }t\rightarrow\infty.
\end{eqnarray}
\end{Coro}


%___________________________________________________________________________________________
%
%\subsection{Propiedades de los Procesos de Renovaci\'on}
%___________________________________________________________________________________________
%

Los tiempos $T_{n}$ est\'an relacionados con los conteos de $N\left(t\right)$ por

\begin{eqnarray*}
\left\{N\left(t\right)\geq n\right\}&=&\left\{T_{n}\leq t\right\}\\
T_{N\left(t\right)}\leq &t&<T_{N\left(t\right)+1},
\end{eqnarray*}

adem\'as $N\left(T_{n}\right)=n$, y 

\begin{eqnarray*}
N\left(t\right)=\max\left\{n:T_{n}\leq t\right\}=\min\left\{n:T_{n+1}>t\right\}
\end{eqnarray*}

Por propiedades de la convoluci\'on se sabe que

\begin{eqnarray*}
P\left\{T_{n}\leq t\right\}=F^{n\star}\left(t\right)
\end{eqnarray*}
que es la $n$-\'esima convoluci\'on de $F$. Entonces 

\begin{eqnarray*}
\left\{N\left(t\right)\geq n\right\}&=&\left\{T_{n}\leq t\right\}\\
P\left\{N\left(t\right)\leq n\right\}&=&1-F^{\left(n+1\right)\star}\left(t\right)
\end{eqnarray*}

Adem\'as usando el hecho de que $\esp\left[N\left(t\right)\right]=\sum_{n=1}^{\infty}P\left\{N\left(t\right)\geq n\right\}$
se tiene que

\begin{eqnarray*}
\esp\left[N\left(t\right)\right]=\sum_{n=1}^{\infty}F^{n\star}\left(t\right)
\end{eqnarray*}

\begin{Prop}
Para cada $t\geq0$, la funci\'on generadora de momentos $\esp\left[e^{\alpha N\left(t\right)}\right]$ existe para alguna $\alpha$ en una vecindad del 0, y de aqu\'i que $\esp\left[N\left(t\right)^{m}\right]<\infty$, para $m\geq1$.
\end{Prop}


\begin{Note}
Si el primer tiempo de renovaci\'on $\xi_{1}$ no tiene la misma distribuci\'on que el resto de las $\xi_{n}$, para $n\geq2$, a $N\left(t\right)$ se le llama Proceso de Renovaci\'on retardado, donde si $\xi$ tiene distribuci\'on $G$, entonces el tiempo $T_{n}$ de la $n$-\'esima renovaci\'on tiene distribuci\'on $G\star F^{\left(n-1\right)\star}\left(t\right)$
\end{Note}


\begin{Teo}
Para una constante $\mu\leq\infty$ ( o variable aleatoria), las siguientes expresiones son equivalentes:

\begin{eqnarray}
lim_{n\rightarrow\infty}n^{-1}T_{n}&=&\mu,\textrm{ c.s.}\\
lim_{t\rightarrow\infty}t^{-1}N\left(t\right)&=&1/\mu,\textrm{ c.s.}
\end{eqnarray}
\end{Teo}


Es decir, $T_{n}$ satisface la Ley Fuerte de los Grandes N\'umeros s\'i y s\'olo s\'i $N\left/t\right)$ la cumple.


\begin{Coro}[Ley Fuerte de los Grandes N\'umeros para Procesos de Renovaci\'on]
Si $N\left(t\right)$ es un proceso de renovaci\'on cuyos tiempos de inter-renovaci\'on tienen media $\mu\leq\infty$, entonces
\begin{eqnarray}
t^{-1}N\left(t\right)\rightarrow 1/\mu,\textrm{ c.s. cuando }t\rightarrow\infty.
\end{eqnarray}

\end{Coro}


Considerar el proceso estoc\'astico de valores reales $\left\{Z\left(t\right):t\geq0\right\}$ en el mismo espacio de probabilidad que $N\left(t\right)$

\begin{Def}
Para el proceso $\left\{Z\left(t\right):t\geq0\right\}$ se define la fluctuaci\'on m\'axima de $Z\left(t\right)$ en el intervalo $\left(T_{n-1},T_{n}\right]$:
\begin{eqnarray*}
M_{n}=\sup_{T_{n-1}<t\leq T_{n}}|Z\left(t\right)-Z\left(T_{n-1}\right)|
\end{eqnarray*}
\end{Def}

\begin{Teo}
Sup\'ongase que $n^{-1}T_{n}\rightarrow\mu$ c.s. cuando $n\rightarrow\infty$, donde $\mu\leq\infty$ es una constante o variable aleatoria. Sea $a$ una constante o variable aleatoria que puede ser infinita cuando $\mu$ es finita, y considere las expresiones l\'imite:
\begin{eqnarray}
lim_{n\rightarrow\infty}n^{-1}Z\left(T_{n}\right)&=&a,\textrm{ c.s.}\\
lim_{t\rightarrow\infty}t^{-1}Z\left(t\right)&=&a/\mu,\textrm{ c.s.}
\end{eqnarray}
La segunda expresi\'on implica la primera. Conversamente, la primera implica la segunda si el proceso $Z\left(t\right)$ es creciente, o si $lim_{n\rightarrow\infty}n^{-1}M_{n}=0$ c.s.
\end{Teo}

\begin{Coro}
Si $N\left(t\right)$ es un proceso de renovaci\'on, y $\left(Z\left(T_{n}\right)-Z\left(T_{n-1}\right),M_{n}\right)$, para $n\geq1$, son variables aleatorias independientes e id\'enticamente distribuidas con media finita, entonces,
\begin{eqnarray}
lim_{t\rightarrow\infty}t^{-1}Z\left(t\right)\rightarrow\frac{\esp\left[Z\left(T_{1}\right)-Z\left(T_{0}\right)\right]}{\esp\left[T_{1}\right]},\textrm{ c.s. cuando  }t\rightarrow\infty.
\end{eqnarray}
\end{Coro}

%___________________________________________________________________________________________
%
%\subsection{Propiedades de los Procesos de Renovaci\'on}
%___________________________________________________________________________________________
%

Los tiempos $T_{n}$ est\'an relacionados con los conteos de $N\left(t\right)$ por

\begin{eqnarray*}
\left\{N\left(t\right)\geq n\right\}&=&\left\{T_{n}\leq t\right\}\\
T_{N\left(t\right)}\leq &t&<T_{N\left(t\right)+1},
\end{eqnarray*}

adem\'as $N\left(T_{n}\right)=n$, y 

\begin{eqnarray*}
N\left(t\right)=\max\left\{n:T_{n}\leq t\right\}=\min\left\{n:T_{n+1}>t\right\}
\end{eqnarray*}

Por propiedades de la convoluci\'on se sabe que

\begin{eqnarray*}
P\left\{T_{n}\leq t\right\}=F^{n\star}\left(t\right)
\end{eqnarray*}
que es la $n$-\'esima convoluci\'on de $F$. Entonces 

\begin{eqnarray*}
\left\{N\left(t\right)\geq n\right\}&=&\left\{T_{n}\leq t\right\}\\
P\left\{N\left(t\right)\leq n\right\}&=&1-F^{\left(n+1\right)\star}\left(t\right)
\end{eqnarray*}

Adem\'as usando el hecho de que $\esp\left[N\left(t\right)\right]=\sum_{n=1}^{\infty}P\left\{N\left(t\right)\geq n\right\}$
se tiene que

\begin{eqnarray*}
\esp\left[N\left(t\right)\right]=\sum_{n=1}^{\infty}F^{n\star}\left(t\right)
\end{eqnarray*}

\begin{Prop}
Para cada $t\geq0$, la funci\'on generadora de momentos $\esp\left[e^{\alpha N\left(t\right)}\right]$ existe para alguna $\alpha$ en una vecindad del 0, y de aqu\'i que $\esp\left[N\left(t\right)^{m}\right]<\infty$, para $m\geq1$.
\end{Prop}


\begin{Note}
Si el primer tiempo de renovaci\'on $\xi_{1}$ no tiene la misma distribuci\'on que el resto de las $\xi_{n}$, para $n\geq2$, a $N\left(t\right)$ se le llama Proceso de Renovaci\'on retardado, donde si $\xi$ tiene distribuci\'on $G$, entonces el tiempo $T_{n}$ de la $n$-\'esima renovaci\'on tiene distribuci\'on $G\star F^{\left(n-1\right)\star}\left(t\right)$
\end{Note}


\begin{Teo}
Para una constante $\mu\leq\infty$ ( o variable aleatoria), las siguientes expresiones son equivalentes:

\begin{eqnarray}
lim_{n\rightarrow\infty}n^{-1}T_{n}&=&\mu,\textrm{ c.s.}\\
lim_{t\rightarrow\infty}t^{-1}N\left(t\right)&=&1/\mu,\textrm{ c.s.}
\end{eqnarray}
\end{Teo}


Es decir, $T_{n}$ satisface la Ley Fuerte de los Grandes N\'umeros s\'i y s\'olo s\'i $N\left/t\right)$ la cumple.


\begin{Coro}[Ley Fuerte de los Grandes N\'umeros para Procesos de Renovaci\'on]
Si $N\left(t\right)$ es un proceso de renovaci\'on cuyos tiempos de inter-renovaci\'on tienen media $\mu\leq\infty$, entonces
\begin{eqnarray}
t^{-1}N\left(t\right)\rightarrow 1/\mu,\textrm{ c.s. cuando }t\rightarrow\infty.
\end{eqnarray}

\end{Coro}


Considerar el proceso estoc\'astico de valores reales $\left\{Z\left(t\right):t\geq0\right\}$ en el mismo espacio de probabilidad que $N\left(t\right)$

\begin{Def}
Para el proceso $\left\{Z\left(t\right):t\geq0\right\}$ se define la fluctuaci\'on m\'axima de $Z\left(t\right)$ en el intervalo $\left(T_{n-1},T_{n}\right]$:
\begin{eqnarray*}
M_{n}=\sup_{T_{n-1}<t\leq T_{n}}|Z\left(t\right)-Z\left(T_{n-1}\right)|
\end{eqnarray*}
\end{Def}

\begin{Teo}
Sup\'ongase que $n^{-1}T_{n}\rightarrow\mu$ c.s. cuando $n\rightarrow\infty$, donde $\mu\leq\infty$ es una constante o variable aleatoria. Sea $a$ una constante o variable aleatoria que puede ser infinita cuando $\mu$ es finita, y considere las expresiones l\'imite:
\begin{eqnarray}
lim_{n\rightarrow\infty}n^{-1}Z\left(T_{n}\right)&=&a,\textrm{ c.s.}\\
lim_{t\rightarrow\infty}t^{-1}Z\left(t\right)&=&a/\mu,\textrm{ c.s.}
\end{eqnarray}
La segunda expresi\'on implica la primera. Conversamente, la primera implica la segunda si el proceso $Z\left(t\right)$ es creciente, o si $lim_{n\rightarrow\infty}n^{-1}M_{n}=0$ c.s.
\end{Teo}

\begin{Coro}
Si $N\left(t\right)$ es un proceso de renovaci\'on, y $\left(Z\left(T_{n}\right)-Z\left(T_{n-1}\right),M_{n}\right)$, para $n\geq1$, son variables aleatorias independientes e id\'enticamente distribuidas con media finita, entonces,
\begin{eqnarray}
lim_{t\rightarrow\infty}t^{-1}Z\left(t\right)\rightarrow\frac{\esp\left[Z\left(T_{1}\right)-Z\left(T_{0}\right)\right]}{\esp\left[T_{1}\right]},\textrm{ c.s. cuando  }t\rightarrow\infty.
\end{eqnarray}
\end{Coro}
%___________________________________________________________________________________________
%
%\subsection{Propiedades de los Procesos de Renovaci\'on}
%___________________________________________________________________________________________
%

Los tiempos $T_{n}$ est\'an relacionados con los conteos de $N\left(t\right)$ por

\begin{eqnarray*}
\left\{N\left(t\right)\geq n\right\}&=&\left\{T_{n}\leq t\right\}\\
T_{N\left(t\right)}\leq &t&<T_{N\left(t\right)+1},
\end{eqnarray*}

adem\'as $N\left(T_{n}\right)=n$, y 

\begin{eqnarray*}
N\left(t\right)=\max\left\{n:T_{n}\leq t\right\}=\min\left\{n:T_{n+1}>t\right\}
\end{eqnarray*}

Por propiedades de la convoluci\'on se sabe que

\begin{eqnarray*}
P\left\{T_{n}\leq t\right\}=F^{n\star}\left(t\right)
\end{eqnarray*}
que es la $n$-\'esima convoluci\'on de $F$. Entonces 

\begin{eqnarray*}
\left\{N\left(t\right)\geq n\right\}&=&\left\{T_{n}\leq t\right\}\\
P\left\{N\left(t\right)\leq n\right\}&=&1-F^{\left(n+1\right)\star}\left(t\right)
\end{eqnarray*}

Adem\'as usando el hecho de que $\esp\left[N\left(t\right)\right]=\sum_{n=1}^{\infty}P\left\{N\left(t\right)\geq n\right\}$
se tiene que

\begin{eqnarray*}
\esp\left[N\left(t\right)\right]=\sum_{n=1}^{\infty}F^{n\star}\left(t\right)
\end{eqnarray*}

\begin{Prop}
Para cada $t\geq0$, la funci\'on generadora de momentos $\esp\left[e^{\alpha N\left(t\right)}\right]$ existe para alguna $\alpha$ en una vecindad del 0, y de aqu\'i que $\esp\left[N\left(t\right)^{m}\right]<\infty$, para $m\geq1$.
\end{Prop}


\begin{Note}
Si el primer tiempo de renovaci\'on $\xi_{1}$ no tiene la misma distribuci\'on que el resto de las $\xi_{n}$, para $n\geq2$, a $N\left(t\right)$ se le llama Proceso de Renovaci\'on retardado, donde si $\xi$ tiene distribuci\'on $G$, entonces el tiempo $T_{n}$ de la $n$-\'esima renovaci\'on tiene distribuci\'on $G\star F^{\left(n-1\right)\star}\left(t\right)$
\end{Note}


\begin{Teo}
Para una constante $\mu\leq\infty$ ( o variable aleatoria), las siguientes expresiones son equivalentes:

\begin{eqnarray}
lim_{n\rightarrow\infty}n^{-1}T_{n}&=&\mu,\textrm{ c.s.}\\
lim_{t\rightarrow\infty}t^{-1}N\left(t\right)&=&1/\mu,\textrm{ c.s.}
\end{eqnarray}
\end{Teo}


Es decir, $T_{n}$ satisface la Ley Fuerte de los Grandes N\'umeros s\'i y s\'olo s\'i $N\left/t\right)$ la cumple.


\begin{Coro}[Ley Fuerte de los Grandes N\'umeros para Procesos de Renovaci\'on]
Si $N\left(t\right)$ es un proceso de renovaci\'on cuyos tiempos de inter-renovaci\'on tienen media $\mu\leq\infty$, entonces
\begin{eqnarray}
t^{-1}N\left(t\right)\rightarrow 1/\mu,\textrm{ c.s. cuando }t\rightarrow\infty.
\end{eqnarray}

\end{Coro}


Considerar el proceso estoc\'astico de valores reales $\left\{Z\left(t\right):t\geq0\right\}$ en el mismo espacio de probabilidad que $N\left(t\right)$

\begin{Def}
Para el proceso $\left\{Z\left(t\right):t\geq0\right\}$ se define la fluctuaci\'on m\'axima de $Z\left(t\right)$ en el intervalo $\left(T_{n-1},T_{n}\right]$:
\begin{eqnarray*}
M_{n}=\sup_{T_{n-1}<t\leq T_{n}}|Z\left(t\right)-Z\left(T_{n-1}\right)|
\end{eqnarray*}
\end{Def}

\begin{Teo}
Sup\'ongase que $n^{-1}T_{n}\rightarrow\mu$ c.s. cuando $n\rightarrow\infty$, donde $\mu\leq\infty$ es una constante o variable aleatoria. Sea $a$ una constante o variable aleatoria que puede ser infinita cuando $\mu$ es finita, y considere las expresiones l\'imite:
\begin{eqnarray}
lim_{n\rightarrow\infty}n^{-1}Z\left(T_{n}\right)&=&a,\textrm{ c.s.}\\
lim_{t\rightarrow\infty}t^{-1}Z\left(t\right)&=&a/\mu,\textrm{ c.s.}
\end{eqnarray}
La segunda expresi\'on implica la primera. Conversamente, la primera implica la segunda si el proceso $Z\left(t\right)$ es creciente, o si $lim_{n\rightarrow\infty}n^{-1}M_{n}=0$ c.s.
\end{Teo}

\begin{Coro}
Si $N\left(t\right)$ es un proceso de renovaci\'on, y $\left(Z\left(T_{n}\right)-Z\left(T_{n-1}\right),M_{n}\right)$, para $n\geq1$, son variables aleatorias independientes e id\'enticamente distribuidas con media finita, entonces,
\begin{eqnarray}
lim_{t\rightarrow\infty}t^{-1}Z\left(t\right)\rightarrow\frac{\esp\left[Z\left(T_{1}\right)-Z\left(T_{0}\right)\right]}{\esp\left[T_{1}\right]},\textrm{ c.s. cuando  }t\rightarrow\infty.
\end{eqnarray}
\end{Coro}


%___________________________________________________________________________________________
%
%\subsection{Funci\'on de Renovaci\'on}
%___________________________________________________________________________________________
%


\begin{Def}
Sea $h\left(t\right)$ funci\'on de valores reales en $\rea$ acotada en intervalos finitos e igual a cero para $t<0$ La ecuaci\'on de renovaci\'on para $h\left(t\right)$ y la distribuci\'on $F$ es

\begin{eqnarray}%\label{Ec.Renovacion}
H\left(t\right)=h\left(t\right)+\int_{\left[0,t\right]}H\left(t-s\right)dF\left(s\right)\textrm{,    }t\geq0,
\end{eqnarray}
donde $H\left(t\right)$ es una funci\'on de valores reales. Esto es $H=h+F\star H$. Decimos que $H\left(t\right)$ es soluci\'on de esta ecuaci\'on si satisface la ecuaci\'on, y es acotada en intervalos finitos e iguales a cero para $t<0$.
\end{Def}

\begin{Prop}
La funci\'on $U\star h\left(t\right)$ es la \'unica soluci\'on de la ecuaci\'on de renovaci\'on (\ref{Ec.Renovacion}).
\end{Prop}

\begin{Teo}[Teorema Renovaci\'on Elemental]
\begin{eqnarray*}
t^{-1}U\left(t\right)\rightarrow 1/\mu\textrm{,    cuando }t\rightarrow\infty.
\end{eqnarray*}
\end{Teo}

%___________________________________________________________________________________________
%
%\subsection{Funci\'on de Renovaci\'on}
%___________________________________________________________________________________________
%


Sup\'ongase que $N\left(t\right)$ es un proceso de renovaci\'on con distribuci\'on $F$ con media finita $\mu$.

\begin{Def}
La funci\'on de renovaci\'on asociada con la distribuci\'on $F$, del proceso $N\left(t\right)$, es
\begin{eqnarray*}
U\left(t\right)=\sum_{n=1}^{\infty}F^{n\star}\left(t\right),\textrm{   }t\geq0,
\end{eqnarray*}
donde $F^{0\star}\left(t\right)=\indora\left(t\geq0\right)$.
\end{Def}


\begin{Prop}
Sup\'ongase que la distribuci\'on de inter-renovaci\'on $F$ tiene densidad $f$. Entonces $U\left(t\right)$ tambi\'en tiene densidad, para $t>0$, y es $U^{'}\left(t\right)=\sum_{n=0}^{\infty}f^{n\star}\left(t\right)$. Adem\'as
\begin{eqnarray*}
\prob\left\{N\left(t\right)>N\left(t-\right)\right\}=0\textrm{,   }t\geq0.
\end{eqnarray*}
\end{Prop}

\begin{Def}
La Transformada de Laplace-Stieljes de $F$ est\'a dada por

\begin{eqnarray*}
\hat{F}\left(\alpha\right)=\int_{\rea_{+}}e^{-\alpha t}dF\left(t\right)\textrm{,  }\alpha\geq0.
\end{eqnarray*}
\end{Def}

Entonces

\begin{eqnarray*}
\hat{U}\left(\alpha\right)=\sum_{n=0}^{\infty}\hat{F^{n\star}}\left(\alpha\right)=\sum_{n=0}^{\infty}\hat{F}\left(\alpha\right)^{n}=\frac{1}{1-\hat{F}\left(\alpha\right)}.
\end{eqnarray*}


\begin{Prop}
La Transformada de Laplace $\hat{U}\left(\alpha\right)$ y $\hat{F}\left(\alpha\right)$ determina una a la otra de manera \'unica por la relaci\'on $\hat{U}\left(\alpha\right)=\frac{1}{1-\hat{F}\left(\alpha\right)}$.
\end{Prop}


\begin{Note}
Un proceso de renovaci\'on $N\left(t\right)$ cuyos tiempos de inter-renovaci\'on tienen media finita, es un proceso Poisson con tasa $\lambda$ si y s\'olo s\'i $\esp\left[U\left(t\right)\right]=\lambda t$, para $t\geq0$.
\end{Note}


\begin{Teo}
Sea $N\left(t\right)$ un proceso puntual simple con puntos de localizaci\'on $T_{n}$ tal que $\eta\left(t\right)=\esp\left[N\left(\right)\right]$ es finita para cada $t$. Entonces para cualquier funci\'on $f:\rea_{+}\rightarrow\rea$,
\begin{eqnarray*}
\esp\left[\sum_{n=1}^{N\left(\right)}f\left(T_{n}\right)\right]=\int_{\left(0,t\right]}f\left(s\right)d\eta\left(s\right)\textrm{,  }t\geq0,
\end{eqnarray*}
suponiendo que la integral exista. Adem\'as si $X_{1},X_{2},\ldots$ son variables aleatorias definidas en el mismo espacio de probabilidad que el proceso $N\left(t\right)$ tal que $\esp\left[X_{n}|T_{n}=s\right]=f\left(s\right)$, independiente de $n$. Entonces
\begin{eqnarray*}
\esp\left[\sum_{n=1}^{N\left(t\right)}X_{n}\right]=\int_{\left(0,t\right]}f\left(s\right)d\eta\left(s\right)\textrm{,  }t\geq0,
\end{eqnarray*} 
suponiendo que la integral exista. 
\end{Teo}

\begin{Coro}[Identidad de Wald para Renovaciones]
Para el proceso de renovaci\'on $N\left(t\right)$,
\begin{eqnarray*}
\esp\left[T_{N\left(t\right)+1}\right]=\mu\esp\left[N\left(t\right)+1\right]\textrm{,  }t\geq0,
\end{eqnarray*}  
\end{Coro}

%______________________________________________________________________
%\subsection{Procesos de Renovaci\'on}
%______________________________________________________________________

\begin{Def}%\label{Def.Tn}
Sean $0\leq T_{1}\leq T_{2}\leq \ldots$ son tiempos aleatorios infinitos en los cuales ocurren ciertos eventos. El n\'umero de tiempos $T_{n}$ en el intervalo $\left[0,t\right)$ es

\begin{eqnarray}
N\left(t\right)=\sum_{n=1}^{\infty}\indora\left(T_{n}\leq t\right),
\end{eqnarray}
para $t\geq0$.
\end{Def}

Si se consideran los puntos $T_{n}$ como elementos de $\rea_{+}$, y $N\left(t\right)$ es el n\'umero de puntos en $\rea$. El proceso denotado por $\left\{N\left(t\right):t\geq0\right\}$, denotado por $N\left(t\right)$, es un proceso puntual en $\rea_{+}$. Los $T_{n}$ son los tiempos de ocurrencia, el proceso puntual $N\left(t\right)$ es simple si su n\'umero de ocurrencias son distintas: $0<T_{1}<T_{2}<\ldots$ casi seguramente.

\begin{Def}
Un proceso puntual $N\left(t\right)$ es un proceso de renovaci\'on si los tiempos de interocurrencia $\xi_{n}=T_{n}-T_{n-1}$, para $n\geq1$, son independientes e identicamente distribuidos con distribuci\'on $F$, donde $F\left(0\right)=0$ y $T_{0}=0$. Los $T_{n}$ son llamados tiempos de renovaci\'on, referente a la independencia o renovaci\'on de la informaci\'on estoc\'astica en estos tiempos. Los $\xi_{n}$ son los tiempos de inter-renovaci\'on, y $N\left(t\right)$ es el n\'umero de renovaciones en el intervalo $\left[0,t\right)$
\end{Def}


\begin{Note}
Para definir un proceso de renovaci\'on para cualquier contexto, solamente hay que especificar una distribuci\'on $F$, con $F\left(0\right)=0$, para los tiempos de inter-renovaci\'on. La funci\'on $F$ en turno degune las otra variables aleatorias. De manera formal, existe un espacio de probabilidad y una sucesi\'on de variables aleatorias $\xi_{1},\xi_{2},\ldots$ definidas en este con distribuci\'on $F$. Entonces las otras cantidades son $T_{n}=\sum_{k=1}^{n}\xi_{k}$ y $N\left(t\right)=\sum_{n=1}^{\infty}\indora\left(T_{n}\leq t\right)$, donde $T_{n}\rightarrow\infty$ casi seguramente por la Ley Fuerte de los Grandes Números.
\end{Note}

%___________________________________________________________________________________________
%
%\subsection{Renewal and Regenerative Processes: Serfozo\cite{Serfozo}}
%___________________________________________________________________________________________
%
\begin{Def}%\label{Def.Tn}
Sean $0\leq T_{1}\leq T_{2}\leq \ldots$ son tiempos aleatorios infinitos en los cuales ocurren ciertos eventos. El n\'umero de tiempos $T_{n}$ en el intervalo $\left[0,t\right)$ es

\begin{eqnarray}
N\left(t\right)=\sum_{n=1}^{\infty}\indora\left(T_{n}\leq t\right),
\end{eqnarray}
para $t\geq0$.
\end{Def}

Si se consideran los puntos $T_{n}$ como elementos de $\rea_{+}$, y $N\left(t\right)$ es el n\'umero de puntos en $\rea$. El proceso denotado por $\left\{N\left(t\right):t\geq0\right\}$, denotado por $N\left(t\right)$, es un proceso puntual en $\rea_{+}$. Los $T_{n}$ son los tiempos de ocurrencia, el proceso puntual $N\left(t\right)$ es simple si su n\'umero de ocurrencias son distintas: $0<T_{1}<T_{2}<\ldots$ casi seguramente.

\begin{Def}
Un proceso puntual $N\left(t\right)$ es un proceso de renovaci\'on si los tiempos de interocurrencia $\xi_{n}=T_{n}-T_{n-1}$, para $n\geq1$, son independientes e identicamente distribuidos con distribuci\'on $F$, donde $F\left(0\right)=0$ y $T_{0}=0$. Los $T_{n}$ son llamados tiempos de renovaci\'on, referente a la independencia o renovaci\'on de la informaci\'on estoc\'astica en estos tiempos. Los $\xi_{n}$ son los tiempos de inter-renovaci\'on, y $N\left(t\right)$ es el n\'umero de renovaciones en el intervalo $\left[0,t\right)$
\end{Def}


\begin{Note}
Para definir un proceso de renovaci\'on para cualquier contexto, solamente hay que especificar una distribuci\'on $F$, con $F\left(0\right)=0$, para los tiempos de inter-renovaci\'on. La funci\'on $F$ en turno degune las otra variables aleatorias. De manera formal, existe un espacio de probabilidad y una sucesi\'on de variables aleatorias $\xi_{1},\xi_{2},\ldots$ definidas en este con distribuci\'on $F$. Entonces las otras cantidades son $T_{n}=\sum_{k=1}^{n}\xi_{k}$ y $N\left(t\right)=\sum_{n=1}^{\infty}\indora\left(T_{n}\leq t\right)$, donde $T_{n}\rightarrow\infty$ casi seguramente por la Ley Fuerte de los Grandes N\'umeros.
\end{Note}







Los tiempos $T_{n}$ est\'an relacionados con los conteos de $N\left(t\right)$ por

\begin{eqnarray*}
\left\{N\left(t\right)\geq n\right\}&=&\left\{T_{n}\leq t\right\}\\
T_{N\left(t\right)}\leq &t&<T_{N\left(t\right)+1},
\end{eqnarray*}

adem\'as $N\left(T_{n}\right)=n$, y 

\begin{eqnarray*}
N\left(t\right)=\max\left\{n:T_{n}\leq t\right\}=\min\left\{n:T_{n+1}>t\right\}
\end{eqnarray*}

Por propiedades de la convoluci\'on se sabe que

\begin{eqnarray*}
P\left\{T_{n}\leq t\right\}=F^{n\star}\left(t\right)
\end{eqnarray*}
que es la $n$-\'esima convoluci\'on de $F$. Entonces 

\begin{eqnarray*}
\left\{N\left(t\right)\geq n\right\}&=&\left\{T_{n}\leq t\right\}\\
P\left\{N\left(t\right)\leq n\right\}&=&1-F^{\left(n+1\right)\star}\left(t\right)
\end{eqnarray*}

Adem\'as usando el hecho de que $\esp\left[N\left(t\right)\right]=\sum_{n=1}^{\infty}P\left\{N\left(t\right)\geq n\right\}$
se tiene que

\begin{eqnarray*}
\esp\left[N\left(t\right)\right]=\sum_{n=1}^{\infty}F^{n\star}\left(t\right)
\end{eqnarray*}

\begin{Prop}
Para cada $t\geq0$, la funci\'on generadora de momentos $\esp\left[e^{\alpha N\left(t\right)}\right]$ existe para alguna $\alpha$ en una vecindad del 0, y de aqu\'i que $\esp\left[N\left(t\right)^{m}\right]<\infty$, para $m\geq1$.
\end{Prop}

\begin{Ejem}[\textbf{Proceso Poisson}]

Suponga que se tienen tiempos de inter-renovaci\'on \textit{i.i.d.} del proceso de renovaci\'on $N\left(t\right)$ tienen distribuci\'on exponencial $F\left(t\right)=q-e^{-\lambda t}$ con tasa $\lambda$. Entonces $N\left(t\right)$ es un proceso Poisson con tasa $\lambda$.

\end{Ejem}


\begin{Note}
Si el primer tiempo de renovaci\'on $\xi_{1}$ no tiene la misma distribuci\'on que el resto de las $\xi_{n}$, para $n\geq2$, a $N\left(t\right)$ se le llama Proceso de Renovaci\'on retardado, donde si $\xi$ tiene distribuci\'on $G$, entonces el tiempo $T_{n}$ de la $n$-\'esima renovaci\'on tiene distribuci\'on $G\star F^{\left(n-1\right)\star}\left(t\right)$
\end{Note}


\begin{Teo}
Para una constante $\mu\leq\infty$ ( o variable aleatoria), las siguientes expresiones son equivalentes:

\begin{eqnarray}
lim_{n\rightarrow\infty}n^{-1}T_{n}&=&\mu,\textrm{ c.s.}\\
lim_{t\rightarrow\infty}t^{-1}N\left(t\right)&=&1/\mu,\textrm{ c.s.}
\end{eqnarray}
\end{Teo}


Es decir, $T_{n}$ satisface la Ley Fuerte de los Grandes N\'umeros s\'i y s\'olo s\'i $N\left/t\right)$ la cumple.


\begin{Coro}[Ley Fuerte de los Grandes N\'umeros para Procesos de Renovaci\'on]
Si $N\left(t\right)$ es un proceso de renovaci\'on cuyos tiempos de inter-renovaci\'on tienen media $\mu\leq\infty$, entonces
\begin{eqnarray}
t^{-1}N\left(t\right)\rightarrow 1/\mu,\textrm{ c.s. cuando }t\rightarrow\infty.
\end{eqnarray}

\end{Coro}


Considerar el proceso estoc\'astico de valores reales $\left\{Z\left(t\right):t\geq0\right\}$ en el mismo espacio de probabilidad que $N\left(t\right)$

\begin{Def}
Para el proceso $\left\{Z\left(t\right):t\geq0\right\}$ se define la fluctuaci\'on m\'axima de $Z\left(t\right)$ en el intervalo $\left(T_{n-1},T_{n}\right]$:
\begin{eqnarray*}
M_{n}=\sup_{T_{n-1}<t\leq T_{n}}|Z\left(t\right)-Z\left(T_{n-1}\right)|
\end{eqnarray*}
\end{Def}

\begin{Teo}
Sup\'ongase que $n^{-1}T_{n}\rightarrow\mu$ c.s. cuando $n\rightarrow\infty$, donde $\mu\leq\infty$ es una constante o variable aleatoria. Sea $a$ una constante o variable aleatoria que puede ser infinita cuando $\mu$ es finita, y considere las expresiones l\'imite:
\begin{eqnarray}
lim_{n\rightarrow\infty}n^{-1}Z\left(T_{n}\right)&=&a,\textrm{ c.s.}\\
lim_{t\rightarrow\infty}t^{-1}Z\left(t\right)&=&a/\mu,\textrm{ c.s.}
\end{eqnarray}
La segunda expresi\'on implica la primera. Conversamente, la primera implica la segunda si el proceso $Z\left(t\right)$ es creciente, o si $lim_{n\rightarrow\infty}n^{-1}M_{n}=0$ c.s.
\end{Teo}

\begin{Coro}
Si $N\left(t\right)$ es un proceso de renovaci\'on, y $\left(Z\left(T_{n}\right)-Z\left(T_{n-1}\right),M_{n}\right)$, para $n\geq1$, son variables aleatorias independientes e id\'enticamente distribuidas con media finita, entonces,
\begin{eqnarray}
lim_{t\rightarrow\infty}t^{-1}Z\left(t\right)\rightarrow\frac{\esp\left[Z\left(T_{1}\right)-Z\left(T_{0}\right)\right]}{\esp\left[T_{1}\right]},\textrm{ c.s. cuando  }t\rightarrow\infty.
\end{eqnarray}
\end{Coro}


Sup\'ongase que $N\left(t\right)$ es un proceso de renovaci\'on con distribuci\'on $F$ con media finita $\mu$.

\begin{Def}
La funci\'on de renovaci\'on asociada con la distribuci\'on $F$, del proceso $N\left(t\right)$, es
\begin{eqnarray*}
U\left(t\right)=\sum_{n=1}^{\infty}F^{n\star}\left(t\right),\textrm{   }t\geq0,
\end{eqnarray*}
donde $F^{0\star}\left(t\right)=\indora\left(t\geq0\right)$.
\end{Def}


\begin{Prop}
Sup\'ongase que la distribuci\'on de inter-renovaci\'on $F$ tiene densidad $f$. Entonces $U\left(t\right)$ tambi\'en tiene densidad, para $t>0$, y es $U^{'}\left(t\right)=\sum_{n=0}^{\infty}f^{n\star}\left(t\right)$. Adem\'as
\begin{eqnarray*}
\prob\left\{N\left(t\right)>N\left(t-\right)\right\}=0\textrm{,   }t\geq0.
\end{eqnarray*}
\end{Prop}

\begin{Def}
La Transformada de Laplace-Stieljes de $F$ est\'a dada por

\begin{eqnarray*}
\hat{F}\left(\alpha\right)=\int_{\rea_{+}}e^{-\alpha t}dF\left(t\right)\textrm{,  }\alpha\geq0.
\end{eqnarray*}
\end{Def}

Entonces

\begin{eqnarray*}
\hat{U}\left(\alpha\right)=\sum_{n=0}^{\infty}\hat{F^{n\star}}\left(\alpha\right)=\sum_{n=0}^{\infty}\hat{F}\left(\alpha\right)^{n}=\frac{1}{1-\hat{F}\left(\alpha\right)}.
\end{eqnarray*}


\begin{Prop}
La Transformada de Laplace $\hat{U}\left(\alpha\right)$ y $\hat{F}\left(\alpha\right)$ determina una a la otra de manera \'unica por la relaci\'on $\hat{U}\left(\alpha\right)=\frac{1}{1-\hat{F}\left(\alpha\right)}$.
\end{Prop}


\begin{Note}
Un proceso de renovaci\'on $N\left(t\right)$ cuyos tiempos de inter-renovaci\'on tienen media finita, es un proceso Poisson con tasa $\lambda$ si y s\'olo s\'i $\esp\left[U\left(t\right)\right]=\lambda t$, para $t\geq0$.
\end{Note}


\begin{Teo}
Sea $N\left(t\right)$ un proceso puntual simple con puntos de localizaci\'on $T_{n}$ tal que $\eta\left(t\right)=\esp\left[N\left(\right)\right]$ es finita para cada $t$. Entonces para cualquier funci\'on $f:\rea_{+}\rightarrow\rea$,
\begin{eqnarray*}
\esp\left[\sum_{n=1}^{N\left(\right)}f\left(T_{n}\right)\right]=\int_{\left(0,t\right]}f\left(s\right)d\eta\left(s\right)\textrm{,  }t\geq0,
\end{eqnarray*}
suponiendo que la integral exista. Adem\'as si $X_{1},X_{2},\ldots$ son variables aleatorias definidas en el mismo espacio de probabilidad que el proceso $N\left(t\right)$ tal que $\esp\left[X_{n}|T_{n}=s\right]=f\left(s\right)$, independiente de $n$. Entonces
\begin{eqnarray*}
\esp\left[\sum_{n=1}^{N\left(t\right)}X_{n}\right]=\int_{\left(0,t\right]}f\left(s\right)d\eta\left(s\right)\textrm{,  }t\geq0,
\end{eqnarray*} 
suponiendo que la integral exista. 
\end{Teo}

\begin{Coro}[Identidad de Wald para Renovaciones]
Para el proceso de renovaci\'on $N\left(t\right)$,
\begin{eqnarray*}
\esp\left[T_{N\left(t\right)+1}\right]=\mu\esp\left[N\left(t\right)+1\right]\textrm{,  }t\geq0,
\end{eqnarray*}  
\end{Coro}


\begin{Def}
Sea $h\left(t\right)$ funci\'on de valores reales en $\rea$ acotada en intervalos finitos e igual a cero para $t<0$ La ecuaci\'on de renovaci\'on para $h\left(t\right)$ y la distribuci\'on $F$ es

\begin{eqnarray}%\label{Ec.Renovacion}
H\left(t\right)=h\left(t\right)+\int_{\left[0,t\right]}H\left(t-s\right)dF\left(s\right)\textrm{,    }t\geq0,
\end{eqnarray}
donde $H\left(t\right)$ es una funci\'on de valores reales. Esto es $H=h+F\star H$. Decimos que $H\left(t\right)$ es soluci\'on de esta ecuaci\'on si satisface la ecuaci\'on, y es acotada en intervalos finitos e iguales a cero para $t<0$.
\end{Def}

\begin{Prop}
La funci\'on $U\star h\left(t\right)$ es la \'unica soluci\'on de la ecuaci\'on de renovaci\'on (\ref{Ec.Renovacion}).
\end{Prop}

\begin{Teo}[Teorema Renovaci\'on Elemental]
\begin{eqnarray*}
t^{-1}U\left(t\right)\rightarrow 1/\mu\textrm{,    cuando }t\rightarrow\infty.
\end{eqnarray*}
\end{Teo}



Sup\'ongase que $N\left(t\right)$ es un proceso de renovaci\'on con distribuci\'on $F$ con media finita $\mu$.

\begin{Def}
La funci\'on de renovaci\'on asociada con la distribuci\'on $F$, del proceso $N\left(t\right)$, es
\begin{eqnarray*}
U\left(t\right)=\sum_{n=1}^{\infty}F^{n\star}\left(t\right),\textrm{   }t\geq0,
\end{eqnarray*}
donde $F^{0\star}\left(t\right)=\indora\left(t\geq0\right)$.
\end{Def}


\begin{Prop}
Sup\'ongase que la distribuci\'on de inter-renovaci\'on $F$ tiene densidad $f$. Entonces $U\left(t\right)$ tambi\'en tiene densidad, para $t>0$, y es $U^{'}\left(t\right)=\sum_{n=0}^{\infty}f^{n\star}\left(t\right)$. Adem\'as
\begin{eqnarray*}
\prob\left\{N\left(t\right)>N\left(t-\right)\right\}=0\textrm{,   }t\geq0.
\end{eqnarray*}
\end{Prop}

\begin{Def}
La Transformada de Laplace-Stieljes de $F$ est\'a dada por

\begin{eqnarray*}
\hat{F}\left(\alpha\right)=\int_{\rea_{+}}e^{-\alpha t}dF\left(t\right)\textrm{,  }\alpha\geq0.
\end{eqnarray*}
\end{Def}

Entonces

\begin{eqnarray*}
\hat{U}\left(\alpha\right)=\sum_{n=0}^{\infty}\hat{F^{n\star}}\left(\alpha\right)=\sum_{n=0}^{\infty}\hat{F}\left(\alpha\right)^{n}=\frac{1}{1-\hat{F}\left(\alpha\right)}.
\end{eqnarray*}


\begin{Prop}
La Transformada de Laplace $\hat{U}\left(\alpha\right)$ y $\hat{F}\left(\alpha\right)$ determina una a la otra de manera \'unica por la relaci\'on $\hat{U}\left(\alpha\right)=\frac{1}{1-\hat{F}\left(\alpha\right)}$.
\end{Prop}


\begin{Note}
Un proceso de renovaci\'on $N\left(t\right)$ cuyos tiempos de inter-renovaci\'on tienen media finita, es un proceso Poisson con tasa $\lambda$ si y s\'olo s\'i $\esp\left[U\left(t\right)\right]=\lambda t$, para $t\geq0$.
\end{Note}


\begin{Teo}
Sea $N\left(t\right)$ un proceso puntual simple con puntos de localizaci\'on $T_{n}$ tal que $\eta\left(t\right)=\esp\left[N\left(\right)\right]$ es finita para cada $t$. Entonces para cualquier funci\'on $f:\rea_{+}\rightarrow\rea$,
\begin{eqnarray*}
\esp\left[\sum_{n=1}^{N\left(\right)}f\left(T_{n}\right)\right]=\int_{\left(0,t\right]}f\left(s\right)d\eta\left(s\right)\textrm{,  }t\geq0,
\end{eqnarray*}
suponiendo que la integral exista. Adem\'as si $X_{1},X_{2},\ldots$ son variables aleatorias definidas en el mismo espacio de probabilidad que el proceso $N\left(t\right)$ tal que $\esp\left[X_{n}|T_{n}=s\right]=f\left(s\right)$, independiente de $n$. Entonces
\begin{eqnarray*}
\esp\left[\sum_{n=1}^{N\left(t\right)}X_{n}\right]=\int_{\left(0,t\right]}f\left(s\right)d\eta\left(s\right)\textrm{,  }t\geq0,
\end{eqnarray*} 
suponiendo que la integral exista. 
\end{Teo}

\begin{Coro}[Identidad de Wald para Renovaciones]
Para el proceso de renovaci\'on $N\left(t\right)$,
\begin{eqnarray*}
\esp\left[T_{N\left(t\right)+1}\right]=\mu\esp\left[N\left(t\right)+1\right]\textrm{,  }t\geq0,
\end{eqnarray*}  
\end{Coro}


\begin{Def}
Sea $h\left(t\right)$ funci\'on de valores reales en $\rea$ acotada en intervalos finitos e igual a cero para $t<0$ La ecuaci\'on de renovaci\'on para $h\left(t\right)$ y la distribuci\'on $F$ es

\begin{eqnarray}%\label{Ec.Renovacion}
H\left(t\right)=h\left(t\right)+\int_{\left[0,t\right]}H\left(t-s\right)dF\left(s\right)\textrm{,    }t\geq0,
\end{eqnarray}
donde $H\left(t\right)$ es una funci\'on de valores reales. Esto es $H=h+F\star H$. Decimos que $H\left(t\right)$ es soluci\'on de esta ecuaci\'on si satisface la ecuaci\'on, y es acotada en intervalos finitos e iguales a cero para $t<0$.
\end{Def}

\begin{Prop}
La funci\'on $U\star h\left(t\right)$ es la \'unica soluci\'on de la ecuaci\'on de renovaci\'on (\ref{Ec.Renovacion}).
\end{Prop}

\begin{Teo}[Teorema Renovaci\'on Elemental]
\begin{eqnarray*}
t^{-1}U\left(t\right)\rightarrow 1/\mu\textrm{,    cuando }t\rightarrow\infty.
\end{eqnarray*}
\end{Teo}


\begin{Note} Una funci\'on $h:\rea_{+}\rightarrow\rea$ es Directamente Riemann Integrable en los siguientes casos:
\begin{itemize}
\item[a)] $h\left(t\right)\geq0$ es decreciente y Riemann Integrable.
\item[b)] $h$ es continua excepto posiblemente en un conjunto de Lebesgue de medida 0, y $|h\left(t\right)|\leq b\left(t\right)$, donde $b$ es DRI.
\end{itemize}
\end{Note}

\begin{Teo}[Teorema Principal de Renovaci\'on]
Si $F$ es no aritm\'etica y $h\left(t\right)$ es Directamente Riemann Integrable (DRI), entonces

\begin{eqnarray*}
lim_{t\rightarrow\infty}U\star h=\frac{1}{\mu}\int_{\rea_{+}}h\left(s\right)ds.
\end{eqnarray*}
\end{Teo}

\begin{Prop}
Cualquier funci\'on $H\left(t\right)$ acotada en intervalos finitos y que es 0 para $t<0$ puede expresarse como
\begin{eqnarray*}
H\left(t\right)=U\star h\left(t\right)\textrm{,  donde }h\left(t\right)=H\left(t\right)-F\star H\left(t\right)
\end{eqnarray*}
\end{Prop}

\begin{Def}
Un proceso estoc\'astico $X\left(t\right)$ es crudamente regenerativo en un tiempo aleatorio positivo $T$ si
\begin{eqnarray*}
\esp\left[X\left(T+t\right)|T\right]=\esp\left[X\left(t\right)\right]\textrm{, para }t\geq0,\end{eqnarray*}
y con las esperanzas anteriores finitas.
\end{Def}

\begin{Prop}
Sup\'ongase que $X\left(t\right)$ es un proceso crudamente regenerativo en $T$, que tiene distribuci\'on $F$. Si $\esp\left[X\left(t\right)\right]$ es acotado en intervalos finitos, entonces
\begin{eqnarray*}
\esp\left[X\left(t\right)\right]=U\star h\left(t\right)\textrm{,  donde }h\left(t\right)=\esp\left[X\left(t\right)\indora\left(T>t\right)\right].
\end{eqnarray*}
\end{Prop}

\begin{Teo}[Regeneraci\'on Cruda]
Sup\'ongase que $X\left(t\right)$ es un proceso con valores positivo crudamente regenerativo en $T$, y def\'inase $M=\sup\left\{|X\left(t\right)|:t\leq T\right\}$. Si $T$ es no aritm\'etico y $M$ y $MT$ tienen media finita, entonces
\begin{eqnarray*}
lim_{t\rightarrow\infty}\esp\left[X\left(t\right)\right]=\frac{1}{\mu}\int_{\rea_{+}}h\left(s\right)ds,
\end{eqnarray*}
donde $h\left(t\right)=\esp\left[X\left(t\right)\indora\left(T>t\right)\right]$.
\end{Teo}


\begin{Note} Una funci\'on $h:\rea_{+}\rightarrow\rea$ es Directamente Riemann Integrable en los siguientes casos:
\begin{itemize}
\item[a)] $h\left(t\right)\geq0$ es decreciente y Riemann Integrable.
\item[b)] $h$ es continua excepto posiblemente en un conjunto de Lebesgue de medida 0, y $|h\left(t\right)|\leq b\left(t\right)$, donde $b$ es DRI.
\end{itemize}
\end{Note}

\begin{Teo}[Teorema Principal de Renovaci\'on]
Si $F$ es no aritm\'etica y $h\left(t\right)$ es Directamente Riemann Integrable (DRI), entonces

\begin{eqnarray*}
lim_{t\rightarrow\infty}U\star h=\frac{1}{\mu}\int_{\rea_{+}}h\left(s\right)ds.
\end{eqnarray*}
\end{Teo}

\begin{Prop}
Cualquier funci\'on $H\left(t\right)$ acotada en intervalos finitos y que es 0 para $t<0$ puede expresarse como
\begin{eqnarray*}
H\left(t\right)=U\star h\left(t\right)\textrm{,  donde }h\left(t\right)=H\left(t\right)-F\star H\left(t\right)
\end{eqnarray*}
\end{Prop}

\begin{Def}
Un proceso estoc\'astico $X\left(t\right)$ es crudamente regenerativo en un tiempo aleatorio positivo $T$ si
\begin{eqnarray*}
\esp\left[X\left(T+t\right)|T\right]=\esp\left[X\left(t\right)\right]\textrm{, para }t\geq0,\end{eqnarray*}
y con las esperanzas anteriores finitas.
\end{Def}

\begin{Prop}
Sup\'ongase que $X\left(t\right)$ es un proceso crudamente regenerativo en $T$, que tiene distribuci\'on $F$. Si $\esp\left[X\left(t\right)\right]$ es acotado en intervalos finitos, entonces
\begin{eqnarray*}
\esp\left[X\left(t\right)\right]=U\star h\left(t\right)\textrm{,  donde }h\left(t\right)=\esp\left[X\left(t\right)\indora\left(T>t\right)\right].
\end{eqnarray*}
\end{Prop}

\begin{Teo}[Regeneraci\'on Cruda]
Sup\'ongase que $X\left(t\right)$ es un proceso con valores positivo crudamente regenerativo en $T$, y def\'inase $M=\sup\left\{|X\left(t\right)|:t\leq T\right\}$. Si $T$ es no aritm\'etico y $M$ y $MT$ tienen media finita, entonces
\begin{eqnarray*}
lim_{t\rightarrow\infty}\esp\left[X\left(t\right)\right]=\frac{1}{\mu}\int_{\rea_{+}}h\left(s\right)ds,
\end{eqnarray*}
donde $h\left(t\right)=\esp\left[X\left(t\right)\indora\left(T>t\right)\right]$.
\end{Teo}

\begin{Def}
Para el proceso $\left\{\left(N\left(t\right),X\left(t\right)\right):t\geq0\right\}$, sus trayectoria muestrales en el intervalo de tiempo $\left[T_{n-1},T_{n}\right)$ est\'an descritas por
\begin{eqnarray*}
\zeta_{n}=\left(\xi_{n},\left\{X\left(T_{n-1}+t\right):0\leq t<\xi_{n}\right\}\right)
\end{eqnarray*}
Este $\zeta_{n}$ es el $n$-\'esimo segmento del proceso. El proceso es regenerativo sobre los tiempos $T_{n}$ si sus segmentos $\zeta_{n}$ son independientes e id\'enticamennte distribuidos.
\end{Def}


\begin{Note}
Si $\tilde{X}\left(t\right)$ con espacio de estados $\tilde{S}$ es regenerativo sobre $T_{n}$, entonces $X\left(t\right)=f\left(\tilde{X}\left(t\right)\right)$ tambi\'en es regenerativo sobre $T_{n}$, para cualquier funci\'on $f:\tilde{S}\rightarrow S$.
\end{Note}

\begin{Note}
Los procesos regenerativos son crudamente regenerativos, pero no al rev\'es.
\end{Note}


\begin{Note}
Un proceso estoc\'astico a tiempo continuo o discreto es regenerativo si existe un proceso de renovaci\'on  tal que los segmentos del proceso entre tiempos de renovaci\'on sucesivos son i.i.d., es decir, para $\left\{X\left(t\right):t\geq0\right\}$ proceso estoc\'astico a tiempo continuo con espacio de estados $S$, espacio m\'etrico.
\end{Note}

Para $\left\{X\left(t\right):t\geq0\right\}$ Proceso Estoc\'astico a tiempo continuo con estado de espacios $S$, que es un espacio m\'etrico, con trayectorias continuas por la derecha y con l\'imites por la izquierda c.s. Sea $N\left(t\right)$ un proceso de renovaci\'on en $\rea_{+}$ definido en el mismo espacio de probabilidad que $X\left(t\right)$, con tiempos de renovaci\'on $T$ y tiempos de inter-renovaci\'on $\xi_{n}=T_{n}-T_{n-1}$, con misma distribuci\'on $F$ de media finita $\mu$.



\begin{Def}
Para el proceso $\left\{\left(N\left(t\right),X\left(t\right)\right):t\geq0\right\}$, sus trayectoria muestrales en el intervalo de tiempo $\left[T_{n-1},T_{n}\right)$ est\'an descritas por
\begin{eqnarray*}
\zeta_{n}=\left(\xi_{n},\left\{X\left(T_{n-1}+t\right):0\leq t<\xi_{n}\right\}\right)
\end{eqnarray*}
Este $\zeta_{n}$ es el $n$-\'esimo segmento del proceso. El proceso es regenerativo sobre los tiempos $T_{n}$ si sus segmentos $\zeta_{n}$ son independientes e id\'enticamennte distribuidos.
\end{Def}

\begin{Note}
Un proceso regenerativo con media de la longitud de ciclo finita es llamado positivo recurrente.
\end{Note}

\begin{Teo}[Procesos Regenerativos]
Suponga que el proceso
\end{Teo}


\begin{Def}[Renewal Process Trinity]
Para un proceso de renovaci\'on $N\left(t\right)$, los siguientes procesos proveen de informaci\'on sobre los tiempos de renovaci\'on.
\begin{itemize}
\item $A\left(t\right)=t-T_{N\left(t\right)}$, el tiempo de recurrencia hacia atr\'as al tiempo $t$, que es el tiempo desde la \'ultima renovaci\'on para $t$.

\item $B\left(t\right)=T_{N\left(t\right)+1}-t$, el tiempo de recurrencia hacia adelante al tiempo $t$, residual del tiempo de renovaci\'on, que es el tiempo para la pr\'oxima renovaci\'on despu\'es de $t$.

\item $L\left(t\right)=\xi_{N\left(t\right)+1}=A\left(t\right)+B\left(t\right)$, la longitud del intervalo de renovaci\'on que contiene a $t$.
\end{itemize}
\end{Def}

\begin{Note}
El proceso tridimensional $\left(A\left(t\right),B\left(t\right),L\left(t\right)\right)$ es regenerativo sobre $T_{n}$, y por ende cada proceso lo es. Cada proceso $A\left(t\right)$ y $B\left(t\right)$ son procesos de MArkov a tiempo continuo con trayectorias continuas por partes en el espacio de estados $\rea_{+}$. Una expresi\'on conveniente para su distribuci\'on conjunta es, para $0\leq x<t,y\geq0$
\begin{equation}\label{NoRenovacion}
P\left\{A\left(t\right)>x,B\left(t\right)>y\right\}=
P\left\{N\left(t+y\right)-N\left((t-x)\right)=0\right\}
\end{equation}
\end{Note}

\begin{Ejem}[Tiempos de recurrencia Poisson]
Si $N\left(t\right)$ es un proceso Poisson con tasa $\lambda$, entonces de la expresi\'on (\ref{NoRenovacion}) se tiene que

\begin{eqnarray*}
\begin{array}{lc}
P\left\{A\left(t\right)>x,B\left(t\right)>y\right\}=e^{-\lambda\left(x+y\right)},&0\leq x<t,y\geq0,
\end{array}
\end{eqnarray*}
que es la probabilidad Poisson de no renovaciones en un intervalo de longitud $x+y$.

\end{Ejem}

\begin{Note}
Una cadena de Markov erg\'odica tiene la propiedad de ser estacionaria si la distribuci\'on de su estado al tiempo $0$ es su distribuci\'on estacionaria.
\end{Note}


\begin{Def}
Un proceso estoc\'astico a tiempo continuo $\left\{X\left(t\right):t\geq0\right\}$ en un espacio general es estacionario si sus distribuciones finito dimensionales son invariantes bajo cualquier  traslado: para cada $0\leq s_{1}<s_{2}<\cdots<s_{k}$ y $t\geq0$,
\begin{eqnarray*}
\left(X\left(s_{1}+t\right),\ldots,X\left(s_{k}+t\right)\right)=_{d}\left(X\left(s_{1}\right),\ldots,X\left(s_{k}\right)\right).
\end{eqnarray*}
\end{Def}

\begin{Note}
Un proceso de Markov es estacionario si $X\left(t\right)=_{d}X\left(0\right)$, $t\geq0$.
\end{Note}

Considerese el proceso $N\left(t\right)=\sum_{n}\indora\left(\tau_{n}\leq t\right)$ en $\rea_{+}$, con puntos $0<\tau_{1}<\tau_{2}<\cdots$.

\begin{Prop}
Si $N$ es un proceso puntual estacionario y $\esp\left[N\left(1\right)\right]<\infty$, entonces $\esp\left[N\left(t\right)\right]=t\esp\left[N\left(1\right)\right]$, $t\geq0$

\end{Prop}

\begin{Teo}
Los siguientes enunciados son equivalentes
\begin{itemize}
\item[i)] El proceso retardado de renovaci\'on $N$ es estacionario.

\item[ii)] EL proceso de tiempos de recurrencia hacia adelante $B\left(t\right)$ es estacionario.


\item[iii)] $\esp\left[N\left(t\right)\right]=t/\mu$,


\item[iv)] $G\left(t\right)=F_{e}\left(t\right)=\frac{1}{\mu}\int_{0}^{t}\left[1-F\left(s\right)\right]ds$
\end{itemize}
Cuando estos enunciados son ciertos, $P\left\{B\left(t\right)\leq x\right\}=F_{e}\left(x\right)$, para $t,x\geq0$.

\end{Teo}

\begin{Note}
Una consecuencia del teorema anterior es que el Proceso Poisson es el \'unico proceso sin retardo que es estacionario.
\end{Note}

\begin{Coro}
El proceso de renovaci\'on $N\left(t\right)$ sin retardo, y cuyos tiempos de inter renonaci\'on tienen media finita, es estacionario si y s\'olo si es un proceso Poisson.

\end{Coro}


%________________________________________________________________________
%\subsection{Procesos Regenerativos}
%________________________________________________________________________



\begin{Note}
Si $\tilde{X}\left(t\right)$ con espacio de estados $\tilde{S}$ es regenerativo sobre $T_{n}$, entonces $X\left(t\right)=f\left(\tilde{X}\left(t\right)\right)$ tambi\'en es regenerativo sobre $T_{n}$, para cualquier funci\'on $f:\tilde{S}\rightarrow S$.
\end{Note}

\begin{Note}
Los procesos regenerativos son crudamente regenerativos, pero no al rev\'es.
\end{Note}
%\subsection*{Procesos Regenerativos: Sigman\cite{Sigman1}}
\begin{Def}[Definici\'on Cl\'asica]
Un proceso estoc\'astico $X=\left\{X\left(t\right):t\geq0\right\}$ es llamado regenerativo is existe una variable aleatoria $R_{1}>0$ tal que
\begin{itemize}
\item[i)] $\left\{X\left(t+R_{1}\right):t\geq0\right\}$ es independiente de $\left\{\left\{X\left(t\right):t<R_{1}\right\},\right\}$
\item[ii)] $\left\{X\left(t+R_{1}\right):t\geq0\right\}$ es estoc\'asticamente equivalente a $\left\{X\left(t\right):t>0\right\}$
\end{itemize}

Llamamos a $R_{1}$ tiempo de regeneraci\'on, y decimos que $X$ se regenera en este punto.
\end{Def}

$\left\{X\left(t+R_{1}\right)\right\}$ es regenerativo con tiempo de regeneraci\'on $R_{2}$, independiente de $R_{1}$ pero con la misma distribuci\'on que $R_{1}$. Procediendo de esta manera se obtiene una secuencia de variables aleatorias independientes e id\'enticamente distribuidas $\left\{R_{n}\right\}$ llamados longitudes de ciclo. Si definimos a $Z_{k}\equiv R_{1}+R_{2}+\cdots+R_{k}$, se tiene un proceso de renovaci\'on llamado proceso de renovaci\'on encajado para $X$.




\begin{Def}
Para $x$ fijo y para cada $t\geq0$, sea $I_{x}\left(t\right)=1$ si $X\left(t\right)\leq x$,  $I_{x}\left(t\right)=0$ en caso contrario, y def\'inanse los tiempos promedio
\begin{eqnarray*}
\overline{X}&=&lim_{t\rightarrow\infty}\frac{1}{t}\int_{0}^{\infty}X\left(u\right)du\\
\prob\left(X_{\infty}\leq x\right)&=&lim_{t\rightarrow\infty}\frac{1}{t}\int_{0}^{\infty}I_{x}\left(u\right)du,
\end{eqnarray*}
cuando estos l\'imites existan.
\end{Def}

Como consecuencia del teorema de Renovaci\'on-Recompensa, se tiene que el primer l\'imite  existe y es igual a la constante
\begin{eqnarray*}
\overline{X}&=&\frac{\esp\left[\int_{0}^{R_{1}}X\left(t\right)dt\right]}{\esp\left[R_{1}\right]},
\end{eqnarray*}
suponiendo que ambas esperanzas son finitas.

\begin{Note}
\begin{itemize}
\item[a)] Si el proceso regenerativo $X$ es positivo recurrente y tiene trayectorias muestrales no negativas, entonces la ecuaci\'on anterior es v\'alida.
\item[b)] Si $X$ es positivo recurrente regenerativo, podemos construir una \'unica versi\'on estacionaria de este proceso, $X_{e}=\left\{X_{e}\left(t\right)\right\}$, donde $X_{e}$ es un proceso estoc\'astico regenerativo y estrictamente estacionario, con distribuci\'on marginal distribuida como $X_{\infty}$
\end{itemize}
\end{Note}

%________________________________________________________________________
%\subsection{Procesos Regenerativos}
%________________________________________________________________________

Para $\left\{X\left(t\right):t\geq0\right\}$ Proceso Estoc\'astico a tiempo continuo con estado de espacios $S$, que es un espacio m\'etrico, con trayectorias continuas por la derecha y con l\'imites por la izquierda c.s. Sea $N\left(t\right)$ un proceso de renovaci\'on en $\rea_{+}$ definido en el mismo espacio de probabilidad que $X\left(t\right)$, con tiempos de renovaci\'on $T$ y tiempos de inter-renovaci\'on $\xi_{n}=T_{n}-T_{n-1}$, con misma distribuci\'on $F$ de media finita $\mu$.



\begin{Def}
Para el proceso $\left\{\left(N\left(t\right),X\left(t\right)\right):t\geq0\right\}$, sus trayectoria muestrales en el intervalo de tiempo $\left[T_{n-1},T_{n}\right)$ est\'an descritas por
\begin{eqnarray*}
\zeta_{n}=\left(\xi_{n},\left\{X\left(T_{n-1}+t\right):0\leq t<\xi_{n}\right\}\right)
\end{eqnarray*}
Este $\zeta_{n}$ es el $n$-\'esimo segmento del proceso. El proceso es regenerativo sobre los tiempos $T_{n}$ si sus segmentos $\zeta_{n}$ son independientes e id\'enticamennte distribuidos.
\end{Def}


\begin{Note}
Si $\tilde{X}\left(t\right)$ con espacio de estados $\tilde{S}$ es regenerativo sobre $T_{n}$, entonces $X\left(t\right)=f\left(\tilde{X}\left(t\right)\right)$ tambi\'en es regenerativo sobre $T_{n}$, para cualquier funci\'on $f:\tilde{S}\rightarrow S$.
\end{Note}

\begin{Note}
Los procesos regenerativos son crudamente regenerativos, pero no al rev\'es.
\end{Note}

\begin{Def}[Definici\'on Cl\'asica]
Un proceso estoc\'astico $X=\left\{X\left(t\right):t\geq0\right\}$ es llamado regenerativo is existe una variable aleatoria $R_{1}>0$ tal que
\begin{itemize}
\item[i)] $\left\{X\left(t+R_{1}\right):t\geq0\right\}$ es independiente de $\left\{\left\{X\left(t\right):t<R_{1}\right\},\right\}$
\item[ii)] $\left\{X\left(t+R_{1}\right):t\geq0\right\}$ es estoc\'asticamente equivalente a $\left\{X\left(t\right):t>0\right\}$
\end{itemize}

Llamamos a $R_{1}$ tiempo de regeneraci\'on, y decimos que $X$ se regenera en este punto.
\end{Def}

$\left\{X\left(t+R_{1}\right)\right\}$ es regenerativo con tiempo de regeneraci\'on $R_{2}$, independiente de $R_{1}$ pero con la misma distribuci\'on que $R_{1}$. Procediendo de esta manera se obtiene una secuencia de variables aleatorias independientes e id\'enticamente distribuidas $\left\{R_{n}\right\}$ llamados longitudes de ciclo. Si definimos a $Z_{k}\equiv R_{1}+R_{2}+\cdots+R_{k}$, se tiene un proceso de renovaci\'on llamado proceso de renovaci\'on encajado para $X$.

\begin{Note}
Un proceso regenerativo con media de la longitud de ciclo finita es llamado positivo recurrente.
\end{Note}


\begin{Def}
Para $x$ fijo y para cada $t\geq0$, sea $I_{x}\left(t\right)=1$ si $X\left(t\right)\leq x$,  $I_{x}\left(t\right)=0$ en caso contrario, y def\'inanse los tiempos promedio
\begin{eqnarray*}
\overline{X}&=&lim_{t\rightarrow\infty}\frac{1}{t}\int_{0}^{\infty}X\left(u\right)du\\
\prob\left(X_{\infty}\leq x\right)&=&lim_{t\rightarrow\infty}\frac{1}{t}\int_{0}^{\infty}I_{x}\left(u\right)du,
\end{eqnarray*}
cuando estos l\'imites existan.
\end{Def}

Como consecuencia del teorema de Renovaci\'on-Recompensa, se tiene que el primer l\'imite  existe y es igual a la constante
\begin{eqnarray*}
\overline{X}&=&\frac{\esp\left[\int_{0}^{R_{1}}X\left(t\right)dt\right]}{\esp\left[R_{1}\right]},
\end{eqnarray*}
suponiendo que ambas esperanzas son finitas.

\begin{Note}
\begin{itemize}
\item[a)] Si el proceso regenerativo $X$ es positivo recurrente y tiene trayectorias muestrales no negativas, entonces la ecuaci\'on anterior es v\'alida.
\item[b)] Si $X$ es positivo recurrente regenerativo, podemos construir una \'unica versi\'on estacionaria de este proceso, $X_{e}=\left\{X_{e}\left(t\right)\right\}$, donde $X_{e}$ es un proceso estoc\'astico regenerativo y estrictamente estacionario, con distribuci\'on marginal distribuida como $X_{\infty}$
\end{itemize}
\end{Note}

%__________________________________________________________________________________________
%\subsection{Procesos Regenerativos Estacionarios - Stidham \cite{Stidham}}
%__________________________________________________________________________________________


Un proceso estoc\'astico a tiempo continuo $\left\{V\left(t\right),t\geq0\right\}$ es un proceso regenerativo si existe una sucesi\'on de variables aleatorias independientes e id\'enticamente distribuidas $\left\{X_{1},X_{2},\ldots\right\}$, sucesi\'on de renovaci\'on, tal que para cualquier conjunto de Borel $A$, 

\begin{eqnarray*}
\prob\left\{V\left(t\right)\in A|X_{1}+X_{2}+\cdots+X_{R\left(t\right)}=s,\left\{V\left(\tau\right),\tau<s\right\}\right\}=\prob\left\{V\left(t-s\right)\in A|X_{1}>t-s\right\},
\end{eqnarray*}
para todo $0\leq s\leq t$, donde $R\left(t\right)=\max\left\{X_{1}+X_{2}+\cdots+X_{j}\leq t\right\}=$n\'umero de renovaciones ({\emph{puntos de regeneraci\'on}}) que ocurren en $\left[0,t\right]$. El intervalo $\left[0,X_{1}\right)$ es llamado {\emph{primer ciclo de regeneraci\'on}} de $\left\{V\left(t \right),t\geq0\right\}$, $\left[X_{1},X_{1}+X_{2}\right)$ el {\emph{segundo ciclo de regeneraci\'on}}, y as\'i sucesivamente.

Sea $X=X_{1}$ y sea $F$ la funci\'on de distrbuci\'on de $X$


\begin{Def}
Se define el proceso estacionario, $\left\{V^{*}\left(t\right),t\geq0\right\}$, para $\left\{V\left(t\right),t\geq0\right\}$ por

\begin{eqnarray*}
\prob\left\{V\left(t\right)\in A\right\}=\frac{1}{\esp\left[X\right]}\int_{0}^{\infty}\prob\left\{V\left(t+x\right)\in A|X>x\right\}\left(1-F\left(x\right)\right)dx,
\end{eqnarray*} 
para todo $t\geq0$ y todo conjunto de Borel $A$.
\end{Def}

\begin{Def}
Una distribuci\'on se dice que es {\emph{aritm\'etica}} si todos sus puntos de incremento son m\'ultiplos de la forma $0,\lambda, 2\lambda,\ldots$ para alguna $\lambda>0$ entera.
\end{Def}


\begin{Def}
Una modificaci\'on medible de un proceso $\left\{V\left(t\right),t\geq0\right\}$, es una versi\'on de este, $\left\{V\left(t,w\right)\right\}$ conjuntamente medible para $t\geq0$ y para $w\in S$, $S$ espacio de estados para $\left\{V\left(t\right),t\geq0\right\}$.
\end{Def}

\begin{Teo}
Sea $\left\{V\left(t\right),t\geq\right\}$ un proceso regenerativo no negativo con modificaci\'on medible. Sea $\esp\left[X\right]<\infty$. Entonces el proceso estacionario dado por la ecuaci\'on anterior est\'a bien definido y tiene funci\'on de distribuci\'on independiente de $t$, adem\'as
\begin{itemize}
\item[i)] \begin{eqnarray*}
\esp\left[V^{*}\left(0\right)\right]&=&\frac{\esp\left[\int_{0}^{X}V\left(s\right)ds\right]}{\esp\left[X\right]}\end{eqnarray*}
\item[ii)] Si $\esp\left[V^{*}\left(0\right)\right]<\infty$, equivalentemente, si $\esp\left[\int_{0}^{X}V\left(s\right)ds\right]<\infty$,entonces
\begin{eqnarray*}
\frac{\int_{0}^{t}V\left(s\right)ds}{t}\rightarrow\frac{\esp\left[\int_{0}^{X}V\left(s\right)ds\right]}{\esp\left[X\right]}
\end{eqnarray*}
con probabilidad 1 y en media, cuando $t\rightarrow\infty$.
\end{itemize}
\end{Teo}
%
%___________________________________________________________________________________________
%\vspace{5.5cm}
%\chapter{Cadenas de Markov estacionarias}
%\vspace{-1.0cm}


%__________________________________________________________________________________________
%\subsection{Procesos Regenerativos Estacionarios - Stidham \cite{Stidham}}
%__________________________________________________________________________________________


Un proceso estoc\'astico a tiempo continuo $\left\{V\left(t\right),t\geq0\right\}$ es un proceso regenerativo si existe una sucesi\'on de variables aleatorias independientes e id\'enticamente distribuidas $\left\{X_{1},X_{2},\ldots\right\}$, sucesi\'on de renovaci\'on, tal que para cualquier conjunto de Borel $A$, 

\begin{eqnarray*}
\prob\left\{V\left(t\right)\in A|X_{1}+X_{2}+\cdots+X_{R\left(t\right)}=s,\left\{V\left(\tau\right),\tau<s\right\}\right\}=\prob\left\{V\left(t-s\right)\in A|X_{1}>t-s\right\},
\end{eqnarray*}
para todo $0\leq s\leq t$, donde $R\left(t\right)=\max\left\{X_{1}+X_{2}+\cdots+X_{j}\leq t\right\}=$n\'umero de renovaciones ({\emph{puntos de regeneraci\'on}}) que ocurren en $\left[0,t\right]$. El intervalo $\left[0,X_{1}\right)$ es llamado {\emph{primer ciclo de regeneraci\'on}} de $\left\{V\left(t \right),t\geq0\right\}$, $\left[X_{1},X_{1}+X_{2}\right)$ el {\emph{segundo ciclo de regeneraci\'on}}, y as\'i sucesivamente.

Sea $X=X_{1}$ y sea $F$ la funci\'on de distrbuci\'on de $X$


\begin{Def}
Se define el proceso estacionario, $\left\{V^{*}\left(t\right),t\geq0\right\}$, para $\left\{V\left(t\right),t\geq0\right\}$ por

\begin{eqnarray*}
\prob\left\{V\left(t\right)\in A\right\}=\frac{1}{\esp\left[X\right]}\int_{0}^{\infty}\prob\left\{V\left(t+x\right)\in A|X>x\right\}\left(1-F\left(x\right)\right)dx,
\end{eqnarray*} 
para todo $t\geq0$ y todo conjunto de Borel $A$.
\end{Def}

\begin{Def}
Una distribuci\'on se dice que es {\emph{aritm\'etica}} si todos sus puntos de incremento son m\'ultiplos de la forma $0,\lambda, 2\lambda,\ldots$ para alguna $\lambda>0$ entera.
\end{Def}


\begin{Def}
Una modificaci\'on medible de un proceso $\left\{V\left(t\right),t\geq0\right\}$, es una versi\'on de este, $\left\{V\left(t,w\right)\right\}$ conjuntamente medible para $t\geq0$ y para $w\in S$, $S$ espacio de estados para $\left\{V\left(t\right),t\geq0\right\}$.
\end{Def}

\begin{Teo}
Sea $\left\{V\left(t\right),t\geq\right\}$ un proceso regenerativo no negativo con modificaci\'on medible. Sea $\esp\left[X\right]<\infty$. Entonces el proceso estacionario dado por la ecuaci\'on anterior est\'a bien definido y tiene funci\'on de distribuci\'on independiente de $t$, adem\'as
\begin{itemize}
\item[i)] \begin{eqnarray*}
\esp\left[V^{*}\left(0\right)\right]&=&\frac{\esp\left[\int_{0}^{X}V\left(s\right)ds\right]}{\esp\left[X\right]}\end{eqnarray*}
\item[ii)] Si $\esp\left[V^{*}\left(0\right)\right]<\infty$, equivalentemente, si $\esp\left[\int_{0}^{X}V\left(s\right)ds\right]<\infty$,entonces
\begin{eqnarray*}
\frac{\int_{0}^{t}V\left(s\right)ds}{t}\rightarrow\frac{\esp\left[\int_{0}^{X}V\left(s\right)ds\right]}{\esp\left[X\right]}
\end{eqnarray*}
con probabilidad 1 y en media, cuando $t\rightarrow\infty$.
\end{itemize}
\end{Teo}

Para $\left\{X\left(t\right):t\geq0\right\}$ Proceso Estoc\'astico a tiempo continuo con estado de espacios $S$, que es un espacio m\'etrico, con trayectorias continuas por la derecha y con l\'imites por la izquierda c.s. Sea $N\left(t\right)$ un proceso de renovaci\'on en $\rea_{+}$ definido en el mismo espacio de probabilidad que $X\left(t\right)$, con tiempos de renovaci\'on $T$ y tiempos de inter-renovaci\'on $\xi_{n}=T_{n}-T_{n-1}$, con misma distribuci\'on $F$ de media finita $\mu$.


%______________________________________________________________________
%\subsection{Ejemplos, Notas importantes}


Sean $T_{1},T_{2},\ldots$ los puntos donde las longitudes de las colas de la red de sistemas de visitas c\'iclicas son cero simult\'aneamente, cuando la cola $Q_{j}$ es visitada por el servidor para dar servicio, es decir, $L_{1}\left(T_{i}\right)=0,L_{2}\left(T_{i}\right)=0,\hat{L}_{1}\left(T_{i}\right)=0$ y $\hat{L}_{2}\left(T_{i}\right)=0$, a estos puntos se les denominar\'a puntos regenerativos. Sea la funci\'on generadora de momentos para $L_{i}$, el n\'umero de usuarios en la cola $Q_{i}\left(z\right)$ en cualquier momento, est\'a dada por el tiempo promedio de $z^{L_{i}\left(t\right)}$ sobre el ciclo regenerativo definido anteriormente:

\begin{eqnarray*}
Q_{i}\left(z\right)&=&\esp\left[z^{L_{i}\left(t\right)}\right]=\frac{\esp\left[\sum_{m=1}^{M_{i}}\sum_{t=\tau_{i}\left(m\right)}^{\tau_{i}\left(m+1\right)-1}z^{L_{i}\left(t\right)}\right]}{\esp\left[\sum_{m=1}^{M_{i}}\tau_{i}\left(m+1\right)-\tau_{i}\left(m\right)\right]}
\end{eqnarray*}

$M_{i}$ es un tiempo de paro en el proceso regenerativo con $\esp\left[M_{i}\right]<\infty$\footnote{En Stidham\cite{Stidham} y Heyman  se muestra que una condici\'on suficiente para que el proceso regenerativo 
estacionario sea un procesoo estacionario es que el valor esperado del tiempo del ciclo regenerativo sea finito, es decir: $\esp\left[\sum_{m=1}^{M_{i}}C_{i}^{(m)}\right]<\infty$, como cada $C_{i}^{(m)}$ contiene intervalos de r\'eplica positivos, se tiene que $\esp\left[M_{i}\right]<\infty$, adem\'as, como $M_{i}>0$, se tiene que la condici\'on anterior es equivalente a tener que $\esp\left[C_{i}\right]<\infty$,
por lo tanto una condici\'on suficiente para la existencia del proceso regenerativo est\'a dada por $\sum_{k=1}^{N}\mu_{k}<1.$}, se sigue del lema de Wald que:


\begin{eqnarray*}
\esp\left[\sum_{m=1}^{M_{i}}\sum_{t=\tau_{i}\left(m\right)}^{\tau_{i}\left(m+1\right)-1}z^{L_{i}\left(t\right)}\right]&=&\esp\left[M_{i}\right]\esp\left[\sum_{t=\tau_{i}\left(m\right)}^{\tau_{i}\left(m+1\right)-1}z^{L_{i}\left(t\right)}\right]\\
\esp\left[\sum_{m=1}^{M_{i}}\tau_{i}\left(m+1\right)-\tau_{i}\left(m\right)\right]&=&\esp\left[M_{i}\right]\esp\left[\tau_{i}\left(m+1\right)-\tau_{i}\left(m\right)\right]
\end{eqnarray*}

por tanto se tiene que


\begin{eqnarray*}
Q_{i}\left(z\right)&=&\frac{\esp\left[\sum_{t=\tau_{i}\left(m\right)}^{\tau_{i}\left(m+1\right)-1}z^{L_{i}\left(t\right)}\right]}{\esp\left[\tau_{i}\left(m+1\right)-\tau_{i}\left(m\right)\right]}
\end{eqnarray*}

observar que el denominador es simplemente la duraci\'on promedio del tiempo del ciclo.


Haciendo las siguientes sustituciones en la ecuaci\'on (\ref{Corolario2}): $n\rightarrow t-\tau_{i}\left(m\right)$, $T \rightarrow \overline{\tau}_{i}\left(m\right)-\tau_{i}\left(m\right)$, $L_{n}\rightarrow L_{i}\left(t\right)$ y $F\left(z\right)=\esp\left[z^{L_{0}}\right]\rightarrow F_{i}\left(z\right)=\esp\left[z^{L_{i}\tau_{i}\left(m\right)}\right]$, se puede ver que

\begin{eqnarray}\label{Eq.Arribos.Primera}
\esp\left[\sum_{n=0}^{T-1}z^{L_{n}}\right]=
\esp\left[\sum_{t=\tau_{i}\left(m\right)}^{\overline{\tau}_{i}\left(m\right)-1}z^{L_{i}\left(t\right)}\right]
=z\frac{F_{i}\left(z\right)-1}{z-P_{i}\left(z\right)}
\end{eqnarray}

Por otra parte durante el tiempo de intervisita para la cola $i$, $L_{i}\left(t\right)$ solamente se incrementa de manera que el incremento por intervalo de tiempo est\'a dado por la funci\'on generadora de probabilidades de $P_{i}\left(z\right)$, por tanto la suma sobre el tiempo de intervisita puede evaluarse como:

\begin{eqnarray*}
\esp\left[\sum_{t=\tau_{i}\left(m\right)}^{\tau_{i}\left(m+1\right)-1}z^{L_{i}\left(t\right)}\right]&=&\esp\left[\sum_{t=\tau_{i}\left(m\right)}^{\tau_{i}\left(m+1\right)-1}\left\{P_{i}\left(z\right)\right\}^{t-\overline{\tau}_{i}\left(m\right)}\right]=\frac{1-\esp\left[\left\{P_{i}\left(z\right)\right\}^{\tau_{i}\left(m+1\right)-\overline{\tau}_{i}\left(m\right)}\right]}{1-P_{i}\left(z\right)}\\
&=&\frac{1-I_{i}\left[P_{i}\left(z\right)\right]}{1-P_{i}\left(z\right)}
\end{eqnarray*}
por tanto

\begin{eqnarray*}
\esp\left[\sum_{t=\tau_{i}\left(m\right)}^{\tau_{i}\left(m+1\right)-1}z^{L_{i}\left(t\right)}\right]&=&
\frac{1-F_{i}\left(z\right)}{1-P_{i}\left(z\right)}
\end{eqnarray*}

Por lo tanto

\begin{eqnarray*}
Q_{i}\left(z\right)&=&\frac{\esp\left[\sum_{t=\tau_{i}\left(m\right)}^{\tau_{i}
\left(m+1\right)-1}z^{L_{i}\left(t\right)}\right]}{\esp\left[\tau_{i}\left(m+1\right)-\tau_{i}\left(m\right)\right]}\\
&=&\frac{1}{\esp\left[\tau_{i}\left(m+1\right)-\tau_{i}\left(m\right)\right]}
\left\{
\esp\left[\sum_{t=\tau_{i}\left(m\right)}^{\overline{\tau}_{i}\left(m\right)-1}
z^{L_{i}\left(t\right)}\right]
+\esp\left[\sum_{t=\overline{\tau}_{i}\left(m\right)}^{\tau_{i}\left(m+1\right)-1}
z^{L_{i}\left(t\right)}\right]\right\}\\
&=&\frac{1}{\esp\left[\tau_{i}\left(m+1\right)-\tau_{i}\left(m\right)\right]}
\left\{
z\frac{F_{i}\left(z\right)-1}{z-P_{i}\left(z\right)}+\frac{1-F_{i}\left(z\right)}
{1-P_{i}\left(z\right)}
\right\}
\end{eqnarray*}

es decir

\begin{equation}
Q_{i}\left(z\right)=\frac{1}{\esp\left[C_{i}\right]}\cdot\frac{1-F_{i}\left(z\right)}{P_{i}\left(z\right)-z}\cdot\frac{\left(1-z\right)P_{i}\left(z\right)}{1-P_{i}\left(z\right)}
\end{equation}

\begin{Teo}
Dada una Red de Sistemas de Visitas C\'iclicas (RSVC), conformada por dos Sistemas de Visitas C\'iclicas (SVC), donde cada uno de ellos consta de dos colas tipo $M/M/1$. Los dos sistemas est\'an comunicados entre s\'i por medio de la transferencia de usuarios entre las colas $Q_{1}\leftrightarrow Q_{3}$ y $Q_{2}\leftrightarrow Q_{4}$. Se definen los eventos para los procesos de arribos al tiempo $t$, $A_{j}\left(t\right)=\left\{0 \textrm{ arribos en }Q_{j}\textrm{ al tiempo }t\right\}$ para alg\'un tiempo $t\geq0$ y $Q_{j}$ la cola $j$-\'esima en la RSVC, para $j=1,2,3,4$.  Existe un intervalo $I\neq\emptyset$ tal que para $T^{*}\in I$, tal que $\prob\left\{A_{1}\left(T^{*}\right),A_{2}\left(Tt^{*}\right),
A_{3}\left(T^{*}\right),A_{4}\left(T^{*}\right)|T^{*}\in I\right\}>0$.
\end{Teo}

\begin{proof}
Sin p\'erdida de generalidad podemos considerar como base del an\'alisis a la cola $Q_{1}$ del primer sistema que conforma la RSVC.

Sea $n>0$, ciclo en el primer sistema en el que se sabe que $L_{j}\left(\overline{\tau}_{1}\left(n\right)\right)=0$, pues la pol\'itica de servicio con que atienden los servidores es la exhaustiva. Como es sabido, para trasladarse a la siguiente cola, el servidor incurre en un tiempo de traslado $r_{1}\left(n\right)>0$, entonces tenemos que $\tau_{2}\left(n\right)=\overline{\tau}_{1}\left(n\right)+r_{1}\left(n\right)$.


Definamos el intervalo $I_{1}\equiv\left[\overline{\tau}_{1}\left(n\right),\tau_{2}\left(n\right)\right]$ de longitud $\xi_{1}=r_{1}\left(n\right)$. Dado que los tiempos entre arribo son exponenciales con tasa $\tilde{\mu}_{1}=\mu_{1}+\hat{\mu}_{1}$ ($\mu_{1}$ son los arribos a $Q_{1}$ por primera vez al sistema, mientras que $\hat{\mu}_{1}$ son los arribos de traslado procedentes de $Q_{3}$) se tiene que la probabilidad del evento $A_{1}\left(t\right)$ est\'a dada por 

\begin{equation}
\prob\left\{A_{1}\left(t\right)|t\in I_{1}\left(n\right)\right\}=e^{-\tilde{\mu}_{1}\xi_{1}\left(n\right)}.
\end{equation} 

Por otra parte, para la cola $Q_{2}$, el tiempo $\overline{\tau}_{2}\left(n-1\right)$ es tal que $L_{2}\left(\overline{\tau}_{2}\left(n-1\right)\right)=0$, es decir, es el tiempo en que la cola queda totalmente vac\'ia en el ciclo anterior a $n$. Entonces tenemos un sgundo intervalo $I_{2}\equiv\left[\overline{\tau}_{2}\left(n-1\right),\tau_{2}\left(n\right)\right]$. Por lo tanto la probabilidad del evento $A_{2}\left(t\right)$ tiene probabilidad dada por

\begin{equation}
\prob\left\{A_{2}\left(t\right)|t\in I_{2}\left(n\right)\right\}=e^{-\tilde{\mu}_{2}\xi_{2}\left(n\right)},
\end{equation} 

donde $\xi_{2}\left(n\right)=\tau_{2}\left(n\right)-\overline{\tau}_{2}\left(n-1\right)$.



Entonces, se tiene que

\begin{eqnarray*}
\prob\left\{A_{1}\left(t\right),A_{2}\left(t\right)|t\in I_{1}\left(n\right)\right\}&=&
\prob\left\{A_{1}\left(t\right)|t\in I_{1}\left(n\right)\right\}
\prob\left\{A_{2}\left(t\right)|t\in I_{1}\left(n\right)\right\}\\
&\geq&
\prob\left\{A_{1}\left(t\right)|t\in I_{1}\left(n\right)\right\}
\prob\left\{A_{2}\left(t\right)|t\in I_{2}\left(n\right)\right\}\\
&=&e^{-\tilde{\mu}_{1}\xi_{1}\left(n\right)}e^{-\tilde{\mu}_{2}\xi_{2}\left(n\right)}
=e^{-\left[\tilde{\mu}_{1}\xi_{1}\left(n\right)+\tilde{\mu}_{2}\xi_{2}\left(n\right)\right]}.
\end{eqnarray*}


es decir, 

\begin{equation}
\prob\left\{A_{1}\left(t\right),A_{2}\left(t\right)|t\in I_{1}\left(n\right)\right\}
=e^{-\left[\tilde{\mu}_{1}\xi_{1}\left(n\right)+\tilde{\mu}_{2}\xi_{2}
\left(n\right)\right]}>0.
\end{equation}

En lo que respecta a la relaci\'on entre los dos SVC que conforman la RSVC, se afirma que existe $m>0$ tal que $\overline{\tau}_{3}\left(m\right)<\tau_{2}\left(n\right)<\tau_{4}\left(m\right)$.

Para $Q_{3}$ sea $I_{3}=\left[\overline{\tau}_{3}\left(m\right),\tau_{4}\left(m\right)\right]$ con longitud  $\xi_{3}\left(m\right)=r_{3}\left(m\right)$, entonces 

\begin{equation}
\prob\left\{A_{3}\left(t\right)|t\in I_{3}\left(n\right)\right\}=e^{-\tilde{\mu}_{3}\xi_{3}\left(n\right)}.
\end{equation} 

An\'alogamente que como se hizo para $Q_{2}$, tenemos que para $Q_{4}$ se tiene el intervalo $I_{4}=\left[\overline{\tau}_{4}\left(m-1\right),\tau_{4}\left(m\right)\right]$ con longitud $\xi_{4}\left(m\right)=\tau_{4}\left(m\right)-\overline{\tau}_{4}\left(m-1\right)$, entonces


\begin{equation}
\prob\left\{A_{4}\left(t\right)|t\in I_{4}\left(m\right)\right\}=e^{-\tilde{\mu}_{4}\xi_{4}\left(n\right)}.
\end{equation} 

Al igual que para el primer sistema, dado que $I_{3}\left(m\right)\subset I_{4}\left(m\right)$, se tiene que

\begin{eqnarray*}
\xi_{3}\left(m\right)\leq\xi_{4}\left(m\right)&\Leftrightarrow& -\xi_{3}\left(m\right)\geq-\xi_{4}\left(m\right)
\\
-\tilde{\mu}_{4}\xi_{3}\left(m\right)\geq-\tilde{\mu}_{4}\xi_{4}\left(m\right)&\Leftrightarrow&
e^{-\tilde{\mu}_{4}\xi_{3}\left(m\right)}\geq e^{-\tilde{\mu}_{4}\xi_{4}\left(m\right)}\\
\prob\left\{A_{4}\left(t\right)|t\in I_{3}\left(m\right)\right\}&\geq&
\prob\left\{A_{4}\left(t\right)|t\in I_{4}\left(m\right)\right\}
\end{eqnarray*}

Entonces, dado que los eventos $A_{3}$ y $A_{4}$ son independientes, se tiene que

\begin{eqnarray*}
\prob\left\{A_{3}\left(t\right),A_{4}\left(t\right)|t\in I_{3}\left(m\right)\right\}&=&
\prob\left\{A_{3}\left(t\right)|t\in I_{3}\left(m\right)\right\}
\prob\left\{A_{4}\left(t\right)|t\in I_{3}\left(m\right)\right\}\\
&\geq&
\prob\left\{A_{3}\left(t\right)|t\in I_{3}\left(n\right)\right\}
\prob\left\{A_{4}\left(t\right)|t\in I_{4}\left(n\right)\right\}\\
&=&e^{-\tilde{\mu}_{3}\xi_{3}\left(m\right)}e^{-\tilde{\mu}_{4}\xi_{4}
\left(m\right)}
=e^{-\left[\tilde{\mu}_{3}\xi_{3}\left(m\right)+\tilde{\mu}_{4}\xi_{4}
\left(m\right)\right]}.
\end{eqnarray*}


es decir, 

\begin{equation}
\prob\left\{A_{3}\left(t\right),A_{4}\left(t\right)|t\in I_{3}\left(m\right)\right\}
=e^{-\left[\tilde{\mu}_{3}\xi_{3}\left(m\right)+\tilde{\mu}_{4}\xi_{4}
\left(m\right)\right]}>0.
\end{equation}

Por construcci\'on se tiene que $I\left(n,m\right)\equiv I_{1}\left(n\right)\cap I_{3}\left(m\right)\neq\emptyset$,entonces en particular se tienen las contenciones $I\left(n,m\right)\subseteq I_{1}\left(n\right)$ y $I\left(n,m\right)\subseteq I_{3}\left(m\right)$, por lo tanto si definimos $\xi_{n,m}\equiv\ell\left(I\left(n,m\right)\right)$ tenemos que

\begin{eqnarray*}
\xi_{n,m}\leq\xi_{1}\left(n\right)\textrm{ y }\xi_{n,m}\leq\xi_{3}\left(m\right)\textrm{ entonces }
-\xi_{n,m}\geq-\xi_{1}\left(n\right)\textrm{ y }-\xi_{n,m}\leq-\xi_{3}\left(m\right)\\
\end{eqnarray*}
por lo tanto tenemos las desigualdades 



\begin{eqnarray*}
\begin{array}{ll}
-\tilde{\mu}_{1}\xi_{n,m}\geq-\tilde{\mu}_{1}\xi_{1}\left(n\right),&
-\tilde{\mu}_{2}\xi_{n,m}\geq-\tilde{\mu}_{2}\xi_{1}\left(n\right)
\geq-\tilde{\mu}_{2}\xi_{2}\left(n\right),\\
-\tilde{\mu}_{3}\xi_{n,m}\geq-\tilde{\mu}_{3}\xi_{3}\left(m\right),&
-\tilde{\mu}_{4}\xi_{n,m}\geq-\tilde{\mu}_{4}\xi_{3}\left(m\right)
\geq-\tilde{\mu}_{4}\xi_{4}\left(m\right).
\end{array}
\end{eqnarray*}

Sea $T^{*}\in I_{n,m}$, entonces dado que en particular $T^{*}\in I_{1}\left(n\right)$ se cumple con probabilidad positiva que no hay arribos a las colas $Q_{1}$ y $Q_{2}$, en consecuencia, tampoco hay usuarios de transferencia para $Q_{3}$ y $Q_{4}$, es decir, $\tilde{\mu}_{1}=\mu_{1}$, $\tilde{\mu}_{2}=\mu_{2}$, $\tilde{\mu}_{3}=\mu_{3}$, $\tilde{\mu}_{4}=\mu_{4}$, es decir, los eventos $Q_{1}$ y $Q_{3}$ son condicionalmente independientes en el intervalo $I_{n,m}$; lo mismo ocurre para las colas $Q_{2}$ y $Q_{4}$, por lo tanto tenemos que


\begin{eqnarray}
\begin{array}{l}
\prob\left\{A_{1}\left(T^{*}\right),A_{2}\left(T^{*}\right),
A_{3}\left(T^{*}\right),A_{4}\left(T^{*}\right)|T^{*}\in I_{n,m}\right\}
=\prod_{j=1}^{4}\prob\left\{A_{j}\left(T^{*}\right)|T^{*}\in I_{n,m}\right\}\\
\geq\prob\left\{A_{1}\left(T^{*}\right)|T^{*}\in I_{1}\left(n\right)\right\}
\prob\left\{A_{2}\left(T^{*}\right)|T^{*}\in I_{2}\left(n\right)\right\}
\prob\left\{A_{3}\left(T^{*}\right)|T^{*}\in I_{3}\left(m\right)\right\}
\prob\left\{A_{4}\left(T^{*}\right)|T^{*}\in I_{4}\left(m\right)\right\}\\
=e^{-\mu_{1}\xi_{1}\left(n\right)}
e^{-\mu_{2}\xi_{2}\left(n\right)}
e^{-\mu_{3}\xi_{3}\left(m\right)}
e^{-\mu_{4}\xi_{4}\left(m\right)}
=e^{-\left[\tilde{\mu}_{1}\xi_{1}\left(n\right)
+\tilde{\mu}_{2}\xi_{2}\left(n\right)
+\tilde{\mu}_{3}\xi_{3}\left(m\right)
+\tilde{\mu}_{4}\xi_{4}
\left(m\right)\right]}>0.
\end{array}
\end{eqnarray}
\end{proof}


Estos resultados aparecen en Daley (1968) \cite{Daley68} para $\left\{T_{n}\right\}$ intervalos de inter-arribo, $\left\{D_{n}\right\}$ intervalos de inter-salida y $\left\{S_{n}\right\}$ tiempos de servicio.

\begin{itemize}
\item Si el proceso $\left\{T_{n}\right\}$ es Poisson, el proceso $\left\{D_{n}\right\}$ es no correlacionado si y s\'olo si es un proceso Poisso, lo cual ocurre si y s\'olo si $\left\{S_{n}\right\}$ son exponenciales negativas.

\item Si $\left\{S_{n}\right\}$ son exponenciales negativas, $\left\{D_{n}\right\}$ es un proceso de renovaci\'on  si y s\'olo si es un proceso Poisson, lo cual ocurre si y s\'olo si $\left\{T_{n}\right\}$ es un proceso Poisson.

\item $\esp\left(D_{n}\right)=\esp\left(T_{n}\right)$.

\item Para un sistema de visitas $GI/M/1$ se tiene el siguiente teorema:

\begin{Teo}
En un sistema estacionario $GI/M/1$ los intervalos de interpartida tienen
\begin{eqnarray*}
\esp\left(e^{-\theta D_{n}}\right)&=&\mu\left(\mu+\theta\right)^{-1}\left[\delta\theta
-\mu\left(1-\delta\right)\alpha\left(\theta\right)\right]
\left[\theta-\mu\left(1-\delta\right)^{-1}\right]\\
\alpha\left(\theta\right)&=&\esp\left[e^{-\theta T_{0}}\right]\\
var\left(D_{n}\right)&=&var\left(T_{0}\right)-\left(\tau^{-1}-\delta^{-1}\right)
2\delta\left(\esp\left(S_{0}\right)\right)^{2}\left(1-\delta\right)^{-1}.
\end{eqnarray*}
\end{Teo}



\begin{Teo}
El proceso de salida de un sistema de colas estacionario $GI/M/1$ es un proceso de renovaci\'on si y s\'olo si el proceso de entrada es un proceso Poisson, en cuyo caso el proceso de salida es un proceso Poisson.
\end{Teo}


\begin{Teo}
Los intervalos de interpartida $\left\{D_{n}\right\}$ de un sistema $M/G/1$ estacionario son no correlacionados si y s\'olo si la distribuci\'on de los tiempos de servicio es exponencial negativa, es decir, el sistema es de tipo  $M/M/1$.

\end{Teo}



\end{itemize}


%\section{Resultados para Procesos de Salida}

En Sigman, Thorison y Wolff \cite{Sigman2} prueban que para la existencia de un una sucesi\'on infinita no decreciente de tiempos de regeneraci\'on $\tau_{1}\leq\tau_{2}\leq\cdots$ en los cuales el proceso se regenera, basta un tiempo de regeneraci\'on $R_{1}$, donde $R_{j}=\tau_{j}-\tau_{j-1}$. Para tal efecto se requiere la existencia de un espacio de probabilidad $\left(\Omega,\mathcal{F},\prob\right)$, y proceso estoc\'astico $\textit{X}=\left\{X\left(t\right):t\geq0\right\}$ con espacio de estados $\left(S,\mathcal{R}\right)$, con $\mathcal{R}$ $\sigma$-\'algebra.

\begin{Prop}
Si existe una variable aleatoria no negativa $R_{1}$ tal que $\theta_{R\footnotesize{1}}X=_{D}X$, entonces $\left(\Omega,\mathcal{F},\prob\right)$ puede extenderse para soportar una sucesi\'on estacionaria de variables aleatorias $R=\left\{R_{k}:k\geq1\right\}$, tal que para $k\geq1$,
\begin{eqnarray*}
\theta_{k}\left(X,R\right)=_{D}\left(X,R\right).
\end{eqnarray*}

Adem\'as, para $k\geq1$, $\theta_{k}R$ es condicionalmente independiente de $\left(X,R_{1},\ldots,R_{k}\right)$, dado $\theta_{\tau k}X$.

\end{Prop}


\begin{itemize}
\item Doob en 1953 demostr\'o que el estado estacionario de un proceso de partida en un sistema de espera $M/G/\infty$, es Poisson con la misma tasa que el proceso de arribos.

\item Burke en 1968, fue el primero en demostrar que el estado estacionario de un proceso de salida de una cola $M/M/s$ es un proceso Poisson.

\item Disney en 1973 obtuvo el siguiente resultado:

\begin{Teo}
Para el sistema de espera $M/G/1/L$ con disciplina FIFO, el proceso $\textbf{I}$ es un proceso de renovaci\'on si y s\'olo si el proceso denominado longitud de la cola es estacionario y se cumple cualquiera de los siguientes casos:

\begin{itemize}
\item[a)] Los tiempos de servicio son identicamente cero;
\item[b)] $L=0$, para cualquier proceso de servicio $S$;
\item[c)] $L=1$ y $G=D$;
\item[d)] $L=\infty$ y $G=M$.
\end{itemize}
En estos casos, respectivamente, las distribuciones de interpartida $P\left\{T_{n+1}-T_{n}\leq t\right\}$ son


\begin{itemize}
\item[a)] $1-e^{-\lambda t}$, $t\geq0$;
\item[b)] $1-e^{-\lambda t}*F\left(t\right)$, $t\geq0$;
\item[c)] $1-e^{-\lambda t}*\indora_{d}\left(t\right)$, $t\geq0$;
\item[d)] $1-e^{-\lambda t}*F\left(t\right)$, $t\geq0$.
\end{itemize}
\end{Teo}


\item Finch (1959) mostr\'o que para los sistemas $M/G/1/L$, con $1\leq L\leq \infty$ con distribuciones de servicio dos veces diferenciable, solamente el sistema $M/M/1/\infty$ tiene proceso de salida de renovaci\'on estacionario.

\item King (1971) demostro que un sistema de colas estacionario $M/G/1/1$ tiene sus tiempos de interpartida sucesivas $D_{n}$ y $D_{n+1}$ son independientes, si y s\'olo si, $G=D$, en cuyo caso le proceso de salida es de renovaci\'on.

\item Disney (1973) demostr\'o que el \'unico sistema estacionario $M/G/1/L$, que tiene proceso de salida de renovaci\'on  son los sistemas $M/M/1$ y $M/D/1/1$.



\item El siguiente resultado es de Disney y Koning (1985)
\begin{Teo}
En un sistema de espera $M/G/s$, el estado estacionario del proceso de salida es un proceso Poisson para cualquier distribuci\'on de los tiempos de servicio si el sistema tiene cualquiera de las siguientes cuatro propiedades.

\begin{itemize}
\item[a)] $s=\infty$
\item[b)] La disciplina de servicio es de procesador compartido.
\item[c)] La disciplina de servicio es LCFS y preemptive resume, esto se cumple para $L<\infty$
\item[d)] $G=M$.
\end{itemize}

\end{Teo}

\item El siguiente resultado es de Alamatsaz (1983)

\begin{Teo}
En cualquier sistema de colas $GI/G/1/L$ con $1\leq L<\infty$ y distribuci\'on de interarribos $A$ y distribuci\'on de los tiempos de servicio $B$, tal que $A\left(0\right)=0$, $A\left(t\right)\left(1-B\left(t\right)\right)>0$ para alguna $t>0$ y $B\left(t\right)$ para toda $t>0$, es imposible que el proceso de salida estacionario sea de renovaci\'on.
\end{Teo}

\end{itemize}

Estos resultados aparecen en Daley (1968) \cite{Daley68} para $\left\{T_{n}\right\}$ intervalos de inter-arribo, $\left\{D_{n}\right\}$ intervalos de inter-salida y $\left\{S_{n}\right\}$ tiempos de servicio.

\begin{itemize}
\item Si el proceso $\left\{T_{n}\right\}$ es Poisson, el proceso $\left\{D_{n}\right\}$ es no correlacionado si y s\'olo si es un proceso Poisso, lo cual ocurre si y s\'olo si $\left\{S_{n}\right\}$ son exponenciales negativas.

\item Si $\left\{S_{n}\right\}$ son exponenciales negativas, $\left\{D_{n}\right\}$ es un proceso de renovaci\'on  si y s\'olo si es un proceso Poisson, lo cual ocurre si y s\'olo si $\left\{T_{n}\right\}$ es un proceso Poisson.

\item $\esp\left(D_{n}\right)=\esp\left(T_{n}\right)$.

\item Para un sistema de visitas $GI/M/1$ se tiene el siguiente teorema:

\begin{Teo}
En un sistema estacionario $GI/M/1$ los intervalos de interpartida tienen
\begin{eqnarray*}
\esp\left(e^{-\theta D_{n}}\right)&=&\mu\left(\mu+\theta\right)^{-1}\left[\delta\theta
-\mu\left(1-\delta\right)\alpha\left(\theta\right)\right]
\left[\theta-\mu\left(1-\delta\right)^{-1}\right]\\
\alpha\left(\theta\right)&=&\esp\left[e^{-\theta T_{0}}\right]\\
var\left(D_{n}\right)&=&var\left(T_{0}\right)-\left(\tau^{-1}-\delta^{-1}\right)
2\delta\left(\esp\left(S_{0}\right)\right)^{2}\left(1-\delta\right)^{-1}.
\end{eqnarray*}
\end{Teo}



\begin{Teo}
El proceso de salida de un sistema de colas estacionario $GI/M/1$ es un proceso de renovaci\'on si y s\'olo si el proceso de entrada es un proceso Poisson, en cuyo caso el proceso de salida es un proceso Poisson.
\end{Teo}


\begin{Teo}
Los intervalos de interpartida $\left\{D_{n}\right\}$ de un sistema $M/G/1$ estacionario son no correlacionados si y s\'olo si la distribuci\'on de los tiempos de servicio es exponencial negativa, es decir, el sistema es de tipo  $M/M/1$.

\end{Teo}



\end{itemize}
%\newpage
%________________________________________________________________________
%\section{Redes de Sistemas de Visitas C\'iclicas}
%________________________________________________________________________

Sean $Q_{1},Q_{2},Q_{3}$ y $Q_{4}$ en una Red de Sistemas de Visitas C\'iclicas (RSVC). Supongamos que cada una de las colas es del tipo $M/M/1$ con tasa de arribo $\mu_{i}$ y que la transferencia de usuarios entre los dos sistemas ocurre entre $Q_{1}\leftrightarrow Q_{3}$ y $Q_{2}\leftrightarrow Q_{4}$ con respectiva tasa de arribo igual a la tasa de salida $\hat{\mu}_{i}=\mu_{i}$, esto se sabe por lo desarrollado en la secci\'on anterior.  

Consideremos, sin p\'erdida de generalidad como base del an\'alisis, la cola $Q_{1}$ adem\'as supongamos al servidor lo comenzamos a observar una vez que termina de atender a la misma para desplazarse y llegar a $Q_{2}$, es decir al tiempo $\tau_{2}$.

Sea $n\in\nat$, $n>0$, ciclo del servidor en que regresa a $Q_{1}$ para dar servicio y atender conforme a la pol\'itica exhaustiva, entonces se tiene que $\overline{\tau}_{1}\left(n\right)$ es el tiempo del servidor en el sistema 1 en que termina de dar servicio a todos los usuarios presentes en la cola, por lo tanto se cumple que $L_{1}\left(\overline{\tau}_{1}\left(n\right)\right)=0$, entonces el servidor para llegar a $Q_{2}$ incurre en un tiempo de traslado $r_{1}$ y por tanto se cumple que $\tau_{2}\left(n\right)=\overline{\tau}_{1}\left(n\right)+r_{1}$. Dado que los tiempos entre arribos son exponenciales se cumple que 

\begin{eqnarray*}
\prob\left\{0 \textrm{ arribos en }Q_{1}\textrm{ en el intervalo }\left[\overline{\tau}_{1}\left(n\right),\overline{\tau}_{1}\left(n\right)+r_{1}\right]\right\}=e^{-\tilde{\mu}_{1}r_{1}},\\
\prob\left\{0 \textrm{ arribos en }Q_{2}\textrm{ en el intervalo }\left[\overline{\tau}_{1}\left(n\right),\overline{\tau}_{1}\left(n\right)+r_{1}\right]\right\}=e^{-\tilde{\mu}_{2}r_{1}}.
\end{eqnarray*}

El evento que nos interesa consiste en que no haya arribos desde que el servidor abandon\'o $Q_{2}$ y regresa nuevamente para dar servicio, es decir en el intervalo de tiempo $\left[\overline{\tau}_{2}\left(n-1\right),\tau_{2}\left(n\right)\right]$. Entonces, si hacemos


\begin{eqnarray*}
\varphi_{1}\left(n\right)&\equiv&\overline{\tau}_{1}\left(n\right)+r_{1}=\overline{\tau}_{2}\left(n-1\right)+r_{1}+r_{2}+\overline{\tau}_{1}\left(n\right)-\tau_{1}\left(n\right)\\
&=&\overline{\tau}_{2}\left(n-1\right)+\overline{\tau}_{1}\left(n\right)-\tau_{1}\left(n\right)+r,,
\end{eqnarray*}

y longitud del intervalo

\begin{eqnarray*}
\xi&\equiv&\overline{\tau}_{1}\left(n\right)+r_{1}-\overline{\tau}_{2}\left(n-1\right)
=\overline{\tau}_{2}\left(n-1\right)+\overline{\tau}_{1}\left(n\right)-\tau_{1}\left(n\right)+r-\overline{\tau}_{2}\left(n-1\right)\\
&=&\overline{\tau}_{1}\left(n\right)-\tau_{1}\left(n\right)+r.
\end{eqnarray*}


Entonces, determinemos la probabilidad del evento no arribos a $Q_{2}$ en $\left[\overline{\tau}_{2}\left(n-1\right),\varphi_{1}\left(n\right)\right]$:

\begin{eqnarray}
\prob\left\{0 \textrm{ arribos en }Q_{2}\textrm{ en el intervalo }\left[\overline{\tau}_{2}\left(n-1\right),\varphi_{1}\left(n\right)\right]\right\}
=e^{-\tilde{\mu}_{2}\xi}.
\end{eqnarray}

De manera an\'aloga, tenemos que la probabilidad de no arribos a $Q_{1}$ en $\left[\overline{\tau}_{2}\left(n-1\right),\varphi_{1}\left(n\right)\right]$ esta dada por

\begin{eqnarray}
\prob\left\{0 \textrm{ arribos en }Q_{1}\textrm{ en el intervalo }\left[\overline{\tau}_{2}\left(n-1\right),\varphi_{1}\left(n\right)\right]\right\}
=e^{-\tilde{\mu}_{1}\xi},
\end{eqnarray}

\begin{eqnarray}
\prob\left\{0 \textrm{ arribos en }Q_{2}\textrm{ en el intervalo }\left[\overline{\tau}_{2}\left(n-1\right),\varphi_{1}\left(n\right)\right]\right\}
=e^{-\tilde{\mu}_{2}\xi}.
\end{eqnarray}

Por tanto 

\begin{eqnarray}
\begin{array}{l}
\prob\left\{0 \textrm{ arribos en }Q_{1}\textrm{ y }Q_{2}\textrm{ en el intervalo }\left[\overline{\tau}_{2}\left(n-1\right),\varphi_{1}\left(n\right)\right]\right\}\\
=\prob\left\{0 \textrm{ arribos en }Q_{1}\textrm{ en el intervalo }\left[\overline{\tau}_{2}\left(n-1\right),\varphi_{1}\left(n\right)\right]\right\}\\
\times
\prob\left\{0 \textrm{ arribos en }Q_{2}\textrm{ en el intervalo }\left[\overline{\tau}_{2}\left(n-1\right),\varphi_{1}\left(n\right)\right]\right\}=e^{-\tilde{\mu}_{1}\xi}e^{-\tilde{\mu}_{2}\xi}
=e^{-\tilde{\mu}\xi}.
\end{array}
\end{eqnarray}

Para el segundo sistema, consideremos nuevamente $\overline{\tau}_{1}\left(n\right)+r_{1}$, sin p\'erdida de generalidad podemos suponer que existe $m>0$ tal que $\overline{\tau}_{3}\left(m\right)<\overline{\tau}_{1}+r_{1}<\tau_{4}\left(m\right)$, entonces

\begin{eqnarray}
\prob\left\{0 \textrm{ arribos en }Q_{3}\textrm{ en el intervalo }\left[\overline{\tau}_{3}\left(m\right),\overline{\tau}_{1}\left(n\right)+r_{1}\right]\right\}
=e^{-\tilde{\mu}_{3}\xi_{3}},
\end{eqnarray}
donde 
\begin{eqnarray}
\xi_{3}=\overline{\tau}_{1}\left(n\right)+r_{1}-\overline{\tau}_{3}\left(m\right)=
\overline{\tau}_{1}\left(n\right)-\overline{\tau}_{3}\left(m\right)+r_{1},
\end{eqnarray}

mientras que para $Q_{4}$ al igual que con $Q_{2}$ escribiremos $\tau_{4}\left(m\right)$ en t\'erminos de $\overline{\tau}_{4}\left(m-1\right)$:

$\varphi_{2}\equiv\tau_{4}\left(m\right)=\overline{\tau}_{4}\left(m-1\right)+r_{4}+\overline{\tau}_{3}\left(m\right)
-\tau_{3}\left(m\right)+r_{3}=\overline{\tau}_{4}\left(m-1\right)+\overline{\tau}_{3}\left(m\right)
-\tau_{3}\left(m\right)+\hat{r}$, adem\'as,

$\xi_{2}\equiv\varphi_{2}\left(m\right)-\overline{\tau}_{1}-r_{1}=\overline{\tau}_{4}\left(m-1\right)+\overline{\tau}_{3}\left(m\right)
-\tau_{3}\left(m\right)-\overline{\tau}_{1}\left(n\right)+\hat{r}-r_{1}$. 

Entonces


\begin{eqnarray}
\prob\left\{0 \textrm{ arribos en }Q_{4}\textrm{ en el intervalo }\left[\overline{\tau}_{1}\left(n\right)+r_{1},\varphi_{2}\left(m\right)\right]\right\}
=e^{-\tilde{\mu}_{4}\xi_{2}},
\end{eqnarray}

mientras que para $Q_{3}$ se tiene que 

\begin{eqnarray}
\prob\left\{0 \textrm{ arribos en }Q_{3}\textrm{ en el intervalo }\left[\overline{\tau}_{1}\left(n\right)+r_{1},\varphi_{2}\left(m\right)\right]\right\}
=e^{-\tilde{\mu}_{3}\xi_{2}}
\end{eqnarray}

Por tanto

\begin{eqnarray}
\prob\left\{0 \textrm{ arribos en }Q_{3}\wedge Q_{4}\textrm{ en el intervalo }\left[\overline{\tau}_{1}\left(n\right)+r_{1},\varphi_{2}\left(m\right)\right]\right\}
=e^{-\hat{\mu}\xi_{2}}
\end{eqnarray}
donde $\hat{\mu}=\tilde{\mu}_{3}+\tilde{\mu}_{4}$.

Ahora, definamos los intervalos $\mathcal{I}_{1}=\left[\overline{\tau}_{1}\left(n\right)+r_{1},\varphi_{1}\left(n\right)\right]$  y $\mathcal{I}_{2}=\left[\overline{\tau}_{1}\left(n\right)+r_{1},\varphi_{2}\left(m\right)\right]$, entonces, sea $\mathcal{I}=\mathcal{I}_{1}\cap\mathcal{I}_{2}$ el intervalo donde cada una de las colas se encuentran vac\'ias, es decir, si tomamos $T^{*}\in\mathcal{I}$, entonces  $L_{1}\left(T^{*}\right)=L_{2}\left(T^{*}\right)=L_{3}\left(T^{*}\right)=L_{4}\left(T^{*}\right)=0$.

Ahora, dado que por construcci\'on $\mathcal{I}\neq\emptyset$ y que para $T^{*}\in\mathcal{I}$ en ninguna de las colas han llegado usuarios, se tiene que no hay transferencia entre las colas, por lo tanto, el sistema 1 y el sistema 2 son condicionalmente independientes en $\mathcal{I}$, es decir

\begin{eqnarray}
\prob\left\{L_{1}\left(T^{*}\right),L_{2}\left(T^{*}\right),L_{3}\left(T^{*}\right),L_{4}\left(T^{*}\right)|T^{*}\in\mathcal{I}\right\}=\prod_{j=1}^{4}\prob\left\{L_{j}\left(T^{*}\right)\right\},
\end{eqnarray}

para $T^{*}\in\mathcal{I}$. 

%\newpage























%________________________________________________________________________
%\section{Procesos Regenerativos}
%________________________________________________________________________

%________________________________________________________________________
%\subsection*{Procesos Regenerativos Sigman, Thorisson y Wolff \cite{Sigman1}}
%________________________________________________________________________


\begin{Def}[Definici\'on Cl\'asica]
Un proceso estoc\'astico $X=\left\{X\left(t\right):t\geq0\right\}$ es llamado regenerativo is existe una variable aleatoria $R_{1}>0$ tal que
\begin{itemize}
\item[i)] $\left\{X\left(t+R_{1}\right):t\geq0\right\}$ es independiente de $\left\{\left\{X\left(t\right):t<R_{1}\right\},\right\}$
\item[ii)] $\left\{X\left(t+R_{1}\right):t\geq0\right\}$ es estoc\'asticamente equivalente a $\left\{X\left(t\right):t>0\right\}$
\end{itemize}

Llamamos a $R_{1}$ tiempo de regeneraci\'on, y decimos que $X$ se regenera en este punto.
\end{Def}

$\left\{X\left(t+R_{1}\right)\right\}$ es regenerativo con tiempo de regeneraci\'on $R_{2}$, independiente de $R_{1}$ pero con la misma distribuci\'on que $R_{1}$. Procediendo de esta manera se obtiene una secuencia de variables aleatorias independientes e id\'enticamente distribuidas $\left\{R_{n}\right\}$ llamados longitudes de ciclo. Si definimos a $Z_{k}\equiv R_{1}+R_{2}+\cdots+R_{k}$, se tiene un proceso de renovaci\'on llamado proceso de renovaci\'on encajado para $X$.


\begin{Note}
La existencia de un primer tiempo de regeneraci\'on, $R_{1}$, implica la existencia de una sucesi\'on completa de estos tiempos $R_{1},R_{2}\ldots,$ que satisfacen la propiedad deseada \cite{Sigman2}.
\end{Note}


\begin{Note} Para la cola $GI/GI/1$ los usuarios arriban con tiempos $t_{n}$ y son atendidos con tiempos de servicio $S_{n}$, los tiempos de arribo forman un proceso de renovaci\'on  con tiempos entre arribos independientes e identicamente distribuidos (\texttt{i.i.d.})$T_{n}=t_{n}-t_{n-1}$, adem\'as los tiempos de servicio son \texttt{i.i.d.} e independientes de los procesos de arribo. Por \textit{estable} se entiende que $\esp S_{n}<\esp T_{n}<\infty$.
\end{Note}
 


\begin{Def}
Para $x$ fijo y para cada $t\geq0$, sea $I_{x}\left(t\right)=1$ si $X\left(t\right)\leq x$,  $I_{x}\left(t\right)=0$ en caso contrario, y def\'inanse los tiempos promedio
\begin{eqnarray*}
\overline{X}&=&lim_{t\rightarrow\infty}\frac{1}{t}\int_{0}^{\infty}X\left(u\right)du\\
\prob\left(X_{\infty}\leq x\right)&=&lim_{t\rightarrow\infty}\frac{1}{t}\int_{0}^{\infty}I_{x}\left(u\right)du,
\end{eqnarray*}
cuando estos l\'imites existan.
\end{Def}

Como consecuencia del teorema de Renovaci\'on-Recompensa, se tiene que el primer l\'imite  existe y es igual a la constante
\begin{eqnarray*}
\overline{X}&=&\frac{\esp\left[\int_{0}^{R_{1}}X\left(t\right)dt\right]}{\esp\left[R_{1}\right]},
\end{eqnarray*}
suponiendo que ambas esperanzas son finitas.
 
\begin{Note}
Funciones de procesos regenerativos son regenerativas, es decir, si $X\left(t\right)$ es regenerativo y se define el proceso $Y\left(t\right)$ por $Y\left(t\right)=f\left(X\left(t\right)\right)$ para alguna funci\'on Borel medible $f\left(\cdot\right)$. Adem\'as $Y$ es regenerativo con los mismos tiempos de renovaci\'on que $X$. 

En general, los tiempos de renovaci\'on, $Z_{k}$ de un proceso regenerativo no requieren ser tiempos de paro con respecto a la evoluci\'on de $X\left(t\right)$.
\end{Note} 

\begin{Note}
Una funci\'on de un proceso de Markov, usualmente no ser\'a un proceso de Markov, sin embargo ser\'a regenerativo si el proceso de Markov lo es.
\end{Note}

 
\begin{Note}
Un proceso regenerativo con media de la longitud de ciclo finita es llamado positivo recurrente.
\end{Note}


\begin{Note}
\begin{itemize}
\item[a)] Si el proceso regenerativo $X$ es positivo recurrente y tiene trayectorias muestrales no negativas, entonces la ecuaci\'on anterior es v\'alida.
\item[b)] Si $X$ es positivo recurrente regenerativo, podemos construir una \'unica versi\'on estacionaria de este proceso, $X_{e}=\left\{X_{e}\left(t\right)\right\}$, donde $X_{e}$ es un proceso estoc\'astico regenerativo y estrictamente estacionario, con distribuci\'on marginal distribuida como $X_{\infty}$
\end{itemize}
\end{Note}


%__________________________________________________________________________________________
%\subsection*{Procesos Regenerativos Estacionarios - Stidham \cite{Stidham}}
%__________________________________________________________________________________________


Un proceso estoc\'astico a tiempo continuo $\left\{V\left(t\right),t\geq0\right\}$ es un proceso regenerativo si existe una sucesi\'on de variables aleatorias independientes e id\'enticamente distribuidas $\left\{X_{1},X_{2},\ldots\right\}$, sucesi\'on de renovaci\'on, tal que para cualquier conjunto de Borel $A$, 

\begin{eqnarray*}
\prob\left\{V\left(t\right)\in A|X_{1}+X_{2}+\cdots+X_{R\left(t\right)}=s,\left\{V\left(\tau\right),\tau<s\right\}\right\}=\prob\left\{V\left(t-s\right)\in A|X_{1}>t-s\right\},
\end{eqnarray*}
para todo $0\leq s\leq t$, donde $R\left(t\right)=\max\left\{X_{1}+X_{2}+\cdots+X_{j}\leq t\right\}=$n\'umero de renovaciones ({\emph{puntos de regeneraci\'on}}) que ocurren en $\left[0,t\right]$. El intervalo $\left[0,X_{1}\right)$ es llamado {\emph{primer ciclo de regeneraci\'on}} de $\left\{V\left(t \right),t\geq0\right\}$, $\left[X_{1},X_{1}+X_{2}\right)$ el {\emph{segundo ciclo de regeneraci\'on}}, y as\'i sucesivamente.

Sea $X=X_{1}$ y sea $F$ la funci\'on de distrbuci\'on de $X$


\begin{Def}
Se define el proceso estacionario, $\left\{V^{*}\left(t\right),t\geq0\right\}$, para $\left\{V\left(t\right),t\geq0\right\}$ por

\begin{eqnarray*}
\prob\left\{V\left(t\right)\in A\right\}=\frac{1}{\esp\left[X\right]}\int_{0}^{\infty}\prob\left\{V\left(t+x\right)\in A|X>x\right\}\left(1-F\left(x\right)\right)dx,
\end{eqnarray*} 
para todo $t\geq0$ y todo conjunto de Borel $A$.
\end{Def}

\begin{Def}
Una distribuci\'on se dice que es {\emph{aritm\'etica}} si todos sus puntos de incremento son m\'ultiplos de la forma $0,\lambda, 2\lambda,\ldots$ para alguna $\lambda>0$ entera.
\end{Def}


\begin{Def}
Una modificaci\'on medible de un proceso $\left\{V\left(t\right),t\geq0\right\}$, es una versi\'on de este, $\left\{V\left(t,w\right)\right\}$ conjuntamente medible para $t\geq0$ y para $w\in S$, $S$ espacio de estados para $\left\{V\left(t\right),t\geq0\right\}$.
\end{Def}

\begin{Teo}
Sea $\left\{V\left(t\right),t\geq\right\}$ un proceso regenerativo no negativo con modificaci\'on medible. Sea $\esp\left[X\right]<\infty$. Entonces el proceso estacionario dado por la ecuaci\'on anterior est\'a bien definido y tiene funci\'on de distribuci\'on independiente de $t$, adem\'as
\begin{itemize}
\item[i)] \begin{eqnarray*}
\esp\left[V^{*}\left(0\right)\right]&=&\frac{\esp\left[\int_{0}^{X}V\left(s\right)ds\right]}{\esp\left[X\right]}\end{eqnarray*}
\item[ii)] Si $\esp\left[V^{*}\left(0\right)\right]<\infty$, equivalentemente, si $\esp\left[\int_{0}^{X}V\left(s\right)ds\right]<\infty$,entonces
\begin{eqnarray*}
\frac{\int_{0}^{t}V\left(s\right)ds}{t}\rightarrow\frac{\esp\left[\int_{0}^{X}V\left(s\right)ds\right]}{\esp\left[X\right]}
\end{eqnarray*}
con probabilidad 1 y en media, cuando $t\rightarrow\infty$.
\end{itemize}
\end{Teo}

\begin{Coro}
Sea $\left\{V\left(t\right),t\geq0\right\}$ un proceso regenerativo no negativo, con modificaci\'on medible. Si $\esp <\infty$, $F$ es no-aritm\'etica, y para todo $x\geq0$, $P\left\{V\left(t\right)\leq x,C>x\right\}$ es de variaci\'on acotada como funci\'on de $t$ en cada intervalo finito $\left[0,\tau\right]$, entonces $V\left(t\right)$ converge en distribuci\'on  cuando $t\rightarrow\infty$ y $$\esp V=\frac{\esp \int_{0}^{X}V\left(s\right)ds}{\esp X}$$
Donde $V$ tiene la distribuci\'on l\'imite de $V\left(t\right)$ cuando $t\rightarrow\infty$.

\end{Coro}

Para el caso discreto se tienen resultados similares.



%______________________________________________________________________
%\section{Procesos de Renovaci\'on}
%______________________________________________________________________

\begin{Def}\label{Def.Tn}
Sean $0\leq T_{1}\leq T_{2}\leq \ldots$ son tiempos aleatorios infinitos en los cuales ocurren ciertos eventos. El n\'umero de tiempos $T_{n}$ en el intervalo $\left[0,t\right)$ es

\begin{eqnarray}
N\left(t\right)=\sum_{n=1}^{\infty}\indora\left(T_{n}\leq t\right),
\end{eqnarray}
para $t\geq0$.
\end{Def}

Si se consideran los puntos $T_{n}$ como elementos de $\rea_{+}$, y $N\left(t\right)$ es el n\'umero de puntos en $\rea$. El proceso denotado por $\left\{N\left(t\right):t\geq0\right\}$, denotado por $N\left(t\right)$, es un proceso puntual en $\rea_{+}$. Los $T_{n}$ son los tiempos de ocurrencia, el proceso puntual $N\left(t\right)$ es simple si su n\'umero de ocurrencias son distintas: $0<T_{1}<T_{2}<\ldots$ casi seguramente.

\begin{Def}
Un proceso puntual $N\left(t\right)$ es un proceso de renovaci\'on si los tiempos de interocurrencia $\xi_{n}=T_{n}-T_{n-1}$, para $n\geq1$, son independientes e identicamente distribuidos con distribuci\'on $F$, donde $F\left(0\right)=0$ y $T_{0}=0$. Los $T_{n}$ son llamados tiempos de renovaci\'on, referente a la independencia o renovaci\'on de la informaci\'on estoc\'astica en estos tiempos. Los $\xi_{n}$ son los tiempos de inter-renovaci\'on, y $N\left(t\right)$ es el n\'umero de renovaciones en el intervalo $\left[0,t\right)$
\end{Def}


\begin{Note}
Para definir un proceso de renovaci\'on para cualquier contexto, solamente hay que especificar una distribuci\'on $F$, con $F\left(0\right)=0$, para los tiempos de inter-renovaci\'on. La funci\'on $F$ en turno degune las otra variables aleatorias. De manera formal, existe un espacio de probabilidad y una sucesi\'on de variables aleatorias $\xi_{1},\xi_{2},\ldots$ definidas en este con distribuci\'on $F$. Entonces las otras cantidades son $T_{n}=\sum_{k=1}^{n}\xi_{k}$ y $N\left(t\right)=\sum_{n=1}^{\infty}\indora\left(T_{n}\leq t\right)$, donde $T_{n}\rightarrow\infty$ casi seguramente por la Ley Fuerte de los Grandes Números.
\end{Note}

%___________________________________________________________________________________________
%
%\subsection*{Teorema Principal de Renovaci\'on}
%___________________________________________________________________________________________
%

\begin{Note} Una funci\'on $h:\rea_{+}\rightarrow\rea$ es Directamente Riemann Integrable en los siguientes casos:
\begin{itemize}
\item[a)] $h\left(t\right)\geq0$ es decreciente y Riemann Integrable.
\item[b)] $h$ es continua excepto posiblemente en un conjunto de Lebesgue de medida 0, y $|h\left(t\right)|\leq b\left(t\right)$, donde $b$ es DRI.
\end{itemize}
\end{Note}

\begin{Teo}[Teorema Principal de Renovaci\'on]
Si $F$ es no aritm\'etica y $h\left(t\right)$ es Directamente Riemann Integrable (DRI), entonces

\begin{eqnarray*}
lim_{t\rightarrow\infty}U\star h=\frac{1}{\mu}\int_{\rea_{+}}h\left(s\right)ds.
\end{eqnarray*}
\end{Teo}

\begin{Prop}
Cualquier funci\'on $H\left(t\right)$ acotada en intervalos finitos y que es 0 para $t<0$ puede expresarse como
\begin{eqnarray*}
H\left(t\right)=U\star h\left(t\right)\textrm{,  donde }h\left(t\right)=H\left(t\right)-F\star H\left(t\right)
\end{eqnarray*}
\end{Prop}

\begin{Def}
Un proceso estoc\'astico $X\left(t\right)$ es crudamente regenerativo en un tiempo aleatorio positivo $T$ si
\begin{eqnarray*}
\esp\left[X\left(T+t\right)|T\right]=\esp\left[X\left(t\right)\right]\textrm{, para }t\geq0,\end{eqnarray*}
y con las esperanzas anteriores finitas.
\end{Def}

\begin{Prop}
Sup\'ongase que $X\left(t\right)$ es un proceso crudamente regenerativo en $T$, que tiene distribuci\'on $F$. Si $\esp\left[X\left(t\right)\right]$ es acotado en intervalos finitos, entonces
\begin{eqnarray*}
\esp\left[X\left(t\right)\right]=U\star h\left(t\right)\textrm{,  donde }h\left(t\right)=\esp\left[X\left(t\right)\indora\left(T>t\right)\right].
\end{eqnarray*}
\end{Prop}

\begin{Teo}[Regeneraci\'on Cruda]
Sup\'ongase que $X\left(t\right)$ es un proceso con valores positivo crudamente regenerativo en $T$, y def\'inase $M=\sup\left\{|X\left(t\right)|:t\leq T\right\}$. Si $T$ es no aritm\'etico y $M$ y $MT$ tienen media finita, entonces
\begin{eqnarray*}
lim_{t\rightarrow\infty}\esp\left[X\left(t\right)\right]=\frac{1}{\mu}\int_{\rea_{+}}h\left(s\right)ds,
\end{eqnarray*}
donde $h\left(t\right)=\esp\left[X\left(t\right)\indora\left(T>t\right)\right]$.
\end{Teo}

%___________________________________________________________________________________________
%
%\subsection*{Propiedades de los Procesos de Renovaci\'on}
%___________________________________________________________________________________________
%

Los tiempos $T_{n}$ est\'an relacionados con los conteos de $N\left(t\right)$ por

\begin{eqnarray*}
\left\{N\left(t\right)\geq n\right\}&=&\left\{T_{n}\leq t\right\}\\
T_{N\left(t\right)}\leq &t&<T_{N\left(t\right)+1},
\end{eqnarray*}

adem\'as $N\left(T_{n}\right)=n$, y 

\begin{eqnarray*}
N\left(t\right)=\max\left\{n:T_{n}\leq t\right\}=\min\left\{n:T_{n+1}>t\right\}
\end{eqnarray*}

Por propiedades de la convoluci\'on se sabe que

\begin{eqnarray*}
P\left\{T_{n}\leq t\right\}=F^{n\star}\left(t\right)
\end{eqnarray*}
que es la $n$-\'esima convoluci\'on de $F$. Entonces 

\begin{eqnarray*}
\left\{N\left(t\right)\geq n\right\}&=&\left\{T_{n}\leq t\right\}\\
P\left\{N\left(t\right)\leq n\right\}&=&1-F^{\left(n+1\right)\star}\left(t\right)
\end{eqnarray*}

Adem\'as usando el hecho de que $\esp\left[N\left(t\right)\right]=\sum_{n=1}^{\infty}P\left\{N\left(t\right)\geq n\right\}$
se tiene que

\begin{eqnarray*}
\esp\left[N\left(t\right)\right]=\sum_{n=1}^{\infty}F^{n\star}\left(t\right)
\end{eqnarray*}

\begin{Prop}
Para cada $t\geq0$, la funci\'on generadora de momentos $\esp\left[e^{\alpha N\left(t\right)}\right]$ existe para alguna $\alpha$ en una vecindad del 0, y de aqu\'i que $\esp\left[N\left(t\right)^{m}\right]<\infty$, para $m\geq1$.
\end{Prop}


\begin{Note}
Si el primer tiempo de renovaci\'on $\xi_{1}$ no tiene la misma distribuci\'on que el resto de las $\xi_{n}$, para $n\geq2$, a $N\left(t\right)$ se le llama Proceso de Renovaci\'on retardado, donde si $\xi$ tiene distribuci\'on $G$, entonces el tiempo $T_{n}$ de la $n$-\'esima renovaci\'on tiene distribuci\'on $G\star F^{\left(n-1\right)\star}\left(t\right)$
\end{Note}


\begin{Teo}
Para una constante $\mu\leq\infty$ ( o variable aleatoria), las siguientes expresiones son equivalentes:

\begin{eqnarray}
lim_{n\rightarrow\infty}n^{-1}T_{n}&=&\mu,\textrm{ c.s.}\\
lim_{t\rightarrow\infty}t^{-1}N\left(t\right)&=&1/\mu,\textrm{ c.s.}
\end{eqnarray}
\end{Teo}


Es decir, $T_{n}$ satisface la Ley Fuerte de los Grandes N\'umeros s\'i y s\'olo s\'i $N\left/t\right)$ la cumple.


\begin{Coro}[Ley Fuerte de los Grandes N\'umeros para Procesos de Renovaci\'on]
Si $N\left(t\right)$ es un proceso de renovaci\'on cuyos tiempos de inter-renovaci\'on tienen media $\mu\leq\infty$, entonces
\begin{eqnarray}
t^{-1}N\left(t\right)\rightarrow 1/\mu,\textrm{ c.s. cuando }t\rightarrow\infty.
\end{eqnarray}

\end{Coro}


Considerar el proceso estoc\'astico de valores reales $\left\{Z\left(t\right):t\geq0\right\}$ en el mismo espacio de probabilidad que $N\left(t\right)$

\begin{Def}
Para el proceso $\left\{Z\left(t\right):t\geq0\right\}$ se define la fluctuaci\'on m\'axima de $Z\left(t\right)$ en el intervalo $\left(T_{n-1},T_{n}\right]$:
\begin{eqnarray*}
M_{n}=\sup_{T_{n-1}<t\leq T_{n}}|Z\left(t\right)-Z\left(T_{n-1}\right)|
\end{eqnarray*}
\end{Def}

\begin{Teo}
Sup\'ongase que $n^{-1}T_{n}\rightarrow\mu$ c.s. cuando $n\rightarrow\infty$, donde $\mu\leq\infty$ es una constante o variable aleatoria. Sea $a$ una constante o variable aleatoria que puede ser infinita cuando $\mu$ es finita, y considere las expresiones l\'imite:
\begin{eqnarray}
lim_{n\rightarrow\infty}n^{-1}Z\left(T_{n}\right)&=&a,\textrm{ c.s.}\\
lim_{t\rightarrow\infty}t^{-1}Z\left(t\right)&=&a/\mu,\textrm{ c.s.}
\end{eqnarray}
La segunda expresi\'on implica la primera. Conversamente, la primera implica la segunda si el proceso $Z\left(t\right)$ es creciente, o si $lim_{n\rightarrow\infty}n^{-1}M_{n}=0$ c.s.
\end{Teo}

\begin{Coro}
Si $N\left(t\right)$ es un proceso de renovaci\'on, y $\left(Z\left(T_{n}\right)-Z\left(T_{n-1}\right),M_{n}\right)$, para $n\geq1$, son variables aleatorias independientes e id\'enticamente distribuidas con media finita, entonces,
\begin{eqnarray}
lim_{t\rightarrow\infty}t^{-1}Z\left(t\right)\rightarrow\frac{\esp\left[Z\left(T_{1}\right)-Z\left(T_{0}\right)\right]}{\esp\left[T_{1}\right]},\textrm{ c.s. cuando  }t\rightarrow\infty.
\end{eqnarray}
\end{Coro}



%___________________________________________________________________________________________
%
%\subsection{Propiedades de los Procesos de Renovaci\'on}
%___________________________________________________________________________________________
%

Los tiempos $T_{n}$ est\'an relacionados con los conteos de $N\left(t\right)$ por

\begin{eqnarray*}
\left\{N\left(t\right)\geq n\right\}&=&\left\{T_{n}\leq t\right\}\\
T_{N\left(t\right)}\leq &t&<T_{N\left(t\right)+1},
\end{eqnarray*}

adem\'as $N\left(T_{n}\right)=n$, y 

\begin{eqnarray*}
N\left(t\right)=\max\left\{n:T_{n}\leq t\right\}=\min\left\{n:T_{n+1}>t\right\}
\end{eqnarray*}

Por propiedades de la convoluci\'on se sabe que

\begin{eqnarray*}
P\left\{T_{n}\leq t\right\}=F^{n\star}\left(t\right)
\end{eqnarray*}
que es la $n$-\'esima convoluci\'on de $F$. Entonces 

\begin{eqnarray*}
\left\{N\left(t\right)\geq n\right\}&=&\left\{T_{n}\leq t\right\}\\
P\left\{N\left(t\right)\leq n\right\}&=&1-F^{\left(n+1\right)\star}\left(t\right)
\end{eqnarray*}

Adem\'as usando el hecho de que $\esp\left[N\left(t\right)\right]=\sum_{n=1}^{\infty}P\left\{N\left(t\right)\geq n\right\}$
se tiene que

\begin{eqnarray*}
\esp\left[N\left(t\right)\right]=\sum_{n=1}^{\infty}F^{n\star}\left(t\right)
\end{eqnarray*}

\begin{Prop}
Para cada $t\geq0$, la funci\'on generadora de momentos $\esp\left[e^{\alpha N\left(t\right)}\right]$ existe para alguna $\alpha$ en una vecindad del 0, y de aqu\'i que $\esp\left[N\left(t\right)^{m}\right]<\infty$, para $m\geq1$.
\end{Prop}


\begin{Note}
Si el primer tiempo de renovaci\'on $\xi_{1}$ no tiene la misma distribuci\'on que el resto de las $\xi_{n}$, para $n\geq2$, a $N\left(t\right)$ se le llama Proceso de Renovaci\'on retardado, donde si $\xi$ tiene distribuci\'on $G$, entonces el tiempo $T_{n}$ de la $n$-\'esima renovaci\'on tiene distribuci\'on $G\star F^{\left(n-1\right)\star}\left(t\right)$
\end{Note}


\begin{Teo}
Para una constante $\mu\leq\infty$ ( o variable aleatoria), las siguientes expresiones son equivalentes:

\begin{eqnarray}
lim_{n\rightarrow\infty}n^{-1}T_{n}&=&\mu,\textrm{ c.s.}\\
lim_{t\rightarrow\infty}t^{-1}N\left(t\right)&=&1/\mu,\textrm{ c.s.}
\end{eqnarray}
\end{Teo}


Es decir, $T_{n}$ satisface la Ley Fuerte de los Grandes N\'umeros s\'i y s\'olo s\'i $N\left/t\right)$ la cumple.


\begin{Coro}[Ley Fuerte de los Grandes N\'umeros para Procesos de Renovaci\'on]
Si $N\left(t\right)$ es un proceso de renovaci\'on cuyos tiempos de inter-renovaci\'on tienen media $\mu\leq\infty$, entonces
\begin{eqnarray}
t^{-1}N\left(t\right)\rightarrow 1/\mu,\textrm{ c.s. cuando }t\rightarrow\infty.
\end{eqnarray}

\end{Coro}


Considerar el proceso estoc\'astico de valores reales $\left\{Z\left(t\right):t\geq0\right\}$ en el mismo espacio de probabilidad que $N\left(t\right)$

\begin{Def}
Para el proceso $\left\{Z\left(t\right):t\geq0\right\}$ se define la fluctuaci\'on m\'axima de $Z\left(t\right)$ en el intervalo $\left(T_{n-1},T_{n}\right]$:
\begin{eqnarray*}
M_{n}=\sup_{T_{n-1}<t\leq T_{n}}|Z\left(t\right)-Z\left(T_{n-1}\right)|
\end{eqnarray*}
\end{Def}

\begin{Teo}
Sup\'ongase que $n^{-1}T_{n}\rightarrow\mu$ c.s. cuando $n\rightarrow\infty$, donde $\mu\leq\infty$ es una constante o variable aleatoria. Sea $a$ una constante o variable aleatoria que puede ser infinita cuando $\mu$ es finita, y considere las expresiones l\'imite:
\begin{eqnarray}
lim_{n\rightarrow\infty}n^{-1}Z\left(T_{n}\right)&=&a,\textrm{ c.s.}\\
lim_{t\rightarrow\infty}t^{-1}Z\left(t\right)&=&a/\mu,\textrm{ c.s.}
\end{eqnarray}
La segunda expresi\'on implica la primera. Conversamente, la primera implica la segunda si el proceso $Z\left(t\right)$ es creciente, o si $lim_{n\rightarrow\infty}n^{-1}M_{n}=0$ c.s.
\end{Teo}

\begin{Coro}
Si $N\left(t\right)$ es un proceso de renovaci\'on, y $\left(Z\left(T_{n}\right)-Z\left(T_{n-1}\right),M_{n}\right)$, para $n\geq1$, son variables aleatorias independientes e id\'enticamente distribuidas con media finita, entonces,
\begin{eqnarray}
lim_{t\rightarrow\infty}t^{-1}Z\left(t\right)\rightarrow\frac{\esp\left[Z\left(T_{1}\right)-Z\left(T_{0}\right)\right]}{\esp\left[T_{1}\right]},\textrm{ c.s. cuando  }t\rightarrow\infty.
\end{eqnarray}
\end{Coro}


%___________________________________________________________________________________________
%
%\subsection{Propiedades de los Procesos de Renovaci\'on}
%___________________________________________________________________________________________
%

Los tiempos $T_{n}$ est\'an relacionados con los conteos de $N\left(t\right)$ por

\begin{eqnarray*}
\left\{N\left(t\right)\geq n\right\}&=&\left\{T_{n}\leq t\right\}\\
T_{N\left(t\right)}\leq &t&<T_{N\left(t\right)+1},
\end{eqnarray*}

adem\'as $N\left(T_{n}\right)=n$, y 

\begin{eqnarray*}
N\left(t\right)=\max\left\{n:T_{n}\leq t\right\}=\min\left\{n:T_{n+1}>t\right\}
\end{eqnarray*}

Por propiedades de la convoluci\'on se sabe que

\begin{eqnarray*}
P\left\{T_{n}\leq t\right\}=F^{n\star}\left(t\right)
\end{eqnarray*}
que es la $n$-\'esima convoluci\'on de $F$. Entonces 

\begin{eqnarray*}
\left\{N\left(t\right)\geq n\right\}&=&\left\{T_{n}\leq t\right\}\\
P\left\{N\left(t\right)\leq n\right\}&=&1-F^{\left(n+1\right)\star}\left(t\right)
\end{eqnarray*}

Adem\'as usando el hecho de que $\esp\left[N\left(t\right)\right]=\sum_{n=1}^{\infty}P\left\{N\left(t\right)\geq n\right\}$
se tiene que

\begin{eqnarray*}
\esp\left[N\left(t\right)\right]=\sum_{n=1}^{\infty}F^{n\star}\left(t\right)
\end{eqnarray*}

\begin{Prop}
Para cada $t\geq0$, la funci\'on generadora de momentos $\esp\left[e^{\alpha N\left(t\right)}\right]$ existe para alguna $\alpha$ en una vecindad del 0, y de aqu\'i que $\esp\left[N\left(t\right)^{m}\right]<\infty$, para $m\geq1$.
\end{Prop}


\begin{Note}
Si el primer tiempo de renovaci\'on $\xi_{1}$ no tiene la misma distribuci\'on que el resto de las $\xi_{n}$, para $n\geq2$, a $N\left(t\right)$ se le llama Proceso de Renovaci\'on retardado, donde si $\xi$ tiene distribuci\'on $G$, entonces el tiempo $T_{n}$ de la $n$-\'esima renovaci\'on tiene distribuci\'on $G\star F^{\left(n-1\right)\star}\left(t\right)$
\end{Note}


\begin{Teo}
Para una constante $\mu\leq\infty$ ( o variable aleatoria), las siguientes expresiones son equivalentes:

\begin{eqnarray}
lim_{n\rightarrow\infty}n^{-1}T_{n}&=&\mu,\textrm{ c.s.}\\
lim_{t\rightarrow\infty}t^{-1}N\left(t\right)&=&1/\mu,\textrm{ c.s.}
\end{eqnarray}
\end{Teo}


Es decir, $T_{n}$ satisface la Ley Fuerte de los Grandes N\'umeros s\'i y s\'olo s\'i $N\left/t\right)$ la cumple.


\begin{Coro}[Ley Fuerte de los Grandes N\'umeros para Procesos de Renovaci\'on]
Si $N\left(t\right)$ es un proceso de renovaci\'on cuyos tiempos de inter-renovaci\'on tienen media $\mu\leq\infty$, entonces
\begin{eqnarray}
t^{-1}N\left(t\right)\rightarrow 1/\mu,\textrm{ c.s. cuando }t\rightarrow\infty.
\end{eqnarray}

\end{Coro}


Considerar el proceso estoc\'astico de valores reales $\left\{Z\left(t\right):t\geq0\right\}$ en el mismo espacio de probabilidad que $N\left(t\right)$

\begin{Def}
Para el proceso $\left\{Z\left(t\right):t\geq0\right\}$ se define la fluctuaci\'on m\'axima de $Z\left(t\right)$ en el intervalo $\left(T_{n-1},T_{n}\right]$:
\begin{eqnarray*}
M_{n}=\sup_{T_{n-1}<t\leq T_{n}}|Z\left(t\right)-Z\left(T_{n-1}\right)|
\end{eqnarray*}
\end{Def}

\begin{Teo}
Sup\'ongase que $n^{-1}T_{n}\rightarrow\mu$ c.s. cuando $n\rightarrow\infty$, donde $\mu\leq\infty$ es una constante o variable aleatoria. Sea $a$ una constante o variable aleatoria que puede ser infinita cuando $\mu$ es finita, y considere las expresiones l\'imite:
\begin{eqnarray}
lim_{n\rightarrow\infty}n^{-1}Z\left(T_{n}\right)&=&a,\textrm{ c.s.}\\
lim_{t\rightarrow\infty}t^{-1}Z\left(t\right)&=&a/\mu,\textrm{ c.s.}
\end{eqnarray}
La segunda expresi\'on implica la primera. Conversamente, la primera implica la segunda si el proceso $Z\left(t\right)$ es creciente, o si $lim_{n\rightarrow\infty}n^{-1}M_{n}=0$ c.s.
\end{Teo}

\begin{Coro}
Si $N\left(t\right)$ es un proceso de renovaci\'on, y $\left(Z\left(T_{n}\right)-Z\left(T_{n-1}\right),M_{n}\right)$, para $n\geq1$, son variables aleatorias independientes e id\'enticamente distribuidas con media finita, entonces,
\begin{eqnarray}
lim_{t\rightarrow\infty}t^{-1}Z\left(t\right)\rightarrow\frac{\esp\left[Z\left(T_{1}\right)-Z\left(T_{0}\right)\right]}{\esp\left[T_{1}\right]},\textrm{ c.s. cuando  }t\rightarrow\infty.
\end{eqnarray}
\end{Coro}

%___________________________________________________________________________________________
%
%\subsection{Propiedades de los Procesos de Renovaci\'on}
%___________________________________________________________________________________________
%

Los tiempos $T_{n}$ est\'an relacionados con los conteos de $N\left(t\right)$ por

\begin{eqnarray*}
\left\{N\left(t\right)\geq n\right\}&=&\left\{T_{n}\leq t\right\}\\
T_{N\left(t\right)}\leq &t&<T_{N\left(t\right)+1},
\end{eqnarray*}

adem\'as $N\left(T_{n}\right)=n$, y 

\begin{eqnarray*}
N\left(t\right)=\max\left\{n:T_{n}\leq t\right\}=\min\left\{n:T_{n+1}>t\right\}
\end{eqnarray*}

Por propiedades de la convoluci\'on se sabe que

\begin{eqnarray*}
P\left\{T_{n}\leq t\right\}=F^{n\star}\left(t\right)
\end{eqnarray*}
que es la $n$-\'esima convoluci\'on de $F$. Entonces 

\begin{eqnarray*}
\left\{N\left(t\right)\geq n\right\}&=&\left\{T_{n}\leq t\right\}\\
P\left\{N\left(t\right)\leq n\right\}&=&1-F^{\left(n+1\right)\star}\left(t\right)
\end{eqnarray*}

Adem\'as usando el hecho de que $\esp\left[N\left(t\right)\right]=\sum_{n=1}^{\infty}P\left\{N\left(t\right)\geq n\right\}$
se tiene que

\begin{eqnarray*}
\esp\left[N\left(t\right)\right]=\sum_{n=1}^{\infty}F^{n\star}\left(t\right)
\end{eqnarray*}

\begin{Prop}
Para cada $t\geq0$, la funci\'on generadora de momentos $\esp\left[e^{\alpha N\left(t\right)}\right]$ existe para alguna $\alpha$ en una vecindad del 0, y de aqu\'i que $\esp\left[N\left(t\right)^{m}\right]<\infty$, para $m\geq1$.
\end{Prop}


\begin{Note}
Si el primer tiempo de renovaci\'on $\xi_{1}$ no tiene la misma distribuci\'on que el resto de las $\xi_{n}$, para $n\geq2$, a $N\left(t\right)$ se le llama Proceso de Renovaci\'on retardado, donde si $\xi$ tiene distribuci\'on $G$, entonces el tiempo $T_{n}$ de la $n$-\'esima renovaci\'on tiene distribuci\'on $G\star F^{\left(n-1\right)\star}\left(t\right)$
\end{Note}


\begin{Teo}
Para una constante $\mu\leq\infty$ ( o variable aleatoria), las siguientes expresiones son equivalentes:

\begin{eqnarray}
lim_{n\rightarrow\infty}n^{-1}T_{n}&=&\mu,\textrm{ c.s.}\\
lim_{t\rightarrow\infty}t^{-1}N\left(t\right)&=&1/\mu,\textrm{ c.s.}
\end{eqnarray}
\end{Teo}


Es decir, $T_{n}$ satisface la Ley Fuerte de los Grandes N\'umeros s\'i y s\'olo s\'i $N\left/t\right)$ la cumple.


\begin{Coro}[Ley Fuerte de los Grandes N\'umeros para Procesos de Renovaci\'on]
Si $N\left(t\right)$ es un proceso de renovaci\'on cuyos tiempos de inter-renovaci\'on tienen media $\mu\leq\infty$, entonces
\begin{eqnarray}
t^{-1}N\left(t\right)\rightarrow 1/\mu,\textrm{ c.s. cuando }t\rightarrow\infty.
\end{eqnarray}

\end{Coro}


Considerar el proceso estoc\'astico de valores reales $\left\{Z\left(t\right):t\geq0\right\}$ en el mismo espacio de probabilidad que $N\left(t\right)$

\begin{Def}
Para el proceso $\left\{Z\left(t\right):t\geq0\right\}$ se define la fluctuaci\'on m\'axima de $Z\left(t\right)$ en el intervalo $\left(T_{n-1},T_{n}\right]$:
\begin{eqnarray*}
M_{n}=\sup_{T_{n-1}<t\leq T_{n}}|Z\left(t\right)-Z\left(T_{n-1}\right)|
\end{eqnarray*}
\end{Def}

\begin{Teo}
Sup\'ongase que $n^{-1}T_{n}\rightarrow\mu$ c.s. cuando $n\rightarrow\infty$, donde $\mu\leq\infty$ es una constante o variable aleatoria. Sea $a$ una constante o variable aleatoria que puede ser infinita cuando $\mu$ es finita, y considere las expresiones l\'imite:
\begin{eqnarray}
lim_{n\rightarrow\infty}n^{-1}Z\left(T_{n}\right)&=&a,\textrm{ c.s.}\\
lim_{t\rightarrow\infty}t^{-1}Z\left(t\right)&=&a/\mu,\textrm{ c.s.}
\end{eqnarray}
La segunda expresi\'on implica la primera. Conversamente, la primera implica la segunda si el proceso $Z\left(t\right)$ es creciente, o si $lim_{n\rightarrow\infty}n^{-1}M_{n}=0$ c.s.
\end{Teo}

\begin{Coro}
Si $N\left(t\right)$ es un proceso de renovaci\'on, y $\left(Z\left(T_{n}\right)-Z\left(T_{n-1}\right),M_{n}\right)$, para $n\geq1$, son variables aleatorias independientes e id\'enticamente distribuidas con media finita, entonces,
\begin{eqnarray}
lim_{t\rightarrow\infty}t^{-1}Z\left(t\right)\rightarrow\frac{\esp\left[Z\left(T_{1}\right)-Z\left(T_{0}\right)\right]}{\esp\left[T_{1}\right]},\textrm{ c.s. cuando  }t\rightarrow\infty.
\end{eqnarray}
\end{Coro}
%___________________________________________________________________________________________
%
%\subsection{Propiedades de los Procesos de Renovaci\'on}
%___________________________________________________________________________________________
%

Los tiempos $T_{n}$ est\'an relacionados con los conteos de $N\left(t\right)$ por

\begin{eqnarray*}
\left\{N\left(t\right)\geq n\right\}&=&\left\{T_{n}\leq t\right\}\\
T_{N\left(t\right)}\leq &t&<T_{N\left(t\right)+1},
\end{eqnarray*}

adem\'as $N\left(T_{n}\right)=n$, y 

\begin{eqnarray*}
N\left(t\right)=\max\left\{n:T_{n}\leq t\right\}=\min\left\{n:T_{n+1}>t\right\}
\end{eqnarray*}

Por propiedades de la convoluci\'on se sabe que

\begin{eqnarray*}
P\left\{T_{n}\leq t\right\}=F^{n\star}\left(t\right)
\end{eqnarray*}
que es la $n$-\'esima convoluci\'on de $F$. Entonces 

\begin{eqnarray*}
\left\{N\left(t\right)\geq n\right\}&=&\left\{T_{n}\leq t\right\}\\
P\left\{N\left(t\right)\leq n\right\}&=&1-F^{\left(n+1\right)\star}\left(t\right)
\end{eqnarray*}

Adem\'as usando el hecho de que $\esp\left[N\left(t\right)\right]=\sum_{n=1}^{\infty}P\left\{N\left(t\right)\geq n\right\}$
se tiene que

\begin{eqnarray*}
\esp\left[N\left(t\right)\right]=\sum_{n=1}^{\infty}F^{n\star}\left(t\right)
\end{eqnarray*}

\begin{Prop}
Para cada $t\geq0$, la funci\'on generadora de momentos $\esp\left[e^{\alpha N\left(t\right)}\right]$ existe para alguna $\alpha$ en una vecindad del 0, y de aqu\'i que $\esp\left[N\left(t\right)^{m}\right]<\infty$, para $m\geq1$.
\end{Prop}


\begin{Note}
Si el primer tiempo de renovaci\'on $\xi_{1}$ no tiene la misma distribuci\'on que el resto de las $\xi_{n}$, para $n\geq2$, a $N\left(t\right)$ se le llama Proceso de Renovaci\'on retardado, donde si $\xi$ tiene distribuci\'on $G$, entonces el tiempo $T_{n}$ de la $n$-\'esima renovaci\'on tiene distribuci\'on $G\star F^{\left(n-1\right)\star}\left(t\right)$
\end{Note}


\begin{Teo}
Para una constante $\mu\leq\infty$ ( o variable aleatoria), las siguientes expresiones son equivalentes:

\begin{eqnarray}
lim_{n\rightarrow\infty}n^{-1}T_{n}&=&\mu,\textrm{ c.s.}\\
lim_{t\rightarrow\infty}t^{-1}N\left(t\right)&=&1/\mu,\textrm{ c.s.}
\end{eqnarray}
\end{Teo}


Es decir, $T_{n}$ satisface la Ley Fuerte de los Grandes N\'umeros s\'i y s\'olo s\'i $N\left/t\right)$ la cumple.


\begin{Coro}[Ley Fuerte de los Grandes N\'umeros para Procesos de Renovaci\'on]
Si $N\left(t\right)$ es un proceso de renovaci\'on cuyos tiempos de inter-renovaci\'on tienen media $\mu\leq\infty$, entonces
\begin{eqnarray}
t^{-1}N\left(t\right)\rightarrow 1/\mu,\textrm{ c.s. cuando }t\rightarrow\infty.
\end{eqnarray}

\end{Coro}


Considerar el proceso estoc\'astico de valores reales $\left\{Z\left(t\right):t\geq0\right\}$ en el mismo espacio de probabilidad que $N\left(t\right)$

\begin{Def}
Para el proceso $\left\{Z\left(t\right):t\geq0\right\}$ se define la fluctuaci\'on m\'axima de $Z\left(t\right)$ en el intervalo $\left(T_{n-1},T_{n}\right]$:
\begin{eqnarray*}
M_{n}=\sup_{T_{n-1}<t\leq T_{n}}|Z\left(t\right)-Z\left(T_{n-1}\right)|
\end{eqnarray*}
\end{Def}

\begin{Teo}
Sup\'ongase que $n^{-1}T_{n}\rightarrow\mu$ c.s. cuando $n\rightarrow\infty$, donde $\mu\leq\infty$ es una constante o variable aleatoria. Sea $a$ una constante o variable aleatoria que puede ser infinita cuando $\mu$ es finita, y considere las expresiones l\'imite:
\begin{eqnarray}
lim_{n\rightarrow\infty}n^{-1}Z\left(T_{n}\right)&=&a,\textrm{ c.s.}\\
lim_{t\rightarrow\infty}t^{-1}Z\left(t\right)&=&a/\mu,\textrm{ c.s.}
\end{eqnarray}
La segunda expresi\'on implica la primera. Conversamente, la primera implica la segunda si el proceso $Z\left(t\right)$ es creciente, o si $lim_{n\rightarrow\infty}n^{-1}M_{n}=0$ c.s.
\end{Teo}

\begin{Coro}
Si $N\left(t\right)$ es un proceso de renovaci\'on, y $\left(Z\left(T_{n}\right)-Z\left(T_{n-1}\right),M_{n}\right)$, para $n\geq1$, son variables aleatorias independientes e id\'enticamente distribuidas con media finita, entonces,
\begin{eqnarray}
lim_{t\rightarrow\infty}t^{-1}Z\left(t\right)\rightarrow\frac{\esp\left[Z\left(T_{1}\right)-Z\left(T_{0}\right)\right]}{\esp\left[T_{1}\right]},\textrm{ c.s. cuando  }t\rightarrow\infty.
\end{eqnarray}
\end{Coro}


%___________________________________________________________________________________________
%
%\subsection*{Funci\'on de Renovaci\'on}
%___________________________________________________________________________________________
%


\begin{Def}
Sea $h\left(t\right)$ funci\'on de valores reales en $\rea$ acotada en intervalos finitos e igual a cero para $t<0$ La ecuaci\'on de renovaci\'on para $h\left(t\right)$ y la distribuci\'on $F$ es

\begin{eqnarray}\label{Ec.Renovacion}
H\left(t\right)=h\left(t\right)+\int_{\left[0,t\right]}H\left(t-s\right)dF\left(s\right)\textrm{,    }t\geq0,
\end{eqnarray}
donde $H\left(t\right)$ es una funci\'on de valores reales. Esto es $H=h+F\star H$. Decimos que $H\left(t\right)$ es soluci\'on de esta ecuaci\'on si satisface la ecuaci\'on, y es acotada en intervalos finitos e iguales a cero para $t<0$.
\end{Def}

\begin{Prop}
La funci\'on $U\star h\left(t\right)$ es la \'unica soluci\'on de la ecuaci\'on de renovaci\'on (\ref{Ec.Renovacion}).
\end{Prop}

\begin{Teo}[Teorema Renovaci\'on Elemental]
\begin{eqnarray*}
t^{-1}U\left(t\right)\rightarrow 1/\mu\textrm{,    cuando }t\rightarrow\infty.
\end{eqnarray*}
\end{Teo}

%___________________________________________________________________________________________
%
%\subsection{Funci\'on de Renovaci\'on}
%___________________________________________________________________________________________
%


Sup\'ongase que $N\left(t\right)$ es un proceso de renovaci\'on con distribuci\'on $F$ con media finita $\mu$.

\begin{Def}
La funci\'on de renovaci\'on asociada con la distribuci\'on $F$, del proceso $N\left(t\right)$, es
\begin{eqnarray*}
U\left(t\right)=\sum_{n=1}^{\infty}F^{n\star}\left(t\right),\textrm{   }t\geq0,
\end{eqnarray*}
donde $F^{0\star}\left(t\right)=\indora\left(t\geq0\right)$.
\end{Def}


\begin{Prop}
Sup\'ongase que la distribuci\'on de inter-renovaci\'on $F$ tiene densidad $f$. Entonces $U\left(t\right)$ tambi\'en tiene densidad, para $t>0$, y es $U^{'}\left(t\right)=\sum_{n=0}^{\infty}f^{n\star}\left(t\right)$. Adem\'as
\begin{eqnarray*}
\prob\left\{N\left(t\right)>N\left(t-\right)\right\}=0\textrm{,   }t\geq0.
\end{eqnarray*}
\end{Prop}

\begin{Def}
La Transformada de Laplace-Stieljes de $F$ est\'a dada por

\begin{eqnarray*}
\hat{F}\left(\alpha\right)=\int_{\rea_{+}}e^{-\alpha t}dF\left(t\right)\textrm{,  }\alpha\geq0.
\end{eqnarray*}
\end{Def}

Entonces

\begin{eqnarray*}
\hat{U}\left(\alpha\right)=\sum_{n=0}^{\infty}\hat{F^{n\star}}\left(\alpha\right)=\sum_{n=0}^{\infty}\hat{F}\left(\alpha\right)^{n}=\frac{1}{1-\hat{F}\left(\alpha\right)}.
\end{eqnarray*}


\begin{Prop}
La Transformada de Laplace $\hat{U}\left(\alpha\right)$ y $\hat{F}\left(\alpha\right)$ determina una a la otra de manera \'unica por la relaci\'on $\hat{U}\left(\alpha\right)=\frac{1}{1-\hat{F}\left(\alpha\right)}$.
\end{Prop}


\begin{Note}
Un proceso de renovaci\'on $N\left(t\right)$ cuyos tiempos de inter-renovaci\'on tienen media finita, es un proceso Poisson con tasa $\lambda$ si y s\'olo s\'i $\esp\left[U\left(t\right)\right]=\lambda t$, para $t\geq0$.
\end{Note}


\begin{Teo}
Sea $N\left(t\right)$ un proceso puntual simple con puntos de localizaci\'on $T_{n}$ tal que $\eta\left(t\right)=\esp\left[N\left(\right)\right]$ es finita para cada $t$. Entonces para cualquier funci\'on $f:\rea_{+}\rightarrow\rea$,
\begin{eqnarray*}
\esp\left[\sum_{n=1}^{N\left(\right)}f\left(T_{n}\right)\right]=\int_{\left(0,t\right]}f\left(s\right)d\eta\left(s\right)\textrm{,  }t\geq0,
\end{eqnarray*}
suponiendo que la integral exista. Adem\'as si $X_{1},X_{2},\ldots$ son variables aleatorias definidas en el mismo espacio de probabilidad que el proceso $N\left(t\right)$ tal que $\esp\left[X_{n}|T_{n}=s\right]=f\left(s\right)$, independiente de $n$. Entonces
\begin{eqnarray*}
\esp\left[\sum_{n=1}^{N\left(t\right)}X_{n}\right]=\int_{\left(0,t\right]}f\left(s\right)d\eta\left(s\right)\textrm{,  }t\geq0,
\end{eqnarray*} 
suponiendo que la integral exista. 
\end{Teo}

\begin{Coro}[Identidad de Wald para Renovaciones]
Para el proceso de renovaci\'on $N\left(t\right)$,
\begin{eqnarray*}
\esp\left[T_{N\left(t\right)+1}\right]=\mu\esp\left[N\left(t\right)+1\right]\textrm{,  }t\geq0,
\end{eqnarray*}  
\end{Coro}

%______________________________________________________________________
%\subsection{Procesos de Renovaci\'on}
%______________________________________________________________________

\begin{Def}\label{Def.Tn}
Sean $0\leq T_{1}\leq T_{2}\leq \ldots$ son tiempos aleatorios infinitos en los cuales ocurren ciertos eventos. El n\'umero de tiempos $T_{n}$ en el intervalo $\left[0,t\right)$ es

\begin{eqnarray}
N\left(t\right)=\sum_{n=1}^{\infty}\indora\left(T_{n}\leq t\right),
\end{eqnarray}
para $t\geq0$.
\end{Def}

Si se consideran los puntos $T_{n}$ como elementos de $\rea_{+}$, y $N\left(t\right)$ es el n\'umero de puntos en $\rea$. El proceso denotado por $\left\{N\left(t\right):t\geq0\right\}$, denotado por $N\left(t\right)$, es un proceso puntual en $\rea_{+}$. Los $T_{n}$ son los tiempos de ocurrencia, el proceso puntual $N\left(t\right)$ es simple si su n\'umero de ocurrencias son distintas: $0<T_{1}<T_{2}<\ldots$ casi seguramente.

\begin{Def}
Un proceso puntual $N\left(t\right)$ es un proceso de renovaci\'on si los tiempos de interocurrencia $\xi_{n}=T_{n}-T_{n-1}$, para $n\geq1$, son independientes e identicamente distribuidos con distribuci\'on $F$, donde $F\left(0\right)=0$ y $T_{0}=0$. Los $T_{n}$ son llamados tiempos de renovaci\'on, referente a la independencia o renovaci\'on de la informaci\'on estoc\'astica en estos tiempos. Los $\xi_{n}$ son los tiempos de inter-renovaci\'on, y $N\left(t\right)$ es el n\'umero de renovaciones en el intervalo $\left[0,t\right)$
\end{Def}


\begin{Note}
Para definir un proceso de renovaci\'on para cualquier contexto, solamente hay que especificar una distribuci\'on $F$, con $F\left(0\right)=0$, para los tiempos de inter-renovaci\'on. La funci\'on $F$ en turno degune las otra variables aleatorias. De manera formal, existe un espacio de probabilidad y una sucesi\'on de variables aleatorias $\xi_{1},\xi_{2},\ldots$ definidas en este con distribuci\'on $F$. Entonces las otras cantidades son $T_{n}=\sum_{k=1}^{n}\xi_{k}$ y $N\left(t\right)=\sum_{n=1}^{\infty}\indora\left(T_{n}\leq t\right)$, donde $T_{n}\rightarrow\infty$ casi seguramente por la Ley Fuerte de los Grandes Números.
\end{Note}

\begin{Def}\label{Def.Tn}
Sean $0\leq T_{1}\leq T_{2}\leq \ldots$ son tiempos aleatorios infinitos en los cuales ocurren ciertos eventos. El n\'umero de tiempos $T_{n}$ en el intervalo $\left[0,t\right)$ es

\begin{eqnarray}
N\left(t\right)=\sum_{n=1}^{\infty}\indora\left(T_{n}\leq t\right),
\end{eqnarray}
para $t\geq0$.
\end{Def}

Si se consideran los puntos $T_{n}$ como elementos de $\rea_{+}$, y $N\left(t\right)$ es el n\'umero de puntos en $\rea$. El proceso denotado por $\left\{N\left(t\right):t\geq0\right\}$, denotado por $N\left(t\right)$, es un proceso puntual en $\rea_{+}$. Los $T_{n}$ son los tiempos de ocurrencia, el proceso puntual $N\left(t\right)$ es simple si su n\'umero de ocurrencias son distintas: $0<T_{1}<T_{2}<\ldots$ casi seguramente.

\begin{Def}
Un proceso puntual $N\left(t\right)$ es un proceso de renovaci\'on si los tiempos de interocurrencia $\xi_{n}=T_{n}-T_{n-1}$, para $n\geq1$, son independientes e identicamente distribuidos con distribuci\'on $F$, donde $F\left(0\right)=0$ y $T_{0}=0$. Los $T_{n}$ son llamados tiempos de renovaci\'on, referente a la independencia o renovaci\'on de la informaci\'on estoc\'astica en estos tiempos. Los $\xi_{n}$ son los tiempos de inter-renovaci\'on, y $N\left(t\right)$ es el n\'umero de renovaciones en el intervalo $\left[0,t\right)$
\end{Def}


\begin{Note}
Para definir un proceso de renovaci\'on para cualquier contexto, solamente hay que especificar una distribuci\'on $F$, con $F\left(0\right)=0$, para los tiempos de inter-renovaci\'on. La funci\'on $F$ en turno degune las otra variables aleatorias. De manera formal, existe un espacio de probabilidad y una sucesi\'on de variables aleatorias $\xi_{1},\xi_{2},\ldots$ definidas en este con distribuci\'on $F$. Entonces las otras cantidades son $T_{n}=\sum_{k=1}^{n}\xi_{k}$ y $N\left(t\right)=\sum_{n=1}^{\infty}\indora\left(T_{n}\leq t\right)$, donde $T_{n}\rightarrow\infty$ casi seguramente por la Ley Fuerte de los Grandes N\'umeros.
\end{Note}







Los tiempos $T_{n}$ est\'an relacionados con los conteos de $N\left(t\right)$ por

\begin{eqnarray*}
\left\{N\left(t\right)\geq n\right\}&=&\left\{T_{n}\leq t\right\}\\
T_{N\left(t\right)}\leq &t&<T_{N\left(t\right)+1},
\end{eqnarray*}

adem\'as $N\left(T_{n}\right)=n$, y 

\begin{eqnarray*}
N\left(t\right)=\max\left\{n:T_{n}\leq t\right\}=\min\left\{n:T_{n+1}>t\right\}
\end{eqnarray*}

Por propiedades de la convoluci\'on se sabe que

\begin{eqnarray*}
P\left\{T_{n}\leq t\right\}=F^{n\star}\left(t\right)
\end{eqnarray*}
que es la $n$-\'esima convoluci\'on de $F$. Entonces 

\begin{eqnarray*}
\left\{N\left(t\right)\geq n\right\}&=&\left\{T_{n}\leq t\right\}\\
P\left\{N\left(t\right)\leq n\right\}&=&1-F^{\left(n+1\right)\star}\left(t\right)
\end{eqnarray*}

Adem\'as usando el hecho de que $\esp\left[N\left(t\right)\right]=\sum_{n=1}^{\infty}P\left\{N\left(t\right)\geq n\right\}$
se tiene que

\begin{eqnarray*}
\esp\left[N\left(t\right)\right]=\sum_{n=1}^{\infty}F^{n\star}\left(t\right)
\end{eqnarray*}

\begin{Prop}
Para cada $t\geq0$, la funci\'on generadora de momentos $\esp\left[e^{\alpha N\left(t\right)}\right]$ existe para alguna $\alpha$ en una vecindad del 0, y de aqu\'i que $\esp\left[N\left(t\right)^{m}\right]<\infty$, para $m\geq1$.
\end{Prop}

\begin{Ejem}[\textbf{Proceso Poisson}]

Suponga que se tienen tiempos de inter-renovaci\'on \textit{i.i.d.} del proceso de renovaci\'on $N\left(t\right)$ tienen distribuci\'on exponencial $F\left(t\right)=q-e^{-\lambda t}$ con tasa $\lambda$. Entonces $N\left(t\right)$ es un proceso Poisson con tasa $\lambda$.

\end{Ejem}


\begin{Note}
Si el primer tiempo de renovaci\'on $\xi_{1}$ no tiene la misma distribuci\'on que el resto de las $\xi_{n}$, para $n\geq2$, a $N\left(t\right)$ se le llama Proceso de Renovaci\'on retardado, donde si $\xi$ tiene distribuci\'on $G$, entonces el tiempo $T_{n}$ de la $n$-\'esima renovaci\'on tiene distribuci\'on $G\star F^{\left(n-1\right)\star}\left(t\right)$
\end{Note}


\begin{Teo}
Para una constante $\mu\leq\infty$ ( o variable aleatoria), las siguientes expresiones son equivalentes:

\begin{eqnarray}
lim_{n\rightarrow\infty}n^{-1}T_{n}&=&\mu,\textrm{ c.s.}\\
lim_{t\rightarrow\infty}t^{-1}N\left(t\right)&=&1/\mu,\textrm{ c.s.}
\end{eqnarray}
\end{Teo}


Es decir, $T_{n}$ satisface la Ley Fuerte de los Grandes N\'umeros s\'i y s\'olo s\'i $N\left/t\right)$ la cumple.


\begin{Coro}[Ley Fuerte de los Grandes N\'umeros para Procesos de Renovaci\'on]
Si $N\left(t\right)$ es un proceso de renovaci\'on cuyos tiempos de inter-renovaci\'on tienen media $\mu\leq\infty$, entonces
\begin{eqnarray}
t^{-1}N\left(t\right)\rightarrow 1/\mu,\textrm{ c.s. cuando }t\rightarrow\infty.
\end{eqnarray}

\end{Coro}


Considerar el proceso estoc\'astico de valores reales $\left\{Z\left(t\right):t\geq0\right\}$ en el mismo espacio de probabilidad que $N\left(t\right)$

\begin{Def}
Para el proceso $\left\{Z\left(t\right):t\geq0\right\}$ se define la fluctuaci\'on m\'axima de $Z\left(t\right)$ en el intervalo $\left(T_{n-1},T_{n}\right]$:
\begin{eqnarray*}
M_{n}=\sup_{T_{n-1}<t\leq T_{n}}|Z\left(t\right)-Z\left(T_{n-1}\right)|
\end{eqnarray*}
\end{Def}

\begin{Teo}
Sup\'ongase que $n^{-1}T_{n}\rightarrow\mu$ c.s. cuando $n\rightarrow\infty$, donde $\mu\leq\infty$ es una constante o variable aleatoria. Sea $a$ una constante o variable aleatoria que puede ser infinita cuando $\mu$ es finita, y considere las expresiones l\'imite:
\begin{eqnarray}
lim_{n\rightarrow\infty}n^{-1}Z\left(T_{n}\right)&=&a,\textrm{ c.s.}\\
lim_{t\rightarrow\infty}t^{-1}Z\left(t\right)&=&a/\mu,\textrm{ c.s.}
\end{eqnarray}
La segunda expresi\'on implica la primera. Conversamente, la primera implica la segunda si el proceso $Z\left(t\right)$ es creciente, o si $lim_{n\rightarrow\infty}n^{-1}M_{n}=0$ c.s.
\end{Teo}

\begin{Coro}
Si $N\left(t\right)$ es un proceso de renovaci\'on, y $\left(Z\left(T_{n}\right)-Z\left(T_{n-1}\right),M_{n}\right)$, para $n\geq1$, son variables aleatorias independientes e id\'enticamente distribuidas con media finita, entonces,
\begin{eqnarray}
lim_{t\rightarrow\infty}t^{-1}Z\left(t\right)\rightarrow\frac{\esp\left[Z\left(T_{1}\right)-Z\left(T_{0}\right)\right]}{\esp\left[T_{1}\right]},\textrm{ c.s. cuando  }t\rightarrow\infty.
\end{eqnarray}
\end{Coro}


Sup\'ongase que $N\left(t\right)$ es un proceso de renovaci\'on con distribuci\'on $F$ con media finita $\mu$.

\begin{Def}
La funci\'on de renovaci\'on asociada con la distribuci\'on $F$, del proceso $N\left(t\right)$, es
\begin{eqnarray*}
U\left(t\right)=\sum_{n=1}^{\infty}F^{n\star}\left(t\right),\textrm{   }t\geq0,
\end{eqnarray*}
donde $F^{0\star}\left(t\right)=\indora\left(t\geq0\right)$.
\end{Def}


\begin{Prop}
Sup\'ongase que la distribuci\'on de inter-renovaci\'on $F$ tiene densidad $f$. Entonces $U\left(t\right)$ tambi\'en tiene densidad, para $t>0$, y es $U^{'}\left(t\right)=\sum_{n=0}^{\infty}f^{n\star}\left(t\right)$. Adem\'as
\begin{eqnarray*}
\prob\left\{N\left(t\right)>N\left(t-\right)\right\}=0\textrm{,   }t\geq0.
\end{eqnarray*}
\end{Prop}

\begin{Def}
La Transformada de Laplace-Stieljes de $F$ est\'a dada por

\begin{eqnarray*}
\hat{F}\left(\alpha\right)=\int_{\rea_{+}}e^{-\alpha t}dF\left(t\right)\textrm{,  }\alpha\geq0.
\end{eqnarray*}
\end{Def}

Entonces

\begin{eqnarray*}
\hat{U}\left(\alpha\right)=\sum_{n=0}^{\infty}\hat{F^{n\star}}\left(\alpha\right)=\sum_{n=0}^{\infty}\hat{F}\left(\alpha\right)^{n}=\frac{1}{1-\hat{F}\left(\alpha\right)}.
\end{eqnarray*}


\begin{Prop}
La Transformada de Laplace $\hat{U}\left(\alpha\right)$ y $\hat{F}\left(\alpha\right)$ determina una a la otra de manera \'unica por la relaci\'on $\hat{U}\left(\alpha\right)=\frac{1}{1-\hat{F}\left(\alpha\right)}$.
\end{Prop}


\begin{Note}
Un proceso de renovaci\'on $N\left(t\right)$ cuyos tiempos de inter-renovaci\'on tienen media finita, es un proceso Poisson con tasa $\lambda$ si y s\'olo s\'i $\esp\left[U\left(t\right)\right]=\lambda t$, para $t\geq0$.
\end{Note}


\begin{Teo}
Sea $N\left(t\right)$ un proceso puntual simple con puntos de localizaci\'on $T_{n}$ tal que $\eta\left(t\right)=\esp\left[N\left(\right)\right]$ es finita para cada $t$. Entonces para cualquier funci\'on $f:\rea_{+}\rightarrow\rea$,
\begin{eqnarray*}
\esp\left[\sum_{n=1}^{N\left(\right)}f\left(T_{n}\right)\right]=\int_{\left(0,t\right]}f\left(s\right)d\eta\left(s\right)\textrm{,  }t\geq0,
\end{eqnarray*}
suponiendo que la integral exista. Adem\'as si $X_{1},X_{2},\ldots$ son variables aleatorias definidas en el mismo espacio de probabilidad que el proceso $N\left(t\right)$ tal que $\esp\left[X_{n}|T_{n}=s\right]=f\left(s\right)$, independiente de $n$. Entonces
\begin{eqnarray*}
\esp\left[\sum_{n=1}^{N\left(t\right)}X_{n}\right]=\int_{\left(0,t\right]}f\left(s\right)d\eta\left(s\right)\textrm{,  }t\geq0,
\end{eqnarray*} 
suponiendo que la integral exista. 
\end{Teo}

\begin{Coro}[Identidad de Wald para Renovaciones]
Para el proceso de renovaci\'on $N\left(t\right)$,
\begin{eqnarray*}
\esp\left[T_{N\left(t\right)+1}\right]=\mu\esp\left[N\left(t\right)+1\right]\textrm{,  }t\geq0,
\end{eqnarray*}  
\end{Coro}


\begin{Def}
Sea $h\left(t\right)$ funci\'on de valores reales en $\rea$ acotada en intervalos finitos e igual a cero para $t<0$ La ecuaci\'on de renovaci\'on para $h\left(t\right)$ y la distribuci\'on $F$ es

\begin{eqnarray}\label{Ec.Renovacion}
H\left(t\right)=h\left(t\right)+\int_{\left[0,t\right]}H\left(t-s\right)dF\left(s\right)\textrm{,    }t\geq0,
\end{eqnarray}
donde $H\left(t\right)$ es una funci\'on de valores reales. Esto es $H=h+F\star H$. Decimos que $H\left(t\right)$ es soluci\'on de esta ecuaci\'on si satisface la ecuaci\'on, y es acotada en intervalos finitos e iguales a cero para $t<0$.
\end{Def}

\begin{Prop}
La funci\'on $U\star h\left(t\right)$ es la \'unica soluci\'on de la ecuaci\'on de renovaci\'on (\ref{Ec.Renovacion}).
\end{Prop}

\begin{Teo}[Teorema Renovaci\'on Elemental]
\begin{eqnarray*}
t^{-1}U\left(t\right)\rightarrow 1/\mu\textrm{,    cuando }t\rightarrow\infty.
\end{eqnarray*}
\end{Teo}



Sup\'ongase que $N\left(t\right)$ es un proceso de renovaci\'on con distribuci\'on $F$ con media finita $\mu$.

\begin{Def}
La funci\'on de renovaci\'on asociada con la distribuci\'on $F$, del proceso $N\left(t\right)$, es
\begin{eqnarray*}
U\left(t\right)=\sum_{n=1}^{\infty}F^{n\star}\left(t\right),\textrm{   }t\geq0,
\end{eqnarray*}
donde $F^{0\star}\left(t\right)=\indora\left(t\geq0\right)$.
\end{Def}


\begin{Prop}
Sup\'ongase que la distribuci\'on de inter-renovaci\'on $F$ tiene densidad $f$. Entonces $U\left(t\right)$ tambi\'en tiene densidad, para $t>0$, y es $U^{'}\left(t\right)=\sum_{n=0}^{\infty}f^{n\star}\left(t\right)$. Adem\'as
\begin{eqnarray*}
\prob\left\{N\left(t\right)>N\left(t-\right)\right\}=0\textrm{,   }t\geq0.
\end{eqnarray*}
\end{Prop}

\begin{Def}
La Transformada de Laplace-Stieljes de $F$ est\'a dada por

\begin{eqnarray*}
\hat{F}\left(\alpha\right)=\int_{\rea_{+}}e^{-\alpha t}dF\left(t\right)\textrm{,  }\alpha\geq0.
\end{eqnarray*}
\end{Def}

Entonces

\begin{eqnarray*}
\hat{U}\left(\alpha\right)=\sum_{n=0}^{\infty}\hat{F^{n\star}}\left(\alpha\right)=\sum_{n=0}^{\infty}\hat{F}\left(\alpha\right)^{n}=\frac{1}{1-\hat{F}\left(\alpha\right)}.
\end{eqnarray*}


\begin{Prop}
La Transformada de Laplace $\hat{U}\left(\alpha\right)$ y $\hat{F}\left(\alpha\right)$ determina una a la otra de manera \'unica por la relaci\'on $\hat{U}\left(\alpha\right)=\frac{1}{1-\hat{F}\left(\alpha\right)}$.
\end{Prop}


\begin{Note}
Un proceso de renovaci\'on $N\left(t\right)$ cuyos tiempos de inter-renovaci\'on tienen media finita, es un proceso Poisson con tasa $\lambda$ si y s\'olo s\'i $\esp\left[U\left(t\right)\right]=\lambda t$, para $t\geq0$.
\end{Note}


\begin{Teo}
Sea $N\left(t\right)$ un proceso puntual simple con puntos de localizaci\'on $T_{n}$ tal que $\eta\left(t\right)=\esp\left[N\left(\right)\right]$ es finita para cada $t$. Entonces para cualquier funci\'on $f:\rea_{+}\rightarrow\rea$,
\begin{eqnarray*}
\esp\left[\sum_{n=1}^{N\left(\right)}f\left(T_{n}\right)\right]=\int_{\left(0,t\right]}f\left(s\right)d\eta\left(s\right)\textrm{,  }t\geq0,
\end{eqnarray*}
suponiendo que la integral exista. Adem\'as si $X_{1},X_{2},\ldots$ son variables aleatorias definidas en el mismo espacio de probabilidad que el proceso $N\left(t\right)$ tal que $\esp\left[X_{n}|T_{n}=s\right]=f\left(s\right)$, independiente de $n$. Entonces
\begin{eqnarray*}
\esp\left[\sum_{n=1}^{N\left(t\right)}X_{n}\right]=\int_{\left(0,t\right]}f\left(s\right)d\eta\left(s\right)\textrm{,  }t\geq0,
\end{eqnarray*} 
suponiendo que la integral exista. 
\end{Teo}

\begin{Coro}[Identidad de Wald para Renovaciones]
Para el proceso de renovaci\'on $N\left(t\right)$,
\begin{eqnarray*}
\esp\left[T_{N\left(t\right)+1}\right]=\mu\esp\left[N\left(t\right)+1\right]\textrm{,  }t\geq0,
\end{eqnarray*}  
\end{Coro}


\begin{Def}
Sea $h\left(t\right)$ funci\'on de valores reales en $\rea$ acotada en intervalos finitos e igual a cero para $t<0$ La ecuaci\'on de renovaci\'on para $h\left(t\right)$ y la distribuci\'on $F$ es

\begin{eqnarray}\label{Ec.Renovacion}
H\left(t\right)=h\left(t\right)+\int_{\left[0,t\right]}H\left(t-s\right)dF\left(s\right)\textrm{,    }t\geq0,
\end{eqnarray}
donde $H\left(t\right)$ es una funci\'on de valores reales. Esto es $H=h+F\star H$. Decimos que $H\left(t\right)$ es soluci\'on de esta ecuaci\'on si satisface la ecuaci\'on, y es acotada en intervalos finitos e iguales a cero para $t<0$.
\end{Def}

\begin{Prop}
La funci\'on $U\star h\left(t\right)$ es la \'unica soluci\'on de la ecuaci\'on de renovaci\'on (\ref{Ec.Renovacion}).
\end{Prop}

\begin{Teo}[Teorema Renovaci\'on Elemental]
\begin{eqnarray*}
t^{-1}U\left(t\right)\rightarrow 1/\mu\textrm{,    cuando }t\rightarrow\infty.
\end{eqnarray*}
\end{Teo}


\begin{Note} Una funci\'on $h:\rea_{+}\rightarrow\rea$ es Directamente Riemann Integrable en los siguientes casos:
\begin{itemize}
\item[a)] $h\left(t\right)\geq0$ es decreciente y Riemann Integrable.
\item[b)] $h$ es continua excepto posiblemente en un conjunto de Lebesgue de medida 0, y $|h\left(t\right)|\leq b\left(t\right)$, donde $b$ es DRI.
\end{itemize}
\end{Note}

\begin{Teo}[Teorema Principal de Renovaci\'on]
Si $F$ es no aritm\'etica y $h\left(t\right)$ es Directamente Riemann Integrable (DRI), entonces

\begin{eqnarray*}
lim_{t\rightarrow\infty}U\star h=\frac{1}{\mu}\int_{\rea_{+}}h\left(s\right)ds.
\end{eqnarray*}
\end{Teo}

\begin{Prop}
Cualquier funci\'on $H\left(t\right)$ acotada en intervalos finitos y que es 0 para $t<0$ puede expresarse como
\begin{eqnarray*}
H\left(t\right)=U\star h\left(t\right)\textrm{,  donde }h\left(t\right)=H\left(t\right)-F\star H\left(t\right)
\end{eqnarray*}
\end{Prop}

\begin{Def}
Un proceso estoc\'astico $X\left(t\right)$ es crudamente regenerativo en un tiempo aleatorio positivo $T$ si
\begin{eqnarray*}
\esp\left[X\left(T+t\right)|T\right]=\esp\left[X\left(t\right)\right]\textrm{, para }t\geq0,\end{eqnarray*}
y con las esperanzas anteriores finitas.
\end{Def}

\begin{Prop}
Sup\'ongase que $X\left(t\right)$ es un proceso crudamente regenerativo en $T$, que tiene distribuci\'on $F$. Si $\esp\left[X\left(t\right)\right]$ es acotado en intervalos finitos, entonces
\begin{eqnarray*}
\esp\left[X\left(t\right)\right]=U\star h\left(t\right)\textrm{,  donde }h\left(t\right)=\esp\left[X\left(t\right)\indora\left(T>t\right)\right].
\end{eqnarray*}
\end{Prop}

\begin{Teo}[Regeneraci\'on Cruda]
Sup\'ongase que $X\left(t\right)$ es un proceso con valores positivo crudamente regenerativo en $T$, y def\'inase $M=\sup\left\{|X\left(t\right)|:t\leq T\right\}$. Si $T$ es no aritm\'etico y $M$ y $MT$ tienen media finita, entonces
\begin{eqnarray*}
lim_{t\rightarrow\infty}\esp\left[X\left(t\right)\right]=\frac{1}{\mu}\int_{\rea_{+}}h\left(s\right)ds,
\end{eqnarray*}
donde $h\left(t\right)=\esp\left[X\left(t\right)\indora\left(T>t\right)\right]$.
\end{Teo}


\begin{Note} Una funci\'on $h:\rea_{+}\rightarrow\rea$ es Directamente Riemann Integrable en los siguientes casos:
\begin{itemize}
\item[a)] $h\left(t\right)\geq0$ es decreciente y Riemann Integrable.
\item[b)] $h$ es continua excepto posiblemente en un conjunto de Lebesgue de medida 0, y $|h\left(t\right)|\leq b\left(t\right)$, donde $b$ es DRI.
\end{itemize}
\end{Note}

\begin{Teo}[Teorema Principal de Renovaci\'on]
Si $F$ es no aritm\'etica y $h\left(t\right)$ es Directamente Riemann Integrable (DRI), entonces

\begin{eqnarray*}
lim_{t\rightarrow\infty}U\star h=\frac{1}{\mu}\int_{\rea_{+}}h\left(s\right)ds.
\end{eqnarray*}
\end{Teo}

\begin{Prop}
Cualquier funci\'on $H\left(t\right)$ acotada en intervalos finitos y que es 0 para $t<0$ puede expresarse como
\begin{eqnarray*}
H\left(t\right)=U\star h\left(t\right)\textrm{,  donde }h\left(t\right)=H\left(t\right)-F\star H\left(t\right)
\end{eqnarray*}
\end{Prop}

\begin{Def}
Un proceso estoc\'astico $X\left(t\right)$ es crudamente regenerativo en un tiempo aleatorio positivo $T$ si
\begin{eqnarray*}
\esp\left[X\left(T+t\right)|T\right]=\esp\left[X\left(t\right)\right]\textrm{, para }t\geq0,\end{eqnarray*}
y con las esperanzas anteriores finitas.
\end{Def}

\begin{Prop}
Sup\'ongase que $X\left(t\right)$ es un proceso crudamente regenerativo en $T$, que tiene distribuci\'on $F$. Si $\esp\left[X\left(t\right)\right]$ es acotado en intervalos finitos, entonces
\begin{eqnarray*}
\esp\left[X\left(t\right)\right]=U\star h\left(t\right)\textrm{,  donde }h\left(t\right)=\esp\left[X\left(t\right)\indora\left(T>t\right)\right].
\end{eqnarray*}
\end{Prop}

\begin{Teo}[Regeneraci\'on Cruda]
Sup\'ongase que $X\left(t\right)$ es un proceso con valores positivo crudamente regenerativo en $T$, y def\'inase $M=\sup\left\{|X\left(t\right)|:t\leq T\right\}$. Si $T$ es no aritm\'etico y $M$ y $MT$ tienen media finita, entonces
\begin{eqnarray*}
lim_{t\rightarrow\infty}\esp\left[X\left(t\right)\right]=\frac{1}{\mu}\int_{\rea_{+}}h\left(s\right)ds,
\end{eqnarray*}
donde $h\left(t\right)=\esp\left[X\left(t\right)\indora\left(T>t\right)\right]$.
\end{Teo}

\begin{Def}
Para el proceso $\left\{\left(N\left(t\right),X\left(t\right)\right):t\geq0\right\}$, sus trayectoria muestrales en el intervalo de tiempo $\left[T_{n-1},T_{n}\right)$ est\'an descritas por
\begin{eqnarray*}
\zeta_{n}=\left(\xi_{n},\left\{X\left(T_{n-1}+t\right):0\leq t<\xi_{n}\right\}\right)
\end{eqnarray*}
Este $\zeta_{n}$ es el $n$-\'esimo segmento del proceso. El proceso es regenerativo sobre los tiempos $T_{n}$ si sus segmentos $\zeta_{n}$ son independientes e id\'enticamennte distribuidos.
\end{Def}


\begin{Note}
Si $\tilde{X}\left(t\right)$ con espacio de estados $\tilde{S}$ es regenerativo sobre $T_{n}$, entonces $X\left(t\right)=f\left(\tilde{X}\left(t\right)\right)$ tambi\'en es regenerativo sobre $T_{n}$, para cualquier funci\'on $f:\tilde{S}\rightarrow S$.
\end{Note}

\begin{Note}
Los procesos regenerativos son crudamente regenerativos, pero no al rev\'es.
\end{Note}


\begin{Note}
Un proceso estoc\'astico a tiempo continuo o discreto es regenerativo si existe un proceso de renovaci\'on  tal que los segmentos del proceso entre tiempos de renovaci\'on sucesivos son i.i.d., es decir, para $\left\{X\left(t\right):t\geq0\right\}$ proceso estoc\'astico a tiempo continuo con espacio de estados $S$, espacio m\'etrico.
\end{Note}

Para $\left\{X\left(t\right):t\geq0\right\}$ Proceso Estoc\'astico a tiempo continuo con estado de espacios $S$, que es un espacio m\'etrico, con trayectorias continuas por la derecha y con l\'imites por la izquierda c.s. Sea $N\left(t\right)$ un proceso de renovaci\'on en $\rea_{+}$ definido en el mismo espacio de probabilidad que $X\left(t\right)$, con tiempos de renovaci\'on $T$ y tiempos de inter-renovaci\'on $\xi_{n}=T_{n}-T_{n-1}$, con misma distribuci\'on $F$ de media finita $\mu$.



\begin{Def}
Para el proceso $\left\{\left(N\left(t\right),X\left(t\right)\right):t\geq0\right\}$, sus trayectoria muestrales en el intervalo de tiempo $\left[T_{n-1},T_{n}\right)$ est\'an descritas por
\begin{eqnarray*}
\zeta_{n}=\left(\xi_{n},\left\{X\left(T_{n-1}+t\right):0\leq t<\xi_{n}\right\}\right)
\end{eqnarray*}
Este $\zeta_{n}$ es el $n$-\'esimo segmento del proceso. El proceso es regenerativo sobre los tiempos $T_{n}$ si sus segmentos $\zeta_{n}$ son independientes e id\'enticamennte distribuidos.
\end{Def}

\begin{Note}
Un proceso regenerativo con media de la longitud de ciclo finita es llamado positivo recurrente.
\end{Note}

\begin{Teo}[Procesos Regenerativos]
Suponga que el proceso
\end{Teo}


\begin{Def}[Renewal Process Trinity]
Para un proceso de renovaci\'on $N\left(t\right)$, los siguientes procesos proveen de informaci\'on sobre los tiempos de renovaci\'on.
\begin{itemize}
\item $A\left(t\right)=t-T_{N\left(t\right)}$, el tiempo de recurrencia hacia atr\'as al tiempo $t$, que es el tiempo desde la \'ultima renovaci\'on para $t$.

\item $B\left(t\right)=T_{N\left(t\right)+1}-t$, el tiempo de recurrencia hacia adelante al tiempo $t$, residual del tiempo de renovaci\'on, que es el tiempo para la pr\'oxima renovaci\'on despu\'es de $t$.

\item $L\left(t\right)=\xi_{N\left(t\right)+1}=A\left(t\right)+B\left(t\right)$, la longitud del intervalo de renovaci\'on que contiene a $t$.
\end{itemize}
\end{Def}

\begin{Note}
El proceso tridimensional $\left(A\left(t\right),B\left(t\right),L\left(t\right)\right)$ es regenerativo sobre $T_{n}$, y por ende cada proceso lo es. Cada proceso $A\left(t\right)$ y $B\left(t\right)$ son procesos de MArkov a tiempo continuo con trayectorias continuas por partes en el espacio de estados $\rea_{+}$. Una expresi\'on conveniente para su distribuci\'on conjunta es, para $0\leq x<t,y\geq0$
\begin{equation}\label{NoRenovacion}
P\left\{A\left(t\right)>x,B\left(t\right)>y\right\}=
P\left\{N\left(t+y\right)-N\left((t-x)\right)=0\right\}
\end{equation}
\end{Note}

\begin{Ejem}[Tiempos de recurrencia Poisson]
Si $N\left(t\right)$ es un proceso Poisson con tasa $\lambda$, entonces de la expresi\'on (\ref{NoRenovacion}) se tiene que

\begin{eqnarray*}
\begin{array}{lc}
P\left\{A\left(t\right)>x,B\left(t\right)>y\right\}=e^{-\lambda\left(x+y\right)},&0\leq x<t,y\geq0,
\end{array}
\end{eqnarray*}
que es la probabilidad Poisson de no renovaciones en un intervalo de longitud $x+y$.

\end{Ejem}

\begin{Note}
Una cadena de Markov erg\'odica tiene la propiedad de ser estacionaria si la distribuci\'on de su estado al tiempo $0$ es su distribuci\'on estacionaria.
\end{Note}


\begin{Def}
Un proceso estoc\'astico a tiempo continuo $\left\{X\left(t\right):t\geq0\right\}$ en un espacio general es estacionario si sus distribuciones finito dimensionales son invariantes bajo cualquier  traslado: para cada $0\leq s_{1}<s_{2}<\cdots<s_{k}$ y $t\geq0$,
\begin{eqnarray*}
\left(X\left(s_{1}+t\right),\ldots,X\left(s_{k}+t\right)\right)=_{d}\left(X\left(s_{1}\right),\ldots,X\left(s_{k}\right)\right).
\end{eqnarray*}
\end{Def}

\begin{Note}
Un proceso de Markov es estacionario si $X\left(t\right)=_{d}X\left(0\right)$, $t\geq0$.
\end{Note}

Considerese el proceso $N\left(t\right)=\sum_{n}\indora\left(\tau_{n}\leq t\right)$ en $\rea_{+}$, con puntos $0<\tau_{1}<\tau_{2}<\cdots$.

\begin{Prop}
Si $N$ es un proceso puntual estacionario y $\esp\left[N\left(1\right)\right]<\infty$, entonces $\esp\left[N\left(t\right)\right]=t\esp\left[N\left(1\right)\right]$, $t\geq0$

\end{Prop}

\begin{Teo}
Los siguientes enunciados son equivalentes
\begin{itemize}
\item[i)] El proceso retardado de renovaci\'on $N$ es estacionario.

\item[ii)] EL proceso de tiempos de recurrencia hacia adelante $B\left(t\right)$ es estacionario.


\item[iii)] $\esp\left[N\left(t\right)\right]=t/\mu$,


\item[iv)] $G\left(t\right)=F_{e}\left(t\right)=\frac{1}{\mu}\int_{0}^{t}\left[1-F\left(s\right)\right]ds$
\end{itemize}
Cuando estos enunciados son ciertos, $P\left\{B\left(t\right)\leq x\right\}=F_{e}\left(x\right)$, para $t,x\geq0$.

\end{Teo}

\begin{Note}
Una consecuencia del teorema anterior es que el Proceso Poisson es el \'unico proceso sin retardo que es estacionario.
\end{Note}

\begin{Coro}
El proceso de renovaci\'on $N\left(t\right)$ sin retardo, y cuyos tiempos de inter renonaci\'on tienen media finita, es estacionario si y s\'olo si es un proceso Poisson.

\end{Coro}

%______________________________________________________________________

%\section{Ejemplos, Notas importantes}
%______________________________________________________________________
%\section*{Ap\'endice A}
%__________________________________________________________________

%________________________________________________________________________
%\subsection*{Procesos Regenerativos}
%________________________________________________________________________



\begin{Note}
Si $\tilde{X}\left(t\right)$ con espacio de estados $\tilde{S}$ es regenerativo sobre $T_{n}$, entonces $X\left(t\right)=f\left(\tilde{X}\left(t\right)\right)$ tambi\'en es regenerativo sobre $T_{n}$, para cualquier funci\'on $f:\tilde{S}\rightarrow S$.
\end{Note}

\begin{Note}
Los procesos regenerativos son crudamente regenerativos, pero no al rev\'es.
\end{Note}
%\subsection*{Procesos Regenerativos: Sigman\cite{Sigman1}}
\begin{Def}[Definici\'on Cl\'asica]
Un proceso estoc\'astico $X=\left\{X\left(t\right):t\geq0\right\}$ es llamado regenerativo is existe una variable aleatoria $R_{1}>0$ tal que
\begin{itemize}
\item[i)] $\left\{X\left(t+R_{1}\right):t\geq0\right\}$ es independiente de $\left\{\left\{X\left(t\right):t<R_{1}\right\},\right\}$
\item[ii)] $\left\{X\left(t+R_{1}\right):t\geq0\right\}$ es estoc\'asticamente equivalente a $\left\{X\left(t\right):t>0\right\}$
\end{itemize}

Llamamos a $R_{1}$ tiempo de regeneraci\'on, y decimos que $X$ se regenera en este punto.
\end{Def}

$\left\{X\left(t+R_{1}\right)\right\}$ es regenerativo con tiempo de regeneraci\'on $R_{2}$, independiente de $R_{1}$ pero con la misma distribuci\'on que $R_{1}$. Procediendo de esta manera se obtiene una secuencia de variables aleatorias independientes e id\'enticamente distribuidas $\left\{R_{n}\right\}$ llamados longitudes de ciclo. Si definimos a $Z_{k}\equiv R_{1}+R_{2}+\cdots+R_{k}$, se tiene un proceso de renovaci\'on llamado proceso de renovaci\'on encajado para $X$.




\begin{Def}
Para $x$ fijo y para cada $t\geq0$, sea $I_{x}\left(t\right)=1$ si $X\left(t\right)\leq x$,  $I_{x}\left(t\right)=0$ en caso contrario, y def\'inanse los tiempos promedio
\begin{eqnarray*}
\overline{X}&=&lim_{t\rightarrow\infty}\frac{1}{t}\int_{0}^{\infty}X\left(u\right)du\\
\prob\left(X_{\infty}\leq x\right)&=&lim_{t\rightarrow\infty}\frac{1}{t}\int_{0}^{\infty}I_{x}\left(u\right)du,
\end{eqnarray*}
cuando estos l\'imites existan.
\end{Def}

Como consecuencia del teorema de Renovaci\'on-Recompensa, se tiene que el primer l\'imite  existe y es igual a la constante
\begin{eqnarray*}
\overline{X}&=&\frac{\esp\left[\int_{0}^{R_{1}}X\left(t\right)dt\right]}{\esp\left[R_{1}\right]},
\end{eqnarray*}
suponiendo que ambas esperanzas son finitas.

\begin{Note}
\begin{itemize}
\item[a)] Si el proceso regenerativo $X$ es positivo recurrente y tiene trayectorias muestrales no negativas, entonces la ecuaci\'on anterior es v\'alida.
\item[b)] Si $X$ es positivo recurrente regenerativo, podemos construir una \'unica versi\'on estacionaria de este proceso, $X_{e}=\left\{X_{e}\left(t\right)\right\}$, donde $X_{e}$ es un proceso estoc\'astico regenerativo y estrictamente estacionario, con distribuci\'on marginal distribuida como $X_{\infty}$
\end{itemize}
\end{Note}

Para $\left\{X\left(t\right):t\geq0\right\}$ Proceso Estoc\'astico a tiempo continuo con estado de espacios $S$, que es un espacio m\'etrico, con trayectorias continuas por la derecha y con l\'imites por la izquierda c.s. Sea $N\left(t\right)$ un proceso de renovaci\'on en $\rea_{+}$ definido en el mismo espacio de probabilidad que $X\left(t\right)$, con tiempos de renovaci\'on $T$ y tiempos de inter-renovaci\'on $\xi_{n}=T_{n}-T_{n-1}$, con misma distribuci\'on $F$ de media finita $\mu$.


\begin{Def}
Para el proceso $\left\{\left(N\left(t\right),X\left(t\right)\right):t\geq0\right\}$, sus trayectoria muestrales en el intervalo de tiempo $\left[T_{n-1},T_{n}\right)$ est\'an descritas por
\begin{eqnarray*}
\zeta_{n}=\left(\xi_{n},\left\{X\left(T_{n-1}+t\right):0\leq t<\xi_{n}\right\}\right)
\end{eqnarray*}
Este $\zeta_{n}$ es el $n$-\'esimo segmento del proceso. El proceso es regenerativo sobre los tiempos $T_{n}$ si sus segmentos $\zeta_{n}$ son independientes e id\'enticamennte distribuidos.
\end{Def}


\begin{Note}
Si $\tilde{X}\left(t\right)$ con espacio de estados $\tilde{S}$ es regenerativo sobre $T_{n}$, entonces $X\left(t\right)=f\left(\tilde{X}\left(t\right)\right)$ tambi\'en es regenerativo sobre $T_{n}$, para cualquier funci\'on $f:\tilde{S}\rightarrow S$.
\end{Note}

\begin{Note}
Los procesos regenerativos son crudamente regenerativos, pero no al rev\'es.
\end{Note}

\begin{Def}[Definici\'on Cl\'asica]
Un proceso estoc\'astico $X=\left\{X\left(t\right):t\geq0\right\}$ es llamado regenerativo is existe una variable aleatoria $R_{1}>0$ tal que
\begin{itemize}
\item[i)] $\left\{X\left(t+R_{1}\right):t\geq0\right\}$ es independiente de $\left\{\left\{X\left(t\right):t<R_{1}\right\},\right\}$
\item[ii)] $\left\{X\left(t+R_{1}\right):t\geq0\right\}$ es estoc\'asticamente equivalente a $\left\{X\left(t\right):t>0\right\}$
\end{itemize}

Llamamos a $R_{1}$ tiempo de regeneraci\'on, y decimos que $X$ se regenera en este punto.
\end{Def}

$\left\{X\left(t+R_{1}\right)\right\}$ es regenerativo con tiempo de regeneraci\'on $R_{2}$, independiente de $R_{1}$ pero con la misma distribuci\'on que $R_{1}$. Procediendo de esta manera se obtiene una secuencia de variables aleatorias independientes e id\'enticamente distribuidas $\left\{R_{n}\right\}$ llamados longitudes de ciclo. Si definimos a $Z_{k}\equiv R_{1}+R_{2}+\cdots+R_{k}$, se tiene un proceso de renovaci\'on llamado proceso de renovaci\'on encajado para $X$.

\begin{Note}
Un proceso regenerativo con media de la longitud de ciclo finita es llamado positivo recurrente.
\end{Note}


\begin{Def}
Para $x$ fijo y para cada $t\geq0$, sea $I_{x}\left(t\right)=1$ si $X\left(t\right)\leq x$,  $I_{x}\left(t\right)=0$ en caso contrario, y def\'inanse los tiempos promedio
\begin{eqnarray*}
\overline{X}&=&lim_{t\rightarrow\infty}\frac{1}{t}\int_{0}^{\infty}X\left(u\right)du\\
\prob\left(X_{\infty}\leq x\right)&=&lim_{t\rightarrow\infty}\frac{1}{t}\int_{0}^{\infty}I_{x}\left(u\right)du,
\end{eqnarray*}
cuando estos l\'imites existan.
\end{Def}

Como consecuencia del teorema de Renovaci\'on-Recompensa, se tiene que el primer l\'imite  existe y es igual a la constante
\begin{eqnarray*}
\overline{X}&=&\frac{\esp\left[\int_{0}^{R_{1}}X\left(t\right)dt\right]}{\esp\left[R_{1}\right]},
\end{eqnarray*}
suponiendo que ambas esperanzas son finitas.

\begin{Note}
\begin{itemize}
\item[a)] Si el proceso regenerativo $X$ es positivo recurrente y tiene trayectorias muestrales no negativas, entonces la ecuaci\'on anterior es v\'alida.
\item[b)] Si $X$ es positivo recurrente regenerativo, podemos construir una \'unica versi\'on estacionaria de este proceso, $X_{e}=\left\{X_{e}\left(t\right)\right\}$, donde $X_{e}$ es un proceso estoc\'astico regenerativo y estrictamente estacionario, con distribuci\'on marginal distribuida como $X_{\infty}$
\end{itemize}
\end{Note}

%__________________________________________________________________________________________
%\subsection{Procesos Regenerativos Estacionarios - Stidham \cite{Stidham}}
%__________________________________________________________________________________________


Un proceso estoc\'astico a tiempo continuo $\left\{V\left(t\right),t\geq0\right\}$ es un proceso regenerativo si existe una sucesi\'on de variables aleatorias independientes e id\'enticamente distribuidas $\left\{X_{1},X_{2},\ldots\right\}$, sucesi\'on de renovaci\'on, tal que para cualquier conjunto de Borel $A$, 

\begin{eqnarray*}
\prob\left\{V\left(t\right)\in A|X_{1}+X_{2}+\cdots+X_{R\left(t\right)}=s,\left\{V\left(\tau\right),\tau<s\right\}\right\}=\prob\left\{V\left(t-s\right)\in A|X_{1}>t-s\right\},
\end{eqnarray*}
para todo $0\leq s\leq t$, donde $R\left(t\right)=\max\left\{X_{1}+X_{2}+\cdots+X_{j}\leq t\right\}=$n\'umero de renovaciones ({\emph{puntos de regeneraci\'on}}) que ocurren en $\left[0,t\right]$. El intervalo $\left[0,X_{1}\right)$ es llamado {\emph{primer ciclo de regeneraci\'on}} de $\left\{V\left(t \right),t\geq0\right\}$, $\left[X_{1},X_{1}+X_{2}\right)$ el {\emph{segundo ciclo de regeneraci\'on}}, y as\'i sucesivamente.

Sea $X=X_{1}$ y sea $F$ la funci\'on de distrbuci\'on de $X$


\begin{Def}
Se define el proceso estacionario, $\left\{V^{*}\left(t\right),t\geq0\right\}$, para $\left\{V\left(t\right),t\geq0\right\}$ por

\begin{eqnarray*}
\prob\left\{V\left(t\right)\in A\right\}=\frac{1}{\esp\left[X\right]}\int_{0}^{\infty}\prob\left\{V\left(t+x\right)\in A|X>x\right\}\left(1-F\left(x\right)\right)dx,
\end{eqnarray*} 
para todo $t\geq0$ y todo conjunto de Borel $A$.
\end{Def}

\begin{Def}
Una distribuci\'on se dice que es {\emph{aritm\'etica}} si todos sus puntos de incremento son m\'ultiplos de la forma $0,\lambda, 2\lambda,\ldots$ para alguna $\lambda>0$ entera.
\end{Def}


\begin{Def}
Una modificaci\'on medible de un proceso $\left\{V\left(t\right),t\geq0\right\}$, es una versi\'on de este, $\left\{V\left(t,w\right)\right\}$ conjuntamente medible para $t\geq0$ y para $w\in S$, $S$ espacio de estados para $\left\{V\left(t\right),t\geq0\right\}$.
\end{Def}

\begin{Teo}
Sea $\left\{V\left(t\right),t\geq\right\}$ un proceso regenerativo no negativo con modificaci\'on medible. Sea $\esp\left[X\right]<\infty$. Entonces el proceso estacionario dado por la ecuaci\'on anterior est\'a bien definido y tiene funci\'on de distribuci\'on independiente de $t$, adem\'as
\begin{itemize}
\item[i)] \begin{eqnarray*}
\esp\left[V^{*}\left(0\right)\right]&=&\frac{\esp\left[\int_{0}^{X}V\left(s\right)ds\right]}{\esp\left[X\right]}\end{eqnarray*}
\item[ii)] Si $\esp\left[V^{*}\left(0\right)\right]<\infty$, equivalentemente, si $\esp\left[\int_{0}^{X}V\left(s\right)ds\right]<\infty$,entonces
\begin{eqnarray*}
\frac{\int_{0}^{t}V\left(s\right)ds}{t}\rightarrow\frac{\esp\left[\int_{0}^{X}V\left(s\right)ds\right]}{\esp\left[X\right]}
\end{eqnarray*}
con probabilidad 1 y en media, cuando $t\rightarrow\infty$.
\end{itemize}
\end{Teo}
%
%___________________________________________________________________________________________
%\vspace{5.5cm}
%\chapter{Cadenas de Markov estacionarias}
%\vspace{-1.0cm}


%__________________________________________________________________________________________
%\subsection{Procesos Regenerativos Estacionarios - Stidham \cite{Stidham}}
%__________________________________________________________________________________________


Un proceso estoc\'astico a tiempo continuo $\left\{V\left(t\right),t\geq0\right\}$ es un proceso regenerativo si existe una sucesi\'on de variables aleatorias independientes e id\'enticamente distribuidas $\left\{X_{1},X_{2},\ldots\right\}$, sucesi\'on de renovaci\'on, tal que para cualquier conjunto de Borel $A$, 

\begin{eqnarray*}
\prob\left\{V\left(t\right)\in A|X_{1}+X_{2}+\cdots+X_{R\left(t\right)}=s,\left\{V\left(\tau\right),\tau<s\right\}\right\}=\prob\left\{V\left(t-s\right)\in A|X_{1}>t-s\right\},
\end{eqnarray*}
para todo $0\leq s\leq t$, donde $R\left(t\right)=\max\left\{X_{1}+X_{2}+\cdots+X_{j}\leq t\right\}=$n\'umero de renovaciones ({\emph{puntos de regeneraci\'on}}) que ocurren en $\left[0,t\right]$. El intervalo $\left[0,X_{1}\right)$ es llamado {\emph{primer ciclo de regeneraci\'on}} de $\left\{V\left(t \right),t\geq0\right\}$, $\left[X_{1},X_{1}+X_{2}\right)$ el {\emph{segundo ciclo de regeneraci\'on}}, y as\'i sucesivamente.

Sea $X=X_{1}$ y sea $F$ la funci\'on de distrbuci\'on de $X$


\begin{Def}
Se define el proceso estacionario, $\left\{V^{*}\left(t\right),t\geq0\right\}$, para $\left\{V\left(t\right),t\geq0\right\}$ por

\begin{eqnarray*}
\prob\left\{V\left(t\right)\in A\right\}=\frac{1}{\esp\left[X\right]}\int_{0}^{\infty}\prob\left\{V\left(t+x\right)\in A|X>x\right\}\left(1-F\left(x\right)\right)dx,
\end{eqnarray*} 
para todo $t\geq0$ y todo conjunto de Borel $A$.
\end{Def}

\begin{Def}
Una distribuci\'on se dice que es {\emph{aritm\'etica}} si todos sus puntos de incremento son m\'ultiplos de la forma $0,\lambda, 2\lambda,\ldots$ para alguna $\lambda>0$ entera.
\end{Def}


\begin{Def}
Una modificaci\'on medible de un proceso $\left\{V\left(t\right),t\geq0\right\}$, es una versi\'on de este, $\left\{V\left(t,w\right)\right\}$ conjuntamente medible para $t\geq0$ y para $w\in S$, $S$ espacio de estados para $\left\{V\left(t\right),t\geq0\right\}$.
\end{Def}

\begin{Teo}
Sea $\left\{V\left(t\right),t\geq\right\}$ un proceso regenerativo no negativo con modificaci\'on medible. Sea $\esp\left[X\right]<\infty$. Entonces el proceso estacionario dado por la ecuaci\'on anterior est\'a bien definido y tiene funci\'on de distribuci\'on independiente de $t$, adem\'as
\begin{itemize}
\item[i)] \begin{eqnarray*}
\esp\left[V^{*}\left(0\right)\right]&=&\frac{\esp\left[\int_{0}^{X}V\left(s\right)ds\right]}{\esp\left[X\right]}\end{eqnarray*}
\item[ii)] Si $\esp\left[V^{*}\left(0\right)\right]<\infty$, equivalentemente, si $\esp\left[\int_{0}^{X}V\left(s\right)ds\right]<\infty$,entonces
\begin{eqnarray*}
\frac{\int_{0}^{t}V\left(s\right)ds}{t}\rightarrow\frac{\esp\left[\int_{0}^{X}V\left(s\right)ds\right]}{\esp\left[X\right]}
\end{eqnarray*}
con probabilidad 1 y en media, cuando $t\rightarrow\infty$.
\end{itemize}
\end{Teo}

Sea la funci\'on generadora de momentos para $L_{i}$, el n\'umero de usuarios en la cola $Q_{i}\left(z\right)$ en cualquier momento, est\'a dada por el tiempo promedio de $z^{L_{i}\left(t\right)}$ sobre el ciclo regenerativo definido anteriormente. Entonces 



Es decir, es posible determinar las longitudes de las colas a cualquier tiempo $t$. Entonces, determinando el primer momento es posible ver que


\begin{Def}
El tiempo de Ciclo $C_{i}$ es el periodo de tiempo que comienza cuando la cola $i$ es visitada por primera vez en un ciclo, y termina cuando es visitado nuevamente en el pr\'oximo ciclo. La duraci\'on del mismo est\'a dada por $\tau_{i}\left(m+1\right)-\tau_{i}\left(m\right)$, o equivalentemente $\overline{\tau}_{i}\left(m+1\right)-\overline{\tau}_{i}\left(m\right)$ bajo condiciones de estabilidad.
\end{Def}


\begin{Def}
El tiempo de intervisita $I_{i}$ es el periodo de tiempo que comienza cuando se ha completado el servicio en un ciclo y termina cuando es visitada nuevamente en el pr\'oximo ciclo. Su  duraci\'on del mismo est\'a dada por $\tau_{i}\left(m+1\right)-\overline{\tau}_{i}\left(m\right)$.
\end{Def}

La duraci\'on del tiempo de intervisita es $\tau_{i}\left(m+1\right)-\overline{\tau}\left(m\right)$. Dado que el n\'umero de usuarios presentes en $Q_{i}$ al tiempo $t=\tau_{i}\left(m+1\right)$ es igual al n\'umero de arribos durante el intervalo de tiempo $\left[\overline{\tau}\left(m\right),\tau_{i}\left(m+1\right)\right]$ se tiene que


\begin{eqnarray*}
\esp\left[z_{i}^{L_{i}\left(\tau_{i}\left(m+1\right)\right)}\right]=\esp\left[\left\{P_{i}\left(z_{i}\right)\right\}^{\tau_{i}\left(m+1\right)-\overline{\tau}\left(m\right)}\right]
\end{eqnarray*}

entonces, si $I_{i}\left(z\right)=\esp\left[z^{\tau_{i}\left(m+1\right)-\overline{\tau}\left(m\right)}\right]$
se tiene que $F_{i}\left(z\right)=I_{i}\left[P_{i}\left(z\right)\right]$
para $i=1,2$.

Conforme a la definici\'on dada al principio del cap\'itulo, definici\'on (\ref{Def.Tn}), sean $T_{1},T_{2},\ldots$ los puntos donde las longitudes de las colas de la red de sistemas de visitas c\'iclicas son cero simult\'aneamente, cuando la cola $Q_{j}$ es visitada por el servidor para dar servicio, es decir, $L_{1}\left(T_{i}\right)=0,L_{2}\left(T_{i}\right)=0,\hat{L}_{1}\left(T_{i}\right)=0$ y $\hat{L}_{2}\left(T_{i}\right)=0$, a estos puntos se les denominar\'a puntos regenerativos. Entonces, 

\begin{Def}
Al intervalo de tiempo entre dos puntos regenerativos se le llamar\'a ciclo regenerativo.
\end{Def}

\begin{Def}
Para $T_{i}$ se define, $M_{i}$, el n\'umero de ciclos de visita a la cola $Q_{l}$, durante el ciclo regenerativo, es decir, $M_{i}$ es un proceso de renovaci\'on.
\end{Def}

\begin{Def}
Para cada uno de los $M_{i}$'s, se definen a su vez la duraci\'on de cada uno de estos ciclos de visita en el ciclo regenerativo, $C_{i}^{(m)}$, para $m=1,2,\ldots,M_{i}$, que a su vez, tambi\'en es n proceso de renovaci\'on.
\end{Def}

\footnote{In Stidham and  Heyman \cite{Stidham} shows that is sufficient for the regenerative process to be stationary that the mean regenerative cycle time is finite: $\esp\left[\sum_{m=1}^{M_{i}}C_{i}^{(m)}\right]<\infty$, 


 como cada $C_{i}^{(m)}$ contiene intervalos de r\'eplica positivos, se tiene que $\esp\left[M_{i}\right]<\infty$, adem\'as, como $M_{i}>0$, se tiene que la condici\'on anterior es equivalente a tener que $\esp\left[C_{i}\right]<\infty$,
por lo tanto una condici\'on suficiente para la existencia del proceso regenerativo est\'a dada por $\sum_{k=1}^{N}\mu_{k}<1.$}

Para $\left\{X\left(t\right):t\geq0\right\}$ Proceso Estoc\'astico a tiempo continuo con estado de espacios $S$, que es un espacio m\'etrico, con trayectorias continuas por la derecha y con l\'imites por la izquierda c.s. Sea $N\left(t\right)$ un proceso de renovaci\'on en $\rea_{+}$ definido en el mismo espacio de probabilidad que $X\left(t\right)$, con tiempos de renovaci\'on $T$ y tiempos de inter-renovaci\'on $\xi_{n}=T_{n}-T_{n-1}$, con misma distribuci\'on $F$ de media finita $\mu$.

\begin{Def}
Un elemento aleatorio en un espacio medible $\left(E,\mathcal{E}\right)$ en un espacio de probabilidad $\left(\Omega,\mathcal{F},\prob\right)$ a $\left(E,\mathcal{E}\right)$, es decir,
para $A\in \mathcal{E}$,  se tiene que $\left\{Y\in A\right\}\in\mathcal{F}$, donde $\left\{Y\in A\right\}:=\left\{w\in\Omega:Y\left(w\right)\in A\right\}=:Y^{-1}A$.
\end{Def}

\begin{Note}
Tambi\'en se dice que $Y$ est\'a soportado por el espacio de probabilidad $\left(\Omega,\mathcal{F},\prob\right)$ y que $Y$ es un mapeo medible de $\Omega$ en $E$, es decir, es $\mathcal{F}/\mathcal{E}$ medible.
\end{Note}

\begin{Def}
Para cada $i\in \mathbb{I}$ sea $P_{i}$ una medida de probabilidad en un espacio medible $\left(E_{i},\mathcal{E}_{i}\right)$. Se define el espacio producto
$\otimes_{i\in\mathbb{I}}\left(E_{i},\mathcal{E}_{i}\right):=\left(\prod_{i\in\mathbb{I}}E_{i},\otimes_{i\in\mathbb{I}}\mathcal{E}_{i}\right)$, donde $\prod_{i\in\mathbb{I}}E_{i}$ es el producto cartesiano de los $E_{i}$'s, y $\otimes_{i\in\mathbb{I}}\mathcal{E}_{i}$ es la $\sigma$-\'algebra producto, es decir, es la $\sigma$-\'algebra m\'as peque\~na en $\prod_{i\in\mathbb{I}}E_{i}$ que hace al $i$-\'esimo mapeo proyecci\'on en $E_{i}$ medible para toda $i\in\mathbb{I}$ es la $\sigma$-\'algebra inducida por los mapeos proyecci\'on. $$\otimes_{i\in\mathbb{I}}\mathcal{E}_{i}:=\sigma\left\{\left\{y:y_{i}\in A\right\}:i\in\mathbb{I}\textrm{ y }A\in\mathcal{E}_{i}\right\}.$$
\end{Def}

\begin{Def}
Un espacio de probabilidad $\left(\tilde{\Omega},\tilde{\mathcal{F}},\tilde{\prob}\right)$ es una extensi\'on de otro espacio de probabilidad $\left(\Omega,\mathcal{F},\prob\right)$ si $\left(\tilde{\Omega},\tilde{\mathcal{F}},\tilde{\prob}\right)$ soporta un elemento aleatorio $\xi\in\left(\Omega,\mathcal{F}\right)$ que tienen a $\prob$ como distribuci\'on.
\end{Def}

\begin{Teo}
Sea $\mathbb{I}$ un conjunto de \'indices arbitrario. Para cada $i\in\mathbb{I}$ sea $P_{i}$ una medida de probabilidad en un espacio medible $\left(E_{i},\mathcal{E}_{i}\right)$. Entonces existe una \'unica medida de probabilidad $\otimes_{i\in\mathbb{I}}P_{i}$ en $\otimes_{i\in\mathbb{I}}\left(E_{i},\mathcal{E}_{i}\right)$ tal que 

\begin{eqnarray*}
\otimes_{i\in\mathbb{I}}P_{i}\left(y\in\prod_{i\in\mathbb{I}}E_{i}:y_{i}\in A_{i_{1}},\ldots,y_{n}\in A_{i_{n}}\right)=P_{i_{1}}\left(A_{i_{n}}\right)\cdots P_{i_{n}}\left(A_{i_{n}}\right)
\end{eqnarray*}
para todos los enteros $n>0$, toda $i_{1},\ldots,i_{n}\in\mathbb{I}$ y todo $A_{i_{1}}\in\mathcal{E}_{i_{1}},\ldots,A_{i_{n}}\in\mathcal{E}_{i_{n}}$
\end{Teo}

La medida $\otimes_{i\in\mathbb{I}}P_{i}$ es llamada la medida producto y $\otimes_{i\in\mathbb{I}}\left(E_{i},\mathcal{E}_{i},P_{i}\right):=\left(\prod_{i\in\mathbb{I}},E_{i},\otimes_{i\in\mathbb{I}}\mathcal{E}_{i},\otimes_{i\in\mathbb{I}}P_{i}\right)$, es llamado espacio de probabilidad producto.


\begin{Def}
Un espacio medible $\left(E,\mathcal{E}\right)$ es \textit{Polaco} si existe una m\'etrica en $E$ tal que $E$ es completo, es decir cada sucesi\'on de Cauchy converge a un l\'imite en $E$, y \textit{separable}, $E$ tienen un subconjunto denso numerable, y tal que $\mathcal{E}$ es generado por conjuntos abiertos.
\end{Def}


\begin{Def}
Dos espacios medibles $\left(E,\mathcal{E}\right)$ y $\left(G,\mathcal{G}\right)$ son Borel equivalentes \textit{isomorfos} si existe una biyecci\'on $f:E\rightarrow G$ tal que $f$ es $\mathcal{E}/\mathcal{G}$ medible y su inversa $f^{-1}$ es $\mathcal{G}/\mathcal{E}$ medible. La biyecci\'on es una equivalencia de Borel.
\end{Def}

\begin{Def}
Un espacio medible  $\left(E,\mathcal{E}\right)$ es un \textit{espacio est\'andar} si es Borel equivalente a $\left(G,\mathcal{G}\right)$, donde $G$ es un subconjunto de Borel de $\left[0,1\right]$ y $\mathcal{G}$ son los subconjuntos de Borel de $G$.
\end{Def}

\begin{Note}
Cualquier espacio Polaco es un espacio est\'andar.
\end{Note}


\begin{Def}
Un proceso estoc\'astico con conjunto de \'indices $\mathbb{I}$ y espacio de estados $\left(E,\mathcal{E}\right)$ es una familia $Z=\left(\mathbb{Z}_{s}\right)_{s\in\mathbb{I}}$ donde $\mathbb{Z}_{s}$ son elementos aleatorios definidos en un espacio de probabilidad com\'un $\left(\Omega,\mathcal{F},\prob\right)$ y todos toman valores en $\left(E,\mathcal{E}\right)$.
\end{Def}

\begin{Def}
Un proceso estoc\'astico \textit{one-sided contiuous time} (\textbf{PEOSCT}) es un proceso estoc\'astico con conjunto de \'indices $\mathbb{I}=\left[0,\infty\right)$.
\end{Def}


Sea $\left(E^{\mathbb{I}},\mathcal{E}^{\mathbb{I}}\right)$ denota el espacio producto $\left(E^{\mathbb{I}},\mathcal{E}^{\mathbb{I}}\right):=\otimes_{s\in\mathbb{I}}\left(E,\mathcal{E}\right)$. Vamos a considerar $\mathbb{Z}$ como un mapeo aleatorio, es decir, como un elemento aleatorio en $\left(E^{\mathbb{I}},\mathcal{E}^{\mathbb{I}}\right)$ definido por $Z\left(w\right)=\left(Z_{s}\left(w\right)\right)_{s\in\mathbb{I}}$ y $w\in\Omega$.

\begin{Note}
La distribuci\'on de un proceso estoc\'astico $Z$ es la distribuci\'on de $Z$ como un elemento aleatorio en $\left(E^{\mathbb{I}},\mathcal{E}^{\mathbb{I}}\right)$. La distribuci\'on de $Z$ esta determinada de manera \'unica por las distribuciones finito dimensionales.
\end{Note}

\begin{Note}
En particular cuando $Z$ toma valores reales, es decir, $\left(E,\mathcal{E}\right)=\left(\mathbb{R},\mathcal{B}\right)$ las distribuciones finito dimensionales est\'an determinadas por las funciones de distribuci\'on finito dimensionales

\begin{eqnarray}
\prob\left(Z_{t_{1}}\leq x_{1},\ldots,Z_{t_{n}}\leq x_{n}\right),x_{1},\ldots,x_{n}\in\mathbb{R},t_{1},\ldots,t_{n}\in\mathbb{I},n\geq1.
\end{eqnarray}
\end{Note}

\begin{Note}
Para espacios polacos $\left(E,\mathcal{E}\right)$ el Teorema de Consistencia de Kolmogorov asegura que dada una colecci\'on de distribuciones finito dimensionales consistentes, siempre existe un proceso estoc\'astico que posee tales distribuciones finito dimensionales.
\end{Note}


\begin{Def}
Las trayectorias de $Z$ son las realizaciones $Z\left(w\right)$ para $w\in\Omega$ del mapeo aleatorio $Z$.
\end{Def}

\begin{Note}
Algunas restricciones se imponen sobre las trayectorias, por ejemplo que sean continuas por la derecha, o continuas por la derecha con l\'imites por la izquierda, o de manera m\'as general, se pedir\'a que caigan en alg\'un subconjunto $H$ de $E^{\mathbb{I}}$. En este caso es natural considerar a $Z$ como un elemento aleatorio que no est\'a en $\left(E^{\mathbb{I}},\mathcal{E}^{\mathbb{I}}\right)$ sino en $\left(H,\mathcal{H}\right)$, donde $\mathcal{H}$ es la $\sigma$-\'algebra generada por los mapeos proyecci\'on que toman a $z\in H$ a $z_{t}\in E$ para $t\in\mathbb{I}$. A $\mathcal{H}$ se le conoce como la traza de $H$ en $E^{\mathbb{I}}$, es decir,
\begin{eqnarray}
\mathcal{H}:=E^{\mathbb{I}}\cap H:=\left\{A\cap H:A\in E^{\mathbb{I}}\right\}.
\end{eqnarray}
\end{Note}


\begin{Note}
$Z$ tiene trayectorias con valores en $H$ y cada $Z_{t}$ es un mapeo medible de $\left(\Omega,\mathcal{F}\right)$ a $\left(H,\mathcal{H}\right)$. Cuando se considera un espacio de trayectorias en particular $H$, al espacio $\left(H,\mathcal{H}\right)$ se le llama el espacio de trayectorias de $Z$.
\end{Note}

\begin{Note}
La distribuci\'on del proceso estoc\'astico $Z$ con espacio de trayectorias $\left(H,\mathcal{H}\right)$ es la distribuci\'on de $Z$ como  un elemento aleatorio en $\left(H,\mathcal{H}\right)$. La distribuci\'on, nuevemente, est\'a determinada de manera \'unica por las distribuciones finito dimensionales.
\end{Note}


\begin{Def}
Sea $Z$ un PEOSCT  con espacio de estados $\left(E,\mathcal{E}\right)$ y sea $T$ un tiempo aleatorio en $\left[0,\infty\right)$. Por $Z_{T}$ se entiende el mapeo con valores en $E$ definido en $\Omega$ en la manera obvia:
\begin{eqnarray*}
Z_{T}\left(w\right):=Z_{T\left(w\right)}\left(w\right). w\in\Omega.
\end{eqnarray*}
\end{Def}

\begin{Def}
Un PEOSCT $Z$ es conjuntamente medible (\textbf{CM}) si el mapeo que toma $\left(w,t\right)\in\Omega\times\left[0,\infty\right)$ a $Z_{t}\left(w\right)\in E$ es $\mathcal{F}\otimes\mathcal{B}\left[0,\infty\right)/\mathcal{E}$ medible.
\end{Def}

\begin{Note}
Un PEOSCT-CM implica que el proceso es medible, dado que $Z_{T}$ es una composici\'on  de dos mapeos continuos: el primero que toma $w$ en $\left(w,T\left(w\right)\right)$ es $\mathcal{F}/\mathcal{F}\otimes\mathcal{B}\left[0,\infty\right)$ medible, mientras que el segundo toma $\left(w,T\left(w\right)\right)$ en $Z_{T\left(w\right)}\left(w\right)$ es $\mathcal{F}\otimes\mathcal{B}\left[0,\infty\right)/\mathcal{E}$ medible.
\end{Note}


\begin{Def}
Un PEOSCT con espacio de estados $\left(H,\mathcal{H}\right)$ es can\'onicamente conjuntamente medible (\textbf{CCM}) si el mapeo $\left(z,t\right)\in H\times\left[0,\infty\right)$ en $Z_{t}\in E$ es $\mathcal{H}\otimes\mathcal{B}\left[0,\infty\right)/\mathcal{E}$ medible.
\end{Def}

\begin{Note}
Un PEOSCTCCM implica que el proceso es CM, dado que un PECCM $Z$ es un mapeo de $\Omega\times\left[0,\infty\right)$ a $E$, es la composici\'on de dos mapeos medibles: el primero, toma $\left(w,t\right)$ en $\left(Z\left(w\right),t\right)$ es $\mathcal{F}\otimes\mathcal{B}\left[0,\infty\right)/\mathcal{H}\otimes\mathcal{B}\left[0,\infty\right)$ medible, y el segundo que toma $\left(Z\left(w\right),t\right)$  en $Z_{t}\left(w\right)$ es $\mathcal{H}\otimes\mathcal{B}\left[0,\infty\right)/\mathcal{E}$ medible. Por tanto CCM es una condici\'on m\'as fuerte que CM.
\end{Note}

\begin{Def}
Un conjunto de trayectorias $H$ de un PEOSCT $Z$, es internamente shift-invariante (\textbf{ISI}) si 
\begin{eqnarray*}
\left\{\left(z_{t+s}\right)_{s\in\left[0,\infty\right)}:z\in H\right\}=H\textrm{, }t\in\left[0,\infty\right).
\end{eqnarray*}
\end{Def}


\begin{Def}
Dado un PEOSCTISI, se define el mapeo-shift $\theta_{t}$, $t\in\left[0,\infty\right)$, de $H$ a $H$ por 
\begin{eqnarray*}
\theta_{t}z=\left(z_{t+s}\right)_{s\in\left[0,\infty\right)}\textrm{, }z\in H.
\end{eqnarray*}
\end{Def}

\begin{Def}
Se dice que un proceso $Z$ es shift-medible (\textbf{SM}) si $Z$ tiene un conjunto de trayectorias $H$ que es ISI y adem\'as el mapeo que toma $\left(z,t\right)\in H\times\left[0,\infty\right)$ en $\theta_{t}z\in H$ es $\mathcal{H}\otimes\mathcal{B}\left[0,\infty\right)/\mathcal{H}$ medible.
\end{Def}

\begin{Note}
Un proceso estoc\'astico con conjunto de trayectorias $H$ ISI es shift-medible si y s\'olo si es CCM
\end{Note}

\begin{Note}
\begin{itemize}
\item Dado el espacio polaco $\left(E,\mathcal{E}\right)$ se tiene el  conjunto de trayectorias $D_{E}\left[0,\infty\right)$ que es ISI, entonces cumpe con ser CCM.

\item Si $G$ es abierto, podemos cubrirlo por bolas abiertas cuay cerradura este contenida en $G$, y como $G$ es segundo numerable como subespacio de $E$, lo podemos cubrir por una cantidad numerable de bolas abiertas.

\end{itemize}
\end{Note}


\begin{Note}
Los procesos estoc\'asticos $Z$ a tiempo discreto con espacio de estados polaco, tambi\'en tiene un espacio de trayectorias polaco y por tanto tiene distribuciones condicionales regulares.
\end{Note}

\begin{Teo}
El producto numerable de espacios polacos es polaco.
\end{Teo}


\begin{Def}
Sea $\left(\Omega,\mathcal{F},\prob\right)$ espacio de probabilidad que soporta al proceso $Z=\left(Z_{s}\right)_{s\in\left[0,\infty\right)}$ y $S=\left(S_{k}\right)_{0}^{\infty}$ donde $Z$ es un PEOSCTM con espacio de estados $\left(E,\mathcal{E}\right)$  y espacio de trayectorias $\left(H,\mathcal{H}\right)$  y adem\'as $S$ es una sucesi\'on de tiempos aleatorios one-sided que satisfacen la condici\'on $0\leq S_{0}<S_{1}<\cdots\rightarrow\infty$. Considerando $S$ como un mapeo medible de $\left(\Omega,\mathcal{F}\right)$ al espacio sucesi\'on $\left(L,\mathcal{L}\right)$, donde 
\begin{eqnarray*}
L=\left\{\left(s_{k}\right)_{0}^{\infty}\in\left[0,\infty\right)^{\left\{0,1,\ldots\right\}}:s_{0}<s_{1}<\cdots\rightarrow\infty\right\},
\end{eqnarray*}
donde $\mathcal{L}$ son los subconjuntos de Borel de $L$, es decir, $\mathcal{L}=L\cap\mathcal{B}^{\left\{0,1,\ldots\right\}}$.

As\'i el par $\left(Z,S\right)$ es un mapeo medible de  $\left(\Omega,\mathcal{F}\right)$ en $\left(H\times L,\mathcal{H}\otimes\mathcal{L}\right)$. El par $\mathcal{H}\otimes\mathcal{L}^{+}$ denotar\'a la clase de todas las funciones medibles de $\left(H\times L,\mathcal{H}\otimes\mathcal{L}\right)$ en $\left(\left[0,\infty\right),\mathcal{B}\left[0,\infty\right)\right)$.
\end{Def}


\begin{Def}
Sea $\theta_{t}$ el mapeo-shift conjunto de $H\times L$ en $H\times L$ dado por
\begin{eqnarray*}
\theta_{t}\left(z,\left(s_{k}\right)_{0}^{\infty}\right)=\theta_{t}\left(z,\left(s_{n_{t-}+k}-t\right)_{0}^{\infty}\right)
\end{eqnarray*}
donde 
$n_{t-}=inf\left\{n\geq1:s_{n}\geq t\right\}$.
\end{Def}

\begin{Note}
Con la finalidad de poder realizar los shift's sin complicaciones de medibilidad, se supondr\'a que $Z$ es shit-medible, es decir, el conjunto de trayectorias $H$ es invariante bajo shifts del tiempo y el mapeo que toma $\left(z,t\right)\in H\times\left[0,\infty\right)$ en $z_{t}\in E$ es $\mathcal{H}\otimes\mathcal{B}\left[0,\infty\right)/\mathcal{E}$ medible.
\end{Note}

\begin{Def}
Dado un proceso \textbf{PEOSSM} (Proceso Estoc\'astico One Side Shift Medible) $Z$, se dice regenerativo cl\'asico con tiempos de regeneraci\'on $S$ si 

\begin{eqnarray*}
\theta_{S_{n}}\left(Z,S\right)=\left(Z^{0},S^{0}\right),n\geq0
\end{eqnarray*}
y adem\'as $\theta_{S_{n}}\left(Z,S\right)$ es independiente de $\left(\left(Z_{s}\right)s\in\left[0,S_{n}\right),S_{0},\ldots,S_{n}\right)$
Si lo anterior se cumple, al par $\left(Z,S\right)$ se le llama regenerativo cl\'asico.
\end{Def}

\begin{Note}
Si el par $\left(Z,S\right)$ es regenerativo cl\'asico, entonces las longitudes de los ciclos $X_{1},X_{2},\ldots,$ son i.i.d. e independientes de la longitud del retraso $S_{0}$, es decir, $S$ es un proceso de renovaci\'on. Las longitudes de los ciclos tambi\'en son llamados tiempos de inter-regeneraci\'on y tiempos de ocurrencia.

\end{Note}

\begin{Teo}
Sup\'ongase que el par $\left(Z,S\right)$ es regenerativo cl\'asico con $\esp\left[X_{1}\right]<\infty$. Entonces $\left(Z^{*},S^{*}\right)$ en el teorema 2.1 es una versi\'on estacionaria de $\left(Z,S\right)$. Adem\'as, si $X_{1}$ es lattice con span $d$, entonces $\left(Z^{**},S^{**}\right)$ en el teorema 2.2 es una versi\'on periodicamente estacionaria de $\left(Z,S\right)$ con periodo $d$.

\end{Teo}

\begin{Def}
Una variable aleatoria $X_{1}$ es \textit{spread out} si existe una $n\geq1$ y una  funci\'on $f\in\mathcal{B}^{+}$ tal que $\int_{\rea}f\left(x\right)dx>0$ con $X_{2},X_{3},\ldots,X_{n}$ copias i.i.d  de $X_{1}$, $$\prob\left(X_{1}+\cdots+X_{n}\in B\right)\geq\int_{B}f\left(x\right)dx$$ para $B\in\mathcal{B}$.

\end{Def}



\begin{Def}
Dado un proceso estoc\'astico $Z$ se le llama \textit{wide-sense regenerative} (\textbf{WSR}) con tiempos de regeneraci\'on $S$ si $\theta_{S_{n}}\left(Z,S\right)=\left(Z^{0},S^{0}\right)$ para $n\geq0$ en distribuci\'on y $\theta_{S_{n}}\left(Z,S\right)$ es independiente de $\left(S_{0},S_{1},\ldots,S_{n}\right)$ para $n\geq0$.
Se dice que el par $\left(Z,S\right)$ es WSR si lo anterior se cumple.
\end{Def}


\begin{Note}
\begin{itemize}
\item El proceso de trayectorias $\left(\theta_{s}Z\right)_{s\in\left[0,\infty\right)}$ es WSR con tiempos de regeneraci\'on $S$ pero no es regenerativo cl\'asico.

\item Si $Z$ es cualquier proceso estacionario y $S$ es un proceso de renovaci\'on que es independiente de $Z$, entonces $\left(Z,S\right)$ es WSR pero en general no es regenerativo cl\'asico

\end{itemize}

\end{Note}


\begin{Note}
Para cualquier proceso estoc\'astico $Z$, el proceso de trayectorias $\left(\theta_{s}Z\right)_{s\in\left[0,\infty\right)}$ es siempre un proceso de Markov.
\end{Note}



\begin{Teo}
Supongase que el par $\left(Z,S\right)$ es WSR con $\esp\left[X_{1}\right]<\infty$. Entonces $\left(Z^{*},S^{*}\right)$ en el teorema 2.1 es una versi\'on estacionaria de 
$\left(Z,S\right)$.
\end{Teo}


\begin{Teo}
Supongase que $\left(Z,S\right)$ es cycle-stationary con $\esp\left[X_{1}\right]<\infty$. Sea $U$ distribuida uniformemente en $\left[0,1\right)$ e independiente de $\left(Z^{0},S^{0}\right)$ y sea $\prob^{*}$ la medida de probabilidad en $\left(\Omega,\prob\right)$ definida por $$d\prob^{*}=\frac{X_{1}}{\esp\left[X_{1}\right]}d\prob$$. Sea $\left(Z^{*},S^{*}\right)$ con distribuci\'on $\prob^{*}\left(\theta_{UX_{1}}\left(Z^{0},S^{0}\right)\in\cdot\right)$. Entonces $\left(Z^{}*,S^{*}\right)$ es estacionario,
\begin{eqnarray*}
\esp\left[f\left(Z^{*},S^{*}\right)\right]=\esp\left[\int_{0}^{X_{1}}f\left(\theta_{s}\left(Z^{0},S^{0}\right)\right)ds\right]/\esp\left[X_{1}\right]
\end{eqnarray*}
$f\in\mathcal{H}\otimes\mathcal{L}^{+}$, and $S_{0}^{*}$ es continuo con funci\'on distribuci\'on $G_{\infty}$ definida por $$G_{\infty}\left(x\right):=\frac{\esp\left[X_{1}\right]\wedge x}{\esp\left[X_{1}\right]}$$ para $x\geq0$ y densidad $\prob\left[X_{1}>x\right]/\esp\left[X_{1}\right]$, con $x\geq0$.

\end{Teo}


\begin{Teo}
Sea $Z$ un Proceso Estoc\'astico un lado shift-medible \textit{one-sided shift-measurable stochastic process}, (PEOSSM),
y $S_{0}$ y $S_{1}$ tiempos aleatorios tales que $0\leq S_{0}<S_{1}$ y
\begin{equation}
\theta_{S_{1}}Z=\theta_{S_{0}}Z\textrm{ en distribuci\'on}.
\end{equation}

Entonces el espacio de probabilidad subyacente $\left(\Omega,\mathcal{F},\prob\right)$ puede extenderse para soportar una sucesi\'on de tiempos aleatorios $S$ tales que

\begin{eqnarray}
\theta_{S_{n}}\left(Z,S\right)=\left(Z^{0},S^{0}\right),n\geq0,\textrm{ en distribuci\'on},\\
\left(Z,S_{0},S_{1}\right)\textrm{ depende de }\left(X_{2},X_{3},\ldots\right)\textrm{ solamente a traves de }\theta_{S_{1}}Z.
\end{eqnarray}
\end{Teo}





%_________________________________________________________________________
%
%\subsection{Una vez que se tiene estabilidad}
%_________________________________________________________________________
%

Also the intervisit time $I_{i}$ is defined as the period beginning at the time of its service completion in a cycle and ending at the time when it is polled in the next cycle; its duration is given by $\tau_{i}\left(m+1\right)-\overline{\tau}_{i}\left(m\right)$.

So we the following are still true 

\begin{eqnarray}
\begin{array}{ll}
\esp\left[L_{i}\right]=\mu_{i}\esp\left[I_{i}\right], &
\esp\left[C_{i}\right]=\frac{f_{i}\left(i\right)}{\mu_{i}\left(1-\mu_{i}\right)},\\
\esp\left[S_{i}\right]=\mu_{i}\esp\left[C_{i}\right],&
\esp\left[I_{i}\right]=\left(1-\mu_{i}\right)\esp\left[C_{i}\right],\\
Var\left[L_{i}\right]= \mu_{i}^{2}Var\left[I_{i}\right]+\sigma^{2}\esp\left[I_{i}\right],& 
Var\left[C_{i}\right]=\frac{Var\left[L_{i}^{*}\right]}{\mu_{i}^{2}\left(1-\mu_{i}\right)^{2}},\\
Var\left[S_{i}\right]= \frac{Var\left[L_{i}^{*}\right]}{\left(1-\mu_{i}\right)^{2}}+\frac{\sigma^{2}\esp\left[L_{i}^{*}\right]}{\left(1-\mu_{i}\right)^{3}},&
Var\left[I_{i}\right]= \frac{Var\left[L_{i}^{*}\right]}{\mu_{i}^{2}}-\frac{\sigma_{i}^{2}}{\mu_{i}^{2}}f_{i}\left(i\right).
\end{array}
\end{eqnarray}
\begin{Def}
El tiempo de Ciclo $C_{i}$ es el periodo de tiempo que comienza cuando la cola $i$ es visitada por primera vez en un ciclo, y termina cuando es visitado nuevamente en el pr\'oximo ciclo. La duraci\'on del mismo est\'a dada por $\tau_{i}\left(m+1\right)-\tau_{i}\left(m\right)$, o equivalentemente $\overline{\tau}_{i}\left(m+1\right)-\overline{\tau}_{i}\left(m\right)$ bajo condiciones de estabilidad.
\end{Def}


\begin{Def}
El tiempo de intervisita $I_{i}$ es el periodo de tiempo que comienza cuando se ha completado el servicio en un ciclo y termina cuando es visitada nuevamente en el pr\'oximo ciclo. Su  duraci\'on del mismo est\'a dada por $\tau_{i}\left(m+1\right)-\overline{\tau}_{i}\left(m\right)$.
\end{Def}

La duraci\'on del tiempo de intervisita es $\tau_{i}\left(m+1\right)-\overline{\tau}\left(m\right)$. Dado que el n\'umero de usuarios presentes en $Q_{i}$ al tiempo $t=\tau_{i}\left(m+1\right)$ es igual al n\'umero de arribos durante el intervalo de tiempo $\left[\overline{\tau}\left(m\right),\tau_{i}\left(m+1\right)\right]$ se tiene que


\begin{eqnarray*}
\esp\left[z_{i}^{L_{i}\left(\tau_{i}\left(m+1\right)\right)}\right]=\esp\left[\left\{P_{i}\left(z_{i}\right)\right\}^{\tau_{i}\left(m+1\right)-\overline{\tau}\left(m\right)}\right]
\end{eqnarray*}

entonces, si $I_{i}\left(z\right)=\esp\left[z^{\tau_{i}\left(m+1\right)-\overline{\tau}\left(m\right)}\right]$
se tiene que $F_{i}\left(z\right)=I_{i}\left[P_{i}\left(z\right)\right]$
para $i=1,2$.

Conforme a la definici\'on dada al principio del cap\'itulo, definici\'on (\ref{Def.Tn}), sean $T_{1},T_{2},\ldots$ los puntos donde las longitudes de las colas de la red de sistemas de visitas c\'iclicas son cero simult\'aneamente, cuando la cola $Q_{j}$ es visitada por el servidor para dar servicio, es decir, $L_{1}\left(T_{i}\right)=0,L_{2}\left(T_{i}\right)=0,\hat{L}_{1}\left(T_{i}\right)=0$ y $\hat{L}_{2}\left(T_{i}\right)=0$, a estos puntos se les denominar\'a puntos regenerativos. Entonces, 

\begin{Def}
Al intervalo de tiempo entre dos puntos regenerativos se le llamar\'a ciclo regenerativo.
\end{Def}

\begin{Def}
Para $T_{i}$ se define, $M_{i}$, el n\'umero de ciclos de visita a la cola $Q_{l}$, durante el ciclo regenerativo, es decir, $M_{i}$ es un proceso de renovaci\'on.
\end{Def}

\begin{Def}
Para cada uno de los $M_{i}$'s, se definen a su vez la duraci\'on de cada uno de estos ciclos de visita en el ciclo regenerativo, $C_{i}^{(m)}$, para $m=1,2,\ldots,M_{i}$, que a su vez, tambi\'en es n proceso de renovaci\'on.
\end{Def}


Sea la funci\'on generadora de momentos para $L_{i}$, el n\'umero de usuarios en la cola $Q_{i}\left(z\right)$ en cualquier momento, est\'a dada por el tiempo promedio de $z^{L_{i}\left(t\right)}$ sobre el ciclo regenerativo definido anteriormente:

\begin{eqnarray*}
Q_{i}\left(z\right)&=&\esp\left[z^{L_{i}\left(t\right)}\right]=\frac{\esp\left[\sum_{m=1}^{M_{i}}\sum_{t=\tau_{i}\left(m\right)}^{\tau_{i}\left(m+1\right)-1}z^{L_{i}\left(t\right)}\right]}{\esp\left[\sum_{m=1}^{M_{i}}\tau_{i}\left(m+1\right)-\tau_{i}\left(m\right)\right]}
\end{eqnarray*}

$M_{i}$ es un tiempo de paro en el proceso regenerativo con $\esp\left[M_{i}\right]<\infty$\footnote{En Stidham\cite{Stidham} y Heyman  se muestra que una condici\'on suficiente para que el proceso regenerativo 
estacionario sea un procesoo estacionario es que el valor esperado del tiempo del ciclo regenerativo sea finito, es decir: $\esp\left[\sum_{m=1}^{M_{i}}C_{i}^{(m)}\right]<\infty$, como cada $C_{i}^{(m)}$ contiene intervalos de r\'eplica positivos, se tiene que $\esp\left[M_{i}\right]<\infty$, adem\'as, como $M_{i}>0$, se tiene que la condici\'on anterior es equivalente a tener que $\esp\left[C_{i}\right]<\infty$,
por lo tanto una condici\'on suficiente para la existencia del proceso regenerativo est\'a dada por $\sum_{k=1}^{N}\mu_{k}<1.$}, se sigue del lema de Wald que:


\begin{eqnarray*}
\esp\left[\sum_{m=1}^{M_{i}}\sum_{t=\tau_{i}\left(m\right)}^{\tau_{i}\left(m+1\right)-1}z^{L_{i}\left(t\right)}\right]&=&\esp\left[M_{i}\right]\esp\left[\sum_{t=\tau_{i}\left(m\right)}^{\tau_{i}\left(m+1\right)-1}z^{L_{i}\left(t\right)}\right]\\
\esp\left[\sum_{m=1}^{M_{i}}\tau_{i}\left(m+1\right)-\tau_{i}\left(m\right)\right]&=&\esp\left[M_{i}\right]\esp\left[\tau_{i}\left(m+1\right)-\tau_{i}\left(m\right)\right]
\end{eqnarray*}

por tanto se tiene que


\begin{eqnarray*}
Q_{i}\left(z\right)&=&\frac{\esp\left[\sum_{t=\tau_{i}\left(m\right)}^{\tau_{i}\left(m+1\right)-1}z^{L_{i}\left(t\right)}\right]}{\esp\left[\tau_{i}\left(m+1\right)-\tau_{i}\left(m\right)\right]}
\end{eqnarray*}

observar que el denominador es simplemente la duraci\'on promedio del tiempo del ciclo.


Haciendo las siguientes sustituciones en la ecuaci\'on (\ref{Corolario2}): $n\rightarrow t-\tau_{i}\left(m\right)$, $T \rightarrow \overline{\tau}_{i}\left(m\right)-\tau_{i}\left(m\right)$, $L_{n}\rightarrow L_{i}\left(t\right)$ y $F\left(z\right)=\esp\left[z^{L_{0}}\right]\rightarrow F_{i}\left(z\right)=\esp\left[z^{L_{i}\tau_{i}\left(m\right)}\right]$, se puede ver que

\begin{eqnarray}\label{Eq.Arribos.Primera}
\esp\left[\sum_{n=0}^{T-1}z^{L_{n}}\right]=
\esp\left[\sum_{t=\tau_{i}\left(m\right)}^{\overline{\tau}_{i}\left(m\right)-1}z^{L_{i}\left(t\right)}\right]
=z\frac{F_{i}\left(z\right)-1}{z-P_{i}\left(z\right)}
\end{eqnarray}

Por otra parte durante el tiempo de intervisita para la cola $i$, $L_{i}\left(t\right)$ solamente se incrementa de manera que el incremento por intervalo de tiempo est\'a dado por la funci\'on generadora de probabilidades de $P_{i}\left(z\right)$, por tanto la suma sobre el tiempo de intervisita puede evaluarse como:

\begin{eqnarray*}
\esp\left[\sum_{t=\tau_{i}\left(m\right)}^{\tau_{i}\left(m+1\right)-1}z^{L_{i}\left(t\right)}\right]&=&\esp\left[\sum_{t=\tau_{i}\left(m\right)}^{\tau_{i}\left(m+1\right)-1}\left\{P_{i}\left(z\right)\right\}^{t-\overline{\tau}_{i}\left(m\right)}\right]=\frac{1-\esp\left[\left\{P_{i}\left(z\right)\right\}^{\tau_{i}\left(m+1\right)-\overline{\tau}_{i}\left(m\right)}\right]}{1-P_{i}\left(z\right)}\\
&=&\frac{1-I_{i}\left[P_{i}\left(z\right)\right]}{1-P_{i}\left(z\right)}
\end{eqnarray*}
por tanto

\begin{eqnarray*}
\esp\left[\sum_{t=\tau_{i}\left(m\right)}^{\tau_{i}\left(m+1\right)-1}z^{L_{i}\left(t\right)}\right]&=&
\frac{1-F_{i}\left(z\right)}{1-P_{i}\left(z\right)}
\end{eqnarray*}

Por lo tanto

\begin{eqnarray*}
Q_{i}\left(z\right)&=&\frac{\esp\left[\sum_{t=\tau_{i}\left(m\right)}^{\tau_{i}\left(m+1\right)-1}z^{L_{i}\left(t\right)}\right]}{\esp\left[\tau_{i}\left(m+1\right)-\tau_{i}\left(m\right)\right]}
=\frac{1}{\esp\left[\tau_{i}\left(m+1\right)-\tau_{i}\left(m\right)\right]}
\esp\left[\sum_{t=\tau_{i}\left(m\right)}^{\tau_{i}\left(m+1\right)-1}z^{L_{i}\left(t\right)}\right]\\
&=&\frac{1}{\esp\left[\tau_{i}\left(m+1\right)-\tau_{i}\left(m\right)\right]}
\esp\left[\sum_{t=\tau_{i}\left(m\right)}^{\overline{\tau}_{i}\left(m\right)-1}z^{L_{i}\left(t\right)}
+\sum_{t=\overline{\tau}_{i}\left(m\right)}^{\tau_{i}\left(m+1\right)-1}z^{L_{i}\left(t\right)}\right]\\
&=&\frac{1}{\esp\left[\tau_{i}\left(m+1\right)-\tau_{i}\left(m\right)\right]}\left\{
\esp\left[\sum_{t=\tau_{i}\left(m\right)}^{\overline{\tau}_{i}\left(m\right)-1}z^{L_{i}\left(t\right)}\right]
+\esp\left[\sum_{t=\overline{\tau}_{i}\left(m\right)}^{\tau_{i}\left(m+1\right)-1}z^{L_{i}\left(t\right)}\right]\right\}\\
&=&\frac{1}{\esp\left[\tau_{i}\left(m+1\right)-\tau_{i}\left(m\right)\right]}\left\{
z\frac{F_{i}\left(z\right)-1}{z-P_{i}\left(z\right)}+\frac{1-F_{i}\left(z\right)}{1-P_{i}\left(z\right)}
\right\}\\
&=&\frac{1}{\esp\left[C_{i}\right]}\cdot\frac{1-F_{i}\left(z\right)}{P_{i}\left(z\right)-z}\cdot\frac{\left(1-z\right)P_{i}\left(z\right)}{1-P_{i}\left(z\right)}
\end{eqnarray*}

es decir

\begin{equation}
Q_{i}\left(z\right)=\frac{1}{\esp\left[C_{i}\right]}\cdot\frac{1-F_{i}\left(z\right)}{P_{i}\left(z\right)-z}\cdot\frac{\left(1-z\right)P_{i}\left(z\right)}{1-P_{i}\left(z\right)}
\end{equation}


Si hacemos:

\begin{eqnarray}
S\left(z\right)&=&1-F\left(z\right)\\
T\left(z\right)&=&z-P\left(z\right)\\
U\left(z\right)&=&1-P\left(z\right)
\end{eqnarray}
entonces 

\begin{eqnarray}
\esp\left[C_{i}\right]Q\left(z\right)=\frac{\left(z-1\right)S\left(z\right)P\left(z\right)}{T\left(z\right)U\left(z\right)}
\end{eqnarray}

A saber, si $a_{k}=P\left\{L\left(t\right)=k\right\}$
\begin{eqnarray*}
S\left(z\right)=1-F\left(z\right)=1-\sum_{k=0}^{+\infty}a_{k}z^{k}
\end{eqnarray*}
entonces

%\begin{eqnarray}
%\begin{array}{ll}
%S^{'}\left(z\right)=-\sum_{k=1}^{+\infty}ka_{k}z^{k-1},& %S^{(1)}\left(1\right)=-\sum_{k=1}^{+\infty}ka_{k}=-\esp\left[L\left(t\right)\right],\\
%S^{''}\left(z\right)=-\sum_{k=2}^{+\infty}k(k-1)a_{k}z^{k-2},& S^{(2)}\left(1\right)=-\sum_{k=2}^{+\infty}k(k-1)a_{k}=\esp\left[L\left(L-1\right)\right],\\
%S^{'''}\left(z\right)=-\sum_{k=3}^{+\infty}k(k-1)(k-2)a_{k}z^{k-3},&
%S^{(3)}\left(1\right)=-\sum_{k=3}^{+\infty}k(k-1)(k-2)a_{k}\\
%&=-\esp\left[L\left(L-1\right)\left(L-2\right)\right]\\
%&=-\esp\left[L^{3}\right]+3-\esp\left[L^{2}\right]-2-\esp\left[L\right];
%\end{array}
%\end{eqnarray}

$S^{'}\left(z\right)=-\sum_{k=1}^{+\infty}ka_{k}z^{k-1}$, por tanto $S^{(1)}\left(1\right)=-\sum_{k=1}^{+\infty}ka_{k}=-\esp\left[L\left(t\right)\right]$,
luego $S^{''}\left(z\right)=-\sum_{k=2}^{+\infty}k(k-1)a_{k}z^{k-2}$ y $S^{(2)}\left(1\right)=-\sum_{k=2}^{+\infty}k(k-1)a_{k}=\esp\left[L\left(L-1\right)\right]$;
de la misma manera $S^{'''}\left(z\right)=-\sum_{k=3}^{+\infty}k(k-1)(k-2)a_{k}z^{k-3}$ y $S^{(3)}\left(1\right)=-\sum_{k=3}^{+\infty}k(k-1)(k-2)a_{k}=-\esp\left[L\left(L-1\right)\left(L-2\right)\right]
=-\esp\left[L^{3}\right]+3-\esp\left[L^{2}\right]-2-\esp\left[L\right]$. 

Es decir

\begin{eqnarray*}
S^{(1)}\left(1\right)&=&-\esp\left[L\left(t\right)\right],\\ S^{(2)}\left(1\right)&=&-\esp\left[L\left(L-1\right)\right]
=-\esp\left[L^{2}\right]+\esp\left[L\right],\\
S^{(3)}\left(1\right)&=&-\esp\left[L\left(L-1\right)\left(L-2\right)\right]
=-\esp\left[L^{3}\right]+3\esp\left[L^{2}\right]-2\esp\left[L\right].
\end{eqnarray*}


Expandiendo alrededor de $z=1$

\begin{eqnarray*}
S\left(z\right)&=&S\left(1\right)+\frac{S^{'}\left(1\right)}{1!}\left(z-1\right)+\frac{S^{''}\left(1\right)}{2!}\left(z-1\right)^{2}+\frac{S^{'''}\left(1\right)}{3!}\left(z-1\right)^{3}+\ldots+\\
&=&\left(z-1\right)\left\{S^{'}\left(1\right)+\frac{S^{''}\left(1\right)}{2!}\left(z-1\right)+\frac{S^{'''}\left(1\right)}{3!}\left(z-1\right)^{2}+\ldots+\right\}\\
&=&\left(z-1\right)R_{1}\left(z\right)
\end{eqnarray*}
con $R_{1}\left(z\right)\neq0$, pues

\begin{eqnarray}\label{Eq.R1}
R_{1}\left(z\right)=-\esp\left[L\right]
\end{eqnarray}
entonces

\begin{eqnarray}
R_{1}\left(z\right)&=&S^{'}\left(1\right)+\frac{S^{''}\left(1\right)}{2!}\left(z-1\right)+\frac{S^{'''}\left(1\right)}{3!}\left(z-1\right)^{2}+\frac{S^{iv}\left(1\right)}{4!}\left(z-1\right)^{3}+\ldots+
\end{eqnarray}
Calculando las derivadas y evaluando en $z=1$

\begin{eqnarray}
R_{1}\left(1\right)&=&S^{(1)}\left(1\right)=-\esp\left[L\right]\\
R_{1}^{(1)}\left(1\right)&=&\frac{1}{2}S^{(2)}\left(1\right)=-\frac{1}{2}\esp\left[L^{2}\right]+\frac{1}{2}\esp\left[L\right]\\
R_{1}^{(2)}\left(1\right)&=&\frac{2}{3!}S^{(3)}\left(1\right)
=-\frac{1}{3}\esp\left[L^{3}\right]+\esp\left[L^{2}\right]-\frac{2}{3}\esp\left[L\right]
\end{eqnarray}

De manera an\'aloga se puede ver que para $T\left(z\right)=z-P\left(z\right)$ se puede encontrar una expanci\'on alrededor de $z=1$

Expandiendo alrededor de $z=1$

\begin{eqnarray*}
T\left(z\right)&=&T\left(1\right)+\frac{T^{'}\left(1\right)}{1!}\left(z-1\right)+\frac{T^{''}\left(1\right)}{2!}\left(z-1\right)^{2}+\frac{T^{'''}\left(1\right)}{3!}\left(z-1\right)^{3}+\ldots+\\
&=&\left(z-1\right)\left\{T^{'}\left(1\right)+\frac{T^{''}\left(1\right)}{2!}\left(z-1\right)+\frac{T^{'''}\left(1\right)}{3!}\left(z-1\right)^{2}+\ldots+\right\}\\
&=&\left(z-1\right)R_{2}\left(z\right)
\end{eqnarray*}

donde 
\begin{eqnarray*}
T^{(1)}\left(1\right)&=&-\esp\left[X\left(t\right)\right]=-\mu,\\ T^{(2)}\left(1\right)&=&-\esp\left[X\left(X-1\right)\right]
=-\esp\left[X^{2}\right]+\esp\left[X\right]=-\esp\left[X^{2}\right]+\mu,\\
T^{(3)}\left(1\right)&=&-\esp\left[X\left(X-1\right)\left(X-2\right)\right]
=-\esp\left[X^{3}\right]+3\esp\left[X^{2}\right]-2\esp\left[X\right]\\
&=&-\esp\left[X^{3}\right]+3\esp\left[X^{2}\right]-2\mu.
\end{eqnarray*}

Por lo tanto $R_{2}\left(1\right)\neq0$, pues

\begin{eqnarray}\label{Eq.R2}
R_{2}\left(1\right)=1-\esp\left[X\right]=1-\mu
\end{eqnarray}
entonces

\begin{eqnarray}
R_{2}\left(z\right)&=&T^{'}\left(1\right)+\frac{T^{''}\left(1\right)}{2!}\left(z-1\right)+\frac{T^{'''}\left(1\right)}{3!}\left(z-1\right)^{2}+\frac{T^{(iv)}\left(1\right)}{4!}\left(z-1\right)^{3}+\ldots+
\end{eqnarray}
Calculando las derivadas y evaluando en $z=1$

\begin{eqnarray}
R_{2}\left(1\right)&=&T^{(1)}\left(1\right)=1-\mu\\
R_{2}^{(1)}\left(1\right)&=&\frac{1}{2}T^{(2)}\left(1\right)=-\frac{1}{2}\esp\left[X^{2}\right]+\frac{1}{2}\mu\\
R_{2}^{(2)}\left(1\right)&=&\frac{2}{3!}T^{(3)}\left(1\right)
=-\frac{1}{3}\esp\left[X^{3}\right]+\esp\left[X^{2}\right]-\frac{2}{3}\mu
\end{eqnarray}

Finalmente para de manera an\'aloga se puede ver que para $U\left(z\right)=1-P\left(z\right)$ se puede encontrar una expanci\'on alrededor de $z=1$

\begin{eqnarray*}
U\left(z\right)&=&U\left(1\right)+\frac{U^{'}\left(1\right)}{1!}\left(z-1\right)+\frac{U^{''}\left(1\right)}{2!}\left(z-1\right)^{2}+\frac{U^{'''}\left(1\right)}{3!}\left(z-1\right)^{3}+\ldots+\\
&=&\left(z-1\right)\left\{U^{'}\left(1\right)+\frac{U^{''}\left(1\right)}{2!}\left(z-1\right)+\frac{U^{'''}\left(1\right)}{3!}\left(z-1\right)^{2}+\ldots+\right\}\\
&=&\left(z-1\right)R_{3}\left(z\right)
\end{eqnarray*}

donde 
\begin{eqnarray*}
U^{(1)}\left(1\right)&=&-\esp\left[X\left(t\right)\right]=-\mu,\\ U^{(2)}\left(1\right)&=&-\esp\left[X\left(X-1\right)\right]
=-\esp\left[X^{2}\right]+\esp\left[X\right]=-\esp\left[X^{2}\right]+\mu,\\
U^{(3)}\left(1\right)&=&-\esp\left[X\left(X-1\right)\left(X-2\right)\right]
=-\esp\left[X^{3}\right]+3\esp\left[X^{2}\right]-2\esp\left[X\right]\\
&=&-\esp\left[X^{3}\right]+3\esp\left[X^{2}\right]-2\mu.
\end{eqnarray*}

Por lo tanto $R_{3}\left(1\right)\neq0$, pues

\begin{eqnarray}\label{Eq.R2}
R_{3}\left(1\right)=-\esp\left[X\right]=-\mu
\end{eqnarray}
entonces

\begin{eqnarray}
R_{3}\left(z\right)&=&U^{'}\left(1\right)+\frac{U^{''}\left(1\right)}{2!}\left(z-1\right)+\frac{U^{'''}\left(1\right)}{3!}\left(z-1\right)^{2}+\frac{U^{(iv)}\left(1\right)}{4!}\left(z-1\right)^{3}+\ldots+
\end{eqnarray}

Calculando las derivadas y evaluando en $z=1$

\begin{eqnarray}
R_{3}\left(1\right)&=&U^{(1)}\left(1\right)=-\mu\\
R_{3}^{(1)}\left(1\right)&=&\frac{1}{2}U^{(2)}\left(1\right)=-\frac{1}{2}\esp\left[X^{2}\right]+\frac{1}{2}\mu\\
R_{3}^{(2)}\left(1\right)&=&\frac{2}{3!}U^{(3)}\left(1\right)
=-\frac{1}{3}\esp\left[X^{3}\right]+\esp\left[X^{2}\right]-\frac{2}{3}\mu
\end{eqnarray}

Por lo tanto

\begin{eqnarray}
\esp\left[C_{i}\right]Q\left(z\right)&=&\frac{\left(z-1\right)\left(z-1\right)R_{1}\left(z\right)P\left(z\right)}{\left(z-1\right)R_{2}\left(z\right)\left(z-1\right)R_{3}\left(z\right)}
=\frac{R_{1}\left(z\right)P\left(z\right)}{R_{2}\left(z\right)R_{3}\left(z\right)}\equiv\frac{R_{1}P}{R_{2}R_{3}}
\end{eqnarray}

Entonces

\begin{eqnarray}\label{Eq.Primer.Derivada.Q}
\left[\frac{R_{1}\left(z\right)P\left(z\right)}{R_{2}\left(z\right)R_{3}\left(z\right)}\right]^{'}&=&\frac{PR_{2}R_{3}R_{1}^{'}
+R_{1}R_{2}R_{3}P^{'}-R_{3}R_{1}PR_{2}-R_{2}R_{1}PR_{3}^{'}}{\left(R_{2}R_{3}\right)^{2}}
\end{eqnarray}
Evaluando en $z=1$
\begin{eqnarray*}
&=&\frac{R_{2}(1)R_{3}(1)R_{1}^{(1)}(1)+R_{1}(1)R_{2}(1)R_{3}(1)P^{'}(1)-R_{3}(1)R_{1}(1)R_{2}(1)^{(1)}-R_{2}(1)R_{1}(1)R_{3}^{'}(1)}{\left(R_{2}(1)R_{3}(1)\right)^{2}}\\
&=&\frac{1}{\left(1-\mu\right)^{2}\mu^{2}}\left\{\left(-\frac{1}{2}\esp L^{2}+\frac{1}{2}\esp L\right)\left(1-\mu\right)\left(-\mu\right)+\left(-\esp L\right)\left(1-\mu\right)\left(-\mu\right)\mu\right.\\
&&\left.-\left(-\frac{1}{2}\esp X^{2}+\frac{1}{2}\mu\right)\left(-\mu\right)\left(-\esp L\right)-\left(1-\mu\right)\left(-\esp L\right)\left(-\frac{1}{2}\esp X^{2}+\frac{1}{2}\mu\right)\right\}\\
&=&\frac{1}{\left(1-\mu\right)^{2}\mu^{2}}\left\{\left(-\frac{1}{2}\esp L^{2}+\frac{1}{2}\esp L\right)\left(\mu^{2}-\mu\right)
+\left(\mu^{2}-\mu^{3}\right)\esp L\right.\\
&&\left.-\mu\esp L\left(-\frac{1}{2}\esp X^{2}+\frac{1}{2}\mu\right)
+\left(\esp L-\mu\esp L\right)\left(-\frac{1}{2}\esp X^{2}+\frac{1}{2}\mu\right)\right\}\\
&=&\frac{1}{\left(1-\mu\right)^{2}\mu^{2}}\left\{-\frac{1}{2}\mu^{2}\esp L^{2}
+\frac{1}{2}\mu\esp L^{2}
+\frac{1}{2}\mu^{2}\esp L
-\mu^{3}\esp L
+\mu\esp L\esp X^{2}
-\frac{1}{2}\esp L\esp X^{2}\right\}\\
&=&\frac{1}{\left(1-\mu\right)^{2}\mu^{2}}\left\{
\frac{1}{2}\mu\esp L^{2}\left(1-\mu\right)
+\esp L\left(\frac{1}{2}-\mu\right)\left(\mu^{2}-\esp X^{2}\right)\right\}\\
&=&\frac{1}{2\mu\left(1-\mu\right)}\esp L^{2}-\frac{\frac{1}{2}-\mu}{\left(1-\mu\right)^{2}\mu^{2}}\sigma^{2}\esp L
\end{eqnarray*}

por lo tanto (para Takagi)

\begin{eqnarray*}
Q^{(1)}=\frac{1}{\esp C}\left\{\frac{1}{2\mu\left(1-\mu\right)}\esp L^{2}-\frac{\frac{1}{2}-\mu}{\left(1-\mu\right)^{2}\mu^{2}}\sigma^{2}\esp L\right\}
\end{eqnarray*}
donde 

\begin{eqnarray*}
\esp C = \frac{\esp L}{\mu\left(1-\mu\right)}
\end{eqnarray*}
entonces

\begin{eqnarray*}
Q^{(1)}&=&\frac{1}{2}\frac{\esp L^{2}}{\esp L}-\frac{\frac{1}{2}-\mu}{\left(1-\mu\right)\mu}\sigma^{2}
=\frac{\esp L^{2}}{2\esp L}-\frac{\sigma^{2}}{2}\left\{\frac{2\mu-1}{\left(1-\mu\right)\mu}\right\}\\
&=&\frac{\esp L^{2}}{2\esp L}+\frac{\sigma^{2}}{2}\left\{\frac{1}{1-\mu}+\frac{1}{\mu}\right\}
\end{eqnarray*}

Mientras que para nosotros

\begin{eqnarray*}
Q^{(1)}=\frac{1}{\mu\left(1-\mu\right)}\frac{\esp L^{2}}{2\esp C}
-\sigma^{2}\frac{\esp L}{2\esp C}\cdot\frac{1-2\mu}{\left(1-\mu\right)^{2}\mu^{2}}
\end{eqnarray*}

Retomando la ecuaci\'on (\ref{Eq.Primer.Derivada.Q})

\begin{eqnarray*}
\left[\frac{R_{1}\left(z\right)P\left(z\right)}{R_{2}\left(z\right)R_{3}\left(z\right)}\right]^{'}&=&\frac{PR_{2}R_{3}R_{1}^{'}
+R_{1}R_{2}R_{3}P^{'}-R_{3}R_{1}PR_{2}-R_{2}R_{1}PR_{3}^{'}}{\left(R_{2}R_{3}\right)^{2}}
=\frac{F\left(z\right)}{G\left(z\right)}
\end{eqnarray*}

donde 

\begin{eqnarray*}
F\left(z\right)&=&PR_{2}R_{3}R_{1}^{'}
+R_{1}R_{2}R_{3}P^{'}-R_{3}R_{1}PR_{2}^{'}-R_{2}R_{1}PR_{3}^{'}\\
G\left(z\right)&=&R_{2}^{2}R_{3}^{2}\\
G^{2}\left(z\right)&=&R_{2}^{4}R_{3}^{4}=\left(1-\mu\right)^{4}\mu^{4}
\end{eqnarray*}
y por tanto

\begin{eqnarray*}
G^{'}\left(z\right)&=&2R_{2}R_{3}\left[R_{2}^{'}R_{3}+R_{2}R_{3}^{'}\right]\\
G^{'}\left(1\right)&=&-2\left(1-\mu\right)\mu\left[\left(-\frac{1}{2}\esp\left[X^{2}\right]+\frac{1}{2}\mu\right)\left(-\mu\right)+\left(1-\mu\right)\left(-\frac{1}{2}\esp\left[X^{2}\right]+\frac{1}{2}\mu\right)\right]
\end{eqnarray*}


\begin{eqnarray*}
F^{'}\left(z\right)&=&\left[\left(R_{2}R_{3}\right)R_{1}^{''}
-\left(R_{1}R_{3}\right)R_{2}^{''}
-\left(R_{1}R_{2}\right)R_{3}^{''}
-2\left(R_{2}^{'}R_{3}^{'}\right)R_{1}\right]P
+2\left(R_{2}R_{3}\right)R_{1}^{'}P^{'}
+\left(R_{1}R_{2}R_{3}\right)P^{''}
\end{eqnarray*}

Por lo tanto, encontremos $F^{'}\left(z\right)G\left(z\right)+F\left(z\right)G^{'}\left(z\right)$:

\begin{eqnarray*}
&&F^{'}\left(z\right)G\left(z\right)+F\left(z\right)G^{'}\left(z\right)=
\left\{\left[\left(R_{2}R_{3}\right)R_{1}^{''}
-\left(R_{1}R_{3}\right)R_{2}^{''}
-\left(R_{1}R_{2}\right)R_{3}^{''}
-2\left(R_{2}^{'}R_{3}^{'}\right)R_{1}\right]P\right.\\
&&\left.+2\left(R_{2}R_{3}\right)R_{1}^{'}P^{'}
+\left(R_{1}R_{2}R_{3}\right)P^{''}\right\}R_{2}^{2}R_{3}^{2}
-\left\{\left[PR_{2}R_{3}R_{1}^{'}+R_{1}R_{2}R_{3}P^{'}
-R_{3}R_{1}PR_{2}^{'}\right.\right.\\
&&\left.\left.
-R_{2}R_{1}PR_{3}^{'}\right]\left[2R_{2}R_{3}\left(R_{2}^{'}R_{3}+R_{2}R_{3}^{'}\right)\right]\right\}
\end{eqnarray*}
Evaluando en $z=1$

\begin{eqnarray*}
&=&\left(1+R_{3}\right)^{3}R_{3}^{3}R_{1}^{''}-\left(1+R_{3}\right)^{2}R_{1}R_{3}^{3}R_{3}^{''}
-\left(1+R_{3}\right)^{3}R_{3}^{2}R_{1}R_{3}^{''}-2\left(1+R_{3}\right)^{2}R_{3}^{2}
\left(R_{3}^{'}\right)^{2}\\
&+&2\left(1+R_{3}\right)^{3}R_{3}^{3}R_{1}^{'}P^{'}
+\left(1+R_{3}\right)^{3}R_{3}^{3}R_{1}P^{''}
-2\left(1+R_{3}\right)^{2}R_{3}^{2}\left(1+2R_{3}\right)R_{3}^{'}R_{1}^{'}\\
&-&2\left(1+R_{3}\right)^{2}R_{3}^{2}R_{1}R_{3}^{'}\left(1+2R_{3}\right)P^{'}
+2\left(1+R_{3}\right)\left(1+2R_{3}\right)R_{3}^{3}R_{1}\left(R_{3}^{'}\right)^{2}\\
&+&2\left(1+R_{3}\right)^{2}\left(1+2R_{3}\right)R_{1}R_{3}R_{3}^{'}\\
&=&-\left(1-\mu\right)^{3}\mu^{3}R_{1}^{''}-\left(1-\mu\right)^{2}\mu^{2}R_{1}\left(1-2\mu\right)R_{3}^{''}
-\left(1-\mu\right)^{3}\mu^{3}R_{1}P^{''}\\
&+&2\left(1-\mu\right)\mu^{2}\left[\left(1-2\mu\right)R_{1}-\left(1-\mu\right)\right]\left(R_{3}^{'}\right)^{2}
-2\left(1-\mu\right)^{2}\mu R_{1}\left(1-2\mu\right)R_{3}^{'}\\
&-&2\left(1-\mu\right)^{3}\mu^{4}R_{1}^{'}-2\mu\left(1-\mu\right)\left(1-2\mu\right)R_{3}^{'}R_{1}^{'}
-2\mu^{3}\left(1-\mu\right)^{2}\left(1-2\mu\right)R_{1}R_{1}^{'}
\end{eqnarray*}

por tanto

\begin{eqnarray*}
\left[\frac{F\left(z\right)}{G\left(z\right)}\right]^{'}&=&\frac{1}{\mu^{3}\left(1-\mu\right)^{3}}\left\{
-\left(1-\mu\right)^{2}\mu^{2}R_{1}^{''}-\mu\left(1-\mu\right)\left(1-2\mu\right)R_{1}R_{3}^{''}
-\mu^{2}\left(1-\mu\right)^{2}R_{1}P^{''}\right.\\
&&\left.+2\mu\left[\left(1-2\mu\right)R_{1}-\left(1-\mu\right)\right]\left(R_{3}^{'}\right)^{2}
-2\left(1-\mu\right)\left(1-2\mu\right)R_{1}R_{3}^{'}-2\mu^{3}\left(1-\mu\right)^{2}R_{1}^{'}\right.\\
&&\left.-2\left(1-2\mu\right)R_{3}^{'}R_{1}^{'}-2\mu^{2}\left(1-\mu\right)\left(1-2\mu\right)R_{1}R_{1}^{'}\right\}
\end{eqnarray*}

recordemos que


\begin{eqnarray*}
R_{1}&=&-\esp L\\
R_{3}&=& -\mu\\
R_{1}^{'}&=&-\frac{1}{2}\esp L^{2}+\frac{1}{2}\esp L\\
R_{3}^{'}&=&-\frac{1}{2}\esp X^{2}+\frac{1}{2}\mu\\
R_{1}^{''}&=&-\frac{1}{3}\esp L^{3}+\esp L^{2}-\frac{2}{3}\esp L\\
R_{3}^{''}&=&-\frac{1}{3}\esp X^{3}+\esp X^{2}-\frac{2}{3}\mu\\
R_{1}R_{3}^{'}&=&\frac{1}{2}\esp X^{2}\esp L-\frac{1}{2}\esp X\esp L\\
R_{1}R_{1}^{'}&=&\frac{1}{2}\esp L^{2}\esp L+\frac{1}{2}\esp^{2}L\\
R_{3}^{'}R_{1}^{'}&=&\frac{1}{4}\esp X^{2}\esp L^{2}-\frac{1}{4}\esp X^{2}\esp L-\frac{1}{4}\esp L^{2}\esp X+\frac{1}{4}\esp X\esp L\\
R_{1}R_{3}^{''}&=&\frac{1}{6}\esp X^{3}\esp L^{2}-\frac{1}{6}\esp X^{3}\esp L-\frac{1}{2}\esp L^{2}\esp X^{2}+\frac{1}{2}\esp X^{2}\esp L+\frac{1}{3}\esp X\esp L^{2}-\frac{1}{3}\esp X\esp L\\
R_{1}P^{''}&=&-\esp X^{2}\esp L\\
\left(R_{3}^{'}\right)^{2}&=&\frac{1}{4}\esp^{2}X^{2}-\frac{1}{2}\esp X^{2}\esp X+\frac{1}{4}\esp^{2} X
\end{eqnarray*}




\begin{Def}
Let $L_{i}^{*}$ be the number of users at queue $Q_{i}$ when it is polled, then
\begin{eqnarray}
\begin{array}{cc}
\esp\left[L_{i}^{*}\right]=f_{i}\left(i\right), &
Var\left[L_{i}^{*}\right]=f_{i}\left(i,i\right)+\esp\left[L_{i}^{*}\right]-\esp\left[L_{i}^{*}\right]^{2}.
\end{array}
\end{eqnarray}
\end{Def}

\begin{Def}
The cycle time $C_{i}$ for the queue $Q_{i}$ is the period beginning at the time when it is polled in a cycle and ending at the time when it is polled in the next cycle; it's duration is given by $\tau_{i}\left(m+1\right)-\tau_{i}\left(m\right)$, equivalently $\overline{\tau}_{i}\left(m+1\right)-\overline{\tau}_{i}\left(m\right)$ under steady state assumption.
\end{Def}

\begin{Def}
The intervisit time $I_{i}$ is defined as the period beginning at the time of its service completion in a cycle and ending at the time when it is polled in the next cycle; its duration is given by $\tau_{i}\left(m+1\right)-\overline{\tau}_{i}\left(m\right)$.
\end{Def}

The intervisit time duration $\tau_{i}\left(m+1\right)-\overline{\tau}\left(m\right)$ given the number of users found at queue $Q_{i}$ at time $t=\tau_{i}\left(m+1\right)$ is equal to the number of arrivals during the preceding intervisit time $\left[\overline{\tau}\left(m\right),\tau_{i}\left(m+1\right)\right]$. 

So we have



\begin{eqnarray*}
\esp\left[z_{i}^{L_{i}\left(\tau_{i}\left(m+1\right)\right)}\right]=\esp\left[\left\{P_{i}\left(z_{i}\right)\right\}^{\tau_{i}\left(m+1\right)-\overline{\tau}\left(m\right)}\right]
\end{eqnarray*}

if $I_{i}\left(z\right)=\esp\left[z^{\tau_{i}\left(m+1\right)-\overline{\tau}\left(m\right)}\right]$
we have $F_{i}\left(z\right)=I_{i}\left[P_{i}\left(z\right)\right]$
for $i=1,2$. Futhermore can be proved that

\begin{eqnarray}
\begin{array}{ll}
\esp\left[L_{i}\right]=\mu_{i}\esp\left[I_{i}\right], &
\esp\left[C_{i}\right]=\frac{f_{i}\left(i\right)}{\mu_{i}\left(1-\mu_{i}\right)},\\
\esp\left[S_{i}\right]=\mu_{i}\esp\left[C_{i}\right],&
\esp\left[I_{i}\right]=\left(1-\mu_{i}\right)\esp\left[C_{i}\right],\\
Var\left[L_{i}\right]= \mu_{i}^{2}Var\left[I_{i}\right]+\sigma^{2}\esp\left[I_{i}\right],& 
Var\left[C_{i}\right]=\frac{Var\left[L_{i}^{*}\right]}{\mu_{i}^{2}\left(1-\mu_{i}\right)^{2}},\\
Var\left[S_{i}\right]= \frac{Var\left[L_{i}^{*}\right]}{\left(1-\mu_{i}\right)^{2}}+\frac{\sigma^{2}\esp\left[L_{i}^{*}\right]}{\left(1-\mu_{i}\right)^{3}},&
Var\left[I_{i}\right]= \frac{Var\left[L_{i}^{*}\right]}{\mu_{i}^{2}}-\frac{\sigma_{i}^{2}}{\mu_{i}^{2}}f_{i}\left(i\right).
\end{array}
\end{eqnarray}

Let consider the points when the process $\left[L_{1}\left(1\right),L_{2}\left(1\right),L_{3}\left(1\right),L_{4}\left(1\right)
\right]$ becomes zero at the same time, this points, $T_{1},T_{2},\ldots$ will be denoted as regeneration points, then we have that

\begin{Def}
the interval between two such succesive regeneration points will be called regenerative cycle.
\end{Def}

\begin{Def}
Para $T_{i}$ se define, $M_{i}$, el n\'umero de ciclos de visita a la cola $Q_{l}$, durante el ciclo regenerativo, es decir, $M_{i}$ es un proceso de renovaci\'on.
\end{Def}

\begin{Def}
Para cada uno de los $M_{i}$'s, se definen a su vez la duraci\'on de cada uno de estos ciclos de visita en el ciclo regenerativo, $C_{i}^{(m)}$, para $m=1,2,\ldots,M_{i}$, que a su vez, tambi\'en es n proceso de renovaci\'on.
\end{Def}



Sea la funci\'on generadora de momentos para $L_{i}$, el n\'umero de usuarios en la cola $Q_{i}\left(z\right)$ en cualquier momento, est\'a dada por el tiempo promedio de $z^{L_{i}\left(t\right)}$ sobre el ciclo regenerativo definido anteriormente. Entonces 

\begin{equation}\label{Eq.Longitud.Tiempo.t}
Q_{i}\left(z\right)=\frac{1}{\esp\left[C_{i}\right]}\cdot\frac{1-F_{i}\left(z\right)}{P_{i}\left(z\right)-z}\cdot\frac{\left(1-z\right)P_{i}\left(z\right)}{1-P_{i}\left(z\right)}.
\end{equation}

Es decir, es posible determinar las longitudes de las colas a cualquier tiempo $t$. Entonces, determinando el primer momento es posible ver que


$M_{i}$ is an stopping time for the regenerative process with $\esp\left[M_{i}\right]<\infty$, from Wald's lemma follows that:


\begin{eqnarray*}
\esp\left[\sum_{m=1}^{M_{i}}\sum_{t=\tau_{i}\left(m\right)}^{\tau_{i}\left(m+1\right)-1}z^{L_{i}\left(t\right)}\right]&=&\esp\left[M_{i}\right]\esp\left[\sum_{t=\tau_{i}\left(m\right)}^{\tau_{i}\left(m+1\right)-1}z^{L_{i}\left(t\right)}\right]\\
\esp\left[\sum_{m=1}^{M_{i}}\tau_{i}\left(m+1\right)-\tau_{i}\left(m\right)\right]&=&\esp\left[M_{i}\right]\esp\left[\tau_{i}\left(m+1\right)-\tau_{i}\left(m\right)\right]
\end{eqnarray*}
therefore 

\begin{eqnarray*}
Q_{i}\left(z\right)&=&\frac{\esp\left[\sum_{t=\tau_{i}\left(m\right)}^{\tau_{i}\left(m+1\right)-1}z^{L_{i}\left(t\right)}\right]}{\esp\left[\tau_{i}\left(m+1\right)-\tau_{i}\left(m\right)\right]}
\end{eqnarray*}

Doing the following substitutions en (\ref{Corolario2}): $n\rightarrow t-\tau_{i}\left(m\right)$, $T \rightarrow \overline{\tau}_{i}\left(m\right)-\tau_{i}\left(m\right)$, $L_{n}\rightarrow L_{i}\left(t\right)$ and $F\left(z\right)=\esp\left[z^{L_{0}}\right]\rightarrow F_{i}\left(z\right)=\esp\left[z^{L_{i}\tau_{i}\left(m\right)}\right]$, 
we obtain

\begin{eqnarray}\label{Eq.Arribos.Primera}
\esp\left[\sum_{n=0}^{T-1}z^{L_{n}}\right]=
\esp\left[\sum_{t=\tau_{i}\left(m\right)}^{\overline{\tau}_{i}\left(m\right)-1}z^{L_{i}\left(t\right)}\right]
=z\frac{F_{i}\left(z\right)-1}{z-P_{i}\left(z\right)}
\end{eqnarray}



Por otra parte durante el tiempo de intervisita para la cola $i$, $L_{i}\left(t\right)$ solamente se incrementa de manera que el incremento por intervalo de tiempo est\'a dado por la funci\'on generadora de probabilidades de $P_{i}\left(z\right)$, por tanto la suma sobre el tiempo de intervisita puede evaluarse como:

\begin{eqnarray*}
\esp\left[\sum_{t=\tau_{i}\left(m\right)}^{\tau_{i}\left(m+1\right)-1}z^{L_{i}\left(t\right)}\right]&=&\esp\left[\sum_{t=\tau_{i}\left(m\right)}^{\tau_{i}\left(m+1\right)-1}\left\{P_{i}\left(z\right)\right\}^{t-\overline{\tau}_{i}\left(m\right)}\right]=\frac{1-\esp\left[\left\{P_{i}\left(z\right)\right\}^{\tau_{i}\left(m+1\right)-\overline{\tau}_{i}\left(m\right)}\right]}{1-P_{i}\left(z\right)}\\
&=&\frac{1-I_{i}\left[P_{i}\left(z\right)\right]}{1-P_{i}\left(z\right)}
\end{eqnarray*}
por tanto

\begin{eqnarray*}
\esp\left[\sum_{t=\tau_{i}\left(m\right)}^{\tau_{i}\left(m+1\right)-1}z^{L_{i}\left(t\right)}\right]&=&
\frac{1-F_{i}\left(z\right)}{1-P_{i}\left(z\right)}
\end{eqnarray*}

Por lo tanto

\begin{eqnarray*}
Q_{i}\left(z\right)&=&\frac{\esp\left[\sum_{t=\tau_{i}\left(m\right)}^{\tau_{i}\left(m+1\right)-1}z^{L_{i}\left(t\right)}\right]}{\esp\left[\tau_{i}\left(m+1\right)-\tau_{i}\left(m\right)\right]}
=\frac{1}{\esp\left[\tau_{i}\left(m+1\right)-\tau_{i}\left(m\right)\right]}
\esp\left[\sum_{t=\tau_{i}\left(m\right)}^{\tau_{i}\left(m+1\right)-1}z^{L_{i}\left(t\right)}\right]\\
&=&\frac{1}{\esp\left[\tau_{i}\left(m+1\right)-\tau_{i}\left(m\right)\right]}
\esp\left[\sum_{t=\tau_{i}\left(m\right)}^{\overline{\tau}_{i}\left(m\right)-1}z^{L_{i}\left(t\right)}
+\sum_{t=\overline{\tau}_{i}\left(m\right)}^{\tau_{i}\left(m+1\right)-1}z^{L_{i}\left(t\right)}\right]\\
&=&\frac{1}{\esp\left[\tau_{i}\left(m+1\right)-\tau_{i}\left(m\right)\right]}\left\{
\esp\left[\sum_{t=\tau_{i}\left(m\right)}^{\overline{\tau}_{i}\left(m\right)-1}z^{L_{i}\left(t\right)}\right]
+\esp\left[\sum_{t=\overline{\tau}_{i}\left(m\right)}^{\tau_{i}\left(m+1\right)-1}z^{L_{i}\left(t\right)}\right]\right\}\\
&=&\frac{1}{\esp\left[\tau_{i}\left(m+1\right)-\tau_{i}\left(m\right)\right]}\left\{
z\frac{F_{i}\left(z\right)-1}{z-P_{i}\left(z\right)}+\frac{1-F_{i}\left(z\right)}{1-P_{i}\left(z\right)}
\right\}\\
&=&\frac{1}{\esp\left[C_{i}\right]}\cdot\frac{1-F_{i}\left(z\right)}{P_{i}\left(z\right)-z}\cdot\frac{\left(1-z\right)P_{i}\left(z\right)}{1-P_{i}\left(z\right)}
\end{eqnarray*}

es decir

\begin{eqnarray}
\begin{array}{ll}
S^{'}\left(z\right)=-\sum_{k=1}^{+\infty}ka_{k}z^{k-1},& S^{(1)}\left(1\right)=-\sum_{k=1}^{+\infty}ka_{k}=-\esp\left[L\left(t\right)\right],\\
S^{''}\left(z\right)=-\sum_{k=2}^{+\infty}k(k-1)a_{k}z^{k-2},& S^{(2)}\left(1\right)=-\sum_{k=2}^{+\infty}k(k-1)a_{k}=\esp\left[L\left(L-1\right)\right],\\
S^{'''}\left(z\right)=-\sum_{k=3}^{+\infty}k(k-1)(k-2)a_{k}z^{k-3},&
S^{(3)}\left(1\right)=-\sum_{k=3}^{+\infty}k(k-1)(k-2)a_{k}\\
&=-\esp\left[L\left(L-1\right)\left(L-2\right)\right]\\
&=-\esp\left[L^{3}\right]+3-\esp\left[L^{2}\right]-2-\esp\left[L\right];
\end{array}
\end{eqnarray}






%\section{Existencia de Tiempos de Regeneraci\'on}
%___________________________________________________________
%


\subsection{Introduction}
%______________________________________________________________________

A cyclic polling system consists of multiple queues that are served by a single server in cyclic order. Users arrive at each queue according to independent processes, which also are independent of the service times. The server attends each queue according to a service policy previously established. The most commonly service policies studied are the exhaustive, gated and the k-limited. The exhaustive policy consists in attending all users until the queue is emptied. When the server finishes, it moves to the next queue incurring in a switchover time that is an independent and identically distributed random variable. An exhaustive analysis have been made in this subject. For an overview of the literature on polling systems, their applications and standard results we refer to surveys such as: \cite{Boxma, Kleinrock, LevySidi, Semenova, TakagiI, Takagi}. 

Bos and Boon \cite{BosBoon} published a report where they studied a Network of Polling Systems applied to a traffic problem, there they analyzed a network of intersections and followed a path in it. Their objective was to predict if the costumers can pass through the network in a finite time or not. The buffer occupancy method was used in this analysis and simulation techniques were also used to verify the results. It is important to remark that the heavy traffic case was studied in this report, as well as the cyclic case was not considered.

In this work, we study a Network of Cyclic Polling Systems (NCPS) that consists of two cyclic polling systems, each of them conformed by two queues attended by a single server. We apply the buffer occupancy method described by Kleinrock and Takagi \cite{TakagiI}. This method is based on the use of the Probability Generating Function (PGF) of the joint distribution function of the queues lengths at the moment the server starts a visit period in each of the queues that conform the system.

We present a theorem that guarantees the stability for the NCPS under specific conditions, also we obtain explicit expressions for the queue lengths at the moment the server arrives. With this results we obtain the queue lengths of the NCPS at any time for the servers.

We believe these results can be generalized for the continuous case and from the point of view of applications, the results are useful because they allow us to obtain analytical expressions for the performance measures, and also give us the keys to determine waiting times and queue lengths for any time during the operation of the network. Initially our main goal was studying the system of public transportation, which can be seen as a network consisting of several cyclic polling systems.

%_________________________________________________________________________
%
\subsection{Construcci\'on del Modelo e Hip\'otesis}
%_________________________________________________________________________
%
Consider a Network consisting of two cyclic polling systems with two queues each, $Q_{1}, Q_{2}$ for the first system and $\hat{Q}_{1},\hat{Q}_{2}$ for the second one, each of them with infinite-sized buffer. In each system a single server visits the queues in cyclic order, where it applies the exhaustive policy, i.e., when the server polls a queue, it serves all the customers present until the queue becomes empty. This case is illustrated in \texttt{Figure 1}. 

The second system's users at queue 2, can moves to the first system after being attended, also we assume that the network is open; that is, all customers eventually leave the network. As usually in polling systems theory we assume the arrivals in each queue are Poisson processes from with independent identical distributed (i.i.d.) inter arrival exponential times. The service times are exponential independent and identically distributed random variables. Finally upon completion of a visit at any queue, the servers incurs in a random switchover time according to an arbitrary distribution. We define a cycle to be the time interval between two consecutive polling instants, the time period in a cycle during which the server is attending a queue is called a service period. We are considering the case where the server visit the queues in cyclic order.

Time is slotted with slot size equal to the service time of a fixed costumer, we call the time interval $\left[t,t+1\right]$ the $t$-th slot. The arrival processes are denoted by $X_{1}\left(t\right),X_{2}\left(t\right)$ for the first system and $\hat{X}_{1}\left(t\right)$, $\hat{X}_{2}\left(t\right)$ for the second, the arrival rate at $Q_{i}$ and $\hat{Q}_{i}$ is denoted by $\mu_{i}$ and $\hat{\mu}_{i}$ respectively, with the condition $\mu_{i}<1$ and $\hat{\mu}_{i}<1$. The second system's users pass to the first one according to a process $Y_{2}$, with arrival rate $\tilde{\mu}_{2}$. 

Let's denote by $\tau_{i}$ the polling instant at queue $Q_{1}$ and by $\overline{\tau}_{i}$ the instant when the servers leaves to queue and starts a switchover time. Like the rest of the random variables the swithcover period is an i.i.d random variable $R_{i}$ with general distribution. 


To determine the length of the queues, i.e., the number of users in the queue at the moment the server arrives we define the process $L_{i}$ and $\hat{L}_{i}$ for the first and second system, respectively, in the sequel we use the buffer occupancy method to obtain the generating function, first and second moments of queue size distributions at polling instants. At each of the queues in the network the number of users is the number of users at the time the server arrives plus the numbers of users from the other system. 


In order to obtain the joint probability generating function (PGF) for the number or users residing in queue $i$ when the queue is polled in the NCPS, we define for each of the arrival processes $X_{1},X_{2},\hat{X}_{1},\hat{X}_{2},Y_{2}$, and $\tilde{X}_{2}$ with $\tilde{X}_{2}=X_{2}+Y_{2}$, their PGF

\begin{eqnarray*}
\begin{array}{cc}
P_{i}\left(z_{i}\right)=\esp\left[z_{i}^{X_{i}\left(t\right)}\right],&
\hat{P}_{i}\left(w_{i}\right)=\esp\left[w_{i}^{\hat{X}_{i}\left(t\right)}\right]
\end{array}
\end{eqnarray*}
for $i=1,2$, and
\begin{eqnarray*}
\begin{array}{cc}
\check{P}_{2}\left(z_{2}\right)=\esp\left[z_{2}^{Y_{2}\left(t\right)}\right],& \tilde{P}_{2}\left(z_{2}\right)=\esp\left[z_{2}^{\tilde{X}_{2}\left(t\right)}
\right],
\end{array}
\end{eqnarray*}

for $i=1,2$, and
\begin{eqnarray*} 
\begin{array}{cc}
\check{\mu}_{2}=\esp\left[Y_{2}\left(t\right)\right]=\check{P}_{2}^{(1)}
\left(1\right),&
\tilde{\mu}_{2}=\esp\left[\tilde{X}_{2}\left(t\right)\right]
=\tilde{P}_{2}^{(1)}\left(1\right).
\end{array}
\end{eqnarray*} The PGF For the service time is defined by:

\begin{eqnarray*}
\begin{array}{cc}
S_{i}\left(z_{i}\right)=\esp\left[z_{i}^{\overline{\tau}_{i}-\tau_{i}}
\right], &
\hat{S}_{i}\left(w_{i}\right)=\esp\left[w_{i}^{\overline{\zeta}_{i}-\zeta_{i}}\right]
\end{array}
\end{eqnarray*} with first moment 
\begin{eqnarray*}
\begin{array}{cc}
s_{i}=\esp\left[\overline{\tau}_{i}-\tau_{i}\right],&\hat{s}_{i}=\esp\left[\overline{\zeta}_{i}-\zeta_{i}\right]
\end{array}
\end{eqnarray*} for $i=1,2$. In a similar manner the PGF for the switchover time of the server from the moment it ends to attend a queue, to the time of arrival to the next queue is given by 
\begin{eqnarray*}
\begin{array}{cc}
R_{i}\left(z_{i}\right)=\esp\left[z_{1}^{\tau_{i+1}-\overline{\tau}_{i}}\right],&
\hat{R}_{i}\left(w_{i}\right)=\esp\left[w_{i}^{\zeta_{i+1}-\overline{\zeta}_{i}}\right]
\end{array}
\end{eqnarray*} with first moment 

\begin{eqnarray*}
\begin{array}{cc}
r_{i}=\esp\left[\tau_{i+1}-\overline{\tau}_{i}\right],&
\hat{r}_{i}=\esp\left[\zeta_{i+1}-\overline{\zeta}_{i}\right]
\end{array}
\end{eqnarray*} for $i=1,2$. The number of users in the queue at times $\overline{\tau}_{1},\overline{\tau}_{2}, \overline{\zeta}_{1},\overline{\zeta}_{2}$, it's zero, i.e.,
 $L_{i}\left(\overline{\tau_{i}}\right)=0,$ and $\hat{L}_{i}\left(\overline{\zeta_{i}}\right)=0$ for $i=1,2$. Then the number of users in the queue of the second system at the moment the server ends attending in the queue is given by the number of users present at the moment it arrives plus the number of arrivals during the service time, i.e.,
$$\hat{L}_{i}\left(\overline{\tau}_{j}\right)=\hat{L}_{i}\left(\tau_{j}\right)+\hat{X}_{i}\left(\overline{\tau}_{j}-\tau_{j}\right),$$
for $i,j=1,2$, meanwhile for the first system : $$L_{1}\left(\overline{\tau}_{j}\right)=L_{1}\left(\tau_{j}\right)+X_{1}\left(\overline{\tau}_{j}-\tau_{j}\right).$$ Specifically for the second queue of the first system we need to consider the users of transfer becoming from the second queue in the second system while the server its in the other queue attending, it means that this users have been already attended by the server before they can go to the first queue:

\begin{equation}\label{Eq.UsuariosTotalesZ2}
L_{2}\left(\overline{\tau}_{1}\right)=L_{2}\left(\tau_{1}\right)+X_{2}\left(\overline{\tau}_{1}-\tau_{1}\right)+Y_{2}\left(\overline{\tau}_{1}-\tau_{1}\right).
\end{equation}

As is know, the gambler's ruin problem can be used to model the server's busy period in a cyclic polling system, so let $\tilde{L}_{0}\geq0$ be the number of users present at the moment the server arrive to start attending, also let $T$ be the time the server need to attend the users in the queue starting with $\tilde{L}_{0}$ users. Suppose the gambler has two independent and simultaneous moves, such events are independent and identical to each other for each realization. The gain on the $n$-th game is $\tilde{\mathsf{X}}_{n}=\mathsf{X}_{n}+\mathsf{Y}_{n}$ units from which is substracted a playing fee of 1 unit for each move. His PGF is given by $F\left(z\right)=\esp\left[z^{\tilde{L}_{0}}\right]$, futhermore
%$\tilde{\mathrm{X}}$, $\tilde{\mathit{X}}$, $\tilde{\mathcal{X}}$, $\tilde{\mathfrak{X}}$,$\tilde{\mathbb{X}}$,$\tilde{\mathtt{X}}$,$\tilde{\mathsf{X}}$,

$$\tilde{P}\left(z\right)=\esp\left[z^{\tilde{\mathsf{X}}_{n}}\right]=\esp\left[z^{\mathsf{X}_{n}+\mathsf{X}_{n}}\right]=\esp\left[z^{\mathsf{X}_{n}}z^{\mathsf{X}_{n}}\right]=\esp\left[z^{\mathsf{X}_{n}}\right]\esp\left[z^{\mathsf{X}_{n}}\right]=P\left(z\right)\check{P}\left(z\right),$$ with $\tilde{\mu}=\esp\left[\tilde{\mathsf{X}}_{n}\right]=\tilde{P}\left[z\right]<1$. If  $\tilde{L}_{n}$ denotes the capital remaining after the $n$-th game, then $\tilde{L}_{n}=\tilde{L}_{0}+\tilde{\mathsf{X}}_{1}+\tilde{\mathsf{X}}_{2}+\cdots+\tilde{\mathsf{X}}_{n}-2n$. The result that relates the gambler's ruin problem with the busy period of the server it's a generalization of the result given in Takagi \cite{Takagi} chapter 3.

\begin{Prop}
Let's $G_{n}\left(z\right)$ and $G\left(z,w\right)$ defined as in 
(\ref{Eq.3.16.b.2S}), then $G_{n}\left(z\right)=\frac{1}{z}\left[G_{n-1}\left(z\right)-G_{n-1}\left(0\right)\right]\tilde{P}\left(z\right)$. Futhermore $G\left(z,w\right)=\frac{zF\left(z\right)-wP\left(z\right)G\left(0,w\right)}{z-wR\left(z\right)}$, with a unique pole in the unit circle, also the pole is of the form $z=\theta\left(w\right)$ and satisfies 
\begin{multicols}{3}
\begin{itemize}
\item[i)]$\tilde{\theta}\left(1\right)=1$,

\item[ii)] $\tilde{\theta}^{(1)}\left(1\right)=\frac{1}{1-\tilde{\mu}}$,

\item[iii)]
$\tilde{\theta}^{(2)}\left(1\right)=\frac{\tilde{\mu}}{\left(1-\tilde{\mu}\right)^{2}}+\frac{\tilde{\sigma}}{\left(1-\tilde{\mu}\right)^{3}}$.
\end{itemize}
\end{multicols}
\end{Prop}
%_________________________________________________________________________
%
\subsection{Description of the model: Probability Generating Function}
%_________________________________________________________________________
%

In order to model the network of cyclic polling system it's necessary to consider the users arrivals to each queue in one of the system, but on times the other system's server arrival, $\zeta_{i}$. In the case of the first system and the server arrives to a queue in the second one: $$F_{i,j}\left(z_{i};\zeta_{j}\right)=\esp\left[z_{i}^{L_{i}\left(\zeta_{j}\right)}\right]=
\sum_{k=0}^{\infty}\prob\left[L_{i}\left(\zeta_{j}\right)
=k\right]z_{i}^{k},$$ for $i,j=1,2$. Now consider the case of the queues in the second system and the server arrive to a queue in the first system $$\hat{F}_{i,j}\left(w_{i};\tau_{j}\right)=\esp\left[w_{i}^{\hat{L}_{i}\left(\tau_{j}\right)}\right] =\sum_{k=0}^{\infty}\prob\left[\hat{L}_{i}\left(\tau_{j}\right)
=k\right]w_{i}^{k},$$ for $i,j=1,2$. With the developed we can define the joint PGF for the second system:
$$\esp\left[w_{1}^{\hat{L}_{1}\left(\tau_{j}\right)}w_{2}^{\hat{L}_{2}\left(\tau_{j}\right)}\right]
=\esp\left[w_{1}^{\hat{L}_{1}\left(\tau_{j}\right)}\right]
\esp\left[w_{2}^{\hat{L}_{2}\left(\tau_{j}\right)}\right]=\hat{F}_{1,j}\left(w_{1};\tau_{j}\right)\hat{F}_{2,j}\left(w_{2};\tau_{j}\right)\equiv\hat{\mathbf{F}}_{j}\left(w_{1},w_{2};\tau_{j}\right).$$
%\end{eqnarray*}

In a similar manner we define the joint PGF for the first system, and the second system's server:
%\begin{eqnarray*}
$$\esp\left[z_{1}^{L_{1}\left(\zeta_{j}\right)}z_{2}^{L_{2}\left(\zeta_{j}\right)}\right]
=\esp\left[z_{1}^{L_{1}\left(\zeta_{j}\right)}\right]
\esp\left[z_{2}^{L_{2}\left(\zeta_{j}\right)}\right]=F_{1,j}\left(z_{1};\zeta_{j}\right)F_{2,j}\left(z_{2};\zeta_{j}\right)\equiv\mathbf{F}_{j}\left(z_{1},z_{2};\zeta_{j}\right).$$
%\end{eqnarray*}

Now we proceed to determine the joint PGF for the times that the server visit each queue in their corresponding system, i.e., $t=\left\{\tau_{1},\tau_{2},\zeta_{1},\zeta_{2}\right\}$:

\begin{eqnarray}\label{Eq.Conjuntas}
\begin{array}{l}
\mathbf{F}_{j}\left(z_{1},z_{2},w_{1},w_{2}\right)=\esp\left[\prod_{i=1}^{2}z_{i}^{L_{i}\left(\tau_{j}
\right)}\prod_{i=1}^{2}w_{i}^{\hat{L}_{i}\left(\tau_{j}\right)}\right],\\
\hat{\mathbf{F}}_{j}\left(z_{1},z_{2},w_{1},w_{2}\right)=\esp\left[\prod_{i=1}^{2}z_{i}^{L_{i}
\left(\zeta_{j}\right)}\prod_{i=1}^{2}w_{i}^{\hat{L}_{i}\left(\zeta_{j}\right)}\right],
\end{array}
\end{eqnarray} for $j=1,2$. Then with the purpose of find the number of users present in the netwotk when the server ends attending one of the queues in any of the systems we have that

\begin{eqnarray*}
&&\esp\left[z_{1}^{L_{1}\left(\overline{\tau}_{1}\right)}z_{2}^{L_{2}\left(\overline{\tau}_{1}\right)}w_{1}^{\hat{L}_{1}\left(\overline{\tau}_{1}\right)}w_{2}^{\hat{L}_{2}\left(\overline{\tau}_{1}\right)}\right]
=\esp\left[z_{2}^{L_{2}\left(\overline{\tau}_{1}\right)}w_{1}^{\hat{L}_{1}\left(\overline{\tau}_{1}
\right)}w_{2}^{\hat{L}_{2}\left(\overline{\tau}_{1}\right)}\right]\\
&=&\esp\left[z_{2}^{L_{2}\left(\tau_{1}\right)+X_{2}\left(\overline{\tau}_{1}-\tau_{1}\right)+Y_{2}\left(\overline{\tau}_{1}-\tau_{1}\right)}w_{1}^{\hat{L}_{1}\left(\tau_{1}\right)+\hat{X}_{1}\left(\overline{\tau}_{1}-\tau_{1}\right)}w_{2}^{\hat{L}_{2}\left(\tau_{1}\right)+\hat{X}_{2}\left(\overline{\tau}_{1}-\tau_{1}\right)}\right]
\end{eqnarray*}

using the equation (\ref{Eq.UsuariosTotalesZ2}) we have


\begin{eqnarray*}
&=&\esp\left[z_{2}^{L_{2}\left(\tau_{1}\right)}z_{2}^{X_{2}\left(\overline{\tau}_{1}-\tau_{1}\right)}z_{2}^{Y_{2}\left(\overline{\tau}_{1}-\tau_{1}\right)}w_{1}^{\hat{L}_{1}\left(\tau_{1}\right)}w_{1}^{\hat{X}_{1}\left(\overline{\tau}_{1}-\tau_{1}\right)}w_{2}^{\hat{L}_{2}\left(\tau_{1}\right)}w_{2}^{\hat{X}_{2}\left(\overline{\tau}_{1}-\tau_{1}\right)}\right]\\
&=&\esp\left[z_{2}^{L_{2}\left(\tau_{1}\right)}\left\{w_{1}^{\hat{L}_{1}\left(\tau_{1}\right)}w_{2}^{\hat{L}_{2}\left(\tau_{1}\right)}\right\}\left\{z_{2}^{X_{2}\left(\overline{\tau}_{1}-\tau_{1}\right)}
z_{2}^{Y_{2}\left(\overline{\tau}_{1}-\tau_{1}\right)}w_{1}^{\hat{X}_{1}\left(\overline{\tau}_{1}-\tau_{1}\right)}w_{2}^{\hat{X}_{2}\left(\overline{\tau}_{1}-\tau_{1}\right)}\right\}\right]
\end{eqnarray*}

applying the fact that the arrivals processes in the queues in each systems are independent:

$$=\esp\left[z_{2}^{L_{2}\left(\tau_{1}\right)}\left\{z_{2}^{X_{2}\left(\overline{\tau}_{1}-\tau_{1}\right)}z_{2}^{Y_{2}\left(\overline{\tau}_{1}-
\tau_{1}\right)}w_{1}^{\hat{X}_{1}\left(\overline{\tau}_{1}-\tau_{1}\right)}w_{2}^{\hat{X}_{2}\left(\overline{\tau}_{1}-\tau_{1}\right)}\right\}\right]
\esp\left[w_{1}^{\hat{L}_{1}\left(\tau_{1}\right)}w_{2}^{\hat{L}_{2}\left(\tau_{1}\right)}\right]$$ given that the arrival processes in the queues are independent, it's possible to separate the expectation for the arrival processes in $Q_{1}$ and $Q_{2}$ at time $\tau_{1}$, which is the time the server visits $Q_{1}$. Considering
$\tilde{X}_{2}\left(z_{2}\right)=X_{2}\left(z_{2}\right)+Y_{2}\left(z_{2}\right)$ we have


\begin{eqnarray*}
\begin{array}{l}
=\esp\left[z_{2}^{L_{2}\left(\tau_{1}\right)}\left\{z_{2}^{\tilde{X}_{2}\left(\overline{\tau}_{1}-\tau_{1}\right)}w_{1}^{\hat{X}_{1}\left(\overline{\tau}_{1}
-\tau_{1}\right)}
w_{2}^{\hat{X}_{2}\left(\overline{\tau}_{1}-\tau_{1}\right)}\right\}\right]\esp\left[w_{1}^{\hat{L}_{1}\left(\tau_{1}\right)}
w_{2}^{\hat{L}_{2}\left(\tau_{1}\right)}\right]\\
=\esp\left[z_{2}^{L_{2}\left(\tau_{1}\right)}\left\{\tilde{P}_{2}\left(z_{2}\right)
^{\overline{\tau}_{1}-\tau_{1}}\hat{P}_{1}\left(w_{1}\right)^{\overline{\tau}_{1}-
\tau_{1}}\hat{P}_{2}\left(w_{2}\right)^{\overline{\tau}_{1}-\tau_{1}}\right\}\right]
\esp\left[w_{1}^{\hat{L}_{1}\left(\tau_{1}\right)}w_{2}^{\hat{L}_{2}\left(\tau_{1}\right)}\right]\\
=\esp\left[z_{2}^{L_{2}\left(\tau_{1}\right)}\left\{\tilde{P}_{2}\left(z_{2}\right)\hat{P}_{1}\left(w_{1}\right)\hat{P}_{2}\left(w_{2}\right)\right\}^{\overline{\tau}_{1}-\tau_{1}}\right]\esp\left[w_{1}^{\hat{L}_{1}\left(\tau_{1}\right)}w_{2}^{\hat{L}_{2}\left(\tau_{1}\right)}\right]\\
=\esp\left[z_{2}^{L_{2}\left(\tau_{1}\right)}\theta_{1}\left(\tilde{P}_{2}\left(z_{2}\right)\hat{P}_{1}\left(w_{1}\right)\hat{P}_{2}\left(w_{2}\right)\right)
^{L_{1}\left(\tau_{1}\right)}\right]\esp\left[w_{1}^{\hat{L}_{1}\left(\tau_{1}\right)}w_{2}^{\hat{L}_{2}\left(\tau_{1}\right)}\right]\\
=F_{1}\left(\theta_{1}\left(\tilde{P}_{2}\left(z_{2}\right)\hat{P}_{1}\left(w_{1}\right)\hat{P}_{2}\left(w_{2}\right)\right),z_{2}\right)\cdot
\hat{F}_{1}\left(w_{1},w_{2};\tau_{1}\right)\\
\equiv \mathbf{F}_{1}\left(\theta_{1}\left(\tilde{P}_{2}\left(z_{2}\right)\hat{P}_{1}\left(w_{1}\right)\hat{P}_{2}\left(w_{2}\right)\right),z_{2},w_{1},w_{2}\right).
\end{array}
\end{eqnarray*}

The last equalities  are true because the number of arrivals to $\hat{Q}_{2}$ 
during the time interval $\left[\tau_{1},\overline{\tau}_{1}\right]$ still haven't been attended by the server in the system 2, then the users can't pass to the first system through the queue $Q_{2}$. Therefore the number of users switching from $\hat{Q}_{2}$ to $Q_{2}$ during the time interval $\left[\tau_{1},\overline{\tau}_{1}\right]$ depends on the policy of transfer between the two systems, according to the last section
%{\small{
\begin{eqnarray*}\label{Eq.Fs}
\begin{array}{l}
\esp\left[z_{1}^{L_{1}\left(\overline{\tau}_{1}\right)}z_{2}^{L_{2}\left(\overline{\tau}_{1}
\right)}w_{1}^{\hat{L}_{1}\left(\overline{\tau}_{1}\right)}w_{2}^{\hat{L}_{2}\left(
\overline{\tau}_{1}\right)}\right]
=\mathbf{F}_{1}\left(\theta_{1}\left(\tilde{P}_{2}\left(z_{2}\right)
\hat{P}_{1}\left(w_{1}\right)\hat{P}_{2}\left(w_{2}\right)\right),z_{2},w_{1},w_{2}\right)\\
\equiv F_{1}\left(\theta_{1}\left(\tilde{P}_{2}\left(z_{2}\right)\hat{P}_{1}\left(w_{1}\right)\hat{P}_{2}\left(w_{2}\right)\right),z_{2}\right)\hat{F}_{1}\left(w_{1},w_{2};\tau_{1}\right).
\end{array}
\end{eqnarray*}%}}

Using similar reasoning for the rest of the server's arrival times we have that

\begin{eqnarray*}
\esp\left[z_{1}^{L_{1}\left(\overline{\tau}_{2}\right)}z_{2}^{L_{2}\left(\overline{\tau}_{2}\right)}w_{1}^{\hat{L}_{1}\left(\overline{\tau}_{2}\right)}w_{2}^{\hat{L}_{2}\left(\overline{\tau}_{2}\right)}\right]&=&F_{2}\left(z_{1},\tilde{\theta}_{2}\left(P_{1}\left(z_{1}\right)\hat{P}_{1}\left(w_{1}\right)\hat{P}_{2}\left(w_{2}\right)\right)\right)
\hat{F}_{2}\left(w_{1},w_{2};\tau_{2}\right)\\
&\equiv& \mathbf{F}_{2}\left(z_{1},\tilde{\theta}_{2}\left(P_{1}\left(z_{1}\right)\hat{P}_{1}\left(w_{1}\right)\hat{P}_{2}\left(w_{2}\right)\right),w_{1},w_{2}\right),\\
\esp\left[z_{1}^{L_{1}\left(\overline{\zeta}_{1}\right)}z_{2}^{L_{2}\left(\overline{\zeta}_{1}
\right)}w_{1}^{\hat{L}_{1}\left(\overline{\zeta}_{1}\right)}w_{2}^{\hat{L}_{2}\left(\overline{\zeta}_{1}\right)}\right]
&=&F_{1}\left(z_{1},z_{2};\zeta_{1}\right)\hat{F}_{1}\left(\hat{\theta}_{1}\left(P_{1}\left(z_{1}\right)\tilde{P}_{2}\left(z_{2}\right)\hat{P}_{2}\left(w_{2}\right)\right),w_{2}\right)\\
&\equiv&\hat{\mathbf{F}}_{1}\left(z_{1},z_{2},\hat{\theta}_{1}\left(P_{1}\left(z_{1}\right)\tilde{P}_{2}\left(z_{2}\right)\hat{P}_{2}\left(w_{2}\right)\right),w_{2}\right),\\
\esp\left[z_{1}^{L_{1}\left(\overline{\zeta}_{2}\right)}z_{2}^{L_{2}\left(\overline{\zeta}_{2}\right)}w_{1}^{\hat{L}_{1}\left(\overline{\zeta}_{2}\right)}w_{2}^{\hat{L}_{2}\left(\overline{\zeta}_{2}\right)}\right]
&=&F_{2}\left(z_{1},z_{2};\zeta_{2}\right)\hat{F}_{2}\left(w_{1},\hat{\theta}_{2}\left(P_{1}\left(z_{1}\right)\tilde{P}_{2}\left(z_{2}\right)\hat{P}_{1}\left(w_{1}\right)\right)\right)\\
&\equiv&\hat{\mathbf{F}}_{2}\left(z_{1},z_{2},w_{1},\hat{\theta}_{2}\left(P_{1}\left(z_{1}\right)\tilde{P}_{2}\left(z_{2}\right)\hat{P}_{1}\left(w_{1}\right)\right)\right).
\end{eqnarray*}

Now we are in conditions to obtain the recursive equations that model the NCPS. We need to consider the switchover times that the server need to translate from one queue to another and, the number or user presents in the system at the time the server leaves to the queue to start attending the next. Thus far developed, we can find that for the NCPS:

\begin{eqnarray}\label{Recursive.Equations.First.Casse}
\begin{array}{r}
\mathbf{F}_{2}\left(z_{1},z_{2},w_{1},w_{2}\right)=R_{1}\left(P_{1}\left(z_{1}\right)\tilde{P}_{2}
\left(z_{2}\right)\prod_{i=1}^{2}
\hat{P}_{i}\left(w_{i}\right)\right)\mathbf{F}_{1}\left(\theta_{1}\left(\tilde{P}_{2}\left(z_{2}
\right)\hat{P}_{1}\left(w_{1}\right)\hat{P}_{2}\left(w_{2}\right)\right),z_{2},w_{1},w_{2}\right),\\
\mathbf{F}_{1}\left(z_{1},z_{2},w_{1},w_{2}\right)=R_{2}\left(P_{1}\left(z_{1}\right)\tilde{P}_{2}
\left(z_{2}\right)\prod_{i=1}^{2}
\hat{P}_{i}\left(w_{i}\right)\right)\mathbf{F}_{2}\left(z_{1},\tilde{\theta}_{2}\left(P_{1}\left(z_{1}\right)\hat{P}_{1}\left(w_{1}\right)\hat{P}_{2}\left(w_{2}
\right)\right),w_{1},w_{2}\right),\\
\hat{\mathbf{F}}_{2}\left(z_{1},z_{2},w_{1},w_{2}\right)=\hat{R}_{1}\left(P_{1}\left(z_{1}\right)\tilde{P}_{2}\left(z_{2}\right)\prod_{i=1}^{2}
\hat{P}_{i}\left(w_{i}\right)\right)\hat{\mathbf{F}}_{1}\left(z_{1},z_{2},\hat{\theta}_{1}\left(P_{1}\left(z_{1}\right)\tilde{P}_{2}\left(z_{2}\right)\hat{P}_{2}\left(w_{2}
\right)\right),w_{2}\right),\\
\hat{\mathbf{F}}_{1}\left(z_{1},z_{2},w_{1},w_{2}\right)=\hat{R}_{2}\left(P_{1}\left(z_{1}\right)\tilde{P}_{2}\left(z_{2}\right)\prod_{i=1}^{2}
\hat{P}_{i}\left(w_{i}\right)\right)\hat{\mathbf{F}}_{2}\left(z_{1},z_{2},w_{1},\hat{\theta}_{2}\left(P_{1}\left(z_{1}\right)\tilde{P}_{2}\left(z_{2}\right)\hat{P}_{1}\left(w_{1}
\right)\right)\right).
\end{array}
\end{eqnarray}



%_____________________________________________________
%\subsubsection{Server Switchover times}
%_____________________________________________________
It's necessary to give an step ahead, considering the case illustrated in \texttt{Figure 2}, where just like before, the server's switchover times are given by the generals equations
$R_{i}\left(\mathbf{z,w}\right)=R_{i}\left(\tilde{P}_{1}\left(z_{1}\right)
\tilde{P}_{2}\left(z_{2}\right)\tilde{P}_{3}\left(z_{3}\right)
\tilde{P}_{4}\left(z_{4}\right)\right)$, with first order derivatives given by $D_{i}R_{i}=r_{i}\tilde{\mu}_{i}$, and second order partial derivatives $D_{j}D_{i}R_{k}=R_{k}^{(2)}\tilde{\mu}_{i}\tilde{\mu}_{j}+\indora_{i=j}r_{k}P_{i}^{(2)}+\indora_{i\neq j}r_{k}\tilde{\mu}_{i}\tilde{\mu}_{j}$ for any $i,j,k$. According to the equations given before and the queue lengths for the other system's server times, we can obtain general expressions

\begin{eqnarray}\label{Ec.Gral.Primer.Momento.Ind.Exh}
\begin{array}{ll}
D_{j}\mathbf{F}_{i}\left(z_{1},z_{2};\tau_{i+2}\right)=\indora_{j\leq2}F_{j,i+2}^{(1)},&
D_{j}\mathbf{F}_{i}\left(z_{3},z_{4};\tau_{i-2}\right)=\indora_{j\geq3}F_{j,i-2}^{(1)},
\end{array}
\end{eqnarray}

for $i,j=1,2,3,4$; with second order derivatives given by

\begin{eqnarray}\label{Ec.Gral.Segundo.Momento.Ind.Exh}
\begin{array}{l}
D_{j}D_{i}\mathbf{F}_{k}\left(z_{1},z_{2};\tau_{k+2}\right)=\indora_{i\geq3}\indora_{j=i}F_{i,k+2}^{(2)}+\indora_{i\geq 3}\indora_{j\neq i}F_{j,k-2}^{(1)}F_{i,k+2}^{(1)},\\
D_{j}D_{i}\mathbf{F}_{k}\left(z_{3},z_{4};\tau_{k-2}\right)=\indora_{i\geq3}\indora_{j=i}F_{i,k-2}^{(2)}+\indora_{i\geq 3}\indora_{j\neq i}F_{j,k-2}^{(1)}F_{i,k-2}^{(1)}.
\end{array}
\end{eqnarray}


 According with the developed at the moment, we can get the recursive equations which are of the following form

\begin{eqnarray}\label{General.System.Double.Transfer}
\begin{array}{l}
\mathbf{F}_{1}\left(z_{1},z_{2},z_{3},z_{4}\right)=R_{2}\left(\prod_{i=1}^{4}\tilde{P}_{i}\left(z_{i}
\right)\right)\mathbf{F}_{2}\left(z_{1},\tilde{\theta}_{2}\left(\tilde{P}_{1}\left(z_{1}\right)\tilde{P}_{3}\left(z_{3}\right)\tilde{P}_{4}
\left(z_{4}\right)\right),z_{3},z_{4}\right),\\
\mathbf{F}_{2}\left(z_{1},z_{2},z_{3},z_{4}\right)=R_{1}\left(\prod_{i=1}^{4}\tilde{P}_{i}\left(z_{i}
\right)\right)
\mathbf{F}_{1}\left(\tilde{\theta}_{1}\left(\tilde{P}_{2}\left(z_{2}\right)\tilde{P}_{3}\left(z_{3}
\right)\tilde{P}_{4}\left(z_{4}\right)\right),z_{2},z_{3},z_{4}\right),\\
\mathbf{F}_{3}\left(z_{1},z_{2},z_{3},z_{4}\right)=R_{4}\left(\prod_{i=1}^{4}\tilde{P}_{i}\left(z_{i}
\right)\right)\mathbf{F}_{4}\left(z_{1},z_{2},z_{3},\tilde{\theta}_{4}\left(\tilde{P}_{1}\left(z_{1}\right)\tilde{P}_{2}\left(z_{2}\right)\tilde{P}_{3}\left(z_{3}\right)
\right)\right),\\
\mathbf{F}_{4}\left(z_{1},z_{2},z_{3},z_{4}\right)=R_{3}\left(\prod_{i=1}^{4}\tilde{P}_{i}\left(z_{i}
\right)\right)
\mathbf{F}_{3}\left(z_{1},z_{2},\tilde{\theta}_{3}\left(\tilde{P}_{1}\left(z_{1}\right)\tilde{P}_{2}\left(z_{2}\right)\tilde{P}_{4}
\left(z_{4}\right)\right),z_{4}\right).
\end{array}
\end{eqnarray}
%_________________________________________________________________________
%
%\subsection{Hipotesis sobre las colas}
%_________________________________________________________________________
%


So we have the first theorem

\begin{Teo}
Suppose  $\tilde{\mu}=\tilde{\mu}_{1}+\tilde{\mu}_{2}<1$, $\hat{\mu}=\tilde{\mu}_{3}+\tilde{\mu}_{4}<1$, then the number of users in the queues conforming the network of cyclic polling system (\ref{General.System.Double.Transfer}), when the server visit a queue can be found solving the linear system given by equations (\ref{Ec.Primer.Orden.General.Impar}) and (\ref{Ec.Primer.Orden.General.Par}):

\begin{eqnarray}\label{Ec.Primer.Orden.General.Impar}
\begin{array}{l}
f_{j}\left(i\right)=r_{j+1}\tilde{\mu}_{i}
+\indora_{i\neq j+1}f_{j+1}\left(j+1\right)\frac{\tilde{\mu}_{i}}{1-\tilde{\mu}_{j+1}}
+\indora_{i=j}f_{j+1}\left(i\right)
+\indora_{j=1}\indora_{i\geq3}F_{i,j+1}^{(1)}
+\indora_{j=3}\indora_{i\leq2}F_{i,j+1}^{(1)}
\end{array}
\end{eqnarray}
$j=1,3$ and $i=1,2,3,4$.

\begin{eqnarray}\label{Ec.Primer.Orden.General.Par}
\begin{array}{l}
f_{j}\left(i\right)=r_{j-1}\tilde{\mu}_{i}
+\indora_{i\neq j-1}f_{j-1}\left(j-1\right)\frac{\tilde{\mu}_{i}}{1-\tilde{\mu}_{j-1}}
+\indora_{i=j}f_{j-1}\left(i\right)
+\indora_{j=2}\indora_{i\geq3}F_{i,j-1}^{(1)}
+\indora_{j=4}\indora_{i\leq2}F_{i,j-1}^{(1)}
\end{array}
\end{eqnarray}
$j=2,4$ and $i=1,2,3,4$, whose solutions are:
%{\footnotesize{


\begin{eqnarray}
\begin{array}{l}
f_{i}\left(j\right)=\left(\indora_{j=i-1}+\indora_{j=i+1}\right)r_{j}\tilde{\mu}_{j}+\indora_{i=j}\left(\indora_{i\leq2}\frac{r\tilde{\mu}_{i}\left(1-\tilde{\mu}_{i}\right)}{1-\tilde{\mu}}+\indora_{i\geq2}\frac{\hat{r}\tilde{\mu}_{i}\left(1-\tilde{\mu}_{i}\right)}{1-\hat{\mu}}\right)\\
+\indora_{i=1}\indora_{j\geq3}\left(\tilde{\mu}_{j}\left(r_{i+1}+\frac{r\tilde{\mu}_{i+1}}{1-\tilde{\mu}}\right)+F_{j,i+1}^{(1)}\right)
+\indora_{i=3}\indora_{j\geq3}\left(\tilde{\mu}_{j}\left(r_{i+1}+\frac{\hat{r}\tilde{\mu}_{i+1}}{1-\hat{\mu}}\right)+F_{j,i+1}^{(1)}\right)\\
+\indora_{i=2}\indora_{j\leq2}\left(\tilde{\mu}_{j}\left(r_{i-1}+\frac{r\tilde{\mu}_{i-1}}{1-\tilde{\mu}}\right)+F_{j,i-1}^{(1)}\right)
+\indora_{i=4}\indora_{j\leq2}\left(\tilde{\mu}_{j}\left(r_{i-1}+\frac{\hat{r}\tilde{\mu}_{i-1}}{1-\hat{\mu}}\right)+F_{j,i-1}^{(1)}\right).
\end{array}
\end{eqnarray}
\end{Teo}
%______________________________________________________________________

\begin{Teo}
For the system given in (\ref{General.System.Double.Transfer}) we have that the second moments are in their general form

%{\small{
\begin{eqnarray}\label{Eq.Gral.Second.Order.Exhaustive}
\begin{array}{r}
f_{1}\left(i,k\right)=D_{k}D_{i}\left(R_{2}+\mathbf{F}_{2}+\indora_{i\geq3}\mathbf{F}_{4}\right)
+D_{i}R_{2}D_{k}\left(\mathbf{F}_{2}+\indora_{k\geq3}\mathbf{F}_{4}\right)
+D_{i}F_{2}D_{k}\left(R_{2}+\indora_{k\geq3}\mathbf{F}_{4}\right)\\
+\indora_{i\geq3}D_{i}\mathbf{F}_{4}D_{k}\left(R_{2}+\mathbf{F}_{2}\right)\\
f_{2}\left(i,k\right)=D_{k}D_{i}\left(R_{1}+\mathbf{F}_{1}+\indora_{i\geq3}\mathbf{F}_{3}\right)+D_{i}R_{1}D_{k}\left(\mathbf{F}_{1}+\indora_{k\geq3}\mathbf{F}_{3}\right)+D_{i}\mathbf{F}_{1}D_{k}\left(R_{1}+\indora_{k\geq3}\mathbf{F}_{3}\right)\\
+\indora_{i\geq3}D_{i}\mathbf{F}_{3}D_{k}\left(R_{1}+\mathbf{F}_{1}\right)\\
f_{3}\left(i,k\right)=D_{k}D_{i}\left(R_{4}+\indora_{i\leq2}\mathbf{F}_{2}+\mathbf{F}_{4}\right)+D_{i}\tilde{R}_{4}D_{k}\left(\indora_{k\leq2}\mathbf{F}_{2}+\mathbf{F}_{4}\right)+D_{i}\mathbf{F}_{4}D_{k}\left(R_{4}+\indora_{k\leq2}\mathbf{F}_{2}\right)\\
+\indora_{i\leq2}D_{i}\mathbf{F}_{2}D_{k}\left(R_{4}+\mathbf{F}_{4}\right)\\
f_{4}\left(i,k\right)=D_{k}D_{i}\left(R_{3}+\indora_{i\leq2}\mathbf{F}_{1}+\mathbf{F}_{3}\right)+D_{i}R_{3}D_{k}\left(\indora_{k\leq2}\mathbf{F}_{1}+\mathbf{F}_{3}\right)+D_{i}\mathbf{F}_{3}D_{k}\left(R_{3}+\indora_{k\leq2}\mathbf{F}_{1}\right)\\
+\indora_{i\leq2}D_{i}\mathbf{F}_{1}D_{k}\left(R_{3}+\mathbf{F}_{3}\right)
\end{array}
\end{eqnarray}%}}

\end{Teo}


\begin{Coro}\label{Coro.Second.Order.Eqs}
Conforming the equations given in (\ref{Eq.Gral.Second.Order.Exhaustive}) the second order moments are obtained solving the linear systems given by  (\ref{System.Second.Order.Moments.uno}). These solutions are 

\begin{eqnarray}\label{Sol.System.Second.Order.Exhaustive}
\begin{array}{ll}
f_{1}\left(1,1\right)=b_{3},&
f_{2}\left(2,2\right)=\frac{b_{2}}{1-b_{1}},\\
f_{1}\left(1,3\right)=a_{4}\left(\frac{b_{2}}{1-b_{1}}\right)+a_{5}K_{12}+K_{3},&
f_{1}\left(1,4\right)=a_{6}\left(\frac{b_{2}}{1-b_{1}}\right)+a_{7}K_{12}+K_{4},\\
f_{1}\left(3,3\right)=a_{8}\left(\frac{b_{2}}{1-b_{1}}\right)+K_{8},&
f_{1}\left(3,4\right)=a_{9}\left(\frac{b_{2}}{1-b_{1}}\right)+K_{9},\\
f_{1}\left(4,4\right)=a_{10}\left(\frac{b_{2}}{1-b_{1}}\right)+a_{5}K_{12}+K_{10},&
f_{2}\left(2,3\right)=a_{14}b_{3}+a_{15}K_{2}+K_{16},\\
f_{2}\left(2,4\right)=a_{16}b_{3}+a_{17}K_{2}+K_{17},&
f_{2}\left(3,3\right)=a_{18}b_{3}+K_{18},\\
f_{2}\left(3,4\right)=a_{19}b_{3}+K_{19},&
f_{2}\left(4,4\right)=a_{20}b_{3}+K_{20},\\
f_{3}\left(3,3\right)=\frac{b_{5}}{1-b_{4}},&
f_{4}\left(4,4\right)=b_{6},\\
f_{3}\left(1,1\right)=a_{21}b_{6}+K_{21},&
f_{3}\left(1,2\right)=a_{22}b_{6}+K_{22},\\
f_{3}\left(1,3\right)=a_{23}b_{6}+a_{24}K_{39}+K_{23},&
f_{3}\left(2,2\right)=a_{25}b_{6}+K_{25},\\
f_{3}\left(2,3\right)=a_{26}b_{6}+a_{27}K_{39}+K_{26},&
f_{4}\left(1,1\right)=a_{31}\left(\frac{b_{5}}{1-b_{4}}\right)+K_{31},\\
f_{4}\left(1,2\right)=a_{32}\left(\frac{b_{5}}{1-b_{4}}\right)+K_{32},&
f_{4}\left(1,4\right)=a_{33}\left(\frac{b_{5}}{1-b_{4}}\right)+a_{34}K_{29}+K_{31},\\
f_{4}\left(2,2\right)=a_{35}\left(\frac{b_{5}}{1-b_{4}}\right)+K_{35},&
f_{4}\left(2,4\right)=a_{36}\left(\frac{b_{5}}{1-b_{4}}\right)+a_{37}K_{29}+K_{37}.
\end{array}
\end{eqnarray}

where
\begin{eqnarray*}
\begin{array}{lll}
N_{1}=a_{2}K_{12}+a_{3}K_{11}+K_{1},&
N_{2}=a_{12}K_{2}+a_{13}K_{5}+K_{15},&
b_{1}=a_{1}a_{11},\\
b_{2}=a_{11}N_{1}+N_{2},&
b_{3}=a_{1}\left(\frac{b_{2}}{1-b_{1}}\right)+N_{1},&
N_{3}=a_{29}K_{39}+a_{30}K_{38}+K_{28},\\
N_{4}=a_{39}K_{29}+a_{40}K_{30}+K_{40},&
b_{4}=a_{28}a_{38},&
b_{5}=a_{28}N_{4}+N_{3},\\
&b_{6}=a_{38}\left(\frac{b_{5}}{1-b_{4}}\right)+N_{4}.&
\end{array}
\end{eqnarray*}

\end{Coro}
The values for the $a_{i}$'s and $K_{i}$ can be found in \textit{Appendix B}. Finally 

\begin{Def}
Let $L_{i}^{*}$ be the number of users at queue $Q_{i}$ when it is polled, then
\begin{eqnarray}
\begin{array}{cc}
\esp\left[L_{i}^{*}\right]=f_{i}\left(i\right), &
Var\left[L_{i}^{*}\right]=f_{i}\left(i,i\right)+\esp\left[L_{i}^{*}\right]-\esp\left[L_{i}^{*}\right]^{2}.
\end{array}
\end{eqnarray}
\end{Def}

%_________________________________________________________________________
%
\subsection{Stability Analysis}
%_________________________________________________________________________
%

We are interested in determine the queue lengths at any time, not just when the server arrives to the queue to start attending according to the exhaustive policy. For this purpose we need to make assumptions over the processes involved in order to guarantee the stability of the Network.



First of all we are going to assume the arrival processes are Poisson, the service time are exponential. In 1973 Disney \cite{Disney} prove that the only stationary system $M/G/1/L$, with renewal departure process are the $M/M/1$ y $M/D/1/1$ systems, also this implies that the output process is Poisson with the same rate of the arrival process. The switchover times has no particular distribution, the only condition they have to satisfy is the first moment finite.

Sigman, Thorison and Wolff \cite{Sigman2} proved that if there is a first regeneration time then exist a non decreasing infinite sequence of regeneration times. With this in consideration we have the following theorem 


\begin{Teo}\label{First.Regeneration.Time.Theorem}
Given a Network of Cyclic Polling Systems (NCPS) conformed by two cyclic polling systems, each of them with $M/M/1$ queues. Both systems are related by users transfer between the queues $Q_{1},Q_{3}$ and $Q_{2},Q_{4}$. Suppose $\tilde{\mu},\hat{\mu}<1$. Let's define the following events for the arrival processes at time $t$: $A_{j}\left(t\right)=\left\{0 \textrm{ arrivals on }Q_{j}\textrm{ at time }t\right\}$, for some $t\geq0$ and queue $Q_{j}$ in the NCPS for $j=1,2,3,4$. Then there exist an non empty interval $I$ such that for $T^{*}\in I$ the $\prob\left\{A_{1}\left(T^{*}\right),A_{2}\left(Tt^{*}\right),
A_{3}\left(T^{*}\right),A_{4}\left(T^{*}\right)|T^{*}\in I\right\}>0$ is satisfied.

\end{Teo}
\begin{proof}

Without of loss of generality we are going to consider as base of the analysis the queue $Q_{1}$ from the first system.

Let's $n\geq1$ cycle for the first system, so let's be $\overline{\tau}_{1}\left(n\right)$ time the server ends attending en queue $Q_{1}$, it means 
$L_{j}\left(\overline{\tau}_{1}\left(n\right)\right)=0$. The server incurrs in a switchover time to traslate to the other queue, which is a random variable whose realitation is $r_{1}\left(n\right)>0$, then we have that $\tau_{2}\left(n\right)=\overline{\tau}_{1}\left(n\right)+r_{1}\left(n\right)$.

Let's be $I_{1}\equiv\left[\overline{\tau}_{1}\left(n\right),\tau_{2}\left(n\right)\right]$ the intreval with length $\xi_{1}=r_{1}\left(n\right)$. Given that the arrival times are exponentials with rate $\tilde{\mu}_{1}=\mu_{1}+\hat{\mu}_{1}$ and the transfer users process from queue $Q_{3}$ are exponentials with rate $\hat{\mu}_{1}$, we have that the event $A_{1}\left(t\right)$ has probability given by 

\begin{equation}
\prob\left\{A_{1}\left(t\right)|t\in I_{1}\left(n\right)\right\}=e^{-\tilde{\mu}_{1}\xi_{1}\left(n\right)}.
\end{equation} 

In the other side, for the queue $Q_{2}$, the time 
$\overline{\tau}_{2}\left(n-1\right)$ is such that 
$L_{2}\left(\overline{\tau}_{2}\left(n-1\right)\right)=0$, it means, it's the time when the queue is emptied by the server en the previous cycle. So we have a second time interval $I_{2}\equiv\left[\overline{\tau}_{2}\left(n-1\right),\tau_{2}\left(n\right)\right]$ so the event $A_{2}\left(t\right)$ has probability

\begin{equation}
\prob\left\{A_{2}\left(t\right)|t\in I_{2}\left(n\right)\right\}=e^{-\tilde{\mu}_{2}\xi_{2}\left(n\right)},
\end{equation} 
with length 
$\xi_{2}\left(n\right)=\tau_{2}\left(n\right)-\overline{\tau}_{2}\left(n-1\right)$. Given the time intervals construction we have that $I_{1}\left(n\right)\subset I_{2}\left(n\right)$, therefore  $\xi_{1}\left(n\right)\leq\xi_{2}\left(n\right)$ so $-\xi_{1}\left(n\right)\geq-\xi_{2}\left(n\right)$ then $-\tilde{\mu}_{2}\xi_{1}\left(n\right)\geq-\tilde{\mu}_{2}\xi_{2}\left(n\right)$ and finally $e^{-\tilde{\mu}_{2}\xi_{1}\left(n\right)}\geq e^{-\tilde{\mu}_{2}\xi_{2}\left(n\right)}$, then

\begin{equation}
\prob\left\{A_{2}\left(t\right)|t\in I_{1}\left(n\right)\right\}\geq
\prob\left\{A_{2}\left(t\right)|t\in I_{2}\left(n\right)\right\}.
\end{equation}

Now we can determine the joint conditional probability on the interval $I_{1}\left(n\right)$
\begin{eqnarray*}
\prob\left\{A_{1}\left(t\right),A_{2}\left(t\right)|t\in I_{1}\left(n\right)\right\}&=&
\prob\left\{A_{1}\left(t\right)|t\in I_{1}\left(n\right)\right\}
\prob\left\{A_{2}\left(t\right)|t\in I_{1}\left(n\right)\right\}\\
&\geq&
\prob\left\{A_{1}\left(t\right)|t\in I_{1}\left(n\right)\right\}
\prob\left\{A_{2}\left(t\right)|t\in I_{2}\left(n\right)\right\}\\
&=&e^{-\tilde{\mu}_{1}\xi_{1}\left(n\right)}e^{-\tilde{\mu}_{2}\xi_{2}\left(n\right)}
=e^{-\left[\tilde{\mu}_{1}\xi_{1}\left(n\right)+\tilde{\mu}_{2}\xi_{2}\left(n\right)\right]}.
\end{eqnarray*}

It means 
\begin{equation}
\prob\left\{A_{1}\left(t\right),A_{2}\left(t\right)|t\in I_{1}\left(n\right)\right\}
=e^{-\left[\tilde{\mu}_{1}\xi_{1}\left(n\right)+\tilde{\mu}_{2}\xi_{2}
\left(n\right)\right]}>0.
\end{equation}

With respect the relation between both systems, there exists some $m\geq1$ such that $\tau_{3}\left(m\right)<\tau_{2}\left(n\right)<\tau_{4}\left(m\right)$ therefore we have the following cases for $\tau_{2}\left(n\right)$:

\begin{multicols}{2}
\begin{itemize}
\item[a)] $\tau_{3}\left(m\right)<\tau_{2}\left(n\right)<\overline{\tau}_{3}\left(m\right)$,

\item[b)] $\overline{\tau}_{3}\left(m\right)<\tau_{2}\left(n\right)
<\tau_{4}\left(m\right)$,

\item[c)] $\tau_{4}\left(m\right)<\tau_{2}\left(n\right)<
\overline{\tau}_{4}\left(m\right)$,

\item[d)] $\overline{\tau}_{4}\left(m\right)<\tau_{2}\left(n\right)
<\tau_{3}\left(m+1\right)$.
\end{itemize}
\end{multicols}

First consider the time interval $I_{3}\left(m\right)\equiv\left[\tau_{3}\left(m\right),\overline{\tau}_{3}\left(m\right)\right]$ such that $\tau_{2}\left(n\right)\in I_{3}\left(m\right)$, with length $\xi_{3}\equiv\overline{\tau}_{3}\left(m\right)-\tau_{3}\left(m\right)$, then we have for the queue $Q_{3}$
\begin{equation}
\prob\left\{A_{3}\left(t\right)|t\in I_{3}\left(m\right)\right\}=e^{-\tilde{\mu}_{3}\xi_{3}\left(m\right)}.
\end{equation} 

whereas for $Q_{4}$ lets consider the time interval $I_{4}\left(m\right)\equiv\left[\tau_{4}\left(m-1\right),\overline{\tau}_{3}\left(m\right)\right]$, then we have that $I_{3}\left(m\right)\subset I_{4}\left(m\right)$, therefore in a similar manner that we have done for $Q_{1}$ and $Q_{2}$ we obtain


\begin{equation}
\prob\left\{A_{4}\left(t\right)|t\in I_{3}\left(m\right)\right\}\geq
\prob\left\{A_{4}\left(t\right)|t\in I_{4}\left(m\right)\right\}
\end{equation}

and

\begin{equation}
\prob\left\{A_{3}\left(t\right),A_{4}\left(t\right)|t\in I_{3}\left(m\right)\right\}\geq
e^{-\left(\tilde{\mu}_{3}\xi_{3}\left(m\right)+\tilde{\mu}_{4}\xi_{4}\left(m\right)\right)}>0.
\end{equation}


For the rest of the cases the demonstration is similar. It means we always can find a time interval where we can guarantee there is no arrivals to the queues in each system with positive probability.  


By construction we have that $I\left(n,m\right)\equiv I_{1}\left(n\right)\cap I_{3}\left(m\right)\neq\emptyset$, then in particular we have the following contentions $I\left(n,m\right)\subseteq I_{1}\left(n\right)$ and $I\left(n,m\right)\subseteq I_{3}\left(m\right)$, therefore if we define $\xi\left(n,m\right)$ as the length of the interval $I\left(n,m\right)$ we have $\xi\left(n,m\right)\leq\xi_{1}\left(n\right)$, $\xi\left(n,m\right)\leq\xi_{3}\left(m\right)$, then $-\xi\left(n,m\right)\geq-\xi_{1}\left(n\right)$ and finally $-\xi\left(n,m\right)\leq-\xi_{3}\left(m\right)$ therefore we have the following
\begin{multicols}{2}
\begin{enumerate}
\item $-\tilde{\mu}_{1}\xi_{n,m}\geq-\tilde{\mu}_{1}\xi_{1}\left(n\right)$,
\item $-\tilde{\mu}_{2}\xi_{n,m}\geq-\tilde{\mu}_{2}\xi_{1}\left(n\right)
\geq-\tilde{\mu}_{2}\xi_{2}\left(n\right)$,
\item $-\tilde{\mu}_{3}\xi_{n,m}\geq-\tilde{\mu}_{3}\xi_{3}\left(m\right)$,
\item $-\tilde{\mu}_{4}\xi_{n,m}\geq-\tilde{\mu}_{4}\xi_{3}\left(m\right)
\geq-\tilde{\mu}_{4}\xi_{4}\left(m\right).$
\end{enumerate}
\end{multicols}

Let's $T^{*}\in I\left(n,m\right)$, then given that in particular $T^{*}\in I_{1}\left(n\right)$, there is no arrivals to the queues $Q_{1}$ and $Q_{2}$, therefore there is no transfer users from $Q_{3}$ and $Q_{4}$, it means, $\tilde{\mu}_{1}=\mu_{1}$, $\tilde{\mu}_{2}=\mu_{2}$, $\tilde{\mu}_{3}=\mu_{3}$, $\tilde{\mu}_{4}=\mu_{4}$, thats it, the events $A_{1}$ and $A_{3}$ are conditionally independent in the interval $I\left(n,m\right)$; the same goes for the events $A_{2}$ and $A_{4}$, therefore we have
%\small{
\begin{eqnarray}
\begin{array}{l}
\prob\left\{A_{1}\left(T^{*}\right),A_{2}\left(T^{*}\right),
A_{3}\left(T^{*}\right),A_{4}\left(T^{*}\right)|T^{*}\in I\left(n,m\right)\right\}
=\prod_{j=1}^{4}\prob\left\{A_{j}\left(T^{*}\right)|T^{*}\in I\left(n,m\right)\right\}\\
\geq\prob\left\{A_{1}\left(T^{*}\right)|T^{*}\in I_{1}\left(n\right)\right\}
\prob\left\{A_{2}\left(T^{*}\right)|T^{*}\in I_{2}\left(n\right)\right\}
\prob\left\{A_{3}\left(T^{*}\right)|T^{*}\in I_{3}\left(m\right)\right\}
\prob\left\{A_{4}\left(T^{*}\right)|T^{*}\in I_{4}\left(m\right)\right\}\\
=e^{-\mu_{1}\xi_{1}\left(n\right)}
e^{-\mu_{2}\xi_{2}\left(n\right)}
e^{-\mu_{3}\xi_{3}\left(m\right)}
e^{-\mu_{4}\xi_{4}\left(m\right)}
=e^{-\left[\tilde{\mu}_{1}\xi_{1}\left(n\right)
+\tilde{\mu}_{2}\xi_{2}\left(n\right)
+\tilde{\mu}_{3}\xi_{3}\left(m\right)
+\tilde{\mu}_{4}\xi_{4}
\left(m\right)\right]}>0.
\end{array}
\end{eqnarray}

Now we only need to prove that for $n\ge1$, there exist an $m\geq1$ such that the cases mentioned before are satisfied: 

\begin{multicols}{2}
\begin{itemize}
\item[a)] $\tau_{3}\left(m\right)<\tau_{2}\left(n\right)<\overline{\tau}_{3}\left(m\right)$,

\item[b)] $\overline{\tau}_{3}\left(m\right)<\tau_{2}\left(n\right)
<\tau_{4}\left(m\right)$,

\item[c)] $\tau_{4}\left(m\right)<\tau_{2}\left(n\right)<
\overline{\tau}_{4}\left(m\right)$,

\item[d)] $\overline{\tau}_{4}\left(m\right)<\tau_{2}\left(n\right)
<\tau_{3}\left(m+1\right)$.
\end{itemize}
\end{multicols}
We only give the proof for the fist case, for the rest the demonstration are similar. Suppose there is no $m\geq1$, with $I_{1}\left(n\right)\cap I_{3}\left(m\right)\neq\emptyset$, it means that for all $m\geq1$, $I_{1}\left(n\right)\cap I_{3}\left(m\right)=\emptyset$, then we have only two cases

\begin{itemize}
\item[a)] $\tau_{2}\left(n\right)\leq\tau_{3}\left(m\right)$: Recall that $\tau_{2}\left(m\right)=\overline{\tau}_{1}+r_{1}\left(m\right)$ 
where each of the random variables are such that $\esp\left[\overline{\tau}_{1}\left(n\right)-\tau_{1}\left(n\right)\right]<\infty$, $\esp\left[R_{1}\right]<\infty$ y $\esp\left[\tau_{3}\left(m\right)\right]<\infty$, which contradicts the fact that there is no such $m\geq1$.

\item[b)] $\tau_{2}\left(n\right)\geq\overline{\tau}_{3}\left(m\right)$: the reasoning is similar to the previous given.

\end{itemize}

\end{proof}


According to the stablished in Sigman, Thorison and Wolff \cite{Sigman2} theorem (\ref{First.Regeneration.Time.Theorem}) allow us to ensure that there is an infinite sequence of regeneration times, let $T_{1},T_{2},\ldots$ considered as the regeneration points, then we have that just like in Takagi \cite{Takagi}, the following definition

\begin{Def}
the interval between two such succesive regeneration points will be called regenerative cycle.
\end{Def}

And for the regeneration points 

\begin{Def}
Let $M_{i}$ be the number of polling cycles in a regenerative cycle.
\end{Def}

\begin{Def}
Considering the $M_{i}$'s, the duration of the $m$-th polling cycle in a regeneration cycle will be denoted by $C_{i}^{(m)}$, for $m=1,2,\ldots,M_{i}$.
\end{Def}

And finally, the mean polling cycle time is defined by

\begin{Def}
\begin{equation}
\esp\left[C_{i}\right]=\frac{\sum_{m=1}^{M_{i}}\esp\left[C_{i}^{(m)}\right]}{\esp\left[M_{i}\right]}
\end{equation}
\end{Def}

\begin{Teo}
The process $\left\{C_{i}:i=1,2,\ldots,M_{i}\right\}$ is a regenerative process. Also there exists a regenerative and stationary process as function of this process.
\end{Teo}

With this in mind let denote by $L_{i}$ the number of users at queue $Q_{i}$ at arbitrary times. Their generating probability function  will be denoted by $Q_{i}\left(z\right)$ which is also given by the time average of $z^{L_{i}\left(t\right)}$ over the regenerative cycled defined before so we have

\begin{eqnarray*}
Q_{i}\left(z\right)&=&\esp\left[z^{L_{i}\left(t\right)}\right]=\frac{\esp\left[\sum_{m=1}^{M_{i}}\sum_{t=\tau_{i}\left(m\right)}^{\tau_{i}\left(m+1\right)-1}z^{L_{i}\left(t\right)}\right]}{\esp\left[\sum_{m=1}^{M_{i}}\tau_{i}\left(m+1\right)-\tau_{i}\left(m\right)\right]}
\end{eqnarray*}

which can be rewritten as

\begin{equation}\label{Eq.Long.Caulquier.Tiempo}
Q_{i}\left(z\right)=\frac{1}{\esp\left[C_{i}\right]}\cdot\frac{1-F_{i}\left(z\right)}{P_{i}\left(z\right)-z}\cdot\frac{\left(1-z\right)P_{i}\left(z\right)}{1-P_{i}\left(z\right)}
\end{equation}

If we define the following
\begin{eqnarray}
\begin{array}{ccc}
S\left(z\right)=1-F\left(z\right),&
T\left(z\right)=z-P\left(z\right),&
U\left(z\right)=1-P\left(z\right).
\end{array}
\end{eqnarray}
then 

\begin{eqnarray}
\esp\left[C_{i}\right]Q\left(z\right)=\frac{\left(z-1\right)S\left(z\right)P\left(z\right)}{T\left(z\right)U\left(z\right)}.
\end{eqnarray}

Where if we define $a_{k}=P\left\{L\left(t\right)=k\right\}$ then 
\begin{eqnarray*}
S\left(z\right)=1-F\left(z\right)=1-\sum_{k=0}^{+\infty}a_{k}z^{k}
\end{eqnarray*}
therefore $S^{'}\left(z\right)=-\sum_{k=1}^{+\infty}ka_{k}z^{k-1}$, with $S^{(1)}\left(1\right)=-\sum_{k=1}^{+\infty}ka_{k}=-\esp\left[L\left(t\right)\right]$,
and $S^{''}\left(z\right)=-\sum_{k=2}^{+\infty}k(k-1)a_{k}z^{k-2}$ so  $S^{(2)}\left(1\right)=-\sum_{k=2}^{+\infty}k(k-1)a_{k}=\esp\left[L\left(L-1\right)\right]$;
in the same way we obtain $S^{'''}\left(z\right)=-\sum_{k=3}^{+\infty}k(k-1)(k-2)a_{k}z^{k-3}$ and $S^{(3)}\left(1\right)=-\sum_{k=3}^{+\infty}k(k-1)(k-2)a_{k}=-\esp\left[L\left(L-1\right)\left(L-2\right)\right]
=-\esp\left[L^{3}\right]+3-\esp\left[L^{2}\right]-2-\esp\left[L\right]$. 

it means

\begin{eqnarray}
\begin{array}{l}
S^{(1)}\left(1\right)=-\esp\left[L\left(t\right)\right],\\ S^{(2)}\left(1\right)=-\esp\left[L\left(L-1\right)\right]
=-\esp\left[L^{2}\right]+\esp\left[L\right],\\
S^{(3)}\left(1\right)=-\esp\left[L\left(L-1\right)\left(L-2\right)\right]
=-\esp\left[L^{3}\right]+3\esp\left[L^{2}\right]-2\esp\left[L\right].
\end{array}
\end{eqnarray}


expanding around $z=1$

\begin{eqnarray*}
S\left(z\right)&=&S\left(1\right)+\frac{S^{'}\left(1\right)}{1!}\left(z-1\right)+\frac{S^{''}\left(1\right)}{2!}\left(z-1\right)^{2}+\frac{S^{'''}\left(1\right)}{3!}\left(z-1\right)^{3}+\ldots+\\
&=&\left(z-1\right)\left\{S^{'}\left(1\right)+\frac{S^{''}\left(1\right)}{2!}\left(z-1\right)+\frac{S^{'''}\left(1\right)}{3!}\left(z-1\right)^{2}+\ldots+\right\}\\
&=&\left(z-1\right)R_{1}\left(z\right)
\end{eqnarray*}
with $R_{1}\left(z\right)\neq0$, given that $R_{1}\left(z\right)=-\esp\left[L\right]$ then

\begin{eqnarray}
R_{1}\left(z\right)&=&S^{'}\left(1\right)+\frac{S^{''}\left(1\right)}{2!}\left(z-1\right)+\frac{S^{'''}\left(1\right)}{3!}\left(z-1\right)^{2}+\frac{S^{iv}\left(1\right)}{4!}\left(z-1\right)^{3}+\ldots+
\end{eqnarray}
Calculating the derivatives and evaluating in $z=1$

\begin{eqnarray}
\begin{array}{l}
R_{1}\left(1\right)=S^{(1)}\left(1\right)=-\esp\left[L\right]\\
R_{1}^{(1)}\left(1\right)=\frac{1}{2}S^{(2)}\left(1\right)=-\frac{1}{2}\esp\left[L^{2}\right]+\frac{1}{2}\esp\left[L\right]\\
R_{1}^{(2)}\left(1\right)=\frac{2}{3!}S^{(3)}\left(1\right)
=-\frac{1}{3}\esp\left[L^{3}\right]+\esp\left[L^{2}\right]-\frac{2}{3}\esp\left[L\right]
\end{array}
\end{eqnarray}

In a similar manner for $T\left(z\right)=z-P\left(z\right)$ can be found an expansion around $z=1$:

\begin{eqnarray*}
T\left(z\right)&=&T\left(1\right)+\frac{T^{'}\left(1\right)}{1!}\left(z-1\right)+\frac{T^{''}\left(1\right)}{2!}\left(z-1\right)^{2}+\frac{T^{'''}\left(1\right)}{3!}\left(z-1\right)^{3}+\ldots+\\
&=&\left(z-1\right)\left\{T^{'}\left(1\right)+\frac{T^{''}\left(1\right)}{2!}\left(z-1\right)+\frac{T^{'''}\left(1\right)}{3!}\left(z-1\right)^{2}+\ldots+\right\}\\
&=&\left(z-1\right)R_{2}\left(z\right)
\end{eqnarray*}

where
\begin{eqnarray}
\begin{array}{l}
T^{(1)}\left(1\right)=-\esp\left[X\left(t\right)\right]=-\mu,\\ T^{(2)}\left(1\right)=-\esp\left[X\left(X-1\right)\right]
=-\esp\left[X^{2}\right]+\esp\left[X\right]=-\esp\left[X^{2}\right]+\mu,\\
T^{(3)}\left(1\right)=-\esp\left[X\left(X-1\right)\left(X-2\right)\right]
=-\esp\left[X^{3}\right]+3\esp\left[X^{2}\right]-2\esp\left[X\right]\\
=-\esp\left[X^{3}\right]+3\esp\left[X^{2}\right]-2\mu.
\end{array}
\end{eqnarray}

therefore $R_{2}\left(1\right)\neq0$, because

\begin{eqnarray}\label{Eq.R2}
R_{2}\left(1\right)=1-\esp\left[X\right]=1-\mu
\end{eqnarray}
then 

\begin{eqnarray}
R_{2}\left(z\right)&=&T^{'}\left(1\right)+\frac{T^{''}\left(1\right)}{2!}\left(z-1\right)+\frac{T^{'''}\left(1\right)}{3!}\left(z-1\right)^{2}+\frac{T^{(iv)}\left(1\right)}{4!}\left(z-1\right)^{3}+\ldots+
\end{eqnarray}
Calculating the derivatives and evaluating $z=1$

\begin{eqnarray}
\begin{array}{l}
R_{2}\left(1\right)=T^{(1)}\left(1\right)=1-\mu\\
R_{2}^{(1)}\left(1\right)=\frac{1}{2}T^{(2)}\left(1\right)=-\frac{1}{2}\esp\left[X^{2}\right]+\frac{1}{2}\mu\\
R_{2}^{(2)}\left(1\right)=\frac{2}{3!}T^{(3)}\left(1\right)
=-\frac{1}{3}\esp\left[X^{3}\right]+\esp\left[X^{2}\right]-\frac{2}{3}\mu
\end{array}
\end{eqnarray}
Finally proceeding in analogous manner for $U\left(z\right)=1-P\left(z\right)$ also can be found an expansion around $z=1$

\begin{eqnarray*}
\begin{array}{l}
U\left(z\right)=U\left(1\right)+\frac{U^{'}\left(1\right)}{1!}\left(z-1\right)+\frac{U^{''}\left(1\right)}{2!}\left(z-1\right)^{2}+\frac{U^{'''}\left(1\right)}{3!}\left(z-1\right)^{3}+\ldots+\\
=\left(z-1\right)\left\{U^{'}\left(1\right)+\frac{U^{''}\left(1\right)}{2!}\left(z-1\right)+\frac{U^{'''}\left(1\right)}{3!}\left(z-1\right)^{2}+\ldots+\right\}
=\left(z-1\right)R_{3}\left(z\right)
\end{array}
\end{eqnarray*}

where

\begin{eqnarray*}
\begin{array}{l}
U^{(1)}\left(1\right)=-\esp\left[X\left(t\right)\right]=-\mu,\\ U^{(2)}\left(1\right)=-\esp\left[X\left(X-1\right)\right]
=-\esp\left[X^{2}\right]+\esp\left[X\right]=-\esp\left[X^{2}\right]+\mu,\\
U^{(3)}\left(1\right)=-\esp\left[X\left(X-1\right)\left(X-2\right)\right]
=-\esp\left[X^{3}\right]+3\esp\left[X^{2}\right]-2\esp\left[X\right]\\
=-\esp\left[X^{3}\right]+3\esp\left[X^{2}\right]-2\mu.
\end{array}
\end{eqnarray*}

Therefore $R_{3}\left(1\right)\neq0$, because

\begin{eqnarray}\label{Eq.R3}
R_{3}\left(1\right)=-\esp\left[X\right]=-\mu
\end{eqnarray}
then

\begin{eqnarray}
R_{3}\left(z\right)&=&U^{'}\left(1\right)+\frac{U^{''}\left(1\right)}{2!}\left(z-1\right)+\frac{U^{'''}\left(1\right)}{3!}\left(z-1\right)^{2}+\frac{U^{(iv)}\left(1\right)}{4!}\left(z-1\right)^{3}+\ldots+
\end{eqnarray}

calculating the derivatives and evaluating in $z=1$

\begin{eqnarray}
\begin{array}{l}
R_{3}\left(1\right)=U^{(1)}\left(1\right)=-\mu\\
R_{3}^{(1)}\left(1\right)=\frac{1}{2}U^{(2)}\left(1\right)=-\frac{1}{2}\esp\left[X^{2}\right]+\frac{1}{2}\mu\\
R_{3}^{(2)}\left(1\right)=\frac{2}{3!}U^{(3)}\left(1\right)
=-\frac{1}{3}\esp\left[X^{3}\right]+\esp\left[X^{2}\right]-\frac{2}{3}\mu
\end{array}
\end{eqnarray}

Then we have that 

\begin{eqnarray}
\esp\left[C_{i}\right]Q\left(z\right)&=&\frac{\left(z-1\right)\left(z-1\right)R_{1}\left(z\right)P\left(z\right)}{\left(z-1\right)R_{2}\left(z\right)\left(z-1\right)R_{3}\left(z\right)}
=\frac{R_{1}\left(z\right)P\left(z\right)}{R_{2}\left(z\right)R_{3}\left(z\right)}\equiv\frac{R_{1}P}{R_{2}R_{3}}
\end{eqnarray}

Calcuating the derivative with respect $z$

\begin{eqnarray}\label{Ec.Primer.Derivada.Q}
\left[\frac{R_{1}\left(z\right)P\left(z\right)}{R_{2}\left(z\right)R_{3}\left(z\right)}\right]^{'}&=&\frac{PR_{2}R_{3}R_{1}^{'}
+R_{1}R_{2}R_{3}P^{'}-R_{3}R_{1}PR_{2}-R_{2}R_{1}PR_{3}^{'}}{\left(R_{2}R_{3}\right)^{2}}
\end{eqnarray}
evauatong in $z=1$
\begin{eqnarray*}
&=&\frac{R_{2}(1)R_{3}(1)R_{1}^{(1)}(1)+R_{1}(1)R_{2}(1)R_{3}(1)P^{'}(1)-R_{3}(1)R_{1}(1)R_{2}(1)^{(1)}-R_{2}(1)R_{1}(1)R_{3}^{'}(1)}{\left(R_{2}(1)R_{3}(1)\right)^{2}}\\
&=&\frac{1}{\left(1-\mu\right)^{2}\mu^{2}}\left\{\left(-\frac{1}{2}\esp L^{2}+\frac{1}{2}\esp L\right)\left(1-\mu\right)\left(-\mu\right)+\left(-\esp L\right)\left(1-\mu\right)\left(-\mu\right)\mu\right.\\
&&\left.-\left(-\frac{1}{2}\esp X^{2}+\frac{1}{2}\mu\right)\left(-\mu\right)\left(-\esp L\right)-\left(1-\mu\right)\left(-\esp L\right)\left(-\frac{1}{2}\esp X^{2}+\frac{1}{2}\mu\right)\right\}\\
&=&\frac{1}{\left(1-\mu\right)^{2}\mu^{2}}\left\{\left(-\frac{1}{2}\esp L^{2}+\frac{1}{2}\esp L\right)\left(\mu^{2}-\mu\right)
+\left(\mu^{2}-\mu^{3}\right)\esp L\right.\\
&&\left.-\mu\esp L\left(-\frac{1}{2}\esp X^{2}+\frac{1}{2}\mu\right)
+\left(\esp L-\mu\esp L\right)\left(-\frac{1}{2}\esp X^{2}+\frac{1}{2}\mu\right)\right\}\\
&=&\frac{1}{\left(1-\mu\right)^{2}\mu^{2}}\left\{-\frac{1}{2}\mu^{2}\esp L^{2}
+\frac{1}{2}\mu\esp L^{2}
+\frac{1}{2}\mu^{2}\esp L
-\mu^{3}\esp L
+\mu\esp L\esp X^{2}
-\frac{1}{2}\esp L\esp X^{2}\right\}\\
&=&\frac{1}{\left(1-\mu\right)^{2}\mu^{2}}\left\{
\frac{1}{2}\mu\esp L^{2}\left(1-\mu\right)
+\esp L\left(\frac{1}{2}-\mu\right)\left(\mu^{2}-\esp X^{2}\right)\right\}\\
&=&\frac{1}{2\mu\left(1-\mu\right)}\esp L^{2}-\frac{\frac{1}{2}-\mu}{\left(1-\mu\right)^{2}\mu^{2}}\sigma^{2}\esp L
\end{eqnarray*}

Then we get (Takagi's formula)

\begin{eqnarray*}
Q^{(1)}=\frac{1}{\esp C}\left\{\frac{1}{2\mu\left(1-\mu\right)}\esp L^{2}-\frac{\frac{1}{2}-\mu}{\left(1-\mu\right)^{2}\mu^{2}}\sigma^{2}\esp L\right\}
\end{eqnarray*}
with

\begin{eqnarray*}
\esp C = \frac{\esp L}{\mu\left(1-\mu\right)}
\end{eqnarray*}
therefore

\begin{eqnarray*}
Q^{(1)}&=&\frac{1}{2}\frac{\esp L^{2}}{\esp L}-\frac{\frac{1}{2}-\mu}{\left(1-\mu\right)\mu}\sigma^{2}
=\frac{\esp L^{2}}{2\esp L}-\frac{\sigma^{2}}{2}\left\{\frac{2\mu-1}{\left(1-\mu\right)\mu}\right\}\\
&=&\frac{\esp L^{2}}{2\esp L}+\frac{\sigma^{2}}{2}\left\{\frac{1}{1-\mu}+\frac{1}{\mu}\right\}
\end{eqnarray*}

while for us

\begin{eqnarray*}
Q^{(1)}=\frac{1}{\mu\left(1-\mu\right)}\frac{\esp L^{2}}{2\esp C}
-\sigma^{2}\frac{\esp L}{2\esp C}\cdot\frac{1-2\mu}{\left(1-\mu\right)^{2}\mu^{2}}.
\end{eqnarray*}

Now, reconsider the equation (\ref{Ec.Primer.Derivada.Q})

\begin{eqnarray*}
\left[\frac{R_{1}\left(z\right)P\left(z\right)}{R_{2}\left(z\right)R_{3}\left(z\right)}\right]^{'}&=&\frac{PR_{2}R_{3}R_{1}^{'}
+R_{1}R_{2}R_{3}P^{'}-R_{3}R_{1}PR_{2}-R_{2}R_{1}PR_{3}^{'}}{\left(R_{2}R_{3}\right)^{2}}
\equiv\frac{F\left(z\right)}{G\left(z\right)}
\end{eqnarray*}

where

\begin{eqnarray}
\begin{array}{l}
F\left(z\right)=PR_{2}R_{3}R_{1}^{'}
+R_{1}R_{2}R_{3}P^{'}-R_{3}R_{1}PR_{2}^{'}-R_{2}R_{1}PR_{3}^{'}\\
G\left(z\right)=R_{2}^{2}R_{3}^{2}\\
G^{2}\left(z\right)=R_{2}^{4}R_{3}^{4}=\left(1-\mu\right)^{4}\mu^{4}
\end{array}
\end{eqnarray}
so 

\begin{eqnarray}
\begin{array}{l}
G^{'}\left(z\right)=2R_{2}R_{3}\left[R_{2}^{'}R_{3}+R_{2}R_{3}^{'}\right]\\
G^{'}\left(1\right)=-2\left(1-\mu\right)\mu\left[\left(-\frac{1}{2}\esp\left[X^{2}\right]+\frac{1}{2}\mu\right)\left(-\mu\right)+\left(1-\mu\right)\left(-\frac{1}{2}\esp\left[X^{2}\right]+\frac{1}{2}\mu\right)\right]\\
F^{'}\left(z\right)=\left[\left(R_{2}R_{3}\right)R_{1}^{''}
-\left(R_{1}R_{3}\right)R_{2}^{''}
-\left(R_{1}R_{2}\right)R_{3}^{''}
-2\left(R_{2}^{'}R_{3}^{'}\right)R_{1}\right]P
+2\left(R_{2}R_{3}\right)R_{1}^{'}P^{'}
+\left(R_{1}R_{2}R_{3}\right)P^{''}.
\end{array}
\end{eqnarray}

Now, let us calculate $F^{'}\left(z\right)G\left(z\right)+F\left(z\right)G^{'}\left(z\right)$:

\begin{eqnarray*}
&&F^{'}\left(z\right)G\left(z\right)+F\left(z\right)G^{'}\left(z\right)=
\left\{\left[\left(R_{2}R_{3}\right)R_{1}^{''}
-\left(R_{1}R_{3}\right)R_{2}^{''}
-\left(R_{1}R_{2}\right)R_{3}^{''}
-2\left(R_{2}^{'}R_{3}^{'}\right)R_{1}\right]P\right.\\
&&\left.+2\left(R_{2}R_{3}\right)R_{1}^{'}P^{'}
+\left(R_{1}R_{2}R_{3}\right)P^{''}\right\}R_{2}^{2}R_{3}^{2}
-\left\{\left[PR_{2}R_{3}R_{1}^{'}+R_{1}R_{2}R_{3}P^{'}
-R_{3}R_{1}PR_{2}^{'}\right.\right.\\
&&\left.\left.
-R_{2}R_{1}PR_{3}^{'}\right]\left[2R_{2}R_{3}\left(R_{2}^{'}R_{3}+R_{2}R_{3}^{'}\right)\right]\right\}
\end{eqnarray*}
now evaluate in $z=1$

\begin{eqnarray*}
F^{'}\left(1\right)G\left(1\right)&+&F\left(1\right)G^{'}\left(1\right)
=\left(1+R_{3}\right)^{3}R_{3}^{3}R_{1}^{''}-\left(1+R_{3}\right)^{2}R_{1}R_{3}^{3}R_{3}^{''}
-\left(1+R_{3}\right)^{3}R_{3}^{2}R_{1}R_{3}^{''}\\
&-&2\left(1+R_{3}\right)^{2}R_{3}^{2}
\left(R_{3}^{'}\right)^{2}\\
&+&2\left(1+R_{3}\right)^{3}R_{3}^{3}R_{1}^{'}P^{'}
+\left(1+R_{3}\right)^{3}R_{3}^{3}R_{1}P^{''}
-2\left(1+R_{3}\right)^{2}R_{3}^{2}\left(1+2R_{3}\right)R_{3}^{'}R_{1}^{'}\\
&-&2\left(1+R_{3}\right)^{2}R_{3}^{2}R_{1}R_{3}^{'}\left(1+2R_{3}\right)P^{'}
+2\left(1+R_{3}\right)\left(1+2R_{3}\right)R_{3}^{3}R_{1}\left(R_{3}^{'}\right)^{2}\\
&+&2\left(1+R_{3}\right)^{2}\left(1+2R_{3}\right)R_{1}R_{3}R_{3}^{'}\\
&=&-\left(1-\mu\right)^{3}\mu^{3}R_{1}^{''}-\left(1-\mu\right)^{2}\mu^{2}R_{1}\left(1-2\mu\right)R_{3}^{''}
-\left(1-\mu\right)^{3}\mu^{3}R_{1}P^{''}\\
&+&2\left(1-\mu\right)\mu^{2}\left[\left(1-2\mu\right)R_{1}-\left(1-\mu\right)\right]\left(R_{3}^{'}\right)^{2}
-2\left(1-\mu\right)^{2}\mu R_{1}\left(1-2\mu\right)R_{3}^{'}\\
&-&2\left(1-\mu\right)^{3}\mu^{4}R_{1}^{'}-2\mu\left(1-\mu\right)\left(1-2\mu\right)R_{3}^{'}R_{1}^{'}
-2\mu^{3}\left(1-\mu\right)^{2}\left(1-2\mu\right)R_{1}R_{1}^{'}
\end{eqnarray*}
therefore
\begin{eqnarray*}
\left[\frac{F\left(z\right)}{G\left(z\right)}\right]^{'}&=&\frac{1}{\mu^{3}\left(1-\mu\right)^{3}}\left\{
-\left(1-\mu\right)^{2}\mu^{2}R_{1}^{''}-\mu\left(1-\mu\right)\left(1-2\mu\right)R_{1}R_{3}^{''}
-\mu^{2}\left(1-\mu\right)^{2}R_{1}P^{''}\right.\\
&&\left.+2\mu\left[\left(1-2\mu\right)R_{1}-\left(1-\mu\right)\right]\left(R_{3}^{'}\right)^{2}
-2\left(1-\mu\right)\left(1-2\mu\right)R_{1}R_{3}^{'}-2\mu^{3}\left(1-\mu\right)^{2}R_{1}^{'}\right.\\
&&\left.-2\left(1-2\mu\right)R_{3}^{'}R_{1}^{'}-2\mu^{2}\left(1-\mu\right)\left(1-2\mu\right)R_{1}R_{1}^{'}\right\}
\end{eqnarray*}
recall that


\begin{eqnarray*}
R_{1}&=&-\esp L\\
R_{3}&=& -\mu\\
R_{1}^{'}&=&-\frac{1}{2}\esp L^{2}+\frac{1}{2}\esp L\\
R_{3}^{'}&=&-\frac{1}{2}\esp X^{2}+\frac{1}{2}\mu\\
R_{1}^{''}&=&-\frac{1}{3}\esp L^{3}+\esp L^{2}-\frac{2}{3}\esp L\\
R_{3}^{''}&=&-\frac{1}{3}\esp X^{3}+\esp X^{2}-\frac{2}{3}\mu\\
R_{1}R_{3}^{'}&=&\frac{1}{2}\esp X^{2}\esp L-\frac{1}{2}\esp X\esp L\\
R_{1}R_{1}^{'}&=&\frac{1}{2}\esp L^{2}\esp L+\frac{1}{2}\esp^{2}L\\
R_{3}^{'}R_{1}^{'}&=&\frac{1}{4}\esp X^{2}\esp L^{2}-\frac{1}{4}\esp X^{2}\esp L-\frac{1}{4}\esp L^{2}\esp X+\frac{1}{4}\esp X\esp L\\
R_{1}R_{3}^{''}&=&\frac{1}{6}\esp X^{3}\esp L^{2}-\frac{1}{6}\esp X^{3}\esp L-\frac{1}{2}\esp L^{2}\esp X^{2}+\frac{1}{2}\esp X^{2}\esp L+\frac{1}{3}\esp X\esp L^{2}-\frac{1}{3}\esp X\esp L\\
R_{1}P^{''}&=&-\esp X^{2}\esp L\\
\left(R_{3}^{'}\right)^{2}&=&\frac{1}{4}\esp^{2}X^{2}-\frac{1}{2}\esp X^{2}\esp X+\frac{1}{4}\esp^{2} X
\end{eqnarray*}









\newpage
%______________________________________________________________________
\subsection{Appendix A: General Case Calculations Exhaustive Policy}\label{Secc.Append.B}
%______________________________________________________________________

%_______________________________________________________________
%\subsection{Calculations}
%_______________________________________________________________


Remember the equations given in equations (\ref{Ec.Gral.Primer.Momento.Ind.Exh}) and (\ref{Eq.Gral.Second.Order.Exhaustive}) for the first and second order partial derivatives respectively. The first moments equations for the queue lengths as before for the times the server arrives to the queue to start attending are obtained solving the system given by $f_{1}\left(i\right)=D_{i}R_{2}+D_{i}\mathbf{F}_{2}+\indora_{i\geq3}D_{i}\mathbf{F}_{4}$, similar expressions of the queues for the rest give us the linear system



\begin{eqnarray*}
\begin{array}{ll}
f_{1}\left(1\right)=r_{2}\tilde{\mu}_{1}+\frac{\tilde{\mu}_{1}}{1-\tilde{\mu}_{2}}f_{2}\left(2\right)+f_{2}\left(1\right),&
f_{1}\left(2\right)=r_{2}\tilde{\mu}_{2},\\
f_{1}\left(3\right)=r_{2}\tilde{\mu}_{3}+\frac{\tilde{\mu}_{3}}{1-\tilde{\mu}_{2}}f_{2}\left(2\right)+F_{3,2}^{(1)}\left(1\right),&
f_{1}\left(4\right)=r_{2}\tilde{\mu}_{4}+\frac{\tilde{\mu}_{4}}{1-\tilde{\mu}_{2}}f_{2}\left(2\right)+F_{4,2}^{(1)}\left(1\right),\\
f_{2}\left(1\right)=r_{1}\tilde{\mu}_{1},&
f_{2}\left(2\right)=r_{1}\tilde{\mu}_{2}+\frac{\tilde{\mu}_{2}}{1-\tilde{\mu}_{1}}f_{1}\left(1\right)+f_{1}\left(2\right),\\
f_{2}\left(3\right)=r_{1}\tilde{\mu}_{3}+\frac{\tilde{\mu}_{3}}{1-\tilde{\mu}_{1}}f_{1}\left(1\right)+F_{3,1}^{(1)}\left(1\right),&
f_{2}\left(4\right)=r_{1}\tilde{\mu}_{4}+\frac{\tilde{\mu}_{4}}{1-\tilde{\mu}_{1}}f_{1}\left(1\right)+F_{4,1}^{(1)}\left(1\right),\\
f_{3}\left(1\right)=\tilde{r}_{4}\tilde{\mu}_{1}+\frac{\tilde{\mu}_{1}}{1-\tilde{\mu}_{4}}f_{4}\left(4\right)+F_{1,4}^{(1)}\left(1\right),&
f_{3}\left(2\right)=\tilde{r}_{4}\tilde{\mu}_{2}+\frac{\tilde{\mu}_{2}}{1-\tilde{\mu}_{4}}f_{4}\left(4\right)+F_{2,4}^{(1)}\left(1\right),\\
f_{3}\left(3\right)=\tilde{r}_{4}\tilde{\mu}_{3}+\frac{\tilde{\mu}_{3}}{1-\tilde{\mu}_{4}}f_{4}\left(4\right)+f_{4}\left(3\right),&
f_{3}\left(4\right)=\tilde{r}_{4}\tilde{\mu}_{4}\\
f_{4}\left(1\right)=\tilde{r}_{3}\tilde{\mu}_{1}+\frac{\tilde{\mu}_{1}}{1-\tilde{\mu}_{3}}f_{3}\left(3\right)+F_{1,3}^{(1)}\left(1\right),&
f_{4}\left(2\right)=\tilde{r}_{3}\mu_{2}+\frac{\tilde{\mu}_{2}}{1-\tilde{\mu}_{3}}f_{3}\left(3\right)+F_{2,3}^{(1)}\left(1\right),\\
f_{4}\left(3\right)=\tilde{r}_{3}\tilde{\mu}_{3},&
f_{4}\left(4\right)=\tilde{r}_{3}\tilde{\mu}_{4}+\frac{\tilde{\mu}_{4}}{1-\tilde{\mu}_{3}}f_{3}\left(3\right)+f_{3}\left(4\right),\\
\end{array}
\end{eqnarray*}

Then we have that if $\mu=\tilde{\mu}_{1}+\tilde{\mu}_{2}<1$, $\hat{\mu}=\tilde{\mu}_{3}+\tilde{\mu}_{4}<1$, $r=r_{1}+r_{2}$ and $\hat{r}=\tilde{r}_{3}+\tilde{r}_{4}$  the system's solution are obtained by direct calculations:

\begin{eqnarray*}
\begin{array}{ll}
f_{2}\left(1\right)=r_{1}\tilde{\mu}_{1},&
f_{1}\left(2\right)=r_{2}\tilde{\mu}_{2},\\
f_{3}\left(4\right)=r_{4}\tilde{\mu}_{4},&
f_{4}\left(3\right)=r_{3}\tilde{\mu}_{3},\\
f_{1}\left(1\right)=r\frac{\tilde{\mu}_{1}\left(1-\tilde{\mu}_{1}\right)}{1-\mu},&
f_{2}\left(2\right)=r\frac{\tilde{\mu}_{2}\left(1-\tilde{\mu}_{2}\right)}{1-\mu},\\
f_{1}\left(3\right)=\tilde{\mu}_{3}\left(r_{2}+\frac{r\tilde{\mu}_{2}}{1-\mu}\right)+F_{3,2}^{(1)}\left(1\right),&
f_{1}\left(4\right)=\tilde{\mu}_{4}\left(r_{2}+\frac{r\tilde{\mu}_{2}}{1-\mu}\right)+F_{4,2}^{(1)}\left(1\right),\\
f_{2}\left(3\right)=\tilde{\mu}_{3}\left(r_{1}+\frac{r\tilde{\mu}_{1}}{1-\tilde{\mu}}\right)+F_{3,1}^{(1)}\left(1\right),&
f_{2}\left(4\right)=\tilde{\mu}_{4}\left(r_{1}+\frac{r\tilde{\mu}_{1}}{1-\mu}\right)+F_{4,,1}^{(1)}\left(1\right),\\
f_{3}\left(1\right)=\tilde{\mu}_{1}\left(r_{4}+\frac{\hat{r}\tilde{\mu}_{4}}{1-\hat{\mu}}\right)+F_{1,4}^{(1)}\left(1\right),&
f_{3}\left(2\right)=\tilde{\mu}_{2}\left(r_{4}+\frac{\hat{r}\tilde{\mu}_{4}}{1-\hat{\mu}}\right)+F_{2,4}^{(1)}\left(1\right),\\
f_{3}\left(3\right)=\hat{r}\frac{\tilde{\mu}_{3}\left(1-\tilde{\mu}_{3}\right)}{1-\hat{\mu}},&
f_{4}\left(1\right)=\tilde{\mu}_{1}\left(r_{3}+\frac{\hat{r}\tilde{\mu}_{3}}{1-\hat{\mu}}\right)+F_{1,3}^{(1)}\left(1\right),\\
f_{4}\left(2\right)=\tilde{\mu}_{2}\left(r_{3}+\frac{\hat{r}\tilde{\mu}_{3}}{1-\hat{\mu}}\right)+F_{2,3}^{(1)}\left(1\right),&
f_{4}\left(4\right)=\hat{r}\frac{\tilde{\mu}_{4}\left(1-\tilde{\mu}_{4}\right)}{1-\hat{\mu}}.
\end{array}
\end{eqnarray*}

Now, developing the equations given in (\ref{Eq.Gral.Second.Order.Exhaustive}) we obtain for instance $f_{1}\left(1,1\right)=\left(\frac{\tilde{\mu}_{1}}{1-\tilde{\mu}_{2}}\right)^{2}f_{2}\left(2,2\right)
+2\frac{\tilde{\mu}_{1}}{1-\tilde{\mu}_{2}}f_{2}\left(2,1\right)
+f_{2}\left(1,1\right)
+\tilde{\mu}_{1}^{2}\left(R_{2}^{(2)}+f_{2}\left(2\right)\theta_{2}^{(2)}\right)
+\tilde{P}_{1}^{(2)}\left(\frac{f_{2}\left(2\right)}{1-\tilde{\mu}_{2}}+r_{2}\right)+2r_{2}\tilde{\mu}_{2}f_{2}\left(1\right)$; similar reasoning lead us the following general expressions

\begin{eqnarray}\label{Eq.Sdo.Orden.Exh.uno}
\begin{array}{l}
f_{1}\left(i,j\right)=\indora_{i=1}f_{2}\left(1,1\right)
+\left[\left(1-\indora_{i=j=3}\right)\indora_{i+j\leq6}\indora_{i\leq j}\frac{\mu_{j}}{1-\tilde{\mu}_{2}}
+\left(1-\indora_{i=j=3}\right)\indora_{i+j\leq6}\indora_{i>j}\frac{\mu_{i}}{1-\tilde{\mu}_{2}}\right.\\
\left.+\indora_{i=1}\frac{\mu_{i}}{1-\tilde{\mu}_{2}}\right]f_{2}\left(1,2\right)+\indora_{i,j\neq2}\left(\frac{1}{1-\tilde{\mu}_{2}}\right)^{2}\mu_{i}\mu_{j}f_{2}\left(2,2\right)
+\left[\indora_{i,j\neq2}\tilde{\theta}_{2}^{(2)}\tilde{\mu}_{i}\tilde{\mu}_{j}
+\indora_{i,j\neq2}\indora_{i=j}\frac{\tilde{P}_{i}^{(2)}}{1-\tilde{\mu}_{2}}\right.\\
\left.+\indora_{i,j\neq2}\indora_{i\neq j}\frac{\tilde{\mu}_{i}\tilde{\mu}_{j}}{1-\tilde{\mu}_{2}}\right]f_{2}\left(2\right)
+\left[r_{2}\tilde{\mu}_{i}
+\indora_{i\geq3}F_{i,2}^{(1)}\right]f_{2}\left(j\right)
+\left[r_{2}\tilde{\mu}_{j}
+\indora_{j\geq3}F_{j,2}^{(1)}\right]f_{2}\left(i\right)\\
+\left[R_{2}^{(2)}
+\indora_{i=j}r_{2}\right]\tilde{\mu}_{i}\mu_{j}+\indora_{j\geq3}F_{j,2}^{(1)}\left[\indora_{j\neq i}F_{i,2}^{(1)}
+r_{2}\tilde{\mu}_{i}\right]
+r_{2}\left[\indora_{i=j}P_{i}^{(2)}
+\indora_{i\geq3}F_{i,2}^{(1)}\tilde{\mu}_{j}\right]\\
+\indora_{i\geq3}\indora_{j=i}F_{i,2}^{(2)}
\end{array}
\end{eqnarray}

in a similar manner we obtain expressions for $f_{2}\left(i,j\right)$, $f_{3}\left(i,j\right)$ and $f_{4}\left(i,j\right)$

for $i,k=1,2,3,4$; from which we obtain the linear equations system
\begin{eqnarray}\label{System.Second.Order.Moments.uno}
\begin{array}{ll}
f_{1}\left(1,1\right)=a_{1}f_{2}\left(2,2\right)
+a_{2}f_{2}\left(2,1\right)
+a_{3}f_{2}\left(1,1\right)
+K_{1},&
f_{1}\left(1,2\right)=K_{2},\\
f_{1}\left(1,3\right)=a_{4}f_{2}\left(2,2\right)+a_{5}f\left(2,1\right)+K_{3},&
f_{1}\left(1,4\right)=a_{6}f_{2}\left(2,2\right)+a_{7}f_{2}\left(2,1\right)+K_{4},\end{array}
\end{eqnarray}
for the rest equations, similar reasoning lead us to a linear system equations whose solutions are described in corolary (\ref{Coro.Second.Order.Eqs}) with coefficients given by, we just show a few of them


%Which can be reduced to solve the system given in (\ref{System.Second.Order.Moments.uno}) and (\ref{System.Second.Order.Moments.dos}).

with values for $a_{i}$ and $K_{i}$  
%{\small{
\begin{eqnarray}\label{Coefficients.Ais.Exh.uno}
\begin{array}{llll}
a_{1}=\left(\frac{\tilde{\mu}_{1}}{1-\tilde{\mu}_{2}}\right)^{2},&
a_{2}=\frac{2\tilde{\mu}_{1}}{1-\tilde{\mu}_{2}},&
a_{3}=1,&
a_{4}=\left(\frac{1}{1-\tilde{\mu}_{2}}\right)^{2}\tilde{\mu}_{1}\tilde{\mu}_{3},\\
a_{5}=\frac{\tilde{\mu}_{3}}{1-\tilde{\mu}_{2}},&
a_{6}=\left(\frac{1}{1-\tilde{\mu}_{2}}\right)^{2}\tilde{\mu}_{1}\tilde{\mu}_{4},&
a_{7}=\frac{\tilde{\mu}_{4}}{1-\tilde{\mu}_{2}},&\\
\end{array}
\end{eqnarray}%}}





\begin{eqnarray}\label{Coefficients.kis.Exh.uno}
\begin{array}{l}
K_{1}=\tilde{\mu}_{1}^{2}\left(R_{2}^{(2)}+f_{2}\left(2\right)\theta_{2}^{(2)}\right)
+\tilde{P}_{1}^{(2)}\left(\frac{f_{2}\left(2\right)}{1-\tilde{\mu}_{2}}+r_{2}\right)
+2r_{2}\tilde{\mu}_{2}f_{2}\left(1\right),\\
K_{2}=\tilde{\mu}_{1}\tilde{\mu}_{2}\left[R_{2}^{(2)}
+r_{2}\right]
+r_{2}\left[\tilde{\mu}_{1}f_{2}\left(2\right)
+\tilde{\mu}_{2}f_{2}\left(1\right)\right],\\
K_{3}=\tilde{\mu}_{1}\tilde{\mu}_{3}\left[R_{2}^{(2)}+r_{2}+f_{2}\left(2\right)\left(\tilde{\theta}_{2}^{(2)}+\frac{1}{1-\tilde{\mu}_{2}}\right)\right]
+r_{2}\tilde{\mu}_{1}\left[F_{3,2}^{(1)}+f_{2}\left(1\right)\right]
+\left[r_{2}\tilde{\mu}_{3}+F_{3,2}^{(1)}\right]f_{2}\left(1\right),\\
K_{4}=\tilde{\mu}_{1}\tilde{\mu}_{4}\left[R_{2}^{(2)}
+r_{2}+f_{2}\left(2\right)\left(\tilde{\theta}_{2}^{(2)}
+\frac{1}{1-\tilde{\mu}_{2}}\right)\right]
+r_{2}\tilde{\mu}_{1}\left[f_{2}\left(4\right)+F_{4,2}^{(1)}\right]
+f_{2}\left(1\right)\left[r_{2}\tilde{\mu}_{4}+F_{4,2}^{(1)}\right],
\end{array}
\end{eqnarray}

\newpage
%______________________________________________________________________
\subsection{Appendix B: Stability Analysis for a NCPS}
%__________________________________________________________________
%
\begin{Teo}
Dada una Red de Sistemas de Visitas C\'iclicas (RSVC), conformada por dos Sistemas de Visitas C\'iclicas (SVC), donde cada uno de ellos consta de dos colas tipo $M/M/1$. Los dos sistemas est\'an comunicados entre s\'i por medio de la transferencia de usuarios entre las colas $Q_{1}\leftrightarrow Q_{3}$ y $Q_{2}\leftrightarrow Q_{4}$. Se definen los eventos para los procesos de arribos al tiempo $t$, $A_{j}\left(t\right)=\left\{0 \textrm{ arribos en }Q_{j}\textrm{ al tiempo }t\right\}$ para alg\'un tiempo $t\geq0$ y $Q_{j}$ la cola $j$-\'esima en la RSVC, para $j=1,2,3,4$.  Existe un intervalo $I\neq\emptyset$ tal que para $T^{*}\in I$, tal que $\prob\left\{A_{1}\left(T^{*}\right),A_{2}\left(Tt^{*}\right),
A_{3}\left(T^{*}\right),A_{4}\left(T^{*}\right)|T^{*}\in I\right\}>0$.
\end{Teo}



\begin{proof}
Sin p\'erdida de generalidad podemos considerar como base del an\'alisis a la cola $Q_{1}$ del primer sistema que conforma la RSVC.\medskip 

Sea $n\geq1$, ciclo en el primer sistema en el que se sabe que $L_{j}\left(\overline{\tau}_{1}\left(n\right)\right)=0$, pues la pol\'itica de servicio con que atienden los servidores es la exhaustiva. Como es sabido, para trasladarse a la siguiente cola, el servidor incurre en un tiempo de traslado $r_{1}\left(n\right)>0$, entonces tenemos que $\tau_{2}\left(n\right)=\overline{\tau}_{1}\left(n\right)+r_{1}\left(n\right)$.\medskip 


Definamos el intervalo $I_{1}\equiv\left[\overline{\tau}_{1}\left(n\right),\tau_{2}\left(n\right)\right]$ de longitud $\xi_{1}=r_{1}\left(n\right)$.

Dado que los tiempos entre arribo son exponenciales con tasa $\tilde{\mu}_{1}=\mu_{1}+\hat{\mu}_{1}$ ($\mu_{1}$ son los arribos a $Q_{1}$ por primera vez al sistema, mientras que $\hat{\mu}_{1}$ son los arribos de traslado procedentes de $Q_{3}$) se tiene que la probabilidad del evento $A_{1}\left(t\right)$ est\'a dada por 

\begin{equation}
\prob\left\{A_{1}\left(t\right)|t\in I_{1}\left(n\right)\right\}=e^{-\tilde{\mu}_{1}\xi_{1}\left(n\right)}.
\end{equation} 


Por otra parte, para la cola $Q_{2}$ el tiempo $\overline{\tau}_{2}\left(n-1\right)$ es tal que $L_{2}\left(\overline{\tau}_{2}\left(n-1\right)\right)=0$, es decir, es el tiempo en que la cola queda totalmente vac\'ia en el ciclo anterior a $n$. \medskip 


Entonces tenemos un sgundo intervalo $I_{2}\equiv\left[\overline{\tau}_{2}\left(n-1\right),\tau_{2}\left(n\right)\right]$. Por lo tanto la probabilidad del evento $A_{2}\left(t\right)$ tiene probabilidad dada por

\begin{eqnarray}
\prob\left\{A_{2}\left(t\right)|t\in I_{2}\left(n\right)\right\}=e^{-\tilde{\mu}_{2}\xi_{2}\left(n\right)},\\
\xi_{2}\left(n\right)=\tau_{2}\left(n\right)-\overline{\tau}_{2}\left(n-1\right)
\end{eqnarray}
%\end{equation} 

%donde $$.

Ahora, dado que $I_{1}\left(n\right)\subset I_{2}\left(n\right)$, se tiene que

\begin{eqnarray*}
\xi_{1}\left(n\right)\leq\xi_{2}\left(n\right)&\Leftrightarrow& -\xi_{1}\left(n\right)\geq-\xi_{2}\left(n\right)
\\
-\tilde{\mu}_{2}\xi_{1}\left(n\right)\geq-\tilde{\mu}_{2}\xi_{2}\left(n\right)&\Leftrightarrow&
e^{-\tilde{\mu}_{2}\xi_{1}\left(n\right)}\geq e^{-\tilde{\mu}_{2}\xi_{2}\left(n\right)}\\
\prob\left\{A_{2}\left(t\right)|t\in I_{1}\left(n\right)\right\}&\geq&
\prob\left\{A_{2}\left(t\right)|t\in I_{2}\left(n\right)\right\}.
\end{eqnarray*}


Entonces se tiene que
\small{
\begin{eqnarray*}
\prob\left\{A_{1}\left(t\right),A_{2}\left(t\right)|t\in I_{1}\left(n\right)\right\}&=&
\prob\left\{A_{1}\left(t\right)|t\in I_{1}\left(n\right)\right\}
\prob\left\{A_{2}\left(t\right)|t\in I_{1}\left(n\right)\right\}\\
&\geq&
\prob\left\{A_{1}\left(t\right)|t\in I_{1}\left(n\right)\right\}
\prob\left\{A_{2}\left(t\right)|t\in I_{2}\left(n\right)\right\}\\
&=&e^{-\tilde{\mu}_{1}\xi_{1}\left(n\right)}e^{-\tilde{\mu}_{2}\xi_{2}\left(n\right)}
=e^{-\left[\tilde{\mu}_{1}\xi_{1}\left(n\right)+\tilde{\mu}_{2}\xi_{2}\left(n\right)\right]}.
\end{eqnarray*}}


Es decir, 

\begin{equation}
\prob\left\{A_{1}\left(t\right),A_{2}\left(t\right)|t\in I_{1}\left(n\right)\right\}
=e^{-\left[\tilde{\mu}_{1}\xi_{1}\left(n\right)+\tilde{\mu}_{2}\xi_{2}
\left(n\right)\right]}>0.
\end{equation}
En lo que respecta a la relaci\'on entre los dos SVC que conforman la RSVC para alg\'un $m\geq1$ se tiene que $\tau_{3}\left(m\right)<\tau_{2}\left(n\right)<\tau_{4}\left(m\right)$ por lo tanto se cumple cualquiera de los siguientes cuatro casos
\begin{itemize}
\item[a)] $\tau_{3}\left(m\right)<\tau_{2}\left(n\right)<\overline{\tau}_{3}\left(m\right)$

\item[b)] $\overline{\tau}_{3}\left(m\right)<\tau_{2}\left(n\right)
<\tau_{4}\left(m\right)$

\item[c)] $\tau_{4}\left(m\right)<\tau_{2}\left(n\right)<
\overline{\tau}_{4}\left(m\right)$

\item[d)] $\overline{\tau}_{4}\left(m\right)<\tau_{2}\left(n\right)
<\tau_{3}\left(m+1\right)$
\end{itemize}


Sea el intervalo $I_{3}\left(m\right)\equiv\left[\tau_{3}\left(m\right),\overline{\tau}_{3}\left(m\right)\right]$ tal que $\tau_{2}\left(n\right)\in I_{3}\left(m\right)$, con longitud de intervalo $\xi_{3}\equiv\overline{\tau}_{3}\left(m\right)-\tau_{3}\left(m\right)$, entonces se tiene que para $Q_{3}$
\begin{equation}
\prob\left\{A_{3}\left(t\right)|t\in I_{3}\left(m\right)\right\}=e^{-\tilde{\mu}_{3}\xi_{3}\left(m\right)}.
\end{equation} 

mientras que para $Q_{4}$ consideremos el intervalo $I_{4}\left(m\right)\equiv\left[\tau_{4}\left(m-1\right),\overline{\tau}_{3}\left(m\right)\right]$, entonces por construcci\'on  $I_{3}\left(m\right)\subset I_{4}\left(m\right)$, por lo tanto


\begin{eqnarray*}
\xi_{3}\left(m\right)\leq\xi_{4}\left(m\right)&\Leftrightarrow& -\xi_{3}\left(m\right)\geq-\xi_{4}\left(m\right)
\\
-\tilde{\mu}_{4}\xi_{3}\left(m\right)\geq-\tilde{\mu}_{4}\xi_{4}\left(m\right)&\Leftrightarrow&
e^{-\tilde{\mu}_{4}\xi_{3}\left(m\right)}\geq e^{-\tilde{\mu}_{4}\xi_{4}\left(n\right)}\\
\prob\left\{A_{4}\left(t\right)|t\in I_{3}\left(m\right)\right\}&\geq&
\prob\left\{A_{4}\left(t\right)|t\in I_{4}\left(m\right)\right\}.
\end{eqnarray*}



Entonces se tiene que
\small{
\begin{eqnarray*}
\prob\left\{A_{3}\left(t\right),A_{4}\left(t\right)|t\in I_{3}\left(m\right)\right\}&=&
\prob\left\{A_{3}\left(t\right)|t\in I_{3}\left(m\right)\right\}
\prob\left\{A_{4}\left(t\right)|t\in I_{3}\left(m\right)\right\}\\
&\geq&
\prob\left\{A_{3}\left(t\right)|t\in I_{3}\left(m\right)\right\}
\prob\left\{A_{4}\left(t\right)|t\in I_{4}\left(m\right)\right\}\\
&=&e^{-\tilde{\mu}_{3}\xi_{3}\left(m\right)}e^{-\tilde{\mu}_{4}\xi_{4}
\left(m\right)}
=e^{-\left(\tilde{\mu}_{3}\xi_{3}\left(m\right)+\tilde{\mu}_{4}\xi_{4}\left(m\right)\right)}.
\end{eqnarray*}}

Es decir, 

\begin{equation}
\prob\left\{A_{3}\left(t\right),A_{4}\left(t\right)|t\in I_{3}\left(m\right)\right\}\geq
e^{-\left(\tilde{\mu}_{3}\xi_{3}\left(m\right)+\tilde{\mu}_{4}\xi_{4}\left(m\right)\right)}>0.
\end{equation}


Sea el intervalo $I_{3}\left(m\right)\equiv\left[\overline{\tau}_{3}\left(m\right),\tau_{4}\left(m\right)\right]$ con longitud $\xi_{3}\equiv\tau_{4}\left(m\right)-\overline{\tau}_{3}\left(m\right)$, entonces se tiene que para $Q_{3}$
\begin{equation}
\prob\left\{A_{3}\left(t\right)|t\in I_{3}\left(m\right)\right\}=e^{-\tilde{\mu}_{3}\xi_{3}\left(m\right)}.
\end{equation} 

mientras que para $Q_{4}$ consideremos el intervalo $I_{4}\left(m\right)\equiv\left[\overline{\tau}_{4}\left(m-1\right),\tau_{4}\left(m\right)\right]$, entonces por construcci\'on  $I_{3}\left(m\right)\subset I_{4}\left(m\right)$, y al igual que en el caso anterior se tiene que 

\begin{eqnarray*}
\xi_{3}\left(m\right)\leq\xi_{4}\left(m\right)&\Leftrightarrow& -\xi_{3}\left(m\right)\geq-\xi_{4}\left(m\right)
\\
-\tilde{\mu}_{4}\xi_{3}\left(m\right)\geq-\tilde{\mu}_{4}\xi_{4}\left(m\right)&\Leftrightarrow&
e^{-\tilde{\mu}_{4}\xi_{3}\left(m\right)}\geq e^{-\tilde{\mu}_{4}\xi_{4}\left(n\right)}\\
\prob\left\{A_{4}\left(t\right)|t\in I_{3}\left(m\right)\right\}&\geq&
\prob\left\{A_{4}\left(t\right)|t\in I_{4}\left(m\right)\right\}.
\end{eqnarray*}


Entonces se tiene que
\small{
\begin{eqnarray*}
\prob\left\{A_{3}\left(t\right),A_{4}\left(t\right)|t\in I_{3}\left(m\right)\right\}&=&
\prob\left\{A_{3}\left(t\right)|t\in I_{3}\left(m\right)\right\}
\prob\left\{A_{4}\left(t\right)|t\in I_{3}\left(m\right)\right\}\\
&\geq&
\prob\left\{A_{3}\left(t\right)|t\in I_{3}\left(m\right)\right\}
\prob\left\{A_{4}\left(t\right)|t\in I_{4}\left(m\right)\right\}\\
&=&e^{-\tilde{\mu}_{3}\xi_{3}\left(m\right)}e^{-\tilde{\mu}_{4}\xi_{4}\left(m\right)}
=e^{-\left(\tilde{\mu}_{3}\xi_{3}\left(m\right)+\tilde{\mu}_{4}\xi_{4}\left(m\right)\right)}.
\end{eqnarray*}}

Es decir, 

\begin{equation}
\prob\left\{A_{3}\left(t\right),A_{4}\left(t\right)|t\in I_{4}\left(m\right)\right\}\geq
e^{-\left(\tilde{\mu}_{3}+\tilde{\mu}_{4}\right)\xi_{3}\left(m\right)}>0.
\end{equation}


Para el intervalo $I_{3}\left(m\right)=\left[\tau_{4}\left(m\right),\overline{\tau}_{4}\left(m\right)\right]$, se tiene que este caso es an\'alogo al caso (a).


Para el intevalo $I_{3}\left(m\right)\equiv\left[\overline{\tau}_{4}\left(m\right),\tau_{4}\left(m+1\right)\right]$, se tiene que es an\'alogo al caso (b).


Por construcci\'on se tiene que $I\left(n,m\right)\equiv I_{1}\left(n\right)\cap I_{3}\left(m\right)\neq\emptyset$,entonces en particular se tienen las contenciones $I\left(n,m\right)\subseteq I_{1}\left(n\right)$ y $I\left(n,m\right)\subseteq I_{3}\left(m\right)$, por lo tanto si definimos $\xi_{n,m}\equiv\ell\left(I\left(n,m\right)\right)$ tenemos que

\begin{eqnarray*}
\xi_{n,m}\leq\xi_{1}\left(n\right)\textrm{ y }\xi_{n,m}\leq\xi_{3}\left(m\right)\textrm{ entonces }\\
-\xi_{n,m}\geq-\xi_{1}\left(n\right)\textrm{ y }-\xi_{n,m}\leq-\xi_{3}\left(m\right)\\
\end{eqnarray*}
por lo tanto tenemos las desigualdades 


\begin{eqnarray*}
\begin{array}{ll}
-\tilde{\mu}_{1}\xi_{n,m}\geq-\tilde{\mu}_{1}\xi_{1}\left(n\right),&
-\tilde{\mu}_{2}\xi_{n,m}\geq-\tilde{\mu}_{2}\xi_{1}\left(n\right)
\geq-\tilde{\mu}_{2}\xi_{2}\left(n\right),\\
-\tilde{\mu}_{3}\xi_{n,m}\geq-\tilde{\mu}_{3}\xi_{3}\left(m\right),&
-\tilde{\mu}_{4}\xi_{n,m}\geq-\tilde{\mu}_{4}\xi_{3}\left(m\right)
\geq-\tilde{\mu}_{4}\xi_{4}\left(m\right).
\end{array}
\end{eqnarray*}

Sea $T^{*}\in I\left(n,m\right)$, entonces dado que en particular $T^{*}\in I_{1}\left(n\right)$, se cumple con probabilidad positiva que no hay arribos a las colas $Q_{1}$ y $Q_{2}$, en consecuencia, tampoco hay usuarios de transferencia para $Q_{3}$ y $Q_{4}$, es decir, $\tilde{\mu}_{1}=\mu_{1}$, $\tilde{\mu}_{2}=\mu_{2}$, $\tilde{\mu}_{3}=\mu_{3}$, $\tilde{\mu}_{4}=\mu_{4}$, es decir, los eventos $Q_{1}$ y $Q_{3}$ son condicionalmente independientes en el intervalo $I\left(n,m\right)$; lo mismo ocurre para las colas $Q_{2}$ y $Q_{4}$, por lo tanto tenemos que
%\small{
\begin{eqnarray}
\begin{array}{l}
\prob\left\{A_{1}\left(T^{*}\right),A_{2}\left(T^{*}\right),
A_{3}\left(T^{*}\right),A_{4}\left(T^{*}\right)|T^{*}\in I\left(n,m\right)\right\}\\
=\prod_{j=1}^{4}\prob\left\{A_{j}\left(T^{*}\right)|T^{*}\in I\left(n,m\right)\right\}\\
\geq\prob\left\{A_{1}\left(T^{*}\right)|T^{*}\in I_{1}\left(n\right)\right\}
\prob\left\{A_{2}\left(T^{*}\right)|T^{*}\in I_{2}\left(n\right)\right\}\\
\prob\left\{A_{3}\left(T^{*}\right)|T^{*}\in I_{3}\left(m\right)\right\}
\prob\left\{A_{4}\left(T^{*}\right)|T^{*}\in I_{4}\left(m\right)\right\}\\
=e^{-\mu_{1}\xi_{1}\left(n\right)}
e^{-\mu_{2}\xi_{2}\left(n\right)}
e^{-\mu_{3}\xi_{3}\left(m\right)}
e^{-\mu_{4}\xi_{4}\left(m\right)}\\
=e^{-\left[\tilde{\mu}_{1}\xi_{1}\left(n\right)
+\tilde{\mu}_{2}\xi_{2}\left(n\right)
+\tilde{\mu}_{3}\xi_{3}\left(m\right)
+\tilde{\mu}_{4}\xi_{4}
\left(m\right)\right]}>0.
\end{array}
\end{eqnarray}


Ahora solo resta demostrar que para $n\ge1$, existe $m\geq1$ tal que se cumplen cualquiera de los cuatro casos arriba mencionados: 

\begin{itemize}
\item[a)] $\tau_{3}\left(m\right)<\tau_{2}\left(n\right)<\overline{\tau}_{3}\left(m\right)$

\item[b)] $\overline{\tau}_{3}\left(m\right)<\tau_{2}\left(n\right)
<\tau_{4}\left(m\right)$

\item[c)] $\tau_{4}\left(m\right)<\tau_{2}\left(n\right)<
\overline{\tau}_{4}\left(m\right)$

\item[d)] $\overline{\tau}_{4}\left(m\right)<\tau_{2}\left(n\right)
<\tau_{3}\left(m+1\right)$
\end{itemize}

Consideremos nuevamente el primer caso. Supongamos que no existe $m\geq1$, tal que $I_{1}\left(n\right)\cap I_{3}\left(m\right)\neq\emptyset$, es decir, para toda $m\geq1$, $I_{1}\left(n\right)\cap I_{3}\left(m\right)=\emptyset$, entonces se tiene que ocurren cualquiera de los dos casos

\begin{itemize}
\item[a)] $\tau_{2}\left(n\right)\leq\tau_{3}\left(m\right)$: Recordemos que $\tau_{2}\left(m\right)=\overline{\tau}_{1}+r_{1}\left(m\right)$ donde cada una de las variables aleatorias son tales que $\esp\left[\overline{\tau}_{1}\left(n\right)-\tau_{1}\left(n\right)\right]<\infty$, $\esp\left[R_{1}\right]<\infty$ y $\esp\left[\tau_{3}\left(m\right)\right]<\infty$, lo cual contradice el hecho de que no exista un ciclo $m\geq1$ que satisfaga la condici\'on deseada.

\item[b)] $\tau_{2}\left(n\right)\geq\overline{\tau}_{3}\left(m\right)$: por un argumento similar al anterior se tiene que no es posible que no exista un ciclo $m\geq1$ tal que satisaface la condici\'on deseada.

\end{itemize}

Para el resto de los casos la demostraci\'on es an\'aloga. Por lo tanto, se tiene que efectivamente existe $m\geq1$ tal que $\tau_{3}\left(m\right)<\tau_{2}\left(n\right)<\tau_{4}\left(m\right)$.
\end{proof}

%_________________________________________________________________________
%
\section{Introduction.1}


%_________________________________________________________________________
%

\section{Preliminaries and notation.1}

%_________________________________________________________________________
%
\section{Networks of cyclic polling systems.1}
\subsection{Description}


\subsection{Stability}


\subsection{Queue lengths at any time}


%_________________________________________________________________________
%
\section{Concluding remarks}



%_________________________________________________________________________
%



%_________________________________________________________________________
%








\section{Primera Versi\'on}
%______________________________________________________________________
\subsection{Introduction.2}
%______________________________________________________________________
A cyclic polling system consists of multiple queues that are served by a single server in cyclic order. Users arrive at each queue according to independent processes, which also are independent of the service times. The server attends each queue according to a service policy previously established. The most commonly service policies studied are the exhaustive, gated and the k-limited. The exhaustive policy consists in attending all users until the queue is emptied. When the server finishes, it moves to the next queue incurring in a switchover time that is an independent and identically distributed random variable. An exhaustive analysis have been made in this subject. For an overview of the literature on polling systems, their applications and standard results we refer to surveys such as: \cite{Boxma, Kleinrock, LevySidi, Semenova, TakagiI, Takagi}. 

Bos and Boon \cite{BosBoon} published a report where they studied a Network of Polling Systems applied to a traffic problem, there they analyzed a network of intersections and followed a path in it. Their objective was to predict if the costumers can pass through the network in a finite time or not. The buffer occupancy method was used in this analysis and simulation techniques were also used to verify the results. It is important to remark that the heavy traffic case was studied in this report, as well as the cyclic case was not considered.

Our main contributions in this work can be summarized as follows: under the assumption of a stable network, we obtain explicit expressions for the queue lengths at the moment of the server's arrival, these results are presented in theorem 1. Futhermore, we derive explicit expressions for the first and second moments of the queue lenghts. In the polling systems literature the determination of the PGF of the joint queue length for the cyclic polling system at the times the server arrives to the queue, has been widely studied, so we present the PGF of the joint queue length of the NCPS. 

We believe these results can be generalized for any numbers of systems and any number of queues for the exhaustive service discipline. From a theoretical perspective, this is interesting, because with the work developed in this article it will be possible to analyze the continuous case considering regenerative arrival processes. From the point of view of applications, the results are useful because they allow us to obtain analytical expressions for the performance measures, and also give us the keys to determine waiting times and queue lengths for any time during the operation of the network. Initially our main goal was studying the system of public transportation, which can be seen as a network consisting of several cyclic polling systems. 

In this work, we study a Network of Cyclic Polling Systems (NCPS) that consists of two cyclic polling systems, each of them conformed by two queues attended by a single server. We apply the buffer occupancy method described by Kleinrock and Takagi \cite{TakagiI}. They use the Probability Generating Function (PGF) of the joint distribution function of the queues lengths at the moment the server starts a visit period in each of the queues that conform the system. We begin our analysis by extending the result of the gambler's ruin to the case of two independent games at the same time. Exploiting the similarities between both cases, we get closed expressions for the first and second moments of the time needed to empty the queue. The main objective is to obtain the PGF of the joint queue length distributions at polling instants for the NCPS in order to determine the second order moments.

The remainder of this paper is structured as follows. In Section 2, we provide a detailed model description and the necessary notation. Section 3 derives the expressions for the first and second order moments of the queue's lengths under the assumption of a steady state system. For the exhaustive policy, in the bidirectional case we present the details of the theorems in \textit{appendix A} and \textit{B}, finally we present closed form expressions for some performance measures.


%______________________________________________________________________
\subsection{Preliminaries and notation}
%______________________________________________________________________

Consider a Network consisting of two cyclic polling systems with two queues each, $Q_{1}, Q_{2}$ for the first system and $\hat{Q}_{1},\hat{Q}_{2}$ for the second one, each of them with infinite-sized buffer. In each system a single server visits the queues in cyclic order, where it applies the exhaustive policy, i.e., when the server polls a queue, it serves all the customers present until the queue becomes empty. This case is ilustrated in \texttt{Figure 1}. 

The second system's users at queue 2, can moves to the first system after being attended, also we assume that the network is open; that is, all customers eventually leave the network. As usually in polling systems theory we assume the arrivals in each queue are Poisson processes from the arrival with independent identical distributed (i.i.d.) interarrival times, their service times are also i.i.d. and finally upon completion of a visit at any queue, the servers incurs in a random switchover time according to an arbitrary distribution.  We define a cycle to be the time interval between two consecutive polling instants, the time period in a cycle during which the server is attending a queue is called a service period. The queues are visited in cyclic order.

Time is slotted with slot size equal to the service time of a fixed costumer, we call the time interval $\left[t,t+1\right]$ the $t$-th slot. The arrival processes are denoted by $X_{1}\left(t\right),X_{2}\left(t\right)$ for the first system and $\hat{X}_{1}\left(t\right)$, $\hat{X}_{2}\left(t\right)$ for the second, the arrival rate at $Q_{i}$ and $\hat{Q}_{i}$ is denoted by $\mu_{i}$ and $\hat{\mu}_{i}$ respectively, with the condition $\mu_{i}<1$ and $\hat{\mu}_{i}<1$. The second system's users pass to the first one according to a process $Y_{2}$, with arrival rate $\tilde{\mu}_{2}$. 

The service time for the queue $Q_{i}$ is a random variable $\tau_{i}$ with process defined by $S_{i}$. In similar manner the switchover period following the service of queue $i$ is an independent random variable $R_{i}$ with general distribution. To determine the length of the queues, i.e., the number of users in the queue at the moment the server arrives we define the process $L_{i}$ and $\hat{L}_{i}$ for the first and second system, respectively, in the sequel we use the buffer occupancy method to obtain the generating function, first and second moments of queue size distributions at polling instants. At each of the queues in the network the number of users is the number of users at the time the server arrives plus the numbers of arrivals during the service time. In order to obtain the joint probability generating function (PGF) for the number or users residing in queue $i$ when the queue is polled in the NCPS, we define for each of the arrival processes $X_{1},X_{2},\hat{X}_{1},\hat{X}_{2},Y_{2}$, and $\tilde{X}_{2}$ with $\tilde{X}_{2}=X_{2}+Y_{2}$, their PGF

\begin{eqnarray*}
\begin{array}{cc}
P_{i}\left(z_{i}\right)=\esp\left[z_{i}^{X_{i}\left(t\right)}\right],&
\hat{P}_{i}\left(w_{i}\right)=\esp\left[w_{i}^{\hat{X}_{i}\left(t\right)}\right]
\end{array}
\end{eqnarray*}
for $i=1,2$, and
\begin{eqnarray*}
\begin{array}{cc}
\check{P}_{2}\left(z_{2}\right)=\esp\left[z_{2}^{Y_{2}\left(t\right)}\right],& \tilde{P}_{2}\left(z_{2}\right)=\esp\left[z_{2}^{\tilde{X}_{2}\left(t\right)}
\right],
\end{array}
\end{eqnarray*}

for $i=1,2$, and
\begin{eqnarray*} 
\begin{array}{cc}
\check{\mu}_{2}=\esp\left[Y_{2}\left(t\right)\right]=\check{P}_{2}^{(1)}
\left(1\right),&
\tilde{\mu}_{2}=\esp\left[\tilde{X}_{2}\left(t\right)\right]
=\tilde{P}_{2}^{(1)}\left(1\right).
\end{array}
\end{eqnarray*} The PGF For the service time is defined by:

\begin{eqnarray*}
\begin{array}{cc}
S_{i}\left(z_{i}\right)=\esp\left[z_{i}^{\overline{\tau}_{i}-\tau_{i}}
\right], &
\hat{S}_{i}\left(w_{i}\right)=\esp\left[w_{i}^{\overline{\zeta}_{i}-\zeta_{i}}\right]
\end{array}
\end{eqnarray*} with first moment 
\begin{eqnarray*}
\begin{array}{cc}
s_{i}=\esp\left[\overline{\tau}_{i}-\tau_{i}\right],&\hat{s}_{i}=\esp\left[\overline{\zeta}_{i}-\zeta_{i}\right]
\end{array}
\end{eqnarray*} for $i=1,2$. In a similar manner the PGF for the switchover time of the server from the moment it ends to attend a queue, to the time of arrival to the next queue is given by 
\begin{eqnarray*}
\begin{array}{cc}
R_{i}\left(z_{i}\right)=\esp\left[z_{1}^{\tau_{i+1}-\overline{\tau}_{i}}\right],&
\hat{R}_{i}\left(w_{i}\right)=\esp\left[w_{i}^{\zeta_{i+1}-\overline{\zeta}_{i}}\right]
\end{array}
\end{eqnarray*} with first moment 

\begin{eqnarray*}
\begin{array}{cc}
r_{i}=\esp\left[\tau_{i+1}-\overline{\tau}_{i}\right],&
\hat{r}_{i}=\esp\left[\zeta_{i+1}-\overline{\zeta}_{i}\right]
\end{array}
\end{eqnarray*} for $i=1,2$. The number of users in the queue at times $\overline{\tau}_{1},\overline{\tau}_{2}, \overline{\zeta}_{1},\overline{\zeta}_{2}$, it's zero, i.e.,
 $L_{i}\left(\overline{\tau_{i}}\right)=0,$ and $\hat{L}_{i}\left(\overline{\zeta_{i}}\right)=0$ for $i=1,2$. Then the number of users in the queue of the second system at the moment the server ends attending in the queue is given by the number of users present at the moment it arrives plus the number of arrivals during the service time, i.e.,
$$\hat{L}_{i}\left(\overline{\tau}_{j}\right)=\hat{L}_{i}\left(\tau_{j}\right)+\hat{X}_{i}\left(\overline{\tau}_{j}-\tau_{j}\right),$$
for $i,j=1,2$, meanwhile for the first system : $$L_{1}\left(\overline{\tau}_{j}\right)=L_{1}\left(\tau_{j}\right)+X_{1}\left(\overline{\tau}_{j}-\tau_{j}\right).$$ Specifically for the second queue of the first system we need to consider the users of transfer becoming from the second queue in the second system while the server its in the other queue attending, it means that this users have been already attended by the server before they can go to the first queue:

\begin{equation}\label{Eq.UsuariosTotalesZ2}
L_{2}\left(\overline{\tau}_{1}\right)=L_{2}\left(\tau_{1}\right)+X_{2}\left(\overline{\tau}_{1}-\tau_{1}\right)+Y_{2}\left(\overline{\tau}_{1}-\tau_{1}\right).
\end{equation}

As is know, the gambler's ruin problem can be used to model the server's busy period in a cyclic polling system, so let $\tilde{L}_{0}\geq0$ be the number of users present at the moment the server arrive to start attending, also let $T$ be the time the server need to attend the users in the queue starting with $\tilde{L}_{0}$ users. Suppose the gambler has two independent and simultaneous moves, such events are independent and identical to each other for each realization. The gain on the $n$-th game is $\tilde{\mathsf{X}}_{n}=\mathsf{X}_{n}+\mathsf{Y}_{n}$ units from which is substracted a playing fee of 1 unit for each move. His PGF is given by $F\left(z\right)=\esp\left[z^{\tilde{L}_{0}}\right]$, futhermore
%$\tilde{\mathrm{X}}$, $\tilde{\mathit{X}}$, $\tilde{\mathcal{X}}$, $\tilde{\mathfrak{X}}$,$\tilde{\mathbb{X}}$,$\tilde{\mathtt{X}}$,$\tilde{\mathsf{X}}$,

$$\tilde{P}\left(z\right)=\esp\left[z^{\tilde{\mathsf{X}}_{n}}\right]=\esp\left[z^{\mathsf{X}_{n}+\mathsf{X}_{n}}\right]=\esp\left[z^{\mathsf{X}_{n}}z^{\mathsf{X}_{n}}\right]=\esp\left[z^{\mathsf{X}_{n}}\right]\esp\left[z^{\mathsf{X}_{n}}\right]=P\left(z\right)\check{P}\left(z\right),$$ with $\tilde{\mu}=\esp\left[\tilde{\mathsf{X}}_{n}\right]=\tilde{P}\left[z\right]<1$. If  $\tilde{L}_{n}$ denotes the capital remaining after the $n$-th game, then $\tilde{L}_{n}=\tilde{L}_{0}+\tilde{\mathsf{X}}_{1}+\tilde{\mathsf{X}}_{2}+\cdots+\tilde{\mathsf{X}}_{n}-2n$. The result that relates the gambler's ruin problem with the busy period of the server it's a generalization of the result given in Takagi \cite{Takagi} chapter 3.

\begin{Prop}
Let's $G_{n}\left(z\right)$ and $G\left(z,w\right)$ defined as in 
(\ref{Eq.3.16.b.2S}), then $G_{n}\left(z\right)=\frac{1}{z}\left[G_{n-1}\left(z\right)-G_{n-1}\left(0\right)\right]\tilde{P}\left(z\right)$. Futhermore $G\left(z,w\right)=\frac{zF\left(z\right)-wP\left(z\right)G\left(0,w\right)}{z-wR\left(z\right)}$, with a unique pole in the unit circle, also the pole is of the form $z=\theta\left(w\right)$ and satisfies 
\begin{multicols}{3}
\begin{itemize}
\item[i)]$\tilde{\theta}\left(1\right)=1$,

\item[ii)] $\tilde{\theta}^{(1)}\left(1\right)=\frac{1}{1-\tilde{\mu}}$,

\item[iii)]
$\tilde{\theta}^{(2)}\left(1\right)=\frac{\tilde{\mu}}{\left(1-\tilde{\mu}\right)^{2}}+\frac{\tilde{\sigma}}{\left(1-\tilde{\mu}\right)^{3}}$.
\end{itemize}
\end{multicols}
\end{Prop}

%______________________________________________________________________
\subsection{Networks of cyclic polling systems}
%______________________________________________________________________
%______________________________________________________________________
\subsection*{Description of the model}
%______________________________________________________________________


In order to model the network of cyclic polling system it's necessary to 
consider the users arrivals to each queue in one of the system, but on times the other system's server arrival, $\zeta_{i}$. In the case of the first system and the server arrives to a queue in the second one: $$F_{i,j}\left(z_{i};\zeta_{j}\right)=\esp\left[z_{i}^{L_{i}\left(\zeta_{j}\right)}\right]=
\sum_{k=0}^{\infty}\prob\left[L_{i}\left(\zeta_{j}\right)
=k\right]z_{i}^{k},$$ for $i,j=1,2$. Now consider the case of the queues in the second system and the server arrive to a queue in the first system $$\hat{F}_{i,j}\left(w_{i};\tau_{j}\right)=\esp\left[w_{i}^{\hat{L}_{i}\left(\tau_{j}\right)}\right] =\sum_{k=0}^{\infty}\prob\left[\hat{L}_{i}\left(\tau_{j}\right)
=k\right]w_{i}^{k},$$ for $i,j=1,2$. With the developed we can define the joint PGF for the second system:
$$\esp\left[w_{1}^{\hat{L}_{1}\left(\tau_{j}\right)}w_{2}^{\hat{L}_{2}\left(\tau_{j}\right)}\right]
=\esp\left[w_{1}^{\hat{L}_{1}\left(\tau_{j}\right)}\right]
\esp\left[w_{2}^{\hat{L}_{2}\left(\tau_{j}\right)}\right]=\hat{F}_{1,j}\left(w_{1};\tau_{j}\right)\hat{F}_{2,j}\left(w_{2};\tau_{j}\right)\equiv\hat{\mathbf{F}}_{j}\left(w_{1},w_{2};\tau_{j}\right).$$
%\end{eqnarray*}

In a similar manner we defin the joint PGF for the first system, and the second system's server:
%\begin{eqnarray*}
$$\esp\left[z_{1}^{L_{1}\left(\zeta_{j}\right)}z_{2}^{L_{2}\left(\zeta_{j}\right)}\right]
=\esp\left[z_{1}^{L_{1}\left(\zeta_{j}\right)}\right]
\esp\left[z_{2}^{L_{2}\left(\zeta_{j}\right)}\right]=F_{1,j}\left(z_{1};\zeta_{j}\right)F_{2,j}\left(z_{2};\zeta_{j}\right)\equiv\mathbf{F}_{j}\left(z_{1},z_{2};\zeta_{j}\right).$$
%\end{eqnarray*}

Now we proceed to determine the joint PGF for the times that the server visit each queue in their corresponding system, i.e., $t=\left\{\tau_{1},\tau_{2},\zeta_{1},\zeta_{2}\right\}$:

\begin{eqnarray}\label{Eq.Conjuntas}
\begin{array}{l}
\mathbf{F}_{j}\left(z_{1},z_{2},w_{1},w_{2}\right)=\esp\left[\prod_{i=1}^{2}z_{i}^{L_{i}\left(\tau_{j}
\right)}\prod_{i=1}^{2}w_{i}^{\hat{L}_{i}\left(\tau_{j}\right)}\right],\\
\hat{\mathbf{F}}_{j}\left(z_{1},z_{2},w_{1},w_{2}\right)=\esp\left[\prod_{i=1}^{2}z_{i}^{L_{i}
\left(\zeta_{j}\right)}\prod_{i=1}^{2}w_{i}^{\hat{L}_{i}\left(\zeta_{j}\right)}\right],
\end{array}
\end{eqnarray} for $j=1,2$. Then with the purpose of find the number of users present in the netwotk when the server ends attending one of the queues in any of the systems we have that

\begin{eqnarray*}
&&\esp\left[z_{1}^{L_{1}\left(\overline{\tau}_{1}\right)}z_{2}^{L_{2}\left(\overline{\tau}_{1}\right)}w_{1}^{\hat{L}_{1}\left(\overline{\tau}_{1}\right)}w_{2}^{\hat{L}_{2}\left(\overline{\tau}_{1}\right)}\right]
=\esp\left[z_{2}^{L_{2}\left(\overline{\tau}_{1}\right)}w_{1}^{\hat{L}_{1}\left(\overline{\tau}_{1}
\right)}w_{2}^{\hat{L}_{2}\left(\overline{\tau}_{1}\right)}\right]\\
&=&\esp\left[z_{2}^{L_{2}\left(\tau_{1}\right)+X_{2}\left(\overline{\tau}_{1}-\tau_{1}\right)+Y_{2}\left(\overline{\tau}_{1}-\tau_{1}\right)}w_{1}^{\hat{L}_{1}\left(\tau_{1}\right)+\hat{X}_{1}\left(\overline{\tau}_{1}-\tau_{1}\right)}w_{2}^{\hat{L}_{2}\left(\tau_{1}\right)+\hat{X}_{2}\left(\overline{\tau}_{1}-\tau_{1}\right)}\right]
\end{eqnarray*}

using the equation (\ref{Eq.UsuariosTotalesZ2}) we have


\begin{eqnarray*}
&=&\esp\left[z_{2}^{L_{2}\left(\tau_{1}\right)}z_{2}^{X_{2}\left(\overline{\tau}_{1}-\tau_{1}\right)}z_{2}^{Y_{2}\left(\overline{\tau}_{1}-\tau_{1}\right)}w_{1}^{\hat{L}_{1}\left(\tau_{1}\right)}w_{1}^{\hat{X}_{1}\left(\overline{\tau}_{1}-\tau_{1}\right)}w_{2}^{\hat{L}_{2}\left(\tau_{1}\right)}w_{2}^{\hat{X}_{2}\left(\overline{\tau}_{1}-\tau_{1}\right)}\right]\\
&=&\esp\left[z_{2}^{L_{2}\left(\tau_{1}\right)}\left\{w_{1}^{\hat{L}_{1}\left(\tau_{1}\right)}w_{2}^{\hat{L}_{2}\left(\tau_{1}\right)}\right\}\left\{z_{2}^{X_{2}\left(\overline{\tau}_{1}-\tau_{1}\right)}
z_{2}^{Y_{2}\left(\overline{\tau}_{1}-\tau_{1}\right)}w_{1}^{\hat{X}_{1}\left(\overline{\tau}_{1}-\tau_{1}\right)}w_{2}^{\hat{X}_{2}\left(\overline{\tau}_{1}-\tau_{1}\right)}\right\}\right]
\end{eqnarray*}

applying the fact that the arrivals processes in the queues in each systems are independent:

$$=\esp\left[z_{2}^{L_{2}\left(\tau_{1}\right)}\left\{z_{2}^{X_{2}\left(\overline{\tau}_{1}-\tau_{1}\right)}z_{2}^{Y_{2}\left(\overline{\tau}_{1}-
\tau_{1}\right)}w_{1}^{\hat{X}_{1}\left(\overline{\tau}_{1}-\tau_{1}\right)}w_{2}^{\hat{X}_{2}\left(\overline{\tau}_{1}-\tau_{1}\right)}\right\}\right]
\esp\left[w_{1}^{\hat{L}_{1}\left(\tau_{1}\right)}w_{2}^{\hat{L}_{2}\left(\tau_{1}\right)}\right]$$ given that the arrival processes in the queues are independent, it's possible to separate the expectation for the arrival processes in $Q_{1}$ and $Q_{2}$ at time $\tau_{1}$, which is the time the server visits $Q_{1}$. Considering
$\tilde{X}_{2}\left(z_{2}\right)=X_{2}\left(z_{2}\right)+Y_{2}\left(z_{2}\right)$ we have


\begin{eqnarray*}
\begin{array}{l}
=\esp\left[z_{2}^{L_{2}\left(\tau_{1}\right)}\left\{z_{2}^{\tilde{X}_{2}\left(\overline{\tau}_{1}-\tau_{1}\right)}w_{1}^{\hat{X}_{1}\left(\overline{\tau}_{1}
-\tau_{1}\right)}
w_{2}^{\hat{X}_{2}\left(\overline{\tau}_{1}-\tau_{1}\right)}\right\}\right]\esp\left[w_{1}^{\hat{L}_{1}\left(\tau_{1}\right)}
w_{2}^{\hat{L}_{2}\left(\tau_{1}\right)}\right]\\
=\esp\left[z_{2}^{L_{2}\left(\tau_{1}\right)}\left\{\tilde{P}_{2}\left(z_{2}\right)
^{\overline{\tau}_{1}-\tau_{1}}\hat{P}_{1}\left(w_{1}\right)^{\overline{\tau}_{1}-
\tau_{1}}\hat{P}_{2}\left(w_{2}\right)^{\overline{\tau}_{1}-\tau_{1}}\right\}\right]
\esp\left[w_{1}^{\hat{L}_{1}\left(\tau_{1}\right)}w_{2}^{\hat{L}_{2}\left(\tau_{1}\right)}\right]\\
=\esp\left[z_{2}^{L_{2}\left(\tau_{1}\right)}\left\{\tilde{P}_{2}\left(z_{2}\right)\hat{P}_{1}\left(w_{1}\right)\hat{P}_{2}\left(w_{2}\right)\right\}^{\overline{\tau}_{1}-\tau_{1}}\right]\esp\left[w_{1}^{\hat{L}_{1}\left(\tau_{1}\right)}w_{2}^{\hat{L}_{2}\left(\tau_{1}\right)}\right]\\
=\esp\left[z_{2}^{L_{2}\left(\tau_{1}\right)}\theta_{1}\left(\tilde{P}_{2}\left(z_{2}\right)\hat{P}_{1}\left(w_{1}\right)\hat{P}_{2}\left(w_{2}\right)\right)
^{L_{1}\left(\tau_{1}\right)}\right]\esp\left[w_{1}^{\hat{L}_{1}\left(\tau_{1}\right)}w_{2}^{\hat{L}_{2}\left(\tau_{1}\right)}\right]\\
=F_{1}\left(\theta_{1}\left(\tilde{P}_{2}\left(z_{2}\right)\hat{P}_{1}\left(w_{1}\right)\hat{P}_{2}\left(w_{2}\right)\right),z_{2}\right)\cdot
\hat{F}_{1}\left(w_{1},w_{2};\tau_{1}\right)\\
\equiv \mathbf{F}_{1}\left(\theta_{1}\left(\tilde{P}_{2}\left(z_{2}\right)\hat{P}_{1}\left(w_{1}\right)\hat{P}_{2}\left(w_{2}\right)\right),z_{2},w_{1},w_{2}\right).
\end{array}
\end{eqnarray*}

The last equalities  are true because the number of arrivals to $\hat{Q}_{2}$ 
during the time interval $\left[\tau_{1},\overline{\tau}_{1}\right]$ still haven't been attended by the server in the system 2, then the users can't pass to the first system through the queue $Q_{2}$. Therefore the number of users switching from $\hat{Q}_{2}$ to $Q_{2}$ during the time interval $\left[\tau_{1},\overline{\tau}_{1}\right]$ depends on the policy of transfer between the two systems, according to the last section
%{\small{
\begin{eqnarray*}\label{Eq.Fs}
\begin{array}{l}
\esp\left[z_{1}^{L_{1}\left(\overline{\tau}_{1}\right)}z_{2}^{L_{2}\left(\overline{\tau}_{1}
\right)}w_{1}^{\hat{L}_{1}\left(\overline{\tau}_{1}\right)}w_{2}^{\hat{L}_{2}\left(
\overline{\tau}_{1}\right)}\right]
=\mathbf{F}_{1}\left(\theta_{1}\left(\tilde{P}_{2}\left(z_{2}\right)
\hat{P}_{1}\left(w_{1}\right)\hat{P}_{2}\left(w_{2}\right)\right),z_{2},w_{1},w_{2}\right)\\
\equiv F_{1}\left(\theta_{1}\left(\tilde{P}_{2}\left(z_{2}\right)\hat{P}_{1}\left(w_{1}\right)\hat{P}_{2}\left(w_{2}\right)\right),z_{2}\right)\hat{F}_{1}\left(w_{1},w_{2};\tau_{1}\right).
\end{array}
\end{eqnarray*}%}}

Using similar reasoning for the rest of the server's arrival times we have that

\begin{eqnarray*}
\esp\left[z_{1}^{L_{1}\left(\overline{\tau}_{2}\right)}z_{2}^{L_{2}\left(\overline{\tau}_{2}\right)}w_{1}^{\hat{L}_{1}\left(\overline{\tau}_{2}\right)}w_{2}^{\hat{L}_{2}\left(\overline{\tau}_{2}\right)}\right]&=&F_{2}\left(z_{1},\tilde{\theta}_{2}\left(P_{1}\left(z_{1}\right)\hat{P}_{1}\left(w_{1}\right)\hat{P}_{2}\left(w_{2}\right)\right)\right)
\hat{F}_{2}\left(w_{1},w_{2};\tau_{2}\right)\\
&\equiv& \mathbf{F}_{2}\left(z_{1},\tilde{\theta}_{2}\left(P_{1}\left(z_{1}\right)\hat{P}_{1}\left(w_{1}\right)\hat{P}_{2}\left(w_{2}\right)\right),w_{1},w_{2}\right),\\
\esp\left[z_{1}^{L_{1}\left(\overline{\zeta}_{1}\right)}z_{2}^{L_{2}\left(\overline{\zeta}_{1}
\right)}w_{1}^{\hat{L}_{1}\left(\overline{\zeta}_{1}\right)}w_{2}^{\hat{L}_{2}\left(\overline{\zeta}_{1}\right)}\right]
&=&F_{1}\left(z_{1},z_{2};\zeta_{1}\right)\hat{F}_{1}\left(\hat{\theta}_{1}\left(P_{1}\left(z_{1}\right)\tilde{P}_{2}\left(z_{2}\right)\hat{P}_{2}\left(w_{2}\right)\right),w_{2}\right)\\
&\equiv&\hat{\mathbf{F}}_{1}\left(z_{1},z_{2},\hat{\theta}_{1}\left(P_{1}\left(z_{1}\right)\tilde{P}_{2}\left(z_{2}\right)\hat{P}_{2}\left(w_{2}\right)\right),w_{2}\right),\\
\esp\left[z_{1}^{L_{1}\left(\overline{\zeta}_{2}\right)}z_{2}^{L_{2}\left(\overline{\zeta}_{2}\right)}w_{1}^{\hat{L}_{1}\left(\overline{\zeta}_{2}\right)}w_{2}^{\hat{L}_{2}\left(\overline{\zeta}_{2}\right)}\right]
&=&F_{2}\left(z_{1},z_{2};\zeta_{2}\right)\hat{F}_{2}\left(w_{1},\hat{\theta}_{2}\left(P_{1}\left(z_{1}\right)\tilde{P}_{2}\left(z_{2}\right)\hat{P}_{1}\left(w_{1}\right)\right)\right)\\
&\equiv&\hat{\mathbf{F}}_{2}\left(z_{1},z_{2},w_{1},\hat{\theta}_{2}\left(P_{1}\left(z_{1}\right)\tilde{P}_{2}\left(z_{2}\right)\hat{P}_{1}\left(w_{1}\right)\right)\right).
\end{eqnarray*}

Now we are in conditions to obtain the recursive equations that model the NCPS. We need to consider the switchover times that the server need to translate from one queue to another and, the number or user presents in the system at the time the server leaves to the queue to start attending the next. Thus far developed, we can find that for the NCPS:

\begin{eqnarray}\label{Recursive.Equations.First.Casse}
\begin{array}{r}
\mathbf{F}_{2}\left(z_{1},z_{2},w_{1},w_{2}\right)=R_{1}\left(P_{1}\left(z_{1}\right)\tilde{P}_{2}
\left(z_{2}\right)\prod_{i=1}^{2}
\hat{P}_{i}\left(w_{i}\right)\right)\mathbf{F}_{1}\left(\theta_{1}\left(\tilde{P}_{2}\left(z_{2}
\right)\hat{P}_{1}\left(w_{1}\right)\hat{P}_{2}\left(w_{2}\right)\right),z_{2},w_{1},w_{2}\right),\\
\mathbf{F}_{1}\left(z_{1},z_{2},w_{1},w_{2}\right)=R_{2}\left(P_{1}\left(z_{1}\right)\tilde{P}_{2}
\left(z_{2}\right)\prod_{i=1}^{2}
\hat{P}_{i}\left(w_{i}\right)\right)\mathbf{F}_{2}\left(z_{1},\tilde{\theta}_{2}\left(P_{1}\left(z_{1}\right)\hat{P}_{1}\left(w_{1}\right)\hat{P}_{2}\left(w_{2}
\right)\right),w_{1},w_{2}\right),\\
\hat{\mathbf{F}}_{2}\left(z_{1},z_{2},w_{1},w_{2}\right)=\hat{R}_{1}\left(P_{1}\left(z_{1}\right)\tilde{P}_{2}\left(z_{2}\right)\prod_{i=1}^{2}
\hat{P}_{i}\left(w_{i}\right)\right)\hat{\mathbf{F}}_{1}\left(z_{1},z_{2},\hat{\theta}_{1}\left(P_{1}\left(z_{1}\right)\tilde{P}_{2}\left(z_{2}\right)\hat{P}_{2}\left(w_{2}
\right)\right),w_{2}\right),\\
\hat{\mathbf{F}}_{1}\left(z_{1},z_{2},w_{1},w_{2}\right)=\hat{R}_{2}\left(P_{1}\left(z_{1}\right)\tilde{P}_{2}\left(z_{2}\right)\prod_{i=1}^{2}
\hat{P}_{i}\left(w_{i}\right)\right)\hat{\mathbf{F}}_{2}\left(z_{1},z_{2},w_{1},\hat{\theta}_{2}\left(P_{1}\left(z_{1}\right)\tilde{P}_{2}\left(z_{2}\right)\hat{P}_{1}\left(w_{1}
\right)\right)\right).
\end{array}
\end{eqnarray}


%_____________________________________________________
%\subsubsection{Server Switchover times}
%_____________________________________________________
It's necessary to give an step ahead, considering the case illustrated in \texttt{Figure 2}, where just like before, the server's switchover times are given by the generals equations
$R_{i}\left(\mathbf{z,w}\right)=R_{i}\left(\tilde{P}_{1}\left(z_{1}\right)
\tilde{P}_{2}\left(z_{2}\right)\tilde{P}_{3}\left(z_{3}\right)
\tilde{P}_{4}\left(z_{4}\right)\right)$, with first order derivatives given by $D_{i}R_{i}=r_{i}\tilde{\mu}_{i}$, and second order partial derivatives $D_{j}D_{i}R_{k}=R_{k}^{(2)}\tilde{\mu}_{i}\tilde{\mu}_{j}+\indora_{i=j}r_{k}P_{i}^{(2)}+\indora_{i\neq j}r_{k}\tilde{\mu}_{i}\tilde{\mu}_{j}$ for any $i,j,k$. According to the equations given before and the queue lengths for the other system's server times, we can obtain general expressions

\begin{eqnarray}\label{Ec.Gral.Primer.Momento.Ind.Exh}
\begin{array}{ll}
D_{j}\mathbf{F}_{i}\left(z_{1},z_{2};\tau_{i+2}\right)=\indora_{j\leq2}F_{j,i+2}^{(1)},&
D_{j}\mathbf{F}_{i}\left(z_{3},z_{4};\tau_{i-2}\right)=\indora_{j\geq3}F_{j,i-2}^{(1)},
\end{array}
\end{eqnarray}

for $i,j=1,2,3,4$; with second order derivatives given by

\begin{eqnarray}\label{Ec.Gral.Segundo.Momento.Ind.Exh}
\begin{array}{l}
D_{j}D_{i}\mathbf{F}_{k}\left(z_{1},z_{2};\tau_{k+2}\right)=\indora_{i\geq3}\indora_{j=i}F_{i,k+2}^{(2)}+\indora_{i\geq 3}\indora_{j\neq i}F_{j,k-2}^{(1)}F_{i,k+2}^{(1)},\\
D_{j}D_{i}\mathbf{F}_{k}\left(z_{3},z_{4};\tau_{k-2}\right)=\indora_{i\geq3}\indora_{j=i}F_{i,k-2}^{(2)}+\indora_{i\geq 3}\indora_{j\neq i}F_{j,k-2}^{(1)}F_{i,k-2}^{(1)}.
\end{array}
\end{eqnarray}


 According with the developed at the moment, we can get the recursive equations which are of the following form

\begin{eqnarray}\label{General.System.Double.Transfer}
\begin{array}{l}
\mathbf{F}_{1}\left(z_{1},z_{2},z_{3},z_{4}\right)=R_{2}\left(\prod_{i=1}^{4}\tilde{P}_{i}\left(z_{i}
\right)\right)\mathbf{F}_{2}\left(z_{1},\tilde{\theta}_{2}\left(\tilde{P}_{1}\left(z_{1}\right)\tilde{P}_{3}\left(z_{3}\right)\tilde{P}_{4}
\left(z_{4}\right)\right),z_{3},z_{4}\right),\\
\mathbf{F}_{2}\left(z_{1},z_{2},z_{3},z_{4}\right)=R_{1}\left(\prod_{i=1}^{4}\tilde{P}_{i}\left(z_{i}
\right)\right)
\mathbf{F}_{1}\left(\tilde{\theta}_{1}\left(\tilde{P}_{2}\left(z_{2}\right)\tilde{P}_{3}\left(z_{3}
\right)\tilde{P}_{4}\left(z_{4}\right)\right),z_{2},z_{3},z_{4}\right),\\
\mathbf{F}_{3}\left(z_{1},z_{2},z_{3},z_{4}\right)=R_{4}\left(\prod_{i=1}^{4}\tilde{P}_{i}\left(z_{i}
\right)\right)\mathbf{F}_{4}\left(z_{1},z_{2},z_{3},\tilde{\theta}_{4}\left(\tilde{P}_{1}\left(z_{1}\right)\tilde{P}_{2}\left(z_{2}\right)\tilde{P}_{3}\left(z_{3}\right)
\right)\right),\\
\mathbf{F}_{4}\left(z_{1},z_{2},z_{3},z_{4}\right)=R_{3}\left(\prod_{i=1}^{4}\tilde{P}_{i}\left(z_{i}
\right)\right)
\mathbf{F}_{3}\left(z_{1},z_{2},\tilde{\theta}_{3}\left(\tilde{P}_{1}\left(z_{1}\right)\tilde{P}_{2}\left(z_{2}\right)\tilde{P}_{4}
\left(z_{4}\right)\right),z_{4}\right).
\end{array}
\end{eqnarray}

So we have the first theorem

\begin{Teo}
Suppose  $\tilde{\mu}=\tilde{\mu}_{1}+\tilde{\mu}_{2}<1$, $\hat{\mu}=\tilde{\mu}_{3}+\tilde{\mu}_{4}<1$, then the number of users in the queues conforming the network of cyclic polling system (\ref{General.System.Double.Transfer}), when the server visit a queue can be found solving the linear system given by equations (\ref{Ec.Primer.Orden.General.Impar}) and (\ref{Ec.Primer.Orden.General.Par}):

\begin{eqnarray}\label{Ec.Primer.Orden.General.Impar}
\begin{array}{l}
f_{j}\left(i\right)=r_{j+1}\tilde{\mu}_{i}
+\indora_{i\neq j+1}f_{j+1}\left(j+1\right)\frac{\tilde{\mu}_{i}}{1-\tilde{\mu}_{j+1}}
+\indora_{i=j}f_{j+1}\left(i\right)
+\indora_{j=1}\indora_{i\geq3}F_{i,j+1}^{(1)}
+\indora_{j=3}\indora_{i\leq2}F_{i,j+1}^{(1)}
\end{array}
\end{eqnarray}
$j=1,3$ and $i=1,2,3,4$.

\begin{eqnarray}\label{Ec.Primer.Orden.General.Par}
\begin{array}{l}
f_{j}\left(i\right)=r_{j-1}\tilde{\mu}_{i}
+\indora_{i\neq j-1}f_{j-1}\left(j-1\right)\frac{\tilde{\mu}_{i}}{1-\tilde{\mu}_{j-1}}
+\indora_{i=j}f_{j-1}\left(i\right)
+\indora_{j=2}\indora_{i\geq3}F_{i,j-1}^{(1)}
+\indora_{j=4}\indora_{i\leq2}F_{i,j-1}^{(1)}
\end{array}
\end{eqnarray}
$j=2,4$ and $i=1,2,3,4$, whose solutions are:
%{\footnotesize{


\begin{eqnarray}
\begin{array}{l}
f_{i}\left(j\right)=\left(\indora_{j=i-1}+\indora_{j=i+1}\right)r_{j}\tilde{\mu}_{j}+\indora_{i=j}\left(\indora_{i\leq2}\frac{r\tilde{\mu}_{i}\left(1-\tilde{\mu}_{i}\right)}{1-\tilde{\mu}}+\indora_{i\geq2}\frac{\hat{r}\tilde{\mu}_{i}\left(1-\tilde{\mu}_{i}\right)}{1-\hat{\mu}}\right)\\
+\indora_{i=1}\indora_{j\geq3}\left(\tilde{\mu}_{j}\left(r_{i+1}+\frac{r\tilde{\mu}_{i+1}}{1-\tilde{\mu}}\right)+F_{j,i+1}^{(1)}\right)
+\indora_{i=3}\indora_{j\geq3}\left(\tilde{\mu}_{j}\left(r_{i+1}+\frac{\hat{r}\tilde{\mu}_{i+1}}{1-\hat{\mu}}\right)+F_{j,i+1}^{(1)}\right)\\
+\indora_{i=2}\indora_{j\leq2}\left(\tilde{\mu}_{j}\left(r_{i-1}+\frac{r\tilde{\mu}_{i-1}}{1-\tilde{\mu}}\right)+F_{j,i-1}^{(1)}\right)
+\indora_{i=4}\indora_{j\leq2}\left(\tilde{\mu}_{j}\left(r_{i-1}+\frac{\hat{r}\tilde{\mu}_{i-1}}{1-\hat{\mu}}\right)+F_{j,i-1}^{(1)}\right).
\end{array}
\end{eqnarray}
\end{Teo}
%______________________________________________________________________

\begin{Teo}
For the system given in (\ref{General.System.Double.Transfer}) we have that the second moments are in their general form

%{\small{
\begin{eqnarray}\label{Eq.Gral.Second.Order.Exhaustive}
\begin{array}{r}
f_{1}\left(i,k\right)=D_{k}D_{i}\left(R_{2}+\mathbf{F}_{2}+\indora_{i\geq3}\mathbf{F}_{4}\right)
+D_{i}R_{2}D_{k}\left(\mathbf{F}_{2}+\indora_{k\geq3}\mathbf{F}_{4}\right)
+D_{i}F_{2}D_{k}\left(R_{2}+\indora_{k\geq3}\mathbf{F}_{4}\right)\\
+\indora_{i\geq3}D_{i}\mathbf{F}_{4}D_{k}\left(R_{2}+\mathbf{F}_{2}\right)\\
f_{2}\left(i,k\right)=D_{k}D_{i}\left(R_{1}+\mathbf{F}_{1}+\indora_{i\geq3}\mathbf{F}_{3}\right)+D_{i}R_{1}D_{k}\left(\mathbf{F}_{1}+\indora_{k\geq3}\mathbf{F}_{3}\right)+D_{i}\mathbf{F}_{1}D_{k}\left(R_{1}+\indora_{k\geq3}\mathbf{F}_{3}\right)\\
+\indora_{i\geq3}D_{i}\mathbf{F}_{3}D_{k}\left(R_{1}+\mathbf{F}_{1}\right)\\
f_{3}\left(i,k\right)=D_{k}D_{i}\left(R_{4}+\indora_{i\leq2}\mathbf{F}_{2}+\mathbf{F}_{4}\right)+D_{i}\tilde{R}_{4}D_{k}\left(\indora_{k\leq2}\mathbf{F}_{2}+\mathbf{F}_{4}\right)+D_{i}\mathbf{F}_{4}D_{k}\left(R_{4}+\indora_{k\leq2}\mathbf{F}_{2}\right)\\
+\indora_{i\leq2}D_{i}\mathbf{F}_{2}D_{k}\left(R_{4}+\mathbf{F}_{4}\right)\\
f_{4}\left(i,k\right)=D_{k}D_{i}\left(R_{3}+\indora_{i\leq2}\mathbf{F}_{1}+\mathbf{F}_{3}\right)+D_{i}R_{3}D_{k}\left(\indora_{k\leq2}\mathbf{F}_{1}+\mathbf{F}_{3}\right)+D_{i}\mathbf{F}_{3}D_{k}\left(R_{3}+\indora_{k\leq2}\mathbf{F}_{1}\right)\\
+\indora_{i\leq2}D_{i}\mathbf{F}_{1}D_{k}\left(R_{3}+\mathbf{F}_{3}\right)
\end{array}
\end{eqnarray}%}}

\end{Teo}


\begin{Coro}\label{Coro.Second.Order.Eqs}
Conforming the equations given in (\ref{Eq.Gral.Second.Order.Exhaustive}) the second order moments are obtained solving the linear systems given by  (\ref{System.Second.Order.Moments.uno}). These solutions are 

\begin{eqnarray}\label{Sol.System.Second.Order.Exhaustive}
\begin{array}{ll}
f_{1}\left(1,1\right)=b_{3},&
f_{2}\left(2,2\right)=\frac{b_{2}}{1-b_{1}},\\
f_{1}\left(1,3\right)=a_{4}\left(\frac{b_{2}}{1-b_{1}}\right)+a_{5}K_{12}+K_{3},&
f_{1}\left(1,4\right)=a_{6}\left(\frac{b_{2}}{1-b_{1}}\right)+a_{7}K_{12}+K_{4},\\
f_{1}\left(3,3\right)=a_{8}\left(\frac{b_{2}}{1-b_{1}}\right)+K_{8},&
f_{1}\left(3,4\right)=a_{9}\left(\frac{b_{2}}{1-b_{1}}\right)+K_{9},\\
f_{1}\left(4,4\right)=a_{10}\left(\frac{b_{2}}{1-b_{1}}\right)+a_{5}K_{12}+K_{10},&
f_{2}\left(2,3\right)=a_{14}b_{3}+a_{15}K_{2}+K_{16},\\
f_{2}\left(2,4\right)=a_{16}b_{3}+a_{17}K_{2}+K_{17},&
f_{2}\left(3,3\right)=a_{18}b_{3}+K_{18},\\
f_{2}\left(3,4\right)=a_{19}b_{3}+K_{19},&
f_{2}\left(4,4\right)=a_{20}b_{3}+K_{20},\\
f_{3}\left(3,3\right)=\frac{b_{5}}{1-b_{4}},&
f_{4}\left(4,4\right)=b_{6},\\
f_{3}\left(1,1\right)=a_{21}b_{6}+K_{21},&
f_{3}\left(1,2\right)=a_{22}b_{6}+K_{22},\\
f_{3}\left(1,3\right)=a_{23}b_{6}+a_{24}K_{39}+K_{23},&
f_{3}\left(2,2\right)=a_{25}b_{6}+K_{25},\\
f_{3}\left(2,3\right)=a_{26}b_{6}+a_{27}K_{39}+K_{26},&
f_{4}\left(1,1\right)=a_{31}\left(\frac{b_{5}}{1-b_{4}}\right)+K_{31},\\
f_{4}\left(1,2\right)=a_{32}\left(\frac{b_{5}}{1-b_{4}}\right)+K_{32},&
f_{4}\left(1,4\right)=a_{33}\left(\frac{b_{5}}{1-b_{4}}\right)+a_{34}K_{29}+K_{31},\\
f_{4}\left(2,2\right)=a_{35}\left(\frac{b_{5}}{1-b_{4}}\right)+K_{35},&
f_{4}\left(2,4\right)=a_{36}\left(\frac{b_{5}}{1-b_{4}}\right)+a_{37}K_{29}+K_{37}.
\end{array}
\end{eqnarray}

where
\begin{eqnarray*}
\begin{array}{lll}
N_{1}=a_{2}K_{12}+a_{3}K_{11}+K_{1},&
N_{2}=a_{12}K_{2}+a_{13}K_{5}+K_{15},&
b_{1}=a_{1}a_{11},\\
b_{2}=a_{11}N_{1}+N_{2},&
b_{3}=a_{1}\left(\frac{b_{2}}{1-b_{1}}\right)+N_{1},&
N_{3}=a_{29}K_{39}+a_{30}K_{38}+K_{28},\\
N_{4}=a_{39}K_{29}+a_{40}K_{30}+K_{40},&
b_{4}=a_{28}a_{38},&
b_{5}=a_{28}N_{4}+N_{3},\\
&b_{6}=a_{38}\left(\frac{b_{5}}{1-b_{4}}\right)+N_{4}.&
\end{array}
\end{eqnarray*}

\end{Coro}
The values for the $a_{i}$'s and $K_{i}$ can be found in \textit{Appendix B}. 
%____________________________________________________________________
\subsection{Additional measures of performance}
%____________________________________________________________________
%\subsubsection*{Exhaustive case}
%____________________________________________________________________
\begin{Def}
Let $L_{i}^{*}$ be the number of users at queue $Q_{i}$ when it is polled, then
\begin{eqnarray}
\begin{array}{cc}
\esp\left[L_{i}^{*}\right]=f_{i}\left(i\right), &
Var\left[L_{i}^{*}\right]=f_{i}\left(i,i\right)+\esp\left[L_{i}^{*}\right]-\esp\left[L_{i}^{*}\right]^{2}.
\end{array}
\end{eqnarray}
\end{Def}

\begin{Def}
The cycle time $C_{i}$ for the queue $Q_{i}$ is the period beginning at the time when it is polled in a cycle and ending at the time when it is polled in the next cycle; it's duration is given by $\tau_{i}\left(m+1\right)-\tau_{i}\left(m\right)$, equivalently $\overline{\tau}_{i}\left(m+1\right)-\overline{\tau}_{i}\left(m\right)$ under steady state assumption.
\end{Def}

\begin{Def}
The intervisit time $I_{i}$ is defined as the period beginning at the time of its service completion in a cycle and ending at the time when it is polled in the next cycle; its duration is given by $\tau_{i}\left(m+1\right)-\overline{\tau}_{i}\left(m\right)$.
\end{Def}

The intervisit time duration $\tau_{i}\left(m+1\right)-\overline{\tau}\left(m\right)$ given the number of users found at queue $Q_{i}$ at time $t=\tau_{i}\left(m+1\right)$ is equal to the number of arrivals during the preceding intervisit time $\left[\overline{\tau}\left(m\right),\tau_{i}\left(m+1\right)\right]$. 

So we have



\begin{eqnarray*}
\esp\left[z_{i}^{L_{i}\left(\tau_{i}\left(m+1\right)\right)}\right]=\esp\left[\left\{P_{i}\left(z_{i}\right)\right\}^{\tau_{i}\left(m+1\right)-\overline{\tau}\left(m\right)}\right]
\end{eqnarray*}

if $I_{i}\left(z\right)=\esp\left[z^{\tau_{i}\left(m+1\right)-\overline{\tau}\left(m\right)}\right]$
we have $F_{i}\left(z\right)=I_{i}\left[P_{i}\left(z\right)\right]$
for $i=1,2$. Futhermore can be proved that

\begin{eqnarray}
\begin{array}{ll}
\esp\left[L_{i}\right]=\mu_{i}\esp\left[I_{i}\right], &
\esp\left[C_{i}\right]=\frac{f_{i}\left(i\right)}{\mu_{i}\left(1-\mu_{i}\right)},\\
\esp\left[S_{i}\right]=\mu_{i}\esp\left[C_{i}\right],&
\esp\left[I_{i}\right]=\left(1-\mu_{i}\right)\esp\left[C_{i}\right],\\
Var\left[L_{i}\right]= \mu_{i}^{2}Var\left[I_{i}\right]+\sigma^{2}\esp\left[I_{i}\right],& 
Var\left[C_{i}\right]=\frac{Var\left[L_{i}^{*}\right]}{\mu_{i}^{2}\left(1-\mu_{i}\right)^{2}},\\
Var\left[S_{i}\right]= \frac{Var\left[L_{i}^{*}\right]}{\left(1-\mu_{i}\right)^{2}}+\frac{\sigma^{2}\esp\left[L_{i}^{*}\right]}{\left(1-\mu_{i}\right)^{3}},&
Var\left[I_{i}\right]= \frac{Var\left[L_{i}^{*}\right]}{\mu_{i}^{2}}-\frac{\sigma_{i}^{2}}{\mu_{i}^{2}}f_{i}\left(i\right).
\end{array}
\end{eqnarray}

Let consider the points when the process $\left[L_{1}\left(1\right),L_{2}\left(1\right),L_{3}\left(1\right),L_{4}\left(1\right)
\right]$ becomes zero at the same time, this points, $T_{1},T_{2},\ldots$ will be denoted as regeneration points, then we have that

\begin{Def}
the interval between two such succesive regeneration points will be called regenerative cycle.
\end{Def}

\begin{Def}
Para $T_{i}$ se define, $M_{i}$, el n\'umero de ciclos de visita a la cola $Q_{l}$, durante el ciclo regenerativo, es decir, $M_{i}$ es un proceso de renovaci\'on.
\end{Def}

\begin{Def}
Para cada uno de los $M_{i}$'s, se definen a su vez la duraci\'on de cada uno de estos ciclos de visita en el ciclo regenerativo, $C_{i}^{(m)}$, para $m=1,2,\ldots,M_{i}$, que a su vez, tambi\'en es n proceso de renovaci\'on.
\end{Def}



Sea la funci\'on generadora de momentos para $L_{i}$, el n\'umero de usuarios en la cola $Q_{i}\left(z\right)$ en cualquier momento, est\'a dada por el tiempo promedio de $z^{L_{i}\left(t\right)}$ sobre el ciclo regenerativo definido anteriormente. Entonces 

\begin{equation}\label{Eq.Longitud.Tiempo.t}
Q_{i}\left(z\right)=\frac{1}{\esp\left[C_{i}\right]}\cdot\frac{1-F_{i}\left(z\right)}{P_{i}\left(z\right)-z}\cdot\frac{\left(1-z\right)P_{i}\left(z\right)}{1-P_{i}\left(z\right)}.
\end{equation}

Es decir, es posible determinar las longitudes de las colas a cualquier tiempo $t$. Entonces, determinando el primer momento es posible ver que


%______________________________________________________________________
\subsection{Numerical Examples}
%_______________________________________________________________________
We are going to consider the following parameters for the network of cyclic polling systems, for the arrival processes the rates are: 0.1 and 0.2 for the first system in queue 1 and 2 respectively, 0.4 and 0.1 for the second system in queue 3 and 4 respectively. The switchover times  from queue to queue in both systems are considered uniformly distributed over the interval $\left[0,3\right]$. So we have the following:
\begin{center}
\begin{tabular}{c|ccccccccc}
$i$&$\mu_{i}$&$\hat{\mu}_{i}$&
$\tilde{\mu}_{i}$&
$P_{i}^{(2)}$&$ \tilde{P}_{i}^{(2)}$&
$r_{i}$&$R_{i}^{(2)}$&$\tilde{\theta}_{i}$&$\tilde{\theta}_{i}^{(2)}$\\\hline
1&$\frac{1}{10}$&$\frac{1}{10}$&0.5&
$\frac{11}{100}$&0.75&1.5&3&2&75\\
2&$\frac{2}{10}$&$\frac{2}{10}$&
0.3&$\frac{24}{100}$&0.39&
1.5&3&1.4285&45\\
3&$\frac{4}{10}$&$\frac{4}{10}$&0.5&
$\frac{56}{100}$&0.75&
1.5&3&2&75\\ 
4&$\frac{1}{10}$&$\frac{1}{10}$&
0.3&$\frac{11}{100}$&
0.39&1.5&3&1.4285&45\\
\end{tabular}

\end{center}

Consider the case where the arrival processes is considered poisson with rate 
then the queues lengths for the network are


\begin{center}\begin{tabular}{c|cccc}
 $f_{i}\left(j\right)$ &  1 &  2 &  3 & 4  \\\hline  
1 &  3.75 & 0.45  & 4.66 & 2.80 \\ 
2 &  0.75 &  3.15 &  5.50 & 3.30\\ 
3 &  4.66 &  2.80 &  3.75 & 0.45\\ 
4 &  5.50 &  3.30 &  0.75 & 3.15
\end{tabular}\end{center}

and the second order moments are 

\begin{center}\begin{tabular}{c|cccc}
 $f_{i}\left(i,i\right)$ & 1& 2 &  3 & 4 \\\hline  
$i$ &  79.5346&40.06249&92.81625&70.65135
\end{tabular}\end{center}
%\newpage
here we show the numerical results for the measure of performance


\begin{center}\begin{tabular}{c|ccccccc}
$i$&$var\left[L_{i}^{*}\right]$&$\esp\left[S_{i}\right]$&
$var\left[S_{i}\right]$&$\esp\left[I_{i}\right]$&
$var\left[I_{i}\right]$&$\esp\left[C_{i}\right]$&
$var\left[C_{i}\right]$\\\hline
1&69.22&7.5&
291.88&7.5&
269.38&15&
1107.55\\
2&33.28&4.5&
70.69&10.5&
359.38&15&
754.875\\
3&82.50&7.5&
345.01&7.5&
322.51&15&
1320.06\\
4&63.87&4.5&
133.12&10.5&
699.26&15&
1448.50
\end{tabular}\end{center}
%\newpage
%______________________________________________________________________
%\section{Concluding Remarks}
%______________________________________________________________________


Sean $T_{1},T_{2},\ldots$ los puntos donde las longitudes de las colas de la red de sistemas de visitas c\'iclicas son cero simult\'aneamente, cuando la cola $Q_{j}$ es visitada por el servidor para dar servicio, es decir, $L_{1}\left(T_{i}\right)=0,L_{2}\left(T_{i}\right)=0,\hat{L}_{1}\left(T_{i}\right)=0$ y $\hat{L}_{2}\left(T_{i}\right)=0$, a estos puntos se les denominar\'a puntos regenerativos. Sea la funci\'on generadora de momentos para $L_{i}$, el n\'umero de usuarios en la cola $Q_{i}\left(z\right)$ en cualquier momento, est\'a dada por el tiempo promedio de $z^{L_{i}\left(t\right)}$ sobre el ciclo regenerativo definido anteriormente:

\begin{eqnarray*}
Q_{i}\left(z\right)&=&\esp\left[z^{L_{i}\left(t\right)}\right]=\frac{\esp\left[\sum_{m=1}^{M_{i}}\sum_{t=\tau_{i}\left(m\right)}^{\tau_{i}\left(m+1\right)-1}z^{L_{i}\left(t\right)}\right]}{\esp\left[\sum_{m=1}^{M_{i}}\tau_{i}\left(m+1\right)-\tau_{i}\left(m\right)\right]}
\end{eqnarray*}

$M_{i}$ es un tiempo de paro en el proceso regenerativo con $\esp\left[M_{i}\right]<\infty$\footnote{En Stidham\cite{Stidham} y Heyman  se muestra que una condici\'on suficiente para que el proceso regenerativo 
estacionario sea un procesoo estacionario es que el valor esperado del tiempo del ciclo regenerativo sea finito, es decir: $\esp\left[\sum_{m=1}^{M_{i}}C_{i}^{(m)}\right]<\infty$, como cada $C_{i}^{(m)}$ contiene intervalos de r\'eplica positivos, se tiene que $\esp\left[M_{i}\right]<\infty$, adem\'as, como $M_{i}>0$, se tiene que la condici\'on anterior es equivalente a tener que $\esp\left[C_{i}\right]<\infty$,
por lo tanto una condici\'on suficiente para la existencia del proceso regenerativo est\'a dada por $\sum_{k=1}^{N}\mu_{k}<1.$}, se sigue del lema de Wald que:


\begin{eqnarray*}
\esp\left[\sum_{m=1}^{M_{i}}\sum_{t=\tau_{i}\left(m\right)}^{\tau_{i}\left(m+1\right)-1}z^{L_{i}\left(t\right)}\right]&=&\esp\left[M_{i}\right]\esp\left[\sum_{t=\tau_{i}\left(m\right)}^{\tau_{i}\left(m+1\right)-1}z^{L_{i}\left(t\right)}\right]\\
\esp\left[\sum_{m=1}^{M_{i}}\tau_{i}\left(m+1\right)-\tau_{i}\left(m\right)\right]&=&\esp\left[M_{i}\right]\esp\left[\tau_{i}\left(m+1\right)-\tau_{i}\left(m\right)\right]
\end{eqnarray*}

por tanto se tiene que


\begin{eqnarray*}
Q_{i}\left(z\right)&=&\frac{\esp\left[\sum_{t=\tau_{i}\left(m\right)}^{\tau_{i}\left(m+1\right)-1}z^{L_{i}\left(t\right)}\right]}{\esp\left[\tau_{i}\left(m+1\right)-\tau_{i}\left(m\right)\right]}
\end{eqnarray*}

observar que el denominador es simplemente la duraci\'on promedio del tiempo del ciclo.


Haciendo las siguientes sustituciones en la ecuaci\'on (\ref{Corolario2}): $n\rightarrow t-\tau_{i}\left(m\right)$, $T \rightarrow \overline{\tau}_{i}\left(m\right)-\tau_{i}\left(m\right)$, $L_{n}\rightarrow L_{i}\left(t\right)$ y $F\left(z\right)=\esp\left[z^{L_{0}}\right]\rightarrow F_{i}\left(z\right)=\esp\left[z^{L_{i}\tau_{i}\left(m\right)}\right]$, se puede ver que

\begin{eqnarray}\label{Eq.Arribos.Primera}
\esp\left[\sum_{n=0}^{T-1}z^{L_{n}}\right]=
\esp\left[\sum_{t=\tau_{i}\left(m\right)}^{\overline{\tau}_{i}\left(m\right)-1}z^{L_{i}\left(t\right)}\right]
=z\frac{F_{i}\left(z\right)-1}{z-P_{i}\left(z\right)}
\end{eqnarray}

Por otra parte durante el tiempo de intervisita para la cola $i$, $L_{i}\left(t\right)$ solamente se incrementa de manera que el incremento por intervalo de tiempo est\'a dado por la funci\'on generadora de probabilidades de $P_{i}\left(z\right)$, por tanto la suma sobre el tiempo de intervisita puede evaluarse como:

\begin{eqnarray*}
\esp\left[\sum_{t=\tau_{i}\left(m\right)}^{\tau_{i}\left(m+1\right)-1}z^{L_{i}\left(t\right)}\right]&=&\esp\left[\sum_{t=\tau_{i}\left(m\right)}^{\tau_{i}\left(m+1\right)-1}\left\{P_{i}\left(z\right)\right\}^{t-\overline{\tau}_{i}\left(m\right)}\right]=\frac{1-\esp\left[\left\{P_{i}\left(z\right)\right\}^{\tau_{i}\left(m+1\right)-\overline{\tau}_{i}\left(m\right)}\right]}{1-P_{i}\left(z\right)}\\
&=&\frac{1-I_{i}\left[P_{i}\left(z\right)\right]}{1-P_{i}\left(z\right)}
\end{eqnarray*}
por tanto

\begin{eqnarray*}
\esp\left[\sum_{t=\tau_{i}\left(m\right)}^{\tau_{i}\left(m+1\right)-1}z^{L_{i}\left(t\right)}\right]&=&
\frac{1-F_{i}\left(z\right)}{1-P_{i}\left(z\right)}
\end{eqnarray*}

Por lo tanto

\begin{eqnarray*}
Q_{i}\left(z\right)&=&\frac{\esp\left[\sum_{t=\tau_{i}\left(m\right)}^{\tau_{i}
\left(m+1\right)-1}z^{L_{i}\left(t\right)}\right]}{\esp\left[\tau_{i}\left(m+1\right)-\tau_{i}\left(m\right)\right]}\\
&=&\frac{1}{\esp\left[\tau_{i}\left(m+1\right)-\tau_{i}\left(m\right)\right]}
\left\{
\esp\left[\sum_{t=\tau_{i}\left(m\right)}^{\overline{\tau}_{i}\left(m\right)-1}
z^{L_{i}\left(t\right)}\right]
+\esp\left[\sum_{t=\overline{\tau}_{i}\left(m\right)}^{\tau_{i}\left(m+1\right)-1}
z^{L_{i}\left(t\right)}\right]\right\}\\
&=&\frac{1}{\esp\left[\tau_{i}\left(m+1\right)-\tau_{i}\left(m\right)\right]}
\left\{
z\frac{F_{i}\left(z\right)-1}{z-P_{i}\left(z\right)}+\frac{1-F_{i}\left(z\right)}
{1-P_{i}\left(z\right)}
\right\}
\end{eqnarray*}

es decir

\begin{equation}
Q_{i}\left(z\right)=\frac{1}{\esp\left[C_{i}\right]}\cdot\frac{1-F_{i}\left(z\right)}{P_{i}\left(z\right)-z}\cdot\frac{\left(1-z\right)P_{i}\left(z\right)}{1-P_{i}\left(z\right)}
\end{equation}




%______________________________________________________________________
\section{General Case Calculations Exhaustive Policy}\label{Secc.Append.B}
%______________________________________________________________________

%_______________________________________________________________
%\subsection{Calculations}
%_______________________________________________________________


Remember the equations given in equations (\ref{Ec.Gral.Primer.Momento.Ind.Exh}) and (\ref{Eq.Gral.Second.Order.Exhaustive}) for the first and second order partial derivatives respectively. The first moments equations for the queue lengths as before for the times the server arrives to the queue to start attending are obtained solving the system given by $f_{1}\left(i\right)=D_{i}R_{2}+D_{i}\mathbf{F}_{2}+\indora_{i\geq3}D_{i}\mathbf{F}_{4}$, similar expressions of the queues for the rest give us the linear system



\begin{eqnarray*}
\begin{array}{ll}
f_{1}\left(1\right)=r_{2}\tilde{\mu}_{1}+\frac{\tilde{\mu}_{1}}{1-\tilde{\mu}_{2}}f_{2}\left(2\right)+f_{2}\left(1\right),&
f_{1}\left(2\right)=r_{2}\tilde{\mu}_{2},\\
f_{1}\left(3\right)=r_{2}\tilde{\mu}_{3}+\frac{\tilde{\mu}_{3}}{1-\tilde{\mu}_{2}}f_{2}\left(2\right)+F_{3,2}^{(1)}\left(1\right),&
f_{1}\left(4\right)=r_{2}\tilde{\mu}_{4}+\frac{\tilde{\mu}_{4}}{1-\tilde{\mu}_{2}}f_{2}\left(2\right)+F_{4,2}^{(1)}\left(1\right),\\
f_{2}\left(1\right)=r_{1}\tilde{\mu}_{1},&
f_{2}\left(2\right)=r_{1}\tilde{\mu}_{2}+\frac{\tilde{\mu}_{2}}{1-\tilde{\mu}_{1}}f_{1}\left(1\right)+f_{1}\left(2\right),\\
f_{2}\left(3\right)=r_{1}\tilde{\mu}_{3}+\frac{\tilde{\mu}_{3}}{1-\tilde{\mu}_{1}}f_{1}\left(1\right)+F_{3,1}^{(1)}\left(1\right),&
f_{2}\left(4\right)=r_{1}\tilde{\mu}_{4}+\frac{\tilde{\mu}_{4}}{1-\tilde{\mu}_{1}}f_{1}\left(1\right)+F_{4,1}^{(1)}\left(1\right),\\
f_{3}\left(1\right)=\tilde{r}_{4}\tilde{\mu}_{1}+\frac{\tilde{\mu}_{1}}{1-\tilde{\mu}_{4}}f_{4}\left(4\right)+F_{1,4}^{(1)}\left(1\right),&
f_{3}\left(2\right)=\tilde{r}_{4}\tilde{\mu}_{2}+\frac{\tilde{\mu}_{2}}{1-\tilde{\mu}_{4}}f_{4}\left(4\right)+F_{2,4}^{(1)}\left(1\right),\\
f_{3}\left(3\right)=\tilde{r}_{4}\tilde{\mu}_{3}+\frac{\tilde{\mu}_{3}}{1-\tilde{\mu}_{4}}f_{4}\left(4\right)+f_{4}\left(3\right),&
f_{3}\left(4\right)=\tilde{r}_{4}\tilde{\mu}_{4}\\
f_{4}\left(1\right)=\tilde{r}_{3}\tilde{\mu}_{1}+\frac{\tilde{\mu}_{1}}{1-\tilde{\mu}_{3}}f_{3}\left(3\right)+F_{1,3}^{(1)}\left(1\right),&
f_{4}\left(2\right)=\tilde{r}_{3}\mu_{2}+\frac{\tilde{\mu}_{2}}{1-\tilde{\mu}_{3}}f_{3}\left(3\right)+F_{2,3}^{(1)}\left(1\right),\\
f_{4}\left(3\right)=\tilde{r}_{3}\tilde{\mu}_{3},&
f_{4}\left(4\right)=\tilde{r}_{3}\tilde{\mu}_{4}+\frac{\tilde{\mu}_{4}}{1-\tilde{\mu}_{3}}f_{3}\left(3\right)+f_{3}\left(4\right),\\
\end{array}
\end{eqnarray*}

Then we have that if $\mu=\tilde{\mu}_{1}+\tilde{\mu}_{2}<1$, $\hat{\mu}=\tilde{\mu}_{3}+\tilde{\mu}_{4}<1$, $r=r_{1}+r_{2}$ and $\hat{r}=\tilde{r}_{3}+\tilde{r}_{4}$  the system's solution are obtained by direct calculations:

\begin{eqnarray*}
\begin{array}{ll}
f_{2}\left(1\right)=r_{1}\tilde{\mu}_{1},&
f_{1}\left(2\right)=r_{2}\tilde{\mu}_{2},\\
f_{3}\left(4\right)=r_{4}\tilde{\mu}_{4},&
f_{4}\left(3\right)=r_{3}\tilde{\mu}_{3},\\
f_{1}\left(1\right)=r\frac{\tilde{\mu}_{1}\left(1-\tilde{\mu}_{1}\right)}{1-\mu},&
f_{2}\left(2\right)=r\frac{\tilde{\mu}_{2}\left(1-\tilde{\mu}_{2}\right)}{1-\mu},\\
f_{1}\left(3\right)=\tilde{\mu}_{3}\left(r_{2}+\frac{r\tilde{\mu}_{2}}{1-\mu}\right)+F_{3,2}^{(1)}\left(1\right),&
f_{1}\left(4\right)=\tilde{\mu}_{4}\left(r_{2}+\frac{r\tilde{\mu}_{2}}{1-\mu}\right)+F_{4,2}^{(1)}\left(1\right),\\
f_{2}\left(3\right)=\tilde{\mu}_{3}\left(r_{1}+\frac{r\tilde{\mu}_{1}}{1-\tilde{\mu}}\right)+F_{3,1}^{(1)}\left(1\right),&
f_{2}\left(4\right)=\tilde{\mu}_{4}\left(r_{1}+\frac{r\tilde{\mu}_{1}}{1-\mu}\right)+F_{4,,1}^{(1)}\left(1\right),\\
f_{3}\left(1\right)=\tilde{\mu}_{1}\left(r_{4}+\frac{\hat{r}\tilde{\mu}_{4}}{1-\hat{\mu}}\right)+F_{1,4}^{(1)}\left(1\right),&
f_{3}\left(2\right)=\tilde{\mu}_{2}\left(r_{4}+\frac{\hat{r}\tilde{\mu}_{4}}{1-\hat{\mu}}\right)+F_{2,4}^{(1)}\left(1\right),\\
f_{3}\left(3\right)=\hat{r}\frac{\tilde{\mu}_{3}\left(1-\tilde{\mu}_{3}\right)}{1-\hat{\mu}},&
f_{4}\left(1\right)=\tilde{\mu}_{1}\left(r_{3}+\frac{\hat{r}\tilde{\mu}_{3}}{1-\hat{\mu}}\right)+F_{1,3}^{(1)}\left(1\right),\\
f_{4}\left(2\right)=\tilde{\mu}_{2}\left(r_{3}+\frac{\hat{r}\tilde{\mu}_{3}}{1-\hat{\mu}}\right)+F_{2,3}^{(1)}\left(1\right),&
f_{4}\left(4\right)=\hat{r}\frac{\tilde{\mu}_{4}\left(1-\tilde{\mu}_{4}\right)}{1-\hat{\mu}}.
\end{array}
\end{eqnarray*}

Now, developing the equations given in (\ref{Eq.Gral.Second.Order.Exhaustive}) we obtain for instance $f_{1}\left(1,1\right)=\left(\frac{\tilde{\mu}_{1}}{1-\tilde{\mu}_{2}}\right)^{2}f_{2}\left(2,2\right)
+2\frac{\tilde{\mu}_{1}}{1-\tilde{\mu}_{2}}f_{2}\left(2,1\right)
+f_{2}\left(1,1\right)
+\tilde{\mu}_{1}^{2}\left(R_{2}^{(2)}+f_{2}\left(2\right)\theta_{2}^{(2)}\right)
+\tilde{P}_{1}^{(2)}\left(\frac{f_{2}\left(2\right)}{1-\tilde{\mu}_{2}}+r_{2}\right)+2r_{2}\tilde{\mu}_{2}f_{2}\left(1\right)$; similar reasoning lead us the following general expressions

\begin{eqnarray}\label{Eq.Sdo.Orden.Exh.uno}
\begin{array}{l}
f_{1}\left(i,j\right)=\indora_{i=1}f_{2}\left(1,1\right)
+\left[\left(1-\indora_{i=j=3}\right)\indora_{i+j\leq6}\indora_{i\leq j}\frac{\mu_{j}}{1-\tilde{\mu}_{2}}
+\left(1-\indora_{i=j=3}\right)\indora_{i+j\leq6}\indora_{i>j}\frac{\mu_{i}}{1-\tilde{\mu}_{2}}\right.\\
\left.+\indora_{i=1}\frac{\mu_{i}}{1-\tilde{\mu}_{2}}\right]f_{2}\left(1,2\right)+\indora_{i,j\neq2}\left(\frac{1}{1-\tilde{\mu}_{2}}\right)^{2}\mu_{i}\mu_{j}f_{2}\left(2,2\right)
+\left[\indora_{i,j\neq2}\tilde{\theta}_{2}^{(2)}\tilde{\mu}_{i}\tilde{\mu}_{j}
+\indora_{i,j\neq2}\indora_{i=j}\frac{\tilde{P}_{i}^{(2)}}{1-\tilde{\mu}_{2}}\right.\\
\left.+\indora_{i,j\neq2}\indora_{i\neq j}\frac{\tilde{\mu}_{i}\tilde{\mu}_{j}}{1-\tilde{\mu}_{2}}\right]f_{2}\left(2\right)
+\left[r_{2}\tilde{\mu}_{i}
+\indora_{i\geq3}F_{i,2}^{(1)}\right]f_{2}\left(j\right)
+\left[r_{2}\tilde{\mu}_{j}
+\indora_{j\geq3}F_{j,2}^{(1)}\right]f_{2}\left(i\right)\\
+\left[R_{2}^{(2)}
+\indora_{i=j}r_{2}\right]\tilde{\mu}_{i}\mu_{j}+\indora_{j\geq3}F_{j,2}^{(1)}\left[\indora_{j\neq i}F_{i,2}^{(1)}
+r_{2}\tilde{\mu}_{i}\right]
+r_{2}\left[\indora_{i=j}P_{i}^{(2)}
+\indora_{i\geq3}F_{i,2}^{(1)}\tilde{\mu}_{j}\right]\\
+\indora_{i\geq3}\indora_{j=i}F_{i,2}^{(2)}
\end{array}
\end{eqnarray}

in a similar manner we obtain expressions for $f_{2}\left(i,j\right)$, $f_{3}\left(i,j\right)$ and $f_{4}\left(i,j\right)$

for $i,k=1,2,3,4$; from which we obtain the linear equations system
\begin{eqnarray}\label{System.Second.Order.Moments.uno}
\begin{array}{ll}
f_{1}\left(1,1\right)=a_{1}f_{2}\left(2,2\right)
+a_{2}f_{2}\left(2,1\right)
+a_{3}f_{2}\left(1,1\right)
+K_{1},&
f_{1}\left(1,2\right)=K_{2},\\
f_{1}\left(1,3\right)=a_{4}f_{2}\left(2,2\right)+a_{5}f\left(2,1\right)+K_{3},&
f_{1}\left(1,4\right)=a_{6}f_{2}\left(2,2\right)+a_{7}f_{2}\left(2,1\right)+K_{4},\end{array}
\end{eqnarray}
for the rest equations, similar reasoning lead us to a linear system equations whose solutions are described in corolary (\ref{Coro.Second.Order.Eqs}) with coefficients given by, we just show a few of them


%Which can be reduced to solve the system given in (\ref{System.Second.Order.Moments.uno}) and (\ref{System.Second.Order.Moments.dos}).

with values for $a_{i}$ and $K_{i}$  
%{\small{
\begin{eqnarray}\label{Coefficients.Ais.Exh.uno}
\begin{array}{llll}
a_{1}=\left(\frac{\tilde{\mu}_{1}}{1-\tilde{\mu}_{2}}\right)^{2},&
a_{2}=\frac{2\tilde{\mu}_{1}}{1-\tilde{\mu}_{2}},&
a_{3}=1,&
a_{4}=\left(\frac{1}{1-\tilde{\mu}_{2}}\right)^{2}\tilde{\mu}_{1}\tilde{\mu}_{3},\\
a_{5}=\frac{\tilde{\mu}_{3}}{1-\tilde{\mu}_{2}},&
a_{6}=\left(\frac{1}{1-\tilde{\mu}_{2}}\right)^{2}\tilde{\mu}_{1}\tilde{\mu}_{4},&
a_{7}=\frac{\tilde{\mu}_{4}}{1-\tilde{\mu}_{2}},&\\
\end{array}
\end{eqnarray}%}}





\begin{eqnarray}\label{Coefficients.kis.Exh.uno}
\begin{array}{l}
K_{1}=\tilde{\mu}_{1}^{2}\left(R_{2}^{(2)}+f_{2}\left(2\right)\theta_{2}^{(2)}\right)
+\tilde{P}_{1}^{(2)}\left(\frac{f_{2}\left(2\right)}{1-\tilde{\mu}_{2}}+r_{2}\right)
+2r_{2}\tilde{\mu}_{2}f_{2}\left(1\right),\\
K_{2}=\tilde{\mu}_{1}\tilde{\mu}_{2}\left[R_{2}^{(2)}
+r_{2}\right]
+r_{2}\left[\tilde{\mu}_{1}f_{2}\left(2\right)
+\tilde{\mu}_{2}f_{2}\left(1\right)\right],\\
K_{3}=\tilde{\mu}_{1}\tilde{\mu}_{3}\left[R_{2}^{(2)}+r_{2}+f_{2}\left(2\right)\left(\tilde{\theta}_{2}^{(2)}+\frac{1}{1-\tilde{\mu}_{2}}\right)\right]
+r_{2}\tilde{\mu}_{1}\left[F_{3,2}^{(1)}+f_{2}\left(1\right)\right]
+\left[r_{2}\tilde{\mu}_{3}+F_{3,2}^{(1)}\right]f_{2}\left(1\right),\\
K_{4}=\tilde{\mu}_{1}\tilde{\mu}_{4}\left[R_{2}^{(2)}
+r_{2}+f_{2}\left(2\right)\left(\tilde{\theta}_{2}^{(2)}
+\frac{1}{1-\tilde{\mu}_{2}}\right)\right]
+r_{2}\tilde{\mu}_{1}\left[f_{2}\left(4\right)+F_{4,2}^{(1)}\right]
+f_{2}\left(1\right)\left[r_{2}\tilde{\mu}_{4}+F_{4,2}^{(1)}\right],
\end{array}
\end{eqnarray}
%\newpage
%________________________________________________________________________
\section{Redes de Sistemas de Visitas C\'iclicas Segundo Intento}
%________________________________________________________________________

\begin{Teo}
Dada una Red de Sistemas de Visitas C\'iclicas (RSVC), conformada por dos Sistemas de Visitas C\'iclicas (SVC), donde cada uno de ellos consta de dos colas tipo $M/M/1$. Los dos sistemas est\'an comunicados entre s\'i por medio de la transferencia de usuarios entre las colas $Q_{1}\leftrightarrow Q_{3}$ y $Q_{2}\leftrightarrow Q_{4}$. Se definen los eventos para los procesos de arribos al tiempo $t$, $A_{j}\left(t\right)=\left\{0 \textrm{ arribos en }Q_{j}\textrm{ al tiempo }t\right\}$ para alg\'un tiempo $t\geq0$ y $Q_{j}$ la cola $j$-\'esima en la RSVC, para $j=1,2,3,4$.  Existe un intervalo $I\neq\emptyset$ tal que para $T^{*}\in I$, tal que $\prob\left\{A_{1}\left(T^{*}\right),A_{2}\left(Tt^{*}\right),
A_{3}\left(T^{*}\right),A_{4}\left(T^{*}\right)|T^{*}\in I\right\}>0$.
\end{Teo}

\begin{proof}
Sin p\'erdida de generalidad podemos considerar como base del an\'alisis a la cola $Q_{1}$ del primer sistema que conforma la RSVC.

Sea $n\geq1$, ciclo en el primer sistema en el que se sabe que $L_{j}\left(\overline{\tau}_{1}\left(n\right)\right)=0$, pues la pol\'itica de servicio con que atienden los servidores es la exhaustiva. Como es sabido, para trasladarse a la siguiente cola, el servidor incurre en un tiempo de traslado $r_{1}\left(n\right)>0$, entonces tenemos que $\tau_{2}\left(n\right)=\overline{\tau}_{1}\left(n\right)+r_{1}\left(n\right)$.


Definamos el intervalo $I_{1}\equiv\left[\overline{\tau}_{1}\left(n\right),\tau_{2}\left(n\right)\right]$ de longitud $\xi_{1}=r_{1}\left(n\right)$. Dado que los tiempos entre arribo son exponenciales con tasa $\tilde{\mu}_{1}=\mu_{1}+\hat{\mu}_{1}$ ($\mu_{1}$ son los arribos a $Q_{1}$ por primera vez al sistema, mientras que $\hat{\mu}_{1}$ son los arribos de traslado procedentes de $Q_{3}$) se tiene que la probabilidad del evento $A_{1}\left(t\right)$ est\'a dada por 

\begin{equation}
\prob\left\{A_{1}\left(t\right)|t\in I_{1}\left(n\right)\right\}=e^{-\tilde{\mu}_{1}\xi_{1}\left(n\right)}.
\end{equation} 

Por otra parte, para la cola $Q_{2}$ el tiempo $\overline{\tau}_{2}\left(n-1\right)$ es tal que $L_{2}\left(\overline{\tau}_{2}\left(n-1\right)\right)=0$, es decir, es el tiempo en que la cola queda totalmente vac\'ia en el ciclo anterior a $n$. Entonces tenemos un sgundo intervalo $I_{2}\equiv\left[\overline{\tau}_{2}\left(n-1\right),\tau_{2}\left(n\right)\right]$. Por lo tanto la probabilidad del evento $A_{2}\left(t\right)$ tiene probabilidad dada por

\begin{equation}
\prob\left\{A_{2}\left(t\right)|t\in I_{2}\left(n\right)\right\}=e^{-\tilde{\mu}_{2}\xi_{2}\left(n\right)},
\end{equation} 

donde $\xi_{2}\left(n\right)=\tau_{2}\left(n\right)-\overline{\tau}_{2}\left(n-1\right)$.


Ahora, dado que $I_{1}\left(n\right)\subset I_{2}\left(n\right)$, se tiene que

\begin{eqnarray*}
\xi_{1}\left(n\right)\leq\xi_{2}\left(n\right)&\Leftrightarrow& -\xi_{1}\left(n\right)\geq-\xi_{2}\left(n\right)
\\
-\tilde{\mu}_{2}\xi_{1}\left(n\right)\geq-\tilde{\mu}_{2}\xi_{2}\left(n\right)&\Leftrightarrow&
e^{-\tilde{\mu}_{2}\xi_{1}\left(n\right)}\geq e^{-\tilde{\mu}_{2}\xi_{2}\left(n\right)}\\
\prob\left\{A_{2}\left(t\right)|t\in I_{1}\left(n\right)\right\}&\geq&
\prob\left\{A_{2}\left(t\right)|t\in I_{2}\left(n\right)\right\}.
\end{eqnarray*}


Entonces se tiene que

\begin{eqnarray*}
\prob\left\{A_{1}\left(t\right),A_{2}\left(t\right)|t\in I_{1}\left(n\right)\right\}&=&
\prob\left\{A_{1}\left(t\right)|t\in I_{1}\left(n\right)\right\}
\prob\left\{A_{2}\left(t\right)|t\in I_{1}\left(n\right)\right\}\\
&\geq&
\prob\left\{A_{1}\left(t\right)|t\in I_{1}\left(n\right)\right\}
\prob\left\{A_{2}\left(t\right)|t\in I_{2}\left(n\right)\right\}\\
&=&e^{-\tilde{\mu}_{1}\xi_{1}\left(n\right)}e^{-\tilde{\mu}_{2}\xi_{2}\left(n\right)}
=e^{-\left[\tilde{\mu}_{1}\xi_{1}\left(n\right)+\tilde{\mu}_{2}\xi_{2}\left(n\right)\right]}.
\end{eqnarray*}


Es decir, 

\begin{equation}
\prob\left\{A_{1}\left(t\right),A_{2}\left(t\right)|t\in I_{1}\left(n\right)\right\}
=e^{-\left[\tilde{\mu}_{1}\xi_{1}\left(n\right)+\tilde{\mu}_{2}\xi_{2}
\left(n\right)\right]}>0.
\end{equation}

En lo que respecta a la relaci\'on entre los dos SVC que conforman la RSVC, se afirma que existe $m\geq1$ tal que $\overline{\tau}_{3}\left(m\right)<\tau_{2}\left(n\right)<\tau_{4}\left(m\right)$.

Para $Q_{3}$ sea $I_{3}=\left[\overline{\tau}_{3}\left(m\right),\tau_{4}\left(m\right)\right]$ con longitud  $\xi_{3}\left(m\right)=r_{3}\left(m\right)$, entonces 

\begin{equation}
\prob\left\{A_{3}\left(t\right)|t\in I_{3}\left(n\right)\right\}=e^{-\tilde{\mu}_{3}\xi_{3}\left(n\right)}.
\end{equation} 

An\'alogamente que como se hizo para $Q_{2}$, tenemos que para $Q_{4}$ se tiene el intervalo $I_{4}=\left[\overline{\tau}_{4}\left(m-1\right),\tau_{4}\left(m\right)\right]$ con longitud $\xi_{4}\left(m\right)=\tau_{4}\left(m\right)-\overline{\tau}_{4}\left(m-1\right)$, entonces


\begin{equation}
\prob\left\{A_{4}\left(t\right)|t\in I_{4}\left(m\right)\right\}=e^{-\tilde{\mu}_{4}\xi_{4}\left(n\right)}.
\end{equation} 

Al igual que para el primer sistema, dado que $I_{3}\left(m\right)\subset I_{4}\left(m\right)$, se tiene que

\begin{eqnarray*}
\xi_{3}\left(m\right)\leq\xi_{4}\left(m\right)&\Leftrightarrow& -\xi_{3}\left(m\right)\geq-\xi_{4}\left(m\right)
\\
-\tilde{\mu}_{4}\xi_{3}\left(m\right)\geq-\tilde{\mu}_{4}\xi_{4}\left(m\right)&\Leftrightarrow&
e^{-\tilde{\mu}_{4}\xi_{3}\left(m\right)}\geq e^{-\tilde{\mu}_{4}\xi_{4}\left(m\right)}\\
\prob\left\{A_{4}\left(t\right)|t\in I_{3}\left(m\right)\right\}&\geq&
\prob\left\{A_{4}\left(t\right)|t\in I_{4}\left(m\right)\right\}
\end{eqnarray*}

Entonces, dado que los eventos $A_{3}$ y $A_{4}$ son independientes, se tiene que

\begin{eqnarray*}
\prob\left\{A_{3}\left(t\right),A_{4}\left(t\right)|t\in I_{3}\left(m\right)\right\}&=&
\prob\left\{A_{3}\left(t\right)|t\in I_{3}\left(m\right)\right\}
\prob\left\{A_{4}\left(t\right)|t\in I_{3}\left(m\right)\right\}\\
&\geq&
\prob\left\{A_{3}\left(t\right)|t\in I_{3}\left(n\right)\right\}
\prob\left\{A_{4}\left(t\right)|t\in I_{4}\left(n\right)\right\}\\
&=&e^{-\tilde{\mu}_{3}\xi_{3}\left(m\right)}e^{-\tilde{\mu}_{4}\xi_{4}
\left(m\right)}
=e^{-\left[\tilde{\mu}_{3}\xi_{3}\left(m\right)+\tilde{\mu}_{4}\xi_{4}
\left(m\right)\right]}.
\end{eqnarray*}


es decir, 

\begin{equation}
\prob\left\{A_{3}\left(t\right),A_{4}\left(t\right)|t\in I_{3}\left(m\right)\right\}
=e^{-\left[\tilde{\mu}_{3}\xi_{3}\left(m\right)+\tilde{\mu}_{4}\xi_{4}
\left(m\right)\right]}>0.
\end{equation}

Por construcci\'on se tiene que $I\left(n,m\right)\equiv I_{1}\left(n\right)\cap I_{3}\left(m\right)\neq\emptyset$,entonces en particular se tienen las contenciones $I\left(n,m\right)\subseteq I_{1}\left(n\right)$ y $I\left(n,m\right)\subseteq I_{3}\left(m\right)$, por lo tanto si definimos $\xi_{n,m}\equiv\ell\left(I\left(n,m\right)\right)$ tenemos que

\begin{eqnarray*}
\xi_{n,m}\leq\xi_{1}\left(n\right)\textrm{ y }\xi_{n,m}\leq\xi_{3}\left(m\right)\textrm{ entonces }
-\xi_{n,m}\geq-\xi_{1}\left(n\right)\textrm{ y }-\xi_{n,m}\leq-\xi_{3}\left(m\right)\\
\end{eqnarray*}
por lo tanto tenemos las desigualdades 



\begin{eqnarray*}
\begin{array}{ll}
-\tilde{\mu}_{1}\xi_{n,m}\geq-\tilde{\mu}_{1}\xi_{1}\left(n\right),&
-\tilde{\mu}_{2}\xi_{n,m}\geq-\tilde{\mu}_{2}\xi_{1}\left(n\right)
\geq-\tilde{\mu}_{2}\xi_{2}\left(n\right),\\
-\tilde{\mu}_{3}\xi_{n,m}\geq-\tilde{\mu}_{3}\xi_{3}\left(m\right),&
-\tilde{\mu}_{4}\xi_{n,m}\geq-\tilde{\mu}_{4}\xi_{3}\left(m\right)
\geq-\tilde{\mu}_{4}\xi_{4}\left(m\right).
\end{array}
\end{eqnarray*}

Sea $T^{*}\in I_{n,m}$, entonces dado que en particular $T^{*}\in I_{1}\left(n\right)$ se cumple con probabilidad positiva que no hay arribos a las colas $Q_{1}$ y $Q_{2}$, en consecuencia, tampoco hay usuarios de transferencia para $Q_{3}$ y $Q_{4}$, es decir, $\tilde{\mu}_{1}=\mu_{1}$, $\tilde{\mu}_{2}=\mu_{2}$, $\tilde{\mu}_{3}=\mu_{3}$, $\tilde{\mu}_{4}=\mu_{4}$, es decir, los eventos $Q_{1}$ y $Q_{3}$ son condicionalmente independientes en el intervalo $I_{n,m}$; lo mismo ocurre para las colas $Q_{2}$ y $Q_{4}$, por lo tanto tenemos que


\begin{eqnarray}
\begin{array}{l}
\prob\left\{A_{1}\left(T^{*}\right),A_{2}\left(T^{*}\right),
A_{3}\left(T^{*}\right),A_{4}\left(T^{*}\right)|T^{*}\in I_{n,m}\right\}
=\prod_{j=1}^{4}\prob\left\{A_{j}\left(T^{*}\right)|T^{*}\in I_{n,m}\right\}\\
\geq\prob\left\{A_{1}\left(T^{*}\right)|T^{*}\in I_{1}\left(n\right)\right\}
\prob\left\{A_{2}\left(T^{*}\right)|T^{*}\in I_{2}\left(n\right)\right\}
\prob\left\{A_{3}\left(T^{*}\right)|T^{*}\in I_{3}\left(m\right)\right\}
\prob\left\{A_{4}\left(T^{*}\right)|T^{*}\in I_{4}\left(m\right)\right\}\\
=e^{-\mu_{1}\xi_{1}\left(n\right)}
e^{-\mu_{2}\xi_{2}\left(n\right)}
e^{-\mu_{3}\xi_{3}\left(m\right)}
e^{-\mu_{4}\xi_{4}\left(m\right)}
=e^{-\left[\tilde{\mu}_{1}\xi_{1}\left(n\right)
+\tilde{\mu}_{2}\xi_{2}\left(n\right)
+\tilde{\mu}_{3}\xi_{3}\left(m\right)
+\tilde{\mu}_{4}\xi_{4}
\left(m\right)\right]}>0.
\end{array}
\end{eqnarray}


Ahora solo resta demostrar que para $n\ge1$, existe $m\geq1$ tal que $\overline{\tau}_{3}\left(m\right)<\tau_{2}\left(n\right)<\tau_{4}\left(m\right)$.

Supongamos que no existe $m\geq1$, tal que $I_{1}\left(n\right)\cap I_{3}\left(m\right)\neq\emptyset$, es decir, para toda $m\geq1$, $I_{1}\left(n\right)\cap I_{3}\left(m\right)=\emptyset$, entonces ocurren cualquiera de los dos casos
%\begin{center}
\begin{itemize}
\item[a)] $\tau_{2}\left(n\right)<\overline{\tau}_{3}\left(m\right)$, 


\item[b)] $\tau_{4}\left(m\right)<\overline{\tau}_{1}\left(n\right)$.

\end{itemize}
%\end{center}

Supongamos que para toda  $n\geq1$ y $m\geq1$ se  tiene $\tau_{2}\left(n\right)<\overline{\tau}_{3}\left(m\right)$. Dado que $\esp\left[\tau_{2}\left(n\right)\right]<\infty$ y $\esp\left[\tau_{3}\left(m\right)\right]<\infty$ entonces $\prob\left\{\tau_{2}\left(n\right)=\infty\right\}=\prob\left\{\tau_{3}\left(m\right)=\infty\right\}=0$. Adem\'as se sabe que $\esp\left[R_{1}\right]<\infty$ y por tanto $\prob\left\{r_{1}\left(n\right)=\infty\right\}=0$, entonces se tiene que $L_{4}\left(\tau_{4}\left(m-1\right)\right)=\infty$ lo cual es una contradicci\'on con el hecho de que la cola $Q_{4}$ es estable y estacionaria, \'o se tiene que $L_{4}\left(\tau_{4}\left(m-1\right)\right)=0$ que es una contradicci\'on  pues el proceso de arribos es no vac\'io. Para el segundo caso la demostraci\'on es an\'aloga.

\end{proof}


%\newpage







%______________________________________________________________________
%\section{Ap\'endice A}
%__________________________________________________________________
%_________________________________________________________________________

\section{Resultados para Procesos de Salida}





%________________________________________________________________________
\subsubsection{Redes de Sistemas de Visitas C\'iclicas}
%________________________________________________________________________



%________________________________________________________
%
\subsection{Convergencia}
%___________________________________________________________________________________________
%

%___________________________________________________________________________________________
%
\subsection{Billingsley: Espacios Producto}
%___________________________________________________________________________________________
%


Sea $S=S^{'}\times S^{''}$ el espacio producto de los espacios m\'etricos $S^{'}$ y $S^{''}$. Si $S$ es separable, que a su vez requiere que ambos espacios sean separables, entonces las $\sigma$-\'algebras $\mathcal{S}$, $\mathcal{S}^{'}$ y $\mathcal{S}^{''}$ de los conjuntos de Borel en este espacio est\'an relacionados por $\mathcal{S}=\mathcal{S}^{'}\times \mathcal{S}^{'}$, 
%___________________________________________________________________________________________
\subsection{Funcion Generadora de Probabilidades Conjunta}
%___________________________________________________________________________________________

Sea $T>0$, y sea $\left\{T_{j}\right\}$ partici\'on del intervalo $\left[0,T\right]$. Sea $T_{k}$ elemento arbitrario de la partici\'on. Sean $m$ y $n$ ciclos del servidor, se definen los siguientes procesos:


\begin{itemize}
\item $L_{j}\left(t\right)$ para la longitud de la cola al tiempo $T_{k}$.

\item $A_{j}\left(t\right)$ el proceso de los residuales de los tiempos de arribo para el siguiente usuario al tiempo $T_{k}$.


\item $B_{j}\left(t\right)$ el proceso de los residuales de los tiempos de servicio al tiempo $T_{k}$.


\item $B_{j}^{0}\left(t\right)$ el proceso de los residuales de los tiempos de traslado del servidor al tiempo $T_{k}$.


\item $C_{j}\left(t\right)$ el n\'umero de usuarios atendidos por el servidor al tiempo $T_{k}$.
\end{itemize}


Entonces se tiene la siguiente funci\'on generadora de probabilidades


\begin{eqnarray}
\mathcal{G}=\esp\left[\prod_{j=1}^{4}L_{j}\left(T_{k}\right)\prod_{j=1}^{4}A_{j}\left(T_{k}\right)\prod_{j=1}^{4}B_{j}\left(T_{k}\right)\prod_{j=1}^{4}B_{j}^{0}\left(T_{k}\right)
\prod_{j=1}^{4}C_{j}\left(T_{k}\right)\right]
\end{eqnarray}

Para $T_{k}$ se tienen los siguientes casos:
\begin{multicols}{2}
\begin{enumerate}
\item $T_{k}\leq\overline{\tau}_{1}\left(m\right)$
\item $\overline{\tau}_{1}\left(m\right)<T_{k}\leq \tau_{2}\left(m\right)$
\item $\overline{\tau}_{2}\left(m\right)<T_{k}\leq \overline{\tau}_{2}\left(m\right)$
\item $\tau_{2}\left(m\right)<T_{k}$
\end{enumerate}
\end{multicols}

Lo cual nos dan los casos que enunciamos de manera exhaustiva a continuaci\'on:
\begin{itemize}
%_____________________________________________________________________________________________
\item Primer caso
%_____________________________________________________________________________________________
\begin{itemize}
\item[1.a)] $T_{k}\leq\overline{\tau}_{1}\left(m\right)$ y $T_{k}\leq\overline{\tau}_{3}\left(n\right)$:
\begin{eqnarray*}
L_{1}\left(T_{k}\right)&=&L_{1}\left(\tau_{1}\left(m\right)\right)+X_{1}\left(T_{k}-\tau_{1}\left(m\right)\right)+Y_{1}
\left(T_{k}-\tau_{1}\left(m\right)\right)-C_{1}\left(T_{k}\right)\\
L_{2}\left(T_{k}\right)&=&X_{2}\left(T_{k}-\overline{\tau}_{2}\left(m-1\right)\right)\\
L_{3}\left(T_{k}\right)&=&L_{3}\left(\tau_{3}\left(n\right)\right)+X_{3}\left(T_{k}-\tau_{3}\left(n\right)\right)+Y_{3}\left(T_{k}-\tau_{3}\left(n\right)\right)-C_{3}\left(T_{k}\right)\\
L_{4}\left(T_{k}\right)&=&X_{4}\left(T_{k}-\overline{\tau}_{4}\left(n-1\right)\right)
\end{eqnarray*}


\item[1.b)] $T_{k}\leq\overline{\tau}_{1}\left(m\right)$ y $\overline{\tau}_{3}\left(n\right)<T_{k}\leq\tau_{4}\left(n\right)$:
\begin{eqnarray*}
L_{1}\left(T_{k}\right)&=&L_{1}\left(\tau_{1}\left(m\right)\right)+X_{1}\left(T_{k}-\tau_{1}\left(m\right)\right)+Y_{1}
\left(T_{k}-\tau_{1}\left(m\right)\right)-C_{1}\left(T_{k}\right)\\
L_{2}\left(T_{k}\right)&=&X_{2}\left(T_{k}-\overline{\tau}_{2}\left(m-1\right)\right)\\
L_{3}\left(T_{k}\right)&=&X_{3}\left(T_{k}-\overline{\tau}_{3}\left(n\right)\right)\\
L_{4}\left(T_{k}\right)&=&L_{4}\left(\tau_{4}\left(n\right)\right)
\end{eqnarray*}

\item[1.c)] $T_{k}\leq\overline{\tau}_{1}\left(m\right)$ y $\tau_{4}\left(n\right)<T_{k}\leq\overline{\tau}_{4}\left(n\right)$:
\begin{eqnarray*}
L_{1}\left(T_{k}\right)&=&L_{1}\left(\tau_{1}\left(m\right)\right)+X_{1}\left(T_{k}-\tau_{1}\left(m\right)\right)+Y_{1}
\left(T_{k}-\tau_{1}\left(m\right)\right)-C_{1}\left(T_{k}\right)\\
L_{2}\left(T_{k}\right)&=&X_{2}\left(T_{k}-\overline{\tau}_{2}\left(m-1\right)\right)\\
L_{3}\left(T_{k}\right)&=&X_{3}\left(T_{k}-\overline{\tau}_{3}\left(n\right)\right)\\
L_{4}\left(T_{k}\right)&=&L_{4}\left(\tau_{4}\left(n\right)\right)+X_{4}\left(T_{k}-\tau_{4}\left(n\right)\right)+Y_{4}
\left(T_{k}-\tau_{4}\left(n\right)\right)-C_{4}\left(T_{k}\right)
\end{eqnarray*}

\item[1.d)] $T_{k}\leq\overline{\tau}_{1}\left(m\right)$ y $\overline{\tau}_{4}\left(n\right)<T_{k}$:
\begin{eqnarray*}
L_{1}\left(T_{k}\right)&=&L_{1}\left(\tau_{1}\left(m\right)\right)+X_{1}\left(T_{k}-\tau_{1}\left(m\right)\right)+Y_{1}
\left(T_{k}-\tau_{1}\left(m\right)\right)-C_{1}\left(T_{k}\right)\\
L_{2}\left(T_{k}\right)&=&X_{2}\left(T_{k}-\overline{\tau}_{2}\left(m-1\right)\right)\\
L_{3}\left(T_{k}\right)&=&X_{3}\left(T_{k}-\overline{\tau}_{3}\left(n\right)\right)\\
L_{4}\left(T_{k}\right)&=&X_{4}\left(T_{k}-\overline{\tau}_{4}\left(n\right)\right)
\end{eqnarray*}

\end{itemize}
%_____________________________________________________________________________________________
\item Segundo caso
%_____________________________________________________________________________________________
\begin{itemize}
\item[2.a)] $\overline{\tau}_{1}\left(m\right)<T_{k}\leq\tau_{2}\left(m\right)$ y $T_{k}\leq\overline{\tau}_{3}\left(n\right)$:
\begin{eqnarray*}
L_{1}\left(T_{k}\right)&=&X_{1}\left(T_{k}-\overline{\tau}_{1}\left(m\right)\right)\\
L_{2}\left(T_{k}\right)&=&L_{2}\left(\tau_{2}\left(m\right)\right)\\
L_{3}\left(T_{k}\right)&=&L_{3}\left(\tau_{3}\left(n\right)\right)+X_{3}\left(T_{k}-\tau_{3}\left(n\right)\right)+Y_{3}\left(T_{k}-\tau_{3}\left(n\right)\right)-C_{3}\left(T_{k}\right)\\
L_{4}\left(T_{k}\right)&=&X_{4}\left(T_{k}-\overline{\tau}_{4}\left(n-1\right)\right)
\end{eqnarray*}

\item[2.b)] $\overline{\tau}_{1}\left(m\right)<T_{k}\leq\tau_{2}\left(m\right)$ y $\overline{\tau}_{3}\left(n\right)<T_{k}\leq\tau_{4}\left(n\right)$:
\begin{eqnarray*}
L_{1}\left(T_{k}\right)&=&X_{1}\left(T_{k}-\overline{\tau}_{1}\left(m\right)\right)\\
L_{2}\left(T_{k}\right)&=&L_{2}\left(\tau_{2}\left(m\right)\right)\\
L_{3}\left(T_{k}\right)&=&X_{3}\left(T_{k}-\overline{\tau}_{3}\left(n\right)\right)\\
L_{4}\left(T_{k}\right)&=&L_{4}\left(\tau_{4}\left(n\right)\right)
\end{eqnarray*}


\item[2.c)] $\overline{\tau}_{1}\left(m\right)<T_{k}\leq\tau_{2}\left(m\right)$ y $\tau_{4}\left(n\right)<T_{k}\leq\overline{\tau}_{4}\left(n\right)$:
\begin{eqnarray*}
L_{1}\left(T_{k}\right)&=&X_{1}\left(T_{k}-\overline{\tau}_{1}\left(m\right)\right)\\
L_{2}\left(T_{k}\right)&=&L_{2}\left(\tau_{2}\left(m\right)\right)\\
L_{3}\left(T_{k}\right)&=&X_{3}\left(T_{k}-\overline{\tau}_{3}\left(m\right)\right)\\
L_{4}\left(T_{k}\right)&=&L_{4}\left(\tau_{4}\left(n\right)\right)+X_{4}\left(T_{k}-\tau_{4}\left(n\right)\right)+Y_{4}
\left(T_{k}-\tau_{4}\left(n\right)\right)-C_{4}\left(T_{k}\right)
\end{eqnarray*}

\item[2.d)] $\overline{\tau}_{1}\left(m\right)<T_{k}\leq\tau_{2}\left(m\right)$ y $\overline{\tau}_{4}\left(n\right)<T_{k}$:
\begin{eqnarray*}
L_{1}\left(T_{k}\right)&=&X_{1}\left(T_{k}-\overline{\tau}_{1}\left(m\right)\right)\\
L_{2}\left(T_{k}\right)&=&L_{2}\left(\tau_{2}\left(m\right)\right)\\
L_{3}\left(T_{k}\right)&=&X_{3}\left(T_{k}-\overline{\tau}_{3}\left(n\right)\right)\\
L_{4}\left(T_{k}\right)&=&X_{4}\left(T_{k}-\overline{\tau}_{4}\left(n\right)\right)
\end{eqnarray*}
\end{itemize}

%_____________________________________________________________________________________________
\item Tercer caso
%_____________________________________________________________________________________________
\begin{itemize}
\item[3.a)] $\tau_{2}\left(m\right)<T_{k}\leq \overline{\tau}_{2}\left(m\right)$ y $T_{k}\leq\overline{\tau}_{3}\left(n\right)$:
\begin{eqnarray*}
L_{1}\left(T_{k}\right)&=&X_{1}\left(T_{k}-\overline{\tau}_{1}\left(m\right)\right)\\
L_{2}\left(T_{k}\right)&=&L_{2}\left(\tau_{2}\left(m\right)\right)+X_{2}\left(T_{k}-\tau_{2}\left(m\right)\right)+Y_{2}
\left(T_{k}-\tau_{2}\left(m\right)\right)-C_{2}\left(T_{k}\right)\\
L_{3}\left(T_{k}\right)&=&L_{3}\left(\tau_{3}\left(n\right)\right)+X_{3}\left(T_{k}-\tau_{3}\left(n\right)\right)+Y_{3}\left(T_{k}-\tau_{3}\left(n\right)\right)-C_{3}\left(T_{k}\right)\\
L_{4}\left(T_{k}\right)&=&X_{4}\left(T_{k}-\overline{\tau}_{4}\left(n-1\right)\right)
\end{eqnarray*}

\item[3.b)] $\tau_{2}\left(m\right)<T_{k}\leq \overline{\tau}_{2}\left(m\right)$ y $\overline{\tau}_{3}\left(n\right)<T_{k}\leq\tau_{4}\left(n\right)$:
\begin{eqnarray*}
L_{1}\left(T_{k}\right)&=&X_{1}\left(T_{k}-\overline{\tau}_{1}\left(m\right)\right)\\
L_{2}\left(T_{k}\right)&=&L_{2}\left(\tau_{2}\left(m\right)\right)+X_{2}\left(T_{k}-\tau_{2}\left(m\right)\right)+Y_{2}
\left(T_{k}-\tau_{2}\left(m\right)\right)-C_{2}\left(T_{k}\right)\\
L_{3}\left(T_{k}\right)&=&X_{3}\left(T_{k}-\overline{\tau}_{3}\left(n\right)\right)\\
L_{4}\left(T_{k}\right)&=&L_{4}\left(\tau_{4}\left(n\right)\right)
\end{eqnarray*}


\item[3.c)] $\tau_{2}\left(m\right)<T_{k}\leq \overline{\tau}_{2}\left(m\right)$ y $\tau_{4}\left(n\right)<T_{k}\leq\overline{\tau}_{4}\left(n\right)$:
\begin{eqnarray*}
L_{1}\left(T_{k}\right)&=&X_{1}\left(T_{k}-\overline{\tau}_{1}\left(m\right)\right)\\
L_{2}\left(T_{k}\right)&=&L_{2}\left(\tau_{2}\left(m\right)\right)+X_{2}\left(T_{k}-\tau_{2}\left(m\right)\right)+Y_{2}
\left(T_{k}-\tau_{2}\left(m\right)\right)-C_{2}\left(T_{k}\right)\\
L_{3}\left(T_{k}\right)&=&X_{3}\left(T_{k}-\overline{\tau}_{3}\left(n\right)\right)\\
L_{4}\left(T_{k}\right)&=&L_{4}\left(\tau_{4}\left(n\right)\right)+X_{4}\left(T_{k}-\tau_{4}\left(n\right)\right)+Y_{4}
\left(T_{k}-\tau_{4}\left(n\right)\right)-C_{4}\left(T_{k}\right)
\end{eqnarray*}

\item[3.d)] $\tau_{2}\left(m\right)<T_{k}\leq \overline{\tau}_{2}\left(m\right)$ y $\overline{\tau}_{4}\left(n\right)<T_{k}$:
\begin{eqnarray*}
L_{1}\left(T_{k}\right)&=&X_{1}\left(T_{k}-\overline{\tau}_{1}\left(m\right)\right)\\
L_{2}\left(T_{k}\right)&=&L_{2}\left(\tau_{2}\left(m\right)\right)+X_{2}\left(T_{k}-\tau_{2}\left(m\right)\right)+Y_{2}
\left(T_{k}-\tau_{2}\left(m\right)\right)-C_{2}\left(T_{k}\right)\\
L_{3}\left(T_{k}\right)&=&X_{3}\left(T_{k}-\overline{\tau}_{3}\left(n\right)\right)\\
L_{4}\left(T_{k}\right)&=&X_{4}\left(T_{k}-\overline{\tau}_{4}\left(n\right)\right)
\end{eqnarray*}
\end{itemize}
%_____________________________________________________________________________________________
\item Cuarto caso
%_____________________________________________________________________________________________
\begin{itemize}
\item[4.a)] $\overline{\tau}_{2}\left(m\right)<T_{k}$ y $T_{k}\leq\overline{\tau}_{3}\left(n\right)$:
\begin{eqnarray*}
L_{1}\left(T_{k}\right)&=&X_{1}\left(T_{k}-\overline{\tau}_{1}\left(m\right)\right)\\
L_{2}\left(T_{k}\right)&=&X_{2}\left(T_{k}-\overline{\tau}_{2}\left(m\right)\right)\\
L_{3}\left(T_{k}\right)&=&L_{3}\left(\tau_{3}\left(n\right)\right)+X_{3}\left(T_{k}-\tau_{3}\left(n\right)\right)+Y_{3}\left(T_{k}-\tau_{3}\left(n\right)\right)-C_{3}\left(T_{k}\right)\\
L_{4}\left(T_{k}\right)&=&X_{4}\left(T_{k}-\overline{\tau}_{4}\left(n-1\right)\right)
\end{eqnarray*}

\item[4.b)] $\overline{\tau}_{2}\left(m\right)<T_{k}$ y $\overline{\tau}_{3}\left(n\right)<T_{k}\leq\tau_{4}$:
\begin{eqnarray*}
L_{1}\left(T_{k}\right)&=&X_{1}\left(T_{k}-\overline{\tau}_{1}\left(m\right)\right)\\
L_{2}\left(T_{k}\right)&=&X_{2}\left(T_{k}-\overline{\tau}_{2}\left(m\right)\right)\\
L_{3}\left(T_{k}\right)&=&X_{3}\left(T_{k}-\overline{\tau}_{3}\left(n\right)\right)\\
L_{4}\left(T_{k}\right)&=&L_{4}\left(\tau_{4}\left(n\right)\right)
\end{eqnarray*}


\item[4.c)] $\overline{\tau}_{2}\left(m\right)<T_{k}$ y $\tau_{4}\left(n\right)<T_{k}\leq\overline{\tau}_{4}\left(n\right)$:
\begin{eqnarray*}
L_{1}\left(T_{k}\right)&=&X_{1}\left(T_{k}-\overline{\tau}_{1}\left(m\right)\right)\\
L_{2}\left(T_{k}\right)&=&X_{2}\left(T_{k}-\overline{\tau}_{2}\left(m\right)\right)\\
L_{3}\left(T_{k}\right)&=&X_{3}\left(T_{k}-\overline{\tau}_{3}\left(n\right)\right)\\
L_{4}\left(T_{k}\right)&=&L_{4}\left(\tau_{4}\left(n\right)\right)+X_{4}\left(T_{k}-\tau_{4}\left(n\right)\right)+Y_{4}
\left(T_{k}-\tau_{4}\left(n\right)\right)-C_{4}\left(T_{k}\right)
\end{eqnarray*}

\item[4.d)] $\overline{\tau}_{2}<T_{k}$ y $\overline{\tau}_{4}<T_{k}$:
\begin{eqnarray*}
L_{1}\left(T_{k}\right)&=&X_{1}\left(T_{k}-\overline{\tau}_{1}\left(m\right)\right)\\
L_{2}\left(T_{k}\right)&=&X_{2}\left(T_{k}-\overline{\tau}_{2}\left(m\right)\right)\\
L_{3}\left(T_{k}\right)&=&X_{3}\left(T_{k}-\overline{\tau}_{3}\left(n\right)\right)\\
L_{4}\left(T_{k}\right)&=&X_{4}\left(T_{k}-\overline{\tau}_{4}\left(n\right)\right)
\end{eqnarray*}
\end{itemize}


\end{itemize}
\newpage
%_____________________________________________________
\subsection{Puntos de Renovaci\'on}
%_____________________________________________________

Para cada cola $Q_{i}$ se tienen los procesos de arribo a la cola, para estas, los tiempos de arribo est\'an dados por $$\left\{T_{1}^{i},T_{2}^{i},\ldots,T_{k}^{i},\ldots\right\},$$ entonces, consideremos solamente los primeros tiempos de arribo a cada una de las colas, es decir, $$\left\{T_{1}^{1},T_{1}^{2},T_{1}^{3},T_{1}^{4}\right\},$$ se sabe que cada uno de estos tiempos se distribuye de manera exponencial con par\'ametro $1/mu_{i}$. Adem\'as se sabe que para $$T^{*}=\min\left\{T_{1}^{1},T_{1}^{2},T_{1}^{3},T_{1}^{4}\right\},$$ $T^{*}$ se distribuye de manera exponencial con par\'ametro $$\mu^{*}=\sum_{i=1}^{4}\mu_{i}.$$ Ahora, dado que 
\begin{center}
\begin{tabular}{lcl}
$\tilde{r}=r_{1}+r_{2}$ & y &$\hat{r}=r_{3}+r_{4}.$
\end{tabular}
\end{center}


Supongamos que $$\tilde{r},\hat{r}<\mu^{*},$$ entonces si tomamos $$r^{*}=\min\left\{\tilde{r},\hat{r}\right\},$$ se tiene que para  $$t^{*}\in\left(0,r^{*}\right)$$ se cumple que 
\begin{center}
\begin{tabular}{lcl}
$\tau_{1}\left(1\right)=0$ & y por tanto & $\overline{\tau}_{1}=0,$
\end{tabular}
\end{center}
entonces para la segunda cola en este primer ciclo se cumple que $$\tau_{2}=\overline{\tau}_{1}+r_{1}=r_{1}<\mu^{*},$$ y por tanto se tiene que  $$\overline{\tau}_{2}=\tau_{2}.$$ Por lo tanto, nuevamente para la primer cola en el segundo ciclo $$\tau_{1}\left(2\right)=\tau_{2}\left(1\right)+r_{2}=\tilde{r}<\mu^{*}.$$ An\'alogamente para el segundo sistema se tiene que ambas colas est\'an vac\'ias, es decir, existe un valor $t^{*}$ tal que en el intervalo $\left(0,t^{*}\right)$ no ha llegado ning\'un usuario, es decir, $$L_{i}\left(t^{*}\right)=0$$ para $i=1,2,3,4$.

\subsection{Resultados para Procesos de Salida}

%______________________________________________________________
\section{Redes de Sistemas de Visitas C\'iclicas}
%______________________________________________________________



Para cada cola $Q_{j}$, $j=1,\ldots,4$, se tienen los siguientes procesos

\begin{itemize}
\item $L_{j}\left(t\right)$ el n\'umero de usuarios presentes en la cola al tiempo $t$.

\item $A_{j}\left(t\right)$ el residual del tiempo de arribo del siguiente usuario.

\item $B_{j}\left(t\right)$ el residual del tiempo de servicio del usuario que est\'a siendo atendido.

\item $C_{j}\left(t\right)$ el residual del tiempo de traslado del servidor entre una cola y otra, en caso de que se encuentre dando servicio se considera $C_{j}\left(t\right)=0$.
\end{itemize}
para $j=1,\ldots,4$. Con base en lo anterior se tienen los procesos

\begin{eqnarray}\label{Procesos.RSVC}
L\left(t\right)=\left(L_{j}\left(t\right)\right)_{j=1}^{4},
A\left(t\right)=\left(A_{j}\left(t\right)\right)_{j=1}^{4}, B\left(t\right)=\left(B_{j}\left(t\right)\right)_{j=1}^{4}
\textrm{ y } C\left(t\right)=\left(C_{j}\left(t\right)\right)_{j=1}^{4}.
\end{eqnarray}




Por lo tanto se tiene el proceso estoc\'astico
\begin{eqnarray}\label{Proceso.Estocastico.Z}
\mathbb{Z}=\left(L\left(t\right),A\left(t\right),
B\left(t\right),C\left(t\right)\right)
\end{eqnarray}


Para los procesos residuales de los tiempos de traslado, servicio y de arribos, su espacio de estados es un subconjunto de $\rea_{+}=\left[0,\infty\right)$, es decir, $E\subset\left[0,\infty\right)$ y por tanto $\mathcal{E}\subset\mathcal{B}\left[0,\infty\right)$, luego el espacio $\left(E,\mathcal{E}\right)$ es un espacio polaco.


Para cada proceso de residuales se tienen los siguientes espacios producto

\begin{itemize}
\item[a)] Para $A\left(t\right)=\left(A_{j}\left(t\right)\right)_{j=1}^{4}$ se tiene el espacio producto $\left(E_{2},\mathcal{E}_{2}\right)=\otimes_{j=1}^{4}\left(E_{j},\mathcal{E}_{j}\right)$,
\item[b)] Para $B\left(t\right)=\left(B_{j}\left(t\right)\right)_{j=1}^{4}$ se tiene el espacio producto $\left(E_{3},\mathcal{E}_{3}\right)=\otimes_{j=1}^{4}\left(E_{j},\mathcal{E}_{j}\right)$,

\item[c)] Para $C\left(t\right)=\left(C_{j}\left(t\right)\right)_{j=1}^{4}$ se tiene el espacio producto $\left(E_{4},\mathcal{E}_{4}\right)=\otimes_{j=1}^{4}\left(E_{j},\mathcal{E}_{j}\right)$.
\end{itemize}


En lo que respecta al proceso $L\left(t\right)=\left(L_{j}\left(t\right)\right)_{j=1}^{4}$
 el proceso de estados $E_{j}\subset\mathbb{N}$ y $\mathcal{E}_{j}\subset\sigma\left(E\right)$, por lo tanto el espacio producto $\left(E_{1},\mathcal{E}_{1}\right)=\otimes_{j=1}^{4}\left(E_{j},\mathcal{E}_{j}\right)$ que adem\'as tambi\'en resulta ser polaco. Entonces con los espacios productos $\left(E_{i},\mathcal{E}_{i}\right)_{i=1}^{4}$, se define el espacio producto $\left(E,\mathcal{E}\right)=\otimes_{i=1}^{4} \left(E_{i},\mathcal{E}_{i}\right)$ que nuevamente resulta ser un espacio polaco. De acuerdo con Thorisson existe un espacio de probabilidad $\left(\Omega,\mathcal{F},\prob\right)$ en el que est\'a definido el proceso estoc\'astico definido en  (\ref{Proceso.Estocastico.Z}) que toma valores en $\left(E,\mathcal{E}\right)$.
 
 
Con la finalidad de analizar las propiedades del proceso $\mathbb{Z}$ consideremos el conjunto de \'indices $\mathbb{I}=\left[0,\infty\right)$, entonces tenemos el elemento aleatorio $\mathbb{Z}=\left(Z\right)_{s\in\mathbb{I}}$ que est\'a definido en el espacio de probabilidad $\left(\Omega,\mathcal{F},\prob\right)$ y con valores en $\left(E,\mathcal{E}\right)$. El proceso $Z$ as\'i definido es un PEOSCT conforme a la definici\'on dada en 

\begin{Teo}
Dada una Red de Sistemas de Visitas C\'iclicas (RSVC), conformada por dos Sistemas de Visitas C\'iclicas (SVC), donde cada uno de ellos consta de dos colas tipo $M/M/1$. Los dos sistemas est\'an comunicados entre s\'i por medio de la transferencia de usuarios entre las colas $Q_{1}\leftrightarrow Q_{3}$ y $Q_{2}\leftrightarrow Q_{4}$. Se definen los eventos para los procesos de arribos al tiempo $t$, $A_{j}\left(t\right)=\left\{0 \textrm{ arribos en }Q_{j}\textrm{ al tiempo }t\right\}$ para alg\'un tiempo $t\geq0$ y $Q_{j}$ la cola $j$-\'esima en la RSVC, para $j=1,2,3,4$.  Existe un intervalo $I\neq\emptyset$ tal que para $T^{*}\in I$, tal que $\prob\left\{A_{1}\left(T^{*}\right),A_{2}\left(Tt^{*}\right),
A_{3}\left(T^{*}\right),A_{4}\left(T^{*}\right)|T^{*}\in I\right\}>0$.
\end{Teo}



\begin{proof}
Sin p\'erdida de generalidad podemos considerar como base del an\'alisis a la cola $Q_{1}$ del primer sistema que conforma la RSVC.\medskip 

Sea $n\geq1$, ciclo en el primer sistema en el que se sabe que $L_{j}\left(\overline{\tau}_{1}\left(n\right)\right)=0$, pues la pol\'itica de servicio con que atienden los servidores es la exhaustiva. Como es sabido, para trasladarse a la siguiente cola, el servidor incurre en un tiempo de traslado $r_{1}\left(n\right)>0$, entonces tenemos que $\tau_{2}\left(n\right)=\overline{\tau}_{1}\left(n\right)+r_{1}\left(n\right)$.\medskip 


Definamos el intervalo $I_{1}\equiv\left[\overline{\tau}_{1}\left(n\right),\tau_{2}\left(n\right)\right]$ de longitud $\xi_{1}=r_{1}\left(n\right)$.

Dado que los tiempos entre arribo son exponenciales con tasa $\tilde{\mu}_{1}=\mu_{1}+\hat{\mu}_{1}$ ($\mu_{1}$ son los arribos a $Q_{1}$ por primera vez al sistema, mientras que $\hat{\mu}_{1}$ son los arribos de traslado procedentes de $Q_{3}$) se tiene que la probabilidad del evento $A_{1}\left(t\right)$ est\'a dada por 

\begin{equation}
\prob\left\{A_{1}\left(t\right)|t\in I_{1}\left(n\right)\right\}=e^{-\tilde{\mu}_{1}\xi_{1}\left(n\right)}.
\end{equation} 


Por otra parte, para la cola $Q_{2}$ el tiempo $\overline{\tau}_{2}\left(n-1\right)$ es tal que $L_{2}\left(\overline{\tau}_{2}\left(n-1\right)\right)=0$, es decir, es el tiempo en que la cola queda totalmente vac\'ia en el ciclo anterior a $n$. \medskip 


Entonces tenemos un sgundo intervalo $I_{2}\equiv\left[\overline{\tau}_{2}\left(n-1\right),\tau_{2}\left(n\right)\right]$. Por lo tanto la probabilidad del evento $A_{2}\left(t\right)$ tiene probabilidad dada por

\begin{eqnarray}
\prob\left\{A_{2}\left(t\right)|t\in I_{2}\left(n\right)\right\}=e^{-\tilde{\mu}_{2}\xi_{2}\left(n\right)},\\
\xi_{2}\left(n\right)=\tau_{2}\left(n\right)-\overline{\tau}_{2}\left(n-1\right)
\end{eqnarray}
%\end{equation} 

%donde $$.

Ahora, dado que $I_{1}\left(n\right)\subset I_{2}\left(n\right)$, se tiene que

\begin{eqnarray*}
\xi_{1}\left(n\right)\leq\xi_{2}\left(n\right)&\Leftrightarrow& -\xi_{1}\left(n\right)\geq-\xi_{2}\left(n\right)
\\
-\tilde{\mu}_{2}\xi_{1}\left(n\right)\geq-\tilde{\mu}_{2}\xi_{2}\left(n\right)&\Leftrightarrow&
e^{-\tilde{\mu}_{2}\xi_{1}\left(n\right)}\geq e^{-\tilde{\mu}_{2}\xi_{2}\left(n\right)}\\
\prob\left\{A_{2}\left(t\right)|t\in I_{1}\left(n\right)\right\}&\geq&
\prob\left\{A_{2}\left(t\right)|t\in I_{2}\left(n\right)\right\}.
\end{eqnarray*}


Entonces se tiene que
\small{
\begin{eqnarray*}
\prob\left\{A_{1}\left(t\right),A_{2}\left(t\right)|t\in I_{1}\left(n\right)\right\}&=&
\prob\left\{A_{1}\left(t\right)|t\in I_{1}\left(n\right)\right\}
\prob\left\{A_{2}\left(t\right)|t\in I_{1}\left(n\right)\right\}\\
&\geq&
\prob\left\{A_{1}\left(t\right)|t\in I_{1}\left(n\right)\right\}
\prob\left\{A_{2}\left(t\right)|t\in I_{2}\left(n\right)\right\}\\
&=&e^{-\tilde{\mu}_{1}\xi_{1}\left(n\right)}e^{-\tilde{\mu}_{2}\xi_{2}\left(n\right)}
=e^{-\left[\tilde{\mu}_{1}\xi_{1}\left(n\right)+\tilde{\mu}_{2}\xi_{2}\left(n\right)\right]}.
\end{eqnarray*}}


Es decir, 

\begin{equation}
\prob\left\{A_{1}\left(t\right),A_{2}\left(t\right)|t\in I_{1}\left(n\right)\right\}
=e^{-\left[\tilde{\mu}_{1}\xi_{1}\left(n\right)+\tilde{\mu}_{2}\xi_{2}
\left(n\right)\right]}>0.
\end{equation}
En lo que respecta a la relaci\'on entre los dos SVC que conforman la RSVC para alg\'un $m\geq1$ se tiene que $\tau_{3}\left(m\right)<\tau_{2}\left(n\right)<\tau_{4}\left(m\right)$ por lo tanto se cumple cualquiera de los siguientes cuatro casos
\begin{itemize}
\item[a)] $\tau_{3}\left(m\right)<\tau_{2}\left(n\right)<\overline{\tau}_{3}\left(m\right)$

\item[b)] $\overline{\tau}_{3}\left(m\right)<\tau_{2}\left(n\right)
<\tau_{4}\left(m\right)$

\item[c)] $\tau_{4}\left(m\right)<\tau_{2}\left(n\right)<
\overline{\tau}_{4}\left(m\right)$

\item[d)] $\overline{\tau}_{4}\left(m\right)<\tau_{2}\left(n\right)
<\tau_{3}\left(m+1\right)$
\end{itemize}


Sea el intervalo $I_{3}\left(m\right)\equiv\left[\tau_{3}\left(m\right),\overline{\tau}_{3}\left(m\right)\right]$ tal que $\tau_{2}\left(n\right)\in I_{3}\left(m\right)$, con longitud de intervalo $\xi_{3}\equiv\overline{\tau}_{3}\left(m\right)-\tau_{3}\left(m\right)$, entonces se tiene que para $Q_{3}$
\begin{equation}
\prob\left\{A_{3}\left(t\right)|t\in I_{3}\left(m\right)\right\}=e^{-\tilde{\mu}_{3}\xi_{3}\left(m\right)}.
\end{equation} 

mientras que para $Q_{4}$ consideremos el intervalo $I_{4}\left(m\right)\equiv\left[\tau_{4}\left(m-1\right),\overline{\tau}_{3}\left(m\right)\right]$, entonces por construcci\'on  $I_{3}\left(m\right)\subset I_{4}\left(m\right)$, por lo tanto


\begin{eqnarray*}
\xi_{3}\left(m\right)\leq\xi_{4}\left(m\right)&\Leftrightarrow& -\xi_{3}\left(m\right)\geq-\xi_{4}\left(m\right)
\\
-\tilde{\mu}_{4}\xi_{3}\left(m\right)\geq-\tilde{\mu}_{4}\xi_{4}\left(m\right)&\Leftrightarrow&
e^{-\tilde{\mu}_{4}\xi_{3}\left(m\right)}\geq e^{-\tilde{\mu}_{4}\xi_{4}\left(n\right)}\\
\prob\left\{A_{4}\left(t\right)|t\in I_{3}\left(m\right)\right\}&\geq&
\prob\left\{A_{4}\left(t\right)|t\in I_{4}\left(m\right)\right\}.
\end{eqnarray*}



Entonces se tiene que
\small{
\begin{eqnarray*}
\prob\left\{A_{3}\left(t\right),A_{4}\left(t\right)|t\in I_{3}\left(m\right)\right\}&=&
\prob\left\{A_{3}\left(t\right)|t\in I_{3}\left(m\right)\right\}
\prob\left\{A_{4}\left(t\right)|t\in I_{3}\left(m\right)\right\}\\
&\geq&
\prob\left\{A_{3}\left(t\right)|t\in I_{3}\left(m\right)\right\}
\prob\left\{A_{4}\left(t\right)|t\in I_{4}\left(m\right)\right\}\\
&=&e^{-\tilde{\mu}_{3}\xi_{3}\left(m\right)}e^{-\tilde{\mu}_{4}\xi_{4}
\left(m\right)}
=e^{-\left(\tilde{\mu}_{3}\xi_{3}\left(m\right)+\tilde{\mu}_{4}\xi_{4}\left(m\right)\right)}.
\end{eqnarray*}}

Es decir, 

\begin{equation}
\prob\left\{A_{3}\left(t\right),A_{4}\left(t\right)|t\in I_{3}\left(m\right)\right\}\geq
e^{-\left(\tilde{\mu}_{3}\xi_{3}\left(m\right)+\tilde{\mu}_{4}\xi_{4}\left(m\right)\right)}>0.
\end{equation}


Sea el intervalo $I_{3}\left(m\right)\equiv\left[\overline{\tau}_{3}\left(m\right),\tau_{4}\left(m\right)\right]$ con longitud $\xi_{3}\equiv\tau_{4}\left(m\right)-\overline{\tau}_{3}\left(m\right)$, entonces se tiene que para $Q_{3}$
\begin{equation}
\prob\left\{A_{3}\left(t\right)|t\in I_{3}\left(m\right)\right\}=e^{-\tilde{\mu}_{3}\xi_{3}\left(m\right)}.
\end{equation} 

mientras que para $Q_{4}$ consideremos el intervalo $I_{4}\left(m\right)\equiv\left[\overline{\tau}_{4}\left(m-1\right),\tau_{4}\left(m\right)\right]$, entonces por construcci\'on  $I_{3}\left(m\right)\subset I_{4}\left(m\right)$, y al igual que en el caso anterior se tiene que 

\begin{eqnarray*}
\xi_{3}\left(m\right)\leq\xi_{4}\left(m\right)&\Leftrightarrow& -\xi_{3}\left(m\right)\geq-\xi_{4}\left(m\right)
\\
-\tilde{\mu}_{4}\xi_{3}\left(m\right)\geq-\tilde{\mu}_{4}\xi_{4}\left(m\right)&\Leftrightarrow&
e^{-\tilde{\mu}_{4}\xi_{3}\left(m\right)}\geq e^{-\tilde{\mu}_{4}\xi_{4}\left(n\right)}\\
\prob\left\{A_{4}\left(t\right)|t\in I_{3}\left(m\right)\right\}&\geq&
\prob\left\{A_{4}\left(t\right)|t\in I_{4}\left(m\right)\right\}.
\end{eqnarray*}


Entonces se tiene que
\small{
\begin{eqnarray*}
\prob\left\{A_{3}\left(t\right),A_{4}\left(t\right)|t\in I_{3}\left(m\right)\right\}&=&
\prob\left\{A_{3}\left(t\right)|t\in I_{3}\left(m\right)\right\}
\prob\left\{A_{4}\left(t\right)|t\in I_{3}\left(m\right)\right\}\\
&\geq&
\prob\left\{A_{3}\left(t\right)|t\in I_{3}\left(m\right)\right\}
\prob\left\{A_{4}\left(t\right)|t\in I_{4}\left(m\right)\right\}\\
&=&e^{-\tilde{\mu}_{3}\xi_{3}\left(m\right)}e^{-\tilde{\mu}_{4}\xi_{4}\left(m\right)}
=e^{-\left(\tilde{\mu}_{3}\xi_{3}\left(m\right)+\tilde{\mu}_{4}\xi_{4}\left(m\right)\right)}.
\end{eqnarray*}}

Es decir, 

\begin{equation}
\prob\left\{A_{3}\left(t\right),A_{4}\left(t\right)|t\in I_{4}\left(m\right)\right\}\geq
e^{-\left(\tilde{\mu}_{3}+\tilde{\mu}_{4}\right)\xi_{3}\left(m\right)}>0.
\end{equation}


Para el intervalo $I_{3}\left(m\right)=\left[\tau_{4}\left(m\right),\overline{\tau}_{4}\left(m\right)\right]$, se tiene que este caso es an\'alogo al caso (a).


Para el intevalo $I_{3}\left(m\right)\equiv\left[\overline{\tau}_{4}\left(m\right),\tau_{4}\left(m+1\right)\right]$, se tiene que es an\'alogo al caso (b).


Por construcci\'on se tiene que $I\left(n,m\right)\equiv I_{1}\left(n\right)\cap I_{3}\left(m\right)\neq\emptyset$,entonces en particular se tienen las contenciones $I\left(n,m\right)\subseteq I_{1}\left(n\right)$ y $I\left(n,m\right)\subseteq I_{3}\left(m\right)$, por lo tanto si definimos $\xi_{n,m}\equiv\ell\left(I\left(n,m\right)\right)$ tenemos que

\begin{eqnarray*}
\xi_{n,m}\leq\xi_{1}\left(n\right)\textrm{ y }\xi_{n,m}\leq\xi_{3}\left(m\right)\textrm{ entonces }\\
-\xi_{n,m}\geq-\xi_{1}\left(n\right)\textrm{ y }-\xi_{n,m}\leq-\xi_{3}\left(m\right)\\
\end{eqnarray*}
por lo tanto tenemos las desigualdades 


\begin{eqnarray*}
\begin{array}{ll}
-\tilde{\mu}_{1}\xi_{n,m}\geq-\tilde{\mu}_{1}\xi_{1}\left(n\right),&
-\tilde{\mu}_{2}\xi_{n,m}\geq-\tilde{\mu}_{2}\xi_{1}\left(n\right)
\geq-\tilde{\mu}_{2}\xi_{2}\left(n\right),\\
-\tilde{\mu}_{3}\xi_{n,m}\geq-\tilde{\mu}_{3}\xi_{3}\left(m\right),&
-\tilde{\mu}_{4}\xi_{n,m}\geq-\tilde{\mu}_{4}\xi_{3}\left(m\right)
\geq-\tilde{\mu}_{4}\xi_{4}\left(m\right).
\end{array}
\end{eqnarray*}

Sea $T^{*}\in I\left(n,m\right)$, entonces dado que en particular $T^{*}\in I_{1}\left(n\right)$, se cumple con probabilidad positiva que no hay arribos a las colas $Q_{1}$ y $Q_{2}$, en consecuencia, tampoco hay usuarios de transferencia para $Q_{3}$ y $Q_{4}$, es decir, $\tilde{\mu}_{1}=\mu_{1}$, $\tilde{\mu}_{2}=\mu_{2}$, $\tilde{\mu}_{3}=\mu_{3}$, $\tilde{\mu}_{4}=\mu_{4}$, es decir, los eventos $Q_{1}$ y $Q_{3}$ son condicionalmente independientes en el intervalo $I\left(n,m\right)$; lo mismo ocurre para las colas $Q_{2}$ y $Q_{4}$, por lo tanto tenemos que
%\small{
\begin{eqnarray}
\begin{array}{l}
\prob\left\{A_{1}\left(T^{*}\right),A_{2}\left(T^{*}\right),
A_{3}\left(T^{*}\right),A_{4}\left(T^{*}\right)|T^{*}\in I\left(n,m\right)\right\}\\
=\prod_{j=1}^{4}\prob\left\{A_{j}\left(T^{*}\right)|T^{*}\in I\left(n,m\right)\right\}\\
\geq\prob\left\{A_{1}\left(T^{*}\right)|T^{*}\in I_{1}\left(n\right)\right\}
\prob\left\{A_{2}\left(T^{*}\right)|T^{*}\in I_{2}\left(n\right)\right\}\\
\prob\left\{A_{3}\left(T^{*}\right)|T^{*}\in I_{3}\left(m\right)\right\}
\prob\left\{A_{4}\left(T^{*}\right)|T^{*}\in I_{4}\left(m\right)\right\}\\
=e^{-\mu_{1}\xi_{1}\left(n\right)}
e^{-\mu_{2}\xi_{2}\left(n\right)}
e^{-\mu_{3}\xi_{3}\left(m\right)}
e^{-\mu_{4}\xi_{4}\left(m\right)}\\
=e^{-\left[\tilde{\mu}_{1}\xi_{1}\left(n\right)
+\tilde{\mu}_{2}\xi_{2}\left(n\right)
+\tilde{\mu}_{3}\xi_{3}\left(m\right)
+\tilde{\mu}_{4}\xi_{4}
\left(m\right)\right]}>0.
\end{array}
\end{eqnarray}


Ahora solo resta demostrar que para $n\ge1$, existe $m\geq1$ tal que se cumplen cualquiera de los cuatro casos arriba mencionados: 

\begin{itemize}
\item[a)] $\tau_{3}\left(m\right)<\tau_{2}\left(n\right)<\overline{\tau}_{3}\left(m\right)$

\item[b)] $\overline{\tau}_{3}\left(m\right)<\tau_{2}\left(n\right)
<\tau_{4}\left(m\right)$

\item[c)] $\tau_{4}\left(m\right)<\tau_{2}\left(n\right)<
\overline{\tau}_{4}\left(m\right)$

\item[d)] $\overline{\tau}_{4}\left(m\right)<\tau_{2}\left(n\right)
<\tau_{3}\left(m+1\right)$
\end{itemize}

Consideremos nuevamente el primer caso. Supongamos que no existe $m\geq1$, tal que $I_{1}\left(n\right)\cap I_{3}\left(m\right)\neq\emptyset$, es decir, para toda $m\geq1$, $I_{1}\left(n\right)\cap I_{3}\left(m\right)=\emptyset$, entonces se tiene que ocurren cualquiera de los dos casos

\begin{itemize}
\item[a)] $\tau_{2}\left(n\right)\leq\tau_{3}\left(m\right)$: Recordemos que $\tau_{2}\left(m\right)=\overline{\tau}_{1}+r_{1}\left(m\right)$ donde cada una de las variables aleatorias son tales que $\esp\left[\overline{\tau}_{1}\left(n\right)-\tau_{1}\left(n\right)\right]<\infty$, $\esp\left[R_{1}\right]<\infty$ y $\esp\left[\tau_{3}\left(m\right)\right]<\infty$, lo cual contradice el hecho de que no exista un ciclo $m\geq1$ que satisfaga la condici\'on deseada.

\item[b)] $\tau_{2}\left(n\right)\geq\overline{\tau}_{3}\left(m\right)$: por un argumento similar al anterior se tiene que no es posible que no exista un ciclo $m\geq1$ tal que satisaface la condici\'on deseada.

\end{itemize}

Para el resto de los casos la demostraci\'on es an\'aloga. Por lo tanto, se tiene que efectivamente existe $m\geq1$ tal que $\tau_{3}\left(m\right)<\tau_{2}\left(n\right)<\tau_{4}\left(m\right)$.
\end{proof}
\newpage

En Sigman, Thorison y Wolff \cite{Sigman2} prueban que para la existencia de un una sucesi\'on infinita no decreciente de tiempos de regeneraci\'on $\tau_{1}\leq\tau_{2}\leq\cdots$ en los cuales el proceso se regenera, basta un tiempo de regeneraci\'on $R_{1}$, donde $R_{j}=\tau_{j}-\tau_{j-1}$. Para tal efecto se requiere la existencia de un espacio de probabilidad $\left(\Omega,\mathcal{F},\prob\right)$, y proceso estoc\'astico $\textit{X}=\left\{X\left(t\right):t\geq0\right\}$ con espacio de estados $\left(S,\mathcal{R}\right)$, con $\mathcal{R}$ $\sigma$-\'algebra.

\begin{Prop}
Si existe una variable aleatoria no negativa $R_{1}$ tal que $\theta_{R1}X=_{D}X$, entonces $\left(\Omega,\mathcal{F},\prob\right)$ puede extenderse para soportar una sucesi\'on estacionaria de variables aleatorias $R=\left\{R_{k}:k\geq1\right\}$, tal que para $k\geq1$,
\begin{eqnarray*}
\theta_{k}\left(X,R\right)=_{D}\left(X,R\right).
\end{eqnarray*}

Adem\'as, para $k\geq1$, $\theta_{k}R$ es condicionalmente independiente de $\left(X,R_{1},\ldots,R_{k}\right)$, dado $\theta_{\tau k}X$.

\end{Prop}


\begin{itemize}
\item Doob en 1953 demostr\'o que el estado estacionario de un proceso de partida en un sistema de espera $M/G/\infty$, es Poisson con la misma tasa que el proceso de arribos.

\item Burke en 1968, fue el primero en demostrar que el estado estacionario de un proceso de salida de una cola $M/M/s$ es un proceso Poisson.

\item Disney en 1973 obtuvo el siguiente resultado:

\begin{Teo}
Para el sistema de espera $M/G/1/L$ con disciplina FIFO, el proceso $\textbf{I}$ es un proceso de renovaci\'on si y s\'olo si el proceso denominado longitud de la cola es estacionario y se cumple cualquiera de los siguientes casos:

\begin{itemize}
\item[a)] Los tiempos de servicio son identicamente cero;
\item[b)] $L=0$, para cualquier proceso de servicio $S$;
\item[c)] $L=1$ y $G=D$;
\item[d)] $L=\infty$ y $G=M$.
\end{itemize}
En estos casos, respectivamente, las distribuciones de interpartida $P\left\{T_{n+1}-T_{n}\leq t\right\}$ son


\begin{itemize}
\item[a)] $1-e^{-\lambda t}$, $t\geq0$;
\item[b)] $1-e^{-\lambda t}*F\left(t\right)$, $t\geq0$;
\item[c)] $1-e^{-\lambda t}*\indora_{d}\left(t\right)$, $t\geq0$;
\item[d)] $1-e^{-\lambda t}*F\left(t\right)$, $t\geq0$.
\end{itemize}
\end{Teo}


\item Finch (1959) mostr\'o que para los sistemas $M/G/1/L$, con $1\leq L\leq \infty$ con distribuciones de servicio dos veces diferenciable, solamente el sistema $M/M/1/\infty$ tiene proceso de salida de renovaci\'on estacionario.

\item King (1971) demostro que un sistema de colas estacionario $M/G/1/1$ tiene sus tiempos de interpartida sucesivas $D_{n}$ y $D_{n+1}$ son independientes, si y s\'olo si, $G=D$, en cuyo caso le proceso de salida es de renovaci\'on.

\item Disney (1973) demostr\'o que el \'unico sistema estacionario $M/G/1/L$, que tiene proceso de salida de renovaci\'on  son los sistemas $M/M/1$ y $M/D/1/1$.



\item El siguiente resultado es de Disney y Koning (1985)
\begin{Teo}
En un sistema de espera $M/G/s$, el estado estacionario del proceso de salida es un proceso Poisson para cualquier distribuci\'on de los tiempos de servicio si el sistema tiene cualquiera de las siguientes cuatro propiedades.

\begin{itemize}
\item[a)] $s=\infty$
\item[b)] La disciplina de servicio es de procesador compartido.
\item[c)] La disciplina de servicio es LCFS y preemptive resume, esto se cumple para $L<\infty$
\item[d)] $G=M$.
\end{itemize}

\end{Teo}

\item El siguiente resultado es de Alamatsaz (1983)

\begin{Teo}
En cualquier sistema de colas $GI/G/1/L$ con $1\leq L<\infty$ y distribuci\'on de interarribos $A$ y distribuci\'on de los tiempos de servicio $B$, tal que $A\left(0\right)=0$, $A\left(t\right)\left(1-B\left(t\right)\right)>0$ para alguna $t>0$ y $B\left(t\right)$ para toda $t>0$, es imposible que el proceso de salida estacionario sea de renovaci\'on.
\end{Teo}

\end{itemize}



%________________________________________________________________________
%\subsection{Procesos Regenerativos Sigman, Thorisson y Wolff \cite{Sigman1}}
%________________________________________________________________________


\begin{Def}[Definici\'on Cl\'asica]
Un proceso estoc\'astico $X=\left\{X\left(t\right):t\geq0\right\}$ es llamado regenerativo is existe una variable aleatoria $R_{1}>0$ tal que
\begin{itemize}
\item[i)] $\left\{X\left(t+R_{1}\right):t\geq0\right\}$ es independiente de $\left\{\left\{X\left(t\right):t<R_{1}\right\},\right\}$
\item[ii)] $\left\{X\left(t+R_{1}\right):t\geq0\right\}$ es estoc\'asticamente equivalente a $\left\{X\left(t\right):t>0\right\}$
\end{itemize}

Llamamos a $R_{1}$ tiempo de regeneraci\'on, y decimos que $X$ se regenera en este punto.
\end{Def}

$\left\{X\left(t+R_{1}\right)\right\}$ es regenerativo con tiempo de regeneraci\'on $R_{2}$, independiente de $R_{1}$ pero con la misma distribuci\'on que $R_{1}$. Procediendo de esta manera se obtiene una secuencia de variables aleatorias independientes e id\'enticamente distribuidas $\left\{R_{n}\right\}$ llamados longitudes de ciclo. Si definimos a $Z_{k}\equiv R_{1}+R_{2}+\cdots+R_{k}$, se tiene un proceso de renovaci\'on llamado proceso de renovaci\'on encajado para $X$.


\begin{Note}
La existencia de un primer tiempo de regeneraci\'on, $R_{1}$, implica la existencia de una sucesi\'on completa de estos tiempos $R_{1},R_{2}\ldots,$ que satisfacen la propiedad deseada \cite{Sigman2}.
\end{Note}


\begin{Note} Para la cola $GI/GI/1$ los usuarios arriban con tiempos $t_{n}$ y son atendidos con tiempos de servicio $S_{n}$, los tiempos de arribo forman un proceso de renovaci\'on  con tiempos entre arribos independientes e identicamente distribuidos (\texttt{i.i.d.})$T_{n}=t_{n}-t_{n-1}$, adem\'as los tiempos de servicio son \texttt{i.i.d.} e independientes de los procesos de arribo. Por \textit{estable} se entiende que $\esp S_{n}<\esp T_{n}<\infty$.
\end{Note}
 


\begin{Def}
Para $x$ fijo y para cada $t\geq0$, sea $I_{x}\left(t\right)=1$ si $X\left(t\right)\leq x$,  $I_{x}\left(t\right)=0$ en caso contrario, y def\'inanse los tiempos promedio
\begin{eqnarray*}
\overline{X}&=&lim_{t\rightarrow\infty}\frac{1}{t}\int_{0}^{\infty}X\left(u\right)du\\
\prob\left(X_{\infty}\leq x\right)&=&lim_{t\rightarrow\infty}\frac{1}{t}\int_{0}^{\infty}I_{x}\left(u\right)du,
\end{eqnarray*}
cuando estos l\'imites existan.
\end{Def}

Como consecuencia del teorema de Renovaci\'on-Recompensa, se tiene que el primer l\'imite  existe y es igual a la constante
\begin{eqnarray*}
\overline{X}&=&\frac{\esp\left[\int_{0}^{R_{1}}X\left(t\right)dt\right]}{\esp\left[R_{1}\right]},
\end{eqnarray*}
suponiendo que ambas esperanzas son finitas.
 
\begin{Note}
Funciones de procesos regenerativos son regenerativas, es decir, si $X\left(t\right)$ es regenerativo y se define el proceso $Y\left(t\right)$ por $Y\left(t\right)=f\left(X\left(t\right)\right)$ para alguna funci\'on Borel medible $f\left(\cdot\right)$. Adem\'as $Y$ es regenerativo con los mismos tiempos de renovaci\'on que $X$. 

En general, los tiempos de renovaci\'on, $Z_{k}$ de un proceso regenerativo no requieren ser tiempos de paro con respecto a la evoluci\'on de $X\left(t\right)$.
\end{Note} 

\begin{Note}
Una funci\'on de un proceso de Markov, usualmente no ser\'a un proceso de Markov, sin embargo ser\'a regenerativo si el proceso de Markov lo es.
\end{Note}

 
\begin{Note}
Un proceso regenerativo con media de la longitud de ciclo finita es llamado positivo recurrente.
\end{Note}


\begin{Note}
\begin{itemize}
\item[a)] Si el proceso regenerativo $X$ es positivo recurrente y tiene trayectorias muestrales no negativas, entonces la ecuaci\'on anterior es v\'alida.
\item[b)] Si $X$ es positivo recurrente regenerativo, podemos construir una \'unica versi\'on estacionaria de este proceso, $X_{e}=\left\{X_{e}\left(t\right)\right\}$, donde $X_{e}$ es un proceso estoc\'astico regenerativo y estrictamente estacionario, con distribuci\'on marginal distribuida como $X_{\infty}$
\end{itemize}
\end{Note}


%__________________________________________________________________________________________
%\subsection{Procesos Regenerativos Estacionarios - Stidham \cite{Stidham}}
%__________________________________________________________________________________________


Un proceso estoc\'astico a tiempo continuo $\left\{V\left(t\right),t\geq0\right\}$ es un proceso regenerativo si existe una sucesi\'on de variables aleatorias independientes e id\'enticamente distribuidas $\left\{X_{1},X_{2},\ldots\right\}$, sucesi\'on de renovaci\'on, tal que para cualquier conjunto de Borel $A$, 

\begin{eqnarray*}
\prob\left\{V\left(t\right)\in A|X_{1}+X_{2}+\cdots+X_{R\left(t\right)}=s,\left\{V\left(\tau\right),\tau<s\right\}\right\}=\prob\left\{V\left(t-s\right)\in A|X_{1}>t-s\right\},
\end{eqnarray*}
para todo $0\leq s\leq t$, donde $R\left(t\right)=\max\left\{X_{1}+X_{2}+\cdots+X_{j}\leq t\right\}=$n\'umero de renovaciones ({\emph{puntos de regeneraci\'on}}) que ocurren en $\left[0,t\right]$. El intervalo $\left[0,X_{1}\right)$ es llamado {\emph{primer ciclo de regeneraci\'on}} de $\left\{V\left(t \right),t\geq0\right\}$, $\left[X_{1},X_{1}+X_{2}\right)$ el {\emph{segundo ciclo de regeneraci\'on}}, y as\'i sucesivamente.

Sea $X=X_{1}$ y sea $F$ la funci\'on de distrbuci\'on de $X$


\begin{Def}
Se define el proceso estacionario, $\left\{V^{*}\left(t\right),t\geq0\right\}$, para $\left\{V\left(t\right),t\geq0\right\}$ por

\begin{eqnarray*}
\prob\left\{V\left(t\right)\in A\right\}=\frac{1}{\esp\left[X\right]}\int_{0}^{\infty}\prob\left\{V\left(t+x\right)\in A|X>x\right\}\left(1-F\left(x\right)\right)dx,
\end{eqnarray*} 
para todo $t\geq0$ y todo conjunto de Borel $A$.
\end{Def}

\begin{Def}
Una distribuci\'on se dice que es {\emph{aritm\'etica}} si todos sus puntos de incremento son m\'ultiplos de la forma $0,\lambda, 2\lambda,\ldots$ para alguna $\lambda>0$ entera.
\end{Def}


\begin{Def}
Una modificaci\'on medible de un proceso $\left\{V\left(t\right),t\geq0\right\}$, es una versi\'on de este, $\left\{V\left(t,w\right)\right\}$ conjuntamente medible para $t\geq0$ y para $w\in S$, $S$ espacio de estados para $\left\{V\left(t\right),t\geq0\right\}$.
\end{Def}

\begin{Teo}
Sea $\left\{V\left(t\right),t\geq\right\}$ un proceso regenerativo no negativo con modificaci\'on medible. Sea $\esp\left[X\right]<\infty$. Entonces el proceso estacionario dado por la ecuaci\'on anterior est\'a bien definido y tiene funci\'on de distribuci\'on independiente de $t$, adem\'as
\begin{itemize}
\item[i)] \begin{eqnarray*}
\esp\left[V^{*}\left(0\right)\right]&=&\frac{\esp\left[\int_{0}^{X}V\left(s\right)ds\right]}{\esp\left[X\right]}\end{eqnarray*}
\item[ii)] Si $\esp\left[V^{*}\left(0\right)\right]<\infty$, equivalentemente, si $\esp\left[\int_{0}^{X}V\left(s\right)ds\right]<\infty$,entonces
\begin{eqnarray*}
\frac{\int_{0}^{t}V\left(s\right)ds}{t}\rightarrow\frac{\esp\left[\int_{0}^{X}V\left(s\right)ds\right]}{\esp\left[X\right]}
\end{eqnarray*}
con probabilidad 1 y en media, cuando $t\rightarrow\infty$.
\end{itemize}
\end{Teo}

\begin{Coro}
Sea $\left\{V\left(t\right),t\geq0\right\}$ un proceso regenerativo no negativo, con modificaci\'on medible. Si $\esp <\infty$, $F$ es no-aritm\'etica, y para todo $x\geq0$, $P\left\{V\left(t\right)\leq x,C>x\right\}$ es de variaci\'on acotada como funci\'on de $t$ en cada intervalo finito $\left[0,\tau\right]$, entonces $V\left(t\right)$ converge en distribuci\'on  cuando $t\rightarrow\infty$ y $$\esp V=\frac{\esp \int_{0}^{X}V\left(s\right)ds}{\esp X}$$
Donde $V$ tiene la distribuci\'on l\'imite de $V\left(t\right)$ cuando $t\rightarrow\infty$.

\end{Coro}

Para el caso discreto se tienen resultados similares.



%______________________________________________________________________
%\subsection{Procesos de Renovaci\'on}
%______________________________________________________________________

\begin{Def}%\label{Def.Tn}
Sean $0\leq T_{1}\leq T_{2}\leq \ldots$ son tiempos aleatorios infinitos en los cuales ocurren ciertos eventos. El n\'umero de tiempos $T_{n}$ en el intervalo $\left[0,t\right)$ es

\begin{eqnarray}
N\left(t\right)=\sum_{n=1}^{\infty}\indora\left(T_{n}\leq t\right),
\end{eqnarray}
para $t\geq0$.
\end{Def}

Si se consideran los puntos $T_{n}$ como elementos de $\rea_{+}$, y $N\left(t\right)$ es el n\'umero de puntos en $\rea$. El proceso denotado por $\left\{N\left(t\right):t\geq0\right\}$, denotado por $N\left(t\right)$, es un proceso puntual en $\rea_{+}$. Los $T_{n}$ son los tiempos de ocurrencia, el proceso puntual $N\left(t\right)$ es simple si su n\'umero de ocurrencias son distintas: $0<T_{1}<T_{2}<\ldots$ casi seguramente.

\begin{Def}
Un proceso puntual $N\left(t\right)$ es un proceso de renovaci\'on si los tiempos de interocurrencia $\xi_{n}=T_{n}-T_{n-1}$, para $n\geq1$, son independientes e identicamente distribuidos con distribuci\'on $F$, donde $F\left(0\right)=0$ y $T_{0}=0$. Los $T_{n}$ son llamados tiempos de renovaci\'on, referente a la independencia o renovaci\'on de la informaci\'on estoc\'astica en estos tiempos. Los $\xi_{n}$ son los tiempos de inter-renovaci\'on, y $N\left(t\right)$ es el n\'umero de renovaciones en el intervalo $\left[0,t\right)$
\end{Def}


\begin{Note}
Para definir un proceso de renovaci\'on para cualquier contexto, solamente hay que especificar una distribuci\'on $F$, con $F\left(0\right)=0$, para los tiempos de inter-renovaci\'on. La funci\'on $F$ en turno degune las otra variables aleatorias. De manera formal, existe un espacio de probabilidad y una sucesi\'on de variables aleatorias $\xi_{1},\xi_{2},\ldots$ definidas en este con distribuci\'on $F$. Entonces las otras cantidades son $T_{n}=\sum_{k=1}^{n}\xi_{k}$ y $N\left(t\right)=\sum_{n=1}^{\infty}\indora\left(T_{n}\leq t\right)$, donde $T_{n}\rightarrow\infty$ casi seguramente por la Ley Fuerte de los Grandes Números.
\end{Note}

%___________________________________________________________________________________________
%
%\subsection{Teorema Principal de Renovaci\'on}
%___________________________________________________________________________________________
%

\begin{Note} Una funci\'on $h:\rea_{+}\rightarrow\rea$ es Directamente Riemann Integrable en los siguientes casos:
\begin{itemize}
\item[a)] $h\left(t\right)\geq0$ es decreciente y Riemann Integrable.
\item[b)] $h$ es continua excepto posiblemente en un conjunto de Lebesgue de medida 0, y $|h\left(t\right)|\leq b\left(t\right)$, donde $b$ es DRI.
\end{itemize}
\end{Note}

\begin{Teo}[Teorema Principal de Renovaci\'on]
Si $F$ es no aritm\'etica y $h\left(t\right)$ es Directamente Riemann Integrable (DRI), entonces

\begin{eqnarray*}
lim_{t\rightarrow\infty}U\star h=\frac{1}{\mu}\int_{\rea_{+}}h\left(s\right)ds.
\end{eqnarray*}
\end{Teo}

\begin{Prop}
Cualquier funci\'on $H\left(t\right)$ acotada en intervalos finitos y que es 0 para $t<0$ puede expresarse como
\begin{eqnarray*}
H\left(t\right)=U\star h\left(t\right)\textrm{,  donde }h\left(t\right)=H\left(t\right)-F\star H\left(t\right)
\end{eqnarray*}
\end{Prop}

\begin{Def}
Un proceso estoc\'astico $X\left(t\right)$ es crudamente regenerativo en un tiempo aleatorio positivo $T$ si
\begin{eqnarray*}
\esp\left[X\left(T+t\right)|T\right]=\esp\left[X\left(t\right)\right]\textrm{, para }t\geq0,\end{eqnarray*}
y con las esperanzas anteriores finitas.
\end{Def}

\begin{Prop}
Sup\'ongase que $X\left(t\right)$ es un proceso crudamente regenerativo en $T$, que tiene distribuci\'on $F$. Si $\esp\left[X\left(t\right)\right]$ es acotado en intervalos finitos, entonces
\begin{eqnarray*}
\esp\left[X\left(t\right)\right]=U\star h\left(t\right)\textrm{,  donde }h\left(t\right)=\esp\left[X\left(t\right)\indora\left(T>t\right)\right].
\end{eqnarray*}
\end{Prop}

\begin{Teo}[Regeneraci\'on Cruda]
Sup\'ongase que $X\left(t\right)$ es un proceso con valores positivo crudamente regenerativo en $T$, y def\'inase $M=\sup\left\{|X\left(t\right)|:t\leq T\right\}$. Si $T$ es no aritm\'etico y $M$ y $MT$ tienen media finita, entonces
\begin{eqnarray*}
lim_{t\rightarrow\infty}\esp\left[X\left(t\right)\right]=\frac{1}{\mu}\int_{\rea_{+}}h\left(s\right)ds,
\end{eqnarray*}
donde $h\left(t\right)=\esp\left[X\left(t\right)\indora\left(T>t\right)\right]$.
\end{Teo}

%___________________________________________________________________________________________
%
%\subsection{Propiedades de los Procesos de Renovaci\'on}
%___________________________________________________________________________________________
%

Los tiempos $T_{n}$ est\'an relacionados con los conteos de $N\left(t\right)$ por

\begin{eqnarray*}
\left\{N\left(t\right)\geq n\right\}&=&\left\{T_{n}\leq t\right\}\\
T_{N\left(t\right)}\leq &t&<T_{N\left(t\right)+1},
\end{eqnarray*}

adem\'as $N\left(T_{n}\right)=n$, y 

\begin{eqnarray*}
N\left(t\right)=\max\left\{n:T_{n}\leq t\right\}=\min\left\{n:T_{n+1}>t\right\}
\end{eqnarray*}

Por propiedades de la convoluci\'on se sabe que

\begin{eqnarray*}
P\left\{T_{n}\leq t\right\}=F^{n\star}\left(t\right)
\end{eqnarray*}
que es la $n$-\'esima convoluci\'on de $F$. Entonces 

\begin{eqnarray*}
\left\{N\left(t\right)\geq n\right\}&=&\left\{T_{n}\leq t\right\}\\
P\left\{N\left(t\right)\leq n\right\}&=&1-F^{\left(n+1\right)\star}\left(t\right)
\end{eqnarray*}

Adem\'as usando el hecho de que $\esp\left[N\left(t\right)\right]=\sum_{n=1}^{\infty}P\left\{N\left(t\right)\geq n\right\}$
se tiene que

\begin{eqnarray*}
\esp\left[N\left(t\right)\right]=\sum_{n=1}^{\infty}F^{n\star}\left(t\right)
\end{eqnarray*}

\begin{Prop}
Para cada $t\geq0$, la funci\'on generadora de momentos $\esp\left[e^{\alpha N\left(t\right)}\right]$ existe para alguna $\alpha$ en una vecindad del 0, y de aqu\'i que $\esp\left[N\left(t\right)^{m}\right]<\infty$, para $m\geq1$.
\end{Prop}


\begin{Note}
Si el primer tiempo de renovaci\'on $\xi_{1}$ no tiene la misma distribuci\'on que el resto de las $\xi_{n}$, para $n\geq2$, a $N\left(t\right)$ se le llama Proceso de Renovaci\'on retardado, donde si $\xi$ tiene distribuci\'on $G$, entonces el tiempo $T_{n}$ de la $n$-\'esima renovaci\'on tiene distribuci\'on $G\star F^{\left(n-1\right)\star}\left(t\right)$
\end{Note}


\begin{Teo}
Para una constante $\mu\leq\infty$ ( o variable aleatoria), las siguientes expresiones son equivalentes:

\begin{eqnarray}
lim_{n\rightarrow\infty}n^{-1}T_{n}&=&\mu,\textrm{ c.s.}\\
lim_{t\rightarrow\infty}t^{-1}N\left(t\right)&=&1/\mu,\textrm{ c.s.}
\end{eqnarray}
\end{Teo}


Es decir, $T_{n}$ satisface la Ley Fuerte de los Grandes N\'umeros s\'i y s\'olo s\'i $N\left/t\right)$ la cumple.


\begin{Coro}[Ley Fuerte de los Grandes N\'umeros para Procesos de Renovaci\'on]
Si $N\left(t\right)$ es un proceso de renovaci\'on cuyos tiempos de inter-renovaci\'on tienen media $\mu\leq\infty$, entonces
\begin{eqnarray}
t^{-1}N\left(t\right)\rightarrow 1/\mu,\textrm{ c.s. cuando }t\rightarrow\infty.
\end{eqnarray}

\end{Coro}


Considerar el proceso estoc\'astico de valores reales $\left\{Z\left(t\right):t\geq0\right\}$ en el mismo espacio de probabilidad que $N\left(t\right)$

\begin{Def}
Para el proceso $\left\{Z\left(t\right):t\geq0\right\}$ se define la fluctuaci\'on m\'axima de $Z\left(t\right)$ en el intervalo $\left(T_{n-1},T_{n}\right]$:
\begin{eqnarray*}
M_{n}=\sup_{T_{n-1}<t\leq T_{n}}|Z\left(t\right)-Z\left(T_{n-1}\right)|
\end{eqnarray*}
\end{Def}

\begin{Teo}
Sup\'ongase que $n^{-1}T_{n}\rightarrow\mu$ c.s. cuando $n\rightarrow\infty$, donde $\mu\leq\infty$ es una constante o variable aleatoria. Sea $a$ una constante o variable aleatoria que puede ser infinita cuando $\mu$ es finita, y considere las expresiones l\'imite:
\begin{eqnarray}
lim_{n\rightarrow\infty}n^{-1}Z\left(T_{n}\right)&=&a,\textrm{ c.s.}\\
lim_{t\rightarrow\infty}t^{-1}Z\left(t\right)&=&a/\mu,\textrm{ c.s.}
\end{eqnarray}
La segunda expresi\'on implica la primera. Conversamente, la primera implica la segunda si el proceso $Z\left(t\right)$ es creciente, o si $lim_{n\rightarrow\infty}n^{-1}M_{n}=0$ c.s.
\end{Teo}

\begin{Coro}
Si $N\left(t\right)$ es un proceso de renovaci\'on, y $\left(Z\left(T_{n}\right)-Z\left(T_{n-1}\right),M_{n}\right)$, para $n\geq1$, son variables aleatorias independientes e id\'enticamente distribuidas con media finita, entonces,
\begin{eqnarray}
lim_{t\rightarrow\infty}t^{-1}Z\left(t\right)\rightarrow\frac{\esp\left[Z\left(T_{1}\right)-Z\left(T_{0}\right)\right]}{\esp\left[T_{1}\right]},\textrm{ c.s. cuando  }t\rightarrow\infty.
\end{eqnarray}
\end{Coro}



%___________________________________________________________________________________________
%
%\subsection{Propiedades de los Procesos de Renovaci\'on}
%___________________________________________________________________________________________
%

Los tiempos $T_{n}$ est\'an relacionados con los conteos de $N\left(t\right)$ por

\begin{eqnarray*}
\left\{N\left(t\right)\geq n\right\}&=&\left\{T_{n}\leq t\right\}\\
T_{N\left(t\right)}\leq &t&<T_{N\left(t\right)+1},
\end{eqnarray*}

adem\'as $N\left(T_{n}\right)=n$, y 

\begin{eqnarray*}
N\left(t\right)=\max\left\{n:T_{n}\leq t\right\}=\min\left\{n:T_{n+1}>t\right\}
\end{eqnarray*}

Por propiedades de la convoluci\'on se sabe que

\begin{eqnarray*}
P\left\{T_{n}\leq t\right\}=F^{n\star}\left(t\right)
\end{eqnarray*}
que es la $n$-\'esima convoluci\'on de $F$. Entonces 

\begin{eqnarray*}
\left\{N\left(t\right)\geq n\right\}&=&\left\{T_{n}\leq t\right\}\\
P\left\{N\left(t\right)\leq n\right\}&=&1-F^{\left(n+1\right)\star}\left(t\right)
\end{eqnarray*}

Adem\'as usando el hecho de que $\esp\left[N\left(t\right)\right]=\sum_{n=1}^{\infty}P\left\{N\left(t\right)\geq n\right\}$
se tiene que

\begin{eqnarray*}
\esp\left[N\left(t\right)\right]=\sum_{n=1}^{\infty}F^{n\star}\left(t\right)
\end{eqnarray*}

\begin{Prop}
Para cada $t\geq0$, la funci\'on generadora de momentos $\esp\left[e^{\alpha N\left(t\right)}\right]$ existe para alguna $\alpha$ en una vecindad del 0, y de aqu\'i que $\esp\left[N\left(t\right)^{m}\right]<\infty$, para $m\geq1$.
\end{Prop}


\begin{Note}
Si el primer tiempo de renovaci\'on $\xi_{1}$ no tiene la misma distribuci\'on que el resto de las $\xi_{n}$, para $n\geq2$, a $N\left(t\right)$ se le llama Proceso de Renovaci\'on retardado, donde si $\xi$ tiene distribuci\'on $G$, entonces el tiempo $T_{n}$ de la $n$-\'esima renovaci\'on tiene distribuci\'on $G\star F^{\left(n-1\right)\star}\left(t\right)$
\end{Note}


\begin{Teo}
Para una constante $\mu\leq\infty$ ( o variable aleatoria), las siguientes expresiones son equivalentes:

\begin{eqnarray}
lim_{n\rightarrow\infty}n^{-1}T_{n}&=&\mu,\textrm{ c.s.}\\
lim_{t\rightarrow\infty}t^{-1}N\left(t\right)&=&1/\mu,\textrm{ c.s.}
\end{eqnarray}
\end{Teo}


Es decir, $T_{n}$ satisface la Ley Fuerte de los Grandes N\'umeros s\'i y s\'olo s\'i $N\left/t\right)$ la cumple.


\begin{Coro}[Ley Fuerte de los Grandes N\'umeros para Procesos de Renovaci\'on]
Si $N\left(t\right)$ es un proceso de renovaci\'on cuyos tiempos de inter-renovaci\'on tienen media $\mu\leq\infty$, entonces
\begin{eqnarray}
t^{-1}N\left(t\right)\rightarrow 1/\mu,\textrm{ c.s. cuando }t\rightarrow\infty.
\end{eqnarray}

\end{Coro}


Considerar el proceso estoc\'astico de valores reales $\left\{Z\left(t\right):t\geq0\right\}$ en el mismo espacio de probabilidad que $N\left(t\right)$

\begin{Def}
Para el proceso $\left\{Z\left(t\right):t\geq0\right\}$ se define la fluctuaci\'on m\'axima de $Z\left(t\right)$ en el intervalo $\left(T_{n-1},T_{n}\right]$:
\begin{eqnarray*}
M_{n}=\sup_{T_{n-1}<t\leq T_{n}}|Z\left(t\right)-Z\left(T_{n-1}\right)|
\end{eqnarray*}
\end{Def}

\begin{Teo}
Sup\'ongase que $n^{-1}T_{n}\rightarrow\mu$ c.s. cuando $n\rightarrow\infty$, donde $\mu\leq\infty$ es una constante o variable aleatoria. Sea $a$ una constante o variable aleatoria que puede ser infinita cuando $\mu$ es finita, y considere las expresiones l\'imite:
\begin{eqnarray}
lim_{n\rightarrow\infty}n^{-1}Z\left(T_{n}\right)&=&a,\textrm{ c.s.}\\
lim_{t\rightarrow\infty}t^{-1}Z\left(t\right)&=&a/\mu,\textrm{ c.s.}
\end{eqnarray}
La segunda expresi\'on implica la primera. Conversamente, la primera implica la segunda si el proceso $Z\left(t\right)$ es creciente, o si $lim_{n\rightarrow\infty}n^{-1}M_{n}=0$ c.s.
\end{Teo}

\begin{Coro}
Si $N\left(t\right)$ es un proceso de renovaci\'on, y $\left(Z\left(T_{n}\right)-Z\left(T_{n-1}\right),M_{n}\right)$, para $n\geq1$, son variables aleatorias independientes e id\'enticamente distribuidas con media finita, entonces,
\begin{eqnarray}
lim_{t\rightarrow\infty}t^{-1}Z\left(t\right)\rightarrow\frac{\esp\left[Z\left(T_{1}\right)-Z\left(T_{0}\right)\right]}{\esp\left[T_{1}\right]},\textrm{ c.s. cuando  }t\rightarrow\infty.
\end{eqnarray}
\end{Coro}


%___________________________________________________________________________________________
%
%\subsection{Propiedades de los Procesos de Renovaci\'on}
%___________________________________________________________________________________________
%

Los tiempos $T_{n}$ est\'an relacionados con los conteos de $N\left(t\right)$ por

\begin{eqnarray*}
\left\{N\left(t\right)\geq n\right\}&=&\left\{T_{n}\leq t\right\}\\
T_{N\left(t\right)}\leq &t&<T_{N\left(t\right)+1},
\end{eqnarray*}

adem\'as $N\left(T_{n}\right)=n$, y 

\begin{eqnarray*}
N\left(t\right)=\max\left\{n:T_{n}\leq t\right\}=\min\left\{n:T_{n+1}>t\right\}
\end{eqnarray*}

Por propiedades de la convoluci\'on se sabe que

\begin{eqnarray*}
P\left\{T_{n}\leq t\right\}=F^{n\star}\left(t\right)
\end{eqnarray*}
que es la $n$-\'esima convoluci\'on de $F$. Entonces 

\begin{eqnarray*}
\left\{N\left(t\right)\geq n\right\}&=&\left\{T_{n}\leq t\right\}\\
P\left\{N\left(t\right)\leq n\right\}&=&1-F^{\left(n+1\right)\star}\left(t\right)
\end{eqnarray*}

Adem\'as usando el hecho de que $\esp\left[N\left(t\right)\right]=\sum_{n=1}^{\infty}P\left\{N\left(t\right)\geq n\right\}$
se tiene que

\begin{eqnarray*}
\esp\left[N\left(t\right)\right]=\sum_{n=1}^{\infty}F^{n\star}\left(t\right)
\end{eqnarray*}

\begin{Prop}
Para cada $t\geq0$, la funci\'on generadora de momentos $\esp\left[e^{\alpha N\left(t\right)}\right]$ existe para alguna $\alpha$ en una vecindad del 0, y de aqu\'i que $\esp\left[N\left(t\right)^{m}\right]<\infty$, para $m\geq1$.
\end{Prop}


\begin{Note}
Si el primer tiempo de renovaci\'on $\xi_{1}$ no tiene la misma distribuci\'on que el resto de las $\xi_{n}$, para $n\geq2$, a $N\left(t\right)$ se le llama Proceso de Renovaci\'on retardado, donde si $\xi$ tiene distribuci\'on $G$, entonces el tiempo $T_{n}$ de la $n$-\'esima renovaci\'on tiene distribuci\'on $G\star F^{\left(n-1\right)\star}\left(t\right)$
\end{Note}


\begin{Teo}
Para una constante $\mu\leq\infty$ ( o variable aleatoria), las siguientes expresiones son equivalentes:

\begin{eqnarray}
lim_{n\rightarrow\infty}n^{-1}T_{n}&=&\mu,\textrm{ c.s.}\\
lim_{t\rightarrow\infty}t^{-1}N\left(t\right)&=&1/\mu,\textrm{ c.s.}
\end{eqnarray}
\end{Teo}


Es decir, $T_{n}$ satisface la Ley Fuerte de los Grandes N\'umeros s\'i y s\'olo s\'i $N\left/t\right)$ la cumple.


\begin{Coro}[Ley Fuerte de los Grandes N\'umeros para Procesos de Renovaci\'on]
Si $N\left(t\right)$ es un proceso de renovaci\'on cuyos tiempos de inter-renovaci\'on tienen media $\mu\leq\infty$, entonces
\begin{eqnarray}
t^{-1}N\left(t\right)\rightarrow 1/\mu,\textrm{ c.s. cuando }t\rightarrow\infty.
\end{eqnarray}

\end{Coro}


Considerar el proceso estoc\'astico de valores reales $\left\{Z\left(t\right):t\geq0\right\}$ en el mismo espacio de probabilidad que $N\left(t\right)$

\begin{Def}
Para el proceso $\left\{Z\left(t\right):t\geq0\right\}$ se define la fluctuaci\'on m\'axima de $Z\left(t\right)$ en el intervalo $\left(T_{n-1},T_{n}\right]$:
\begin{eqnarray*}
M_{n}=\sup_{T_{n-1}<t\leq T_{n}}|Z\left(t\right)-Z\left(T_{n-1}\right)|
\end{eqnarray*}
\end{Def}

\begin{Teo}
Sup\'ongase que $n^{-1}T_{n}\rightarrow\mu$ c.s. cuando $n\rightarrow\infty$, donde $\mu\leq\infty$ es una constante o variable aleatoria. Sea $a$ una constante o variable aleatoria que puede ser infinita cuando $\mu$ es finita, y considere las expresiones l\'imite:
\begin{eqnarray}
lim_{n\rightarrow\infty}n^{-1}Z\left(T_{n}\right)&=&a,\textrm{ c.s.}\\
lim_{t\rightarrow\infty}t^{-1}Z\left(t\right)&=&a/\mu,\textrm{ c.s.}
\end{eqnarray}
La segunda expresi\'on implica la primera. Conversamente, la primera implica la segunda si el proceso $Z\left(t\right)$ es creciente, o si $lim_{n\rightarrow\infty}n^{-1}M_{n}=0$ c.s.
\end{Teo}

\begin{Coro}
Si $N\left(t\right)$ es un proceso de renovaci\'on, y $\left(Z\left(T_{n}\right)-Z\left(T_{n-1}\right),M_{n}\right)$, para $n\geq1$, son variables aleatorias independientes e id\'enticamente distribuidas con media finita, entonces,
\begin{eqnarray}
lim_{t\rightarrow\infty}t^{-1}Z\left(t\right)\rightarrow\frac{\esp\left[Z\left(T_{1}\right)-Z\left(T_{0}\right)\right]}{\esp\left[T_{1}\right]},\textrm{ c.s. cuando  }t\rightarrow\infty.
\end{eqnarray}
\end{Coro}

%___________________________________________________________________________________________
%
%\subsection{Propiedades de los Procesos de Renovaci\'on}
%___________________________________________________________________________________________
%

Los tiempos $T_{n}$ est\'an relacionados con los conteos de $N\left(t\right)$ por

\begin{eqnarray*}
\left\{N\left(t\right)\geq n\right\}&=&\left\{T_{n}\leq t\right\}\\
T_{N\left(t\right)}\leq &t&<T_{N\left(t\right)+1},
\end{eqnarray*}

adem\'as $N\left(T_{n}\right)=n$, y 

\begin{eqnarray*}
N\left(t\right)=\max\left\{n:T_{n}\leq t\right\}=\min\left\{n:T_{n+1}>t\right\}
\end{eqnarray*}

Por propiedades de la convoluci\'on se sabe que

\begin{eqnarray*}
P\left\{T_{n}\leq t\right\}=F^{n\star}\left(t\right)
\end{eqnarray*}
que es la $n$-\'esima convoluci\'on de $F$. Entonces 

\begin{eqnarray*}
\left\{N\left(t\right)\geq n\right\}&=&\left\{T_{n}\leq t\right\}\\
P\left\{N\left(t\right)\leq n\right\}&=&1-F^{\left(n+1\right)\star}\left(t\right)
\end{eqnarray*}

Adem\'as usando el hecho de que $\esp\left[N\left(t\right)\right]=\sum_{n=1}^{\infty}P\left\{N\left(t\right)\geq n\right\}$
se tiene que

\begin{eqnarray*}
\esp\left[N\left(t\right)\right]=\sum_{n=1}^{\infty}F^{n\star}\left(t\right)
\end{eqnarray*}

\begin{Prop}
Para cada $t\geq0$, la funci\'on generadora de momentos $\esp\left[e^{\alpha N\left(t\right)}\right]$ existe para alguna $\alpha$ en una vecindad del 0, y de aqu\'i que $\esp\left[N\left(t\right)^{m}\right]<\infty$, para $m\geq1$.
\end{Prop}


\begin{Note}
Si el primer tiempo de renovaci\'on $\xi_{1}$ no tiene la misma distribuci\'on que el resto de las $\xi_{n}$, para $n\geq2$, a $N\left(t\right)$ se le llama Proceso de Renovaci\'on retardado, donde si $\xi$ tiene distribuci\'on $G$, entonces el tiempo $T_{n}$ de la $n$-\'esima renovaci\'on tiene distribuci\'on $G\star F^{\left(n-1\right)\star}\left(t\right)$
\end{Note}


\begin{Teo}
Para una constante $\mu\leq\infty$ ( o variable aleatoria), las siguientes expresiones son equivalentes:

\begin{eqnarray}
lim_{n\rightarrow\infty}n^{-1}T_{n}&=&\mu,\textrm{ c.s.}\\
lim_{t\rightarrow\infty}t^{-1}N\left(t\right)&=&1/\mu,\textrm{ c.s.}
\end{eqnarray}
\end{Teo}


Es decir, $T_{n}$ satisface la Ley Fuerte de los Grandes N\'umeros s\'i y s\'olo s\'i $N\left/t\right)$ la cumple.


\begin{Coro}[Ley Fuerte de los Grandes N\'umeros para Procesos de Renovaci\'on]
Si $N\left(t\right)$ es un proceso de renovaci\'on cuyos tiempos de inter-renovaci\'on tienen media $\mu\leq\infty$, entonces
\begin{eqnarray}
t^{-1}N\left(t\right)\rightarrow 1/\mu,\textrm{ c.s. cuando }t\rightarrow\infty.
\end{eqnarray}

\end{Coro}


Considerar el proceso estoc\'astico de valores reales $\left\{Z\left(t\right):t\geq0\right\}$ en el mismo espacio de probabilidad que $N\left(t\right)$

\begin{Def}
Para el proceso $\left\{Z\left(t\right):t\geq0\right\}$ se define la fluctuaci\'on m\'axima de $Z\left(t\right)$ en el intervalo $\left(T_{n-1},T_{n}\right]$:
\begin{eqnarray*}
M_{n}=\sup_{T_{n-1}<t\leq T_{n}}|Z\left(t\right)-Z\left(T_{n-1}\right)|
\end{eqnarray*}
\end{Def}

\begin{Teo}
Sup\'ongase que $n^{-1}T_{n}\rightarrow\mu$ c.s. cuando $n\rightarrow\infty$, donde $\mu\leq\infty$ es una constante o variable aleatoria. Sea $a$ una constante o variable aleatoria que puede ser infinita cuando $\mu$ es finita, y considere las expresiones l\'imite:
\begin{eqnarray}
lim_{n\rightarrow\infty}n^{-1}Z\left(T_{n}\right)&=&a,\textrm{ c.s.}\\
lim_{t\rightarrow\infty}t^{-1}Z\left(t\right)&=&a/\mu,\textrm{ c.s.}
\end{eqnarray}
La segunda expresi\'on implica la primera. Conversamente, la primera implica la segunda si el proceso $Z\left(t\right)$ es creciente, o si $lim_{n\rightarrow\infty}n^{-1}M_{n}=0$ c.s.
\end{Teo}

\begin{Coro}
Si $N\left(t\right)$ es un proceso de renovaci\'on, y $\left(Z\left(T_{n}\right)-Z\left(T_{n-1}\right),M_{n}\right)$, para $n\geq1$, son variables aleatorias independientes e id\'enticamente distribuidas con media finita, entonces,
\begin{eqnarray}
lim_{t\rightarrow\infty}t^{-1}Z\left(t\right)\rightarrow\frac{\esp\left[Z\left(T_{1}\right)-Z\left(T_{0}\right)\right]}{\esp\left[T_{1}\right]},\textrm{ c.s. cuando  }t\rightarrow\infty.
\end{eqnarray}
\end{Coro}
%___________________________________________________________________________________________
%
%\subsection{Propiedades de los Procesos de Renovaci\'on}
%___________________________________________________________________________________________
%

Los tiempos $T_{n}$ est\'an relacionados con los conteos de $N\left(t\right)$ por

\begin{eqnarray*}
\left\{N\left(t\right)\geq n\right\}&=&\left\{T_{n}\leq t\right\}\\
T_{N\left(t\right)}\leq &t&<T_{N\left(t\right)+1},
\end{eqnarray*}

adem\'as $N\left(T_{n}\right)=n$, y 

\begin{eqnarray*}
N\left(t\right)=\max\left\{n:T_{n}\leq t\right\}=\min\left\{n:T_{n+1}>t\right\}
\end{eqnarray*}

Por propiedades de la convoluci\'on se sabe que

\begin{eqnarray*}
P\left\{T_{n}\leq t\right\}=F^{n\star}\left(t\right)
\end{eqnarray*}
que es la $n$-\'esima convoluci\'on de $F$. Entonces 

\begin{eqnarray*}
\left\{N\left(t\right)\geq n\right\}&=&\left\{T_{n}\leq t\right\}\\
P\left\{N\left(t\right)\leq n\right\}&=&1-F^{\left(n+1\right)\star}\left(t\right)
\end{eqnarray*}

Adem\'as usando el hecho de que $\esp\left[N\left(t\right)\right]=\sum_{n=1}^{\infty}P\left\{N\left(t\right)\geq n\right\}$
se tiene que

\begin{eqnarray*}
\esp\left[N\left(t\right)\right]=\sum_{n=1}^{\infty}F^{n\star}\left(t\right)
\end{eqnarray*}

\begin{Prop}
Para cada $t\geq0$, la funci\'on generadora de momentos $\esp\left[e^{\alpha N\left(t\right)}\right]$ existe para alguna $\alpha$ en una vecindad del 0, y de aqu\'i que $\esp\left[N\left(t\right)^{m}\right]<\infty$, para $m\geq1$.
\end{Prop}


\begin{Note}
Si el primer tiempo de renovaci\'on $\xi_{1}$ no tiene la misma distribuci\'on que el resto de las $\xi_{n}$, para $n\geq2$, a $N\left(t\right)$ se le llama Proceso de Renovaci\'on retardado, donde si $\xi$ tiene distribuci\'on $G$, entonces el tiempo $T_{n}$ de la $n$-\'esima renovaci\'on tiene distribuci\'on $G\star F^{\left(n-1\right)\star}\left(t\right)$
\end{Note}


\begin{Teo}
Para una constante $\mu\leq\infty$ ( o variable aleatoria), las siguientes expresiones son equivalentes:

\begin{eqnarray}
lim_{n\rightarrow\infty}n^{-1}T_{n}&=&\mu,\textrm{ c.s.}\\
lim_{t\rightarrow\infty}t^{-1}N\left(t\right)&=&1/\mu,\textrm{ c.s.}
\end{eqnarray}
\end{Teo}


Es decir, $T_{n}$ satisface la Ley Fuerte de los Grandes N\'umeros s\'i y s\'olo s\'i $N\left/t\right)$ la cumple.


\begin{Coro}[Ley Fuerte de los Grandes N\'umeros para Procesos de Renovaci\'on]
Si $N\left(t\right)$ es un proceso de renovaci\'on cuyos tiempos de inter-renovaci\'on tienen media $\mu\leq\infty$, entonces
\begin{eqnarray}
t^{-1}N\left(t\right)\rightarrow 1/\mu,\textrm{ c.s. cuando }t\rightarrow\infty.
\end{eqnarray}

\end{Coro}


Considerar el proceso estoc\'astico de valores reales $\left\{Z\left(t\right):t\geq0\right\}$ en el mismo espacio de probabilidad que $N\left(t\right)$

\begin{Def}
Para el proceso $\left\{Z\left(t\right):t\geq0\right\}$ se define la fluctuaci\'on m\'axima de $Z\left(t\right)$ en el intervalo $\left(T_{n-1},T_{n}\right]$:
\begin{eqnarray*}
M_{n}=\sup_{T_{n-1}<t\leq T_{n}}|Z\left(t\right)-Z\left(T_{n-1}\right)|
\end{eqnarray*}
\end{Def}

\begin{Teo}
Sup\'ongase que $n^{-1}T_{n}\rightarrow\mu$ c.s. cuando $n\rightarrow\infty$, donde $\mu\leq\infty$ es una constante o variable aleatoria. Sea $a$ una constante o variable aleatoria que puede ser infinita cuando $\mu$ es finita, y considere las expresiones l\'imite:
\begin{eqnarray}
lim_{n\rightarrow\infty}n^{-1}Z\left(T_{n}\right)&=&a,\textrm{ c.s.}\\
lim_{t\rightarrow\infty}t^{-1}Z\left(t\right)&=&a/\mu,\textrm{ c.s.}
\end{eqnarray}
La segunda expresi\'on implica la primera. Conversamente, la primera implica la segunda si el proceso $Z\left(t\right)$ es creciente, o si $lim_{n\rightarrow\infty}n^{-1}M_{n}=0$ c.s.
\end{Teo}

\begin{Coro}
Si $N\left(t\right)$ es un proceso de renovaci\'on, y $\left(Z\left(T_{n}\right)-Z\left(T_{n-1}\right),M_{n}\right)$, para $n\geq1$, son variables aleatorias independientes e id\'enticamente distribuidas con media finita, entonces,
\begin{eqnarray}
lim_{t\rightarrow\infty}t^{-1}Z\left(t\right)\rightarrow\frac{\esp\left[Z\left(T_{1}\right)-Z\left(T_{0}\right)\right]}{\esp\left[T_{1}\right]},\textrm{ c.s. cuando  }t\rightarrow\infty.
\end{eqnarray}
\end{Coro}


%___________________________________________________________________________________________
%
%\subsection{Funci\'on de Renovaci\'on}
%___________________________________________________________________________________________
%


\begin{Def}
Sea $h\left(t\right)$ funci\'on de valores reales en $\rea$ acotada en intervalos finitos e igual a cero para $t<0$ La ecuaci\'on de renovaci\'on para $h\left(t\right)$ y la distribuci\'on $F$ es

\begin{eqnarray}%\label{Ec.Renovacion}
H\left(t\right)=h\left(t\right)+\int_{\left[0,t\right]}H\left(t-s\right)dF\left(s\right)\textrm{,    }t\geq0,
\end{eqnarray}
donde $H\left(t\right)$ es una funci\'on de valores reales. Esto es $H=h+F\star H$. Decimos que $H\left(t\right)$ es soluci\'on de esta ecuaci\'on si satisface la ecuaci\'on, y es acotada en intervalos finitos e iguales a cero para $t<0$.
\end{Def}

\begin{Prop}
La funci\'on $U\star h\left(t\right)$ es la \'unica soluci\'on de la ecuaci\'on de renovaci\'on (\ref{Ec.Renovacion}).
\end{Prop}

\begin{Teo}[Teorema Renovaci\'on Elemental]
\begin{eqnarray*}
t^{-1}U\left(t\right)\rightarrow 1/\mu\textrm{,    cuando }t\rightarrow\infty.
\end{eqnarray*}
\end{Teo}

%___________________________________________________________________________________________
%
%\subsection{Funci\'on de Renovaci\'on}
%___________________________________________________________________________________________
%


Sup\'ongase que $N\left(t\right)$ es un proceso de renovaci\'on con distribuci\'on $F$ con media finita $\mu$.

\begin{Def}
La funci\'on de renovaci\'on asociada con la distribuci\'on $F$, del proceso $N\left(t\right)$, es
\begin{eqnarray*}
U\left(t\right)=\sum_{n=1}^{\infty}F^{n\star}\left(t\right),\textrm{   }t\geq0,
\end{eqnarray*}
donde $F^{0\star}\left(t\right)=\indora\left(t\geq0\right)$.
\end{Def}


\begin{Prop}
Sup\'ongase que la distribuci\'on de inter-renovaci\'on $F$ tiene densidad $f$. Entonces $U\left(t\right)$ tambi\'en tiene densidad, para $t>0$, y es $U^{'}\left(t\right)=\sum_{n=0}^{\infty}f^{n\star}\left(t\right)$. Adem\'as
\begin{eqnarray*}
\prob\left\{N\left(t\right)>N\left(t-\right)\right\}=0\textrm{,   }t\geq0.
\end{eqnarray*}
\end{Prop}

\begin{Def}
La Transformada de Laplace-Stieljes de $F$ est\'a dada por

\begin{eqnarray*}
\hat{F}\left(\alpha\right)=\int_{\rea_{+}}e^{-\alpha t}dF\left(t\right)\textrm{,  }\alpha\geq0.
\end{eqnarray*}
\end{Def}

Entonces

\begin{eqnarray*}
\hat{U}\left(\alpha\right)=\sum_{n=0}^{\infty}\hat{F^{n\star}}\left(\alpha\right)=\sum_{n=0}^{\infty}\hat{F}\left(\alpha\right)^{n}=\frac{1}{1-\hat{F}\left(\alpha\right)}.
\end{eqnarray*}


\begin{Prop}
La Transformada de Laplace $\hat{U}\left(\alpha\right)$ y $\hat{F}\left(\alpha\right)$ determina una a la otra de manera \'unica por la relaci\'on $\hat{U}\left(\alpha\right)=\frac{1}{1-\hat{F}\left(\alpha\right)}$.
\end{Prop}


\begin{Note}
Un proceso de renovaci\'on $N\left(t\right)$ cuyos tiempos de inter-renovaci\'on tienen media finita, es un proceso Poisson con tasa $\lambda$ si y s\'olo s\'i $\esp\left[U\left(t\right)\right]=\lambda t$, para $t\geq0$.
\end{Note}


\begin{Teo}
Sea $N\left(t\right)$ un proceso puntual simple con puntos de localizaci\'on $T_{n}$ tal que $\eta\left(t\right)=\esp\left[N\left(\right)\right]$ es finita para cada $t$. Entonces para cualquier funci\'on $f:\rea_{+}\rightarrow\rea$,
\begin{eqnarray*}
\esp\left[\sum_{n=1}^{N\left(\right)}f\left(T_{n}\right)\right]=\int_{\left(0,t\right]}f\left(s\right)d\eta\left(s\right)\textrm{,  }t\geq0,
\end{eqnarray*}
suponiendo que la integral exista. Adem\'as si $X_{1},X_{2},\ldots$ son variables aleatorias definidas en el mismo espacio de probabilidad que el proceso $N\left(t\right)$ tal que $\esp\left[X_{n}|T_{n}=s\right]=f\left(s\right)$, independiente de $n$. Entonces
\begin{eqnarray*}
\esp\left[\sum_{n=1}^{N\left(t\right)}X_{n}\right]=\int_{\left(0,t\right]}f\left(s\right)d\eta\left(s\right)\textrm{,  }t\geq0,
\end{eqnarray*} 
suponiendo que la integral exista. 
\end{Teo}

\begin{Coro}[Identidad de Wald para Renovaciones]
Para el proceso de renovaci\'on $N\left(t\right)$,
\begin{eqnarray*}
\esp\left[T_{N\left(t\right)+1}\right]=\mu\esp\left[N\left(t\right)+1\right]\textrm{,  }t\geq0,
\end{eqnarray*}  
\end{Coro}

%______________________________________________________________________
%\subsection{Procesos de Renovaci\'on}
%______________________________________________________________________

\begin{Def}%\label{Def.Tn}
Sean $0\leq T_{1}\leq T_{2}\leq \ldots$ son tiempos aleatorios infinitos en los cuales ocurren ciertos eventos. El n\'umero de tiempos $T_{n}$ en el intervalo $\left[0,t\right)$ es

\begin{eqnarray}
N\left(t\right)=\sum_{n=1}^{\infty}\indora\left(T_{n}\leq t\right),
\end{eqnarray}
para $t\geq0$.
\end{Def}

Si se consideran los puntos $T_{n}$ como elementos de $\rea_{+}$, y $N\left(t\right)$ es el n\'umero de puntos en $\rea$. El proceso denotado por $\left\{N\left(t\right):t\geq0\right\}$, denotado por $N\left(t\right)$, es un proceso puntual en $\rea_{+}$. Los $T_{n}$ son los tiempos de ocurrencia, el proceso puntual $N\left(t\right)$ es simple si su n\'umero de ocurrencias son distintas: $0<T_{1}<T_{2}<\ldots$ casi seguramente.

\begin{Def}
Un proceso puntual $N\left(t\right)$ es un proceso de renovaci\'on si los tiempos de interocurrencia $\xi_{n}=T_{n}-T_{n-1}$, para $n\geq1$, son independientes e identicamente distribuidos con distribuci\'on $F$, donde $F\left(0\right)=0$ y $T_{0}=0$. Los $T_{n}$ son llamados tiempos de renovaci\'on, referente a la independencia o renovaci\'on de la informaci\'on estoc\'astica en estos tiempos. Los $\xi_{n}$ son los tiempos de inter-renovaci\'on, y $N\left(t\right)$ es el n\'umero de renovaciones en el intervalo $\left[0,t\right)$
\end{Def}


\begin{Note}
Para definir un proceso de renovaci\'on para cualquier contexto, solamente hay que especificar una distribuci\'on $F$, con $F\left(0\right)=0$, para los tiempos de inter-renovaci\'on. La funci\'on $F$ en turno degune las otra variables aleatorias. De manera formal, existe un espacio de probabilidad y una sucesi\'on de variables aleatorias $\xi_{1},\xi_{2},\ldots$ definidas en este con distribuci\'on $F$. Entonces las otras cantidades son $T_{n}=\sum_{k=1}^{n}\xi_{k}$ y $N\left(t\right)=\sum_{n=1}^{\infty}\indora\left(T_{n}\leq t\right)$, donde $T_{n}\rightarrow\infty$ casi seguramente por la Ley Fuerte de los Grandes Números.
\end{Note}

%___________________________________________________________________________________________
%
%\subsection{Renewal and Regenerative Processes: Serfozo\cite{Serfozo}}
%___________________________________________________________________________________________
%
\begin{Def}%\label{Def.Tn}
Sean $0\leq T_{1}\leq T_{2}\leq \ldots$ son tiempos aleatorios infinitos en los cuales ocurren ciertos eventos. El n\'umero de tiempos $T_{n}$ en el intervalo $\left[0,t\right)$ es

\begin{eqnarray}
N\left(t\right)=\sum_{n=1}^{\infty}\indora\left(T_{n}\leq t\right),
\end{eqnarray}
para $t\geq0$.
\end{Def}

Si se consideran los puntos $T_{n}$ como elementos de $\rea_{+}$, y $N\left(t\right)$ es el n\'umero de puntos en $\rea$. El proceso denotado por $\left\{N\left(t\right):t\geq0\right\}$, denotado por $N\left(t\right)$, es un proceso puntual en $\rea_{+}$. Los $T_{n}$ son los tiempos de ocurrencia, el proceso puntual $N\left(t\right)$ es simple si su n\'umero de ocurrencias son distintas: $0<T_{1}<T_{2}<\ldots$ casi seguramente.

\begin{Def}
Un proceso puntual $N\left(t\right)$ es un proceso de renovaci\'on si los tiempos de interocurrencia $\xi_{n}=T_{n}-T_{n-1}$, para $n\geq1$, son independientes e identicamente distribuidos con distribuci\'on $F$, donde $F\left(0\right)=0$ y $T_{0}=0$. Los $T_{n}$ son llamados tiempos de renovaci\'on, referente a la independencia o renovaci\'on de la informaci\'on estoc\'astica en estos tiempos. Los $\xi_{n}$ son los tiempos de inter-renovaci\'on, y $N\left(t\right)$ es el n\'umero de renovaciones en el intervalo $\left[0,t\right)$
\end{Def}


\begin{Note}
Para definir un proceso de renovaci\'on para cualquier contexto, solamente hay que especificar una distribuci\'on $F$, con $F\left(0\right)=0$, para los tiempos de inter-renovaci\'on. La funci\'on $F$ en turno degune las otra variables aleatorias. De manera formal, existe un espacio de probabilidad y una sucesi\'on de variables aleatorias $\xi_{1},\xi_{2},\ldots$ definidas en este con distribuci\'on $F$. Entonces las otras cantidades son $T_{n}=\sum_{k=1}^{n}\xi_{k}$ y $N\left(t\right)=\sum_{n=1}^{\infty}\indora\left(T_{n}\leq t\right)$, donde $T_{n}\rightarrow\infty$ casi seguramente por la Ley Fuerte de los Grandes N\'umeros.
\end{Note}







Los tiempos $T_{n}$ est\'an relacionados con los conteos de $N\left(t\right)$ por

\begin{eqnarray*}
\left\{N\left(t\right)\geq n\right\}&=&\left\{T_{n}\leq t\right\}\\
T_{N\left(t\right)}\leq &t&<T_{N\left(t\right)+1},
\end{eqnarray*}

adem\'as $N\left(T_{n}\right)=n$, y 

\begin{eqnarray*}
N\left(t\right)=\max\left\{n:T_{n}\leq t\right\}=\min\left\{n:T_{n+1}>t\right\}
\end{eqnarray*}

Por propiedades de la convoluci\'on se sabe que

\begin{eqnarray*}
P\left\{T_{n}\leq t\right\}=F^{n\star}\left(t\right)
\end{eqnarray*}
que es la $n$-\'esima convoluci\'on de $F$. Entonces 

\begin{eqnarray*}
\left\{N\left(t\right)\geq n\right\}&=&\left\{T_{n}\leq t\right\}\\
P\left\{N\left(t\right)\leq n\right\}&=&1-F^{\left(n+1\right)\star}\left(t\right)
\end{eqnarray*}

Adem\'as usando el hecho de que $\esp\left[N\left(t\right)\right]=\sum_{n=1}^{\infty}P\left\{N\left(t\right)\geq n\right\}$
se tiene que

\begin{eqnarray*}
\esp\left[N\left(t\right)\right]=\sum_{n=1}^{\infty}F^{n\star}\left(t\right)
\end{eqnarray*}

\begin{Prop}
Para cada $t\geq0$, la funci\'on generadora de momentos $\esp\left[e^{\alpha N\left(t\right)}\right]$ existe para alguna $\alpha$ en una vecindad del 0, y de aqu\'i que $\esp\left[N\left(t\right)^{m}\right]<\infty$, para $m\geq1$.
\end{Prop}

\begin{Ejem}[\textbf{Proceso Poisson}]

Suponga que se tienen tiempos de inter-renovaci\'on \textit{i.i.d.} del proceso de renovaci\'on $N\left(t\right)$ tienen distribuci\'on exponencial $F\left(t\right)=q-e^{-\lambda t}$ con tasa $\lambda$. Entonces $N\left(t\right)$ es un proceso Poisson con tasa $\lambda$.

\end{Ejem}


\begin{Note}
Si el primer tiempo de renovaci\'on $\xi_{1}$ no tiene la misma distribuci\'on que el resto de las $\xi_{n}$, para $n\geq2$, a $N\left(t\right)$ se le llama Proceso de Renovaci\'on retardado, donde si $\xi$ tiene distribuci\'on $G$, entonces el tiempo $T_{n}$ de la $n$-\'esima renovaci\'on tiene distribuci\'on $G\star F^{\left(n-1\right)\star}\left(t\right)$
\end{Note}


\begin{Teo}
Para una constante $\mu\leq\infty$ ( o variable aleatoria), las siguientes expresiones son equivalentes:

\begin{eqnarray}
lim_{n\rightarrow\infty}n^{-1}T_{n}&=&\mu,\textrm{ c.s.}\\
lim_{t\rightarrow\infty}t^{-1}N\left(t\right)&=&1/\mu,\textrm{ c.s.}
\end{eqnarray}
\end{Teo}


Es decir, $T_{n}$ satisface la Ley Fuerte de los Grandes N\'umeros s\'i y s\'olo s\'i $N\left/t\right)$ la cumple.


\begin{Coro}[Ley Fuerte de los Grandes N\'umeros para Procesos de Renovaci\'on]
Si $N\left(t\right)$ es un proceso de renovaci\'on cuyos tiempos de inter-renovaci\'on tienen media $\mu\leq\infty$, entonces
\begin{eqnarray}
t^{-1}N\left(t\right)\rightarrow 1/\mu,\textrm{ c.s. cuando }t\rightarrow\infty.
\end{eqnarray}

\end{Coro}


Considerar el proceso estoc\'astico de valores reales $\left\{Z\left(t\right):t\geq0\right\}$ en el mismo espacio de probabilidad que $N\left(t\right)$

\begin{Def}
Para el proceso $\left\{Z\left(t\right):t\geq0\right\}$ se define la fluctuaci\'on m\'axima de $Z\left(t\right)$ en el intervalo $\left(T_{n-1},T_{n}\right]$:
\begin{eqnarray*}
M_{n}=\sup_{T_{n-1}<t\leq T_{n}}|Z\left(t\right)-Z\left(T_{n-1}\right)|
\end{eqnarray*}
\end{Def}

\begin{Teo}
Sup\'ongase que $n^{-1}T_{n}\rightarrow\mu$ c.s. cuando $n\rightarrow\infty$, donde $\mu\leq\infty$ es una constante o variable aleatoria. Sea $a$ una constante o variable aleatoria que puede ser infinita cuando $\mu$ es finita, y considere las expresiones l\'imite:
\begin{eqnarray}
lim_{n\rightarrow\infty}n^{-1}Z\left(T_{n}\right)&=&a,\textrm{ c.s.}\\
lim_{t\rightarrow\infty}t^{-1}Z\left(t\right)&=&a/\mu,\textrm{ c.s.}
\end{eqnarray}
La segunda expresi\'on implica la primera. Conversamente, la primera implica la segunda si el proceso $Z\left(t\right)$ es creciente, o si $lim_{n\rightarrow\infty}n^{-1}M_{n}=0$ c.s.
\end{Teo}

\begin{Coro}
Si $N\left(t\right)$ es un proceso de renovaci\'on, y $\left(Z\left(T_{n}\right)-Z\left(T_{n-1}\right),M_{n}\right)$, para $n\geq1$, son variables aleatorias independientes e id\'enticamente distribuidas con media finita, entonces,
\begin{eqnarray}
lim_{t\rightarrow\infty}t^{-1}Z\left(t\right)\rightarrow\frac{\esp\left[Z\left(T_{1}\right)-Z\left(T_{0}\right)\right]}{\esp\left[T_{1}\right]},\textrm{ c.s. cuando  }t\rightarrow\infty.
\end{eqnarray}
\end{Coro}


Sup\'ongase que $N\left(t\right)$ es un proceso de renovaci\'on con distribuci\'on $F$ con media finita $\mu$.

\begin{Def}
La funci\'on de renovaci\'on asociada con la distribuci\'on $F$, del proceso $N\left(t\right)$, es
\begin{eqnarray*}
U\left(t\right)=\sum_{n=1}^{\infty}F^{n\star}\left(t\right),\textrm{   }t\geq0,
\end{eqnarray*}
donde $F^{0\star}\left(t\right)=\indora\left(t\geq0\right)$.
\end{Def}


\begin{Prop}
Sup\'ongase que la distribuci\'on de inter-renovaci\'on $F$ tiene densidad $f$. Entonces $U\left(t\right)$ tambi\'en tiene densidad, para $t>0$, y es $U^{'}\left(t\right)=\sum_{n=0}^{\infty}f^{n\star}\left(t\right)$. Adem\'as
\begin{eqnarray*}
\prob\left\{N\left(t\right)>N\left(t-\right)\right\}=0\textrm{,   }t\geq0.
\end{eqnarray*}
\end{Prop}

\begin{Def}
La Transformada de Laplace-Stieljes de $F$ est\'a dada por

\begin{eqnarray*}
\hat{F}\left(\alpha\right)=\int_{\rea_{+}}e^{-\alpha t}dF\left(t\right)\textrm{,  }\alpha\geq0.
\end{eqnarray*}
\end{Def}

Entonces

\begin{eqnarray*}
\hat{U}\left(\alpha\right)=\sum_{n=0}^{\infty}\hat{F^{n\star}}\left(\alpha\right)=\sum_{n=0}^{\infty}\hat{F}\left(\alpha\right)^{n}=\frac{1}{1-\hat{F}\left(\alpha\right)}.
\end{eqnarray*}


\begin{Prop}
La Transformada de Laplace $\hat{U}\left(\alpha\right)$ y $\hat{F}\left(\alpha\right)$ determina una a la otra de manera \'unica por la relaci\'on $\hat{U}\left(\alpha\right)=\frac{1}{1-\hat{F}\left(\alpha\right)}$.
\end{Prop}


\begin{Note}
Un proceso de renovaci\'on $N\left(t\right)$ cuyos tiempos de inter-renovaci\'on tienen media finita, es un proceso Poisson con tasa $\lambda$ si y s\'olo s\'i $\esp\left[U\left(t\right)\right]=\lambda t$, para $t\geq0$.
\end{Note}


\begin{Teo}
Sea $N\left(t\right)$ un proceso puntual simple con puntos de localizaci\'on $T_{n}$ tal que $\eta\left(t\right)=\esp\left[N\left(\right)\right]$ es finita para cada $t$. Entonces para cualquier funci\'on $f:\rea_{+}\rightarrow\rea$,
\begin{eqnarray*}
\esp\left[\sum_{n=1}^{N\left(\right)}f\left(T_{n}\right)\right]=\int_{\left(0,t\right]}f\left(s\right)d\eta\left(s\right)\textrm{,  }t\geq0,
\end{eqnarray*}
suponiendo que la integral exista. Adem\'as si $X_{1},X_{2},\ldots$ son variables aleatorias definidas en el mismo espacio de probabilidad que el proceso $N\left(t\right)$ tal que $\esp\left[X_{n}|T_{n}=s\right]=f\left(s\right)$, independiente de $n$. Entonces
\begin{eqnarray*}
\esp\left[\sum_{n=1}^{N\left(t\right)}X_{n}\right]=\int_{\left(0,t\right]}f\left(s\right)d\eta\left(s\right)\textrm{,  }t\geq0,
\end{eqnarray*} 
suponiendo que la integral exista. 
\end{Teo}

\begin{Coro}[Identidad de Wald para Renovaciones]
Para el proceso de renovaci\'on $N\left(t\right)$,
\begin{eqnarray*}
\esp\left[T_{N\left(t\right)+1}\right]=\mu\esp\left[N\left(t\right)+1\right]\textrm{,  }t\geq0,
\end{eqnarray*}  
\end{Coro}


\begin{Def}
Sea $h\left(t\right)$ funci\'on de valores reales en $\rea$ acotada en intervalos finitos e igual a cero para $t<0$ La ecuaci\'on de renovaci\'on para $h\left(t\right)$ y la distribuci\'on $F$ es

\begin{eqnarray}%\label{Ec.Renovacion}
H\left(t\right)=h\left(t\right)+\int_{\left[0,t\right]}H\left(t-s\right)dF\left(s\right)\textrm{,    }t\geq0,
\end{eqnarray}
donde $H\left(t\right)$ es una funci\'on de valores reales. Esto es $H=h+F\star H$. Decimos que $H\left(t\right)$ es soluci\'on de esta ecuaci\'on si satisface la ecuaci\'on, y es acotada en intervalos finitos e iguales a cero para $t<0$.
\end{Def}

\begin{Prop}
La funci\'on $U\star h\left(t\right)$ es la \'unica soluci\'on de la ecuaci\'on de renovaci\'on (\ref{Ec.Renovacion}).
\end{Prop}

\begin{Teo}[Teorema Renovaci\'on Elemental]
\begin{eqnarray*}
t^{-1}U\left(t\right)\rightarrow 1/\mu\textrm{,    cuando }t\rightarrow\infty.
\end{eqnarray*}
\end{Teo}



Sup\'ongase que $N\left(t\right)$ es un proceso de renovaci\'on con distribuci\'on $F$ con media finita $\mu$.

\begin{Def}
La funci\'on de renovaci\'on asociada con la distribuci\'on $F$, del proceso $N\left(t\right)$, es
\begin{eqnarray*}
U\left(t\right)=\sum_{n=1}^{\infty}F^{n\star}\left(t\right),\textrm{   }t\geq0,
\end{eqnarray*}
donde $F^{0\star}\left(t\right)=\indora\left(t\geq0\right)$.
\end{Def}


\begin{Prop}
Sup\'ongase que la distribuci\'on de inter-renovaci\'on $F$ tiene densidad $f$. Entonces $U\left(t\right)$ tambi\'en tiene densidad, para $t>0$, y es $U^{'}\left(t\right)=\sum_{n=0}^{\infty}f^{n\star}\left(t\right)$. Adem\'as
\begin{eqnarray*}
\prob\left\{N\left(t\right)>N\left(t-\right)\right\}=0\textrm{,   }t\geq0.
\end{eqnarray*}
\end{Prop}

\begin{Def}
La Transformada de Laplace-Stieljes de $F$ est\'a dada por

\begin{eqnarray*}
\hat{F}\left(\alpha\right)=\int_{\rea_{+}}e^{-\alpha t}dF\left(t\right)\textrm{,  }\alpha\geq0.
\end{eqnarray*}
\end{Def}

Entonces

\begin{eqnarray*}
\hat{U}\left(\alpha\right)=\sum_{n=0}^{\infty}\hat{F^{n\star}}\left(\alpha\right)=\sum_{n=0}^{\infty}\hat{F}\left(\alpha\right)^{n}=\frac{1}{1-\hat{F}\left(\alpha\right)}.
\end{eqnarray*}


\begin{Prop}
La Transformada de Laplace $\hat{U}\left(\alpha\right)$ y $\hat{F}\left(\alpha\right)$ determina una a la otra de manera \'unica por la relaci\'on $\hat{U}\left(\alpha\right)=\frac{1}{1-\hat{F}\left(\alpha\right)}$.
\end{Prop}


\begin{Note}
Un proceso de renovaci\'on $N\left(t\right)$ cuyos tiempos de inter-renovaci\'on tienen media finita, es un proceso Poisson con tasa $\lambda$ si y s\'olo s\'i $\esp\left[U\left(t\right)\right]=\lambda t$, para $t\geq0$.
\end{Note}


\begin{Teo}
Sea $N\left(t\right)$ un proceso puntual simple con puntos de localizaci\'on $T_{n}$ tal que $\eta\left(t\right)=\esp\left[N\left(\right)\right]$ es finita para cada $t$. Entonces para cualquier funci\'on $f:\rea_{+}\rightarrow\rea$,
\begin{eqnarray*}
\esp\left[\sum_{n=1}^{N\left(\right)}f\left(T_{n}\right)\right]=\int_{\left(0,t\right]}f\left(s\right)d\eta\left(s\right)\textrm{,  }t\geq0,
\end{eqnarray*}
suponiendo que la integral exista. Adem\'as si $X_{1},X_{2},\ldots$ son variables aleatorias definidas en el mismo espacio de probabilidad que el proceso $N\left(t\right)$ tal que $\esp\left[X_{n}|T_{n}=s\right]=f\left(s\right)$, independiente de $n$. Entonces
\begin{eqnarray*}
\esp\left[\sum_{n=1}^{N\left(t\right)}X_{n}\right]=\int_{\left(0,t\right]}f\left(s\right)d\eta\left(s\right)\textrm{,  }t\geq0,
\end{eqnarray*} 
suponiendo que la integral exista. 
\end{Teo}

\begin{Coro}[Identidad de Wald para Renovaciones]
Para el proceso de renovaci\'on $N\left(t\right)$,
\begin{eqnarray*}
\esp\left[T_{N\left(t\right)+1}\right]=\mu\esp\left[N\left(t\right)+1\right]\textrm{,  }t\geq0,
\end{eqnarray*}  
\end{Coro}


\begin{Def}
Sea $h\left(t\right)$ funci\'on de valores reales en $\rea$ acotada en intervalos finitos e igual a cero para $t<0$ La ecuaci\'on de renovaci\'on para $h\left(t\right)$ y la distribuci\'on $F$ es

\begin{eqnarray}%\label{Ec.Renovacion}
H\left(t\right)=h\left(t\right)+\int_{\left[0,t\right]}H\left(t-s\right)dF\left(s\right)\textrm{,    }t\geq0,
\end{eqnarray}
donde $H\left(t\right)$ es una funci\'on de valores reales. Esto es $H=h+F\star H$. Decimos que $H\left(t\right)$ es soluci\'on de esta ecuaci\'on si satisface la ecuaci\'on, y es acotada en intervalos finitos e iguales a cero para $t<0$.
\end{Def}

\begin{Prop}
La funci\'on $U\star h\left(t\right)$ es la \'unica soluci\'on de la ecuaci\'on de renovaci\'on (\ref{Ec.Renovacion}).
\end{Prop}

\begin{Teo}[Teorema Renovaci\'on Elemental]
\begin{eqnarray*}
t^{-1}U\left(t\right)\rightarrow 1/\mu\textrm{,    cuando }t\rightarrow\infty.
\end{eqnarray*}
\end{Teo}


\begin{Note} Una funci\'on $h:\rea_{+}\rightarrow\rea$ es Directamente Riemann Integrable en los siguientes casos:
\begin{itemize}
\item[a)] $h\left(t\right)\geq0$ es decreciente y Riemann Integrable.
\item[b)] $h$ es continua excepto posiblemente en un conjunto de Lebesgue de medida 0, y $|h\left(t\right)|\leq b\left(t\right)$, donde $b$ es DRI.
\end{itemize}
\end{Note}

\begin{Teo}[Teorema Principal de Renovaci\'on]
Si $F$ es no aritm\'etica y $h\left(t\right)$ es Directamente Riemann Integrable (DRI), entonces

\begin{eqnarray*}
lim_{t\rightarrow\infty}U\star h=\frac{1}{\mu}\int_{\rea_{+}}h\left(s\right)ds.
\end{eqnarray*}
\end{Teo}

\begin{Prop}
Cualquier funci\'on $H\left(t\right)$ acotada en intervalos finitos y que es 0 para $t<0$ puede expresarse como
\begin{eqnarray*}
H\left(t\right)=U\star h\left(t\right)\textrm{,  donde }h\left(t\right)=H\left(t\right)-F\star H\left(t\right)
\end{eqnarray*}
\end{Prop}

\begin{Def}
Un proceso estoc\'astico $X\left(t\right)$ es crudamente regenerativo en un tiempo aleatorio positivo $T$ si
\begin{eqnarray*}
\esp\left[X\left(T+t\right)|T\right]=\esp\left[X\left(t\right)\right]\textrm{, para }t\geq0,\end{eqnarray*}
y con las esperanzas anteriores finitas.
\end{Def}

\begin{Prop}
Sup\'ongase que $X\left(t\right)$ es un proceso crudamente regenerativo en $T$, que tiene distribuci\'on $F$. Si $\esp\left[X\left(t\right)\right]$ es acotado en intervalos finitos, entonces
\begin{eqnarray*}
\esp\left[X\left(t\right)\right]=U\star h\left(t\right)\textrm{,  donde }h\left(t\right)=\esp\left[X\left(t\right)\indora\left(T>t\right)\right].
\end{eqnarray*}
\end{Prop}

\begin{Teo}[Regeneraci\'on Cruda]
Sup\'ongase que $X\left(t\right)$ es un proceso con valores positivo crudamente regenerativo en $T$, y def\'inase $M=\sup\left\{|X\left(t\right)|:t\leq T\right\}$. Si $T$ es no aritm\'etico y $M$ y $MT$ tienen media finita, entonces
\begin{eqnarray*}
lim_{t\rightarrow\infty}\esp\left[X\left(t\right)\right]=\frac{1}{\mu}\int_{\rea_{+}}h\left(s\right)ds,
\end{eqnarray*}
donde $h\left(t\right)=\esp\left[X\left(t\right)\indora\left(T>t\right)\right]$.
\end{Teo}


\begin{Note} Una funci\'on $h:\rea_{+}\rightarrow\rea$ es Directamente Riemann Integrable en los siguientes casos:
\begin{itemize}
\item[a)] $h\left(t\right)\geq0$ es decreciente y Riemann Integrable.
\item[b)] $h$ es continua excepto posiblemente en un conjunto de Lebesgue de medida 0, y $|h\left(t\right)|\leq b\left(t\right)$, donde $b$ es DRI.
\end{itemize}
\end{Note}

\begin{Teo}[Teorema Principal de Renovaci\'on]
Si $F$ es no aritm\'etica y $h\left(t\right)$ es Directamente Riemann Integrable (DRI), entonces

\begin{eqnarray*}
lim_{t\rightarrow\infty}U\star h=\frac{1}{\mu}\int_{\rea_{+}}h\left(s\right)ds.
\end{eqnarray*}
\end{Teo}

\begin{Prop}
Cualquier funci\'on $H\left(t\right)$ acotada en intervalos finitos y que es 0 para $t<0$ puede expresarse como
\begin{eqnarray*}
H\left(t\right)=U\star h\left(t\right)\textrm{,  donde }h\left(t\right)=H\left(t\right)-F\star H\left(t\right)
\end{eqnarray*}
\end{Prop}

\begin{Def}
Un proceso estoc\'astico $X\left(t\right)$ es crudamente regenerativo en un tiempo aleatorio positivo $T$ si
\begin{eqnarray*}
\esp\left[X\left(T+t\right)|T\right]=\esp\left[X\left(t\right)\right]\textrm{, para }t\geq0,\end{eqnarray*}
y con las esperanzas anteriores finitas.
\end{Def}

\begin{Prop}
Sup\'ongase que $X\left(t\right)$ es un proceso crudamente regenerativo en $T$, que tiene distribuci\'on $F$. Si $\esp\left[X\left(t\right)\right]$ es acotado en intervalos finitos, entonces
\begin{eqnarray*}
\esp\left[X\left(t\right)\right]=U\star h\left(t\right)\textrm{,  donde }h\left(t\right)=\esp\left[X\left(t\right)\indora\left(T>t\right)\right].
\end{eqnarray*}
\end{Prop}

\begin{Teo}[Regeneraci\'on Cruda]
Sup\'ongase que $X\left(t\right)$ es un proceso con valores positivo crudamente regenerativo en $T$, y def\'inase $M=\sup\left\{|X\left(t\right)|:t\leq T\right\}$. Si $T$ es no aritm\'etico y $M$ y $MT$ tienen media finita, entonces
\begin{eqnarray*}
lim_{t\rightarrow\infty}\esp\left[X\left(t\right)\right]=\frac{1}{\mu}\int_{\rea_{+}}h\left(s\right)ds,
\end{eqnarray*}
donde $h\left(t\right)=\esp\left[X\left(t\right)\indora\left(T>t\right)\right]$.
\end{Teo}

\begin{Def}
Para el proceso $\left\{\left(N\left(t\right),X\left(t\right)\right):t\geq0\right\}$, sus trayectoria muestrales en el intervalo de tiempo $\left[T_{n-1},T_{n}\right)$ est\'an descritas por
\begin{eqnarray*}
\zeta_{n}=\left(\xi_{n},\left\{X\left(T_{n-1}+t\right):0\leq t<\xi_{n}\right\}\right)
\end{eqnarray*}
Este $\zeta_{n}$ es el $n$-\'esimo segmento del proceso. El proceso es regenerativo sobre los tiempos $T_{n}$ si sus segmentos $\zeta_{n}$ son independientes e id\'enticamennte distribuidos.
\end{Def}


\begin{Note}
Si $\tilde{X}\left(t\right)$ con espacio de estados $\tilde{S}$ es regenerativo sobre $T_{n}$, entonces $X\left(t\right)=f\left(\tilde{X}\left(t\right)\right)$ tambi\'en es regenerativo sobre $T_{n}$, para cualquier funci\'on $f:\tilde{S}\rightarrow S$.
\end{Note}

\begin{Note}
Los procesos regenerativos son crudamente regenerativos, pero no al rev\'es.
\end{Note}


\begin{Note}
Un proceso estoc\'astico a tiempo continuo o discreto es regenerativo si existe un proceso de renovaci\'on  tal que los segmentos del proceso entre tiempos de renovaci\'on sucesivos son i.i.d., es decir, para $\left\{X\left(t\right):t\geq0\right\}$ proceso estoc\'astico a tiempo continuo con espacio de estados $S$, espacio m\'etrico.
\end{Note}

Para $\left\{X\left(t\right):t\geq0\right\}$ Proceso Estoc\'astico a tiempo continuo con estado de espacios $S$, que es un espacio m\'etrico, con trayectorias continuas por la derecha y con l\'imites por la izquierda c.s. Sea $N\left(t\right)$ un proceso de renovaci\'on en $\rea_{+}$ definido en el mismo espacio de probabilidad que $X\left(t\right)$, con tiempos de renovaci\'on $T$ y tiempos de inter-renovaci\'on $\xi_{n}=T_{n}-T_{n-1}$, con misma distribuci\'on $F$ de media finita $\mu$.



\begin{Def}
Para el proceso $\left\{\left(N\left(t\right),X\left(t\right)\right):t\geq0\right\}$, sus trayectoria muestrales en el intervalo de tiempo $\left[T_{n-1},T_{n}\right)$ est\'an descritas por
\begin{eqnarray*}
\zeta_{n}=\left(\xi_{n},\left\{X\left(T_{n-1}+t\right):0\leq t<\xi_{n}\right\}\right)
\end{eqnarray*}
Este $\zeta_{n}$ es el $n$-\'esimo segmento del proceso. El proceso es regenerativo sobre los tiempos $T_{n}$ si sus segmentos $\zeta_{n}$ son independientes e id\'enticamennte distribuidos.
\end{Def}

\begin{Note}
Un proceso regenerativo con media de la longitud de ciclo finita es llamado positivo recurrente.
\end{Note}

\begin{Teo}[Procesos Regenerativos]
Suponga que el proceso
\end{Teo}


\begin{Def}[Renewal Process Trinity]
Para un proceso de renovaci\'on $N\left(t\right)$, los siguientes procesos proveen de informaci\'on sobre los tiempos de renovaci\'on.
\begin{itemize}
\item $A\left(t\right)=t-T_{N\left(t\right)}$, el tiempo de recurrencia hacia atr\'as al tiempo $t$, que es el tiempo desde la \'ultima renovaci\'on para $t$.

\item $B\left(t\right)=T_{N\left(t\right)+1}-t$, el tiempo de recurrencia hacia adelante al tiempo $t$, residual del tiempo de renovaci\'on, que es el tiempo para la pr\'oxima renovaci\'on despu\'es de $t$.

\item $L\left(t\right)=\xi_{N\left(t\right)+1}=A\left(t\right)+B\left(t\right)$, la longitud del intervalo de renovaci\'on que contiene a $t$.
\end{itemize}
\end{Def}

\begin{Note}
El proceso tridimensional $\left(A\left(t\right),B\left(t\right),L\left(t\right)\right)$ es regenerativo sobre $T_{n}$, y por ende cada proceso lo es. Cada proceso $A\left(t\right)$ y $B\left(t\right)$ son procesos de MArkov a tiempo continuo con trayectorias continuas por partes en el espacio de estados $\rea_{+}$. Una expresi\'on conveniente para su distribuci\'on conjunta es, para $0\leq x<t,y\geq0$
\begin{equation}\label{NoRenovacion}
P\left\{A\left(t\right)>x,B\left(t\right)>y\right\}=
P\left\{N\left(t+y\right)-N\left((t-x)\right)=0\right\}
\end{equation}
\end{Note}

\begin{Ejem}[Tiempos de recurrencia Poisson]
Si $N\left(t\right)$ es un proceso Poisson con tasa $\lambda$, entonces de la expresi\'on (\ref{NoRenovacion}) se tiene que

\begin{eqnarray*}
\begin{array}{lc}
P\left\{A\left(t\right)>x,B\left(t\right)>y\right\}=e^{-\lambda\left(x+y\right)},&0\leq x<t,y\geq0,
\end{array}
\end{eqnarray*}
que es la probabilidad Poisson de no renovaciones en un intervalo de longitud $x+y$.

\end{Ejem}

\begin{Note}
Una cadena de Markov erg\'odica tiene la propiedad de ser estacionaria si la distribuci\'on de su estado al tiempo $0$ es su distribuci\'on estacionaria.
\end{Note}


\begin{Def}
Un proceso estoc\'astico a tiempo continuo $\left\{X\left(t\right):t\geq0\right\}$ en un espacio general es estacionario si sus distribuciones finito dimensionales son invariantes bajo cualquier  traslado: para cada $0\leq s_{1}<s_{2}<\cdots<s_{k}$ y $t\geq0$,
\begin{eqnarray*}
\left(X\left(s_{1}+t\right),\ldots,X\left(s_{k}+t\right)\right)=_{d}\left(X\left(s_{1}\right),\ldots,X\left(s_{k}\right)\right).
\end{eqnarray*}
\end{Def}

\begin{Note}
Un proceso de Markov es estacionario si $X\left(t\right)=_{d}X\left(0\right)$, $t\geq0$.
\end{Note}

Considerese el proceso $N\left(t\right)=\sum_{n}\indora\left(\tau_{n}\leq t\right)$ en $\rea_{+}$, con puntos $0<\tau_{1}<\tau_{2}<\cdots$.

\begin{Prop}
Si $N$ es un proceso puntual estacionario y $\esp\left[N\left(1\right)\right]<\infty$, entonces $\esp\left[N\left(t\right)\right]=t\esp\left[N\left(1\right)\right]$, $t\geq0$

\end{Prop}

\begin{Teo}
Los siguientes enunciados son equivalentes
\begin{itemize}
\item[i)] El proceso retardado de renovaci\'on $N$ es estacionario.

\item[ii)] EL proceso de tiempos de recurrencia hacia adelante $B\left(t\right)$ es estacionario.


\item[iii)] $\esp\left[N\left(t\right)\right]=t/\mu$,


\item[iv)] $G\left(t\right)=F_{e}\left(t\right)=\frac{1}{\mu}\int_{0}^{t}\left[1-F\left(s\right)\right]ds$
\end{itemize}
Cuando estos enunciados son ciertos, $P\left\{B\left(t\right)\leq x\right\}=F_{e}\left(x\right)$, para $t,x\geq0$.

\end{Teo}

\begin{Note}
Una consecuencia del teorema anterior es que el Proceso Poisson es el \'unico proceso sin retardo que es estacionario.
\end{Note}

\begin{Coro}
El proceso de renovaci\'on $N\left(t\right)$ sin retardo, y cuyos tiempos de inter renonaci\'on tienen media finita, es estacionario si y s\'olo si es un proceso Poisson.

\end{Coro}


%________________________________________________________________________
%\subsection{Procesos Regenerativos}
%________________________________________________________________________



\begin{Note}
Si $\tilde{X}\left(t\right)$ con espacio de estados $\tilde{S}$ es regenerativo sobre $T_{n}$, entonces $X\left(t\right)=f\left(\tilde{X}\left(t\right)\right)$ tambi\'en es regenerativo sobre $T_{n}$, para cualquier funci\'on $f:\tilde{S}\rightarrow S$.
\end{Note}

\begin{Note}
Los procesos regenerativos son crudamente regenerativos, pero no al rev\'es.
\end{Note}
%\subsection*{Procesos Regenerativos: Sigman\cite{Sigman1}}
\begin{Def}[Definici\'on Cl\'asica]
Un proceso estoc\'astico $X=\left\{X\left(t\right):t\geq0\right\}$ es llamado regenerativo is existe una variable aleatoria $R_{1}>0$ tal que
\begin{itemize}
\item[i)] $\left\{X\left(t+R_{1}\right):t\geq0\right\}$ es independiente de $\left\{\left\{X\left(t\right):t<R_{1}\right\},\right\}$
\item[ii)] $\left\{X\left(t+R_{1}\right):t\geq0\right\}$ es estoc\'asticamente equivalente a $\left\{X\left(t\right):t>0\right\}$
\end{itemize}

Llamamos a $R_{1}$ tiempo de regeneraci\'on, y decimos que $X$ se regenera en este punto.
\end{Def}

$\left\{X\left(t+R_{1}\right)\right\}$ es regenerativo con tiempo de regeneraci\'on $R_{2}$, independiente de $R_{1}$ pero con la misma distribuci\'on que $R_{1}$. Procediendo de esta manera se obtiene una secuencia de variables aleatorias independientes e id\'enticamente distribuidas $\left\{R_{n}\right\}$ llamados longitudes de ciclo. Si definimos a $Z_{k}\equiv R_{1}+R_{2}+\cdots+R_{k}$, se tiene un proceso de renovaci\'on llamado proceso de renovaci\'on encajado para $X$.




\begin{Def}
Para $x$ fijo y para cada $t\geq0$, sea $I_{x}\left(t\right)=1$ si $X\left(t\right)\leq x$,  $I_{x}\left(t\right)=0$ en caso contrario, y def\'inanse los tiempos promedio
\begin{eqnarray*}
\overline{X}&=&lim_{t\rightarrow\infty}\frac{1}{t}\int_{0}^{\infty}X\left(u\right)du\\
\prob\left(X_{\infty}\leq x\right)&=&lim_{t\rightarrow\infty}\frac{1}{t}\int_{0}^{\infty}I_{x}\left(u\right)du,
\end{eqnarray*}
cuando estos l\'imites existan.
\end{Def}

Como consecuencia del teorema de Renovaci\'on-Recompensa, se tiene que el primer l\'imite  existe y es igual a la constante
\begin{eqnarray*}
\overline{X}&=&\frac{\esp\left[\int_{0}^{R_{1}}X\left(t\right)dt\right]}{\esp\left[R_{1}\right]},
\end{eqnarray*}
suponiendo que ambas esperanzas son finitas.

\begin{Note}
\begin{itemize}
\item[a)] Si el proceso regenerativo $X$ es positivo recurrente y tiene trayectorias muestrales no negativas, entonces la ecuaci\'on anterior es v\'alida.
\item[b)] Si $X$ es positivo recurrente regenerativo, podemos construir una \'unica versi\'on estacionaria de este proceso, $X_{e}=\left\{X_{e}\left(t\right)\right\}$, donde $X_{e}$ es un proceso estoc\'astico regenerativo y estrictamente estacionario, con distribuci\'on marginal distribuida como $X_{\infty}$
\end{itemize}
\end{Note}

%________________________________________________________________________
%\subsection{Procesos Regenerativos}
%________________________________________________________________________

Para $\left\{X\left(t\right):t\geq0\right\}$ Proceso Estoc\'astico a tiempo continuo con estado de espacios $S$, que es un espacio m\'etrico, con trayectorias continuas por la derecha y con l\'imites por la izquierda c.s. Sea $N\left(t\right)$ un proceso de renovaci\'on en $\rea_{+}$ definido en el mismo espacio de probabilidad que $X\left(t\right)$, con tiempos de renovaci\'on $T$ y tiempos de inter-renovaci\'on $\xi_{n}=T_{n}-T_{n-1}$, con misma distribuci\'on $F$ de media finita $\mu$.



\begin{Def}
Para el proceso $\left\{\left(N\left(t\right),X\left(t\right)\right):t\geq0\right\}$, sus trayectoria muestrales en el intervalo de tiempo $\left[T_{n-1},T_{n}\right)$ est\'an descritas por
\begin{eqnarray*}
\zeta_{n}=\left(\xi_{n},\left\{X\left(T_{n-1}+t\right):0\leq t<\xi_{n}\right\}\right)
\end{eqnarray*}
Este $\zeta_{n}$ es el $n$-\'esimo segmento del proceso. El proceso es regenerativo sobre los tiempos $T_{n}$ si sus segmentos $\zeta_{n}$ son independientes e id\'enticamennte distribuidos.
\end{Def}


\begin{Note}
Si $\tilde{X}\left(t\right)$ con espacio de estados $\tilde{S}$ es regenerativo sobre $T_{n}$, entonces $X\left(t\right)=f\left(\tilde{X}\left(t\right)\right)$ tambi\'en es regenerativo sobre $T_{n}$, para cualquier funci\'on $f:\tilde{S}\rightarrow S$.
\end{Note}

\begin{Note}
Los procesos regenerativos son crudamente regenerativos, pero no al rev\'es.
\end{Note}

\begin{Def}[Definici\'on Cl\'asica]
Un proceso estoc\'astico $X=\left\{X\left(t\right):t\geq0\right\}$ es llamado regenerativo is existe una variable aleatoria $R_{1}>0$ tal que
\begin{itemize}
\item[i)] $\left\{X\left(t+R_{1}\right):t\geq0\right\}$ es independiente de $\left\{\left\{X\left(t\right):t<R_{1}\right\},\right\}$
\item[ii)] $\left\{X\left(t+R_{1}\right):t\geq0\right\}$ es estoc\'asticamente equivalente a $\left\{X\left(t\right):t>0\right\}$
\end{itemize}

Llamamos a $R_{1}$ tiempo de regeneraci\'on, y decimos que $X$ se regenera en este punto.
\end{Def}

$\left\{X\left(t+R_{1}\right)\right\}$ es regenerativo con tiempo de regeneraci\'on $R_{2}$, independiente de $R_{1}$ pero con la misma distribuci\'on que $R_{1}$. Procediendo de esta manera se obtiene una secuencia de variables aleatorias independientes e id\'enticamente distribuidas $\left\{R_{n}\right\}$ llamados longitudes de ciclo. Si definimos a $Z_{k}\equiv R_{1}+R_{2}+\cdots+R_{k}$, se tiene un proceso de renovaci\'on llamado proceso de renovaci\'on encajado para $X$.

\begin{Note}
Un proceso regenerativo con media de la longitud de ciclo finita es llamado positivo recurrente.
\end{Note}


\begin{Def}
Para $x$ fijo y para cada $t\geq0$, sea $I_{x}\left(t\right)=1$ si $X\left(t\right)\leq x$,  $I_{x}\left(t\right)=0$ en caso contrario, y def\'inanse los tiempos promedio
\begin{eqnarray*}
\overline{X}&=&lim_{t\rightarrow\infty}\frac{1}{t}\int_{0}^{\infty}X\left(u\right)du\\
\prob\left(X_{\infty}\leq x\right)&=&lim_{t\rightarrow\infty}\frac{1}{t}\int_{0}^{\infty}I_{x}\left(u\right)du,
\end{eqnarray*}
cuando estos l\'imites existan.
\end{Def}

Como consecuencia del teorema de Renovaci\'on-Recompensa, se tiene que el primer l\'imite  existe y es igual a la constante
\begin{eqnarray*}
\overline{X}&=&\frac{\esp\left[\int_{0}^{R_{1}}X\left(t\right)dt\right]}{\esp\left[R_{1}\right]},
\end{eqnarray*}
suponiendo que ambas esperanzas son finitas.

\begin{Note}
\begin{itemize}
\item[a)] Si el proceso regenerativo $X$ es positivo recurrente y tiene trayectorias muestrales no negativas, entonces la ecuaci\'on anterior es v\'alida.
\item[b)] Si $X$ es positivo recurrente regenerativo, podemos construir una \'unica versi\'on estacionaria de este proceso, $X_{e}=\left\{X_{e}\left(t\right)\right\}$, donde $X_{e}$ es un proceso estoc\'astico regenerativo y estrictamente estacionario, con distribuci\'on marginal distribuida como $X_{\infty}$
\end{itemize}
\end{Note}

%__________________________________________________________________________________________
%\subsection{Procesos Regenerativos Estacionarios - Stidham \cite{Stidham}}
%__________________________________________________________________________________________


Un proceso estoc\'astico a tiempo continuo $\left\{V\left(t\right),t\geq0\right\}$ es un proceso regenerativo si existe una sucesi\'on de variables aleatorias independientes e id\'enticamente distribuidas $\left\{X_{1},X_{2},\ldots\right\}$, sucesi\'on de renovaci\'on, tal que para cualquier conjunto de Borel $A$, 

\begin{eqnarray*}
\prob\left\{V\left(t\right)\in A|X_{1}+X_{2}+\cdots+X_{R\left(t\right)}=s,\left\{V\left(\tau\right),\tau<s\right\}\right\}=\prob\left\{V\left(t-s\right)\in A|X_{1}>t-s\right\},
\end{eqnarray*}
para todo $0\leq s\leq t$, donde $R\left(t\right)=\max\left\{X_{1}+X_{2}+\cdots+X_{j}\leq t\right\}=$n\'umero de renovaciones ({\emph{puntos de regeneraci\'on}}) que ocurren en $\left[0,t\right]$. El intervalo $\left[0,X_{1}\right)$ es llamado {\emph{primer ciclo de regeneraci\'on}} de $\left\{V\left(t \right),t\geq0\right\}$, $\left[X_{1},X_{1}+X_{2}\right)$ el {\emph{segundo ciclo de regeneraci\'on}}, y as\'i sucesivamente.

Sea $X=X_{1}$ y sea $F$ la funci\'on de distrbuci\'on de $X$


\begin{Def}
Se define el proceso estacionario, $\left\{V^{*}\left(t\right),t\geq0\right\}$, para $\left\{V\left(t\right),t\geq0\right\}$ por

\begin{eqnarray*}
\prob\left\{V\left(t\right)\in A\right\}=\frac{1}{\esp\left[X\right]}\int_{0}^{\infty}\prob\left\{V\left(t+x\right)\in A|X>x\right\}\left(1-F\left(x\right)\right)dx,
\end{eqnarray*} 
para todo $t\geq0$ y todo conjunto de Borel $A$.
\end{Def}

\begin{Def}
Una distribuci\'on se dice que es {\emph{aritm\'etica}} si todos sus puntos de incremento son m\'ultiplos de la forma $0,\lambda, 2\lambda,\ldots$ para alguna $\lambda>0$ entera.
\end{Def}


\begin{Def}
Una modificaci\'on medible de un proceso $\left\{V\left(t\right),t\geq0\right\}$, es una versi\'on de este, $\left\{V\left(t,w\right)\right\}$ conjuntamente medible para $t\geq0$ y para $w\in S$, $S$ espacio de estados para $\left\{V\left(t\right),t\geq0\right\}$.
\end{Def}

\begin{Teo}
Sea $\left\{V\left(t\right),t\geq\right\}$ un proceso regenerativo no negativo con modificaci\'on medible. Sea $\esp\left[X\right]<\infty$. Entonces el proceso estacionario dado por la ecuaci\'on anterior est\'a bien definido y tiene funci\'on de distribuci\'on independiente de $t$, adem\'as
\begin{itemize}
\item[i)] \begin{eqnarray*}
\esp\left[V^{*}\left(0\right)\right]&=&\frac{\esp\left[\int_{0}^{X}V\left(s\right)ds\right]}{\esp\left[X\right]}\end{eqnarray*}
\item[ii)] Si $\esp\left[V^{*}\left(0\right)\right]<\infty$, equivalentemente, si $\esp\left[\int_{0}^{X}V\left(s\right)ds\right]<\infty$,entonces
\begin{eqnarray*}
\frac{\int_{0}^{t}V\left(s\right)ds}{t}\rightarrow\frac{\esp\left[\int_{0}^{X}V\left(s\right)ds\right]}{\esp\left[X\right]}
\end{eqnarray*}
con probabilidad 1 y en media, cuando $t\rightarrow\infty$.
\end{itemize}
\end{Teo}
%
%___________________________________________________________________________________________
%\vspace{5.5cm}
%\chapter{Cadenas de Markov estacionarias}
%\vspace{-1.0cm}


%__________________________________________________________________________________________
%\subsection{Procesos Regenerativos Estacionarios - Stidham \cite{Stidham}}
%__________________________________________________________________________________________


Un proceso estoc\'astico a tiempo continuo $\left\{V\left(t\right),t\geq0\right\}$ es un proceso regenerativo si existe una sucesi\'on de variables aleatorias independientes e id\'enticamente distribuidas $\left\{X_{1},X_{2},\ldots\right\}$, sucesi\'on de renovaci\'on, tal que para cualquier conjunto de Borel $A$, 

\begin{eqnarray*}
\prob\left\{V\left(t\right)\in A|X_{1}+X_{2}+\cdots+X_{R\left(t\right)}=s,\left\{V\left(\tau\right),\tau<s\right\}\right\}=\prob\left\{V\left(t-s\right)\in A|X_{1}>t-s\right\},
\end{eqnarray*}
para todo $0\leq s\leq t$, donde $R\left(t\right)=\max\left\{X_{1}+X_{2}+\cdots+X_{j}\leq t\right\}=$n\'umero de renovaciones ({\emph{puntos de regeneraci\'on}}) que ocurren en $\left[0,t\right]$. El intervalo $\left[0,X_{1}\right)$ es llamado {\emph{primer ciclo de regeneraci\'on}} de $\left\{V\left(t \right),t\geq0\right\}$, $\left[X_{1},X_{1}+X_{2}\right)$ el {\emph{segundo ciclo de regeneraci\'on}}, y as\'i sucesivamente.

Sea $X=X_{1}$ y sea $F$ la funci\'on de distrbuci\'on de $X$


\begin{Def}
Se define el proceso estacionario, $\left\{V^{*}\left(t\right),t\geq0\right\}$, para $\left\{V\left(t\right),t\geq0\right\}$ por

\begin{eqnarray*}
\prob\left\{V\left(t\right)\in A\right\}=\frac{1}{\esp\left[X\right]}\int_{0}^{\infty}\prob\left\{V\left(t+x\right)\in A|X>x\right\}\left(1-F\left(x\right)\right)dx,
\end{eqnarray*} 
para todo $t\geq0$ y todo conjunto de Borel $A$.
\end{Def}

\begin{Def}
Una distribuci\'on se dice que es {\emph{aritm\'etica}} si todos sus puntos de incremento son m\'ultiplos de la forma $0,\lambda, 2\lambda,\ldots$ para alguna $\lambda>0$ entera.
\end{Def}


\begin{Def}
Una modificaci\'on medible de un proceso $\left\{V\left(t\right),t\geq0\right\}$, es una versi\'on de este, $\left\{V\left(t,w\right)\right\}$ conjuntamente medible para $t\geq0$ y para $w\in S$, $S$ espacio de estados para $\left\{V\left(t\right),t\geq0\right\}$.
\end{Def}

\begin{Teo}
Sea $\left\{V\left(t\right),t\geq\right\}$ un proceso regenerativo no negativo con modificaci\'on medible. Sea $\esp\left[X\right]<\infty$. Entonces el proceso estacionario dado por la ecuaci\'on anterior est\'a bien definido y tiene funci\'on de distribuci\'on independiente de $t$, adem\'as
\begin{itemize}
\item[i)] \begin{eqnarray*}
\esp\left[V^{*}\left(0\right)\right]&=&\frac{\esp\left[\int_{0}^{X}V\left(s\right)ds\right]}{\esp\left[X\right]}\end{eqnarray*}
\item[ii)] Si $\esp\left[V^{*}\left(0\right)\right]<\infty$, equivalentemente, si $\esp\left[\int_{0}^{X}V\left(s\right)ds\right]<\infty$,entonces
\begin{eqnarray*}
\frac{\int_{0}^{t}V\left(s\right)ds}{t}\rightarrow\frac{\esp\left[\int_{0}^{X}V\left(s\right)ds\right]}{\esp\left[X\right]}
\end{eqnarray*}
con probabilidad 1 y en media, cuando $t\rightarrow\infty$.
\end{itemize}
\end{Teo}

Para $\left\{X\left(t\right):t\geq0\right\}$ Proceso Estoc\'astico a tiempo continuo con estado de espacios $S$, que es un espacio m\'etrico, con trayectorias continuas por la derecha y con l\'imites por la izquierda c.s. Sea $N\left(t\right)$ un proceso de renovaci\'on en $\rea_{+}$ definido en el mismo espacio de probabilidad que $X\left(t\right)$, con tiempos de renovaci\'on $T$ y tiempos de inter-renovaci\'on $\xi_{n}=T_{n}-T_{n-1}$, con misma distribuci\'on $F$ de media finita $\mu$.


%______________________________________________________________________
%\subsection{Ejemplos, Notas importantes}


Sean $T_{1},T_{2},\ldots$ los puntos donde las longitudes de las colas de la red de sistemas de visitas c\'iclicas son cero simult\'aneamente, cuando la cola $Q_{j}$ es visitada por el servidor para dar servicio, es decir, $L_{1}\left(T_{i}\right)=0,L_{2}\left(T_{i}\right)=0,\hat{L}_{1}\left(T_{i}\right)=0$ y $\hat{L}_{2}\left(T_{i}\right)=0$, a estos puntos se les denominar\'a puntos regenerativos. Sea la funci\'on generadora de momentos para $L_{i}$, el n\'umero de usuarios en la cola $Q_{i}\left(z\right)$ en cualquier momento, est\'a dada por el tiempo promedio de $z^{L_{i}\left(t\right)}$ sobre el ciclo regenerativo definido anteriormente:

\begin{eqnarray*}
Q_{i}\left(z\right)&=&\esp\left[z^{L_{i}\left(t\right)}\right]=\frac{\esp\left[\sum_{m=1}^{M_{i}}\sum_{t=\tau_{i}\left(m\right)}^{\tau_{i}\left(m+1\right)-1}z^{L_{i}\left(t\right)}\right]}{\esp\left[\sum_{m=1}^{M_{i}}\tau_{i}\left(m+1\right)-\tau_{i}\left(m\right)\right]}
\end{eqnarray*}

$M_{i}$ es un tiempo de paro en el proceso regenerativo con $\esp\left[M_{i}\right]<\infty$\footnote{En Stidham\cite{Stidham} y Heyman  se muestra que una condici\'on suficiente para que el proceso regenerativo 
estacionario sea un procesoo estacionario es que el valor esperado del tiempo del ciclo regenerativo sea finito, es decir: $\esp\left[\sum_{m=1}^{M_{i}}C_{i}^{(m)}\right]<\infty$, como cada $C_{i}^{(m)}$ contiene intervalos de r\'eplica positivos, se tiene que $\esp\left[M_{i}\right]<\infty$, adem\'as, como $M_{i}>0$, se tiene que la condici\'on anterior es equivalente a tener que $\esp\left[C_{i}\right]<\infty$,
por lo tanto una condici\'on suficiente para la existencia del proceso regenerativo est\'a dada por $\sum_{k=1}^{N}\mu_{k}<1.$}, se sigue del lema de Wald que:


\begin{eqnarray*}
\esp\left[\sum_{m=1}^{M_{i}}\sum_{t=\tau_{i}\left(m\right)}^{\tau_{i}\left(m+1\right)-1}z^{L_{i}\left(t\right)}\right]&=&\esp\left[M_{i}\right]\esp\left[\sum_{t=\tau_{i}\left(m\right)}^{\tau_{i}\left(m+1\right)-1}z^{L_{i}\left(t\right)}\right]\\
\esp\left[\sum_{m=1}^{M_{i}}\tau_{i}\left(m+1\right)-\tau_{i}\left(m\right)\right]&=&\esp\left[M_{i}\right]\esp\left[\tau_{i}\left(m+1\right)-\tau_{i}\left(m\right)\right]
\end{eqnarray*}

por tanto se tiene que


\begin{eqnarray*}
Q_{i}\left(z\right)&=&\frac{\esp\left[\sum_{t=\tau_{i}\left(m\right)}^{\tau_{i}\left(m+1\right)-1}z^{L_{i}\left(t\right)}\right]}{\esp\left[\tau_{i}\left(m+1\right)-\tau_{i}\left(m\right)\right]}
\end{eqnarray*}

observar que el denominador es simplemente la duraci\'on promedio del tiempo del ciclo.


Haciendo las siguientes sustituciones en la ecuaci\'on (\ref{Corolario2}): $n\rightarrow t-\tau_{i}\left(m\right)$, $T \rightarrow \overline{\tau}_{i}\left(m\right)-\tau_{i}\left(m\right)$, $L_{n}\rightarrow L_{i}\left(t\right)$ y $F\left(z\right)=\esp\left[z^{L_{0}}\right]\rightarrow F_{i}\left(z\right)=\esp\left[z^{L_{i}\tau_{i}\left(m\right)}\right]$, se puede ver que

\begin{eqnarray}\label{Eq.Arribos.Primera}
\esp\left[\sum_{n=0}^{T-1}z^{L_{n}}\right]=
\esp\left[\sum_{t=\tau_{i}\left(m\right)}^{\overline{\tau}_{i}\left(m\right)-1}z^{L_{i}\left(t\right)}\right]
=z\frac{F_{i}\left(z\right)-1}{z-P_{i}\left(z\right)}
\end{eqnarray}

Por otra parte durante el tiempo de intervisita para la cola $i$, $L_{i}\left(t\right)$ solamente se incrementa de manera que el incremento por intervalo de tiempo est\'a dado por la funci\'on generadora de probabilidades de $P_{i}\left(z\right)$, por tanto la suma sobre el tiempo de intervisita puede evaluarse como:

\begin{eqnarray*}
\esp\left[\sum_{t=\tau_{i}\left(m\right)}^{\tau_{i}\left(m+1\right)-1}z^{L_{i}\left(t\right)}\right]&=&\esp\left[\sum_{t=\tau_{i}\left(m\right)}^{\tau_{i}\left(m+1\right)-1}\left\{P_{i}\left(z\right)\right\}^{t-\overline{\tau}_{i}\left(m\right)}\right]=\frac{1-\esp\left[\left\{P_{i}\left(z\right)\right\}^{\tau_{i}\left(m+1\right)-\overline{\tau}_{i}\left(m\right)}\right]}{1-P_{i}\left(z\right)}\\
&=&\frac{1-I_{i}\left[P_{i}\left(z\right)\right]}{1-P_{i}\left(z\right)}
\end{eqnarray*}
por tanto

\begin{eqnarray*}
\esp\left[\sum_{t=\tau_{i}\left(m\right)}^{\tau_{i}\left(m+1\right)-1}z^{L_{i}\left(t\right)}\right]&=&
\frac{1-F_{i}\left(z\right)}{1-P_{i}\left(z\right)}
\end{eqnarray*}

Por lo tanto

\begin{eqnarray*}
Q_{i}\left(z\right)&=&\frac{\esp\left[\sum_{t=\tau_{i}\left(m\right)}^{\tau_{i}
\left(m+1\right)-1}z^{L_{i}\left(t\right)}\right]}{\esp\left[\tau_{i}\left(m+1\right)-\tau_{i}\left(m\right)\right]}\\
&=&\frac{1}{\esp\left[\tau_{i}\left(m+1\right)-\tau_{i}\left(m\right)\right]}
\left\{
\esp\left[\sum_{t=\tau_{i}\left(m\right)}^{\overline{\tau}_{i}\left(m\right)-1}
z^{L_{i}\left(t\right)}\right]
+\esp\left[\sum_{t=\overline{\tau}_{i}\left(m\right)}^{\tau_{i}\left(m+1\right)-1}
z^{L_{i}\left(t\right)}\right]\right\}\\
&=&\frac{1}{\esp\left[\tau_{i}\left(m+1\right)-\tau_{i}\left(m\right)\right]}
\left\{
z\frac{F_{i}\left(z\right)-1}{z-P_{i}\left(z\right)}+\frac{1-F_{i}\left(z\right)}
{1-P_{i}\left(z\right)}
\right\}
\end{eqnarray*}

es decir

\begin{equation}
Q_{i}\left(z\right)=\frac{1}{\esp\left[C_{i}\right]}\cdot\frac{1-F_{i}\left(z\right)}{P_{i}\left(z\right)-z}\cdot\frac{\left(1-z\right)P_{i}\left(z\right)}{1-P_{i}\left(z\right)}
\end{equation}

\begin{Teo}
Dada una Red de Sistemas de Visitas C\'iclicas (RSVC), conformada por dos Sistemas de Visitas C\'iclicas (SVC), donde cada uno de ellos consta de dos colas tipo $M/M/1$. Los dos sistemas est\'an comunicados entre s\'i por medio de la transferencia de usuarios entre las colas $Q_{1}\leftrightarrow Q_{3}$ y $Q_{2}\leftrightarrow Q_{4}$. Se definen los eventos para los procesos de arribos al tiempo $t$, $A_{j}\left(t\right)=\left\{0 \textrm{ arribos en }Q_{j}\textrm{ al tiempo }t\right\}$ para alg\'un tiempo $t\geq0$ y $Q_{j}$ la cola $j$-\'esima en la RSVC, para $j=1,2,3,4$.  Existe un intervalo $I\neq\emptyset$ tal que para $T^{*}\in I$, tal que $\prob\left\{A_{1}\left(T^{*}\right),A_{2}\left(Tt^{*}\right),
A_{3}\left(T^{*}\right),A_{4}\left(T^{*}\right)|T^{*}\in I\right\}>0$.
\end{Teo}

\begin{proof}
Sin p\'erdida de generalidad podemos considerar como base del an\'alisis a la cola $Q_{1}$ del primer sistema que conforma la RSVC.

Sea $n>0$, ciclo en el primer sistema en el que se sabe que $L_{j}\left(\overline{\tau}_{1}\left(n\right)\right)=0$, pues la pol\'itica de servicio con que atienden los servidores es la exhaustiva. Como es sabido, para trasladarse a la siguiente cola, el servidor incurre en un tiempo de traslado $r_{1}\left(n\right)>0$, entonces tenemos que $\tau_{2}\left(n\right)=\overline{\tau}_{1}\left(n\right)+r_{1}\left(n\right)$.


Definamos el intervalo $I_{1}\equiv\left[\overline{\tau}_{1}\left(n\right),\tau_{2}\left(n\right)\right]$ de longitud $\xi_{1}=r_{1}\left(n\right)$. Dado que los tiempos entre arribo son exponenciales con tasa $\tilde{\mu}_{1}=\mu_{1}+\hat{\mu}_{1}$ ($\mu_{1}$ son los arribos a $Q_{1}$ por primera vez al sistema, mientras que $\hat{\mu}_{1}$ son los arribos de traslado procedentes de $Q_{3}$) se tiene que la probabilidad del evento $A_{1}\left(t\right)$ est\'a dada por 

\begin{equation}
\prob\left\{A_{1}\left(t\right)|t\in I_{1}\left(n\right)\right\}=e^{-\tilde{\mu}_{1}\xi_{1}\left(n\right)}.
\end{equation} 

Por otra parte, para la cola $Q_{2}$, el tiempo $\overline{\tau}_{2}\left(n-1\right)$ es tal que $L_{2}\left(\overline{\tau}_{2}\left(n-1\right)\right)=0$, es decir, es el tiempo en que la cola queda totalmente vac\'ia en el ciclo anterior a $n$. Entonces tenemos un sgundo intervalo $I_{2}\equiv\left[\overline{\tau}_{2}\left(n-1\right),\tau_{2}\left(n\right)\right]$. Por lo tanto la probabilidad del evento $A_{2}\left(t\right)$ tiene probabilidad dada por

\begin{equation}
\prob\left\{A_{2}\left(t\right)|t\in I_{2}\left(n\right)\right\}=e^{-\tilde{\mu}_{2}\xi_{2}\left(n\right)},
\end{equation} 

donde $\xi_{2}\left(n\right)=\tau_{2}\left(n\right)-\overline{\tau}_{2}\left(n-1\right)$.



Entonces, se tiene que

\begin{eqnarray*}
\prob\left\{A_{1}\left(t\right),A_{2}\left(t\right)|t\in I_{1}\left(n\right)\right\}&=&
\prob\left\{A_{1}\left(t\right)|t\in I_{1}\left(n\right)\right\}
\prob\left\{A_{2}\left(t\right)|t\in I_{1}\left(n\right)\right\}\\
&\geq&
\prob\left\{A_{1}\left(t\right)|t\in I_{1}\left(n\right)\right\}
\prob\left\{A_{2}\left(t\right)|t\in I_{2}\left(n\right)\right\}\\
&=&e^{-\tilde{\mu}_{1}\xi_{1}\left(n\right)}e^{-\tilde{\mu}_{2}\xi_{2}\left(n\right)}
=e^{-\left[\tilde{\mu}_{1}\xi_{1}\left(n\right)+\tilde{\mu}_{2}\xi_{2}\left(n\right)\right]}.
\end{eqnarray*}


es decir, 

\begin{equation}
\prob\left\{A_{1}\left(t\right),A_{2}\left(t\right)|t\in I_{1}\left(n\right)\right\}
=e^{-\left[\tilde{\mu}_{1}\xi_{1}\left(n\right)+\tilde{\mu}_{2}\xi_{2}
\left(n\right)\right]}>0.
\end{equation}

En lo que respecta a la relaci\'on entre los dos SVC que conforman la RSVC, se afirma que existe $m>0$ tal que $\overline{\tau}_{3}\left(m\right)<\tau_{2}\left(n\right)<\tau_{4}\left(m\right)$.

Para $Q_{3}$ sea $I_{3}=\left[\overline{\tau}_{3}\left(m\right),\tau_{4}\left(m\right)\right]$ con longitud  $\xi_{3}\left(m\right)=r_{3}\left(m\right)$, entonces 

\begin{equation}
\prob\left\{A_{3}\left(t\right)|t\in I_{3}\left(n\right)\right\}=e^{-\tilde{\mu}_{3}\xi_{3}\left(n\right)}.
\end{equation} 

An\'alogamente que como se hizo para $Q_{2}$, tenemos que para $Q_{4}$ se tiene el intervalo $I_{4}=\left[\overline{\tau}_{4}\left(m-1\right),\tau_{4}\left(m\right)\right]$ con longitud $\xi_{4}\left(m\right)=\tau_{4}\left(m\right)-\overline{\tau}_{4}\left(m-1\right)$, entonces


\begin{equation}
\prob\left\{A_{4}\left(t\right)|t\in I_{4}\left(m\right)\right\}=e^{-\tilde{\mu}_{4}\xi_{4}\left(n\right)}.
\end{equation} 

Al igual que para el primer sistema, dado que $I_{3}\left(m\right)\subset I_{4}\left(m\right)$, se tiene que

\begin{eqnarray*}
\xi_{3}\left(m\right)\leq\xi_{4}\left(m\right)&\Leftrightarrow& -\xi_{3}\left(m\right)\geq-\xi_{4}\left(m\right)
\\
-\tilde{\mu}_{4}\xi_{3}\left(m\right)\geq-\tilde{\mu}_{4}\xi_{4}\left(m\right)&\Leftrightarrow&
e^{-\tilde{\mu}_{4}\xi_{3}\left(m\right)}\geq e^{-\tilde{\mu}_{4}\xi_{4}\left(m\right)}\\
\prob\left\{A_{4}\left(t\right)|t\in I_{3}\left(m\right)\right\}&\geq&
\prob\left\{A_{4}\left(t\right)|t\in I_{4}\left(m\right)\right\}
\end{eqnarray*}

Entonces, dado que los eventos $A_{3}$ y $A_{4}$ son independientes, se tiene que

\begin{eqnarray*}
\prob\left\{A_{3}\left(t\right),A_{4}\left(t\right)|t\in I_{3}\left(m\right)\right\}&=&
\prob\left\{A_{3}\left(t\right)|t\in I_{3}\left(m\right)\right\}
\prob\left\{A_{4}\left(t\right)|t\in I_{3}\left(m\right)\right\}\\
&\geq&
\prob\left\{A_{3}\left(t\right)|t\in I_{3}\left(n\right)\right\}
\prob\left\{A_{4}\left(t\right)|t\in I_{4}\left(n\right)\right\}\\
&=&e^{-\tilde{\mu}_{3}\xi_{3}\left(m\right)}e^{-\tilde{\mu}_{4}\xi_{4}
\left(m\right)}
=e^{-\left[\tilde{\mu}_{3}\xi_{3}\left(m\right)+\tilde{\mu}_{4}\xi_{4}
\left(m\right)\right]}.
\end{eqnarray*}


es decir, 

\begin{equation}
\prob\left\{A_{3}\left(t\right),A_{4}\left(t\right)|t\in I_{3}\left(m\right)\right\}
=e^{-\left[\tilde{\mu}_{3}\xi_{3}\left(m\right)+\tilde{\mu}_{4}\xi_{4}
\left(m\right)\right]}>0.
\end{equation}

Por construcci\'on se tiene que $I\left(n,m\right)\equiv I_{1}\left(n\right)\cap I_{3}\left(m\right)\neq\emptyset$,entonces en particular se tienen las contenciones $I\left(n,m\right)\subseteq I_{1}\left(n\right)$ y $I\left(n,m\right)\subseteq I_{3}\left(m\right)$, por lo tanto si definimos $\xi_{n,m}\equiv\ell\left(I\left(n,m\right)\right)$ tenemos que

\begin{eqnarray*}
\xi_{n,m}\leq\xi_{1}\left(n\right)\textrm{ y }\xi_{n,m}\leq\xi_{3}\left(m\right)\textrm{ entonces }
-\xi_{n,m}\geq-\xi_{1}\left(n\right)\textrm{ y }-\xi_{n,m}\leq-\xi_{3}\left(m\right)\\
\end{eqnarray*}
por lo tanto tenemos las desigualdades 



\begin{eqnarray*}
\begin{array}{ll}
-\tilde{\mu}_{1}\xi_{n,m}\geq-\tilde{\mu}_{1}\xi_{1}\left(n\right),&
-\tilde{\mu}_{2}\xi_{n,m}\geq-\tilde{\mu}_{2}\xi_{1}\left(n\right)
\geq-\tilde{\mu}_{2}\xi_{2}\left(n\right),\\
-\tilde{\mu}_{3}\xi_{n,m}\geq-\tilde{\mu}_{3}\xi_{3}\left(m\right),&
-\tilde{\mu}_{4}\xi_{n,m}\geq-\tilde{\mu}_{4}\xi_{3}\left(m\right)
\geq-\tilde{\mu}_{4}\xi_{4}\left(m\right).
\end{array}
\end{eqnarray*}

Sea $T^{*}\in I_{n,m}$, entonces dado que en particular $T^{*}\in I_{1}\left(n\right)$ se cumple con probabilidad positiva que no hay arribos a las colas $Q_{1}$ y $Q_{2}$, en consecuencia, tampoco hay usuarios de transferencia para $Q_{3}$ y $Q_{4}$, es decir, $\tilde{\mu}_{1}=\mu_{1}$, $\tilde{\mu}_{2}=\mu_{2}$, $\tilde{\mu}_{3}=\mu_{3}$, $\tilde{\mu}_{4}=\mu_{4}$, es decir, los eventos $Q_{1}$ y $Q_{3}$ son condicionalmente independientes en el intervalo $I_{n,m}$; lo mismo ocurre para las colas $Q_{2}$ y $Q_{4}$, por lo tanto tenemos que


\begin{eqnarray}
\begin{array}{l}
\prob\left\{A_{1}\left(T^{*}\right),A_{2}\left(T^{*}\right),
A_{3}\left(T^{*}\right),A_{4}\left(T^{*}\right)|T^{*}\in I_{n,m}\right\}
=\prod_{j=1}^{4}\prob\left\{A_{j}\left(T^{*}\right)|T^{*}\in I_{n,m}\right\}\\
\geq\prob\left\{A_{1}\left(T^{*}\right)|T^{*}\in I_{1}\left(n\right)\right\}
\prob\left\{A_{2}\left(T^{*}\right)|T^{*}\in I_{2}\left(n\right)\right\}
\prob\left\{A_{3}\left(T^{*}\right)|T^{*}\in I_{3}\left(m\right)\right\}
\prob\left\{A_{4}\left(T^{*}\right)|T^{*}\in I_{4}\left(m\right)\right\}\\
=e^{-\mu_{1}\xi_{1}\left(n\right)}
e^{-\mu_{2}\xi_{2}\left(n\right)}
e^{-\mu_{3}\xi_{3}\left(m\right)}
e^{-\mu_{4}\xi_{4}\left(m\right)}
=e^{-\left[\tilde{\mu}_{1}\xi_{1}\left(n\right)
+\tilde{\mu}_{2}\xi_{2}\left(n\right)
+\tilde{\mu}_{3}\xi_{3}\left(m\right)
+\tilde{\mu}_{4}\xi_{4}
\left(m\right)\right]}>0.
\end{array}
\end{eqnarray}
\end{proof}


Estos resultados aparecen en Daley (1968) \cite{Daley68} para $\left\{T_{n}\right\}$ intervalos de inter-arribo, $\left\{D_{n}\right\}$ intervalos de inter-salida y $\left\{S_{n}\right\}$ tiempos de servicio.

\begin{itemize}
\item Si el proceso $\left\{T_{n}\right\}$ es Poisson, el proceso $\left\{D_{n}\right\}$ es no correlacionado si y s\'olo si es un proceso Poisso, lo cual ocurre si y s\'olo si $\left\{S_{n}\right\}$ son exponenciales negativas.

\item Si $\left\{S_{n}\right\}$ son exponenciales negativas, $\left\{D_{n}\right\}$ es un proceso de renovaci\'on  si y s\'olo si es un proceso Poisson, lo cual ocurre si y s\'olo si $\left\{T_{n}\right\}$ es un proceso Poisson.

\item $\esp\left(D_{n}\right)=\esp\left(T_{n}\right)$.

\item Para un sistema de visitas $GI/M/1$ se tiene el siguiente teorema:

\begin{Teo}
En un sistema estacionario $GI/M/1$ los intervalos de interpartida tienen
\begin{eqnarray*}
\esp\left(e^{-\theta D_{n}}\right)&=&\mu\left(\mu+\theta\right)^{-1}\left[\delta\theta
-\mu\left(1-\delta\right)\alpha\left(\theta\right)\right]
\left[\theta-\mu\left(1-\delta\right)^{-1}\right]\\
\alpha\left(\theta\right)&=&\esp\left[e^{-\theta T_{0}}\right]\\
var\left(D_{n}\right)&=&var\left(T_{0}\right)-\left(\tau^{-1}-\delta^{-1}\right)
2\delta\left(\esp\left(S_{0}\right)\right)^{2}\left(1-\delta\right)^{-1}.
\end{eqnarray*}
\end{Teo}



\begin{Teo}
El proceso de salida de un sistema de colas estacionario $GI/M/1$ es un proceso de renovaci\'on si y s\'olo si el proceso de entrada es un proceso Poisson, en cuyo caso el proceso de salida es un proceso Poisson.
\end{Teo}


\begin{Teo}
Los intervalos de interpartida $\left\{D_{n}\right\}$ de un sistema $M/G/1$ estacionario son no correlacionados si y s\'olo si la distribuci\'on de los tiempos de servicio es exponencial negativa, es decir, el sistema es de tipo  $M/M/1$.

\end{Teo}



\end{itemize}


%\section{Resultados para Procesos de Salida}

En Sigman, Thorison y Wolff \cite{Sigman2} prueban que para la existencia de un una sucesi\'on infinita no decreciente de tiempos de regeneraci\'on $\tau_{1}\leq\tau_{2}\leq\cdots$ en los cuales el proceso se regenera, basta un tiempo de regeneraci\'on $R_{1}$, donde $R_{j}=\tau_{j}-\tau_{j-1}$. Para tal efecto se requiere la existencia de un espacio de probabilidad $\left(\Omega,\mathcal{F},\prob\right)$, y proceso estoc\'astico $\textit{X}=\left\{X\left(t\right):t\geq0\right\}$ con espacio de estados $\left(S,\mathcal{R}\right)$, con $\mathcal{R}$ $\sigma$-\'algebra.

\begin{Prop}
Si existe una variable aleatoria no negativa $R_{1}$ tal que $\theta_{R\footnotesize{1}}X=_{D}X$, entonces $\left(\Omega,\mathcal{F},\prob\right)$ puede extenderse para soportar una sucesi\'on estacionaria de variables aleatorias $R=\left\{R_{k}:k\geq1\right\}$, tal que para $k\geq1$,
\begin{eqnarray*}
\theta_{k}\left(X,R\right)=_{D}\left(X,R\right).
\end{eqnarray*}

Adem\'as, para $k\geq1$, $\theta_{k}R$ es condicionalmente independiente de $\left(X,R_{1},\ldots,R_{k}\right)$, dado $\theta_{\tau k}X$.

\end{Prop}


\begin{itemize}
\item Doob en 1953 demostr\'o que el estado estacionario de un proceso de partida en un sistema de espera $M/G/\infty$, es Poisson con la misma tasa que el proceso de arribos.

\item Burke en 1968, fue el primero en demostrar que el estado estacionario de un proceso de salida de una cola $M/M/s$ es un proceso Poisson.

\item Disney en 1973 obtuvo el siguiente resultado:

\begin{Teo}
Para el sistema de espera $M/G/1/L$ con disciplina FIFO, el proceso $\textbf{I}$ es un proceso de renovaci\'on si y s\'olo si el proceso denominado longitud de la cola es estacionario y se cumple cualquiera de los siguientes casos:

\begin{itemize}
\item[a)] Los tiempos de servicio son identicamente cero;
\item[b)] $L=0$, para cualquier proceso de servicio $S$;
\item[c)] $L=1$ y $G=D$;
\item[d)] $L=\infty$ y $G=M$.
\end{itemize}
En estos casos, respectivamente, las distribuciones de interpartida $P\left\{T_{n+1}-T_{n}\leq t\right\}$ son


\begin{itemize}
\item[a)] $1-e^{-\lambda t}$, $t\geq0$;
\item[b)] $1-e^{-\lambda t}*F\left(t\right)$, $t\geq0$;
\item[c)] $1-e^{-\lambda t}*\indora_{d}\left(t\right)$, $t\geq0$;
\item[d)] $1-e^{-\lambda t}*F\left(t\right)$, $t\geq0$.
\end{itemize}
\end{Teo}


\item Finch (1959) mostr\'o que para los sistemas $M/G/1/L$, con $1\leq L\leq \infty$ con distribuciones de servicio dos veces diferenciable, solamente el sistema $M/M/1/\infty$ tiene proceso de salida de renovaci\'on estacionario.

\item King (1971) demostro que un sistema de colas estacionario $M/G/1/1$ tiene sus tiempos de interpartida sucesivas $D_{n}$ y $D_{n+1}$ son independientes, si y s\'olo si, $G=D$, en cuyo caso le proceso de salida es de renovaci\'on.

\item Disney (1973) demostr\'o que el \'unico sistema estacionario $M/G/1/L$, que tiene proceso de salida de renovaci\'on  son los sistemas $M/M/1$ y $M/D/1/1$.



\item El siguiente resultado es de Disney y Koning (1985)
\begin{Teo}
En un sistema de espera $M/G/s$, el estado estacionario del proceso de salida es un proceso Poisson para cualquier distribuci\'on de los tiempos de servicio si el sistema tiene cualquiera de las siguientes cuatro propiedades.

\begin{itemize}
\item[a)] $s=\infty$
\item[b)] La disciplina de servicio es de procesador compartido.
\item[c)] La disciplina de servicio es LCFS y preemptive resume, esto se cumple para $L<\infty$
\item[d)] $G=M$.
\end{itemize}

\end{Teo}

\item El siguiente resultado es de Alamatsaz (1983)

\begin{Teo}
En cualquier sistema de colas $GI/G/1/L$ con $1\leq L<\infty$ y distribuci\'on de interarribos $A$ y distribuci\'on de los tiempos de servicio $B$, tal que $A\left(0\right)=0$, $A\left(t\right)\left(1-B\left(t\right)\right)>0$ para alguna $t>0$ y $B\left(t\right)$ para toda $t>0$, es imposible que el proceso de salida estacionario sea de renovaci\'on.
\end{Teo}

\end{itemize}

Estos resultados aparecen en Daley (1968) \cite{Daley68} para $\left\{T_{n}\right\}$ intervalos de inter-arribo, $\left\{D_{n}\right\}$ intervalos de inter-salida y $\left\{S_{n}\right\}$ tiempos de servicio.

\begin{itemize}
\item Si el proceso $\left\{T_{n}\right\}$ es Poisson, el proceso $\left\{D_{n}\right\}$ es no correlacionado si y s\'olo si es un proceso Poisso, lo cual ocurre si y s\'olo si $\left\{S_{n}\right\}$ son exponenciales negativas.

\item Si $\left\{S_{n}\right\}$ son exponenciales negativas, $\left\{D_{n}\right\}$ es un proceso de renovaci\'on  si y s\'olo si es un proceso Poisson, lo cual ocurre si y s\'olo si $\left\{T_{n}\right\}$ es un proceso Poisson.

\item $\esp\left(D_{n}\right)=\esp\left(T_{n}\right)$.

\item Para un sistema de visitas $GI/M/1$ se tiene el siguiente teorema:

\begin{Teo}
En un sistema estacionario $GI/M/1$ los intervalos de interpartida tienen
\begin{eqnarray*}
\esp\left(e^{-\theta D_{n}}\right)&=&\mu\left(\mu+\theta\right)^{-1}\left[\delta\theta
-\mu\left(1-\delta\right)\alpha\left(\theta\right)\right]
\left[\theta-\mu\left(1-\delta\right)^{-1}\right]\\
\alpha\left(\theta\right)&=&\esp\left[e^{-\theta T_{0}}\right]\\
var\left(D_{n}\right)&=&var\left(T_{0}\right)-\left(\tau^{-1}-\delta^{-1}\right)
2\delta\left(\esp\left(S_{0}\right)\right)^{2}\left(1-\delta\right)^{-1}.
\end{eqnarray*}
\end{Teo}



\begin{Teo}
El proceso de salida de un sistema de colas estacionario $GI/M/1$ es un proceso de renovaci\'on si y s\'olo si el proceso de entrada es un proceso Poisson, en cuyo caso el proceso de salida es un proceso Poisson.
\end{Teo}


\begin{Teo}
Los intervalos de interpartida $\left\{D_{n}\right\}$ de un sistema $M/G/1$ estacionario son no correlacionados si y s\'olo si la distribuci\'on de los tiempos de servicio es exponencial negativa, es decir, el sistema es de tipo  $M/M/1$.

\end{Teo}



\end{itemize}
%\newpage
%________________________________________________________________________
%\section{Redes de Sistemas de Visitas C\'iclicas}
%________________________________________________________________________

Sean $Q_{1},Q_{2},Q_{3}$ y $Q_{4}$ en una Red de Sistemas de Visitas C\'iclicas (RSVC). Supongamos que cada una de las colas es del tipo $M/M/1$ con tasa de arribo $\mu_{i}$ y que la transferencia de usuarios entre los dos sistemas ocurre entre $Q_{1}\leftrightarrow Q_{3}$ y $Q_{2}\leftrightarrow Q_{4}$ con respectiva tasa de arribo igual a la tasa de salida $\hat{\mu}_{i}=\mu_{i}$, esto se sabe por lo desarrollado en la secci\'on anterior.  

Consideremos, sin p\'erdida de generalidad como base del an\'alisis, la cola $Q_{1}$ adem\'as supongamos al servidor lo comenzamos a observar una vez que termina de atender a la misma para desplazarse y llegar a $Q_{2}$, es decir al tiempo $\tau_{2}$.

Sea $n\in\nat$, $n>0$, ciclo del servidor en que regresa a $Q_{1}$ para dar servicio y atender conforme a la pol\'itica exhaustiva, entonces se tiene que $\overline{\tau}_{1}\left(n\right)$ es el tiempo del servidor en el sistema 1 en que termina de dar servicio a todos los usuarios presentes en la cola, por lo tanto se cumple que $L_{1}\left(\overline{\tau}_{1}\left(n\right)\right)=0$, entonces el servidor para llegar a $Q_{2}$ incurre en un tiempo de traslado $r_{1}$ y por tanto se cumple que $\tau_{2}\left(n\right)=\overline{\tau}_{1}\left(n\right)+r_{1}$. Dado que los tiempos entre arribos son exponenciales se cumple que 

\begin{eqnarray*}
\prob\left\{0 \textrm{ arribos en }Q_{1}\textrm{ en el intervalo }\left[\overline{\tau}_{1}\left(n\right),\overline{\tau}_{1}\left(n\right)+r_{1}\right]\right\}=e^{-\tilde{\mu}_{1}r_{1}},\\
\prob\left\{0 \textrm{ arribos en }Q_{2}\textrm{ en el intervalo }\left[\overline{\tau}_{1}\left(n\right),\overline{\tau}_{1}\left(n\right)+r_{1}\right]\right\}=e^{-\tilde{\mu}_{2}r_{1}}.
\end{eqnarray*}

El evento que nos interesa consiste en que no haya arribos desde que el servidor abandon\'o $Q_{2}$ y regresa nuevamente para dar servicio, es decir en el intervalo de tiempo $\left[\overline{\tau}_{2}\left(n-1\right),\tau_{2}\left(n\right)\right]$. Entonces, si hacemos


\begin{eqnarray*}
\varphi_{1}\left(n\right)&\equiv&\overline{\tau}_{1}\left(n\right)+r_{1}=\overline{\tau}_{2}\left(n-1\right)+r_{1}+r_{2}+\overline{\tau}_{1}\left(n\right)-\tau_{1}\left(n\right)\\
&=&\overline{\tau}_{2}\left(n-1\right)+\overline{\tau}_{1}\left(n\right)-\tau_{1}\left(n\right)+r,,
\end{eqnarray*}

y longitud del intervalo

\begin{eqnarray*}
\xi&\equiv&\overline{\tau}_{1}\left(n\right)+r_{1}-\overline{\tau}_{2}\left(n-1\right)
=\overline{\tau}_{2}\left(n-1\right)+\overline{\tau}_{1}\left(n\right)-\tau_{1}\left(n\right)+r-\overline{\tau}_{2}\left(n-1\right)\\
&=&\overline{\tau}_{1}\left(n\right)-\tau_{1}\left(n\right)+r.
\end{eqnarray*}


Entonces, determinemos la probabilidad del evento no arribos a $Q_{2}$ en $\left[\overline{\tau}_{2}\left(n-1\right),\varphi_{1}\left(n\right)\right]$:

\begin{eqnarray}
\prob\left\{0 \textrm{ arribos en }Q_{2}\textrm{ en el intervalo }\left[\overline{\tau}_{2}\left(n-1\right),\varphi_{1}\left(n\right)\right]\right\}
=e^{-\tilde{\mu}_{2}\xi}.
\end{eqnarray}

De manera an\'aloga, tenemos que la probabilidad de no arribos a $Q_{1}$ en $\left[\overline{\tau}_{2}\left(n-1\right),\varphi_{1}\left(n\right)\right]$ esta dada por

\begin{eqnarray}
\prob\left\{0 \textrm{ arribos en }Q_{1}\textrm{ en el intervalo }\left[\overline{\tau}_{2}\left(n-1\right),\varphi_{1}\left(n\right)\right]\right\}
=e^{-\tilde{\mu}_{1}\xi},
\end{eqnarray}

\begin{eqnarray}
\prob\left\{0 \textrm{ arribos en }Q_{2}\textrm{ en el intervalo }\left[\overline{\tau}_{2}\left(n-1\right),\varphi_{1}\left(n\right)\right]\right\}
=e^{-\tilde{\mu}_{2}\xi}.
\end{eqnarray}

Por tanto 

\begin{eqnarray}
\begin{array}{l}
\prob\left\{0 \textrm{ arribos en }Q_{1}\textrm{ y }Q_{2}\textrm{ en el intervalo }\left[\overline{\tau}_{2}\left(n-1\right),\varphi_{1}\left(n\right)\right]\right\}\\
=\prob\left\{0 \textrm{ arribos en }Q_{1}\textrm{ en el intervalo }\left[\overline{\tau}_{2}\left(n-1\right),\varphi_{1}\left(n\right)\right]\right\}\\
\times
\prob\left\{0 \textrm{ arribos en }Q_{2}\textrm{ en el intervalo }\left[\overline{\tau}_{2}\left(n-1\right),\varphi_{1}\left(n\right)\right]\right\}=e^{-\tilde{\mu}_{1}\xi}e^{-\tilde{\mu}_{2}\xi}
=e^{-\tilde{\mu}\xi}.
\end{array}
\end{eqnarray}

Para el segundo sistema, consideremos nuevamente $\overline{\tau}_{1}\left(n\right)+r_{1}$, sin p\'erdida de generalidad podemos suponer que existe $m>0$ tal que $\overline{\tau}_{3}\left(m\right)<\overline{\tau}_{1}+r_{1}<\tau_{4}\left(m\right)$, entonces

\begin{eqnarray}
\prob\left\{0 \textrm{ arribos en }Q_{3}\textrm{ en el intervalo }\left[\overline{\tau}_{3}\left(m\right),\overline{\tau}_{1}\left(n\right)+r_{1}\right]\right\}
=e^{-\tilde{\mu}_{3}\xi_{3}},
\end{eqnarray}
donde 
\begin{eqnarray}
\xi_{3}=\overline{\tau}_{1}\left(n\right)+r_{1}-\overline{\tau}_{3}\left(m\right)=
\overline{\tau}_{1}\left(n\right)-\overline{\tau}_{3}\left(m\right)+r_{1},
\end{eqnarray}

mientras que para $Q_{4}$ al igual que con $Q_{2}$ escribiremos $\tau_{4}\left(m\right)$ en t\'erminos de $\overline{\tau}_{4}\left(m-1\right)$:

$\varphi_{2}\equiv\tau_{4}\left(m\right)=\overline{\tau}_{4}\left(m-1\right)+r_{4}+\overline{\tau}_{3}\left(m\right)
-\tau_{3}\left(m\right)+r_{3}=\overline{\tau}_{4}\left(m-1\right)+\overline{\tau}_{3}\left(m\right)
-\tau_{3}\left(m\right)+\hat{r}$, adem\'as,

$\xi_{2}\equiv\varphi_{2}\left(m\right)-\overline{\tau}_{1}-r_{1}=\overline{\tau}_{4}\left(m-1\right)+\overline{\tau}_{3}\left(m\right)
-\tau_{3}\left(m\right)-\overline{\tau}_{1}\left(n\right)+\hat{r}-r_{1}$. 

Entonces


\begin{eqnarray}
\prob\left\{0 \textrm{ arribos en }Q_{4}\textrm{ en el intervalo }\left[\overline{\tau}_{1}\left(n\right)+r_{1},\varphi_{2}\left(m\right)\right]\right\}
=e^{-\tilde{\mu}_{4}\xi_{2}},
\end{eqnarray}

mientras que para $Q_{3}$ se tiene que 

\begin{eqnarray}
\prob\left\{0 \textrm{ arribos en }Q_{3}\textrm{ en el intervalo }\left[\overline{\tau}_{1}\left(n\right)+r_{1},\varphi_{2}\left(m\right)\right]\right\}
=e^{-\tilde{\mu}_{3}\xi_{2}}
\end{eqnarray}

Por tanto

\begin{eqnarray}
\prob\left\{0 \textrm{ arribos en }Q_{3}\wedge Q_{4}\textrm{ en el intervalo }\left[\overline{\tau}_{1}\left(n\right)+r_{1},\varphi_{2}\left(m\right)\right]\right\}
=e^{-\hat{\mu}\xi_{2}}
\end{eqnarray}
donde $\hat{\mu}=\tilde{\mu}_{3}+\tilde{\mu}_{4}$.

Ahora, definamos los intervalos $\mathcal{I}_{1}=\left[\overline{\tau}_{1}\left(n\right)+r_{1},\varphi_{1}\left(n\right)\right]$  y $\mathcal{I}_{2}=\left[\overline{\tau}_{1}\left(n\right)+r_{1},\varphi_{2}\left(m\right)\right]$, entonces, sea $\mathcal{I}=\mathcal{I}_{1}\cap\mathcal{I}_{2}$ el intervalo donde cada una de las colas se encuentran vac\'ias, es decir, si tomamos $T^{*}\in\mathcal{I}$, entonces  $L_{1}\left(T^{*}\right)=L_{2}\left(T^{*}\right)=L_{3}\left(T^{*}\right)=L_{4}\left(T^{*}\right)=0$.

Ahora, dado que por construcci\'on $\mathcal{I}\neq\emptyset$ y que para $T^{*}\in\mathcal{I}$ en ninguna de las colas han llegado usuarios, se tiene que no hay transferencia entre las colas, por lo tanto, el sistema 1 y el sistema 2 son condicionalmente independientes en $\mathcal{I}$, es decir

\begin{eqnarray}
\prob\left\{L_{1}\left(T^{*}\right),L_{2}\left(T^{*}\right),L_{3}\left(T^{*}\right),L_{4}\left(T^{*}\right)|T^{*}\in\mathcal{I}\right\}=\prod_{j=1}^{4}\prob\left\{L_{j}\left(T^{*}\right)\right\},
\end{eqnarray}

para $T^{*}\in\mathcal{I}$. 

%\newpage
























Sea la funci\'on generadora de momentos para $L_{i}$, el n\'umero de usuarios en la cola $Q_{i}\left(z\right)$ en cualquier momento, est\'a dada por el tiempo promedio de $z^{L_{i}\left(t\right)}$ sobre el ciclo regenerativo definido anteriormente. Entonces 



Es decir, es posible determinar las longitudes de las colas a cualquier tiempo $t$. Entonces, determinando el primer momento es posible ver que







%______________________________________________________________________
\section{Introduction}
%______________________________________________________________________

A cyclic polling system consists of multiple queues that are served by a single server in cyclic order. Users arrive at each queue according to independent processes, which also are independent of the service times. The server attends each queue according to a service policy previously established. The most commonly service policies studied are the exhaustive, gated and the k-limited. The exhaustive policy consists in attending all users until the queue is emptied. When the server finishes, it moves to the next queue incurring in a switchover time that is an independent and identically distributed random variable. An exhaustive analysis have been made in this subject. For an overview of the literature on polling systems, their applications and standard results we refer to surveys such as: \cite{Boxma, Kleinrock, LevySidi, Semenova, TakagiI, Takagi}. 

Bos and Boon \cite{BosBoon} published a report where they studied a Network of Polling Systems applied to a traffic problem, there they analyzed a network of intersections and followed a path in it. Their objective was to predict if the costumers can pass through the network in a finite time or not. The buffer occupancy method was used in this analysis and simulation techniques were also used to verify the results. It is important to remark that the heavy traffic case was studied in this report, as well as the cyclic case was not considered.

In this work, we study a Network of Cyclic Polling Systems (NCPS) that consists of two cyclic polling systems, each of them conformed by two queues attended by a single server. We apply the buffer occupancy method described by Kleinrock and Takagi \cite{TakagiI}. This method is based on the use of the Probability Generating Function (PGF) of the joint distribution function of the queues lengths at the moment the server starts a visit period in each of the queues that conform the system.

We present a theorem that guarantees the stability for the NCPS under specific conditions, also we obtain explicit expressions for the queue lengths at the moment the server arrives. With this results we obtain the queue lengths of the NCPS at any time for the servers.

We believe these results can be generalized for the continuous case and from the point of view of applications, the results are useful because they allow us to obtain analytical expressions for the performance measures, and also give us the keys to determine waiting times and queue lengths for any time during the operation of the network. Initially our main goal was studying the system of public transportation, which can be seen as a network consisting of several cyclic polling systems.

%_________________________________________________________________________
%
\section{Construcci\'on del Modelo e Hip\'otesis}
%_________________________________________________________________________
%

Consider a Network consisting of two cyclic polling systems with two queues each, $Q_{1}, Q_{2}$ for the first system and $\hat{Q}_{1},\hat{Q}_{2}$ for the second one, each of them with infinite-sized buffer. In each system a single server visits the queues in cyclic order, where it applies the exhaustive policy, i.e., when the server polls a queue, it serves all the customers present until the queue becomes empty. This case is illustrated in \texttt{Figure 1}. 

The second system's users at queue 2, can moves to the first system after being attended, also we assume that the network is open; that is, all customers eventually leave the network. As usually in polling systems theory we assume the arrivals in each queue are Poisson processes from with independent identical distributed (i.i.d.) inter arrival exponential times. The service times are exponential independent and identically distributed random variables. Finally upon completion of a visit at any queue, the servers incurs in a random switchover time according to an arbitrary distribution. We define a cycle to be the time interval between two consecutive polling instants, the time period in a cycle during which the server is attending a queue is called a service period. We are considering the case where the server visit the queues in cyclic order.

Time is slotted with slot size equal to the service time of a fixed costumer, we call the time interval $\left[t,t+1\right]$ the $t$-th slot. The arrival processes are denoted by $X_{1}\left(t\right),X_{2}\left(t\right)$ for the first system and $\hat{X}_{1}\left(t\right)$, $\hat{X}_{2}\left(t\right)$ for the second, the arrival rate at $Q_{i}$ and $\hat{Q}_{i}$ is denoted by $\mu_{i}$ and $\hat{\mu}_{i}$ respectively, with the condition $\mu_{i}<1$ and $\hat{\mu}_{i}<1$. The second system's users pass to the first one according to a process $Y_{2}$, with arrival rate $\tilde{\mu}_{2}$. 

Let's denote by $\tau_{i}$ the polling instant at queue $Q_{1}$ and by $\overline{\tau}_{i}$ the instant when the servers leaves to queue and starts a switchover time. Like the rest of the random variables the swithcover period is an i.i.d random variable $R_{i}$ with general distribution. 


To determine the length of the queues, i.e., the number of users in the queue at the moment the server arrives we define the process $L_{i}$ and $\hat{L}_{i}$ for the first and second system, respectively, in the sequel we use the buffer occupancy method to obtain the generating function, first and second moments of queue size distributions at polling instants. At each of the queues in the network the number of users is the number of users at the time the server arrives plus the numbers of users from the other system. 


In order to obtain the joint probability generating function (PGF) for the number or users residing in queue $i$ when the queue is polled in the NCPS, we define for each of the arrival processes $X_{1},X_{2},\hat{X}_{1},\hat{X}_{2},Y_{2}$, and $\tilde{X}_{2}$ with $\tilde{X}_{2}=X_{2}+Y_{2}$, their PGF

\begin{eqnarray*}
\begin{array}{cc}
P_{i}\left(z_{i}\right)=\esp\left[z_{i}^{X_{i}\left(t\right)}\right],&
\hat{P}_{i}\left(w_{i}\right)=\esp\left[w_{i}^{\hat{X}_{i}\left(t\right)}\right]
\end{array}
\end{eqnarray*}
for $i=1,2$, and
\begin{eqnarray*}
\begin{array}{cc}
\check{P}_{2}\left(z_{2}\right)=\esp\left[z_{2}^{Y_{2}\left(t\right)}\right],& \tilde{P}_{2}\left(z_{2}\right)=\esp\left[z_{2}^{\tilde{X}_{2}\left(t\right)}
\right],
\end{array}
\end{eqnarray*}

for $i=1,2$, and
\begin{eqnarray*} 
\begin{array}{cc}
\check{\mu}_{2}=\esp\left[Y_{2}\left(t\right)\right]=\check{P}_{2}^{(1)}
\left(1\right),&
\tilde{\mu}_{2}=\esp\left[\tilde{X}_{2}\left(t\right)\right]
=\tilde{P}_{2}^{(1)}\left(1\right).
\end{array}
\end{eqnarray*} The PGF For the service time is defined by:

\begin{eqnarray*}
\begin{array}{cc}
S_{i}\left(z_{i}\right)=\esp\left[z_{i}^{\overline{\tau}_{i}-\tau_{i}}
\right], &
\hat{S}_{i}\left(w_{i}\right)=\esp\left[w_{i}^{\overline{\zeta}_{i}-\zeta_{i}}\right]
\end{array}
\end{eqnarray*} with first moment 
\begin{eqnarray*}
\begin{array}{cc}
s_{i}=\esp\left[\overline{\tau}_{i}-\tau_{i}\right],&\hat{s}_{i}=\esp\left[\overline{\zeta}_{i}-\zeta_{i}\right]
\end{array}
\end{eqnarray*} for $i=1,2$. In a similar manner the PGF for the switchover time of the server from the moment it ends to attend a queue, to the time of arrival to the next queue is given by 
\begin{eqnarray*}
\begin{array}{cc}
R_{i}\left(z_{i}\right)=\esp\left[z_{1}^{\tau_{i+1}-\overline{\tau}_{i}}\right],&
\hat{R}_{i}\left(w_{i}\right)=\esp\left[w_{i}^{\zeta_{i+1}-\overline{\zeta}_{i}}\right]
\end{array}
\end{eqnarray*} with first moment 

\begin{eqnarray*}
\begin{array}{cc}
r_{i}=\esp\left[\tau_{i+1}-\overline{\tau}_{i}\right],&
\hat{r}_{i}=\esp\left[\zeta_{i+1}-\overline{\zeta}_{i}\right]
\end{array}
\end{eqnarray*} for $i=1,2$. The number of users in the queue at times $\overline{\tau}_{1},\overline{\tau}_{2}, \overline{\zeta}_{1},\overline{\zeta}_{2}$, it's zero, i.e.,
 $L_{i}\left(\overline{\tau_{i}}\right)=0,$ and $\hat{L}_{i}\left(\overline{\zeta_{i}}\right)=0$ for $i=1,2$. Then the number of users in the queue of the second system at the moment the server ends attending in the queue is given by the number of users present at the moment it arrives plus the number of arrivals during the service time, i.e.,
$$\hat{L}_{i}\left(\overline{\tau}_{j}\right)=\hat{L}_{i}\left(\tau_{j}\right)+\hat{X}_{i}\left(\overline{\tau}_{j}-\tau_{j}\right),$$
for $i,j=1,2$, meanwhile for the first system : $$L_{1}\left(\overline{\tau}_{j}\right)=L_{1}\left(\tau_{j}\right)+X_{1}\left(\overline{\tau}_{j}-\tau_{j}\right).$$ Specifically for the second queue of the first system we need to consider the users of transfer becoming from the second queue in the second system while the server its in the other queue attending, it means that this users have been already attended by the server before they can go to the first queue:

\begin{equation}\label{Eq.UsuariosTotalesZ2}
L_{2}\left(\overline{\tau}_{1}\right)=L_{2}\left(\tau_{1}\right)+X_{2}\left(\overline{\tau}_{1}-\tau_{1}\right)+Y_{2}\left(\overline{\tau}_{1}-\tau_{1}\right).
\end{equation}

As is know, the gambler's ruin problem can be used to model the server's busy period in a cyclic polling system, so let $\tilde{L}_{0}\geq0$ be the number of users present at the moment the server arrive to start attending, also let $T$ be the time the server need to attend the users in the queue starting with $\tilde{L}_{0}$ users. Suppose the gambler has two independent and simultaneous moves, such events are independent and identical to each other for each realization. The gain on the $n$-th game is $\tilde{\mathsf{X}}_{n}=\mathsf{X}_{n}+\mathsf{Y}_{n}$ units from which is substracted a playing fee of 1 unit for each move. His PGF is given by $F\left(z\right)=\esp\left[z^{\tilde{L}_{0}}\right]$, futhermore
%$\tilde{\mathrm{X}}$, $\tilde{\mathit{X}}$, $\tilde{\mathcal{X}}$, $\tilde{\mathfrak{X}}$,$\tilde{\mathbb{X}}$,$\tilde{\mathtt{X}}$,$\tilde{\mathsf{X}}$,

$$\tilde{P}\left(z\right)=\esp\left[z^{\tilde{\mathsf{X}}_{n}}\right]=\esp\left[z^{\mathsf{X}_{n}+\mathsf{X}_{n}}\right]=\esp\left[z^{\mathsf{X}_{n}}z^{\mathsf{X}_{n}}\right]=\esp\left[z^{\mathsf{X}_{n}}\right]\esp\left[z^{\mathsf{X}_{n}}\right]=P\left(z\right)\check{P}\left(z\right),$$ with $\tilde{\mu}=\esp\left[\tilde{\mathsf{X}}_{n}\right]=\tilde{P}\left[z\right]<1$. If  $\tilde{L}_{n}$ denotes the capital remaining after the $n$-th game, then $\tilde{L}_{n}=\tilde{L}_{0}+\tilde{\mathsf{X}}_{1}+\tilde{\mathsf{X}}_{2}+\cdots+\tilde{\mathsf{X}}_{n}-2n$. The result that relates the gambler's ruin problem with the busy period of the server it's a generalization of the result given in Takagi \cite{Takagi} chapter 3.

\begin{Prop}
Let's $G_{n}\left(z\right)$ and $G\left(z,w\right)$ defined as in 
(\ref{Eq.3.16.b.2S}), then $G_{n}\left(z\right)=\frac{1}{z}\left[G_{n-1}\left(z\right)-G_{n-1}\left(0\right)\right]\tilde{P}\left(z\right)$. Futhermore $G\left(z,w\right)=\frac{zF\left(z\right)-wP\left(z\right)G\left(0,w\right)}{z-wR\left(z\right)}$, with a unique pole in the unit circle, also the pole is of the form $z=\theta\left(w\right)$ and satisfies 
\begin{multicols}{3}
\begin{itemize}
\item[i)]$\tilde{\theta}\left(1\right)=1$,

\item[ii)] $\tilde{\theta}^{(1)}\left(1\right)=\frac{1}{1-\tilde{\mu}}$,

\item[iii)]
$\tilde{\theta}^{(2)}\left(1\right)=\frac{\tilde{\mu}}{\left(1-\tilde{\mu}\right)^{2}}+\frac{\tilde{\sigma}}{\left(1-\tilde{\mu}\right)^{3}}$.
\end{itemize}
\end{multicols}
\end{Prop}
%_________________________________________________________________________
%
\subsection{Description of the model: Probability Generating Function}
%_________________________________________________________________________
%

In order to model the network of cyclic polling system it's necessary to consider the users arrivals to each queue in one of the system, but on times the other system's server arrival, $\zeta_{i}$. In the case of the first system and the server arrives to a queue in the second one: $$F_{i,j}\left(z_{i};\zeta_{j}\right)=\esp\left[z_{i}^{L_{i}\left(\zeta_{j}\right)}\right]=
\sum_{k=0}^{\infty}\prob\left[L_{i}\left(\zeta_{j}\right)
=k\right]z_{i}^{k},$$ for $i,j=1,2$. Now consider the case of the queues in the second system and the server arrive to a queue in the first system $$\hat{F}_{i,j}\left(w_{i};\tau_{j}\right)=\esp\left[w_{i}^{\hat{L}_{i}\left(\tau_{j}\right)}\right] =\sum_{k=0}^{\infty}\prob\left[\hat{L}_{i}\left(\tau_{j}\right)
=k\right]w_{i}^{k},$$ for $i,j=1,2$. With the developed we can define the joint PGF for the second system:
$$\esp\left[w_{1}^{\hat{L}_{1}\left(\tau_{j}\right)}w_{2}^{\hat{L}_{2}\left(\tau_{j}\right)}\right]
=\esp\left[w_{1}^{\hat{L}_{1}\left(\tau_{j}\right)}\right]
\esp\left[w_{2}^{\hat{L}_{2}\left(\tau_{j}\right)}\right]=\hat{F}_{1,j}\left(w_{1};\tau_{j}\right)\hat{F}_{2,j}\left(w_{2};\tau_{j}\right)\equiv\hat{\mathbf{F}}_{j}\left(w_{1},w_{2};\tau_{j}\right).$$
%\end{eqnarray*}

In a similar manner we define the joint PGF for the first system, and the second system's server:
%\begin{eqnarray*}
$$\esp\left[z_{1}^{L_{1}\left(\zeta_{j}\right)}z_{2}^{L_{2}\left(\zeta_{j}\right)}\right]
=\esp\left[z_{1}^{L_{1}\left(\zeta_{j}\right)}\right]
\esp\left[z_{2}^{L_{2}\left(\zeta_{j}\right)}\right]=F_{1,j}\left(z_{1};\zeta_{j}\right)F_{2,j}\left(z_{2};\zeta_{j}\right)\equiv\mathbf{F}_{j}\left(z_{1},z_{2};\zeta_{j}\right).$$
%\end{eqnarray*}

Now we proceed to determine the joint PGF for the times that the server visit each queue in their corresponding system, i.e., $t=\left\{\tau_{1},\tau_{2},\zeta_{1},\zeta_{2}\right\}$:

\begin{eqnarray}\label{Eq.Conjuntas}
\begin{array}{l}
\mathbf{F}_{j}\left(z_{1},z_{2},w_{1},w_{2}\right)=\esp\left[\prod_{i=1}^{2}z_{i}^{L_{i}\left(\tau_{j}
\right)}\prod_{i=1}^{2}w_{i}^{\hat{L}_{i}\left(\tau_{j}\right)}\right],\\
\hat{\mathbf{F}}_{j}\left(z_{1},z_{2},w_{1},w_{2}\right)=\esp\left[\prod_{i=1}^{2}z_{i}^{L_{i}
\left(\zeta_{j}\right)}\prod_{i=1}^{2}w_{i}^{\hat{L}_{i}\left(\zeta_{j}\right)}\right],
\end{array}
\end{eqnarray} for $j=1,2$. Then with the purpose of find the number of users present in the netwotk when the server ends attending one of the queues in any of the systems we have that

\begin{eqnarray*}
&&\esp\left[z_{1}^{L_{1}\left(\overline{\tau}_{1}\right)}z_{2}^{L_{2}\left(\overline{\tau}_{1}\right)}w_{1}^{\hat{L}_{1}\left(\overline{\tau}_{1}\right)}w_{2}^{\hat{L}_{2}\left(\overline{\tau}_{1}\right)}\right]
=\esp\left[z_{2}^{L_{2}\left(\overline{\tau}_{1}\right)}w_{1}^{\hat{L}_{1}\left(\overline{\tau}_{1}
\right)}w_{2}^{\hat{L}_{2}\left(\overline{\tau}_{1}\right)}\right]\\
&=&\esp\left[z_{2}^{L_{2}\left(\tau_{1}\right)+X_{2}\left(\overline{\tau}_{1}-\tau_{1}\right)+Y_{2}\left(\overline{\tau}_{1}-\tau_{1}\right)}w_{1}^{\hat{L}_{1}\left(\tau_{1}\right)+\hat{X}_{1}\left(\overline{\tau}_{1}-\tau_{1}\right)}w_{2}^{\hat{L}_{2}\left(\tau_{1}\right)+\hat{X}_{2}\left(\overline{\tau}_{1}-\tau_{1}\right)}\right]
\end{eqnarray*}

using the equation (\ref{Eq.UsuariosTotalesZ2}) we have


\begin{eqnarray*}
&=&\esp\left[z_{2}^{L_{2}\left(\tau_{1}\right)}z_{2}^{X_{2}\left(\overline{\tau}_{1}-\tau_{1}\right)}z_{2}^{Y_{2}\left(\overline{\tau}_{1}-\tau_{1}\right)}w_{1}^{\hat{L}_{1}\left(\tau_{1}\right)}w_{1}^{\hat{X}_{1}\left(\overline{\tau}_{1}-\tau_{1}\right)}w_{2}^{\hat{L}_{2}\left(\tau_{1}\right)}w_{2}^{\hat{X}_{2}\left(\overline{\tau}_{1}-\tau_{1}\right)}\right]\\
&=&\esp\left[z_{2}^{L_{2}\left(\tau_{1}\right)}\left\{w_{1}^{\hat{L}_{1}\left(\tau_{1}\right)}w_{2}^{\hat{L}_{2}\left(\tau_{1}\right)}\right\}\left\{z_{2}^{X_{2}\left(\overline{\tau}_{1}-\tau_{1}\right)}
z_{2}^{Y_{2}\left(\overline{\tau}_{1}-\tau_{1}\right)}w_{1}^{\hat{X}_{1}\left(\overline{\tau}_{1}-\tau_{1}\right)}w_{2}^{\hat{X}_{2}\left(\overline{\tau}_{1}-\tau_{1}\right)}\right\}\right]
\end{eqnarray*}

applying the fact that the arrivals processes in the queues in each systems are independent:

$$=\esp\left[z_{2}^{L_{2}\left(\tau_{1}\right)}\left\{z_{2}^{X_{2}\left(\overline{\tau}_{1}-\tau_{1}\right)}z_{2}^{Y_{2}\left(\overline{\tau}_{1}-
\tau_{1}\right)}w_{1}^{\hat{X}_{1}\left(\overline{\tau}_{1}-\tau_{1}\right)}w_{2}^{\hat{X}_{2}\left(\overline{\tau}_{1}-\tau_{1}\right)}\right\}\right]
\esp\left[w_{1}^{\hat{L}_{1}\left(\tau_{1}\right)}w_{2}^{\hat{L}_{2}\left(\tau_{1}\right)}\right]$$ given that the arrival processes in the queues are independent, it's possible to separate the expectation for the arrival processes in $Q_{1}$ and $Q_{2}$ at time $\tau_{1}$, which is the time the server visits $Q_{1}$. Considering
$\tilde{X}_{2}\left(z_{2}\right)=X_{2}\left(z_{2}\right)+Y_{2}\left(z_{2}\right)$ we have


\begin{eqnarray*}
\begin{array}{l}
=\esp\left[z_{2}^{L_{2}\left(\tau_{1}\right)}\left\{z_{2}^{\tilde{X}_{2}\left(\overline{\tau}_{1}-\tau_{1}\right)}w_{1}^{\hat{X}_{1}\left(\overline{\tau}_{1}
-\tau_{1}\right)}
w_{2}^{\hat{X}_{2}\left(\overline{\tau}_{1}-\tau_{1}\right)}\right\}\right]\esp\left[w_{1}^{\hat{L}_{1}\left(\tau_{1}\right)}
w_{2}^{\hat{L}_{2}\left(\tau_{1}\right)}\right]\\
=\esp\left[z_{2}^{L_{2}\left(\tau_{1}\right)}\left\{\tilde{P}_{2}\left(z_{2}\right)
^{\overline{\tau}_{1}-\tau_{1}}\hat{P}_{1}\left(w_{1}\right)^{\overline{\tau}_{1}-
\tau_{1}}\hat{P}_{2}\left(w_{2}\right)^{\overline{\tau}_{1}-\tau_{1}}\right\}\right]
\esp\left[w_{1}^{\hat{L}_{1}\left(\tau_{1}\right)}w_{2}^{\hat{L}_{2}\left(\tau_{1}\right)}\right]\\
=\esp\left[z_{2}^{L_{2}\left(\tau_{1}\right)}\left\{\tilde{P}_{2}\left(z_{2}\right)\hat{P}_{1}\left(w_{1}\right)\hat{P}_{2}\left(w_{2}\right)\right\}^{\overline{\tau}_{1}-\tau_{1}}\right]\esp\left[w_{1}^{\hat{L}_{1}\left(\tau_{1}\right)}w_{2}^{\hat{L}_{2}\left(\tau_{1}\right)}\right]\\
=\esp\left[z_{2}^{L_{2}\left(\tau_{1}\right)}\theta_{1}\left(\tilde{P}_{2}\left(z_{2}\right)\hat{P}_{1}\left(w_{1}\right)\hat{P}_{2}\left(w_{2}\right)\right)
^{L_{1}\left(\tau_{1}\right)}\right]\esp\left[w_{1}^{\hat{L}_{1}\left(\tau_{1}\right)}w_{2}^{\hat{L}_{2}\left(\tau_{1}\right)}\right]\\
=F_{1}\left(\theta_{1}\left(\tilde{P}_{2}\left(z_{2}\right)\hat{P}_{1}\left(w_{1}\right)\hat{P}_{2}\left(w_{2}\right)\right),z_{2}\right)\cdot
\hat{F}_{1}\left(w_{1},w_{2};\tau_{1}\right)\\
\equiv \mathbf{F}_{1}\left(\theta_{1}\left(\tilde{P}_{2}\left(z_{2}\right)\hat{P}_{1}\left(w_{1}\right)\hat{P}_{2}\left(w_{2}\right)\right),z_{2},w_{1},w_{2}\right).
\end{array}
\end{eqnarray*}

The last equalities  are true because the number of arrivals to $\hat{Q}_{2}$ 
during the time interval $\left[\tau_{1},\overline{\tau}_{1}\right]$ still haven't been attended by the server in the system 2, then the users can't pass to the first system through the queue $Q_{2}$. Therefore the number of users switching from $\hat{Q}_{2}$ to $Q_{2}$ during the time interval $\left[\tau_{1},\overline{\tau}_{1}\right]$ depends on the policy of transfer between the two systems, according to the last section
%{\small{
\begin{eqnarray*}\label{Eq.Fs}
\begin{array}{l}
\esp\left[z_{1}^{L_{1}\left(\overline{\tau}_{1}\right)}z_{2}^{L_{2}\left(\overline{\tau}_{1}
\right)}w_{1}^{\hat{L}_{1}\left(\overline{\tau}_{1}\right)}w_{2}^{\hat{L}_{2}\left(
\overline{\tau}_{1}\right)}\right]
=\mathbf{F}_{1}\left(\theta_{1}\left(\tilde{P}_{2}\left(z_{2}\right)
\hat{P}_{1}\left(w_{1}\right)\hat{P}_{2}\left(w_{2}\right)\right),z_{2},w_{1},w_{2}\right)\\
\equiv F_{1}\left(\theta_{1}\left(\tilde{P}_{2}\left(z_{2}\right)\hat{P}_{1}\left(w_{1}\right)\hat{P}_{2}\left(w_{2}\right)\right),z_{2}\right)\hat{F}_{1}\left(w_{1},w_{2};\tau_{1}\right).
\end{array}
\end{eqnarray*}%}}

Using similar reasoning for the rest of the server's arrival times we have that

\begin{eqnarray*}
\esp\left[z_{1}^{L_{1}\left(\overline{\tau}_{2}\right)}z_{2}^{L_{2}\left(\overline{\tau}_{2}\right)}w_{1}^{\hat{L}_{1}\left(\overline{\tau}_{2}\right)}w_{2}^{\hat{L}_{2}\left(\overline{\tau}_{2}\right)}\right]&=&F_{2}\left(z_{1},\tilde{\theta}_{2}\left(P_{1}\left(z_{1}\right)\hat{P}_{1}\left(w_{1}\right)\hat{P}_{2}\left(w_{2}\right)\right)\right)
\hat{F}_{2}\left(w_{1},w_{2};\tau_{2}\right)\\
&\equiv& \mathbf{F}_{2}\left(z_{1},\tilde{\theta}_{2}\left(P_{1}\left(z_{1}\right)\hat{P}_{1}\left(w_{1}\right)\hat{P}_{2}\left(w_{2}\right)\right),w_{1},w_{2}\right),\\
\esp\left[z_{1}^{L_{1}\left(\overline{\zeta}_{1}\right)}z_{2}^{L_{2}\left(\overline{\zeta}_{1}
\right)}w_{1}^{\hat{L}_{1}\left(\overline{\zeta}_{1}\right)}w_{2}^{\hat{L}_{2}\left(\overline{\zeta}_{1}\right)}\right]
&=&F_{1}\left(z_{1},z_{2};\zeta_{1}\right)\hat{F}_{1}\left(\hat{\theta}_{1}\left(P_{1}\left(z_{1}\right)\tilde{P}_{2}\left(z_{2}\right)\hat{P}_{2}\left(w_{2}\right)\right),w_{2}\right)\\
&\equiv&\hat{\mathbf{F}}_{1}\left(z_{1},z_{2},\hat{\theta}_{1}\left(P_{1}\left(z_{1}\right)\tilde{P}_{2}\left(z_{2}\right)\hat{P}_{2}\left(w_{2}\right)\right),w_{2}\right),\\
\esp\left[z_{1}^{L_{1}\left(\overline{\zeta}_{2}\right)}z_{2}^{L_{2}\left(\overline{\zeta}_{2}\right)}w_{1}^{\hat{L}_{1}\left(\overline{\zeta}_{2}\right)}w_{2}^{\hat{L}_{2}\left(\overline{\zeta}_{2}\right)}\right]
&=&F_{2}\left(z_{1},z_{2};\zeta_{2}\right)\hat{F}_{2}\left(w_{1},\hat{\theta}_{2}\left(P_{1}\left(z_{1}\right)\tilde{P}_{2}\left(z_{2}\right)\hat{P}_{1}\left(w_{1}\right)\right)\right)\\
&\equiv&\hat{\mathbf{F}}_{2}\left(z_{1},z_{2},w_{1},\hat{\theta}_{2}\left(P_{1}\left(z_{1}\right)\tilde{P}_{2}\left(z_{2}\right)\hat{P}_{1}\left(w_{1}\right)\right)\right).
\end{eqnarray*}

Now we are in conditions to obtain the recursive equations that model the NCPS. We need to consider the switchover times that the server need to translate from one queue to another and, the number or user presents in the system at the time the server leaves to the queue to start attending the next. Thus far developed, we can find that for the NCPS:

\begin{eqnarray}\label{Recursive.Equations.First.Casse}
\begin{array}{r}
\mathbf{F}_{2}\left(z_{1},z_{2},w_{1},w_{2}\right)=R_{1}\left(P_{1}\left(z_{1}\right)\tilde{P}_{2}
\left(z_{2}\right)\prod_{i=1}^{2}
\hat{P}_{i}\left(w_{i}\right)\right)\mathbf{F}_{1}\left(\theta_{1}\left(\tilde{P}_{2}\left(z_{2}
\right)\hat{P}_{1}\left(w_{1}\right)\hat{P}_{2}\left(w_{2}\right)\right),z_{2},w_{1},w_{2}\right),\\
\mathbf{F}_{1}\left(z_{1},z_{2},w_{1},w_{2}\right)=R_{2}\left(P_{1}\left(z_{1}\right)\tilde{P}_{2}
\left(z_{2}\right)\prod_{i=1}^{2}
\hat{P}_{i}\left(w_{i}\right)\right)\mathbf{F}_{2}\left(z_{1},\tilde{\theta}_{2}\left(P_{1}\left(z_{1}\right)\hat{P}_{1}\left(w_{1}\right)\hat{P}_{2}\left(w_{2}
\right)\right),w_{1},w_{2}\right),\\
\hat{\mathbf{F}}_{2}\left(z_{1},z_{2},w_{1},w_{2}\right)=\hat{R}_{1}\left(P_{1}\left(z_{1}\right)\tilde{P}_{2}\left(z_{2}\right)\prod_{i=1}^{2}
\hat{P}_{i}\left(w_{i}\right)\right)\hat{\mathbf{F}}_{1}\left(z_{1},z_{2},\hat{\theta}_{1}\left(P_{1}\left(z_{1}\right)\tilde{P}_{2}\left(z_{2}\right)\hat{P}_{2}\left(w_{2}
\right)\right),w_{2}\right),\\
\hat{\mathbf{F}}_{1}\left(z_{1},z_{2},w_{1},w_{2}\right)=\hat{R}_{2}\left(P_{1}\left(z_{1}\right)\tilde{P}_{2}\left(z_{2}\right)\prod_{i=1}^{2}
\hat{P}_{i}\left(w_{i}\right)\right)\hat{\mathbf{F}}_{2}\left(z_{1},z_{2},w_{1},\hat{\theta}_{2}\left(P_{1}\left(z_{1}\right)\tilde{P}_{2}\left(z_{2}\right)\hat{P}_{1}\left(w_{1}
\right)\right)\right).
\end{array}
\end{eqnarray}



%_____________________________________________________
%\subsubsection{Server Switchover times}
%_____________________________________________________
It's necessary to give an step ahead, considering the case illustrated in \texttt{Figure 2}, where just like before, the server's switchover times are given by the generals equations
$R_{i}\left(\mathbf{z,w}\right)=R_{i}\left(\tilde{P}_{1}\left(z_{1}\right)
\tilde{P}_{2}\left(z_{2}\right)\tilde{P}_{3}\left(z_{3}\right)
\tilde{P}_{4}\left(z_{4}\right)\right)$, with first order derivatives given by $D_{i}R_{i}=r_{i}\tilde{\mu}_{i}$, and second order partial derivatives $D_{j}D_{i}R_{k}=R_{k}^{(2)}\tilde{\mu}_{i}\tilde{\mu}_{j}+\indora_{i=j}r_{k}P_{i}^{(2)}+\indora_{i\neq j}r_{k}\tilde{\mu}_{i}\tilde{\mu}_{j}$ for any $i,j,k$. According to the equations given before and the queue lengths for the other system's server times, we can obtain general expressions

\begin{eqnarray}\label{Ec.Gral.Primer.Momento.Ind.Exh}
\begin{array}{ll}
D_{j}\mathbf{F}_{i}\left(z_{1},z_{2};\tau_{i+2}\right)=\indora_{j\leq2}F_{j,i+2}^{(1)},&
D_{j}\mathbf{F}_{i}\left(z_{3},z_{4};\tau_{i-2}\right)=\indora_{j\geq3}F_{j,i-2}^{(1)},
\end{array}
\end{eqnarray}

for $i,j=1,2,3,4$; with second order derivatives given by

\begin{eqnarray}\label{Ec.Gral.Segundo.Momento.Ind.Exh}
\begin{array}{l}
D_{j}D_{i}\mathbf{F}_{k}\left(z_{1},z_{2};\tau_{k+2}\right)=\indora_{i\geq3}\indora_{j=i}F_{i,k+2}^{(2)}+\indora_{i\geq 3}\indora_{j\neq i}F_{j,k-2}^{(1)}F_{i,k+2}^{(1)},\\
D_{j}D_{i}\mathbf{F}_{k}\left(z_{3},z_{4};\tau_{k-2}\right)=\indora_{i\geq3}\indora_{j=i}F_{i,k-2}^{(2)}+\indora_{i\geq 3}\indora_{j\neq i}F_{j,k-2}^{(1)}F_{i,k-2}^{(1)}.
\end{array}
\end{eqnarray}


 According with the developed at the moment, we can get the recursive equations which are of the following form

\begin{eqnarray}\label{General.System.Double.Transfer}
\begin{array}{l}
\mathbf{F}_{1}\left(z_{1},z_{2},z_{3},z_{4}\right)=R_{2}\left(\prod_{i=1}^{4}\tilde{P}_{i}\left(z_{i}
\right)\right)\mathbf{F}_{2}\left(z_{1},\tilde{\theta}_{2}\left(\tilde{P}_{1}\left(z_{1}\right)\tilde{P}_{3}\left(z_{3}\right)\tilde{P}_{4}
\left(z_{4}\right)\right),z_{3},z_{4}\right),\\
\mathbf{F}_{2}\left(z_{1},z_{2},z_{3},z_{4}\right)=R_{1}\left(\prod_{i=1}^{4}\tilde{P}_{i}\left(z_{i}
\right)\right)
\mathbf{F}_{1}\left(\tilde{\theta}_{1}\left(\tilde{P}_{2}\left(z_{2}\right)\tilde{P}_{3}\left(z_{3}
\right)\tilde{P}_{4}\left(z_{4}\right)\right),z_{2},z_{3},z_{4}\right),\\
\mathbf{F}_{3}\left(z_{1},z_{2},z_{3},z_{4}\right)=R_{4}\left(\prod_{i=1}^{4}\tilde{P}_{i}\left(z_{i}
\right)\right)\mathbf{F}_{4}\left(z_{1},z_{2},z_{3},\tilde{\theta}_{4}\left(\tilde{P}_{1}\left(z_{1}\right)\tilde{P}_{2}\left(z_{2}\right)\tilde{P}_{3}\left(z_{3}\right)
\right)\right),\\
\mathbf{F}_{4}\left(z_{1},z_{2},z_{3},z_{4}\right)=R_{3}\left(\prod_{i=1}^{4}\tilde{P}_{i}\left(z_{i}
\right)\right)
\mathbf{F}_{3}\left(z_{1},z_{2},\tilde{\theta}_{3}\left(\tilde{P}_{1}\left(z_{1}\right)\tilde{P}_{2}\left(z_{2}\right)\tilde{P}_{4}
\left(z_{4}\right)\right),z_{4}\right).
\end{array}
\end{eqnarray}
%_________________________________________________________________________
%
%\subsection{Hipotesis sobre las colas}
%_________________________________________________________________________
%


So we have the first theorem

\begin{Teo}
Suppose  $\tilde{\mu}=\tilde{\mu}_{1}+\tilde{\mu}_{2}<1$, $\hat{\mu}=\tilde{\mu}_{3}+\tilde{\mu}_{4}<1$, then the number of users in the queues conforming the network of cyclic polling system (\ref{General.System.Double.Transfer}), when the server visit a queue can be found solving the linear system given by equations (\ref{Ec.Primer.Orden.General.Impar}) and (\ref{Ec.Primer.Orden.General.Par}):

\begin{eqnarray}\label{Ec.Primer.Orden.General.Impar}
\begin{array}{l}
f_{j}\left(i\right)=r_{j+1}\tilde{\mu}_{i}
+\indora_{i\neq j+1}f_{j+1}\left(j+1\right)\frac{\tilde{\mu}_{i}}{1-\tilde{\mu}_{j+1}}
+\indora_{i=j}f_{j+1}\left(i\right)
+\indora_{j=1}\indora_{i\geq3}F_{i,j+1}^{(1)}
+\indora_{j=3}\indora_{i\leq2}F_{i,j+1}^{(1)}
\end{array}
\end{eqnarray}
$j=1,3$ and $i=1,2,3,4$.

\begin{eqnarray}\label{Ec.Primer.Orden.General.Par}
\begin{array}{l}
f_{j}\left(i\right)=r_{j-1}\tilde{\mu}_{i}
+\indora_{i\neq j-1}f_{j-1}\left(j-1\right)\frac{\tilde{\mu}_{i}}{1-\tilde{\mu}_{j-1}}
+\indora_{i=j}f_{j-1}\left(i\right)
+\indora_{j=2}\indora_{i\geq3}F_{i,j-1}^{(1)}
+\indora_{j=4}\indora_{i\leq2}F_{i,j-1}^{(1)}
\end{array}
\end{eqnarray}
$j=2,4$ and $i=1,2,3,4$, whose solutions are:
%{\footnotesize{


\begin{eqnarray}
\begin{array}{l}
f_{i}\left(j\right)=\left(\indora_{j=i-1}+\indora_{j=i+1}\right)r_{j}\tilde{\mu}_{j}+\indora_{i=j}\left(\indora_{i\leq2}\frac{r\tilde{\mu}_{i}\left(1-\tilde{\mu}_{i}\right)}{1-\tilde{\mu}}+\indora_{i\geq2}\frac{\hat{r}\tilde{\mu}_{i}\left(1-\tilde{\mu}_{i}\right)}{1-\hat{\mu}}\right)\\
+\indora_{i=1}\indora_{j\geq3}\left(\tilde{\mu}_{j}\left(r_{i+1}+\frac{r\tilde{\mu}_{i+1}}{1-\tilde{\mu}}\right)+F_{j,i+1}^{(1)}\right)
+\indora_{i=3}\indora_{j\geq3}\left(\tilde{\mu}_{j}\left(r_{i+1}+\frac{\hat{r}\tilde{\mu}_{i+1}}{1-\hat{\mu}}\right)+F_{j,i+1}^{(1)}\right)\\
+\indora_{i=2}\indora_{j\leq2}\left(\tilde{\mu}_{j}\left(r_{i-1}+\frac{r\tilde{\mu}_{i-1}}{1-\tilde{\mu}}\right)+F_{j,i-1}^{(1)}\right)
+\indora_{i=4}\indora_{j\leq2}\left(\tilde{\mu}_{j}\left(r_{i-1}+\frac{\hat{r}\tilde{\mu}_{i-1}}{1-\hat{\mu}}\right)+F_{j,i-1}^{(1)}\right).
\end{array}
\end{eqnarray}
\end{Teo}
%______________________________________________________________________

\begin{Teo}
For the system given in (\ref{General.System.Double.Transfer}) we have that the second moments are in their general form

%{\small{
\begin{eqnarray}\label{Eq.Gral.Second.Order.Exhaustive}
\begin{array}{r}
f_{1}\left(i,k\right)=D_{k}D_{i}\left(R_{2}+\mathbf{F}_{2}+\indora_{i\geq3}\mathbf{F}_{4}\right)
+D_{i}R_{2}D_{k}\left(\mathbf{F}_{2}+\indora_{k\geq3}\mathbf{F}_{4}\right)
+D_{i}F_{2}D_{k}\left(R_{2}+\indora_{k\geq3}\mathbf{F}_{4}\right)\\
+\indora_{i\geq3}D_{i}\mathbf{F}_{4}D_{k}\left(R_{2}+\mathbf{F}_{2}\right)\\
f_{2}\left(i,k\right)=D_{k}D_{i}\left(R_{1}+\mathbf{F}_{1}+\indora_{i\geq3}\mathbf{F}_{3}\right)+D_{i}R_{1}D_{k}\left(\mathbf{F}_{1}+\indora_{k\geq3}\mathbf{F}_{3}\right)+D_{i}\mathbf{F}_{1}D_{k}\left(R_{1}+\indora_{k\geq3}\mathbf{F}_{3}\right)\\
+\indora_{i\geq3}D_{i}\mathbf{F}_{3}D_{k}\left(R_{1}+\mathbf{F}_{1}\right)\\
f_{3}\left(i,k\right)=D_{k}D_{i}\left(R_{4}+\indora_{i\leq2}\mathbf{F}_{2}+\mathbf{F}_{4}\right)+D_{i}\tilde{R}_{4}D_{k}\left(\indora_{k\leq2}\mathbf{F}_{2}+\mathbf{F}_{4}\right)+D_{i}\mathbf{F}_{4}D_{k}\left(R_{4}+\indora_{k\leq2}\mathbf{F}_{2}\right)\\
+\indora_{i\leq2}D_{i}\mathbf{F}_{2}D_{k}\left(R_{4}+\mathbf{F}_{4}\right)\\
f_{4}\left(i,k\right)=D_{k}D_{i}\left(R_{3}+\indora_{i\leq2}\mathbf{F}_{1}+\mathbf{F}_{3}\right)+D_{i}R_{3}D_{k}\left(\indora_{k\leq2}\mathbf{F}_{1}+\mathbf{F}_{3}\right)+D_{i}\mathbf{F}_{3}D_{k}\left(R_{3}+\indora_{k\leq2}\mathbf{F}_{1}\right)\\
+\indora_{i\leq2}D_{i}\mathbf{F}_{1}D_{k}\left(R_{3}+\mathbf{F}_{3}\right)
\end{array}
\end{eqnarray}%}}

\end{Teo}


\begin{Coro}\label{Coro.Second.Order.Eqs}
Conforming the equations given in (\ref{Eq.Gral.Second.Order.Exhaustive}) the second order moments are obtained solving the linear systems given by  (\ref{System.Second.Order.Moments.uno}). These solutions are 

\begin{eqnarray}\label{Sol.System.Second.Order.Exhaustive}
\begin{array}{ll}
f_{1}\left(1,1\right)=b_{3},&
f_{2}\left(2,2\right)=\frac{b_{2}}{1-b_{1}},\\
f_{1}\left(1,3\right)=a_{4}\left(\frac{b_{2}}{1-b_{1}}\right)+a_{5}K_{12}+K_{3},&
f_{1}\left(1,4\right)=a_{6}\left(\frac{b_{2}}{1-b_{1}}\right)+a_{7}K_{12}+K_{4},\\
f_{1}\left(3,3\right)=a_{8}\left(\frac{b_{2}}{1-b_{1}}\right)+K_{8},&
f_{1}\left(3,4\right)=a_{9}\left(\frac{b_{2}}{1-b_{1}}\right)+K_{9},\\
f_{1}\left(4,4\right)=a_{10}\left(\frac{b_{2}}{1-b_{1}}\right)+a_{5}K_{12}+K_{10},&
f_{2}\left(2,3\right)=a_{14}b_{3}+a_{15}K_{2}+K_{16},\\
f_{2}\left(2,4\right)=a_{16}b_{3}+a_{17}K_{2}+K_{17},&
f_{2}\left(3,3\right)=a_{18}b_{3}+K_{18},\\
f_{2}\left(3,4\right)=a_{19}b_{3}+K_{19},&
f_{2}\left(4,4\right)=a_{20}b_{3}+K_{20},\\
f_{3}\left(3,3\right)=\frac{b_{5}}{1-b_{4}},&
f_{4}\left(4,4\right)=b_{6},\\
f_{3}\left(1,1\right)=a_{21}b_{6}+K_{21},&
f_{3}\left(1,2\right)=a_{22}b_{6}+K_{22},\\
f_{3}\left(1,3\right)=a_{23}b_{6}+a_{24}K_{39}+K_{23},&
f_{3}\left(2,2\right)=a_{25}b_{6}+K_{25},\\
f_{3}\left(2,3\right)=a_{26}b_{6}+a_{27}K_{39}+K_{26},&
f_{4}\left(1,1\right)=a_{31}\left(\frac{b_{5}}{1-b_{4}}\right)+K_{31},\\
f_{4}\left(1,2\right)=a_{32}\left(\frac{b_{5}}{1-b_{4}}\right)+K_{32},&
f_{4}\left(1,4\right)=a_{33}\left(\frac{b_{5}}{1-b_{4}}\right)+a_{34}K_{29}+K_{31},\\
f_{4}\left(2,2\right)=a_{35}\left(\frac{b_{5}}{1-b_{4}}\right)+K_{35},&
f_{4}\left(2,4\right)=a_{36}\left(\frac{b_{5}}{1-b_{4}}\right)+a_{37}K_{29}+K_{37}.
\end{array}
\end{eqnarray}

where
\begin{eqnarray*}
\begin{array}{lll}
N_{1}=a_{2}K_{12}+a_{3}K_{11}+K_{1},&
N_{2}=a_{12}K_{2}+a_{13}K_{5}+K_{15},&
b_{1}=a_{1}a_{11},\\
b_{2}=a_{11}N_{1}+N_{2},&
b_{3}=a_{1}\left(\frac{b_{2}}{1-b_{1}}\right)+N_{1},&
N_{3}=a_{29}K_{39}+a_{30}K_{38}+K_{28},\\
N_{4}=a_{39}K_{29}+a_{40}K_{30}+K_{40},&
b_{4}=a_{28}a_{38},&
b_{5}=a_{28}N_{4}+N_{3},\\
&b_{6}=a_{38}\left(\frac{b_{5}}{1-b_{4}}\right)+N_{4}.&
\end{array}
\end{eqnarray*}

\end{Coro}
The values for the $a_{i}$'s and $K_{i}$ can be found in \textit{Appendix B}. Finally 

\begin{Def}
Let $L_{i}^{*}$ be the number of users at queue $Q_{i}$ when it is polled, then
\begin{eqnarray}
\begin{array}{cc}
\esp\left[L_{i}^{*}\right]=f_{i}\left(i\right), &
Var\left[L_{i}^{*}\right]=f_{i}\left(i,i\right)+\esp\left[L_{i}^{*}\right]-\esp\left[L_{i}^{*}\right]^{2}.
\end{array}
\end{eqnarray}
\end{Def}

%_________________________________________________________________________
%
\subsection{Stability Analysis}
%_________________________________________________________________________
%

We are interested in determine the queue lengths at any time, not just when the server arrives to the queue to start attending according to the exhaustive policy. For this purpose we need to make assumptions over the processes involved in order to guarantee the stability of the Network.



First of all we are going to assume the arrival processes are Poisson, the service time are exponential. In 1973 Disney \cite{Disney} prove that the only stationary system $M/G/1/L$, with renewal departure process are the $M/M/1$ y $M/D/1/1$ systems, also this implies that the output process is Poisson with the same rate of the arrival process. The switchover times has no particular distribution, the only condition they have to satisfy is the first moment finite.

Sigman, Thorison and Wolff \cite{Sigman2} proved that if there is a first regeneration time then exist a non decreasing infinite sequence of regeneration times. With this in consideration we have the following theorem 


\begin{Teo}\label{First.Regeneration.Time.Theorem}
Given a Network of Cyclic Polling Systems (NCPS) conformed by two cyclic polling systems, each of them with $M/M/1$ queues. Both systems are related by users transfer between the queues $Q_{1},Q_{3}$ and $Q_{2},Q_{4}$. Suppose $\tilde{\mu},\hat{\mu}<1$. Let's define the following events for the arrival processes at time $t$: $A_{j}\left(t\right)=\left\{0 \textrm{ arrivals on }Q_{j}\textrm{ at time }t\right\}$, for some $t\geq0$ and queue $Q_{j}$ in the NCPS for $j=1,2,3,4$. Then there exist an non empty interval $I$ such that for $T^{*}\in I$ the $\prob\left\{A_{1}\left(T^{*}\right),A_{2}\left(Tt^{*}\right),
A_{3}\left(T^{*}\right),A_{4}\left(T^{*}\right)|T^{*}\in I\right\}>0$ is satisfied.

\end{Teo}
\begin{proof}

Without of loss of generality we are going to consider as base of the analysis the queue $Q_{1}$ from the first system.

Let's $n\geq1$ cycle for the first system, so let's be $\overline{\tau}_{1}\left(n\right)$ time the server ends attending en queue $Q_{1}$, it means 
$L_{j}\left(\overline{\tau}_{1}\left(n\right)\right)=0$. The server incurrs in a switchover time to traslate to the other queue, which is a random variable whose realitation is $r_{1}\left(n\right)>0$, then we have that $\tau_{2}\left(n\right)=\overline{\tau}_{1}\left(n\right)+r_{1}\left(n\right)$.

Let's be $I_{1}\equiv\left[\overline{\tau}_{1}\left(n\right),\tau_{2}\left(n\right)\right]$ the intreval with length $\xi_{1}=r_{1}\left(n\right)$. Given that the arrival times are exponentials with rate $\tilde{\mu}_{1}=\mu_{1}+\hat{\mu}_{1}$ and the transfer users process from queue $Q_{3}$ are exponentials with rate $\hat{\mu}_{1}$, we have that the event $A_{1}\left(t\right)$ has probability given by 

\begin{equation}
\prob\left\{A_{1}\left(t\right)|t\in I_{1}\left(n\right)\right\}=e^{-\tilde{\mu}_{1}\xi_{1}\left(n\right)}.
\end{equation} 

In the other side, for the queue $Q_{2}$, the time 
$\overline{\tau}_{2}\left(n-1\right)$ is such that 
$L_{2}\left(\overline{\tau}_{2}\left(n-1\right)\right)=0$, it means, it's the time when the queue is emptied by the server en the previous cycle. So we have a second time interval $I_{2}\equiv\left[\overline{\tau}_{2}\left(n-1\right),\tau_{2}\left(n\right)\right]$ so the event $A_{2}\left(t\right)$ has probability

\begin{equation}
\prob\left\{A_{2}\left(t\right)|t\in I_{2}\left(n\right)\right\}=e^{-\tilde{\mu}_{2}\xi_{2}\left(n\right)},
\end{equation} 
with length 
$\xi_{2}\left(n\right)=\tau_{2}\left(n\right)-\overline{\tau}_{2}\left(n-1\right)$. Given the time intervals construction we have that $I_{1}\left(n\right)\subset I_{2}\left(n\right)$, therefore  $\xi_{1}\left(n\right)\leq\xi_{2}\left(n\right)$ so $-\xi_{1}\left(n\right)\geq-\xi_{2}\left(n\right)$ then $-\tilde{\mu}_{2}\xi_{1}\left(n\right)\geq-\tilde{\mu}_{2}\xi_{2}\left(n\right)$ and finally $e^{-\tilde{\mu}_{2}\xi_{1}\left(n\right)}\geq e^{-\tilde{\mu}_{2}\xi_{2}\left(n\right)}$, then

\begin{equation}
\prob\left\{A_{2}\left(t\right)|t\in I_{1}\left(n\right)\right\}\geq
\prob\left\{A_{2}\left(t\right)|t\in I_{2}\left(n\right)\right\}.
\end{equation}

Now we can determine the joint conditional probability on the interval $I_{1}\left(n\right)$
\begin{eqnarray*}
\prob\left\{A_{1}\left(t\right),A_{2}\left(t\right)|t\in I_{1}\left(n\right)\right\}&=&
\prob\left\{A_{1}\left(t\right)|t\in I_{1}\left(n\right)\right\}
\prob\left\{A_{2}\left(t\right)|t\in I_{1}\left(n\right)\right\}\\
&\geq&
\prob\left\{A_{1}\left(t\right)|t\in I_{1}\left(n\right)\right\}
\prob\left\{A_{2}\left(t\right)|t\in I_{2}\left(n\right)\right\}\\
&=&e^{-\tilde{\mu}_{1}\xi_{1}\left(n\right)}e^{-\tilde{\mu}_{2}\xi_{2}\left(n\right)}
=e^{-\left[\tilde{\mu}_{1}\xi_{1}\left(n\right)+\tilde{\mu}_{2}\xi_{2}\left(n\right)\right]}.
\end{eqnarray*}

It means 
\begin{equation}
\prob\left\{A_{1}\left(t\right),A_{2}\left(t\right)|t\in I_{1}\left(n\right)\right\}
=e^{-\left[\tilde{\mu}_{1}\xi_{1}\left(n\right)+\tilde{\mu}_{2}\xi_{2}
\left(n\right)\right]}>0.
\end{equation}

With respect the relation between both systems, there exists some $m\geq1$ such that $\tau_{3}\left(m\right)<\tau_{2}\left(n\right)<\tau_{4}\left(m\right)$ therefore we have the following cases for $\tau_{2}\left(n\right)$:

\begin{multicols}{2}
\begin{itemize}
\item[a)] $\tau_{3}\left(m\right)<\tau_{2}\left(n\right)<\overline{\tau}_{3}\left(m\right)$,

\item[b)] $\overline{\tau}_{3}\left(m\right)<\tau_{2}\left(n\right)
<\tau_{4}\left(m\right)$,

\item[c)] $\tau_{4}\left(m\right)<\tau_{2}\left(n\right)<
\overline{\tau}_{4}\left(m\right)$,

\item[d)] $\overline{\tau}_{4}\left(m\right)<\tau_{2}\left(n\right)
<\tau_{3}\left(m+1\right)$.
\end{itemize}
\end{multicols}

First consider the time interval $I_{3}\left(m\right)\equiv\left[\tau_{3}\left(m\right),\overline{\tau}_{3}\left(m\right)\right]$ such that $\tau_{2}\left(n\right)\in I_{3}\left(m\right)$, with length $\xi_{3}\equiv\overline{\tau}_{3}\left(m\right)-\tau_{3}\left(m\right)$, then we have for the queue $Q_{3}$
\begin{equation}
\prob\left\{A_{3}\left(t\right)|t\in I_{3}\left(m\right)\right\}=e^{-\tilde{\mu}_{3}\xi_{3}\left(m\right)}.
\end{equation} 

whereas for $Q_{4}$ lets consider the time interval $I_{4}\left(m\right)\equiv\left[\tau_{4}\left(m-1\right),\overline{\tau}_{3}\left(m\right)\right]$, then we have that $I_{3}\left(m\right)\subset I_{4}\left(m\right)$, therefore in a similar manner that we have done for $Q_{1}$ and $Q_{2}$ we obtain


\begin{equation}
\prob\left\{A_{4}\left(t\right)|t\in I_{3}\left(m\right)\right\}\geq
\prob\left\{A_{4}\left(t\right)|t\in I_{4}\left(m\right)\right\}
\end{equation}

and

\begin{equation}
\prob\left\{A_{3}\left(t\right),A_{4}\left(t\right)|t\in I_{3}\left(m\right)\right\}\geq
e^{-\left(\tilde{\mu}_{3}\xi_{3}\left(m\right)+\tilde{\mu}_{4}\xi_{4}\left(m\right)\right)}>0.
\end{equation}


For the rest of the cases the demonstration is similar. It means we always can find a time interval where we can guarantee there is no arrivals to the queues in each system with positive probability.  


By construction we have that $I\left(n,m\right)\equiv I_{1}\left(n\right)\cap I_{3}\left(m\right)\neq\emptyset$, then in particular we have the following contentions $I\left(n,m\right)\subseteq I_{1}\left(n\right)$ and $I\left(n,m\right)\subseteq I_{3}\left(m\right)$, therefore if we define $\xi\left(n,m\right)$ as the length of the interval $I\left(n,m\right)$ we have $\xi\left(n,m\right)\leq\xi_{1}\left(n\right)$, $\xi\left(n,m\right)\leq\xi_{3}\left(m\right)$, then $-\xi\left(n,m\right)\geq-\xi_{1}\left(n\right)$ and finally $-\xi\left(n,m\right)\leq-\xi_{3}\left(m\right)$ therefore we have the following
\begin{multicols}{2}
\begin{enumerate}
\item $-\tilde{\mu}_{1}\xi_{n,m}\geq-\tilde{\mu}_{1}\xi_{1}\left(n\right)$,
\item $-\tilde{\mu}_{2}\xi_{n,m}\geq-\tilde{\mu}_{2}\xi_{1}\left(n\right)
\geq-\tilde{\mu}_{2}\xi_{2}\left(n\right)$,
\item $-\tilde{\mu}_{3}\xi_{n,m}\geq-\tilde{\mu}_{3}\xi_{3}\left(m\right)$,
\item $-\tilde{\mu}_{4}\xi_{n,m}\geq-\tilde{\mu}_{4}\xi_{3}\left(m\right)
\geq-\tilde{\mu}_{4}\xi_{4}\left(m\right).$
\end{enumerate}
\end{multicols}

Let's $T^{*}\in I\left(n,m\right)$, then given that in particular $T^{*}\in I_{1}\left(n\right)$, there is no arrivals to the queues $Q_{1}$ and $Q_{2}$, therefore there is no transfer users from $Q_{3}$ and $Q_{4}$, it means, $\tilde{\mu}_{1}=\mu_{1}$, $\tilde{\mu}_{2}=\mu_{2}$, $\tilde{\mu}_{3}=\mu_{3}$, $\tilde{\mu}_{4}=\mu_{4}$, thats it, the events $A_{1}$ and $A_{3}$ are conditionally independent in the interval $I\left(n,m\right)$; the same goes for the events $A_{2}$ and $A_{4}$, therefore we have
%\small{
\begin{eqnarray}
\begin{array}{l}
\prob\left\{A_{1}\left(T^{*}\right),A_{2}\left(T^{*}\right),
A_{3}\left(T^{*}\right),A_{4}\left(T^{*}\right)|T^{*}\in I\left(n,m\right)\right\}
=\prod_{j=1}^{4}\prob\left\{A_{j}\left(T^{*}\right)|T^{*}\in I\left(n,m\right)\right\}\\
\geq\prob\left\{A_{1}\left(T^{*}\right)|T^{*}\in I_{1}\left(n\right)\right\}
\prob\left\{A_{2}\left(T^{*}\right)|T^{*}\in I_{2}\left(n\right)\right\}
\prob\left\{A_{3}\left(T^{*}\right)|T^{*}\in I_{3}\left(m\right)\right\}
\prob\left\{A_{4}\left(T^{*}\right)|T^{*}\in I_{4}\left(m\right)\right\}\\
=e^{-\mu_{1}\xi_{1}\left(n\right)}
e^{-\mu_{2}\xi_{2}\left(n\right)}
e^{-\mu_{3}\xi_{3}\left(m\right)}
e^{-\mu_{4}\xi_{4}\left(m\right)}
=e^{-\left[\tilde{\mu}_{1}\xi_{1}\left(n\right)
+\tilde{\mu}_{2}\xi_{2}\left(n\right)
+\tilde{\mu}_{3}\xi_{3}\left(m\right)
+\tilde{\mu}_{4}\xi_{4}
\left(m\right)\right]}>0.
\end{array}
\end{eqnarray}

Now we only need to prove that for $n\ge1$, there exist an $m\geq1$ such that the cases mentioned before are satisfied: 

\begin{multicols}{2}
\begin{itemize}
\item[a)] $\tau_{3}\left(m\right)<\tau_{2}\left(n\right)<\overline{\tau}_{3}\left(m\right)$,

\item[b)] $\overline{\tau}_{3}\left(m\right)<\tau_{2}\left(n\right)
<\tau_{4}\left(m\right)$,

\item[c)] $\tau_{4}\left(m\right)<\tau_{2}\left(n\right)<
\overline{\tau}_{4}\left(m\right)$,

\item[d)] $\overline{\tau}_{4}\left(m\right)<\tau_{2}\left(n\right)
<\tau_{3}\left(m+1\right)$.
\end{itemize}
\end{multicols}
We only give the proof for the fist case, for the rest the demonstration are similar. Suppose there is no $m\geq1$, with $I_{1}\left(n\right)\cap I_{3}\left(m\right)\neq\emptyset$, it means that for all $m\geq1$, $I_{1}\left(n\right)\cap I_{3}\left(m\right)=\emptyset$, then we have only two cases

\begin{itemize}
\item[a)] $\tau_{2}\left(n\right)\leq\tau_{3}\left(m\right)$: Recall that $\tau_{2}\left(m\right)=\overline{\tau}_{1}+r_{1}\left(m\right)$ 
where each of the random variables are such that $\esp\left[\overline{\tau}_{1}\left(n\right)-\tau_{1}\left(n\right)\right]<\infty$, $\esp\left[R_{1}\right]<\infty$ y $\esp\left[\tau_{3}\left(m\right)\right]<\infty$, which contradicts the fact that there is no such $m\geq1$.

\item[b)] $\tau_{2}\left(n\right)\geq\overline{\tau}_{3}\left(m\right)$: the reasoning is similar to the previous given.

\end{itemize}

\end{proof}






\newpage
%______________________________________________________________________
\section{ General Case Calculations Exhaustive Policy}\label{Secc.Append.B}
%______________________________________________________________________

%_______________________________________________________________
%\subsection{Calculations}
%_______________________________________________________________


Remember the equations given in equations (\ref{Ec.Gral.Primer.Momento.Ind.Exh}) and (\ref{Eq.Gral.Second.Order.Exhaustive}) for the first and second order partial derivatives respectively. The first moments equations for the queue lengths as before for the times the server arrives to the queue to start attending are obtained solving the system given by $f_{1}\left(i\right)=D_{i}R_{2}+D_{i}\mathbf{F}_{2}+\indora_{i\geq3}D_{i}\mathbf{F}_{4}$, similar expressions of the queues for the rest give us the linear system



\begin{eqnarray*}
\begin{array}{ll}
f_{1}\left(1\right)=r_{2}\tilde{\mu}_{1}+\frac{\tilde{\mu}_{1}}{1-\tilde{\mu}_{2}}f_{2}\left(2\right)+f_{2}\left(1\right),&
f_{1}\left(2\right)=r_{2}\tilde{\mu}_{2},\\
f_{1}\left(3\right)=r_{2}\tilde{\mu}_{3}+\frac{\tilde{\mu}_{3}}{1-\tilde{\mu}_{2}}f_{2}\left(2\right)+F_{3,2}^{(1)}\left(1\right),&
f_{1}\left(4\right)=r_{2}\tilde{\mu}_{4}+\frac{\tilde{\mu}_{4}}{1-\tilde{\mu}_{2}}f_{2}\left(2\right)+F_{4,2}^{(1)}\left(1\right),\\
f_{2}\left(1\right)=r_{1}\tilde{\mu}_{1},&
f_{2}\left(2\right)=r_{1}\tilde{\mu}_{2}+\frac{\tilde{\mu}_{2}}{1-\tilde{\mu}_{1}}f_{1}\left(1\right)+f_{1}\left(2\right),\\
f_{2}\left(3\right)=r_{1}\tilde{\mu}_{3}+\frac{\tilde{\mu}_{3}}{1-\tilde{\mu}_{1}}f_{1}\left(1\right)+F_{3,1}^{(1)}\left(1\right),&
f_{2}\left(4\right)=r_{1}\tilde{\mu}_{4}+\frac{\tilde{\mu}_{4}}{1-\tilde{\mu}_{1}}f_{1}\left(1\right)+F_{4,1}^{(1)}\left(1\right),\\
f_{3}\left(1\right)=\tilde{r}_{4}\tilde{\mu}_{1}+\frac{\tilde{\mu}_{1}}{1-\tilde{\mu}_{4}}f_{4}\left(4\right)+F_{1,4}^{(1)}\left(1\right),&
f_{3}\left(2\right)=\tilde{r}_{4}\tilde{\mu}_{2}+\frac{\tilde{\mu}_{2}}{1-\tilde{\mu}_{4}}f_{4}\left(4\right)+F_{2,4}^{(1)}\left(1\right),\\
f_{3}\left(3\right)=\tilde{r}_{4}\tilde{\mu}_{3}+\frac{\tilde{\mu}_{3}}{1-\tilde{\mu}_{4}}f_{4}\left(4\right)+f_{4}\left(3\right),&
f_{3}\left(4\right)=\tilde{r}_{4}\tilde{\mu}_{4}\\
f_{4}\left(1\right)=\tilde{r}_{3}\tilde{\mu}_{1}+\frac{\tilde{\mu}_{1}}{1-\tilde{\mu}_{3}}f_{3}\left(3\right)+F_{1,3}^{(1)}\left(1\right),&
f_{4}\left(2\right)=\tilde{r}_{3}\mu_{2}+\frac{\tilde{\mu}_{2}}{1-\tilde{\mu}_{3}}f_{3}\left(3\right)+F_{2,3}^{(1)}\left(1\right),\\
f_{4}\left(3\right)=\tilde{r}_{3}\tilde{\mu}_{3},&
f_{4}\left(4\right)=\tilde{r}_{3}\tilde{\mu}_{4}+\frac{\tilde{\mu}_{4}}{1-\tilde{\mu}_{3}}f_{3}\left(3\right)+f_{3}\left(4\right),\\
\end{array}
\end{eqnarray*}

Then we have that if $\mu=\tilde{\mu}_{1}+\tilde{\mu}_{2}<1$, $\hat{\mu}=\tilde{\mu}_{3}+\tilde{\mu}_{4}<1$, $r=r_{1}+r_{2}$ and $\hat{r}=\tilde{r}_{3}+\tilde{r}_{4}$  the system's solution are obtained by direct calculations:

\begin{eqnarray*}
\begin{array}{ll}
f_{2}\left(1\right)=r_{1}\tilde{\mu}_{1},&
f_{1}\left(2\right)=r_{2}\tilde{\mu}_{2},\\
f_{3}\left(4\right)=r_{4}\tilde{\mu}_{4},&
f_{4}\left(3\right)=r_{3}\tilde{\mu}_{3},\\
f_{1}\left(1\right)=r\frac{\tilde{\mu}_{1}\left(1-\tilde{\mu}_{1}\right)}{1-\mu},&
f_{2}\left(2\right)=r\frac{\tilde{\mu}_{2}\left(1-\tilde{\mu}_{2}\right)}{1-\mu},\\
f_{1}\left(3\right)=\tilde{\mu}_{3}\left(r_{2}+\frac{r\tilde{\mu}_{2}}{1-\mu}\right)+F_{3,2}^{(1)}\left(1\right),&
f_{1}\left(4\right)=\tilde{\mu}_{4}\left(r_{2}+\frac{r\tilde{\mu}_{2}}{1-\mu}\right)+F_{4,2}^{(1)}\left(1\right),\\
f_{2}\left(3\right)=\tilde{\mu}_{3}\left(r_{1}+\frac{r\tilde{\mu}_{1}}{1-\tilde{\mu}}\right)+F_{3,1}^{(1)}\left(1\right),&
f_{2}\left(4\right)=\tilde{\mu}_{4}\left(r_{1}+\frac{r\tilde{\mu}_{1}}{1-\mu}\right)+F_{4,,1}^{(1)}\left(1\right),\\
f_{3}\left(1\right)=\tilde{\mu}_{1}\left(r_{4}+\frac{\hat{r}\tilde{\mu}_{4}}{1-\hat{\mu}}\right)+F_{1,4}^{(1)}\left(1\right),&
f_{3}\left(2\right)=\tilde{\mu}_{2}\left(r_{4}+\frac{\hat{r}\tilde{\mu}_{4}}{1-\hat{\mu}}\right)+F_{2,4}^{(1)}\left(1\right),\\
f_{3}\left(3\right)=\hat{r}\frac{\tilde{\mu}_{3}\left(1-\tilde{\mu}_{3}\right)}{1-\hat{\mu}},&
f_{4}\left(1\right)=\tilde{\mu}_{1}\left(r_{3}+\frac{\hat{r}\tilde{\mu}_{3}}{1-\hat{\mu}}\right)+F_{1,3}^{(1)}\left(1\right),\\
f_{4}\left(2\right)=\tilde{\mu}_{2}\left(r_{3}+\frac{\hat{r}\tilde{\mu}_{3}}{1-\hat{\mu}}\right)+F_{2,3}^{(1)}\left(1\right),&
f_{4}\left(4\right)=\hat{r}\frac{\tilde{\mu}_{4}\left(1-\tilde{\mu}_{4}\right)}{1-\hat{\mu}}.
\end{array}
\end{eqnarray*}

Now, developing the equations given in (\ref{Eq.Gral.Second.Order.Exhaustive}) we obtain for instance $f_{1}\left(1,1\right)=\left(\frac{\tilde{\mu}_{1}}{1-\tilde{\mu}_{2}}\right)^{2}f_{2}\left(2,2\right)
+2\frac{\tilde{\mu}_{1}}{1-\tilde{\mu}_{2}}f_{2}\left(2,1\right)
+f_{2}\left(1,1\right)
+\tilde{\mu}_{1}^{2}\left(R_{2}^{(2)}+f_{2}\left(2\right)\theta_{2}^{(2)}\right)
+\tilde{P}_{1}^{(2)}\left(\frac{f_{2}\left(2\right)}{1-\tilde{\mu}_{2}}+r_{2}\right)+2r_{2}\tilde{\mu}_{2}f_{2}\left(1\right)$; similar reasoning lead us the following general expressions

\begin{eqnarray}\label{Eq.Sdo.Orden.Exh.uno}
\begin{array}{l}
f_{1}\left(i,j\right)=\indora_{i=1}f_{2}\left(1,1\right)
+\left[\left(1-\indora_{i=j=3}\right)\indora_{i+j\leq6}\indora_{i\leq j}\frac{\mu_{j}}{1-\tilde{\mu}_{2}}
+\left(1-\indora_{i=j=3}\right)\indora_{i+j\leq6}\indora_{i>j}\frac{\mu_{i}}{1-\tilde{\mu}_{2}}\right.\\
\left.+\indora_{i=1}\frac{\mu_{i}}{1-\tilde{\mu}_{2}}\right]f_{2}\left(1,2\right)+\indora_{i,j\neq2}\left(\frac{1}{1-\tilde{\mu}_{2}}\right)^{2}\mu_{i}\mu_{j}f_{2}\left(2,2\right)
+\left[\indora_{i,j\neq2}\tilde{\theta}_{2}^{(2)}\tilde{\mu}_{i}\tilde{\mu}_{j}
+\indora_{i,j\neq2}\indora_{i=j}\frac{\tilde{P}_{i}^{(2)}}{1-\tilde{\mu}_{2}}\right.\\
\left.+\indora_{i,j\neq2}\indora_{i\neq j}\frac{\tilde{\mu}_{i}\tilde{\mu}_{j}}{1-\tilde{\mu}_{2}}\right]f_{2}\left(2\right)
+\left[r_{2}\tilde{\mu}_{i}
+\indora_{i\geq3}F_{i,2}^{(1)}\right]f_{2}\left(j\right)
+\left[r_{2}\tilde{\mu}_{j}
+\indora_{j\geq3}F_{j,2}^{(1)}\right]f_{2}\left(i\right)\\
+\left[R_{2}^{(2)}
+\indora_{i=j}r_{2}\right]\tilde{\mu}_{i}\mu_{j}+\indora_{j\geq3}F_{j,2}^{(1)}\left[\indora_{j\neq i}F_{i,2}^{(1)}
+r_{2}\tilde{\mu}_{i}\right]
+r_{2}\left[\indora_{i=j}P_{i}^{(2)}
+\indora_{i\geq3}F_{i,2}^{(1)}\tilde{\mu}_{j}\right]\\
+\indora_{i\geq3}\indora_{j=i}F_{i,2}^{(2)}
\end{array}
\end{eqnarray}

in a similar manner we obtain expressions for $f_{2}\left(i,j\right)$, $f_{3}\left(i,j\right)$ and $f_{4}\left(i,j\right)$

for $i,k=1,2,3,4$; from which we obtain the linear equations system
\begin{eqnarray}\label{System.Second.Order.Moments.uno}
\begin{array}{ll}
f_{1}\left(1,1\right)=a_{1}f_{2}\left(2,2\right)
+a_{2}f_{2}\left(2,1\right)
+a_{3}f_{2}\left(1,1\right)
+K_{1},&
f_{1}\left(1,2\right)=K_{2},\\
f_{1}\left(1,3\right)=a_{4}f_{2}\left(2,2\right)+a_{5}f\left(2,1\right)+K_{3},&
f_{1}\left(1,4\right)=a_{6}f_{2}\left(2,2\right)+a_{7}f_{2}\left(2,1\right)+K_{4},\end{array}
\end{eqnarray}
for the rest equations, similar reasoning lead us to a linear system equations whose solutions are described in corolary (\ref{Coro.Second.Order.Eqs}) with coefficients given by, we just show a few of them


%Which can be reduced to solve the system given in (\ref{System.Second.Order.Moments.uno}) and (\ref{System.Second.Order.Moments.dos}).

with values for $a_{i}$ and $K_{i}$  
%{\small{
\begin{eqnarray}\label{Coefficients.Ais.Exh.uno}
\begin{array}{llll}
a_{1}=\left(\frac{\tilde{\mu}_{1}}{1-\tilde{\mu}_{2}}\right)^{2},&
a_{2}=\frac{2\tilde{\mu}_{1}}{1-\tilde{\mu}_{2}},&
a_{3}=1,&
a_{4}=\left(\frac{1}{1-\tilde{\mu}_{2}}\right)^{2}\tilde{\mu}_{1}\tilde{\mu}_{3},\\
a_{5}=\frac{\tilde{\mu}_{3}}{1-\tilde{\mu}_{2}},&
a_{6}=\left(\frac{1}{1-\tilde{\mu}_{2}}\right)^{2}\tilde{\mu}_{1}\tilde{\mu}_{4},&
a_{7}=\frac{\tilde{\mu}_{4}}{1-\tilde{\mu}_{2}},&\\
\end{array}
\end{eqnarray}%}}





\begin{eqnarray}\label{Coefficients.kis.Exh.uno}
\begin{array}{l}
K_{1}=\tilde{\mu}_{1}^{2}\left(R_{2}^{(2)}+f_{2}\left(2\right)\theta_{2}^{(2)}\right)
+\tilde{P}_{1}^{(2)}\left(\frac{f_{2}\left(2\right)}{1-\tilde{\mu}_{2}}+r_{2}\right)
+2r_{2}\tilde{\mu}_{2}f_{2}\left(1\right),\\
K_{2}=\tilde{\mu}_{1}\tilde{\mu}_{2}\left[R_{2}^{(2)}
+r_{2}\right]
+r_{2}\left[\tilde{\mu}_{1}f_{2}\left(2\right)
+\tilde{\mu}_{2}f_{2}\left(1\right)\right],\\
K_{3}=\tilde{\mu}_{1}\tilde{\mu}_{3}\left[R_{2}^{(2)}+r_{2}+f_{2}\left(2\right)\left(\tilde{\theta}_{2}^{(2)}+\frac{1}{1-\tilde{\mu}_{2}}\right)\right]
+r_{2}\tilde{\mu}_{1}\left[F_{3,2}^{(1)}+f_{2}\left(1\right)\right]
+\left[r_{2}\tilde{\mu}_{3}+F_{3,2}^{(1)}\right]f_{2}\left(1\right),\\
K_{4}=\tilde{\mu}_{1}\tilde{\mu}_{4}\left[R_{2}^{(2)}
+r_{2}+f_{2}\left(2\right)\left(\tilde{\theta}_{2}^{(2)}
+\frac{1}{1-\tilde{\mu}_{2}}\right)\right]
+r_{2}\tilde{\mu}_{1}\left[f_{2}\left(4\right)+F_{4,2}^{(1)}\right]
+f_{2}\left(1\right)\left[r_{2}\tilde{\mu}_{4}+F_{4,2}^{(1)}\right],
\end{array}
\end{eqnarray}

\newpage
%______________________________________________________________________
\section{ Stability Analysis for a NCPS}
%__________________________________________________________________
%
\begin{Teo}
Dada una Red de Sistemas de Visitas C\'iclicas (RSVC), conformada por dos Sistemas de Visitas C\'iclicas (SVC), donde cada uno de ellos consta de dos colas tipo $M/M/1$. Los dos sistemas est\'an comunicados entre s\'i por medio de la transferencia de usuarios entre las colas $Q_{1}\leftrightarrow Q_{3}$ y $Q_{2}\leftrightarrow Q_{4}$. Se definen los eventos para los procesos de arribos al tiempo $t$, $A_{j}\left(t\right)=\left\{0 \textrm{ arribos en }Q_{j}\textrm{ al tiempo }t\right\}$ para alg\'un tiempo $t\geq0$ y $Q_{j}$ la cola $j$-\'esima en la RSVC, para $j=1,2,3,4$.  Existe un intervalo $I\neq\emptyset$ tal que para $T^{*}\in I$, tal que $\prob\left\{A_{1}\left(T^{*}\right),A_{2}\left(Tt^{*}\right),
A_{3}\left(T^{*}\right),A_{4}\left(T^{*}\right)|T^{*}\in I\right\}>0$.
\end{Teo}



\begin{proof}
Sin p\'erdida de generalidad podemos considerar como base del an\'alisis a la cola $Q_{1}$ del primer sistema que conforma la RSVC.\medskip 

Sea $n\geq1$, ciclo en el primer sistema en el que se sabe que $L_{j}\left(\overline{\tau}_{1}\left(n\right)\right)=0$, pues la pol\'itica de servicio con que atienden los servidores es la exhaustiva. Como es sabido, para trasladarse a la siguiente cola, el servidor incurre en un tiempo de traslado $r_{1}\left(n\right)>0$, entonces tenemos que $\tau_{2}\left(n\right)=\overline{\tau}_{1}\left(n\right)+r_{1}\left(n\right)$.\medskip 


Definamos el intervalo $I_{1}\equiv\left[\overline{\tau}_{1}\left(n\right),\tau_{2}\left(n\right)\right]$ de longitud $\xi_{1}=r_{1}\left(n\right)$.

Dado que los tiempos entre arribo son exponenciales con tasa $\tilde{\mu}_{1}=\mu_{1}+\hat{\mu}_{1}$ ($\mu_{1}$ son los arribos a $Q_{1}$ por primera vez al sistema, mientras que $\hat{\mu}_{1}$ son los arribos de traslado procedentes de $Q_{3}$) se tiene que la probabilidad del evento $A_{1}\left(t\right)$ est\'a dada por 

\begin{equation}
\prob\left\{A_{1}\left(t\right)|t\in I_{1}\left(n\right)\right\}=e^{-\tilde{\mu}_{1}\xi_{1}\left(n\right)}.
\end{equation} 


Por otra parte, para la cola $Q_{2}$ el tiempo $\overline{\tau}_{2}\left(n-1\right)$ es tal que $L_{2}\left(\overline{\tau}_{2}\left(n-1\right)\right)=0$, es decir, es el tiempo en que la cola queda totalmente vac\'ia en el ciclo anterior a $n$. \medskip 


Entonces tenemos un sgundo intervalo $I_{2}\equiv\left[\overline{\tau}_{2}\left(n-1\right),\tau_{2}\left(n\right)\right]$. Por lo tanto la probabilidad del evento $A_{2}\left(t\right)$ tiene probabilidad dada por

\begin{eqnarray}
\prob\left\{A_{2}\left(t\right)|t\in I_{2}\left(n\right)\right\}=e^{-\tilde{\mu}_{2}\xi_{2}\left(n\right)},\\
\xi_{2}\left(n\right)=\tau_{2}\left(n\right)-\overline{\tau}_{2}\left(n-1\right)
\end{eqnarray}
%\end{equation} 

%donde $$.

Ahora, dado que $I_{1}\left(n\right)\subset I_{2}\left(n\right)$, se tiene que

\begin{eqnarray*}
\xi_{1}\left(n\right)\leq\xi_{2}\left(n\right)&\Leftrightarrow& -\xi_{1}\left(n\right)\geq-\xi_{2}\left(n\right)
\\
-\tilde{\mu}_{2}\xi_{1}\left(n\right)\geq-\tilde{\mu}_{2}\xi_{2}\left(n\right)&\Leftrightarrow&
e^{-\tilde{\mu}_{2}\xi_{1}\left(n\right)}\geq e^{-\tilde{\mu}_{2}\xi_{2}\left(n\right)}\\
\prob\left\{A_{2}\left(t\right)|t\in I_{1}\left(n\right)\right\}&\geq&
\prob\left\{A_{2}\left(t\right)|t\in I_{2}\left(n\right)\right\}.
\end{eqnarray*}


Entonces se tiene que
\small{
\begin{eqnarray*}
\prob\left\{A_{1}\left(t\right),A_{2}\left(t\right)|t\in I_{1}\left(n\right)\right\}&=&
\prob\left\{A_{1}\left(t\right)|t\in I_{1}\left(n\right)\right\}
\prob\left\{A_{2}\left(t\right)|t\in I_{1}\left(n\right)\right\}\\
&\geq&
\prob\left\{A_{1}\left(t\right)|t\in I_{1}\left(n\right)\right\}
\prob\left\{A_{2}\left(t\right)|t\in I_{2}\left(n\right)\right\}\\
&=&e^{-\tilde{\mu}_{1}\xi_{1}\left(n\right)}e^{-\tilde{\mu}_{2}\xi_{2}\left(n\right)}
=e^{-\left[\tilde{\mu}_{1}\xi_{1}\left(n\right)+\tilde{\mu}_{2}\xi_{2}\left(n\right)\right]}.
\end{eqnarray*}}


Es decir, 

\begin{equation}
\prob\left\{A_{1}\left(t\right),A_{2}\left(t\right)|t\in I_{1}\left(n\right)\right\}
=e^{-\left[\tilde{\mu}_{1}\xi_{1}\left(n\right)+\tilde{\mu}_{2}\xi_{2}
\left(n\right)\right]}>0.
\end{equation}
En lo que respecta a la relaci\'on entre los dos SVC que conforman la RSVC para alg\'un $m\geq1$ se tiene que $\tau_{3}\left(m\right)<\tau_{2}\left(n\right)<\tau_{4}\left(m\right)$ por lo tanto se cumple cualquiera de los siguientes cuatro casos
\begin{itemize}
\item[a)] $\tau_{3}\left(m\right)<\tau_{2}\left(n\right)<\overline{\tau}_{3}\left(m\right)$

\item[b)] $\overline{\tau}_{3}\left(m\right)<\tau_{2}\left(n\right)
<\tau_{4}\left(m\right)$

\item[c)] $\tau_{4}\left(m\right)<\tau_{2}\left(n\right)<
\overline{\tau}_{4}\left(m\right)$

\item[d)] $\overline{\tau}_{4}\left(m\right)<\tau_{2}\left(n\right)
<\tau_{3}\left(m+1\right)$
\end{itemize}


Sea el intervalo $I_{3}\left(m\right)\equiv\left[\tau_{3}\left(m\right),\overline{\tau}_{3}\left(m\right)\right]$ tal que $\tau_{2}\left(n\right)\in I_{3}\left(m\right)$, con longitud de intervalo $\xi_{3}\equiv\overline{\tau}_{3}\left(m\right)-\tau_{3}\left(m\right)$, entonces se tiene que para $Q_{3}$
\begin{equation}
\prob\left\{A_{3}\left(t\right)|t\in I_{3}\left(m\right)\right\}=e^{-\tilde{\mu}_{3}\xi_{3}\left(m\right)}.
\end{equation} 

mientras que para $Q_{4}$ consideremos el intervalo $I_{4}\left(m\right)\equiv\left[\tau_{4}\left(m-1\right),\overline{\tau}_{3}\left(m\right)\right]$, entonces por construcci\'on  $I_{3}\left(m\right)\subset I_{4}\left(m\right)$, por lo tanto


\begin{eqnarray*}
\xi_{3}\left(m\right)\leq\xi_{4}\left(m\right)&\Leftrightarrow& -\xi_{3}\left(m\right)\geq-\xi_{4}\left(m\right)
\\
-\tilde{\mu}_{4}\xi_{3}\left(m\right)\geq-\tilde{\mu}_{4}\xi_{4}\left(m\right)&\Leftrightarrow&
e^{-\tilde{\mu}_{4}\xi_{3}\left(m\right)}\geq e^{-\tilde{\mu}_{4}\xi_{4}\left(n\right)}\\
\prob\left\{A_{4}\left(t\right)|t\in I_{3}\left(m\right)\right\}&\geq&
\prob\left\{A_{4}\left(t\right)|t\in I_{4}\left(m\right)\right\}.
\end{eqnarray*}



Entonces se tiene que
\small{
\begin{eqnarray*}
\prob\left\{A_{3}\left(t\right),A_{4}\left(t\right)|t\in I_{3}\left(m\right)\right\}&=&
\prob\left\{A_{3}\left(t\right)|t\in I_{3}\left(m\right)\right\}
\prob\left\{A_{4}\left(t\right)|t\in I_{3}\left(m\right)\right\}\\
&\geq&
\prob\left\{A_{3}\left(t\right)|t\in I_{3}\left(m\right)\right\}
\prob\left\{A_{4}\left(t\right)|t\in I_{4}\left(m\right)\right\}\\
&=&e^{-\tilde{\mu}_{3}\xi_{3}\left(m\right)}e^{-\tilde{\mu}_{4}\xi_{4}
\left(m\right)}
=e^{-\left(\tilde{\mu}_{3}\xi_{3}\left(m\right)+\tilde{\mu}_{4}\xi_{4}\left(m\right)\right)}.
\end{eqnarray*}}

Es decir, 

\begin{equation}
\prob\left\{A_{3}\left(t\right),A_{4}\left(t\right)|t\in I_{3}\left(m\right)\right\}\geq
e^{-\left(\tilde{\mu}_{3}\xi_{3}\left(m\right)+\tilde{\mu}_{4}\xi_{4}\left(m\right)\right)}>0.
\end{equation}


Sea el intervalo $I_{3}\left(m\right)\equiv\left[\overline{\tau}_{3}\left(m\right),\tau_{4}\left(m\right)\right]$ con longitud $\xi_{3}\equiv\tau_{4}\left(m\right)-\overline{\tau}_{3}\left(m\right)$, entonces se tiene que para $Q_{3}$
\begin{equation}
\prob\left\{A_{3}\left(t\right)|t\in I_{3}\left(m\right)\right\}=e^{-\tilde{\mu}_{3}\xi_{3}\left(m\right)}.
\end{equation} 

mientras que para $Q_{4}$ consideremos el intervalo $I_{4}\left(m\right)\equiv\left[\overline{\tau}_{4}\left(m-1\right),\tau_{4}\left(m\right)\right]$, entonces por construcci\'on  $I_{3}\left(m\right)\subset I_{4}\left(m\right)$, y al igual que en el caso anterior se tiene que 

\begin{eqnarray*}
\xi_{3}\left(m\right)\leq\xi_{4}\left(m\right)&\Leftrightarrow& -\xi_{3}\left(m\right)\geq-\xi_{4}\left(m\right)
\\
-\tilde{\mu}_{4}\xi_{3}\left(m\right)\geq-\tilde{\mu}_{4}\xi_{4}\left(m\right)&\Leftrightarrow&
e^{-\tilde{\mu}_{4}\xi_{3}\left(m\right)}\geq e^{-\tilde{\mu}_{4}\xi_{4}\left(n\right)}\\
\prob\left\{A_{4}\left(t\right)|t\in I_{3}\left(m\right)\right\}&\geq&
\prob\left\{A_{4}\left(t\right)|t\in I_{4}\left(m\right)\right\}.
\end{eqnarray*}


Entonces se tiene que
\small{
\begin{eqnarray*}
\prob\left\{A_{3}\left(t\right),A_{4}\left(t\right)|t\in I_{3}\left(m\right)\right\}&=&
\prob\left\{A_{3}\left(t\right)|t\in I_{3}\left(m\right)\right\}
\prob\left\{A_{4}\left(t\right)|t\in I_{3}\left(m\right)\right\}\\
&\geq&
\prob\left\{A_{3}\left(t\right)|t\in I_{3}\left(m\right)\right\}
\prob\left\{A_{4}\left(t\right)|t\in I_{4}\left(m\right)\right\}\\
&=&e^{-\tilde{\mu}_{3}\xi_{3}\left(m\right)}e^{-\tilde{\mu}_{4}\xi_{4}\left(m\right)}
=e^{-\left(\tilde{\mu}_{3}\xi_{3}\left(m\right)+\tilde{\mu}_{4}\xi_{4}\left(m\right)\right)}.
\end{eqnarray*}}

Es decir, 

\begin{equation}
\prob\left\{A_{3}\left(t\right),A_{4}\left(t\right)|t\in I_{4}\left(m\right)\right\}\geq
e^{-\left(\tilde{\mu}_{3}+\tilde{\mu}_{4}\right)\xi_{3}\left(m\right)}>0.
\end{equation}


Para el intervalo $I_{3}\left(m\right)=\left[\tau_{4}\left(m\right),\overline{\tau}_{4}\left(m\right)\right]$, se tiene que este caso es an\'alogo al caso (a).


Para el intevalo $I_{3}\left(m\right)\equiv\left[\overline{\tau}_{4}\left(m\right),\tau_{4}\left(m+1\right)\right]$, se tiene que es an\'alogo al caso (b).


Por construcci\'on se tiene que $I\left(n,m\right)\equiv I_{1}\left(n\right)\cap I_{3}\left(m\right)\neq\emptyset$,entonces en particular se tienen las contenciones $I\left(n,m\right)\subseteq I_{1}\left(n\right)$ y $I\left(n,m\right)\subseteq I_{3}\left(m\right)$, por lo tanto si definimos $\xi_{n,m}\equiv\ell\left(I\left(n,m\right)\right)$ tenemos que

\begin{eqnarray*}
\xi_{n,m}\leq\xi_{1}\left(n\right)\textrm{ y }\xi_{n,m}\leq\xi_{3}\left(m\right)\textrm{ entonces }\\
-\xi_{n,m}\geq-\xi_{1}\left(n\right)\textrm{ y }-\xi_{n,m}\leq-\xi_{3}\left(m\right)\\
\end{eqnarray*}
por lo tanto tenemos las desigualdades 


\begin{eqnarray*}
\begin{array}{ll}
-\tilde{\mu}_{1}\xi_{n,m}\geq-\tilde{\mu}_{1}\xi_{1}\left(n\right),&
-\tilde{\mu}_{2}\xi_{n,m}\geq-\tilde{\mu}_{2}\xi_{1}\left(n\right)
\geq-\tilde{\mu}_{2}\xi_{2}\left(n\right),\\
-\tilde{\mu}_{3}\xi_{n,m}\geq-\tilde{\mu}_{3}\xi_{3}\left(m\right),&
-\tilde{\mu}_{4}\xi_{n,m}\geq-\tilde{\mu}_{4}\xi_{3}\left(m\right)
\geq-\tilde{\mu}_{4}\xi_{4}\left(m\right).
\end{array}
\end{eqnarray*}

Sea $T^{*}\in I\left(n,m\right)$, entonces dado que en particular $T^{*}\in I_{1}\left(n\right)$, se cumple con probabilidad positiva que no hay arribos a las colas $Q_{1}$ y $Q_{2}$, en consecuencia, tampoco hay usuarios de transferencia para $Q_{3}$ y $Q_{4}$, es decir, $\tilde{\mu}_{1}=\mu_{1}$, $\tilde{\mu}_{2}=\mu_{2}$, $\tilde{\mu}_{3}=\mu_{3}$, $\tilde{\mu}_{4}=\mu_{4}$, es decir, los eventos $Q_{1}$ y $Q_{3}$ son condicionalmente independientes en el intervalo $I\left(n,m\right)$; lo mismo ocurre para las colas $Q_{2}$ y $Q_{4}$, por lo tanto tenemos que
%\small{
\begin{eqnarray}
\begin{array}{l}
\prob\left\{A_{1}\left(T^{*}\right),A_{2}\left(T^{*}\right),
A_{3}\left(T^{*}\right),A_{4}\left(T^{*}\right)|T^{*}\in I\left(n,m\right)\right\}\\
=\prod_{j=1}^{4}\prob\left\{A_{j}\left(T^{*}\right)|T^{*}\in I\left(n,m\right)\right\}\\
\geq\prob\left\{A_{1}\left(T^{*}\right)|T^{*}\in I_{1}\left(n\right)\right\}
\prob\left\{A_{2}\left(T^{*}\right)|T^{*}\in I_{2}\left(n\right)\right\}\\
\prob\left\{A_{3}\left(T^{*}\right)|T^{*}\in I_{3}\left(m\right)\right\}
\prob\left\{A_{4}\left(T^{*}\right)|T^{*}\in I_{4}\left(m\right)\right\}\\
=e^{-\mu_{1}\xi_{1}\left(n\right)}
e^{-\mu_{2}\xi_{2}\left(n\right)}
e^{-\mu_{3}\xi_{3}\left(m\right)}
e^{-\mu_{4}\xi_{4}\left(m\right)}\\
=e^{-\left[\tilde{\mu}_{1}\xi_{1}\left(n\right)
+\tilde{\mu}_{2}\xi_{2}\left(n\right)
+\tilde{\mu}_{3}\xi_{3}\left(m\right)
+\tilde{\mu}_{4}\xi_{4}
\left(m\right)\right]}>0.
\end{array}
\end{eqnarray}


Ahora solo resta demostrar que para $n\ge1$, existe $m\geq1$ tal que se cumplen cualquiera de los cuatro casos arriba mencionados: 

\begin{itemize}
\item[a)] $\tau_{3}\left(m\right)<\tau_{2}\left(n\right)<\overline{\tau}_{3}\left(m\right)$

\item[b)] $\overline{\tau}_{3}\left(m\right)<\tau_{2}\left(n\right)
<\tau_{4}\left(m\right)$

\item[c)] $\tau_{4}\left(m\right)<\tau_{2}\left(n\right)<
\overline{\tau}_{4}\left(m\right)$

\item[d)] $\overline{\tau}_{4}\left(m\right)<\tau_{2}\left(n\right)
<\tau_{3}\left(m+1\right)$
\end{itemize}

Consideremos nuevamente el primer caso. Supongamos que no existe $m\geq1$, tal que $I_{1}\left(n\right)\cap I_{3}\left(m\right)\neq\emptyset$, es decir, para toda $m\geq1$, $I_{1}\left(n\right)\cap I_{3}\left(m\right)=\emptyset$, entonces se tiene que ocurren cualquiera de los dos casos

\begin{itemize}
\item[a)] $\tau_{2}\left(n\right)\leq\tau_{3}\left(m\right)$: Recordemos que $\tau_{2}\left(m\right)=\overline{\tau}_{1}+r_{1}\left(m\right)$ donde cada una de las variables aleatorias son tales que $\esp\left[\overline{\tau}_{1}\left(n\right)-\tau_{1}\left(n\right)\right]<\infty$, $\esp\left[R_{1}\right]<\infty$ y $\esp\left[\tau_{3}\left(m\right)\right]<\infty$, lo cual contradice el hecho de que no exista un ciclo $m\geq1$ que satisfaga la condici\'on deseada.

\item[b)] $\tau_{2}\left(n\right)\geq\overline{\tau}_{3}\left(m\right)$: por un argumento similar al anterior se tiene que no es posible que no exista un ciclo $m\geq1$ tal que satisaface la condici\'on deseada.

\end{itemize}

Para el resto de los casos la demostraci\'on es an\'aloga. Por lo tanto, se tiene que efectivamente existe $m\geq1$ tal que $\tau_{3}\left(m\right)<\tau_{2}\left(n\right)<\tau_{4}\left(m\right)$.
\end{proof}
\newpage

%_________________________________________________________________________
%
\section{ Output Process and Regenerative Processes}
%_________________________________________________________________________
%
En Sigman, Thorison y Wolff \cite{Sigman2} prueban que para la existencia de un una sucesi\'on infinita no decreciente de tiempos de regeneraci\'on $\tau_{1}\leq\tau_{2}\leq\cdots$ en los cuales el proceso se regenera, basta un tiempo de regeneraci\'on $R_{1}$, donde $R_{j}=\tau_{j}-\tau_{j-1}$. Para tal efecto se requiere la existencia de un espacio de probabilidad $\left(\Omega,\mathcal{F},\prob\right)$, y proceso estoc\'astico $\textit{X}=\left\{X\left(t\right):t\geq0\right\}$ con espacio de estados $\left(S,\mathcal{R}\right)$, con $\mathcal{R}$ $\sigma$-\'algebra.

\begin{Prop}
Si existe una variable aleatoria no negativa $R_{1}$ tal que $\theta_{R1}X=_{D}X$, entonces $\left(\Omega,\mathcal{F},\prob\right)$ puede extenderse para soportar una sucesi\'on estacionaria de variables aleatorias $R=\left\{R_{k}:k\geq1\right\}$, tal que para $k\geq1$,
\begin{eqnarray*}
\theta_{k}\left(X,R\right)=_{D}\left(X,R\right).
\end{eqnarray*}

Adem\'as, para $k\geq1$, $\theta_{k}R$ es condicionalmente independiente de $\left(X,R_{1},\ldots,R_{k}\right)$, dado $\theta_{\tau k}X$.

\end{Prop}


\begin{itemize}
\item Doob en 1953 demostr\'o que el estado estacionario de un proceso de partida en un sistema de espera $M/G/\infty$, es Poisson con la misma tasa que el proceso de arribos.

\item Burke en 1968, fue el primero en demostrar que el estado estacionario de un proceso de salida de una cola $M/M/s$ es un proceso Poisson.

\item Disney en 1973 obtuvo el siguiente resultado:

\begin{Teo}
Para el sistema de espera $M/G/1/L$ con disciplina FIFO, el proceso $\textbf{I}$ es un proceso de renovaci\'on si y s\'olo si el proceso denominado longitud de la cola es estacionario y se cumple cualquiera de los siguientes casos:

\begin{itemize}
\item[a)] Los tiempos de servicio son identicamente cero;
\item[b)] $L=0$, para cualquier proceso de servicio $S$;
\item[c)] $L=1$ y $G=D$;
\item[d)] $L=\infty$ y $G=M$.
\end{itemize}
En estos casos, respectivamente, las distribuciones de interpartida $P\left\{T_{n+1}-T_{n}\leq t\right\}$ son


\begin{itemize}
\item[a)] $1-e^{-\lambda t}$, $t\geq0$;
\item[b)] $1-e^{-\lambda t}*F\left(t\right)$, $t\geq0$;
\item[c)] $1-e^{-\lambda t}*\indora_{d}\left(t\right)$, $t\geq0$;
\item[d)] $1-e^{-\lambda t}*F\left(t\right)$, $t\geq0$.
\end{itemize}
\end{Teo}


\item Finch (1959) mostr\'o que para los sistemas $M/G/1/L$, con $1\leq L\leq \infty$ con distribuciones de servicio dos veces diferenciable, solamente el sistema $M/M/1/\infty$ tiene proceso de salida de renovaci\'on estacionario.

\item King (1971) demostro que un sistema de colas estacionario $M/G/1/1$ tiene sus tiempos de interpartida sucesivas $D_{n}$ y $D_{n+1}$ son independientes, si y s\'olo si, $G=D$, en cuyo caso le proceso de salida es de renovaci\'on.

\item Disney (1973) demostr\'o que el \'unico sistema estacionario $M/G/1/L$, que tiene proceso de salida de renovaci\'on  son los sistemas $M/M/1$ y $M/D/1/1$.



\item El siguiente resultado es de Disney y Koning (1985)
\begin{Teo}
En un sistema de espera $M/G/s$, el estado estacionario del proceso de salida es un proceso Poisson para cualquier distribuci\'on de los tiempos de servicio si el sistema tiene cualquiera de las siguientes cuatro propiedades.

\begin{itemize}
\item[a)] $s=\infty$
\item[b)] La disciplina de servicio es de procesador compartido.
\item[c)] La disciplina de servicio es LCFS y preemptive resume, esto se cumple para $L<\infty$
\item[d)] $G=M$.
\end{itemize}

\end{Teo}

\item El siguiente resultado es de Alamatsaz (1983)

\begin{Teo}
En cualquier sistema de colas $GI/G/1/L$ con $1\leq L<\infty$ y distribuci\'on de interarribos $A$ y distribuci\'on de los tiempos de servicio $B$, tal que $A\left(0\right)=0$, $A\left(t\right)\left(1-B\left(t\right)\right)>0$ para alguna $t>0$ y $B\left(t\right)$ para toda $t>0$, es imposible que el proceso de salida estacionario sea de renovaci\'on.
\end{Teo}

\end{itemize}






%______________________________________________________________________
%\subsection{Ejemplos, Notas importantes}


Sean $T_{1},T_{2},\ldots$ los puntos donde las longitudes de las colas de la red de sistemas de visitas c\'iclicas son cero simult\'aneamente, cuando la cola $Q_{j}$ es visitada por el servidor para dar servicio, es decir, $L_{1}\left(T_{i}\right)=0,L_{2}\left(T_{i}\right)=0,\hat{L}_{1}\left(T_{i}\right)=0$ y $\hat{L}_{2}\left(T_{i}\right)=0$, a estos puntos se les denominar\'a puntos regenerativos. Sea la funci\'on generadora de momentos para $L_{i}$, el n\'umero de usuarios en la cola $Q_{i}\left(z\right)$ en cualquier momento, est\'a dada por el tiempo promedio de $z^{L_{i}\left(t\right)}$ sobre el ciclo regenerativo definido anteriormente:

\begin{eqnarray*}
Q_{i}\left(z\right)&=&\esp\left[z^{L_{i}\left(t\right)}\right]=\frac{\esp\left[\sum_{m=1}^{M_{i}}\sum_{t=\tau_{i}\left(m\right)}^{\tau_{i}\left(m+1\right)-1}z^{L_{i}\left(t\right)}\right]}{\esp\left[\sum_{m=1}^{M_{i}}\tau_{i}\left(m+1\right)-\tau_{i}\left(m\right)\right]}
\end{eqnarray*}

$M_{i}$ es un tiempo de paro en el proceso regenerativo con $\esp\left[M_{i}\right]<\infty$\footnote{En Stidham\cite{Stidham} y Heyman  se muestra que una condici\'on suficiente para que el proceso regenerativo 
estacionario sea un procesoo estacionario es que el valor esperado del tiempo del ciclo regenerativo sea finito, es decir: $\esp\left[\sum_{m=1}^{M_{i}}C_{i}^{(m)}\right]<\infty$, como cada $C_{i}^{(m)}$ contiene intervalos de r\'eplica positivos, se tiene que $\esp\left[M_{i}\right]<\infty$, adem\'as, como $M_{i}>0$, se tiene que la condici\'on anterior es equivalente a tener que $\esp\left[C_{i}\right]<\infty$,
por lo tanto una condici\'on suficiente para la existencia del proceso regenerativo est\'a dada por $\sum_{k=1}^{N}\mu_{k}<1.$}, se sigue del lema de Wald que:


\begin{eqnarray*}
\esp\left[\sum_{m=1}^{M_{i}}\sum_{t=\tau_{i}\left(m\right)}^{\tau_{i}\left(m+1\right)-1}z^{L_{i}\left(t\right)}\right]&=&\esp\left[M_{i}\right]\esp\left[\sum_{t=\tau_{i}\left(m\right)}^{\tau_{i}\left(m+1\right)-1}z^{L_{i}\left(t\right)}\right]\\
\esp\left[\sum_{m=1}^{M_{i}}\tau_{i}\left(m+1\right)-\tau_{i}\left(m\right)\right]&=&\esp\left[M_{i}\right]\esp\left[\tau_{i}\left(m+1\right)-\tau_{i}\left(m\right)\right]
\end{eqnarray*}

por tanto se tiene que


\begin{eqnarray*}
Q_{i}\left(z\right)&=&\frac{\esp\left[\sum_{t=\tau_{i}\left(m\right)}^{\tau_{i}\left(m+1\right)-1}z^{L_{i}\left(t\right)}\right]}{\esp\left[\tau_{i}\left(m+1\right)-\tau_{i}\left(m\right)\right]}
\end{eqnarray*}

observar que el denominador es simplemente la duraci\'on promedio del tiempo del ciclo.


Haciendo las siguientes sustituciones en la ecuaci\'on (\ref{Corolario2}): $n\rightarrow t-\tau_{i}\left(m\right)$, $T \rightarrow \overline{\tau}_{i}\left(m\right)-\tau_{i}\left(m\right)$, $L_{n}\rightarrow L_{i}\left(t\right)$ y $F\left(z\right)=\esp\left[z^{L_{0}}\right]\rightarrow F_{i}\left(z\right)=\esp\left[z^{L_{i}\tau_{i}\left(m\right)}\right]$, se puede ver que

\begin{eqnarray}\label{Eq.Arribos.Primera}
\esp\left[\sum_{n=0}^{T-1}z^{L_{n}}\right]=
\esp\left[\sum_{t=\tau_{i}\left(m\right)}^{\overline{\tau}_{i}\left(m\right)-1}z^{L_{i}\left(t\right)}\right]
=z\frac{F_{i}\left(z\right)-1}{z-P_{i}\left(z\right)}
\end{eqnarray}

Por otra parte durante el tiempo de intervisita para la cola $i$, $L_{i}\left(t\right)$ solamente se incrementa de manera que el incremento por intervalo de tiempo est\'a dado por la funci\'on generadora de probabilidades de $P_{i}\left(z\right)$, por tanto la suma sobre el tiempo de intervisita puede evaluarse como:

\begin{eqnarray*}
\esp\left[\sum_{t=\tau_{i}\left(m\right)}^{\tau_{i}\left(m+1\right)-1}z^{L_{i}\left(t\right)}\right]&=&\esp\left[\sum_{t=\tau_{i}\left(m\right)}^{\tau_{i}\left(m+1\right)-1}\left\{P_{i}\left(z\right)\right\}^{t-\overline{\tau}_{i}\left(m\right)}\right]=\frac{1-\esp\left[\left\{P_{i}\left(z\right)\right\}^{\tau_{i}\left(m+1\right)-\overline{\tau}_{i}\left(m\right)}\right]}{1-P_{i}\left(z\right)}\\
&=&\frac{1-I_{i}\left[P_{i}\left(z\right)\right]}{1-P_{i}\left(z\right)}
\end{eqnarray*}
por tanto

\begin{eqnarray*}
\esp\left[\sum_{t=\tau_{i}\left(m\right)}^{\tau_{i}\left(m+1\right)-1}z^{L_{i}\left(t\right)}\right]&=&
\frac{1-F_{i}\left(z\right)}{1-P_{i}\left(z\right)}
\end{eqnarray*}

Por lo tanto

\begin{eqnarray*}
Q_{i}\left(z\right)&=&\frac{\esp\left[\sum_{t=\tau_{i}\left(m\right)}^{\tau_{i}
\left(m+1\right)-1}z^{L_{i}\left(t\right)}\right]}{\esp\left[\tau_{i}\left(m+1\right)-\tau_{i}\left(m\right)\right]}\\
&=&\frac{1}{\esp\left[\tau_{i}\left(m+1\right)-\tau_{i}\left(m\right)\right]}
\left\{
\esp\left[\sum_{t=\tau_{i}\left(m\right)}^{\overline{\tau}_{i}\left(m\right)-1}
z^{L_{i}\left(t\right)}\right]
+\esp\left[\sum_{t=\overline{\tau}_{i}\left(m\right)}^{\tau_{i}\left(m+1\right)-1}
z^{L_{i}\left(t\right)}\right]\right\}\\
&=&\frac{1}{\esp\left[\tau_{i}\left(m+1\right)-\tau_{i}\left(m\right)\right]}
\left\{
z\frac{F_{i}\left(z\right)-1}{z-P_{i}\left(z\right)}+\frac{1-F_{i}\left(z\right)}
{1-P_{i}\left(z\right)}
\right\}
\end{eqnarray*}

es decir

\begin{equation}
Q_{i}\left(z\right)=\frac{1}{\esp\left[C_{i}\right]}\cdot\frac{1-F_{i}\left(z\right)}{P_{i}\left(z\right)-z}\cdot\frac{\left(1-z\right)P_{i}\left(z\right)}{1-P_{i}\left(z\right)}
\end{equation}

\begin{Teo}
Dada una Red de Sistemas de Visitas C\'iclicas (RSVC), conformada por dos Sistemas de Visitas C\'iclicas (SVC), donde cada uno de ellos consta de dos colas tipo $M/M/1$. Los dos sistemas est\'an comunicados entre s\'i por medio de la transferencia de usuarios entre las colas $Q_{1}\leftrightarrow Q_{3}$ y $Q_{2}\leftrightarrow Q_{4}$. Se definen los eventos para los procesos de arribos al tiempo $t$, $A_{j}\left(t\right)=\left\{0 \textrm{ arribos en }Q_{j}\textrm{ al tiempo }t\right\}$ para alg\'un tiempo $t\geq0$ y $Q_{j}$ la cola $j$-\'esima en la RSVC, para $j=1,2,3,4$.  Existe un intervalo $I\neq\emptyset$ tal que para $T^{*}\in I$, tal que $\prob\left\{A_{1}\left(T^{*}\right),A_{2}\left(Tt^{*}\right),
A_{3}\left(T^{*}\right),A_{4}\left(T^{*}\right)|T^{*}\in I\right\}>0$.
\end{Teo}

\begin{proof}
Sin p\'erdida de generalidad podemos considerar como base del an\'alisis a la cola $Q_{1}$ del primer sistema que conforma la RSVC.

Sea $n>0$, ciclo en el primer sistema en el que se sabe que $L_{j}\left(\overline{\tau}_{1}\left(n\right)\right)=0$, pues la pol\'itica de servicio con que atienden los servidores es la exhaustiva. Como es sabido, para trasladarse a la siguiente cola, el servidor incurre en un tiempo de traslado $r_{1}\left(n\right)>0$, entonces tenemos que $\tau_{2}\left(n\right)=\overline{\tau}_{1}\left(n\right)+r_{1}\left(n\right)$.


Definamos el intervalo $I_{1}\equiv\left[\overline{\tau}_{1}\left(n\right),\tau_{2}\left(n\right)\right]$ de longitud $\xi_{1}=r_{1}\left(n\right)$. Dado que los tiempos entre arribo son exponenciales con tasa $\tilde{\mu}_{1}=\mu_{1}+\hat{\mu}_{1}$ ($\mu_{1}$ son los arribos a $Q_{1}$ por primera vez al sistema, mientras que $\hat{\mu}_{1}$ son los arribos de traslado procedentes de $Q_{3}$) se tiene que la probabilidad del evento $A_{1}\left(t\right)$ est\'a dada por 

\begin{equation}
\prob\left\{A_{1}\left(t\right)|t\in I_{1}\left(n\right)\right\}=e^{-\tilde{\mu}_{1}\xi_{1}\left(n\right)}.
\end{equation} 

Por otra parte, para la cola $Q_{2}$, el tiempo $\overline{\tau}_{2}\left(n-1\right)$ es tal que $L_{2}\left(\overline{\tau}_{2}\left(n-1\right)\right)=0$, es decir, es el tiempo en que la cola queda totalmente vac\'ia en el ciclo anterior a $n$. Entonces tenemos un sgundo intervalo $I_{2}\equiv\left[\overline{\tau}_{2}\left(n-1\right),\tau_{2}\left(n\right)\right]$. Por lo tanto la probabilidad del evento $A_{2}\left(t\right)$ tiene probabilidad dada por

\begin{equation}
\prob\left\{A_{2}\left(t\right)|t\in I_{2}\left(n\right)\right\}=e^{-\tilde{\mu}_{2}\xi_{2}\left(n\right)},
\end{equation} 

donde $\xi_{2}\left(n\right)=\tau_{2}\left(n\right)-\overline{\tau}_{2}\left(n-1\right)$.



Entonces, se tiene que

\begin{eqnarray*}
\prob\left\{A_{1}\left(t\right),A_{2}\left(t\right)|t\in I_{1}\left(n\right)\right\}&=&
\prob\left\{A_{1}\left(t\right)|t\in I_{1}\left(n\right)\right\}
\prob\left\{A_{2}\left(t\right)|t\in I_{1}\left(n\right)\right\}\\
&\geq&
\prob\left\{A_{1}\left(t\right)|t\in I_{1}\left(n\right)\right\}
\prob\left\{A_{2}\left(t\right)|t\in I_{2}\left(n\right)\right\}\\
&=&e^{-\tilde{\mu}_{1}\xi_{1}\left(n\right)}e^{-\tilde{\mu}_{2}\xi_{2}\left(n\right)}
=e^{-\left[\tilde{\mu}_{1}\xi_{1}\left(n\right)+\tilde{\mu}_{2}\xi_{2}\left(n\right)\right]}.
\end{eqnarray*}


es decir, 

\begin{equation}
\prob\left\{A_{1}\left(t\right),A_{2}\left(t\right)|t\in I_{1}\left(n\right)\right\}
=e^{-\left[\tilde{\mu}_{1}\xi_{1}\left(n\right)+\tilde{\mu}_{2}\xi_{2}
\left(n\right)\right]}>0.
\end{equation}

En lo que respecta a la relaci\'on entre los dos SVC que conforman la RSVC, se afirma que existe $m>0$ tal que $\overline{\tau}_{3}\left(m\right)<\tau_{2}\left(n\right)<\tau_{4}\left(m\right)$.

Para $Q_{3}$ sea $I_{3}=\left[\overline{\tau}_{3}\left(m\right),\tau_{4}\left(m\right)\right]$ con longitud  $\xi_{3}\left(m\right)=r_{3}\left(m\right)$, entonces 

\begin{equation}
\prob\left\{A_{3}\left(t\right)|t\in I_{3}\left(n\right)\right\}=e^{-\tilde{\mu}_{3}\xi_{3}\left(n\right)}.
\end{equation} 

An\'alogamente que como se hizo para $Q_{2}$, tenemos que para $Q_{4}$ se tiene el intervalo $I_{4}=\left[\overline{\tau}_{4}\left(m-1\right),\tau_{4}\left(m\right)\right]$ con longitud $\xi_{4}\left(m\right)=\tau_{4}\left(m\right)-\overline{\tau}_{4}\left(m-1\right)$, entonces


\begin{equation}
\prob\left\{A_{4}\left(t\right)|t\in I_{4}\left(m\right)\right\}=e^{-\tilde{\mu}_{4}\xi_{4}\left(n\right)}.
\end{equation} 

Al igual que para el primer sistema, dado que $I_{3}\left(m\right)\subset I_{4}\left(m\right)$, se tiene que

\begin{eqnarray*}
\xi_{3}\left(m\right)\leq\xi_{4}\left(m\right)&\Leftrightarrow& -\xi_{3}\left(m\right)\geq-\xi_{4}\left(m\right)
\\
-\tilde{\mu}_{4}\xi_{3}\left(m\right)\geq-\tilde{\mu}_{4}\xi_{4}\left(m\right)&\Leftrightarrow&
e^{-\tilde{\mu}_{4}\xi_{3}\left(m\right)}\geq e^{-\tilde{\mu}_{4}\xi_{4}\left(m\right)}\\
\prob\left\{A_{4}\left(t\right)|t\in I_{3}\left(m\right)\right\}&\geq&
\prob\left\{A_{4}\left(t\right)|t\in I_{4}\left(m\right)\right\}
\end{eqnarray*}

Entonces, dado que los eventos $A_{3}$ y $A_{4}$ son independientes, se tiene que

\begin{eqnarray*}
\prob\left\{A_{3}\left(t\right),A_{4}\left(t\right)|t\in I_{3}\left(m\right)\right\}&=&
\prob\left\{A_{3}\left(t\right)|t\in I_{3}\left(m\right)\right\}
\prob\left\{A_{4}\left(t\right)|t\in I_{3}\left(m\right)\right\}\\
&\geq&
\prob\left\{A_{3}\left(t\right)|t\in I_{3}\left(n\right)\right\}
\prob\left\{A_{4}\left(t\right)|t\in I_{4}\left(n\right)\right\}\\
&=&e^{-\tilde{\mu}_{3}\xi_{3}\left(m\right)}e^{-\tilde{\mu}_{4}\xi_{4}
\left(m\right)}
=e^{-\left[\tilde{\mu}_{3}\xi_{3}\left(m\right)+\tilde{\mu}_{4}\xi_{4}
\left(m\right)\right]}.
\end{eqnarray*}


es decir, 

\begin{equation}
\prob\left\{A_{3}\left(t\right),A_{4}\left(t\right)|t\in I_{3}\left(m\right)\right\}
=e^{-\left[\tilde{\mu}_{3}\xi_{3}\left(m\right)+\tilde{\mu}_{4}\xi_{4}
\left(m\right)\right]}>0.
\end{equation}

Por construcci\'on se tiene que $I\left(n,m\right)\equiv I_{1}\left(n\right)\cap I_{3}\left(m\right)\neq\emptyset$,entonces en particular se tienen las contenciones $I\left(n,m\right)\subseteq I_{1}\left(n\right)$ y $I\left(n,m\right)\subseteq I_{3}\left(m\right)$, por lo tanto si definimos $\xi_{n,m}\equiv\ell\left(I\left(n,m\right)\right)$ tenemos que

\begin{eqnarray*}
\xi_{n,m}\leq\xi_{1}\left(n\right)\textrm{ y }\xi_{n,m}\leq\xi_{3}\left(m\right)\textrm{ entonces }
-\xi_{n,m}\geq-\xi_{1}\left(n\right)\textrm{ y }-\xi_{n,m}\leq-\xi_{3}\left(m\right)\\
\end{eqnarray*}
por lo tanto tenemos las desigualdades 



\begin{eqnarray*}
\begin{array}{ll}
-\tilde{\mu}_{1}\xi_{n,m}\geq-\tilde{\mu}_{1}\xi_{1}\left(n\right),&
-\tilde{\mu}_{2}\xi_{n,m}\geq-\tilde{\mu}_{2}\xi_{1}\left(n\right)
\geq-\tilde{\mu}_{2}\xi_{2}\left(n\right),\\
-\tilde{\mu}_{3}\xi_{n,m}\geq-\tilde{\mu}_{3}\xi_{3}\left(m\right),&
-\tilde{\mu}_{4}\xi_{n,m}\geq-\tilde{\mu}_{4}\xi_{3}\left(m\right)
\geq-\tilde{\mu}_{4}\xi_{4}\left(m\right).
\end{array}
\end{eqnarray*}

Sea $T^{*}\in I_{n,m}$, entonces dado que en particular $T^{*}\in I_{1}\left(n\right)$ se cumple con probabilidad positiva que no hay arribos a las colas $Q_{1}$ y $Q_{2}$, en consecuencia, tampoco hay usuarios de transferencia para $Q_{3}$ y $Q_{4}$, es decir, $\tilde{\mu}_{1}=\mu_{1}$, $\tilde{\mu}_{2}=\mu_{2}$, $\tilde{\mu}_{3}=\mu_{3}$, $\tilde{\mu}_{4}=\mu_{4}$, es decir, los eventos $Q_{1}$ y $Q_{3}$ son condicionalmente independientes en el intervalo $I_{n,m}$; lo mismo ocurre para las colas $Q_{2}$ y $Q_{4}$, por lo tanto tenemos que


\begin{eqnarray}
\begin{array}{l}
\prob\left\{A_{1}\left(T^{*}\right),A_{2}\left(T^{*}\right),
A_{3}\left(T^{*}\right),A_{4}\left(T^{*}\right)|T^{*}\in I_{n,m}\right\}
=\prod_{j=1}^{4}\prob\left\{A_{j}\left(T^{*}\right)|T^{*}\in I_{n,m}\right\}\\
\geq\prob\left\{A_{1}\left(T^{*}\right)|T^{*}\in I_{1}\left(n\right)\right\}
\prob\left\{A_{2}\left(T^{*}\right)|T^{*}\in I_{2}\left(n\right)\right\}
\prob\left\{A_{3}\left(T^{*}\right)|T^{*}\in I_{3}\left(m\right)\right\}
\prob\left\{A_{4}\left(T^{*}\right)|T^{*}\in I_{4}\left(m\right)\right\}\\
=e^{-\mu_{1}\xi_{1}\left(n\right)}
e^{-\mu_{2}\xi_{2}\left(n\right)}
e^{-\mu_{3}\xi_{3}\left(m\right)}
e^{-\mu_{4}\xi_{4}\left(m\right)}
=e^{-\left[\tilde{\mu}_{1}\xi_{1}\left(n\right)
+\tilde{\mu}_{2}\xi_{2}\left(n\right)
+\tilde{\mu}_{3}\xi_{3}\left(m\right)
+\tilde{\mu}_{4}\xi_{4}
\left(m\right)\right]}>0.
\end{array}
\end{eqnarray}
\end{proof}


Estos resultados aparecen en Daley (1968) \cite{Daley68} para $\left\{T_{n}\right\}$ intervalos de inter-arribo, $\left\{D_{n}\right\}$ intervalos de inter-salida y $\left\{S_{n}\right\}$ tiempos de servicio.

\begin{itemize}
\item Si el proceso $\left\{T_{n}\right\}$ es Poisson, el proceso $\left\{D_{n}\right\}$ es no correlacionado si y s\'olo si es un proceso Poisso, lo cual ocurre si y s\'olo si $\left\{S_{n}\right\}$ son exponenciales negativas.

\item Si $\left\{S_{n}\right\}$ son exponenciales negativas, $\left\{D_{n}\right\}$ es un proceso de renovaci\'on  si y s\'olo si es un proceso Poisson, lo cual ocurre si y s\'olo si $\left\{T_{n}\right\}$ es un proceso Poisson.

\item $\esp\left(D_{n}\right)=\esp\left(T_{n}\right)$.

\item Para un sistema de visitas $GI/M/1$ se tiene el siguiente teorema:

\begin{Teo}
En un sistema estacionario $GI/M/1$ los intervalos de interpartida tienen
\begin{eqnarray*}
\esp\left(e^{-\theta D_{n}}\right)&=&\mu\left(\mu+\theta\right)^{-1}\left[\delta\theta
-\mu\left(1-\delta\right)\alpha\left(\theta\right)\right]
\left[\theta-\mu\left(1-\delta\right)^{-1}\right]\\
\alpha\left(\theta\right)&=&\esp\left[e^{-\theta T_{0}}\right]\\
var\left(D_{n}\right)&=&var\left(T_{0}\right)-\left(\tau^{-1}-\delta^{-1}\right)
2\delta\left(\esp\left(S_{0}\right)\right)^{2}\left(1-\delta\right)^{-1}.
\end{eqnarray*}
\end{Teo}



\begin{Teo}
El proceso de salida de un sistema de colas estacionario $GI/M/1$ es un proceso de renovaci\'on si y s\'olo si el proceso de entrada es un proceso Poisson, en cuyo caso el proceso de salida es un proceso Poisson.
\end{Teo}


\begin{Teo}
Los intervalos de interpartida $\left\{D_{n}\right\}$ de un sistema $M/G/1$ estacionario son no correlacionados si y s\'olo si la distribuci\'on de los tiempos de servicio es exponencial negativa, es decir, el sistema es de tipo  $M/M/1$.

\end{Teo}



\end{itemize}


\section{Resultados para Procesos de Salida}

En Sigman, Thorison y Wolff \cite{Sigman2} prueban que para la existencia de un una sucesi\'on infinita no decreciente de tiempos de regeneraci\'on $\tau_{1}\leq\tau_{2}\leq\cdots$ en los cuales el proceso se regenera, basta un tiempo de regeneraci\'on $R_{1}$, donde $R_{j}=\tau_{j}-\tau_{j-1}$. Para tal efecto se requiere la existencia de un espacio de probabilidad $\left(\Omega,\mathcal{F},\prob\right)$, y proceso estoc\'astico $\textit{X}=\left\{X\left(t\right):t\geq0\right\}$ con espacio de estados $\left(S,\mathcal{R}\right)$, con $\mathcal{R}$ $\sigma$-\'algebra.

\begin{Prop}
Si existe una variable aleatoria no negativa $R_{1}$ tal que $\theta_{R\footnotesize{1}}X=_{D}X$, entonces $\left(\Omega,\mathcal{F},\prob\right)$ puede extenderse para soportar una sucesi\'on estacionaria de variables aleatorias $R=\left\{R_{k}:k\geq1\right\}$, tal que para $k\geq1$,
\begin{eqnarray*}
\theta_{k}\left(X,R\right)=_{D}\left(X,R\right).
\end{eqnarray*}

Adem\'as, para $k\geq1$, $\theta_{k}R$ es condicionalmente independiente de $\left(X,R_{1},\ldots,R_{k}\right)$, dado $\theta_{\tau k}X$.

\end{Prop}


\begin{itemize}
\item Doob en 1953 demostr\'o que el estado estacionario de un proceso de partida en un sistema de espera $M/G/\infty$, es Poisson con la misma tasa que el proceso de arribos.

\item Burke en 1968, fue el primero en demostrar que el estado estacionario de un proceso de salida de una cola $M/M/s$ es un proceso Poisson.

\item Disney en 1973 obtuvo el siguiente resultado:

\begin{Teo}
Para el sistema de espera $M/G/1/L$ con disciplina FIFO, el proceso $\textbf{I}$ es un proceso de renovaci\'on si y s\'olo si el proceso denominado longitud de la cola es estacionario y se cumple cualquiera de los siguientes casos:

\begin{itemize}
\item[a)] Los tiempos de servicio son identicamente cero;
\item[b)] $L=0$, para cualquier proceso de servicio $S$;
\item[c)] $L=1$ y $G=D$;
\item[d)] $L=\infty$ y $G=M$.
\end{itemize}
En estos casos, respectivamente, las distribuciones de interpartida $P\left\{T_{n+1}-T_{n}\leq t\right\}$ son


\begin{itemize}
\item[a)] $1-e^{-\lambda t}$, $t\geq0$;
\item[b)] $1-e^{-\lambda t}*F\left(t\right)$, $t\geq0$;
\item[c)] $1-e^{-\lambda t}*\indora_{d}\left(t\right)$, $t\geq0$;
\item[d)] $1-e^{-\lambda t}*F\left(t\right)$, $t\geq0$.
\end{itemize}
\end{Teo}


\item Finch (1959) mostr\'o que para los sistemas $M/G/1/L$, con $1\leq L\leq \infty$ con distribuciones de servicio dos veces diferenciable, solamente el sistema $M/M/1/\infty$ tiene proceso de salida de renovaci\'on estacionario.

\item King (1971) demostro que un sistema de colas estacionario $M/G/1/1$ tiene sus tiempos de interpartida sucesivas $D_{n}$ y $D_{n+1}$ son independientes, si y s\'olo si, $G=D$, en cuyo caso le proceso de salida es de renovaci\'on.

\item Disney (1973) demostr\'o que el \'unico sistema estacionario $M/G/1/L$, que tiene proceso de salida de renovaci\'on  son los sistemas $M/M/1$ y $M/D/1/1$.



\item El siguiente resultado es de Disney y Koning (1985)
\begin{Teo}
En un sistema de espera $M/G/s$, el estado estacionario del proceso de salida es un proceso Poisson para cualquier distribuci\'on de los tiempos de servicio si el sistema tiene cualquiera de las siguientes cuatro propiedades.

\begin{itemize}
\item[a)] $s=\infty$
\item[b)] La disciplina de servicio es de procesador compartido.
\item[c)] La disciplina de servicio es LCFS y preemptive resume, esto se cumple para $L<\infty$
\item[d)] $G=M$.
\end{itemize}

\end{Teo}

\item El siguiente resultado es de Alamatsaz (1983)

\begin{Teo}
En cualquier sistema de colas $GI/G/1/L$ con $1\leq L<\infty$ y distribuci\'on de interarribos $A$ y distribuci\'on de los tiempos de servicio $B$, tal que $A\left(0\right)=0$, $A\left(t\right)\left(1-B\left(t\right)\right)>0$ para alguna $t>0$ y $B\left(t\right)$ para toda $t>0$, es imposible que el proceso de salida estacionario sea de renovaci\'on.
\end{Teo}

\end{itemize}

Estos resultados aparecen en Daley (1968) \cite{Daley68} para $\left\{T_{n}\right\}$ intervalos de inter-arribo, $\left\{D_{n}\right\}$ intervalos de inter-salida y $\left\{S_{n}\right\}$ tiempos de servicio.

\begin{itemize}
\item Si el proceso $\left\{T_{n}\right\}$ es Poisson, el proceso $\left\{D_{n}\right\}$ es no correlacionado si y s\'olo si es un proceso Poisso, lo cual ocurre si y s\'olo si $\left\{S_{n}\right\}$ son exponenciales negativas.

\item Si $\left\{S_{n}\right\}$ son exponenciales negativas, $\left\{D_{n}\right\}$ es un proceso de renovaci\'on  si y s\'olo si es un proceso Poisson, lo cual ocurre si y s\'olo si $\left\{T_{n}\right\}$ es un proceso Poisson.

\item $\esp\left(D_{n}\right)=\esp\left(T_{n}\right)$.

\item Para un sistema de visitas $GI/M/1$ se tiene el siguiente teorema:

\begin{Teo}
En un sistema estacionario $GI/M/1$ los intervalos de interpartida tienen
\begin{eqnarray*}
\esp\left(e^{-\theta D_{n}}\right)&=&\mu\left(\mu+\theta\right)^{-1}\left[\delta\theta
-\mu\left(1-\delta\right)\alpha\left(\theta\right)\right]
\left[\theta-\mu\left(1-\delta\right)^{-1}\right]\\
\alpha\left(\theta\right)&=&\esp\left[e^{-\theta T_{0}}\right]\\
var\left(D_{n}\right)&=&var\left(T_{0}\right)-\left(\tau^{-1}-\delta^{-1}\right)
2\delta\left(\esp\left(S_{0}\right)\right)^{2}\left(1-\delta\right)^{-1}.
\end{eqnarray*}
\end{Teo}



\begin{Teo}
El proceso de salida de un sistema de colas estacionario $GI/M/1$ es un proceso de renovaci\'on si y s\'olo si el proceso de entrada es un proceso Poisson, en cuyo caso el proceso de salida es un proceso Poisson.
\end{Teo}


\begin{Teo}
Los intervalos de interpartida $\left\{D_{n}\right\}$ de un sistema $M/G/1$ estacionario son no correlacionados si y s\'olo si la distribuci\'on de los tiempos de servicio es exponencial negativa, es decir, el sistema es de tipo  $M/M/1$.

\end{Teo}



\end{itemize}
%\newpage
%________________________________________________________________________
%\section{Redes de Sistemas de Visitas C\'iclicas}
%________________________________________________________________________

Sean $Q_{1},Q_{2},Q_{3}$ y $Q_{4}$ en una Red de Sistemas de Visitas C\'iclicas (RSVC). Supongamos que cada una de las colas es del tipo $M/M/1$ con tasa de arribo $\mu_{i}$ y que la transferencia de usuarios entre los dos sistemas ocurre entre $Q_{1}\leftrightarrow Q_{3}$ y $Q_{2}\leftrightarrow Q_{4}$ con respectiva tasa de arribo igual a la tasa de salida $\hat{\mu}_{i}=\mu_{i}$, esto se sabe por lo desarrollado en la secci\'on anterior.  

Consideremos, sin p\'erdida de generalidad como base del an\'alisis, la cola $Q_{1}$ adem\'as supongamos al servidor lo comenzamos a observar una vez que termina de atender a la misma para desplazarse y llegar a $Q_{2}$, es decir al tiempo $\tau_{2}$.

Sea $n\in\nat$, $n>0$, ciclo del servidor en que regresa a $Q_{1}$ para dar servicio y atender conforme a la pol\'itica exhaustiva, entonces se tiene que $\overline{\tau}_{1}\left(n\right)$ es el tiempo del servidor en el sistema 1 en que termina de dar servicio a todos los usuarios presentes en la cola, por lo tanto se cumple que $L_{1}\left(\overline{\tau}_{1}\left(n\right)\right)=0$, entonces el servidor para llegar a $Q_{2}$ incurre en un tiempo de traslado $r_{1}$ y por tanto se cumple que $\tau_{2}\left(n\right)=\overline{\tau}_{1}\left(n\right)+r_{1}$. Dado que los tiempos entre arribos son exponenciales se cumple que 

\begin{eqnarray*}
\prob\left\{0 \textrm{ arribos en }Q_{1}\textrm{ en el intervalo }\left[\overline{\tau}_{1}\left(n\right),\overline{\tau}_{1}\left(n\right)+r_{1}\right]\right\}=e^{-\tilde{\mu}_{1}r_{1}},\\
\prob\left\{0 \textrm{ arribos en }Q_{2}\textrm{ en el intervalo }\left[\overline{\tau}_{1}\left(n\right),\overline{\tau}_{1}\left(n\right)+r_{1}\right]\right\}=e^{-\tilde{\mu}_{2}r_{1}}.
\end{eqnarray*}

El evento que nos interesa consiste en que no haya arribos desde que el servidor abandon\'o $Q_{2}$ y regresa nuevamente para dar servicio, es decir en el intervalo de tiempo $\left[\overline{\tau}_{2}\left(n-1\right),\tau_{2}\left(n\right)\right]$. Entonces, si hacemos


\begin{eqnarray*}
\varphi_{1}\left(n\right)&\equiv&\overline{\tau}_{1}\left(n\right)+r_{1}=\overline{\tau}_{2}\left(n-1\right)+r_{1}+r_{2}+\overline{\tau}_{1}\left(n\right)-\tau_{1}\left(n\right)\\
&=&\overline{\tau}_{2}\left(n-1\right)+\overline{\tau}_{1}\left(n\right)-\tau_{1}\left(n\right)+r,,
\end{eqnarray*}

y longitud del intervalo

\begin{eqnarray*}
\xi&\equiv&\overline{\tau}_{1}\left(n\right)+r_{1}-\overline{\tau}_{2}\left(n-1\right)
=\overline{\tau}_{2}\left(n-1\right)+\overline{\tau}_{1}\left(n\right)-\tau_{1}\left(n\right)+r-\overline{\tau}_{2}\left(n-1\right)\\
&=&\overline{\tau}_{1}\left(n\right)-\tau_{1}\left(n\right)+r.
\end{eqnarray*}


Entonces, determinemos la probabilidad del evento no arribos a $Q_{2}$ en $\left[\overline{\tau}_{2}\left(n-1\right),\varphi_{1}\left(n\right)\right]$:

\begin{eqnarray}
\prob\left\{0 \textrm{ arribos en }Q_{2}\textrm{ en el intervalo }\left[\overline{\tau}_{2}\left(n-1\right),\varphi_{1}\left(n\right)\right]\right\}
=e^{-\tilde{\mu}_{2}\xi}.
\end{eqnarray}

De manera an\'aloga, tenemos que la probabilidad de no arribos a $Q_{1}$ en $\left[\overline{\tau}_{2}\left(n-1\right),\varphi_{1}\left(n\right)\right]$ esta dada por

\begin{eqnarray}
\prob\left\{0 \textrm{ arribos en }Q_{1}\textrm{ en el intervalo }\left[\overline{\tau}_{2}\left(n-1\right),\varphi_{1}\left(n\right)\right]\right\}
=e^{-\tilde{\mu}_{1}\xi},
\end{eqnarray}

\begin{eqnarray}
\prob\left\{0 \textrm{ arribos en }Q_{2}\textrm{ en el intervalo }\left[\overline{\tau}_{2}\left(n-1\right),\varphi_{1}\left(n\right)\right]\right\}
=e^{-\tilde{\mu}_{2}\xi}.
\end{eqnarray}

Por tanto 

\begin{eqnarray}
\begin{array}{l}
\prob\left\{0 \textrm{ arribos en }Q_{1}\textrm{ y }Q_{2}\textrm{ en el intervalo }\left[\overline{\tau}_{2}\left(n-1\right),\varphi_{1}\left(n\right)\right]\right\}\\
=\prob\left\{0 \textrm{ arribos en }Q_{1}\textrm{ en el intervalo }\left[\overline{\tau}_{2}\left(n-1\right),\varphi_{1}\left(n\right)\right]\right\}\\
\times
\prob\left\{0 \textrm{ arribos en }Q_{2}\textrm{ en el intervalo }\left[\overline{\tau}_{2}\left(n-1\right),\varphi_{1}\left(n\right)\right]\right\}=e^{-\tilde{\mu}_{1}\xi}e^{-\tilde{\mu}_{2}\xi}
=e^{-\tilde{\mu}\xi}.
\end{array}
\end{eqnarray}

Para el segundo sistema, consideremos nuevamente $\overline{\tau}_{1}\left(n\right)+r_{1}$, sin p\'erdida de generalidad podemos suponer que existe $m>0$ tal que $\overline{\tau}_{3}\left(m\right)<\overline{\tau}_{1}+r_{1}<\tau_{4}\left(m\right)$, entonces

\begin{eqnarray}
\prob\left\{0 \textrm{ arribos en }Q_{3}\textrm{ en el intervalo }\left[\overline{\tau}_{3}\left(m\right),\overline{\tau}_{1}\left(n\right)+r_{1}\right]\right\}
=e^{-\tilde{\mu}_{3}\xi_{3}},
\end{eqnarray}
donde 
\begin{eqnarray}
\xi_{3}=\overline{\tau}_{1}\left(n\right)+r_{1}-\overline{\tau}_{3}\left(m\right)=
\overline{\tau}_{1}\left(n\right)-\overline{\tau}_{3}\left(m\right)+r_{1},
\end{eqnarray}

mientras que para $Q_{4}$ al igual que con $Q_{2}$ escribiremos $\tau_{4}\left(m\right)$ en t\'erminos de $\overline{\tau}_{4}\left(m-1\right)$:

$\varphi_{2}\equiv\tau_{4}\left(m\right)=\overline{\tau}_{4}\left(m-1\right)+r_{4}+\overline{\tau}_{3}\left(m\right)
-\tau_{3}\left(m\right)+r_{3}=\overline{\tau}_{4}\left(m-1\right)+\overline{\tau}_{3}\left(m\right)
-\tau_{3}\left(m\right)+\hat{r}$, adem\'as,

$\xi_{2}\equiv\varphi_{2}\left(m\right)-\overline{\tau}_{1}-r_{1}=\overline{\tau}_{4}\left(m-1\right)+\overline{\tau}_{3}\left(m\right)
-\tau_{3}\left(m\right)-\overline{\tau}_{1}\left(n\right)+\hat{r}-r_{1}$. 

Entonces


\begin{eqnarray}
\prob\left\{0 \textrm{ arribos en }Q_{4}\textrm{ en el intervalo }\left[\overline{\tau}_{1}\left(n\right)+r_{1},\varphi_{2}\left(m\right)\right]\right\}
=e^{-\tilde{\mu}_{4}\xi_{2}},
\end{eqnarray}

mientras que para $Q_{3}$ se tiene que 

\begin{eqnarray}
\prob\left\{0 \textrm{ arribos en }Q_{3}\textrm{ en el intervalo }\left[\overline{\tau}_{1}\left(n\right)+r_{1},\varphi_{2}\left(m\right)\right]\right\}
=e^{-\tilde{\mu}_{3}\xi_{2}}
\end{eqnarray}

Por tanto

\begin{eqnarray}
\prob\left\{0 \textrm{ arribos en }Q_{3}\wedge Q_{4}\textrm{ en el intervalo }\left[\overline{\tau}_{1}\left(n\right)+r_{1},\varphi_{2}\left(m\right)\right]\right\}
=e^{-\hat{\mu}\xi_{2}}
\end{eqnarray}
donde $\hat{\mu}=\tilde{\mu}_{3}+\tilde{\mu}_{4}$.

Ahora, definamos los intervalos $\mathcal{I}_{1}=\left[\overline{\tau}_{1}\left(n\right)+r_{1},\varphi_{1}\left(n\right)\right]$  y $\mathcal{I}_{2}=\left[\overline{\tau}_{1}\left(n\right)+r_{1},\varphi_{2}\left(m\right)\right]$, entonces, sea $\mathcal{I}=\mathcal{I}_{1}\cap\mathcal{I}_{2}$ el intervalo donde cada una de las colas se encuentran vac\'ias, es decir, si tomamos $T^{*}\in\mathcal{I}$, entonces  $L_{1}\left(T^{*}\right)=L_{2}\left(T^{*}\right)=L_{3}\left(T^{*}\right)=L_{4}\left(T^{*}\right)=0$.

Ahora, dado que por construcci\'on $\mathcal{I}\neq\emptyset$ y que para $T^{*}\in\mathcal{I}$ en ninguna de las colas han llegado usuarios, se tiene que no hay transferencia entre las colas, por lo tanto, el sistema 1 y el sistema 2 son condicionalmente independientes en $\mathcal{I}$, es decir

\begin{eqnarray}
\prob\left\{L_{1}\left(T^{*}\right),L_{2}\left(T^{*}\right),L_{3}\left(T^{*}\right),L_{4}\left(T^{*}\right)|T^{*}\in\mathcal{I}\right\}=\prod_{j=1}^{4}\prob\left\{L_{j}\left(T^{*}\right)\right\},
\end{eqnarray}

para $T^{*}\in\mathcal{I}$. 

%\newpage























%________________________________________________________
%
\section{Convergencia}
%___________________________________________________________________________________________
%

%___________________________________________________________________________________________
%
\subsection{Billingsley: Espacios Producto}
%___________________________________________________________________________________________
%


Sea $S=S^{'}\times S^{''}$ el espacio producto de los espacios m\'etricos $S^{'}$ y $S^{''}$. Si $S$ es separable, que a su vez requiere que ambos espacios sean separables, entonces las $\sigma$-\'algebras $\mathcal{S}$, $\mathcal{S}^{'}$ y $\mathcal{S}^{''}$ de los conjuntos de Borel en este espacio est\'an relacionados por $\mathcal{S}=\mathcal{S}^{'}\times \mathcal{S}^{'}$, 
%___________________________________________________________________________________________
\section{Funcion Generadora de Probabilidades Conjunta}
%___________________________________________________________________________________________

Sea $T>0$, y sea $\left\{T_{j}\right\}$ partici\'on del intervalo $\left[0,T\right]$. Sea $T_{k}$ elemento arbitrario de la partici\'on. Sean $m$ y $n$ ciclos del servidor, se definen los siguientes procesos:


\begin{itemize}
\item $L_{j}\left(t\right)$ para la longitud de la cola al tiempo $T_{k}$.

\item $A_{j}\left(t\right)$ el proceso de los residuales de los tiempos de arribo para el siguiente usuario al tiempo $T_{k}$.


\item $B_{j}\left(t\right)$ el proceso de los residuales de los tiempos de servicio al tiempo $T_{k}$.


\item $B_{j}^{0}\left(t\right)$ el proceso de los residuales de los tiempos de traslado del servidor al tiempo $T_{k}$.


\item $C_{j}\left(t\right)$ el n\'umero de usuarios atendidos por el servidor al tiempo $T_{k}$.
\end{itemize}


Entonces se tiene la siguiente funci\'on generadora de probabilidades


\begin{eqnarray}
\mathcal{G}=\esp\left[\prod_{j=1}^{4}L_{j}\left(T_{k}\right)\prod_{j=1}^{4}A_{j}\left(T_{k}\right)\prod_{j=1}^{4}B_{j}\left(T_{k}\right)\prod_{j=1}^{4}B_{j}^{0}\left(T_{k}\right)
\prod_{j=1}^{4}C_{j}\left(T_{k}\right)\right]
\end{eqnarray}

Para $T_{k}$ se tienen los siguientes casos:
\begin{multicols}{2}
\begin{enumerate}
\item $T_{k}\leq\overline{\tau}_{1}\left(m\right)$
\item $\overline{\tau}_{1}\left(m\right)<T_{k}\leq \tau_{2}\left(m\right)$
\item $\overline{\tau}_{2}\left(m\right)<T_{k}\leq \overline{\tau}_{2}\left(m\right)$
\item $\tau_{2}\left(m\right)<T_{k}$
\end{enumerate}
\end{multicols}

Lo cual nos dan los casos que enunciamos de manera exhaustiva a continuaci\'on:
\begin{itemize}
%_____________________________________________________________________________________________
\item Primer caso
%_____________________________________________________________________________________________
\begin{itemize}
\item[1.a)] $T_{k}\leq\overline{\tau}_{1}\left(m\right)$ y $T_{k}\leq\overline{\tau}_{3}\left(n\right)$:
\begin{eqnarray*}
L_{1}\left(T_{k}\right)&=&L_{1}\left(\tau_{1}\left(m\right)\right)+X_{1}\left(T_{k}-\tau_{1}\left(m\right)\right)+Y_{1}
\left(T_{k}-\tau_{1}\left(m\right)\right)-C_{1}\left(T_{k}\right)\\
L_{2}\left(T_{k}\right)&=&X_{2}\left(T_{k}-\overline{\tau}_{2}\left(m-1\right)\right)\\
L_{3}\left(T_{k}\right)&=&L_{3}\left(\tau_{3}\left(n\right)\right)+X_{3}\left(T_{k}-\tau_{3}\left(n\right)\right)+Y_{3}\left(T_{k}-\tau_{3}\left(n\right)\right)-C_{3}\left(T_{k}\right)\\
L_{4}\left(T_{k}\right)&=&X_{4}\left(T_{k}-\overline{\tau}_{4}\left(n-1\right)\right)
\end{eqnarray*}


\item[1.b)] $T_{k}\leq\overline{\tau}_{1}\left(m\right)$ y $\overline{\tau}_{3}\left(n\right)<T_{k}\leq\tau_{4}\left(n\right)$:
\begin{eqnarray*}
L_{1}\left(T_{k}\right)&=&L_{1}\left(\tau_{1}\left(m\right)\right)+X_{1}\left(T_{k}-\tau_{1}\left(m\right)\right)+Y_{1}
\left(T_{k}-\tau_{1}\left(m\right)\right)-C_{1}\left(T_{k}\right)\\
L_{2}\left(T_{k}\right)&=&X_{2}\left(T_{k}-\overline{\tau}_{2}\left(m-1\right)\right)\\
L_{3}\left(T_{k}\right)&=&X_{3}\left(T_{k}-\overline{\tau}_{3}\left(n\right)\right)\\
L_{4}\left(T_{k}\right)&=&L_{4}\left(\tau_{4}\left(n\right)\right)
\end{eqnarray*}

\item[1.c)] $T_{k}\leq\overline{\tau}_{1}\left(m\right)$ y $\tau_{4}\left(n\right)<T_{k}\leq\overline{\tau}_{4}\left(n\right)$:
\begin{eqnarray*}
L_{1}\left(T_{k}\right)&=&L_{1}\left(\tau_{1}\left(m\right)\right)+X_{1}\left(T_{k}-\tau_{1}\left(m\right)\right)+Y_{1}
\left(T_{k}-\tau_{1}\left(m\right)\right)-C_{1}\left(T_{k}\right)\\
L_{2}\left(T_{k}\right)&=&X_{2}\left(T_{k}-\overline{\tau}_{2}\left(m-1\right)\right)\\
L_{3}\left(T_{k}\right)&=&X_{3}\left(T_{k}-\overline{\tau}_{3}\left(n\right)\right)\\
L_{4}\left(T_{k}\right)&=&L_{4}\left(\tau_{4}\left(n\right)\right)+X_{4}\left(T_{k}-\tau_{4}\left(n\right)\right)+Y_{4}
\left(T_{k}-\tau_{4}\left(n\right)\right)-C_{4}\left(T_{k}\right)
\end{eqnarray*}

\item[1.d)] $T_{k}\leq\overline{\tau}_{1}\left(m\right)$ y $\overline{\tau}_{4}\left(n\right)<T_{k}$:
\begin{eqnarray*}
L_{1}\left(T_{k}\right)&=&L_{1}\left(\tau_{1}\left(m\right)\right)+X_{1}\left(T_{k}-\tau_{1}\left(m\right)\right)+Y_{1}
\left(T_{k}-\tau_{1}\left(m\right)\right)-C_{1}\left(T_{k}\right)\\
L_{2}\left(T_{k}\right)&=&X_{2}\left(T_{k}-\overline{\tau}_{2}\left(m-1\right)\right)\\
L_{3}\left(T_{k}\right)&=&X_{3}\left(T_{k}-\overline{\tau}_{3}\left(n\right)\right)\\
L_{4}\left(T_{k}\right)&=&X_{4}\left(T_{k}-\overline{\tau}_{4}\left(n\right)\right)
\end{eqnarray*}

\end{itemize}
%_____________________________________________________________________________________________
\item Segundo caso
%_____________________________________________________________________________________________
\begin{itemize}
\item[2.a)] $\overline{\tau}_{1}\left(m\right)<T_{k}\leq\tau_{2}\left(m\right)$ y $T_{k}\leq\overline{\tau}_{3}\left(n\right)$:
\begin{eqnarray*}
L_{1}\left(T_{k}\right)&=&X_{1}\left(T_{k}-\overline{\tau}_{1}\left(m\right)\right)\\
L_{2}\left(T_{k}\right)&=&L_{2}\left(\tau_{2}\left(m\right)\right)\\
L_{3}\left(T_{k}\right)&=&L_{3}\left(\tau_{3}\left(n\right)\right)+X_{3}\left(T_{k}-\tau_{3}\left(n\right)\right)+Y_{3}\left(T_{k}-\tau_{3}\left(n\right)\right)-C_{3}\left(T_{k}\right)\\
L_{4}\left(T_{k}\right)&=&X_{4}\left(T_{k}-\overline{\tau}_{4}\left(n-1\right)\right)
\end{eqnarray*}

\item[2.b)] $\overline{\tau}_{1}\left(m\right)<T_{k}\leq\tau_{2}\left(m\right)$ y $\overline{\tau}_{3}\left(n\right)<T_{k}\leq\tau_{4}\left(n\right)$:
\begin{eqnarray*}
L_{1}\left(T_{k}\right)&=&X_{1}\left(T_{k}-\overline{\tau}_{1}\left(m\right)\right)\\
L_{2}\left(T_{k}\right)&=&L_{2}\left(\tau_{2}\left(m\right)\right)\\
L_{3}\left(T_{k}\right)&=&X_{3}\left(T_{k}-\overline{\tau}_{3}\left(n\right)\right)\\
L_{4}\left(T_{k}\right)&=&L_{4}\left(\tau_{4}\left(n\right)\right)
\end{eqnarray*}


\item[2.c)] $\overline{\tau}_{1}\left(m\right)<T_{k}\leq\tau_{2}\left(m\right)$ y $\tau_{4}\left(n\right)<T_{k}\leq\overline{\tau}_{4}\left(n\right)$:
\begin{eqnarray*}
L_{1}\left(T_{k}\right)&=&X_{1}\left(T_{k}-\overline{\tau}_{1}\left(m\right)\right)\\
L_{2}\left(T_{k}\right)&=&L_{2}\left(\tau_{2}\left(m\right)\right)\\
L_{3}\left(T_{k}\right)&=&X_{3}\left(T_{k}-\overline{\tau}_{3}\left(m\right)\right)\\
L_{4}\left(T_{k}\right)&=&L_{4}\left(\tau_{4}\left(n\right)\right)+X_{4}\left(T_{k}-\tau_{4}\left(n\right)\right)+Y_{4}
\left(T_{k}-\tau_{4}\left(n\right)\right)-C_{4}\left(T_{k}\right)
\end{eqnarray*}

\item[2.d)] $\overline{\tau}_{1}\left(m\right)<T_{k}\leq\tau_{2}\left(m\right)$ y $\overline{\tau}_{4}\left(n\right)<T_{k}$:
\begin{eqnarray*}
L_{1}\left(T_{k}\right)&=&X_{1}\left(T_{k}-\overline{\tau}_{1}\left(m\right)\right)\\
L_{2}\left(T_{k}\right)&=&L_{2}\left(\tau_{2}\left(m\right)\right)\\
L_{3}\left(T_{k}\right)&=&X_{3}\left(T_{k}-\overline{\tau}_{3}\left(n\right)\right)\\
L_{4}\left(T_{k}\right)&=&X_{4}\left(T_{k}-\overline{\tau}_{4}\left(n\right)\right)
\end{eqnarray*}
\end{itemize}

%_____________________________________________________________________________________________
\item Tercer caso
%_____________________________________________________________________________________________
\begin{itemize}
\item[3.a)] $\tau_{2}\left(m\right)<T_{k}\leq \overline{\tau}_{2}\left(m\right)$ y $T_{k}\leq\overline{\tau}_{3}\left(n\right)$:
\begin{eqnarray*}
L_{1}\left(T_{k}\right)&=&X_{1}\left(T_{k}-\overline{\tau}_{1}\left(m\right)\right)\\
L_{2}\left(T_{k}\right)&=&L_{2}\left(\tau_{2}\left(m\right)\right)+X_{2}\left(T_{k}-\tau_{2}\left(m\right)\right)+Y_{2}
\left(T_{k}-\tau_{2}\left(m\right)\right)-C_{2}\left(T_{k}\right)\\
L_{3}\left(T_{k}\right)&=&L_{3}\left(\tau_{3}\left(n\right)\right)+X_{3}\left(T_{k}-\tau_{3}\left(n\right)\right)+Y_{3}\left(T_{k}-\tau_{3}\left(n\right)\right)-C_{3}\left(T_{k}\right)\\
L_{4}\left(T_{k}\right)&=&X_{4}\left(T_{k}-\overline{\tau}_{4}\left(n-1\right)\right)
\end{eqnarray*}

\item[3.b)] $\tau_{2}\left(m\right)<T_{k}\leq \overline{\tau}_{2}\left(m\right)$ y $\overline{\tau}_{3}\left(n\right)<T_{k}\leq\tau_{4}\left(n\right)$:
\begin{eqnarray*}
L_{1}\left(T_{k}\right)&=&X_{1}\left(T_{k}-\overline{\tau}_{1}\left(m\right)\right)\\
L_{2}\left(T_{k}\right)&=&L_{2}\left(\tau_{2}\left(m\right)\right)+X_{2}\left(T_{k}-\tau_{2}\left(m\right)\right)+Y_{2}
\left(T_{k}-\tau_{2}\left(m\right)\right)-C_{2}\left(T_{k}\right)\\
L_{3}\left(T_{k}\right)&=&X_{3}\left(T_{k}-\overline{\tau}_{3}\left(n\right)\right)\\
L_{4}\left(T_{k}\right)&=&L_{4}\left(\tau_{4}\left(n\right)\right)
\end{eqnarray*}


\item[3.c)] $\tau_{2}\left(m\right)<T_{k}\leq \overline{\tau}_{2}\left(m\right)$ y $\tau_{4}\left(n\right)<T_{k}\leq\overline{\tau}_{4}\left(n\right)$:
\begin{eqnarray*}
L_{1}\left(T_{k}\right)&=&X_{1}\left(T_{k}-\overline{\tau}_{1}\left(m\right)\right)\\
L_{2}\left(T_{k}\right)&=&L_{2}\left(\tau_{2}\left(m\right)\right)+X_{2}\left(T_{k}-\tau_{2}\left(m\right)\right)+Y_{2}
\left(T_{k}-\tau_{2}\left(m\right)\right)-C_{2}\left(T_{k}\right)\\
L_{3}\left(T_{k}\right)&=&X_{3}\left(T_{k}-\overline{\tau}_{3}\left(n\right)\right)\\
L_{4}\left(T_{k}\right)&=&L_{4}\left(\tau_{4}\left(n\right)\right)+X_{4}\left(T_{k}-\tau_{4}\left(n\right)\right)+Y_{4}
\left(T_{k}-\tau_{4}\left(n\right)\right)-C_{4}\left(T_{k}\right)
\end{eqnarray*}

\item[3.d)] $\tau_{2}\left(m\right)<T_{k}\leq \overline{\tau}_{2}\left(m\right)$ y $\overline{\tau}_{4}\left(n\right)<T_{k}$:
\begin{eqnarray*}
L_{1}\left(T_{k}\right)&=&X_{1}\left(T_{k}-\overline{\tau}_{1}\left(m\right)\right)\\
L_{2}\left(T_{k}\right)&=&L_{2}\left(\tau_{2}\left(m\right)\right)+X_{2}\left(T_{k}-\tau_{2}\left(m\right)\right)+Y_{2}
\left(T_{k}-\tau_{2}\left(m\right)\right)-C_{2}\left(T_{k}\right)\\
L_{3}\left(T_{k}\right)&=&X_{3}\left(T_{k}-\overline{\tau}_{3}\left(n\right)\right)\\
L_{4}\left(T_{k}\right)&=&X_{4}\left(T_{k}-\overline{\tau}_{4}\left(n\right)\right)
\end{eqnarray*}
\end{itemize}
%_____________________________________________________________________________________________
\item Cuarto caso
%_____________________________________________________________________________________________
\begin{itemize}
\item[4.a)] $\overline{\tau}_{2}\left(m\right)<T_{k}$ y $T_{k}\leq\overline{\tau}_{3}\left(n\right)$:
\begin{eqnarray*}
L_{1}\left(T_{k}\right)&=&X_{1}\left(T_{k}-\overline{\tau}_{1}\left(m\right)\right)\\
L_{2}\left(T_{k}\right)&=&X_{2}\left(T_{k}-\overline{\tau}_{2}\left(m\right)\right)\\
L_{3}\left(T_{k}\right)&=&L_{3}\left(\tau_{3}\left(n\right)\right)+X_{3}\left(T_{k}-\tau_{3}\left(n\right)\right)+Y_{3}\left(T_{k}-\tau_{3}\left(n\right)\right)-C_{3}\left(T_{k}\right)\\
L_{4}\left(T_{k}\right)&=&X_{4}\left(T_{k}-\overline{\tau}_{4}\left(n-1\right)\right)
\end{eqnarray*}

\item[4.b)] $\overline{\tau}_{2}\left(m\right)<T_{k}$ y $\overline{\tau}_{3}\left(n\right)<T_{k}\leq\tau_{4}$:
\begin{eqnarray*}
L_{1}\left(T_{k}\right)&=&X_{1}\left(T_{k}-\overline{\tau}_{1}\left(m\right)\right)\\
L_{2}\left(T_{k}\right)&=&X_{2}\left(T_{k}-\overline{\tau}_{2}\left(m\right)\right)\\
L_{3}\left(T_{k}\right)&=&X_{3}\left(T_{k}-\overline{\tau}_{3}\left(n\right)\right)\\
L_{4}\left(T_{k}\right)&=&L_{4}\left(\tau_{4}\left(n\right)\right)
\end{eqnarray*}


\item[4.c)] $\overline{\tau}_{2}\left(m\right)<T_{k}$ y $\tau_{4}\left(n\right)<T_{k}\leq\overline{\tau}_{4}\left(n\right)$:
\begin{eqnarray*}
L_{1}\left(T_{k}\right)&=&X_{1}\left(T_{k}-\overline{\tau}_{1}\left(m\right)\right)\\
L_{2}\left(T_{k}\right)&=&X_{2}\left(T_{k}-\overline{\tau}_{2}\left(m\right)\right)\\
L_{3}\left(T_{k}\right)&=&X_{3}\left(T_{k}-\overline{\tau}_{3}\left(n\right)\right)\\
L_{4}\left(T_{k}\right)&=&L_{4}\left(\tau_{4}\left(n\right)\right)+X_{4}\left(T_{k}-\tau_{4}\left(n\right)\right)+Y_{4}
\left(T_{k}-\tau_{4}\left(n\right)\right)-C_{4}\left(T_{k}\right)
\end{eqnarray*}

\item[4.d)] $\overline{\tau}_{2}<T_{k}$ y $\overline{\tau}_{4}<T_{k}$:
\begin{eqnarray*}
L_{1}\left(T_{k}\right)&=&X_{1}\left(T_{k}-\overline{\tau}_{1}\left(m\right)\right)\\
L_{2}\left(T_{k}\right)&=&X_{2}\left(T_{k}-\overline{\tau}_{2}\left(m\right)\right)\\
L_{3}\left(T_{k}\right)&=&X_{3}\left(T_{k}-\overline{\tau}_{3}\left(n\right)\right)\\
L_{4}\left(T_{k}\right)&=&X_{4}\left(T_{k}-\overline{\tau}_{4}\left(n\right)\right)
\end{eqnarray*}
\end{itemize}


\end{itemize}
\newpage
%_____________________________________________________
\section{Puntos de Renovaci\'on}
%_____________________________________________________

Para cada cola $Q_{i}$ se tienen los procesos de arribo a la cola, para estas, los tiempos de arribo est\'an dados por $$\left\{T_{1}^{i},T_{2}^{i},\ldots,T_{k}^{i},\ldots\right\},$$ entonces, consideremos solamente los primeros tiempos de arribo a cada una de las colas, es decir, $$\left\{T_{1}^{1},T_{1}^{2},T_{1}^{3},T_{1}^{4}\right\},$$ se sabe que cada uno de estos tiempos se distribuye de manera exponencial con par\'ametro $1/mu_{i}$. Adem\'as se sabe que para $$T^{*}=\min\left\{T_{1}^{1},T_{1}^{2},T_{1}^{3},T_{1}^{4}\right\},$$ $T^{*}$ se distribuye de manera exponencial con par\'ametro $$\mu^{*}=\sum_{i=1}^{4}\mu_{i}.$$ Ahora, dado que 
\begin{center}
\begin{tabular}{lcl}
$\tilde{r}=r_{1}+r_{2}$ & y &$\hat{r}=r_{3}+r_{4}.$
\end{tabular}
\end{center}


Supongamos que $$\tilde{r},\hat{r}<\mu^{*},$$ entonces si tomamos $$r^{*}=\min\left\{\tilde{r},\hat{r}\right\},$$ se tiene que para  $$t^{*}\in\left(0,r^{*}\right)$$ se cumple que 
\begin{center}
\begin{tabular}{lcl}
$\tau_{1}\left(1\right)=0$ & y por tanto & $\overline{\tau}_{1}=0,$
\end{tabular}
\end{center}
entonces para la segunda cola en este primer ciclo se cumple que $$\tau_{2}=\overline{\tau}_{1}+r_{1}=r_{1}<\mu^{*},$$ y por tanto se tiene que  $$\overline{\tau}_{2}=\tau_{2}.$$ Por lo tanto, nuevamente para la primer cola en el segundo ciclo $$\tau_{1}\left(2\right)=\tau_{2}\left(1\right)+r_{2}=\tilde{r}<\mu^{*}.$$ An\'alogamente para el segundo sistema se tiene que ambas colas est\'an vac\'ias, es decir, existe un valor $t^{*}$ tal que en el intervalo $\left(0,t^{*}\right)$ no ha llegado ning\'un usuario, es decir, $$L_{i}\left(t^{*}\right)=0$$ para $i=1,2,3,4$.

\section{Resultados para Procesos de Salida}

En \cite{Sigman2} prueban que para la existencia de un una sucesi\'on infinita no decreciente de tiempos de regeneraci\'on $\tau_{1}\leq\tau_{2}\leq\cdots$ en los cuales el proceso se regenera, basta un tiempo de regeneraci\'on $R_{1}$, donde $R_{j}=\tau_{j}-\tau_{j-1}$. Para tal efecto se requiere la existencia de un espacio de probabilidad $\left(\Omega,\mathcal{F},\prob\right)$, y proceso estoc\'astico $\textit{X}=\left\{X\left(t\right):t\geq0\right\}$ con espacio de estados $\left(S,\mathcal{R}\right)$, con $\mathcal{R}$ $\sigma$-\'algebra.

\begin{Prop}
Si existe una variable aleatoria no negativa $R_{1}$ tal que $\theta_{R\footnotesize{1}}X=_{D}X$, entonces $\left(\Omega,\mathcal{F},\prob\right)$ puede extenderse para soportar una sucesi\'on estacionaria de variables aleatorias $R=\left\{R_{k}:k\geq1\right\}$, tal que para $k\geq1$,
\begin{eqnarray*}
\theta_{k}\left(X,R\right)=_{D}\left(X,R\right).
\end{eqnarray*}

Adem\'as, para $k\geq1$, $\theta_{k}R$ es condicionalmente independiente de $\left(X,R_{1},\ldots,R_{k}\right)$, dado $\theta_{\tau k}X$.

\end{Prop}


\begin{itemize}
\item Doob en 1953 demostr\'o que el estado estacionario de un proceso de partida en un sistema de espera $M/G/\infty$, es Poisson con la misma tasa que el proceso de arribos.

\item Burke en 1968, fue el primero en demostrar que el estado estacionario de un proceso de salida de una cola $M/M/s$ es un proceso Poisson.

\item Disney en 1973 obtuvo el siguiente resultado:

\begin{Teo}
Para el sistema de espera $M/G/1/L$ con disciplina FIFO, el proceso $\textbf{I}$ es un proceso de renovaci\'on si y s\'olo si el proceso denominado longitud de la cola es estacionario y se cumple cualquiera de los siguientes casos:

\begin{itemize}
\item[a)] Los tiempos de servicio son identicamente cero;
\item[b)] $L=0$, para cualquier proceso de servicio $S$;
\item[c)] $L=1$ y $G=D$;
\item[d)] $L=\infty$ y $G=M$.
\end{itemize}
En estos casos, respectivamente, las distribuciones de interpartida $P\left\{T_{n+1}-T_{n}\leq t\right\}$ son


\begin{itemize}
\item[a)] $1-e^{-\lambda t}$, $t\geq0$;
\item[b)] $1-e^{-\lambda t}*F\left(t\right)$, $t\geq0$;
\item[c)] $1-e^{-\lambda t}*\indora_{d}\left(t\right)$, $t\geq0$;
\item[d)] $1-e^{-\lambda t}*F\left(t\right)$, $t\geq0$.
\end{itemize}
\end{Teo}


\item Finch (1959) mostr\'o que para los sistemas $M/G/1/L$, con $1\leq L\leq \infty$ con distribuciones de servicio dos veces diferenciable, solamente el sistema $M/M/1/\infty$ tiene proceso de salida de renovaci\'on estacionario.

\item King (1971) demostro que un sistema de colas estacionario $M/G/1/1$ tiene sus tiempos de interpartida sucesivas $D_{n}$ y $D_{n+1}$ son independientes, si y s\'olo si, $G=D$, en cuyo caso le proceso de salida es de renovaci\'on.

\item Disney (1973) demostr\'o que el \'unico sistema estacionario $M/G/1/L$, que tiene proceso de salida de renovaci\'on  son los sistemas $M/M/1$ y $M/D/1/1$.



\item El siguiente resultado es de Disney y Koning (1985)
\begin{Teo}
En un sistema de espera $M/G/s$, el estado estacionario del proceso de salida es un proceso Poisson para cualquier distribuci\'on de los tiempos de servicio si el sistema tiene cualquiera de las siguientes cuatro propiedades.

\begin{itemize}
\item[a)] $s=\infty$
\item[b)] La disciplina de servicio es de procesador compartido.
\item[c)] La disciplina de servicio es LCFS y preemptive resume, esto se cumple para $L<\infty$
\item[d)] $G=M$.
\end{itemize}

\end{Teo}

\item El siguiente resultado es de Alamatsaz (1983)

\begin{Teo}
En cualquier sistema de colas $GI/G/1/L$ con $1\leq L<\infty$ y distribuci\'on de interarribos $A$ y distribuci\'on de los tiempos de servicio $B$, tal que $A\left(0\right)=0$, $A\left(t\right)\left(1-B\left(t\right)\right)>0$ para alguna $t>0$ y $B\left(t\right)$ para toda $t>0$, es imposible que el proceso de salida estacionario sea de renovaci\'on.
\end{Teo}

\end{itemize}

Estos resultados aparecen en Daley (1968) \cite{Daley68} para $\left\{T_{n}\right\}$ intervalos de inter-arribo, $\left\{D_{n}\right\}$ intervalos de inter-salida y $\left\{S_{n}\right\}$ tiempos de servicio.

\begin{itemize}
\item Si el proceso $\left\{T_{n}\right\}$ es Poisson, el proceso $\left\{D_{n}\right\}$ es no correlacionado si y s\'olo si es un proceso Poisso, lo cual ocurre si y s\'olo si $\left\{S_{n}\right\}$ son exponenciales negativas.

\item Si $\left\{S_{n}\right\}$ son exponenciales negativas, $\left\{D_{n}\right\}$ es un proceso de renovaci\'on  si y s\'olo si es un proceso Poisson, lo cual ocurre si y s\'olo si $\left\{T_{n}\right\}$ es un proceso Poisson.

\item $\esp\left(D_{n}\right)=\esp\left(T_{n}\right)$.

\item Para un sistema de visitas $GI/M/1$ se tiene el siguiente teorema:

\begin{Teo}
En un sistema estacionario $GI/M/1$ los intervalos de interpartida tienen
\begin{eqnarray*}
\esp\left(e^{-\theta D_{n}}\right)&=&\mu\left(\mu+\theta\right)^{-1}\left[\delta\theta
-\mu\left(1-\delta\right)\alpha\left(\theta\right)\right]
\left[\theta-\mu\left(1-\delta\right)^{-1}\right]\\
\alpha\left(\theta\right)&=&\esp\left[e^{-\theta T_{0}}\right]\\
var\left(D_{n}\right)&=&var\left(T_{0}\right)-\left(\tau^{-1}-\delta^{-1}\right)
2\delta\left(\esp\left(S_{0}\right)\right)^{2}\left(1-\delta\right)^{-1}.
\end{eqnarray*}
\end{Teo}



\begin{Teo}
El proceso de salida de un sistema de colas estacionario $GI/M/1$ es un proceso de renovaci\'on si y s\'olo si el proceso de entrada es un proceso Poisson, en cuyo caso el proceso de salida es un proceso Poisson.
\end{Teo}


\begin{Teo}
Los intervalos de interpartida $\left\{D_{n}\right\}$ de un sistema $M/G/1$ estacionario son no correlacionados si y s\'olo si la distribuci\'on de los tiempos de servicio es exponencial negativa, es decir, el sistema es de tipo  $M/M/1$.

\end{Teo}



\end{itemize}


%sudo apt-get install ubuntu-restricted-extras


\section{Resultados para Procesos de Salida}

En \cite{Sigman2} prueban que para la existencia de un una sucesi\'on infinita no decreciente de tiempos de regeneraci\'on $\tau_{1}\leq\tau_{2}\leq\cdots$ en los cuales el proceso se regenera, basta un tiempo de regeneraci\'on $R_{1}$, donde $R_{j}=\tau_{j}-\tau_{j-1}$. Para tal efecto se requiere la existencia de un espacio de probabilidad $\left(\Omega,\mathcal{F},\prob\right)$, y proceso estoc\'astico $\textit{X}=\left\{X\left(t\right):t\geq0\right\}$ con espacio de estados $\left(S,\mathcal{R}\right)$, con $\mathcal{R}$ $\sigma$-\'algebra.

\begin{Prop}
Si existe una variable aleatoria no negativa $R_{1}$ tal que $\theta_{R\footnotesize{1}}X=_{D}X$, entonces $\left(\Omega,\mathcal{F},\prob\right)$ puede extenderse para soportar una sucesi\'on estacionaria de variables aleatorias $R=\left\{R_{k}:k\geq1\right\}$, tal que para $k\geq1$,
\begin{eqnarray*}
\theta_{k}\left(X,R\right)=_{D}\left(X,R\right).
\end{eqnarray*}

Adem\'as, para $k\geq1$, $\theta_{k}R$ es condicionalmente independiente de $\left(X,R_{1},\ldots,R_{k}\right)$, dado $\theta_{\tau k}X$.

\end{Prop}


\begin{itemize}
\item Doob en 1953 demostr\'o que el estado estacionario de un proceso de partida en un sistema de espera $M/G/\infty$, es Poisson con la misma tasa que el proceso de arribos.

\item Burke en 1968, fue el primero en demostrar que el estado estacionario de un proceso de salida de una cola $M/M/s$ es un proceso Poisson.

\item Disney en 1973 obtuvo el siguiente resultado:

\begin{Teo}
Para el sistema de espera $M/G/1/L$ con disciplina FIFO, el proceso $\textbf{I}$ es un proceso de renovaci\'on si y s\'olo si el proceso denominado longitud de la cola es estacionario y se cumple cualquiera de los siguientes casos:

\begin{itemize}
\item[a)] Los tiempos de servicio son identicamente cero;
\item[b)] $L=0$, para cualquier proceso de servicio $S$;
\item[c)] $L=1$ y $G=D$;
\item[d)] $L=\infty$ y $G=M$.
\end{itemize}
En estos casos, respectivamente, las distribuciones de interpartida $P\left\{T_{n+1}-T_{n}\leq t\right\}$ son


\begin{itemize}
\item[a)] $1-e^{-\lambda t}$, $t\geq0$;
\item[b)] $1-e^{-\lambda t}*F\left(t\right)$, $t\geq0$;
\item[c)] $1-e^{-\lambda t}*\indora_{d}\left(t\right)$, $t\geq0$;
\item[d)] $1-e^{-\lambda t}*F\left(t\right)$, $t\geq0$.
\end{itemize}
\end{Teo}


\item Finch (1959) mostr\'o que para los sistemas $M/G/1/L$, con $1\leq L\leq \infty$ con distribuciones de servicio dos veces diferenciable, solamente el sistema $M/M/1/\infty$ tiene proceso de salida de renovaci\'on estacionario.

\item King (1971) demostro que un sistema de colas estacionario $M/G/1/1$ tiene sus tiempos de interpartida sucesivas $D_{n}$ y $D_{n+1}$ son independientes, si y s\'olo si, $G=D$, en cuyo caso le proceso de salida es de renovaci\'on.

\item Disney (1973) demostr\'o que el \'unico sistema estacionario $M/G/1/L$, que tiene proceso de salida de renovaci\'on  son los sistemas $M/M/1$ y $M/D/1/1$.



\item El siguiente resultado es de Disney y Koning (1985)
\begin{Teo}
En un sistema de espera $M/G/s$, el estado estacionario del proceso de salida es un proceso Poisson para cualquier distribuci\'on de los tiempos de servicio si el sistema tiene cualquiera de las siguientes cuatro propiedades.

\begin{itemize}
\item[a)] $s=\infty$
\item[b)] La disciplina de servicio es de procesador compartido.
\item[c)] La disciplina de servicio es LCFS y preemptive resume, esto se cumple para $L<\infty$
\item[d)] $G=M$.
\end{itemize}

\end{Teo}

\item El siguiente resultado es de Alamatsaz (1983)

\begin{Teo}
En cualquier sistema de colas $GI/G/1/L$ con $1\leq L<\infty$ y distribuci\'on de interarribos $A$ y distribuci\'on de los tiempos de servicio $B$, tal que $A\left(0\right)=0$, $A\left(t\right)\left(1-B\left(t\right)\right)>0$ para alguna $t>0$ y $B\left(t\right)$ para toda $t>0$, es imposible que el proceso de salida estacionario sea de renovaci\'on.
\end{Teo}

\end{itemize}

Estos resultados aparecen en Daley (1968) \cite{Daley68} para $\left\{T_{n}\right\}$ intervalos de inter-arribo, $\left\{D_{n}\right\}$ intervalos de inter-salida y $\left\{S_{n}\right\}$ tiempos de servicio.

\begin{itemize}
\item Si el proceso $\left\{T_{n}\right\}$ es Poisson, el proceso $\left\{D_{n}\right\}$ es no correlacionado si y s\'olo si es un proceso Poisso, lo cual ocurre si y s\'olo si $\left\{S_{n}\right\}$ son exponenciales negativas.

\item Si $\left\{S_{n}\right\}$ son exponenciales negativas, $\left\{D_{n}\right\}$ es un proceso de renovaci\'on  si y s\'olo si es un proceso Poisson, lo cual ocurre si y s\'olo si $\left\{T_{n}\right\}$ es un proceso Poisson.

\item $\esp\left(D_{n}\right)=\esp\left(T_{n}\right)$.

\item Para un sistema de visitas $GI/M/1$ se tiene el siguiente teorema:

\begin{Teo}
En un sistema estacionario $GI/M/1$ los intervalos de interpartida tienen
\begin{eqnarray*}
\esp\left(e^{-\theta D_{n}}\right)&=&\mu\left(\mu+\theta\right)^{-1}\left[\delta\theta
-\mu\left(1-\delta\right)\alpha\left(\theta\right)\right]
\left[\theta-\mu\left(1-\delta\right)^{-1}\right]\\
\alpha\left(\theta\right)&=&\esp\left[e^{-\theta T_{0}}\right]\\
var\left(D_{n}\right)&=&var\left(T_{0}\right)-\left(\tau^{-1}-\delta^{-1}\right)
2\delta\left(\esp\left(S_{0}\right)\right)^{2}\left(1-\delta\right)^{-1}.
\end{eqnarray*}
\end{Teo}



\begin{Teo}
El proceso de salida de un sistema de colas estacionario $GI/M/1$ es un proceso de renovaci\'on si y s\'olo si el proceso de entrada es un proceso Poisson, en cuyo caso el proceso de salida es un proceso Poisson.
\end{Teo}


\begin{Teo}
Los intervalos de interpartida $\left\{D_{n}\right\}$ de un sistema $M/G/1$ estacionario son no correlacionados si y s\'olo si la distribuci\'on de los tiempos de servicio es exponencial negativa, es decir, el sistema es de tipo  $M/M/1$.

\end{Teo}



\end{itemize}



\subsection{Por resolver}

derivando con respecto a $z$



\begin{eqnarray*}
\frac{d Q_{i}\left(z\right)}{d z}&=&\frac{\left(1-F_{i}\left(z\right)\right)P_{i}\left(z\right)}{\esp\left[C_{i}\right]\left(1-P_{i}\left(z\right)\right)\left(P_{i}\left(z\right)-z\right)}\\
&-&\frac{\left(1-z\right)P_{i}\left(z\right)F_{i}^{'}\left(z\right)}{\esp\left[C_{i}\right]\left(1-P_{i}\left(z\right)\right)\left(P_{i}\left(z\right)-z\right)}\\
&-&\frac{\left(1-z\right)\left(1-F_{i}\left(z\right)\right)P_{i}\left(z\right)\left(P_{i}^{'}\left(z\right)-1\right)}{\esp\left[C_{i}\right]\left(1-P_{i}\left(z\right)\right)\left(P_{i}\left(z\right)-z\right)^{2}}\\
&+&\frac{\left(1-z\right)\left(1-F_{i}\left(z\right)\right)P_{i}^{'}\left(z\right)}{\esp\left[C_{i}\right]\left(1-P_{i}\left(z\right)\right)\left(P_{i}\left(z\right)-z\right)}\\
&+&\frac{\left(1-z\right)\left(1-F_{i}\left(z\right)\right)P_{i}\left(z\right)P_{i}^{'}\left(z\right)}{\esp\left[C_{i}\right]\left(1-P_{i}\left(z\right)\right)^{2}\left(P_{i}\left(z\right)-z\right)}
\end{eqnarray*}

Calculando el l\'imite cuando $z\rightarrow1^{+}$:
\begin{eqnarray}
Q_{i}^{(1)}\left(z\right)=\lim_{z\rightarrow1^{+}}\frac{d Q_{i}\left(z\right)}{dz}&=&\lim_{z\rightarrow1}\frac{\left(1-F_{i}\left(z\right)\right)P_{i}\left(z\right)}{\esp\left[C_{i}\right]\left(1-P_{i}\left(z\right)\right)\left(P_{i}\left(z\right)-z\right)}\\
&-&\lim_{z\rightarrow1^{+}}\frac{\left(1-z\right)P_{i}\left(z\right)F_{i}^{'}\left(z\right)}{\esp\left[C_{i}\right]\left(1-P_{i}\left(z\right)\right)\left(P_{i}\left(z\right)-z\right)}\\
&-&\lim_{z\rightarrow1^{+}}\frac{\left(1-z\right)\left(1-F_{i}\left(z\right)\right)P_{i}\left(z\right)\left(P_{i}^{'}\left(z\right)-1\right)}{\esp\left[C_{i}\right]\left(1-P_{i}\left(z\right)\right)\left(P_{i}\left(z\right)-z\right)^{2}}\\
&+&\lim_{z\rightarrow1^{+}}\frac{\left(1-z\right)\left(1-F_{i}\left(z\right)\right)P_{i}^{'}\left(z\right)}{\esp\left[C_{i}\right]\left(1-P_{i}\left(z\right)\right)\left(P_{i}\left(z\right)-z\right)}\\
&+&\lim_{z\rightarrow1^{+}}\frac{\left(1-z\right)\left(1-F_{i}\left(z\right)\right)P_{i}\left(z\right)P_{i}^{'}\left(z\right)}{\esp\left[C_{i}\right]\left(1-P_{i}\left(z\right)\right)^{2}\left(P_{i}\left(z\right)-z\right)}
\end{eqnarray}

Entonces:
%______________________________________________________

\begin{eqnarray*}
\lim_{z\rightarrow1^{+}}\frac{\left(1-F_{i}\left(z\right)\right)P_{i}\left(z\right)}{\left(1-P_{i}\left(z\right)\right)\left(P_{i}\left(z\right)-z\right)}&=&\lim_{z\rightarrow1^{+}}\frac{\frac{d}{dz}\left[\left(1-F_{i}\left(z\right)\right)P_{i}\left(z\right)\right]}{\frac{d}{dz}\left[\left(1-P_{i}\left(z\right)\right)\left(-z+P_{i}\left(z\right)\right)\right]}\\
&=&\lim_{z\rightarrow1^{+}}\frac{-P_{i}\left(z\right)F_{i}^{'}\left(z\right)+\left(1-F_{i}\left(z\right)\right)P_{i}^{'}\left(z\right)}{\left(1-P_{i}\left(z\right)\right)\left(-1+P_{i}^{'}\left(z\right)\right)-\left(-z+P_{i}\left(z\right)\right)P_{i}^{'}\left(z\right)}
\end{eqnarray*}


%______________________________________________________


\begin{eqnarray*}
\lim_{z\rightarrow1^{+}}\frac{\left(1-z\right)P_{i}\left(z\right)F_{i}^{'}\left(z\right)}{\left(1-P_{i}\left(z\right)\right)\left(P_{i}\left(z\right)-z\right)}&=&\lim_{z\rightarrow1^{+}}\frac{\frac{d}{dz}\left[\left(1-z\right)P_{i}\left(z\right)F_{i}^{'}\left(z\right)\right]}{\frac{d}{dz}\left[\left(1-P_{i}\left(z\right)\right)\left(P_{i}\left(z\right)-z\right)\right]}\\
&=&\lim_{z\rightarrow1^{+}}\frac{-P_{i}\left(z\right) F_{i}^{'}\left(z\right)+(1-z) F_{i}^{'}\left(z\right) P_{i}^{'}\left(z\right)+(1-z) P_{i}\left(z\right)F_{i}^{''}\left(z\right)}{\left(1-P_{i}\left(z\right)\right)\left(-1+P_{i}^{'}\left(z\right)\right)-\left(-z+P_{i}\left(z\right)\right)P_{i}^{'}\left(z\right)}
\end{eqnarray*}


%______________________________________________________

\begin{eqnarray*}
&&\lim_{z\rightarrow1^{+}}\frac{\left(1-z\right)\left(1-F_{i}\left(z\right)\right)P_{i}\left(z\right)\left(P_{i}^{'}\left(z\right)-1\right)}{\left(1-P_{i}\left(z\right)\right)\left(P_{i}\left(z\right)-z\right)^{2}}=\lim_{z\rightarrow1^{+}}\frac{\frac{d}{dz}\left[\left(1-z\right)\left(1-F_{i}\left(z\right)\right)P_{i}\left(z\right)\left(P_{i}^{'}\left(z\right)-1\right)\right]}{\frac{d}{dz}\left[\left(1-P_{i}\left(z\right)\right)\left(P_{i}\left(z\right)-z\right)^{2}\right]}\\
&=&\lim_{z\rightarrow1^{+}}\frac{-\left(1-F_{i}\left(z\right)\right) P_{i}\left(z\right)\left(-1+P_{i}^{'}\left(z\right)\right)-(1-z) P_{i}\left(z\right)F_{i}^{'}\left(z\right)\left(-1+P_{i}^{'}\left(z\right)\right)}{2\left(1-P_{i}\left(z\right)\right)\left(-z+P_{i}\left(z\right)\right) \left(-1+P_{i}^{'}\left(z\right)\right)-\left(-z+P_{i}\left(z\right)\right)^2 P_{i}^{'}\left(z\right)}\\
&+&\lim_{z\rightarrow1^{+}}\frac{+(1-z) \left(1-F_{i}\left(z\right)\right) \left(-1+P_{i}^{'}\left(z\right)\right) P_{i}^{'}\left(z\right)}{{2\left(1-P_{i}\left(z\right)\right)\left(-z+P_{i}\left(z\right)\right) \left(-1+P_{i}^{'}\left(z\right)\right)-\left(-z+P_{i}\left(z\right)\right)^2 P_{i}^{'}\left(z\right)}}\\
&+&\lim_{z\rightarrow1^{+}}\frac{+(1-z) \left(1-F_{i}\left(z\right)\right) P_{i}\left(z\right)P_{i}^{''}\left(z\right)}{{2\left(1-P_{i}\left(z\right)\right)\left(-z+P_{i}\left(z\right)\right) \left(-1+P_{i}^{'}\left(z\right)\right)-\left(-z+P_{i}\left(z\right)\right)^2 P_{i}^{'}\left(z\right)}}
\end{eqnarray*}











%______________________________________________________
\begin{eqnarray*}
&&\lim_{z\rightarrow1^{+}}\frac{\left(1-z\right)\left(1-F_{i}\left(z\right)\right)P_{i}^{'}\left(z\right)}{\left(1-P_{i}\left(z\right)\right)\left(P_{i}\left(z\right)-z\right)}=\lim_{z\rightarrow1^{+}}\frac{\frac{d}{dz}\left[\left(1-z\right)\left(1-F_{i}\left(z\right)\right)P_{i}^{'}\left(z\right)\right]}{\frac{d}{dz}\left[\left(1-P_{i}\left(z\right)\right)\left(P_{i}\left(z\right)-z\right)\right]}\\
&=&\lim_{z\rightarrow1^{+}}\frac{-\left(1-F_{i}\left(z\right)\right) P_{i}^{'}\left(z\right)-(1-z) F_{i}^{'}\left(z\right) P_{i}^{'}\left(z\right)+(1-z) \left(1-F_{i}\left(z\right)\right) P_{i}^{''}\left(z\right)}{\left(1-P_{i}\left(z\right)\right) \left(-1+P_{i}^{'}\left(z\right)\right)-\left(-z+P_{i}\left(z\right)\right) P_{i}^{'}\left(z\right)}\frac{}{}
\end{eqnarray*}

%______________________________________________________
\begin{eqnarray*}
&&\lim_{z\rightarrow1^{+}}\frac{\left(1-z\right)\left(1-F_{i}\left(z\right)\right)P_{i}\left(z\right)P_{i}^{'}\left(z\right)}{\left(1-P_{i}\left(z\right)\right)^{2}\left(P_{i}\left(z\right)-z\right)}=\lim_{z\rightarrow1^{+}}\frac{\frac{d}{dz}\left[\left(1-z\right)\left(1-F_{i}\left(z\right)\right)P_{i}\left(z\right)P_{i}^{'}\left(z\right)\right]}{\frac{d}{dz}\left[\left(1-P_{i}\left(z\right)\right)^{2}\left(P_{i}\left(z\right)-z\right)\right]}\\
&=&\lim_{z\rightarrow1^{+}}\frac{-\left(1-F_{i}\left(z\right)\right) P_{i}\left(z\right) P_{i}^{'}\left(z\right)-(1-z) P_{i}\left(z\right) F_{i}^{'}\left(z\right)P_i'[z]}{\left(1-P_{i}\left(z\right)\right)^2 \left(-1+P_{i}^{'}\left(z\right)\right)-2 \left(1-P_{i}\left(z\right)\right) \left(-z+P_{i}\left(z\right)\right) P_{i}^{'}\left(z\right)}\\
&+&\lim_{z\rightarrow1^{+}}\frac{(1-z) \left(1-F_{i}\left(z\right)\right) P_{i}^{'}\left(z\right)^2+(1-z) \left(1-F_{i}\left(z\right)\right) P_{i}\left(z\right) P_{i}^{''}\left(z\right)}{\left(1-P_{i}\left(z\right)\right)^2 \left(-1+P_{i}^{'}\left(z\right)\right)-2 \left(1-P_{i}\left(z\right)\right) \left(-z+P_{i}\left(z\right)\right) P_{i}^{'}\left(z\right)}\\
\end{eqnarray*}

%\section{Por resolver}



\begin{eqnarray*}
&&\frac{\partial Q_{i}\left(z\right)}{\partial z}=\frac{1}{\esp\left[C_{i}\right]}\frac{\partial}{\partial z}\left\{\frac{1-F_{i}\left(z\right)}{P_{i}\left(z\right)-z}\cdot\frac{\left(1-z\right)P_{i}\left(z\right)}{1-P_{i}\left(z\right)}\right\}\\
&=&\frac{1}{\esp\left[C_{i}\right]}\left\{\frac{\partial}{\partial z}\left(\frac{1-F_{i}\left(z\right)}{P_{i}\left(z\right)-z}\right)\cdot\frac{\left(1-z\right)P_{i}\left(z\right)}{1-P_{i}\left(z\right)}+\frac{1-F_{i}\left(z\right)}{P_{i}\left(z\right)-z}\cdot\frac{\partial}{\partial z}\left(\frac{\left(1-z\right)P_{i}\left(z\right)}{1-P_{i}\left(z\right)}\right)\right\}\\
&=&\frac{1}{\esp\left[C_{i}\right]}\cdot\frac{\left(1-z\right)P_{i}\left(z\right)}{1-P_{i}\left(z\right)}\cdot\frac{\partial}{\partial z}\left(\frac{1-F_{i}\left(z\right)}{P_{i}\left(z\right)-z}\right)+\frac{1}{\esp\left[C_{i}\right]}\cdot\frac{1-F_{i}\left(z\right)}{P_{i}\left(z\right)-z}\cdot\frac{\partial}{\partial z}\left(\frac{\left(1-z\right)P_{i}\left(z\right)}{1-P_{i}\left(z\right)}\right)\\
&=&\frac{1}{\esp\left[C_{i}\right]}\cdot\frac{\left(1-z\right)P_{i}\left(z\right)}{1-P_{i}\left(z\right)}\cdot\frac{-F_{i}^{'}\left(z\right)\left(P_{i}\left(z\right)-z\right)-\left(1-F_{i}\left(z\right)\right)\left(P_{i}^{'}\left(z\right)-1\right)}{\left(P_{i}\left(z\right)-z\right)^{2}}\\
&+&\frac{1}{\esp\left[C_{i}\right]}\cdot\frac{1-F_{i}\left(z\right)}{P_{i}\left(z\right)-z}\cdot\frac{\left(1-z\right)P_{i}^{'}\left(z\right)-P_{i}\left(z\right)}{\left(1-P_{i}\left(z\right)\right)^{2}}
\end{eqnarray*}



\begin{eqnarray*}
Q_{i}^{(1)}\left(z\right)&=& \frac{\left(1-F_{i}\left(z\right)\right)P_{i}\left(z\right)}{\esp\left[C_{i}\right]\left(1-P_{i}\left(z\right)\right)\left(P_{i}\left(z\right)-z\right)}
-\frac{\left(1-z\right)P_{i}\left(z\right)F_{i}^{'}\left(z\right)}{\esp\left[C_{i}\right]\left(1-P_{i}\left(z\right)\right)\left(P_{i}\left(z\right)-z\right)}\\
&-&\frac{\left(1-z\right)\left(1-F_{i}\left(z\right)\right)P_{i}\left(z\right)\left(P_{i}^{'}\left(z\right)-1\right)}{\esp\left[C_{i}\right]\left(1-P_{i}\left(z\right)\right)\left(P_{i}\left(z\right)-z\right)^{2}}+\frac{\left(1-z\right)\left(1-F_{i}\left(z\right)\right)P_{i}^{'}\left(z\right)}{\esp\left[C_{i}\right]\left(1-P_{i}\left(z\right)\right)\left(P_{i}\left(z\right)-z\right)}\\
&+&\frac{\left(1-z\right)\left(1-F_{i}\left(z\right)\right)P_{i}\left(z\right)P_{i}^{'}\left(z\right)}{\esp\left[C_{i}\right]\left(1-P_{i}\left(z\right)\right)^{2}\left(P_{i}\left(z\right)-z\right)}
\end{eqnarray*}
%___________________________________________________________________________________________
%\subsection{Operaciones Matemathica: Tiempos de Espera}
%___________________________________________________________________________________________
Sea
$V_{i}\left(z\right)=\frac{1}{\esp\left[C_{i}\right]}\frac{I_{i}\left(z\right)-1}{z-P_{i}\left(z\right)}$

%{\esp\lef[I_{i}\right]}\frac{1-\mu_{i}}{z-P_{i}\left(z\right)}

\begin{eqnarray*}
\frac{\partial V_{i}\left(z\right)}{\partial z}&=&\frac{1}{\esp\left[C_{i}\right]}\left[\frac{I_{i}{'}\left(z\right)\left(z-P_{i}\left(z\right)\right)}{z-P_{i}\left(z\right)}-\frac{\left(I_{i}\left(z\right)-1\right)\left(1-P_{i}{'}\left(z\right)\right)}{\left(z-P_{i}\left(z\right)\right)^{2}}\right]
\end{eqnarray*}


La FGP para el tiempo de espera para cualquier usuario en la cola est\'a dada por:
\[U_{i}\left(z\right)=\frac{1}{\esp\left[C_{i}\right]}\cdot\frac{1-P_{i}\left(z\right)}{z-P_{i}\left(z\right)}\cdot\frac{I_{i}\left(z\right)-1}{1-z}\]

entonces


\begin{eqnarray*}
\frac{d}{dz}V_{i}\left(z\right)&=&\frac{1}{\esp\left[C_{i}\right]}\left\{\frac{d}{dz}\left(\frac{1-P_{i}\left(z\right)}{z-P_{i}\left(z\right)}\right)\frac{I_{i}\left(z\right)-1}{1-z}+\frac{1-P_{i}\left(z\right)}{z-P_{i}\left(z\right)}\frac{d}{dz}\left(\frac{I_{i}\left(z\right)-1}{1-z}\right)\right\}\\
&=&\frac{1}{\esp\left[C_{i}\right]}\left\{\frac{-P_{i}\left(z\right)\left(z-P_{i}\left(z\right)\right)-\left(1-P_{i}\left(z\right)\right)\left(1-P_{i}^{'}\left(z\right)\right)}{\left(z-P_{i}\left(z\right)\right)^{2}}\cdot\frac{I_{i}\left(z\right)-1}{1-z}\right\}\\
&+&\frac{1}{\esp\left[C_{i}\right]}\left\{\frac{1-P_{i}\left(z\right)}{z-P_{i}\left(z\right)}\cdot\frac{I_{i}^{'}\left(z\right)\left(1-z\right)+\left(I_{i}\left(z\right)-1\right)}{\left(1-z\right)^{2}}\right\}
\end{eqnarray*}
%\frac{I_{i}\left(z\right)-1}{1-z}
%+\frac{1-P_{i}\left(z\right)}{z-P_{i}\frac{d}{dz}\left(\frac{I_{i}\left(z\right)-1}{1-z}\right)


\begin{eqnarray*}
\frac{\partial U_{i}\left(z\right)}{\partial z}&=&\frac{(-1+I_{i}[z]) (1-P_{i}[z])}{(1-z)^2 \esp[I_{i}] (z-P_{i}[z])}+\frac{(1-P_{i}[z]) I_{i}^{'}[z]}{(1-z) \esp[I_{i}] (z-P_{i}[z])}-\frac{(-1+I_{i}[z]) (1-P_{i}[z])\left(1-P{'}[z]\right)}{(1-z) \esp[I_{i}] (z-P_{i}[z])^2}\\
&-&\frac{(-1+I_{i}[z]) P_{i}{'}[z]}{(1-z) \esp[I_{i}](z-P_{i}[z])}
\end{eqnarray*}

%___________________________________________________________________________________________
%\section{Tiempos de Ciclo e Intervisita}
%___________________________________________________________________________________________


%___________________________________________________________________________________________
%\subsection{Longitudes de la Cola en cualquier tiempo}
%___________________________________________________________________________________________

Sea
$V_{i}\left(z\right)=\frac{1}{\esp\left[C_{i}\right]}\frac{I_{i}\left(z\right)-1}{z-P_{i}\left(z\right)}$

%{\esp\lef[I_{i}\right]}\frac{1-\mu_{i}}{z-P_{i}\left(z\right)}

\begin{eqnarray*}
\frac{\partial V_{i}\left(z\right)}{\partial z}&=&\frac{1}{\esp\left[C_{i}\right]}\left[\frac{I_{i}{'}\left(z\right)\left(z-P_{i}\left(z\right)\right)}{z-P_{i}\left(z\right)}-\frac{\left(I_{i}\left(z\right)-1\right)\left(1-P_{i}{'}\left(z\right)\right)}{\left(z-P_{i}\left(z\right)\right)^{2}}\right]
\end{eqnarray*}


La FGP para el tiempo de espera para cualquier usuario en la cola est\'a dada por:
\[U_{i}\left(z\right)=\frac{1}{\esp\left[C_{i}\right]}\cdot\frac{1-P_{i}\left(z\right)}{z-P_{i}\left(z\right)}\cdot\frac{I_{i}\left(z\right)-1}{1-z}\]

entonces


\begin{eqnarray*}
\frac{d}{dz}V_{i}\left(z\right)&=&\frac{1}{\esp\left[C_{i}\right]}\left\{\frac{d}{dz}\left(\frac{1-P_{i}\left(z\right)}{z-P_{i}\left(z\right)}\right)\frac{I_{i}\left(z\right)-1}{1-z}+\frac{1-P_{i}\left(z\right)}{z-P_{i}\left(z\right)}\frac{d}{dz}\left(\frac{I_{i}\left(z\right)-1}{1-z}\right)\right\}\\
&=&\frac{1}{\esp\left[C_{i}\right]}\left\{\frac{-P_{i}\left(z\right)\left(z-P_{i}\left(z\right)\right)-\left(1-P_{i}\left(z\right)\right)\left(1-P_{i}^{'}\left(z\right)\right)}{\left(z-P_{i}\left(z\right)\right)^{2}}\cdot\frac{I_{i}\left(z\right)-1}{1-z}\right\}\\
&+&\frac{1}{\esp\left[C_{i}\right]}\left\{\frac{1-P_{i}\left(z\right)}{z-P_{i}\left(z\right)}\cdot\frac{I_{i}^{'}\left(z\right)\left(1-z\right)+\left(I_{i}\left(z\right)-1\right)}{\left(1-z\right)^{2}}\right\}
\end{eqnarray*}
%\frac{I_{i}\left(z\right)-1}{1-z}
%+\frac{1-P_{i}\left(z\right)}{z-P_{i}\frac{d}{dz}\left(\frac{I_{i}\left(z\right)-1}{1-z}\right)


\begin{eqnarray*}
\frac{\partial U_{i}\left(z\right)}{\partial z}&=&\frac{(-1+I_{i}[z]) (1-P_{i}[z])}{(1-z)^2 \esp[I_{i}] (z-P_{i}[z])}+\frac{(1-P_{i}[z]) I_{i}^{'}[z]}{(1-z) \esp[I_{i}] (z-P_{i}[z])}-\frac{(-1+I_{i}[z]) (1-P_{i}[z])\left(1-P{'}[z]\right)}{(1-z) \esp[I_{i}] (z-P_{i}[z])^2}\\
&-&\frac{(-1+I_{i}[z]) P_{i}{'}[z]}{(1-z) \esp[I_{i}](z-P_{i}[z])}
\end{eqnarray*}



\begin{eqnarray*}
&&\frac{\partial Q_{i}\left(z\right)}{\partial z}=\frac{1}{\esp\left[C_{i}\right]}\frac{\partial}{\partial z}\left\{\frac{1-F_{i}\left(z\right)}{P_{i}\left(z\right)-z}\cdot\frac{\left(1-z\right)P_{i}\left(z\right)}{1-P_{i}\left(z\right)}\right\}\\
&=&\frac{1}{\esp\left[C_{i}\right]}\left\{\frac{\partial}{\partial z}\left(\frac{1-F_{i}\left(z\right)}{P_{i}\left(z\right)-z}\right)\cdot\frac{\left(1-z\right)P_{i}\left(z\right)}{1-P_{i}\left(z\right)}+\frac{1-F_{i}\left(z\right)}{P_{i}\left(z\right)-z}\cdot\frac{\partial}{\partial z}\left(\frac{\left(1-z\right)P_{i}\left(z\right)}{1-P_{i}\left(z\right)}\right)\right\}\\
&=&\frac{1}{\esp\left[C_{i}\right]}\cdot\frac{\left(1-z\right)P_{i}\left(z\right)}{1-P_{i}\left(z\right)}\cdot\frac{\partial}{\partial z}\left(\frac{1-F_{i}\left(z\right)}{P_{i}\left(z\right)-z}\right)+\frac{1}{\esp\left[C_{i}\right]}\cdot\frac{1-F_{i}\left(z\right)}{P_{i}\left(z\right)-z}\cdot\frac{\partial}{\partial z}\left(\frac{\left(1-z\right)P_{i}\left(z\right)}{1-P_{i}\left(z\right)}\right)\\
&=&\frac{1}{\esp\left[C_{i}\right]}\cdot\frac{\left(1-z\right)P_{i}\left(z\right)}{1-P_{i}\left(z\right)}\cdot\frac{-F_{i}^{'}\left(z\right)\left(P_{i}\left(z\right)-z\right)-\left(1-F_{i}\left(z\right)\right)\left(P_{i}^{'}\left(z\right)-1\right)}{\left(P_{i}\left(z\right)-z\right)^{2}}\\
&+&\frac{1}{\esp\left[C_{i}\right]}\cdot\frac{1-F_{i}\left(z\right)}{P_{i}\left(z\right)-z}\cdot\frac{\left(1-z\right)P_{i}^{'}\left(z\right)-P_{i}\left(z\right)}{\left(1-P_{i}\left(z\right)\right)^{2}}
\end{eqnarray*}



\begin{eqnarray*}
Q_{i}^{(1)}\left(z\right)&=& \frac{\left(1-F_{i}\left(z\right)\right)P_{i}\left(z\right)}{\esp\left[C_{i}\right]\left(1-P_{i}\left(z\right)\right)\left(P_{i}\left(z\right)-z\right)}
-\frac{\left(1-z\right)P_{i}\left(z\right)F_{i}^{'}\left(z\right)}{\esp\left[C_{i}\right]\left(1-P_{i}\left(z\right)\right)\left(P_{i}\left(z\right)-z\right)}\\
&-&\frac{\left(1-z\right)\left(1-F_{i}\left(z\right)\right)P_{i}\left(z\right)\left(P_{i}^{'}\left(z\right)-1\right)}{\esp\left[C_{i}\right]\left(1-P_{i}\left(z\right)\right)\left(P_{i}\left(z\right)-z\right)^{2}}+\frac{\left(1-z\right)\left(1-F_{i}\left(z\right)\right)P_{i}^{'}\left(z\right)}{\esp\left[C_{i}\right]\left(1-P_{i}\left(z\right)\right)\left(P_{i}\left(z\right)-z\right)}\\
&+&\frac{\left(1-z\right)\left(1-F_{i}\left(z\right)\right)P_{i}\left(z\right)P_{i}^{'}\left(z\right)}{\esp\left[C_{i}\right]\left(1-P_{i}\left(z\right)\right)^{2}\left(P_{i}\left(z\right)-z\right)}
\end{eqnarray*}
%___________________________________________________________________________________________
\subsection{Operaciones Matemathica: Tiempos de Espera}
%___________________________________________________________________________________________
Sea
$V_{i}\left(z\right)=\frac{1}{\esp\left[C_{i}\right]}\frac{I_{i}\left(z\right)-1}{z-P_{i}\left(z\right)}$

%{\esp\lef[I_{i}\right]}\frac{1-\mu_{i}}{z-P_{i}\left(z\right)}

\begin{eqnarray*}
\frac{\partial V_{i}\left(z\right)}{\partial z}&=&\frac{1}{\esp\left[C_{i}\right]}\left[\frac{I_{i}{'}\left(z\right)\left(z-P_{i}\left(z\right)\right)}{z-P_{i}\left(z\right)}-\frac{\left(I_{i}\left(z\right)-1\right)\left(1-P_{i}{'}\left(z\right)\right)}{\left(z-P_{i}\left(z\right)\right)^{2}}\right]
\end{eqnarray*}


La FGP para el tiempo de espera para cualquier usuario en la cola est\'a dada por:
\[U_{i}\left(z\right)=\frac{1}{\esp\left[C_{i}\right]}\cdot\frac{1-P_{i}\left(z\right)}{z-P_{i}\left(z\right)}\cdot\frac{I_{i}\left(z\right)-1}{1-z}\]

entonces


\begin{eqnarray*}
\frac{d}{dz}V_{i}\left(z\right)&=&\frac{1}{\esp\left[C_{i}\right]}\left\{\frac{d}{dz}\left(\frac{1-P_{i}\left(z\right)}{z-P_{i}\left(z\right)}\right)\frac{I_{i}\left(z\right)-1}{1-z}+\frac{1-P_{i}\left(z\right)}{z-P_{i}\left(z\right)}\frac{d}{dz}\left(\frac{I_{i}\left(z\right)-1}{1-z}\right)\right\}\\
&=&\frac{1}{\esp\left[C_{i}\right]}\left\{\frac{-P_{i}\left(z\right)\left(z-P_{i}\left(z\right)\right)-\left(1-P_{i}\left(z\right)\right)\left(1-P_{i}^{'}\left(z\right)\right)}{\left(z-P_{i}\left(z\right)\right)^{2}}\cdot\frac{I_{i}\left(z\right)-1}{1-z}\right\}\\
&+&\frac{1}{\esp\left[C_{i}\right]}\left\{\frac{1-P_{i}\left(z\right)}{z-P_{i}\left(z\right)}\cdot\frac{I_{i}^{'}\left(z\right)\left(1-z\right)+\left(I_{i}\left(z\right)-1\right)}{\left(1-z\right)^{2}}\right\}
\end{eqnarray*}
%\frac{I_{i}\left(z\right)-1}{1-z}
%+\frac{1-P_{i}\left(z\right)}{z-P_{i}\frac{d}{dz}\left(\frac{I_{i}\left(z\right)-1}{1-z}\right)


\begin{eqnarray*}
\frac{\partial U_{i}\left(z\right)}{\partial z}&=&\frac{(-1+I_{i}[z]) (1-P_{i}[z])}{(1-z)^2 \esp[I_{i}] (z-P_{i}[z])}+\frac{(1-P_{i}[z]) I_{i}^{'}[z]}{(1-z) \esp[I_{i}] (z-P_{i}[z])}-\frac{(-1+I_{i}[z]) (1-P_{i}[z])\left(1-P{'}[z]\right)}{(1-z) \esp[I_{i}] (z-P_{i}[z])^2}\\
&-&\frac{(-1+I_{i}[z]) P_{i}{'}[z]}{(1-z) \esp[I_{i}](z-P_{i}[z])}
\end{eqnarray*}


%_______________________________________________________________________________________________________
\section{Ya revisado}
%_______________________________________________________________________________________________________


Def\'inanse los puntos de regenaraci\'on  en el proceso $\left[L_{1}\left(t\right),L_{2}\left(t\right),\ldots,L_{N}\left(t\right)\right]$. Los puntos cuando la cola $i$ es visitada y todos los $L_{j}\left(\tau_{i}\left(m\right)\right)=0$ para $i=1,2$  son puntos de regeneraci\'on. Se llama ciclo regenerativo al intervalo entre dos puntos regenerativos sucesivos.

Sea $M_{i}$  el n\'umero de ciclos de visita en un ciclo regenerativo, y sea $C_{i}^{(m)}$, para $m=1,2,\ldots,M_{i}$ la duraci\'on del $m$-\'esimo ciclo de visita en un ciclo regenerativo. Se define el ciclo del tiempo de visita promedio $\esp\left[C_{i}\right]$ como

\begin{eqnarray*}
\esp\left[C_{i}\right]&=&\frac{\esp\left[\sum_{m=1}^{M_{i}}C_{i}^{(m)}\right]}{\esp\left[M_{i}\right]}
\end{eqnarray*}




Sea la funci\'on generadora de momentos para $L_{i}$, el n\'umero de usuarios en la cola $Q_{i}\left(z\right)$ en cualquier momento, est\'a dada por el tiempo promedio de $z^{L_{i}\left(t\right)}$ sobre el ciclo regenerativo definido anteriormente:

\begin{eqnarray*}
Q_{i}\left(z\right)&=&\esp\left[z^{L_{i}\left(t\right)}\right]=\frac{\esp\left[\sum_{m=1}^{M_{i}}\sum_{t=\tau_{i}\left(m\right)}^{\tau_{i}\left(m+1\right)-1}z^{L_{i}\left(t\right)}\right]}{\esp\left[\sum_{m=1}^{M_{i}}\tau_{i}\left(m+1\right)-\tau_{i}\left(m\right)\right]}
\end{eqnarray*}

$M_{i}$ es un tiempo de paro en el proceso regenerativo con $\esp\left[M_{i}\right]<\infty$, se sigue del lema de Wald que:


\begin{eqnarray*}
\esp\left[\sum_{m=1}^{M_{i}}\sum_{t=\tau_{i}\left(m\right)}^{\tau_{i}\left(m+1\right)-1}z^{L_{i}\left(t\right)}\right]&=&\esp\left[M_{i}\right]\esp\left[\sum_{t=\tau_{i}\left(m\right)}^{\tau_{i}\left(m+1\right)-1}z^{L_{i}\left(t\right)}\right]\\
\esp\left[\sum_{m=1}^{M_{i}}\tau_{i}\left(m+1\right)-\tau_{i}\left(m\right)\right]&=&\esp\left[M_{i}\right]\esp\left[\tau_{i}\left(m+1\right)-\tau_{i}\left(m\right)\right]
\end{eqnarray*}

por tanto se tiene que


\begin{eqnarray*}
Q_{i}\left(z\right)&=&\frac{\esp\left[\sum_{t=\tau_{i}\left(m\right)}^{\tau_{i}\left(m+1\right)-1}z^{L_{i}\left(t\right)}\right]}{\esp\left[\tau_{i}\left(m+1\right)-\tau_{i}\left(m\right)\right]}
\end{eqnarray*}

observar que el denominador es simplemente la duraci\'on promedio del tiempo del ciclo.


Se puede demostrar (ver Hideaki Takagi 1986) que

\begin{eqnarray*}
\esp\left[\sum_{t=\tau_{i}\left(m\right)}^{\tau_{i}\left(m+1\right)-1}z^{L_{i}\left(t\right)}\right]=z\frac{F_{i}\left(z\right)-1}{z-P_{i}\left(z\right)}
\end{eqnarray*}

Durante el tiempo de intervisita para la cola $i$, $L_{i}\left(t\right)$ solamente se incrementa de manera que el incremento por intervalo de tiempo est\'a dado por la funci\'on generadora de probabilidades de $P_{i}\left(z\right)$, por tanto la suma sobre el tiempo de intervisita puede evaluarse como:

\begin{eqnarray*}
\esp\left[\sum_{t=\tau_{i}\left(m\right)}^{\tau_{i}\left(m+1\right)-1}z^{L_{i}\left(t\right)}\right]&=&\esp\left[\sum_{t=\tau_{i}\left(m\right)}^{\tau_{i}\left(m+1\right)-1}\left\{P_{i}\left(z\right)\right\}^{t-\overline{\tau}_{i}\left(m\right)}\right]=\frac{1-\esp\left[\left\{P_{i}\left(z\right)\right\}^{\tau_{i}\left(m+1\right)-\overline{\tau}_{i}\left(m\right)}\right]}{1-P_{i}\left(z\right)}\\
&=&\frac{1-I_{i}\left[P_{i}\left(z\right)\right]}{1-P_{i}\left(z\right)}
\end{eqnarray*}
por tanto

\begin{eqnarray*}
\esp\left[\sum_{t=\tau_{i}\left(m\right)}^{\tau_{i}\left(m+1\right)-1}z^{L_{i}\left(t\right)}\right]&=&\frac{1-F_{i}\left(z\right)}{1-P_{i}\left(z\right)}
\end{eqnarray*}

Haciendo uso de lo hasta ahora desarrollado se tiene que

\begin{eqnarray*}
Q_{i}\left(z\right)&=&\frac{1}{\esp\left[C_{i}\right]}\cdot\frac{1-F_{i}\left(z\right)}{P_{i}\left(z\right)-z}\cdot\frac{\left(1-z\right)P_{i}\left(z\right)}{1-P_{i}\left(z\right)}\\
&=&\frac{\mu_{i}\left(1-\mu_{i}\right)}{f_{i}\left(i\right)}\cdot\frac{1-F_{i}\left(z\right)}{P_{i}\left(z\right)-z}\cdot\frac{\left(1-z\right)P_{i}\left(z\right)}{1-P_{i}\left(z\right)}
\end{eqnarray*}

\begin{Def}
Sea $L_{i}^{*}$el n\'umero de usuarios en la cola $Q_{i}$ cuando es visitada por el servidor para dar servicio, entonces

\begin{eqnarray}
\esp\left[L_{i}^{*}\right]&=&f_{i}\left(i\right)\\
Var\left[L_{i}^{*}\right]&=&f_{i}\left(i,i\right)+\esp\left[L_{i}^{*}\right]-\esp\left[L_{i}^{*}\right]^{2}.
\end{eqnarray}

\end{Def}


\begin{Def}
El tiempo de intervisita $I_{i}$ es el periodo de tiempo que comienza cuando se ha completado el servicio en un ciclo y termina cuando es visitada nuevamente en el pr\'oximo ciclo. Su  duraci\'on del mismo est\'a dada por $\tau_{i}\left(m+1\right)-\overline{\tau}_{i}\left(m\right)$.
\end{Def}


Recordemos las siguientes expresiones:

\begin{eqnarray*}
S_{i}\left(z\right)&=&\esp\left[z^{\overline{\tau}_{i}\left(m\right)-\tau_{i}\left(m\right)}\right]=F_{i}\left(\theta\left(z\right)\right),\\
F\left(z\right)&=&\esp\left[z^{L_{0}}\right],\\
P\left(z\right)&=&\esp\left[z^{X_{n}}\right],\\
F_{i}\left(z\right)&=&\esp\left[z^{L_{i}\left(\tau_{i}\left(m\right)\right)}\right],
\theta_{i}\left(z\right)-zP_{i}
\end{eqnarray*}

entonces 

\begin{eqnarray*}
\esp\left[S_{i}\right]&=&\frac{\esp\left[L_{i}^{*}\right]}{1-\mu_{i}}=\frac{f_{i}\left(i\right)}{1-\mu_{i}},\\
Var\left[S_{i}\right]&=&\frac{Var\left[L_{i}^{*}\right]}{\left(1-\mu_{i}\right)^{2}}+\frac{\sigma^{2}\esp\left[L_{i}^{*}\right]}{\left(1-\mu_{i}\right)^{3}}
\end{eqnarray*}

donde recordemos que

\begin{eqnarray*}
Var\left[L_{i}^{*}\right]&=&f_{i}\left(i,i\right)+f_{i}\left(i\right)-f_{i}\left(i\right)^{2}.
\end{eqnarray*}

La duraci\'on del tiempo de intervisita es $\tau_{i}\left(m+1\right)-\overline{\tau}\left(m\right)$. Dado que el n\'umero de usuarios presentes en $Q_{i}$ al tiempo $t=\tau_{i}\left(m+1\right)$ es igual al n\'umero de arribos durante el intervalo de tiempo $\left[\overline{\tau}\left(m\right),\tau_{i}\left(m+1\right)\right]$ se tiene que


\begin{eqnarray*}
\esp\left[z_{i}^{L_{i}\left(\tau_{i}\left(m+1\right)\right)}\right]=\esp\left[\left\{P_{i}\left(z_{i}\right)\right\}^{\tau_{i}\left(m+1\right)-\overline{\tau}\left(m\right)}\right]
\end{eqnarray*}

entonces, si \begin{eqnarray*}I_{i}\left(z\right)&=&\esp\left[z^{\tau_{i}\left(m+1\right)-\overline{\tau}\left(m\right)}\right]\end{eqnarray*} se tienen que

\begin{eqnarray*}
F_{i}\left(z\right)=I_{i}\left[P_{i}\left(z\right)\right]
\end{eqnarray*}
para $i=1,2$, por tanto



\begin{eqnarray*}
\esp\left[L_{i}^{*}\right]&=&\mu_{i}\esp\left[I_{i}\right]\\
Var\left[L_{i}^{*}\right]&=&\mu_{i}^{2}Var\left[I_{i}\right]+\sigma^{2}\esp\left[I_{i}\right]
\end{eqnarray*}
para $i=1,2$, por tanto


\begin{eqnarray*}
\esp\left[I_{i}\right]&=&\frac{f_{i}\left(i\right)}{\mu_{i}},
\end{eqnarray*}
adem\'as

\begin{eqnarray*}
Var\left[I_{i}\right]&=&\frac{Var\left[L_{i}^{*}\right]}{\mu_{i}^{2}}-\frac{\sigma_{i}^{2}}{\mu_{i}^{2}}f_{i}\left(i\right).
\end{eqnarray*}


Si  $C_{i}\left(z\right)=\esp\left[z^{\overline{\tau}\left(m+1\right)-\overline{\tau}_{i}\left(m\right)}\right]$el tiempo de duraci\'on del ciclo, entonces, por lo hasta ahora establecido, se tiene que

\begin{eqnarray*}
C_{i}\left(z\right)=I_{i}\left[\theta_{i}\left(z\right)\right],
\end{eqnarray*}
entonces

\begin{eqnarray*}
\esp\left[C_{i}\right]&=&\esp\left[I_{i}\right]\esp\left[\theta_{i}\left(z\right)\right]=\frac{\esp\left[L_{i}^{*}\right]}{\mu_{i}}\frac{1}{1-\mu_{i}}=\frac{f_{i}\left(i\right)}{\mu_{i}\left(1-\mu_{i}\right)}\\
Var\left[C_{i}\right]&=&\frac{Var\left[L_{i}^{*}\right]}{\mu_{i}^{2}\left(1-\mu_{i}\right)^{2}}.
\end{eqnarray*}

Por tanto se tienen las siguientes igualdades


\begin{eqnarray*}
\esp\left[S_{i}\right]&=&\mu_{i}\esp\left[C_{i}\right],\\
\esp\left[I_{i}\right]&=&\left(1-\mu_{i}\right)\esp\left[C_{i}\right]\\
\end{eqnarray*}

derivando con respecto a $z$



\begin{eqnarray*}
\frac{d Q_{i}\left(z\right)}{d z}&=&\frac{\left(1-F_{i}\left(z\right)\right)P_{i}\left(z\right)}{\esp\left[C_{i}\right]\left(1-P_{i}\left(z\right)\right)\left(P_{i}\left(z\right)-z\right)}\\
&-&\frac{\left(1-z\right)P_{i}\left(z\right)F_{i}^{'}\left(z\right)}{\esp\left[C_{i}\right]\left(1-P_{i}\left(z\right)\right)\left(P_{i}\left(z\right)-z\right)}\\
&-&\frac{\left(1-z\right)\left(1-F_{i}\left(z\right)\right)P_{i}\left(z\right)\left(P_{i}^{'}\left(z\right)-1\right)}{\esp\left[C_{i}\right]\left(1-P_{i}\left(z\right)\right)\left(P_{i}\left(z\right)-z\right)^{2}}\\
&+&\frac{\left(1-z\right)\left(1-F_{i}\left(z\right)\right)P_{i}^{'}\left(z\right)}{\esp\left[C_{i}\right]\left(1-P_{i}\left(z\right)\right)\left(P_{i}\left(z\right)-z\right)}\\
&+&\frac{\left(1-z\right)\left(1-F_{i}\left(z\right)\right)P_{i}\left(z\right)P_{i}^{'}\left(z\right)}{\esp\left[C_{i}\right]\left(1-P_{i}\left(z\right)\right)^{2}\left(P_{i}\left(z\right)-z\right)}
\end{eqnarray*}

Calculando el l\'imite cuando $z\rightarrow1^{+}$:
\begin{eqnarray}
Q_{i}^{(1)}\left(z\right)=\lim_{z\rightarrow1^{+}}\frac{d Q_{i}\left(z\right)}{dz}&=&\lim_{z\rightarrow1}\frac{\left(1-F_{i}\left(z\right)\right)P_{i}\left(z\right)}{\esp\left[C_{i}\right]\left(1-P_{i}\left(z\right)\right)\left(P_{i}\left(z\right)-z\right)}\\
&-&\lim_{z\rightarrow1^{+}}\frac{\left(1-z\right)P_{i}\left(z\right)F_{i}^{'}\left(z\right)}{\esp\left[C_{i}\right]\left(1-P_{i}\left(z\right)\right)\left(P_{i}\left(z\right)-z\right)}\\
&-&\lim_{z\rightarrow1^{+}}\frac{\left(1-z\right)\left(1-F_{i}\left(z\right)\right)P_{i}\left(z\right)\left(P_{i}^{'}\left(z\right)-1\right)}{\esp\left[C_{i}\right]\left(1-P_{i}\left(z\right)\right)\left(P_{i}\left(z\right)-z\right)^{2}}\\
&+&\lim_{z\rightarrow1^{+}}\frac{\left(1-z\right)\left(1-F_{i}\left(z\right)\right)P_{i}^{'}\left(z\right)}{\esp\left[C_{i}\right]\left(1-P_{i}\left(z\right)\right)\left(P_{i}\left(z\right)-z\right)}\\
&+&\lim_{z\rightarrow1^{+}}\frac{\left(1-z\right)\left(1-F_{i}\left(z\right)\right)P_{i}\left(z\right)P_{i}^{'}\left(z\right)}{\esp\left[C_{i}\right]\left(1-P_{i}\left(z\right)\right)^{2}\left(P_{i}\left(z\right)-z\right)}
\end{eqnarray}

Entonces:
%______________________________________________________

\begin{eqnarray*}
\lim_{z\rightarrow1^{+}}\frac{\left(1-F_{i}\left(z\right)\right)P_{i}\left(z\right)}{\left(1-P_{i}\left(z\right)\right)\left(P_{i}\left(z\right)-z\right)}&=&\lim_{z\rightarrow1^{+}}\frac{\frac{d}{dz}\left[\left(1-F_{i}\left(z\right)\right)P_{i}\left(z\right)\right]}{\frac{d}{dz}\left[\left(1-P_{i}\left(z\right)\right)\left(-z+P_{i}\left(z\right)\right)\right]}\\
&=&\lim_{z\rightarrow1^{+}}\frac{-P_{i}\left(z\right)F_{i}^{'}\left(z\right)+\left(1-F_{i}\left(z\right)\right)P_{i}^{'}\left(z\right)}{\left(1-P_{i}\left(z\right)\right)\left(-1+P_{i}^{'}\left(z\right)\right)-\left(-z+P_{i}\left(z\right)\right)P_{i}^{'}\left(z\right)}
\end{eqnarray*}


%______________________________________________________


\begin{eqnarray*}
\lim_{z\rightarrow1^{+}}\frac{\left(1-z\right)P_{i}\left(z\right)F_{i}^{'}\left(z\right)}{\left(1-P_{i}\left(z\right)\right)\left(P_{i}\left(z\right)-z\right)}&=&\lim_{z\rightarrow1^{+}}\frac{\frac{d}{dz}\left[\left(1-z\right)P_{i}\left(z\right)F_{i}^{'}\left(z\right)\right]}{\frac{d}{dz}\left[\left(1-P_{i}\left(z\right)\right)\left(P_{i}\left(z\right)-z\right)\right]}\\
&=&\lim_{z\rightarrow1^{+}}\frac{-P_{i}\left(z\right) F_{i}^{'}\left(z\right)+(1-z) F_{i}^{'}\left(z\right) P_{i}^{'}\left(z\right)+(1-z) P_{i}\left(z\right)F_{i}^{''}\left(z\right)}{\left(1-P_{i}\left(z\right)\right)\left(-1+P_{i}^{'}\left(z\right)\right)-\left(-z+P_{i}\left(z\right)\right)P_{i}^{'}\left(z\right)}
\end{eqnarray*}


%______________________________________________________

\begin{eqnarray*}
&&\lim_{z\rightarrow1^{+}}\frac{\left(1-z\right)\left(1-F_{i}\left(z\right)\right)P_{i}\left(z\right)\left(P_{i}^{'}\left(z\right)-1\right)}{\left(1-P_{i}\left(z\right)\right)\left(P_{i}\left(z\right)-z\right)^{2}}=\lim_{z\rightarrow1^{+}}\frac{\frac{d}{dz}\left[\left(1-z\right)\left(1-F_{i}\left(z\right)\right)P_{i}\left(z\right)\left(P_{i}^{'}\left(z\right)-1\right)\right]}{\frac{d}{dz}\left[\left(1-P_{i}\left(z\right)\right)\left(P_{i}\left(z\right)-z\right)^{2}\right]}\\
&=&\lim_{z\rightarrow1^{+}}\frac{-\left(1-F_{i}\left(z\right)\right) P_{i}\left(z\right)\left(-1+P_{i}^{'}\left(z\right)\right)-(1-z) P_{i}\left(z\right)F_{i}^{'}\left(z\right)\left(-1+P_{i}^{'}\left(z\right)\right)}{2\left(1-P_{i}\left(z\right)\right)\left(-z+P_{i}\left(z\right)\right) \left(-1+P_{i}^{'}\left(z\right)\right)-\left(-z+P_{i}\left(z\right)\right)^2 P_{i}^{'}\left(z\right)}\\
&+&\lim_{z\rightarrow1^{+}}\frac{+(1-z) \left(1-F_{i}\left(z\right)\right) \left(-1+P_{i}^{'}\left(z\right)\right) P_{i}^{'}\left(z\right)}{{2\left(1-P_{i}\left(z\right)\right)\left(-z+P_{i}\left(z\right)\right) \left(-1+P_{i}^{'}\left(z\right)\right)-\left(-z+P_{i}\left(z\right)\right)^2 P_{i}^{'}\left(z\right)}}\\
&+&\lim_{z\rightarrow1^{+}}\frac{+(1-z) \left(1-F_{i}\left(z\right)\right) P_{i}\left(z\right)P_{i}^{''}\left(z\right)}{{2\left(1-P_{i}\left(z\right)\right)\left(-z+P_{i}\left(z\right)\right) \left(-1+P_{i}^{'}\left(z\right)\right)-\left(-z+P_{i}\left(z\right)\right)^2 P_{i}^{'}\left(z\right)}}
\end{eqnarray*}











%______________________________________________________
\begin{eqnarray*}
&&\lim_{z\rightarrow1^{+}}\frac{\left(1-z\right)\left(1-F_{i}\left(z\right)\right)P_{i}^{'}\left(z\right)}{\left(1-P_{i}\left(z\right)\right)\left(P_{i}\left(z\right)-z\right)}=\lim_{z\rightarrow1^{+}}\frac{\frac{d}{dz}\left[\left(1-z\right)\left(1-F_{i}\left(z\right)\right)P_{i}^{'}\left(z\right)\right]}{\frac{d}{dz}\left[\left(1-P_{i}\left(z\right)\right)\left(P_{i}\left(z\right)-z\right)\right]}\\
&=&\lim_{z\rightarrow1^{+}}\frac{-\left(1-F_{i}\left(z\right)\right) P_{i}^{'}\left(z\right)-(1-z) F_{i}^{'}\left(z\right) P_{i}^{'}\left(z\right)+(1-z) \left(1-F_{i}\left(z\right)\right) P_{i}^{''}\left(z\right)}{\left(1-P_{i}\left(z\right)\right) \left(-1+P_{i}^{'}\left(z\right)\right)-\left(-z+P_{i}\left(z\right)\right) P_{i}^{'}\left(z\right)}\frac{}{}
\end{eqnarray*}

%______________________________________________________
\begin{eqnarray*}
&&\lim_{z\rightarrow1^{+}}\frac{\left(1-z\right)\left(1-F_{i}\left(z\right)\right)P_{i}\left(z\right)P_{i}^{'}\left(z\right)}{\left(1-P_{i}\left(z\right)\right)^{2}\left(P_{i}\left(z\right)-z\right)}=\lim_{z\rightarrow1^{+}}\frac{\frac{d}{dz}\left[\left(1-z\right)\left(1-F_{i}\left(z\right)\right)P_{i}\left(z\right)P_{i}^{'}\left(z\right)\right]}{\frac{d}{dz}\left[\left(1-P_{i}\left(z\right)\right)^{2}\left(P_{i}\left(z\right)-z\right)\right]}\\
&=&\lim_{z\rightarrow1^{+}}\frac{-\left(1-F_{i}\left(z\right)\right) P_{i}\left(z\right) P_{i}^{'}\left(z\right)-(1-z) P_{i}\left(z\right) F_{i}^{'}\left(z\right)P_i'[z]}{\left(1-P_{i}\left(z\right)\right)^2 \left(-1+P_{i}^{'}\left(z\right)\right)-2 \left(1-P_{i}\left(z\right)\right) \left(-z+P_{i}\left(z\right)\right) P_{i}^{'}\left(z\right)}\\
&+&\lim_{z\rightarrow1^{+}}\frac{(1-z) \left(1-F_{i}\left(z\right)\right) P_{i}^{'}\left(z\right)^2+(1-z) \left(1-F_{i}\left(z\right)\right) P_{i}\left(z\right) P_{i}^{''}\left(z\right)}{\left(1-P_{i}\left(z\right)\right)^2 \left(-1+P_{i}^{'}\left(z\right)\right)-2 \left(1-P_{i}\left(z\right)\right) \left(-z+P_{i}\left(z\right)\right) P_{i}^{'}\left(z\right)}\\
\end{eqnarray*}



En nuestra notaci\'on $V\left(t\right)\equiv C_{i}$ y $X_{i}=C_{i}^{(m)}$ para nuestra segunda definici\'on, mientras que para la primera la notaci\'on es: $X\left(t\right)\equiv C_{i}$ y $R_{i}\equiv C_{i}^{(m)}$.


%___________________________________________________________________________________________
%\section{Tiempos de Ciclo e Intervisita}
%___________________________________________________________________________________________


\begin{Def}
Sea $L_{i}^{*}$el n\'umero de usuarios en la cola $Q_{i}$ cuando es visitada por el servidor para dar servicio, entonces

\begin{eqnarray}
\esp\left[L_{i}^{*}\right]&=&f_{i}\left(i\right)\\
Var\left[L_{i}^{*}\right]&=&f_{i}\left(i,i\right)+\esp\left[L_{i}^{*}\right]-\esp\left[L_{i}^{*}\right]^{2}.
\end{eqnarray}

\end{Def}

\begin{Def}
El tiempo de Ciclo $C_{i}$ es e periodo de tiempo que comienza cuando la cola $i$ es visitada por primera vez en un ciclo, y termina cuando es visitado nuevamente en el pr\'oximo ciclo. La duraci\'on del mismo est\'a dada por $\tau_{i}\left(m+1\right)-\tau_{i}\left(m\right)$, o equivalentemente $\overline{\tau}_{i}\left(m+1\right)-\overline{\tau}_{i}\left(m\right)$ bajo condiciones de estabilidad.
\end{Def}

\begin{Def}
El tiempo de intervisita $I_{i}$ es el periodo de tiempo que comienza cuando se ha completado el servicio en un ciclo y termina cuando es visitada nuevamente en el pr\'oximo ciclo. Su  duraci\'on del mismo est\'a dada por $\tau_{i}\left(m+1\right)-\overline{\tau}_{i}\left(m\right)$.
\end{Def}


Recordemos las siguientes expresiones:

\begin{eqnarray*}
S_{i}\left(z\right)&=&\esp\left[z^{\overline{\tau}_{i}\left(m\right)-\tau_{i}\left(m\right)}\right]=F_{i}\left(\theta\left(z\right)\right),\\
F\left(z\right)&=&\esp\left[z^{L_{0}}\right],\\
P\left(z\right)&=&\esp\left[z^{X_{n}}\right],\\
F_{i}\left(z\right)&=&\esp\left[z^{L_{i}\left(\tau_{i}\left(m\right)\right)}\right],
\theta_{i}\left(z\right)-zP_{i}
\end{eqnarray*}

entonces 

\begin{eqnarray*}
\esp\left[S_{i}\right]&=&\frac{\esp\left[L_{i}^{*}\right]}{1-\mu_{i}}=\frac{f_{i}\left(i\right)}{1-\mu_{i}},\\
Var\left[S_{i}\right]&=&\frac{Var\left[L_{i}^{*}\right]}{\left(1-\mu_{i}\right)^{2}}+\frac{\sigma^{2}\esp\left[L_{i}^{*}\right]}{\left(1-\mu_{i}\right)^{3}}
\end{eqnarray*}

donde recordemos que

\begin{eqnarray*}
Var\left[L_{i}^{*}\right]&=&f_{i}\left(i,i\right)+f_{i}\left(i\right)-f_{i}\left(i\right)^{2}.
\end{eqnarray*}

La duraci\'on del tiempo de intervisita es $\tau_{i}\left(m+1\right)-\overline{\tau}\left(m\right)$. Dado que el n\'umero de usuarios presentes en $Q_{i}$ al tiempo $t=\tau_{i}\left(m+1\right)$ es igual al n\'umero de arribos durante el intervalo de tiempo $\left[\overline{\tau}\left(m\right),\tau_{i}\left(m+1\right)\right]$ se tiene que


\begin{eqnarray*}
\esp\left[z_{i}^{L_{i}\left(\tau_{i}\left(m+1\right)\right)}\right]=\esp\left[\left\{P_{i}\left(z_{i}\right)\right\}^{\tau_{i}\left(m+1\right)-\overline{\tau}\left(m\right)}\right]
\end{eqnarray*}

entonces, si \begin{eqnarray*}I_{i}\left(z\right)&=&\esp\left[z^{\tau_{i}\left(m+1\right)-\overline{\tau}\left(m\right)}\right]\end{eqnarray*} se tienen que

\begin{eqnarray*}
F_{i}\left(z\right)=I_{i}\left[P_{i}\left(z\right)\right]
\end{eqnarray*}
para $i=1,2$, por tanto



\begin{eqnarray*}
\esp\left[L_{i}^{*}\right]&=&\mu_{i}\esp\left[I_{i}\right]\\
Var\left[L_{i}^{*}\right]&=&\mu_{i}^{2}Var\left[I_{i}\right]+\sigma^{2}\esp\left[I_{i}\right]
\end{eqnarray*}
para $i=1,2$, por tanto


\begin{eqnarray*}
\esp\left[I_{i}\right]&=&\frac{f_{i}\left(i\right)}{\mu_{i}},
\end{eqnarray*}
adem\'as

\begin{eqnarray*}
Var\left[I_{i}\right]&=&\frac{Var\left[L_{i}^{*}\right]}{\mu_{i}^{2}}-\frac{\sigma_{i}^{2}}{\mu_{i}^{2}}f_{i}\left(i\right).
\end{eqnarray*}


Si  $C_{i}\left(z\right)=\esp\left[z^{\overline{\tau}\left(m+1\right)-\overline{\tau}_{i}\left(m\right)}\right]$el tiempo de duraci\'on del ciclo, entonces, por lo hasta ahora establecido, se tiene que

\begin{eqnarray*}
C_{i}\left(z\right)=I_{i}\left[\theta_{i}\left(z\right)\right],
\end{eqnarray*}
entonces

\begin{eqnarray*}
\esp\left[C_{i}\right]&=&\esp\left[I_{i}\right]\esp\left[\theta_{i}\left(z\right)\right]=\frac{\esp\left[L_{i}^{*}\right]}{\mu_{i}}\frac{1}{1-\mu_{i}}=\frac{f_{i}\left(i\right)}{\mu_{i}\left(1-\mu_{i}\right)}\\
Var\left[C_{i}\right]&=&\frac{Var\left[L_{i}^{*}\right]}{\mu_{i}^{2}\left(1-\mu_{i}\right)^{2}}.
\end{eqnarray*}

Por tanto se tienen las siguientes igualdades


\begin{eqnarray*}
\esp\left[S_{i}\right]&=&\mu_{i}\esp\left[C_{i}\right],\\
\esp\left[I_{i}\right]&=&\left(1-\mu_{i}\right)\esp\left[C_{i}\right]\\
\end{eqnarray*}

Def\'inanse los puntos de regenaraci\'on  en el proceso $\left[L_{1}\left(t\right),L_{2}\left(t\right),\ldots,L_{N}\left(t\right)\right]$. Los puntos cuando la cola $i$ es visitada y todos los $L_{j}\left(\tau_{i}\left(m\right)\right)=0$ para $i=1,2$  son puntos de regeneraci\'on. Se llama ciclo regenerativo al intervalo entre dos puntos regenerativos sucesivos.

Sea $M_{i}$  el n\'umero de ciclos de visita en un ciclo regenerativo, y sea $C_{i}^{(m)}$, para $m=1,2,\ldots,M_{i}$ la duraci\'on del $m$-\'esimo ciclo de visita en un ciclo regenerativo. Se define el ciclo del tiempo de visita promedio $\esp\left[C_{i}\right]$ como

\begin{eqnarray*}
\esp\left[C_{i}\right]&=&\frac{\esp\left[\sum_{m=1}^{M_{i}}C_{i}^{(m)}\right]}{\esp\left[M_{i}\right]}
\end{eqnarray*}


En Stid72 y Heym82 se muestra que una condici\'on suficiente para que el proceso regenerativo 
estacionario sea un procesoo estacionario es que el valor esperado del tiempo del ciclo regenerativo sea finito:

\begin{eqnarray*}
\esp\left[\sum_{m=1}^{M_{i}}C_{i}^{(m)}\right]<\infty.
\end{eqnarray*}

como cada $C_{i}^{(m)}$ contiene intervalos de r\'eplica positivos, se tiene que $\esp\left[M_{i}\right]<\infty$, adem\'as, como $M_{i}>0$, se tiene que la condici\'on anterior es equivalente a tener que 

\begin{eqnarray*}
\esp\left[C_{i}\right]<\infty,
\end{eqnarray*}
por lo tanto una condici\'on suficiente para la existencia del proceso regenerativo est\'a dada por

\begin{eqnarray*}
\sum_{k=1}^{N}\mu_{k}<1.
\end{eqnarray*}



\begin{Note}\label{Cita1.Stidham}
En Stidham\cite{Stidham} y Heyman  se muestra que una condici\'on suficiente para que el proceso regenerativo 
estacionario sea un procesoo estacionario es que el valor esperado del tiempo del ciclo regenerativo sea finito:

\begin{eqnarray*}
\esp\left[\sum_{m=1}^{M_{i}}C_{i}^{(m)}\right]<\infty.
\end{eqnarray*}

como cada $C_{i}^{(m)}$ contiene intervalos de r\'eplica positivos, se tiene que $\esp\left[M_{i}\right]<\infty$, adem\'as, como $M_{i}>0$, se tiene que la condici\'on anterior es equivalente a tener que 

\begin{eqnarray*}
\esp\left[C_{i}\right]<\infty,
\end{eqnarray*}
por lo tanto una condici\'on suficiente para la existencia del proceso regenerativo est\'a dada por

\begin{eqnarray*}
\sum_{k=1}^{N}\mu_{k}<1.
\end{eqnarray*}

{\centering{\Huge{\textbf{Nota incompleta!!}}}}
\end{Note}

%_______________________________________________________________________________________
\subsection{Procesos de Renovaci\'on y Regenerativos}
%_______________________________________________________________________________________



Se puede demostrar (ver Hideaki Takagi 1986) que

\begin{eqnarray*}
\esp\left[\sum_{t=\tau_{i}\left(m\right)}^{\tau_{i}\left(m+1\right)-1}z^{L_{i}\left(t\right)}\right]=z\frac{F_{i}\left(z\right)-1}{z-P_{i}\left(z\right)}
\end{eqnarray*}

Durante el tiempo de intervisita para la cola $i$, $L_{i}\left(t\right)$ solamente se incrementa de manera que el incremento por intervalo de tiempo est\'a dado por la funci\'on generadora de probabilidades de $P_{i}\left(z\right)$, por tanto la suma sobre el tiempo de intervisita puede evaluarse como:

\begin{eqnarray*}
\esp\left[\sum_{t=\tau_{i}\left(m\right)}^{\tau_{i}\left(m+1\right)-1}z^{L_{i}\left(t\right)}\right]&=&\esp\left[\sum_{t=\tau_{i}\left(m\right)}^{\tau_{i}\left(m+1\right)-1}\left\{P_{i}\left(z\right)\right\}^{t-\overline{\tau}_{i}\left(m\right)}\right]=\frac{1-\esp\left[\left\{P_{i}\left(z\right)\right\}^{\tau_{i}\left(m+1\right)-\overline{\tau}_{i}\left(m\right)}\right]}{1-P_{i}\left(z\right)}\\
&=&\frac{1-I_{i}\left[P_{i}\left(z\right)\right]}{1-P_{i}\left(z\right)}
\end{eqnarray*}
por tanto

\begin{eqnarray*}
\esp\left[\sum_{t=\tau_{i}\left(m\right)}^{\tau_{i}\left(m+1\right)-1}z^{L_{i}\left(t\right)}\right]&=&\frac{1-F_{i}\left(z\right)}{1-P_{i}\left(z\right)}
\end{eqnarray*}

Haciendo uso de lo hasta ahora desarrollado se tiene que



%___________________________________________________________________________________________
%\subsection{Longitudes de la Cola en cualquier tiempo}
%___________________________________________________________________________________________
Sea 
\begin{eqnarray*}
Q_{i}\left(z\right)&=&\frac{1}{\esp\left[C_{i}\right]}\cdot\frac{1-F_{i}\left(z\right)}{P_{i}\left(z\right)-z}\cdot\frac{\left(1-z\right)P_{i}\left(z\right)}{1-P_{i}\left(z\right)}
\end{eqnarray*}

Consideremos una cola de la red de sistemas de visitas c\'iclicas fija, $Q_{l}$.


Conforme a la definici\'on dada al principio del cap\'itulo, definici\'on (\ref{Def.Tn}), sean $T_{1},T_{2},\ldots$ los puntos donde las longitudes de las colas de la red de sistemas de visitas c\'iclicas son cero simult\'aneamente, cuando la cola $Q_{l}$ es visitada por el servidor para dar servicio, es decir, $L_{1}\left(T_{i}\right)=0,L_{2}\left(T_{i}\right)=0,\hat{L}_{1}\left(T_{i}\right)=0$ y $\hat{L}_{2}\left(T_{i}\right)=0$, a estos puntos se les denominar\'a puntos regenerativos. Entonces, 

\begin{Def}
Al intervalo de tiempo entre dos puntos regenerativos se le llamar\'a ciclo regenerativo.
\end{Def}

\begin{Def}
Para $T_{i}$ se define, $M_{i}$, el n\'umero de ciclos de visita a la cola $Q_{l}$, durante el ciclo regenerativo, es decir, $M_{i}$ es un proceso de renovaci\'on.
\end{Def}

\begin{Def}
Para cada uno de los $M_{i}$'s, se definen a su vez la duraci\'on de cada uno de estos ciclos de visita en el ciclo regenerativo, $C_{i}^{(m)}$, para $m=1,2,\ldots,M_{i}$, que a su vez, tambi\'en es n proceso de renovaci\'on.
\end{Def}

En nuestra notaci\'on $V\left(t\right)\equiv C_{i}$ y $X_{i}=C_{i}^{(m)}$ para nuestra segunda definici\'on, mientras que para la primera la notaci\'on es: $X\left(t\right)\equiv C_{i}$ y $R_{i}\equiv C_{i}^{(m)}$.


%___________________________________________________________________________________________
%\subsection{Tiempos de Ciclo e Intervisita}
%___________________________________________________________________________________________


\begin{Def}
Sea $L_{i}^{*}$el n\'umero de usuarios en la cola $Q_{i}$ cuando es visitada por el servidor para dar servicio, entonces

\begin{eqnarray}
\esp\left[L_{i}^{*}\right]&=&f_{i}\left(i\right)\\
Var\left[L_{i}^{*}\right]&=&f_{i}\left(i,i\right)+\esp\left[L_{i}^{*}\right]-\esp\left[L_{i}^{*}\right]^{2}.
\end{eqnarray}

\end{Def}

\begin{Def}
El tiempo de Ciclo $C_{i}$ es e periodo de tiempo que comienza cuando la cola $i$ es visitada por primera vez en un ciclo, y termina cuando es visitado nuevamente en el pr\'oximo ciclo. La duraci\'on del mismo est\'a dada por $\tau_{i}\left(m+1\right)-\tau_{i}\left(m\right)$, o equivalentemente $\overline{\tau}_{i}\left(m+1\right)-\overline{\tau}_{i}\left(m\right)$ bajo condiciones de estabilidad.
\end{Def}



Recordemos las siguientes expresiones:

\begin{eqnarray*}
S_{i}\left(z\right)&=&\esp\left[z^{\overline{\tau}_{i}\left(m\right)-\tau_{i}\left(m\right)}\right]=F_{i}\left(\theta\left(z\right)\right),\\
F\left(z\right)&=&\esp\left[z^{L_{0}}\right],\\
P\left(z\right)&=&\esp\left[z^{X_{n}}\right],\\
F_{i}\left(z\right)&=&\esp\left[z^{L_{i}\left(\tau_{i}\left(m\right)\right)}\right],
\theta_{i}\left(z\right)-zP_{i}
\end{eqnarray*}

entonces 

\begin{eqnarray*}
\esp\left[S_{i}\right]&=&\frac{\esp\left[L_{i}^{*}\right]}{1-\mu_{i}}=\frac{f_{i}\left(i\right)}{1-\mu_{i}},\\
Var\left[S_{i}\right]&=&\frac{Var\left[L_{i}^{*}\right]}{\left(1-\mu_{i}\right)^{2}}+\frac{\sigma^{2}\esp\left[L_{i}^{*}\right]}{\left(1-\mu_{i}\right)^{3}}
\end{eqnarray*}

donde recordemos que

\begin{eqnarray*}
Var\left[L_{i}^{*}\right]&=&f_{i}\left(i,i\right)+f_{i}\left(i\right)-f_{i}\left(i\right)^{2}.
\end{eqnarray*}

 por tanto


\begin{eqnarray*}
\esp\left[I_{i}\right]&=&\frac{f_{i}\left(i\right)}{\mu_{i}},
\end{eqnarray*}
adem\'as

\begin{eqnarray*}
Var\left[I_{i}\right]&=&\frac{Var\left[L_{i}^{*}\right]}{\mu_{i}^{2}}-\frac{\sigma_{i}^{2}}{\mu_{i}^{2}}f_{i}\left(i\right).
\end{eqnarray*}


Si  $C_{i}\left(z\right)=\esp\left[z^{\overline{\tau}\left(m+1\right)-\overline{\tau}_{i}\left(m\right)}\right]$el tiempo de duraci\'on del ciclo, entonces, por lo hasta ahora establecido, se tiene que

\begin{eqnarray*}
C_{i}\left(z\right)=I_{i}\left[\theta_{i}\left(z\right)\right],
\end{eqnarray*}
entonces

\begin{eqnarray*}
\esp\left[C_{i}\right]&=&\esp\left[I_{i}\right]\esp\left[\theta_{i}\left(z\right)\right]=\frac{\esp\left[L_{i}^{*}\right]}{\mu_{i}}\frac{1}{1-\mu_{i}}=\frac{f_{i}\left(i\right)}{\mu_{i}\left(1-\mu_{i}\right)}\\
Var\left[C_{i}\right]&=&\frac{Var\left[L_{i}^{*}\right]}{\mu_{i}^{2}\left(1-\mu_{i}\right)^{2}}.
\end{eqnarray*}

Por tanto se tienen las siguientes igualdades


\begin{eqnarray*}
\esp\left[S_{i}\right]&=&\mu_{i}\esp\left[C_{i}\right],\\
\esp\left[I_{i}\right]&=&\left(1-\mu_{i}\right)\esp\left[C_{i}\right]\\
\end{eqnarray*}

Def\'inanse los puntos de regenaraci\'on  en el proceso $\left[L_{1}\left(t\right),L_{2}\left(t\right),\ldots,L_{N}\left(t\right)\right]$. Los puntos cuando la cola $i$ es visitada y todos los $L_{j}\left(\tau_{i}\left(m\right)\right)=0$ para $i=1,2$  son puntos de regeneraci\'on. Se llama ciclo regenerativo al intervalo entre dos puntos regenerativos sucesivos.

Sea $M_{i}$  el n\'umero de ciclos de visita en un ciclo regenerativo, y sea $C_{i}^{(m)}$, para $m=1,2,\ldots,M_{i}$ la duraci\'on del $m$-\'esimo ciclo de visita en un ciclo regenerativo. Se define el ciclo del tiempo de visita promedio $\esp\left[C_{i}\right]$ como

\begin{eqnarray*}
\esp\left[C_{i}\right]&=&\frac{\esp\left[\sum_{m=1}^{M_{i}}C_{i}^{(m)}\right]}{\esp\left[M_{i}\right]}
\end{eqnarray*}


En Stid72 y Heym82 se muestra que una condici\'on suficiente para que el proceso regenerativo 
estacionario sea un procesoo estacionario es que el valor esperado del tiempo del ciclo regenerativo sea finito:

\begin{eqnarray*}
\esp\left[\sum_{m=1}^{M_{i}}C_{i}^{(m)}\right]<\infty.
\end{eqnarray*}

como cada $C_{i}^{(m)}$ contiene intervalos de r\'eplica positivos, se tiene que $\esp\left[M_{i}\right]<\infty$, adem\'as, como $M_{i}>0$, se tiene que la condici\'on anterior es equivalente a tener que 

\begin{eqnarray*}
\esp\left[C_{i}\right]<\infty,
\end{eqnarray*}
por lo tanto una condici\'on suficiente para la existencia del proceso regenerativo est\'a dada por

\begin{eqnarray*}
\sum_{k=1}^{N}\mu_{k}<1.
\end{eqnarray*}

Sea la funci\'on generadora de momentos para $L_{i}$, el n\'umero de usuarios en la cola $Q_{i}\left(z\right)$ en cualquier momento, est\'a dada por el tiempo promedio de $z^{L_{i}\left(t\right)}$ sobre el ciclo regenerativo definido anteriormente:

\begin{eqnarray*}
Q_{i}\left(z\right)&=&\esp\left[z^{L_{i}\left(t\right)}\right]=\frac{\esp\left[\sum_{m=1}^{M_{i}}\sum_{t=\tau_{i}\left(m\right)}^{\tau_{i}\left(m+1\right)-1}z^{L_{i}\left(t\right)}\right]}{\esp\left[\sum_{m=1}^{M_{i}}\tau_{i}\left(m+1\right)-\tau_{i}\left(m\right)\right]}
\end{eqnarray*}

$M_{i}$ es un tiempo de paro en el proceso regenerativo con $\esp\left[M_{i}\right]<\infty$, se sigue del lema de Wald que:


\begin{eqnarray*}
\esp\left[\sum_{m=1}^{M_{i}}\sum_{t=\tau_{i}\left(m\right)}^{\tau_{i}\left(m+1\right)-1}z^{L_{i}\left(t\right)}\right]&=&\esp\left[M_{i}\right]\esp\left[\sum_{t=\tau_{i}\left(m\right)}^{\tau_{i}\left(m+1\right)-1}z^{L_{i}\left(t\right)}\right]\\
\esp\left[\sum_{m=1}^{M_{i}}\tau_{i}\left(m+1\right)-\tau_{i}\left(m\right)\right]&=&\esp\left[M_{i}\right]\esp\left[\tau_{i}\left(m+1\right)-\tau_{i}\left(m\right)\right]
\end{eqnarray*}

por tanto se tiene que


\begin{eqnarray*}
Q_{i}\left(z\right)&=&\frac{\esp\left[\sum_{t=\tau_{i}\left(m\right)}^{\tau_{i}\left(m+1\right)-1}z^{L_{i}\left(t\right)}\right]}{\esp\left[\tau_{i}\left(m+1\right)-\tau_{i}\left(m\right)\right]}
\end{eqnarray*}

observar que el denominador es simplemente la duraci\'on promedio del tiempo del ciclo.


Se puede demostrar (ver Hideaki Takagi 1986) que

\begin{eqnarray*}
\esp\left[\sum_{t=\tau_{i}\left(m\right)}^{\tau_{i}\left(m+1\right)-1}z^{L_{i}\left(t\right)}\right]=z\frac{F_{i}\left(z\right)-1}{z-P_{i}\left(z\right)}
\end{eqnarray*}

Durante el tiempo de intervisita para la cola $i$, $L_{i}\left(t\right)$ solamente se incrementa de manera que el incremento por intervalo de tiempo est\'a dado por la funci\'on generadora de probabilidades de $P_{i}\left(z\right)$, por tanto la suma sobre el tiempo de intervisita puede evaluarse como:

\begin{eqnarray*}
\esp\left[\sum_{t=\tau_{i}\left(m\right)}^{\tau_{i}\left(m+1\right)-1}z^{L_{i}\left(t\right)}\right]&=&\esp\left[\sum_{t=\tau_{i}\left(m\right)}^{\tau_{i}\left(m+1\right)-1}\left\{P_{i}\left(z\right)\right\}^{t-\overline{\tau}_{i}\left(m\right)}\right]=\frac{1-\esp\left[\left\{P_{i}\left(z\right)\right\}^{\tau_{i}\left(m+1\right)-\overline{\tau}_{i}\left(m\right)}\right]}{1-P_{i}\left(z\right)}\\
&=&\frac{1-I_{i}\left[P_{i}\left(z\right)\right]}{1-P_{i}\left(z\right)}
\end{eqnarray*}
por tanto

\begin{eqnarray*}
\esp\left[\sum_{t=\tau_{i}\left(m\right)}^{\tau_{i}\left(m+1\right)-1}z^{L_{i}\left(t\right)}\right]&=&\frac{1-F_{i}\left(z\right)}{1-P_{i}\left(z\right)}
\end{eqnarray*}

Haciendo uso de lo hasta ahora desarrollado se tiene que

\begin{eqnarray*}
Q_{i}\left(z\right)&=&\frac{1}{\esp\left[C_{i}\right]}\cdot\frac{1-F_{i}\left(z\right)}{P_{i}\left(z\right)-z}\cdot\frac{\left(1-z\right)P_{i}\left(z\right)}{1-P_{i}\left(z\right)}\\
&=&\frac{\mu_{i}\left(1-\mu_{i}\right)}{f_{i}\left(i\right)}\cdot\frac{1-F_{i}\left(z\right)}{P_{i}\left(z\right)-z}\cdot\frac{\left(1-z\right)P_{i}\left(z\right)}{1-P_{i}\left(z\right)}
\end{eqnarray*}

derivando con respecto a $z$



\begin{eqnarray*}
\frac{d Q_{i}\left(z\right)}{d z}&=&\frac{\left(1-F_{i}\left(z\right)\right)P_{i}\left(z\right)}{\esp\left[C_{i}\right]\left(1-P_{i}\left(z\right)\right)\left(P_{i}\left(z\right)-z\right)}\\
&-&\frac{\left(1-z\right)P_{i}\left(z\right)F_{i}^{'}\left(z\right)}{\esp\left[C_{i}\right]\left(1-P_{i}\left(z\right)\right)\left(P_{i}\left(z\right)-z\right)}\\
&-&\frac{\left(1-z\right)\left(1-F_{i}\left(z\right)\right)P_{i}\left(z\right)\left(P_{i}^{'}\left(z\right)-1\right)}{\esp\left[C_{i}\right]\left(1-P_{i}\left(z\right)\right)\left(P_{i}\left(z\right)-z\right)^{2}}\\
&+&\frac{\left(1-z\right)\left(1-F_{i}\left(z\right)\right)P_{i}^{'}\left(z\right)}{\esp\left[C_{i}\right]\left(1-P_{i}\left(z\right)\right)\left(P_{i}\left(z\right)-z\right)}\\
&+&\frac{\left(1-z\right)\left(1-F_{i}\left(z\right)\right)P_{i}\left(z\right)P_{i}^{'}\left(z\right)}{\esp\left[C_{i}\right]\left(1-P_{i}\left(z\right)\right)^{2}\left(P_{i}\left(z\right)-z\right)}
\end{eqnarray*}

Calculando el l\'imite cuando $z\rightarrow1^{+}$:
\begin{eqnarray}
Q_{i}^{(1)}\left(z\right)=\lim_{z\rightarrow1^{+}}\frac{d Q_{i}\left(z\right)}{dz}&=&\lim_{z\rightarrow1}\frac{\left(1-F_{i}\left(z\right)\right)P_{i}\left(z\right)}{\esp\left[C_{i}\right]\left(1-P_{i}\left(z\right)\right)\left(P_{i}\left(z\right)-z\right)}\\
&-&\lim_{z\rightarrow1^{+}}\frac{\left(1-z\right)P_{i}\left(z\right)F_{i}^{'}\left(z\right)}{\esp\left[C_{i}\right]\left(1-P_{i}\left(z\right)\right)\left(P_{i}\left(z\right)-z\right)}\\
&-&\lim_{z\rightarrow1^{+}}\frac{\left(1-z\right)\left(1-F_{i}\left(z\right)\right)P_{i}\left(z\right)\left(P_{i}^{'}\left(z\right)-1\right)}{\esp\left[C_{i}\right]\left(1-P_{i}\left(z\right)\right)\left(P_{i}\left(z\right)-z\right)^{2}}\\
&+&\lim_{z\rightarrow1^{+}}\frac{\left(1-z\right)\left(1-F_{i}\left(z\right)\right)P_{i}^{'}\left(z\right)}{\esp\left[C_{i}\right]\left(1-P_{i}\left(z\right)\right)\left(P_{i}\left(z\right)-z\right)}\\
&+&\lim_{z\rightarrow1^{+}}\frac{\left(1-z\right)\left(1-F_{i}\left(z\right)\right)P_{i}\left(z\right)P_{i}^{'}\left(z\right)}{\esp\left[C_{i}\right]\left(1-P_{i}\left(z\right)\right)^{2}\left(P_{i}\left(z\right)-z\right)}
\end{eqnarray}

Entonces:
%______________________________________________________

\begin{eqnarray*}
\lim_{z\rightarrow1^{+}}\frac{\left(1-F_{i}\left(z\right)\right)P_{i}\left(z\right)}{\left(1-P_{i}\left(z\right)\right)\left(P_{i}\left(z\right)-z\right)}&=&\lim_{z\rightarrow1^{+}}\frac{\frac{d}{dz}\left[\left(1-F_{i}\left(z\right)\right)P_{i}\left(z\right)\right]}{\frac{d}{dz}\left[\left(1-P_{i}\left(z\right)\right)\left(-z+P_{i}\left(z\right)\right)\right]}\\
&=&\lim_{z\rightarrow1^{+}}\frac{-P_{i}\left(z\right)F_{i}^{'}\left(z\right)+\left(1-F_{i}\left(z\right)\right)P_{i}^{'}\left(z\right)}{\left(1-P_{i}\left(z\right)\right)\left(-1+P_{i}^{'}\left(z\right)\right)-\left(-z+P_{i}\left(z\right)\right)P_{i}^{'}\left(z\right)}
\end{eqnarray*}


%______________________________________________________


\begin{eqnarray*}
\lim_{z\rightarrow1^{+}}\frac{\left(1-z\right)P_{i}\left(z\right)F_{i}^{'}\left(z\right)}{\left(1-P_{i}\left(z\right)\right)\left(P_{i}\left(z\right)-z\right)}&=&\lim_{z\rightarrow1^{+}}\frac{\frac{d}{dz}\left[\left(1-z\right)P_{i}\left(z\right)F_{i}^{'}\left(z\right)\right]}{\frac{d}{dz}\left[\left(1-P_{i}\left(z\right)\right)\left(P_{i}\left(z\right)-z\right)\right]}\\
&=&\lim_{z\rightarrow1^{+}}\frac{-P_{i}\left(z\right) F_{i}^{'}\left(z\right)+(1-z) F_{i}^{'}\left(z\right) P_{i}^{'}\left(z\right)+(1-z) P_{i}\left(z\right)F_{i}^{''}\left(z\right)}{\left(1-P_{i}\left(z\right)\right)\left(-1+P_{i}^{'}\left(z\right)\right)-\left(-z+P_{i}\left(z\right)\right)P_{i}^{'}\left(z\right)}
\end{eqnarray*}


%______________________________________________________

\begin{eqnarray*}
&&\lim_{z\rightarrow1^{+}}\frac{\left(1-z\right)\left(1-F_{i}\left(z\right)\right)P_{i}\left(z\right)\left(P_{i}^{'}\left(z\right)-1\right)}{\left(1-P_{i}\left(z\right)\right)\left(P_{i}\left(z\right)-z\right)^{2}}=\lim_{z\rightarrow1^{+}}\frac{\frac{d}{dz}\left[\left(1-z\right)\left(1-F_{i}\left(z\right)\right)P_{i}\left(z\right)\left(P_{i}^{'}\left(z\right)-1\right)\right]}{\frac{d}{dz}\left[\left(1-P_{i}\left(z\right)\right)\left(P_{i}\left(z\right)-z\right)^{2}\right]}\\
&=&\lim_{z\rightarrow1^{+}}\frac{-\left(1-F_{i}\left(z\right)\right) P_{i}\left(z\right)\left(-1+P_{i}^{'}\left(z\right)\right)-(1-z) P_{i}\left(z\right)F_{i}^{'}\left(z\right)\left(-1+P_{i}^{'}\left(z\right)\right)}{2\left(1-P_{i}\left(z\right)\right)\left(-z+P_{i}\left(z\right)\right) \left(-1+P_{i}^{'}\left(z\right)\right)-\left(-z+P_{i}\left(z\right)\right)^2 P_{i}^{'}\left(z\right)}\\
&+&\lim_{z\rightarrow1^{+}}\frac{+(1-z) \left(1-F_{i}\left(z\right)\right) \left(-1+P_{i}^{'}\left(z\right)\right) P_{i}^{'}\left(z\right)}{{2\left(1-P_{i}\left(z\right)\right)\left(-z+P_{i}\left(z\right)\right) \left(-1+P_{i}^{'}\left(z\right)\right)-\left(-z+P_{i}\left(z\right)\right)^2 P_{i}^{'}\left(z\right)}}\\
&+&\lim_{z\rightarrow1^{+}}\frac{+(1-z) \left(1-F_{i}\left(z\right)\right) P_{i}\left(z\right)P_{i}^{''}\left(z\right)}{{2\left(1-P_{i}\left(z\right)\right)\left(-z+P_{i}\left(z\right)\right) \left(-1+P_{i}^{'}\left(z\right)\right)-\left(-z+P_{i}\left(z\right)\right)^2 P_{i}^{'}\left(z\right)}}
\end{eqnarray*}











%______________________________________________________
\begin{eqnarray*}
&&\lim_{z\rightarrow1^{+}}\frac{\left(1-z\right)\left(1-F_{i}\left(z\right)\right)P_{i}^{'}\left(z\right)}{\left(1-P_{i}\left(z\right)\right)\left(P_{i}\left(z\right)-z\right)}=\lim_{z\rightarrow1^{+}}\frac{\frac{d}{dz}\left[\left(1-z\right)\left(1-F_{i}\left(z\right)\right)P_{i}^{'}\left(z\right)\right]}{\frac{d}{dz}\left[\left(1-P_{i}\left(z\right)\right)\left(P_{i}\left(z\right)-z\right)\right]}\\
&=&\lim_{z\rightarrow1^{+}}\frac{-\left(1-F_{i}\left(z\right)\right) P_{i}^{'}\left(z\right)-(1-z) F_{i}^{'}\left(z\right) P_{i}^{'}\left(z\right)+(1-z) \left(1-F_{i}\left(z\right)\right) P_{i}^{''}\left(z\right)}{\left(1-P_{i}\left(z\right)\right) \left(-1+P_{i}^{'}\left(z\right)\right)-\left(-z+P_{i}\left(z\right)\right) P_{i}^{'}\left(z\right)}\frac{}{}
\end{eqnarray*}

%______________________________________________________
\begin{eqnarray*}
&&\lim_{z\rightarrow1^{+}}\frac{\left(1-z\right)\left(1-F_{i}\left(z\right)\right)P_{i}\left(z\right)P_{i}^{'}\left(z\right)}{\left(1-P_{i}\left(z\right)\right)^{2}\left(P_{i}\left(z\right)-z\right)}=\lim_{z\rightarrow1^{+}}\frac{\frac{d}{dz}\left[\left(1-z\right)\left(1-F_{i}\left(z\right)\right)P_{i}\left(z\right)P_{i}^{'}\left(z\right)\right]}{\frac{d}{dz}\left[\left(1-P_{i}\left(z\right)\right)^{2}\left(P_{i}\left(z\right)-z\right)\right]}\\
&=&\lim_{z\rightarrow1^{+}}\frac{-\left(1-F_{i}\left(z\right)\right) P_{i}\left(z\right) P_{i}^{'}\left(z\right)-(1-z) P_{i}\left(z\right) F_{i}^{'}\left(z\right)P_i'[z]}{\left(1-P_{i}\left(z\right)\right)^2 \left(-1+P_{i}^{'}\left(z\right)\right)-2 \left(1-P_{i}\left(z\right)\right) \left(-z+P_{i}\left(z\right)\right) P_{i}^{'}\left(z\right)}\\
&+&\lim_{z\rightarrow1^{+}}\frac{(1-z) \left(1-F_{i}\left(z\right)\right) P_{i}^{'}\left(z\right)^2+(1-z) \left(1-F_{i}\left(z\right)\right) P_{i}\left(z\right) P_{i}^{''}\left(z\right)}{\left(1-P_{i}\left(z\right)\right)^2 \left(-1+P_{i}^{'}\left(z\right)\right)-2 \left(1-P_{i}\left(z\right)\right) \left(-z+P_{i}\left(z\right)\right) P_{i}^{'}\left(z\right)}\\
\end{eqnarray*}
%___________________________________________________________________________________________
%\subsection{Tiempos de Ciclo e Intervisita}
%___________________________________________________________________________________________


\begin{Def}
Sea $L_{i}^{*}$el n\'umero de usuarios en la cola $Q_{i}$ cuando es visitada por el servidor para dar servicio, entonces

\begin{eqnarray}
\esp\left[L_{i}^{*}\right]&=&f_{i}\left(i\right)\\
Var\left[L_{i}^{*}\right]&=&f_{i}\left(i,i\right)+\esp\left[L_{i}^{*}\right]-\esp\left[L_{i}^{*}\right]^{2}.
\end{eqnarray}

\end{Def}

\begin{Def}
El tiempo de Ciclo $C_{i}$ es e periodo de tiempo que comienza cuando la cola $i$ es visitada por primera vez en un ciclo, y termina cuando es visitado nuevamente en el pr\'oximo ciclo. La duraci\'on del mismo est\'a dada por $\tau_{i}\left(m+1\right)-\tau_{i}\left(m\right)$, o equivalentemente $\overline{\tau}_{i}\left(m+1\right)-\overline{\tau}_{i}\left(m\right)$ bajo condiciones de estabilidad.
\end{Def}

\begin{Def}
El tiempo de intervisita $I_{i}$ es el periodo de tiempo que comienza cuando se ha completado el servicio en un ciclo y termina cuando es visitada nuevamente en el pr\'oximo ciclo. Su  duraci\'on del mismo est\'a dada por $\tau_{i}\left(m+1\right)-\overline{\tau}_{i}\left(m\right)$.
\end{Def}


Recordemos las siguientes expresiones:

\begin{eqnarray*}
S_{i}\left(z\right)&=&\esp\left[z^{\overline{\tau}_{i}\left(m\right)-\tau_{i}\left(m\right)}\right]=F_{i}\left(\theta\left(z\right)\right),\\
F\left(z\right)&=&\esp\left[z^{L_{0}}\right],\\
P\left(z\right)&=&\esp\left[z^{X_{n}}\right],\\
F_{i}\left(z\right)&=&\esp\left[z^{L_{i}\left(\tau_{i}\left(m\right)\right)}\right],
\theta_{i}\left(z\right)-zP_{i}
\end{eqnarray*}

entonces 

\begin{eqnarray*}
\esp\left[S_{i}\right]&=&\frac{\esp\left[L_{i}^{*}\right]}{1-\mu_{i}}=\frac{f_{i}\left(i\right)}{1-\mu_{i}},\\
Var\left[S_{i}\right]&=&\frac{Var\left[L_{i}^{*}\right]}{\left(1-\mu_{i}\right)^{2}}+\frac{\sigma^{2}\esp\left[L_{i}^{*}\right]}{\left(1-\mu_{i}\right)^{3}}
\end{eqnarray*}

donde recordemos que

\begin{eqnarray*}
Var\left[L_{i}^{*}\right]&=&f_{i}\left(i,i\right)+f_{i}\left(i\right)-f_{i}\left(i\right)^{2}.
\end{eqnarray*}

La duraci\'on del tiempo de intervisita es $\tau_{i}\left(m+1\right)-\overline{\tau}\left(m\right)$. Dado que el n\'umero de usuarios presentes en $Q_{i}$ al tiempo $t=\tau_{i}\left(m+1\right)$ es igual al n\'umero de arribos durante el intervalo de tiempo $\left[\overline{\tau}\left(m\right),\tau_{i}\left(m+1\right)\right]$ se tiene que


\begin{eqnarray*}
\esp\left[z_{i}^{L_{i}\left(\tau_{i}\left(m+1\right)\right)}\right]=\esp\left[\left\{P_{i}\left(z_{i}\right)\right\}^{\tau_{i}\left(m+1\right)-\overline{\tau}\left(m\right)}\right]
\end{eqnarray*}

entonces, si \begin{eqnarray*}I_{i}\left(z\right)&=&\esp\left[z^{\tau_{i}\left(m+1\right)-\overline{\tau}\left(m\right)}\right]\end{eqnarray*} se tienen que

\begin{eqnarray*}
F_{i}\left(z\right)=I_{i}\left[P_{i}\left(z\right)\right]
\end{eqnarray*}
para $i=1,2$, por tanto



\begin{eqnarray*}
\esp\left[L_{i}^{*}\right]&=&\mu_{i}\esp\left[I_{i}\right]\\
Var\left[L_{i}^{*}\right]&=&\mu_{i}^{2}Var\left[I_{i}\right]+\sigma^{2}\esp\left[I_{i}\right]
\end{eqnarray*}
para $i=1,2$, por tanto


\begin{eqnarray*}
\esp\left[I_{i}\right]&=&\frac{f_{i}\left(i\right)}{\mu_{i}},
\end{eqnarray*}
adem\'as

\begin{eqnarray*}
Var\left[I_{i}\right]&=&\frac{Var\left[L_{i}^{*}\right]}{\mu_{i}^{2}}-\frac{\sigma_{i}^{2}}{\mu_{i}^{2}}f_{i}\left(i\right).
\end{eqnarray*}


Si  $C_{i}\left(z\right)=\esp\left[z^{\overline{\tau}\left(m+1\right)-\overline{\tau}_{i}\left(m\right)}\right]$el tiempo de duraci\'on del ciclo, entonces, por lo hasta ahora establecido, se tiene que

\begin{eqnarray*}
C_{i}\left(z\right)=I_{i}\left[\theta_{i}\left(z\right)\right],
\end{eqnarray*}
entonces

\begin{eqnarray*}
\esp\left[C_{i}\right]&=&\esp\left[I_{i}\right]\esp\left[\theta_{i}\left(z\right)\right]=\frac{\esp\left[L_{i}^{*}\right]}{\mu_{i}}\frac{1}{1-\mu_{i}}=\frac{f_{i}\left(i\right)}{\mu_{i}\left(1-\mu_{i}\right)}\\
Var\left[C_{i}\right]&=&\frac{Var\left[L_{i}^{*}\right]}{\mu_{i}^{2}\left(1-\mu_{i}\right)^{2}}.
\end{eqnarray*}

Por tanto se tienen las siguientes igualdades


\begin{eqnarray*}
\esp\left[S_{i}\right]&=&\mu_{i}\esp\left[C_{i}\right],\\
\esp\left[I_{i}\right]&=&\left(1-\mu_{i}\right)\esp\left[C_{i}\right]\\
\end{eqnarray*}

Def\'inanse los puntos de regenaraci\'on  en el proceso $\left[L_{1}\left(t\right),L_{2}\left(t\right),\ldots,L_{N}\left(t\right)\right]$. Los puntos cuando la cola $i$ es visitada y todos los $L_{j}\left(\tau_{i}\left(m\right)\right)=0$ para $i=1,2$  son puntos de regeneraci\'on. Se llama ciclo regenerativo al intervalo entre dos puntos regenerativos sucesivos.

Sea $M_{i}$  el n\'umero de ciclos de visita en un ciclo regenerativo, y sea $C_{i}^{(m)}$, para $m=1,2,\ldots,M_{i}$ la duraci\'on del $m$-\'esimo ciclo de visita en un ciclo regenerativo. Se define el ciclo del tiempo de visita promedio $\esp\left[C_{i}\right]$ como

\begin{eqnarray*}
\esp\left[C_{i}\right]&=&\frac{\esp\left[\sum_{m=1}^{M_{i}}C_{i}^{(m)}\right]}{\esp\left[M_{i}\right]}
\end{eqnarray*}


En Stid72 y Heym82 se muestra que una condici\'on suficiente para que el proceso regenerativo 
estacionario sea un procesoo estacionario es que el valor esperado del tiempo del ciclo regenerativo sea finito:

\begin{eqnarray*}
\esp\left[\sum_{m=1}^{M_{i}}C_{i}^{(m)}\right]<\infty.
\end{eqnarray*}

como cada $C_{i}^{(m)}$ contiene intervalos de r\'eplica positivos, se tiene que $\esp\left[M_{i}\right]<\infty$, adem\'as, como $M_{i}>0$, se tiene que la condici\'on anterior es equivalente a tener que 

\begin{eqnarray*}
\esp\left[C_{i}\right]<\infty,
\end{eqnarray*}
por lo tanto una condici\'on suficiente para la existencia del proceso regenerativo est\'a dada por

\begin{eqnarray*}
\sum_{k=1}^{N}\mu_{k}<1.
\end{eqnarray*}

Sea la funci\'on generadora de momentos para $L_{i}$, el n\'umero de usuarios en la cola $Q_{i}\left(z\right)$ en cualquier momento, est\'a dada por el tiempo promedio de $z^{L_{i}\left(t\right)}$ sobre el ciclo regenerativo definido anteriormente:

\begin{eqnarray*}
Q_{i}\left(z\right)&=&\esp\left[z^{L_{i}\left(t\right)}\right]=\frac{\esp\left[\sum_{m=1}^{M_{i}}\sum_{t=\tau_{i}\left(m\right)}^{\tau_{i}\left(m+1\right)-1}z^{L_{i}\left(t\right)}\right]}{\esp\left[\sum_{m=1}^{M_{i}}\tau_{i}\left(m+1\right)-\tau_{i}\left(m\right)\right]}
\end{eqnarray*}

$M_{i}$ es un tiempo de paro en el proceso regenerativo con $\esp\left[M_{i}\right]<\infty$, se sigue del lema de Wald que:


\begin{eqnarray*}
\esp\left[\sum_{m=1}^{M_{i}}\sum_{t=\tau_{i}\left(m\right)}^{\tau_{i}\left(m+1\right)-1}z^{L_{i}\left(t\right)}\right]&=&\esp\left[M_{i}\right]\esp\left[\sum_{t=\tau_{i}\left(m\right)}^{\tau_{i}\left(m+1\right)-1}z^{L_{i}\left(t\right)}\right]\\
\esp\left[\sum_{m=1}^{M_{i}}\tau_{i}\left(m+1\right)-\tau_{i}\left(m\right)\right]&=&\esp\left[M_{i}\right]\esp\left[\tau_{i}\left(m+1\right)-\tau_{i}\left(m\right)\right]
\end{eqnarray*}

por tanto se tiene que


\begin{eqnarray*}
Q_{i}\left(z\right)&=&\frac{\esp\left[\sum_{t=\tau_{i}\left(m\right)}^{\tau_{i}\left(m+1\right)-1}z^{L_{i}\left(t\right)}\right]}{\esp\left[\tau_{i}\left(m+1\right)-\tau_{i}\left(m\right)\right]}
\end{eqnarray*}

observar que el denominador es simplemente la duraci\'on promedio del tiempo del ciclo.


Se puede demostrar (ver Hideaki Takagi 1986) que

\begin{eqnarray*}
\esp\left[\sum_{t=\tau_{i}\left(m\right)}^{\tau_{i}\left(m+1\right)-1}z^{L_{i}\left(t\right)}\right]=z\frac{F_{i}\left(z\right)-1}{z-P_{i}\left(z\right)}
\end{eqnarray*}

Durante el tiempo de intervisita para la cola $i$, $L_{i}\left(t\right)$ solamente se incrementa de manera que el incremento por intervalo de tiempo est\'a dado por la funci\'on generadora de probabilidades de $P_{i}\left(z\right)$, por tanto la suma sobre el tiempo de intervisita puede evaluarse como:

\begin{eqnarray*}
\esp\left[\sum_{t=\tau_{i}\left(m\right)}^{\tau_{i}\left(m+1\right)-1}z^{L_{i}\left(t\right)}\right]&=&\esp\left[\sum_{t=\tau_{i}\left(m\right)}^{\tau_{i}\left(m+1\right)-1}\left\{P_{i}\left(z\right)\right\}^{t-\overline{\tau}_{i}\left(m\right)}\right]=\frac{1-\esp\left[\left\{P_{i}\left(z\right)\right\}^{\tau_{i}\left(m+1\right)-\overline{\tau}_{i}\left(m\right)}\right]}{1-P_{i}\left(z\right)}\\
&=&\frac{1-I_{i}\left[P_{i}\left(z\right)\right]}{1-P_{i}\left(z\right)}
\end{eqnarray*}
por tanto

\begin{eqnarray*}
\esp\left[\sum_{t=\tau_{i}\left(m\right)}^{\tau_{i}\left(m+1\right)-1}z^{L_{i}\left(t\right)}\right]&=&\frac{1-F_{i}\left(z\right)}{1-P_{i}\left(z\right)}
\end{eqnarray*}

Haciendo uso de lo hasta ahora desarrollado se tiene que

\begin{eqnarray*}
Q_{i}\left(z\right)&=&\frac{1}{\esp\left[C_{i}\right]}\cdot\frac{1-F_{i}\left(z\right)}{P_{i}\left(z\right)-z}\cdot\frac{\left(1-z\right)P_{i}\left(z\right)}{1-P_{i}\left(z\right)}\\
&=&\frac{\mu_{i}\left(1-\mu_{i}\right)}{f_{i}\left(i\right)}\cdot\frac{1-F_{i}\left(z\right)}{P_{i}\left(z\right)-z}\cdot\frac{\left(1-z\right)P_{i}\left(z\right)}{1-P_{i}\left(z\right)}
\end{eqnarray*}


%___________________________________________________________________________________________
%\subsection{Longitudes de la Cola en cualquier tiempo}
%___________________________________________________________________________________________

Sea
$V_{i}\left(z\right)=\frac{1}{\esp\left[C_{i}\right]}\frac{I_{i}\left(z\right)-1}{z-P_{i}\left(z\right)}$

%{\esp\lef[I_{i}\right]}\frac{1-\mu_{i}}{z-P_{i}\left(z\right)}

\begin{eqnarray*}
\frac{\partial V_{i}\left(z\right)}{\partial z}&=&\frac{1}{\esp\left[C_{i}\right]}\left[\frac{I_{i}{'}\left(z\right)\left(z-P_{i}\left(z\right)\right)}{z-P_{i}\left(z\right)}-\frac{\left(I_{i}\left(z\right)-1\right)\left(1-P_{i}{'}\left(z\right)\right)}{\left(z-P_{i}\left(z\right)\right)^{2}}\right]
\end{eqnarray*}


La FGP para el tiempo de espera para cualquier usuario en la cola est\'a dada por:
\[U_{i}\left(z\right)=\frac{1}{\esp\left[C_{i}\right]}\cdot\frac{1-P_{i}\left(z\right)}{z-P_{i}\left(z\right)}\cdot\frac{I_{i}\left(z\right)-1}{1-z}\]

entonces


\begin{eqnarray*}
\frac{d}{dz}V_{i}\left(z\right)&=&\frac{1}{\esp\left[C_{i}\right]}\left\{\frac{d}{dz}\left(\frac{1-P_{i}\left(z\right)}{z-P_{i}\left(z\right)}\right)\frac{I_{i}\left(z\right)-1}{1-z}+\frac{1-P_{i}\left(z\right)}{z-P_{i}\left(z\right)}\frac{d}{dz}\left(\frac{I_{i}\left(z\right)-1}{1-z}\right)\right\}\\
&=&\frac{1}{\esp\left[C_{i}\right]}\left\{\frac{-P_{i}\left(z\right)\left(z-P_{i}\left(z\right)\right)-\left(1-P_{i}\left(z\right)\right)\left(1-P_{i}^{'}\left(z\right)\right)}{\left(z-P_{i}\left(z\right)\right)^{2}}\cdot\frac{I_{i}\left(z\right)-1}{1-z}\right\}\\
&+&\frac{1}{\esp\left[C_{i}\right]}\left\{\frac{1-P_{i}\left(z\right)}{z-P_{i}\left(z\right)}\cdot\frac{I_{i}^{'}\left(z\right)\left(1-z\right)+\left(I_{i}\left(z\right)-1\right)}{\left(1-z\right)^{2}}\right\}
\end{eqnarray*}
%\frac{I_{i}\left(z\right)-1}{1-z}
%+\frac{1-P_{i}\left(z\right)}{z-P_{i}\frac{d}{dz}\left(\frac{I_{i}\left(z\right)-1}{1-z}\right)


\begin{eqnarray*}
\frac{\partial U_{i}\left(z\right)}{\partial z}&=&\frac{(-1+I_{i}[z]) (1-P_{i}[z])}{(1-z)^2 \esp[I_{i}] (z-P_{i}[z])}+\frac{(1-P_{i}[z]) I_{i}^{'}[z]}{(1-z) \esp[I_{i}] (z-P_{i}[z])}-\frac{(-1+I_{i}[z]) (1-P_{i}[z])\left(1-P{'}[z]\right)}{(1-z) \esp[I_{i}] (z-P_{i}[z])^2}\\
&-&\frac{(-1+I_{i}[z]) P_{i}{'}[z]}{(1-z) \esp[I_{i}](z-P_{i}[z])}
\end{eqnarray*}
%___________________________________________________________________________________________
%\subsection{Longitudes de la Cola en cualquier tiempo}
%___________________________________________________________________________________________
Sea 
\begin{eqnarray*}
Q_{i}\left(z\right)&=&\frac{1}{\esp\left[C_{i}\right]}\cdot\frac{1-F_{i}\left(z\right)}{P_{i}\left(z\right)-z}\cdot\frac{\left(1-z\right)P_{i}\left(z\right)}{1-P_{i}\left(z\right)}
\end{eqnarray*}

derivando con respecto a $z$



\begin{eqnarray*}
\frac{d Q_{i}\left(z\right)}{d z}&=&\frac{\left(1-F_{i}\left(z\right)\right)P_{i}\left(z\right)}{\esp\left[C_{i}\right]\left(1-P_{i}\left(z\right)\right)\left(P_{i}\left(z\right)-z\right)}\\
&-&\frac{\left(1-z\right)P_{i}\left(z\right)F_{i}^{'}\left(z\right)}{\esp\left[C_{i}\right]\left(1-P_{i}\left(z\right)\right)\left(P_{i}\left(z\right)-z\right)}\\
&-&\frac{\left(1-z\right)\left(1-F_{i}\left(z\right)\right)P_{i}\left(z\right)\left(P_{i}^{'}\left(z\right)-1\right)}{\esp\left[C_{i}\right]\left(1-P_{i}\left(z\right)\right)\left(P_{i}\left(z\right)-z\right)^{2}}\\
&+&\frac{\left(1-z\right)\left(1-F_{i}\left(z\right)\right)P_{i}^{'}\left(z\right)}{\esp\left[C_{i}\right]\left(1-P_{i}\left(z\right)\right)\left(P_{i}\left(z\right)-z\right)}\\
&+&\frac{\left(1-z\right)\left(1-F_{i}\left(z\right)\right)P_{i}\left(z\right)P_{i}^{'}\left(z\right)}{\esp\left[C_{i}\right]\left(1-P_{i}\left(z\right)\right)^{2}\left(P_{i}\left(z\right)-z\right)}
\end{eqnarray*}

Calculando el l\'imite cuando $z\rightarrow1^{+}$:
\begin{eqnarray}
Q_{i}^{(1)}\left(z\right)=\lim_{z\rightarrow1^{+}}\frac{d Q_{i}\left(z\right)}{dz}&=&\lim_{z\rightarrow1}\frac{\left(1-F_{i}\left(z\right)\right)P_{i}\left(z\right)}{\esp\left[C_{i}\right]\left(1-P_{i}\left(z\right)\right)\left(P_{i}\left(z\right)-z\right)}\\
&-&\lim_{z\rightarrow1^{+}}\frac{\left(1-z\right)P_{i}\left(z\right)F_{i}^{'}\left(z\right)}{\esp\left[C_{i}\right]\left(1-P_{i}\left(z\right)\right)\left(P_{i}\left(z\right)-z\right)}\\
&-&\lim_{z\rightarrow1^{+}}\frac{\left(1-z\right)\left(1-F_{i}\left(z\right)\right)P_{i}\left(z\right)\left(P_{i}^{'}\left(z\right)-1\right)}{\esp\left[C_{i}\right]\left(1-P_{i}\left(z\right)\right)\left(P_{i}\left(z\right)-z\right)^{2}}\\
&+&\lim_{z\rightarrow1^{+}}\frac{\left(1-z\right)\left(1-F_{i}\left(z\right)\right)P_{i}^{'}\left(z\right)}{\esp\left[C_{i}\right]\left(1-P_{i}\left(z\right)\right)\left(P_{i}\left(z\right)-z\right)}\\
&+&\lim_{z\rightarrow1^{+}}\frac{\left(1-z\right)\left(1-F_{i}\left(z\right)\right)P_{i}\left(z\right)P_{i}^{'}\left(z\right)}{\esp\left[C_{i}\right]\left(1-P_{i}\left(z\right)\right)^{2}\left(P_{i}\left(z\right)-z\right)}
\end{eqnarray}

Entonces:
%______________________________________________________

\begin{eqnarray*}
\lim_{z\rightarrow1^{+}}\frac{\left(1-F_{i}\left(z\right)\right)P_{i}\left(z\right)}{\left(1-P_{i}\left(z\right)\right)\left(P_{i}\left(z\right)-z\right)}&=&\lim_{z\rightarrow1^{+}}\frac{\frac{d}{dz}\left[\left(1-F_{i}\left(z\right)\right)P_{i}\left(z\right)\right]}{\frac{d}{dz}\left[\left(1-P_{i}\left(z\right)\right)\left(-z+P_{i}\left(z\right)\right)\right]}\\
&=&\lim_{z\rightarrow1^{+}}\frac{-P_{i}\left(z\right)F_{i}^{'}\left(z\right)+\left(1-F_{i}\left(z\right)\right)P_{i}^{'}\left(z\right)}{\left(1-P_{i}\left(z\right)\right)\left(-1+P_{i}^{'}\left(z\right)\right)-\left(-z+P_{i}\left(z\right)\right)P_{i}^{'}\left(z\right)}
\end{eqnarray*}


%______________________________________________________


\begin{eqnarray*}
\lim_{z\rightarrow1^{+}}\frac{\left(1-z\right)P_{i}\left(z\right)F_{i}^{'}\left(z\right)}{\left(1-P_{i}\left(z\right)\right)\left(P_{i}\left(z\right)-z\right)}&=&\lim_{z\rightarrow1^{+}}\frac{\frac{d}{dz}\left[\left(1-z\right)P_{i}\left(z\right)F_{i}^{'}\left(z\right)\right]}{\frac{d}{dz}\left[\left(1-P_{i}\left(z\right)\right)\left(P_{i}\left(z\right)-z\right)\right]}\\
&=&\lim_{z\rightarrow1^{+}}\frac{-P_{i}\left(z\right) F_{i}^{'}\left(z\right)+(1-z) F_{i}^{'}\left(z\right) P_{i}^{'}\left(z\right)+(1-z) P_{i}\left(z\right)F_{i}^{''}\left(z\right)}{\left(1-P_{i}\left(z\right)\right)\left(-1+P_{i}^{'}\left(z\right)\right)-\left(-z+P_{i}\left(z\right)\right)P_{i}^{'}\left(z\right)}
\end{eqnarray*}


%______________________________________________________

\begin{eqnarray*}
&&\lim_{z\rightarrow1^{+}}\frac{\left(1-z\right)\left(1-F_{i}\left(z\right)\right)P_{i}\left(z\right)\left(P_{i}^{'}\left(z\right)-1\right)}{\left(1-P_{i}\left(z\right)\right)\left(P_{i}\left(z\right)-z\right)^{2}}=\lim_{z\rightarrow1^{+}}\frac{\frac{d}{dz}\left[\left(1-z\right)\left(1-F_{i}\left(z\right)\right)P_{i}\left(z\right)\left(P_{i}^{'}\left(z\right)-1\right)\right]}{\frac{d}{dz}\left[\left(1-P_{i}\left(z\right)\right)\left(P_{i}\left(z\right)-z\right)^{2}\right]}\\
&=&\lim_{z\rightarrow1^{+}}\frac{-\left(1-F_{i}\left(z\right)\right) P_{i}\left(z\right)\left(-1+P_{i}^{'}\left(z\right)\right)-(1-z) P_{i}\left(z\right)F_{i}^{'}\left(z\right)\left(-1+P_{i}^{'}\left(z\right)\right)}{2\left(1-P_{i}\left(z\right)\right)\left(-z+P_{i}\left(z\right)\right) \left(-1+P_{i}^{'}\left(z\right)\right)-\left(-z+P_{i}\left(z\right)\right)^2 P_{i}^{'}\left(z\right)}\\
&+&\lim_{z\rightarrow1^{+}}\frac{+(1-z) \left(1-F_{i}\left(z\right)\right) \left(-1+P_{i}^{'}\left(z\right)\right) P_{i}^{'}\left(z\right)}{{2\left(1-P_{i}\left(z\right)\right)\left(-z+P_{i}\left(z\right)\right) \left(-1+P_{i}^{'}\left(z\right)\right)-\left(-z+P_{i}\left(z\right)\right)^2 P_{i}^{'}\left(z\right)}}\\
&+&\lim_{z\rightarrow1^{+}}\frac{+(1-z) \left(1-F_{i}\left(z\right)\right) P_{i}\left(z\right)P_{i}^{''}\left(z\right)}{{2\left(1-P_{i}\left(z\right)\right)\left(-z+P_{i}\left(z\right)\right) \left(-1+P_{i}^{'}\left(z\right)\right)-\left(-z+P_{i}\left(z\right)\right)^2 P_{i}^{'}\left(z\right)}}
\end{eqnarray*}











%______________________________________________________
\begin{eqnarray*}
&&\lim_{z\rightarrow1^{+}}\frac{\left(1-z\right)\left(1-F_{i}\left(z\right)\right)P_{i}^{'}\left(z\right)}{\left(1-P_{i}\left(z\right)\right)\left(P_{i}\left(z\right)-z\right)}=\lim_{z\rightarrow1^{+}}\frac{\frac{d}{dz}\left[\left(1-z\right)\left(1-F_{i}\left(z\right)\right)P_{i}^{'}\left(z\right)\right]}{\frac{d}{dz}\left[\left(1-P_{i}\left(z\right)\right)\left(P_{i}\left(z\right)-z\right)\right]}\\
&=&\lim_{z\rightarrow1^{+}}\frac{-\left(1-F_{i}\left(z\right)\right) P_{i}^{'}\left(z\right)-(1-z) F_{i}^{'}\left(z\right) P_{i}^{'}\left(z\right)+(1-z) \left(1-F_{i}\left(z\right)\right) P_{i}^{''}\left(z\right)}{\left(1-P_{i}\left(z\right)\right) \left(-1+P_{i}^{'}\left(z\right)\right)-\left(-z+P_{i}\left(z\right)\right) P_{i}^{'}\left(z\right)}\frac{}{}
\end{eqnarray*}

%______________________________________________________
\begin{eqnarray*}
&&\lim_{z\rightarrow1^{+}}\frac{\left(1-z\right)\left(1-F_{i}\left(z\right)\right)P_{i}\left(z\right)P_{i}^{'}\left(z\right)}{\left(1-P_{i}\left(z\right)\right)^{2}\left(P_{i}\left(z\right)-z\right)}=\lim_{z\rightarrow1^{+}}\frac{\frac{d}{dz}\left[\left(1-z\right)\left(1-F_{i}\left(z\right)\right)P_{i}\left(z\right)P_{i}^{'}\left(z\right)\right]}{\frac{d}{dz}\left[\left(1-P_{i}\left(z\right)\right)^{2}\left(P_{i}\left(z\right)-z\right)\right]}\\
&=&\lim_{z\rightarrow1^{+}}\frac{-\left(1-F_{i}\left(z\right)\right) P_{i}\left(z\right) P_{i}^{'}\left(z\right)-(1-z) P_{i}\left(z\right) F_{i}^{'}\left(z\right)P_i'[z]}{\left(1-P_{i}\left(z\right)\right)^2 \left(-1+P_{i}^{'}\left(z\right)\right)-2 \left(1-P_{i}\left(z\right)\right) \left(-z+P_{i}\left(z\right)\right) P_{i}^{'}\left(z\right)}\\
&+&\lim_{z\rightarrow1^{+}}\frac{(1-z) \left(1-F_{i}\left(z\right)\right) P_{i}^{'}\left(z\right)^2+(1-z) \left(1-F_{i}\left(z\right)\right) P_{i}\left(z\right) P_{i}^{''}\left(z\right)}{\left(1-P_{i}\left(z\right)\right)^2 \left(-1+P_{i}^{'}\left(z\right)\right)-2 \left(1-P_{i}\left(z\right)\right) \left(-z+P_{i}\left(z\right)\right) P_{i}^{'}\left(z\right)}\\
\end{eqnarray*}




%_______________________________________________________________________________________________________
\section{Tiempo de Ciclo Promedio}
%_______________________________________________________________________________________________________

Consideremos una cola de la red de sistemas de visitas c\'iclicas fija, $Q_{l}$.


Conforme a la definici\'on dada al principio del cap\'itulo, definici\'on (\ref{Def.Tn}), sean $T_{1},T_{2},\ldots$ los puntos donde las longitudes de las colas de la red de sistemas de visitas c\'iclicas son cero simult\'aneamente, cuando la cola $Q_{l}$ es visitada por el servidor para dar servicio, es decir, $L_{1}\left(T_{i}\right)=0,L_{2}\left(T_{i}\right)=0,\hat{L}_{1}\left(T_{i}\right)=0$ y $\hat{L}_{2}\left(T_{i}\right)=0$, a estos puntos se les denominar\'a puntos regenerativos. Entonces, 

\begin{Def}
Al intervalo de tiempo entre dos puntos regenerativos se le llamar\'a ciclo regenerativo.
\end{Def}

\begin{Def}
Para $T_{i}$ se define, $M_{i}$, el n\'umero de ciclos de visita a la cola $Q_{l}$, durante el ciclo regenerativo, es decir, $M_{i}$ es un proceso de renovaci\'on.
\end{Def}

\begin{Def}
Para cada uno de los $M_{i}$'s, se definen a su vez la duraci\'on de cada uno de estos ciclos de visita en el ciclo regenerativo, $C_{i}^{(m)}$, para $m=1,2,\ldots,M_{i}$, que a su vez, tambi\'en es n proceso de renovaci\'on.
\end{Def}

En nuestra notaci\'on $V\left(t\right)\equiv C_{i}$ y $X_{i}=C_{i}^{(m)}$ para nuestra segunda definici\'on, mientras que para la primera la notaci\'on es: $X\left(t\right)\equiv C_{i}$ y $R_{i}\equiv C_{i}^{(m)}$.


%___________________________________________________________________________________________
\section{Tiempos de Ciclo e Intervisita}
%___________________________________________________________________________________________


\begin{Def}
Sea $L_{i}^{*}$el n\'umero de usuarios en la cola $Q_{i}$ cuando es visitada por el servidor para dar servicio, entonces

\begin{eqnarray}
\esp\left[L_{i}^{*}\right]&=&f_{i}\left(i\right)\\
Var\left[L_{i}^{*}\right]&=&f_{i}\left(i,i\right)+\esp\left[L_{i}^{*}\right]-\esp\left[L_{i}^{*}\right]^{2}.
\end{eqnarray}

\end{Def}

\begin{Def}
El tiempo de Ciclo $C_{i}$ es e periodo de tiempo que comienza cuando la cola $i$ es visitada por primera vez en un ciclo, y termina cuando es visitado nuevamente en el pr\'oximo ciclo. La duraci\'on del mismo est\'a dada por $\tau_{i}\left(m+1\right)-\tau_{i}\left(m\right)$, o equivalentemente $\overline{\tau}_{i}\left(m+1\right)-\overline{\tau}_{i}\left(m\right)$ bajo condiciones de estabilidad.
\end{Def}

\begin{Def}
El tiempo de intervisita $I_{i}$ es el periodo de tiempo que comienza cuando se ha completado el servicio en un ciclo y termina cuando es visitada nuevamente en el pr\'oximo ciclo. Su  duraci\'on del mismo est\'a dada por $\tau_{i}\left(m+1\right)-\overline{\tau}_{i}\left(m\right)$.
\end{Def}


Recordemos las siguientes expresiones:

\begin{eqnarray*}
S_{i}\left(z\right)&=&\esp\left[z^{\overline{\tau}_{i}\left(m\right)-\tau_{i}\left(m\right)}\right]=F_{i}\left(\theta\left(z\right)\right),\\
F\left(z\right)&=&\esp\left[z^{L_{0}}\right],\\
P\left(z\right)&=&\esp\left[z^{X_{n}}\right],\\
F_{i}\left(z\right)&=&\esp\left[z^{L_{i}\left(\tau_{i}\left(m\right)\right)}\right],
\theta_{i}\left(z\right)-zP_{i}
\end{eqnarray*}

entonces 

\begin{eqnarray*}
\esp\left[S_{i}\right]&=&\frac{\esp\left[L_{i}^{*}\right]}{1-\mu_{i}}=\frac{f_{i}\left(i\right)}{1-\mu_{i}},\\
Var\left[S_{i}\right]&=&\frac{Var\left[L_{i}^{*}\right]}{\left(1-\mu_{i}\right)^{2}}+\frac{\sigma^{2}\esp\left[L_{i}^{*}\right]}{\left(1-\mu_{i}\right)^{3}}
\end{eqnarray*}

donde recordemos que

\begin{eqnarray*}
Var\left[L_{i}^{*}\right]&=&f_{i}\left(i,i\right)+f_{i}\left(i\right)-f_{i}\left(i\right)^{2}.
\end{eqnarray*}

La duraci\'on del tiempo de intervisita es $\tau_{i}\left(m+1\right)-\overline{\tau}\left(m\right)$. Dado que el n\'umero de usuarios presentes en $Q_{i}$ al tiempo $t=\tau_{i}\left(m+1\right)$ es igual al n\'umero de arribos durante el intervalo de tiempo $\left[\overline{\tau}\left(m\right),\tau_{i}\left(m+1\right)\right]$ se tiene que


\begin{eqnarray*}
\esp\left[z_{i}^{L_{i}\left(\tau_{i}\left(m+1\right)\right)}\right]=\esp\left[\left\{P_{i}\left(z_{i}\right)\right\}^{\tau_{i}\left(m+1\right)-\overline{\tau}\left(m\right)}\right]
\end{eqnarray*}

entonces, si \begin{eqnarray*}I_{i}\left(z\right)&=&\esp\left[z^{\tau_{i}\left(m+1\right)-\overline{\tau}\left(m\right)}\right]\end{eqnarray*} se tienen que

\begin{eqnarray*}
F_{i}\left(z\right)=I_{i}\left[P_{i}\left(z\right)\right]
\end{eqnarray*}
para $i=1,2$, por tanto



\begin{eqnarray*}
\esp\left[L_{i}^{*}\right]&=&\mu_{i}\esp\left[I_{i}\right]\\
Var\left[L_{i}^{*}\right]&=&\mu_{i}^{2}Var\left[I_{i}\right]+\sigma^{2}\esp\left[I_{i}\right]
\end{eqnarray*}
para $i=1,2$, por tanto


\begin{eqnarray*}
\esp\left[I_{i}\right]&=&\frac{f_{i}\left(i\right)}{\mu_{i}},
\end{eqnarray*}
adem\'as

\begin{eqnarray*}
Var\left[I_{i}\right]&=&\frac{Var\left[L_{i}^{*}\right]}{\mu_{i}^{2}}-\frac{\sigma_{i}^{2}}{\mu_{i}^{2}}f_{i}\left(i\right).
\end{eqnarray*}


Si  $C_{i}\left(z\right)=\esp\left[z^{\overline{\tau}\left(m+1\right)-\overline{\tau}_{i}\left(m\right)}\right]$el tiempo de duraci\'on del ciclo, entonces, por lo hasta ahora establecido, se tiene que

\begin{eqnarray*}
C_{i}\left(z\right)=I_{i}\left[\theta_{i}\left(z\right)\right],
\end{eqnarray*}
entonces

\begin{eqnarray*}
\esp\left[C_{i}\right]&=&\esp\left[I_{i}\right]\esp\left[\theta_{i}\left(z\right)\right]=\frac{\esp\left[L_{i}^{*}\right]}{\mu_{i}}\frac{1}{1-\mu_{i}}=\frac{f_{i}\left(i\right)}{\mu_{i}\left(1-\mu_{i}\right)}\\
Var\left[C_{i}\right]&=&\frac{Var\left[L_{i}^{*}\right]}{\mu_{i}^{2}\left(1-\mu_{i}\right)^{2}}.
\end{eqnarray*}

Por tanto se tienen las siguientes igualdades


\begin{eqnarray*}
\esp\left[S_{i}\right]&=&\mu_{i}\esp\left[C_{i}\right],\\
\esp\left[I_{i}\right]&=&\left(1-\mu_{i}\right)\esp\left[C_{i}\right]\\
\end{eqnarray*}

Def\'inanse los puntos de regenaraci\'on  en el proceso $\left[L_{1}\left(t\right),L_{2}\left(t\right),\ldots,L_{N}\left(t\right)\right]$. Los puntos cuando la cola $i$ es visitada y todos los $L_{j}\left(\tau_{i}\left(m\right)\right)=0$ para $i=1,2$  son puntos de regeneraci\'on. Se llama ciclo regenerativo al intervalo entre dos puntos regenerativos sucesivos.

Sea $M_{i}$  el n\'umero de ciclos de visita en un ciclo regenerativo, y sea $C_{i}^{(m)}$, para $m=1,2,\ldots,M_{i}$ la duraci\'on del $m$-\'esimo ciclo de visita en un ciclo regenerativo. Se define el ciclo del tiempo de visita promedio $\esp\left[C_{i}\right]$ como

\begin{eqnarray*}
\esp\left[C_{i}\right]&=&\frac{\esp\left[\sum_{m=1}^{M_{i}}C_{i}^{(m)}\right]}{\esp\left[M_{i}\right]}
\end{eqnarray*}


En Stid72 y Heym82 se muestra que una condici\'on suficiente para que el proceso regenerativo 
estacionario sea un procesoo estacionario es que el valor esperado del tiempo del ciclo regenerativo sea finito:

\begin{eqnarray*}
\esp\left[\sum_{m=1}^{M_{i}}C_{i}^{(m)}\right]<\infty.
\end{eqnarray*}

como cada $C_{i}^{(m)}$ contiene intervalos de r\'eplica positivos, se tiene que $\esp\left[M_{i}\right]<\infty$, adem\'as, como $M_{i}>0$, se tiene que la condici\'on anterior es equivalente a tener que 

\begin{eqnarray*}
\esp\left[C_{i}\right]<\infty,
\end{eqnarray*}
por lo tanto una condici\'on suficiente para la existencia del proceso regenerativo est\'a dada por

\begin{eqnarray*}
\sum_{k=1}^{N}\mu_{k}<1.
\end{eqnarray*}

Sea la funci\'on generadora de momentos para $L_{i}$, el n\'umero de usuarios en la cola $Q_{i}\left(z\right)$ en cualquier momento, est\'a dada por el tiempo promedio de $z^{L_{i}\left(t\right)}$ sobre el ciclo regenerativo definido anteriormente:

\begin{eqnarray*}
Q_{i}\left(z\right)&=&\esp\left[z^{L_{i}\left(t\right)}\right]=\frac{\esp\left[\sum_{m=1}^{M_{i}}\sum_{t=\tau_{i}\left(m\right)}^{\tau_{i}\left(m+1\right)-1}z^{L_{i}\left(t\right)}\right]}{\esp\left[\sum_{m=1}^{M_{i}}\tau_{i}\left(m+1\right)-\tau_{i}\left(m\right)\right]}
\end{eqnarray*}

$M_{i}$ es un tiempo de paro en el proceso regenerativo con $\esp\left[M_{i}\right]<\infty$, se sigue del lema de Wald que:


\begin{eqnarray*}
\esp\left[\sum_{m=1}^{M_{i}}\sum_{t=\tau_{i}\left(m\right)}^{\tau_{i}\left(m+1\right)-1}z^{L_{i}\left(t\right)}\right]&=&\esp\left[M_{i}\right]\esp\left[\sum_{t=\tau_{i}\left(m\right)}^{\tau_{i}\left(m+1\right)-1}z^{L_{i}\left(t\right)}\right]\\
\esp\left[\sum_{m=1}^{M_{i}}\tau_{i}\left(m+1\right)-\tau_{i}\left(m\right)\right]&=&\esp\left[M_{i}\right]\esp\left[\tau_{i}\left(m+1\right)-\tau_{i}\left(m\right)\right]
\end{eqnarray*}

por tanto se tiene que


\begin{eqnarray*}
Q_{i}\left(z\right)&=&\frac{\esp\left[\sum_{t=\tau_{i}\left(m\right)}^{\tau_{i}\left(m+1\right)-1}z^{L_{i}\left(t\right)}\right]}{\esp\left[\tau_{i}\left(m+1\right)-\tau_{i}\left(m\right)\right]}
\end{eqnarray*}

observar que el denominador es simplemente la duraci\'on promedio del tiempo del ciclo.


Se puede demostrar (ver Hideaki Takagi 1986) que

\begin{eqnarray*}
\esp\left[\sum_{t=\tau_{i}\left(m\right)}^{\tau_{i}\left(m+1\right)-1}z^{L_{i}\left(t\right)}\right]=z\frac{F_{i}\left(z\right)-1}{z-P_{i}\left(z\right)}
\end{eqnarray*}

Durante el tiempo de intervisita para la cola $i$, $L_{i}\left(t\right)$ solamente se incrementa de manera que el incremento por intervalo de tiempo est\'a dado por la funci\'on generadora de probabilidades de $P_{i}\left(z\right)$, por tanto la suma sobre el tiempo de intervisita puede evaluarse como:

\begin{eqnarray*}
\esp\left[\sum_{t=\tau_{i}\left(m\right)}^{\tau_{i}\left(m+1\right)-1}z^{L_{i}\left(t\right)}\right]&=&\esp\left[\sum_{t=\tau_{i}\left(m\right)}^{\tau_{i}\left(m+1\right)-1}\left\{P_{i}\left(z\right)\right\}^{t-\overline{\tau}_{i}\left(m\right)}\right]=\frac{1-\esp\left[\left\{P_{i}\left(z\right)\right\}^{\tau_{i}\left(m+1\right)-\overline{\tau}_{i}\left(m\right)}\right]}{1-P_{i}\left(z\right)}\\
&=&\frac{1-I_{i}\left[P_{i}\left(z\right)\right]}{1-P_{i}\left(z\right)}
\end{eqnarray*}
por tanto

\begin{eqnarray*}
\esp\left[\sum_{t=\tau_{i}\left(m\right)}^{\tau_{i}\left(m+1\right)-1}z^{L_{i}\left(t\right)}\right]&=&\frac{1-F_{i}\left(z\right)}{1-P_{i}\left(z\right)}
\end{eqnarray*}

Haciendo uso de lo hasta ahora desarrollado se tiene que

\begin{eqnarray*}
Q_{i}\left(z\right)&=&\frac{1}{\esp\left[C_{i}\right]}\cdot\frac{1-F_{i}\left(z\right)}{P_{i}\left(z\right)-z}\cdot\frac{\left(1-z\right)P_{i}\left(z\right)}{1-P_{i}\left(z\right)}\\
&=&\frac{\mu_{i}\left(1-\mu_{i}\right)}{f_{i}\left(i\right)}\cdot\frac{1-F_{i}\left(z\right)}{P_{i}\left(z\right)-z}\cdot\frac{\left(1-z\right)P_{i}\left(z\right)}{1-P_{i}\left(z\right)}
\end{eqnarray*}


%___________________________________________________________________________________________
\subsection{Longitudes de la Cola en cualquier tiempo}
%___________________________________________________________________________________________
Sea 
\begin{eqnarray*}
Q_{i}\left(z\right)&=&\frac{1}{\esp\left[C_{i}\right]}\cdot\frac{1-F_{i}\left(z\right)}{P_{i}\left(z\right)-z}\cdot\frac{\left(1-z\right)P_{i}\left(z\right)}{1-P_{i}\left(z\right)}
\end{eqnarray*}

derivando con respecto a $z$



\begin{eqnarray*}
\frac{d Q_{i}\left(z\right)}{d z}&=&\frac{\left(1-F_{i}\left(z\right)\right)P_{i}\left(z\right)}{\esp\left[C_{i}\right]\left(1-P_{i}\left(z\right)\right)\left(P_{i}\left(z\right)-z\right)}\\
&-&\frac{\left(1-z\right)P_{i}\left(z\right)F_{i}^{'}\left(z\right)}{\esp\left[C_{i}\right]\left(1-P_{i}\left(z\right)\right)\left(P_{i}\left(z\right)-z\right)}\\
&-&\frac{\left(1-z\right)\left(1-F_{i}\left(z\right)\right)P_{i}\left(z\right)\left(P_{i}^{'}\left(z\right)-1\right)}{\esp\left[C_{i}\right]\left(1-P_{i}\left(z\right)\right)\left(P_{i}\left(z\right)-z\right)^{2}}\\
&+&\frac{\left(1-z\right)\left(1-F_{i}\left(z\right)\right)P_{i}^{'}\left(z\right)}{\esp\left[C_{i}\right]\left(1-P_{i}\left(z\right)\right)\left(P_{i}\left(z\right)-z\right)}\\
&+&\frac{\left(1-z\right)\left(1-F_{i}\left(z\right)\right)P_{i}\left(z\right)P_{i}^{'}\left(z\right)}{\esp\left[C_{i}\right]\left(1-P_{i}\left(z\right)\right)^{2}\left(P_{i}\left(z\right)-z\right)}
\end{eqnarray*}

Calculando el l\'imite cuando $z\rightarrow1^{+}$:
\begin{eqnarray}
Q_{i}^{(1)}\left(z\right)=\lim_{z\rightarrow1^{+}}\frac{d Q_{i}\left(z\right)}{dz}&=&\lim_{z\rightarrow1}\frac{\left(1-F_{i}\left(z\right)\right)P_{i}\left(z\right)}{\esp\left[C_{i}\right]\left(1-P_{i}\left(z\right)\right)\left(P_{i}\left(z\right)-z\right)}\\
&-&\lim_{z\rightarrow1^{+}}\frac{\left(1-z\right)P_{i}\left(z\right)F_{i}^{'}\left(z\right)}{\esp\left[C_{i}\right]\left(1-P_{i}\left(z\right)\right)\left(P_{i}\left(z\right)-z\right)}\\
&-&\lim_{z\rightarrow1^{+}}\frac{\left(1-z\right)\left(1-F_{i}\left(z\right)\right)P_{i}\left(z\right)\left(P_{i}^{'}\left(z\right)-1\right)}{\esp\left[C_{i}\right]\left(1-P_{i}\left(z\right)\right)\left(P_{i}\left(z\right)-z\right)^{2}}\\
&+&\lim_{z\rightarrow1^{+}}\frac{\left(1-z\right)\left(1-F_{i}\left(z\right)\right)P_{i}^{'}\left(z\right)}{\esp\left[C_{i}\right]\left(1-P_{i}\left(z\right)\right)\left(P_{i}\left(z\right)-z\right)}\\
&+&\lim_{z\rightarrow1^{+}}\frac{\left(1-z\right)\left(1-F_{i}\left(z\right)\right)P_{i}\left(z\right)P_{i}^{'}\left(z\right)}{\esp\left[C_{i}\right]\left(1-P_{i}\left(z\right)\right)^{2}\left(P_{i}\left(z\right)-z\right)}
\end{eqnarray}

Entonces:
%______________________________________________________

\begin{eqnarray*}
\lim_{z\rightarrow1^{+}}\frac{\left(1-F_{i}\left(z\right)\right)P_{i}\left(z\right)}{\left(1-P_{i}\left(z\right)\right)\left(P_{i}\left(z\right)-z\right)}&=&\lim_{z\rightarrow1^{+}}\frac{\frac{d}{dz}\left[\left(1-F_{i}\left(z\right)\right)P_{i}\left(z\right)\right]}{\frac{d}{dz}\left[\left(1-P_{i}\left(z\right)\right)\left(-z+P_{i}\left(z\right)\right)\right]}\\
&=&\lim_{z\rightarrow1^{+}}\frac{-P_{i}\left(z\right)F_{i}^{'}\left(z\right)+\left(1-F_{i}\left(z\right)\right)P_{i}^{'}\left(z\right)}{\left(1-P_{i}\left(z\right)\right)\left(-1+P_{i}^{'}\left(z\right)\right)-\left(-z+P_{i}\left(z\right)\right)P_{i}^{'}\left(z\right)}
\end{eqnarray*}


%______________________________________________________


\begin{eqnarray*}
\lim_{z\rightarrow1^{+}}\frac{\left(1-z\right)P_{i}\left(z\right)F_{i}^{'}\left(z\right)}{\left(1-P_{i}\left(z\right)\right)\left(P_{i}\left(z\right)-z\right)}&=&\lim_{z\rightarrow1^{+}}\frac{\frac{d}{dz}\left[\left(1-z\right)P_{i}\left(z\right)F_{i}^{'}\left(z\right)\right]}{\frac{d}{dz}\left[\left(1-P_{i}\left(z\right)\right)\left(P_{i}\left(z\right)-z\right)\right]}\\
&=&\lim_{z\rightarrow1^{+}}\frac{-P_{i}\left(z\right) F_{i}^{'}\left(z\right)+(1-z) F_{i}^{'}\left(z\right) P_{i}^{'}\left(z\right)+(1-z) P_{i}\left(z\right)F_{i}^{''}\left(z\right)}{\left(1-P_{i}\left(z\right)\right)\left(-1+P_{i}^{'}\left(z\right)\right)-\left(-z+P_{i}\left(z\right)\right)P_{i}^{'}\left(z\right)}
\end{eqnarray*}


%______________________________________________________

\begin{eqnarray*}
&&\lim_{z\rightarrow1^{+}}\frac{\left(1-z\right)\left(1-F_{i}\left(z\right)\right)P_{i}\left(z\right)\left(P_{i}^{'}\left(z\right)-1\right)}{\left(1-P_{i}\left(z\right)\right)\left(P_{i}\left(z\right)-z\right)^{2}}=\lim_{z\rightarrow1^{+}}\frac{\frac{d}{dz}\left[\left(1-z\right)\left(1-F_{i}\left(z\right)\right)P_{i}\left(z\right)\left(P_{i}^{'}\left(z\right)-1\right)\right]}{\frac{d}{dz}\left[\left(1-P_{i}\left(z\right)\right)\left(P_{i}\left(z\right)-z\right)^{2}\right]}\\
&=&\lim_{z\rightarrow1^{+}}\frac{-\left(1-F_{i}\left(z\right)\right) P_{i}\left(z\right)\left(-1+P_{i}^{'}\left(z\right)\right)-(1-z) P_{i}\left(z\right)F_{i}^{'}\left(z\right)\left(-1+P_{i}^{'}\left(z\right)\right)}{2\left(1-P_{i}\left(z\right)\right)\left(-z+P_{i}\left(z\right)\right) \left(-1+P_{i}^{'}\left(z\right)\right)-\left(-z+P_{i}\left(z\right)\right)^2 P_{i}^{'}\left(z\right)}\\
&+&\lim_{z\rightarrow1^{+}}\frac{+(1-z) \left(1-F_{i}\left(z\right)\right) \left(-1+P_{i}^{'}\left(z\right)\right) P_{i}^{'}\left(z\right)}{{2\left(1-P_{i}\left(z\right)\right)\left(-z+P_{i}\left(z\right)\right) \left(-1+P_{i}^{'}\left(z\right)\right)-\left(-z+P_{i}\left(z\right)\right)^2 P_{i}^{'}\left(z\right)}}\\
&+&\lim_{z\rightarrow1^{+}}\frac{+(1-z) \left(1-F_{i}\left(z\right)\right) P_{i}\left(z\right)P_{i}^{''}\left(z\right)}{{2\left(1-P_{i}\left(z\right)\right)\left(-z+P_{i}\left(z\right)\right) \left(-1+P_{i}^{'}\left(z\right)\right)-\left(-z+P_{i}\left(z\right)\right)^2 P_{i}^{'}\left(z\right)}}
\end{eqnarray*}











%______________________________________________________
\begin{eqnarray*}
&&\lim_{z\rightarrow1^{+}}\frac{\left(1-z\right)\left(1-F_{i}\left(z\right)\right)P_{i}^{'}\left(z\right)}{\left(1-P_{i}\left(z\right)\right)\left(P_{i}\left(z\right)-z\right)}=\lim_{z\rightarrow1^{+}}\frac{\frac{d}{dz}\left[\left(1-z\right)\left(1-F_{i}\left(z\right)\right)P_{i}^{'}\left(z\right)\right]}{\frac{d}{dz}\left[\left(1-P_{i}\left(z\right)\right)\left(P_{i}\left(z\right)-z\right)\right]}\\
&=&\lim_{z\rightarrow1^{+}}\frac{-\left(1-F_{i}\left(z\right)\right) P_{i}^{'}\left(z\right)-(1-z) F_{i}^{'}\left(z\right) P_{i}^{'}\left(z\right)+(1-z) \left(1-F_{i}\left(z\right)\right) P_{i}^{''}\left(z\right)}{\left(1-P_{i}\left(z\right)\right) \left(-1+P_{i}^{'}\left(z\right)\right)-\left(-z+P_{i}\left(z\right)\right) P_{i}^{'}\left(z\right)}\frac{}{}
\end{eqnarray*}

%______________________________________________________
\begin{eqnarray*}
&&\lim_{z\rightarrow1^{+}}\frac{\left(1-z\right)\left(1-F_{i}\left(z\right)\right)P_{i}\left(z\right)P_{i}^{'}\left(z\right)}{\left(1-P_{i}\left(z\right)\right)^{2}\left(P_{i}\left(z\right)-z\right)}=\lim_{z\rightarrow1^{+}}\frac{\frac{d}{dz}\left[\left(1-z\right)\left(1-F_{i}\left(z\right)\right)P_{i}\left(z\right)P_{i}^{'}\left(z\right)\right]}{\frac{d}{dz}\left[\left(1-P_{i}\left(z\right)\right)^{2}\left(P_{i}\left(z\right)-z\right)\right]}\\
&=&\lim_{z\rightarrow1^{+}}\frac{-\left(1-F_{i}\left(z\right)\right) P_{i}\left(z\right) P_{i}^{'}\left(z\right)-(1-z) P_{i}\left(z\right) F_{i}^{'}\left(z\right)P_i'[z]}{\left(1-P_{i}\left(z\right)\right)^2 \left(-1+P_{i}^{'}\left(z\right)\right)-2 \left(1-P_{i}\left(z\right)\right) \left(-z+P_{i}\left(z\right)\right) P_{i}^{'}\left(z\right)}\\
&+&\lim_{z\rightarrow1^{+}}\frac{(1-z) \left(1-F_{i}\left(z\right)\right) P_{i}^{'}\left(z\right)^2+(1-z) \left(1-F_{i}\left(z\right)\right) P_{i}\left(z\right) P_{i}^{''}\left(z\right)}{\left(1-P_{i}\left(z\right)\right)^2 \left(-1+P_{i}^{'}\left(z\right)\right)-2 \left(1-P_{i}\left(z\right)\right) \left(-z+P_{i}\left(z\right)\right) P_{i}^{'}\left(z\right)}\\
\end{eqnarray*}

\section{Por resolver}



\begin{eqnarray*}
&&\frac{\partial Q_{i}\left(z\right)}{\partial z}=\frac{1}{\esp\left[C_{i}\right]}\frac{\partial}{\partial z}\left\{\frac{1-F_{i}\left(z\right)}{P_{i}\left(z\right)-z}\cdot\frac{\left(1-z\right)P_{i}\left(z\right)}{1-P_{i}\left(z\right)}\right\}\\
&=&\frac{1}{\esp\left[C_{i}\right]}\left\{\frac{\partial}{\partial z}\left(\frac{1-F_{i}\left(z\right)}{P_{i}\left(z\right)-z}\right)\cdot\frac{\left(1-z\right)P_{i}\left(z\right)}{1-P_{i}\left(z\right)}+\frac{1-F_{i}\left(z\right)}{P_{i}\left(z\right)-z}\cdot\frac{\partial}{\partial z}\left(\frac{\left(1-z\right)P_{i}\left(z\right)}{1-P_{i}\left(z\right)}\right)\right\}\\
&=&\frac{1}{\esp\left[C_{i}\right]}\cdot\frac{\left(1-z\right)P_{i}\left(z\right)}{1-P_{i}\left(z\right)}\cdot\frac{\partial}{\partial z}\left(\frac{1-F_{i}\left(z\right)}{P_{i}\left(z\right)-z}\right)+\frac{1}{\esp\left[C_{i}\right]}\cdot\frac{1-F_{i}\left(z\right)}{P_{i}\left(z\right)-z}\cdot\frac{\partial}{\partial z}\left(\frac{\left(1-z\right)P_{i}\left(z\right)}{1-P_{i}\left(z\right)}\right)\\
&=&\frac{1}{\esp\left[C_{i}\right]}\cdot\frac{\left(1-z\right)P_{i}\left(z\right)}{1-P_{i}\left(z\right)}\cdot\frac{-F_{i}^{'}\left(z\right)\left(P_{i}\left(z\right)-z\right)-\left(1-F_{i}\left(z\right)\right)\left(P_{i}^{'}\left(z\right)-1\right)}{\left(P_{i}\left(z\right)-z\right)^{2}}\\
&+&\frac{1}{\esp\left[C_{i}\right]}\cdot\frac{1-F_{i}\left(z\right)}{P_{i}\left(z\right)-z}\cdot\frac{\left(1-z\right)P_{i}^{'}\left(z\right)-P_{i}\left(z\right)}{\left(1-P_{i}\left(z\right)\right)^{2}}
\end{eqnarray*}



\begin{eqnarray*}
Q_{i}^{(1)}\left(z\right)&=& \frac{\left(1-F_{i}\left(z\right)\right)P_{i}\left(z\right)}{\esp\left[C_{i}\right]\left(1-P_{i}\left(z\right)\right)\left(P_{i}\left(z\right)-z\right)}
-\frac{\left(1-z\right)P_{i}\left(z\right)F_{i}^{'}\left(z\right)}{\esp\left[C_{i}\right]\left(1-P_{i}\left(z\right)\right)\left(P_{i}\left(z\right)-z\right)}\\
&-&\frac{\left(1-z\right)\left(1-F_{i}\left(z\right)\right)P_{i}\left(z\right)\left(P_{i}^{'}\left(z\right)-1\right)}{\esp\left[C_{i}\right]\left(1-P_{i}\left(z\right)\right)\left(P_{i}\left(z\right)-z\right)^{2}}+\frac{\left(1-z\right)\left(1-F_{i}\left(z\right)\right)P_{i}^{'}\left(z\right)}{\esp\left[C_{i}\right]\left(1-P_{i}\left(z\right)\right)\left(P_{i}\left(z\right)-z\right)}\\
&+&\frac{\left(1-z\right)\left(1-F_{i}\left(z\right)\right)P_{i}\left(z\right)P_{i}^{'}\left(z\right)}{\esp\left[C_{i}\right]\left(1-P_{i}\left(z\right)\right)^{2}\left(P_{i}\left(z\right)-z\right)}
\end{eqnarray*}
%___________________________________________________________________________________________
%\subsection{Operaciones Matemathica: Tiempos de Espera}
%___________________________________________________________________________________________
Sea
$V_{i}\left(z\right)=\frac{1}{\esp\left[C_{i}\right]}\frac{I_{i}\left(z\right)-1}{z-P_{i}\left(z\right)}$

%{\esp\lef[I_{i}\right]}\frac{1-\mu_{i}}{z-P_{i}\left(z\right)}

\begin{eqnarray*}
\frac{\partial V_{i}\left(z\right)}{\partial z}&=&\frac{1}{\esp\left[C_{i}\right]}\left[\frac{I_{i}{'}\left(z\right)\left(z-P_{i}\left(z\right)\right)}{z-P_{i}\left(z\right)}-\frac{\left(I_{i}\left(z\right)-1\right)\left(1-P_{i}{'}\left(z\right)\right)}{\left(z-P_{i}\left(z\right)\right)^{2}}\right]
\end{eqnarray*}


La FGP para el tiempo de espera para cualquier usuario en la cola est\'a dada por:
\[U_{i}\left(z\right)=\frac{1}{\esp\left[C_{i}\right]}\cdot\frac{1-P_{i}\left(z\right)}{z-P_{i}\left(z\right)}\cdot\frac{I_{i}\left(z\right)-1}{1-z}\]

entonces


\begin{eqnarray*}
\frac{d}{dz}V_{i}\left(z\right)&=&\frac{1}{\esp\left[C_{i}\right]}\left\{\frac{d}{dz}\left(\frac{1-P_{i}\left(z\right)}{z-P_{i}\left(z\right)}\right)\frac{I_{i}\left(z\right)-1}{1-z}+\frac{1-P_{i}\left(z\right)}{z-P_{i}\left(z\right)}\frac{d}{dz}\left(\frac{I_{i}\left(z\right)-1}{1-z}\right)\right\}\\
&=&\frac{1}{\esp\left[C_{i}\right]}\left\{\frac{-P_{i}\left(z\right)\left(z-P_{i}\left(z\right)\right)-\left(1-P_{i}\left(z\right)\right)\left(1-P_{i}^{'}\left(z\right)\right)}{\left(z-P_{i}\left(z\right)\right)^{2}}\cdot\frac{I_{i}\left(z\right)-1}{1-z}\right\}\\
&+&\frac{1}{\esp\left[C_{i}\right]}\left\{\frac{1-P_{i}\left(z\right)}{z-P_{i}\left(z\right)}\cdot\frac{I_{i}^{'}\left(z\right)\left(1-z\right)+\left(I_{i}\left(z\right)-1\right)}{\left(1-z\right)^{2}}\right\}
\end{eqnarray*}
%\frac{I_{i}\left(z\right)-1}{1-z}
%+\frac{1-P_{i}\left(z\right)}{z-P_{i}\frac{d}{dz}\left(\frac{I_{i}\left(z\right)-1}{1-z}\right)


\begin{eqnarray*}
\frac{\partial U_{i}\left(z\right)}{\partial z}&=&\frac{(-1+I_{i}[z]) (1-P_{i}[z])}{(1-z)^2 \esp[I_{i}] (z-P_{i}[z])}+\frac{(1-P_{i}[z]) I_{i}^{'}[z]}{(1-z) \esp[I_{i}] (z-P_{i}[z])}-\frac{(-1+I_{i}[z]) (1-P_{i}[z])\left(1-P{'}[z]\right)}{(1-z) \esp[I_{i}] (z-P_{i}[z])^2}\\
&-&\frac{(-1+I_{i}[z]) P_{i}{'}[z]}{(1-z) \esp[I_{i}](z-P_{i}[z])}
\end{eqnarray*}

%___________________________________________________________________________________________
\section{Tiempos de Ciclo e Intervisita}
%___________________________________________________________________________________________


\begin{Def}
Sea $L_{i}^{*}$el n\'umero de usuarios en la cola $Q_{i}$ cuando es visitada por el servidor para dar servicio, entonces

\begin{eqnarray}
\esp\left[L_{i}^{*}\right]&=&f_{i}\left(i\right)\\
Var\left[L_{i}^{*}\right]&=&f_{i}\left(i,i\right)+\esp\left[L_{i}^{*}\right]-\esp\left[L_{i}^{*}\right]^{2}.
\end{eqnarray}

\end{Def}

\begin{Def}
El tiempo de Ciclo $C_{i}$ es e periodo de tiempo que comienza cuando la cola $i$ es visitada por primera vez en un ciclo, y termina cuando es visitado nuevamente en el pr\'oximo ciclo. La duraci\'on del mismo est\'a dada por $\tau_{i}\left(m+1\right)-\tau_{i}\left(m\right)$, o equivalentemente $\overline{\tau}_{i}\left(m+1\right)-\overline{\tau}_{i}\left(m\right)$ bajo condiciones de estabilidad.
\end{Def}

\begin{Def}
El tiempo de intervisita $I_{i}$ es el periodo de tiempo que comienza cuando se ha completado el servicio en un ciclo y termina cuando es visitada nuevamente en el pr\'oximo ciclo. Su  duraci\'on del mismo est\'a dada por $\tau_{i}\left(m+1\right)-\overline{\tau}_{i}\left(m\right)$.
\end{Def}


Recordemos las siguientes expresiones:

\begin{eqnarray*}
S_{i}\left(z\right)&=&\esp\left[z^{\overline{\tau}_{i}\left(m\right)-\tau_{i}\left(m\right)}\right]=F_{i}\left(\theta\left(z\right)\right),\\
F\left(z\right)&=&\esp\left[z^{L_{0}}\right],\\
P\left(z\right)&=&\esp\left[z^{X_{n}}\right],\\
F_{i}\left(z\right)&=&\esp\left[z^{L_{i}\left(\tau_{i}\left(m\right)\right)}\right],
\theta_{i}\left(z\right)-zP_{i}
\end{eqnarray*}

entonces 

\begin{eqnarray*}
\esp\left[S_{i}\right]&=&\frac{\esp\left[L_{i}^{*}\right]}{1-\mu_{i}}=\frac{f_{i}\left(i\right)}{1-\mu_{i}},\\
Var\left[S_{i}\right]&=&\frac{Var\left[L_{i}^{*}\right]}{\left(1-\mu_{i}\right)^{2}}+\frac{\sigma^{2}\esp\left[L_{i}^{*}\right]}{\left(1-\mu_{i}\right)^{3}}
\end{eqnarray*}

donde recordemos que

\begin{eqnarray*}
Var\left[L_{i}^{*}\right]&=&f_{i}\left(i,i\right)+f_{i}\left(i\right)-f_{i}\left(i\right)^{2}.
\end{eqnarray*}

La duraci\'on del tiempo de intervisita es $\tau_{i}\left(m+1\right)-\overline{\tau}\left(m\right)$. Dado que el n\'umero de usuarios presentes en $Q_{i}$ al tiempo $t=\tau_{i}\left(m+1\right)$ es igual al n\'umero de arribos durante el intervalo de tiempo $\left[\overline{\tau}\left(m\right),\tau_{i}\left(m+1\right)\right]$ se tiene que


\begin{eqnarray*}
\esp\left[z_{i}^{L_{i}\left(\tau_{i}\left(m+1\right)\right)}\right]=\esp\left[\left\{P_{i}\left(z_{i}\right)\right\}^{\tau_{i}\left(m+1\right)-\overline{\tau}\left(m\right)}\right]
\end{eqnarray*}

entonces, si \begin{eqnarray*}I_{i}\left(z\right)&=&\esp\left[z^{\tau_{i}\left(m+1\right)-\overline{\tau}\left(m\right)}\right]\end{eqnarray*} se tienen que

\begin{eqnarray*}
F_{i}\left(z\right)=I_{i}\left[P_{i}\left(z\right)\right]
\end{eqnarray*}
para $i=1,2$, por tanto



\begin{eqnarray*}
\esp\left[L_{i}^{*}\right]&=&\mu_{i}\esp\left[I_{i}\right]\\
Var\left[L_{i}^{*}\right]&=&\mu_{i}^{2}Var\left[I_{i}\right]+\sigma^{2}\esp\left[I_{i}\right]
\end{eqnarray*}
para $i=1,2$, por tanto


\begin{eqnarray*}
\esp\left[I_{i}\right]&=&\frac{f_{i}\left(i\right)}{\mu_{i}},
\end{eqnarray*}
adem\'as

\begin{eqnarray*}
Var\left[I_{i}\right]&=&\frac{Var\left[L_{i}^{*}\right]}{\mu_{i}^{2}}-\frac{\sigma_{i}^{2}}{\mu_{i}^{2}}f_{i}\left(i\right).
\end{eqnarray*}


Si  $C_{i}\left(z\right)=\esp\left[z^{\overline{\tau}\left(m+1\right)-\overline{\tau}_{i}\left(m\right)}\right]$el tiempo de duraci\'on del ciclo, entonces, por lo hasta ahora establecido, se tiene que

\begin{eqnarray*}
C_{i}\left(z\right)=I_{i}\left[\theta_{i}\left(z\right)\right],
\end{eqnarray*}
entonces

\begin{eqnarray*}
\esp\left[C_{i}\right]&=&\esp\left[I_{i}\right]\esp\left[\theta_{i}\left(z\right)\right]=\frac{\esp\left[L_{i}^{*}\right]}{\mu_{i}}\frac{1}{1-\mu_{i}}=\frac{f_{i}\left(i\right)}{\mu_{i}\left(1-\mu_{i}\right)}\\
Var\left[C_{i}\right]&=&\frac{Var\left[L_{i}^{*}\right]}{\mu_{i}^{2}\left(1-\mu_{i}\right)^{2}}.
\end{eqnarray*}

Por tanto se tienen las siguientes igualdades


\begin{eqnarray*}
\esp\left[S_{i}\right]&=&\mu_{i}\esp\left[C_{i}\right],\\
\esp\left[I_{i}\right]&=&\left(1-\mu_{i}\right)\esp\left[C_{i}\right]\\
\end{eqnarray*}

Def\'inanse los puntos de regenaraci\'on  en el proceso $\left[L_{1}\left(t\right),L_{2}\left(t\right),\ldots,L_{N}\left(t\right)\right]$. Los puntos cuando la cola $i$ es visitada y todos los $L_{j}\left(\tau_{i}\left(m\right)\right)=0$ para $i=1,2$  son puntos de regeneraci\'on. Se llama ciclo regenerativo al intervalo entre dos puntos regenerativos sucesivos.

Sea $M_{i}$  el n\'umero de ciclos de visita en un ciclo regenerativo, y sea $C_{i}^{(m)}$, para $m=1,2,\ldots,M_{i}$ la duraci\'on del $m$-\'esimo ciclo de visita en un ciclo regenerativo. Se define el ciclo del tiempo de visita promedio $\esp\left[C_{i}\right]$ como

\begin{eqnarray*}
\esp\left[C_{i}\right]&=&\frac{\esp\left[\sum_{m=1}^{M_{i}}C_{i}^{(m)}\right]}{\esp\left[M_{i}\right]}
\end{eqnarray*}


En Stid72 y Heym82 se muestra que una condici\'on suficiente para que el proceso regenerativo 
estacionario sea un procesoo estacionario es que el valor esperado del tiempo del ciclo regenerativo sea finito:

\begin{eqnarray*}
\esp\left[\sum_{m=1}^{M_{i}}C_{i}^{(m)}\right]<\infty.
\end{eqnarray*}

como cada $C_{i}^{(m)}$ contiene intervalos de r\'eplica positivos, se tiene que $\esp\left[M_{i}\right]<\infty$, adem\'as, como $M_{i}>0$, se tiene que la condici\'on anterior es equivalente a tener que 

\begin{eqnarray*}
\esp\left[C_{i}\right]<\infty,
\end{eqnarray*}
por lo tanto una condici\'on suficiente para la existencia del proceso regenerativo est\'a dada por

\begin{eqnarray*}
\sum_{k=1}^{N}\mu_{k}<1.
\end{eqnarray*}

Sea la funci\'on generadora de momentos para $L_{i}$, el n\'umero de usuarios en la cola $Q_{i}\left(z\right)$ en cualquier momento, est\'a dada por el tiempo promedio de $z^{L_{i}\left(t\right)}$ sobre el ciclo regenerativo definido anteriormente:

\begin{eqnarray*}
Q_{i}\left(z\right)&=&\esp\left[z^{L_{i}\left(t\right)}\right]=\frac{\esp\left[\sum_{m=1}^{M_{i}}\sum_{t=\tau_{i}\left(m\right)}^{\tau_{i}\left(m+1\right)-1}z^{L_{i}\left(t\right)}\right]}{\esp\left[\sum_{m=1}^{M_{i}}\tau_{i}\left(m+1\right)-\tau_{i}\left(m\right)\right]}
\end{eqnarray*}

$M_{i}$ es un tiempo de paro en el proceso regenerativo con $\esp\left[M_{i}\right]<\infty$, se sigue del lema de Wald que:


\begin{eqnarray*}
\esp\left[\sum_{m=1}^{M_{i}}\sum_{t=\tau_{i}\left(m\right)}^{\tau_{i}\left(m+1\right)-1}z^{L_{i}\left(t\right)}\right]&=&\esp\left[M_{i}\right]\esp\left[\sum_{t=\tau_{i}\left(m\right)}^{\tau_{i}\left(m+1\right)-1}z^{L_{i}\left(t\right)}\right]\\
\esp\left[\sum_{m=1}^{M_{i}}\tau_{i}\left(m+1\right)-\tau_{i}\left(m\right)\right]&=&\esp\left[M_{i}\right]\esp\left[\tau_{i}\left(m+1\right)-\tau_{i}\left(m\right)\right]
\end{eqnarray*}

por tanto se tiene que


\begin{eqnarray*}
Q_{i}\left(z\right)&=&\frac{\esp\left[\sum_{t=\tau_{i}\left(m\right)}^{\tau_{i}\left(m+1\right)-1}z^{L_{i}\left(t\right)}\right]}{\esp\left[\tau_{i}\left(m+1\right)-\tau_{i}\left(m\right)\right]}
\end{eqnarray*}

observar que el denominador es simplemente la duraci\'on promedio del tiempo del ciclo.


Se puede demostrar (ver Hideaki Takagi 1986) que

\begin{eqnarray*}
\esp\left[\sum_{t=\tau_{i}\left(m\right)}^{\tau_{i}\left(m+1\right)-1}z^{L_{i}\left(t\right)}\right]=z\frac{F_{i}\left(z\right)-1}{z-P_{i}\left(z\right)}
\end{eqnarray*}

Durante el tiempo de intervisita para la cola $i$, $L_{i}\left(t\right)$ solamente se incrementa de manera que el incremento por intervalo de tiempo est\'a dado por la funci\'on generadora de probabilidades de $P_{i}\left(z\right)$, por tanto la suma sobre el tiempo de intervisita puede evaluarse como:

\begin{eqnarray*}
\esp\left[\sum_{t=\tau_{i}\left(m\right)}^{\tau_{i}\left(m+1\right)-1}z^{L_{i}\left(t\right)}\right]&=&\esp\left[\sum_{t=\tau_{i}\left(m\right)}^{\tau_{i}\left(m+1\right)-1}\left\{P_{i}\left(z\right)\right\}^{t-\overline{\tau}_{i}\left(m\right)}\right]=\frac{1-\esp\left[\left\{P_{i}\left(z\right)\right\}^{\tau_{i}\left(m+1\right)-\overline{\tau}_{i}\left(m\right)}\right]}{1-P_{i}\left(z\right)}\\
&=&\frac{1-I_{i}\left[P_{i}\left(z\right)\right]}{1-P_{i}\left(z\right)}
\end{eqnarray*}
por tanto

\begin{eqnarray*}
\esp\left[\sum_{t=\tau_{i}\left(m\right)}^{\tau_{i}\left(m+1\right)-1}z^{L_{i}\left(t\right)}\right]&=&\frac{1-F_{i}\left(z\right)}{1-P_{i}\left(z\right)}
\end{eqnarray*}

Haciendo uso de lo hasta ahora desarrollado se tiene que

\begin{eqnarray*}
Q_{i}\left(z\right)&=&\frac{1}{\esp\left[C_{i}\right]}\cdot\frac{1-F_{i}\left(z\right)}{P_{i}\left(z\right)-z}\cdot\frac{\left(1-z\right)P_{i}\left(z\right)}{1-P_{i}\left(z\right)}\\
&=&\frac{\mu_{i}\left(1-\mu_{i}\right)}{f_{i}\left(i\right)}\cdot\frac{1-F_{i}\left(z\right)}{P_{i}\left(z\right)-z}\cdot\frac{\left(1-z\right)P_{i}\left(z\right)}{1-P_{i}\left(z\right)}
\end{eqnarray*}

derivando con respecto a $z$



\begin{eqnarray*}
\frac{d Q_{i}\left(z\right)}{d z}&=&\frac{\left(1-F_{i}\left(z\right)\right)P_{i}\left(z\right)}{\esp\left[C_{i}\right]\left(1-P_{i}\left(z\right)\right)\left(P_{i}\left(z\right)-z\right)}\\
&-&\frac{\left(1-z\right)P_{i}\left(z\right)F_{i}^{'}\left(z\right)}{\esp\left[C_{i}\right]\left(1-P_{i}\left(z\right)\right)\left(P_{i}\left(z\right)-z\right)}\\
&-&\frac{\left(1-z\right)\left(1-F_{i}\left(z\right)\right)P_{i}\left(z\right)\left(P_{i}^{'}\left(z\right)-1\right)}{\esp\left[C_{i}\right]\left(1-P_{i}\left(z\right)\right)\left(P_{i}\left(z\right)-z\right)^{2}}\\
&+&\frac{\left(1-z\right)\left(1-F_{i}\left(z\right)\right)P_{i}^{'}\left(z\right)}{\esp\left[C_{i}\right]\left(1-P_{i}\left(z\right)\right)\left(P_{i}\left(z\right)-z\right)}\\
&+&\frac{\left(1-z\right)\left(1-F_{i}\left(z\right)\right)P_{i}\left(z\right)P_{i}^{'}\left(z\right)}{\esp\left[C_{i}\right]\left(1-P_{i}\left(z\right)\right)^{2}\left(P_{i}\left(z\right)-z\right)}
\end{eqnarray*}

Calculando el l\'imite cuando $z\rightarrow1^{+}$:
\begin{eqnarray}
Q_{i}^{(1)}\left(z\right)=\lim_{z\rightarrow1^{+}}\frac{d Q_{i}\left(z\right)}{dz}&=&\lim_{z\rightarrow1}\frac{\left(1-F_{i}\left(z\right)\right)P_{i}\left(z\right)}{\esp\left[C_{i}\right]\left(1-P_{i}\left(z\right)\right)\left(P_{i}\left(z\right)-z\right)}\\
&-&\lim_{z\rightarrow1^{+}}\frac{\left(1-z\right)P_{i}\left(z\right)F_{i}^{'}\left(z\right)}{\esp\left[C_{i}\right]\left(1-P_{i}\left(z\right)\right)\left(P_{i}\left(z\right)-z\right)}\\
&-&\lim_{z\rightarrow1^{+}}\frac{\left(1-z\right)\left(1-F_{i}\left(z\right)\right)P_{i}\left(z\right)\left(P_{i}^{'}\left(z\right)-1\right)}{\esp\left[C_{i}\right]\left(1-P_{i}\left(z\right)\right)\left(P_{i}\left(z\right)-z\right)^{2}}\\
&+&\lim_{z\rightarrow1^{+}}\frac{\left(1-z\right)\left(1-F_{i}\left(z\right)\right)P_{i}^{'}\left(z\right)}{\esp\left[C_{i}\right]\left(1-P_{i}\left(z\right)\right)\left(P_{i}\left(z\right)-z\right)}\\
&+&\lim_{z\rightarrow1^{+}}\frac{\left(1-z\right)\left(1-F_{i}\left(z\right)\right)P_{i}\left(z\right)P_{i}^{'}\left(z\right)}{\esp\left[C_{i}\right]\left(1-P_{i}\left(z\right)\right)^{2}\left(P_{i}\left(z\right)-z\right)}
\end{eqnarray}

Entonces:
%______________________________________________________

\begin{eqnarray*}
\lim_{z\rightarrow1^{+}}\frac{\left(1-F_{i}\left(z\right)\right)P_{i}\left(z\right)}{\left(1-P_{i}\left(z\right)\right)\left(P_{i}\left(z\right)-z\right)}&=&\lim_{z\rightarrow1^{+}}\frac{\frac{d}{dz}\left[\left(1-F_{i}\left(z\right)\right)P_{i}\left(z\right)\right]}{\frac{d}{dz}\left[\left(1-P_{i}\left(z\right)\right)\left(-z+P_{i}\left(z\right)\right)\right]}\\
&=&\lim_{z\rightarrow1^{+}}\frac{-P_{i}\left(z\right)F_{i}^{'}\left(z\right)+\left(1-F_{i}\left(z\right)\right)P_{i}^{'}\left(z\right)}{\left(1-P_{i}\left(z\right)\right)\left(-1+P_{i}^{'}\left(z\right)\right)-\left(-z+P_{i}\left(z\right)\right)P_{i}^{'}\left(z\right)}
\end{eqnarray*}


%______________________________________________________


\begin{eqnarray*}
\lim_{z\rightarrow1^{+}}\frac{\left(1-z\right)P_{i}\left(z\right)F_{i}^{'}\left(z\right)}{\left(1-P_{i}\left(z\right)\right)\left(P_{i}\left(z\right)-z\right)}&=&\lim_{z\rightarrow1^{+}}\frac{\frac{d}{dz}\left[\left(1-z\right)P_{i}\left(z\right)F_{i}^{'}\left(z\right)\right]}{\frac{d}{dz}\left[\left(1-P_{i}\left(z\right)\right)\left(P_{i}\left(z\right)-z\right)\right]}\\
&=&\lim_{z\rightarrow1^{+}}\frac{-P_{i}\left(z\right) F_{i}^{'}\left(z\right)+(1-z) F_{i}^{'}\left(z\right) P_{i}^{'}\left(z\right)+(1-z) P_{i}\left(z\right)F_{i}^{''}\left(z\right)}{\left(1-P_{i}\left(z\right)\right)\left(-1+P_{i}^{'}\left(z\right)\right)-\left(-z+P_{i}\left(z\right)\right)P_{i}^{'}\left(z\right)}
\end{eqnarray*}


%______________________________________________________

\begin{eqnarray*}
&&\lim_{z\rightarrow1^{+}}\frac{\left(1-z\right)\left(1-F_{i}\left(z\right)\right)P_{i}\left(z\right)\left(P_{i}^{'}\left(z\right)-1\right)}{\left(1-P_{i}\left(z\right)\right)\left(P_{i}\left(z\right)-z\right)^{2}}=\lim_{z\rightarrow1^{+}}\frac{\frac{d}{dz}\left[\left(1-z\right)\left(1-F_{i}\left(z\right)\right)P_{i}\left(z\right)\left(P_{i}^{'}\left(z\right)-1\right)\right]}{\frac{d}{dz}\left[\left(1-P_{i}\left(z\right)\right)\left(P_{i}\left(z\right)-z\right)^{2}\right]}\\
&=&\lim_{z\rightarrow1^{+}}\frac{-\left(1-F_{i}\left(z\right)\right) P_{i}\left(z\right)\left(-1+P_{i}^{'}\left(z\right)\right)-(1-z) P_{i}\left(z\right)F_{i}^{'}\left(z\right)\left(-1+P_{i}^{'}\left(z\right)\right)}{2\left(1-P_{i}\left(z\right)\right)\left(-z+P_{i}\left(z\right)\right) \left(-1+P_{i}^{'}\left(z\right)\right)-\left(-z+P_{i}\left(z\right)\right)^2 P_{i}^{'}\left(z\right)}\\
&+&\lim_{z\rightarrow1^{+}}\frac{+(1-z) \left(1-F_{i}\left(z\right)\right) \left(-1+P_{i}^{'}\left(z\right)\right) P_{i}^{'}\left(z\right)}{{2\left(1-P_{i}\left(z\right)\right)\left(-z+P_{i}\left(z\right)\right) \left(-1+P_{i}^{'}\left(z\right)\right)-\left(-z+P_{i}\left(z\right)\right)^2 P_{i}^{'}\left(z\right)}}\\
&+&\lim_{z\rightarrow1^{+}}\frac{+(1-z) \left(1-F_{i}\left(z\right)\right) P_{i}\left(z\right)P_{i}^{''}\left(z\right)}{{2\left(1-P_{i}\left(z\right)\right)\left(-z+P_{i}\left(z\right)\right) \left(-1+P_{i}^{'}\left(z\right)\right)-\left(-z+P_{i}\left(z\right)\right)^2 P_{i}^{'}\left(z\right)}}
\end{eqnarray*}











%______________________________________________________
\begin{eqnarray*}
&&\lim_{z\rightarrow1^{+}}\frac{\left(1-z\right)\left(1-F_{i}\left(z\right)\right)P_{i}^{'}\left(z\right)}{\left(1-P_{i}\left(z\right)\right)\left(P_{i}\left(z\right)-z\right)}=\lim_{z\rightarrow1^{+}}\frac{\frac{d}{dz}\left[\left(1-z\right)\left(1-F_{i}\left(z\right)\right)P_{i}^{'}\left(z\right)\right]}{\frac{d}{dz}\left[\left(1-P_{i}\left(z\right)\right)\left(P_{i}\left(z\right)-z\right)\right]}\\
&=&\lim_{z\rightarrow1^{+}}\frac{-\left(1-F_{i}\left(z\right)\right) P_{i}^{'}\left(z\right)-(1-z) F_{i}^{'}\left(z\right) P_{i}^{'}\left(z\right)+(1-z) \left(1-F_{i}\left(z\right)\right) P_{i}^{''}\left(z\right)}{\left(1-P_{i}\left(z\right)\right) \left(-1+P_{i}^{'}\left(z\right)\right)-\left(-z+P_{i}\left(z\right)\right) P_{i}^{'}\left(z\right)}\frac{}{}
\end{eqnarray*}

%______________________________________________________
\begin{eqnarray*}
&&\lim_{z\rightarrow1^{+}}\frac{\left(1-z\right)\left(1-F_{i}\left(z\right)\right)P_{i}\left(z\right)P_{i}^{'}\left(z\right)}{\left(1-P_{i}\left(z\right)\right)^{2}\left(P_{i}\left(z\right)-z\right)}=\lim_{z\rightarrow1^{+}}\frac{\frac{d}{dz}\left[\left(1-z\right)\left(1-F_{i}\left(z\right)\right)P_{i}\left(z\right)P_{i}^{'}\left(z\right)\right]}{\frac{d}{dz}\left[\left(1-P_{i}\left(z\right)\right)^{2}\left(P_{i}\left(z\right)-z\right)\right]}\\
&=&\lim_{z\rightarrow1^{+}}\frac{-\left(1-F_{i}\left(z\right)\right) P_{i}\left(z\right) P_{i}^{'}\left(z\right)-(1-z) P_{i}\left(z\right) F_{i}^{'}\left(z\right)P_i'[z]}{\left(1-P_{i}\left(z\right)\right)^2 \left(-1+P_{i}^{'}\left(z\right)\right)-2 \left(1-P_{i}\left(z\right)\right) \left(-z+P_{i}\left(z\right)\right) P_{i}^{'}\left(z\right)}\\
&+&\lim_{z\rightarrow1^{+}}\frac{(1-z) \left(1-F_{i}\left(z\right)\right) P_{i}^{'}\left(z\right)^2+(1-z) \left(1-F_{i}\left(z\right)\right) P_{i}\left(z\right) P_{i}^{''}\left(z\right)}{\left(1-P_{i}\left(z\right)\right)^2 \left(-1+P_{i}^{'}\left(z\right)\right)-2 \left(1-P_{i}\left(z\right)\right) \left(-z+P_{i}\left(z\right)\right) P_{i}^{'}\left(z\right)}\\
\end{eqnarray*}

%___________________________________________________________________________________________
\subsection{Longitudes de la Cola en cualquier tiempo}
%___________________________________________________________________________________________

Sea
$V_{i}\left(z\right)=\frac{1}{\esp\left[C_{i}\right]}\frac{I_{i}\left(z\right)-1}{z-P_{i}\left(z\right)}$

%{\esp\lef[I_{i}\right]}\frac{1-\mu_{i}}{z-P_{i}\left(z\right)}

\begin{eqnarray*}
\frac{\partial V_{i}\left(z\right)}{\partial z}&=&\frac{1}{\esp\left[C_{i}\right]}\left[\frac{I_{i}{'}\left(z\right)\left(z-P_{i}\left(z\right)\right)}{z-P_{i}\left(z\right)}-\frac{\left(I_{i}\left(z\right)-1\right)\left(1-P_{i}{'}\left(z\right)\right)}{\left(z-P_{i}\left(z\right)\right)^{2}}\right]
\end{eqnarray*}


La FGP para el tiempo de espera para cualquier usuario en la cola est\'a dada por:
\[U_{i}\left(z\right)=\frac{1}{\esp\left[C_{i}\right]}\cdot\frac{1-P_{i}\left(z\right)}{z-P_{i}\left(z\right)}\cdot\frac{I_{i}\left(z\right)-1}{1-z}\]

entonces


\begin{eqnarray*}
\frac{d}{dz}V_{i}\left(z\right)&=&\frac{1}{\esp\left[C_{i}\right]}\left\{\frac{d}{dz}\left(\frac{1-P_{i}\left(z\right)}{z-P_{i}\left(z\right)}\right)\frac{I_{i}\left(z\right)-1}{1-z}+\frac{1-P_{i}\left(z\right)}{z-P_{i}\left(z\right)}\frac{d}{dz}\left(\frac{I_{i}\left(z\right)-1}{1-z}\right)\right\}\\
&=&\frac{1}{\esp\left[C_{i}\right]}\left\{\frac{-P_{i}\left(z\right)\left(z-P_{i}\left(z\right)\right)-\left(1-P_{i}\left(z\right)\right)\left(1-P_{i}^{'}\left(z\right)\right)}{\left(z-P_{i}\left(z\right)\right)^{2}}\cdot\frac{I_{i}\left(z\right)-1}{1-z}\right\}\\
&+&\frac{1}{\esp\left[C_{i}\right]}\left\{\frac{1-P_{i}\left(z\right)}{z-P_{i}\left(z\right)}\cdot\frac{I_{i}^{'}\left(z\right)\left(1-z\right)+\left(I_{i}\left(z\right)-1\right)}{\left(1-z\right)^{2}}\right\}
\end{eqnarray*}
%\frac{I_{i}\left(z\right)-1}{1-z}
%+\frac{1-P_{i}\left(z\right)}{z-P_{i}\frac{d}{dz}\left(\frac{I_{i}\left(z\right)-1}{1-z}\right)


\begin{eqnarray*}
\frac{\partial U_{i}\left(z\right)}{\partial z}&=&\frac{(-1+I_{i}[z]) (1-P_{i}[z])}{(1-z)^2 \esp[I_{i}] (z-P_{i}[z])}+\frac{(1-P_{i}[z]) I_{i}^{'}[z]}{(1-z) \esp[I_{i}] (z-P_{i}[z])}-\frac{(-1+I_{i}[z]) (1-P_{i}[z])\left(1-P{'}[z]\right)}{(1-z) \esp[I_{i}] (z-P_{i}[z])^2}\\
&-&\frac{(-1+I_{i}[z]) P_{i}{'}[z]}{(1-z) \esp[I_{i}](z-P_{i}[z])}
\end{eqnarray*}
%___________________________________________________________________________________________
%\vspace{5.5cm}
\section{Preliminares: Modelos de Flujo}
%\vspace{-1.0cm}
%___________________________________________________________________________________________
%
\subsection{Procesos Regenerativos}
%_____________________________________________________

Si $x$ es el n{\'u}mero de usuarios en la cola al comienzo del
periodo de servicio y $N_{s}\left(x\right)=N\left(x\right)$ es el
n{\'u}mero de usuarios que son atendidos con la pol{\'\i}tica $s$,
{\'u}nica en nuestro caso, durante un periodo de servicio,
entonces se asume que:
\begin{itemize}
\item[(S1.)]
\begin{equation}\label{S1}
lim_{x\rightarrow\infty}\esp\left[N\left(x\right)\right]=\overline{N}>0.
\end{equation}
\item[(S2.)]
\begin{equation}\label{S2}
\esp\left[N\left(x\right)\right]\leq \overline{N}, \end{equation}
para cualquier valor de $x$. \item La $n$-{\'e}sima ocurrencia va
acompa{\~n}ada con el tiempo de cambio de longitud
$\delta_{j,j+1}\left(n\right)$, independientes e id{\'e}nticamente
distribuidas, con
$\esp\left[\delta_{j,j+1}\left(1\right)\right]\geq0$. \item Se
define
\begin{equation}
\delta^{*}:=\sum_{j,j+1}\esp\left[\delta_{j,j+1}\left(1\right)\right].
\end{equation}

\item Los tiempos de inter-arribo a la cola $k$,son de la forma
$\left\{\xi_{k}\left(n\right)\right\}_{n\geq1}$, con la propiedad
de que son independientes e id{\'e}nticamente distribuidos.

\item Los tiempos de servicio
$\left\{\eta_{k}\left(n\right)\right\}_{n\geq1}$ tienen la
propiedad de ser independientes e id{\'e}nticamente distribuidos.

\item Se define la tasa de arribo a la $k$-{\'e}sima cola como
$\lambda_{k}=1/\esp\left[\xi_{k}\left(1\right)\right]$ y
adem{\'a}s se define

\item la tasa de servicio para la $k$-{\'e}sima cola como
$\mu_{k}=1/\esp\left[\eta_{k}\left(1\right)\right]$

\item tambi{\'e}n se define $\rho_{k}=\lambda_{k}/\mu_{k}$, donde
es necesario que $\rho<1$ para cuestiones de estabilidad.

\item De las pol{\'\i}ticas posibles solamente consideraremos la
pol{\'\i}tica cerrada (Gated).
\end{itemize}

Las Colas C\'iclicas se pueden describir por medio de un proceso
de Markov $\left(X\left(t\right)\right)_{t\in\rea}$, donde el
estado del sistema al tiempo $t\geq0$ est\'a dado por
\begin{equation}
X\left(t\right)=\left(Q\left(t\right),A\left(t\right),H\left(t\right),B\left(t\right),B^{0}\left(t\right),C\left(t\right)\right)
\end{equation}
definido en el espacio producto:
\begin{equation}
\mathcal{X}=\mathbb{Z}^{K}\times\rea_{+}^{K}\times\left(\left\{1,2,\ldots,K\right\}\times\left\{1,2,\ldots,S\right\}\right)^{M}\times\rea_{+}^{K}\times\rea_{+}^{K}\times\mathbb{Z}^{K},
\end{equation}

\begin{itemize}
\item $Q\left(t\right)=\left(Q_{k}\left(t\right),1\leq k\leq
K\right)$, es el n\'umero de usuarios en la cola $k$, incluyendo
aquellos que est\'an siendo atendidos provenientes de la
$k$-\'esima cola.

\item $A\left(t\right)=\left(A_{k}\left(t\right),1\leq k\leq
K\right)$, son los residuales de los tiempos de arribo en la cola
$k$. \item $H\left(t\right)$ es el par ordenado que consiste en la
cola que esta siendo atendida y la pol\'itica de servicio que se
utilizar\'a.

\item $B\left(t\right)$ es el tiempo de servicio residual.

\item $B^{0}\left(t\right)$ es el tiempo residual del cambio de
cola.

\item $C\left(t\right)$ indica el n\'umero de usuarios atendidos
durante la visita del servidor a la cola dada en
$H\left(t\right)$.
\end{itemize}

$A_{k}\left(t\right),B_{m}\left(t\right)$ y
$B_{m}^{0}\left(t\right)$ se suponen continuas por la derecha y
que satisfacen la propiedad fuerte de Markov, (\cite{Dai})

\begin{itemize}
\item Los tiempos de interarribo a la cola $k$,son de la forma
$\left\{\xi_{k}\left(n\right)\right\}_{n\geq1}$, con la propiedad
de que son independientes e id{\'e}nticamente distribuidos.

\item Los tiempos de servicio
$\left\{\eta_{k}\left(n\right)\right\}_{n\geq1}$ tienen la
propiedad de ser independientes e id{\'e}nticamente distribuidos.

\item Se define la tasa de arribo a la $k$-{\'e}sima cola como
$\lambda_{k}=1/\esp\left[\xi_{k}\left(1\right)\right]$ y
adem{\'a}s se define

\item la tasa de servicio para la $k$-{\'e}sima cola como
$\mu_{k}=1/\esp\left[\eta_{k}\left(1\right)\right]$

\item tambi{\'e}n se define $\rho_{k}=\lambda_{k}/\mu_{k}$, donde
es necesario que $\rho<1$ para cuestiones de estabilidad.

\item De las pol{\'\i}ticas posibles solamente consideraremos la
pol{\'\i}tica cerrada (Gated).
\end{itemize}

%\section{Preliminares}



Sup\'ongase que el sistema consta de varias colas a los cuales
llegan uno o varios servidores a dar servicio a los usuarios
esperando en la cola.\\


Si $x$ es el n\'umero de usuarios en la cola al comienzo del
periodo de servicio y $N_{s}\left(x\right)=N\left(x\right)$ es el
n\'umero de usuarios que son atendidos con la pol\'itica $s$,
\'unica en nuestro caso, durante un periodo de servicio, entonces
se asume que:
\begin{itemize}
\item[1)]\label{S1}$lim_{x\rightarrow\infty}\esp\left[N\left(x\right)\right]=\overline{N}>0$
\item[2)]\label{S2}$\esp\left[N\left(x\right)\right]\leq\overline{N}$para
cualquier valor de $x$.
\end{itemize}
La manera en que atiende el servidor $m$-\'esimo, en este caso en
espec\'ifico solo lo ilustraremos con un s\'olo servidor, es la
siguiente:
\begin{itemize}
\item Al t\'ermino de la visita a la cola $j$, el servidor se
cambia a la cola $j^{'}$ con probabilidad
$r_{j,j^{'}}^{m}=r_{j,j^{'}}$

\item La $n$-\'esima ocurrencia va acompa\~nada con el tiempo de
cambio de longitud $\delta_{j,j^{'}}\left(n\right)$,
independientes e id\'enticamente distribuidas, con
$\esp\left[\delta_{j,j^{'}}\left(1\right)\right]\geq0$.

\item Sea $\left\{p_{j}\right\}$ la distribuci\'on invariante
estacionaria \'unica para la Cadena de Markov con matriz de
transici\'on $\left(r_{j,j^{'}}\right)$.

\item Finalmente, se define
\begin{equation}
\delta^{*}:=\sum_{j,j^{'}}p_{j}r_{j,j^{'}}\esp\left[\delta_{j,j^{'}}\left(i\right)\right].
\end{equation}
\end{itemize}

Veamos un caso muy espec\'ifico en el cual los tiempos de arribo a cada una de las colas se comportan de acuerdo a un proceso Poisson de la forma
$\left\{\xi_{k}\left(n\right)\right\}_{n\geq1}$, y los tiempos de servicio en cada una de las colas son variables aleatorias distribuidas exponencialmente e id\'enticamente distribuidas
$\left\{\eta_{k}\left(n\right)\right\}_{n\geq1}$, donde ambos procesos adem\'as cumplen la condici\'on de ser independientes entre si. Para la $k$-\'esima cola se define la tasa de arribo a la como
$\lambda_{k}=1/\esp\left[\xi_{k}\left(1\right)\right]$ y la tasa
de servicio como
$\mu_{k}=1/\esp\left[\eta_{k}\left(1\right)\right]$, finalmente se
define la carga de la cola como $\rho_{k}=\lambda_{k}/\mu_{k}$,
donde se pide que $\rho<1$, para garantizar la estabilidad del sistema.\\

Se denotar\'a por $Q_{k}\left(t\right)$ el n\'umero de usuarios en la cola $k$,
$A_{k}\left(t\right)$ los residuales de los tiempos entre arribos a la cola $k$;
para cada servidor $m$, se denota por $B_{m}\left(t\right)$ los residuales de los tiempos de servicio al tiempo $t$; $B_{m}^{0}\left(t\right)$ son los residuales de los tiempos de traslado de la cola $k$ a la pr\'oxima por atender, al tiempo $t$, finalmente sea $C_{m}\left(t\right)$ el n\'umero de usuarios atendidos durante la visita del servidor a la cola $k$ al tiempo $t$.\\


En este sentido el proceso para el sistema de visitas se puede definir como:

\begin{equation}\label{Esp.Edos.Down}
X\left(t\right)^{T}=\left(Q_{k}\left(t\right),A_{k}\left(t\right),B_{m}\left(t\right),B_{m}^{0}\left(t\right),C_{m}\left(t\right)\right)
\end{equation}
para $k=1,\ldots,K$ y $m=1,2,\ldots,M$. $X$ evoluciona en el
espacio de estados:
$X=\ent_{+}^{K}\times\rea_{+}^{K}\times\left(\left\{1,2,\ldots,K\right\}\times\left\{1,2,\ldots,S\right\}\right)^{M}\times\rea_{+}^{K}\times\ent_{+}^{K}$.\\

El sistema aqu\'i descrito debe de cumplir con los siguientes supuestos b\'asicos de un sistema de visitas:

Antes enunciemos los supuestos que regir\'an en la red.

\begin{itemize}
\item[A1)] $\xi_{1},\ldots,\xi_{K},\eta_{1},\ldots,\eta_{K}$ son
mutuamente independientes y son sucesiones independientes e
id\'enticamente distribuidas.

\item[A2)] Para alg\'un entero $p\geq1$
\begin{eqnarray*}
\esp\left[\xi_{l}\left(1\right)^{p+1}\right]<\infty\textrm{ para }l\in\mathcal{A}\textrm{ y }\\
\esp\left[\eta_{k}\left(1\right)^{p+1}\right]<\infty\textrm{ para
}k=1,\ldots,K.
\end{eqnarray*}
donde $\mathcal{A}$ es la clase de posibles arribos.

\item[A3)] Para $k=1,2,\ldots,K$ existe una funci\'on positiva
$q_{k}\left(x\right)$ definida en $\rea_{+}$, y un entero $j_{k}$,
tal que
\begin{eqnarray}
P\left(\xi_{k}\left(1\right)\geq x\right)>0\textrm{, para todo }x>0\\
P\left\{a\leq\sum_{i=1}^{j_{k}}\xi_{k}\left(i\right)\leq
b\right\}\geq\int_{a}^{b}q_{k}\left(x\right)dx, \textrm{ }0\leq
a<b.
\end{eqnarray}
\end{itemize}

En particular los procesos de tiempo entre arribos y de servicio
considerados con fines de ilustraci\'on de la metodolog\'ia
cumplen con el supuesto $A2)$ para $p=1$, es decir, ambos procesos
tienen primer y segundo momento finito.

En lo que respecta al supuesto (A3), en Dai y Meyn \cite{DaiSean}
hacen ver que este se puede sustituir por

\begin{itemize}
\item[A3')] Para el Proceso de Markov $X$, cada subconjunto
compacto de $X$ es un conjunto peque\~no, ver definici\'on
\ref{Def.Cto.Peq.}.
\end{itemize}

Es por esta raz\'on que con la finalidad de poder hacer uso de
$A3^{'})$ es necesario recurrir a los Procesos de Harris y en
particular a los Procesos Harris Recurrente:
%_______________________________________________________________________
\subsection{Procesos Harris Recurrente}
%_______________________________________________________________________

Por el supuesto (A1) conforme a Davis \cite{Davis}, se puede
definir el proceso de saltos correspondiente de manera tal que
satisfaga el supuesto (\ref{Sup3.1.Davis}), de hecho la
demostraci\'on est\'a basada en la l\'inea de argumentaci\'on de
Davis, (\cite{Davis}, p\'aginas 362-364).

Entonces se tiene un espacio de estados Markoviano. El espacio de
Markov descrito en Dai y Meyn \cite{DaiSean}

\[\left(\Omega,\mathcal{F},\mathcal{F}_{t},X\left(t\right),\theta_{t},P_{x}\right)\]
es un proceso de Borel Derecho (Sharpe \cite{Sharpe}) en el
espacio de estados medible $\left(X,\mathcal{B}_{X}\right)$. El
Proceso $X=\left\{X\left(t\right),t\geq0\right\}$ tiene
trayectorias continuas por la derecha, est\'a definida en
$\left(\Omega,\mathcal{F}\right)$ y est\'a adaptado a
$\left\{\mathcal{F}_{t},t\geq0\right\}$; la colecci\'on
$\left\{P_{x},x\in \mathbb{X}\right\}$ son medidas de probabilidad
en $\left(\Omega,\mathcal{F}\right)$ tales que para todo $x\in
\mathbb{X}$
\[P_{x}\left\{X\left(0\right)=x\right\}=1\] y
\[E_{x}\left\{f\left(X\circ\theta_{t}\right)|\mathcal{F}_{t}\right\}=E_{X}\left(\tau\right)f\left(X\right)\]
en $\left\{\tau<\infty\right\}$, $P_{x}$-c.s. Donde $\tau$ es un
$\mathcal{F}_{t}$-tiempo de paro
\[\left(X\circ\theta_{\tau}\right)\left(w\right)=\left\{X\left(\tau\left(w\right)+t,w\right),t\geq0\right\}\]
y $f$ es una funci\'on de valores reales acotada y medible con la
$\sigma$-algebra de Kolmogorov generada por los cilindros.\\

Sea $P^{t}\left(x,D\right)$, $D\in\mathcal{B}_{\mathbb{X}}$,
$t\geq0$ probabilidad de transici\'on de $X$ definida como
\[P^{t}\left(x,D\right)=P_{x}\left(X\left(t\right)\in
D\right)\]


\begin{Def}
Una medida no cero $\pi$ en
$\left(\mathbf{X},\mathcal{B}_{\mathbf{X}}\right)$ es {\bf
invariante} para $X$ si $\pi$ es $\sigma$-finita y
\[\pi\left(D\right)=\int_{\mathbf{X}}P^{t}\left(x,D\right)\pi\left(dx\right)\]
para todo $D\in \mathcal{B}_{\mathbf{X}}$, con $t\geq0$.
\end{Def}

\begin{Def}
El proceso de Markov $X$ es llamado Harris recurrente si existe
una medida de probabilidad $\nu$ en
$\left(\mathbf{X},\mathcal{B}_{\mathbf{X}}\right)$, tal que si
$\nu\left(D\right)>0$ y $D\in\mathcal{B}_{\mathbf{X}}$
\[P_{x}\left\{\tau_{D}<\infty\right\}\equiv1\] cuando
$\tau_{D}=inf\left\{t\geq0:X_{t}\in D\right\}$.
\end{Def}

\begin{Note}
\begin{itemize}
\item[i)] Si $X$ es Harris recurrente, entonces existe una \'unica
medida invariante $\pi$ (Getoor \cite{Getoor}).

\item[ii)] Si la medida invariante es finita, entonces puede
normalizarse a una medida de probabilidad, en este caso se le
llama Proceso {\em Harris recurrente positivo}.


\item[iii)] Cuando $X$ es Harris recurrente positivo se dice que
la disciplina de servicio es estable. En este caso $\pi$ denota la
distribuci\'on estacionaria y hacemos
\[P_{\pi}\left(\cdot\right)=\int_{\mathbf{X}}P_{x}\left(\cdot\right)\pi\left(dx\right)\]
y se utiliza $E_{\pi}$ para denotar el operador esperanza
correspondiente.
\end{itemize}
\end{Note}

\begin{Def}\label{Def.Cto.Peq.}
Un conjunto $D\in\mathcal{B_{\mathbf{X}}}$ es llamado peque\~no si
existe un $t>0$, una medida de probabilidad $\nu$ en
$\mathcal{B_{\mathbf{X}}}$, y un $\delta>0$ tal que
\[P^{t}\left(x,A\right)\geq\delta\nu\left(A\right)\] para $x\in
D,A\in\mathcal{B_{X}}$.
\end{Def}

La siguiente serie de resultados vienen enunciados y demostrados
en Dai \cite{Dai}:
\begin{Lema}[Lema 3.1, Dai\cite{Dai}]
Sea $B$ conjunto peque\~no cerrado, supongamos que
$P_{x}\left(\tau_{B}<\infty\right)\equiv1$ y que para alg\'un
$\delta>0$ se cumple que
\begin{equation}\label{Eq.3.1}
\sup\esp_{x}\left[\tau_{B}\left(\delta\right)\right]<\infty,
\end{equation}
donde
$\tau_{B}\left(\delta\right)=inf\left\{t\geq\delta:X\left(t\right)\in
B\right\}$. Entonces, $X$ es un proceso Harris Recurrente
Positivo.
\end{Lema}

\begin{Lema}[Lema 3.1, Dai \cite{Dai}]\label{Lema.3.}
Bajo el supuesto (A3), el conjunto $B=\left\{|x|\leq k\right\}$ es
un conjunto peque\~no cerrado para cualquier $k>0$.
\end{Lema}

\begin{Teo}[Teorema 3.1, Dai\cite{Dai}]\label{Tma.3.1}
Si existe un $\delta>0$ tal que
\begin{equation}
lim_{|x|\rightarrow\infty}\frac{1}{|x|}\esp|X^{x}\left(|x|\delta\right)|=0,
\end{equation}
entonces la ecuaci\'on (\ref{Eq.3.1}) se cumple para
$B=\left\{|x|\leq k\right\}$ con alg\'un $k>0$. En particular, $X$
es Harris Recurrente Positivo.
\end{Teo}

\begin{Note}
En Meyn and Tweedie \cite{MeynTweedie} muestran que si
$P_{x}\left\{\tau_{D}<\infty\right\}\equiv1$ incluso para solo un
conjunto peque\~no, entonces el proceso es Harris Recurrente.
\end{Note}

Entonces, tenemos que el proceso $X$ es un proceso de Markov que
cumple con los supuestos $A1)$-$A3)$, lo que falta de hacer es
construir el Modelo de Flujo bas\'andonos en lo hasta ahora
presentado.
%_______________________________________________________________________
\subsection{Modelo de Flujo}
%_______________________________________________________________________

Dada una condici\'on inicial $x\in\textrm{X}$, sea
$Q_{k}^{x}\left(t\right)$ la longitud de la cola al tiempo $t$,
$T_{m,k}^{x}\left(t\right)$ el tiempo acumulado, al tiempo $t$,
que tarda el servidor $m$ en atender a los usuarios de la cola
$k$. Finalmente sea $T_{m,k}^{x,0}\left(t\right)$ el tiempo
acumulado, al tiempo $t$, que tarda el servidor $m$ en trasladarse
a otra cola a partir de la $k$-\'esima.\\

Sup\'ongase que la funci\'on
$\left(\overline{Q}\left(\cdot\right),\overline{T}_{m}
\left(\cdot\right),\overline{T}_{m}^{0} \left(\cdot\right)\right)$
para $m=1,2,\ldots,M$ es un punto l\'imite de
\begin{equation}\label{Eq.Punto.Limite}
\left(\frac{1}{|x|}Q^{x}\left(|x|t\right),\frac{1}{|x|}T_{m}^{x}\left(|x|t\right),\frac{1}{|x|}T_{m}^{x,0}\left(|x|t\right)\right)
\end{equation}
para $m=1,2,\ldots,M$, cuando $x\rightarrow\infty$. Entonces
$\left(\overline{Q}\left(t\right),\overline{T}_{m}
\left(t\right),\overline{T}_{m}^{0} \left(t\right)\right)$ es un
flujo l\'imite del sistema. Al conjunto de todos las posibles
flujos l\'imite se le llama \textbf{Modelo de Flujo}.\\

El modelo de flujo satisface el siguiente conjunto de ecuaciones:

\begin{equation}\label{Eq.MF.1}
\overline{Q}_{k}\left(t\right)=\overline{Q}_{k}\left(0\right)+\lambda_{k}t-\sum_{m=1}^{M}\mu_{k}\overline{T}_{m,k}\left(t\right)\\
\end{equation}
para $k=1,2,\ldots,K$.\\
\begin{equation}\label{Eq.MF.2}
\overline{Q}_{k}\left(t\right)\geq0\textrm{ para
}k=1,2,\ldots,K,\\
\end{equation}

\begin{equation}\label{Eq.MF.3}
\overline{T}_{m,k}\left(0\right)=0,\textrm{ y }\overline{T}_{m,k}\left(\cdot\right)\textrm{ es no decreciente},\\
\end{equation}
para $k=1,2,\ldots,K$ y $m=1,2,\ldots,M$,\\
\begin{equation}\label{Eq.MF.4}
\sum_{k=1}^{K}\overline{T}_{m,k}^{0}\left(t\right)+\overline{T}_{m,k}\left(t\right)=t\textrm{
para }m=1,2,\ldots,M.\\
\end{equation}

De acuerdo a Dai \cite{Dai}, se tiene que el conjunto de posibles
l\'imites
$\left(\overline{Q}\left(\cdot\right),\overline{T}\left(\cdot\right),\overline{T}^{0}\left(\cdot\right)\right)$,
en el sentido de que deben de satisfacer las ecuaciones
(\ref{Eq.MF.1})-(\ref{Eq.MF.4}), se le llama {\em Modelo de
Flujo}.


\begin{Def}[Definici\'on 4.1, , Dai \cite{Dai}]\label{Def.Modelo.Flujo}
Sea una disciplina de servicio espec\'ifica. Cualquier l\'imite
$\left(\overline{Q}\left(\cdot\right),\overline{T}\left(\cdot\right)\right)$
en (\ref{Eq.Punto.Limite}) es un {\em flujo l\'imite} de la
disciplina. Cualquier soluci\'on (\ref{Eq.MF.1})-(\ref{Eq.MF.4})
es llamado flujo soluci\'on de la disciplina. Se dice que el
modelo de flujo l\'imite, modelo de flujo, de la disciplina de la
cola es estable si existe una constante $\delta>0$ que depende de
$\mu,\lambda$ y $P$ solamente, tal que cualquier flujo l\'imite
con
$|\overline{Q}\left(0\right)|+|\overline{U}|+|\overline{V}|=1$, se
tiene que $\overline{Q}\left(\cdot+\delta\right)\equiv0$.
\end{Def}

Al conjunto de ecuaciones dadas en \ref{Eq.MF.1}-\ref{Eq.MF.4} se
le llama {\em Modelo de flujo} y al conjunto de todas las
soluciones del modelo de flujo
$\left(\overline{Q}\left(\cdot\right),\overline{T}
\left(\cdot\right)\right)$ se le denotar\'a por $\mathcal{Q}$.

Si se hace $|x|\rightarrow\infty$ sin restringir ninguna de las
componentes, tambi\'en se obtienen un modelo de flujo, pero en
este caso el residual de los procesos de arribo y servicio
introducen un retraso:
\begin{Teo}[Teorema 4.2, Dai\cite{Dai}]\label{Tma.4.2.Dai}
Sea una disciplina fija para la cola, suponga que se cumplen las
condiciones (A1))-(A3)). Si el modelo de flujo l\'imite de la
disciplina de la cola es estable, entonces la cadena de Markov $X$
que describe la din\'amica de la red bajo la disciplina es Harris
recurrente positiva.
\end{Teo}

Ahora se procede a escalar el espacio y el tiempo para reducir la
aparente fluctuaci\'on del modelo. Consid\'erese el proceso
\begin{equation}\label{Eq.3.7}
\overline{Q}^{x}\left(t\right)=\frac{1}{|x|}Q^{x}\left(|x|t\right)
\end{equation}
A este proceso se le conoce como el fluido escalado, y cualquier
l\'imite $\overline{Q}^{x}\left(t\right)$ es llamado flujo
l\'imite del proceso de longitud de la cola. Haciendo
$|q|\rightarrow\infty$ mientras se mantiene el resto de las
componentes fijas, cualquier punto l\'imite del proceso de
longitud de la cola normalizado $\overline{Q}^{x}$ es soluci\'on
del siguiente modelo de flujo.


\begin{Def}[Definici\'on 3.3, Dai y Meyn \cite{DaiSean}]
El modelo de flujo es estable si existe un tiempo fijo $t_{0}$ tal
que $\overline{Q}\left(t\right)=0$, con $t\geq t_{0}$, para
cualquier $\overline{Q}\left(\cdot\right)\in\mathcal{Q}$ que
cumple con $|\overline{Q}\left(0\right)|=1$.
\end{Def}

El siguiente resultado se encuentra en Chen \cite{Chen}.
\begin{Lemma}[Lema 3.1, Dai y Meyn \cite{DaiSean}]
Si el modelo de flujo definido por \ref{Eq.MF.1}-\ref{Eq.MF.4} es
estable, entonces el modelo de flujo retrasado es tambi\'en
estable, es decir, existe $t_{0}>0$ tal que
$\overline{Q}\left(t\right)=0$ para cualquier $t\geq t_{0}$, para
cualquier soluci\'on del modelo de flujo retrasado cuya
condici\'on inicial $\overline{x}$ satisface que
$|\overline{x}|=|\overline{Q}\left(0\right)|+|\overline{A}\left(0\right)|+|\overline{B}\left(0\right)|\leq1$.
\end{Lemma}


Ahora ya estamos en condiciones de enunciar los resultados principales:


\begin{Teo}[Teorema 2.1, Down \cite{Down}]\label{Tma2.1.Down}
Suponga que el modelo de flujo es estable, y que se cumplen los supuestos (A1) y (A2), entonces
\begin{itemize}
\item[i)] Para alguna constante $\kappa_{p}$, y para cada
condici\'on inicial $x\in X$
\begin{equation}\label{Estability.Eq1}
limsup_{t\rightarrow\infty}\frac{1}{t}\int_{0}^{t}\esp_{x}\left[|Q\left(s\right)|^{p}\right]ds\leq\kappa_{p},
\end{equation}
donde $p$ es el entero dado en (A2).
\end{itemize}
Si adem\'as se cumple la condici\'on (A3), entonces para cada
condici\'on inicial:
\begin{itemize}
\item[ii)] Los momentos transitorios convergen a su estado
estacionario:
 \begin{equation}\label{Estability.Eq2}
lim_{t\rightarrow\infty}\esp_{x}\left[Q_{k}\left(t\right)^{r}\right]=\esp_{\pi}\left[Q_{k}\left(0\right)^{r}\right]\leq\kappa_{r},
\end{equation}
para $r=1,2,\ldots,p$ y $k=1,2,\ldots,K$. Donde $\pi$ es la
probabilidad invariante para $\mathbf{X}$.

\item[iii)]  El primer momento converge con raz\'on $t^{p-1}$:
\begin{equation}\label{Estability.Eq3}
lim_{t\rightarrow\infty}t^{p-1}|\esp_{x}\left[Q_{k}\left(t\right)\right]-\esp_{\pi}\left[Q_{k}\left(0\right)\right]=0.
\end{equation}

\item[iv)] La {\em Ley Fuerte de los grandes n\'umeros} se cumple:
\begin{equation}\label{Estability.Eq4}
lim_{t\rightarrow\infty}\frac{1}{t}\int_{0}^{t}Q_{k}^{r}\left(s\right)ds=\esp_{\pi}\left[Q_{k}\left(0\right)^{r}\right],\textrm{
}\prob_{x}\textrm{-c.s.}
\end{equation}
para $r=1,2,\ldots,p$ y $k=1,2,\ldots,K$.
\end{itemize}
\end{Teo}

La contribuci\'on de Down a la teor\'ia de los Sistemas de Visitas
C\'iclicas, es la relaci\'on que hay entre la estabilidad del
sistema con el comportamiento de las medidas de desempe\~no, es
decir, la condici\'on suficiente para poder garantizar la
convergencia del proceso de la longitud de la cola as\'i como de
por los menos los dos primeros momentos adem\'as de una versi\'on
de la Ley Fuerte de los Grandes N\'umeros para los sistemas de
visitas.


\begin{Teo}[Teorema 2.3, Down \cite{Down}]\label{Tma2.3.Down}
Considere el siguiente valor:
\begin{equation}\label{Eq.Rho.1serv}
\rho=\sum_{k=1}^{K}\rho_{k}+max_{1\leq j\leq K}\left(\frac{\lambda_{j}}{\sum_{s=1}^{S}p_{js}\overline{N}_{s}}\right)\delta^{*}
\end{equation}
\begin{itemize}
\item[i)] Si $\rho<1$ entonces la red es estable, es decir, se cumple el teorema \ref{Tma2.1.Down}.

\item[ii)] Si $\rho<1$ entonces la red es inestable, es decir, se cumple el teorema \ref{Tma2.2.Down}
\end{itemize}
\end{Teo}

\begin{Teo}
Sea $\left(X_{n},\mathcal{F}_{n},n=0,1,\ldots,\right\}$ Proceso de
Markov con espacio de estados $\left(S_{0},\chi_{0}\right)$
generado por una distribuici\'on inicial $P_{o}$ y probabilidad de
transici\'on $p_{mn}$, para $m,n=0,1,\ldots,$ $m<n$, que por
notaci\'on se escribir\'a como $p\left(m,n,x,B\right)\rightarrow
p_{mn}\left(x,B\right)$. Sea $S$ tiempo de paro relativo a la
$\sigma$-\'algebra $\mathcal{F}_{n}$. Sea $T$ funci\'on medible,
$T:\Omega\rightarrow\left\{0,1,\ldots,\right\}$. Sup\'ongase que
$T\geq S$, entonces $T$ es tiempo de paro. Si $B\in\chi_{0}$,
entonces
\begin{equation}\label{Prop.Fuerte.Markov}
P\left\{X\left(T\right)\in
B,T<\infty|\mathcal{F}\left(S\right)\right\} =
p\left(S,T,X\left(s\right),B\right)
\end{equation}
en $\left\{T<\infty\right\}$.
\end{Teo}


Sea $K$ conjunto numerable y sea $d:K\rightarrow\nat$ funci\'on.
Para $v\in K$, $M_{v}$ es un conjunto abierto de
$\rea^{d\left(v\right)}$. Entonces \[E=\cup_{v\in
K}M_{v}=\left\{\left(v,\zeta\right):v\in K,\zeta\in
M_{v}\right\}.\]

Sea $\mathcal{E}$ la clase de conjuntos medibles en $E$:
\[\mathcal{E}=\left\{\cup_{v\in K}A_{v}:A_{v}\in \mathcal{M}_{v}\right\}.\]

donde $\mathcal{M}$ son los conjuntos de Borel de $M_{v}$.
Entonces $\left(E,\mathcal{E}\right)$ es un espacio de Borel. El
estado del proceso se denotar\'a por
$\mathbf{x}_{t}=\left(v_{t},\zeta_{t}\right)$. La distribuci\'on
de $\left(\mathbf{x}_{t}\right)$ est\'a determinada por por los
siguientes objetos:

\begin{itemize}
\item[i)] Los campos vectoriales $\left(\mathcal{H}_{v},v\in
K\right)$. \item[ii)] Una funci\'on medible $\lambda:E\rightarrow
\rea_{+}$. \item[iii)] Una medida de transici\'on
$Q:\mathcal{E}\times\left(E\cup\Gamma^{*}\right)\rightarrow\left[0,1\right]$
donde
\begin{equation}
\Gamma^{*}=\cup_{v\in K}\partial^{*}M_{v}.
\end{equation}
y
\begin{equation}
\partial^{*}M_{v}=\left\{z\in\partial M_{v}:\mathbf{\mathbf{\phi}_{v}\left(t,\zeta\right)=\mathbf{z}}\textrm{ para alguna }\left(t,\zeta\right)\in\rea_{+}\times M_{v}\right\}.
\end{equation}
$\partial M_{v}$ denota  la frontera de $M_{v}$.
\end{itemize}

El campo vectorial $\left(\mathcal{H}_{v},v\in K\right)$ se supone
tal que para cada $\mathbf{z}\in M_{v}$ existe una \'unica curva
integral $\mathbf{\phi}_{v}\left(t,\zeta\right)$ que satisface la
ecuaci\'on

\begin{equation}
\frac{d}{dt}f\left(\zeta_{t}\right)=\mathcal{H}f\left(\zeta_{t}\right),
\end{equation}
con $\zeta_{0}=\mathbf{z}$, para cualquier funci\'on suave
$f:\rea^{d}\rightarrow\rea$ y $\mathcal{H}$ denota el operador
diferencial de primer orden, con $\mathcal{H}=\mathcal{H}_{v}$ y
$\zeta_{t}=\mathbf{\phi}\left(t,\mathbf{z}\right)$. Adem\'as se
supone que $\mathcal{H}_{v}$ es conservativo, es decir, las curvas
integrales est\'an definidas para todo $t>0$.

Para $\mathbf{x}=\left(v,\zeta\right)\in E$ se denota
\[t^{*}\mathbf{x}=inf\left\{t>0:\mathbf{\phi}_{v}\left(t,\zeta\right)\in\partial^{*}M_{v}\right\}\]

En lo que respecta a la funci\'on $\lambda$, se supondr\'a que
para cada $\left(v,\zeta\right)\in E$ existe un $\epsilon>0$ tal
que la funci\'on
$s\rightarrow\lambda\left(v,\phi_{v}\left(s,\zeta\right)\right)\in
E$ es integrable para $s\in\left[0,\epsilon\right)$. La medida de
transici\'on $Q\left(A;\mathbf{x}\right)$ es una funci\'on medible
de $\mathbf{x}$ para cada $A\in\mathcal{E}$, definida para
$\mathbf{x}\in E\cup\Gamma^{*}$ y es una medida de probabilidad en
$\left(E,\mathcal{E}\right)$ para cada $\mathbf{x}\in E$.

El movimiento del proceso $\left(\mathbf{x}_{t}\right)$ comenzando
en $\mathbf{x}=\left(n,\mathbf{z}\right)\in E$ se puede construir
de la siguiente manera, def\'inase la funci\'on $F$ por

\begin{equation}
F\left(t\right)=\left\{\begin{array}{ll}\\
exp\left(-\int_{0}^{t}\lambda\left(n,\phi_{n}\left(s,\mathbf{z}\right)\right)ds\right), & t<t^{*}\left(\mathbf{x}\right),\\
0, & t\geq t^{*}\left(\mathbf{x}\right)
\end{array}\right.
\end{equation}

Sea $T_{1}$ una variable aleatoria tal que
$\prob\left[T_{1}>t\right]=F\left(t\right)$, ahora sea la variable
aleatoria $\left(N,Z\right)$ con distribuici\'on
$Q\left(\cdot;\phi_{n}\left(T_{1},\mathbf{z}\right)\right)$. La
trayectoria de $\left(\mathbf{x}_{t}\right)$ para $t\leq T_{1}$
es\footnote{Revisar p\'agina 362, y 364 de Davis \cite{Davis}.}
\begin{eqnarray*}
\mathbf{x}_{t}=\left(v_{t},\zeta_{t}\right)=\left\{\begin{array}{ll}
\left(n,\phi_{n}\left(t,\mathbf{z}\right)\right), & t<T_{1},\\
\left(N,\mathbf{Z}\right), & t=t_{1}.
\end{array}\right.
\end{eqnarray*}

Comenzando en $\mathbf{x}_{T_{1}}$ se selecciona el siguiente
tiempo de intersalto $T_{2}-T_{1}$ lugar del post-salto
$\mathbf{x}_{T_{2}}$ de manera similar y as\'i sucesivamente. Este
procedimiento nos da una trayectoria determinista por partes
$\mathbf{x}_{t}$ con tiempos de salto $T_{1},T_{2},\ldots$. Bajo
las condiciones enunciadas para $\lambda,T_{1}>0$  y
$T_{1}-T_{2}>0$ para cada $i$, con probabilidad 1. Se supone que
se cumple la siguiente condici\'on.

\begin{Sup}[Supuesto 3.1, Davis \cite{Davis}]\label{Sup3.1.Davis}
Sea $N_{t}:=\sum_{t}\indora_{\left(t\geq t\right)}$ el n\'umero de
saltos en $\left[0,t\right]$. Entonces
\begin{equation}
\esp\left[N_{t}\right]<\infty\textrm{ para toda }t.
\end{equation}
\end{Sup}

es un proceso de Markov, m\'as a\'un, es un Proceso Fuerte de
Markov, es decir, la Propiedad Fuerte de Markov se cumple para
cualquier tiempo de paro.


Sea $E$ es un espacio m\'etrico separable y la m\'etrica $d$ es
compatible con la topolog\'ia.


\begin{Def}
Un espacio topol\'ogico $E$ es llamado de {\em Rad\'on} si es
homeomorfo a un subconjunto universalmente medible de un espacio
m\'etrico compacto.
\end{Def}

Equivalentemente, la definici\'on de un espacio de Rad\'on puede
encontrarse en los siguientes t\'erminos:


\begin{Def}
$E$ es un espacio de Rad\'on si cada medida finita en
$\left(E,\mathcal{B}\left(E\right)\right)$ es regular interior o
cerrada, {\em tight}.
\end{Def}

\begin{Def}
Una medida finita, $\lambda$ en la $\sigma$-\'algebra de Borel de
un espacio metrizable $E$ se dice cerrada si
\begin{equation}\label{Eq.A2.3}
\lambda\left(E\right)=sup\left\{\lambda\left(K\right):K\textrm{ es
compacto en }E\right\}.
\end{equation}
\end{Def}

El siguiente teorema nos permite tener una mejor caracterizaci\'on
de los espacios de Rad\'on:
\begin{Teo}\label{Tma.A2.2}
Sea $E$ espacio separable metrizable. Entonces $E$ es Radoniano si
y s\'olo s\'i cada medida finita en
$\left(E,\mathcal{B}\left(E\right)\right)$ es cerrada.
\end{Teo}

Sea $E$ espacio de estados, tal que $E$ es un espacio de Rad\'on,
$\mathcal{B}\left(E\right)$ $\sigma$-\'algebra de Borel en $E$,
que se denotar\'a por $\mathcal{E}$.

Sea $\left(X,\mathcal{G},\prob\right)$ espacio de probabilidad,
$I\subset\rea$ conjunto de \'indices. Sea $\mathcal{F}_{\leq t}$
la $\sigma$-\'algebra natural definida como
$\sigma\left\{f\left(X_{r}\right):r\in I, r\leq
t,f\in\mathcal{E}\right\}$. Se considerar\'a una
$\sigma$-\'algebra m\'as general, $ \left(\mathcal{G}_{t}\right)$
tal que $\left(X_{t}\right)$ sea $\mathcal{E}$-adaptado.

\begin{Def}
Una familia $\left(P_{s,t}\right)$ de kernels de Markov en
$\left(E,\mathcal{E}\right)$ indexada por pares $s,t\in I$, con
$s\leq t$ es una funci\'on de transici\'on en $\ER$, si  para todo
$r\leq s< t$ en $I$ y todo $x\in E$,
$B\in\mathcal{E}$\footnote{Ecuaci\'on de Chapman-Kolmogorov}
\begin{equation}\label{Eq.Kernels}
P_{r,t}\left(x,B\right)=\int_{E}P_{r,s}\left(x,dy\right)P_{s,t}\left(y,B\right).
\end{equation}
\end{Def}

Se dice que la funci\'on de transici\'on $\KM$ en $\ER$ es la
funci\'on de transici\'on para un proceso $\PE$  con valores en
$E$ y que satisface la propiedad de
Markov\footnote{\begin{equation}\label{Eq.1.4.S}
\prob\left\{H|\mathcal{G}_{t}\right\}=\prob\left\{H|X_{t}\right\}\textrm{
}H\in p\mathcal{F}_{\geq t}.
\end{equation}} (\ref{Eq.1.4.S}) relativa a $\left(\mathcal{G}_{t}\right)$ si

\begin{equation}\label{Eq.1.6.S}
\prob\left\{f\left(X_{t}\right)|\mathcal{G}_{s}\right\}=P_{s,t}f\left(X_{t}\right)\textrm{
}s\leq t\in I,\textrm{ }f\in b\mathcal{E}.
\end{equation}

\begin{Def}
Una familia $\left(P_{t}\right)_{t\geq0}$ de kernels de Markov en
$\ER$ es llamada {\em Semigrupo de Transici\'on de Markov} o {\em
Semigrupo de Transici\'on} si
\[P_{t+s}f\left(x\right)=P_{t}\left(P_{s}f\right)\left(x\right),\textrm{ }t,s\geq0,\textrm{ }x\in E\textrm{ }f\in b\mathcal{E}.\]
\end{Def}
\begin{Note}
Si la funci\'on de transici\'on $\KM$ es llamada homog\'enea si
$P_{s,t}=P_{t-s}$.
\end{Note}

Un proceso de Markov que satisface la ecuaci\'on (\ref{Eq.1.6.S})
con funci\'on de transici\'on homog\'enea $\left(P_{t}\right)$
tiene la propiedad caracter\'istica
\begin{equation}\label{Eq.1.8.S}
\prob\left\{f\left(X_{t+s}\right)|\mathcal{G}_{t}\right\}=P_{s}f\left(X_{t}\right)\textrm{
}t,s\geq0,\textrm{ }f\in b\mathcal{E}.
\end{equation}
La ecuaci\'on anterior es la {\em Propiedad Simple de Markov} de
$X$ relativa a $\left(P_{t}\right)$.

En este sentido el proceso $\PE$ cumple con la propiedad de Markov
(\ref{Eq.1.8.S}) relativa a
$\left(\Omega,\mathcal{G},\mathcal{G}_{t},\prob\right)$ con
semigrupo de transici\'on $\left(P_{t}\right)$.

\begin{Def}
Un proceso estoc\'astico $\PE$ definido en
$\left(\Omega,\mathcal{G},\prob\right)$ con valores en el espacio
topol\'ogico $E$ es continuo por la derecha si cada trayectoria
muestral $t\rightarrow X_{t}\left(w\right)$ es un mapeo continuo
por la derecha de $I$ en $E$.
\end{Def}

\begin{Def}[HD1]\label{Eq.2.1.S}
Un semigrupo de Markov $\left(P_{t}\right)$ en un espacio de
Rad\'on $E$ se dice que satisface la condici\'on {\em HD1} si,
dada una medida de probabilidad $\mu$ en $E$, existe una
$\sigma$-\'algebra $\mathcal{E^{*}}$ con
$\mathcal{E}\subset\mathcal{E}^{*}$ y
$P_{t}\left(b\mathcal{E}^{*}\right)\subset b\mathcal{E}^{*}$, y un
$\mathcal{E}^{*}$-proceso $E$-valuado continuo por la derecha
$\PE$ en alg\'un espacio de probabilidad filtrado
$\left(\Omega,\mathcal{G},\mathcal{G}_{t},\prob\right)$ tal que
$X=\left(\Omega,\mathcal{G},\mathcal{G}_{t},\prob\right)$ es de
Markov (Homog\'eneo) con semigrupo de transici\'on $(P_{t})$ y
distribuci\'on inicial $\mu$.
\end{Def}

Consid\'erese la colecci\'on de variables aleatorias $X_{t}$
definidas en alg\'un espacio de probabilidad, y una colecci\'on de
medidas $\mathbf{P}^{x}$ tales que
$\mathbf{P}^{x}\left\{X_{0}=x\right\}$, y bajo cualquier
$\mathbf{P}^{x}$, $X_{t}$ es de Markov con semigrupo
$\left(P_{t}\right)$. $\mathbf{P}^{x}$ puede considerarse como la
distribuci\'on condicional de $\mathbf{P}$ dado $X_{0}=x$.

\begin{Def}\label{Def.2.2.S}
Sea $E$ espacio de Rad\'on, $\SG$ semigrupo de Markov en $\ER$. La
colecci\'on
$\mathbf{X}=\left(\Omega,\mathcal{G},\mathcal{G}_{t},X_{t},\theta_{t},\CM\right)$
es un proceso $\mathcal{E}$-Markov continuo por la derecha simple,
con espacio de estados $E$ y semigrupo de transici\'on $\SG$ en
caso de que $\mathbf{X}$ satisfaga las siguientes condiciones:
\begin{itemize}
\item[i)] $\left(\Omega,\mathcal{G},\mathcal{G}_{t}\right)$ es un
espacio de medida filtrado, y $X_{t}$ es un proceso $E$-valuado
continuo por la derecha $\mathcal{E}^{*}$-adaptado a
$\left(\mathcal{G}_{t}\right)$;

\item[ii)] $\left(\theta_{t}\right)_{t\geq0}$ es una colecci\'on
de operadores {\em shift} para $X$, es decir, mapea $\Omega$ en
s\'i mismo satisfaciendo para $t,s\geq0$,

\begin{equation}\label{Eq.Shift}
\theta_{t}\circ\theta_{s}=\theta_{t+s}\textrm{ y
}X_{t}\circ\theta_{t}=X_{t+s};
\end{equation}

\item[iii)] Para cualquier $x\in E$,$\CM\left\{X_{0}=x\right\}=1$,
y el proceso $\PE$ tiene la propiedad de Markov (\ref{Eq.1.8.S})
con semigrupo de transici\'on $\SG$ relativo a
$\left(\Omega,\mathcal{G},\mathcal{G}_{t},\CM\right)$.
\end{itemize}
\end{Def}

\begin{Def}[HD2]\label{Eq.2.2.S}
Para cualquier $\alpha>0$ y cualquier $f\in S^{\alpha}$, el
proceso $t\rightarrow f\left(X_{t}\right)$ es continuo por la
derecha casi seguramente.
\end{Def}

\begin{Def}\label{Def.PD}
Un sistema
$\mathbf{X}=\left(\Omega,\mathcal{G},\mathcal{G}_{t},X_{t},\theta_{t},\CM\right)$
es un proceso derecho en el espacio de Rad\'on $E$ con semigrupo
de transici\'on $\SG$ provisto de:
\begin{itemize}
\item[i)] $\mathbf{X}$ es una realizaci\'on  continua por la
derecha, \ref{Def.2.2.S}, de $\SG$.

\item[ii)] $\mathbf{X}$ satisface la condicion HD2,
\ref{Eq.2.2.S}, relativa a $\mathcal{G}_{t}$.

\item[iii)] $\mathcal{G}_{t}$ es aumentado y continuo por la
derecha.
\end{itemize}
\end{Def}

\begin{Lema}[Lema 4.2, Dai\cite{Dai}]\label{Lema4.2}
Sea $\left\{x_{n}\right\}\subset \mathbf{X}$ con
$|x_{n}|\rightarrow\infty$, conforme $n\rightarrow\infty$. Suponga
que
\[lim_{n\rightarrow\infty}\frac{1}{|x_{n}|}U\left(0\right)=\overline{U}\]
y
\[lim_{n\rightarrow\infty}\frac{1}{|x_{n}|}V\left(0\right)=\overline{V}.\]

Entonces, conforme $n\rightarrow\infty$, casi seguramente

\begin{equation}\label{E1.4.2}
\frac{1}{|x_{n}|}\Phi^{k}\left(\left[|x_{n}|t\right]\right)\rightarrow
P_{k}^{'}t\textrm{, u.o.c.,}
\end{equation}

\begin{equation}\label{E1.4.3}
\frac{1}{|x_{n}|}E^{x_{n}}_{k}\left(|x_{n}|t\right)\rightarrow
\alpha_{k}\left(t-\overline{U}_{k}\right)^{+}\textrm{, u.o.c.,}
\end{equation}

\begin{equation}\label{E1.4.4}
\frac{1}{|x_{n}|}S^{x_{n}}_{k}\left(|x_{n}|t\right)\rightarrow
\mu_{k}\left(t-\overline{V}_{k}\right)^{+}\textrm{, u.o.c.,}
\end{equation}

donde $\left[t\right]$ es la parte entera de $t$ y
$\mu_{k}=1/m_{k}=1/\esp\left[\eta_{k}\left(1\right)\right]$.
\end{Lema}

\begin{Lema}[Lema 4.3, Dai\cite{Dai}]\label{Lema.4.3}
Sea $\left\{x_{n}\right\}\subset \mathbf{X}$ con
$|x_{n}|\rightarrow\infty$, conforme $n\rightarrow\infty$. Suponga
que
\[lim_{n\rightarrow\infty}\frac{1}{|x_{n}|}U\left(0\right)=\overline{U}_{k}\]
y
\[lim_{n\rightarrow\infty}\frac{1}{|x_{n}|}V\left(0\right)=\overline{V}_{k}.\]
\begin{itemize}
\item[a)] Conforme $n\rightarrow\infty$ casi seguramente,
\[lim_{n\rightarrow\infty}\frac{1}{|x_{n}|}U^{x_{n}}_{k}\left(|x_{n}|t\right)=\left(\overline{U}_{k}-t\right)^{+}\textrm{, u.o.c.}\]
y
\[lim_{n\rightarrow\infty}\frac{1}{|x_{n}|}V^{x_{n}}_{k}\left(|x_{n}|t\right)=\left(\overline{V}_{k}-t\right)^{+}.\]

\item[b)] Para cada $t\geq0$ fijo,
\[\left\{\frac{1}{|x_{n}|}U^{x_{n}}_{k}\left(|x_{n}|t\right),|x_{n}|\geq1\right\}\]
y
\[\left\{\frac{1}{|x_{n}|}V^{x_{n}}_{k}\left(|x_{n}|t\right),|x_{n}|\geq1\right\}\]
\end{itemize}
son uniformemente convergentes.
\end{Lema}

$S_{l}^{x}\left(t\right)$ es el n\'umero total de servicios
completados de la clase $l$, si la clase $l$ est\'a dando $t$
unidades de tiempo de servicio. Sea $T_{l}^{x}\left(x\right)$ el
monto acumulado del tiempo de servicio que el servidor
$s\left(l\right)$ gasta en los usuarios de la clase $l$ al tiempo
$t$. Entonces $S_{l}^{x}\left(T_{l}^{x}\left(t\right)\right)$ es
el n\'umero total de servicios completados para la clase $l$ al
tiempo $t$. Una fracci\'on de estos usuarios,
$\Phi_{l}^{x}\left(S_{l}^{x}\left(T_{l}^{x}\left(t\right)\right)\right)$,
se convierte en usuarios de la clase $k$.\\

Entonces, dado lo anterior, se tiene la siguiente representaci\'on
para el proceso de la longitud de la cola:\\

\begin{equation}
Q_{k}^{x}\left(t\right)=_{k}^{x}\left(0\right)+E_{k}^{x}\left(t\right)+\sum_{l=1}^{K}\Phi_{k}^{l}\left(S_{l}^{x}\left(T_{l}^{x}\left(t\right)\right)\right)-S_{k}^{x}\left(T_{k}^{x}\left(t\right)\right)
\end{equation}
para $k=1,\ldots,K$. Para $i=1,\ldots,d$, sea
\[I_{i}^{x}\left(t\right)=t-\sum_{j\in C_{i}}T_{k}^{x}\left(t\right).\]

Entonces $I_{i}^{x}\left(t\right)$ es el monto acumulado del
tiempo que el servidor $i$ ha estado desocupado al tiempo $t$. Se
est\'a asumiendo que las disciplinas satisfacen la ley de
conservaci\'on del trabajo, es decir, el servidor $i$ est\'a en
pausa solamente cuando no hay usuarios en la estaci\'on $i$.
Entonces, se tiene que

\begin{equation}
\int_{0}^{\infty}\left(\sum_{k\in
C_{i}}Q_{k}^{x}\left(t\right)\right)dI_{i}^{x}\left(t\right)=0,
\end{equation}
para $i=1,\ldots,d$.\\

Hacer
\[T^{x}\left(t\right)=\left(T_{1}^{x}\left(t\right),\ldots,T_{K}^{x}\left(t\right)\right)^{'},\]
\[I^{x}\left(t\right)=\left(I_{1}^{x}\left(t\right),\ldots,I_{K}^{x}\left(t\right)\right)^{'}\]
y
\[S^{x}\left(T^{x}\left(t\right)\right)=\left(S_{1}^{x}\left(T_{1}^{x}\left(t\right)\right),\ldots,S_{K}^{x}\left(T_{K}^{x}\left(t\right)\right)\right)^{'}.\]

Para una disciplina que cumple con la ley de conservaci\'on del
trabajo, en forma vectorial, se tiene el siguiente conjunto de
ecuaciones

\begin{equation}\label{Eq.MF.1.3}
Q^{x}\left(t\right)=Q^{x}\left(0\right)+E^{x}\left(t\right)+\sum_{l=1}^{K}\Phi^{l}\left(S_{l}^{x}\left(T_{l}^{x}\left(t\right)\right)\right)-S^{x}\left(T^{x}\left(t\right)\right),\\
\end{equation}

\begin{equation}\label{Eq.MF.2.3}
Q^{x}\left(t\right)\geq0,\\
\end{equation}

\begin{equation}\label{Eq.MF.3.3}
T^{x}\left(0\right)=0,\textrm{ y }\overline{T}^{x}\left(t\right)\textrm{ es no decreciente},\\
\end{equation}

\begin{equation}\label{Eq.MF.4.3}
I^{x}\left(t\right)=et-CT^{x}\left(t\right)\textrm{ es no
decreciente}\\
\end{equation}

\begin{equation}\label{Eq.MF.5.3}
\int_{0}^{\infty}\left(CQ^{x}\left(t\right)\right)dI_{i}^{x}\left(t\right)=0,\\
\end{equation}

\begin{equation}\label{Eq.MF.6.3}
\textrm{Condiciones adicionales en
}\left(\overline{Q}^{x}\left(\cdot\right),\overline{T}^{x}\left(\cdot\right)\right)\textrm{
espec\'ificas de la disciplina de la cola,}
\end{equation}

donde $e$ es un vector de unos de dimensi\'on $d$, $C$ es la
matriz definida por
\[C_{ik}=\left\{\begin{array}{cc}
1,& S\left(k\right)=i,\\
0,& \textrm{ en otro caso}.\\
\end{array}\right.
\]
Es necesario enunciar el siguiente Teorema que se utilizar\'a para
el Teorema \ref{Tma.4.2.Dai}:
\begin{Teo}[Teorema 4.1, Dai \cite{Dai}]
Considere una disciplina que cumpla la ley de conservaci\'on del
trabajo, para casi todas las trayectorias muestrales $\omega$ y
cualquier sucesi\'on de estados iniciales
$\left\{x_{n}\right\}\subset \mathbf{X}$, con
$|x_{n}|\rightarrow\infty$, existe una subsucesi\'on
$\left\{x_{n_{j}}\right\}$ con $|x_{n_{j}}|\rightarrow\infty$ tal
que
\begin{equation}\label{Eq.4.15}
\frac{1}{|x_{n_{j}}|}\left(Q^{x_{n_{j}}}\left(0\right),U^{x_{n_{j}}}\left(0\right),V^{x_{n_{j}}}\left(0\right)\right)\rightarrow\left(\overline{Q}\left(0\right),\overline{U},\overline{V}\right),
\end{equation}

\begin{equation}\label{Eq.4.16}
\frac{1}{|x_{n_{j}}|}\left(Q^{x_{n_{j}}}\left(|x_{n_{j}}|t\right),T^{x_{n_{j}}}\left(|x_{n_{j}}|t\right)\right)\rightarrow\left(\overline{Q}\left(t\right),\overline{T}\left(t\right)\right)\textrm{
u.o.c.}
\end{equation}

Adem\'as,
$\left(\overline{Q}\left(t\right),\overline{T}\left(t\right)\right)$
satisface las siguientes ecuaciones:
\begin{equation}\label{Eq.MF.1.3a}
\overline{Q}\left(t\right)=Q\left(0\right)+\left(\alpha
t-\overline{U}\right)^{+}-\left(I-P\right)^{'}M^{-1}\left(\overline{T}\left(t\right)-\overline{V}\right)^{+},
\end{equation}

\begin{equation}\label{Eq.MF.2.3a}
\overline{Q}\left(t\right)\geq0,\\
\end{equation}

\begin{equation}\label{Eq.MF.3.3a}
\overline{T}\left(t\right)\textrm{ es no decreciente y comienza en cero},\\
\end{equation}

\begin{equation}\label{Eq.MF.4.3a}
\overline{I}\left(t\right)=et-C\overline{T}\left(t\right)\textrm{
es no decreciente,}\\
\end{equation}

\begin{equation}\label{Eq.MF.5.3a}
\int_{0}^{\infty}\left(C\overline{Q}\left(t\right)\right)d\overline{I}\left(t\right)=0,\\
\end{equation}

\begin{equation}\label{Eq.MF.6.3a}
\textrm{Condiciones adicionales en
}\left(\overline{Q}\left(\cdot\right),\overline{T}\left(\cdot\right)\right)\textrm{
especficas de la disciplina de la cola,}
\end{equation}
\end{Teo}


Propiedades importantes para el modelo de flujo retrasado:

\begin{Prop}
 Sea $\left(\overline{Q},\overline{T},\overline{T}^{0}\right)$ un flujo l\'imite de \ref{Eq.4.4} y suponga que cuando $x\rightarrow\infty$ a lo largo de
una subsucesi\'on
\[\left(\frac{1}{|x|}Q_{k}^{x}\left(0\right),\frac{1}{|x|}A_{k}^{x}\left(0\right),\frac{1}{|x|}B_{k}^{x}\left(0\right),\frac{1}{|x|}B_{k}^{x,0}\left(0\right)\right)\rightarrow\left(\overline{Q}_{k}\left(0\right),0,0,0\right)\]
para $k=1,\ldots,K$. EL flujo l\'imite tiene las siguientes
propiedades, donde las propiedades de la derivada se cumplen donde
la derivada exista:
\begin{itemize}
 \item[i)] Los vectores de tiempo ocupado $\overline{T}\left(t\right)$ y $\overline{T}^{0}\left(t\right)$ son crecientes y continuas con
$\overline{T}\left(0\right)=\overline{T}^{0}\left(0\right)=0$.
\item[ii)] Para todo $t\geq0$
\[\sum_{k=1}^{K}\left[\overline{T}_{k}\left(t\right)+\overline{T}_{k}^{0}\left(t\right)\right]=t\]
\item[iii)] Para todo $1\leq k\leq K$
\[\overline{Q}_{k}\left(t\right)=\overline{Q}_{k}\left(0\right)+\alpha_{k}t-\mu_{k}\overline{T}_{k}\left(t\right)\]
\item[iv)]  Para todo $1\leq k\leq K$
\[\dot{{\overline{T}}}_{k}\left(t\right)=\beta_{k}\] para $\overline{Q}_{k}\left(t\right)=0$.
\item[v)] Para todo $k,j$
\[\mu_{k}^{0}\overline{T}_{k}^{0}\left(t\right)=\mu_{j}^{0}\overline{T}_{j}^{0}\left(t\right)\]
\item[vi)]  Para todo $1\leq k\leq K$
\[\mu_{k}\dot{{\overline{T}}}_{k}\left(t\right)=l_{k}\mu_{k}^{0}\dot{{\overline{T}}}_{k}^{0}\left(t\right)\] para $\overline{Q}_{k}\left(t\right)>0$.
\end{itemize}
\end{Prop}

\begin{Lema}[Lema 3.1 \cite{Chen}]\label{Lema3.1}
Si el modelo de flujo es estable, definido por las ecuaciones
(3.8)-(3.13), entonces el modelo de flujo retrasado tambi\'en es
estable.
\end{Lema}

\begin{Teo}[Teorema 5.1 \cite{Chen}]\label{Tma.5.1.Chen}
La red de colas es estable si existe una constante $t_{0}$ que
depende de $\left(\alpha,\mu,T,U\right)$ y $V$ que satisfagan las
ecuaciones (5.1)-(5.5), $Z\left(t\right)=0$, para toda $t\geq
t_{0}$.
\end{Teo}



\begin{Lema}[Lema 5.2 \cite{Gut}]\label{Lema.5.2.Gut}
Sea $\left\{\xi\left(k\right):k\in\ent\right\}$ sucesi\'on de
variables aleatorias i.i.d. con valores en
$\left(0,\infty\right)$, y sea $E\left(t\right)$ el proceso de
conteo
\[E\left(t\right)=max\left\{n\geq1:\xi\left(1\right)+\cdots+\xi\left(n-1\right)\leq t\right\}.\]
Si $E\left[\xi\left(1\right)\right]<\infty$, entonces para
cualquier entero $r\geq1$
\begin{equation}
lim_{t\rightarrow\infty}\esp\left[\left(\frac{E\left(t\right)}{t}\right)^{r}\right]=\left(\frac{1}{E\left[\xi_{1}\right]}\right)^{r}
\end{equation}
de aqu\'i, bajo estas condiciones
\begin{itemize}
\item[a)] Para cualquier $t>0$,
$sup_{t\geq\delta}\esp\left[\left(\frac{E\left(t\right)}{t}\right)^{r}\right]$

\item[b)] Las variables aleatorias
$\left\{\left(\frac{E\left(t\right)}{t}\right)^{r}:t\geq1\right\}$
son uniformemente integrables.
\end{itemize}
\end{Lema}

\begin{Teo}[Teorema 5.1: Ley Fuerte para Procesos de Conteo
\cite{Gut}]\label{Tma.5.1.Gut} Sea
$0<\mu<\esp\left(X_{1}\right]\leq\infty$. entonces

\begin{itemize}
\item[a)] $\frac{N\left(t\right)}{t}\rightarrow\frac{1}{\mu}$
a.s., cuando $t\rightarrow\infty$.


\item[b)]$\esp\left[\frac{N\left(t\right)}{t}\right]^{r}\rightarrow\frac{1}{\mu^{r}}$,
cuando $t\rightarrow\infty$ para todo $r>0$..
\end{itemize}
\end{Teo}


\begin{Prop}[Proposici\'on 5.1 \cite{DaiSean}]\label{Prop.5.1}
Suponga que los supuestos (A1) y (A2) se cumplen, adem\'as suponga
que el modelo de flujo es estable. Entonces existe $t_{0}>0$ tal
que
\begin{equation}\label{Eq.Prop.5.1}
lim_{|x|\rightarrow\infty}\frac{1}{|x|^{p+1}}\esp_{x}\left[|X\left(t_{0}|x|\right)|^{p+1}\right]=0.
\end{equation}

\end{Prop}


\begin{Prop}[Proposici\'on 5.3 \cite{DaiSean}]
Sea $X$ proceso de estados para la red de colas, y suponga que se
cumplen los supuestos (A1) y (A2), entonces para alguna constante
positiva $C_{p+1}<\infty$, $\delta>0$ y un conjunto compacto
$C\subset X$.

\begin{equation}\label{Eq.5.4}
\esp_{x}\left[\int_{0}^{\tau_{C}\left(\delta\right)}\left(1+|X\left(t\right)|^{p}\right)dt\right]\leq
C_{p+1}\left(1+|x|^{p+1}\right)
\end{equation}
\end{Prop}

\begin{Prop}[Proposici\'on 5.4 \cite{DaiSean}]
Sea $X$ un proceso de Markov Borel Derecho en $X$, sea
$f:X\leftarrow\rea_{+}$ y defina para alguna $\delta>0$, y un
conjunto cerrado $C\subset X$
\[V\left(x\right):=\esp_{x}\left[\int_{0}^{\tau_{C}\left(\delta\right)}f\left(X\left(t\right)\right)dt\right]\]
para $x\in X$. Si $V$ es finito en todas partes y uniformemente
acotada en $C$, entonces existe $k<\infty$ tal que
\begin{equation}\label{Eq.5.11}
\frac{1}{t}\esp_{x}\left[V\left(x\right)\right]+\frac{1}{t}\int_{0}^{t}\esp_{x}\left[f\left(X\left(s\right)\right)ds\right]\leq\frac{1}{t}V\left(x\right)+k,
\end{equation}
para $x\in X$ y $t>0$.
\end{Prop}


\begin{Teo}[Teorema 5.5 \cite{DaiSean}]
Suponga que se cumplen (A1) y (A2), adem\'as suponga que el modelo
de flujo es estable. Entonces existe una constante $k_{p}<\infty$
tal que
\begin{equation}\label{Eq.5.13}
\frac{1}{t}\int_{0}^{t}\esp_{x}\left[|Q\left(s\right)|^{p}\right]ds\leq
k_{p}\left\{\frac{1}{t}|x|^{p+1}+1\right\}
\end{equation}
para $t\geq0$, $x\in X$. En particular para cada condici\'on
inicial
\begin{equation}\label{Eq.5.14}
Limsup_{t\rightarrow\infty}\frac{1}{t}\int_{0}^{t}\esp_{x}\left[|Q\left(s\right)|^{p}\right]ds\leq
k_{p}
\end{equation}
\end{Teo}

\begin{Teo}[Teorema 6.2 \cite{DaiSean}]\label{Tma.6.2}
Suponga que se cumplen los supuestos (A1)-(A3) y que el modelo de
flujo es estable, entonces se tiene que
\[\parallel P^{t}\left(c,\cdot\right)-\pi\left(\cdot\right)\parallel_{f_{p}}\rightarrow0\]
para $t\rightarrow\infty$ y $x\in X$. En particular para cada
condici\'on inicial
\[lim_{t\rightarrow\infty}\esp_{x}\left[\left|Q_{t}\right|^{p}\right]=\esp_{\pi}\left[\left|Q_{0}\right|^{p}\right]<\infty\]
\end{Teo}


\begin{Teo}[Teorema 6.3 \cite{DaiSean}]\label{Tma.6.3}
Suponga que se cumplen los supuestos (A1)-(A3) y que el modelo de
flujo es estable, entonces con
$f\left(x\right)=f_{1}\left(x\right)$, se tiene que
\[lim_{t\rightarrow\infty}t^{(p-1)\left|P^{t}\left(c,\cdot\right)-\pi\left(\cdot\right)\right|_{f}=0},\]
para $x\in X$. En particular, para cada condici\'on inicial
\[lim_{t\rightarrow\infty}t^{(p-1)}\left|\esp_{x}\left[Q_{t}\right]-\esp_{\pi}\left[Q_{0}\right]\right|=0.\]
\end{Teo}



\begin{Prop}[Proposici\'on 5.1, Dai y Meyn \cite{DaiSean}]\label{Prop.5.1.DaiSean}
Suponga que los supuestos A1) y A2) son ciertos y que el modelo de
flujo es estable. Entonces existe $t_{0}>0$ tal que
\begin{equation}
lim_{|x|\rightarrow\infty}\frac{1}{|x|^{p+1}}\esp_{x}\left[|X\left(t_{0}|x|\right)|^{p+1}\right]=0
\end{equation}
\end{Prop}

\begin{Lemma}[Lema 5.2, Dai y Meyn, \cite{DaiSean}]\label{Lema.5.2.DaiSean}
 Sea $\left\{\zeta\left(k\right):k\in \mathbb{z}\right\}$ una sucesi\'on independiente e id\'enticamente distribuida que toma valores en $\left(0,\infty\right)$,
y sea
$E\left(t\right)=max\left(n\geq1:\zeta\left(1\right)+\cdots+\zeta\left(n-1\right)\leq
t\right)$. Si $\esp\left[\zeta\left(1\right)\right]<\infty$,
entonces para cualquier entero $r\geq1$
\begin{equation}
 lim_{t\rightarrow\infty}\esp\left[\left(\frac{E\left(t\right)}{t}\right)^{r}\right]=\left(\frac{1}{\esp\left[\zeta_{1}\right]}\right)^{r}.
\end{equation}
Luego, bajo estas condiciones:
\begin{itemize}
 \item[a)] para cualquier $\delta>0$, $\sup_{t\geq\delta}\esp\left[\left(\frac{E\left(t\right)}{t}\right)^{r}\right]<\infty$
\item[b)] las variables aleatorias
$\left\{\left(\frac{E\left(t\right)}{t}\right)^{r}:t\geq1\right\}$
son uniformemente integrables.
\end{itemize}
\end{Lemma}

\begin{Teo}[Teorema 5.5, Dai y Meyn \cite{DaiSean}]\label{Tma.5.5.DaiSean}
Suponga que los supuestos A1) y A2) se cumplen y que el modelo de
flujo es estable. Entonces existe una constante $\kappa_{p}$ tal
que
\begin{equation}
\frac{1}{t}\int_{0}^{t}\esp_{x}\left[|Q\left(s\right)|^{p}\right]ds\leq\kappa_{p}\left\{\frac{1}{t}|x|^{p+1}+1\right\}
\end{equation}
para $t>0$ y $x\in X$. En particular, para cada condici\'on
inicial
\begin{eqnarray*}
\limsup_{t\rightarrow\infty}\frac{1}{t}\int_{0}^{t}\esp_{x}\left[|Q\left(s\right)|^{p}\right]ds\leq\kappa_{p}.
\end{eqnarray*}
\end{Teo}

\begin{Teo}[Teorema 6.2, Dai y Meyn \cite{DaiSean}]\label{Tma.6.2.DaiSean}
Suponga que se cumplen los supuestos A1), A2) y A3) y que el
modelo de flujo es estable. Entonces se tiene que
\begin{equation}
\left\|P^{t}\left(x,\cdot\right)-\pi\left(\cdot\right)\right\|_{f_{p}}\textrm{,
}t\rightarrow\infty,x\in X.
\end{equation}
En particular para cada condici\'on inicial
\begin{eqnarray*}
\lim_{t\rightarrow\infty}\esp_{x}\left[|Q\left(t\right)|^{p}\right]=\esp_{\pi}\left[|Q\left(0\right)|^{p}\right]\leq\kappa_{r}
\end{eqnarray*}
\end{Teo}
\begin{Teo}[Teorema 6.3, Dai y Meyn \cite{DaiSean}]\label{Tma.6.3.DaiSean}
Suponga que se cumplen los supuestos A1), A2) y A3) y que el
modelo de flujo es estable. Entonces con
$f\left(x\right)=f_{1}\left(x\right)$ se tiene
\begin{equation}
\lim_{t\rightarrow\infty}t^{p-1}\left\|P^{t}\left(x,\cdot\right)-\pi\left(\cdot\right)\right\|_{f}=0.
\end{equation}
En particular para cada condici\'on inicial
\begin{eqnarray*}
\lim_{t\rightarrow\infty}t^{p-1}|\esp_{x}\left[Q\left(t\right)\right]-\esp_{\pi}\left[Q\left(0\right)\right]|=0.
\end{eqnarray*}
\end{Teo}

\begin{Teo}[Teorema 6.4, Dai y Meyn, \cite{DaiSean}]\label{Tma.6.4.DaiSean}
Suponga que se cumplen los supuestos A1), A2) y A3) y que el
modelo de flujo es estable. Sea $\nu$ cualquier distribuci\'on de
probabilidad en $\left(X,\mathcal{B}_{X}\right)$, y $\pi$ la
distribuci\'on estacionaria de $X$.
\begin{itemize}
\item[i)] Para cualquier $f:X\leftarrow\rea_{+}$
\begin{equation}
\lim_{t\rightarrow\infty}\frac{1}{t}\int_{o}^{t}f\left(X\left(s\right)\right)ds=\pi\left(f\right):=\int
f\left(x\right)\pi\left(dx\right)
\end{equation}
$\prob$-c.s.

\item[ii)] Para cualquier $f:X\leftarrow\rea_{+}$ con
$\pi\left(|f|\right)<\infty$, la ecuaci\'on anterior se cumple.
\end{itemize}
\end{Teo}

\begin{Teo}[Teorema 2.2, Down \cite{Down}]\label{Tma2.2.Down}
Suponga que el fluido modelo es inestable en el sentido de que
para alguna $\epsilon_{0},c_{0}\geq0$,
\begin{equation}\label{Eq.Inestability}
|Q\left(T\right)|\geq\epsilon_{0}T-c_{0}\textrm{,   }T\geq0,
\end{equation}
para cualquier condici\'on inicial $Q\left(0\right)$, con
$|Q\left(0\right)|=1$. Entonces para cualquier $0<q\leq1$, existe
$B<0$ tal que para cualquier $|x|\geq B$,
\begin{equation}
\prob_{x}\left\{\mathbb{X}\rightarrow\infty\right\}\geq q.
\end{equation}
\end{Teo}



\begin{Def}
Sea $X$ un conjunto y $\mathcal{F}$ una $\sigma$-\'algebra de
subconjuntos de $X$, la pareja $\left(X,\mathcal{F}\right)$ es
llamado espacio medible. Un subconjunto $A$ de $X$ es llamado
medible, o medible con respecto a $\mathcal{F}$, si
$A\in\mathcal{F}$.
\end{Def}

\begin{Def}
Sea $\left(X,\mathcal{F},\mu\right)$ espacio de medida. Se dice
que la medida $\mu$ es $\sigma$-finita si se puede escribir
$X=\bigcup_{n\geq1}X_{n}$ con $X_{n}\in\mathcal{F}$ y
$\mu\left(X_{n}\right)<\infty$.
\end{Def}

\begin{Def}\label{Cto.Borel}
Sea $X$ el conjunto de los n\'umeros reales $\rea$. El \'algebra
de Borel es la $\sigma$-\'algebra $B$ generada por los intervalos
abiertos $\left(a,b\right)\in\rea$. Cualquier conjunto en $B$ es
llamado {\em Conjunto de Borel}.
\end{Def}

\begin{Def}\label{Funcion.Medible}
Una funci\'on $f:X\rightarrow\rea$, es medible si para cualquier
n\'umero real $\alpha$ el conjunto
\[\left\{x\in X:f\left(x\right)>\alpha\right\}\]
pertenece a $\mathcal{F}$. Equivalentemente, se dice que $f$ es
medible si
\[f^{-1}\left(\left(\alpha,\infty\right)\right)=\left\{x\in X:f\left(x\right)>\alpha\right\}\in\mathcal{F}.\]
\end{Def}


\begin{Def}\label{Def.Cilindros}
Sean $\left(\Omega_{i},\mathcal{F}_{i}\right)$, $i=1,2,\ldots,$
espacios medibles y $\Omega=\prod_{i=1}^{\infty}\Omega_{i}$ el
conjunto de todas las sucesiones
$\left(\omega_{1},\omega_{2},\ldots,\right)$ tales que
$\omega_{i}\in\Omega_{i}$, $i=1,2,\ldots,$. Si
$B^{n}\subset\prod_{i=1}^{\infty}\Omega_{i}$, definimos
$B_{n}=\left\{\omega\in\Omega:\left(\omega_{1},\omega_{2},\ldots,\omega_{n}\right)\in
B^{n}\right\}$. Al conjunto $B_{n}$ se le llama {\em cilindro} con
base $B^{n}$, el cilindro es llamado medible si
$B^{n}\in\prod_{i=1}^{\infty}\mathcal{F}_{i}$.
\end{Def}


\begin{Def}\label{Def.Proc.Adaptado}[TSP, Ash \cite{RBA}]
Sea $X\left(t\right),t\geq0$ proceso estoc\'astico, el proceso es
adaptado a la familia de $\sigma$-\'algebras $\mathcal{F}_{t}$,
para $t\geq0$, si para $s<t$ implica que
$\mathcal{F}_{s}\subset\mathcal{F}_{t}$, y $X\left(t\right)$ es
$\mathcal{F}_{t}$-medible para cada $t$. Si no se especifica
$\mathcal{F}_{t}$ entonces se toma $\mathcal{F}_{t}$ como
$\mathcal{F}\left(X\left(s\right),s\leq t\right)$, la m\'as
peque\~na $\sigma$-\'algebra de subconjuntos de $\Omega$ que hace
que cada $X\left(s\right)$, con $s\leq t$ sea Borel medible.
\end{Def}


\begin{Def}\label{Def.Tiempo.Paro}[TSP, Ash \cite{RBA}]
Sea $\left\{\mathcal{F}\left(t\right),t\geq0\right\}$ familia
creciente de sub $\sigma$-\'algebras. es decir,
$\mathcal{F}\left(s\right)\subset\mathcal{F}\left(t\right)$ para
$s\leq t$. Un tiempo de paro para $\mathcal{F}\left(t\right)$ es
una funci\'on $T:\Omega\rightarrow\left[0,\infty\right]$ tal que
$\left\{T\leq t\right\}\in\mathcal{F}\left(t\right)$ para cada
$t\geq0$. Un tiempo de paro para el proceso estoc\'astico
$X\left(t\right),t\geq0$ es un tiempo de paro para las
$\sigma$-\'algebras
$\mathcal{F}\left(t\right)=\mathcal{F}\left(X\left(s\right)\right)$.
\end{Def}

\begin{Def}
Sea $X\left(t\right),t\geq0$ proceso estoc\'astico, con
$\left(S,\chi\right)$ espacio de estados. Se dice que el proceso
es adaptado a $\left\{\mathcal{F}\left(t\right)\right\}$, es
decir, si para cualquier $s,t\in I$, $I$ conjunto de \'indices,
$s<t$, se tiene que
$\mathcal{F}\left(s\right)\subset\mathcal{F}\left(t\right)$ y
$X\left(t\right)$ es $\mathcal{F}\left(t\right)$-medible,
\end{Def}

\begin{Def}
Sea $X\left(t\right),t\geq0$ proceso estoc\'astico, se dice que es
un Proceso de Markov relativo a $\mathcal{F}\left(t\right)$ o que
$\left\{X\left(t\right),\mathcal{F}\left(t\right)\right\}$ es de
Markov si y s\'olo si para cualquier conjunto $B\in\chi$,  y
$s,t\in I$, $s<t$ se cumple que
\begin{equation}\label{Prop.Markov}
P\left\{X\left(t\right)\in
B|\mathcal{F}\left(s\right)\right\}=P\left\{X\left(t\right)\in
B|X\left(s\right)\right\}.
\end{equation}
\end{Def}
\begin{Note}
Si se dice que $\left\{X\left(t\right)\right\}$ es un Proceso de
Markov sin mencionar $\mathcal{F}\left(t\right)$, se asumir\'a que
\begin{eqnarray*}
\mathcal{F}\left(t\right)=\mathcal{F}_{0}\left(t\right)=\mathcal{F}\left(X\left(r\right),r\leq
t\right),
\end{eqnarray*}
entonces la ecuaci\'on (\ref{Prop.Markov}) se puede escribir como
\begin{equation}
P\left\{X\left(t\right)\in B|X\left(r\right),r\leq s\right\} =
P\left\{X\left(t\right)\in B|X\left(s\right)\right\}
\end{equation}
\end{Note}
%_______________________________________________________________
\subsection{Procesos de Estados de Markov}
%_______________________________________________________________

\begin{Teo}
Sea $\left(X_{n},\mathcal{F}_{n},n=0,1,\ldots,\right\}$ Proceso de
Markov con espacio de estados $\left(S_{0},\chi_{0}\right)$
generado por una distribuici\'on inicial $P_{o}$ y probabilidad de
transici\'on $p_{mn}$, para $m,n=0,1,\ldots,$ $m<n$, que por
notaci\'on se escribir\'a como $p\left(m,n,x,B\right)\rightarrow
p_{mn}\left(x,B\right)$. Sea $S$ tiempo de paro relativo a la
$\sigma$-\'algebra $\mathcal{F}_{n}$. Sea $T$ funci\'on medible,
$T:\Omega\rightarrow\left\{0,1,\ldots,\right\}$. Sup\'ongase que
$T\geq S$, entonces $T$ es tiempo de paro. Si $B\in\chi_{0}$,
entonces
\begin{equation}\label{Prop.Fuerte.Markov}
P\left\{X\left(T\right)\in
B,T<\infty|\mathcal{F}\left(S\right)\right\} =
p\left(S,T,X\left(s\right),B\right)
\end{equation}
en $\left\{T<\infty\right\}$.
\end{Teo}


Sea $K$ conjunto numerable y sea $d:K\rightarrow\nat$ funci\'on.
Para $v\in K$, $M_{v}$ es un conjunto abierto de
$\rea^{d\left(v\right)}$. Entonces \[E=\bigcup_{v\in
K}M_{v}=\left\{\left(v,\zeta\right):v\in K,\zeta\in
M_{v}\right\}.\]

Sea $\mathcal{E}$ la clase de conjuntos medibles en $E$:
\[\mathcal{E}=\left\{\bigcup_{v\in K}A_{v}:A_{v}\in \mathcal{M}_{v}\right\}.\]

donde $\mathcal{M}$ son los conjuntos de Borel de $M_{v}$.
Entonces $\left(E,\mathcal{E}\right)$ es un espacio de Borel. El
estado del proceso se denotar\'a por
$\mathbf{x}_{t}=\left(v_{t},\zeta_{t}\right)$. La distribuci\'on
de $\left(\mathbf{x}_{t}\right)$ est\'a determinada por por los
siguientes objetos:

\begin{itemize}
\item[i)] Los campos vectoriales $\left(\mathcal{H}_{v},v\in
K\right)$. \item[ii)] Una funci\'on medible $\lambda:E\rightarrow
\rea_{+}$. \item[iii)] Una medida de transici\'on
$Q:\mathcal{E}\times\left(E\cup\Gamma^{*}\right)\rightarrow\left[0,1\right]$
donde
\begin{equation}
\Gamma^{*}=\bigcup_{v\in K}\partial^{*}M_{v}.
\end{equation}
y
\begin{equation}
\partial^{*}M_{v}=\left\{z\in\partial M_{v}:\mathbf{\mathbf{\phi}_{v}\left(t,\zeta\right)=\mathbf{z}}\textrm{ para alguna }\left(t,\zeta\right)\in\rea_{+}\times M_{v}\right\}.
\end{equation}
$\partial M_{v}$ denota  la frontera de $M_{v}$.
\end{itemize}

El campo vectorial $\left(\mathcal{H}_{v},v\in K\right)$ se supone
tal que para cada $\mathbf{z}\in M_{v}$ existe una \'unica curva
integral $\mathbf{\phi}_{v}\left(t,\zeta\right)$ que satisface la
ecuaci\'on

\begin{equation}
\frac{d}{dt}f\left(\zeta_{t}\right)=\mathcal{H}f\left(\zeta_{t}\right),
\end{equation}
con $\zeta_{0}=\mathbf{z}$, para cualquier funci\'on suave
$f:\rea^{d}\rightarrow\rea$ y $\mathcal{H}$ denota el operador
diferencial de primer orden, con $\mathcal{H}=\mathcal{H}_{v}$ y
$\zeta_{t}=\mathbf{\phi}\left(t,\mathbf{z}\right)$. Adem\'as se
supone que $\mathcal{H}_{v}$ es conservativo, es decir, las curvas
integrales est\'an definidas para todo $t>0$.

Para $\mathbf{x}=\left(v,\zeta\right)\in E$ se denota
\[t^{*}\mathbf{x}=inf\left\{t>0:\mathbf{\phi}_{v}\left(t,\zeta\right)\in\partial^{*}M_{v}\right\}\]

En lo que respecta a la funci\'on $\lambda$, se supondr\'a que
para cada $\left(v,\zeta\right)\in E$ existe un $\epsilon>0$ tal
que la funci\'on
$s\rightarrow\lambda\left(v,\phi_{v}\left(s,\zeta\right)\right)\in
E$ es integrable para $s\in\left[0,\epsilon\right)$. La medida de
transici\'on $Q\left(A;\mathbf{x}\right)$ es una funci\'on medible
de $\mathbf{x}$ para cada $A\in\mathcal{E}$, definida para
$\mathbf{x}\in E\cup\Gamma^{*}$ y es una medida de probabilidad en
$\left(E,\mathcal{E}\right)$ para cada $\mathbf{x}\in E$.

El movimiento del proceso $\left(\mathbf{x}_{t}\right)$ comenzando
en $\mathbf{x}=\left(n,\mathbf{z}\right)\in E$ se puede construir
de la siguiente manera, def\'inase la funci\'on $F$ por

\begin{equation}
F\left(t\right)=\left\{\begin{array}{ll}\\
exp\left(-\int_{0}^{t}\lambda\left(n,\phi_{n}\left(s,\mathbf{z}\right)\right)ds\right), & t<t^{*}\left(\mathbf{x}\right),\\
0, & t\geq t^{*}\left(\mathbf{x}\right)
\end{array}\right.
\end{equation}

Sea $T_{1}$ una variable aleatoria tal que
$\prob\left[T_{1}>t\right]=F\left(t\right)$, ahora sea la variable
aleatoria $\left(N,Z\right)$ con distribuici\'on
$Q\left(\cdot;\phi_{n}\left(T_{1},\mathbf{z}\right)\right)$. La
trayectoria de $\left(\mathbf{x}_{t}\right)$ para $t\leq T_{1}$ es
\begin{eqnarray*}
\mathbf{x}_{t}=\left(v_{t},\zeta_{t}\right)=\left\{\begin{array}{ll}
\left(n,\phi_{n}\left(t,\mathbf{z}\right)\right), & t<T_{1},\\
\left(N,\mathbf{Z}\right), & t=t_{1}.
\end{array}\right.
\end{eqnarray*}

Comenzando en $\mathbf{x}_{T_{1}}$ se selecciona el siguiente
tiempo de intersalto $T_{2}-T_{1}$ lugar del post-salto
$\mathbf{x}_{T_{2}}$ de manera similar y as\'i sucesivamente. Este
procedimiento nos da una trayectoria determinista por partes
$\mathbf{x}_{t}$ con tiempos de salto $T_{1},T_{2},\ldots$. Bajo
las condiciones enunciadas para $\lambda,T_{1}>0$  y
$T_{1}-T_{2}>0$ para cada $i$, con probabilidad 1. Se supone que
se cumple la siguiente condici\'on.

\begin{Sup}[Supuesto 3.1, Davis \cite{Davis}]\label{Sup3.1.Davis}
Sea $N_{t}:=\sum_{t}\indora_{\left(t\geq t\right)}$ el n\'umero de
saltos en $\left[0,t\right]$. Entonces
\begin{equation}
\esp\left[N_{t}\right]<\infty\textrm{ para toda }t.
\end{equation}
\end{Sup}

es un proceso de Markov, m\'as a\'un, es un Proceso Fuerte de
Markov, es decir, la Propiedad Fuerte de Markov\footnote{Revisar
p\'agina 362, y 364 de Davis \cite{Davis}.} se cumple para
cualquier tiempo de paro.
%_________________________________________________________________________
%\renewcommand{\refname}{PROCESOS ESTOC\'ASTICOS}
%\renewcommand{\appendixname}{PROCESOS ESTOC\'ASTICOS}
%\renewcommand{\appendixtocname}{PROCESOS ESTOC\'ASTICOS}
%\renewcommand{\appendixpagename}{PROCESOS ESTOC\'ASTICOS}
%\appendix
%\clearpage % o \cleardoublepage
%\addappheadtotoc
%\appendixpage
%_________________________________________________________________________
\subsection{Teor\'ia General de Procesos Estoc\'asticos}
%_________________________________________________________________________
En esta secci\'on se har\'an las siguientes consideraciones: $E$
es un espacio m\'etrico separable y la m\'etrica $d$ es compatible
con la topolog\'ia.

\begin{Def}
Una medida finita, $\lambda$ en la $\sigma$-\'algebra de Borel de
un espacio metrizable $E$ se dice cerrada si
\begin{equation}\label{Eq.A2.3}
\lambda\left(E\right)=sup\left\{\lambda\left(K\right):K\textrm{ es
compacto en }E\right\}.
\end{equation}
\end{Def}

\begin{Def}
$E$ es un espacio de Rad\'on si cada medida finita en
$\left(E,\mathcal{B}\left(E\right)\right)$ es regular interior o cerrada,
{\em tight}.
\end{Def}


El siguiente teorema nos permite tener una mejor caracterizaci\'on de los espacios de Rad\'on:
\begin{Teo}\label{Tma.A2.2}
Sea $E$ espacio separable metrizable. Entonces $E$ es de Rad\'on
si y s\'olo s\'i cada medida finita en
$\left(E,\mathcal{B}\left(E\right)\right)$ es cerrada.
\end{Teo}

%_________________________________________________________________________________________
\subsection{Propiedades de Markov}
%_________________________________________________________________________________________

Sea $E$ espacio de estados, tal que $E$ es un espacio de Rad\'on, $\mathcal{B}\left(E\right)$ $\sigma$-\'algebra de Borel en $E$, que se denotar\'a por $\mathcal{E}$.

Sea $\left(X,\mathcal{G},\prob\right)$ espacio de probabilidad,
$I\subset\rea$ conjunto de índices. Sea $\mathcal{F}_{\leq t}$ la
$\sigma$-\'algebra natural definida como
$\sigma\left\{f\left(X_{r}\right):r\in I, r\leq
t,f\in\mathcal{E}\right\}$. Se considerar\'a una
$\sigma$-\'algebra m\'as general\footnote{qu\'e se quiere decir
con el t\'ermino: m\'as general?}, $ \left(\mathcal{G}_{t}\right)$
tal que $\left(X_{t}\right)$ sea $\mathcal{E}$-adaptado.

\begin{Def}
Una familia $\left(P_{s,t}\right)$ de kernels de Markov en $\left(E,\mathcal{E}\right)$ indexada por pares $s,t\in I$, con $s\leq t$ es una funci\'on de transici\'on en $\ER$, si  para todo $r\leq s< t$ en $I$ y todo $x\in E$, $B\in\mathcal{E}$
\begin{equation}\label{Eq.Kernels}
P_{r,t}\left(x,B\right)=\int_{E}P_{r,s}\left(x,dy\right)P_{s,t}\left(y,B\right)\footnote{Ecuaci\'on de Chapman-Kolmogorov}.
\end{equation}
\end{Def}

Se dice que la funci\'on de transici\'on $\KM$ en $\ER$ es la funci\'on de transici\'on para un proceso $\PE$  con valores en $E$ y que satisface la propiedad de Markov\footnote{\begin{equation}\label{Eq.1.4.S}
\prob\left\{H|\mathcal{G}_{t}\right\}=\prob\left\{H|X_{t}\right\}\textrm{ }H\in p\mathcal{F}_{\geq t}.
\end{equation}} (\ref{Eq.1.4.S}) relativa a $\left(\mathcal{G}_{t}\right)$ si

\begin{equation}\label{Eq.1.6.S}
\prob\left\{f\left(X_{t}\right)|\mathcal{G}_{s}\right\}=P_{s,t}f\left(X_{t}\right)\textrm{ }s\leq t\in I,\textrm{ }f\in b\mathcal{E}.
\end{equation}

\begin{Def}
Una familia $\left(P_{t}\right)_{t\geq0}$ de kernels de Markov en $\ER$ es llamada {\em Semigrupo de Transici\'on de Markov} o {\em Semigrupo de Transici\'on} si
\[P_{t+s}f\left(x\right)=P_{t}\left(P_{s}f\right)\left(x\right),\textrm{ }t,s\geq0,\textrm{ }x\in E\textrm{ }f\in b\mathcal{E}\footnote{Definir los t\'ermino $b\mathcal{E}$ y $p\mathcal{E}$}.\]
\end{Def}
\begin{Note}
Si la funci\'on de transici\'on $\KM$ es llamada homog\'enea si $P_{s,t}=P_{t-s}$.
\end{Note}

Un proceso de Markov que satisface la ecuaci\'on (\ref{Eq.1.6.S}) con funci\'on de transici\'on homog\'enea $\left(P_{t}\right)$ tiene la propiedad caracter\'istica
\begin{equation}\label{Eq.1.8.S}
\prob\left\{f\left(X_{t+s}\right)|\mathcal{G}_{t}\right\}=P_{s}f\left(X_{t}\right)\textrm{ }t,s\geq0,\textrm{ }f\in b\mathcal{E}.
\end{equation}
La ecuaci\'on anterior es la {\em Propiedad Simple de Markov} de $X$ relativa a $\left(P_{t}\right)$.

En este sentido el proceso $\PE$ cumple con la propiedad de Markov (\ref{Eq.1.8.S}) relativa a $\left(\Omega,\mathcal{G},\mathcal{G}_{t},\prob\right)$ con semigrupo de transici\'on $\left(P_{t}\right)$.
%_________________________________________________________________________________________
\subsection{Primer Condici\'on de Regularidad}
%_________________________________________________________________________________________
%\newcommand{\EM}{\left(\Omega,\mathcal{G},\prob\right)}
%\newcommand{\E4}{\left(\Omega,\mathcal{G},\mathcal{G}_{t},\prob\right)}
\begin{Def}
Un proceso estoc\'astico $\PE$ definido en
$\left(\Omega,\mathcal{G},\prob\right)$ con valores en el espacio
topol\'ogico $E$ es continuo por la derecha si cada trayectoria
muestral $t\rightarrow X_{t}\left(w\right)$ es un mapeo continuo
por la derecha de $I$ en $E$.
\end{Def}

\begin{Def}[HD1]\label{Eq.2.1.S}
Un semigrupo de Markov $\left(P_{t}\right)$ en un espacio de
Rad\'on $E$ se dice que satisface la condici\'on {\em HD1} si,
dada una medida de probabilidad $\mu$ en $E$, existe una
$\sigma$-\'algebra $\mathcal{E^{*}}$ con
$\mathcal{E}\subset\mathcal{E}^{*}$ y
$P_{t}\left(b\mathcal{E}^{*}\right)\subset b\mathcal{E}^{*}$, y un
$\mathcal{E}^{*}$-proceso $E$-valuado continuo por la derecha
$\PE$ en alg\'un espacio de probabilidad filtrado
$\left(\Omega,\mathcal{G},\mathcal{G}_{t},\prob\right)$ tal que
$X=\left(\Omega,\mathcal{G},\mathcal{G}_{t},\prob\right)$ es de
Markov (Homog\'eneo) con semigrupo de transici\'on $(P_{t})$ y
distribuci\'on inicial $\mu$.
\end{Def}

Consid\'erese la colecci\'on de variables aleatorias $X_{t}$
definidas en alg\'un espacio de probabilidad, y una colecci\'on de
medidas $\mathbf{P}^{x}$ tales que
$\mathbf{P}^{x}\left\{X_{0}=x\right\}$, y bajo cualquier
$\mathbf{P}^{x}$, $X_{t}$ es de Markov con semigrupo
$\left(P_{t}\right)$. $\mathbf{P}^{x}$ puede considerarse como la
distribuci\'on condicional de $\mathbf{P}$ dado $X_{0}=x$.

\begin{Def}\label{Def.2.2.S}
Sea $E$ espacio de Rad\'on, $\SG$ semigrupo de Markov en $\ER$. La colecci\'on $\mathbf{X}=\left(\Omega,\mathcal{G},\mathcal{G}_{t},X_{t},\theta_{t},\CM\right)$ es un proceso $\mathcal{E}$-Markov continuo por la derecha simple, con espacio de estados $E$ y semigrupo de transici\'on $\SG$ en caso de que $\mathbf{X}$ satisfaga las siguientes condiciones:
\begin{itemize}
\item[i)] $\left(\Omega,\mathcal{G},\mathcal{G}_{t}\right)$ es un espacio de medida filtrado, y $X_{t}$ es un proceso $E$-valuado continuo por la derecha $\mathcal{E}^{*}$-adaptado a $\left(\mathcal{G}_{t}\right)$;

\item[ii)] $\left(\theta_{t}\right)_{t\geq0}$ es una colecci\'on de operadores {\em shift} para $X$, es decir, mapea $\Omega$ en s\'i mismo satisfaciendo para $t,s\geq0$,

\begin{equation}\label{Eq.Shift}
\theta_{t}\circ\theta_{s}=\theta_{t+s}\textrm{ y }X_{t}\circ\theta_{t}=X_{t+s};
\end{equation}

\item[iii)] Para cualquier $x\in E$,$\CM\left\{X_{0}=x\right\}=1$, y el proceso $\PE$ tiene la propiedad de Markov (\ref{Eq.1.8.S}) con semigrupo de transici\'on $\SG$ relativo a $\left(\Omega,\mathcal{G},\mathcal{G}_{t},\CM\right)$.
\end{itemize}
\end{Def}

\begin{Def}[HD2]\label{Eq.2.2.S}
Para cualquier $\alpha>0$ y cualquier $f\in S^{\alpha}$, el proceso $t\rightarrow f\left(X_{t}\right)$ es continuo por la derecha casi seguramente.
\end{Def}

\begin{Def}\label{Def.PD}
Un sistema $\mathbf{X}=\left(\Omega,\mathcal{G},\mathcal{G}_{t},X_{t},\theta_{t},\CM\right)$ es un proceso derecho en el espacio de Rad\'on $E$ con semigrupo de transici\'on $\SG$ provisto de:
\begin{itemize}
\item[i)] $\mathbf{X}$ es una realizaci\'on  continua por la derecha, \ref{Def.2.2.S}, de $\SG$.

\item[ii)] $\mathbf{X}$ satisface la condicion HD2, \ref{Eq.2.2.S}, relativa a $\mathcal{G}_{t}$.

\item[iii)] $\mathcal{G}_{t}$ es aumentado y continuo por la derecha.
\end{itemize}
\end{Def}


%_________________________________________________________________________
%\renewcommand{\refname}{MODELO DE FLUJO}
%\renewcommand{\appendixname}{MODELO DE FLUJO}
%\renewcommand{\appendixtocname}{MODELO DE FLUJO}
%\renewcommand{\appendixpagename}{MODELO DE FLUJO}
%\appendix
%\clearpage % o \cleardoublepage
%\addappheadtotoc
%\appendixpage

\subsection{Construcci\'on del Modelo de Flujo}


\begin{Lema}[Lema 4.2, Dai\cite{Dai}]\label{Lema4.2}
Sea $\left\{x_{n}\right\}\subset \mathbf{X}$ con
$|x_{n}|\rightarrow\infty$, conforme $n\rightarrow\infty$. Suponga
que
\[lim_{n\rightarrow\infty}\frac{1}{|x_{n}|}U\left(0\right)=\overline{U}\]
y
\[lim_{n\rightarrow\infty}\frac{1}{|x_{n}|}V\left(0\right)=\overline{V}.\]

Entonces, conforme $n\rightarrow\infty$, casi seguramente

\begin{equation}\label{E1.4.2}
\frac{1}{|x_{n}|}\Phi^{k}\left(\left[|x_{n}|t\right]\right)\rightarrow
P_{k}^{'}t\textrm{, u.o.c.,}
\end{equation}

\begin{equation}\label{E1.4.3}
\frac{1}{|x_{n}|}E^{x_{n}}_{k}\left(|x_{n}|t\right)\rightarrow
\alpha_{k}\left(t-\overline{U}_{k}\right)^{+}\textrm{, u.o.c.,}
\end{equation}

\begin{equation}\label{E1.4.4}
\frac{1}{|x_{n}|}S^{x_{n}}_{k}\left(|x_{n}|t\right)\rightarrow
\mu_{k}\left(t-\overline{V}_{k}\right)^{+}\textrm{, u.o.c.,}
\end{equation}

donde $\left[t\right]$ es la parte entera de $t$ y
$\mu_{k}=1/m_{k}=1/\esp\left[\eta_{k}\left(1\right)\right]$.
\end{Lema}

\begin{Lema}[Lema 4.3, Dai\cite{Dai}]\label{Lema.4.3}
Sea $\left\{x_{n}\right\}\subset \mathbf{X}$ con
$|x_{n}|\rightarrow\infty$, conforme $n\rightarrow\infty$. Suponga
que
\[lim_{n\rightarrow\infty}\frac{1}{|x_{n}|}U_{k}\left(0\right)=\overline{U}_{k}\]
y
\[lim_{n\rightarrow\infty}\frac{1}{|x_{n}|}V_{k}\left(0\right)=\overline{V}_{k}.\]
\begin{itemize}
\item[a)] Conforme $n\rightarrow\infty$ casi seguramente,
\[lim_{n\rightarrow\infty}\frac{1}{|x_{n}|}U^{x_{n}}_{k}\left(|x_{n}|t\right)=\left(\overline{U}_{k}-t\right)^{+}\textrm{, u.o.c.}\]
y
\[lim_{n\rightarrow\infty}\frac{1}{|x_{n}|}V^{x_{n}}_{k}\left(|x_{n}|t\right)=\left(\overline{V}_{k}-t\right)^{+}.\]

\item[b)] Para cada $t\geq0$ fijo,
\[\left\{\frac{1}{|x_{n}|}U^{x_{n}}_{k}\left(|x_{n}|t\right),|x_{n}|\geq1\right\}\]
y
\[\left\{\frac{1}{|x_{n}|}V^{x_{n}}_{k}\left(|x_{n}|t\right),|x_{n}|\geq1\right\}\]
\end{itemize}
son uniformemente convergentes.
\end{Lema}

Sea $S_{l}^{x}\left(t\right)$ el n\'umero total de servicios
completados de la clase $l$, si la clase $l$ est\'a dando $t$
unidades de tiempo de servicio. Sea $T_{l}^{x}\left(x\right)$ el
monto acumulado del tiempo de servicio que el servidor
$s\left(l\right)$ gasta en los usuarios de la clase $l$ al tiempo
$t$. Entonces $S_{l}^{x}\left(T_{l}^{x}\left(t\right)\right)$ es
el n\'umero total de servicios completados para la clase $l$ al
tiempo $t$. Una fracci\'on de estos usuarios,
$\Phi_{k}^{x}\left(S_{l}^{x}\left(T_{l}^{x}\left(t\right)\right)\right)$,
se convierte en usuarios de la clase $k$.\\

Entonces, dado lo anterior, se tiene la siguiente representaci\'on
para el proceso de la longitud de la cola:\\

\begin{equation}
Q_{k}^{x}\left(t\right)=Q_{k}^{x}\left(0\right)+E_{k}^{x}\left(t\right)+\sum_{l=1}^{K}\Phi_{k}^{l}\left(S_{l}^{x}\left(T_{l}^{x}\left(t\right)\right)\right)-S_{k}^{x}\left(T_{k}^{x}\left(t\right)\right)
\end{equation}
para $k=1,\ldots,K$. Para $i=1,\ldots,d$, sea
\[I_{i}^{x}\left(t\right)=t-\sum_{j\in C_{i}}T_{k}^{x}\left(t\right).\]

Entonces $I_{i}^{x}\left(t\right)$ es el monto acumulado del
tiempo que el servidor $i$ ha estado desocupado al tiempo $t$. Se
est\'a asumiendo que las disciplinas satisfacen la ley de
conservaci\'on del trabajo, es decir, el servidor $i$ est\'a en
pausa solamente cuando no hay usuarios en la estaci\'on $i$.
Entonces, se tiene que

\begin{equation}
\int_{0}^{\infty}\left(\sum_{k\in
C_{i}}Q_{k}^{x}\left(t\right)\right)dI_{i}^{x}\left(t\right)=0,
\end{equation}
para $i=1,\ldots,d$.\\

Hacer
\[T^{x}\left(t\right)=\left(T_{1}^{x}\left(t\right),\ldots,T_{K}^{x}\left(t\right)\right)^{'},\]
\[I^{x}\left(t\right)=\left(I_{1}^{x}\left(t\right),\ldots,I_{K}^{x}\left(t\right)\right)^{'}\]
y
\[S^{x}\left(T^{x}\left(t\right)\right)=\left(S_{1}^{x}\left(T_{1}^{x}\left(t\right)\right),\ldots,S_{K}^{x}\left(T_{K}^{x}\left(t\right)\right)\right)^{'}.\]

Para una disciplina que cumple con la ley de conservaci\'on del
trabajo, en forma vectorial, se tiene el siguiente conjunto de
ecuaciones

\begin{equation}\label{Eq.MF.1.3}
Q^{x}\left(t\right)=Q^{x}\left(0\right)+E^{x}\left(t\right)+\sum_{l=1}^{K}\Phi^{l}\left(S_{l}^{x}\left(T_{l}^{x}\left(t\right)\right)\right)-S^{x}\left(T^{x}\left(t\right)\right),\\
\end{equation}

\begin{equation}\label{Eq.MF.2.3}
Q^{x}\left(t\right)\geq0,\\
\end{equation}

\begin{equation}\label{Eq.MF.3.3}
T^{x}\left(0\right)=0,\textrm{ y }\overline{T}^{x}\left(t\right)\textrm{ es no decreciente},\\
\end{equation}

\begin{equation}\label{Eq.MF.4.3}
I^{x}\left(t\right)=et-CT^{x}\left(t\right)\textrm{ es no
decreciente}\\
\end{equation}

\begin{equation}\label{Eq.MF.5.3}
\int_{0}^{\infty}\left(CQ^{x}\left(t\right)\right)dI_{i}^{x}\left(t\right)=0,\\
\end{equation}

\begin{equation}\label{Eq.MF.6.3}
\textrm{Condiciones adicionales en
}\left(\overline{Q}^{x}\left(\cdot\right),\overline{T}^{x}\left(\cdot\right)\right)\textrm{
espec\'ificas de la disciplina de la cola,}
\end{equation}

donde $e$ es un vector de unos de dimensi\'on $d$, $C$ es la
matriz definida por
\[C_{ik}=\left\{\begin{array}{cc}
1,& S\left(k\right)=i,\\
0,& \textrm{ en otro caso}.\\
\end{array}\right.
\]
Es necesario enunciar el siguiente Teorema que se utilizar\'a para
el Teorema \ref{Tma.4.2.Dai}:
\begin{Teo}[Teorema 4.1, Dai \cite{Dai}]
Considere una disciplina que cumpla la ley de conservaci\'on del
trabajo, para casi todas las trayectorias muestrales $\omega$ y
cualquier sucesi\'on de estados iniciales
$\left\{x_{n}\right\}\subset \mathbf{X}$, con
$|x_{n}|\rightarrow\infty$, existe una subsucesi\'on
$\left\{x_{n_{j}}\right\}$ con $|x_{n_{j}}|\rightarrow\infty$ tal
que
\begin{equation}\label{Eq.4.15}
\frac{1}{|x_{n_{j}}|}\left(Q^{x_{n_{j}}}\left(0\right),U^{x_{n_{j}}}\left(0\right),V^{x_{n_{j}}}\left(0\right)\right)\rightarrow\left(\overline{Q}\left(0\right),\overline{U},\overline{V}\right),
\end{equation}

\begin{equation}\label{Eq.4.16}
\frac{1}{|x_{n_{j}}|}\left(Q^{x_{n_{j}}}\left(|x_{n_{j}}|t\right),T^{x_{n_{j}}}\left(|x_{n_{j}}|t\right)\right)\rightarrow\left(\overline{Q}\left(t\right),\overline{T}\left(t\right)\right)\textrm{
u.o.c.}
\end{equation}

Adem\'as,
$\left(\overline{Q}\left(t\right),\overline{T}\left(t\right)\right)$
satisface las siguientes ecuaciones:
\begin{equation}\label{Eq.MF.1.3a}
\overline{Q}\left(t\right)=Q\left(0\right)+\left(\alpha
t-\overline{U}\right)^{+}-\left(I-P\right)^{'}M^{-1}\left(\overline{T}\left(t\right)-\overline{V}\right)^{+},
\end{equation}

\begin{equation}\label{Eq.MF.2.3a}
\overline{Q}\left(t\right)\geq0,\\
\end{equation}

\begin{equation}\label{Eq.MF.3.3a}
\overline{T}\left(t\right)\textrm{ es no decreciente y comienza en cero},\\
\end{equation}

\begin{equation}\label{Eq.MF.4.3a}
\overline{I}\left(t\right)=et-C\overline{T}\left(t\right)\textrm{
es no decreciente,}\\
\end{equation}

\begin{equation}\label{Eq.MF.5.3a}
\int_{0}^{\infty}\left(C\overline{Q}\left(t\right)\right)d\overline{I}\left(t\right)=0,\\
\end{equation}

\begin{equation}\label{Eq.MF.6.3a}
\textrm{Condiciones adicionales en
}\left(\overline{Q}\left(\cdot\right),\overline{T}\left(\cdot\right)\right)\textrm{
especficas de la disciplina de la cola,}
\end{equation}
\end{Teo}


Propiedades importantes para el modelo de flujo retrasado:

\begin{Prop}
 Sea $\left(\overline{Q},\overline{T},\overline{T}^{0}\right)$ un flujo l\'imite de \ref{Eq.4.4} y suponga que cuando $x\rightarrow\infty$ a lo largo de
una subsucesi\'on
\[\left(\frac{1}{|x|}Q_{k}^{x}\left(0\right),\frac{1}{|x|}A_{k}^{x}\left(0\right),\frac{1}{|x|}B_{k}^{x}\left(0\right),\frac{1}{|x|}B_{k}^{x,0}\left(0\right)\right)\rightarrow\left(\overline{Q}_{k}\left(0\right),0,0,0\right)\]
para $k=1,\ldots,K$. EL flujo l\'imite tiene las siguientes
propiedades, donde las propiedades de la derivada se cumplen donde
la derivada exista:
\begin{itemize}
 \item[i)] Los vectores de tiempo ocupado $\overline{T}\left(t\right)$ y $\overline{T}^{0}\left(t\right)$ son crecientes y continuas con
$\overline{T}\left(0\right)=\overline{T}^{0}\left(0\right)=0$.
\item[ii)] Para todo $t\geq0$
\[\sum_{k=1}^{K}\left[\overline{T}_{k}\left(t\right)+\overline{T}_{k}^{0}\left(t\right)\right]=t\]
\item[iii)] Para todo $1\leq k\leq K$
\[\overline{Q}_{k}\left(t\right)=\overline{Q}_{k}\left(0\right)+\alpha_{k}t-\mu_{k}\overline{T}_{k}\left(t\right)\]
\item[iv)]  Para todo $1\leq k\leq K$
\[\dot{{\overline{T}}}_{k}\left(t\right)=\beta_{k}\] para $\overline{Q}_{k}\left(t\right)=0$.
\item[v)] Para todo $k,j$
\[\mu_{k}^{0}\overline{T}_{k}^{0}\left(t\right)=\mu_{j}^{0}\overline{T}_{j}^{0}\left(t\right)\]
\item[vi)]  Para todo $1\leq k\leq K$
\[\mu_{k}\dot{{\overline{T}}}_{k}\left(t\right)=l_{k}\mu_{k}^{0}\dot{{\overline{T}}}_{k}^{0}\left(t\right)\] para $\overline{Q}_{k}\left(t\right)>0$.
\end{itemize}
\end{Prop}

\begin{Teo}[Teorema 5.1: Ley Fuerte para Procesos de Conteo
\cite{Gut}]\label{Tma.5.1.Gut} Sea
$0<\mu<\esp\left(X_{1}\right]\leq\infty$. entonces

\begin{itemize}
\item[a)] $\frac{N\left(t\right)}{t}\rightarrow\frac{1}{\mu}$
a.s., cuando $t\rightarrow\infty$.


\item[b)]$\esp\left[\frac{N\left(t\right)}{t}\right]^{r}\rightarrow\frac{1}{\mu^{r}}$,
cuando $t\rightarrow\infty$ para todo $r>0$..
\end{itemize}
\end{Teo}


\begin{Prop}[Proposici\'on 5.3 \cite{DaiSean}]
Sea $X$ proceso de estados para la red de colas, y suponga que se
cumplen los supuestos (A1) y (A2), entonces para alguna constante
positiva $C_{p+1}<\infty$, $\delta>0$ y un conjunto compacto
$C\subset X$.

\begin{equation}\label{Eq.5.4}
\esp_{x}\left[\int_{0}^{\tau_{C}\left(\delta\right)}\left(1+|X\left(t\right)|^{p}\right)dt\right]\leq
C_{p+1}\left(1+|x|^{p+1}\right)
\end{equation}
\end{Prop}

\begin{Prop}[Proposici\'on 5.4 \cite{DaiSean}]
Sea $X$ un proceso de Markov Borel Derecho en $X$, sea
$f:X\leftarrow\rea_{+}$ y defina para alguna $\delta>0$, y un
conjunto cerrado $C\subset X$
\[V\left(x\right):=\esp_{x}\left[\int_{0}^{\tau_{C}\left(\delta\right)}f\left(X\left(t\right)\right)dt\right]\]
para $x\in X$. Si $V$ es finito en todas partes y uniformemente
acotada en $C$, entonces existe $k<\infty$ tal que
\begin{equation}\label{Eq.5.11}
\frac{1}{t}\esp_{x}\left[V\left(x\right)\right]+\frac{1}{t}\int_{0}^{t}\esp_{x}\left[f\left(X\left(s\right)\right)ds\right]\leq\frac{1}{t}V\left(x\right)+k,
\end{equation}
para $x\in X$ y $t>0$.
\end{Prop}


%_________________________________________________________________________
%\renewcommand{\refname}{Ap\'endice D}
%\renewcommand{\appendixname}{ESTABILIDAD}
%\renewcommand{\appendixtocname}{ESTABILIDAD}
%\renewcommand{\appendixpagename}{ESTABILIDAD}
%\appendix
%\clearpage % o \cleardoublepage
%\addappheadtotoc
%\appendixpage

\subsection{Estabilidad}

\begin{Def}[Definici\'on 3.2, Dai y Meyn \cite{DaiSean}]
El modelo de flujo retrasado de una disciplina de servicio en una
red con retraso
$\left(\overline{A}\left(0\right),\overline{B}\left(0\right)\right)\in\rea_{+}^{K+|A|}$
se define como el conjunto de ecuaciones dadas en
\ref{Eq.3.8}-\ref{Eq.3.13}, junto con la condici\'on:
\begin{equation}\label{CondAd.FluidModel}
\overline{Q}\left(t\right)=\overline{Q}\left(0\right)+\left(\alpha
t-\overline{A}\left(0\right)\right)^{+}-\left(I-P^{'}\right)M\left(\overline{T}\left(t\right)-\overline{B}\left(0\right)\right)^{+}
\end{equation}
\end{Def}

entonces si el modelo de flujo retrasado tambi\'en es estable:


\begin{Def}[Definici\'on 3.1, Dai y Meyn \cite{DaiSean}]
Un flujo l\'imite (retrasado) para una red bajo una disciplina de
servicio espec\'ifica se define como cualquier soluci\'on
 $\left(\overline{Q}\left(\cdot\right),\overline{T}\left(\cdot\right)\right)$ de las siguientes ecuaciones, donde
$\overline{Q}\left(t\right)=\left(\overline{Q}_{1}\left(t\right),\ldots,\overline{Q}_{K}\left(t\right)\right)^{'}$
y
$\overline{T}\left(t\right)=\left(\overline{T}_{1}\left(t\right),\ldots,\overline{T}_{K}\left(t\right)\right)^{'}$
\begin{equation}\label{Eq.3.8}
\overline{Q}_{k}\left(t\right)=\overline{Q}_{k}\left(0\right)+\alpha_{k}t-\mu_{k}\overline{T}_{k}\left(t\right)+\sum_{l=1}^{k}P_{lk}\mu_{l}\overline{T}_{l}\left(t\right)\\
\end{equation}
\begin{equation}\label{Eq.3.9}
\overline{Q}_{k}\left(t\right)\geq0\textrm{ para }k=1,2,\ldots,K,\\
\end{equation}
\begin{equation}\label{Eq.3.10}
\overline{T}_{k}\left(0\right)=0,\textrm{ y }\overline{T}_{k}\left(\cdot\right)\textrm{ es no decreciente},\\
\end{equation}
\begin{equation}\label{Eq.3.11}
\overline{I}_{i}\left(t\right)=t-\sum_{k\in C_{i}}\overline{T}_{k}\left(t\right)\textrm{ es no decreciente}\\
\end{equation}
\begin{equation}\label{Eq.3.12}
\overline{I}_{i}\left(\cdot\right)\textrm{ se incrementa al tiempo }t\textrm{ cuando }\sum_{k\in C_{i}}Q_{k}^{x}\left(t\right)dI_{i}^{x}\left(t\right)=0\\
\end{equation}
\begin{equation}\label{Eq.3.13}
\textrm{condiciones adicionales sobre
}\left(Q^{x}\left(\cdot\right),T^{x}\left(\cdot\right)\right)\textrm{
referentes a la disciplina de servicio}
\end{equation}
\end{Def}

\begin{Lema}[Lema 3.1 \cite{Chen}]\label{Lema3.1}
Si el modelo de flujo es estable, definido por las ecuaciones
(3.8)-(3.13), entonces el modelo de flujo retrasado tambin es
estable.
\end{Lema}

\begin{Teo}[Teorema 5.1 \cite{Chen}]\label{Tma.5.1.Chen}
La red de colas es estable si existe una constante $t_{0}$ que
depende de $\left(\alpha,\mu,T,U\right)$ y $V$ que satisfagan las
ecuaciones (5.1)-(5.5), $Z\left(t\right)=0$, para toda $t\geq
t_{0}$.
\end{Teo}

\begin{Prop}[Proposici\'on 5.1, Dai y Meyn \cite{DaiSean}]\label{Prop.5.1.DaiSean}
Suponga que los supuestos A1) y A2) son ciertos y que el modelo de flujo es estable. Entonces existe $t_{0}>0$ tal que
\begin{equation}
lim_{|x|\rightarrow\infty}\frac{1}{|x|^{p+1}}\esp_{x}\left[|X\left(t_{0}|x|\right)|^{p+1}\right]=0
\end{equation}
\end{Prop}

\begin{Lemma}[Lema 5.2, Dai y Meyn \cite{DaiSean}]\label{Lema.5.2.DaiSean}
 Sea $\left\{\zeta\left(k\right):k\in \mathbb{z}\right\}$ una sucesi\'on independiente e id\'enticamente distribuida que toma valores en $\left(0,\infty\right)$,
y sea
$E\left(t\right)=max\left(n\geq1:\zeta\left(1\right)+\cdots+\zeta\left(n-1\right)\leq
t\right)$. Si $\esp\left[\zeta\left(1\right)\right]<\infty$,
entonces para cualquier entero $r\geq1$
\begin{equation}
 lim_{t\rightarrow\infty}\esp\left[\left(\frac{E\left(t\right)}{t}\right)^{r}\right]=\left(\frac{1}{\esp\left[\zeta_{1}\right]}\right)^{r}.
\end{equation}
Luego, bajo estas condiciones:
\begin{itemize}
 \item[a)] para cualquier $\delta>0$, $\sup_{t\geq\delta}\esp\left[\left(\frac{E\left(t\right)}{t}\right)^{r}\right]<\infty$
\item[b)] las variables aleatorias
$\left\{\left(\frac{E\left(t\right)}{t}\right)^{r}:t\geq1\right\}$
son uniformemente integrables.
\end{itemize}
\end{Lemma}

\begin{Teo}[Teorema 5.5, Dai y Meyn \cite{DaiSean}]\label{Tma.5.5.DaiSean}
Suponga que los supuestos A1) y A2) se cumplen y que el modelo de
flujo es estable. Entonces existe una constante $\kappa_{p}$ tal
que
\begin{equation}
\frac{1}{t}\int_{0}^{t}\esp_{x}\left[|Q\left(s\right)|^{p}\right]ds\leq\kappa_{p}\left\{\frac{1}{t}|x|^{p+1}+1\right\}
\end{equation}
para $t>0$ y $x\in X$. En particular, para cada condici\'on
inicial
\begin{eqnarray*}
\limsup_{t\rightarrow\infty}\frac{1}{t}\int_{0}^{t}\esp_{x}\left[|Q\left(s\right)|^{p}\right]ds\leq\kappa_{p}.
\end{eqnarray*}
\end{Teo}

\begin{Teo}[Teorema 6.2, Dai y Meyn \cite{DaiSean}]\label{Tma.6.2.DaiSean}
Suponga que se cumplen los supuestos A1), A2) y A3) y que el
modelo de flujo es estable. Entonces se tiene que
\begin{equation}
\left\|P^{t}\left(x,\cdot\right)-\pi\left(\cdot\right)\right\|_{f_{p}}\textrm{,
}t\rightarrow\infty,x\in X.
\end{equation}
En particular para cada condici\'on inicial
\begin{eqnarray*}
\lim_{t\rightarrow\infty}\esp_{x}\left[|Q\left(t\right)|^{p}\right]=\esp_{\pi}\left[|Q\left(0\right)|^{p}\right]\leq\kappa_{r}
\end{eqnarray*}
\end{Teo}
\begin{Teo}[Teorema 6.3, Dai y Meyn \cite{DaiSean}]\label{Tma.6.3.DaiSean}
Suponga que se cumplen los supuestos A1), A2) y A3) y que el
modelo de flujo es estable. Entonces con
$f\left(x\right)=f_{1}\left(x\right)$ se tiene
\begin{equation}
\lim_{t\rightarrow\infty}t^{p-1}\left\|P^{t}\left(x,\cdot\right)-\pi\left(\cdot\right)\right\|_{f}=0.
\end{equation}
En particular para cada condici\'on inicial
\begin{eqnarray*}
\lim_{t\rightarrow\infty}t^{p-1}|\esp_{x}\left[Q\left(t\right)\right]-\esp_{\pi}\left[Q\left(0\right)\right]|=0.
\end{eqnarray*}
\end{Teo}

\begin{Teo}[Teorema 6.4, Dai y Meyn \cite{DaiSean}]\label{Tma.6.4.DaiSean}
Suponga que se cumplen los supuestos A1), A2) y A3) y que el
modelo de flujo es estable. Sea $\nu$ cualquier distribuci\'on de
probabilidad en $\left(X,\mathcal{B}_{X}\right)$, y $\pi$ la
distribuci\'on estacionaria de $X$.
\begin{itemize}
\item[i)] Para cualquier $f:X\leftarrow\rea_{+}$
\begin{equation}
\lim_{t\rightarrow\infty}\frac{1}{t}\int_{o}^{t}f\left(X\left(s\right)\right)ds=\pi\left(f\right):=\int
f\left(x\right)\pi\left(dx\right)
\end{equation}
$\prob$-c.s.

\item[ii)] Para cualquier $f:X\leftarrow\rea_{+}$ con
$\pi\left(|f|\right)<\infty$, la ecuaci\'on anterior se cumple.
\end{itemize}
\end{Teo}

\begin{Teo}[Teorema 2.2, Down \cite{Down}]\label{Tma2.2.Down}
Suponga que el fluido modelo es inestable en el sentido de que
para alguna $\epsilon_{0},c_{0}\geq0$,
\begin{equation}\label{Eq.Inestability}
|Q\left(T\right)|\geq\epsilon_{0}T-c_{0}\textrm{,   }T\geq0,
\end{equation}
para cualquier condici\'on inicial $Q\left(0\right)$, con
$|Q\left(0\right)|=1$. Entonces para cualquier $0<q\leq1$, existe
$B<0$ tal que para cualquier $|x|\geq B$,
\begin{equation}
\prob_{x}\left\{\mathbb{X}\rightarrow\infty\right\}\geq q.
\end{equation}
\end{Teo}


\begin{Def}
Sea $X$ un conjunto y $\mathcal{F}$ una $\sigma$-\'algebra de
subconjuntos de $X$, la pareja $\left(X,\mathcal{F}\right)$ es
llamado espacio medible. Un subconjunto $A$ de $X$ es llamado
medible, o medible con respecto a $\mathcal{F}$, si
$A\in\mathcal{F}$.
\end{Def}

\begin{Def}
Sea $\left(X,\mathcal{F},\mu\right)$ espacio de medida. Se dice
que la medida $\mu$ es $\sigma$-finita si se puede escribir
$X=\bigcup_{n\geq1}X_{n}$ con $X_{n}\in\mathcal{F}$ y
$\mu\left(X_{n}\right)<\infty$.
\end{Def}

\begin{Def}\label{Cto.Borel}
Sea $X$ el conjunto de los \'umeros reales $\rea$. El \'algebra de
Borel es la $\sigma$-\'algebra $B$ generada por los intervalos
abiertos $\left(a,b\right)\in\rea$. Cualquier conjunto en $B$ es
llamado {\em Conjunto de Borel}.
\end{Def}

\begin{Def}\label{Funcion.Medible}
Una funci\'on $f:X\rightarrow\rea$, es medible si para cualquier
n\'umero real $\alpha$ el conjunto
\[\left\{x\in X:f\left(x\right)>\alpha\right\}\]
pertenece a $X$. Equivalentemente, se dice que $f$ es medible si
\[f^{-1}\left(\left(\alpha,\infty\right)\right)=\left\{x\in X:f\left(x\right)>\alpha\right\}\in\mathcal{F}.\]
\end{Def}


\begin{Def}\label{Def.Cilindros}
Sean $\left(\Omega_{i},\mathcal{F}_{i}\right)$, $i=1,2,\ldots,$
espacios medibles y $\Omega=\prod_{i=1}^{\infty}\Omega_{i}$ el
conjunto de todas las sucesiones
$\left(\omega_{1},\omega_{2},\ldots,\right)$ tales que
$\omega_{i}\in\Omega_{i}$, $i=1,2,\ldots,$. Si
$B^{n}\subset\prod_{i=1}^{\infty}\Omega_{i}$, definimos
$B_{n}=\left\{\omega\in\Omega:\left(\omega_{1},\omega_{2},\ldots,\omega_{n}\right)\in
B^{n}\right\}$. Al conjunto $B_{n}$ se le llama {\em cilindro} con
base $B^{n}$, el cilindro es llamado medible si
$B^{n}\in\prod_{i=1}^{\infty}\mathcal{F}_{i}$.
\end{Def}


\begin{Def}\label{Def.Proc.Adaptado}[TSP, Ash \cite{RBA}]
Sea $X\left(t\right),t\geq0$ proceso estoc\'astico, el proceso es
adaptado a la familia de $\sigma$-\'algebras $\mathcal{F}_{t}$,
para $t\geq0$, si para $s<t$ implica que
$\mathcal{F}_{s}\subset\mathcal{F}_{t}$, y $X\left(t\right)$ es
$\mathcal{F}_{t}$-medible para cada $t$. Si no se especifica
$\mathcal{F}_{t}$ entonces se toma $\mathcal{F}_{t}$ como
$\mathcal{F}\left(X\left(s\right),s\leq t\right)$, la m\'as
peque\~na $\sigma$-\'algebra de subconjuntos de $\Omega$ que hace
que cada $X\left(s\right)$, con $s\leq t$ sea Borel medible.
\end{Def}


\begin{Def}\label{Def.Tiempo.Paro}[TSP, Ash \cite{RBA}]
Sea $\left\{\mathcal{F}\left(t\right),t\geq0\right\}$ familia
creciente de sub $\sigma$-\'algebras. es decir,
$\mathcal{F}\left(s\right)\subset\mathcal{F}\left(t\right)$ para
$s\leq t$. Un tiempo de paro para $\mathcal{F}\left(t\right)$ es
una funci\'on $T:\Omega\rightarrow\left[0,\infty\right]$ tal que
$\left\{T\leq t\right\}\in\mathcal{F}\left(t\right)$ para cada
$t\geq0$. Un tiempo de paro para el proceso estoc\'astico
$X\left(t\right),t\geq0$ es un tiempo de paro para las
$\sigma$-\'algebras
$\mathcal{F}\left(t\right)=\mathcal{F}\left(X\left(s\right)\right)$.
\end{Def}

\begin{Def}
Sea $X\left(t\right),t\geq0$ proceso estoc\'astico, con
$\left(S,\chi\right)$ espacio de estados. Se dice que el proceso
es adaptado a $\left\{\mathcal{F}\left(t\right)\right\}$, es
decir, si para cualquier $s,t\in I$, $I$ conjunto de \'indices,
$s<t$, se tiene que
$\mathcal{F}\left(s\right)\subset\mathcal{F}\left(t\right)$ y
$X\left(t\right)$ es $\mathcal{F}\left(t\right)$-medible,
\end{Def}

\begin{Def}
Sea $X\left(t\right),t\geq0$ proceso estoc\'astico, se dice que es
un Proceso de Markov relativo a $\mathcal{F}\left(t\right)$ o que
$\left\{X\left(t\right),\mathcal{F}\left(t\right)\right\}$ es de
Markov si y s\'olo si para cualquier conjunto $B\in\chi$,  y
$s,t\in I$, $s<t$ se cumple que
\begin{equation}\label{Prop.Markov}
P\left\{X\left(t\right)\in
B|\mathcal{F}\left(s\right)\right\}=P\left\{X\left(t\right)\in
B|X\left(s\right)\right\}.
\end{equation}
\end{Def}
\begin{Note}
Si se dice que $\left\{X\left(t\right)\right\}$ es un Proceso de
Markov sin mencionar $\mathcal{F}\left(t\right)$, se asumir\'a que
\begin{eqnarray*}
\mathcal{F}\left(t\right)=\mathcal{F}_{0}\left(t\right)=\mathcal{F}\left(X\left(r\right),r\leq
t\right),
\end{eqnarray*}
entonces la ecuaci\'on (\ref{Prop.Markov}) se puede escribir como
\begin{equation}
P\left\{X\left(t\right)\in B|X\left(r\right),r\leq s\right\} =
P\left\{X\left(t\right)\in B|X\left(s\right)\right\}
\end{equation}
\end{Note}

\begin{Teo}
Sea $\left(X_{n},\mathcal{F}_{n},n=0,1,\ldots,\right\}$ Proceso de
Markov con espacio de estados $\left(S_{0},\chi_{0}\right)$
generado por una distribuici\'on inicial $P_{o}$ y probabilidad de
transici\'on $p_{mn}$, para $m,n=0,1,\ldots,$ $m<n$, que por
notaci\'on se escribir\'a como $p\left(m,n,x,B\right)\rightarrow
p_{mn}\left(x,B\right)$. Sea $S$ tiempo de paro relativo a la
$\sigma$-\'algebra $\mathcal{F}_{n}$. Sea $T$ funci\'on medible,
$T:\Omega\rightarrow\left\{0,1,\ldots,\right\}$. Sup\'ongase que
$T\geq S$, entonces $T$ es tiempo de paro. Si $B\in\chi_{0}$,
entonces
\begin{equation}\label{Prop.Fuerte.Markov}
P\left\{X\left(T\right)\in
B,T<\infty|\mathcal{F}\left(S\right)\right\} =
p\left(S,T,X\left(s\right),B\right)
\end{equation}
en $\left\{T<\infty\right\}$.
\end{Teo}


Sea $K$ conjunto numerable y sea $d:K\rightarrow\nat$ funci\'on.
Para $v\in K$, $M_{v}$ es un conjunto abierto de
$\rea^{d\left(v\right)}$. Entonces \[E=\cup_{v\in
K}M_{v}=\left\{\left(v,\zeta\right):v\in K,\zeta\in
M_{v}\right\}.\]

Sea $\mathcal{E}$ la clase de conjuntos medibles en $E$:
\[\mathcal{E}=\left\{\cup_{v\in K}A_{v}:A_{v}\in \mathcal{M}_{v}\right\}.\]

donde $\mathcal{M}$ son los conjuntos de Borel de $M_{v}$.
Entonces $\left(E,\mathcal{E}\right)$ es un espacio de Borel. El
estado del proceso se denotar\'a por
$\mathbf{x}_{t}=\left(v_{t},\zeta_{t}\right)$. La distribuci\'on
de $\left(\mathbf{x}_{t}\right)$ est\'a determinada por por los
siguientes objetos:

\begin{itemize}
\item[i)] Los campos vectoriales $\left(\mathcal{H}_{v},v\in
K\right)$. \item[ii)] Una funci\'on medible $\lambda:E\rightarrow
\rea_{+}$. \item[iii)] Una medida de transici\'on
$Q:\mathcal{E}\times\left(E\cup\Gamma^{*}\right)\rightarrow\left[0,1\right]$
donde
\begin{equation}
\Gamma^{*}=\cup_{v\in K}\partial^{*}M_{v}.
\end{equation}
y
\begin{equation}
\partial^{*}M_{v}=\left\{z\in\partial M_{v}:\mathbf{\mathbf{\phi}_{v}\left(t,\zeta\right)=\mathbf{z}}\textrm{ para alguna }\left(t,\zeta\right)\in\rea_{+}\times M_{v}\right\}.
\end{equation}
$\partial M_{v}$ denota  la frontera de $M_{v}$.
\end{itemize}

El campo vectorial $\left(\mathcal{H}_{v},v\in K\right)$ se supone
tal que para cada $\mathbf{z}\in M_{v}$ existe una \'unica curva
integral $\mathbf{\phi}_{v}\left(t,\zeta\right)$ que satisface la
ecuaci\'on

\begin{equation}
\frac{d}{dt}f\left(\zeta_{t}\right)=\mathcal{H}f\left(\zeta_{t}\right),
\end{equation}
con $\zeta_{0}=\mathbf{z}$, para cualquier funci\'on suave
$f:\rea^{d}\rightarrow\rea$ y $\mathcal{H}$ denota el operador
diferencial de primer orden, con $\mathcal{H}=\mathcal{H}_{v}$ y
$\zeta_{t}=\mathbf{\phi}\left(t,\mathbf{z}\right)$. Adem\'as se
supone que $\mathcal{H}_{v}$ es conservativo, es decir, las curvas
integrales est\'an definidas para todo $t>0$.

Para $\mathbf{x}=\left(v,\zeta\right)\in E$ se denota
\[t^{*}\mathbf{x}=inf\left\{t>0:\mathbf{\phi}_{v}\left(t,\zeta\right)\in\partial^{*}M_{v}\right\}\]

En lo que respecta a la funci\'on $\lambda$, se supondr\'a que
para cada $\left(v,\zeta\right)\in E$ existe un $\epsilon>0$ tal
que la funci\'on
$s\rightarrow\lambda\left(v,\phi_{v}\left(s,\zeta\right)\right)\in
E$ es integrable para $s\in\left[0,\epsilon\right)$. La medida de
transici\'on $Q\left(A;\mathbf{x}\right)$ es una funci\'on medible
de $\mathbf{x}$ para cada $A\in\mathcal{E}$, definida para
$\mathbf{x}\in E\cup\Gamma^{*}$ y es una medida de probabilidad en
$\left(E,\mathcal{E}\right)$ para cada $\mathbf{x}\in E$.

El movimiento del proceso $\left(\mathbf{x}_{t}\right)$ comenzando
en $\mathbf{x}=\left(n,\mathbf{z}\right)\in E$ se puede construir
de la siguiente manera, def\'inase la funci\'on $F$ por

\begin{equation}
F\left(t\right)=\left\{\begin{array}{ll}\\
exp\left(-\int_{0}^{t}\lambda\left(n,\phi_{n}\left(s,\mathbf{z}\right)\right)ds\right), & t<t^{*}\left(\mathbf{x}\right),\\
0, & t\geq t^{*}\left(\mathbf{x}\right)
\end{array}\right.
\end{equation}

Sea $T_{1}$ una variable aleatoria tal que
$\prob\left[T_{1}>t\right]=F\left(t\right)$, ahora sea la variable
aleatoria $\left(N,Z\right)$ con distribuici\'on
$Q\left(\cdot;\phi_{n}\left(T_{1},\mathbf{z}\right)\right)$. La
trayectoria de $\left(\mathbf{x}_{t}\right)$ para $t\leq T_{1}$
es\footnote{Revisar p\'agina 362, y 364 de Davis \cite{Davis}.}
\begin{eqnarray*}
\mathbf{x}_{t}=\left(v_{t},\zeta_{t}\right)=\left\{\begin{array}{ll}
\left(n,\phi_{n}\left(t,\mathbf{z}\right)\right), & t<T_{1},\\
\left(N,\mathbf{Z}\right), & t=t_{1}.
\end{array}\right.
\end{eqnarray*}

Comenzando en $\mathbf{x}_{T_{1}}$ se selecciona el siguiente
tiempo de intersalto $T_{2}-T_{1}$ lugar del post-salto
$\mathbf{x}_{T_{2}}$ de manera similar y as\'i sucesivamente. Este
procedimiento nos da una trayectoria determinista por partes
$\mathbf{x}_{t}$ con tiempos de salto $T_{1},T_{2},\ldots$. Bajo
las condiciones enunciadas para $\lambda,T_{1}>0$  y
$T_{1}-T_{2}>0$ para cada $i$, con probabilidad 1. Se supone que
se cumple la siquiente condici\'on.

\begin{Sup}[Supuesto 3.1, Davis \cite{Davis}]\label{Sup3.1.Davis}
Sea $N_{t}:=\sum_{t}\indora_{\left(t\geq t\right)}$ el n\'umero de
saltos en $\left[0,t\right]$. Entonces
\begin{equation}
\esp\left[N_{t}\right]<\infty\textrm{ para toda }t.
\end{equation}
\end{Sup}

es un proceso de Markov, m\'as a\'un, es un Proceso Fuerte de
Markov, es decir, la Propiedad Fuerte de Markov se cumple para
cualquier tiempo de paro.
%_________________________________________________________________________

En esta secci\'on se har\'an las siguientes consideraciones: $E$
es un espacio m\'etrico separable y la m\'etrica $d$ es compatible
con la topolog\'ia.


\begin{Def}
Un espacio topol\'ogico $E$ es llamado {\em Luisin} si es
homeomorfo a un subconjunto de Borel de un espacio m\'etrico
compacto.
\end{Def}

\begin{Def}
Un espacio topol\'ogico $E$ es llamado de {\em Rad\'on} si es
homeomorfo a un subconjunto universalmente medible de un espacio
m\'etrico compacto.
\end{Def}

Equivalentemente, la definici\'on de un espacio de Rad\'on puede
encontrarse en los siguientes t\'erminos:


\begin{Def}
$E$ es un espacio de Rad\'on si cada medida finita en
$\left(E,\mathcal{B}\left(E\right)\right)$ es regular interior o cerrada,
{\em tight}.
\end{Def}

\begin{Def}
Una medida finita, $\lambda$ en la $\sigma$-\'algebra de Borel de
un espacio metrizable $E$ se dice cerrada si
\begin{equation}\label{Eq.A2.3}
\lambda\left(E\right)=sup\left\{\lambda\left(K\right):K\textrm{ es
compacto en }E\right\}.
\end{equation}
\end{Def}

El siguiente teorema nos permite tener una mejor caracterizaci\'on de los espacios de Rad\'on:
\begin{Teo}\label{Tma.A2.2}
Sea $E$ espacio separable metrizable. Entonces $E$ es Radoniano si y s\'olo s\'i cada medida finita en $\left(E,\mathcal{B}\left(E\right)\right)$ es cerrada.
\end{Teo}

%_________________________________________________________________________________________
\subsection{Propiedades de Markov}
%_________________________________________________________________________________________

Sea $E$ espacio de estados, tal que $E$ es un espacio de Rad\'on, $\mathcal{B}\left(E\right)$ $\sigma$-\'algebra de Borel en $E$, que se denotar\'a por $\mathcal{E}$.

Sea $\left(X,\mathcal{G},\prob\right)$ espacio de probabilidad, $I\subset\rea$ conjunto de índices. Sea $\mathcal{F}_{\leq t}$ la $\sigma$-\'algebra natural definida como $\sigma\left\{f\left(X_{r}\right):r\in I, rleq t,f\in\mathcal{E}\right\}$. Se considerar\'a una $\sigma$-\'algebra m\'as general, $ \left(\mathcal{G}_{t}\right)$ tal que $\left(X_{t}\right)$ sea $\mathcal{E}$-adaptado.

\begin{Def}
Una familia $\left(P_{s,t}\right)$ de kernels de Markov en $\left(E,\mathcal{E}\right)$ indexada por pares $s,t\in I$, con $s\leq t$ es una funci\'on de transici\'on en $\ER$, si  para todo $r\leq s< t$ en $I$ y todo $x\in E$, $B\in\mathcal{E}$
\begin{equation}\label{Eq.Kernels}
P_{r,t}\left(x,B\right)=\int_{E}P_{r,s}\left(x,dy\right)P_{s,t}\left(y,B\right)\footnote{Ecuaci\'on de Chapman-Kolmogorov}.
\end{equation}
\end{Def}

Se dice que la funci\'on de transici\'on $\KM$ en $\ER$ es la funci\'on de transici\'on para un proceso $\PE$  con valores en $E$ y que satisface la propiedad de Markov\footnote{\begin{equation}\label{Eq.1.4.S}
\prob\left\{H|\mathcal{G}_{t}\right\}=\prob\left\{H|X_{t}\right\}\textrm{ }H\in p\mathcal{F}_{\geq t}.
\end{equation}} (\ref{Eq.1.4.S}) relativa a $\left(\mathcal{G}_{t}\right)$ si 

\begin{equation}\label{Eq.1.6.S}
\prob\left\{f\left(X_{t}\right)|\mathcal{G}_{s}\right\}=P_{s,t}f\left(X_{t}\right)\textrm{ }s\leq t\in I,\textrm{ }f\in b\mathcal{E}.
\end{equation}

\begin{Def}
Una familia $\left(P_{t}\right)_{t\geq0}$ de kernels de Markov en $\ER$ es llamada {\em Semigrupo de Transici\'on de Markov} o {\em Semigrupo de Transici\'on} si
\[P_{t+s}f\left(x\right)=P_{t}\left(P_{s}f\right)\left(x\right),\textrm{ }t,s\geq0,\textrm{ }x\in E\textrm{ }f\in b\mathcal{E}.\]
\end{Def}
\begin{Note}
Si la funci\'on de transici\'on $\KM$ es llamada homog\'enea si $P_{s,t}=P_{t-s}$.
\end{Note}

Un proceso de Markov que satisface la ecuaci\'on (\ref{Eq.1.6.S}) con funci\'on de transici\'on homog\'enea $\left(P_{t}\right)$ tiene la propiedad caracter\'istica
\begin{equation}\label{Eq.1.8.S}
\prob\left\{f\left(X_{t+s}\right)|\mathcal{G}_{t}\right\}=P_{s}f\left(X_{t}\right)\textrm{ }t,s\geq0,\textrm{ }f\in b\mathcal{E}.
\end{equation}
La ecuaci\'on anterior es la {\em Propiedad Simple de Markov} de $X$ relativa a $\left(P_{t}\right)$.

En este sentido el proceso $\PE$ cumple con la propiedad de Markov (\ref{Eq.1.8.S}) relativa a $\left(\Omega,\mathcal{G},\mathcal{G}_{t},\prob\right)$ con semigrupo de transici\'on $\left(P_{t}\right)$.
%_________________________________________________________________________________________
\subsection{Primer Condici\'on de Regularidad}
%_________________________________________________________________________________________
%\newcommand{\EM}{\left(\Omega,\mathcal{G},\prob\right)}
%\newcommand{\E4}{\left(\Omega,\mathcal{G},\mathcal{G}_{t},\prob\right)}
\begin{Def}
Un proceso estoc\'astico $\PE$ definido en $\left(\Omega,\mathcal{G},\prob\right)$ con valores en el espacio topol\'ogico $E$ es continuo por la derecha si cada trayectoria muestral $t\rightarrow X_{t}\left(w\right)$ es un mapeo continuo por la derecha de $I$ en $E$.
\end{Def}

\begin{Def}[HD1]\label{Eq.2.1.S}
Un semigrupo de Markov $\left/P_{t}\right)$ en un espacio de Rad\'on $E$ se dice que satisface la condici\'on {\em HD1} si, dada una medida de probabilidad $\mu$ en $E$, existe una $\sigma$-\'algebra $\mathcal{E^{*}}$ con $\mathcal{E}\subset\mathcal{E}$ y $P_{t}\left(b\mathcal{E}^{*}\right)\subset b\mathcal{E}^{*}$, y un $\mathcal{E}^{*}$-proceso $E$-valuado continuo por la derecha $\PE$ en alg\'un espacio de probabilidad filtrado $\left(\Omega,\mathcal{G},\mathcal{G}_{t},\prob\right)$ tal que $X=\left(\Omega,\mathcal{G},\mathcal{G}_{t},\prob\right)$ es de Markov (Homog\'eneo) con semigrupo de transici\'on $(P_{t})$ y distribuci\'on inicial $\mu$.
\end{Def}

Considerese la colecci\'on de variables aleatorias $X_{t}$ definidas en alg\'un espacio de probabilidad, y una colecci\'on de medidas $\mathbf{P}^{x}$ tales que $\mathbf{P}^{x}\left\{X_{0}=x\right\}$, y bajo cualquier $\mathbf{P}^{x}$, $X_{t}$ es de Markov con semigrupo $\left(P_{t}\right)$. $\mathbf{P}^{x}$ puede considerarse como la distribuci\'on condicional de $\mathbf{P}$ dado $X_{0}=x$.

\begin{Def}\label{Def.2.2.S}
Sea $E$ espacio de Rad\'on, $\SG$ semigrupo de Markov en $\ER$. La colecci\'on $\mathbf{X}=\left(\Omega,\mathcal{G},\mathcal{G}_{t},X_{t},\theta_{t},\CM\right)$ es un proceso $\mathcal{E}$-Markov continuo por la derecha simple, con espacio de estados $E$ y semigrupo de transici\'on $\SG$ en caso de que $\mathbf{X}$ satisfaga las siguientes condiciones:
\begin{itemize}
\item[i)] $\left(\Omega,\mathcal{G},\mathcal{G}_{t}\right)$ es un espacio de medida filtrado, y $X_{t}$ es un proceso $E$-valuado continuo por la derecha $\mathcal{E}^{*}$-adaptado a $\left(\mathcal{G}_{t}\right)$;

\item[ii)] $\left(\theta_{t}\right)_{t\geq0}$ es una colecci\'on de operadores {\em shift} para $X$, es decir, mapea $\Omega$ en s\'i mismo satisfaciendo para $t,s\geq0$,

\begin{equation}\label{Eq.Shift}
\theta_{t}\circ\theta_{s}=\theta_{t+s}\textrm{ y }X_{t}\circ\theta_{t}=X_{t+s};
\end{equation}

\item[iii)] Para cualquier $x\in E$,$\CM\left\{X_{0}=x\right\}=1$, y el proceso $\PE$ tiene la propiedad de Markov (\ref{Eq.1.8.S}) con semigrupo de transici\'on $\SG$ relativo a $\left(\Omega,\mathcal{G},\mathcal{G}_{t},\CM\right)$.
\end{itemize}
\end{Def}

\begin{Def}[HD2]\label{Eq.2.2.S}
Para cualquier $\alpha>0$ y cualquier $f\in S^{\alpha}$, el proceso $t\rightarrow f\left(X_{t}\right)$ es continuo por la derecha casi seguramente.
\end{Def}

\begin{Def}\label{Def.PD}
Un sistema $\mathbf{X}=\left(\Omega,\mathcal{G},\mathcal{G}_{t},X_{t},\theta_{t},\CM\right)$ es un proceso derecho en el espacio de Rad\'on $E$ con semigrupo de transici\'on $\SG$ provisto de:
\begin{itemize}
\item[i)] $\mathbf{X}$ es una realizaci\'on  continua por la derecha, \ref{Def.2.2.S}, de $\SG$.

\item[ii)] $\mathbf{X}$ satisface la condicion HD2, \ref{Eq.2.2.S}, relativa a $\mathcal{G}_{t}$.

\item[iii)] $\mathcal{G}_{t}$ es aumentado y continuo por la derecha.
\end{itemize}
\end{Def}




\begin{Lema}[Lema 4.2, Dai\cite{Dai}]\label{Lema4.2}
Sea $\left\{x_{n}\right\}\subset \mathbf{X}$ con
$|x_{n}|\rightarrow\infty$, conforme $n\rightarrow\infty$. Suponga
que
\[lim_{n\rightarrow\infty}\frac{1}{|x_{n}|}U\left(0\right)=\overline{U}\]
y
\[lim_{n\rightarrow\infty}\frac{1}{|x_{n}|}V\left(0\right)=\overline{V}.\]

Entonces, conforme $n\rightarrow\infty$, casi seguramente

\begin{equation}\label{E1.4.2}
\frac{1}{|x_{n}|}\Phi^{k}\left(\left[|x_{n}|t\right]\right)\rightarrow
P_{k}^{'}t\textrm{, u.o.c.,}
\end{equation}

\begin{equation}\label{E1.4.3}
\frac{1}{|x_{n}|}E^{x_{n}}_{k}\left(|x_{n}|t\right)\rightarrow
\alpha_{k}\left(t-\overline{U}_{k}\right)^{+}\textrm{, u.o.c.,}
\end{equation}

\begin{equation}\label{E1.4.4}
\frac{1}{|x_{n}|}S^{x_{n}}_{k}\left(|x_{n}|t\right)\rightarrow
\mu_{k}\left(t-\overline{V}_{k}\right)^{+}\textrm{, u.o.c.,}
\end{equation}

donde $\left[t\right]$ es la parte entera de $t$ y
$\mu_{k}=1/m_{k}=1/\esp\left[\eta_{k}\left(1\right)\right]$.
\end{Lema}

\begin{Lema}[Lema 4.3, Dai\cite{Dai}]\label{Lema.4.3}
Sea $\left\{x_{n}\right\}\subset \mathbf{X}$ con
$|x_{n}|\rightarrow\infty$, conforme $n\rightarrow\infty$. Suponga
que
\[lim_{n\rightarrow\infty}\frac{1}{|x_{n}|}U\left(0\right)=\overline{U}_{k}\]
y
\[lim_{n\rightarrow\infty}\frac{1}{|x_{n}|}V\left(0\right)=\overline{V}_{k}.\]
\begin{itemize}
\item[a)] Conforme $n\rightarrow\infty$ casi seguramente,
\[lim_{n\rightarrow\infty}\frac{1}{|x_{n}|}U^{x_{n}}_{k}\left(|x_{n}|t\right)=\left(\overline{U}_{k}-t\right)^{+}\textrm{, u.o.c.}\]
y
\[lim_{n\rightarrow\infty}\frac{1}{|x_{n}|}V^{x_{n}}_{k}\left(|x_{n}|t\right)=\left(\overline{V}_{k}-t\right)^{+}.\]

\item[b)] Para cada $t\geq0$ fijo,
\[\left\{\frac{1}{|x_{n}|}U^{x_{n}}_{k}\left(|x_{n}|t\right),|x_{n}|\geq1\right\}\]
y
\[\left\{\frac{1}{|x_{n}|}V^{x_{n}}_{k}\left(|x_{n}|t\right),|x_{n}|\geq1\right\}\]
\end{itemize}
son uniformemente convergentes.
\end{Lema}

$S_{l}^{x}\left(t\right)$ es el n\'umero total de servicios
completados de la clase $l$, si la clase $l$ est\'a dando $t$
unidades de tiempo de servicio. Sea $T_{l}^{x}\left(x\right)$ el
monto acumulado del tiempo de servicio que el servidor
$s\left(l\right)$ gasta en los usuarios de la clase $l$ al tiempo
$t$. Entonces $S_{l}^{x}\left(T_{l}^{x}\left(t\right)\right)$ es
el n\'umero total de servicios completados para la clase $l$ al
tiempo $t$. Una fracci\'on de estos usuarios,
$\Phi_{l}^{x}\left(S_{l}^{x}\left(T_{l}^{x}\left(t\right)\right)\right)$,
se convierte en usuarios de la clase $k$.\\

Entonces, dado lo anterior, se tiene la siguiente representaci\'on
para el proceso de la longitud de la cola:\\

\begin{equation}
Q_{k}^{x}\left(t\right)=_{k}^{x}\left(0\right)+E_{k}^{x}\left(t\right)+\sum_{l=1}^{K}\Phi_{k}^{l}\left(S_{l}^{x}\left(T_{l}^{x}\left(t\right)\right)\right)-S_{k}^{x}\left(T_{k}^{x}\left(t\right)\right)
\end{equation}
para $k=1,\ldots,K$. Para $i=1,\ldots,d$, sea
\[I_{i}^{x}\left(t\right)=t-\sum_{j\in C_{i}}T_{k}^{x}\left(t\right).\]

Entonces $I_{i}^{x}\left(t\right)$ es el monto acumulado del
tiempo que el servidor $i$ ha estado desocupado al tiempo $t$. Se
est\'a asumiendo que las disciplinas satisfacen la ley de
conservaci\'on del trabajo, es decir, el servidor $i$ est\'a en
pausa solamente cuando no hay usuarios en la estaci\'on $i$.
Entonces, se tiene que

\begin{equation}
\int_{0}^{\infty}\left(\sum_{k\in
C_{i}}Q_{k}^{x}\left(t\right)\right)dI_{i}^{x}\left(t\right)=0,
\end{equation}
para $i=1,\ldots,d$.\\

Hacer
\[T^{x}\left(t\right)=\left(T_{1}^{x}\left(t\right),\ldots,T_{K}^{x}\left(t\right)\right)^{'},\]
\[I^{x}\left(t\right)=\left(I_{1}^{x}\left(t\right),\ldots,I_{K}^{x}\left(t\right)\right)^{'}\]
y
\[S^{x}\left(T^{x}\left(t\right)\right)=\left(S_{1}^{x}\left(T_{1}^{x}\left(t\right)\right),\ldots,S_{K}^{x}\left(T_{K}^{x}\left(t\right)\right)\right)^{'}.\]

Para una disciplina que cumple con la ley de conservaci\'on del
trabajo, en forma vectorial, se tiene el siguiente conjunto de
ecuaciones

\begin{equation}\label{Eq.MF.1.3}
Q^{x}\left(t\right)=Q^{x}\left(0\right)+E^{x}\left(t\right)+\sum_{l=1}^{K}\Phi^{l}\left(S_{l}^{x}\left(T_{l}^{x}\left(t\right)\right)\right)-S^{x}\left(T^{x}\left(t\right)\right),\\
\end{equation}

\begin{equation}\label{Eq.MF.2.3}
Q^{x}\left(t\right)\geq0,\\
\end{equation}

\begin{equation}\label{Eq.MF.3.3}
T^{x}\left(0\right)=0,\textrm{ y }\overline{T}^{x}\left(t\right)\textrm{ es no decreciente},\\
\end{equation}

\begin{equation}\label{Eq.MF.4.3}
I^{x}\left(t\right)=et-CT^{x}\left(t\right)\textrm{ es no
decreciente}\\
\end{equation}

\begin{equation}\label{Eq.MF.5.3}
\int_{0}^{\infty}\left(CQ^{x}\left(t\right)\right)dI_{i}^{x}\left(t\right)=0,\\
\end{equation}

\begin{equation}\label{Eq.MF.6.3}
\textrm{Condiciones adicionales en
}\left(\overline{Q}^{x}\left(\cdot\right),\overline{T}^{x}\left(\cdot\right)\right)\textrm{
espec\'ificas de la disciplina de la cola,}
\end{equation}

donde $e$ es un vector de unos de dimensi\'on $d$, $C$ es la
matriz definida por
\[C_{ik}=\left\{\begin{array}{cc}
1,& S\left(k\right)=i,\\
0,& \textrm{ en otro caso}.\\
\end{array}\right.
\]
Es necesario enunciar el siguiente Teorema que se utilizar\'a para
el Teorema \ref{Tma.4.2.Dai}:
\begin{Teo}[Teorema 4.1, Dai \cite{Dai}]
Considere una disciplina que cumpla la ley de conservaci\'on del
trabajo, para casi todas las trayectorias muestrales $\omega$ y
cualquier sucesi\'on de estados iniciales
$\left\{x_{n}\right\}\subset \mathbf{X}$, con
$|x_{n}|\rightarrow\infty$, existe una subsucesi\'on
$\left\{x_{n_{j}}\right\}$ con $|x_{n_{j}}|\rightarrow\infty$ tal
que
\begin{equation}\label{Eq.4.15}
\frac{1}{|x_{n_{j}}|}\left(Q^{x_{n_{j}}}\left(0\right),U^{x_{n_{j}}}\left(0\right),V^{x_{n_{j}}}\left(0\right)\right)\rightarrow\left(\overline{Q}\left(0\right),\overline{U},\overline{V}\right),
\end{equation}

\begin{equation}\label{Eq.4.16}
\frac{1}{|x_{n_{j}}|}\left(Q^{x_{n_{j}}}\left(|x_{n_{j}}|t\right),T^{x_{n_{j}}}\left(|x_{n_{j}}|t\right)\right)\rightarrow\left(\overline{Q}\left(t\right),\overline{T}\left(t\right)\right)\textrm{
u.o.c.}
\end{equation}

Adem\'as,
$\left(\overline{Q}\left(t\right),\overline{T}\left(t\right)\right)$
satisface las siguientes ecuaciones:
\begin{equation}\label{Eq.MF.1.3a}
\overline{Q}\left(t\right)=Q\left(0\right)+\left(\alpha
t-\overline{U}\right)^{+}-\left(I-P\right)^{'}M^{-1}\left(\overline{T}\left(t\right)-\overline{V}\right)^{+},
\end{equation}

\begin{equation}\label{Eq.MF.2.3a}
\overline{Q}\left(t\right)\geq0,\\
\end{equation}

\begin{equation}\label{Eq.MF.3.3a}
\overline{T}\left(t\right)\textrm{ es no decreciente y comienza en cero},\\
\end{equation}

\begin{equation}\label{Eq.MF.4.3a}
\overline{I}\left(t\right)=et-C\overline{T}\left(t\right)\textrm{
es no decreciente,}\\
\end{equation}

\begin{equation}\label{Eq.MF.5.3a}
\int_{0}^{\infty}\left(C\overline{Q}\left(t\right)\right)d\overline{I}\left(t\right)=0,\\
\end{equation}

\begin{equation}\label{Eq.MF.6.3a}
\textrm{Condiciones adicionales en
}\left(\overline{Q}\left(\cdot\right),\overline{T}\left(\cdot\right)\right)\textrm{
especficas de la disciplina de la cola,}
\end{equation}
\end{Teo}

\begin{Def}[Definici\'on 4.1, , Dai \cite{Dai}]
Sea una disciplina de servicio espec\'ifica. Cualquier l\'imite
$\left(\overline{Q}\left(\cdot\right),\overline{T}\left(\cdot\right)\right)$
en \ref{Eq.4.16} es un {\em flujo l\'imite} de la disciplina.
Cualquier soluci\'on (\ref{Eq.MF.1.3a})-(\ref{Eq.MF.6.3a}) es
llamado flujo soluci\'on de la disciplina. Se dice que el modelo de flujo l\'imite, modelo de flujo, de la disciplina de la cola es estable si existe una constante
$\delta>0$ que depende de $\mu,\alpha$ y $P$ solamente, tal que
cualquier flujo l\'imite con
$|\overline{Q}\left(0\right)|+|\overline{U}|+|\overline{V}|=1$, se
tiene que $\overline{Q}\left(\cdot+\delta\right)\equiv0$.
\end{Def}

\begin{Teo}[Teorema 4.2, Dai\cite{Dai}]\label{Tma.4.2.Dai}
Sea una disciplina fija para la cola, suponga que se cumplen las
condiciones (1.2)-(1.5). Si el modelo de flujo l\'imite de la
disciplina de la cola es estable, entonces la cadena de Markov $X$
que describe la din\'amica de la red bajo la disciplina es Harris
recurrente positiva.
\end{Teo}

Ahora se procede a escalar el espacio y el tiempo para reducir la
aparente fluctuaci\'on del modelo. Consid\'erese el proceso
\begin{equation}\label{Eq.3.7}
\overline{Q}^{x}\left(t\right)=\frac{1}{|x|}Q^{x}\left(|x|t\right)
\end{equation}
A este proceso se le conoce como el fluido escalado, y cualquier l\'imite $\overline{Q}^{x}\left(t\right)$ es llamado flujo l\'imite del proceso de longitud de la cola. Haciendo $|q|\rightarrow\infty$ mientras se mantiene el resto de las componentes fijas, cualquier punto l\'imite del proceso de longitud de la cola normalizado $\overline{Q}^{x}$ es soluci\'on del siguiente modelo de flujo.

Al conjunto de ecuaciones dadas en \ref{Eq.3.8}-\ref{Eq.3.13} se
le llama {\em Modelo de flujo} y al conjunto de todas las
soluciones del modelo de flujo
$\left(\overline{Q}\left(\cdot\right),\overline{T}
\left(\cdot\right)\right)$ se le denotar\'a por $\mathcal{Q}$.

Si se hace $|x|\rightarrow\infty$ sin restringir ninguna de las
componentes, tambi\'en se obtienen un modelo de flujo, pero en
este caso el residual de los procesos de arribo y servicio
introducen un retraso:

\begin{Def}[Definici\'on 3.3, Dai y Meyn \cite{DaiSean}]
El modelo de flujo es estable si existe un tiempo fijo $t_{0}$ tal
que $\overline{Q}\left(t\right)=0$, con $t\geq t_{0}$, para
cualquier $\overline{Q}\left(\cdot\right)\in\mathcal{Q}$ que
cumple con $|\overline{Q}\left(0\right)|=1$.
\end{Def}

El siguiente resultado se encuentra en Chen \cite{Chen}.
\begin{Lemma}[Lema 3.1, Dai y Meyn \cite{DaiSean}]
Si el modelo de flujo definido por \ref{Eq.3.8}-\ref{Eq.3.13} es
estable, entonces el modelo de flujo retrasado es tambi\'en
estable, es decir, existe $t_{0}>0$ tal que
$\overline{Q}\left(t\right)=0$ para cualquier $t\geq t_{0}$, para
cualquier soluci\'on del modelo de flujo retrasado cuya
condici\'on inicial $\overline{x}$ satisface que
$|\overline{x}|=|\overline{Q}\left(0\right)|+|\overline{A}\left(0\right)|+|\overline{B}\left(0\right)|\leq1$.
\end{Lemma}


Propiedades importantes para el modelo de flujo retrasado:

\begin{Prop}
 Sea $\left(\overline{Q},\overline{T},\overline{T}^{0}\right)$ un flujo l\'imite de \ref{Eq.4.4} y suponga que cuando $x\rightarrow\infty$ a lo largo de
una subsucesi\'on
\[\left(\frac{1}{|x|}Q_{k}^{x}\left(0\right),\frac{1}{|x|}A_{k}^{x}\left(0\right),\frac{1}{|x|}B_{k}^{x}\left(0\right),\frac{1}{|x|}B_{k}^{x,0}\left(0\right)\right)\rightarrow\left(\overline{Q}_{k}\left(0\right),0,0,0\right)\]
para $k=1,\ldots,K$. EL flujo l\'imite tiene las siguientes
propiedades, donde las propiedades de la derivada se cumplen donde
la derivada exista:
\begin{itemize}
 \item[i)] Los vectores de tiempo ocupado $\overline{T}\left(t\right)$ y $\overline{T}^{0}\left(t\right)$ son crecientes y continuas con
$\overline{T}\left(0\right)=\overline{T}^{0}\left(0\right)=0$.
\item[ii)] Para todo $t\geq0$
\[\sum_{k=1}^{K}\left[\overline{T}_{k}\left(t\right)+\overline{T}_{k}^{0}\left(t\right)\right]=t\]
\item[iii)] Para todo $1\leq k\leq K$
\[\overline{Q}_{k}\left(t\right)=\overline{Q}_{k}\left(0\right)+\alpha_{k}t-\mu_{k}\overline{T}_{k}\left(t\right)\]
\item[iv)]  Para todo $1\leq k\leq K$
\[\dot{{\overline{T}}}_{k}\left(t\right)=\beta_{k}\] para $\overline{Q}_{k}\left(t\right)=0$.
\item[v)] Para todo $k,j$
\[\mu_{k}^{0}\overline{T}_{k}^{0}\left(t\right)=\mu_{j}^{0}\overline{T}_{j}^{0}\left(t\right)\]
\item[vi)]  Para todo $1\leq k\leq K$
\[\mu_{k}\dot{{\overline{T}}}_{k}\left(t\right)=l_{k}\mu_{k}^{0}\dot{{\overline{T}}}_{k}^{0}\left(t\right)\] para $\overline{Q}_{k}\left(t\right)>0$.
\end{itemize}
\end{Prop}

\begin{Lema}[Lema 3.1 \cite{Chen}]\label{Lema3.1}
Si el modelo de flujo es estable, definido por las ecuaciones
(3.8)-(3.13), entonces el modelo de flujo retrasado tambin es
estable.
\end{Lema}

\begin{Teo}[Teorema 5.2 \cite{Chen}]\label{Tma.5.2}
Si el modelo de flujo lineal correspondiente a la red de cola es
estable, entonces la red de colas es estable.
\end{Teo}

\begin{Teo}[Teorema 5.1 \cite{Chen}]\label{Tma.5.1.Chen}
La red de colas es estable si existe una constante $t_{0}$ que
depende de $\left(\alpha,\mu,T,U\right)$ y $V$ que satisfagan las
ecuaciones (5.1)-(5.5), $Z\left(t\right)=0$, para toda $t\geq
t_{0}$.
\end{Teo}



\begin{Lema}[Lema 5.2 \cite{Gut}]\label{Lema.5.2.Gut}
Sea $\left\{\xi\left(k\right):k\in\ent\right\}$ sucesin de
variables aleatorias i.i.d. con valores en
$\left(0,\infty\right)$, y sea $E\left(t\right)$ el proceso de
conteo
\[E\left(t\right)=max\left\{n\geq1:\xi\left(1\right)+\cdots+\xi\left(n-1\right)\leq t\right\}.\]
Si $E\left[\xi\left(1\right)\right]<\infty$, entonces para
cualquier entero $r\geq1$
\begin{equation}
lim_{t\rightarrow\infty}\esp\left[\left(\frac{E\left(t\right)}{t}\right)^{r}\right]=\left(\frac{1}{E\left[\xi_{1}\right]}\right)^{r}
\end{equation}
de aqu, bajo estas condiciones
\begin{itemize}
\item[a)] Para cualquier $t>0$,
$sup_{t\geq\delta}\esp\left[\left(\frac{E\left(t\right)}{t}\right)^{r}\right]$

\item[b)] Las variables aleatorias
$\left\{\left(\frac{E\left(t\right)}{t}\right)^{r}:t\geq1\right\}$
son uniformemente integrables.
\end{itemize}
\end{Lema}

\begin{Teo}[Teorema 5.1: Ley Fuerte para Procesos de Conteo
\cite{Gut}]\label{Tma.5.1.Gut} Sea
$0<\mu<\esp\left(X_{1}\right]\leq\infty$. entonces

\begin{itemize}
\item[a)] $\frac{N\left(t\right)}{t}\rightarrow\frac{1}{\mu}$
a.s., cuando $t\rightarrow\infty$.


\item[b)]$\esp\left[\frac{N\left(t\right)}{t}\right]^{r}\rightarrow\frac{1}{\mu^{r}}$,
cuando $t\rightarrow\infty$ para todo $r>0$..
\end{itemize}
\end{Teo}


\begin{Prop}[Proposicin 5.1 \cite{DaiSean}]\label{Prop.5.1}
Suponga que los supuestos (A1) y (A2) se cumplen, adems suponga
que el modelo de flujo es estable. Entonces existe $t_{0}>0$ tal
que
\begin{equation}\label{Eq.Prop.5.1}
lim_{|x|\rightarrow\infty}\frac{1}{|x|^{p+1}}\esp_{x}\left[|X\left(t_{0}|x|\right)|^{p+1}\right]=0.
\end{equation}

\end{Prop}


\begin{Prop}[Proposici\'on 5.3 \cite{DaiSean}]
Sea $X$ proceso de estados para la red de colas, y suponga que se
cumplen los supuestos (A1) y (A2), entonces para alguna constante
positiva $C_{p+1}<\infty$, $\delta>0$ y un conjunto compacto
$C\subset X$.

\begin{equation}\label{Eq.5.4}
\esp_{x}\left[\int_{0}^{\tau_{C}\left(\delta\right)}\left(1+|X\left(t\right)|^{p}\right)dt\right]\leq
C_{p+1}\left(1+|x|^{p+1}\right)
\end{equation}
\end{Prop}

\begin{Prop}[Proposici\'on 5.4 \cite{DaiSean}]
Sea $X$ un proceso de Markov Borel Derecho en $X$, sea
$f:X\leftarrow\rea_{+}$ y defina para alguna $\delta>0$, y un
conjunto cerrado $C\subset X$
\[V\left(x\right):=\esp_{x}\left[\int_{0}^{\tau_{C}\left(\delta\right)}f\left(X\left(t\right)\right)dt\right]\]
para $x\in X$. Si $V$ es finito en todas partes y uniformemente
acotada en $C$, entonces existe $k<\infty$ tal que
\begin{equation}\label{Eq.5.11}
\frac{1}{t}\esp_{x}\left[V\left(x\right)\right]+\frac{1}{t}\int_{0}^{t}\esp_{x}\left[f\left(X\left(s\right)\right)ds\right]\leq\frac{1}{t}V\left(x\right)+k,
\end{equation}
para $x\in X$ y $t>0$.
\end{Prop}


\begin{Teo}[Teorema 5.5 \cite{DaiSean}]
Suponga que se cumplen (A1) y (A2), adems suponga que el modelo
de flujo es estable. Entonces existe una constante $k_{p}<\infty$
tal que
\begin{equation}\label{Eq.5.13}
\frac{1}{t}\int_{0}^{t}\esp_{x}\left[|Q\left(s\right)|^{p}\right]ds\leq
k_{p}\left\{\frac{1}{t}|x|^{p+1}+1\right\}
\end{equation}
para $t\geq0$, $x\in X$. En particular para cada condici\'on inicial
\begin{equation}\label{Eq.5.14}
Limsup_{t\rightarrow\infty}\frac{1}{t}\int_{0}^{t}\esp_{x}\left[|Q\left(s\right)|^{p}\right]ds\leq
k_{p}
\end{equation}
\end{Teo}

\begin{Teo}[Teorema 6.2\cite{DaiSean}]\label{Tma.6.2}
Suponga que se cumplen los supuestos (A1)-(A3) y que el modelo de
flujo es estable, entonces se tiene que
\[\parallel P^{t}\left(c,\cdot\right)-\pi\left(\cdot\right)\parallel_{f_{p}}\rightarrow0\]
para $t\rightarrow\infty$ y $x\in X$. En particular para cada
condicin inicial
\[lim_{t\rightarrow\infty}\esp_{x}\left[\left|Q_{t}\right|^{p}\right]=\esp_{\pi}\left[\left|Q_{0}\right|^{p}\right]<\infty\]
\end{Teo}


\begin{Teo}[Teorema 6.3\cite{DaiSean}]\label{Tma.6.3}
Suponga que se cumplen los supuestos (A1)-(A3) y que el modelo de
flujo es estable, entonces con
$f\left(x\right)=f_{1}\left(x\right)$, se tiene que
\[lim_{t\rightarrow\infty}t^{(p-1)\left|P^{t}\left(c,\cdot\right)-\pi\left(\cdot\right)\right|_{f}=0},\]
para $x\in X$. En particular, para cada condicin inicial
\[lim_{t\rightarrow\infty}t^{(p-1)\left|\esp_{x}\left[Q_{t}\right]-\esp_{\pi}\left[Q_{0}\right]\right|=0}.\]
\end{Teo}


\begin{Prop}[Proposici\'on 5.1, Dai y Meyn \cite{DaiSean}]\label{Prop.5.1.DaiSean}
Suponga que los supuestos A1) y A2) son ciertos y que el modelo de flujo es estable. Entonces existe $t_{0}>0$ tal que
\begin{equation}
lim_{|x|\rightarrow\infty}\frac{1}{|x|^{p+1}}\esp_{x}\left[|X\left(t_{0}|x|\right)|^{p+1}\right]=0
\end{equation}
\end{Prop}

\begin{Lemma}[Lema 5.2, Dai y Meyn \cite{DaiSean}]\label{Lema.5.2.DaiSean}
 Sea $\left\{\zeta\left(k\right):k\in \mathbb{z}\right\}$ una sucesi\'on independiente e id\'enticamente distribuida que toma valores en $\left(0,\infty\right)$,
y sea
$E\left(t\right)=max\left(n\geq1:\zeta\left(1\right)+\cdots+\zeta\left(n-1\right)\leq
t\right)$. Si $\esp\left[\zeta\left(1\right)\right]<\infty$,
entonces para cualquier entero $r\geq1$
\begin{equation}
 lim_{t\rightarrow\infty}\esp\left[\left(\frac{E\left(t\right)}{t}\right)^{r}\right]=\left(\frac{1}{\esp\left[\zeta_{1}\right]}\right)^{r}.
\end{equation}
Luego, bajo estas condiciones:
\begin{itemize}
 \item[a)] para cualquier $\delta>0$, $\sup_{t\geq\delta}\esp\left[\left(\frac{E\left(t\right)}{t}\right)^{r}\right]<\infty$
\item[b)] las variables aleatorias
$\left\{\left(\frac{E\left(t\right)}{t}\right)^{r}:t\geq1\right\}$
son uniformemente integrables.
\end{itemize}
\end{Lemma}

\begin{Teo}[Teorema 5.5, Dai y Meyn \cite{DaiSean}]\label{Tma.5.5.DaiSean}
Suponga que los supuestos A1) y A2) se cumplen y que el modelo de
flujo es estable. Entonces existe una constante $\kappa_{p}$ tal
que
\begin{equation}
\frac{1}{t}\int_{0}^{t}\esp_{x}\left[|Q\left(s\right)|^{p}\right]ds\leq\kappa_{p}\left\{\frac{1}{t}|x|^{p+1}+1\right\}
\end{equation}
para $t>0$ y $x\in X$. En particular, para cada condici\'on
inicial
\begin{eqnarray*}
\limsup_{t\rightarrow\infty}\frac{1}{t}\int_{0}^{t}\esp_{x}\left[|Q\left(s\right)|^{p}\right]ds\leq\kappa_{p}.
\end{eqnarray*}
\end{Teo}

\begin{Teo}[Teorema 6.2, Dai y Meyn \cite{DaiSean}]\label{Tma.6.2.DaiSean}
Suponga que se cumplen los supuestos A1), A2) y A3) y que el
modelo de flujo es estable. Entonces se tiene que
\begin{equation}
\left\|P^{t}\left(x,\cdot\right)-\pi\left(\cdot\right)\right\|_{f_{p}}\textrm{,
}t\rightarrow\infty,x\in X.
\end{equation}
En particular para cada condici\'on inicial
\begin{eqnarray*}
\lim_{t\rightarrow\infty}\esp_{x}\left[|Q\left(t\right)|^{p}\right]=\esp_{\pi}\left[|Q\left(0\right)|^{p}\right]\leq\kappa_{r}
\end{eqnarray*}
\end{Teo}
\begin{Teo}[Teorema 6.3, Dai y Meyn \cite{DaiSean}]\label{Tma.6.3.DaiSean}
Suponga que se cumplen los supuestos A1), A2) y A3) y que el
modelo de flujo es estable. Entonces con
$f\left(x\right)=f_{1}\left(x\right)$ se tiene
\begin{equation}
\lim_{t\rightarrow\infty}t^{p-1}\left\|P^{t}\left(x,\cdot\right)-\pi\left(\cdot\right)\right\|_{f}=0.
\end{equation}
En particular para cada condici\'on inicial
\begin{eqnarray*}
\lim_{t\rightarrow\infty}t^{p-1}|\esp_{x}\left[Q\left(t\right)\right]-\esp_{\pi}\left[Q\left(0\right)\right]|=0.
\end{eqnarray*}
\end{Teo}

\begin{Teo}[Teorema 6.4, Dai y Meyn \cite{DaiSean}]\label{Tma.6.4.DaiSean}
Suponga que se cumplen los supuestos A1), A2) y A3) y que el
modelo de flujo es estable. Sea $\nu$ cualquier distribuci\'on de
probabilidad en $\left(X,\mathcal{B}_{X}\right)$, y $\pi$ la
distribuci\'on estacionaria de $X$.
\begin{itemize}
\item[i)] Para cualquier $f:X\leftarrow\rea_{+}$
\begin{equation}
\lim_{t\rightarrow\infty}\frac{1}{t}\int_{o}^{t}f\left(X\left(s\right)\right)ds=\pi\left(f\right):=\int
f\left(x\right)\pi\left(dx\right)
\end{equation}
$\prob$-c.s.

\item[ii)] Para cualquier $f:X\leftarrow\rea_{+}$ con
$\pi\left(|f|\right)<\infty$, la ecuaci\'on anterior se cumple.
\end{itemize}
\end{Teo}

\begin{Teo}[Teorema 2.2, Down \cite{Down}]\label{Tma2.2.Down}
Suponga que el fluido modelo es inestable en el sentido de que
para alguna $\epsilon_{0},c_{0}\geq0$,
\begin{equation}\label{Eq.Inestability}
|Q\left(T\right)|\geq\epsilon_{0}T-c_{0}\textrm{,   }T\geq0,
\end{equation}
para cualquier condici\'on inicial $Q\left(0\right)$, con
$|Q\left(0\right)|=1$. Entonces para cualquier $0<q\leq1$, existe
$B<0$ tal que para cualquier $|x|\geq B$,
\begin{equation}
\prob_{x}\left\{\mathbb{X}\rightarrow\infty\right\}\geq q.
\end{equation}
\end{Teo}



Es necesario hacer los siguientes supuestos sobre el
comportamiento del sistema de visitas c\'iclicas:
\begin{itemize}
\item Los tiempos de interarribo a la $k$-\'esima cola, son de la
forma $\left\{\xi_{k}\left(n\right)\right\}_{n\geq1}$, con la
propiedad de que son independientes e id{\'e}nticamente
distribuidos,
\item Los tiempos de servicio
$\left\{\eta_{k}\left(n\right)\right\}_{n\geq1}$ tienen la
propiedad de ser independientes e id{\'e}nticamente distribuidos,
\item Se define la tasa de arribo a la $k$-{\'e}sima cola como
$\lambda_{k}=1/\esp\left[\xi_{k}\left(1\right)\right]$,
\item la tasa de servicio para la $k$-{\'e}sima cola se define
como $\mu_{k}=1/\esp\left[\eta_{k}\left(1\right)\right]$,
\item tambi{\'e}n se define $\rho_{k}:=\lambda_{k}/\mu_{k}$, la
intensidad de tr\'afico del sistema o carga de la red, donde es
necesario que $\rho<1$ para cuestiones de estabilidad.
\end{itemize}



%_________________________________________________________________________
\subsection{Procesos Fuerte de Markov}
%_________________________________________________________________________
En Dai \cite{Dai} se muestra que para una amplia serie de disciplinas
de servicio el proceso $X$ es un Proceso Fuerte de
Markov, y por tanto se puede asumir que


Para establecer que $X=\left\{X\left(t\right),t\geq0\right\}$ es
un Proceso Fuerte de Markov, se siguen las secciones 2.3 y 2.4 de Kaspi and Mandelbaum \cite{KaspiMandelbaum}. \\

%______________________________________________________________
\subsubsection{Construcci\'on de un Proceso Determinista por partes, Davis
\cite{Davis}}.
%______________________________________________________________

%_________________________________________________________________________
\subsection{Procesos Harris Recurrentes Positivos}
%_________________________________________________________________________
Sea el proceso de Markov $X=\left\{X\left(t\right),t\geq0\right\}$
que describe la din\'amica de la red de colas. En lo que respecta
al supuesto (A3), en Dai y Meyn \cite{DaiSean} y Meyn y Down
\cite{MeynDown} hacen ver que este se puede sustituir por

\begin{itemize}
\item[A3')] Para el Proceso de Markov $X$, cada subconjunto
compacto de $X$ es un conjunto peque\~no.
\end{itemize}

Este supuesto es importante pues es un requisito para deducir la ergodicidad de la red.

%_________________________________________________________________________
\subsection{Construcci\'on de un Modelo de Flujo L\'imite}
%_________________________________________________________________________

Consideremos un caso m\'as simple para poner en contexto lo
anterior: para un sistema de visitas c\'iclicas se tiene que el
estado al tiempo $t$ es
\begin{equation}
X\left(t\right)=\left(Q\left(t\right),U\left(t\right),V\left(t\right)\right),
\end{equation}

donde $Q\left(t\right)$ es el n\'umero de usuarios formados en
cada estaci\'on. $U\left(t\right)$ es el tiempo restante antes de
que la siguiente clase $k$ de usuarios lleguen desde fuera del
sistema, $V\left(t\right)$ es el tiempo restante de servicio para
la clase $k$ de usuarios que est\'an siendo atendidos. Tanto
$U\left(t\right)$ como $V\left(t\right)$ se puede asumir que son
continuas por la derecha.

Sea
$x=\left(Q\left(0\right),U\left(0\right),V\left(0\right)\right)=\left(q,a,b\right)$,
el estado inicial de la red bajo una disciplina espec\'ifica para
la cola. Para $l\in\mathcal{E}$, donde $\mathcal{E}$ es el conjunto de clases de arribos externos, y $k=1,\ldots,K$ se define\\
\begin{eqnarray*}
E_{l}^{x}\left(t\right)&=&max\left\{r:U_{l}\left(0\right)+\xi_{l}\left(1\right)+\cdots+\xi_{l}\left(r-1\right)\leq
t\right\}\textrm{   }t\geq0,\\
S_{k}^{x}\left(t\right)&=&max\left\{r:V_{k}\left(0\right)+\eta_{k}\left(1\right)+\cdots+\eta_{k}\left(r-1\right)\leq
t\right\}\textrm{   }t\geq0.
\end{eqnarray*}

Para cada $k$ y cada $n$ se define

\begin{eqnarray*}\label{Eq.phi}
\Phi^{k}\left(n\right):=\sum_{i=1}^{n}\phi^{k}\left(i\right).
\end{eqnarray*}

donde $\phi^{k}\left(n\right)$ se define como el vector de ruta
para el $n$-\'esimo usuario de la clase $k$ que termina en la
estaci\'on $s\left(k\right)$, la $s$-\'eima componente de
$\phi^{k}\left(n\right)$ es uno si estos usuarios se convierten en
usuarios de la clase $l$ y cero en otro caso, por lo tanto
$\phi^{k}\left(n\right)$ es un vector {\em Bernoulli} de
dimensi\'on $K$ con par\'ametro $P_{k}^{'}$, donde $P_{k}$ denota
el $k$-\'esimo rengl\'on de $P=\left(P_{kl}\right)$.

Se asume que cada para cada $k$ la sucesi\'on $\phi^{k}\left(n\right)=\left\{\phi^{k}\left(n\right),n\geq1\right\}$
es independiente e id\'enticamente distribuida y que las
$\phi^{1}\left(n\right),\ldots,\phi^{K}\left(n\right)$ son
mutuamente independientes, adem\'as de independientes de los
procesos de arribo y de servicio.\\

\begin{Lema}[Lema 4.2, Dai\cite{Dai}]\label{Lema4.2}
Sea $\left\{x_{n}\right\}\subset \mathbf{X}$ con
$|x_{n}|\rightarrow\infty$, conforme $n\rightarrow\infty$. Suponga
que
\[lim_{n\rightarrow\infty}\frac{1}{|x_{n}|}U\left(0\right)=\overline{U}\]
y
\[lim_{n\rightarrow\infty}\frac{1}{|x_{n}|}V\left(0\right)=\overline{V}.\]

Entonces, conforme $n\rightarrow\infty$, casi seguramente

\begin{equation}\label{E1.4.2}
\frac{1}{|x_{n}|}\Phi^{k}\left(\left[|x_{n}|t\right]\right)\rightarrow
P_{k}^{'}t\textrm{, u.o.c.,}
\end{equation}

\begin{equation}\label{E1.4.3}
\frac{1}{|x_{n}|}E^{x_{n}}_{k}\left(|x_{n}|t\right)\rightarrow
\alpha_{k}\left(t-\overline{U}_{k}\right)^{+}\textrm{, u.o.c.,}
\end{equation}

\begin{equation}\label{E1.4.4}
\frac{1}{|x_{n}|}S^{x_{n}}_{k}\left(|x_{n}|t\right)\rightarrow
\mu_{k}\left(t-\overline{V}_{k}\right)^{+}\textrm{, u.o.c.,}
\end{equation}

donde $\left[t\right]$ es la parte entera de $t$ y
$\mu_{k}=1/m_{k}=1/\esp\left[\eta_{k}\left(1\right)\right]$.
\end{Lema}

\begin{Lema}[Lema 4.3, Dai\cite{Dai}]\label{Lema.4.3}
Sea $\left\{x_{n}\right\}\subset \mathbf{X}$ con
$|x_{n}|\rightarrow\infty$, conforme $n\rightarrow\infty$. Suponga
que
\[lim_{n\rightarrow\infty}\frac{1}{|x_{n}|}U\left(0\right)=\overline{U}_{k}\]
y
\[lim_{n\rightarrow\infty}\frac{1}{|x_{n}|}V\left(0\right)=\overline{V}_{k}.\]
\begin{itemize}
\item[a)] Conforme $n\rightarrow\infty$ casi seguramente,
\[lim_{n\rightarrow\infty}\frac{1}{|x_{n}|}U^{x_{n}}_{k}\left(|x_{n}|t\right)=\left(\overline{U}_{k}-t\right)^{+}\textrm{, u.o.c.}\]
y
\[lim_{n\rightarrow\infty}\frac{1}{|x_{n}|}V^{x_{n}}_{k}\left(|x_{n}|t\right)=\left(\overline{V}_{k}-t\right)^{+}.\]

\item[b)] Para cada $t\geq0$ fijo,
\[\left\{\frac{1}{|x_{n}|}U^{x_{n}}_{k}\left(|x_{n}|t\right),|x_{n}|\geq1\right\}\]
y
\[\left\{\frac{1}{|x_{n}|}V^{x_{n}}_{k}\left(|x_{n}|t\right),|x_{n}|\geq1\right\}\]
\end{itemize}
son uniformemente convergentes.
\end{Lema}

$S_{l}^{x}\left(t\right)$ es el n\'umero total de servicios
completados de la clase $l$, si la clase $l$ est\'a dando $t$
unidades de tiempo de servicio. Sea $T_{l}^{x}\left(x\right)$ el
monto acumulado del tiempo de servicio que el servidor
$s\left(l\right)$ gasta en los usuarios de la clase $l$ al tiempo
$t$. Entonces $S_{l}^{x}\left(T_{l}^{x}\left(t\right)\right)$ es
el n\'umero total de servicios completados para la clase $l$ al
tiempo $t$. Una fracci\'on de estos usuarios,
$\Phi_{l}^{x}\left(S_{l}^{x}\left(T_{l}^{x}\left(t\right)\right)\right)$,
se convierte en usuarios de la clase $k$.\\

Entonces, dado lo anterior, se tiene la siguiente representaci\'on
para el proceso de la longitud de la cola:\\

\begin{equation}
Q_{k}^{x}\left(t\right)=_{k}^{x}\left(0\right)+E_{k}^{x}\left(t\right)+\sum_{l=1}^{K}\Phi_{k}^{l}\left(S_{l}^{x}\left(T_{l}^{x}\left(t\right)\right)\right)-S_{k}^{x}\left(T_{k}^{x}\left(t\right)\right)
\end{equation}
para $k=1,\ldots,K$. Para $i=1,\ldots,d$, sea
\[I_{i}^{x}\left(t\right)=t-\sum_{j\in C_{i}}T_{k}^{x}\left(t\right).\]

Entonces $I_{i}^{x}\left(t\right)$ es el monto acumulado del
tiempo que el servidor $i$ ha estado desocupado al tiempo $t$. Se
est\'a asumiendo que las disciplinas satisfacen la ley de
conservaci\'on del trabajo, es decir, el servidor $i$ est\'a en
pausa solamente cuando no hay usuarios en la estaci\'on $i$.
Entonces, se tiene que

\begin{equation}
\int_{0}^{\infty}\left(\sum_{k\in
C_{i}}Q_{k}^{x}\left(t\right)\right)dI_{i}^{x}\left(t\right)=0,
\end{equation}
para $i=1,\ldots,d$.\\

Hacer
\[T^{x}\left(t\right)=\left(T_{1}^{x}\left(t\right),\ldots,T_{K}^{x}\left(t\right)\right)^{'},\]
\[I^{x}\left(t\right)=\left(I_{1}^{x}\left(t\right),\ldots,I_{K}^{x}\left(t\right)\right)^{'}\]
y
\[S^{x}\left(T^{x}\left(t\right)\right)=\left(S_{1}^{x}\left(T_{1}^{x}\left(t\right)\right),\ldots,S_{K}^{x}\left(T_{K}^{x}\left(t\right)\right)\right)^{'}.\]

Para una disciplina que cumple con la ley de conservaci\'on del
trabajo, en forma vectorial, se tiene el siguiente conjunto de
ecuaciones

\begin{equation}\label{Eq.MF.1.3}
Q^{x}\left(t\right)=Q^{x}\left(0\right)+E^{x}\left(t\right)+\sum_{l=1}^{K}\Phi^{l}\left(S_{l}^{x}\left(T_{l}^{x}\left(t\right)\right)\right)-S^{x}\left(T^{x}\left(t\right)\right),\\
\end{equation}

\begin{equation}\label{Eq.MF.2.3}
Q^{x}\left(t\right)\geq0,\\
\end{equation}

\begin{equation}\label{Eq.MF.3.3}
T^{x}\left(0\right)=0,\textrm{ y }\overline{T}^{x}\left(t\right)\textrm{ es no decreciente},\\
\end{equation}

\begin{equation}\label{Eq.MF.4.3}
I^{x}\left(t\right)=et-CT^{x}\left(t\right)\textrm{ es no
decreciente}\\
\end{equation}

\begin{equation}\label{Eq.MF.5.3}
\int_{0}^{\infty}\left(CQ^{x}\left(t\right)\right)dI_{i}^{x}\left(t\right)=0,\\
\end{equation}

\begin{equation}\label{Eq.MF.6.3}
\textrm{Condiciones adicionales en
}\left(\overline{Q}^{x}\left(\cdot\right),\overline{T}^{x}\left(\cdot\right)\right)\textrm{
espec\'ificas de la disciplina de la cola,}
\end{equation}

donde $e$ es un vector de unos de dimensi\'on $d$, $C$ es la
matriz definida por
\[C_{ik}=\left\{\begin{array}{cc}
1,& S\left(k\right)=i,\\
0,& \textrm{ en otro caso}.\\
\end{array}\right.
\]
Es necesario enunciar el siguiente Teorema que se utilizar\'a para
el Teorema \ref{Tma.4.2.Dai}:
\begin{Teo}[Teorema 4.1, Dai \cite{Dai}]
Considere una disciplina que cumpla la ley de conservaci\'on del
trabajo, para casi todas las trayectorias muestrales $\omega$ y
cualquier sucesi\'on de estados iniciales
$\left\{x_{n}\right\}\subset \mathbf{X}$, con
$|x_{n}|\rightarrow\infty$, existe una subsucesi\'on
$\left\{x_{n_{j}}\right\}$ con $|x_{n_{j}}|\rightarrow\infty$ tal
que
\begin{equation}\label{Eq.4.15}
\frac{1}{|x_{n_{j}}|}\left(Q^{x_{n_{j}}}\left(0\right),U^{x_{n_{j}}}\left(0\right),V^{x_{n_{j}}}\left(0\right)\right)\rightarrow\left(\overline{Q}\left(0\right),\overline{U},\overline{V}\right),
\end{equation}

\begin{equation}\label{Eq.4.16}
\frac{1}{|x_{n_{j}}|}\left(Q^{x_{n_{j}}}\left(|x_{n_{j}}|t\right),T^{x_{n_{j}}}\left(|x_{n_{j}}|t\right)\right)\rightarrow\left(\overline{Q}\left(t\right),\overline{T}\left(t\right)\right)\textrm{
u.o.c.}
\end{equation}

Adem\'as,
$\left(\overline{Q}\left(t\right),\overline{T}\left(t\right)\right)$
satisface las siguientes ecuaciones:
\begin{equation}\label{Eq.MF.1.3a}
\overline{Q}\left(t\right)=Q\left(0\right)+\left(\alpha
t-\overline{U}\right)^{+}-\left(I-P\right)^{'}M^{-1}\left(\overline{T}\left(t\right)-\overline{V}\right)^{+},
\end{equation}

\begin{equation}\label{Eq.MF.2.3a}
\overline{Q}\left(t\right)\geq0,\\
\end{equation}

\begin{equation}\label{Eq.MF.3.3a}
\overline{T}\left(t\right)\textrm{ es no decreciente y comienza en cero},\\
\end{equation}

\begin{equation}\label{Eq.MF.4.3a}
\overline{I}\left(t\right)=et-C\overline{T}\left(t\right)\textrm{
es no decreciente,}\\
\end{equation}

\begin{equation}\label{Eq.MF.5.3a}
\int_{0}^{\infty}\left(C\overline{Q}\left(t\right)\right)d\overline{I}\left(t\right)=0,\\
\end{equation}

\begin{equation}\label{Eq.MF.6.3a}
\textrm{Condiciones adicionales en
}\left(\overline{Q}\left(\cdot\right),\overline{T}\left(\cdot\right)\right)\textrm{
especficas de la disciplina de la cola,}
\end{equation}
\end{Teo}

\begin{Def}[Definici\'on 4.1, , Dai \cite{Dai}]
Sea una disciplina de servicio espec\'ifica. Cualquier l\'imite
$\left(\overline{Q}\left(\cdot\right),\overline{T}\left(\cdot\right)\right)$
en \ref{Eq.4.16} es un {\em flujo l\'imite} de la disciplina.
Cualquier soluci\'on (\ref{Eq.MF.1.3a})-(\ref{Eq.MF.6.3a}) es
llamado flujo soluci\'on de la disciplina. Se dice que el modelo de flujo l\'imite, modelo de flujo, de la disciplina de la cola es estable si existe una constante
$\delta>0$ que depende de $\mu,\alpha$ y $P$ solamente, tal que
cualquier flujo l\'imite con
$|\overline{Q}\left(0\right)|+|\overline{U}|+|\overline{V}|=1$, se
tiene que $\overline{Q}\left(\cdot+\delta\right)\equiv0$.
\end{Def}

\begin{Teo}[Teorema 4.2, Dai\cite{Dai}]\label{Tma.4.2.Dai}
Sea una disciplina fija para la cola, suponga que se cumplen las
condiciones (1.2)-(1.5). Si el modelo de flujo l\'imite de la
disciplina de la cola es estable, entonces la cadena de Markov $X$
que describe la din\'amica de la red bajo la disciplina es Harris
recurrente positiva.
\end{Teo}

Ahora se procede a escalar el espacio y el tiempo para reducir la
aparente fluctuaci\'on del modelo. Consid\'erese el proceso
\begin{equation}\label{Eq.3.7}
\overline{Q}^{x}\left(t\right)=\frac{1}{|x|}Q^{x}\left(|x|t\right)
\end{equation}
A este proceso se le conoce como el fluido escalado, y cualquier l\'imite $\overline{Q}^{x}\left(t\right)$ es llamado flujo l\'imite del proceso de longitud de la cola. Haciendo $|q|\rightarrow\infty$ mientras se mantiene el resto de las componentes fijas, cualquier punto l\'imite del proceso de longitud de la cola normalizado $\overline{Q}^{x}$ es soluci\'on del siguiente modelo de flujo.

\begin{Def}[Definici\'on 3.1, Dai y Meyn \cite{DaiSean}]
Un flujo l\'imite (retrasado) para una red bajo una disciplina de
servicio espec\'ifica se define como cualquier soluci\'on
 $\left(\overline{Q}\left(\cdot\right),\overline{T}\left(\cdot\right)\right)$ de las siguientes ecuaciones, donde
$\overline{Q}\left(t\right)=\left(\overline{Q}_{1}\left(t\right),\ldots,\overline{Q}_{K}\left(t\right)\right)^{'}$
y
$\overline{T}\left(t\right)=\left(\overline{T}_{1}\left(t\right),\ldots,\overline{T}_{K}\left(t\right)\right)^{'}$
\begin{equation}\label{Eq.3.8}
\overline{Q}_{k}\left(t\right)=\overline{Q}_{k}\left(0\right)+\alpha_{k}t-\mu_{k}\overline{T}_{k}\left(t\right)+\sum_{l=1}^{k}P_{lk}\mu_{l}\overline{T}_{l}\left(t\right)\\
\end{equation}
\begin{equation}\label{Eq.3.9}
\overline{Q}_{k}\left(t\right)\geq0\textrm{ para }k=1,2,\ldots,K,\\
\end{equation}
\begin{equation}\label{Eq.3.10}
\overline{T}_{k}\left(0\right)=0,\textrm{ y }\overline{T}_{k}\left(\cdot\right)\textrm{ es no decreciente},\\
\end{equation}
\begin{equation}\label{Eq.3.11}
\overline{I}_{i}\left(t\right)=t-\sum_{k\in C_{i}}\overline{T}_{k}\left(t\right)\textrm{ es no decreciente}\\
\end{equation}
\begin{equation}\label{Eq.3.12}
\overline{I}_{i}\left(\cdot\right)\textrm{ se incrementa al tiempo }t\textrm{ cuando }\sum_{k\in C_{i}}Q_{k}^{x}\left(t\right)dI_{i}^{x}\left(t\right)=0\\
\end{equation}
\begin{equation}\label{Eq.3.13}
\textrm{condiciones adicionales sobre
}\left(Q^{x}\left(\cdot\right),T^{x}\left(\cdot\right)\right)\textrm{
referentes a la disciplina de servicio}
\end{equation}
\end{Def}

Al conjunto de ecuaciones dadas en \ref{Eq.3.8}-\ref{Eq.3.13} se
le llama {\em Modelo de flujo} y al conjunto de todas las
soluciones del modelo de flujo
$\left(\overline{Q}\left(\cdot\right),\overline{T}
\left(\cdot\right)\right)$ se le denotar\'a por $\mathcal{Q}$.

Si se hace $|x|\rightarrow\infty$ sin restringir ninguna de las
componentes, tambi\'en se obtienen un modelo de flujo, pero en
este caso el residual de los procesos de arribo y servicio
introducen un retraso:

\begin{Def}[Definici\'on 3.2, Dai y Meyn \cite{DaiSean}]
El modelo de flujo retrasado de una disciplina de servicio en una
red con retraso
$\left(\overline{A}\left(0\right),\overline{B}\left(0\right)\right)\in\rea_{+}^{K+|A|}$
se define como el conjunto de ecuaciones dadas en
\ref{Eq.3.8}-\ref{Eq.3.13}, junto con la condici\'on:
\begin{equation}\label{CondAd.FluidModel}
\overline{Q}\left(t\right)=\overline{Q}\left(0\right)+\left(\alpha
t-\overline{A}\left(0\right)\right)^{+}-\left(I-P^{'}\right)M\left(\overline{T}\left(t\right)-\overline{B}\left(0\right)\right)^{+}
\end{equation}
\end{Def}

\begin{Def}[Definici\'on 3.3, Dai y Meyn \cite{DaiSean}]
El modelo de flujo es estable si existe un tiempo fijo $t_{0}$ tal
que $\overline{Q}\left(t\right)=0$, con $t\geq t_{0}$, para
cualquier $\overline{Q}\left(\cdot\right)\in\mathcal{Q}$ que
cumple con $|\overline{Q}\left(0\right)|=1$.
\end{Def}

El siguiente resultado se encuentra en Chen \cite{Chen}.
\begin{Lemma}[Lema 3.1, Dai y Meyn \cite{DaiSean}]
Si el modelo de flujo definido por \ref{Eq.3.8}-\ref{Eq.3.13} es
estable, entonces el modelo de flujo retrasado es tambi\'en
estable, es decir, existe $t_{0}>0$ tal que
$\overline{Q}\left(t\right)=0$ para cualquier $t\geq t_{0}$, para
cualquier soluci\'on del modelo de flujo retrasado cuya
condici\'on inicial $\overline{x}$ satisface que
$|\overline{x}|=|\overline{Q}\left(0\right)|+|\overline{A}\left(0\right)|+|\overline{B}\left(0\right)|\leq1$.
\end{Lemma}

%_________________________________________________________________________
\subsection{Modelo de Visitas C\'iclicas con un Servidor: Estabilidad}
%_________________________________________________________________________

%_________________________________________________________________________
\subsection{Teorema 2.1}
%_________________________________________________________________________



El resultado principal de Down \cite{Down} que relaciona la estabilidad del modelo de flujo con la estabilidad del sistema original

\begin{Teo}[Teorema 2.1, Down \cite{Down}]\label{Tma.2.1.Down}
Suponga que el modelo de flujo es estable, y que se cumplen los supuestos (A1) y (A2), entonces
\begin{itemize}
\item[i)] Para alguna constante $\kappa_{p}$, y para cada
condici\'on inicial $x\in X$
\begin{equation}\label{Estability.Eq1}
lim_{t\rightarrow\infty}\sup\frac{1}{t}\int_{0}^{t}\esp_{x}\left[|Q\left(s\right)|^{p}\right]ds\leq\kappa_{p},
\end{equation}
donde $p$ es el entero dado en (A2). Si adem\'as se cumple
la condici\'on (A3), entonces para cada condici\'on inicial:

\item[ii)] Los momentos transitorios convergen a su estado estacionario:
 \begin{equation}\label{Estability.Eq2}
lim_{t\rightarrow\infty}\esp_{x}\left[Q_{k}\left(t\right)^{r}\right]=\esp_{\pi}\left[Q_{k}\left(0\right)^{r}\right]\leq\kappa_{r},
\end{equation}
para $r=1,2,\ldots,p$ y $k=1,2,\ldots,K$. Donde $\pi$ es la
probabilidad invariante para $\mathbf{X}$.

\item[iii)]  El primer momento converge con raz\'on $t^{p-1}$:
\begin{equation}\label{Estability.Eq3}
lim_{t\rightarrow\infty}t^{p-1}|\esp_{x}\left[Q_{k}\left(t\right)\right]-\esp_{\pi}\left[Q\left(0\right)\right]=0.
\end{equation}

\item[iv)] La {\em Ley Fuerte de los grandes n\'umeros} se cumple:
\begin{equation}\label{Estability.Eq4}
lim_{t\rightarrow\infty}\frac{1}{t}\int_{0}^{t}Q_{k}^{r}\left(s\right)ds=\esp_{\pi}\left[Q_{k}\left(0\right)^{r}\right],\textrm{
}\prob_{x}\textrm{-c.s.}
\end{equation}
para $r=1,2,\ldots,p$ y $k=1,2,\ldots,K$.
\end{itemize}
\end{Teo}


\begin{Prop}[Proposici\'on 5.1, Dai y Meyn \cite{DaiSean}]\label{Prop.5.1.DaiSean}
Suponga que los supuestos A1) y A2) son ciertos y que el modelo de flujo es estable. Entonces existe $t_{0}>0$ tal que
\begin{equation}
lim_{|x|\rightarrow\infty}\frac{1}{|x|^{p+1}}\esp_{x}\left[|X\left(t_{0}|x|\right)|^{p+1}\right]=0
\end{equation}
\end{Prop}

\begin{Lemma}[Lema 5.2, Dai y Meyn \cite{DaiSean}]\label{Lema.5.2.DaiSean}
 Sea $\left\{\zeta\left(k\right):k\in \mathbb{z}\right\}$ una sucesi\'on independiente e id\'enticamente distribuida que toma valores en $\left(0,\infty\right)$,
y sea
$E\left(t\right)=max\left(n\geq1:\zeta\left(1\right)+\cdots+\zeta\left(n-1\right)\leq
t\right)$. Si $\esp\left[\zeta\left(1\right)\right]<\infty$,
entonces para cualquier entero $r\geq1$
\begin{equation}
 lim_{t\rightarrow\infty}\esp\left[\left(\frac{E\left(t\right)}{t}\right)^{r}\right]=\left(\frac{1}{\esp\left[\zeta_{1}\right]}\right)^{r}.
\end{equation}
Luego, bajo estas condiciones:
\begin{itemize}
 \item[a)] para cualquier $\delta>0$, $\sup_{t\geq\delta}\esp\left[\left(\frac{E\left(t\right)}{t}\right)^{r}\right]<\infty$
\item[b)] las variables aleatorias
$\left\{\left(\frac{E\left(t\right)}{t}\right)^{r}:t\geq1\right\}$
son uniformemente integrables.
\end{itemize}
\end{Lemma}

\begin{Teo}[Teorema 5.5, Dai y Meyn \cite{DaiSean}]\label{Tma.5.5.DaiSean}
Suponga que los supuestos A1) y A2) se cumplen y que el modelo de
flujo es estable. Entonces existe una constante $\kappa_{p}$ tal
que
\begin{equation}
\frac{1}{t}\int_{0}^{t}\esp_{x}\left[|Q\left(s\right)|^{p}\right]ds\leq\kappa_{p}\left\{\frac{1}{t}|x|^{p+1}+1\right\}
\end{equation}
para $t>0$ y $x\in X$. En particular, para cada condici\'on
inicial
\begin{eqnarray*}
\limsup_{t\rightarrow\infty}\frac{1}{t}\int_{0}^{t}\esp_{x}\left[|Q\left(s\right)|^{p}\right]ds\leq\kappa_{p}.
\end{eqnarray*}
\end{Teo}

\begin{Teo}[Teorema 6.2, Dai y Meyn \cite{DaiSean}]\label{Tma.6.2.DaiSean}
Suponga que se cumplen los supuestos A1), A2) y A3) y que el
modelo de flujo es estable. Entonces se tiene que
\begin{equation}
\left\|P^{t}\left(x,\cdot\right)-\pi\left(\cdot\right)\right\|_{f_{p}}\textrm{,
}t\rightarrow\infty,x\in X.
\end{equation}
En particular para cada condici\'on inicial
\begin{eqnarray*}
\lim_{t\rightarrow\infty}\esp_{x}\left[|Q\left(t\right)|^{p}\right]=\esp_{\pi}\left[|Q\left(0\right)|^{p}\right]\leq\kappa_{r}
\end{eqnarray*}
\end{Teo}
\begin{Teo}[Teorema 6.3, Dai y Meyn \cite{DaiSean}]\label{Tma.6.3.DaiSean}
Suponga que se cumplen los supuestos A1), A2) y A3) y que el
modelo de flujo es estable. Entonces con
$f\left(x\right)=f_{1}\left(x\right)$ se tiene
\begin{equation}
\lim_{t\rightarrow\infty}t^{p-1}\left\|P^{t}\left(x,\cdot\right)-\pi\left(\cdot\right)\right\|_{f}=0.
\end{equation}
En particular para cada condici\'on inicial
\begin{eqnarray*}
\lim_{t\rightarrow\infty}t^{p-1}|\esp_{x}\left[Q\left(t\right)\right]-\esp_{\pi}\left[Q\left(0\right)\right]|=0.
\end{eqnarray*}
\end{Teo}

\begin{Teo}[Teorema 6.4, Dai y Meyn \cite{DaiSean}]\label{Tma.6.4.DaiSean}
Suponga que se cumplen los supuestos A1), A2) y A3) y que el
modelo de flujo es estable. Sea $\nu$ cualquier distribuci\'on de
probabilidad en $\left(X,\mathcal{B}_{X}\right)$, y $\pi$ la
distribuci\'on estacionaria de $X$.
\begin{itemize}
\item[i)] Para cualquier $f:X\leftarrow\rea_{+}$
\begin{equation}
\lim_{t\rightarrow\infty}\frac{1}{t}\int_{o}^{t}f\left(X\left(s\right)\right)ds=\pi\left(f\right):=\int
f\left(x\right)\pi\left(dx\right)
\end{equation}
$\prob$-c.s.

\item[ii)] Para cualquier $f:X\leftarrow\rea_{+}$ con
$\pi\left(|f|\right)<\infty$, la ecuaci\'on anterior se cumple.
\end{itemize}
\end{Teo}

%_________________________________________________________________________
\subsection{Teorema 2.2}
%_________________________________________________________________________

\begin{Teo}[Teorema 2.2, Down \cite{Down}]\label{Tma2.2.Down}
Suponga que el fluido modelo es inestable en el sentido de que
para alguna $\epsilon_{0},c_{0}\geq0$,
\begin{equation}\label{Eq.Inestability}
|Q\left(T\right)|\geq\epsilon_{0}T-c_{0}\textrm{,   }T\geq0,
\end{equation}
para cualquier condici\'on inicial $Q\left(0\right)$, con
$|Q\left(0\right)|=1$. Entonces para cualquier $0<q\leq1$, existe
$B<0$ tal que para cualquier $|x|\geq B$,
\begin{equation}
\prob_{x}\left\{\mathbb{X}\rightarrow\infty\right\}\geq q.
\end{equation}
\end{Teo}

%_________________________________________________________________________
\subsection{Teorema 2.3}
%_________________________________________________________________________
\begin{Teo}[Teorema 2.3, Down \cite{Down}]\label{Tma2.3.Down}
Considere el siguiente valor:
\begin{equation}\label{Eq.Rho.1serv}
\rho=\sum_{k=1}^{K}\rho_{k}+max_{1\leq j\leq K}\left(\frac{\lambda_{j}}{\sum_{s=1}^{S}p_{js}\overline{N}_{s}}\right)\delta^{*}
\end{equation}
\begin{itemize}
\item[i)] Si $\rho<1$ entonces la red es estable, es decir, se cumple el teorema \ref{Tma.2.1.Down}.

\item[ii)] Si $\rho<1$ entonces la red es inestable, es decir, se cumple el teorema \ref{Tma2.2.Down}
\end{itemize}
\end{Teo}
%_____________________________________________________________________
\subsection{Definiciones  B\'asicas}
%_____________________________________________________________________
\begin{Def}
Sea $X$ un conjunto y $\mathcal{F}$ una $\sigma$-\'algebra de
subconjuntos de $X$, la pareja $\left(X,\mathcal{F}\right)$ es
llamado espacio medible. Un subconjunto $A$ de $X$ es llamado
medible, o medible con respecto a $\mathcal{F}$, si
$A\in\mathcal{F}$.
\end{Def}

\begin{Def}
Sea $\left(X,\mathcal{F},\mu\right)$ espacio de medida. Se dice
que la medida $\mu$ es $\sigma$-finita si se puede escribir
$X=\bigcup_{n\geq1}X_{n}$ con $X_{n}\in\mathcal{F}$ y
$\mu\left(X_{n}\right)<\infty$.
\end{Def}

\begin{Def}\label{Cto.Borel}
Sea $X$ el conjunto de los \'umeros reales $\rea$. El \'algebra de
Borel es la $\sigma$-\'algebra $B$ generada por los intervalos
abiertos $\left(a,b\right)\in\rea$. Cualquier conjunto en $B$ es
llamado {\em Conjunto de Borel}.
\end{Def}

\begin{Def}\label{Funcion.Medible}
Una funci\'on $f:X\rightarrow\rea$, es medible si para cualquier
n\'umero real $\alpha$ el conjunto
\[\left\{x\in X:f\left(x\right)>\alpha\right\}\]
pertenece a $X$. Equivalentemente, se dice que $f$ es medible si
\[f^{-1}\left(\left(\alpha,\infty\right)\right)=\left\{x\in X:f\left(x\right)>\alpha\right\}\in\mathcal{F}.\]
\end{Def}


\begin{Def}\label{Def.Cilindros}
Sean $\left(\Omega_{i},\mathcal{F}_{i}\right)$, $i=1,2,\ldots,$
espacios medibles y $\Omega=\prod_{i=1}^{\infty}\Omega_{i}$ el
conjunto de todas las sucesiones
$\left(\omega_{1},\omega_{2},\ldots,\right)$ tales que
$\omega_{i}\in\Omega_{i}$, $i=1,2,\ldots,$. Si
$B^{n}\subset\prod_{i=1}^{\infty}\Omega_{i}$, definimos
$B_{n}=\left\{\omega\in\Omega:\left(\omega_{1},\omega_{2},\ldots,\omega_{n}\right)\in
B^{n}\right\}$. Al conjunto $B_{n}$ se le llama {\em cilindro} con
base $B^{n}$, el cilindro es llamado medible si
$B^{n}\in\prod_{i=1}^{\infty}\mathcal{F}_{i}$.
\end{Def}


\begin{Def}\label{Def.Proc.Adaptado}[TSP, Ash \cite{RBA}]
Sea $X\left(t\right),t\geq0$ proceso estoc\'astico, el proceso es
adaptado a la familia de $\sigma$-\'algebras $\mathcal{F}_{t}$,
para $t\geq0$, si para $s<t$ implica que
$\mathcal{F}_{s}\subset\mathcal{F}_{t}$, y $X\left(t\right)$ es
$\mathcal{F}_{t}$-medible para cada $t$. Si no se especifica
$\mathcal{F}_{t}$ entonces se toma $\mathcal{F}_{t}$ como
$\mathcal{F}\left(X\left(s\right),s\leq t\right)$, la m\'as
peque\~na $\sigma$-\'algebra de subconjuntos de $\Omega$ que hace
que cada $X\left(s\right)$, con $s\leq t$ sea Borel medible.
\end{Def}


\begin{Def}\label{Def.Tiempo.Paro}[TSP, Ash \cite{RBA}]
Sea $\left\{\mathcal{F}\left(t\right),t\geq0\right\}$ familia
creciente de sub $\sigma$-\'algebras. es decir,
$\mathcal{F}\left(s\right)\subset\mathcal{F}\left(t\right)$ para
$s\leq t$. Un tiempo de paro para $\mathcal{F}\left(t\right)$ es
una funci\'on $T:\Omega\rightarrow\left[0,\infty\right]$ tal que
$\left\{T\leq t\right\}\in\mathcal{F}\left(t\right)$ para cada
$t\geq0$. Un tiempo de paro para el proceso estoc\'astico
$X\left(t\right),t\geq0$ es un tiempo de paro para las
$\sigma$-\'algebras
$\mathcal{F}\left(t\right)=\mathcal{F}\left(X\left(s\right)\right)$.
\end{Def}

\begin{Def}
Sea $X\left(t\right),t\geq0$ proceso estoc\'astico, con
$\left(S,\chi\right)$ espacio de estados. Se dice que el proceso
es adaptado a $\left\{\mathcal{F}\left(t\right)\right\}$, es
decir, si para cualquier $s,t\in I$, $I$ conjunto de \'indices,
$s<t$, se tiene que
$\mathcal{F}\left(s\right)\subset\mathcal{F}\left(t\right)$ y
$X\left(t\right)$ es $\mathcal{F}\left(t\right)$-medible,
\end{Def}

\begin{Def}
Sea $X\left(t\right),t\geq0$ proceso estoc\'astico, se dice que es
un Proceso de Markov relativo a $\mathcal{F}\left(t\right)$ o que
$\left\{X\left(t\right),\mathcal{F}\left(t\right)\right\}$ es de
Markov si y s\'olo si para cualquier conjunto $B\in\chi$,  y
$s,t\in I$, $s<t$ se cumple que
\begin{equation}\label{Prop.Markov}
P\left\{X\left(t\right)\in
B|\mathcal{F}\left(s\right)\right\}=P\left\{X\left(t\right)\in
B|X\left(s\right)\right\}.
\end{equation}
\end{Def}
\begin{Note}
Si se dice que $\left\{X\left(t\right)\right\}$ es un Proceso de
Markov sin mencionar $\mathcal{F}\left(t\right)$, se asumir\'a que
\begin{eqnarray*}
\mathcal{F}\left(t\right)=\mathcal{F}_{0}\left(t\right)=\mathcal{F}\left(X\left(r\right),r\leq
t\right),
\end{eqnarray*}
entonces la ecuaci\'on (\ref{Prop.Markov}) se puede escribir como
\begin{equation}
P\left\{X\left(t\right)\in B|X\left(r\right),r\leq s\right\} =
P\left\{X\left(t\right)\in B|X\left(s\right)\right\}
\end{equation}
\end{Note}

\begin{Teo}
Sea $\left(X_{n},\mathcal{F}_{n},n=0,1,\ldots,\right\}$ Proceso de
Markov con espacio de estados $\left(S_{0},\chi_{0}\right)$
generado por una distribuici\'on inicial $P_{o}$ y probabilidad de
transici\'on $p_{mn}$, para $m,n=0,1,\ldots,$ $m<n$, que por
notaci\'on se escribir\'a como $p\left(m,n,x,B\right)\rightarrow
p_{mn}\left(x,B\right)$. Sea $S$ tiempo de paro relativo a la
$\sigma$-\'algebra $\mathcal{F}_{n}$. Sea $T$ funci\'on medible,
$T:\Omega\rightarrow\left\{0,1,\ldots,\right\}$. Sup\'ongase que
$T\geq S$, entonces $T$ es tiempo de paro. Si $B\in\chi_{0}$,
entonces
\begin{equation}\label{Prop.Fuerte.Markov}
P\left\{X\left(T\right)\in
B,T<\infty|\mathcal{F}\left(S\right)\right\} =
p\left(S,T,X\left(s\right),B\right)
\end{equation}
en $\left\{T<\infty\right\}$.
\end{Teo}

Propiedades importantes para el modelo de flujo retrasado:

\begin{Prop}
 Sea $\left(\overline{Q},\overline{T},\overline{T}^{0}\right)$ un flujo l\'imite de \ref{Equation.4.4} y suponga que cuando $x\rightarrow\infty$ a lo largo de
una subsucesi\'on
\[\left(\frac{1}{|x|}Q_{k}^{x}\left(0\right),\frac{1}{|x|}A_{k}^{x}\left(0\right),\frac{1}{|x|}B_{k}^{x}\left(0\right),\frac{1}{|x|}B_{k}^{x,0}\left(0\right)\right)\rightarrow\left(\overline{Q}_{k}\left(0\right),0,0,0\right)\]
para $k=1,\ldots,K$. EL flujo l\'imite tiene las siguientes
propiedades, donde las propiedades de la derivada se cumplen donde
la derivada exista:
\begin{itemize}
 \item[i)] Los vectores de tiempo ocupado $\overline{T}\left(t\right)$ y $\overline{T}^{0}\left(t\right)$ son crecientes y continuas con
$\overline{T}\left(0\right)=\overline{T}^{0}\left(0\right)=0$.
\item[ii)] Para todo $t\geq0$
\[\sum_{k=1}^{K}\left[\overline{T}_{k}\left(t\right)+\overline{T}_{k}^{0}\left(t\right)\right]=t\]
\item[iii)] Para todo $1\leq k\leq K$
\[\overline{Q}_{k}\left(t\right)=\overline{Q}_{k}\left(0\right)+\alpha_{k}t-\mu_{k}\overline{T}_{k}\left(t\right)\]
\item[iv)]  Para todo $1\leq k\leq K$
\[\dot{{\overline{T}}}_{k}\left(t\right)=\beta_{k}\] para $\overline{Q}_{k}\left(t\right)=0$.
\item[v)] Para todo $k,j$
\[\mu_{k}^{0}\overline{T}_{k}^{0}\left(t\right)=\mu_{j}^{0}\overline{T}_{j}^{0}\left(t\right)\]
\item[vi)]  Para todo $1\leq k\leq K$
\[\mu_{k}\dot{{\overline{T}}}_{k}\left(t\right)=l_{k}\mu_{k}^{0}\dot{{\overline{T}}}_{k}^{0}\left(t\right)\] para $\overline{Q}_{k}\left(t\right)>0$.
\end{itemize}
\end{Prop}

\begin{Lema}[Lema 3.1 \cite{Chen}]\label{Lema3.1}
Si el modelo de flujo es estable, definido por las ecuaciones
(3.8)-(3.13), entonces el modelo de flujo retrasado tambin es
estable.
\end{Lema}

\begin{Teo}[Teorema 5.2 \cite{Chen}]\label{Tma.5.2}
Si el modelo de flujo lineal correspondiente a la red de cola es
estable, entonces la red de colas es estable.
\end{Teo}

\begin{Teo}[Teorema 5.1 \cite{Chen}]\label{Tma.5.1.Chen}
La red de colas es estable si existe una constante $t_{0}$ que
depende de $\left(\alpha,\mu,T,U\right)$ y $V$ que satisfagan las
ecuaciones (5.1)-(5.5), $Z\left(t\right)=0$, para toda $t\geq
t_{0}$.
\end{Teo}



\begin{Lema}[Lema 5.2 \cite{Gut}]\label{Lema.5.2.Gut}
Sea $\left\{\xi\left(k\right):k\in\ent\right\}$ sucesin de
variables aleatorias i.i.d. con valores en
$\left(0,\infty\right)$, y sea $E\left(t\right)$ el proceso de
conteo
\[E\left(t\right)=max\left\{n\geq1:\xi\left(1\right)+\cdots+\xi\left(n-1\right)\leq t\right\}.\]
Si $E\left[\xi\left(1\right)\right]<\infty$, entonces para
cualquier entero $r\geq1$
\begin{equation}
lim_{t\rightarrow\infty}\esp\left[\left(\frac{E\left(t\right)}{t}\right)^{r}\right]=\left(\frac{1}{E\left[\xi_{1}\right]}\right)^{r}
\end{equation}
de aqu, bajo estas condiciones
\begin{itemize}
\item[a)] Para cualquier $t>0$,
$sup_{t\geq\delta}\esp\left[\left(\frac{E\left(t\right)}{t}\right)^{r}\right]$

\item[b)] Las variables aleatorias
$\left\{\left(\frac{E\left(t\right)}{t}\right)^{r}:t\geq1\right\}$
son uniformemente integrables.
\end{itemize}
\end{Lema}

\begin{Teo}[Teorema 5.1: Ley Fuerte para Procesos de Conteo
\cite{Gut}]\label{Tma.5.1.Gut} Sea
$0<\mu<\esp\left(X_{1}\right]\leq\infty$. entonces

\begin{itemize}
\item[a)] $\frac{N\left(t\right)}{t}\rightarrow\frac{1}{\mu}$
a.s., cuando $t\rightarrow\infty$.


\item[b)]$\esp\left[\frac{N\left(t\right)}{t}\right]^{r}\rightarrow\frac{1}{\mu^{r}}$,
cuando $t\rightarrow\infty$ para todo $r>0$..
\end{itemize}
\end{Teo}


\begin{Prop}[Proposicin 5.1 \cite{DaiSean}]\label{Prop.5.1}
Suponga que los supuestos (A1) y (A2) se cumplen, adems suponga
que el modelo de flujo es estable. Entonces existe $t_{0}>0$ tal
que
\begin{equation}\label{Eq.Prop.5.1}
lim_{|x|\rightarrow\infty}\frac{1}{|x|^{p+1}}\esp_{x}\left[|X\left(t_{0}|x|\right)|^{p+1}\right]=0.
\end{equation}

\end{Prop}


\begin{Prop}[Proposici\'on 5.3 \cite{DaiSean}]
Sea $X$ proceso de estados para la red de colas, y suponga que se
cumplen los supuestos (A1) y (A2), entonces para alguna constante
positiva $C_{p+1}<\infty$, $\delta>0$ y un conjunto compacto
$C\subset X$.

\begin{equation}\label{Eq.5.4}
\esp_{x}\left[\int_{0}^{\tau_{C}\left(\delta\right)}\left(1+|X\left(t\right)|^{p}\right)dt\right]\leq
C_{p+1}\left(1+|x|^{p+1}\right)
\end{equation}
\end{Prop}

\begin{Prop}[Proposici\'on 5.4 \cite{DaiSean}]
Sea $X$ un proceso de Markov Borel Derecho en $X$, sea
$f:X\leftarrow\rea_{+}$ y defina para alguna $\delta>0$, y un
conjunto cerrado $C\subset X$
\[V\left(x\right):=\esp_{x}\left[\int_{0}^{\tau_{C}\left(\delta\right)}f\left(X\left(t\right)\right)dt\right]\]
para $x\in X$. Si $V$ es finito en todas partes y uniformemente
acotada en $C$, entonces existe $k<\infty$ tal que
\begin{equation}\label{Eq.5.11}
\frac{1}{t}\esp_{x}\left[V\left(x\right)\right]+\frac{1}{t}\int_{0}^{t}\esp_{x}\left[f\left(X\left(s\right)\right)ds\right]\leq\frac{1}{t}V\left(x\right)+k,
\end{equation}
para $x\in X$ y $t>0$.
\end{Prop}


\begin{Teo}[Teorema 5.5 \cite{DaiSean}]
Suponga que se cumplen (A1) y (A2), adems suponga que el modelo
de flujo es estable. Entonces existe una constante $k_{p}<\infty$
tal que
\begin{equation}\label{Eq.5.13}
\frac{1}{t}\int_{0}^{t}\esp_{x}\left[|Q\left(s\right)|^{p}\right]ds\leq
k_{p}\left\{\frac{1}{t}|x|^{p+1}+1\right\}
\end{equation}
para $t\geq0$, $x\in X$. En particular para cada condicin inicial
\begin{equation}\label{Eq.5.14}
Limsup_{t\rightarrow\infty}\frac{1}{t}\int_{0}^{t}\esp_{x}\left[|Q\left(s\right)|^{p}\right]ds\leq
k_{p}
\end{equation}
\end{Teo}

\begin{Teo}[Teorema 6.2\cite{DaiSean}]\label{Tma.6.2}
Suponga que se cumplen los supuestos (A1)-(A3) y que el modelo de
flujo es estable, entonces se tiene que
\[\parallel P^{t}\left(c,\cdot\right)-\pi\left(\cdot\right)\parallel_{f_{p}}\rightarrow0\]
para $t\rightarrow\infty$ y $x\in X$. En particular para cada
condicin inicial
\[lim_{t\rightarrow\infty}\esp_{x}\left[\left|Q_{t}\right|^{p}\right]=\esp_{\pi}\left[\left|Q_{0}\right|^{p}\right]<\infty\]
\end{Teo}


\begin{Teo}[Teorema 6.3\cite{DaiSean}]\label{Tma.6.3}
Suponga que se cumplen los supuestos (A1)-(A3) y que el modelo de
flujo es estable, entonces con
$f\left(x\right)=f_{1}\left(x\right)$, se tiene que
\[lim_{t\rightarrow\infty}t^{(p-1)\left|P^{t}\left(c,\cdot\right)-\pi\left(\cdot\right)\right|_{f}=0},\]
para $x\in X$. En particular, para cada condicin inicial
\[lim_{t\rightarrow\infty}t^{(p-1)\left|\esp_{x}\left[Q_{t}\right]-\esp_{\pi}\left[Q_{0}\right]\right|=0}.\]
\end{Teo}



Si $x$ es el n{\'u}mero de usuarios en la cola al comienzo del
periodo de servicio y $N_{s}\left(x\right)=N\left(x\right)$ es el
n{\'u}mero de usuarios que son atendidos con la pol{\'\i}tica $s$,
{\'u}nica en nuestro caso, durante un periodo de servicio,
entonces se asume que:
\begin{itemize}
\item[(S1.)]
\begin{equation}\label{S1}
lim_{x\rightarrow\infty}\esp\left[N\left(x\right)\right]=\overline{N}>0.
\end{equation}
\item[(S2.)]
\begin{equation}\label{S2}
\esp\left[N\left(x\right)\right]\leq \overline{N}, \end{equation}
para cualquier valor de $x$. \item La $n$-{\'e}sima ocurrencia va
acompa{\~n}ada con el tiempo de cambio de longitud
$\delta_{j,j+1}\left(n\right)$, independientes e id{\'e}nticamente
distribuidas, con
$\esp\left[\delta_{j,j+1}\left(1\right)\right]\geq0$. \item Se
define
\begin{equation}
\delta^{*}:=\sum_{j,j+1}\esp\left[\delta_{j,j+1}\left(1\right)\right].
\end{equation}

\item Los tiempos de inter-arribo a la cola $k$,son de la forma
$\left\{\xi_{k}\left(n\right)\right\}_{n\geq1}$, con la propiedad
de que son independientes e id{\'e}nticamente distribuidos.

\item Los tiempos de servicio
$\left\{\eta_{k}\left(n\right)\right\}_{n\geq1}$ tienen la
propiedad de ser independientes e id{\'e}nticamente distribuidos.

\item Se define la tasa de arribo a la $k$-{\'e}sima cola como
$\lambda_{k}=1/\esp\left[\xi_{k}\left(1\right)\right]$ y
adem{\'a}s se define

\item la tasa de servicio para la $k$-{\'e}sima cola como
$\mu_{k}=1/\esp\left[\eta_{k}\left(1\right)\right]$

\item tambi{\'e}n se define $\rho_{k}=\lambda_{k}/\mu_{k}$, donde
es necesario que $\rho<1$ para cuestiones de estabilidad.

\item De las pol{\'\i}ticas posibles solamente consideraremos la
pol{\'\i}tica cerrada (Gated).
\end{itemize}

Las Colas C\'iclicas se pueden describir por medio de un proceso
de Markov $\left(X\left(t\right)\right)_{t\in\rea}$, donde el
estado del sistema al tiempo $t\geq0$ est\'a dado por
\begin{equation}
X\left(t\right)=\left(Q\left(t\right),A\left(t\right),H\left(t\right),B\left(t\right),B^{0}\left(t\right),C\left(t\right)\right)
\end{equation}
definido en el espacio producto:
\begin{equation}
\mathcal{X}=\mathbb{Z}^{K}\times\rea_{+}^{K}\times\left(\left\{1,2,\ldots,K\right\}\times\left\{1,2,\ldots,S\right\}\right)^{M}\times\rea_{+}^{K}\times\rea_{+}^{K}\times\mathbb{Z}^{K},
\end{equation}

\begin{itemize}
\item $Q\left(t\right)=\left(Q_{k}\left(t\right),1\leq k\leq
K\right)$, es el n\'umero de usuarios en la cola $k$, incluyendo
aquellos que est\'an siendo atendidos provenientes de la
$k$-\'esima cola.

\item $A\left(t\right)=\left(A_{k}\left(t\right),1\leq k\leq
K\right)$, son los residuales de los tiempos de arribo en la cola
$k$. \item $H\left(t\right)$ es el par ordenado que consiste en la
cola que esta siendo atendida y la pol\'itica de servicio que se
utilizar\'a.

\item $B\left(t\right)$ es el tiempo de servicio residual.

\item $B^{0}\left(t\right)$ es el tiempo residual del cambio de
cola.

\item $C\left(t\right)$ indica el n\'umero de usuarios atendidos
durante la visita del servidor a la cola dada en
$H\left(t\right)$.
\end{itemize}

$A_{k}\left(t\right),B_{m}\left(t\right)$ y
$B_{m}^{0}\left(t\right)$ se suponen continuas por la derecha y
que satisfacen la propiedad fuerte de Markov, (\cite{Dai})

\begin{itemize}
\item Los tiempos de interarribo a la cola $k$,son de la forma
$\left\{\xi_{k}\left(n\right)\right\}_{n\geq1}$, con la propiedad
de que son independientes e id{\'e}nticamente distribuidos.

\item Los tiempos de servicio
$\left\{\eta_{k}\left(n\right)\right\}_{n\geq1}$ tienen la
propiedad de ser independientes e id{\'e}nticamente distribuidos.

\item Se define la tasa de arribo a la $k$-{\'e}sima cola como
$\lambda_{k}=1/\esp\left[\xi_{k}\left(1\right)\right]$ y
adem{\'a}s se define

\item la tasa de servicio para la $k$-{\'e}sima cola como
$\mu_{k}=1/\esp\left[\eta_{k}\left(1\right)\right]$

\item tambi{\'e}n se define $\rho_{k}=\lambda_{k}/\mu_{k}$, donde
es necesario que $\rho<1$ para cuestiones de estabilidad.

\item De las pol{\'\i}ticas posibles solamente consideraremos la
pol{\'\i}tica cerrada (Gated).
\end{itemize}


%_____________________________________________________


\subsection{Preliminares}



Sup\'ongase que el sistema consta de varias colas a los cuales
llegan uno o varios servidores a dar servicio a los usuarios
esperando en la cola.\\


Si $x$ es el n\'umero de usuarios en la cola al comienzo del
periodo de servicio y $N_{s}\left(x\right)=N\left(x\right)$ es el
n\'umero de usuarios que son atendidos con la pol\'itica $s$,
\'unica en nuestro caso, durante un periodo de servicio, entonces
se asume que:
\begin{itemize}
\item[1)]\label{S1}$lim_{x\rightarrow\infty}\esp\left[N\left(x\right)\right]=\overline{N}>0$
\item[2)]\label{S2}$\esp\left[N\left(x\right)\right]\leq\overline{N}$para
cualquier valor de $x$.
\end{itemize}
La manera en que atiende el servidor $m$-\'esimo, en este caso en
espec\'ifico solo lo ilustraremos con un s\'olo servidor, es la
siguiente:
\begin{itemize}
\item Al t\'ermino de la visita a la cola $j$, el servidor se
cambia a la cola $j^{'}$ con probabilidad
$r_{j,j^{'}}^{m}=r_{j,j^{'}}$

\item La $n$-\'esima ocurrencia va acompa\~nada con el tiempo de
cambio de longitud $\delta_{j,j^{'}}\left(n\right)$,
independientes e id\'enticamente distribuidas, con
$\esp\left[\delta_{j,j^{'}}\left(1\right)\right]\geq0$.

\item Sea $\left\{p_{j}\right\}$ la distribuci\'on invariante
estacionaria \'unica para la Cadena de Markov con matriz de
transici\'on $\left(r_{j,j^{'}}\right)$.

\item Finalmente, se define
\begin{equation}
\delta^{*}:=\sum_{j,j^{'}}p_{j}r_{j,j^{'}}\esp\left[\delta_{j,j^{'}}\left(i\right)\right].
\end{equation}
\end{itemize}

Veamos un caso muy espec\'ifico en el cual los tiempos de arribo a cada una de las colas se comportan de acuerdo a un proceso Poisson de la forma
$\left\{\xi_{k}\left(n\right)\right\}_{n\geq1}$, y los tiempos de servicio en cada una de las colas son variables aleatorias distribuidas exponencialmente e id\'enticamente distribuidas
$\left\{\eta_{k}\left(n\right)\right\}_{n\geq1}$, donde ambos procesos adem\'as cumplen la condici\'on de ser independientes entre si. Para la $k$-\'esima cola se define la tasa de arribo a la como
$\lambda_{k}=1/\esp\left[\xi_{k}\left(1\right)\right]$ y la tasa
de servicio como
$\mu_{k}=1/\esp\left[\eta_{k}\left(1\right)\right]$, finalmente se
define la carga de la cola como $\rho_{k}=\lambda_{k}/\mu_{k}$,
donde se pide que $\rho<1$, para garantizar la estabilidad del sistema.\\

Se denotar\'a por $Q_{k}\left(t\right)$ el n\'umero de usuarios en la cola $k$,
$A_{k}\left(t\right)$ los residuales de los tiempos entre arribos a la cola $k$;
para cada servidor $m$, se denota por $B_{m}\left(t\right)$ los residuales de los tiempos de servicio al tiempo $t$; $B_{m}^{0}\left(t\right)$ son los residuales de los tiempos de traslado de la cola $k$ a la pr\'oxima por atender, al tiempo $t$, finalmente sea $C_{m}\left(t\right)$ el n\'umero de usuarios atendidos durante la visita del servidor a la cola $k$ al tiempo $t$.\\


En este sentido el proceso para el sistema de visitas se puede definir como:

\begin{equation}\label{Esp.Edos.Down}
X\left(t\right)^{T}=\left(Q_{k}\left(t\right),A_{k}\left(t\right),B_{m}\left(t\right),B_{m}^{0}\left(t\right),C_{m}\left(t\right)\right)
\end{equation}
para $k=1,\ldots,K$ y $m=1,2,\ldots,M$. $X$ evoluciona en el
espacio de estados:
$X=\ent_{+}^{K}\times\rea_{+}^{K}\times\left(\left\{1,2,\ldots,K\right\}\times\left\{1,2,\ldots,S\right\}\right)^{M}\times\rea_{+}^{K}\times\ent_{+}^{K}$.\\

El sistema aqu\'i descrito debe de cumplir con los siguientes supuestos b\'asicos de un sistema de visitas:

Antes enunciemos los supuestos que regir\'an en la red.

\begin{itemize}
\item[A1)] $\xi_{1},\ldots,\xi_{K},\eta_{1},\ldots,\eta_{K}$ son
mutuamente independientes y son sucesiones independientes e
id\'enticamente distribuidas.

\item[A2)] Para alg\'un entero $p\geq1$
\begin{eqnarray*}
\esp\left[\xi_{l}\left(1\right)^{p+1}\right]<\infty\textrm{ para }l\in\mathcal{A}\textrm{ y }\\
\esp\left[\eta_{k}\left(1\right)^{p+1}\right]<\infty\textrm{ para
}k=1,\ldots,K.
\end{eqnarray*}
donde $\mathcal{A}$ es la clase de posibles arribos.

\item[A3)] Para $k=1,2,\ldots,K$ existe una funci\'on positiva
$q_{k}\left(x\right)$ definida en $\rea_{+}$, y un entero $j_{k}$,
tal que
\begin{eqnarray}
P\left(\xi_{k}\left(1\right)\geq x\right)>0\textrm{, para todo }x>0\\
P\left\{a\leq\sum_{i=1}^{j_{k}}\xi_{k}\left(i\right)\leq
b\right\}\geq\int_{a}^{b}q_{k}\left(x\right)dx, \textrm{ }0\leq
a<b.
\end{eqnarray}
\end{itemize}

En particular los procesos de tiempo entre arribos y de servicio
considerados con fines de ilustraci\'on de la metodolog\'ia
cumplen con el supuesto $A2)$ para $p=1$, es decir, ambos procesos
tienen primer y segundo momento finito.

En lo que respecta al supuesto (A3), en Dai y Meyn \cite{DaiSean}
hacen ver que este se puede sustituir por

\begin{itemize}
\item[A3')] Para el Proceso de Markov $X$, cada subconjunto
compacto de $X$ es un conjunto peque\~no, ver definici\'on
\ref{Def.Cto.Peq.}.
\end{itemize}

Es por esta raz\'on que con la finalidad de poder hacer uso de
$A3^{'})$ es necesario recurrir a los Procesos de Harris y en
particular a los Procesos Harris Recurrente:
%_______________________________________________________________________
\subsection{Procesos Harris Recurrente}
%_______________________________________________________________________

Por el supuesto (A1) conforme a Davis \cite{Davis}, se puede
definir el proceso de saltos correspondiente de manera tal que
satisfaga el supuesto (\ref{Sup3.1.Davis}), de hecho la
demostraci\'on est\'a basada en la l\'inea de argumentaci\'on de
Davis, (\cite{Davis}, p\'aginas 362-364).

Entonces se tiene un espacio de estados Markoviano. El espacio de
Markov descrito en Dai y Meyn \cite{DaiSean}

\[\left(\Omega,\mathcal{F},\mathcal{F}_{t},X\left(t\right),\theta_{t},P_{x}\right)\]
es un proceso de Borel Derecho (Sharpe \cite{Sharpe}) en el
espacio de estados medible $\left(X,\mathcal{B}_{X}\right)$. El
Proceso $X=\left\{X\left(t\right),t\geq0\right\}$ tiene
trayectorias continuas por la derecha, est\'a definida en
$\left(\Omega,\mathcal{F}\right)$ y est\'a adaptado a
$\left\{\mathcal{F}_{t},t\geq0\right\}$; la colecci\'on
$\left\{P_{x},x\in \mathbb{X}\right\}$ son medidas de probabilidad
en $\left(\Omega,\mathcal{F}\right)$ tales que para todo $x\in
\mathbb{X}$
\[P_{x}\left\{X\left(0\right)=x\right\}=1\] y
\[E_{x}\left\{f\left(X\circ\theta_{t}\right)|\mathcal{F}_{t}\right\}=E_{X}\left(\tau\right)f\left(X\right)\]
en $\left\{\tau<\infty\right\}$, $P_{x}$-c.s. Donde $\tau$ es un
$\mathcal{F}_{t}$-tiempo de paro
\[\left(X\circ\theta_{\tau}\right)\left(w\right)=\left\{X\left(\tau\left(w\right)+t,w\right),t\geq0\right\}\]
y $f$ es una funci\'on de valores reales acotada y medible con la
$\sigma$-algebra de Kolmogorov generada por los cilindros.\\

Sea $P^{t}\left(x,D\right)$, $D\in\mathcal{B}_{\mathbb{X}}$,
$t\geq0$ probabilidad de transici\'on de $X$ definida como
\[P^{t}\left(x,D\right)=P_{x}\left(X\left(t\right)\in
D\right)\]


\begin{Def}
Una medida no cero $\pi$ en
$\left(\mathbf{X},\mathcal{B}_{\mathbf{X}}\right)$ es {\bf
invariante} para $X$ si $\pi$ es $\sigma$-finita y
\[\pi\left(D\right)=\int_{\mathbf{X}}P^{t}\left(x,D\right)\pi\left(dx\right)\]
para todo $D\in \mathcal{B}_{\mathbf{X}}$, con $t\geq0$.
\end{Def}

\begin{Def}
El proceso de Markov $X$ es llamado Harris recurrente si existe
una medida de probabilidad $\nu$ en
$\left(\mathbf{X},\mathcal{B}_{\mathbf{X}}\right)$, tal que si
$\nu\left(D\right)>0$ y $D\in\mathcal{B}_{\mathbf{X}}$
\[P_{x}\left\{\tau_{D}<\infty\right\}\equiv1\] cuando
$\tau_{D}=inf\left\{t\geq0:X_{t}\in D\right\}$.
\end{Def}

\begin{Note}
\begin{itemize}
\item[i)] Si $X$ es Harris recurrente, entonces existe una \'unica
medida invariante $\pi$ (Getoor \cite{Getoor}).

\item[ii)] Si la medida invariante es finita, entonces puede
normalizarse a una medida de probabilidad, en este caso se le
llama Proceso {\em Harris recurrente positivo}.


\item[iii)] Cuando $X$ es Harris recurrente positivo se dice que
la disciplina de servicio es estable. En este caso $\pi$ denota la
distribuci\'on estacionaria y hacemos
\[P_{\pi}\left(\cdot\right)=\int_{\mathbf{X}}P_{x}\left(\cdot\right)\pi\left(dx\right)\]
y se utiliza $E_{\pi}$ para denotar el operador esperanza
correspondiente.
\end{itemize}
\end{Note}

\begin{Def}\label{Def.Cto.Peq.}
Un conjunto $D\in\mathcal{B_{\mathbf{X}}}$ es llamado peque\~no si
existe un $t>0$, una medida de probabilidad $\nu$ en
$\mathcal{B_{\mathbf{X}}}$, y un $\delta>0$ tal que
\[P^{t}\left(x,A\right)\geq\delta\nu\left(A\right)\] para $x\in
D,A\in\mathcal{B_{X}}$.
\end{Def}

La siguiente serie de resultados vienen enunciados y demostrados
en Dai \cite{Dai}:
\begin{Lema}[Lema 3.1, Dai\cite{Dai}]
Sea $B$ conjunto peque\~no cerrado, supongamos que
$P_{x}\left(\tau_{B}<\infty\right)\equiv1$ y que para alg\'un
$\delta>0$ se cumple que
\begin{equation}\label{Eq.3.1}
\sup\esp_{x}\left[\tau_{B}\left(\delta\right)\right]<\infty,
\end{equation}
donde
$\tau_{B}\left(\delta\right)=inf\left\{t\geq\delta:X\left(t\right)\in
B\right\}$. Entonces, $X$ es un proceso Harris Recurrente
Positivo.
\end{Lema}

\begin{Lema}[Lema 3.1, Dai \cite{Dai}]\label{Lema.3.}
Bajo el supuesto (A3), el conjunto $B=\left\{|x|\leq k\right\}$ es
un conjunto peque\~no cerrado para cualquier $k>0$.
\end{Lema}

\begin{Teo}[Teorema 3.1, Dai\cite{Dai}]\label{Tma.3.1}
Si existe un $\delta>0$ tal que
\begin{equation}
lim_{|x|\rightarrow\infty}\frac{1}{|x|}\esp|X^{x}\left(|x|\delta\right)|=0,
\end{equation}
entonces la ecuaci\'on (\ref{Eq.3.1}) se cumple para
$B=\left\{|x|\leq k\right\}$ con alg\'un $k>0$. En particular, $X$
es Harris Recurrente Positivo.
\end{Teo}

\begin{Note}
En Meyn and Tweedie \cite{MeynTweedie} muestran que si
$P_{x}\left\{\tau_{D}<\infty\right\}\equiv1$ incluso para solo un
conjunto peque\~no, entonces el proceso es Harris Recurrente.
\end{Note}

Entonces, tenemos que el proceso $X$ es un proceso de Markov que
cumple con los supuestos $A1)$-$A3)$, lo que falta de hacer es
construir el Modelo de Flujo bas\'andonos en lo hasta ahora
presentado.
%_______________________________________________________________________
\subsection{Modelo de Flujo}
%_______________________________________________________________________

Dada una condici\'on inicial $x\in\textrm{X}$, sea
$Q_{k}^{x}\left(t\right)$ la longitud de la cola al tiempo $t$,
$T_{m,k}^{x}\left(t\right)$ el tiempo acumulado, al tiempo $t$,
que tarda el servidor $m$ en atender a los usuarios de la cola
$k$. Finalmente sea $T_{m,k}^{x,0}\left(t\right)$ el tiempo
acumulado, al tiempo $t$, que tarda el servidor $m$ en trasladarse
a otra cola a partir de la $k$-\'esima.\\

Sup\'ongase que la funci\'on
$\left(\overline{Q}\left(\cdot\right),\overline{T}_{m}
\left(\cdot\right),\overline{T}_{m}^{0} \left(\cdot\right)\right)$
para $m=1,2,\ldots,M$ es un punto l\'imite de
\begin{equation}\label{Eq.Punto.Limite}
\left(\frac{1}{|x|}Q^{x}\left(|x|t\right),\frac{1}{|x|}T_{m}^{x}\left(|x|t\right),\frac{1}{|x|}T_{m}^{x,0}\left(|x|t\right)\right)
\end{equation}
para $m=1,2,\ldots,M$, cuando $x\rightarrow\infty$. Entonces
$\left(\overline{Q}\left(t\right),\overline{T}_{m}
\left(t\right),\overline{T}_{m}^{0} \left(t\right)\right)$ es un
flujo l\'imite del sistema. Al conjunto de todos las posibles
flujos l\'imite se le llama \textbf{Modelo de Flujo}.\\

El modelo de flujo satisface el siguiente conjunto de ecuaciones:

\begin{equation}\label{Eq.MF.1}
\overline{Q}_{k}\left(t\right)=\overline{Q}_{k}\left(0\right)+\lambda_{k}t-\sum_{m=1}^{M}\mu_{k}\overline{T}_{m,k}\left(t\right)\\
\end{equation}
para $k=1,2,\ldots,K$.\\
\begin{equation}\label{Eq.MF.2}
\overline{Q}_{k}\left(t\right)\geq0\textrm{ para
}k=1,2,\ldots,K,\\
\end{equation}

\begin{equation}\label{Eq.MF.3}
\overline{T}_{m,k}\left(0\right)=0,\textrm{ y }\overline{T}_{m,k}\left(\cdot\right)\textrm{ es no decreciente},\\
\end{equation}
para $k=1,2,\ldots,K$ y $m=1,2,\ldots,M$,\\
\begin{equation}\label{Eq.MF.4}
\sum_{k=1}^{K}\overline{T}_{m,k}^{0}\left(t\right)+\overline{T}_{m,k}\left(t\right)=t\textrm{
para }m=1,2,\ldots,M.\\
\end{equation}

De acuerdo a Dai \cite{Dai}, se tiene que el conjunto de posibles
l\'imites
$\left(\overline{Q}\left(\cdot\right),\overline{T}\left(\cdot\right),\overline{T}^{0}\left(\cdot\right)\right)$,
en el sentido de que deben de satisfacer las ecuaciones
(\ref{Eq.MF.1})-(\ref{Eq.MF.4}), se le llama {\em Modelo de
Flujo}.


\begin{Def}[Definici\'on 4.1, , Dai \cite{Dai}]\label{Def.Modelo.Flujo}
Sea una disciplina de servicio espec\'ifica. Cualquier l\'imite
$\left(\overline{Q}\left(\cdot\right),\overline{T}\left(\cdot\right)\right)$
en (\ref{Eq.Punto.Limite}) es un {\em flujo l\'imite} de la
disciplina. Cualquier soluci\'on (\ref{Eq.MF.1})-(\ref{Eq.MF.4})
es llamado flujo soluci\'on de la disciplina. Se dice que el
modelo de flujo l\'imite, modelo de flujo, de la disciplina de la
cola es estable si existe una constante $\delta>0$ que depende de
$\mu,\lambda$ y $P$ solamente, tal que cualquier flujo l\'imite
con
$|\overline{Q}\left(0\right)|+|\overline{U}|+|\overline{V}|=1$, se
tiene que $\overline{Q}\left(\cdot+\delta\right)\equiv0$.
\end{Def}

Al conjunto de ecuaciones dadas en \ref{Eq.MF.1}-\ref{Eq.MF.4} se
le llama {\em Modelo de flujo} y al conjunto de todas las
soluciones del modelo de flujo
$\left(\overline{Q}\left(\cdot\right),\overline{T}
\left(\cdot\right)\right)$ se le denotar\'a por $\mathcal{Q}$.

Si se hace $|x|\rightarrow\infty$ sin restringir ninguna de las
componentes, tambi\'en se obtienen un modelo de flujo, pero en
este caso el residual de los procesos de arribo y servicio
introducen un retraso:
\begin{Teo}[Teorema 4.2, Dai\cite{Dai}]\label{Tma.4.2.Dai}
Sea una disciplina fija para la cola, suponga que se cumplen las
condiciones (A1))-(A3)). Si el modelo de flujo l\'imite de la
disciplina de la cola es estable, entonces la cadena de Markov $X$
que describe la din\'amica de la red bajo la disciplina es Harris
recurrente positiva.
\end{Teo}

Ahora se procede a escalar el espacio y el tiempo para reducir la
aparente fluctuaci\'on del modelo. Consid\'erese el proceso
\begin{equation}\label{Eq.3.7}
\overline{Q}^{x}\left(t\right)=\frac{1}{|x|}Q^{x}\left(|x|t\right)
\end{equation}
A este proceso se le conoce como el fluido escalado, y cualquier
l\'imite $\overline{Q}^{x}\left(t\right)$ es llamado flujo
l\'imite del proceso de longitud de la cola. Haciendo
$|q|\rightarrow\infty$ mientras se mantiene el resto de las
componentes fijas, cualquier punto l\'imite del proceso de
longitud de la cola normalizado $\overline{Q}^{x}$ es soluci\'on
del siguiente modelo de flujo.


\begin{Def}[Definici\'on 3.3, Dai y Meyn \cite{DaiSean}]
El modelo de flujo es estable si existe un tiempo fijo $t_{0}$ tal
que $\overline{Q}\left(t\right)=0$, con $t\geq t_{0}$, para
cualquier $\overline{Q}\left(\cdot\right)\in\mathcal{Q}$ que
cumple con $|\overline{Q}\left(0\right)|=1$.
\end{Def}

El siguiente resultado se encuentra en Chen \cite{Chen}.
\begin{Lemma}[Lema 3.1, Dai y Meyn \cite{DaiSean}]
Si el modelo de flujo definido por \ref{Eq.MF.1}-\ref{Eq.MF.4} es
estable, entonces el modelo de flujo retrasado es tambi\'en
estable, es decir, existe $t_{0}>0$ tal que
$\overline{Q}\left(t\right)=0$ para cualquier $t\geq t_{0}$, para
cualquier soluci\'on del modelo de flujo retrasado cuya
condici\'on inicial $\overline{x}$ satisface que
$|\overline{x}|=|\overline{Q}\left(0\right)|+|\overline{A}\left(0\right)|+|\overline{B}\left(0\right)|\leq1$.
\end{Lemma}


Ahora ya estamos en condiciones de enunciar los resultados principales:


\begin{Teo}[Teorema 2.1, Down \cite{Down}]\label{Tma2.1.Down}
Suponga que el modelo de flujo es estable, y que se cumplen los supuestos (A1) y (A2), entonces
\begin{itemize}
\item[i)] Para alguna constante $\kappa_{p}$, y para cada
condici\'on inicial $x\in X$
\begin{equation}\label{Estability.Eq1}
limsup_{t\rightarrow\infty}\frac{1}{t}\int_{0}^{t}\esp_{x}\left[|Q\left(s\right)|^{p}\right]ds\leq\kappa_{p},
\end{equation}
donde $p$ es el entero dado en (A2).
\end{itemize}
Si adem\'as se cumple la condici\'on (A3), entonces para cada
condici\'on inicial:
\begin{itemize}
\item[ii)] Los momentos transitorios convergen a su estado
estacionario:
 \begin{equation}\label{Estability.Eq2}
lim_{t\rightarrow\infty}\esp_{x}\left[Q_{k}\left(t\right)^{r}\right]=\esp_{\pi}\left[Q_{k}\left(0\right)^{r}\right]\leq\kappa_{r},
\end{equation}
para $r=1,2,\ldots,p$ y $k=1,2,\ldots,K$. Donde $\pi$ es la
probabilidad invariante para $\mathbf{X}$.

\item[iii)]  El primer momento converge con raz\'on $t^{p-1}$:
\begin{equation}\label{Estability.Eq3}
lim_{t\rightarrow\infty}t^{p-1}|\esp_{x}\left[Q_{k}\left(t\right)\right]-\esp_{\pi}\left[Q_{k}\left(0\right)\right]=0.
\end{equation}

\item[iv)] La {\em Ley Fuerte de los grandes n\'umeros} se cumple:
\begin{equation}\label{Estability.Eq4}
lim_{t\rightarrow\infty}\frac{1}{t}\int_{0}^{t}Q_{k}^{r}\left(s\right)ds=\esp_{\pi}\left[Q_{k}\left(0\right)^{r}\right],\textrm{
}\prob_{x}\textrm{-c.s.}
\end{equation}
para $r=1,2,\ldots,p$ y $k=1,2,\ldots,K$.
\end{itemize}
\end{Teo}

La contribuci\'on de Down a la teor\'ia de los Sistemas de Visitas
C\'iclicas, es la relaci\'on que hay entre la estabilidad del
sistema con el comportamiento de las medidas de desempe\~no, es
decir, la condici\'on suficiente para poder garantizar la
convergencia del proceso de la longitud de la cola as\'i como de
por los menos los dos primeros momentos adem\'as de una versi\'on
de la Ley Fuerte de los Grandes N\'umeros para los sistemas de
visitas.


\begin{Teo}[Teorema 2.3, Down \cite{Down}]\label{Tma2.3.Down}
Considere el siguiente valor:
\begin{equation}\label{Eq.Rho.1serv}
\rho=\sum_{k=1}^{K}\rho_{k}+max_{1\leq j\leq K}\left(\frac{\lambda_{j}}{\sum_{s=1}^{S}p_{js}\overline{N}_{s}}\right)\delta^{*}
\end{equation}
\begin{itemize}
\item[i)] Si $\rho<1$ entonces la red es estable, es decir, se cumple el teorema \ref{Tma2.1.Down}.

\item[ii)] Si $\rho<1$ entonces la red es inestable, es decir, se cumple el teorema \ref{Tma2.2.Down}
\end{itemize}
\end{Teo}

\begin{Teo}
Sea $\left(X_{n},\mathcal{F}_{n},n=0,1,\ldots,\right\}$ Proceso de
Markov con espacio de estados $\left(S_{0},\chi_{0}\right)$
generado por una distribuici\'on inicial $P_{o}$ y probabilidad de
transici\'on $p_{mn}$, para $m,n=0,1,\ldots,$ $m<n$, que por
notaci\'on se escribir\'a como $p\left(m,n,x,B\right)\rightarrow
p_{mn}\left(x,B\right)$. Sea $S$ tiempo de paro relativo a la
$\sigma$-\'algebra $\mathcal{F}_{n}$. Sea $T$ funci\'on medible,
$T:\Omega\rightarrow\left\{0,1,\ldots,\right\}$. Sup\'ongase que
$T\geq S$, entonces $T$ es tiempo de paro. Si $B\in\chi_{0}$,
entonces
\begin{equation}\label{Prop.Fuerte.Markov}
P\left\{X\left(T\right)\in
B,T<\infty|\mathcal{F}\left(S\right)\right\} =
p\left(S,T,X\left(s\right),B\right)
\end{equation}
en $\left\{T<\infty\right\}$.
\end{Teo}


Sea $K$ conjunto numerable y sea $d:K\rightarrow\nat$ funci\'on.
Para $v\in K$, $M_{v}$ es un conjunto abierto de
$\rea^{d\left(v\right)}$. Entonces \[E=\cup_{v\in
K}M_{v}=\left\{\left(v,\zeta\right):v\in K,\zeta\in
M_{v}\right\}.\]

Sea $\mathcal{E}$ la clase de conjuntos medibles en $E$:
\[\mathcal{E}=\left\{\cup_{v\in K}A_{v}:A_{v}\in \mathcal{M}_{v}\right\}.\]

donde $\mathcal{M}$ son los conjuntos de Borel de $M_{v}$.
Entonces $\left(E,\mathcal{E}\right)$ es un espacio de Borel. El
estado del proceso se denotar\'a por
$\mathbf{x}_{t}=\left(v_{t},\zeta_{t}\right)$. La distribuci\'on
de $\left(\mathbf{x}_{t}\right)$ est\'a determinada por por los
siguientes objetos:

\begin{itemize}
\item[i)] Los campos vectoriales $\left(\mathcal{H}_{v},v\in
K\right)$. \item[ii)] Una funci\'on medible $\lambda:E\rightarrow
\rea_{+}$. \item[iii)] Una medida de transici\'on
$Q:\mathcal{E}\times\left(E\cup\Gamma^{*}\right)\rightarrow\left[0,1\right]$
donde
\begin{equation}
\Gamma^{*}=\cup_{v\in K}\partial^{*}M_{v}.
\end{equation}
y
\begin{equation}
\partial^{*}M_{v}=\left\{z\in\partial M_{v}:\mathbf{\mathbf{\phi}_{v}\left(t,\zeta\right)=\mathbf{z}}\textrm{ para alguna }\left(t,\zeta\right)\in\rea_{+}\times M_{v}\right\}.
\end{equation}
$\partial M_{v}$ denota  la frontera de $M_{v}$.
\end{itemize}

El campo vectorial $\left(\mathcal{H}_{v},v\in K\right)$ se supone
tal que para cada $\mathbf{z}\in M_{v}$ existe una \'unica curva
integral $\mathbf{\phi}_{v}\left(t,\zeta\right)$ que satisface la
ecuaci\'on

\begin{equation}
\frac{d}{dt}f\left(\zeta_{t}\right)=\mathcal{H}f\left(\zeta_{t}\right),
\end{equation}
con $\zeta_{0}=\mathbf{z}$, para cualquier funci\'on suave
$f:\rea^{d}\rightarrow\rea$ y $\mathcal{H}$ denota el operador
diferencial de primer orden, con $\mathcal{H}=\mathcal{H}_{v}$ y
$\zeta_{t}=\mathbf{\phi}\left(t,\mathbf{z}\right)$. Adem\'as se
supone que $\mathcal{H}_{v}$ es conservativo, es decir, las curvas
integrales est\'an definidas para todo $t>0$.

Para $\mathbf{x}=\left(v,\zeta\right)\in E$ se denota
\[t^{*}\mathbf{x}=inf\left\{t>0:\mathbf{\phi}_{v}\left(t,\zeta\right)\in\partial^{*}M_{v}\right\}\]

En lo que respecta a la funci\'on $\lambda$, se supondr\'a que
para cada $\left(v,\zeta\right)\in E$ existe un $\epsilon>0$ tal
que la funci\'on
$s\rightarrow\lambda\left(v,\phi_{v}\left(s,\zeta\right)\right)\in
E$ es integrable para $s\in\left[0,\epsilon\right)$. La medida de
transici\'on $Q\left(A;\mathbf{x}\right)$ es una funci\'on medible
de $\mathbf{x}$ para cada $A\in\mathcal{E}$, definida para
$\mathbf{x}\in E\cup\Gamma^{*}$ y es una medida de probabilidad en
$\left(E,\mathcal{E}\right)$ para cada $\mathbf{x}\in E$.

El movimiento del proceso $\left(\mathbf{x}_{t}\right)$ comenzando
en $\mathbf{x}=\left(n,\mathbf{z}\right)\in E$ se puede construir
de la siguiente manera, def\'inase la funci\'on $F$ por

\begin{equation}
F\left(t\right)=\left\{\begin{array}{ll}\\
exp\left(-\int_{0}^{t}\lambda\left(n,\phi_{n}\left(s,\mathbf{z}\right)\right)ds\right), & t<t^{*}\left(\mathbf{x}\right),\\
0, & t\geq t^{*}\left(\mathbf{x}\right)
\end{array}\right.
\end{equation}

Sea $T_{1}$ una variable aleatoria tal que
$\prob\left[T_{1}>t\right]=F\left(t\right)$, ahora sea la variable
aleatoria $\left(N,Z\right)$ con distribuici\'on
$Q\left(\cdot;\phi_{n}\left(T_{1},\mathbf{z}\right)\right)$. La
trayectoria de $\left(\mathbf{x}_{t}\right)$ para $t\leq T_{1}$
es\footnote{Revisar p\'agina 362, y 364 de Davis \cite{Davis}.}
\begin{eqnarray*}
\mathbf{x}_{t}=\left(v_{t},\zeta_{t}\right)=\left\{\begin{array}{ll}
\left(n,\phi_{n}\left(t,\mathbf{z}\right)\right), & t<T_{1},\\
\left(N,\mathbf{Z}\right), & t=t_{1}.
\end{array}\right.
\end{eqnarray*}

Comenzando en $\mathbf{x}_{T_{1}}$ se selecciona el siguiente
tiempo de intersalto $T_{2}-T_{1}$ lugar del post-salto
$\mathbf{x}_{T_{2}}$ de manera similar y as\'i sucesivamente. Este
procedimiento nos da una trayectoria determinista por partes
$\mathbf{x}_{t}$ con tiempos de salto $T_{1},T_{2},\ldots$. Bajo
las condiciones enunciadas para $\lambda,T_{1}>0$  y
$T_{1}-T_{2}>0$ para cada $i$, con probabilidad 1. Se supone que
se cumple la siguiente condici\'on.

\begin{Sup}[Supuesto 3.1, Davis \cite{Davis}]\label{Sup3.1.Davis}
Sea $N_{t}:=\sum_{t}\indora_{\left(t\geq t\right)}$ el n\'umero de
saltos en $\left[0,t\right]$. Entonces
\begin{equation}
\esp\left[N_{t}\right]<\infty\textrm{ para toda }t.
\end{equation}
\end{Sup}

es un proceso de Markov, m\'as a\'un, es un Proceso Fuerte de
Markov, es decir, la Propiedad Fuerte de Markov se cumple para
cualquier tiempo de paro.


Sea $E$ es un espacio m\'etrico separable y la m\'etrica $d$ es
compatible con la topolog\'ia.


\begin{Def}
Un espacio topol\'ogico $E$ es llamado de {\em Rad\'on} si es
homeomorfo a un subconjunto universalmente medible de un espacio
m\'etrico compacto.
\end{Def}

Equivalentemente, la definici\'on de un espacio de Rad\'on puede
encontrarse en los siguientes t\'erminos:


\begin{Def}
$E$ es un espacio de Rad\'on si cada medida finita en
$\left(E,\mathcal{B}\left(E\right)\right)$ es regular interior o
cerrada, {\em tight}.
\end{Def}

\begin{Def}
Una medida finita, $\lambda$ en la $\sigma$-\'algebra de Borel de
un espacio metrizable $E$ se dice cerrada si
\begin{equation}\label{Eq.A2.3}
\lambda\left(E\right)=sup\left\{\lambda\left(K\right):K\textrm{ es
compacto en }E\right\}.
\end{equation}
\end{Def}

El siguiente teorema nos permite tener una mejor caracterizaci\'on
de los espacios de Rad\'on:
\begin{Teo}\label{Tma.A2.2}
Sea $E$ espacio separable metrizable. Entonces $E$ es Radoniano si
y s\'olo s\'i cada medida finita en
$\left(E,\mathcal{B}\left(E\right)\right)$ es cerrada.
\end{Teo}

Sea $E$ espacio de estados, tal que $E$ es un espacio de Rad\'on,
$\mathcal{B}\left(E\right)$ $\sigma$-\'algebra de Borel en $E$,
que se denotar\'a por $\mathcal{E}$.

Sea $\left(X,\mathcal{G},\prob\right)$ espacio de probabilidad,
$I\subset\rea$ conjunto de \'indices. Sea $\mathcal{F}_{\leq t}$
la $\sigma$-\'algebra natural definida como
$\sigma\left\{f\left(X_{r}\right):r\in I, r\leq
t,f\in\mathcal{E}\right\}$. Se considerar\'a una
$\sigma$-\'algebra m\'as general, $ \left(\mathcal{G}_{t}\right)$
tal que $\left(X_{t}\right)$ sea $\mathcal{E}$-adaptado.

\begin{Def}
Una familia $\left(P_{s,t}\right)$ de kernels de Markov en
$\left(E,\mathcal{E}\right)$ indexada por pares $s,t\in I$, con
$s\leq t$ es una funci\'on de transici\'on en $\ER$, si  para todo
$r\leq s< t$ en $I$ y todo $x\in E$,
$B\in\mathcal{E}$\footnote{Ecuaci\'on de Chapman-Kolmogorov}
\begin{equation}\label{Eq.Kernels}
P_{r,t}\left(x,B\right)=\int_{E}P_{r,s}\left(x,dy\right)P_{s,t}\left(y,B\right).
\end{equation}
\end{Def}

Se dice que la funci\'on de transici\'on $\KM$ en $\ER$ es la
funci\'on de transici\'on para un proceso $\PE$  con valores en
$E$ y que satisface la propiedad de
Markov\footnote{\begin{equation}\label{Eq.1.4.S}
\prob\left\{H|\mathcal{G}_{t}\right\}=\prob\left\{H|X_{t}\right\}\textrm{
}H\in p\mathcal{F}_{\geq t}.
\end{equation}} (\ref{Eq.1.4.S}) relativa a $\left(\mathcal{G}_{t}\right)$ si

\begin{equation}\label{Eq.1.6.S}
\prob\left\{f\left(X_{t}\right)|\mathcal{G}_{s}\right\}=P_{s,t}f\left(X_{t}\right)\textrm{
}s\leq t\in I,\textrm{ }f\in b\mathcal{E}.
\end{equation}

\begin{Def}
Una familia $\left(P_{t}\right)_{t\geq0}$ de kernels de Markov en
$\ER$ es llamada {\em Semigrupo de Transici\'on de Markov} o {\em
Semigrupo de Transici\'on} si
\[P_{t+s}f\left(x\right)=P_{t}\left(P_{s}f\right)\left(x\right),\textrm{ }t,s\geq0,\textrm{ }x\in E\textrm{ }f\in b\mathcal{E}.\]
\end{Def}
\begin{Note}
Si la funci\'on de transici\'on $\KM$ es llamada homog\'enea si
$P_{s,t}=P_{t-s}$.
\end{Note}

Un proceso de Markov que satisface la ecuaci\'on (\ref{Eq.1.6.S})
con funci\'on de transici\'on homog\'enea $\left(P_{t}\right)$
tiene la propiedad caracter\'istica
\begin{equation}\label{Eq.1.8.S}
\prob\left\{f\left(X_{t+s}\right)|\mathcal{G}_{t}\right\}=P_{s}f\left(X_{t}\right)\textrm{
}t,s\geq0,\textrm{ }f\in b\mathcal{E}.
\end{equation}
La ecuaci\'on anterior es la {\em Propiedad Simple de Markov} de
$X$ relativa a $\left(P_{t}\right)$.

En este sentido el proceso $\PE$ cumple con la propiedad de Markov
(\ref{Eq.1.8.S}) relativa a
$\left(\Omega,\mathcal{G},\mathcal{G}_{t},\prob\right)$ con
semigrupo de transici\'on $\left(P_{t}\right)$.

\begin{Def}
Un proceso estoc\'astico $\PE$ definido en
$\left(\Omega,\mathcal{G},\prob\right)$ con valores en el espacio
topol\'ogico $E$ es continuo por la derecha si cada trayectoria
muestral $t\rightarrow X_{t}\left(w\right)$ es un mapeo continuo
por la derecha de $I$ en $E$.
\end{Def}

\begin{Def}[HD1]\label{Eq.2.1.S}
Un semigrupo de Markov $\left(P_{t}\right)$ en un espacio de
Rad\'on $E$ se dice que satisface la condici\'on {\em HD1} si,
dada una medida de probabilidad $\mu$ en $E$, existe una
$\sigma$-\'algebra $\mathcal{E^{*}}$ con
$\mathcal{E}\subset\mathcal{E}^{*}$ y
$P_{t}\left(b\mathcal{E}^{*}\right)\subset b\mathcal{E}^{*}$, y un
$\mathcal{E}^{*}$-proceso $E$-valuado continuo por la derecha
$\PE$ en alg\'un espacio de probabilidad filtrado
$\left(\Omega,\mathcal{G},\mathcal{G}_{t},\prob\right)$ tal que
$X=\left(\Omega,\mathcal{G},\mathcal{G}_{t},\prob\right)$ es de
Markov (Homog\'eneo) con semigrupo de transici\'on $(P_{t})$ y
distribuci\'on inicial $\mu$.
\end{Def}

Consid\'erese la colecci\'on de variables aleatorias $X_{t}$
definidas en alg\'un espacio de probabilidad, y una colecci\'on de
medidas $\mathbf{P}^{x}$ tales que
$\mathbf{P}^{x}\left\{X_{0}=x\right\}$, y bajo cualquier
$\mathbf{P}^{x}$, $X_{t}$ es de Markov con semigrupo
$\left(P_{t}\right)$. $\mathbf{P}^{x}$ puede considerarse como la
distribuci\'on condicional de $\mathbf{P}$ dado $X_{0}=x$.

\begin{Def}\label{Def.2.2.S}
Sea $E$ espacio de Rad\'on, $\SG$ semigrupo de Markov en $\ER$. La
colecci\'on
$\mathbf{X}=\left(\Omega,\mathcal{G},\mathcal{G}_{t},X_{t},\theta_{t},\CM\right)$
es un proceso $\mathcal{E}$-Markov continuo por la derecha simple,
con espacio de estados $E$ y semigrupo de transici\'on $\SG$ en
caso de que $\mathbf{X}$ satisfaga las siguientes condiciones:
\begin{itemize}
\item[i)] $\left(\Omega,\mathcal{G},\mathcal{G}_{t}\right)$ es un
espacio de medida filtrado, y $X_{t}$ es un proceso $E$-valuado
continuo por la derecha $\mathcal{E}^{*}$-adaptado a
$\left(\mathcal{G}_{t}\right)$;

\item[ii)] $\left(\theta_{t}\right)_{t\geq0}$ es una colecci\'on
de operadores {\em shift} para $X$, es decir, mapea $\Omega$ en
s\'i mismo satisfaciendo para $t,s\geq0$,

\begin{equation}\label{Eq.Shift}
\theta_{t}\circ\theta_{s}=\theta_{t+s}\textrm{ y
}X_{t}\circ\theta_{t}=X_{t+s};
\end{equation}

\item[iii)] Para cualquier $x\in E$,$\CM\left\{X_{0}=x\right\}=1$,
y el proceso $\PE$ tiene la propiedad de Markov (\ref{Eq.1.8.S})
con semigrupo de transici\'on $\SG$ relativo a
$\left(\Omega,\mathcal{G},\mathcal{G}_{t},\CM\right)$.
\end{itemize}
\end{Def}

\begin{Def}[HD2]\label{Eq.2.2.S}
Para cualquier $\alpha>0$ y cualquier $f\in S^{\alpha}$, el
proceso $t\rightarrow f\left(X_{t}\right)$ es continuo por la
derecha casi seguramente.
\end{Def}

\begin{Def}\label{Def.PD}
Un sistema
$\mathbf{X}=\left(\Omega,\mathcal{G},\mathcal{G}_{t},X_{t},\theta_{t},\CM\right)$
es un proceso derecho en el espacio de Rad\'on $E$ con semigrupo
de transici\'on $\SG$ provisto de:
\begin{itemize}
\item[i)] $\mathbf{X}$ es una realizaci\'on  continua por la
derecha, \ref{Def.2.2.S}, de $\SG$.

\item[ii)] $\mathbf{X}$ satisface la condicion HD2,
\ref{Eq.2.2.S}, relativa a $\mathcal{G}_{t}$.

\item[iii)] $\mathcal{G}_{t}$ es aumentado y continuo por la
derecha.
\end{itemize}
\end{Def}

\begin{Lema}[Lema 4.2, Dai\cite{Dai}]\label{Lema4.2}
Sea $\left\{x_{n}\right\}\subset \mathbf{X}$ con
$|x_{n}|\rightarrow\infty$, conforme $n\rightarrow\infty$. Suponga
que
\[lim_{n\rightarrow\infty}\frac{1}{|x_{n}|}U\left(0\right)=\overline{U}\]
y
\[lim_{n\rightarrow\infty}\frac{1}{|x_{n}|}V\left(0\right)=\overline{V}.\]

Entonces, conforme $n\rightarrow\infty$, casi seguramente

\begin{equation}\label{E1.4.2}
\frac{1}{|x_{n}|}\Phi^{k}\left(\left[|x_{n}|t\right]\right)\rightarrow
P_{k}^{'}t\textrm{, u.o.c.,}
\end{equation}

\begin{equation}\label{E1.4.3}
\frac{1}{|x_{n}|}E^{x_{n}}_{k}\left(|x_{n}|t\right)\rightarrow
\alpha_{k}\left(t-\overline{U}_{k}\right)^{+}\textrm{, u.o.c.,}
\end{equation}

\begin{equation}\label{E1.4.4}
\frac{1}{|x_{n}|}S^{x_{n}}_{k}\left(|x_{n}|t\right)\rightarrow
\mu_{k}\left(t-\overline{V}_{k}\right)^{+}\textrm{, u.o.c.,}
\end{equation}

donde $\left[t\right]$ es la parte entera de $t$ y
$\mu_{k}=1/m_{k}=1/\esp\left[\eta_{k}\left(1\right)\right]$.
\end{Lema}

\begin{Lema}[Lema 4.3, Dai\cite{Dai}]\label{Lema.4.3}
Sea $\left\{x_{n}\right\}\subset \mathbf{X}$ con
$|x_{n}|\rightarrow\infty$, conforme $n\rightarrow\infty$. Suponga
que
\[lim_{n\rightarrow\infty}\frac{1}{|x_{n}|}U\left(0\right)=\overline{U}_{k}\]
y
\[lim_{n\rightarrow\infty}\frac{1}{|x_{n}|}V\left(0\right)=\overline{V}_{k}.\]
\begin{itemize}
\item[a)] Conforme $n\rightarrow\infty$ casi seguramente,
\[lim_{n\rightarrow\infty}\frac{1}{|x_{n}|}U^{x_{n}}_{k}\left(|x_{n}|t\right)=\left(\overline{U}_{k}-t\right)^{+}\textrm{, u.o.c.}\]
y
\[lim_{n\rightarrow\infty}\frac{1}{|x_{n}|}V^{x_{n}}_{k}\left(|x_{n}|t\right)=\left(\overline{V}_{k}-t\right)^{+}.\]

\item[b)] Para cada $t\geq0$ fijo,
\[\left\{\frac{1}{|x_{n}|}U^{x_{n}}_{k}\left(|x_{n}|t\right),|x_{n}|\geq1\right\}\]
y
\[\left\{\frac{1}{|x_{n}|}V^{x_{n}}_{k}\left(|x_{n}|t\right),|x_{n}|\geq1\right\}\]
\end{itemize}
son uniformemente convergentes.
\end{Lema}

$S_{l}^{x}\left(t\right)$ es el n\'umero total de servicios
completados de la clase $l$, si la clase $l$ est\'a dando $t$
unidades de tiempo de servicio. Sea $T_{l}^{x}\left(x\right)$ el
monto acumulado del tiempo de servicio que el servidor
$s\left(l\right)$ gasta en los usuarios de la clase $l$ al tiempo
$t$. Entonces $S_{l}^{x}\left(T_{l}^{x}\left(t\right)\right)$ es
el n\'umero total de servicios completados para la clase $l$ al
tiempo $t$. Una fracci\'on de estos usuarios,
$\Phi_{l}^{x}\left(S_{l}^{x}\left(T_{l}^{x}\left(t\right)\right)\right)$,
se convierte en usuarios de la clase $k$.\\

Entonces, dado lo anterior, se tiene la siguiente representaci\'on
para el proceso de la longitud de la cola:\\

\begin{equation}
Q_{k}^{x}\left(t\right)=_{k}^{x}\left(0\right)+E_{k}^{x}\left(t\right)+\sum_{l=1}^{K}\Phi_{k}^{l}\left(S_{l}^{x}\left(T_{l}^{x}\left(t\right)\right)\right)-S_{k}^{x}\left(T_{k}^{x}\left(t\right)\right)
\end{equation}
para $k=1,\ldots,K$. Para $i=1,\ldots,d$, sea
\[I_{i}^{x}\left(t\right)=t-\sum_{j\in C_{i}}T_{k}^{x}\left(t\right).\]

Entonces $I_{i}^{x}\left(t\right)$ es el monto acumulado del
tiempo que el servidor $i$ ha estado desocupado al tiempo $t$. Se
est\'a asumiendo que las disciplinas satisfacen la ley de
conservaci\'on del trabajo, es decir, el servidor $i$ est\'a en
pausa solamente cuando no hay usuarios en la estaci\'on $i$.
Entonces, se tiene que

\begin{equation}
\int_{0}^{\infty}\left(\sum_{k\in
C_{i}}Q_{k}^{x}\left(t\right)\right)dI_{i}^{x}\left(t\right)=0,
\end{equation}
para $i=1,\ldots,d$.\\

Hacer
\[T^{x}\left(t\right)=\left(T_{1}^{x}\left(t\right),\ldots,T_{K}^{x}\left(t\right)\right)^{'},\]
\[I^{x}\left(t\right)=\left(I_{1}^{x}\left(t\right),\ldots,I_{K}^{x}\left(t\right)\right)^{'}\]
y
\[S^{x}\left(T^{x}\left(t\right)\right)=\left(S_{1}^{x}\left(T_{1}^{x}\left(t\right)\right),\ldots,S_{K}^{x}\left(T_{K}^{x}\left(t\right)\right)\right)^{'}.\]

Para una disciplina que cumple con la ley de conservaci\'on del
trabajo, en forma vectorial, se tiene el siguiente conjunto de
ecuaciones

\begin{equation}\label{Eq.MF.1.3}
Q^{x}\left(t\right)=Q^{x}\left(0\right)+E^{x}\left(t\right)+\sum_{l=1}^{K}\Phi^{l}\left(S_{l}^{x}\left(T_{l}^{x}\left(t\right)\right)\right)-S^{x}\left(T^{x}\left(t\right)\right),\\
\end{equation}

\begin{equation}\label{Eq.MF.2.3}
Q^{x}\left(t\right)\geq0,\\
\end{equation}

\begin{equation}\label{Eq.MF.3.3}
T^{x}\left(0\right)=0,\textrm{ y }\overline{T}^{x}\left(t\right)\textrm{ es no decreciente},\\
\end{equation}

\begin{equation}\label{Eq.MF.4.3}
I^{x}\left(t\right)=et-CT^{x}\left(t\right)\textrm{ es no
decreciente}\\
\end{equation}

\begin{equation}\label{Eq.MF.5.3}
\int_{0}^{\infty}\left(CQ^{x}\left(t\right)\right)dI_{i}^{x}\left(t\right)=0,\\
\end{equation}

\begin{equation}\label{Eq.MF.6.3}
\textrm{Condiciones adicionales en
}\left(\overline{Q}^{x}\left(\cdot\right),\overline{T}^{x}\left(\cdot\right)\right)\textrm{
espec\'ificas de la disciplina de la cola,}
\end{equation}

donde $e$ es un vector de unos de dimensi\'on $d$, $C$ es la
matriz definida por
\[C_{ik}=\left\{\begin{array}{cc}
1,& S\left(k\right)=i,\\
0,& \textrm{ en otro caso}.\\
\end{array}\right.
\]
Es necesario enunciar el siguiente Teorema que se utilizar\'a para
el Teorema \ref{Tma.4.2.Dai}:
\begin{Teo}[Teorema 4.1, Dai \cite{Dai}]
Considere una disciplina que cumpla la ley de conservaci\'on del
trabajo, para casi todas las trayectorias muestrales $\omega$ y
cualquier sucesi\'on de estados iniciales
$\left\{x_{n}\right\}\subset \mathbf{X}$, con
$|x_{n}|\rightarrow\infty$, existe una subsucesi\'on
$\left\{x_{n_{j}}\right\}$ con $|x_{n_{j}}|\rightarrow\infty$ tal
que
\begin{equation}\label{Eq.4.15}
\frac{1}{|x_{n_{j}}|}\left(Q^{x_{n_{j}}}\left(0\right),U^{x_{n_{j}}}\left(0\right),V^{x_{n_{j}}}\left(0\right)\right)\rightarrow\left(\overline{Q}\left(0\right),\overline{U},\overline{V}\right),
\end{equation}

\begin{equation}\label{Eq.4.16}
\frac{1}{|x_{n_{j}}|}\left(Q^{x_{n_{j}}}\left(|x_{n_{j}}|t\right),T^{x_{n_{j}}}\left(|x_{n_{j}}|t\right)\right)\rightarrow\left(\overline{Q}\left(t\right),\overline{T}\left(t\right)\right)\textrm{
u.o.c.}
\end{equation}

Adem\'as,
$\left(\overline{Q}\left(t\right),\overline{T}\left(t\right)\right)$
satisface las siguientes ecuaciones:
\begin{equation}\label{Eq.MF.1.3a}
\overline{Q}\left(t\right)=Q\left(0\right)+\left(\alpha
t-\overline{U}\right)^{+}-\left(I-P\right)^{'}M^{-1}\left(\overline{T}\left(t\right)-\overline{V}\right)^{+},
\end{equation}

\begin{equation}\label{Eq.MF.2.3a}
\overline{Q}\left(t\right)\geq0,\\
\end{equation}

\begin{equation}\label{Eq.MF.3.3a}
\overline{T}\left(t\right)\textrm{ es no decreciente y comienza en cero},\\
\end{equation}

\begin{equation}\label{Eq.MF.4.3a}
\overline{I}\left(t\right)=et-C\overline{T}\left(t\right)\textrm{
es no decreciente,}\\
\end{equation}

\begin{equation}\label{Eq.MF.5.3a}
\int_{0}^{\infty}\left(C\overline{Q}\left(t\right)\right)d\overline{I}\left(t\right)=0,\\
\end{equation}

\begin{equation}\label{Eq.MF.6.3a}
\textrm{Condiciones adicionales en
}\left(\overline{Q}\left(\cdot\right),\overline{T}\left(\cdot\right)\right)\textrm{
especficas de la disciplina de la cola,}
\end{equation}
\end{Teo}


Propiedades importantes para el modelo de flujo retrasado:

\begin{Prop}
 Sea $\left(\overline{Q},\overline{T},\overline{T}^{0}\right)$ un flujo l\'imite de \ref{Eq.4.4} y suponga que cuando $x\rightarrow\infty$ a lo largo de
una subsucesi\'on
\[\left(\frac{1}{|x|}Q_{k}^{x}\left(0\right),\frac{1}{|x|}A_{k}^{x}\left(0\right),\frac{1}{|x|}B_{k}^{x}\left(0\right),\frac{1}{|x|}B_{k}^{x,0}\left(0\right)\right)\rightarrow\left(\overline{Q}_{k}\left(0\right),0,0,0\right)\]
para $k=1,\ldots,K$. EL flujo l\'imite tiene las siguientes
propiedades, donde las propiedades de la derivada se cumplen donde
la derivada exista:
\begin{itemize}
 \item[i)] Los vectores de tiempo ocupado $\overline{T}\left(t\right)$ y $\overline{T}^{0}\left(t\right)$ son crecientes y continuas con
$\overline{T}\left(0\right)=\overline{T}^{0}\left(0\right)=0$.
\item[ii)] Para todo $t\geq0$
\[\sum_{k=1}^{K}\left[\overline{T}_{k}\left(t\right)+\overline{T}_{k}^{0}\left(t\right)\right]=t\]
\item[iii)] Para todo $1\leq k\leq K$
\[\overline{Q}_{k}\left(t\right)=\overline{Q}_{k}\left(0\right)+\alpha_{k}t-\mu_{k}\overline{T}_{k}\left(t\right)\]
\item[iv)]  Para todo $1\leq k\leq K$
\[\dot{{\overline{T}}}_{k}\left(t\right)=\beta_{k}\] para $\overline{Q}_{k}\left(t\right)=0$.
\item[v)] Para todo $k,j$
\[\mu_{k}^{0}\overline{T}_{k}^{0}\left(t\right)=\mu_{j}^{0}\overline{T}_{j}^{0}\left(t\right)\]
\item[vi)]  Para todo $1\leq k\leq K$
\[\mu_{k}\dot{{\overline{T}}}_{k}\left(t\right)=l_{k}\mu_{k}^{0}\dot{{\overline{T}}}_{k}^{0}\left(t\right)\] para $\overline{Q}_{k}\left(t\right)>0$.
\end{itemize}
\end{Prop}

\begin{Lema}[Lema 3.1 \cite{Chen}]\label{Lema3.1}
Si el modelo de flujo es estable, definido por las ecuaciones
(3.8)-(3.13), entonces el modelo de flujo retrasado tambi\'en es
estable.
\end{Lema}

\begin{Teo}[Teorema 5.1 \cite{Chen}]\label{Tma.5.1.Chen}
La red de colas es estable si existe una constante $t_{0}$ que
depende de $\left(\alpha,\mu,T,U\right)$ y $V$ que satisfagan las
ecuaciones (5.1)-(5.5), $Z\left(t\right)=0$, para toda $t\geq
t_{0}$.
\end{Teo}



\begin{Lema}[Lema 5.2 \cite{Gut}]\label{Lema.5.2.Gut}
Sea $\left\{\xi\left(k\right):k\in\ent\right\}$ sucesi\'on de
variables aleatorias i.i.d. con valores en
$\left(0,\infty\right)$, y sea $E\left(t\right)$ el proceso de
conteo
\[E\left(t\right)=max\left\{n\geq1:\xi\left(1\right)+\cdots+\xi\left(n-1\right)\leq t\right\}.\]
Si $E\left[\xi\left(1\right)\right]<\infty$, entonces para
cualquier entero $r\geq1$
\begin{equation}
lim_{t\rightarrow\infty}\esp\left[\left(\frac{E\left(t\right)}{t}\right)^{r}\right]=\left(\frac{1}{E\left[\xi_{1}\right]}\right)^{r}
\end{equation}
de aqu\'i, bajo estas condiciones
\begin{itemize}
\item[a)] Para cualquier $t>0$,
$sup_{t\geq\delta}\esp\left[\left(\frac{E\left(t\right)}{t}\right)^{r}\right]$

\item[b)] Las variables aleatorias
$\left\{\left(\frac{E\left(t\right)}{t}\right)^{r}:t\geq1\right\}$
son uniformemente integrables.
\end{itemize}
\end{Lema}

\begin{Teo}[Teorema 5.1: Ley Fuerte para Procesos de Conteo
\cite{Gut}]\label{Tma.5.1.Gut} Sea
$0<\mu<\esp\left(X_{1}\right]\leq\infty$. entonces

\begin{itemize}
\item[a)] $\frac{N\left(t\right)}{t}\rightarrow\frac{1}{\mu}$
a.s., cuando $t\rightarrow\infty$.


\item[b)]$\esp\left[\frac{N\left(t\right)}{t}\right]^{r}\rightarrow\frac{1}{\mu^{r}}$,
cuando $t\rightarrow\infty$ para todo $r>0$..
\end{itemize}
\end{Teo}


\begin{Prop}[Proposici\'on 5.1 \cite{DaiSean}]\label{Prop.5.1}
Suponga que los supuestos (A1) y (A2) se cumplen, adem\'as suponga
que el modelo de flujo es estable. Entonces existe $t_{0}>0$ tal
que
\begin{equation}\label{Eq.Prop.5.1}
lim_{|x|\rightarrow\infty}\frac{1}{|x|^{p+1}}\esp_{x}\left[|X\left(t_{0}|x|\right)|^{p+1}\right]=0.
\end{equation}

\end{Prop}


\begin{Prop}[Proposici\'on 5.3 \cite{DaiSean}]
Sea $X$ proceso de estados para la red de colas, y suponga que se
cumplen los supuestos (A1) y (A2), entonces para alguna constante
positiva $C_{p+1}<\infty$, $\delta>0$ y un conjunto compacto
$C\subset X$.

\begin{equation}\label{Eq.5.4}
\esp_{x}\left[\int_{0}^{\tau_{C}\left(\delta\right)}\left(1+|X\left(t\right)|^{p}\right)dt\right]\leq
C_{p+1}\left(1+|x|^{p+1}\right)
\end{equation}
\end{Prop}

\begin{Prop}[Proposici\'on 5.4 \cite{DaiSean}]
Sea $X$ un proceso de Markov Borel Derecho en $X$, sea
$f:X\leftarrow\rea_{+}$ y defina para alguna $\delta>0$, y un
conjunto cerrado $C\subset X$
\[V\left(x\right):=\esp_{x}\left[\int_{0}^{\tau_{C}\left(\delta\right)}f\left(X\left(t\right)\right)dt\right]\]
para $x\in X$. Si $V$ es finito en todas partes y uniformemente
acotada en $C$, entonces existe $k<\infty$ tal que
\begin{equation}\label{Eq.5.11}
\frac{1}{t}\esp_{x}\left[V\left(x\right)\right]+\frac{1}{t}\int_{0}^{t}\esp_{x}\left[f\left(X\left(s\right)\right)ds\right]\leq\frac{1}{t}V\left(x\right)+k,
\end{equation}
para $x\in X$ y $t>0$.
\end{Prop}


\begin{Teo}[Teorema 5.5 \cite{DaiSean}]
Suponga que se cumplen (A1) y (A2), adem\'as suponga que el modelo
de flujo es estable. Entonces existe una constante $k_{p}<\infty$
tal que
\begin{equation}\label{Eq.5.13}
\frac{1}{t}\int_{0}^{t}\esp_{x}\left[|Q\left(s\right)|^{p}\right]ds\leq
k_{p}\left\{\frac{1}{t}|x|^{p+1}+1\right\}
\end{equation}
para $t\geq0$, $x\in X$. En particular para cada condici\'on
inicial
\begin{equation}\label{Eq.5.14}
Limsup_{t\rightarrow\infty}\frac{1}{t}\int_{0}^{t}\esp_{x}\left[|Q\left(s\right)|^{p}\right]ds\leq
k_{p}
\end{equation}
\end{Teo}

\begin{Teo}[Teorema 6.2 \cite{DaiSean}]\label{Tma.6.2}
Suponga que se cumplen los supuestos (A1)-(A3) y que el modelo de
flujo es estable, entonces se tiene que
\[\parallel P^{t}\left(c,\cdot\right)-\pi\left(\cdot\right)\parallel_{f_{p}}\rightarrow0\]
para $t\rightarrow\infty$ y $x\in X$. En particular para cada
condici\'on inicial
\[lim_{t\rightarrow\infty}\esp_{x}\left[\left|Q_{t}\right|^{p}\right]=\esp_{\pi}\left[\left|Q_{0}\right|^{p}\right]<\infty\]
\end{Teo}


\begin{Teo}[Teorema 6.3 \cite{DaiSean}]\label{Tma.6.3}
Suponga que se cumplen los supuestos (A1)-(A3) y que el modelo de
flujo es estable, entonces con
$f\left(x\right)=f_{1}\left(x\right)$, se tiene que
\[lim_{t\rightarrow\infty}t^{(p-1)\left|P^{t}\left(c,\cdot\right)-\pi\left(\cdot\right)\right|_{f}=0},\]
para $x\in X$. En particular, para cada condici\'on inicial
\[lim_{t\rightarrow\infty}t^{(p-1)}\left|\esp_{x}\left[Q_{t}\right]-\esp_{\pi}\left[Q_{0}\right]\right|=0.\]
\end{Teo}



\begin{Prop}[Proposici\'on 5.1, Dai y Meyn \cite{DaiSean}]\label{Prop.5.1.DaiSean}
Suponga que los supuestos A1) y A2) son ciertos y que el modelo de
flujo es estable. Entonces existe $t_{0}>0$ tal que
\begin{equation}
lim_{|x|\rightarrow\infty}\frac{1}{|x|^{p+1}}\esp_{x}\left[|X\left(t_{0}|x|\right)|^{p+1}\right]=0
\end{equation}
\end{Prop}

\begin{Lemma}[Lema 5.2, Dai y Meyn, \cite{DaiSean}]\label{Lema.5.2.DaiSean}
 Sea $\left\{\zeta\left(k\right):k\in \mathbb{z}\right\}$ una sucesi\'on independiente e id\'enticamente distribuida que toma valores en $\left(0,\infty\right)$,
y sea
$E\left(t\right)=max\left(n\geq1:\zeta\left(1\right)+\cdots+\zeta\left(n-1\right)\leq
t\right)$. Si $\esp\left[\zeta\left(1\right)\right]<\infty$,
entonces para cualquier entero $r\geq1$
\begin{equation}
 lim_{t\rightarrow\infty}\esp\left[\left(\frac{E\left(t\right)}{t}\right)^{r}\right]=\left(\frac{1}{\esp\left[\zeta_{1}\right]}\right)^{r}.
\end{equation}
Luego, bajo estas condiciones:
\begin{itemize}
 \item[a)] para cualquier $\delta>0$, $\sup_{t\geq\delta}\esp\left[\left(\frac{E\left(t\right)}{t}\right)^{r}\right]<\infty$
\item[b)] las variables aleatorias
$\left\{\left(\frac{E\left(t\right)}{t}\right)^{r}:t\geq1\right\}$
son uniformemente integrables.
\end{itemize}
\end{Lemma}

\begin{Teo}[Teorema 5.5, Dai y Meyn \cite{DaiSean}]\label{Tma.5.5.DaiSean}
Suponga que los supuestos A1) y A2) se cumplen y que el modelo de
flujo es estable. Entonces existe una constante $\kappa_{p}$ tal
que
\begin{equation}
\frac{1}{t}\int_{0}^{t}\esp_{x}\left[|Q\left(s\right)|^{p}\right]ds\leq\kappa_{p}\left\{\frac{1}{t}|x|^{p+1}+1\right\}
\end{equation}
para $t>0$ y $x\in X$. En particular, para cada condici\'on
inicial
\begin{eqnarray*}
\limsup_{t\rightarrow\infty}\frac{1}{t}\int_{0}^{t}\esp_{x}\left[|Q\left(s\right)|^{p}\right]ds\leq\kappa_{p}.
\end{eqnarray*}
\end{Teo}

\begin{Teo}[Teorema 6.2, Dai y Meyn \cite{DaiSean}]\label{Tma.6.2.DaiSean}
Suponga que se cumplen los supuestos A1), A2) y A3) y que el
modelo de flujo es estable. Entonces se tiene que
\begin{equation}
\left\|P^{t}\left(x,\cdot\right)-\pi\left(\cdot\right)\right\|_{f_{p}}\textrm{,
}t\rightarrow\infty,x\in X.
\end{equation}
En particular para cada condici\'on inicial
\begin{eqnarray*}
\lim_{t\rightarrow\infty}\esp_{x}\left[|Q\left(t\right)|^{p}\right]=\esp_{\pi}\left[|Q\left(0\right)|^{p}\right]\leq\kappa_{r}
\end{eqnarray*}
\end{Teo}
\begin{Teo}[Teorema 6.3, Dai y Meyn \cite{DaiSean}]\label{Tma.6.3.DaiSean}
Suponga que se cumplen los supuestos A1), A2) y A3) y que el
modelo de flujo es estable. Entonces con
$f\left(x\right)=f_{1}\left(x\right)$ se tiene
\begin{equation}
\lim_{t\rightarrow\infty}t^{p-1}\left\|P^{t}\left(x,\cdot\right)-\pi\left(\cdot\right)\right\|_{f}=0.
\end{equation}
En particular para cada condici\'on inicial
\begin{eqnarray*}
\lim_{t\rightarrow\infty}t^{p-1}|\esp_{x}\left[Q\left(t\right)\right]-\esp_{\pi}\left[Q\left(0\right)\right]|=0.
\end{eqnarray*}
\end{Teo}

\begin{Teo}[Teorema 6.4, Dai y Meyn, \cite{DaiSean}]\label{Tma.6.4.DaiSean}
Suponga que se cumplen los supuestos A1), A2) y A3) y que el
modelo de flujo es estable. Sea $\nu$ cualquier distribuci\'on de
probabilidad en $\left(X,\mathcal{B}_{X}\right)$, y $\pi$ la
distribuci\'on estacionaria de $X$.
\begin{itemize}
\item[i)] Para cualquier $f:X\leftarrow\rea_{+}$
\begin{equation}
\lim_{t\rightarrow\infty}\frac{1}{t}\int_{o}^{t}f\left(X\left(s\right)\right)ds=\pi\left(f\right):=\int
f\left(x\right)\pi\left(dx\right)
\end{equation}
$\prob$-c.s.

\item[ii)] Para cualquier $f:X\leftarrow\rea_{+}$ con
$\pi\left(|f|\right)<\infty$, la ecuaci\'on anterior se cumple.
\end{itemize}
\end{Teo}

\begin{Teo}[Teorema 2.2, Down \cite{Down}]\label{Tma2.2.Down}
Suponga que el fluido modelo es inestable en el sentido de que
para alguna $\epsilon_{0},c_{0}\geq0$,
\begin{equation}\label{Eq.Inestability}
|Q\left(T\right)|\geq\epsilon_{0}T-c_{0}\textrm{,   }T\geq0,
\end{equation}
para cualquier condici\'on inicial $Q\left(0\right)$, con
$|Q\left(0\right)|=1$. Entonces para cualquier $0<q\leq1$, existe
$B<0$ tal que para cualquier $|x|\geq B$,
\begin{equation}
\prob_{x}\left\{\mathbb{X}\rightarrow\infty\right\}\geq q.
\end{equation}
\end{Teo}



\begin{Def}
Sea $X$ un conjunto y $\mathcal{F}$ una $\sigma$-\'algebra de
subconjuntos de $X$, la pareja $\left(X,\mathcal{F}\right)$ es
llamado espacio medible. Un subconjunto $A$ de $X$ es llamado
medible, o medible con respecto a $\mathcal{F}$, si
$A\in\mathcal{F}$.
\end{Def}

\begin{Def}
Sea $\left(X,\mathcal{F},\mu\right)$ espacio de medida. Se dice
que la medida $\mu$ es $\sigma$-finita si se puede escribir
$X=\bigcup_{n\geq1}X_{n}$ con $X_{n}\in\mathcal{F}$ y
$\mu\left(X_{n}\right)<\infty$.
\end{Def}

\begin{Def}\label{Cto.Borel}
Sea $X$ el conjunto de los n\'umeros reales $\rea$. El \'algebra
de Borel es la $\sigma$-\'algebra $B$ generada por los intervalos
abiertos $\left(a,b\right)\in\rea$. Cualquier conjunto en $B$ es
llamado {\em Conjunto de Borel}.
\end{Def}

\begin{Def}\label{Funcion.Medible}
Una funci\'on $f:X\rightarrow\rea$, es medible si para cualquier
n\'umero real $\alpha$ el conjunto
\[\left\{x\in X:f\left(x\right)>\alpha\right\}\]
pertenece a $\mathcal{F}$. Equivalentemente, se dice que $f$ es
medible si
\[f^{-1}\left(\left(\alpha,\infty\right)\right)=\left\{x\in X:f\left(x\right)>\alpha\right\}\in\mathcal{F}.\]
\end{Def}


\begin{Def}\label{Def.Cilindros}
Sean $\left(\Omega_{i},\mathcal{F}_{i}\right)$, $i=1,2,\ldots,$
espacios medibles y $\Omega=\prod_{i=1}^{\infty}\Omega_{i}$ el
conjunto de todas las sucesiones
$\left(\omega_{1},\omega_{2},\ldots,\right)$ tales que
$\omega_{i}\in\Omega_{i}$, $i=1,2,\ldots,$. Si
$B^{n}\subset\prod_{i=1}^{\infty}\Omega_{i}$, definimos
$B_{n}=\left\{\omega\in\Omega:\left(\omega_{1},\omega_{2},\ldots,\omega_{n}\right)\in
B^{n}\right\}$. Al conjunto $B_{n}$ se le llama {\em cilindro} con
base $B^{n}$, el cilindro es llamado medible si
$B^{n}\in\prod_{i=1}^{\infty}\mathcal{F}_{i}$.
\end{Def}


\begin{Def}\label{Def.Proc.Adaptado}[TSP, Ash \cite{RBA}]
Sea $X\left(t\right),t\geq0$ proceso estoc\'astico, el proceso es
adaptado a la familia de $\sigma$-\'algebras $\mathcal{F}_{t}$,
para $t\geq0$, si para $s<t$ implica que
$\mathcal{F}_{s}\subset\mathcal{F}_{t}$, y $X\left(t\right)$ es
$\mathcal{F}_{t}$-medible para cada $t$. Si no se especifica
$\mathcal{F}_{t}$ entonces se toma $\mathcal{F}_{t}$ como
$\mathcal{F}\left(X\left(s\right),s\leq t\right)$, la m\'as
peque\~na $\sigma$-\'algebra de subconjuntos de $\Omega$ que hace
que cada $X\left(s\right)$, con $s\leq t$ sea Borel medible.
\end{Def}


\begin{Def}\label{Def.Tiempo.Paro}[TSP, Ash \cite{RBA}]
Sea $\left\{\mathcal{F}\left(t\right),t\geq0\right\}$ familia
creciente de sub $\sigma$-\'algebras. es decir,
$\mathcal{F}\left(s\right)\subset\mathcal{F}\left(t\right)$ para
$s\leq t$. Un tiempo de paro para $\mathcal{F}\left(t\right)$ es
una funci\'on $T:\Omega\rightarrow\left[0,\infty\right]$ tal que
$\left\{T\leq t\right\}\in\mathcal{F}\left(t\right)$ para cada
$t\geq0$. Un tiempo de paro para el proceso estoc\'astico
$X\left(t\right),t\geq0$ es un tiempo de paro para las
$\sigma$-\'algebras
$\mathcal{F}\left(t\right)=\mathcal{F}\left(X\left(s\right)\right)$.
\end{Def}

\begin{Def}
Sea $X\left(t\right),t\geq0$ proceso estoc\'astico, con
$\left(S,\chi\right)$ espacio de estados. Se dice que el proceso
es adaptado a $\left\{\mathcal{F}\left(t\right)\right\}$, es
decir, si para cualquier $s,t\in I$, $I$ conjunto de \'indices,
$s<t$, se tiene que
$\mathcal{F}\left(s\right)\subset\mathcal{F}\left(t\right)$ y
$X\left(t\right)$ es $\mathcal{F}\left(t\right)$-medible,
\end{Def}

\begin{Def}
Sea $X\left(t\right),t\geq0$ proceso estoc\'astico, se dice que es
un Proceso de Markov relativo a $\mathcal{F}\left(t\right)$ o que
$\left\{X\left(t\right),\mathcal{F}\left(t\right)\right\}$ es de
Markov si y s\'olo si para cualquier conjunto $B\in\chi$,  y
$s,t\in I$, $s<t$ se cumple que
\begin{equation}\label{Prop.Markov}
P\left\{X\left(t\right)\in
B|\mathcal{F}\left(s\right)\right\}=P\left\{X\left(t\right)\in
B|X\left(s\right)\right\}.
\end{equation}
\end{Def}
\begin{Note}
Si se dice que $\left\{X\left(t\right)\right\}$ es un Proceso de
Markov sin mencionar $\mathcal{F}\left(t\right)$, se asumir\'a que
\begin{eqnarray*}
\mathcal{F}\left(t\right)=\mathcal{F}_{0}\left(t\right)=\mathcal{F}\left(X\left(r\right),r\leq
t\right),
\end{eqnarray*}
entonces la ecuaci\'on (\ref{Prop.Markov}) se puede escribir como
\begin{equation}
P\left\{X\left(t\right)\in B|X\left(r\right),r\leq s\right\} =
P\left\{X\left(t\right)\in B|X\left(s\right)\right\}
\end{equation}
\end{Note}
%_______________________________________________________________
\subsection{Procesos de Estados de Markov}
%_______________________________________________________________

\begin{Teo}
Sea $\left(X_{n},\mathcal{F}_{n},n=0,1,\ldots,\right\}$ Proceso de
Markov con espacio de estados $\left(S_{0},\chi_{0}\right)$
generado por una distribuici\'on inicial $P_{o}$ y probabilidad de
transici\'on $p_{mn}$, para $m,n=0,1,\ldots,$ $m<n$, que por
notaci\'on se escribir\'a como $p\left(m,n,x,B\right)\rightarrow
p_{mn}\left(x,B\right)$. Sea $S$ tiempo de paro relativo a la
$\sigma$-\'algebra $\mathcal{F}_{n}$. Sea $T$ funci\'on medible,
$T:\Omega\rightarrow\left\{0,1,\ldots,\right\}$. Sup\'ongase que
$T\geq S$, entonces $T$ es tiempo de paro. Si $B\in\chi_{0}$,
entonces
\begin{equation}\label{Prop.Fuerte.Markov}
P\left\{X\left(T\right)\in
B,T<\infty|\mathcal{F}\left(S\right)\right\} =
p\left(S,T,X\left(s\right),B\right)
\end{equation}
en $\left\{T<\infty\right\}$.
\end{Teo}


Sea $K$ conjunto numerable y sea $d:K\rightarrow\nat$ funci\'on.
Para $v\in K$, $M_{v}$ es un conjunto abierto de
$\rea^{d\left(v\right)}$. Entonces \[E=\bigcup_{v\in
K}M_{v}=\left\{\left(v,\zeta\right):v\in K,\zeta\in
M_{v}\right\}.\]

Sea $\mathcal{E}$ la clase de conjuntos medibles en $E$:
\[\mathcal{E}=\left\{\bigcup_{v\in K}A_{v}:A_{v}\in \mathcal{M}_{v}\right\}.\]

donde $\mathcal{M}$ son los conjuntos de Borel de $M_{v}$.
Entonces $\left(E,\mathcal{E}\right)$ es un espacio de Borel. El
estado del proceso se denotar\'a por
$\mathbf{x}_{t}=\left(v_{t},\zeta_{t}\right)$. La distribuci\'on
de $\left(\mathbf{x}_{t}\right)$ est\'a determinada por por los
siguientes objetos:

\begin{itemize}
\item[i)] Los campos vectoriales $\left(\mathcal{H}_{v},v\in
K\right)$. \item[ii)] Una funci\'on medible $\lambda:E\rightarrow
\rea_{+}$. \item[iii)] Una medida de transici\'on
$Q:\mathcal{E}\times\left(E\cup\Gamma^{*}\right)\rightarrow\left[0,1\right]$
donde
\begin{equation}
\Gamma^{*}=\bigcup_{v\in K}\partial^{*}M_{v}.
\end{equation}
y
\begin{equation}
\partial^{*}M_{v}=\left\{z\in\partial M_{v}:\mathbf{\mathbf{\phi}_{v}\left(t,\zeta\right)=\mathbf{z}}\textrm{ para alguna }\left(t,\zeta\right)\in\rea_{+}\times M_{v}\right\}.
\end{equation}
$\partial M_{v}$ denota  la frontera de $M_{v}$.
\end{itemize}

El campo vectorial $\left(\mathcal{H}_{v},v\in K\right)$ se supone
tal que para cada $\mathbf{z}\in M_{v}$ existe una \'unica curva
integral $\mathbf{\phi}_{v}\left(t,\zeta\right)$ que satisface la
ecuaci\'on

\begin{equation}
\frac{d}{dt}f\left(\zeta_{t}\right)=\mathcal{H}f\left(\zeta_{t}\right),
\end{equation}
con $\zeta_{0}=\mathbf{z}$, para cualquier funci\'on suave
$f:\rea^{d}\rightarrow\rea$ y $\mathcal{H}$ denota el operador
diferencial de primer orden, con $\mathcal{H}=\mathcal{H}_{v}$ y
$\zeta_{t}=\mathbf{\phi}\left(t,\mathbf{z}\right)$. Adem\'as se
supone que $\mathcal{H}_{v}$ es conservativo, es decir, las curvas
integrales est\'an definidas para todo $t>0$.

Para $\mathbf{x}=\left(v,\zeta\right)\in E$ se denota
\[t^{*}\mathbf{x}=inf\left\{t>0:\mathbf{\phi}_{v}\left(t,\zeta\right)\in\partial^{*}M_{v}\right\}\]

En lo que respecta a la funci\'on $\lambda$, se supondr\'a que
para cada $\left(v,\zeta\right)\in E$ existe un $\epsilon>0$ tal
que la funci\'on
$s\rightarrow\lambda\left(v,\phi_{v}\left(s,\zeta\right)\right)\in
E$ es integrable para $s\in\left[0,\epsilon\right)$. La medida de
transici\'on $Q\left(A;\mathbf{x}\right)$ es una funci\'on medible
de $\mathbf{x}$ para cada $A\in\mathcal{E}$, definida para
$\mathbf{x}\in E\cup\Gamma^{*}$ y es una medida de probabilidad en
$\left(E,\mathcal{E}\right)$ para cada $\mathbf{x}\in E$.

El movimiento del proceso $\left(\mathbf{x}_{t}\right)$ comenzando
en $\mathbf{x}=\left(n,\mathbf{z}\right)\in E$ se puede construir
de la siguiente manera, def\'inase la funci\'on $F$ por

\begin{equation}
F\left(t\right)=\left\{\begin{array}{ll}\\
exp\left(-\int_{0}^{t}\lambda\left(n,\phi_{n}\left(s,\mathbf{z}\right)\right)ds\right), & t<t^{*}\left(\mathbf{x}\right),\\
0, & t\geq t^{*}\left(\mathbf{x}\right)
\end{array}\right.
\end{equation}

Sea $T_{1}$ una variable aleatoria tal que
$\prob\left[T_{1}>t\right]=F\left(t\right)$, ahora sea la variable
aleatoria $\left(N,Z\right)$ con distribuici\'on
$Q\left(\cdot;\phi_{n}\left(T_{1},\mathbf{z}\right)\right)$. La
trayectoria de $\left(\mathbf{x}_{t}\right)$ para $t\leq T_{1}$ es
\begin{eqnarray*}
\mathbf{x}_{t}=\left(v_{t},\zeta_{t}\right)=\left\{\begin{array}{ll}
\left(n,\phi_{n}\left(t,\mathbf{z}\right)\right), & t<T_{1},\\
\left(N,\mathbf{Z}\right), & t=t_{1}.
\end{array}\right.
\end{eqnarray*}

Comenzando en $\mathbf{x}_{T_{1}}$ se selecciona el siguiente
tiempo de intersalto $T_{2}-T_{1}$ lugar del post-salto
$\mathbf{x}_{T_{2}}$ de manera similar y as\'i sucesivamente. Este
procedimiento nos da una trayectoria determinista por partes
$\mathbf{x}_{t}$ con tiempos de salto $T_{1},T_{2},\ldots$. Bajo
las condiciones enunciadas para $\lambda,T_{1}>0$  y
$T_{1}-T_{2}>0$ para cada $i$, con probabilidad 1. Se supone que
se cumple la siguiente condici\'on.

\begin{Sup}[Supuesto 3.1, Davis \cite{Davis}]\label{Sup3.1.Davis}
Sea $N_{t}:=\sum_{t}\indora_{\left(t\geq t\right)}$ el n\'umero de
saltos en $\left[0,t\right]$. Entonces
\begin{equation}
\esp\left[N_{t}\right]<\infty\textrm{ para toda }t.
\end{equation}
\end{Sup}

es un proceso de Markov, m\'as a\'un, es un Proceso Fuerte de
Markov, es decir, la Propiedad Fuerte de Markov\footnote{Revisar
p\'agina 362, y 364 de Davis \cite{Davis}.} se cumple para
cualquier tiempo de paro.
%_________________________________________________________________________
%\renewcommand{\refname}{PROCESOS ESTOC\'ASTICOS}
%\renewcommand{\appendixname}{PROCESOS ESTOC\'ASTICOS}
%\renewcommand{\appendixtocname}{PROCESOS ESTOC\'ASTICOS}
%\renewcommand{\appendixpagename}{PROCESOS ESTOC\'ASTICOS}
%\appendix
%\clearpage % o \cleardoublepage
%\addappheadtotoc
%\appendixpage
%_________________________________________________________________________
\subsection{Teor\'ia General de Procesos Estoc\'asticos}
%_________________________________________________________________________
En esta secci\'on se har\'an las siguientes consideraciones: $E$
es un espacio m\'etrico separable y la m\'etrica $d$ es compatible
con la topolog\'ia.

\begin{Def}
Una medida finita, $\lambda$ en la $\sigma$-\'algebra de Borel de
un espacio metrizable $E$ se dice cerrada si
\begin{equation}\label{Eq.A2.3}
\lambda\left(E\right)=sup\left\{\lambda\left(K\right):K\textrm{ es
compacto en }E\right\}.
\end{equation}
\end{Def}

\begin{Def}
$E$ es un espacio de Rad\'on si cada medida finita en
$\left(E,\mathcal{B}\left(E\right)\right)$ es regular interior o cerrada,
{\em tight}.
\end{Def}


El siguiente teorema nos permite tener una mejor caracterizaci\'on de los espacios de Rad\'on:
\begin{Teo}\label{Tma.A2.2}
Sea $E$ espacio separable metrizable. Entonces $E$ es de Rad\'on
si y s\'olo s\'i cada medida finita en
$\left(E,\mathcal{B}\left(E\right)\right)$ es cerrada.
\end{Teo}

%_________________________________________________________________________________________
\subsection{Propiedades de Markov}
%_________________________________________________________________________________________

Sea $E$ espacio de estados, tal que $E$ es un espacio de Rad\'on, $\mathcal{B}\left(E\right)$ $\sigma$-\'algebra de Borel en $E$, que se denotar\'a por $\mathcal{E}$.

Sea $\left(X,\mathcal{G},\prob\right)$ espacio de probabilidad,
$I\subset\rea$ conjunto de índices. Sea $\mathcal{F}_{\leq t}$ la
$\sigma$-\'algebra natural definida como
$\sigma\left\{f\left(X_{r}\right):r\in I, r\leq
t,f\in\mathcal{E}\right\}$. Se considerar\'a una
$\sigma$-\'algebra m\'as general\footnote{qu\'e se quiere decir
con el t\'ermino: m\'as general?}, $ \left(\mathcal{G}_{t}\right)$
tal que $\left(X_{t}\right)$ sea $\mathcal{E}$-adaptado.

\begin{Def}
Una familia $\left(P_{s,t}\right)$ de kernels de Markov en $\left(E,\mathcal{E}\right)$ indexada por pares $s,t\in I$, con $s\leq t$ es una funci\'on de transici\'on en $\ER$, si  para todo $r\leq s< t$ en $I$ y todo $x\in E$, $B\in\mathcal{E}$
\begin{equation}\label{Eq.Kernels}
P_{r,t}\left(x,B\right)=\int_{E}P_{r,s}\left(x,dy\right)P_{s,t}\left(y,B\right)\footnote{Ecuaci\'on de Chapman-Kolmogorov}.
\end{equation}
\end{Def}

Se dice que la funci\'on de transici\'on $\KM$ en $\ER$ es la funci\'on de transici\'on para un proceso $\PE$  con valores en $E$ y que satisface la propiedad de Markov\footnote{\begin{equation}\label{Eq.1.4.S}
\prob\left\{H|\mathcal{G}_{t}\right\}=\prob\left\{H|X_{t}\right\}\textrm{ }H\in p\mathcal{F}_{\geq t}.
\end{equation}} (\ref{Eq.1.4.S}) relativa a $\left(\mathcal{G}_{t}\right)$ si

\begin{equation}\label{Eq.1.6.S}
\prob\left\{f\left(X_{t}\right)|\mathcal{G}_{s}\right\}=P_{s,t}f\left(X_{t}\right)\textrm{ }s\leq t\in I,\textrm{ }f\in b\mathcal{E}.
\end{equation}

\begin{Def}
Una familia $\left(P_{t}\right)_{t\geq0}$ de kernels de Markov en $\ER$ es llamada {\em Semigrupo de Transici\'on de Markov} o {\em Semigrupo de Transici\'on} si
\[P_{t+s}f\left(x\right)=P_{t}\left(P_{s}f\right)\left(x\right),\textrm{ }t,s\geq0,\textrm{ }x\in E\textrm{ }f\in b\mathcal{E}\footnote{Definir los t\'ermino $b\mathcal{E}$ y $p\mathcal{E}$}.\]
\end{Def}
\begin{Note}
Si la funci\'on de transici\'on $\KM$ es llamada homog\'enea si $P_{s,t}=P_{t-s}$.
\end{Note}

Un proceso de Markov que satisface la ecuaci\'on (\ref{Eq.1.6.S}) con funci\'on de transici\'on homog\'enea $\left(P_{t}\right)$ tiene la propiedad caracter\'istica
\begin{equation}\label{Eq.1.8.S}
\prob\left\{f\left(X_{t+s}\right)|\mathcal{G}_{t}\right\}=P_{s}f\left(X_{t}\right)\textrm{ }t,s\geq0,\textrm{ }f\in b\mathcal{E}.
\end{equation}
La ecuaci\'on anterior es la {\em Propiedad Simple de Markov} de $X$ relativa a $\left(P_{t}\right)$.

En este sentido el proceso $\PE$ cumple con la propiedad de Markov (\ref{Eq.1.8.S}) relativa a $\left(\Omega,\mathcal{G},\mathcal{G}_{t},\prob\right)$ con semigrupo de transici\'on $\left(P_{t}\right)$.
%_________________________________________________________________________________________
\subsection{Primer Condici\'on de Regularidad}
%_________________________________________________________________________________________
%\newcommand{\EM}{\left(\Omega,\mathcal{G},\prob\right)}
%\newcommand{\E4}{\left(\Omega,\mathcal{G},\mathcal{G}_{t},\prob\right)}
\begin{Def}
Un proceso estoc\'astico $\PE$ definido en
$\left(\Omega,\mathcal{G},\prob\right)$ con valores en el espacio
topol\'ogico $E$ es continuo por la derecha si cada trayectoria
muestral $t\rightarrow X_{t}\left(w\right)$ es un mapeo continuo
por la derecha de $I$ en $E$.
\end{Def}

\begin{Def}[HD1]\label{Eq.2.1.S}
Un semigrupo de Markov $\left(P_{t}\right)$ en un espacio de
Rad\'on $E$ se dice que satisface la condici\'on {\em HD1} si,
dada una medida de probabilidad $\mu$ en $E$, existe una
$\sigma$-\'algebra $\mathcal{E^{*}}$ con
$\mathcal{E}\subset\mathcal{E}^{*}$ y
$P_{t}\left(b\mathcal{E}^{*}\right)\subset b\mathcal{E}^{*}$, y un
$\mathcal{E}^{*}$-proceso $E$-valuado continuo por la derecha
$\PE$ en alg\'un espacio de probabilidad filtrado
$\left(\Omega,\mathcal{G},\mathcal{G}_{t},\prob\right)$ tal que
$X=\left(\Omega,\mathcal{G},\mathcal{G}_{t},\prob\right)$ es de
Markov (Homog\'eneo) con semigrupo de transici\'on $(P_{t})$ y
distribuci\'on inicial $\mu$.
\end{Def}

Consid\'erese la colecci\'on de variables aleatorias $X_{t}$
definidas en alg\'un espacio de probabilidad, y una colecci\'on de
medidas $\mathbf{P}^{x}$ tales que
$\mathbf{P}^{x}\left\{X_{0}=x\right\}$, y bajo cualquier
$\mathbf{P}^{x}$, $X_{t}$ es de Markov con semigrupo
$\left(P_{t}\right)$. $\mathbf{P}^{x}$ puede considerarse como la
distribuci\'on condicional de $\mathbf{P}$ dado $X_{0}=x$.

\begin{Def}\label{Def.2.2.S}
Sea $E$ espacio de Rad\'on, $\SG$ semigrupo de Markov en $\ER$. La colecci\'on $\mathbf{X}=\left(\Omega,\mathcal{G},\mathcal{G}_{t},X_{t},\theta_{t},\CM\right)$ es un proceso $\mathcal{E}$-Markov continuo por la derecha simple, con espacio de estados $E$ y semigrupo de transici\'on $\SG$ en caso de que $\mathbf{X}$ satisfaga las siguientes condiciones:
\begin{itemize}
\item[i)] $\left(\Omega,\mathcal{G},\mathcal{G}_{t}\right)$ es un espacio de medida filtrado, y $X_{t}$ es un proceso $E$-valuado continuo por la derecha $\mathcal{E}^{*}$-adaptado a $\left(\mathcal{G}_{t}\right)$;

\item[ii)] $\left(\theta_{t}\right)_{t\geq0}$ es una colecci\'on de operadores {\em shift} para $X$, es decir, mapea $\Omega$ en s\'i mismo satisfaciendo para $t,s\geq0$,

\begin{equation}\label{Eq.Shift}
\theta_{t}\circ\theta_{s}=\theta_{t+s}\textrm{ y }X_{t}\circ\theta_{t}=X_{t+s};
\end{equation}

\item[iii)] Para cualquier $x\in E$,$\CM\left\{X_{0}=x\right\}=1$, y el proceso $\PE$ tiene la propiedad de Markov (\ref{Eq.1.8.S}) con semigrupo de transici\'on $\SG$ relativo a $\left(\Omega,\mathcal{G},\mathcal{G}_{t},\CM\right)$.
\end{itemize}
\end{Def}

\begin{Def}[HD2]\label{Eq.2.2.S}
Para cualquier $\alpha>0$ y cualquier $f\in S^{\alpha}$, el proceso $t\rightarrow f\left(X_{t}\right)$ es continuo por la derecha casi seguramente.
\end{Def}

\begin{Def}\label{Def.PD}
Un sistema $\mathbf{X}=\left(\Omega,\mathcal{G},\mathcal{G}_{t},X_{t},\theta_{t},\CM\right)$ es un proceso derecho en el espacio de Rad\'on $E$ con semigrupo de transici\'on $\SG$ provisto de:
\begin{itemize}
\item[i)] $\mathbf{X}$ es una realizaci\'on  continua por la derecha, \ref{Def.2.2.S}, de $\SG$.

\item[ii)] $\mathbf{X}$ satisface la condicion HD2, \ref{Eq.2.2.S}, relativa a $\mathcal{G}_{t}$.

\item[iii)] $\mathcal{G}_{t}$ es aumentado y continuo por la derecha.
\end{itemize}
\end{Def}


%_________________________________________________________________________
%\renewcommand{\refname}{MODELO DE FLUJO}
%\renewcommand{\appendixname}{MODELO DE FLUJO}
%\renewcommand{\appendixtocname}{MODELO DE FLUJO}
%\renewcommand{\appendixpagename}{MODELO DE FLUJO}
%\appendix
%\clearpage % o \cleardoublepage
%\addappheadtotoc
%\appendixpage

\subsection{Construcci\'on del Modelo de Flujo}


\begin{Lema}[Lema 4.2, Dai\cite{Dai}]\label{Lema4.2}
Sea $\left\{x_{n}\right\}\subset \mathbf{X}$ con
$|x_{n}|\rightarrow\infty$, conforme $n\rightarrow\infty$. Suponga
que
\[lim_{n\rightarrow\infty}\frac{1}{|x_{n}|}U\left(0\right)=\overline{U}\]
y
\[lim_{n\rightarrow\infty}\frac{1}{|x_{n}|}V\left(0\right)=\overline{V}.\]

Entonces, conforme $n\rightarrow\infty$, casi seguramente

\begin{equation}\label{E1.4.2}
\frac{1}{|x_{n}|}\Phi^{k}\left(\left[|x_{n}|t\right]\right)\rightarrow
P_{k}^{'}t\textrm{, u.o.c.,}
\end{equation}

\begin{equation}\label{E1.4.3}
\frac{1}{|x_{n}|}E^{x_{n}}_{k}\left(|x_{n}|t\right)\rightarrow
\alpha_{k}\left(t-\overline{U}_{k}\right)^{+}\textrm{, u.o.c.,}
\end{equation}

\begin{equation}\label{E1.4.4}
\frac{1}{|x_{n}|}S^{x_{n}}_{k}\left(|x_{n}|t\right)\rightarrow
\mu_{k}\left(t-\overline{V}_{k}\right)^{+}\textrm{, u.o.c.,}
\end{equation}

donde $\left[t\right]$ es la parte entera de $t$ y
$\mu_{k}=1/m_{k}=1/\esp\left[\eta_{k}\left(1\right)\right]$.
\end{Lema}

\begin{Lema}[Lema 4.3, Dai\cite{Dai}]\label{Lema.4.3}
Sea $\left\{x_{n}\right\}\subset \mathbf{X}$ con
$|x_{n}|\rightarrow\infty$, conforme $n\rightarrow\infty$. Suponga
que
\[lim_{n\rightarrow\infty}\frac{1}{|x_{n}|}U_{k}\left(0\right)=\overline{U}_{k}\]
y
\[lim_{n\rightarrow\infty}\frac{1}{|x_{n}|}V_{k}\left(0\right)=\overline{V}_{k}.\]
\begin{itemize}
\item[a)] Conforme $n\rightarrow\infty$ casi seguramente,
\[lim_{n\rightarrow\infty}\frac{1}{|x_{n}|}U^{x_{n}}_{k}\left(|x_{n}|t\right)=\left(\overline{U}_{k}-t\right)^{+}\textrm{, u.o.c.}\]
y
\[lim_{n\rightarrow\infty}\frac{1}{|x_{n}|}V^{x_{n}}_{k}\left(|x_{n}|t\right)=\left(\overline{V}_{k}-t\right)^{+}.\]

\item[b)] Para cada $t\geq0$ fijo,
\[\left\{\frac{1}{|x_{n}|}U^{x_{n}}_{k}\left(|x_{n}|t\right),|x_{n}|\geq1\right\}\]
y
\[\left\{\frac{1}{|x_{n}|}V^{x_{n}}_{k}\left(|x_{n}|t\right),|x_{n}|\geq1\right\}\]
\end{itemize}
son uniformemente convergentes.
\end{Lema}

Sea $S_{l}^{x}\left(t\right)$ el n\'umero total de servicios
completados de la clase $l$, si la clase $l$ est\'a dando $t$
unidades de tiempo de servicio. Sea $T_{l}^{x}\left(x\right)$ el
monto acumulado del tiempo de servicio que el servidor
$s\left(l\right)$ gasta en los usuarios de la clase $l$ al tiempo
$t$. Entonces $S_{l}^{x}\left(T_{l}^{x}\left(t\right)\right)$ es
el n\'umero total de servicios completados para la clase $l$ al
tiempo $t$. Una fracci\'on de estos usuarios,
$\Phi_{k}^{x}\left(S_{l}^{x}\left(T_{l}^{x}\left(t\right)\right)\right)$,
se convierte en usuarios de la clase $k$.\\

Entonces, dado lo anterior, se tiene la siguiente representaci\'on
para el proceso de la longitud de la cola:\\

\begin{equation}
Q_{k}^{x}\left(t\right)=Q_{k}^{x}\left(0\right)+E_{k}^{x}\left(t\right)+\sum_{l=1}^{K}\Phi_{k}^{l}\left(S_{l}^{x}\left(T_{l}^{x}\left(t\right)\right)\right)-S_{k}^{x}\left(T_{k}^{x}\left(t\right)\right)
\end{equation}
para $k=1,\ldots,K$. Para $i=1,\ldots,d$, sea
\[I_{i}^{x}\left(t\right)=t-\sum_{j\in C_{i}}T_{k}^{x}\left(t\right).\]

Entonces $I_{i}^{x}\left(t\right)$ es el monto acumulado del
tiempo que el servidor $i$ ha estado desocupado al tiempo $t$. Se
est\'a asumiendo que las disciplinas satisfacen la ley de
conservaci\'on del trabajo, es decir, el servidor $i$ est\'a en
pausa solamente cuando no hay usuarios en la estaci\'on $i$.
Entonces, se tiene que

\begin{equation}
\int_{0}^{\infty}\left(\sum_{k\in
C_{i}}Q_{k}^{x}\left(t\right)\right)dI_{i}^{x}\left(t\right)=0,
\end{equation}
para $i=1,\ldots,d$.\\

Hacer
\[T^{x}\left(t\right)=\left(T_{1}^{x}\left(t\right),\ldots,T_{K}^{x}\left(t\right)\right)^{'},\]
\[I^{x}\left(t\right)=\left(I_{1}^{x}\left(t\right),\ldots,I_{K}^{x}\left(t\right)\right)^{'}\]
y
\[S^{x}\left(T^{x}\left(t\right)\right)=\left(S_{1}^{x}\left(T_{1}^{x}\left(t\right)\right),\ldots,S_{K}^{x}\left(T_{K}^{x}\left(t\right)\right)\right)^{'}.\]

Para una disciplina que cumple con la ley de conservaci\'on del
trabajo, en forma vectorial, se tiene el siguiente conjunto de
ecuaciones

\begin{equation}\label{Eq.MF.1.3}
Q^{x}\left(t\right)=Q^{x}\left(0\right)+E^{x}\left(t\right)+\sum_{l=1}^{K}\Phi^{l}\left(S_{l}^{x}\left(T_{l}^{x}\left(t\right)\right)\right)-S^{x}\left(T^{x}\left(t\right)\right),\\
\end{equation}

\begin{equation}\label{Eq.MF.2.3}
Q^{x}\left(t\right)\geq0,\\
\end{equation}

\begin{equation}\label{Eq.MF.3.3}
T^{x}\left(0\right)=0,\textrm{ y }\overline{T}^{x}\left(t\right)\textrm{ es no decreciente},\\
\end{equation}

\begin{equation}\label{Eq.MF.4.3}
I^{x}\left(t\right)=et-CT^{x}\left(t\right)\textrm{ es no
decreciente}\\
\end{equation}

\begin{equation}\label{Eq.MF.5.3}
\int_{0}^{\infty}\left(CQ^{x}\left(t\right)\right)dI_{i}^{x}\left(t\right)=0,\\
\end{equation}

\begin{equation}\label{Eq.MF.6.3}
\textrm{Condiciones adicionales en
}\left(\overline{Q}^{x}\left(\cdot\right),\overline{T}^{x}\left(\cdot\right)\right)\textrm{
espec\'ificas de la disciplina de la cola,}
\end{equation}

donde $e$ es un vector de unos de dimensi\'on $d$, $C$ es la
matriz definida por
\[C_{ik}=\left\{\begin{array}{cc}
1,& S\left(k\right)=i,\\
0,& \textrm{ en otro caso}.\\
\end{array}\right.
\]
Es necesario enunciar el siguiente Teorema que se utilizar\'a para
el Teorema \ref{Tma.4.2.Dai}:
\begin{Teo}[Teorema 4.1, Dai \cite{Dai}]
Considere una disciplina que cumpla la ley de conservaci\'on del
trabajo, para casi todas las trayectorias muestrales $\omega$ y
cualquier sucesi\'on de estados iniciales
$\left\{x_{n}\right\}\subset \mathbf{X}$, con
$|x_{n}|\rightarrow\infty$, existe una subsucesi\'on
$\left\{x_{n_{j}}\right\}$ con $|x_{n_{j}}|\rightarrow\infty$ tal
que
\begin{equation}\label{Eq.4.15}
\frac{1}{|x_{n_{j}}|}\left(Q^{x_{n_{j}}}\left(0\right),U^{x_{n_{j}}}\left(0\right),V^{x_{n_{j}}}\left(0\right)\right)\rightarrow\left(\overline{Q}\left(0\right),\overline{U},\overline{V}\right),
\end{equation}

\begin{equation}\label{Eq.4.16}
\frac{1}{|x_{n_{j}}|}\left(Q^{x_{n_{j}}}\left(|x_{n_{j}}|t\right),T^{x_{n_{j}}}\left(|x_{n_{j}}|t\right)\right)\rightarrow\left(\overline{Q}\left(t\right),\overline{T}\left(t\right)\right)\textrm{
u.o.c.}
\end{equation}

Adem\'as,
$\left(\overline{Q}\left(t\right),\overline{T}\left(t\right)\right)$
satisface las siguientes ecuaciones:
\begin{equation}\label{Eq.MF.1.3a}
\overline{Q}\left(t\right)=Q\left(0\right)+\left(\alpha
t-\overline{U}\right)^{+}-\left(I-P\right)^{'}M^{-1}\left(\overline{T}\left(t\right)-\overline{V}\right)^{+},
\end{equation}

\begin{equation}\label{Eq.MF.2.3a}
\overline{Q}\left(t\right)\geq0,\\
\end{equation}

\begin{equation}\label{Eq.MF.3.3a}
\overline{T}\left(t\right)\textrm{ es no decreciente y comienza en cero},\\
\end{equation}

\begin{equation}\label{Eq.MF.4.3a}
\overline{I}\left(t\right)=et-C\overline{T}\left(t\right)\textrm{
es no decreciente,}\\
\end{equation}

\begin{equation}\label{Eq.MF.5.3a}
\int_{0}^{\infty}\left(C\overline{Q}\left(t\right)\right)d\overline{I}\left(t\right)=0,\\
\end{equation}

\begin{equation}\label{Eq.MF.6.3a}
\textrm{Condiciones adicionales en
}\left(\overline{Q}\left(\cdot\right),\overline{T}\left(\cdot\right)\right)\textrm{
especficas de la disciplina de la cola,}
\end{equation}
\end{Teo}


Propiedades importantes para el modelo de flujo retrasado:

\begin{Prop}
 Sea $\left(\overline{Q},\overline{T},\overline{T}^{0}\right)$ un flujo l\'imite de \ref{Eq.4.4} y suponga que cuando $x\rightarrow\infty$ a lo largo de
una subsucesi\'on
\[\left(\frac{1}{|x|}Q_{k}^{x}\left(0\right),\frac{1}{|x|}A_{k}^{x}\left(0\right),\frac{1}{|x|}B_{k}^{x}\left(0\right),\frac{1}{|x|}B_{k}^{x,0}\left(0\right)\right)\rightarrow\left(\overline{Q}_{k}\left(0\right),0,0,0\right)\]
para $k=1,\ldots,K$. EL flujo l\'imite tiene las siguientes
propiedades, donde las propiedades de la derivada se cumplen donde
la derivada exista:
\begin{itemize}
 \item[i)] Los vectores de tiempo ocupado $\overline{T}\left(t\right)$ y $\overline{T}^{0}\left(t\right)$ son crecientes y continuas con
$\overline{T}\left(0\right)=\overline{T}^{0}\left(0\right)=0$.
\item[ii)] Para todo $t\geq0$
\[\sum_{k=1}^{K}\left[\overline{T}_{k}\left(t\right)+\overline{T}_{k}^{0}\left(t\right)\right]=t\]
\item[iii)] Para todo $1\leq k\leq K$
\[\overline{Q}_{k}\left(t\right)=\overline{Q}_{k}\left(0\right)+\alpha_{k}t-\mu_{k}\overline{T}_{k}\left(t\right)\]
\item[iv)]  Para todo $1\leq k\leq K$
\[\dot{{\overline{T}}}_{k}\left(t\right)=\beta_{k}\] para $\overline{Q}_{k}\left(t\right)=0$.
\item[v)] Para todo $k,j$
\[\mu_{k}^{0}\overline{T}_{k}^{0}\left(t\right)=\mu_{j}^{0}\overline{T}_{j}^{0}\left(t\right)\]
\item[vi)]  Para todo $1\leq k\leq K$
\[\mu_{k}\dot{{\overline{T}}}_{k}\left(t\right)=l_{k}\mu_{k}^{0}\dot{{\overline{T}}}_{k}^{0}\left(t\right)\] para $\overline{Q}_{k}\left(t\right)>0$.
\end{itemize}
\end{Prop}

\begin{Teo}[Teorema 5.1: Ley Fuerte para Procesos de Conteo
\cite{Gut}]\label{Tma.5.1.Gut} Sea
$0<\mu<\esp\left(X_{1}\right]\leq\infty$. entonces

\begin{itemize}
\item[a)] $\frac{N\left(t\right)}{t}\rightarrow\frac{1}{\mu}$
a.s., cuando $t\rightarrow\infty$.


\item[b)]$\esp\left[\frac{N\left(t\right)}{t}\right]^{r}\rightarrow\frac{1}{\mu^{r}}$,
cuando $t\rightarrow\infty$ para todo $r>0$..
\end{itemize}
\end{Teo}


\begin{Prop}[Proposici\'on 5.3 \cite{DaiSean}]
Sea $X$ proceso de estados para la red de colas, y suponga que se
cumplen los supuestos (A1) y (A2), entonces para alguna constante
positiva $C_{p+1}<\infty$, $\delta>0$ y un conjunto compacto
$C\subset X$.

\begin{equation}\label{Eq.5.4}
\esp_{x}\left[\int_{0}^{\tau_{C}\left(\delta\right)}\left(1+|X\left(t\right)|^{p}\right)dt\right]\leq
C_{p+1}\left(1+|x|^{p+1}\right)
\end{equation}
\end{Prop}

\begin{Prop}[Proposici\'on 5.4 \cite{DaiSean}]
Sea $X$ un proceso de Markov Borel Derecho en $X$, sea
$f:X\leftarrow\rea_{+}$ y defina para alguna $\delta>0$, y un
conjunto cerrado $C\subset X$
\[V\left(x\right):=\esp_{x}\left[\int_{0}^{\tau_{C}\left(\delta\right)}f\left(X\left(t\right)\right)dt\right]\]
para $x\in X$. Si $V$ es finito en todas partes y uniformemente
acotada en $C$, entonces existe $k<\infty$ tal que
\begin{equation}\label{Eq.5.11}
\frac{1}{t}\esp_{x}\left[V\left(x\right)\right]+\frac{1}{t}\int_{0}^{t}\esp_{x}\left[f\left(X\left(s\right)\right)ds\right]\leq\frac{1}{t}V\left(x\right)+k,
\end{equation}
para $x\in X$ y $t>0$.
\end{Prop}


%_________________________________________________________________________
%\renewcommand{\refname}{Ap\'endice D}
%\renewcommand{\appendixname}{ESTABILIDAD}
%\renewcommand{\appendixtocname}{ESTABILIDAD}
%\renewcommand{\appendixpagename}{ESTABILIDAD}
%\appendix
%\clearpage % o \cleardoublepage
%\addappheadtotoc
%\appendixpage

\subsection{Estabilidad}

\begin{Def}[Definici\'on 3.2, Dai y Meyn \cite{DaiSean}]
El modelo de flujo retrasado de una disciplina de servicio en una
red con retraso
$\left(\overline{A}\left(0\right),\overline{B}\left(0\right)\right)\in\rea_{+}^{K+|A|}$
se define como el conjunto de ecuaciones dadas en
\ref{Eq.3.8}-\ref{Eq.3.13}, junto con la condici\'on:
\begin{equation}\label{CondAd.FluidModel}
\overline{Q}\left(t\right)=\overline{Q}\left(0\right)+\left(\alpha
t-\overline{A}\left(0\right)\right)^{+}-\left(I-P^{'}\right)M\left(\overline{T}\left(t\right)-\overline{B}\left(0\right)\right)^{+}
\end{equation}
\end{Def}

entonces si el modelo de flujo retrasado tambi\'en es estable:


\begin{Def}[Definici\'on 3.1, Dai y Meyn \cite{DaiSean}]
Un flujo l\'imite (retrasado) para una red bajo una disciplina de
servicio espec\'ifica se define como cualquier soluci\'on
 $\left(\overline{Q}\left(\cdot\right),\overline{T}\left(\cdot\right)\right)$ de las siguientes ecuaciones, donde
$\overline{Q}\left(t\right)=\left(\overline{Q}_{1}\left(t\right),\ldots,\overline{Q}_{K}\left(t\right)\right)^{'}$
y
$\overline{T}\left(t\right)=\left(\overline{T}_{1}\left(t\right),\ldots,\overline{T}_{K}\left(t\right)\right)^{'}$
\begin{equation}\label{Eq.3.8}
\overline{Q}_{k}\left(t\right)=\overline{Q}_{k}\left(0\right)+\alpha_{k}t-\mu_{k}\overline{T}_{k}\left(t\right)+\sum_{l=1}^{k}P_{lk}\mu_{l}\overline{T}_{l}\left(t\right)\\
\end{equation}
\begin{equation}\label{Eq.3.9}
\overline{Q}_{k}\left(t\right)\geq0\textrm{ para }k=1,2,\ldots,K,\\
\end{equation}
\begin{equation}\label{Eq.3.10}
\overline{T}_{k}\left(0\right)=0,\textrm{ y }\overline{T}_{k}\left(\cdot\right)\textrm{ es no decreciente},\\
\end{equation}
\begin{equation}\label{Eq.3.11}
\overline{I}_{i}\left(t\right)=t-\sum_{k\in C_{i}}\overline{T}_{k}\left(t\right)\textrm{ es no decreciente}\\
\end{equation}
\begin{equation}\label{Eq.3.12}
\overline{I}_{i}\left(\cdot\right)\textrm{ se incrementa al tiempo }t\textrm{ cuando }\sum_{k\in C_{i}}Q_{k}^{x}\left(t\right)dI_{i}^{x}\left(t\right)=0\\
\end{equation}
\begin{equation}\label{Eq.3.13}
\textrm{condiciones adicionales sobre
}\left(Q^{x}\left(\cdot\right),T^{x}\left(\cdot\right)\right)\textrm{
referentes a la disciplina de servicio}
\end{equation}
\end{Def}

\begin{Lema}[Lema 3.1 \cite{Chen}]\label{Lema3.1}
Si el modelo de flujo es estable, definido por las ecuaciones
(3.8)-(3.13), entonces el modelo de flujo retrasado tambin es
estable.
\end{Lema}

\begin{Teo}[Teorema 5.1 \cite{Chen}]\label{Tma.5.1.Chen}
La red de colas es estable si existe una constante $t_{0}$ que
depende de $\left(\alpha,\mu,T,U\right)$ y $V$ que satisfagan las
ecuaciones (5.1)-(5.5), $Z\left(t\right)=0$, para toda $t\geq
t_{0}$.
\end{Teo}

\begin{Prop}[Proposici\'on 5.1, Dai y Meyn \cite{DaiSean}]\label{Prop.5.1.DaiSean}
Suponga que los supuestos A1) y A2) son ciertos y que el modelo de flujo es estable. Entonces existe $t_{0}>0$ tal que
\begin{equation}
lim_{|x|\rightarrow\infty}\frac{1}{|x|^{p+1}}\esp_{x}\left[|X\left(t_{0}|x|\right)|^{p+1}\right]=0
\end{equation}
\end{Prop}

\begin{Lemma}[Lema 5.2, Dai y Meyn \cite{DaiSean}]\label{Lema.5.2.DaiSean}
 Sea $\left\{\zeta\left(k\right):k\in \mathbb{z}\right\}$ una sucesi\'on independiente e id\'enticamente distribuida que toma valores en $\left(0,\infty\right)$,
y sea
$E\left(t\right)=max\left(n\geq1:\zeta\left(1\right)+\cdots+\zeta\left(n-1\right)\leq
t\right)$. Si $\esp\left[\zeta\left(1\right)\right]<\infty$,
entonces para cualquier entero $r\geq1$
\begin{equation}
 lim_{t\rightarrow\infty}\esp\left[\left(\frac{E\left(t\right)}{t}\right)^{r}\right]=\left(\frac{1}{\esp\left[\zeta_{1}\right]}\right)^{r}.
\end{equation}
Luego, bajo estas condiciones:
\begin{itemize}
 \item[a)] para cualquier $\delta>0$, $\sup_{t\geq\delta}\esp\left[\left(\frac{E\left(t\right)}{t}\right)^{r}\right]<\infty$
\item[b)] las variables aleatorias
$\left\{\left(\frac{E\left(t\right)}{t}\right)^{r}:t\geq1\right\}$
son uniformemente integrables.
\end{itemize}
\end{Lemma}

\begin{Teo}[Teorema 5.5, Dai y Meyn \cite{DaiSean}]\label{Tma.5.5.DaiSean}
Suponga que los supuestos A1) y A2) se cumplen y que el modelo de
flujo es estable. Entonces existe una constante $\kappa_{p}$ tal
que
\begin{equation}
\frac{1}{t}\int_{0}^{t}\esp_{x}\left[|Q\left(s\right)|^{p}\right]ds\leq\kappa_{p}\left\{\frac{1}{t}|x|^{p+1}+1\right\}
\end{equation}
para $t>0$ y $x\in X$. En particular, para cada condici\'on
inicial
\begin{eqnarray*}
\limsup_{t\rightarrow\infty}\frac{1}{t}\int_{0}^{t}\esp_{x}\left[|Q\left(s\right)|^{p}\right]ds\leq\kappa_{p}.
\end{eqnarray*}
\end{Teo}

\begin{Teo}[Teorema 6.2, Dai y Meyn \cite{DaiSean}]\label{Tma.6.2.DaiSean}
Suponga que se cumplen los supuestos A1), A2) y A3) y que el
modelo de flujo es estable. Entonces se tiene que
\begin{equation}
\left\|P^{t}\left(x,\cdot\right)-\pi\left(\cdot\right)\right\|_{f_{p}}\textrm{,
}t\rightarrow\infty,x\in X.
\end{equation}
En particular para cada condici\'on inicial
\begin{eqnarray*}
\lim_{t\rightarrow\infty}\esp_{x}\left[|Q\left(t\right)|^{p}\right]=\esp_{\pi}\left[|Q\left(0\right)|^{p}\right]\leq\kappa_{r}
\end{eqnarray*}
\end{Teo}
\begin{Teo}[Teorema 6.3, Dai y Meyn \cite{DaiSean}]\label{Tma.6.3.DaiSean}
Suponga que se cumplen los supuestos A1), A2) y A3) y que el
modelo de flujo es estable. Entonces con
$f\left(x\right)=f_{1}\left(x\right)$ se tiene
\begin{equation}
\lim_{t\rightarrow\infty}t^{p-1}\left\|P^{t}\left(x,\cdot\right)-\pi\left(\cdot\right)\right\|_{f}=0.
\end{equation}
En particular para cada condici\'on inicial
\begin{eqnarray*}
\lim_{t\rightarrow\infty}t^{p-1}|\esp_{x}\left[Q\left(t\right)\right]-\esp_{\pi}\left[Q\left(0\right)\right]|=0.
\end{eqnarray*}
\end{Teo}

\begin{Teo}[Teorema 6.4, Dai y Meyn \cite{DaiSean}]\label{Tma.6.4.DaiSean}
Suponga que se cumplen los supuestos A1), A2) y A3) y que el
modelo de flujo es estable. Sea $\nu$ cualquier distribuci\'on de
probabilidad en $\left(X,\mathcal{B}_{X}\right)$, y $\pi$ la
distribuci\'on estacionaria de $X$.
\begin{itemize}
\item[i)] Para cualquier $f:X\leftarrow\rea_{+}$
\begin{equation}
\lim_{t\rightarrow\infty}\frac{1}{t}\int_{o}^{t}f\left(X\left(s\right)\right)ds=\pi\left(f\right):=\int
f\left(x\right)\pi\left(dx\right)
\end{equation}
$\prob$-c.s.

\item[ii)] Para cualquier $f:X\leftarrow\rea_{+}$ con
$\pi\left(|f|\right)<\infty$, la ecuaci\'on anterior se cumple.
\end{itemize}
\end{Teo}

\begin{Teo}[Teorema 2.2, Down \cite{Down}]\label{Tma2.2.Down}
Suponga que el fluido modelo es inestable en el sentido de que
para alguna $\epsilon_{0},c_{0}\geq0$,
\begin{equation}\label{Eq.Inestability}
|Q\left(T\right)|\geq\epsilon_{0}T-c_{0}\textrm{,   }T\geq0,
\end{equation}
para cualquier condici\'on inicial $Q\left(0\right)$, con
$|Q\left(0\right)|=1$. Entonces para cualquier $0<q\leq1$, existe
$B<0$ tal que para cualquier $|x|\geq B$,
\begin{equation}
\prob_{x}\left\{\mathbb{X}\rightarrow\infty\right\}\geq q.
\end{equation}
\end{Teo}


\begin{Def}
Sea $X$ un conjunto y $\mathcal{F}$ una $\sigma$-\'algebra de
subconjuntos de $X$, la pareja $\left(X,\mathcal{F}\right)$ es
llamado espacio medible. Un subconjunto $A$ de $X$ es llamado
medible, o medible con respecto a $\mathcal{F}$, si
$A\in\mathcal{F}$.
\end{Def}

\begin{Def}
Sea $\left(X,\mathcal{F},\mu\right)$ espacio de medida. Se dice
que la medida $\mu$ es $\sigma$-finita si se puede escribir
$X=\bigcup_{n\geq1}X_{n}$ con $X_{n}\in\mathcal{F}$ y
$\mu\left(X_{n}\right)<\infty$.
\end{Def}

\begin{Def}\label{Cto.Borel}
Sea $X$ el conjunto de los \'umeros reales $\rea$. El \'algebra de
Borel es la $\sigma$-\'algebra $B$ generada por los intervalos
abiertos $\left(a,b\right)\in\rea$. Cualquier conjunto en $B$ es
llamado {\em Conjunto de Borel}.
\end{Def}

\begin{Def}\label{Funcion.Medible}
Una funci\'on $f:X\rightarrow\rea$, es medible si para cualquier
n\'umero real $\alpha$ el conjunto
\[\left\{x\in X:f\left(x\right)>\alpha\right\}\]
pertenece a $X$. Equivalentemente, se dice que $f$ es medible si
\[f^{-1}\left(\left(\alpha,\infty\right)\right)=\left\{x\in X:f\left(x\right)>\alpha\right\}\in\mathcal{F}.\]
\end{Def}


\begin{Def}\label{Def.Cilindros}
Sean $\left(\Omega_{i},\mathcal{F}_{i}\right)$, $i=1,2,\ldots,$
espacios medibles y $\Omega=\prod_{i=1}^{\infty}\Omega_{i}$ el
conjunto de todas las sucesiones
$\left(\omega_{1},\omega_{2},\ldots,\right)$ tales que
$\omega_{i}\in\Omega_{i}$, $i=1,2,\ldots,$. Si
$B^{n}\subset\prod_{i=1}^{\infty}\Omega_{i}$, definimos
$B_{n}=\left\{\omega\in\Omega:\left(\omega_{1},\omega_{2},\ldots,\omega_{n}\right)\in
B^{n}\right\}$. Al conjunto $B_{n}$ se le llama {\em cilindro} con
base $B^{n}$, el cilindro es llamado medible si
$B^{n}\in\prod_{i=1}^{\infty}\mathcal{F}_{i}$.
\end{Def}


\begin{Def}\label{Def.Proc.Adaptado}[TSP, Ash \cite{RBA}]
Sea $X\left(t\right),t\geq0$ proceso estoc\'astico, el proceso es
adaptado a la familia de $\sigma$-\'algebras $\mathcal{F}_{t}$,
para $t\geq0$, si para $s<t$ implica que
$\mathcal{F}_{s}\subset\mathcal{F}_{t}$, y $X\left(t\right)$ es
$\mathcal{F}_{t}$-medible para cada $t$. Si no se especifica
$\mathcal{F}_{t}$ entonces se toma $\mathcal{F}_{t}$ como
$\mathcal{F}\left(X\left(s\right),s\leq t\right)$, la m\'as
peque\~na $\sigma$-\'algebra de subconjuntos de $\Omega$ que hace
que cada $X\left(s\right)$, con $s\leq t$ sea Borel medible.
\end{Def}


\begin{Def}\label{Def.Tiempo.Paro}[TSP, Ash \cite{RBA}]
Sea $\left\{\mathcal{F}\left(t\right),t\geq0\right\}$ familia
creciente de sub $\sigma$-\'algebras. es decir,
$\mathcal{F}\left(s\right)\subset\mathcal{F}\left(t\right)$ para
$s\leq t$. Un tiempo de paro para $\mathcal{F}\left(t\right)$ es
una funci\'on $T:\Omega\rightarrow\left[0,\infty\right]$ tal que
$\left\{T\leq t\right\}\in\mathcal{F}\left(t\right)$ para cada
$t\geq0$. Un tiempo de paro para el proceso estoc\'astico
$X\left(t\right),t\geq0$ es un tiempo de paro para las
$\sigma$-\'algebras
$\mathcal{F}\left(t\right)=\mathcal{F}\left(X\left(s\right)\right)$.
\end{Def}

\begin{Def}
Sea $X\left(t\right),t\geq0$ proceso estoc\'astico, con
$\left(S,\chi\right)$ espacio de estados. Se dice que el proceso
es adaptado a $\left\{\mathcal{F}\left(t\right)\right\}$, es
decir, si para cualquier $s,t\in I$, $I$ conjunto de \'indices,
$s<t$, se tiene que
$\mathcal{F}\left(s\right)\subset\mathcal{F}\left(t\right)$ y
$X\left(t\right)$ es $\mathcal{F}\left(t\right)$-medible,
\end{Def}

\begin{Def}
Sea $X\left(t\right),t\geq0$ proceso estoc\'astico, se dice que es
un Proceso de Markov relativo a $\mathcal{F}\left(t\right)$ o que
$\left\{X\left(t\right),\mathcal{F}\left(t\right)\right\}$ es de
Markov si y s\'olo si para cualquier conjunto $B\in\chi$,  y
$s,t\in I$, $s<t$ se cumple que
\begin{equation}\label{Prop.Markov}
P\left\{X\left(t\right)\in
B|\mathcal{F}\left(s\right)\right\}=P\left\{X\left(t\right)\in
B|X\left(s\right)\right\}.
\end{equation}
\end{Def}
\begin{Note}
Si se dice que $\left\{X\left(t\right)\right\}$ es un Proceso de
Markov sin mencionar $\mathcal{F}\left(t\right)$, se asumir\'a que
\begin{eqnarray*}
\mathcal{F}\left(t\right)=\mathcal{F}_{0}\left(t\right)=\mathcal{F}\left(X\left(r\right),r\leq
t\right),
\end{eqnarray*}
entonces la ecuaci\'on (\ref{Prop.Markov}) se puede escribir como
\begin{equation}
P\left\{X\left(t\right)\in B|X\left(r\right),r\leq s\right\} =
P\left\{X\left(t\right)\in B|X\left(s\right)\right\}
\end{equation}
\end{Note}

\begin{Teo}
Sea $\left(X_{n},\mathcal{F}_{n},n=0,1,\ldots,\right\}$ Proceso de
Markov con espacio de estados $\left(S_{0},\chi_{0}\right)$
generado por una distribuici\'on inicial $P_{o}$ y probabilidad de
transici\'on $p_{mn}$, para $m,n=0,1,\ldots,$ $m<n$, que por
notaci\'on se escribir\'a como $p\left(m,n,x,B\right)\rightarrow
p_{mn}\left(x,B\right)$. Sea $S$ tiempo de paro relativo a la
$\sigma$-\'algebra $\mathcal{F}_{n}$. Sea $T$ funci\'on medible,
$T:\Omega\rightarrow\left\{0,1,\ldots,\right\}$. Sup\'ongase que
$T\geq S$, entonces $T$ es tiempo de paro. Si $B\in\chi_{0}$,
entonces
\begin{equation}\label{Prop.Fuerte.Markov}
P\left\{X\left(T\right)\in
B,T<\infty|\mathcal{F}\left(S\right)\right\} =
p\left(S,T,X\left(s\right),B\right)
\end{equation}
en $\left\{T<\infty\right\}$.
\end{Teo}


Sea $K$ conjunto numerable y sea $d:K\rightarrow\nat$ funci\'on.
Para $v\in K$, $M_{v}$ es un conjunto abierto de
$\rea^{d\left(v\right)}$. Entonces \[E=\cup_{v\in
K}M_{v}=\left\{\left(v,\zeta\right):v\in K,\zeta\in
M_{v}\right\}.\]

Sea $\mathcal{E}$ la clase de conjuntos medibles en $E$:
\[\mathcal{E}=\left\{\cup_{v\in K}A_{v}:A_{v}\in \mathcal{M}_{v}\right\}.\]

donde $\mathcal{M}$ son los conjuntos de Borel de $M_{v}$.
Entonces $\left(E,\mathcal{E}\right)$ es un espacio de Borel. El
estado del proceso se denotar\'a por
$\mathbf{x}_{t}=\left(v_{t},\zeta_{t}\right)$. La distribuci\'on
de $\left(\mathbf{x}_{t}\right)$ est\'a determinada por por los
siguientes objetos:

\begin{itemize}
\item[i)] Los campos vectoriales $\left(\mathcal{H}_{v},v\in
K\right)$. \item[ii)] Una funci\'on medible $\lambda:E\rightarrow
\rea_{+}$. \item[iii)] Una medida de transici\'on
$Q:\mathcal{E}\times\left(E\cup\Gamma^{*}\right)\rightarrow\left[0,1\right]$
donde
\begin{equation}
\Gamma^{*}=\cup_{v\in K}\partial^{*}M_{v}.
\end{equation}
y
\begin{equation}
\partial^{*}M_{v}=\left\{z\in\partial M_{v}:\mathbf{\mathbf{\phi}_{v}\left(t,\zeta\right)=\mathbf{z}}\textrm{ para alguna }\left(t,\zeta\right)\in\rea_{+}\times M_{v}\right\}.
\end{equation}
$\partial M_{v}$ denota  la frontera de $M_{v}$.
\end{itemize}

El campo vectorial $\left(\mathcal{H}_{v},v\in K\right)$ se supone
tal que para cada $\mathbf{z}\in M_{v}$ existe una \'unica curva
integral $\mathbf{\phi}_{v}\left(t,\zeta\right)$ que satisface la
ecuaci\'on

\begin{equation}
\frac{d}{dt}f\left(\zeta_{t}\right)=\mathcal{H}f\left(\zeta_{t}\right),
\end{equation}
con $\zeta_{0}=\mathbf{z}$, para cualquier funci\'on suave
$f:\rea^{d}\rightarrow\rea$ y $\mathcal{H}$ denota el operador
diferencial de primer orden, con $\mathcal{H}=\mathcal{H}_{v}$ y
$\zeta_{t}=\mathbf{\phi}\left(t,\mathbf{z}\right)$. Adem\'as se
supone que $\mathcal{H}_{v}$ es conservativo, es decir, las curvas
integrales est\'an definidas para todo $t>0$.

Para $\mathbf{x}=\left(v,\zeta\right)\in E$ se denota
\[t^{*}\mathbf{x}=inf\left\{t>0:\mathbf{\phi}_{v}\left(t,\zeta\right)\in\partial^{*}M_{v}\right\}\]

En lo que respecta a la funci\'on $\lambda$, se supondr\'a que
para cada $\left(v,\zeta\right)\in E$ existe un $\epsilon>0$ tal
que la funci\'on
$s\rightarrow\lambda\left(v,\phi_{v}\left(s,\zeta\right)\right)\in
E$ es integrable para $s\in\left[0,\epsilon\right)$. La medida de
transici\'on $Q\left(A;\mathbf{x}\right)$ es una funci\'on medible
de $\mathbf{x}$ para cada $A\in\mathcal{E}$, definida para
$\mathbf{x}\in E\cup\Gamma^{*}$ y es una medida de probabilidad en
$\left(E,\mathcal{E}\right)$ para cada $\mathbf{x}\in E$.

El movimiento del proceso $\left(\mathbf{x}_{t}\right)$ comenzando
en $\mathbf{x}=\left(n,\mathbf{z}\right)\in E$ se puede construir
de la siguiente manera, def\'inase la funci\'on $F$ por

\begin{equation}
F\left(t\right)=\left\{\begin{array}{ll}\\
exp\left(-\int_{0}^{t}\lambda\left(n,\phi_{n}\left(s,\mathbf{z}\right)\right)ds\right), & t<t^{*}\left(\mathbf{x}\right),\\
0, & t\geq t^{*}\left(\mathbf{x}\right)
\end{array}\right.
\end{equation}

Sea $T_{1}$ una variable aleatoria tal que
$\prob\left[T_{1}>t\right]=F\left(t\right)$, ahora sea la variable
aleatoria $\left(N,Z\right)$ con distribuici\'on
$Q\left(\cdot;\phi_{n}\left(T_{1},\mathbf{z}\right)\right)$. La
trayectoria de $\left(\mathbf{x}_{t}\right)$ para $t\leq T_{1}$
es\footnote{Revisar p\'agina 362, y 364 de Davis \cite{Davis}.}
\begin{eqnarray*}
\mathbf{x}_{t}=\left(v_{t},\zeta_{t}\right)=\left\{\begin{array}{ll}
\left(n,\phi_{n}\left(t,\mathbf{z}\right)\right), & t<T_{1},\\
\left(N,\mathbf{Z}\right), & t=t_{1}.
\end{array}\right.
\end{eqnarray*}

Comenzando en $\mathbf{x}_{T_{1}}$ se selecciona el siguiente
tiempo de intersalto $T_{2}-T_{1}$ lugar del post-salto
$\mathbf{x}_{T_{2}}$ de manera similar y as\'i sucesivamente. Este
procedimiento nos da una trayectoria determinista por partes
$\mathbf{x}_{t}$ con tiempos de salto $T_{1},T_{2},\ldots$. Bajo
las condiciones enunciadas para $\lambda,T_{1}>0$  y
$T_{1}-T_{2}>0$ para cada $i$, con probabilidad 1. Se supone que
se cumple la siquiente condici\'on.

\begin{Sup}[Supuesto 3.1, Davis \cite{Davis}]\label{Sup3.1.Davis}
Sea $N_{t}:=\sum_{t}\indora_{\left(t\geq t\right)}$ el n\'umero de
saltos en $\left[0,t\right]$. Entonces
\begin{equation}
\esp\left[N_{t}\right]<\infty\textrm{ para toda }t.
\end{equation}
\end{Sup}

es un proceso de Markov, m\'as a\'un, es un Proceso Fuerte de
Markov, es decir, la Propiedad Fuerte de Markov se cumple para
cualquier tiempo de paro.
%_________________________________________________________________________

En esta secci\'on se har\'an las siguientes consideraciones: $E$
es un espacio m\'etrico separable y la m\'etrica $d$ es compatible
con la topolog\'ia.


\begin{Def}
Un espacio topol\'ogico $E$ es llamado {\em Luisin} si es
homeomorfo a un subconjunto de Borel de un espacio m\'etrico
compacto.
\end{Def}

\begin{Def}
Un espacio topol\'ogico $E$ es llamado de {\em Rad\'on} si es
homeomorfo a un subconjunto universalmente medible de un espacio
m\'etrico compacto.
\end{Def}

Equivalentemente, la definici\'on de un espacio de Rad\'on puede
encontrarse en los siguientes t\'erminos:


\begin{Def}
$E$ es un espacio de Rad\'on si cada medida finita en
$\left(E,\mathcal{B}\left(E\right)\right)$ es regular interior o cerrada,
{\em tight}.
\end{Def}

\begin{Def}
Una medida finita, $\lambda$ en la $\sigma$-\'algebra de Borel de
un espacio metrizable $E$ se dice cerrada si
\begin{equation}\label{Eq.A2.3}
\lambda\left(E\right)=sup\left\{\lambda\left(K\right):K\textrm{ es
compacto en }E\right\}.
\end{equation}
\end{Def}

El siguiente teorema nos permite tener una mejor caracterizaci\'on de los espacios de Rad\'on:
\begin{Teo}\label{Tma.A2.2}
Sea $E$ espacio separable metrizable. Entonces $E$ es Radoniano si y s\'olo s\'i cada medida finita en $\left(E,\mathcal{B}\left(E\right)\right)$ es cerrada.
\end{Teo}

%_________________________________________________________________________________________
\subsection{Propiedades de Markov}
%_________________________________________________________________________________________

Sea $E$ espacio de estados, tal que $E$ es un espacio de Rad\'on, $\mathcal{B}\left(E\right)$ $\sigma$-\'algebra de Borel en $E$, que se denotar\'a por $\mathcal{E}$.

Sea $\left(X,\mathcal{G},\prob\right)$ espacio de probabilidad, $I\subset\rea$ conjunto de índices. Sea $\mathcal{F}_{\leq t}$ la $\sigma$-\'algebra natural definida como $\sigma\left\{f\left(X_{r}\right):r\in I, rleq t,f\in\mathcal{E}\right\}$. Se considerar\'a una $\sigma$-\'algebra m\'as general, $ \left(\mathcal{G}_{t}\right)$ tal que $\left(X_{t}\right)$ sea $\mathcal{E}$-adaptado.

\begin{Def}
Una familia $\left(P_{s,t}\right)$ de kernels de Markov en $\left(E,\mathcal{E}\right)$ indexada por pares $s,t\in I$, con $s\leq t$ es una funci\'on de transici\'on en $\ER$, si  para todo $r\leq s< t$ en $I$ y todo $x\in E$, $B\in\mathcal{E}$
\begin{equation}\label{Eq.Kernels}
P_{r,t}\left(x,B\right)=\int_{E}P_{r,s}\left(x,dy\right)P_{s,t}\left(y,B\right)\footnote{Ecuaci\'on de Chapman-Kolmogorov}.
\end{equation}
\end{Def}

Se dice que la funci\'on de transici\'on $\KM$ en $\ER$ es la funci\'on de transici\'on para un proceso $\PE$  con valores en $E$ y que satisface la propiedad de Markov\footnote{\begin{equation}\label{Eq.1.4.S}
\prob\left\{H|\mathcal{G}_{t}\right\}=\prob\left\{H|X_{t}\right\}\textrm{ }H\in p\mathcal{F}_{\geq t}.
\end{equation}} (\ref{Eq.1.4.S}) relativa a $\left(\mathcal{G}_{t}\right)$ si 

\begin{equation}\label{Eq.1.6.S}
\prob\left\{f\left(X_{t}\right)|\mathcal{G}_{s}\right\}=P_{s,t}f\left(X_{t}\right)\textrm{ }s\leq t\in I,\textrm{ }f\in b\mathcal{E}.
\end{equation}

\begin{Def}
Una familia $\left(P_{t}\right)_{t\geq0}$ de kernels de Markov en $\ER$ es llamada {\em Semigrupo de Transici\'on de Markov} o {\em Semigrupo de Transici\'on} si
\[P_{t+s}f\left(x\right)=P_{t}\left(P_{s}f\right)\left(x\right),\textrm{ }t,s\geq0,\textrm{ }x\in E\textrm{ }f\in b\mathcal{E}.\]
\end{Def}
\begin{Note}
Si la funci\'on de transici\'on $\KM$ es llamada homog\'enea si $P_{s,t}=P_{t-s}$.
\end{Note}

Un proceso de Markov que satisface la ecuaci\'on (\ref{Eq.1.6.S}) con funci\'on de transici\'on homog\'enea $\left(P_{t}\right)$ tiene la propiedad caracter\'istica
\begin{equation}\label{Eq.1.8.S}
\prob\left\{f\left(X_{t+s}\right)|\mathcal{G}_{t}\right\}=P_{s}f\left(X_{t}\right)\textrm{ }t,s\geq0,\textrm{ }f\in b\mathcal{E}.
\end{equation}
La ecuaci\'on anterior es la {\em Propiedad Simple de Markov} de $X$ relativa a $\left(P_{t}\right)$.

En este sentido el proceso $\PE$ cumple con la propiedad de Markov (\ref{Eq.1.8.S}) relativa a $\left(\Omega,\mathcal{G},\mathcal{G}_{t},\prob\right)$ con semigrupo de transici\'on $\left(P_{t}\right)$.
%_________________________________________________________________________________________
\subsection{Primer Condici\'on de Regularidad}
%_________________________________________________________________________________________
%\newcommand{\EM}{\left(\Omega,\mathcal{G},\prob\right)}
%\newcommand{\E4}{\left(\Omega,\mathcal{G},\mathcal{G}_{t},\prob\right)}
\begin{Def}
Un proceso estoc\'astico $\PE$ definido en $\left(\Omega,\mathcal{G},\prob\right)$ con valores en el espacio topol\'ogico $E$ es continuo por la derecha si cada trayectoria muestral $t\rightarrow X_{t}\left(w\right)$ es un mapeo continuo por la derecha de $I$ en $E$.
\end{Def}

\begin{Def}[HD1]\label{Eq.2.1.S}
Un semigrupo de Markov $\left/P_{t}\right)$ en un espacio de Rad\'on $E$ se dice que satisface la condici\'on {\em HD1} si, dada una medida de probabilidad $\mu$ en $E$, existe una $\sigma$-\'algebra $\mathcal{E^{*}}$ con $\mathcal{E}\subset\mathcal{E}$ y $P_{t}\left(b\mathcal{E}^{*}\right)\subset b\mathcal{E}^{*}$, y un $\mathcal{E}^{*}$-proceso $E$-valuado continuo por la derecha $\PE$ en alg\'un espacio de probabilidad filtrado $\left(\Omega,\mathcal{G},\mathcal{G}_{t},\prob\right)$ tal que $X=\left(\Omega,\mathcal{G},\mathcal{G}_{t},\prob\right)$ es de Markov (Homog\'eneo) con semigrupo de transici\'on $(P_{t})$ y distribuci\'on inicial $\mu$.
\end{Def}

Considerese la colecci\'on de variables aleatorias $X_{t}$ definidas en alg\'un espacio de probabilidad, y una colecci\'on de medidas $\mathbf{P}^{x}$ tales que $\mathbf{P}^{x}\left\{X_{0}=x\right\}$, y bajo cualquier $\mathbf{P}^{x}$, $X_{t}$ es de Markov con semigrupo $\left(P_{t}\right)$. $\mathbf{P}^{x}$ puede considerarse como la distribuci\'on condicional de $\mathbf{P}$ dado $X_{0}=x$.

\begin{Def}\label{Def.2.2.S}
Sea $E$ espacio de Rad\'on, $\SG$ semigrupo de Markov en $\ER$. La colecci\'on $\mathbf{X}=\left(\Omega,\mathcal{G},\mathcal{G}_{t},X_{t},\theta_{t},\CM\right)$ es un proceso $\mathcal{E}$-Markov continuo por la derecha simple, con espacio de estados $E$ y semigrupo de transici\'on $\SG$ en caso de que $\mathbf{X}$ satisfaga las siguientes condiciones:
\begin{itemize}
\item[i)] $\left(\Omega,\mathcal{G},\mathcal{G}_{t}\right)$ es un espacio de medida filtrado, y $X_{t}$ es un proceso $E$-valuado continuo por la derecha $\mathcal{E}^{*}$-adaptado a $\left(\mathcal{G}_{t}\right)$;

\item[ii)] $\left(\theta_{t}\right)_{t\geq0}$ es una colecci\'on de operadores {\em shift} para $X$, es decir, mapea $\Omega$ en s\'i mismo satisfaciendo para $t,s\geq0$,

\begin{equation}\label{Eq.Shift}
\theta_{t}\circ\theta_{s}=\theta_{t+s}\textrm{ y }X_{t}\circ\theta_{t}=X_{t+s};
\end{equation}

\item[iii)] Para cualquier $x\in E$,$\CM\left\{X_{0}=x\right\}=1$, y el proceso $\PE$ tiene la propiedad de Markov (\ref{Eq.1.8.S}) con semigrupo de transici\'on $\SG$ relativo a $\left(\Omega,\mathcal{G},\mathcal{G}_{t},\CM\right)$.
\end{itemize}
\end{Def}

\begin{Def}[HD2]\label{Eq.2.2.S}
Para cualquier $\alpha>0$ y cualquier $f\in S^{\alpha}$, el proceso $t\rightarrow f\left(X_{t}\right)$ es continuo por la derecha casi seguramente.
\end{Def}

\begin{Def}\label{Def.PD}
Un sistema $\mathbf{X}=\left(\Omega,\mathcal{G},\mathcal{G}_{t},X_{t},\theta_{t},\CM\right)$ es un proceso derecho en el espacio de Rad\'on $E$ con semigrupo de transici\'on $\SG$ provisto de:
\begin{itemize}
\item[i)] $\mathbf{X}$ es una realizaci\'on  continua por la derecha, \ref{Def.2.2.S}, de $\SG$.

\item[ii)] $\mathbf{X}$ satisface la condicion HD2, \ref{Eq.2.2.S}, relativa a $\mathcal{G}_{t}$.

\item[iii)] $\mathcal{G}_{t}$ es aumentado y continuo por la derecha.
\end{itemize}
\end{Def}




\begin{Lema}[Lema 4.2, Dai\cite{Dai}]\label{Lema4.2}
Sea $\left\{x_{n}\right\}\subset \mathbf{X}$ con
$|x_{n}|\rightarrow\infty$, conforme $n\rightarrow\infty$. Suponga
que
\[lim_{n\rightarrow\infty}\frac{1}{|x_{n}|}U\left(0\right)=\overline{U}\]
y
\[lim_{n\rightarrow\infty}\frac{1}{|x_{n}|}V\left(0\right)=\overline{V}.\]

Entonces, conforme $n\rightarrow\infty$, casi seguramente

\begin{equation}\label{E1.4.2}
\frac{1}{|x_{n}|}\Phi^{k}\left(\left[|x_{n}|t\right]\right)\rightarrow
P_{k}^{'}t\textrm{, u.o.c.,}
\end{equation}

\begin{equation}\label{E1.4.3}
\frac{1}{|x_{n}|}E^{x_{n}}_{k}\left(|x_{n}|t\right)\rightarrow
\alpha_{k}\left(t-\overline{U}_{k}\right)^{+}\textrm{, u.o.c.,}
\end{equation}

\begin{equation}\label{E1.4.4}
\frac{1}{|x_{n}|}S^{x_{n}}_{k}\left(|x_{n}|t\right)\rightarrow
\mu_{k}\left(t-\overline{V}_{k}\right)^{+}\textrm{, u.o.c.,}
\end{equation}

donde $\left[t\right]$ es la parte entera de $t$ y
$\mu_{k}=1/m_{k}=1/\esp\left[\eta_{k}\left(1\right)\right]$.
\end{Lema}

\begin{Lema}[Lema 4.3, Dai\cite{Dai}]\label{Lema.4.3}
Sea $\left\{x_{n}\right\}\subset \mathbf{X}$ con
$|x_{n}|\rightarrow\infty$, conforme $n\rightarrow\infty$. Suponga
que
\[lim_{n\rightarrow\infty}\frac{1}{|x_{n}|}U\left(0\right)=\overline{U}_{k}\]
y
\[lim_{n\rightarrow\infty}\frac{1}{|x_{n}|}V\left(0\right)=\overline{V}_{k}.\]
\begin{itemize}
\item[a)] Conforme $n\rightarrow\infty$ casi seguramente,
\[lim_{n\rightarrow\infty}\frac{1}{|x_{n}|}U^{x_{n}}_{k}\left(|x_{n}|t\right)=\left(\overline{U}_{k}-t\right)^{+}\textrm{, u.o.c.}\]
y
\[lim_{n\rightarrow\infty}\frac{1}{|x_{n}|}V^{x_{n}}_{k}\left(|x_{n}|t\right)=\left(\overline{V}_{k}-t\right)^{+}.\]

\item[b)] Para cada $t\geq0$ fijo,
\[\left\{\frac{1}{|x_{n}|}U^{x_{n}}_{k}\left(|x_{n}|t\right),|x_{n}|\geq1\right\}\]
y
\[\left\{\frac{1}{|x_{n}|}V^{x_{n}}_{k}\left(|x_{n}|t\right),|x_{n}|\geq1\right\}\]
\end{itemize}
son uniformemente convergentes.
\end{Lema}

$S_{l}^{x}\left(t\right)$ es el n\'umero total de servicios
completados de la clase $l$, si la clase $l$ est\'a dando $t$
unidades de tiempo de servicio. Sea $T_{l}^{x}\left(x\right)$ el
monto acumulado del tiempo de servicio que el servidor
$s\left(l\right)$ gasta en los usuarios de la clase $l$ al tiempo
$t$. Entonces $S_{l}^{x}\left(T_{l}^{x}\left(t\right)\right)$ es
el n\'umero total de servicios completados para la clase $l$ al
tiempo $t$. Una fracci\'on de estos usuarios,
$\Phi_{l}^{x}\left(S_{l}^{x}\left(T_{l}^{x}\left(t\right)\right)\right)$,
se convierte en usuarios de la clase $k$.\\

Entonces, dado lo anterior, se tiene la siguiente representaci\'on
para el proceso de la longitud de la cola:\\

\begin{equation}
Q_{k}^{x}\left(t\right)=_{k}^{x}\left(0\right)+E_{k}^{x}\left(t\right)+\sum_{l=1}^{K}\Phi_{k}^{l}\left(S_{l}^{x}\left(T_{l}^{x}\left(t\right)\right)\right)-S_{k}^{x}\left(T_{k}^{x}\left(t\right)\right)
\end{equation}
para $k=1,\ldots,K$. Para $i=1,\ldots,d$, sea
\[I_{i}^{x}\left(t\right)=t-\sum_{j\in C_{i}}T_{k}^{x}\left(t\right).\]

Entonces $I_{i}^{x}\left(t\right)$ es el monto acumulado del
tiempo que el servidor $i$ ha estado desocupado al tiempo $t$. Se
est\'a asumiendo que las disciplinas satisfacen la ley de
conservaci\'on del trabajo, es decir, el servidor $i$ est\'a en
pausa solamente cuando no hay usuarios en la estaci\'on $i$.
Entonces, se tiene que

\begin{equation}
\int_{0}^{\infty}\left(\sum_{k\in
C_{i}}Q_{k}^{x}\left(t\right)\right)dI_{i}^{x}\left(t\right)=0,
\end{equation}
para $i=1,\ldots,d$.\\

Hacer
\[T^{x}\left(t\right)=\left(T_{1}^{x}\left(t\right),\ldots,T_{K}^{x}\left(t\right)\right)^{'},\]
\[I^{x}\left(t\right)=\left(I_{1}^{x}\left(t\right),\ldots,I_{K}^{x}\left(t\right)\right)^{'}\]
y
\[S^{x}\left(T^{x}\left(t\right)\right)=\left(S_{1}^{x}\left(T_{1}^{x}\left(t\right)\right),\ldots,S_{K}^{x}\left(T_{K}^{x}\left(t\right)\right)\right)^{'}.\]

Para una disciplina que cumple con la ley de conservaci\'on del
trabajo, en forma vectorial, se tiene el siguiente conjunto de
ecuaciones

\begin{equation}\label{Eq.MF.1.3}
Q^{x}\left(t\right)=Q^{x}\left(0\right)+E^{x}\left(t\right)+\sum_{l=1}^{K}\Phi^{l}\left(S_{l}^{x}\left(T_{l}^{x}\left(t\right)\right)\right)-S^{x}\left(T^{x}\left(t\right)\right),\\
\end{equation}

\begin{equation}\label{Eq.MF.2.3}
Q^{x}\left(t\right)\geq0,\\
\end{equation}

\begin{equation}\label{Eq.MF.3.3}
T^{x}\left(0\right)=0,\textrm{ y }\overline{T}^{x}\left(t\right)\textrm{ es no decreciente},\\
\end{equation}

\begin{equation}\label{Eq.MF.4.3}
I^{x}\left(t\right)=et-CT^{x}\left(t\right)\textrm{ es no
decreciente}\\
\end{equation}

\begin{equation}\label{Eq.MF.5.3}
\int_{0}^{\infty}\left(CQ^{x}\left(t\right)\right)dI_{i}^{x}\left(t\right)=0,\\
\end{equation}

\begin{equation}\label{Eq.MF.6.3}
\textrm{Condiciones adicionales en
}\left(\overline{Q}^{x}\left(\cdot\right),\overline{T}^{x}\left(\cdot\right)\right)\textrm{
espec\'ificas de la disciplina de la cola,}
\end{equation}

donde $e$ es un vector de unos de dimensi\'on $d$, $C$ es la
matriz definida por
\[C_{ik}=\left\{\begin{array}{cc}
1,& S\left(k\right)=i,\\
0,& \textrm{ en otro caso}.\\
\end{array}\right.
\]
Es necesario enunciar el siguiente Teorema que se utilizar\'a para
el Teorema \ref{Tma.4.2.Dai}:
\begin{Teo}[Teorema 4.1, Dai \cite{Dai}]
Considere una disciplina que cumpla la ley de conservaci\'on del
trabajo, para casi todas las trayectorias muestrales $\omega$ y
cualquier sucesi\'on de estados iniciales
$\left\{x_{n}\right\}\subset \mathbf{X}$, con
$|x_{n}|\rightarrow\infty$, existe una subsucesi\'on
$\left\{x_{n_{j}}\right\}$ con $|x_{n_{j}}|\rightarrow\infty$ tal
que
\begin{equation}\label{Eq.4.15}
\frac{1}{|x_{n_{j}}|}\left(Q^{x_{n_{j}}}\left(0\right),U^{x_{n_{j}}}\left(0\right),V^{x_{n_{j}}}\left(0\right)\right)\rightarrow\left(\overline{Q}\left(0\right),\overline{U},\overline{V}\right),
\end{equation}

\begin{equation}\label{Eq.4.16}
\frac{1}{|x_{n_{j}}|}\left(Q^{x_{n_{j}}}\left(|x_{n_{j}}|t\right),T^{x_{n_{j}}}\left(|x_{n_{j}}|t\right)\right)\rightarrow\left(\overline{Q}\left(t\right),\overline{T}\left(t\right)\right)\textrm{
u.o.c.}
\end{equation}

Adem\'as,
$\left(\overline{Q}\left(t\right),\overline{T}\left(t\right)\right)$
satisface las siguientes ecuaciones:
\begin{equation}\label{Eq.MF.1.3a}
\overline{Q}\left(t\right)=Q\left(0\right)+\left(\alpha
t-\overline{U}\right)^{+}-\left(I-P\right)^{'}M^{-1}\left(\overline{T}\left(t\right)-\overline{V}\right)^{+},
\end{equation}

\begin{equation}\label{Eq.MF.2.3a}
\overline{Q}\left(t\right)\geq0,\\
\end{equation}

\begin{equation}\label{Eq.MF.3.3a}
\overline{T}\left(t\right)\textrm{ es no decreciente y comienza en cero},\\
\end{equation}

\begin{equation}\label{Eq.MF.4.3a}
\overline{I}\left(t\right)=et-C\overline{T}\left(t\right)\textrm{
es no decreciente,}\\
\end{equation}

\begin{equation}\label{Eq.MF.5.3a}
\int_{0}^{\infty}\left(C\overline{Q}\left(t\right)\right)d\overline{I}\left(t\right)=0,\\
\end{equation}

\begin{equation}\label{Eq.MF.6.3a}
\textrm{Condiciones adicionales en
}\left(\overline{Q}\left(\cdot\right),\overline{T}\left(\cdot\right)\right)\textrm{
especficas de la disciplina de la cola,}
\end{equation}
\end{Teo}

\begin{Def}[Definici\'on 4.1, , Dai \cite{Dai}]
Sea una disciplina de servicio espec\'ifica. Cualquier l\'imite
$\left(\overline{Q}\left(\cdot\right),\overline{T}\left(\cdot\right)\right)$
en \ref{Eq.4.16} es un {\em flujo l\'imite} de la disciplina.
Cualquier soluci\'on (\ref{Eq.MF.1.3a})-(\ref{Eq.MF.6.3a}) es
llamado flujo soluci\'on de la disciplina. Se dice que el modelo de flujo l\'imite, modelo de flujo, de la disciplina de la cola es estable si existe una constante
$\delta>0$ que depende de $\mu,\alpha$ y $P$ solamente, tal que
cualquier flujo l\'imite con
$|\overline{Q}\left(0\right)|+|\overline{U}|+|\overline{V}|=1$, se
tiene que $\overline{Q}\left(\cdot+\delta\right)\equiv0$.
\end{Def}

\begin{Teo}[Teorema 4.2, Dai\cite{Dai}]\label{Tma.4.2.Dai}
Sea una disciplina fija para la cola, suponga que se cumplen las
condiciones (1.2)-(1.5). Si el modelo de flujo l\'imite de la
disciplina de la cola es estable, entonces la cadena de Markov $X$
que describe la din\'amica de la red bajo la disciplina es Harris
recurrente positiva.
\end{Teo}

Ahora se procede a escalar el espacio y el tiempo para reducir la
aparente fluctuaci\'on del modelo. Consid\'erese el proceso
\begin{equation}\label{Eq.3.7}
\overline{Q}^{x}\left(t\right)=\frac{1}{|x|}Q^{x}\left(|x|t\right)
\end{equation}
A este proceso se le conoce como el fluido escalado, y cualquier l\'imite $\overline{Q}^{x}\left(t\right)$ es llamado flujo l\'imite del proceso de longitud de la cola. Haciendo $|q|\rightarrow\infty$ mientras se mantiene el resto de las componentes fijas, cualquier punto l\'imite del proceso de longitud de la cola normalizado $\overline{Q}^{x}$ es soluci\'on del siguiente modelo de flujo.

Al conjunto de ecuaciones dadas en \ref{Eq.3.8}-\ref{Eq.3.13} se
le llama {\em Modelo de flujo} y al conjunto de todas las
soluciones del modelo de flujo
$\left(\overline{Q}\left(\cdot\right),\overline{T}
\left(\cdot\right)\right)$ se le denotar\'a por $\mathcal{Q}$.

Si se hace $|x|\rightarrow\infty$ sin restringir ninguna de las
componentes, tambi\'en se obtienen un modelo de flujo, pero en
este caso el residual de los procesos de arribo y servicio
introducen un retraso:

\begin{Def}[Definici\'on 3.3, Dai y Meyn \cite{DaiSean}]
El modelo de flujo es estable si existe un tiempo fijo $t_{0}$ tal
que $\overline{Q}\left(t\right)=0$, con $t\geq t_{0}$, para
cualquier $\overline{Q}\left(\cdot\right)\in\mathcal{Q}$ que
cumple con $|\overline{Q}\left(0\right)|=1$.
\end{Def}

El siguiente resultado se encuentra en Chen \cite{Chen}.
\begin{Lemma}[Lema 3.1, Dai y Meyn \cite{DaiSean}]
Si el modelo de flujo definido por \ref{Eq.3.8}-\ref{Eq.3.13} es
estable, entonces el modelo de flujo retrasado es tambi\'en
estable, es decir, existe $t_{0}>0$ tal que
$\overline{Q}\left(t\right)=0$ para cualquier $t\geq t_{0}$, para
cualquier soluci\'on del modelo de flujo retrasado cuya
condici\'on inicial $\overline{x}$ satisface que
$|\overline{x}|=|\overline{Q}\left(0\right)|+|\overline{A}\left(0\right)|+|\overline{B}\left(0\right)|\leq1$.
\end{Lemma}


Propiedades importantes para el modelo de flujo retrasado:

\begin{Prop}
 Sea $\left(\overline{Q},\overline{T},\overline{T}^{0}\right)$ un flujo l\'imite de \ref{Eq.4.4} y suponga que cuando $x\rightarrow\infty$ a lo largo de
una subsucesi\'on
\[\left(\frac{1}{|x|}Q_{k}^{x}\left(0\right),\frac{1}{|x|}A_{k}^{x}\left(0\right),\frac{1}{|x|}B_{k}^{x}\left(0\right),\frac{1}{|x|}B_{k}^{x,0}\left(0\right)\right)\rightarrow\left(\overline{Q}_{k}\left(0\right),0,0,0\right)\]
para $k=1,\ldots,K$. EL flujo l\'imite tiene las siguientes
propiedades, donde las propiedades de la derivada se cumplen donde
la derivada exista:
\begin{itemize}
 \item[i)] Los vectores de tiempo ocupado $\overline{T}\left(t\right)$ y $\overline{T}^{0}\left(t\right)$ son crecientes y continuas con
$\overline{T}\left(0\right)=\overline{T}^{0}\left(0\right)=0$.
\item[ii)] Para todo $t\geq0$
\[\sum_{k=1}^{K}\left[\overline{T}_{k}\left(t\right)+\overline{T}_{k}^{0}\left(t\right)\right]=t\]
\item[iii)] Para todo $1\leq k\leq K$
\[\overline{Q}_{k}\left(t\right)=\overline{Q}_{k}\left(0\right)+\alpha_{k}t-\mu_{k}\overline{T}_{k}\left(t\right)\]
\item[iv)]  Para todo $1\leq k\leq K$
\[\dot{{\overline{T}}}_{k}\left(t\right)=\beta_{k}\] para $\overline{Q}_{k}\left(t\right)=0$.
\item[v)] Para todo $k,j$
\[\mu_{k}^{0}\overline{T}_{k}^{0}\left(t\right)=\mu_{j}^{0}\overline{T}_{j}^{0}\left(t\right)\]
\item[vi)]  Para todo $1\leq k\leq K$
\[\mu_{k}\dot{{\overline{T}}}_{k}\left(t\right)=l_{k}\mu_{k}^{0}\dot{{\overline{T}}}_{k}^{0}\left(t\right)\] para $\overline{Q}_{k}\left(t\right)>0$.
\end{itemize}
\end{Prop}

\begin{Lema}[Lema 3.1 \cite{Chen}]\label{Lema3.1}
Si el modelo de flujo es estable, definido por las ecuaciones
(3.8)-(3.13), entonces el modelo de flujo retrasado tambin es
estable.
\end{Lema}

\begin{Teo}[Teorema 5.2 \cite{Chen}]\label{Tma.5.2}
Si el modelo de flujo lineal correspondiente a la red de cola es
estable, entonces la red de colas es estable.
\end{Teo}

\begin{Teo}[Teorema 5.1 \cite{Chen}]\label{Tma.5.1.Chen}
La red de colas es estable si existe una constante $t_{0}$ que
depende de $\left(\alpha,\mu,T,U\right)$ y $V$ que satisfagan las
ecuaciones (5.1)-(5.5), $Z\left(t\right)=0$, para toda $t\geq
t_{0}$.
\end{Teo}



\begin{Lema}[Lema 5.2 \cite{Gut}]\label{Lema.5.2.Gut}
Sea $\left\{\xi\left(k\right):k\in\ent\right\}$ sucesin de
variables aleatorias i.i.d. con valores en
$\left(0,\infty\right)$, y sea $E\left(t\right)$ el proceso de
conteo
\[E\left(t\right)=max\left\{n\geq1:\xi\left(1\right)+\cdots+\xi\left(n-1\right)\leq t\right\}.\]
Si $E\left[\xi\left(1\right)\right]<\infty$, entonces para
cualquier entero $r\geq1$
\begin{equation}
lim_{t\rightarrow\infty}\esp\left[\left(\frac{E\left(t\right)}{t}\right)^{r}\right]=\left(\frac{1}{E\left[\xi_{1}\right]}\right)^{r}
\end{equation}
de aqu, bajo estas condiciones
\begin{itemize}
\item[a)] Para cualquier $t>0$,
$sup_{t\geq\delta}\esp\left[\left(\frac{E\left(t\right)}{t}\right)^{r}\right]$

\item[b)] Las variables aleatorias
$\left\{\left(\frac{E\left(t\right)}{t}\right)^{r}:t\geq1\right\}$
son uniformemente integrables.
\end{itemize}
\end{Lema}

\begin{Teo}[Teorema 5.1: Ley Fuerte para Procesos de Conteo
\cite{Gut}]\label{Tma.5.1.Gut} Sea
$0<\mu<\esp\left(X_{1}\right]\leq\infty$. entonces

\begin{itemize}
\item[a)] $\frac{N\left(t\right)}{t}\rightarrow\frac{1}{\mu}$
a.s., cuando $t\rightarrow\infty$.


\item[b)]$\esp\left[\frac{N\left(t\right)}{t}\right]^{r}\rightarrow\frac{1}{\mu^{r}}$,
cuando $t\rightarrow\infty$ para todo $r>0$..
\end{itemize}
\end{Teo}


\begin{Prop}[Proposicin 5.1 \cite{DaiSean}]\label{Prop.5.1}
Suponga que los supuestos (A1) y (A2) se cumplen, adems suponga
que el modelo de flujo es estable. Entonces existe $t_{0}>0$ tal
que
\begin{equation}\label{Eq.Prop.5.1}
lim_{|x|\rightarrow\infty}\frac{1}{|x|^{p+1}}\esp_{x}\left[|X\left(t_{0}|x|\right)|^{p+1}\right]=0.
\end{equation}

\end{Prop}


\begin{Prop}[Proposici\'on 5.3 \cite{DaiSean}]
Sea $X$ proceso de estados para la red de colas, y suponga que se
cumplen los supuestos (A1) y (A2), entonces para alguna constante
positiva $C_{p+1}<\infty$, $\delta>0$ y un conjunto compacto
$C\subset X$.

\begin{equation}\label{Eq.5.4}
\esp_{x}\left[\int_{0}^{\tau_{C}\left(\delta\right)}\left(1+|X\left(t\right)|^{p}\right)dt\right]\leq
C_{p+1}\left(1+|x|^{p+1}\right)
\end{equation}
\end{Prop}

\begin{Prop}[Proposici\'on 5.4 \cite{DaiSean}]
Sea $X$ un proceso de Markov Borel Derecho en $X$, sea
$f:X\leftarrow\rea_{+}$ y defina para alguna $\delta>0$, y un
conjunto cerrado $C\subset X$
\[V\left(x\right):=\esp_{x}\left[\int_{0}^{\tau_{C}\left(\delta\right)}f\left(X\left(t\right)\right)dt\right]\]
para $x\in X$. Si $V$ es finito en todas partes y uniformemente
acotada en $C$, entonces existe $k<\infty$ tal que
\begin{equation}\label{Eq.5.11}
\frac{1}{t}\esp_{x}\left[V\left(x\right)\right]+\frac{1}{t}\int_{0}^{t}\esp_{x}\left[f\left(X\left(s\right)\right)ds\right]\leq\frac{1}{t}V\left(x\right)+k,
\end{equation}
para $x\in X$ y $t>0$.
\end{Prop}


\begin{Teo}[Teorema 5.5 \cite{DaiSean}]
Suponga que se cumplen (A1) y (A2), adems suponga que el modelo
de flujo es estable. Entonces existe una constante $k_{p}<\infty$
tal que
\begin{equation}\label{Eq.5.13}
\frac{1}{t}\int_{0}^{t}\esp_{x}\left[|Q\left(s\right)|^{p}\right]ds\leq
k_{p}\left\{\frac{1}{t}|x|^{p+1}+1\right\}
\end{equation}
para $t\geq0$, $x\in X$. En particular para cada condici\'on inicial
\begin{equation}\label{Eq.5.14}
Limsup_{t\rightarrow\infty}\frac{1}{t}\int_{0}^{t}\esp_{x}\left[|Q\left(s\right)|^{p}\right]ds\leq
k_{p}
\end{equation}
\end{Teo}

\begin{Teo}[Teorema 6.2\cite{DaiSean}]\label{Tma.6.2}
Suponga que se cumplen los supuestos (A1)-(A3) y que el modelo de
flujo es estable, entonces se tiene que
\[\parallel P^{t}\left(c,\cdot\right)-\pi\left(\cdot\right)\parallel_{f_{p}}\rightarrow0\]
para $t\rightarrow\infty$ y $x\in X$. En particular para cada
condicin inicial
\[lim_{t\rightarrow\infty}\esp_{x}\left[\left|Q_{t}\right|^{p}\right]=\esp_{\pi}\left[\left|Q_{0}\right|^{p}\right]<\infty\]
\end{Teo}


\begin{Teo}[Teorema 6.3\cite{DaiSean}]\label{Tma.6.3}
Suponga que se cumplen los supuestos (A1)-(A3) y que el modelo de
flujo es estable, entonces con
$f\left(x\right)=f_{1}\left(x\right)$, se tiene que
\[lim_{t\rightarrow\infty}t^{(p-1)\left|P^{t}\left(c,\cdot\right)-\pi\left(\cdot\right)\right|_{f}=0},\]
para $x\in X$. En particular, para cada condicin inicial
\[lim_{t\rightarrow\infty}t^{(p-1)\left|\esp_{x}\left[Q_{t}\right]-\esp_{\pi}\left[Q_{0}\right]\right|=0}.\]
\end{Teo}


\begin{Prop}[Proposici\'on 5.1, Dai y Meyn \cite{DaiSean}]\label{Prop.5.1.DaiSean}
Suponga que los supuestos A1) y A2) son ciertos y que el modelo de flujo es estable. Entonces existe $t_{0}>0$ tal que
\begin{equation}
lim_{|x|\rightarrow\infty}\frac{1}{|x|^{p+1}}\esp_{x}\left[|X\left(t_{0}|x|\right)|^{p+1}\right]=0
\end{equation}
\end{Prop}

\begin{Lemma}[Lema 5.2, Dai y Meyn \cite{DaiSean}]\label{Lema.5.2.DaiSean}
 Sea $\left\{\zeta\left(k\right):k\in \mathbb{z}\right\}$ una sucesi\'on independiente e id\'enticamente distribuida que toma valores en $\left(0,\infty\right)$,
y sea
$E\left(t\right)=max\left(n\geq1:\zeta\left(1\right)+\cdots+\zeta\left(n-1\right)\leq
t\right)$. Si $\esp\left[\zeta\left(1\right)\right]<\infty$,
entonces para cualquier entero $r\geq1$
\begin{equation}
 lim_{t\rightarrow\infty}\esp\left[\left(\frac{E\left(t\right)}{t}\right)^{r}\right]=\left(\frac{1}{\esp\left[\zeta_{1}\right]}\right)^{r}.
\end{equation}
Luego, bajo estas condiciones:
\begin{itemize}
 \item[a)] para cualquier $\delta>0$, $\sup_{t\geq\delta}\esp\left[\left(\frac{E\left(t\right)}{t}\right)^{r}\right]<\infty$
\item[b)] las variables aleatorias
$\left\{\left(\frac{E\left(t\right)}{t}\right)^{r}:t\geq1\right\}$
son uniformemente integrables.
\end{itemize}
\end{Lemma}

\begin{Teo}[Teorema 5.5, Dai y Meyn \cite{DaiSean}]\label{Tma.5.5.DaiSean}
Suponga que los supuestos A1) y A2) se cumplen y que el modelo de
flujo es estable. Entonces existe una constante $\kappa_{p}$ tal
que
\begin{equation}
\frac{1}{t}\int_{0}^{t}\esp_{x}\left[|Q\left(s\right)|^{p}\right]ds\leq\kappa_{p}\left\{\frac{1}{t}|x|^{p+1}+1\right\}
\end{equation}
para $t>0$ y $x\in X$. En particular, para cada condici\'on
inicial
\begin{eqnarray*}
\limsup_{t\rightarrow\infty}\frac{1}{t}\int_{0}^{t}\esp_{x}\left[|Q\left(s\right)|^{p}\right]ds\leq\kappa_{p}.
\end{eqnarray*}
\end{Teo}

\begin{Teo}[Teorema 6.2, Dai y Meyn \cite{DaiSean}]\label{Tma.6.2.DaiSean}
Suponga que se cumplen los supuestos A1), A2) y A3) y que el
modelo de flujo es estable. Entonces se tiene que
\begin{equation}
\left\|P^{t}\left(x,\cdot\right)-\pi\left(\cdot\right)\right\|_{f_{p}}\textrm{,
}t\rightarrow\infty,x\in X.
\end{equation}
En particular para cada condici\'on inicial
\begin{eqnarray*}
\lim_{t\rightarrow\infty}\esp_{x}\left[|Q\left(t\right)|^{p}\right]=\esp_{\pi}\left[|Q\left(0\right)|^{p}\right]\leq\kappa_{r}
\end{eqnarray*}
\end{Teo}
\begin{Teo}[Teorema 6.3, Dai y Meyn \cite{DaiSean}]\label{Tma.6.3.DaiSean}
Suponga que se cumplen los supuestos A1), A2) y A3) y que el
modelo de flujo es estable. Entonces con
$f\left(x\right)=f_{1}\left(x\right)$ se tiene
\begin{equation}
\lim_{t\rightarrow\infty}t^{p-1}\left\|P^{t}\left(x,\cdot\right)-\pi\left(\cdot\right)\right\|_{f}=0.
\end{equation}
En particular para cada condici\'on inicial
\begin{eqnarray*}
\lim_{t\rightarrow\infty}t^{p-1}|\esp_{x}\left[Q\left(t\right)\right]-\esp_{\pi}\left[Q\left(0\right)\right]|=0.
\end{eqnarray*}
\end{Teo}

\begin{Teo}[Teorema 6.4, Dai y Meyn \cite{DaiSean}]\label{Tma.6.4.DaiSean}
Suponga que se cumplen los supuestos A1), A2) y A3) y que el
modelo de flujo es estable. Sea $\nu$ cualquier distribuci\'on de
probabilidad en $\left(X,\mathcal{B}_{X}\right)$, y $\pi$ la
distribuci\'on estacionaria de $X$.
\begin{itemize}
\item[i)] Para cualquier $f:X\leftarrow\rea_{+}$
\begin{equation}
\lim_{t\rightarrow\infty}\frac{1}{t}\int_{o}^{t}f\left(X\left(s\right)\right)ds=\pi\left(f\right):=\int
f\left(x\right)\pi\left(dx\right)
\end{equation}
$\prob$-c.s.

\item[ii)] Para cualquier $f:X\leftarrow\rea_{+}$ con
$\pi\left(|f|\right)<\infty$, la ecuaci\'on anterior se cumple.
\end{itemize}
\end{Teo}

\begin{Teo}[Teorema 2.2, Down \cite{Down}]\label{Tma2.2.Down}
Suponga que el fluido modelo es inestable en el sentido de que
para alguna $\epsilon_{0},c_{0}\geq0$,
\begin{equation}\label{Eq.Inestability}
|Q\left(T\right)|\geq\epsilon_{0}T-c_{0}\textrm{,   }T\geq0,
\end{equation}
para cualquier condici\'on inicial $Q\left(0\right)$, con
$|Q\left(0\right)|=1$. Entonces para cualquier $0<q\leq1$, existe
$B<0$ tal que para cualquier $|x|\geq B$,
\begin{equation}
\prob_{x}\left\{\mathbb{X}\rightarrow\infty\right\}\geq q.
\end{equation}
\end{Teo}



Es necesario hacer los siguientes supuestos sobre el
comportamiento del sistema de visitas c\'iclicas:
\begin{itemize}
\item Los tiempos de interarribo a la $k$-\'esima cola, son de la
forma $\left\{\xi_{k}\left(n\right)\right\}_{n\geq1}$, con la
propiedad de que son independientes e id{\'e}nticamente
distribuidos,
\item Los tiempos de servicio
$\left\{\eta_{k}\left(n\right)\right\}_{n\geq1}$ tienen la
propiedad de ser independientes e id{\'e}nticamente distribuidos,
\item Se define la tasa de arribo a la $k$-{\'e}sima cola como
$\lambda_{k}=1/\esp\left[\xi_{k}\left(1\right)\right]$,
\item la tasa de servicio para la $k$-{\'e}sima cola se define
como $\mu_{k}=1/\esp\left[\eta_{k}\left(1\right)\right]$,
\item tambi{\'e}n se define $\rho_{k}:=\lambda_{k}/\mu_{k}$, la
intensidad de tr\'afico del sistema o carga de la red, donde es
necesario que $\rho<1$ para cuestiones de estabilidad.
\end{itemize}



%_________________________________________________________________________
\subsection{Procesos de Estados Markoviano para el Sistema}
%_________________________________________________________________________

%_________________________________________________________________________
\subsection{Procesos Fuerte de Markov}
%_________________________________________________________________________
En Dai \cite{Dai} se muestra que para una amplia serie de disciplinas
de servicio el proceso $X$ es un Proceso Fuerte de
Markov, y por tanto se puede asumir que


Para establecer que $X=\left\{X\left(t\right),t\geq0\right\}$ es
un Proceso Fuerte de Markov, se siguen las secciones 2.3 y 2.4 de Kaspi and Mandelbaum \cite{KaspiMandelbaum}. \\

%______________________________________________________________
\subsubsection{Construcci\'on de un Proceso Determinista por partes, Davis
\cite{Davis}}.
%______________________________________________________________

%_________________________________________________________________________
\subsection{Procesos Harris Recurrentes Positivos}
%_________________________________________________________________________
Sea el proceso de Markov $X=\left\{X\left(t\right),t\geq0\right\}$
que describe la din\'amica de la red de colas. En lo que respecta
al supuesto (A3), en Dai y Meyn \cite{DaiSean} y Meyn y Down
\cite{MeynDown} hacen ver que este se puede sustituir por

\begin{itemize}
\item[A3')] Para el Proceso de Markov $X$, cada subconjunto
compacto de $X$ es un conjunto peque\~no.
\end{itemize}

Este supuesto es importante pues es un requisito para deducir la ergodicidad de la red.

%_________________________________________________________________________
\subsection{Construcci\'on de un Modelo de Flujo L\'imite}
%_________________________________________________________________________

Consideremos un caso m\'as simple para poner en contexto lo
anterior: para un sistema de visitas c\'iclicas se tiene que el
estado al tiempo $t$ es
\begin{equation}
X\left(t\right)=\left(Q\left(t\right),U\left(t\right),V\left(t\right)\right),
\end{equation}

donde $Q\left(t\right)$ es el n\'umero de usuarios formados en
cada estaci\'on. $U\left(t\right)$ es el tiempo restante antes de
que la siguiente clase $k$ de usuarios lleguen desde fuera del
sistema, $V\left(t\right)$ es el tiempo restante de servicio para
la clase $k$ de usuarios que est\'an siendo atendidos. Tanto
$U\left(t\right)$ como $V\left(t\right)$ se puede asumir que son
continuas por la derecha.

Sea
$x=\left(Q\left(0\right),U\left(0\right),V\left(0\right)\right)=\left(q,a,b\right)$,
el estado inicial de la red bajo una disciplina espec\'ifica para
la cola. Para $l\in\mathcal{E}$, donde $\mathcal{E}$ es el conjunto de clases de arribos externos, y $k=1,\ldots,K$ se define\\
\begin{eqnarray*}
E_{l}^{x}\left(t\right)&=&max\left\{r:U_{l}\left(0\right)+\xi_{l}\left(1\right)+\cdots+\xi_{l}\left(r-1\right)\leq
t\right\}\textrm{   }t\geq0,\\
S_{k}^{x}\left(t\right)&=&max\left\{r:V_{k}\left(0\right)+\eta_{k}\left(1\right)+\cdots+\eta_{k}\left(r-1\right)\leq
t\right\}\textrm{   }t\geq0.
\end{eqnarray*}

Para cada $k$ y cada $n$ se define

\begin{eqnarray*}\label{Eq.phi}
\Phi^{k}\left(n\right):=\sum_{i=1}^{n}\phi^{k}\left(i\right).
\end{eqnarray*}

donde $\phi^{k}\left(n\right)$ se define como el vector de ruta
para el $n$-\'esimo usuario de la clase $k$ que termina en la
estaci\'on $s\left(k\right)$, la $s$-\'eima componente de
$\phi^{k}\left(n\right)$ es uno si estos usuarios se convierten en
usuarios de la clase $l$ y cero en otro caso, por lo tanto
$\phi^{k}\left(n\right)$ es un vector {\em Bernoulli} de
dimensi\'on $K$ con par\'ametro $P_{k}^{'}$, donde $P_{k}$ denota
el $k$-\'esimo rengl\'on de $P=\left(P_{kl}\right)$.

Se asume que cada para cada $k$ la sucesi\'on $\phi^{k}\left(n\right)=\left\{\phi^{k}\left(n\right),n\geq1\right\}$
es independiente e id\'enticamente distribuida y que las
$\phi^{1}\left(n\right),\ldots,\phi^{K}\left(n\right)$ son
mutuamente independientes, adem\'as de independientes de los
procesos de arribo y de servicio.\\

\begin{Lema}[Lema 4.2, Dai\cite{Dai}]\label{Lema4.2}
Sea $\left\{x_{n}\right\}\subset \mathbf{X}$ con
$|x_{n}|\rightarrow\infty$, conforme $n\rightarrow\infty$. Suponga
que
\[lim_{n\rightarrow\infty}\frac{1}{|x_{n}|}U\left(0\right)=\overline{U}\]
y
\[lim_{n\rightarrow\infty}\frac{1}{|x_{n}|}V\left(0\right)=\overline{V}.\]

Entonces, conforme $n\rightarrow\infty$, casi seguramente

\begin{equation}\label{E1.4.2}
\frac{1}{|x_{n}|}\Phi^{k}\left(\left[|x_{n}|t\right]\right)\rightarrow
P_{k}^{'}t\textrm{, u.o.c.,}
\end{equation}

\begin{equation}\label{E1.4.3}
\frac{1}{|x_{n}|}E^{x_{n}}_{k}\left(|x_{n}|t\right)\rightarrow
\alpha_{k}\left(t-\overline{U}_{k}\right)^{+}\textrm{, u.o.c.,}
\end{equation}

\begin{equation}\label{E1.4.4}
\frac{1}{|x_{n}|}S^{x_{n}}_{k}\left(|x_{n}|t\right)\rightarrow
\mu_{k}\left(t-\overline{V}_{k}\right)^{+}\textrm{, u.o.c.,}
\end{equation}

donde $\left[t\right]$ es la parte entera de $t$ y
$\mu_{k}=1/m_{k}=1/\esp\left[\eta_{k}\left(1\right)\right]$.
\end{Lema}

\begin{Lema}[Lema 4.3, Dai\cite{Dai}]\label{Lema.4.3}
Sea $\left\{x_{n}\right\}\subset \mathbf{X}$ con
$|x_{n}|\rightarrow\infty$, conforme $n\rightarrow\infty$. Suponga
que
\[lim_{n\rightarrow\infty}\frac{1}{|x_{n}|}U\left(0\right)=\overline{U}_{k}\]
y
\[lim_{n\rightarrow\infty}\frac{1}{|x_{n}|}V\left(0\right)=\overline{V}_{k}.\]
\begin{itemize}
\item[a)] Conforme $n\rightarrow\infty$ casi seguramente,
\[lim_{n\rightarrow\infty}\frac{1}{|x_{n}|}U^{x_{n}}_{k}\left(|x_{n}|t\right)=\left(\overline{U}_{k}-t\right)^{+}\textrm{, u.o.c.}\]
y
\[lim_{n\rightarrow\infty}\frac{1}{|x_{n}|}V^{x_{n}}_{k}\left(|x_{n}|t\right)=\left(\overline{V}_{k}-t\right)^{+}.\]

\item[b)] Para cada $t\geq0$ fijo,
\[\left\{\frac{1}{|x_{n}|}U^{x_{n}}_{k}\left(|x_{n}|t\right),|x_{n}|\geq1\right\}\]
y
\[\left\{\frac{1}{|x_{n}|}V^{x_{n}}_{k}\left(|x_{n}|t\right),|x_{n}|\geq1\right\}\]
\end{itemize}
son uniformemente convergentes.
\end{Lema}

$S_{l}^{x}\left(t\right)$ es el n\'umero total de servicios
completados de la clase $l$, si la clase $l$ est\'a dando $t$
unidades de tiempo de servicio. Sea $T_{l}^{x}\left(x\right)$ el
monto acumulado del tiempo de servicio que el servidor
$s\left(l\right)$ gasta en los usuarios de la clase $l$ al tiempo
$t$. Entonces $S_{l}^{x}\left(T_{l}^{x}\left(t\right)\right)$ es
el n\'umero total de servicios completados para la clase $l$ al
tiempo $t$. Una fracci\'on de estos usuarios,
$\Phi_{l}^{x}\left(S_{l}^{x}\left(T_{l}^{x}\left(t\right)\right)\right)$,
se convierte en usuarios de la clase $k$.\\

Entonces, dado lo anterior, se tiene la siguiente representaci\'on
para el proceso de la longitud de la cola:\\

\begin{equation}
Q_{k}^{x}\left(t\right)=_{k}^{x}\left(0\right)+E_{k}^{x}\left(t\right)+\sum_{l=1}^{K}\Phi_{k}^{l}\left(S_{l}^{x}\left(T_{l}^{x}\left(t\right)\right)\right)-S_{k}^{x}\left(T_{k}^{x}\left(t\right)\right)
\end{equation}
para $k=1,\ldots,K$. Para $i=1,\ldots,d$, sea
\[I_{i}^{x}\left(t\right)=t-\sum_{j\in C_{i}}T_{k}^{x}\left(t\right).\]

Entonces $I_{i}^{x}\left(t\right)$ es el monto acumulado del
tiempo que el servidor $i$ ha estado desocupado al tiempo $t$. Se
est\'a asumiendo que las disciplinas satisfacen la ley de
conservaci\'on del trabajo, es decir, el servidor $i$ est\'a en
pausa solamente cuando no hay usuarios en la estaci\'on $i$.
Entonces, se tiene que

\begin{equation}
\int_{0}^{\infty}\left(\sum_{k\in
C_{i}}Q_{k}^{x}\left(t\right)\right)dI_{i}^{x}\left(t\right)=0,
\end{equation}
para $i=1,\ldots,d$.\\

Hacer
\[T^{x}\left(t\right)=\left(T_{1}^{x}\left(t\right),\ldots,T_{K}^{x}\left(t\right)\right)^{'},\]
\[I^{x}\left(t\right)=\left(I_{1}^{x}\left(t\right),\ldots,I_{K}^{x}\left(t\right)\right)^{'}\]
y
\[S^{x}\left(T^{x}\left(t\right)\right)=\left(S_{1}^{x}\left(T_{1}^{x}\left(t\right)\right),\ldots,S_{K}^{x}\left(T_{K}^{x}\left(t\right)\right)\right)^{'}.\]

Para una disciplina que cumple con la ley de conservaci\'on del
trabajo, en forma vectorial, se tiene el siguiente conjunto de
ecuaciones

\begin{equation}\label{Eq.MF.1.3}
Q^{x}\left(t\right)=Q^{x}\left(0\right)+E^{x}\left(t\right)+\sum_{l=1}^{K}\Phi^{l}\left(S_{l}^{x}\left(T_{l}^{x}\left(t\right)\right)\right)-S^{x}\left(T^{x}\left(t\right)\right),\\
\end{equation}

\begin{equation}\label{Eq.MF.2.3}
Q^{x}\left(t\right)\geq0,\\
\end{equation}

\begin{equation}\label{Eq.MF.3.3}
T^{x}\left(0\right)=0,\textrm{ y }\overline{T}^{x}\left(t\right)\textrm{ es no decreciente},\\
\end{equation}

\begin{equation}\label{Eq.MF.4.3}
I^{x}\left(t\right)=et-CT^{x}\left(t\right)\textrm{ es no
decreciente}\\
\end{equation}

\begin{equation}\label{Eq.MF.5.3}
\int_{0}^{\infty}\left(CQ^{x}\left(t\right)\right)dI_{i}^{x}\left(t\right)=0,\\
\end{equation}

\begin{equation}\label{Eq.MF.6.3}
\textrm{Condiciones adicionales en
}\left(\overline{Q}^{x}\left(\cdot\right),\overline{T}^{x}\left(\cdot\right)\right)\textrm{
espec\'ificas de la disciplina de la cola,}
\end{equation}

donde $e$ es un vector de unos de dimensi\'on $d$, $C$ es la
matriz definida por
\[C_{ik}=\left\{\begin{array}{cc}
1,& S\left(k\right)=i,\\
0,& \textrm{ en otro caso}.\\
\end{array}\right.
\]
Es necesario enunciar el siguiente Teorema que se utilizar\'a para
el Teorema \ref{Tma.4.2.Dai}:
\begin{Teo}[Teorema 4.1, Dai \cite{Dai}]
Considere una disciplina que cumpla la ley de conservaci\'on del
trabajo, para casi todas las trayectorias muestrales $\omega$ y
cualquier sucesi\'on de estados iniciales
$\left\{x_{n}\right\}\subset \mathbf{X}$, con
$|x_{n}|\rightarrow\infty$, existe una subsucesi\'on
$\left\{x_{n_{j}}\right\}$ con $|x_{n_{j}}|\rightarrow\infty$ tal
que
\begin{equation}\label{Eq.4.15}
\frac{1}{|x_{n_{j}}|}\left(Q^{x_{n_{j}}}\left(0\right),U^{x_{n_{j}}}\left(0\right),V^{x_{n_{j}}}\left(0\right)\right)\rightarrow\left(\overline{Q}\left(0\right),\overline{U},\overline{V}\right),
\end{equation}

\begin{equation}\label{Eq.4.16}
\frac{1}{|x_{n_{j}}|}\left(Q^{x_{n_{j}}}\left(|x_{n_{j}}|t\right),T^{x_{n_{j}}}\left(|x_{n_{j}}|t\right)\right)\rightarrow\left(\overline{Q}\left(t\right),\overline{T}\left(t\right)\right)\textrm{
u.o.c.}
\end{equation}

Adem\'as,
$\left(\overline{Q}\left(t\right),\overline{T}\left(t\right)\right)$
satisface las siguientes ecuaciones:
\begin{equation}\label{Eq.MF.1.3a}
\overline{Q}\left(t\right)=Q\left(0\right)+\left(\alpha
t-\overline{U}\right)^{+}-\left(I-P\right)^{'}M^{-1}\left(\overline{T}\left(t\right)-\overline{V}\right)^{+},
\end{equation}

\begin{equation}\label{Eq.MF.2.3a}
\overline{Q}\left(t\right)\geq0,\\
\end{equation}

\begin{equation}\label{Eq.MF.3.3a}
\overline{T}\left(t\right)\textrm{ es no decreciente y comienza en cero},\\
\end{equation}

\begin{equation}\label{Eq.MF.4.3a}
\overline{I}\left(t\right)=et-C\overline{T}\left(t\right)\textrm{
es no decreciente,}\\
\end{equation}

\begin{equation}\label{Eq.MF.5.3a}
\int_{0}^{\infty}\left(C\overline{Q}\left(t\right)\right)d\overline{I}\left(t\right)=0,\\
\end{equation}

\begin{equation}\label{Eq.MF.6.3a}
\textrm{Condiciones adicionales en
}\left(\overline{Q}\left(\cdot\right),\overline{T}\left(\cdot\right)\right)\textrm{
especficas de la disciplina de la cola,}
\end{equation}
\end{Teo}

\begin{Def}[Definici\'on 4.1, , Dai \cite{Dai}]
Sea una disciplina de servicio espec\'ifica. Cualquier l\'imite
$\left(\overline{Q}\left(\cdot\right),\overline{T}\left(\cdot\right)\right)$
en \ref{Eq.4.16} es un {\em flujo l\'imite} de la disciplina.
Cualquier soluci\'on (\ref{Eq.MF.1.3a})-(\ref{Eq.MF.6.3a}) es
llamado flujo soluci\'on de la disciplina. Se dice que el modelo de flujo l\'imite, modelo de flujo, de la disciplina de la cola es estable si existe una constante
$\delta>0$ que depende de $\mu,\alpha$ y $P$ solamente, tal que
cualquier flujo l\'imite con
$|\overline{Q}\left(0\right)|+|\overline{U}|+|\overline{V}|=1$, se
tiene que $\overline{Q}\left(\cdot+\delta\right)\equiv0$.
\end{Def}

\begin{Teo}[Teorema 4.2, Dai\cite{Dai}]\label{Tma.4.2.Dai}
Sea una disciplina fija para la cola, suponga que se cumplen las
condiciones (1.2)-(1.5). Si el modelo de flujo l\'imite de la
disciplina de la cola es estable, entonces la cadena de Markov $X$
que describe la din\'amica de la red bajo la disciplina es Harris
recurrente positiva.
\end{Teo}

Ahora se procede a escalar el espacio y el tiempo para reducir la
aparente fluctuaci\'on del modelo. Consid\'erese el proceso
\begin{equation}\label{Eq.3.7}
\overline{Q}^{x}\left(t\right)=\frac{1}{|x|}Q^{x}\left(|x|t\right)
\end{equation}
A este proceso se le conoce como el fluido escalado, y cualquier l\'imite $\overline{Q}^{x}\left(t\right)$ es llamado flujo l\'imite del proceso de longitud de la cola. Haciendo $|q|\rightarrow\infty$ mientras se mantiene el resto de las componentes fijas, cualquier punto l\'imite del proceso de longitud de la cola normalizado $\overline{Q}^{x}$ es soluci\'on del siguiente modelo de flujo.

\begin{Def}[Definici\'on 3.1, Dai y Meyn \cite{DaiSean}]
Un flujo l\'imite (retrasado) para una red bajo una disciplina de
servicio espec\'ifica se define como cualquier soluci\'on
 $\left(\overline{Q}\left(\cdot\right),\overline{T}\left(\cdot\right)\right)$ de las siguientes ecuaciones, donde
$\overline{Q}\left(t\right)=\left(\overline{Q}_{1}\left(t\right),\ldots,\overline{Q}_{K}\left(t\right)\right)^{'}$
y
$\overline{T}\left(t\right)=\left(\overline{T}_{1}\left(t\right),\ldots,\overline{T}_{K}\left(t\right)\right)^{'}$
\begin{equation}\label{Eq.3.8}
\overline{Q}_{k}\left(t\right)=\overline{Q}_{k}\left(0\right)+\alpha_{k}t-\mu_{k}\overline{T}_{k}\left(t\right)+\sum_{l=1}^{k}P_{lk}\mu_{l}\overline{T}_{l}\left(t\right)\\
\end{equation}
\begin{equation}\label{Eq.3.9}
\overline{Q}_{k}\left(t\right)\geq0\textrm{ para }k=1,2,\ldots,K,\\
\end{equation}
\begin{equation}\label{Eq.3.10}
\overline{T}_{k}\left(0\right)=0,\textrm{ y }\overline{T}_{k}\left(\cdot\right)\textrm{ es no decreciente},\\
\end{equation}
\begin{equation}\label{Eq.3.11}
\overline{I}_{i}\left(t\right)=t-\sum_{k\in C_{i}}\overline{T}_{k}\left(t\right)\textrm{ es no decreciente}\\
\end{equation}
\begin{equation}\label{Eq.3.12}
\overline{I}_{i}\left(\cdot\right)\textrm{ se incrementa al tiempo }t\textrm{ cuando }\sum_{k\in C_{i}}Q_{k}^{x}\left(t\right)dI_{i}^{x}\left(t\right)=0\\
\end{equation}
\begin{equation}\label{Eq.3.13}
\textrm{condiciones adicionales sobre
}\left(Q^{x}\left(\cdot\right),T^{x}\left(\cdot\right)\right)\textrm{
referentes a la disciplina de servicio}
\end{equation}
\end{Def}

Al conjunto de ecuaciones dadas en \ref{Eq.3.8}-\ref{Eq.3.13} se
le llama {\em Modelo de flujo} y al conjunto de todas las
soluciones del modelo de flujo
$\left(\overline{Q}\left(\cdot\right),\overline{T}
\left(\cdot\right)\right)$ se le denotar\'a por $\mathcal{Q}$.

Si se hace $|x|\rightarrow\infty$ sin restringir ninguna de las
componentes, tambi\'en se obtienen un modelo de flujo, pero en
este caso el residual de los procesos de arribo y servicio
introducen un retraso:

\begin{Def}[Definici\'on 3.2, Dai y Meyn \cite{DaiSean}]
El modelo de flujo retrasado de una disciplina de servicio en una
red con retraso
$\left(\overline{A}\left(0\right),\overline{B}\left(0\right)\right)\in\rea_{+}^{K+|A|}$
se define como el conjunto de ecuaciones dadas en
\ref{Eq.3.8}-\ref{Eq.3.13}, junto con la condici\'on:
\begin{equation}\label{CondAd.FluidModel}
\overline{Q}\left(t\right)=\overline{Q}\left(0\right)+\left(\alpha
t-\overline{A}\left(0\right)\right)^{+}-\left(I-P^{'}\right)M\left(\overline{T}\left(t\right)-\overline{B}\left(0\right)\right)^{+}
\end{equation}
\end{Def}

\begin{Def}[Definici\'on 3.3, Dai y Meyn \cite{DaiSean}]
El modelo de flujo es estable si existe un tiempo fijo $t_{0}$ tal
que $\overline{Q}\left(t\right)=0$, con $t\geq t_{0}$, para
cualquier $\overline{Q}\left(\cdot\right)\in\mathcal{Q}$ que
cumple con $|\overline{Q}\left(0\right)|=1$.
\end{Def}

El siguiente resultado se encuentra en Chen \cite{Chen}.
\begin{Lemma}[Lema 3.1, Dai y Meyn \cite{DaiSean}]
Si el modelo de flujo definido por \ref{Eq.3.8}-\ref{Eq.3.13} es
estable, entonces el modelo de flujo retrasado es tambi\'en
estable, es decir, existe $t_{0}>0$ tal que
$\overline{Q}\left(t\right)=0$ para cualquier $t\geq t_{0}$, para
cualquier soluci\'on del modelo de flujo retrasado cuya
condici\'on inicial $\overline{x}$ satisface que
$|\overline{x}|=|\overline{Q}\left(0\right)|+|\overline{A}\left(0\right)|+|\overline{B}\left(0\right)|\leq1$.
\end{Lemma}

%_________________________________________________________________________
\subsection{Modelo de Visitas C\'iclicas con un Servidor: Estabilidad}
%_________________________________________________________________________

%_________________________________________________________________________
\subsection{Teorema 2.1}
%_________________________________________________________________________



El resultado principal de Down \cite{Down} que relaciona la estabilidad del modelo de flujo con la estabilidad del sistema original

\begin{Teo}[Teorema 2.1, Down \cite{Down}]\label{Tma.2.1.Down}
Suponga que el modelo de flujo es estable, y que se cumplen los supuestos (A1) y (A2), entonces
\begin{itemize}
\item[i)] Para alguna constante $\kappa_{p}$, y para cada
condici\'on inicial $x\in X$
\begin{equation}\label{Estability.Eq1}
lim_{t\rightarrow\infty}\sup\frac{1}{t}\int_{0}^{t}\esp_{x}\left[|Q\left(s\right)|^{p}\right]ds\leq\kappa_{p},
\end{equation}
donde $p$ es el entero dado en (A2). Si adem\'as se cumple
la condici\'on (A3), entonces para cada condici\'on inicial:

\item[ii)] Los momentos transitorios convergen a su estado estacionario:
 \begin{equation}\label{Estability.Eq2}
lim_{t\rightarrow\infty}\esp_{x}\left[Q_{k}\left(t\right)^{r}\right]=\esp_{\pi}\left[Q_{k}\left(0\right)^{r}\right]\leq\kappa_{r},
\end{equation}
para $r=1,2,\ldots,p$ y $k=1,2,\ldots,K$. Donde $\pi$ es la
probabilidad invariante para $\mathbf{X}$.

\item[iii)]  El primer momento converge con raz\'on $t^{p-1}$:
\begin{equation}\label{Estability.Eq3}
lim_{t\rightarrow\infty}t^{p-1}|\esp_{x}\left[Q_{k}\left(t\right)\right]-\esp_{\pi}\left[Q\left(0\right)\right]=0.
\end{equation}

\item[iv)] La {\em Ley Fuerte de los grandes n\'umeros} se cumple:
\begin{equation}\label{Estability.Eq4}
lim_{t\rightarrow\infty}\frac{1}{t}\int_{0}^{t}Q_{k}^{r}\left(s\right)ds=\esp_{\pi}\left[Q_{k}\left(0\right)^{r}\right],\textrm{
}\prob_{x}\textrm{-c.s.}
\end{equation}
para $r=1,2,\ldots,p$ y $k=1,2,\ldots,K$.
\end{itemize}
\end{Teo}


\begin{Prop}[Proposici\'on 5.1, Dai y Meyn \cite{DaiSean}]\label{Prop.5.1.DaiSean}
Suponga que los supuestos A1) y A2) son ciertos y que el modelo de flujo es estable. Entonces existe $t_{0}>0$ tal que
\begin{equation}
lim_{|x|\rightarrow\infty}\frac{1}{|x|^{p+1}}\esp_{x}\left[|X\left(t_{0}|x|\right)|^{p+1}\right]=0
\end{equation}
\end{Prop}

\begin{Lemma}[Lema 5.2, Dai y Meyn \cite{DaiSean}]\label{Lema.5.2.DaiSean}
 Sea $\left\{\zeta\left(k\right):k\in \mathbb{z}\right\}$ una sucesi\'on independiente e id\'enticamente distribuida que toma valores en $\left(0,\infty\right)$,
y sea
$E\left(t\right)=max\left(n\geq1:\zeta\left(1\right)+\cdots+\zeta\left(n-1\right)\leq
t\right)$. Si $\esp\left[\zeta\left(1\right)\right]<\infty$,
entonces para cualquier entero $r\geq1$
\begin{equation}
 lim_{t\rightarrow\infty}\esp\left[\left(\frac{E\left(t\right)}{t}\right)^{r}\right]=\left(\frac{1}{\esp\left[\zeta_{1}\right]}\right)^{r}.
\end{equation}
Luego, bajo estas condiciones:
\begin{itemize}
 \item[a)] para cualquier $\delta>0$, $\sup_{t\geq\delta}\esp\left[\left(\frac{E\left(t\right)}{t}\right)^{r}\right]<\infty$
\item[b)] las variables aleatorias
$\left\{\left(\frac{E\left(t\right)}{t}\right)^{r}:t\geq1\right\}$
son uniformemente integrables.
\end{itemize}
\end{Lemma}

\begin{Teo}[Teorema 5.5, Dai y Meyn \cite{DaiSean}]\label{Tma.5.5.DaiSean}
Suponga que los supuestos A1) y A2) se cumplen y que el modelo de
flujo es estable. Entonces existe una constante $\kappa_{p}$ tal
que
\begin{equation}
\frac{1}{t}\int_{0}^{t}\esp_{x}\left[|Q\left(s\right)|^{p}\right]ds\leq\kappa_{p}\left\{\frac{1}{t}|x|^{p+1}+1\right\}
\end{equation}
para $t>0$ y $x\in X$. En particular, para cada condici\'on
inicial
\begin{eqnarray*}
\limsup_{t\rightarrow\infty}\frac{1}{t}\int_{0}^{t}\esp_{x}\left[|Q\left(s\right)|^{p}\right]ds\leq\kappa_{p}.
\end{eqnarray*}
\end{Teo}

\begin{Teo}[Teorema 6.2, Dai y Meyn \cite{DaiSean}]\label{Tma.6.2.DaiSean}
Suponga que se cumplen los supuestos A1), A2) y A3) y que el
modelo de flujo es estable. Entonces se tiene que
\begin{equation}
\left\|P^{t}\left(x,\cdot\right)-\pi\left(\cdot\right)\right\|_{f_{p}}\textrm{,
}t\rightarrow\infty,x\in X.
\end{equation}
En particular para cada condici\'on inicial
\begin{eqnarray*}
\lim_{t\rightarrow\infty}\esp_{x}\left[|Q\left(t\right)|^{p}\right]=\esp_{\pi}\left[|Q\left(0\right)|^{p}\right]\leq\kappa_{r}
\end{eqnarray*}
\end{Teo}
\begin{Teo}[Teorema 6.3, Dai y Meyn \cite{DaiSean}]\label{Tma.6.3.DaiSean}
Suponga que se cumplen los supuestos A1), A2) y A3) y que el
modelo de flujo es estable. Entonces con
$f\left(x\right)=f_{1}\left(x\right)$ se tiene
\begin{equation}
\lim_{t\rightarrow\infty}t^{p-1}\left\|P^{t}\left(x,\cdot\right)-\pi\left(\cdot\right)\right\|_{f}=0.
\end{equation}
En particular para cada condici\'on inicial
\begin{eqnarray*}
\lim_{t\rightarrow\infty}t^{p-1}|\esp_{x}\left[Q\left(t\right)\right]-\esp_{\pi}\left[Q\left(0\right)\right]|=0.
\end{eqnarray*}
\end{Teo}

\begin{Teo}[Teorema 6.4, Dai y Meyn \cite{DaiSean}]\label{Tma.6.4.DaiSean}
Suponga que se cumplen los supuestos A1), A2) y A3) y que el
modelo de flujo es estable. Sea $\nu$ cualquier distribuci\'on de
probabilidad en $\left(X,\mathcal{B}_{X}\right)$, y $\pi$ la
distribuci\'on estacionaria de $X$.
\begin{itemize}
\item[i)] Para cualquier $f:X\leftarrow\rea_{+}$
\begin{equation}
\lim_{t\rightarrow\infty}\frac{1}{t}\int_{o}^{t}f\left(X\left(s\right)\right)ds=\pi\left(f\right):=\int
f\left(x\right)\pi\left(dx\right)
\end{equation}
$\prob$-c.s.

\item[ii)] Para cualquier $f:X\leftarrow\rea_{+}$ con
$\pi\left(|f|\right)<\infty$, la ecuaci\'on anterior se cumple.
\end{itemize}
\end{Teo}

%_________________________________________________________________________
\subsection{Teorema 2.2}
%_________________________________________________________________________

\begin{Teo}[Teorema 2.2, Down \cite{Down}]\label{Tma2.2.Down}
Suponga que el fluido modelo es inestable en el sentido de que
para alguna $\epsilon_{0},c_{0}\geq0$,
\begin{equation}\label{Eq.Inestability}
|Q\left(T\right)|\geq\epsilon_{0}T-c_{0}\textrm{,   }T\geq0,
\end{equation}
para cualquier condici\'on inicial $Q\left(0\right)$, con
$|Q\left(0\right)|=1$. Entonces para cualquier $0<q\leq1$, existe
$B<0$ tal que para cualquier $|x|\geq B$,
\begin{equation}
\prob_{x}\left\{\mathbb{X}\rightarrow\infty\right\}\geq q.
\end{equation}
\end{Teo}

%_________________________________________________________________________
\subsection{Teorema 2.3}
%_________________________________________________________________________
\begin{Teo}[Teorema 2.3, Down \cite{Down}]\label{Tma2.3.Down}
Considere el siguiente valor:
\begin{equation}\label{Eq.Rho.1serv}
\rho=\sum_{k=1}^{K}\rho_{k}+max_{1\leq j\leq K}\left(\frac{\lambda_{j}}{\sum_{s=1}^{S}p_{js}\overline{N}_{s}}\right)\delta^{*}
\end{equation}
\begin{itemize}
\item[i)] Si $\rho<1$ entonces la red es estable, es decir, se cumple el teorema \ref{Tma.2.1.Down}.

\item[ii)] Si $\rho<1$ entonces la red es inestable, es decir, se cumple el teorema \ref{Tma2.2.Down}
\end{itemize}
\end{Teo}
\newpage
%_____________________________________________________________________
\subsection{Definiciones  B\'asicas}
%_____________________________________________________________________
\begin{Def}
Sea $X$ un conjunto y $\mathcal{F}$ una $\sigma$-\'algebra de
subconjuntos de $X$, la pareja $\left(X,\mathcal{F}\right)$ es
llamado espacio medible. Un subconjunto $A$ de $X$ es llamado
medible, o medible con respecto a $\mathcal{F}$, si
$A\in\mathcal{F}$.
\end{Def}

\begin{Def}
Sea $\left(X,\mathcal{F},\mu\right)$ espacio de medida. Se dice
que la medida $\mu$ es $\sigma$-finita si se puede escribir
$X=\bigcup_{n\geq1}X_{n}$ con $X_{n}\in\mathcal{F}$ y
$\mu\left(X_{n}\right)<\infty$.
\end{Def}

\begin{Def}\label{Cto.Borel}
Sea $X$ el conjunto de los \'umeros reales $\rea$. El \'algebra de
Borel es la $\sigma$-\'algebra $B$ generada por los intervalos
abiertos $\left(a,b\right)\in\rea$. Cualquier conjunto en $B$ es
llamado {\em Conjunto de Borel}.
\end{Def}

\begin{Def}\label{Funcion.Medible}
Una funci\'on $f:X\rightarrow\rea$, es medible si para cualquier
n\'umero real $\alpha$ el conjunto
\[\left\{x\in X:f\left(x\right)>\alpha\right\}\]
pertenece a $X$. Equivalentemente, se dice que $f$ es medible si
\[f^{-1}\left(\left(\alpha,\infty\right)\right)=\left\{x\in X:f\left(x\right)>\alpha\right\}\in\mathcal{F}.\]
\end{Def}


\begin{Def}\label{Def.Cilindros}
Sean $\left(\Omega_{i},\mathcal{F}_{i}\right)$, $i=1,2,\ldots,$
espacios medibles y $\Omega=\prod_{i=1}^{\infty}\Omega_{i}$ el
conjunto de todas las sucesiones
$\left(\omega_{1},\omega_{2},\ldots,\right)$ tales que
$\omega_{i}\in\Omega_{i}$, $i=1,2,\ldots,$. Si
$B^{n}\subset\prod_{i=1}^{\infty}\Omega_{i}$, definimos
$B_{n}=\left\{\omega\in\Omega:\left(\omega_{1},\omega_{2},\ldots,\omega_{n}\right)\in
B^{n}\right\}$. Al conjunto $B_{n}$ se le llama {\em cilindro} con
base $B^{n}$, el cilindro es llamado medible si
$B^{n}\in\prod_{i=1}^{\infty}\mathcal{F}_{i}$.
\end{Def}


\begin{Def}\label{Def.Proc.Adaptado}[TSP, Ash \cite{RBA}]
Sea $X\left(t\right),t\geq0$ proceso estoc\'astico, el proceso es
adaptado a la familia de $\sigma$-\'algebras $\mathcal{F}_{t}$,
para $t\geq0$, si para $s<t$ implica que
$\mathcal{F}_{s}\subset\mathcal{F}_{t}$, y $X\left(t\right)$ es
$\mathcal{F}_{t}$-medible para cada $t$. Si no se especifica
$\mathcal{F}_{t}$ entonces se toma $\mathcal{F}_{t}$ como
$\mathcal{F}\left(X\left(s\right),s\leq t\right)$, la m\'as
peque\~na $\sigma$-\'algebra de subconjuntos de $\Omega$ que hace
que cada $X\left(s\right)$, con $s\leq t$ sea Borel medible.
\end{Def}


\begin{Def}\label{Def.Tiempo.Paro}[TSP, Ash \cite{RBA}]
Sea $\left\{\mathcal{F}\left(t\right),t\geq0\right\}$ familia
creciente de sub $\sigma$-\'algebras. es decir,
$\mathcal{F}\left(s\right)\subset\mathcal{F}\left(t\right)$ para
$s\leq t$. Un tiempo de paro para $\mathcal{F}\left(t\right)$ es
una funci\'on $T:\Omega\rightarrow\left[0,\infty\right]$ tal que
$\left\{T\leq t\right\}\in\mathcal{F}\left(t\right)$ para cada
$t\geq0$. Un tiempo de paro para el proceso estoc\'astico
$X\left(t\right),t\geq0$ es un tiempo de paro para las
$\sigma$-\'algebras
$\mathcal{F}\left(t\right)=\mathcal{F}\left(X\left(s\right)\right)$.
\end{Def}

\begin{Def}
Sea $X\left(t\right),t\geq0$ proceso estoc\'astico, con
$\left(S,\chi\right)$ espacio de estados. Se dice que el proceso
es adaptado a $\left\{\mathcal{F}\left(t\right)\right\}$, es
decir, si para cualquier $s,t\in I$, $I$ conjunto de \'indices,
$s<t$, se tiene que
$\mathcal{F}\left(s\right)\subset\mathcal{F}\left(t\right)$ y
$X\left(t\right)$ es $\mathcal{F}\left(t\right)$-medible,
\end{Def}

\begin{Def}
Sea $X\left(t\right),t\geq0$ proceso estoc\'astico, se dice que es
un Proceso de Markov relativo a $\mathcal{F}\left(t\right)$ o que
$\left\{X\left(t\right),\mathcal{F}\left(t\right)\right\}$ es de
Markov si y s\'olo si para cualquier conjunto $B\in\chi$,  y
$s,t\in I$, $s<t$ se cumple que
\begin{equation}\label{Prop.Markov}
P\left\{X\left(t\right)\in
B|\mathcal{F}\left(s\right)\right\}=P\left\{X\left(t\right)\in
B|X\left(s\right)\right\}.
\end{equation}
\end{Def}
\begin{Note}
Si se dice que $\left\{X\left(t\right)\right\}$ es un Proceso de
Markov sin mencionar $\mathcal{F}\left(t\right)$, se asumir\'a que
\begin{eqnarray*}
\mathcal{F}\left(t\right)=\mathcal{F}_{0}\left(t\right)=\mathcal{F}\left(X\left(r\right),r\leq
t\right),
\end{eqnarray*}
entonces la ecuaci\'on (\ref{Prop.Markov}) se puede escribir como
\begin{equation}
P\left\{X\left(t\right)\in B|X\left(r\right),r\leq s\right\} =
P\left\{X\left(t\right)\in B|X\left(s\right)\right\}
\end{equation}
\end{Note}

\begin{Teo}
Sea $\left(X_{n},\mathcal{F}_{n},n=0,1,\ldots,\right\}$ Proceso de
Markov con espacio de estados $\left(S_{0},\chi_{0}\right)$
generado por una distribuici\'on inicial $P_{o}$ y probabilidad de
transici\'on $p_{mn}$, para $m,n=0,1,\ldots,$ $m<n$, que por
notaci\'on se escribir\'a como $p\left(m,n,x,B\right)\rightarrow
p_{mn}\left(x,B\right)$. Sea $S$ tiempo de paro relativo a la
$\sigma$-\'algebra $\mathcal{F}_{n}$. Sea $T$ funci\'on medible,
$T:\Omega\rightarrow\left\{0,1,\ldots,\right\}$. Sup\'ongase que
$T\geq S$, entonces $T$ es tiempo de paro. Si $B\in\chi_{0}$,
entonces
\begin{equation}\label{Prop.Fuerte.Markov}
P\left\{X\left(T\right)\in
B,T<\infty|\mathcal{F}\left(S\right)\right\} =
p\left(S,T,X\left(s\right),B\right)
\end{equation}
en $\left\{T<\infty\right\}$.
\end{Teo}

Propiedades importantes para el modelo de flujo retrasado:

\begin{Prop}
 Sea $\left(\overline{Q},\overline{T},\overline{T}^{0}\right)$ un flujo l\'imite de \ref{Equation.4.4} y suponga que cuando $x\rightarrow\infty$ a lo largo de
una subsucesi\'on
\[\left(\frac{1}{|x|}Q_{k}^{x}\left(0\right),\frac{1}{|x|}A_{k}^{x}\left(0\right),\frac{1}{|x|}B_{k}^{x}\left(0\right),\frac{1}{|x|}B_{k}^{x,0}\left(0\right)\right)\rightarrow\left(\overline{Q}_{k}\left(0\right),0,0,0\right)\]
para $k=1,\ldots,K$. EL flujo l\'imite tiene las siguientes
propiedades, donde las propiedades de la derivada se cumplen donde
la derivada exista:
\begin{itemize}
 \item[i)] Los vectores de tiempo ocupado $\overline{T}\left(t\right)$ y $\overline{T}^{0}\left(t\right)$ son crecientes y continuas con
$\overline{T}\left(0\right)=\overline{T}^{0}\left(0\right)=0$.
\item[ii)] Para todo $t\geq0$
\[\sum_{k=1}^{K}\left[\overline{T}_{k}\left(t\right)+\overline{T}_{k}^{0}\left(t\right)\right]=t\]
\item[iii)] Para todo $1\leq k\leq K$
\[\overline{Q}_{k}\left(t\right)=\overline{Q}_{k}\left(0\right)+\alpha_{k}t-\mu_{k}\overline{T}_{k}\left(t\right)\]
\item[iv)]  Para todo $1\leq k\leq K$
\[\dot{{\overline{T}}}_{k}\left(t\right)=\beta_{k}\] para $\overline{Q}_{k}\left(t\right)=0$.
\item[v)] Para todo $k,j$
\[\mu_{k}^{0}\overline{T}_{k}^{0}\left(t\right)=\mu_{j}^{0}\overline{T}_{j}^{0}\left(t\right)\]
\item[vi)]  Para todo $1\leq k\leq K$
\[\mu_{k}\dot{{\overline{T}}}_{k}\left(t\right)=l_{k}\mu_{k}^{0}\dot{{\overline{T}}}_{k}^{0}\left(t\right)\] para $\overline{Q}_{k}\left(t\right)>0$.
\end{itemize}
\end{Prop}

\begin{Lema}[Lema 3.1 \cite{Chen}]\label{Lema3.1}
Si el modelo de flujo es estable, definido por las ecuaciones
(3.8)-(3.13), entonces el modelo de flujo retrasado tambin es
estable.
\end{Lema}

\begin{Teo}[Teorema 5.2 \cite{Chen}]\label{Tma.5.2}
Si el modelo de flujo lineal correspondiente a la red de cola es
estable, entonces la red de colas es estable.
\end{Teo}

\begin{Teo}[Teorema 5.1 \cite{Chen}]\label{Tma.5.1.Chen}
La red de colas es estable si existe una constante $t_{0}$ que
depende de $\left(\alpha,\mu,T,U\right)$ y $V$ que satisfagan las
ecuaciones (5.1)-(5.5), $Z\left(t\right)=0$, para toda $t\geq
t_{0}$.
\end{Teo}



\begin{Lema}[Lema 5.2 \cite{Gut}]\label{Lema.5.2.Gut}
Sea $\left\{\xi\left(k\right):k\in\ent\right\}$ sucesin de
variables aleatorias i.i.d. con valores en
$\left(0,\infty\right)$, y sea $E\left(t\right)$ el proceso de
conteo
\[E\left(t\right)=max\left\{n\geq1:\xi\left(1\right)+\cdots+\xi\left(n-1\right)\leq t\right\}.\]
Si $E\left[\xi\left(1\right)\right]<\infty$, entonces para
cualquier entero $r\geq1$
\begin{equation}
lim_{t\rightarrow\infty}\esp\left[\left(\frac{E\left(t\right)}{t}\right)^{r}\right]=\left(\frac{1}{E\left[\xi_{1}\right]}\right)^{r}
\end{equation}
de aqu, bajo estas condiciones
\begin{itemize}
\item[a)] Para cualquier $t>0$,
$sup_{t\geq\delta}\esp\left[\left(\frac{E\left(t\right)}{t}\right)^{r}\right]$

\item[b)] Las variables aleatorias
$\left\{\left(\frac{E\left(t\right)}{t}\right)^{r}:t\geq1\right\}$
son uniformemente integrables.
\end{itemize}
\end{Lema}

\begin{Teo}[Teorema 5.1: Ley Fuerte para Procesos de Conteo
\cite{Gut}]\label{Tma.5.1.Gut} Sea
$0<\mu<\esp\left(X_{1}\right]\leq\infty$. entonces

\begin{itemize}
\item[a)] $\frac{N\left(t\right)}{t}\rightarrow\frac{1}{\mu}$
a.s., cuando $t\rightarrow\infty$.


\item[b)]$\esp\left[\frac{N\left(t\right)}{t}\right]^{r}\rightarrow\frac{1}{\mu^{r}}$,
cuando $t\rightarrow\infty$ para todo $r>0$..
\end{itemize}
\end{Teo}


\begin{Prop}[Proposicin 5.1 \cite{DaiSean}]\label{Prop.5.1}
Suponga que los supuestos (A1) y (A2) se cumplen, adems suponga
que el modelo de flujo es estable. Entonces existe $t_{0}>0$ tal
que
\begin{equation}\label{Eq.Prop.5.1}
lim_{|x|\rightarrow\infty}\frac{1}{|x|^{p+1}}\esp_{x}\left[|X\left(t_{0}|x|\right)|^{p+1}\right]=0.
\end{equation}

\end{Prop}


\begin{Prop}[Proposici\'on 5.3 \cite{DaiSean}]
Sea $X$ proceso de estados para la red de colas, y suponga que se
cumplen los supuestos (A1) y (A2), entonces para alguna constante
positiva $C_{p+1}<\infty$, $\delta>0$ y un conjunto compacto
$C\subset X$.

\begin{equation}\label{Eq.5.4}
\esp_{x}\left[\int_{0}^{\tau_{C}\left(\delta\right)}\left(1+|X\left(t\right)|^{p}\right)dt\right]\leq
C_{p+1}\left(1+|x|^{p+1}\right)
\end{equation}
\end{Prop}

\begin{Prop}[Proposici\'on 5.4 \cite{DaiSean}]
Sea $X$ un proceso de Markov Borel Derecho en $X$, sea
$f:X\leftarrow\rea_{+}$ y defina para alguna $\delta>0$, y un
conjunto cerrado $C\subset X$
\[V\left(x\right):=\esp_{x}\left[\int_{0}^{\tau_{C}\left(\delta\right)}f\left(X\left(t\right)\right)dt\right]\]
para $x\in X$. Si $V$ es finito en todas partes y uniformemente
acotada en $C$, entonces existe $k<\infty$ tal que
\begin{equation}\label{Eq.5.11}
\frac{1}{t}\esp_{x}\left[V\left(x\right)\right]+\frac{1}{t}\int_{0}^{t}\esp_{x}\left[f\left(X\left(s\right)\right)ds\right]\leq\frac{1}{t}V\left(x\right)+k,
\end{equation}
para $x\in X$ y $t>0$.
\end{Prop}


\begin{Teo}[Teorema 5.5 \cite{DaiSean}]
Suponga que se cumplen (A1) y (A2), adems suponga que el modelo
de flujo es estable. Entonces existe una constante $k_{p}<\infty$
tal que
\begin{equation}\label{Eq.5.13}
\frac{1}{t}\int_{0}^{t}\esp_{x}\left[|Q\left(s\right)|^{p}\right]ds\leq
k_{p}\left\{\frac{1}{t}|x|^{p+1}+1\right\}
\end{equation}
para $t\geq0$, $x\in X$. En particular para cada condicin inicial
\begin{equation}\label{Eq.5.14}
Limsup_{t\rightarrow\infty}\frac{1}{t}\int_{0}^{t}\esp_{x}\left[|Q\left(s\right)|^{p}\right]ds\leq
k_{p}
\end{equation}
\end{Teo}

\begin{Teo}[Teorema 6.2\cite{DaiSean}]\label{Tma.6.2}
Suponga que se cumplen los supuestos (A1)-(A3) y que el modelo de
flujo es estable, entonces se tiene que
\[\parallel P^{t}\left(c,\cdot\right)-\pi\left(\cdot\right)\parallel_{f_{p}}\rightarrow0\]
para $t\rightarrow\infty$ y $x\in X$. En particular para cada
condicin inicial
\[lim_{t\rightarrow\infty}\esp_{x}\left[\left|Q_{t}\right|^{p}\right]=\esp_{\pi}\left[\left|Q_{0}\right|^{p}\right]<\infty\]
\end{Teo}


\begin{Teo}[Teorema 6.3\cite{DaiSean}]\label{Tma.6.3}
Suponga que se cumplen los supuestos (A1)-(A3) y que el modelo de
flujo es estable, entonces con
$f\left(x\right)=f_{1}\left(x\right)$, se tiene que
\[lim_{t\rightarrow\infty}t^{(p-1)\left|P^{t}\left(c,\cdot\right)-\pi\left(\cdot\right)\right|_{f}=0},\]
para $x\in X$. En particular, para cada condicin inicial
\[lim_{t\rightarrow\infty}t^{(p-1)\left|\esp_{x}\left[Q_{t}\right]-\esp_{\pi}\left[Q_{0}\right]\right|=0}.\]
\end{Teo}



%_____________________________________________________________________________________
\subsection{Proceso de Estados Markoviano para el Sistema}
%_________________________________________________________________________


Sean $Q_{k}\left(t\right)$ el n\'umero de usuarios en la cola $k$,
$A_{k}\left(t\right)$ el tiempo residual de arribos a la cola $k$,
para cada servidor $m$, sea $H_{m}\left(t\right)$ par ordenado que
consiste en la cola que est\'a siendo atendida y la pol\'itica de
servicio que se est\'a utilizando. $B_{m}\left(t\right)$ los
tiempos de servicio residuales, $B_{m}^{0}\left(t\right)$ el
tiempo residual de traslado, $C_{m}\left(t\right)$ el n\'umero de
usuarios atendidos durante la visita del servidor a la cola dada
en $H_{m}\left(t\right)$.

El proceso para el sistema de visitas se puede definir como:

\begin{equation}\label{Esp.Edos.Down}
X\left(t\right)^{T}=\left(Q_{k}\left(t\right),A_{k}\left(t\right),B_{m}\left(t\right),B_{m}^{0}\left(t\right),C_{m}\left(t\right)\right)
\end{equation}
para $k=1,\ldots,K$ y $m=1,2,\ldots,M$. $X$ evoluciona en el
espacio de estados:
$X=\ent_{+}^{K}\times\rea_{+}^{K}\times\left(\left\{1,2,\ldots,K\right\}\times\left\{1,2,\ldots,S\right\}\right)^{M}\times\rea_{+}^{K}\times\rea_{+}^{K}\times\ent_{+}^{K}$.\\

Antes enunciemos los supuestos que regir\'an en la red.


\begin{itemize}
\item[A1)] $\xi_{1},\ldots,\xi_{K},\eta_{1},\ldots,\eta_{K}$ son
mutuamente independientes y son sucesiones independientes e
id\'enticamente distribuidas.

\item[A2)] Para alg\'un entero $p\geq1$
\begin{eqnarray*}
\esp\left[\xi_{l}\left(1\right)^{p+1}\right]<\infty\textrm{ para }l\in\mathcal{A}\textrm{ y }\\
\esp\left[\eta_{k}\left(1\right)^{p+1}\right]<\infty\textrm{ para
}k=1,\ldots,K.
\end{eqnarray*}
donde $\mathcal{A}$ es la clase de posibles arribos.

\item[A3)] Para $k=1,2,\ldots,K$ existe una funci\'on positiva
$q_{k}\left(x\right)$ definida en $\rea_{+}$, y un entero $j_{k}$,
tal que
\begin{eqnarray}
P\left(\xi_{k}\left(1\right)\geq x\right)>0\textrm{, para todo }x>0\\
P\left(\xi_{k}\left(1\right)+\ldots\xi_{k}\left(j_{k}\right)\in dx\right)\geq q_{k}\left(x\right)dx0\textrm{ y }\\
\int_{0}^{\infty}q_{k}\left(x\right)dx>0
\end{eqnarray}
\end{itemize}
%_________________________________________________________________________
\subsection{Procesos Fuerte de Markov}
%_________________________________________________________________________

En Dai \cite{Dai} se muestra que para una amplia serie de
disciplinas de servicio el proceso $X$ es un Proceso Fuerte de
Markov, y por tanto se puede asumir que
\[\left(\left(\Omega,\mathcal{F}\right),\mathcal{F}_{t},X\left(t\right),\theta_{t},P_{x}\right)\]
es un proceso de Borel Derecho, Sharpe \cite{Sharpe}, en el
espacio de estados medible
$\left(X,\mathcal{B}_{X}\right)$.


Se har\'an las siguientes consideraciones: $E$ es un espacio
m\'etrico separable.


\begin{Def}
Un espacio topol\'ogico $E$ es llamado {\em Luisin} si es
homeomorfo a un subconjunto de Borel de un espacio m\'etrico
compacto.
\end{Def}

\begin{Def}
Un espacio topol\'ogico $E$ es llamado de {\em Rad\'on} si es
homeomorfo a un subconjunto universalmente medible de un espacio
m\'etrico compacto.
\end{Def}

Equivalentemente, la definici\'on de un espacio de Rad\'on puede
encontrarse en los siguientes t\'erminos:

\begin{Def}
$E$ es un espacio de Rad\'on si cada medida finita en
$\left(E,\mathcal{B}\left(E\right)\right)$ es regular interior o
cerrada, {\em tight}.
\end{Def}

\begin{Def}
Una medida finita, $\lambda$ en la $\sigma$-\'algebra de Borel de
un espacio metrizable $E$ se dice cerrada si
\begin{equation}\label{Eq.A2.3}
\lambda\left(E\right)=sup\left\{\lambda\left(K\right):K\textrm{ es
compacto en }E\right\}.
\end{equation}
\end{Def}

El siguiente teorema nos permite tener una mejor caracterizaci\'on
de los espacios de Rad\'on:
\begin{Teo}\label{Tma.A2.2}
Sea $E$ espacio separable metrizable. Entonces $E$ es Radoniano si
y s\'olo s\'i cada medida finita en
$\left(E,\mathcal{B}\left(E\right)\right)$ es cerrada.
\end{Teo}

%_________________________________________________________________________________________
\subsection{Propiedades de Markov}
%_________________________________________________________________________________________

Sea $E$ espacio de estados, tal que $E$ es un espacio de Rad\'on,
$\mathcal{B}\left(E\right)$ $\sigma$-\'algebra de Borel en $E$,
que se denotar\'a por $\mathcal{E}$.

Sea $\left(X,\mathcal{G},\prob\right)$ espacio de probabilidad,
$I\subset\rea$ conjunto de índices. Sea $\mathcal{F}_{\leq
t}$ la $\sigma$-\'algebra natural definida como
$\sigma\left\{f\left(X_{r}\right):r\in I, r\leq
t,f\in\mathcal{E}\right\}$. Se considerar\'a una
$\sigma$-\'algebra m\'as general, $ \left(\mathcal{G}_{t}\right)$
tal que $\left(X_{t}\right)$ sea $\mathcal{E}$-adaptado.

\begin{Def}
Una familia $\left(P_{s,t}\right)$ de kernels de Markov en
$\left(E,\mathcal{E}\right)$ indexada por pares $s,t\in I$, con
$s\leq t$ es una funci\'on de transici\'on en $\ER$, si  para todo
$r\leq s< t$ en $I$ y todo $x\in E$, $B\in\mathcal{E}$
\begin{equation}\label{Eq.Kernels}
P_{r,t}\left(x,B\right)=\int_{E}P_{r,s}\left(x,dy\right)P_{s,t}\left(y,B\right)\footnote{Ecuaci\'on
de Chapman-Kolmogorov}.
\end{equation}
\end{Def}

Se dice que la funci\'on de transici\'on $\KM$ en $\ER$ es la
funci\'on de transici\'on para un proceso $\PE$  con valores en
$E$ y que satisface la propiedad de
Markov \footnote{\begin{equation}\label{Eq.1.4.S}
\prob\left\{H|\mathcal{G}_{t}\right\}=\prob\left\{H|X_{t}\right\}\textrm{
}H\in p\mathcal{F}_{\geq t}.
\end{equation}} (\ref{Eq.1.4.S}) relativa a $\left(\mathcal{G}_{t}\right)$ si

\begin{equation}\label{Eq.1.6.S}
\prob\left\{f\left(X_{t}\right)|\mathcal{G}_{s}\right\}=P_{s,t}f\left(X_{t}\right)\textrm{
}s\leq t\in I,\textrm{ }f\in b\mathcal{E}.
\end{equation}

\begin{Def}
Una familia $\left(P_{t}\right)_{t\geq0}$ de kernels de Markov en
$\ER$ es llamada {\em Semigrupo de Transici\'on de Markov} o {\em
Semigrupo de Transici\'on} si
\[P_{t+s}f\left(x\right)=P_{t}\left(P_{s}f\right)\left(x\right),\textrm{ }t,s\geq0,\textrm{ }x\in E\textrm{ }f\in b\mathcal{E}.\]
\end{Def}

\begin{Note}
Si la funci\'on de transici\'on $\KM$ es llamada homog\'enea si
$P_{s,t}=P_{t-s}$.
\end{Note}


Un proceso de Markov que satisface la ecuaci\'on (\ref{Eq.1.6.S})
con funci\'on de transici\'on homog\'enea $\left(P_{t}\right)$
tiene la propiedad caracter\'istica
\begin{equation}\label{Eq.1.8.S}
\prob\left\{f\left(X_{t+s}\right)|\mathcal{G}_{t}\right\}=P_{s}f\left(X_{t}\right)\textrm{
}t,s\geq0,\textrm{ }f\in b\mathcal{E}.
\end{equation}
La ecuaci\'on anterior es la {\em Propiedad Simple de Markov} de
$X$ relativa a $\left(P_{t}\right)$.

En este sentido el proceso $\PE$ cumple con la propiedad de Markov
(\ref{Eq.1.8.S}) relativa a
$\left(\Omega,\mathcal{G},\mathcal{G}_{t},\prob\right)$ con
semigrupo de transici\'on $\left(P_{t}\right)$.

%_________________________________________________________________________________________
\subsection{Primer Condici\'on de Regularidad}
%_________________________________________________________________________________________


\begin{Def}
Un proceso estoc\'astico $\PE$ definido en
$\left(\Omega,\mathcal{G},\prob\right)$ con valores en el espacio
topol\'ogico $E$ es continuo por la derecha si cada trayectoria
muestral $t\rightarrow X_{t}\left(w\right)$ es un mapeo continuo
por la derecha de $I$ en $E$.
\end{Def}

\begin{Def}[HD1]\label{Eq.2.1.S}
Un semigrupo de Markov $\left/P_{t}\right)$ en un espacio de
Rad\'on $E$ se dice que satisface la condici\'on {\em HD1} si,
dada una medida de probabilidad $\mu$ en $E$, existe una
$\sigma$-\'algebra $\mathcal{E^{*}}$ con
$\mathcal{E}\subset\mathcal{E}$ y
$P_{t}\left(b\mathcal{E}^{*}\right)\subset b\mathcal{E}^{*}$, y un
$\mathcal{E}^{*}$-proceso $E$-valuado continuo por la derecha
$\PE$ en alg\'un espacio de probabilidad filtrado
$\left(\Omega,\mathcal{G},\mathcal{G}_{t},\prob\right)$ tal que
$X=\left(\Omega,\mathcal{G},\mathcal{G}_{t},\prob\right)$ es de
Markov (Homog\'eneo) con semigrupo de transici\'on $(P_{t})$ y
distribuci\'on inicial $\mu$.
\end{Def}
Consid\'erese la colecci\'on de variables aleatorias $X_{t}$
definidas en alg\'un espacio de probabilidad, y una colecci\'on de
medidas $\mathbf{P}^{x}$ tales que
$\mathbf{P}^{x}\left\{X_{0}=x\right\}$, y bajo cualquier
$\mathbf{P}^{x}$, $X_{t}$ es de Markov con semigrupo
$\left(P_{t}\right)$. $\mathbf{P}^{x}$ puede considerarse como la
distribuci\'on condicional de $\mathbf{P}$ dado $X_{0}=x$.

\begin{Def}\label{Def.2.2.S}
Sea $E$ espacio de Rad\'on, $\SG$ semigrupo de Markov en $\ER$. La
colecci\'on
$\mathbf{X}=\left(\Omega,\mathcal{G},\mathcal{G}_{t},X_{t},\theta_{t},\CM\right)$
es un proceso $\mathcal{E}$-Markov continuo por la derecha simple,
con espacio de estados $E$ y semigrupo de transici\'on $\SG$ en
caso de que $\mathbf{X}$ satisfaga las siguientes
condiciones:
\begin{itemize}
\item[i)] $\left(\Omega,\mathcal{G},\mathcal{G}_{t}\right)$ es un
espacio de medida filtrado, y $X_{t}$ es un proceso $E$-valuado
continuo por la derecha $\mathcal{E}^{*}$-adaptado a
$\left(\mathcal{G}_{t}\right)$;

\item[ii)] $\left(\theta_{t}\right)_{t\geq0}$ es una colecci\'on
de operadores {\em shift} para $X$, es decir, mapea $\Omega$ en
s\'i mismo satisfaciendo para $t,s\geq0$,

\begin{equation}\label{Eq.Shift}
\theta_{t}\circ\theta_{s}=\theta_{t+s}\textrm{ y
}X_{t}\circ\theta_{t}=X_{t+s};
\end{equation}

\item[iii)] Para cualquier $x\in E$,$\CM\left\{X_{0}=x\right\}=1$,
y el proceso $\PE$ tiene la propiedad de Markov (\ref{Eq.1.8.S})
con semigrupo de transici\'on $\SG$ relativo a
$\left(\Omega,\mathcal{G},\mathcal{G}_{t},\CM\right)$.
\end{itemize}
\end{Def}


\begin{Def}[HD2]\label{Eq.2.2.S}
Para cualquier $\alpha>0$ y cualquier $f\in S^{\alpha}$, el
proceso $t\rightarrow f\left(X_{t}\right)$ es continuo por la
derecha casi seguramente.
\end{Def}

\begin{Def}\label{Def.PD}
Un sistema
$\mathbf{X}=\left(\Omega,\mathcal{G},\mathcal{G}_{t},X_{t},\theta_{t},\CM\right)$
es un proceso derecho en el espacio de Rad\'on $E$ con semigrupo
de transici\'on $\SG$ provisto de:
\begin{itemize}
\item[i)] $\mathbf{X}$ es una realizaci\'on  continua por la
derecha, \ref{Def.2.2.S}, de $\SG$.

\item[ii)] $\mathbf{X}$ satisface la condicion HD2,
\ref{Eq.2.2.S}, relativa a $\mathcal{G}_{t}$.

\item[iii)] $\mathcal{G}_{t}$ es aumentado y continuo por la
derecha.
\end{itemize}
\end{Def}


\begin{Def}
Sea $X$ un conjunto y $\mathcal{F}$ una $\sigma$-\'algebra de
subconjuntos de $X$, la pareja $\left(X,\mathcal{F}\right)$ es
llamado espacio medible. Un subconjunto $A$ de $X$ es llamado
medible, o medible con respecto a $\mathcal{F}$, si
$A\in\mathcal{F}$.
\end{Def}

\begin{Def}
Sea $\left(X,\mathcal{F},\mu\right)$ espacio de medida. Se dice
que la medida $\mu$ es $\sigma$-finita si se puede escribir
$X=\bigcup_{n\geq1}X_{n}$ con $X_{n}\in\mathcal{F}$ y
$\mu\left(X_{n}\right)<\infty$.
\end{Def}

\begin{Def}\label{Cto.Borel}
Sea $X$ el conjunto de los \'umeros reales $\rea$. El \'algebra de
Borel es la $\sigma$-\'algebra $B$ generada por los intervalos
abiertos $\left(a,b\right)\in\rea$. Cualquier conjunto en $B$ es
llamado {\em Conjunto de Borel}.
\end{Def}

\begin{Def}\label{Funcion.Medible}
Una funci\'on $f:X\rightarrow\rea$, es medible si para cualquier
n\'umero real $\alpha$ el conjunto
\[\left\{x\in X:f\left(x\right)>\alpha\right\}\]
pertenece a $X$. Equivalentemente, se dice que $f$ es medible si
\[f^{-1}\left(\left(\alpha,\infty\right)\right)=\left\{x\in X:f\left(x\right)>\alpha\right\}\in\mathcal{F}.\]
\end{Def}


\begin{Def}\label{Def.Cilindros}
Sean $\left(\Omega_{i},\mathcal{F}_{i}\right)$, $i=1,2,\ldots,$
espacios medibles y $\Omega=\prod_{i=1}^{\infty}\Omega_{i}$ el
conjunto de todas las sucesiones
$\left(\omega_{1},\omega_{2},\ldots,\right)$ tales que
$\omega_{i}\in\Omega_{i}$, $i=1,2,\ldots,$. Si
$B^{n}\subset\prod_{i=1}^{\infty}\Omega_{i}$, definimos
$B_{n}=\left\{\omega\in\Omega:\left(\omega_{1},\omega_{2},\ldots,\omega_{n}\right)\in
B^{n}\right\}$. Al conjunto $B_{n}$ se le llama {\em cilindro} con
base $B^{n}$, el cilindro es llamado medible si
$B^{n}\in\prod_{i=1}^{\infty}\mathcal{F}_{i}$.
\end{Def}


\begin{Def}\label{Def.Proc.Adaptado}[TSP, Ash \cite{RBA}]
Sea $X\left(t\right),t\geq0$ proceso estoc\'astico, el proceso es
adaptado a la familia de $\sigma$-\'algebras $\mathcal{F}_{t}$,
para $t\geq0$, si para $s<t$ implica que
$\mathcal{F}_{s}\subset\mathcal{F}_{t}$, y $X\left(t\right)$ es
$\mathcal{F}_{t}$-medible para cada $t$. Si no se especifica
$\mathcal{F}_{t}$ entonces se toma $\mathcal{F}_{t}$ como
$\mathcal{F}\left(X\left(s\right),s\leq t\right)$, la m\'as
peque\~na $\sigma$-\'algebra de subconjuntos de $\Omega$ que hace
que cada $X\left(s\right)$, con $s\leq t$ sea Borel medible.
\end{Def}


\begin{Def}\label{Def.Tiempo.Paro}[TSP, Ash \cite{RBA}]
Sea $\left\{\mathcal{F}\left(t\right),t\geq0\right\}$ familia
creciente de sub $\sigma$-\'algebras. es decir,
$\mathcal{F}\left(s\right)\subset\mathcal{F}\left(t\right)$ para
$s\leq t$. Un tiempo de paro para $\mathcal{F}\left(t\right)$ es
una funci\'on $T:\Omega\rightarrow\left[0,\infty\right]$ tal que
$\left\{T\leq t\right\}\in\mathcal{F}\left(t\right)$ para cada
$t\geq0$. Un tiempo de paro para el proceso estoc\'astico
$X\left(t\right),t\geq0$ es un tiempo de paro para las
$\sigma$-\'algebras
$\mathcal{F}\left(t\right)=\mathcal{F}\left(X\left(s\right)\right)$.
\end{Def}

\begin{Def}
Sea $X\left(t\right),t\geq0$ proceso estoc\'astico, con
$\left(S,\chi\right)$ espacio de estados. Se dice que el proceso
es adaptado a $\left\{\mathcal{F}\left(t\right)\right\}$, es
decir, si para cualquier $s,t\in I$, $I$ conjunto de \'indices,
$s<t$, se tiene que
$\mathcal{F}\left(s\right)\subset\mathcal{F}\left(t\right)$ y
$X\left(t\right)$ es $\mathcal{F}\left(t\right)$-medible,
\end{Def}

\begin{Def}
Sea $X\left(t\right),t\geq0$ proceso estoc\'astico, se dice que es
un Proceso de Markov relativo a $\mathcal{F}\left(t\right)$ o que
$\left\{X\left(t\right),\mathcal{F}\left(t\right)\right\}$ es de
Markov si y s\'olo si para cualquier conjunto $B\in\chi$,  y
$s,t\in I$, $s<t$ se cumple que
\begin{equation}\label{Prop.Markov}
P\left\{X\left(t\right)\in
B|\mathcal{F}\left(s\right)\right\}=P\left\{X\left(t\right)\in
B|X\left(s\right)\right\}.
\end{equation}
\end{Def}
\begin{Note}
Si se dice que $\left\{X\left(t\right)\right\}$ es un Proceso de
Markov sin mencionar $\mathcal{F}\left(t\right)$, se asumir\'a que
\begin{eqnarray*}
\mathcal{F}\left(t\right)=\mathcal{F}_{0}\left(t\right)=\mathcal{F}\left(X\left(r\right),r\leq
t\right),
\end{eqnarray*}
entonces la ecuaci\'on (\ref{Prop.Markov}) se puede escribir como
\begin{equation}
P\left\{X\left(t\right)\in B|X\left(r\right),r\leq s\right\} =
P\left\{X\left(t\right)\in B|X\left(s\right)\right\}
\end{equation}
\end{Note}

\begin{Teo}
Sea $\left(X_{n},\mathcal{F}_{n},n=0,1,\ldots,\right\}$ Proceso de
Markov con espacio de estados $\left(S_{0},\chi_{0}\right)$
generado por una distribuici\'on inicial $P_{o}$ y probabilidad de
transici\'on $p_{mn}$, para $m,n=0,1,\ldots,$ $m<n$, que por
notaci\'on se escribir\'a como $p\left(m,n,x,B\right)\rightarrow
p_{mn}\left(x,B\right)$. Sea $S$ tiempo de paro relativo a la
$\sigma$-\'algebra $\mathcal{F}_{n}$. Sea $T$ funci\'on medible,
$T:\Omega\rightarrow\left\{0,1,\ldots,\right\}$. Sup\'ongase que
$T\geq S$, entonces $T$ es tiempo de paro. Si $B\in\chi_{0}$,
entonces
\begin{equation}\label{Prop.Fuerte.Markov}
P\left\{X\left(T\right)\in
B,T<\infty|\mathcal{F}\left(S\right)\right\} =
p\left(S,T,X\left(s\right),B\right)
\end{equation}
en $\left\{T<\infty\right\}$.
\end{Teo}


Sea $K$ conjunto numerable y sea $d:K\rightarrow\nat$ funci\'on.
Para $v\in K$, $M_{v}$ es un conjunto abierto de
$\rea^{d\left(v\right)}$. Entonces \[E=\cup_{v\in
K}M_{v}=\left\{\left(v,\zeta\right):v\in K,\zeta\in
M_{v}\right\}.\]

Sea $\mathcal{E}$ la clase de conjuntos medibles en $E$:
\[\mathcal{E}=\left\{\cup_{v\in K}A_{v}:A_{v}\in \mathcal{M}_{v}\right\}.\]

donde $\mathcal{M}$ son los conjuntos de Borel de $M_{v}$.
Entonces $\left(E,\mathcal{E}\right)$ es un espacio de Borel. El
estado del proceso se denotar\'a por
$\mathbf{x}_{t}=\left(v_{t},\zeta_{t}\right)$. La distribuci\'on
de $\left(\mathbf{x}_{t}\right)$ est\'a determinada por por los
siguientes objetos:

\begin{itemize}
\item[i)] Los campos vectoriales $\left(\mathcal{H}_{v},v\in
K\right)$. \item[ii)] Una funci\'on medible $\lambda:E\rightarrow
\rea_{+}$. \item[iii)] Una medida de transici\'on
$Q:\mathcal{E}\times\left(E\cup\Gamma^{*}\right)\rightarrow\left[0,1\right]$
donde
\begin{equation}
\Gamma^{*}=\cup_{v\in K}\partial^{*}M_{v}.
\end{equation}
y
\begin{equation}
\partial^{*}M_{v}=\left\{z\in\partial M_{v}:\mathbf{\mathbf{\phi}_{v}\left(t,\zeta\right)=\mathbf{z}}\textrm{ para alguna }\left(t,\zeta\right)\in\rea_{+}\times M_{v}\right\}.
\end{equation}
$\partial M_{v}$ denota  la frontera de $M_{v}$.
\end{itemize}

El campo vectorial $\left(\mathcal{H}_{v},v\in K\right)$ se supone
tal que para cada $\mathbf{z}\in M_{v}$ existe una \'unica curva
integral $\mathbf{\phi}_{v}\left(t,\zeta\right)$ que satisface la
ecuaci\'on

\begin{equation}
\frac{d}{dt}f\left(\zeta_{t}\right)=\mathcal{H}f\left(\zeta_{t}\right),
\end{equation}
con $\zeta_{0}=\mathbf{z}$, para cualquier funci\'on suave
$f:\rea^{d}\rightarrow\rea$ y $\mathcal{H}$ denota el operador
diferencial de primer orden, con $\mathcal{H}=\mathcal{H}_{v}$ y
$\zeta_{t}=\mathbf{\phi}\left(t,\mathbf{z}\right)$. Adem\'as se
supone que $\mathcal{H}_{v}$ es conservativo, es decir, las curvas
integrales est\'an definidas para todo $t>0$.

Para $\mathbf{x}=\left(v,\zeta\right)\in E$ se denota
\[t^{*}\mathbf{x}=inf\left\{t>0:\mathbf{\phi}_{v}\left(t,\zeta\right)\in\partial^{*}M_{v}\right\}\]

En lo que respecta a la funci\'on $\lambda$, se supondr\'a que
para cada $\left(v,\zeta\right)\in E$ existe un $\epsilon>0$ tal
que la funci\'on
$s\rightarrow\lambda\left(v,\phi_{v}\left(s,\zeta\right)\right)\in
E$ es integrable para $s\in\left[0,\epsilon\right)$. La medida de
transici\'on $Q\left(A;\mathbf{x}\right)$ es una funci\'on medible
de $\mathbf{x}$ para cada $A\in\mathcal{E}$, definida para
$\mathbf{x}\in E\cup\Gamma^{*}$ y es una medida de probabilidad en
$\left(E,\mathcal{E}\right)$ para cada $\mathbf{x}\in E$.

El movimiento del proceso $\left(\mathbf{x}_{t}\right)$ comenzando
en $\mathbf{x}=\left(n,\mathbf{z}\right)\in E$ se puede construir
de la siguiente manera, def\'inase la funci\'on $F$ por

\begin{equation}
F\left(t\right)=\left\{\begin{array}{ll}\\
exp\left(-\int_{0}^{t}\lambda\left(n,\phi_{n}\left(s,\mathbf{z}\right)\right)ds\right), & t<t^{*}\left(\mathbf{x}\right),\\
0, & t\geq t^{*}\left(\mathbf{x}\right)
\end{array}\right.
\end{equation}

Sea $T_{1}$ una variable aleatoria tal que
$\prob\left[T_{1}>t\right]=F\left(t\right)$, ahora sea la variable
aleatoria $\left(N,Z\right)$ con distribuici\'on
$Q\left(\cdot;\phi_{n}\left(T_{1},\mathbf{z}\right)\right)$. La
trayectoria de $\left(\mathbf{x}_{t}\right)$ para $t\leq T_{1}$
es\footnote{Revisar p\'agina 362, y 364 de Davis \cite{Davis}.}
\begin{eqnarray*}
\mathbf{x}_{t}=\left(v_{t},\zeta_{t}\right)=\left\{\begin{array}{ll}
\left(n,\phi_{n}\left(t,\mathbf{z}\right)\right), & t<T_{1},\\
\left(N,\mathbf{Z}\right), & t=t_{1}.
\end{array}\right.
\end{eqnarray*}

Comenzando en $\mathbf{x}_{T_{1}}$ se selecciona el siguiente
tiempo de intersalto $T_{2}-T_{1}$ lugar del post-salto
$\mathbf{x}_{T_{2}}$ de manera similar y as\'i sucesivamente. Este
procedimiento nos da una trayectoria determinista por partes
$\mathbf{x}_{t}$ con tiempos de salto $T_{1},T_{2},\ldots$. Bajo
las condiciones enunciadas para $\lambda,T_{1}>0$  y
$T_{1}-T_{2}>0$ para cada $i$, con probabilidad 1. Se supone que
se cumple la siquiente condici\'on.

\begin{Sup}[Supuesto 3.1, Davis \cite{Davis}]\label{Sup3.1.Davis}
Sea $N_{t}:=\sum_{t}\indora_{\left(t\geq t\right)}$ el n\'umero de
saltos en $\left[0,t\right]$. Entonces
\begin{equation}
\esp\left[N_{t}\right]<\infty\textrm{ para toda }t.
\end{equation}
\end{Sup}

es un proceso de Markov, m\'as a\'un, es un Proceso Fuerte de
Markov, es decir, la Propiedad Fuerte de Markov se cumple para
cualquier tiempo de paro.
%_________________________________________________________________________

En esta secci\'on se har\'an las siguientes consideraciones: $E$
es un espacio m\'etrico separable y la m\'etrica $d$ es compatible
con la topolog\'ia.


\begin{Def}
Un espacio topol\'ogico $E$ es llamado {\em Luisin} si es
homeomorfo a un subconjunto de Borel de un espacio m\'etrico
compacto.
\end{Def}

\begin{Def}
Un espacio topol\'ogico $E$ es llamado de {\em Rad\'on} si es
homeomorfo a un subconjunto universalmente medible de un espacio
m\'etrico compacto.
\end{Def}

Equivalentemente, la definici\'on de un espacio de Rad\'on puede
encontrarse en los siguientes t\'erminos:


\begin{Def}
$E$ es un espacio de Rad\'on si cada medida finita en
$\left(E,\mathcal{B}\left(E\right)\right)$ es regular interior o cerrada,
{\em tight}.
\end{Def}

\begin{Def}
Una medida finita, $\lambda$ en la $\sigma$-\'algebra de Borel de
un espacio metrizable $E$ se dice cerrada si
\begin{equation}\label{Eq.A2.3}
\lambda\left(E\right)=sup\left\{\lambda\left(K\right):K\textrm{ es
compacto en }E\right\}.
\end{equation}
\end{Def}

El siguiente teorema nos permite tener una mejor caracterizaci\'on de los espacios de Rad\'on:
\begin{Teo}\label{Tma.A2.2}
Sea $E$ espacio separable metrizable. Entonces $E$ es Radoniano si y s\'olo s\'i cada medida finita en $\left(E,\mathcal{B}\left(E\right)\right)$ es cerrada.
\end{Teo}

%_________________________________________________________________________________________
\subsection{Propiedades de Markov}
%_________________________________________________________________________________________

Sea $E$ espacio de estados, tal que $E$ es un espacio de Rad\'on, $\mathcal{B}\left(E\right)$ $\sigma$-\'algebra de Borel en $E$, que se denotar\'a por $\mathcal{E}$.

Sea $\left(X,\mathcal{G},\prob\right)$ espacio de probabilidad, $I\subset\rea$ conjunto de índices. Sea $\mathcal{F}_{\leq t}$ la $\sigma$-\'algebra natural definida como $\sigma\left\{f\left(X_{r}\right):r\in I, rleq t,f\in\mathcal{E}\right\}$. Se considerar\'a una $\sigma$-\'algebra m\'as general, $ \left(\mathcal{G}_{t}\right)$ tal que $\left(X_{t}\right)$ sea $\mathcal{E}$-adaptado.

\begin{Def}
Una familia $\left(P_{s,t}\right)$ de kernels de Markov en $\left(E,\mathcal{E}\right)$ indexada por pares $s,t\in I$, con $s\leq t$ es una funci\'on de transici\'on en $\ER$, si  para todo $r\leq s< t$ en $I$ y todo $x\in E$, $B\in\mathcal{E}$
\begin{equation}\label{Eq.Kernels}
P_{r,t}\left(x,B\right)=\int_{E}P_{r,s}\left(x,dy\right)P_{s,t}\left(y,B\right)\footnote{Ecuaci\'on de Chapman-Kolmogorov}.
\end{equation}
\end{Def}

Se dice que la funci\'on de transici\'on $\KM$ en $\ER$ es la funci\'on de transici\'on para un proceso $\PE$  con valores en $E$ y que satisface la propiedad de Markov\footnote{\begin{equation}\label{Eq.1.4.S}
\prob\left\{H|\mathcal{G}_{t}\right\}=\prob\left\{H|X_{t}\right\}\textrm{ }H\in p\mathcal{F}_{\geq t}.
\end{equation}} (\ref{Eq.1.4.S}) relativa a $\left(\mathcal{G}_{t}\right)$ si 

\begin{equation}\label{Eq.1.6.S}
\prob\left\{f\left(X_{t}\right)|\mathcal{G}_{s}\right\}=P_{s,t}f\left(X_{t}\right)\textrm{ }s\leq t\in I,\textrm{ }f\in b\mathcal{E}.
\end{equation}

\begin{Def}
Una familia $\left(P_{t}\right)_{t\geq0}$ de kernels de Markov en $\ER$ es llamada {\em Semigrupo de Transici\'on de Markov} o {\em Semigrupo de Transici\'on} si
\[P_{t+s}f\left(x\right)=P_{t}\left(P_{s}f\right)\left(x\right),\textrm{ }t,s\geq0,\textrm{ }x\in E\textrm{ }f\in b\mathcal{E}.\]
\end{Def}
\begin{Note}
Si la funci\'on de transici\'on $\KM$ es llamada homog\'enea si $P_{s,t}=P_{t-s}$.
\end{Note}

Un proceso de Markov que satisface la ecuaci\'on (\ref{Eq.1.6.S}) con funci\'on de transici\'on homog\'enea $\left(P_{t}\right)$ tiene la propiedad caracter\'istica
\begin{equation}\label{Eq.1.8.S}
\prob\left\{f\left(X_{t+s}\right)|\mathcal{G}_{t}\right\}=P_{s}f\left(X_{t}\right)\textrm{ }t,s\geq0,\textrm{ }f\in b\mathcal{E}.
\end{equation}
La ecuaci\'on anterior es la {\em Propiedad Simple de Markov} de $X$ relativa a $\left(P_{t}\right)$.

En este sentido el proceso $\PE$ cumple con la propiedad de Markov (\ref{Eq.1.8.S}) relativa a $\left(\Omega,\mathcal{G},\mathcal{G}_{t},\prob\right)$ con semigrupo de transici\'on $\left(P_{t}\right)$.
%_________________________________________________________________________________________
\subsection{Primer Condici\'on de Regularidad}
%_________________________________________________________________________________________
%\newcommand{\EM}{\left(\Omega,\mathcal{G},\prob\right)}
%\newcommand{\E4}{\left(\Omega,\mathcal{G},\mathcal{G}_{t},\prob\right)}
\begin{Def}
Un proceso estoc\'astico $\PE$ definido en $\left(\Omega,\mathcal{G},\prob\right)$ con valores en el espacio topol\'ogico $E$ es continuo por la derecha si cada trayectoria muestral $t\rightarrow X_{t}\left(w\right)$ es un mapeo continuo por la derecha de $I$ en $E$.
\end{Def}

\begin{Def}[HD1]\label{Eq.2.1.S}
Un semigrupo de Markov $\left/P_{t}\right)$ en un espacio de Rad\'on $E$ se dice que satisface la condici\'on {\em HD1} si, dada una medida de probabilidad $\mu$ en $E$, existe una $\sigma$-\'algebra $\mathcal{E^{*}}$ con $\mathcal{E}\subset\mathcal{E}$ y $P_{t}\left(b\mathcal{E}^{*}\right)\subset b\mathcal{E}^{*}$, y un $\mathcal{E}^{*}$-proceso $E$-valuado continuo por la derecha $\PE$ en alg\'un espacio de probabilidad filtrado $\left(\Omega,\mathcal{G},\mathcal{G}_{t},\prob\right)$ tal que $X=\left(\Omega,\mathcal{G},\mathcal{G}_{t},\prob\right)$ es de Markov (Homog\'eneo) con semigrupo de transici\'on $(P_{t})$ y distribuci\'on inicial $\mu$.
\end{Def}

Considerese la colecci\'on de variables aleatorias $X_{t}$ definidas en alg\'un espacio de probabilidad, y una colecci\'on de medidas $\mathbf{P}^{x}$ tales que $\mathbf{P}^{x}\left\{X_{0}=x\right\}$, y bajo cualquier $\mathbf{P}^{x}$, $X_{t}$ es de Markov con semigrupo $\left(P_{t}\right)$. $\mathbf{P}^{x}$ puede considerarse como la distribuci\'on condicional de $\mathbf{P}$ dado $X_{0}=x$.

\begin{Def}\label{Def.2.2.S}
Sea $E$ espacio de Rad\'on, $\SG$ semigrupo de Markov en $\ER$. La colecci\'on $\mathbf{X}=\left(\Omega,\mathcal{G},\mathcal{G}_{t},X_{t},\theta_{t},\CM\right)$ es un proceso $\mathcal{E}$-Markov continuo por la derecha simple, con espacio de estados $E$ y semigrupo de transici\'on $\SG$ en caso de que $\mathbf{X}$ satisfaga las siguientes condiciones:
\begin{itemize}
\item[i)] $\left(\Omega,\mathcal{G},\mathcal{G}_{t}\right)$ es un espacio de medida filtrado, y $X_{t}$ es un proceso $E$-valuado continuo por la derecha $\mathcal{E}^{*}$-adaptado a $\left(\mathcal{G}_{t}\right)$;

\item[ii)] $\left(\theta_{t}\right)_{t\geq0}$ es una colecci\'on de operadores {\em shift} para $X$, es decir, mapea $\Omega$ en s\'i mismo satisfaciendo para $t,s\geq0$,

\begin{equation}\label{Eq.Shift}
\theta_{t}\circ\theta_{s}=\theta_{t+s}\textrm{ y }X_{t}\circ\theta_{t}=X_{t+s};
\end{equation}

\item[iii)] Para cualquier $x\in E$,$\CM\left\{X_{0}=x\right\}=1$, y el proceso $\PE$ tiene la propiedad de Markov (\ref{Eq.1.8.S}) con semigrupo de transici\'on $\SG$ relativo a $\left(\Omega,\mathcal{G},\mathcal{G}_{t},\CM\right)$.
\end{itemize}
\end{Def}

\begin{Def}[HD2]\label{Eq.2.2.S}
Para cualquier $\alpha>0$ y cualquier $f\in S^{\alpha}$, el proceso $t\rightarrow f\left(X_{t}\right)$ es continuo por la derecha casi seguramente.
\end{Def}

\begin{Def}\label{Def.PD}
Un sistema $\mathbf{X}=\left(\Omega,\mathcal{G},\mathcal{G}_{t},X_{t},\theta_{t},\CM\right)$ es un proceso derecho en el espacio de Rad\'on $E$ con semigrupo de transici\'on $\SG$ provisto de:
\begin{itemize}
\item[i)] $\mathbf{X}$ es una realizaci\'on  continua por la derecha, \ref{Def.2.2.S}, de $\SG$.

\item[ii)] $\mathbf{X}$ satisface la condicion HD2, \ref{Eq.2.2.S}, relativa a $\mathcal{G}_{t}$.

\item[iii)] $\mathcal{G}_{t}$ es aumentado y continuo por la derecha.
\end{itemize}
\end{Def}



\begin{Lema}[Lema 4.2, Dai\cite{Dai}]\label{Lema4.2}
Sea $\left\{x_{n}\right\}\subset \mathbf{X}$ con
$|x_{n}|\rightarrow\infty$, conforme $n\rightarrow\infty$. Suponga
que
\[lim_{n\rightarrow\infty}\frac{1}{|x_{n}|}U\left(0\right)=\overline{U}\]
y
\[lim_{n\rightarrow\infty}\frac{1}{|x_{n}|}V\left(0\right)=\overline{V}.\]

Entonces, conforme $n\rightarrow\infty$, casi seguramente

\begin{equation}\label{E1.4.2}
\frac{1}{|x_{n}|}\Phi^{k}\left(\left[|x_{n}|t\right]\right)\rightarrow
P_{k}^{'}t\textrm{, u.o.c.,}
\end{equation}

\begin{equation}\label{E1.4.3}
\frac{1}{|x_{n}|}E^{x_{n}}_{k}\left(|x_{n}|t\right)\rightarrow
\alpha_{k}\left(t-\overline{U}_{k}\right)^{+}\textrm{, u.o.c.,}
\end{equation}

\begin{equation}\label{E1.4.4}
\frac{1}{|x_{n}|}S^{x_{n}}_{k}\left(|x_{n}|t\right)\rightarrow
\mu_{k}\left(t-\overline{V}_{k}\right)^{+}\textrm{, u.o.c.,}
\end{equation}

donde $\left[t\right]$ es la parte entera de $t$ y
$\mu_{k}=1/m_{k}=1/\esp\left[\eta_{k}\left(1\right)\right]$.
\end{Lema}

\begin{Lema}[Lema 4.3, Dai\cite{Dai}]\label{Lema.4.3}
Sea $\left\{x_{n}\right\}\subset \mathbf{X}$ con
$|x_{n}|\rightarrow\infty$, conforme $n\rightarrow\infty$. Suponga
que
\[lim_{n\rightarrow\infty}\frac{1}{|x_{n}|}U\left(0\right)=\overline{U}_{k}\]
y
\[lim_{n\rightarrow\infty}\frac{1}{|x_{n}|}V\left(0\right)=\overline{V}_{k}.\]
\begin{itemize}
\item[a)] Conforme $n\rightarrow\infty$ casi seguramente,
\[lim_{n\rightarrow\infty}\frac{1}{|x_{n}|}U^{x_{n}}_{k}\left(|x_{n}|t\right)=\left(\overline{U}_{k}-t\right)^{+}\textrm{, u.o.c.}\]
y
\[lim_{n\rightarrow\infty}\frac{1}{|x_{n}|}V^{x_{n}}_{k}\left(|x_{n}|t\right)=\left(\overline{V}_{k}-t\right)^{+}.\]

\item[b)] Para cada $t\geq0$ fijo,
\[\left\{\frac{1}{|x_{n}|}U^{x_{n}}_{k}\left(|x_{n}|t\right),|x_{n}|\geq1\right\}\]
y
\[\left\{\frac{1}{|x_{n}|}V^{x_{n}}_{k}\left(|x_{n}|t\right),|x_{n}|\geq1\right\}\]
\end{itemize}
son uniformemente convergentes.
\end{Lema}

$S_{l}^{x}\left(t\right)$ es el n\'umero total de servicios
completados de la clase $l$, si la clase $l$ est\'a dando $t$
unidades de tiempo de servicio. Sea $T_{l}^{x}\left(x\right)$ el
monto acumulado del tiempo de servicio que el servidor
$s\left(l\right)$ gasta en los usuarios de la clase $l$ al tiempo
$t$. Entonces $S_{l}^{x}\left(T_{l}^{x}\left(t\right)\right)$ es
el n\'umero total de servicios completados para la clase $l$ al
tiempo $t$. Una fracci\'on de estos usuarios,
$\Phi_{l}^{x}\left(S_{l}^{x}\left(T_{l}^{x}\left(t\right)\right)\right)$,
se convierte en usuarios de la clase $k$.\\

Entonces, dado lo anterior, se tiene la siguiente representaci\'on
para el proceso de la longitud de la cola:\\

\begin{equation}
Q_{k}^{x}\left(t\right)=_{k}^{x}\left(0\right)+E_{k}^{x}\left(t\right)+\sum_{l=1}^{K}\Phi_{k}^{l}\left(S_{l}^{x}\left(T_{l}^{x}\left(t\right)\right)\right)-S_{k}^{x}\left(T_{k}^{x}\left(t\right)\right)
\end{equation}
para $k=1,\ldots,K$. Para $i=1,\ldots,d$, sea
\[I_{i}^{x}\left(t\right)=t-\sum_{j\in C_{i}}T_{k}^{x}\left(t\right).\]

Entonces $I_{i}^{x}\left(t\right)$ es el monto acumulado del
tiempo que el servidor $i$ ha estado desocupado al tiempo $t$. Se
est\'a asumiendo que las disciplinas satisfacen la ley de
conservaci\'on del trabajo, es decir, el servidor $i$ est\'a en
pausa solamente cuando no hay usuarios en la estaci\'on $i$.
Entonces, se tiene que

\begin{equation}
\int_{0}^{\infty}\left(\sum_{k\in
C_{i}}Q_{k}^{x}\left(t\right)\right)dI_{i}^{x}\left(t\right)=0,
\end{equation}
para $i=1,\ldots,d$.\\

Hacer
\[T^{x}\left(t\right)=\left(T_{1}^{x}\left(t\right),\ldots,T_{K}^{x}\left(t\right)\right)^{'},\]
\[I^{x}\left(t\right)=\left(I_{1}^{x}\left(t\right),\ldots,I_{K}^{x}\left(t\right)\right)^{'}\]
y
\[S^{x}\left(T^{x}\left(t\right)\right)=\left(S_{1}^{x}\left(T_{1}^{x}\left(t\right)\right),\ldots,S_{K}^{x}\left(T_{K}^{x}\left(t\right)\right)\right)^{'}.\]

Para una disciplina que cumple con la ley de conservaci\'on del
trabajo, en forma vectorial, se tiene el siguiente conjunto de
ecuaciones

\begin{equation}\label{Eq.MF.1.3}
Q^{x}\left(t\right)=Q^{x}\left(0\right)+E^{x}\left(t\right)+\sum_{l=1}^{K}\Phi^{l}\left(S_{l}^{x}\left(T_{l}^{x}\left(t\right)\right)\right)-S^{x}\left(T^{x}\left(t\right)\right),\\
\end{equation}

\begin{equation}\label{Eq.MF.2.3}
Q^{x}\left(t\right)\geq0,\\
\end{equation}

\begin{equation}\label{Eq.MF.3.3}
T^{x}\left(0\right)=0,\textrm{ y }\overline{T}^{x}\left(t\right)\textrm{ es no decreciente},\\
\end{equation}

\begin{equation}\label{Eq.MF.4.3}
I^{x}\left(t\right)=et-CT^{x}\left(t\right)\textrm{ es no
decreciente}\\
\end{equation}

\begin{equation}\label{Eq.MF.5.3}
\int_{0}^{\infty}\left(CQ^{x}\left(t\right)\right)dI_{i}^{x}\left(t\right)=0,\\
\end{equation}

\begin{equation}\label{Eq.MF.6.3}
\textrm{Condiciones adicionales en
}\left(\overline{Q}^{x}\left(\cdot\right),\overline{T}^{x}\left(\cdot\right)\right)\textrm{
espec\'ificas de la disciplina de la cola,}
\end{equation}

donde $e$ es un vector de unos de dimensi\'on $d$, $C$ es la
matriz definida por
\[C_{ik}=\left\{\begin{array}{cc}
1,& S\left(k\right)=i,\\
0,& \textrm{ en otro caso}.\\
\end{array}\right.
\]
Es necesario enunciar el siguiente Teorema que se utilizar\'a para
el Teorema \ref{Tma.4.2.Dai}:
\begin{Teo}[Teorema 4.1, Dai \cite{Dai}]
Considere una disciplina que cumpla la ley de conservaci\'on del
trabajo, para casi todas las trayectorias muestrales $\omega$ y
cualquier sucesi\'on de estados iniciales
$\left\{x_{n}\right\}\subset \mathbf{X}$, con
$|x_{n}|\rightarrow\infty$, existe una subsucesi\'on
$\left\{x_{n_{j}}\right\}$ con $|x_{n_{j}}|\rightarrow\infty$ tal
que
\begin{equation}\label{Eq.4.15}
\frac{1}{|x_{n_{j}}|}\left(Q^{x_{n_{j}}}\left(0\right),U^{x_{n_{j}}}\left(0\right),V^{x_{n_{j}}}\left(0\right)\right)\rightarrow\left(\overline{Q}\left(0\right),\overline{U},\overline{V}\right),
\end{equation}

\begin{equation}\label{Eq.4.16}
\frac{1}{|x_{n_{j}}|}\left(Q^{x_{n_{j}}}\left(|x_{n_{j}}|t\right),T^{x_{n_{j}}}\left(|x_{n_{j}}|t\right)\right)\rightarrow\left(\overline{Q}\left(t\right),\overline{T}\left(t\right)\right)\textrm{
u.o.c.}
\end{equation}

Adem\'as,
$\left(\overline{Q}\left(t\right),\overline{T}\left(t\right)\right)$
satisface las siguientes ecuaciones:
\begin{equation}\label{Eq.MF.1.3a}
\overline{Q}\left(t\right)=Q\left(0\right)+\left(\alpha
t-\overline{U}\right)^{+}-\left(I-P\right)^{'}M^{-1}\left(\overline{T}\left(t\right)-\overline{V}\right)^{+},
\end{equation}

\begin{equation}\label{Eq.MF.2.3a}
\overline{Q}\left(t\right)\geq0,\\
\end{equation}

\begin{equation}\label{Eq.MF.3.3a}
\overline{T}\left(t\right)\textrm{ es no decreciente y comienza en cero},\\
\end{equation}

\begin{equation}\label{Eq.MF.4.3a}
\overline{I}\left(t\right)=et-C\overline{T}\left(t\right)\textrm{
es no decreciente,}\\
\end{equation}

\begin{equation}\label{Eq.MF.5.3a}
\int_{0}^{\infty}\left(C\overline{Q}\left(t\right)\right)d\overline{I}\left(t\right)=0,\\
\end{equation}

\begin{equation}\label{Eq.MF.6.3a}
\textrm{Condiciones adicionales en
}\left(\overline{Q}\left(\cdot\right),\overline{T}\left(\cdot\right)\right)\textrm{
especficas de la disciplina de la cola,}
\end{equation}
\end{Teo}

\begin{Def}[Definici\'on 4.1, , Dai \cite{Dai}]
Sea una disciplina de servicio espec\'ifica. Cualquier l\'imite
$\left(\overline{Q}\left(\cdot\right),\overline{T}\left(\cdot\right)\right)$
en \ref{Eq.4.16} es un {\em flujo l\'imite} de la disciplina.
Cualquier soluci\'on (\ref{Eq.MF.1.3a})-(\ref{Eq.MF.6.3a}) es
llamado flujo soluci\'on de la disciplina. Se dice que el modelo de flujo l\'imite, modelo de flujo, de la disciplina de la cola es estable si existe una constante
$\delta>0$ que depende de $\mu,\alpha$ y $P$ solamente, tal que
cualquier flujo l\'imite con
$|\overline{Q}\left(0\right)|+|\overline{U}|+|\overline{V}|=1$, se
tiene que $\overline{Q}\left(\cdot+\delta\right)\equiv0$.
\end{Def}

\begin{Teo}[Teorema 4.2, Dai\cite{Dai}]\label{Tma.4.2.Dai}
Sea una disciplina fija para la cola, suponga que se cumplen las
condiciones (1.2)-(1.5). Si el modelo de flujo l\'imite de la
disciplina de la cola es estable, entonces la cadena de Markov $X$
que describe la din\'amica de la red bajo la disciplina es Harris
recurrente positiva.
\end{Teo}

Ahora se procede a escalar el espacio y el tiempo para reducir la
aparente fluctuaci\'on del modelo. Consid\'erese el proceso
\begin{equation}\label{Eq.3.7}
\overline{Q}^{x}\left(t\right)=\frac{1}{|x|}Q^{x}\left(|x|t\right)
\end{equation}
A este proceso se le conoce como el fluido escalado, y cualquier l\'imite $\overline{Q}^{x}\left(t\right)$ es llamado flujo l\'imite del proceso de longitud de la cola. Haciendo $|q|\rightarrow\infty$ mientras se mantiene el resto de las componentes fijas, cualquier punto l\'imite del proceso de longitud de la cola normalizado $\overline{Q}^{x}$ es soluci\'on del siguiente modelo de flujo.

Al conjunto de ecuaciones dadas en \ref{Eq.3.8}-\ref{Eq.3.13} se
le llama {\em Modelo de flujo} y al conjunto de todas las
soluciones del modelo de flujo
$\left(\overline{Q}\left(\cdot\right),\overline{T}
\left(\cdot\right)\right)$ se le denotar\'a por $\mathcal{Q}$.

Si se hace $|x|\rightarrow\infty$ sin restringir ninguna de las
componentes, tambi\'en se obtienen un modelo de flujo, pero en
este caso el residual de los procesos de arribo y servicio
introducen un retraso:

\begin{Def}[Definici\'on 3.3, Dai y Meyn \cite{DaiSean}]
El modelo de flujo es estable si existe un tiempo fijo $t_{0}$ tal
que $\overline{Q}\left(t\right)=0$, con $t\geq t_{0}$, para
cualquier $\overline{Q}\left(\cdot\right)\in\mathcal{Q}$ que
cumple con $|\overline{Q}\left(0\right)|=1$.
\end{Def}

El siguiente resultado se encuentra en Chen \cite{Chen}.
\begin{Lemma}[Lema 3.1, Dai y Meyn \cite{DaiSean}]
Si el modelo de flujo definido por \ref{Eq.3.8}-\ref{Eq.3.13} es
estable, entonces el modelo de flujo retrasado es tambi\'en
estable, es decir, existe $t_{0}>0$ tal que
$\overline{Q}\left(t\right)=0$ para cualquier $t\geq t_{0}$, para
cualquier soluci\'on del modelo de flujo retrasado cuya
condici\'on inicial $\overline{x}$ satisface que
$|\overline{x}|=|\overline{Q}\left(0\right)|+|\overline{A}\left(0\right)|+|\overline{B}\left(0\right)|\leq1$.
\end{Lemma}


Propiedades importantes para el modelo de flujo retrasado:

\begin{Prop}
 Sea $\left(\overline{Q},\overline{T},\overline{T}^{0}\right)$ un flujo l\'imite de \ref{Eq.4.4} y suponga que cuando $x\rightarrow\infty$ a lo largo de
una subsucesi\'on
\[\left(\frac{1}{|x|}Q_{k}^{x}\left(0\right),\frac{1}{|x|}A_{k}^{x}\left(0\right),\frac{1}{|x|}B_{k}^{x}\left(0\right),\frac{1}{|x|}B_{k}^{x,0}\left(0\right)\right)\rightarrow\left(\overline{Q}_{k}\left(0\right),0,0,0\right)\]
para $k=1,\ldots,K$. EL flujo l\'imite tiene las siguientes
propiedades, donde las propiedades de la derivada se cumplen donde
la derivada exista:
\begin{itemize}
 \item[i)] Los vectores de tiempo ocupado $\overline{T}\left(t\right)$ y $\overline{T}^{0}\left(t\right)$ son crecientes y continuas con
$\overline{T}\left(0\right)=\overline{T}^{0}\left(0\right)=0$.
\item[ii)] Para todo $t\geq0$
\[\sum_{k=1}^{K}\left[\overline{T}_{k}\left(t\right)+\overline{T}_{k}^{0}\left(t\right)\right]=t\]
\item[iii)] Para todo $1\leq k\leq K$
\[\overline{Q}_{k}\left(t\right)=\overline{Q}_{k}\left(0\right)+\alpha_{k}t-\mu_{k}\overline{T}_{k}\left(t\right)\]
\item[iv)]  Para todo $1\leq k\leq K$
\[\dot{{\overline{T}}}_{k}\left(t\right)=\beta_{k}\] para $\overline{Q}_{k}\left(t\right)=0$.
\item[v)] Para todo $k,j$
\[\mu_{k}^{0}\overline{T}_{k}^{0}\left(t\right)=\mu_{j}^{0}\overline{T}_{j}^{0}\left(t\right)\]
\item[vi)]  Para todo $1\leq k\leq K$
\[\mu_{k}\dot{{\overline{T}}}_{k}\left(t\right)=l_{k}\mu_{k}^{0}\dot{{\overline{T}}}_{k}^{0}\left(t\right)\] para $\overline{Q}_{k}\left(t\right)>0$.
\end{itemize}
\end{Prop}

\begin{Lema}[Lema 3.1 \cite{Chen}]\label{Lema3.1}
Si el modelo de flujo es estable, definido por las ecuaciones
(3.8)-(3.13), entonces el modelo de flujo retrasado tambin es
estable.
\end{Lema}

\begin{Teo}[Teorema 5.2 \cite{Chen}]\label{Tma.5.2}
Si el modelo de flujo lineal correspondiente a la red de cola es
estable, entonces la red de colas es estable.
\end{Teo}

\begin{Teo}[Teorema 5.1 \cite{Chen}]\label{Tma.5.1.Chen}
La red de colas es estable si existe una constante $t_{0}$ que
depende de $\left(\alpha,\mu,T,U\right)$ y $V$ que satisfagan las
ecuaciones (5.1)-(5.5), $Z\left(t\right)=0$, para toda $t\geq
t_{0}$.
\end{Teo}



\begin{Lema}[Lema 5.2 \cite{Gut}]\label{Lema.5.2.Gut}
Sea $\left\{\xi\left(k\right):k\in\ent\right\}$ sucesin de
variables aleatorias i.i.d. con valores en
$\left(0,\infty\right)$, y sea $E\left(t\right)$ el proceso de
conteo
\[E\left(t\right)=max\left\{n\geq1:\xi\left(1\right)+\cdots+\xi\left(n-1\right)\leq t\right\}.\]
Si $E\left[\xi\left(1\right)\right]<\infty$, entonces para
cualquier entero $r\geq1$
\begin{equation}
lim_{t\rightarrow\infty}\esp\left[\left(\frac{E\left(t\right)}{t}\right)^{r}\right]=\left(\frac{1}{E\left[\xi_{1}\right]}\right)^{r}
\end{equation}
de aqu, bajo estas condiciones
\begin{itemize}
\item[a)] Para cualquier $t>0$,
$sup_{t\geq\delta}\esp\left[\left(\frac{E\left(t\right)}{t}\right)^{r}\right]$

\item[b)] Las variables aleatorias
$\left\{\left(\frac{E\left(t\right)}{t}\right)^{r}:t\geq1\right\}$
son uniformemente integrables.
\end{itemize}
\end{Lema}

\begin{Teo}[Teorema 5.1: Ley Fuerte para Procesos de Conteo
\cite{Gut}]\label{Tma.5.1.Gut} Sea
$0<\mu<\esp\left(X_{1}\right]\leq\infty$. entonces

\begin{itemize}
\item[a)] $\frac{N\left(t\right)}{t}\rightarrow\frac{1}{\mu}$
a.s., cuando $t\rightarrow\infty$.


\item[b)]$\esp\left[\frac{N\left(t\right)}{t}\right]^{r}\rightarrow\frac{1}{\mu^{r}}$,
cuando $t\rightarrow\infty$ para todo $r>0$..
\end{itemize}
\end{Teo}


\begin{Prop}[Proposicin 5.1 \cite{DaiSean}]\label{Prop.5.1}
Suponga que los supuestos (A1) y (A2) se cumplen, adems suponga
que el modelo de flujo es estable. Entonces existe $t_{0}>0$ tal
que
\begin{equation}\label{Eq.Prop.5.1}
lim_{|x|\rightarrow\infty}\frac{1}{|x|^{p+1}}\esp_{x}\left[|X\left(t_{0}|x|\right)|^{p+1}\right]=0.
\end{equation}

\end{Prop}


\begin{Prop}[Proposici\'on 5.3 \cite{DaiSean}]
Sea $X$ proceso de estados para la red de colas, y suponga que se
cumplen los supuestos (A1) y (A2), entonces para alguna constante
positiva $C_{p+1}<\infty$, $\delta>0$ y un conjunto compacto
$C\subset X$.

\begin{equation}\label{Eq.5.4}
\esp_{x}\left[\int_{0}^{\tau_{C}\left(\delta\right)}\left(1+|X\left(t\right)|^{p}\right)dt\right]\leq
C_{p+1}\left(1+|x|^{p+1}\right)
\end{equation}
\end{Prop}

\begin{Prop}[Proposici\'on 5.4 \cite{DaiSean}]
Sea $X$ un proceso de Markov Borel Derecho en $X$, sea
$f:X\leftarrow\rea_{+}$ y defina para alguna $\delta>0$, y un
conjunto cerrado $C\subset X$
\[V\left(x\right):=\esp_{x}\left[\int_{0}^{\tau_{C}\left(\delta\right)}f\left(X\left(t\right)\right)dt\right]\]
para $x\in X$. Si $V$ es finito en todas partes y uniformemente
acotada en $C$, entonces existe $k<\infty$ tal que
\begin{equation}\label{Eq.5.11}
\frac{1}{t}\esp_{x}\left[V\left(x\right)\right]+\frac{1}{t}\int_{0}^{t}\esp_{x}\left[f\left(X\left(s\right)\right)ds\right]\leq\frac{1}{t}V\left(x\right)+k,
\end{equation}
para $x\in X$ y $t>0$.
\end{Prop}


\begin{Teo}[Teorema 5.5 \cite{DaiSean}]
Suponga que se cumplen (A1) y (A2), adems suponga que el modelo
de flujo es estable. Entonces existe una constante $k_{p}<\infty$
tal que
\begin{equation}\label{Eq.5.13}
\frac{1}{t}\int_{0}^{t}\esp_{x}\left[|Q\left(s\right)|^{p}\right]ds\leq
k_{p}\left\{\frac{1}{t}|x|^{p+1}+1\right\}
\end{equation}
para $t\geq0$, $x\in X$. En particular para cada condici\'on inicial
\begin{equation}\label{Eq.5.14}
Limsup_{t\rightarrow\infty}\frac{1}{t}\int_{0}^{t}\esp_{x}\left[|Q\left(s\right)|^{p}\right]ds\leq
k_{p}
\end{equation}
\end{Teo}

\begin{Teo}[Teorema 6.2\cite{DaiSean}]\label{Tma.6.2}
Suponga que se cumplen los supuestos (A1)-(A3) y que el modelo de
flujo es estable, entonces se tiene que
\[\parallel P^{t}\left(c,\cdot\right)-\pi\left(\cdot\right)\parallel_{f_{p}}\rightarrow0\]
para $t\rightarrow\infty$ y $x\in X$. En particular para cada
condicin inicial
\[lim_{t\rightarrow\infty}\esp_{x}\left[\left|Q_{t}\right|^{p}\right]=\esp_{\pi}\left[\left|Q_{0}\right|^{p}\right]<\infty\]
\end{Teo}


\begin{Teo}[Teorema 6.3\cite{DaiSean}]\label{Tma.6.3}
Suponga que se cumplen los supuestos (A1)-(A3) y que el modelo de
flujo es estable, entonces con
$f\left(x\right)=f_{1}\left(x\right)$, se tiene que
\[lim_{t\rightarrow\infty}t^{(p-1)\left|P^{t}\left(c,\cdot\right)-\pi\left(\cdot\right)\right|_{f}=0},\]
para $x\in X$. En particular, para cada condicin inicial
\[lim_{t\rightarrow\infty}t^{(p-1)\left|\esp_{x}\left[Q_{t}\right]-\esp_{\pi}\left[Q_{0}\right]\right|=0}.\]
\end{Teo}



\begin{Prop}[Proposici\'on 5.1, Dai y Meyn \cite{DaiSean}]\label{Prop.5.1.DaiSean}
Suponga que los supuestos A1) y A2) son ciertos y que el modelo de flujo es estable. Entonces existe $t_{0}>0$ tal que
\begin{equation}
lim_{|x|\rightarrow\infty}\frac{1}{|x|^{p+1}}\esp_{x}\left[|X\left(t_{0}|x|\right)|^{p+1}\right]=0
\end{equation}
\end{Prop}

\begin{Lemma}[Lema 5.2, Dai y Meyn \cite{DaiSean}]\label{Lema.5.2.DaiSean}
 Sea $\left\{\zeta\left(k\right):k\in \mathbb{z}\right\}$ una sucesi\'on independiente e id\'enticamente distribuida que toma valores en $\left(0,\infty\right)$,
y sea
$E\left(t\right)=max\left(n\geq1:\zeta\left(1\right)+\cdots+\zeta\left(n-1\right)\leq
t\right)$. Si $\esp\left[\zeta\left(1\right)\right]<\infty$,
entonces para cualquier entero $r\geq1$
\begin{equation}
 lim_{t\rightarrow\infty}\esp\left[\left(\frac{E\left(t\right)}{t}\right)^{r}\right]=\left(\frac{1}{\esp\left[\zeta_{1}\right]}\right)^{r}.
\end{equation}
Luego, bajo estas condiciones:
\begin{itemize}
 \item[a)] para cualquier $\delta>0$, $\sup_{t\geq\delta}\esp\left[\left(\frac{E\left(t\right)}{t}\right)^{r}\right]<\infty$
\item[b)] las variables aleatorias
$\left\{\left(\frac{E\left(t\right)}{t}\right)^{r}:t\geq1\right\}$
son uniformemente integrables.
\end{itemize}
\end{Lemma}

\begin{Teo}[Teorema 5.5, Dai y Meyn \cite{DaiSean}]\label{Tma.5.5.DaiSean}
Suponga que los supuestos A1) y A2) se cumplen y que el modelo de
flujo es estable. Entonces existe una constante $\kappa_{p}$ tal
que
\begin{equation}
\frac{1}{t}\int_{0}^{t}\esp_{x}\left[|Q\left(s\right)|^{p}\right]ds\leq\kappa_{p}\left\{\frac{1}{t}|x|^{p+1}+1\right\}
\end{equation}
para $t>0$ y $x\in X$. En particular, para cada condici\'on
inicial
\begin{eqnarray*}
\limsup_{t\rightarrow\infty}\frac{1}{t}\int_{0}^{t}\esp_{x}\left[|Q\left(s\right)|^{p}\right]ds\leq\kappa_{p}.
\end{eqnarray*}
\end{Teo}

\begin{Teo}[Teorema 6.2, Dai y Meyn \cite{DaiSean}]\label{Tma.6.2.DaiSean}
Suponga que se cumplen los supuestos A1), A2) y A3) y que el
modelo de flujo es estable. Entonces se tiene que
\begin{equation}
\left\|P^{t}\left(x,\cdot\right)-\pi\left(\cdot\right)\right\|_{f_{p}}\textrm{,
}t\rightarrow\infty,x\in X.
\end{equation}
En particular para cada condici\'on inicial
\begin{eqnarray*}
\lim_{t\rightarrow\infty}\esp_{x}\left[|Q\left(t\right)|^{p}\right]=\esp_{\pi}\left[|Q\left(0\right)|^{p}\right]\leq\kappa_{r}
\end{eqnarray*}
\end{Teo}
\begin{Teo}[Teorema 6.3, Dai y Meyn \cite{DaiSean}]\label{Tma.6.3.DaiSean}
Suponga que se cumplen los supuestos A1), A2) y A3) y que el
modelo de flujo es estable. Entonces con
$f\left(x\right)=f_{1}\left(x\right)$ se tiene
\begin{equation}
\lim_{t\rightarrow\infty}t^{p-1}\left\|P^{t}\left(x,\cdot\right)-\pi\left(\cdot\right)\right\|_{f}=0.
\end{equation}
En particular para cada condici\'on inicial
\begin{eqnarray*}
\lim_{t\rightarrow\infty}t^{p-1}|\esp_{x}\left[Q\left(t\right)\right]-\esp_{\pi}\left[Q\left(0\right)\right]|=0.
\end{eqnarray*}
\end{Teo}

\begin{Teo}[Teorema 6.4, Dai y Meyn \cite{DaiSean}]\label{Tma.6.4.DaiSean}
Suponga que se cumplen los supuestos A1), A2) y A3) y que el
modelo de flujo es estable. Sea $\nu$ cualquier distribuci\'on de
probabilidad en $\left(X,\mathcal{B}_{X}\right)$, y $\pi$ la
distribuci\'on estacionaria de $X$.
\begin{itemize}
\item[i)] Para cualquier $f:X\leftarrow\rea_{+}$
\begin{equation}
\lim_{t\rightarrow\infty}\frac{1}{t}\int_{o}^{t}f\left(X\left(s\right)\right)ds=\pi\left(f\right):=\int
f\left(x\right)\pi\left(dx\right)
\end{equation}
$\prob$-c.s.

\item[ii)] Para cualquier $f:X\leftarrow\rea_{+}$ con
$\pi\left(|f|\right)<\infty$, la ecuaci\'on anterior se cumple.
\end{itemize}
\end{Teo}

\begin{Teo}[Teorema 2.2, Down \cite{Down}]\label{Tma2.2.Down}
Suponga que el fluido modelo es inestable en el sentido de que
para alguna $\epsilon_{0},c_{0}\geq0$,
\begin{equation}\label{Eq.Inestability}
|Q\left(T\right)|\geq\epsilon_{0}T-c_{0}\textrm{,   }T\geq0,
\end{equation}
para cualquier condici\'on inicial $Q\left(0\right)$, con
$|Q\left(0\right)|=1$. Entonces para cualquier $0<q\leq1$, existe
$B<0$ tal que para cualquier $|x|\geq B$,
\begin{equation}
\prob_{x}\left\{\mathbb{X}\rightarrow\infty\right\}\geq q.
\end{equation}
\end{Teo}


%_________________________________________________________________________
\subsection{Supuestos}
%_________________________________________________________________________
Consideremos el caso en el que se tienen varias colas a las cuales
llegan uno o varios servidores para dar servicio a los usuarios
que se encuentran presentes en la cola, como ya se mencion\'o hay
varios tipos de pol\'iticas de servicio, incluso podr\'ia ocurrir
que la manera en que atiende al resto de las colas sea distinta a
como lo hizo en las anteriores.\\

Para ejemplificar los sistemas de visitas c\'iclicas se
considerar\'a el caso en que a las colas los usuarios son atendidos con
una s\'ola pol\'itica de servicio.\\


Si $\omega$ es el n\'umero de usuarios en la cola al comienzo del
periodo de servicio y $N\left(\omega\right)$ es el n\'umero de
usuarios que son atendidos con una pol\'itica en espec\'ifico
durante el periodo de servicio, entonces se asume que:
\begin{itemize}
\item[1)]\label{S1}$lim_{\omega\rightarrow\infty}\esp\left[N\left(\omega\right)\right]=\overline{N}>0$;
\item[2)]\label{S2}$\esp\left[N\left(\omega\right)\right]\leq\overline{N}$
para cualquier valor de $\omega$.
\end{itemize}
La manera en que atiende el servidor $m$-\'esimo, es la siguiente:
\begin{itemize}
\item Al t\'ermino de la visita a la cola $j$, el servidor cambia
a la cola $j^{'}$ con probabilidad $r_{j,j^{'}}^{m}$

\item La $n$-\'esima vez que el servidor cambia de la cola $j$ a
$j'$, va acompa\~nada con el tiempo de cambio de longitud
$\delta_{j,j^{'}}^{m}\left(n\right)$, con
$\delta_{j,j^{'}}^{m}\left(n\right)$, $n\geq1$, variables
aleatorias independientes e id\'enticamente distribuidas, tales
que $\esp\left[\delta_{j,j^{'}}^{m}\left(1\right)\right]\geq0$.

\item Sea $\left\{p_{j}^{m}\right\}$ la distribuci\'on invariante
estacionaria \'unica para la Cadena de Markov con matriz de
transici\'on $\left(r_{j,j^{'}}^{m}\right)$, se supone que \'esta
existe.

\item Finalmente, se define el tiempo promedio total de traslado
entre las colas como
\begin{equation}
\delta^{*}:=\sum_{j,j^{'}}p_{j}^{m}r_{j,j^{'}}^{m}\esp\left[\delta_{j,j^{'}}^{m}\left(i\right)\right].
\end{equation}
\end{itemize}

Consideremos el caso donde los tiempos entre arribo a cada una de
las colas, $\left\{\xi_{k}\left(n\right)\right\}_{n\geq1}$ son
variables aleatorias independientes a id\'enticamente
distribuidas, y los tiempos de servicio en cada una de las colas
se distribuyen de manera independiente e id\'enticamente
distribuidas $\left\{\eta_{k}\left(n\right)\right\}_{n\geq1}$;
adem\'as ambos procesos cumplen la condici\'on de ser
independientes entre s\'i. Para la $k$-\'esima cola se define la
tasa de arribo por
$\lambda_{k}=1/\esp\left[\xi_{k}\left(1\right)\right]$ y la tasa
de servicio como
$\mu_{k}=1/\esp\left[\eta_{k}\left(1\right)\right]$, finalmente se
define la carga de la cola como $\rho_{k}=\lambda_{k}/\mu_{k}$,
donde se pide que $\rho=\sum_{k=1}^{K}\rho_{k}<1$, para garantizar
la estabilidad del sistema, esto es cierto para las pol\'iticas de
servicio exhaustiva y cerrada, ver Geetor \cite{Getoor}.\\

Si denotamos por
\begin{itemize}
\item $Q_{k}\left(t\right)$ el n\'umero de usuarios presentes en
la cola $k$ al tiempo $t$; \item $A_{k}\left(t\right)$ los
residuales de los tiempos entre arribos a la cola $k$; para cada
servidor $m$; \item $B_{m}\left(t\right)$ denota a los residuales
de los tiempos de servicio al tiempo $t$; \item
$B_{m}^{0}\left(t\right)$ los residuales de los tiempos de
traslado de la cola $k$ a la pr\'oxima por atender al tiempo $t$,

\item sea
$C_{m}\left(t\right)$ el n\'umero de usuarios atendidos durante la
visita del servidor a la cola $k$ al tiempo $t$.
\end{itemize}


En este sentido, el proceso para el sistema de visitas se puede
definir como:

\begin{equation}\label{Esp.Edos.Down}
X\left(t\right)^{T}=\left(Q_{k}\left(t\right),A_{k}\left(t\right),B_{m}\left(t\right),B_{m}^{0}\left(t\right),C_{m}\left(t\right)\right),
\end{equation}
para $k=1,\ldots,K$ y $m=1,2,\ldots,M$, donde $T$ indica que es el
transpuesto del vector que se est\'a definiendo. El proceso $X$
evoluciona en el espacio de estados:
$\mathbb{X}=\ent_{+}^{K}\times\rea_{+}^{K}\times\left(\left\{1,2,\ldots,K\right\}\times\left\{1,2,\ldots,S\right\}\right)^{M}\times\rea_{+}^{K}\times\ent_{+}^{K}$.\\

El sistema aqu\'i descrito debe de cumplir con los siguientes supuestos b\'asicos de un sistema de visitas:
%__________________________________________________________________________
\subsubsection{Supuestos B\'asicos}
%__________________________________________________________________________
\begin{itemize}
\item[A1)] Los procesos
$\xi_{1},\ldots,\xi_{K},\eta_{1},\ldots,\eta_{K}$ son mutuamente
independientes y son sucesiones independientes e id\'enticamente
distribuidas.

\item[A2)] Para alg\'un entero $p\geq1$
\begin{eqnarray*}
\esp\left[\xi_{l}\left(1\right)^{p+1}\right]&<&\infty\textrm{ para }l=1,\ldots,K\textrm{ y }\\
\esp\left[\eta_{k}\left(1\right)^{p+1}\right]&<&\infty\textrm{
para }k=1,\ldots,K.
\end{eqnarray*}
donde $\mathcal{A}$ es la clase de posibles arribos.

\item[A3)] Para cada $k=1,2,\ldots,K$ existe una funci\'on
positiva $q_{k}\left(\cdot\right)$ definida en $\rea_{+}$, y un
entero $j_{k}$, tal que
\begin{eqnarray}
P\left(\xi_{k}\left(1\right)\geq x\right)&>&0\textrm{, para todo }x>0,\\
P\left\{a\leq\sum_{i=1}^{j_{k}}\xi_{k}\left(i\right)\leq
b\right\}&\geq&\int_{a}^{b}q_{k}\left(x\right)dx, \textrm{ }0\leq
a<b.
\end{eqnarray}
\end{itemize}

En lo que respecta al supuesto (A3), en Dai y Meyn \cite{DaiSean}
hacen ver que este se puede sustituir por

\begin{itemize}
\item[A3')] Para el Proceso de Markov $X$, cada subconjunto
compacto del espacio de estados de $X$ es un conjunto peque\~no,
ver definici\'on \ref{Def.Cto.Peq.}.
\end{itemize}

Es por esta raz\'on que con la finalidad de poder hacer uso de
$A3^{'})$ es necesario recurrir a los Procesos de Harris y en
particular a los Procesos Harris Recurrente, ver \cite{Dai,
DaiSean}.
%_______________________________________________________________________
\subsection{Procesos Harris Recurrente}
%_______________________________________________________________________

Por el supuesto (A1) conforme a Davis \cite{Davis}, se puede
definir el proceso de saltos correspondiente de manera tal que
satisfaga el supuesto (A3'), de hecho la demostraci\'on est\'a
basada en la l\'inea de argumentaci\'on de Davis, \cite{Davis},
p\'aginas 362-364.\\

Entonces se tiene un espacio de estados en el cual el proceso $X$
satisface la Propiedad Fuerte de Markov, ver Dai y Meyn
\cite{DaiSean}, dado por

\[\left(\Omega,\mathcal{F},\mathcal{F}_{t},X\left(t\right),\theta_{t},P_{x}\right),\]
adem\'as de ser un proceso de Borel Derecho (Sharpe \cite{Sharpe})
en el espacio de estados medible
$\left(\mathbb{X},\mathcal{B}_\mathbb{X}\right)$. El Proceso
$X=\left\{X\left(t\right),t\geq0\right\}$ tiene trayectorias
continuas por la derecha, est\'a definido en
$\left(\Omega,\mathcal{F}\right)$ y est\'a adaptado a
$\left\{\mathcal{F}_{t},t\geq0\right\}$; la colecci\'on
$\left\{P_{x},x\in \mathbb{X}\right\}$ son medidas de probabilidad
en $\left(\Omega,\mathcal{F}\right)$ tales que para todo $x\in
\mathbb{X}$
\[P_{x}\left\{X\left(0\right)=x\right\}=1,\] y
\[E_{x}\left\{f\left(X\circ\theta_{t}\right)|\mathcal{F}_{t}\right\}=E_{X}\left(\tau\right)f\left(X\right),\]
en $\left\{\tau<\infty\right\}$, $P_{x}$-c.s., con $\theta_{t}$
definido como el operador shift.


Donde $\tau$ es un $\mathcal{F}_{t}$-tiempo de paro
\[\left(X\circ\theta_{\tau}\right)\left(w\right)=\left\{X\left(\tau\left(w\right)+t,w\right),t\geq0\right\},\]
y $f$ es una funci\'on de valores reales acotada y medible, ver \cite{Dai, KaspiMandelbaum}.\\

Sea $P^{t}\left(x,D\right)$, $D\in\mathcal{B}_{\mathbb{X}}$,
$t\geq0$ la probabilidad de transici\'on de $X$ queda definida
como:
\[P^{t}\left(x,D\right)=P_{x}\left(X\left(t\right)\in
D\right).\]


\begin{Def}
Una medida no cero $\pi$ en
$\left(\mathbb{X},\mathcal{B}_{\mathbb{X}}\right)$ es invariante
para $X$ si $\pi$ es $\sigma$-finita y
\[\pi\left(D\right)=\int_{\mathbb{X}}P^{t}\left(x,D\right)\pi\left(dx\right),\]
para todo $D\in \mathcal{B}_{\mathbb{X}}$, con $t\geq0$.
\end{Def}

\begin{Def}
El proceso de Markov $X$ es llamado Harris Recurrente si existe
una medida de probabilidad $\nu$ en
$\left(\mathbb{X},\mathcal{B}_{\mathbb{X}}\right)$, tal que si
$\nu\left(D\right)>0$ y $D\in\mathcal{B}_{\mathbb{X}}$
\[P_{x}\left\{\tau_{D}<\infty\right\}\equiv1,\] cuando
$\tau_{D}=inf\left\{t\geq0:X_{t}\in D\right\}$.
\end{Def}

\begin{Note}
\begin{itemize}
\item[i)] Si $X$ es Harris recurrente, entonces existe una \'unica
medida invariante $\pi$ (Getoor \cite{Getoor}).

\item[ii)] Si la medida invariante es finita, entonces puede
normalizarse a una medida de probabilidad, en este caso al proceso
$X$ se le llama Harris recurrente positivo.


\item[iii)] Cuando $X$ es Harris recurrente positivo se dice que
la disciplina de servicio es estable. En este caso $\pi$ denota la
distribuci\'on estacionaria y hacemos
\[P_{\pi}\left(\cdot\right)=\int_{\mathbf{X}}P_{x}\left(\cdot\right)\pi\left(dx\right),\]
y se utiliza $E_{\pi}$ para denotar el operador esperanza
correspondiente, ver \cite{DaiSean}.
\end{itemize}
\end{Note}

\begin{Def}\label{Def.Cto.Peq.}
Un conjunto $D\in\mathcal{B_{\mathbb{X}}}$ es llamado peque\~no si
existe un $t>0$, una medida de probabilidad $\nu$ en
$\mathcal{B_{\mathbb{X}}}$, y un $\delta>0$ tal que
\[P^{t}\left(x,A\right)\geq\delta\nu\left(A\right),\] para $x\in
D,A\in\mathcal{B_{\mathbb{X}}}$.
\end{Def}

La siguiente serie de resultados vienen enunciados y demostrados
en Dai \cite{Dai}:
\begin{Lema}[Lema 3.1, Dai \cite{Dai}]
Sea $B$ conjunto peque\~no cerrado, supongamos que
$P_{x}\left(\tau_{B}<\infty\right)\equiv1$ y que para alg\'un
$\delta>0$ se cumple que
\begin{equation}\label{Eq.3.1}
\sup\esp_{x}\left[\tau_{B}\left(\delta\right)\right]<\infty,
\end{equation}
donde
$\tau_{B}\left(\delta\right)=inf\left\{t\geq\delta:X\left(t\right)\in
B\right\}$. Entonces, $X$ es un proceso Harris recurrente
positivo.
\end{Lema}

\begin{Lema}[Lema 3.1, Dai \cite{Dai}]\label{Lema.3.}
Bajo el supuesto (A3), el conjunto
$B=\left\{x\in\mathbb{X}:|x|\leq k\right\}$ es un conjunto
peque\~no cerrado para cualquier $k>0$.
\end{Lema}

\begin{Teo}[Teorema 3.1, Dai \cite{Dai}]\label{Tma.3.1}
Si existe un $\delta>0$ tal que
\begin{equation}
lim_{|x|\rightarrow\infty}\frac{1}{|x|}\esp|X^{x}\left(|x|\delta\right)|=0,
\end{equation}
donde $X^{x}$ se utiliza para denotar que el proceso $X$ comienza
a partir de $x$, entonces la ecuaci\'on (\ref{Eq.3.1}) se cumple
para $B=\left\{x\in\mathbb{X}:|x|\leq k\right\}$ con alg\'un
$k>0$. En particular, $X$ es Harris recurrente positivo.
\end{Teo}

Entonces, tenemos que el proceso $X$ es un proceso de Markov que
cumple con los supuestos $A1)$-$A3)$, lo que falta de hacer es
construir el Modelo de Flujo bas\'andonos en lo hasta ahora
presentado.
%_______________________________________________________________________
\subsection{Modelo de Flujo}
%_______________________________________________________________________

Dada una condici\'on inicial $x\in\mathbb{X}$, sea

\begin{itemize}
\item $Q_{k}^{x}\left(t\right)$ la longitud de la cola al tiempo
$t$,

\item $T_{m,k}^{x}\left(t\right)$ el tiempo acumulado, al tiempo
$t$, que tarda el servidor $m$ en atender a los usuarios de la
cola $k$.

\item $T_{m,k}^{x,0}\left(t\right)$ el tiempo acumulado, al tiempo
$t$, que tarda el servidor $m$ en trasladarse a otra cola a partir de la $k$-\'esima.\\
\end{itemize}

Sup\'ongase que la funci\'on
$\left(\overline{Q}\left(\cdot\right),\overline{T}_{m}
\left(\cdot\right),\overline{T}_{m}^{0} \left(\cdot\right)\right)$
para $m=1,2,\ldots,M$ es un punto l\'imite de
\begin{equation}\label{Eq.Punto.Limite}
\left(\frac{1}{|x|}Q^{x}\left(|x|t\right),\frac{1}{|x|}T_{m}^{x}\left(|x|t\right),\frac{1}{|x|}T_{m}^{x,0}\left(|x|t\right)\right)
\end{equation}
para $m=1,2,\ldots,M$, cuando $x\rightarrow\infty$, ver
\cite{Down}. Entonces
$\left(\overline{Q}\left(t\right),\overline{T}_{m}
\left(t\right),\overline{T}_{m}^{0} \left(t\right)\right)$ es un
flujo l\'imite del sistema. Al conjunto de todos las posibles
flujos l\'imite se le llama {\emph{Modelo de Flujo}} y se le
denotar\'a por $\mathcal{Q}$, ver \cite{Down, Dai, DaiSean}.\\

El modelo de flujo satisface el siguiente conjunto de ecuaciones:

\begin{equation}\label{Eq.MF.1}
\overline{Q}_{k}\left(t\right)=\overline{Q}_{k}\left(0\right)+\lambda_{k}t-\sum_{m=1}^{M}\mu_{k}\overline{T}_{m,k}\left(t\right),\\
\end{equation}
para $k=1,2,\ldots,K$.\\
\begin{equation}\label{Eq.MF.2}
\overline{Q}_{k}\left(t\right)\geq0\textrm{ para
}k=1,2,\ldots,K.\\
\end{equation}

\begin{equation}\label{Eq.MF.3}
\overline{T}_{m,k}\left(0\right)=0,\textrm{ y }\overline{T}_{m,k}\left(\cdot\right)\textrm{ es no decreciente},\\
\end{equation}
para $k=1,2,\ldots,K$ y $m=1,2,\ldots,M$.\\
\begin{equation}\label{Eq.MF.4}
\sum_{k=1}^{K}\overline{T}_{m,k}^{0}\left(t\right)+\overline{T}_{m,k}\left(t\right)=t\textrm{
para }m=1,2,\ldots,M.\\
\end{equation}


\begin{Def}[Definici\'on 4.1, Dai \cite{Dai}]\label{Def.Modelo.Flujo}
Sea una disciplina de servicio espec\'ifica. Cualquier l\'imite
$\left(\overline{Q}\left(\cdot\right),\overline{T}\left(\cdot\right),\overline{T}^{0}\left(\cdot\right)\right)$
en (\ref{Eq.Punto.Limite}) es un {\em flujo l\'imite} de la
disciplina. Cualquier soluci\'on (\ref{Eq.MF.1})-(\ref{Eq.MF.4})
es llamado flujo soluci\'on de la disciplina.
\end{Def}

\begin{Def}
Se dice que el modelo de flujo l\'imite, modelo de flujo, de la
disciplina de la cola es estable si existe una constante
$\delta>0$ que depende de $\mu,\lambda$ y $P$ solamente, tal que
cualquier flujo l\'imite con
$|\overline{Q}\left(0\right)|+|\overline{U}|+|\overline{V}|=1$, se
tiene que $\overline{Q}\left(\cdot+\delta\right)\equiv0$.
\end{Def}

Si se hace $|x|\rightarrow\infty$ sin restringir ninguna de las
componentes, tambi\'en se obtienen un modelo de flujo, pero en
este caso el residual de los procesos de arribo y servicio
introducen un retraso:
\begin{Teo}[Teorema 4.2, Dai \cite{Dai}]\label{Tma.4.2.Dai}
Sea una disciplina fija para la cola, suponga que se cumplen las
condiciones (A1)-(A3). Si el modelo de flujo l\'imite de la
disciplina de la cola es estable, entonces la cadena de Markov $X$
que describe la din\'amica de la red bajo la disciplina es Harris
recurrente positiva.
\end{Teo}

Ahora se procede a escalar el espacio y el tiempo para reducir la
aparente fluctuaci\'on del modelo. Consid\'erese el proceso
\begin{equation}\label{Eq.3.7}
\overline{Q}^{x}\left(t\right)=\frac{1}{|x|}Q^{x}\left(|x|t\right).
\end{equation}
A este proceso se le conoce como el flujo escalado, y cualquier
l\'imite $\overline{Q}^{x}\left(t\right)$ es llamado flujo
l\'imite del proceso de longitud de la cola. Haciendo
$|q|\rightarrow\infty$ mientras se mantiene el resto de las
componentes fijas, cualquier punto l\'imite del proceso de
longitud de la cola normalizado $\overline{Q}^{x}$ es soluci\'on
del siguiente modelo de flujo.


\begin{Def}[Definici\'on 3.3, Dai y Meyn \cite{DaiSean}]
El modelo de flujo es estable si existe un tiempo fijo $t_{0}$ tal
que $\overline{Q}\left(t\right)=0$, con $t\geq t_{0}$, para
cualquier $\overline{Q}\left(\cdot\right)\in\mathcal{Q}$ que
cumple con $|\overline{Q}\left(0\right)|=1$.
\end{Def}

\begin{Lemma}[Lema 3.1, Dai y Meyn \cite{DaiSean}]
Si el modelo de flujo definido por (\ref{Eq.MF.1})-(\ref{Eq.MF.4})
es estable, entonces el modelo de flujo retrasado es tambi\'en
estable, es decir, existe $t_{0}>0$ tal que
$\overline{Q}\left(t\right)=0$ para cualquier $t\geq t_{0}$, para
cualquier soluci\'on del modelo de flujo retrasado cuya
condici\'on inicial $\overline{x}$ satisface que
$|\overline{x}|=|\overline{Q}\left(0\right)|+|\overline{A}\left(0\right)|+|\overline{B}\left(0\right)|\leq1$.
\end{Lemma}


Ahora ya estamos en condiciones de enunciar los resultados principales:


\begin{Teo}[Teorema 2.1, Down \cite{Down}]\label{Tma2.1.Down}
Suponga que el modelo de flujo es estable, y que se cumplen los supuestos (A1) y (A2), entonces
\begin{itemize}
\item[i)] Para alguna constante $\kappa_{p}$, y para cada
condici\'on inicial $x\in X$
\begin{equation}\label{Estability.Eq1}
\limsup_{t\rightarrow\infty}\frac{1}{t}\int_{0}^{t}\esp_{x}\left[|Q\left(s\right)|^{p}\right]ds\leq\kappa_{p},
\end{equation}
donde $p$ es el entero dado en (A2).
\end{itemize}
Si adem\'as se cumple la condici\'on (A3), entonces para cada
condici\'on inicial:
\begin{itemize}
\item[ii)] Los momentos transitorios convergen a su estado
estacionario:
 \begin{equation}\label{Estability.Eq2}
lim_{t\rightarrow\infty}\esp_{x}\left[Q_{k}\left(t\right)^{r}\right]=\esp_{\pi}\left[Q_{k}\left(0\right)^{r}\right]\leq\kappa_{r},
\end{equation}
para $r=1,2,\ldots,p$ y $k=1,2,\ldots,K$. Donde $\pi$ es la
probabilidad invariante para $X$.

\item[iii)]  El primer momento converge con raz\'on $t^{p-1}$:
\begin{equation}\label{Estability.Eq3}
lim_{t\rightarrow\infty}t^{p-1}|\esp_{x}\left[Q_{k}\left(t\right)\right]-\esp_{\pi}\left[Q_{k}\left(0\right)\right]|=0.
\end{equation}

\item[iv)] La {\em Ley Fuerte de los grandes n\'umeros} se cumple:
\begin{equation}\label{Estability.Eq4}
lim_{t\rightarrow\infty}\frac{1}{t}\int_{0}^{t}Q_{k}^{r}\left(s\right)ds=\esp_{\pi}\left[Q_{k}\left(0\right)^{r}\right],\textrm{
}\prob_{x}\textrm{-c.s.}
\end{equation}
para $r=1,2,\ldots,p$ y $k=1,2,\ldots,K$.
\end{itemize}
\end{Teo}

La contribuci\'on de Down a la teor\'ia de los {\emph {sistemas de
visitas c\'iclicas}}, es la relaci\'on que hay entre la
estabilidad del sistema con el comportamiento de las medidas de
desempe\~no, es decir, la condici\'on suficiente para poder
garantizar la convergencia del proceso de la longitud de la cola
as\'i como de por los menos los dos primeros momentos adem\'as de
una versi\'on de la Ley Fuerte de los Grandes N\'umeros para los
sistemas de visitas.


\begin{Teo}[Teorema 2.3, Down \cite{Down}]\label{Tma2.3.Down}
Considere el siguiente valor:
\begin{equation}\label{Eq.Rho.1serv}
\rho=\sum_{k=1}^{K}\rho_{k}+max_{1\leq j\leq K}\left(\frac{\lambda_{j}}{\sum_{s=1}^{S}p_{js}\overline{N}_{s}}\right)\delta^{*}
\end{equation}
\begin{itemize}
\item[i)] Si $\rho<1$ entonces la red es estable, es decir, se
cumple el Teorema \ref{Tma2.1.Down}.

\item[ii)] Si $\rho>1$ entonces la red es inestable, es decir, se
cumple el Teorema \ref{Tma2.2.Down}
\end{itemize}
\end{Teo}




Dado el proceso $X=\left\{X\left(t\right),t\geq0\right\}$ definido
en (\ref{Esp.Edos.Down}) que describe la din\'amica del sistema de
visitas c\'iclicas, si $U\left(t\right)$ es el residual de los
tiempos de llegada al tiempo $t$ entre dos usuarios consecutivos y
$V\left(t\right)$ es el residual de los tiempos de servicio al
tiempo $t$ para el usuario que est\'as siendo atendido por el
servidor. Sea $\mathbb{X}$ el espacio de estados que puede tomar
el proceso $X$.


\begin{Lema}[Lema 4.3, Dai\cite{Dai}]\label{Lema.4.3}
Sea $\left\{x_{n}\right\}\subset \mathbf{X}$ con
$|x_{n}|\rightarrow\infty$, conforme $n\rightarrow\infty$. Suponga
que
\[lim_{n\rightarrow\infty}\frac{1}{|x_{n}|}U\left(0\right)=\overline{U}_{k},\]
y
\[lim_{n\rightarrow\infty}\frac{1}{|x_{n}|}V\left(0\right)=\overline{V}_{k}.\]
\begin{itemize}
\item[a)] Conforme $n\rightarrow\infty$ casi seguramente,
\[lim_{n\rightarrow\infty}\frac{1}{|x_{n}|}U^{x_{n}}_{k}\left(|x_{n}|t\right)=\left(\overline{U}_{k}-t\right)^{+}\textrm{, u.o.c.}\]
y
\[lim_{n\rightarrow\infty}\frac{1}{|x_{n}|}V^{x_{n}}_{k}\left(|x_{n}|t\right)=\left(\overline{V}_{k}-t\right)^{+}.\]

\item[b)] Para cada $t\geq0$ fijo,
\[\left\{\frac{1}{|x_{n}|}U^{x_{n}}_{k}\left(|x_{n}|t\right),|x_{n}|\geq1\right\}\]
y
\[\left\{\frac{1}{|x_{n}|}V^{x_{n}}_{k}\left(|x_{n}|t\right),|x_{n}|\geq1\right\}\]
\end{itemize}
son uniformemente convergentes.
\end{Lema}

Sea $e$ es un vector de unos, $C$ es la matriz definida por
\[C_{ik}=\left\{\begin{array}{cc}
1,& S\left(k\right)=i,\\
0,& \textrm{ en otro caso}.\\
\end{array}\right.
\]
Es necesario enunciar el siguiente Teorema que se utilizar\'a para
el Teorema (\ref{Tma.4.2.Dai}):
\begin{Teo}[Teorema 4.1, Dai \cite{Dai}]
Considere una disciplina que cumpla la ley de conservaci\'on, para
casi todas las trayectorias muestrales $\omega$ y cualquier
sucesi\'on de estados iniciales $\left\{x_{n}\right\}\subset
\mathbf{X}$, con $|x_{n}|\rightarrow\infty$, existe una
subsucesi\'on $\left\{x_{n_{j}}\right\}$ con
$|x_{n_{j}}|\rightarrow\infty$ tal que
\begin{equation}\label{Eq.4.15}
\frac{1}{|x_{n_{j}}|}\left(Q^{x_{n_{j}}}\left(0\right),U^{x_{n_{j}}}\left(0\right),V^{x_{n_{j}}}\left(0\right)\right)\rightarrow\left(\overline{Q}\left(0\right),\overline{U},\overline{V}\right),
\end{equation}

\begin{equation}\label{Eq.4.16}
\frac{1}{|x_{n_{j}}|}\left(Q^{x_{n_{j}}}\left(|x_{n_{j}}|t\right),T^{x_{n_{j}}}\left(|x_{n_{j}}|t\right)\right)\rightarrow\left(\overline{Q}\left(t\right),\overline{T}\left(t\right)\right)\textrm{
u.o.c.}
\end{equation}

Adem\'as,
$\left(\overline{Q}\left(t\right),\overline{T}\left(t\right)\right)$
satisface las siguientes ecuaciones:
\begin{equation}\label{Eq.MF.1.3a}
\overline{Q}\left(t\right)=Q\left(0\right)+\left(\alpha
t-\overline{U}\right)^{+}-\left(I-P\right)^{'}M^{-1}\left(\overline{T}\left(t\right)-\overline{V}\right)^{+},
\end{equation}

\begin{equation}\label{Eq.MF.2.3a}
\overline{Q}\left(t\right)\geq0,\\
\end{equation}

\begin{equation}\label{Eq.MF.3.3a}
\overline{T}\left(t\right)\textrm{ es no decreciente y comienza en cero},\\
\end{equation}

\begin{equation}\label{Eq.MF.4.3a}
\overline{I}\left(t\right)=et-C\overline{T}\left(t\right)\textrm{
es no decreciente,}\\
\end{equation}

\begin{equation}\label{Eq.MF.5.3a}
\int_{0}^{\infty}\left(C\overline{Q}\left(t\right)\right)d\overline{I}\left(t\right)=0,\\
\end{equation}

\begin{equation}\label{Eq.MF.6.3a}
\textrm{Condiciones en
}\left(\overline{Q}\left(\cdot\right),\overline{T}\left(\cdot\right)\right)\textrm{
espec\'ificas de la disciplina de la cola,}
\end{equation}
\end{Teo}


Propiedades importantes para el modelo de flujo retrasado:

\begin{Prop}[Proposici\'on 4.2, Dai \cite{Dai}]
 Sea $\left(\overline{Q},\overline{T},\overline{T}^{0}\right)$ un flujo l\'imite de \ref{Eq.Punto.Limite}
 y suponga que cuando $x\rightarrow\infty$ a lo largo de una subsucesi\'on
\[\left(\frac{1}{|x|}Q_{k}^{x}\left(0\right),\frac{1}{|x|}A_{k}^{x}\left(0\right),\frac{1}{|x|}B_{k}^{x}\left(0\right),\frac{1}{|x|}B_{k}^{x,0}\left(0\right)\right)\rightarrow\left(\overline{Q}_{k}\left(0\right),0,0,0\right)\]
para $k=1,\ldots,K$. El flujo l\'imite tiene las siguientes
propiedades, donde las propiedades de la derivada se cumplen donde
la derivada exista:
\begin{itemize}
 \item[i)] Los vectores de tiempo ocupado $\overline{T}\left(t\right)$ y $\overline{T}^{0}\left(t\right)$ son crecientes y continuas con
$\overline{T}\left(0\right)=\overline{T}^{0}\left(0\right)=0$.
\item[ii)] Para todo $t\geq0$
\[\sum_{k=1}^{K}\left[\overline{T}_{k}\left(t\right)+\overline{T}_{k}^{0}\left(t\right)\right]=t.\]
\item[iii)] Para todo $1\leq k\leq K$
\[\overline{Q}_{k}\left(t\right)=\overline{Q}_{k}\left(0\right)+\alpha_{k}t-\mu_{k}\overline{T}_{k}\left(t\right).\]
\item[iv)]  Para todo $1\leq k\leq K$
\[\dot{{\overline{T}}}_{k}\left(t\right)=\rho_{k}\] para $\overline{Q}_{k}\left(t\right)=0$.
\item[v)] Para todo $k,j$
\[\mu_{k}^{0}\overline{T}_{k}^{0}\left(t\right)=\mu_{j}^{0}\overline{T}_{j}^{0}\left(t\right).\]
\item[vi)]  Para todo $1\leq k\leq K$
\[\mu_{k}\dot{{\overline{T}}}_{k}\left(t\right)=l_{k}\mu_{k}^{0}\dot{{\overline{T}}}_{k}^{0}\left(t\right),\] para $\overline{Q}_{k}\left(t\right)>0$.
\end{itemize}
\end{Prop}

\begin{Lema}[Lema 3.1, Chen \cite{Chen}]\label{Lema3.1}
Si el modelo de flujo es estable, definido por las ecuaciones
(3.8)-(3.13), entonces el modelo de flujo retrasado tambi\'en es
estable.
\end{Lema}

\begin{Lema}[Lema 5.2, Gut \cite{Gut}]\label{Lema.5.2.Gut}
Sea $\left\{\xi\left(k\right):k\in\ent\right\}$ sucesi\'on de
variables aleatorias i.i.d. con valores en
$\left(0,\infty\right)$, y sea $E\left(t\right)$ el proceso de
conteo
\[E\left(t\right)=max\left\{n\geq1:\xi\left(1\right)+\cdots+\xi\left(n-1\right)\leq t\right\}.\]
Si $E\left[\xi\left(1\right)\right]<\infty$, entonces para
cualquier entero $r\geq1$
\begin{equation}
lim_{t\rightarrow\infty}\esp\left[\left(\frac{E\left(t\right)}{t}\right)^{r}\right]=\left(\frac{1}{E\left[\xi_{1}\right]}\right)^{r},
\end{equation}
de aqu\'i, bajo estas condiciones
\begin{itemize}
\item[a)] Para cualquier $t>0$,
$sup_{t\geq\delta}\esp\left[\left(\frac{E\left(t\right)}{t}\right)^{r}\right]<\infty$.

\item[b)] Las variables aleatorias
$\left\{\left(\frac{E\left(t\right)}{t}\right)^{r}:t\geq1\right\}$
son uniformemente integrables.
\end{itemize}
\end{Lema}

\begin{Teo}[Teorema 5.1: Ley Fuerte para Procesos de Conteo, Gut
\cite{Gut}]\label{Tma.5.1.Gut} Sea
$0<\mu<\esp\left(X_{1}\right]\leq\infty$. entonces

\begin{itemize}
\item[a)] $\frac{N\left(t\right)}{t}\rightarrow\frac{1}{\mu}$
a.s., cuando $t\rightarrow\infty$.


\item[b)]$\esp\left[\frac{N\left(t\right)}{t}\right]^{r}\rightarrow\frac{1}{\mu^{r}}$,
cuando $t\rightarrow\infty$ para todo $r>0$.
\end{itemize}
\end{Teo}


\begin{Prop}[Proposici\'on 5.1, Dai y Sean \cite{DaiSean}]\label{Prop.5.1}
Suponga que los supuestos (A1) y (A2) se cumplen, adem\'as suponga
que el modelo de flujo es estable. Entonces existe $t_{0}>0$ tal
que
\begin{equation}\label{Eq.Prop.5.1}
lim_{|x|\rightarrow\infty}\frac{1}{|x|^{p+1}}\esp_{x}\left[|X\left(t_{0}|x|\right)|^{p+1}\right]=0.
\end{equation}

\end{Prop}


\begin{Prop}[Proposici\'on 5.3, Dai y Sean \cite{DaiSean}]\label{Prop.5.3.DaiSean}
Sea $X$ proceso de estados para la red de colas, y suponga que se
cumplen los supuestos (A1) y (A2), entonces para alguna constante
positiva $C_{p+1}<\infty$, $\delta>0$ y un conjunto compacto
$C\subset X$.

\begin{equation}\label{Eq.5.4}
\esp_{x}\left[\int_{0}^{\tau_{C}\left(\delta\right)}\left(1+|X\left(t\right)|^{p}\right)dt\right]\leq
C_{p+1}\left(1+|x|^{p+1}\right).
\end{equation}
\end{Prop}

\begin{Prop}[Proposici\'on 5.4, Dai y Sean \cite{DaiSean}]\label{Prop.5.4.DaiSean}
Sea $X$ un proceso de Markov Borel Derecho en $X$, sea
$f:X\leftarrow\rea_{+}$ y defina para alguna $\delta>0$, y un
conjunto cerrado $C\subset X$
\[V\left(x\right):=\esp_{x}\left[\int_{0}^{\tau_{C}\left(\delta\right)}f\left(X\left(t\right)\right)dt\right],\]
para $x\in X$. Si $V$ es finito en todas partes y uniformemente
acotada en $C$, entonces existe $k<\infty$ tal que
\begin{equation}\label{Eq.5.11}
\frac{1}{t}\esp_{x}\left[V\left(x\right)\right]+\frac{1}{t}\int_{0}^{t}\esp_{x}\left[f\left(X\left(s\right)\right)ds\right]\leq\frac{1}{t}V\left(x\right)+k,
\end{equation}
para $x\in X$ y $t>0$.
\end{Prop}


\begin{Teo}[Teorema 5.5, Dai y Sean  \cite{DaiSean}]
Suponga que se cumplen (A1) y (A2), adem\'as suponga que el modelo
de flujo es estable. Entonces existe una constante $k_{p}<\infty$
tal que
\begin{equation}\label{Eq.5.13}
\frac{1}{t}\int_{0}^{t}\esp_{x}\left[|Q\left(s\right)|^{p}\right]ds\leq
k_{p}\left\{\frac{1}{t}|x|^{p+1}+1\right\},
\end{equation}
para $t\geq0$, $x\in X$. En particular para cada condici\'on
inicial
\begin{equation}\label{Eq.5.14}
\limsup_{t\rightarrow\infty}\frac{1}{t}\int_{0}^{t}\esp_{x}\left[|Q\left(s\right)|^{p}\right]ds\leq
k_{p}.
\end{equation}
\end{Teo}

\begin{Teo}[Teorema 6.2 Dai y Sean \cite{DaiSean}]\label{Tma.6.2}
Suponga que se cumplen los supuestos (A1)-(A3) y que el modelo de
flujo es estable, entonces se tiene que
\[\parallel P^{t}\left(x,\cdot\right)-\pi\left(\cdot\right)\parallel_{f_{p}}\rightarrow0,\]
para $t\rightarrow\infty$ y $x\in X$. En particular para cada
condici\'on inicial
\[lim_{t\rightarrow\infty}\esp_{x}\left[\left|Q_{t}\right|^{p}\right]=\esp_{\pi}\left[\left|Q_{0}\right|^{p}\right]<\infty,\]
\end{Teo}

donde

\begin{eqnarray*}
\parallel
P^{t}\left(c,\cdot\right)-\pi\left(\cdot\right)\parallel_{f}=sup_{|g\leq
f|}|\int\pi\left(dy\right)g\left(y\right)-\int
P^{t}\left(x,dy\right)g\left(y\right)|,
\end{eqnarray*}
para $x\in\mathbb{X}$.

\begin{Teo}[Teorema 6.3, Dai y Sean \cite{DaiSean}]\label{Tma.6.3}
Suponga que se cumplen los supuestos (A1)-(A3) y que el modelo de
flujo es estable, entonces con
$f\left(x\right)=f_{1}\left(x\right)$, se tiene que
\[lim_{t\rightarrow\infty}t^{(p-1)}\left|P^{t}\left(c,\cdot\right)-\pi\left(\cdot\right)\right|_{f}=0,\]
para $x\in X$. En particular, para cada condici\'on inicial
\[lim_{t\rightarrow\infty}t^{(p-1)}\left|\esp_{x}\left[Q_{t}\right]-\esp_{\pi}\left[Q_{0}\right]\right|=0.\]
\end{Teo}



\begin{Prop}[Proposici\'on 5.1, Dai y Meyn \cite{DaiSean}]\label{Prop.5.1.DaiSean}
Suponga que los supuestos A1) y A2) son ciertos y que el modelo de
flujo es estable. Entonces existe $t_{0}>0$ tal que
\begin{equation}
lim_{|x|\rightarrow\infty}\frac{1}{|x|^{p+1}}\esp_{x}\left[|X\left(t_{0}|x|\right)|^{p+1}\right]=0.
\end{equation}
\end{Prop}


\begin{Teo}[Teorema 5.5, Dai y Meyn \cite{DaiSean}]\label{Tma.5.5.DaiSean}
Suponga que los supuestos A1) y A2) se cumplen y que el modelo de
flujo es estable. Entonces existe una constante $\kappa_{p}$ tal
que
\begin{equation}
\frac{1}{t}\int_{0}^{t}\esp_{x}\left[|Q\left(s\right)|^{p}\right]ds\leq\kappa_{p}\left\{\frac{1}{t}|x|^{p+1}+1\right\},
\end{equation}
para $t>0$ y $x\in X$. En particular, para cada condici\'on
inicial
\begin{eqnarray*}
\limsup_{t\rightarrow\infty}\frac{1}{t}\int_{0}^{t}\esp_{x}\left[|Q\left(s\right)|^{p}\right]ds\leq\kappa_{p}.
\end{eqnarray*}
\end{Teo}


\begin{Teo}[Teorema 6.4, Dai y Meyn \cite{DaiSean}]\label{Tma.6.4.DaiSean}
Suponga que se cumplen los supuestos A1), A2) y A3) y que el
modelo de flujo es estable. Sea $\nu$ cualquier distribuci\'on de
probabilidad en
$\left(\mathbb{X},\mathcal{B}_{\mathbb{X}}\right)$, y $\pi$ la
distribuci\'on estacionaria de $X$.
\begin{itemize}
\item[i)] Para cualquier $f:X\leftarrow\rea_{+}$
\begin{equation}
\lim_{t\rightarrow\infty}\frac{1}{t}\int_{o}^{t}f\left(X\left(s\right)\right)ds=\pi\left(f\right):=\int
f\left(x\right)\pi\left(dx\right),
\end{equation}
$\prob$-c.s.

\item[ii)] Para cualquier $f:X\leftarrow\rea_{+}$ con
$\pi\left(|f|\right)<\infty$, la ecuaci\'on anterior se cumple.
\end{itemize}
\end{Teo}

\begin{Teo}[Teorema 2.2, Down \cite{Down}]\label{Tma2.2.Down}
Suponga que el fluido modelo es inestable en el sentido de que
para alguna $\epsilon_{0},c_{0}\geq0$,
\begin{equation}\label{Eq.Inestability}
|Q\left(T\right)|\geq\epsilon_{0}T-c_{0}\textrm{,   }T\geq0,
\end{equation}
para cualquier condici\'on inicial $Q\left(0\right)$, con
$|Q\left(0\right)|=1$. Entonces para cualquier $0<q\leq1$, existe
$B<0$ tal que para cualquier $|x|\geq B$,
\begin{equation}
\prob_{x}\left\{\mathbb{X}\rightarrow\infty\right\}\geq q.
\end{equation}
\end{Teo}

\begin{Dem}[Teorema \ref{Tma2.1.Down}] La demostraci\'on de este
teorema se da a continuaci\'on:\\
\begin{itemize}
\item[i)] Utilizando la proposici\'on \ref{Prop.5.3.DaiSean} se
tiene que la proposici\'on \ref{Prop.5.4.DaiSean} es cierta para
$f\left(x\right)=1+|x|^{p}$.

\item[i)] es consecuencia directa del Teorema \ref{Tma.6.2}.

\item[iii)] ver la demostraci\'on dada en Dai y Sean
\cite{DaiSean} p\'aginas 1901-1902.

\item[iv)] ver Dai y Sean \cite{DaiSean} p\'aginas 1902-1903 \'o
\cite{MeynTweedie2}.
\end{itemize}
\end{Dem}

%_____________________________________________________________________
\subsubsection{Modelo de Flujo y Estabilidad}
%_____________________________________________________________________

Para cada $k$ y cada $n$ se define

\numberwithin{equation}{section}
\begin{equation}
\Phi^{k}\left(n\right):=\sum_{i=1}^{n}\phi^{k}\left(i\right).
\end{equation}

suponiendo que el estado inicial de la red es
$x=\left(q,a,b\right)\in X$, entonces para cada $k$

\begin{eqnarray}
E_{k}^{x}\left(t\right):=\max\left\{n\geq0:A_{k}^{x}\left(0\right)+\psi_{k}\left(1\right)+\cdots+\psi_{k}\left(n-1\right)\leq t\right\}\\
S_{k}^{x}\left(t\right):=\max\left\{n\geq0:B_{k}^{x}\left(0\right)+\eta_{k}\left(1\right)+\cdots+\eta_{k}\left(n-1\right)\leq
t\right\}
\end{eqnarray}

Sea $T_{k}^{x}\left(t\right)$ el tiempo acumulado que el servidor
$s\left(k\right)$ ha utilizado en los usuarios de la clase $k$ en
el intervalo $\left[0,t\right]$. Entonces se tienen las siguientes
ecuaciones:

\begin{equation}
Q_{k}^{x}\left(t\right)=Q_{k}^{x}\left(0\right)+E_{k}^{x}\left(t\right)+\sum_{l=1}^{k}\Phi_{k}^{l}S_{l}^{x}\left(T_{l}^{x}\right)-S_{k}^{x}\left(T_{k}^{x}\right)\\
\end{equation}
\begin{equation}
Q^{x}\left(t\right)=\left(Q^{x}_{1}\left(t\right),\ldots,Q^{x}_{K}\left(t\right)\right)^{'}\geq0,\\
\end{equation}
\begin{equation}
T^{x}\left(t\right)=\left(T^{x}_{1}\left(t\right),\ldots,T^{x}_{K}\left(t\right)\right)^{'}\geq0,\textrm{ es no decreciente}\\
\end{equation}
\begin{equation}
I_{i}^{x}\left(t\right)=t-\sum_{k\in C_{i}}T_{k}^{x}\left(t\right)\textrm{ es no decreciente}\\
\end{equation}
\begin{equation}
\int_{0}^{\infty}\sum_{k\in C_{i}}Q_{k}^{x}\left(t\right)dI_{i}^{x}\left(t\right)=0\\
\end{equation}
\begin{equation}
\textrm{condiciones adicionales sobre
}\left(Q^{x}\left(\cdot\right),T^{x}\left(\cdot\right)\right)\textrm{
referentes a la disciplina de servicio}
\end{equation}

Para reducir la fluctuaci\'on del modelo se escala tanto el
espacio como el tiempo, entonces se tiene el proceso:

\begin{equation}
\overline{Q}^{x}\left(t\right)=\frac{1}{|x|}Q^{x}\left(|x|t\right)
\end{equation}
Cualquier l\'imite $\overline{Q}\left(t\right)$ es llamado un
flujo l\'imite del proceso longitud de la cola. Si se hace
$|q|\rightarrow\infty$ y se mantienen las componentes restantes
fijas, de la condici\'on inicial $x$, cualquier punto l\'imite del
proceso normalizado $\overline{Q}^{x}$ es una soluci\'on del
siguiente modelo de flujo, ver \cite{Dai}.

\begin{Def}
Un flujo l\'imite (retrasado) para una red bajo una disciplina de
servicio espec\'ifica se define como cualquier soluci\'on
 $\left(Q^{x}\left(\cdot\right),T^{x}\left(\cdot\right)\right)$ de las siguientes ecuaciones, donde
$\overline{Q}\left(t\right)=\left(\overline{Q}_{1}\left(t\right),\ldots,\overline{Q}_{K}\left(t\right)\right)^{'}$
y
$\overline{T}\left(t\right)=\left(\overline{T}_{1}\left(t\right),\ldots,\overline{T}_{K}\left(t\right)\right)^{'}$
\begin{equation}\label{Eq.3.8}
\overline{Q}_{k}\left(t\right)=\overline{Q}_{k}\left(0\right)+\alpha_{k}t-\mu_{k}\overline{T}_{k}\left(t\right)+\sum_{l=1}^{k}P_{lk}\mu_{l}\overline{T}_{l}\left(t\right)\\
\end{equation}
\begin{equation}\label{Eq.3.9}
\overline{Q}_{k}\left(t\right)\geq0\textrm{ para }k=1,2,\ldots,K,\\
\end{equation}
\begin{equation}\label{Eq.3.10}
\overline{T}_{k}\left(0\right)=0,\textrm{ y }\overline{T}_{k}\left(\cdot\right)\textrm{ es no decreciente},\\
\end{equation}
\begin{equation}\label{Eq.3.11}
\overline{I}_{i}\left(t\right)=t-\sum_{k\in C_{i}}\overline{T}_{k}\left(t\right)\textrm{ es no decreciente}\\
\end{equation}
\begin{equation}\label{Eq.3.12}
\overline{I}_{i}\left(\cdot\right)\textrm{ se incrementa al tiempo}t\textrm{ cuando }\sum_{k\in C_{i}}Q_{k}^{x}\left(t\right)dI_{i}^{x}\left(t\right)=0\\
\end{equation}
\begin{equation}\label{Eq.3.13}
\textrm{condiciones adicionales sobre
}\left(Q^{x}\left(\cdot\right),T^{x}\left(\cdot\right)\right)\textrm{
referentes a la disciplina de servicio}
\end{equation}
\end{Def}

Al conjunto de ecuaciones dadas en \ref{Eq.3.8}-\ref{Eq.3.13} se
le llama {\em Modelo de flujo} y al conjunto de todas las
soluciones del modelo de flujo
$\left(\overline{Q}\left(\cdot\right),\overline{T}
\left(\cdot\right)\right)$ se le denotar\'a por $\mathcal{Q}$.

Si se hace $|x|\rightarrow\infty$ sin restringir ninguna de las
componentes, tambi\'en se obtienen un modelo de flujo, pero en
este caso el residual de los procesos de arribo y servicio
introducen un retraso:

\begin{Def}
El modelo de flujo retrasado de una disciplina de servicio en una
red con retraso
$\left(\overline{A}\left(0\right),\overline{B}\left(0\right)\right)\in\rea_{+}^{K+|A|}$
se define como el conjunto de ecuaciones dadas en
\ref{Eq.3.8}-\ref{Eq.3.13}, junto con la condici\'on:
\begin{equation}\label{CondAd.FluidModel}
\overline{Q}\left(t\right)=\overline{Q}\left(0\right)+\left(\alpha
t-\overline{A}\left(0\right)\right)^{+}-\left(I-P^{'}\right)M\left(\overline{T}\left(t\right)-\overline{B}\left(0\right)\right)^{+}
\end{equation}
\end{Def}

\begin{Def}
El modelo de flujo es estable si existe un tiempo fijo $t_{0}$ tal
que $\overline{Q}\left(t\right)=0$, con $t\geq t_{0}$, para
cualquier $\overline{Q}\left(\cdot\right)\in\mathcal{Q}$ que
cumple con $|\overline{Q}\left(0\right)|=1$.
\end{Def}

El siguiente resultado se encuentra en \cite{Chen}.
\begin{Lemma}
Si el modelo de flujo definido por \ref{Eq.3.8}-\ref{Eq.3.13} es
estable, entonces el modelo de flujo retrasado es tambi\'en
estable, es decir, existe $t_{0}>0$ tal que
$\overline{Q}\left(t\right)=0$ para cualquier $t\geq t_{0}$, para
cualquier soluci\'on del modelo de flujo retrasado cuya
condici\'on inicial $\overline{x}$ satisface que
$|overline{x}|=|\overline{Q}\left(0\right)|+|\overline{A}\left(0\right)|+|\overline{B}\left(0\right)|\leq1$.
\end{Lemma}

%_____________________________________________________________________
\subsubsection{Resultados principales}
%_____________________________________________________________________
Supuestos necesarios sobre la red

\begin{Sup}
\begin{itemize}
\item[A1)] $\psi_{1},\ldots,\psi_{K},\eta_{1},\ldots,\eta_{K}$ son
mutuamente independientes y son sucesiones independientes e
id\'enticamente distribuidas.

\item[A2)] Para alg\'un entero $p\geq1$
\begin{eqnarray*}
\esp\left[\psi_{l}\left(1\right)^{p+1}\right]<\infty\textrm{ para }l\in\mathcal{A}\textrm{ y }\\
\esp\left[\eta_{k}\left(1\right)^{p+1}\right]<\infty\textrm{ para
}k=1,\ldots,K.
\end{eqnarray*}
\item[A3)] El conjunto $\left\{x\in X:|x|=0\right\}$ es un
singleton, y para cada $k\in\mathcal{A}$, existe una funci\'on
positiva $q_{k}\left(x\right)$ definida en $\rea_{+}$, y un entero
$j_{k}$, tal que
\begin{eqnarray}
P\left(\psi_{k}\left(1\right)\geq x\right)>0\textrm{, para todo }x>0\\
P\left(\psi_{k}\left(1\right)+\ldots\psi_{k}\left(j_{k}\right)\in dx\right)\geq q_{k}\left(x\right)dx0\textrm{ y }\\
\int_{0}^{\infty}q_{k}\left(x\right)dx>0
\end{eqnarray}
\end{itemize}
\end{Sup}

El argumento dado en \cite{MaynDown} en el lema
\ref{Lema.34.MeynDown} se puede aplicar para deducir que todos los
subconjuntos compactos de $X$ son peque\~nos.Entonces la
condici\'on $A3)$ se puede generalizar a
\begin{itemize}
\item[A3')] Para el proceso de Markov $X$, cada subconjunto
compacto de $X$ es peque\~no.
\end{itemize}

\begin{Teo}\label{Tma.4.1}
Suponga que el modelo de flujo para una disciplina de servicio es
estable, y suponga adem\'as que las condiciones A1) y A2) se
satisfacen. Entonces:
\begin{itemize}
\item[i)] Para alguna constante $\kappa_{p}$, y para cada
condici\'on inicial $x\in X$
\begin{equation}
\limsup_{t\rightarrow\infty}\frac{1}{t}\int_{0}^{t}\esp_{x}\left[|Q\left(t\right)|^{p}\right]ds\leq\kappa_{p}
\end{equation}
donde $p$ es el entero dado por A2). Suponga adem\'as que A3) o A3')
se cumple, entonces la disciplina de servicio es estable y adem\'as
para cada condici\'on inicial se tiene lo siguiente: \item[ii)] Los
momentos transitorios convergen a sus valores en estado
estacionario:
\begin{equation}
\lim_{t\rightarrow\infty}\esp_{x}\left[Q_{k}\left(t\right)^{r}\right]=\esp_{\pi}\left[Q_{k}\left(0\right)^{r}\right]\leq\kappa_{r}
\end{equation}
para $r=1,\ldots,p$ y $k=1,\ldots,K$. \item[iii)] EL primer
momento converge con raz\'on $t^{p-1}$:
\begin{equation}
\lim_{t\rightarrow\infty}t^{p-1}|\esp_{x}\left[Q\left(t\right)\right]-\esp_{\pi}\left[Q\left(0\right)\right]|=0.
\end{equation}
\item[iv)] Se cumple la Ley Fuerte de los Grandes N\'umeros:
\begin{equation}
\lim_{t\rightarrow\infty}\frac{1}{t}\int_{0}^{t}Q_{k}^{r}\left(s\right)ds=\esp_{\pi}\left[Q_{k}\left(0\right)^{r}\right]
\end{equation}
$\prob$-c.s., para $r=1,\ldots,p$ y $k=1,\ldots,K$.
\end{itemize}
\end{Teo}
\begin{Dem}
La demostraci\'on de este resultado se da aplicando los teoremas
\ref{Tma.5.5}, \ref{Tma.6.2}, \ref{Tma.6.3} y \ref{Tma.6.4}
\end{Dem}

%_____________________________________________________________________
\subsubsection{Definiciones Generales}
%_____________________________________________________________________
Definimos un proceso de estados para la red que depende de la
pol\'itica de servicio utilizada. Bajo cualquier {\em preemptive
buffer priority} disciplina de servicio, el estado
$\mathbb{X}\left(t\right)$ a cualquier tiempo $t$ puede definirse
como
\begin{equation}\label{Eq.Esp.Estados}
\mathbb{X}\left(t\right)=\left(Q_{k}\left(t\right),A_{l}\left(t\right),B_{k}\left(t\right):k=1,2,\ldots,K,l\in\mathcal{A}\right)
\end{equation}
donde $Q_{k}\left(t\right)$ es la longitud de la cola para los
usuarios de la clase $k$, incluyendo aquellos que est\'an siendo
atendidos, $B_{k}\left(t\right)$ son los tiempos de servicio
residuales para los usuarios de la clase $k$ que est\'an en
servicio. Los tiempos de arribo residuales, que son iguales al
tiempo que queda hasta que el pr\'oximo usuario de la clase $k$
llega, se denotan por $A_{k}\left(t\right)$. Tanto
$B_{k}\left(t\right)$ como $A_{k}\left(t\right)$ se suponen
continuos por la derecha.

Sea $\mathbb{X}$ el espacio de estados para el proceso de estados
que por definici\'on es igual  al conjunto de posibles valores
para el estado $\mathbb{X}\left(t\right)$, y sea
$x=\left(q,a,b\right)$ un estado gen\'erico en $\mathbb{X}$, la
componente $q$ determina la posici\'on del usuario en la red,
$|q|$ denota la longitud total de la cola en la red.

Para un estado $x=\left(q,a,b\right)\in\mathbb{X}$ definimos la
{\em norma} de $x$ como $\left\|x\right|=|q|+|a|+|b|$. En
\cite{Dai} se muestra que para una amplia serie de disciplinas de
servicio el proceso $\mathbb{X}$ es un Proceso Fuerte de Markov, y
por tanto se puede asumir que
\[\left(\left(\Omega,\mathcal{F}\right),\mathcal{F}_{t},\mathbb{X}\left(t\right),\theta_{t},P_{x}\right)\]
es un proceso de Borel Derecho en el espacio de estadio medible
$\left(\mathbb{X},\mathcal{B}_{\mathbb{X}}\right)$. El Proceso
$X=\left\{\mathbb{X}\left(t\right),t\geq0\right\}$ tiene
trayectorias continuas por la derecha, est definida en
$\left(\Omega,\mathcal{F}\right)$ y est adaptado a
$\left\{\mathcal{F}_{t},t\geq0\right\}$; $\left\{P_{x},x\in
X\right\}$ son medidas de probabilidad en
$\left(\Omega,\mathcal{F}\right)$ tales que para todo $x\in X$
\[P_{x}\left\{\mathbb{X}\left(0\right)=x\right\}=1\] y
\[E_{x}\left\{f\left(X\circ\theta_{t}\right)|\mathcal{F}_{t}\right\}=E_{X}\left(\tau\right)f\left(X\right)\]
en $\left\{\tau<\infty\right\}$, $P_{x}$-c.s. Donde $\tau$ es un
$\mathcal{F}_{t}$-tiempo de paro
\[\left(X\circ\theta_{\tau}\right)\left(w\right)=\left\{\mathbb{X}\left(\tau\left(w\right)+t,w\right),t\geq0\right\}\]
y $f$ es una funci\'on de valores reales acotada y medible con la
sigma algebra de Kolmogorov generada por los cilindros.

Sea $P^{t}\left(x,D\right)$, $D\in\mathcal{B}_{\mathbb{X}}$,
$t\geq0$ probabilidad de transici\'on de $X$ definida como
\[P^{t}\left(x,D\right)=P_{x}\left(\mathbb{X}\left(t\right)\in
D\right)\]

\begin{Def}
Una medida no cero $\pi$ en
$\left(\mathbb{X},\mathcal{B}_{\mathbb{X}}\right)$ es {\em
invariante} para $X$ si $\pi$ es $\sigma$-finita y
\[\pi\left(D\right)=\int_{X}P^{t}\left(x,D\right)\pi\left(dx\right)\]
para todo $D\in \mathcal{B}_{\mathbb{X}}$, con $t\geq0$.
\end{Def}

\begin{Def}
El proceso de Markov $X$ es llamado {\em Harris recurrente} si
existe una medida de probabilidad $\nu$ en
$\left(\mathbb{X},\mathcal{B}_{\mathbb{X}}\right)$, tal que si
$\nu\left(D\right)>0$ y $D\in\mathcal{B}_{\mathbb{X}}$
\[P_{x}\left\{\tau_{D}<\infty\right\}\equiv1\] cuando
$\tau_{D}=\inf\left\{t\geq0:\mathbb{X}_{t}\in D\right\}$.
\end{Def}

\begin{itemize}
\item Si $X$ es Harris recurrente, entonces una \'unica medida
invariante $\pi$ existe (\cite{Getoor}). \item Si la medida
invariante es finita, entonces puede normalizarse a una medida de
probabilidad, en este caso se le llama {\em Harris recurrente
positiva}. \item Cuando $X$ es Harris recurrente positivo se dice
que la disciplina de servicio es estable. En este caso $\pi$
denota la ditribuci\'on estacionaria y hacemos
\[P_{\pi}\left(\cdot\right)[=\int_{X}P_{x}\left(\cdot\right)\pi\left(dx\right)\]
y se utiliza $E_{\pi}$ para denotar el operador esperanza
correspondiente, as, el proceso
$X=\left\{\mathbb{X}\left(t\right),t\geq0\right\}$ es un proceso
estrictamente estacionario bajo $P_{\pi}$
\end{itemize}

\begin{Def}
Un conjunto $D\in\mathcal{B}_\mathbb{X}$ es llamado peque\~no si
existe un $t>0$, una medida de probabilidad $\nu$ en
$\mathcal{B}_\mathbb{X}$, y un $\delta>0$ tal que
\[P^{t}\left(x,A\right)\geq\delta\nu\left(A\right)\] para $x\in
D,A\in\mathcal{B}_\mathbb{X}$.\footnote{En \cite{MeynTweedie}
muestran que si $P_{x}\left\{\tau_{D}<\infty\right\}\equiv1$
solamente para uno conjunto peque\~no, entonces el proceso es
Harris recurrente}
\end{Def}

%_____________________________________________________________________
\subsubsection{Definiciones y Descripci\'on del Modelo}
%________________________________________________________________________
El modelo est\'a compuesto por $c$ colas de capacidad infinita,
etiquetadas de $1$ a $c$ las cuales son atendidas por $s$
servidores. Los servidores atienden de acuerdo a una cadena de
Markov independiente $\left(X^{i}_{n}\right)_{n}$ con $1\leq i\leq
s$ y $n\in\left\{1,2,\ldots,c\right\}$ con la misma matriz de
transici\'on $r_{k,l}$ y \'unica medida invariante
$\left(p_{k}\right)$. Cada servidor permanece atendiendo en la
cola un periodo llamado de visita y determinada por la pol\'itica de
servicio asignada a la cola.

Los usuarios llegan a la cola $k$ con una tasa $\lambda_{k}$ y son
atendidos a una raz\'on $\mu_{k}$. Las sucesiones de tiempos de
interarribo $\left(\tau_{k}\left(n\right)\right)_{n}$, la de
tiempos de servicio
$\left(\sigma_{k}^{i}\left(n\right)\right)_{n}$ y la de tiempos de
cambio $\left(\sigma_{k,l}^{0,i}\left(n\right)\right)_{n}$
requeridas en la cola $k$ para el servidor $i$ son sucesiones
independientes e id\'enticamente distribuidas con distribuci\'on
general independiente de $i$, con media
$\sigma_{k}=\frac{1}{\mu_{k}}$, respectivamente
$\sigma_{k,l}^{0}=\frac{1}{\mu_{k,l}^{0}}$, e independiente de las
cadenas de Markov $\left(X^{i}_{n}\right)_{n}$. Adem\'as se supone
que los tiempos de interarribo se asume son acotados, para cada
$\rho_{k}=\lambda_{k}\sigma_{k}<s$ para asegurar la estabilidad de
la cola $k$ cuando opera como una cola $M/GM/1$.
%________________________________________________________________________
\subsubsection{Pol\'iticas de Servicio}
%_____________________________________________________________________
Una pol\'itica de servicio determina el n\'umero de usuarios que ser\'an
atendidos sin interrupci\'on en periodo de servicio por los
servidores que atienden a la cola. Para un solo servidor esta se
define a trav\'es de una funci\'on $f$ donde $f\left(x,a\right)$ es el
n\'umero de usuarios que son atendidos sin interrupci\'on cuando el
servidor llega a la cola y encuentra $x$ usuarios esperando dado
el tiempo transcurrido de interarribo $a$. Sea $v\left(x,a\right)$
la duraci\'on del periodo de servicio para una sola condici\'on
inicial $\left(x,a\right)$.

Las pol\'iticas de servicio consideradas satisfacen las siguientes
propiedades:

\begin{itemize}
\item[i)] Hay conservaci\'on del trabajo, es decir
\[v\left(x,a\right)=\sum_{l=1}^{f\left(x,a\right)}\sigma\left(l\right)\]
con $f\left(0,a\right)=v\left(0,a\right)=0$, donde
$\left(\sigma\left(l\right)\right)_{l}$ es una sucesi\'on
independiente e id\'enticamente distribuida de los tiempos de
servicio solicitados. \item[ii)] La selecci\'on de usuarios para se
atendidos es independiente de sus correspondientes tiempos de
servicio y del pasado hasta el inicio del periodo de servicio. As\'i
las distribuci\'on $\left(f,v\right)$ no depende del orden en el
cu\'al son atendidos los usuarios. \item[iii)] La pol\'itica de
servicio es mon\'otona en el sentido de que para cada $a\geq0$ los
n\'umeros $f\left(x,a\right)$ son mon\'otonos en distribuci\'on en $x$ y
su l\'imite en distribuci\'on cuando $x\rightarrow\infty$ es una
variable aleatoria $F^{*0}$ que no depende de $a$. \item[iv)] El
n\'umero de usuarios atendidos por cada servidor es acotado por
$f^{min}\left(x\right)$ de la longitud de la cola $x$ que adem\'as
converge mon\'otonamente en distribuci\'on a $F^{*}$ cuando
$x\rightarrow\infty$
\end{itemize}
%________________________________________________________________________
\subsubsection{Proceso de Estados}
%_____________________________________________________________________
El sistema de colas se describe por medio del proceso de Markov
$\left(X\left(t\right)\right)_{t\in\rea}$ como se define a
continuaci\'on. El estado del sistema al tiempo $t\geq0$ est\'a dado
por
\[X\left(t\right)=\left(Q\left(t\right),P\left(t\right),A\left(t\right),R\left(t\right),C\left(t\right)\right)\]
donde
\begin{itemize}
\item
$Q\left(t\right)=\left(Q_{k}\left(t\right)\right)_{k=1}^{c}$,
n\'umero de usuarios en la cola $k$ al tiempo $t$. \item
$P\left(t\right)=\left(P^{i}\left(t\right)\right)_{i=1}^{s}$, es
la posici\'on del servidor $i$. \item
$A\left(t\right)=\left(A_{k}\left(t\right)\right)_{k=1}^{c}$, es
el residual del tiempo de arribo en la cola $k$ al tiempo $t$.
\item
$R\left(t\right)=\left(R_{k}^{i}\left(t\right),R_{k,l}^{0,i}\left(t\right)\right)_{k,l,i=1}^{c,c,s}$,
el primero es el residual del tiempo de servicio del usuario
atendido por servidor $i$ en la cola $k$ al tiempo $t$, la segunda
componente es el residual del tiempo de cambio del servidor $i$ de
la cola $k$ a la cola $l$ al tiempo $t$. \item
$C\left(t\right)=\left(C_{k}^{i}\left(t\right)\right)_{k,i=1}^{c,s}$,
es la componente correspondiente a la cola $k$ y al servidor $i$
que est\'a determinada por la pol\'itica de servicio en la cola $k$
y que hace al proceso $X\left(t\right)$ un proceso de Markov.
\end{itemize}
Todos los procesos definidos arriba se suponen continuos por la
derecha.

El proceso $X$ tiene la propiedad fuerte de Markov y su espacio de
estados es el espacio producto
\[\mathcal{X}=\nat^{c}\times E^{s}\times \rea_{+}^{c}\times\rea_{+}^{cs}\times\rea_{+}^{c^{2}s}\times \mathcal{C}\] donde $E=\left\{1,2,\ldots,c\right\}^{2}\cup\left\{1,2,\ldots,c\right\}$ y $\mathcal{C}$  depende de las pol\'iticas de servicio.

%_____________________________________________________________________________________
\subsubsection{Introducci{\'o}n}
%_____________________________________________________________________________________
%


Si $x$ es el n{\'u}mero de usuarios en la cola al comienzo del
periodo de servicio y $N_{s}\left(x\right)=N\left(x\right)$ es el
n{\'u}mero de usuarios que son atendidos con la pol{\'\i}tica $s$,
{\'u}nica en nuestro caso durante un periodo de servicio, entonces
se asume que:
\begin{enumerate}
\item
\begin{equation}\label{S1}
lim_{x\rightarrow\infty}\esp\left[N\left(x\right)\right]=\overline{N}>0
\end{equation}
\item
\begin{equation}\label{S2}
\esp\left[N\left(x\right)\right]\leq \overline{N} \end{equation}
para cualquier valor de $x$.
\end{enumerate}
La manera en que atiende el servidor $m$-{\'e}simo, en este caso
en espec{\'\i}fico solo lo ilustraremos con un s{\'o}lo servidor,
es la siguiente:
\begin{itemize}
\item Al t{\'e}rmino de la visita a la cola $j$, el servidor se
cambia a la cola $j^{'}$ con probabilidad
$r_{j,j^{'}}^{m}=r_{j,j^{'}}$

\item La $n$-{\'e}sima ocurencia va acompa{\~n}ada con el tiempo
de cambio de longitud $\delta_{j,j^{'}}\left(n\right)$,
independientes e id{\'e}nticamente distribuidas, con
$\esp\left[\delta_{j,j^{'}}\left(1\right)\right]\geq0$.

\item Sea $\left\{p_{j}\right\}$ la {\'u}nica distribuci{\'o}n
invariante estacionaria para la Cadena de Markov con matriz de
transici{\'o}n $\left(r_{j,j^{'}}\right)$.

\item Finalmente, se define
\begin{equation}
\delta^{*}:=\sum_{j,j^{'}}p_{j}r_{j,j^{'}}\esp\left[\delta_{j,j^{'}}\left(1\right)\right].
\end{equation}
\end{itemize}
%_____________________________________________________________________
\subsubsection{Colas C\'iclicas}
%_____________________________________________________________________
El {\em token passing ring} es una estaci\'on de un solo servidor
con $K$ clases de usuarios. Cada clase tiene su propio regulador
en la estaci\'on. Los usuarios llegan al regulador con raz\'on
$\alpha_{k}$ y son atendidos con taza $\mu_{k}$.

La red se puede modelar como un Proceso de Markov con espacio de
estados continuo, continuo en el tiempo:
\begin{equation}
 X\left(t\right)^{T}=\left(Q_{k}\left(t\right),A_{l}\left(t\right),B_{k}\left(t\right),B_{k}^{0}\left(t\right),C\left(t\right):k=1,\ldots,K,l\in\mathcal{A}\right)
\end{equation}
donde $Q_{k}\left(t\right), B_{k}\left(t\right)$ y
$A_{k}\left(t\right)$ se define como en \ref{Eq.Esp.Estados},
$B_{k}^{0}\left(t\right)$ es el tiempo residual de cambio de la
clase $k$ a la clase $k+1\left(mod K\right)$; $C\left(t\right)$
indica el n\'umero de servicios que han sido comenzados y/o
completados durante la sesi\'on activa del buffer.

Los par\'ametros cruciales son la carga nominal de la cola $k$:
$\beta_{k}=\alpha_{k}/\mu_{k}$ y la carga total es
$\rho_{0}=\sum\beta_{k}$, la media total del tiempo de cambio en
un ciclo del token est\'a definido por
\begin{equation}
 u^{0}=\sum_{k=1}^{K}\esp\left[\eta_{k}^{0}\left(1\right)\right]=\sum_{k=1}^{K}\frac{1}{\mu_{k}^{0}}
\end{equation}

El proceso de la longitud de la cola $Q_{k}^{x}\left(t\right)$ y
el proceso de acumulaci\'on del tiempo de servicio
$T_{k}^{x}\left(t\right)$ para el buffer $k$ y para el estado
inicial $x$ se definen como antes. Sea $T_{k}^{x,0}\left(t\right)$
el tiempo acumulado al tiempo $t$ que el token tarda en cambiar
del buffer $k$ al $k+1\mod K$. Suponga que la funci\'on
$\left(\overline{Q}\left(\cdot\right),\overline{T}\left(\cdot\right),\overline{T}^{0}\left(\cdot\right)\right)$
es un punto l\'imite de
\begin{equation}\label{Eq.4.4}
\left(\frac{1}{|x|}Q^{x}\left(|x|t\right),\frac{1}{|x|}T^{x}\left(|x|t\right),\frac{1}{|x|}T^{x,0}\left(|x|t\right)\right)
\end{equation}
cuando $|x|\rightarrow\infty$. Entonces
$\left(\overline{Q}\left(t\right),\overline{T}\left(t\right),\overline{T}^{0}\left(t\right)\right)$
es un flujo l\'imite retrasado del token ring.

Propiedades importantes para el modelo de flujo retrasado

\begin{Prop}
 Sea $\left(\overline{Q},\overline{T},\overline{T}^{0}\right)$ un flujo l\'imite de \ref{Eq.4.4} y suponga que cuando $x\rightarrow\infty$ a lo largo de
una subsucesi\'on
\[\left(\frac{1}{|x|}Q_{k}^{x}\left(0\right),\frac{1}{|x|}A_{k}^{x}\left(0\right),\frac{1}{|x|}B_{k}^{x}\left(0\right),\frac{1}{|x|}B_{k}^{x,0}\left(0\right)\right)\rightarrow\left(\overline{Q}_{k}\left(0\right),0,0,0\right)\]
para $k=1,\ldots,K$. EL flujo l\'imite tiene las siguientes
propiedades, donde las propiedades de la derivada se cumplen donde
la derivada exista:
\begin{itemize}
 \item[i)] Los vectores de tiempo ocupado $\overline{T}\left(t\right)$ y $\overline{T}^{0}\left(t\right)$ son crecientes y continuas con
$\overline{T}\left(0\right)=\overline{T}^{0}\left(0\right)=0$.
\item[ii)] Para todo $t\geq0$
\[\sum_{k=1}^{K}\left[\overline{T}_{k}\left(t\right)+\overline{T}_{k}^{0}\left(t\right)\right]=t\]
\item[iii)] Para todo $1\leq k\leq K$
\[\overline{Q}_{k}\left(t\right)=\overline{Q}_{k}\left(0\right)+\alpha_{k}t-\mu_{k}\overline{T}_{k}\left(t\right)\]
\item[iv)]  Para todo $1\leq k\leq K$
\[\dot{{\overline{T}}}_{k}\left(t\right)=\beta_{k}\] para $\overline{Q}_{k}\left(t\right)=0$.
\item[v)] Para todo $k,j$
\[\mu_{k}^{0}\overline{T}_{k}^{0}\left(t\right)=\mu_{j}^{0}\overline{T}_{j}^{0}\left(t\right)\]
\item[vi)]  Para todo $1\leq k\leq K$
\[\mu_{k}\dot{{\overline{T}}}_{k}\left(t\right)=l_{k}\mu_{k}^{0}\dot{{\overline{T}}}_{k}^{0}\left(t\right)\] para $\overline{Q}_{k}\left(t\right)>0$.
\end{itemize}
\end{Prop}

%_____________________________________________________________________
\subsubsection{Resultados Previos}
%_____________________________________________________________________

\begin{Lemma}\label{Lema.34.MeynDown}
El proceso estoc\'astico $\Phi$ es un proceso de markov fuerte,
temporalmente homog\'eneo, con trayectorias muestrales continuas
por la derecha, cuyo espacio de estados $Y$ es igual a
$X\times\rea$
\end{Lemma}
\begin{Prop}
 Suponga que los supuestos A1) y A2) son ciertos y que el modelo de flujo es estable. Entonces existe $t_{0}>0$ tal que
\begin{equation}
 lim_{|x|\rightarrow\infty}\frac{1}{|x|^{p+1}}\esp_{x}\left[|X\left(t_{0}|x|\right)|^{p+1}\right]=0
\end{equation}
\end{Prop}

\begin{Lemma}\label{Lema.5.2}
 Sea $\left\{\zeta\left(k\right):k\in \mathbb{z}\right\}$ una sucesi\'on independiente e id\'enticamente distribuida que toma valores en $\left(0,\infty\right)$,
y sea
$E\left(t\right)=max\left(n\geq1:\zeta\left(1\right)+\cdots+\zeta\left(n-1\right)\leq
t\right)$. Si $\esp\left[\zeta\left(1\right)\right]<\infty$,
entonces para cualquier entero $r\geq1$
\begin{equation}
 lim_{t\rightarrow\infty}\esp\left[\left(\frac{E\left(t\right)}{t}\right)^{r}\right]=\left(\frac{1}{\esp\left[\zeta_{1}\right]}\right)^{r}.
\end{equation}
Luego, bajo estas condiciones:
\begin{itemize}
 \item[a)] para cualquier $\delta>0$, $\sup_{t\geq\delta}\esp\left[\left(\frac{E\left(t\right)}{t}\right)^{r}\right]<\infty$
\item[b)] las variables aleatorias
$\left\{\left(\frac{E\left(t\right)}{t}\right)^{r}:t\geq1\right\}$
son uniformemente integrables.
\end{itemize}
\end{Lemma}

\begin{Teo}\label{Tma.5.5}
Suponga que los supuestos A1) y A2) se cumplen y que el modelo de
flujo es estable. Entonces existe una constante $\kappa_{p}$ tal
que
\begin{equation}
\frac{1}{t}\int_{0}^{t}\esp_{x}\left[|Q\left(s\right)|^{p}\right]ds\leq\kappa_{p}\left\{\frac{1}{t}|x|^{p+1}+1\right\}
\end{equation}
para $t>0$ y $x\in X$. En particular, para cada condici\'on inicial
\begin{eqnarray*}
\limsup_{t\rightarrow\infty}\frac{1}{t}\int_{0}^{t}\esp_{x}\left[|Q\left(s\right)|^{p}\right]ds\leq\kappa_{p}.
\end{eqnarray*}
\end{Teo}

\begin{Teo}\label{Tma.6.2}
Suponga que se cumplen los supuestos A1), A2) y A3) y que el
modelo de flujo es estable. Entonces se tiene que
\begin{equation}
|\left|P^{t}\left(x,\cdot\right)-\pi\left(\cdot\right)\right||_{f_{p}}\textrm{,
}t\rightarrow\infty,x\in X.
\end{equation}
En particular para cada condici\'on inicial
\begin{eqnarray*}
\lim_{t\rightarrow\infty}\esp_{x}\left[|Q\left(t\right)|^{p}\right]=\esp_{\pi}\left[|Q\left(0\right)|^{p}\right]\leq\kappa_{r}
\end{eqnarray*}
\end{Teo}
\begin{Teo}\label{Tma.6.3}
Suponga que se cumplen los supuestos A1), A2) y A3) y que el
modelo de flujo es estable. Entonces con
$f\left(x\right)=f_{1}\left(x\right)$ se tiene
\begin{equation}
\lim_{t\rightarrow\infty}t^{p-1}|\left|P^{t}\left(x,\cdot\right)-\pi\left(\cdot\right)\right||_{f}=0.
\end{equation}
En particular para cada condici\'on inicial
\begin{eqnarray*}
\lim_{t\rightarrow\infty}t^{p-1}|\esp_{x}\left[Q\left(t\right)\right]-\esp_{\pi}\left[Q\left(0\right)\right]|=0.
\end{eqnarray*}
\end{Teo}

\begin{Teo}\label{Tma.6.4}
Suponga que se cumplen los supuestos A1), A2) y A3) y que el
modelo de flujo es estable. Sea $\nu$ cualquier distribuci\'on de
probabilidad en $\left(X,\mathcal{B}_{X}\right)$, y $\pi$ la
distribuci\'on estacionaria de $X$.
\begin{itemize}
\item[i)] Para cualquier $f:X\leftarrow\rea_{+}$
\begin{equation}
\lim_{t\rightarrow\infty}\frac{1}{t}\int_{o}^{t}f\left(X\left(s\right)\right)ds=\pi\left(f\right):=\int
f\left(x\right)\pi\left(dx\right)
\end{equation}
$\prob$-c.s. \item[ii)] Para cualquier $f:X\leftarrow\rea_{+}$ con
$\pi\left(|f|\right)<\infty$, la ecuaci\'on anterior se cumple.
\end{itemize}
\end{Teo}

%_____________________________________________________________________________________
%
\subsubsection{Teorema de Estabilidad: Descripci{\'o}n}
%_____________________________________________________________________________________
%


Si $x$ es el n{\'u}mero de usuarios en la cola al comienzo del
periodo de servicio y $N_{s}\left(x\right)=N\left(x\right)$ es el
n{\'u}mero de usuarios que son atendidos con la pol{\'\i}tica $s$,
{\'u}nica en nuestro caso durante un periodo de servicio, entonces
se asume que:
\begin{enumerate}
\item
\begin{equation}\label{S1}
lim_{x\rightarrow\infty}\esp\left[N\left(x\right)\right]=\overline{N}>0
\end{equation}
\item
\begin{equation}\label{S2}
\esp\left[N\left(x\right)\right]\leq \overline{N} \end{equation}
para cualquier valor de $x$.
\end{enumerate}
La manera en que atiende el servidor $m$-{\'e}simo, en este caso
en espec{\'\i}fico solo lo ilustraremos con un s{\'o}lo servidor,
es la siguiente:
\begin{itemize}
\item Al t{\'e}rmino de la visita a la cola $j$, el servidor se
cambia a la cola $j^{'}$ con probabilidad
$r_{j,j^{'}}^{m}=r_{j,j^{'}}$

\item La $n$-{\'e}sima ocurencia va acompa{\~n}ada con el tiempo
de cambio de longitud $\delta_{j,j^{'}}\left(n\right)$,
independientes e id{\'e}nticamente distribuidas, con
$\esp\left[\delta_{j,j^{'}}\left(1\right)\right]\geq0$.

\item Sea $\left\{p_{j}\right\}$ la {\'u}nica distribuci{\'o}n
invariante estacionaria para la Cadena de Markov con matriz de
transici{\'o}n $\left(r_{j,j^{'}}\right)$.

\item Finalmente, se define
\begin{equation}
\delta^{*}:=\sum_{j,j^{'}}p_{j}r_{j,j^{'}}\esp\left[\delta_{j,j^{'}}\left(1\right)\right].
\end{equation}
\end{itemize}

%_________________________________________________________________________
\subsection{Supuestos}
%_________________________________________________________________________
Consideremos el caso en el que se tienen varias colas a las cuales
llegan uno o varios servidores para dar servicio a los usuarios
que se encuentran presentes en la cola, como ya se mencion\'o hay
varios tipos de pol\'iticas de servicio, incluso podr\'ia ocurrir
que la manera en que atiende al resto de las colas sea distinta a
como lo hizo en las anteriores.\\

Para ejemplificar los sistemas de visitas c\'iclicas se
considerar\'a el caso en que a las colas los usuarios son atendidos con
una s\'ola pol\'itica de servicio.\\



Si $\omega$ es el n\'umero de usuarios en la cola al comienzo del
periodo de servicio y $N\left(\omega\right)$ es el n\'umero de
usuarios que son atendidos con una pol\'itica en espec\'ifico
durante el periodo de servicio, entonces se asume que:
\begin{itemize}
\item[1)]\label{S1}$lim_{\omega\rightarrow\infty}\esp\left[N\left(\omega\right)\right]=\overline{N}>0$;
\item[2)]\label{S2}$\esp\left[N\left(\omega\right)\right]\leq\overline{N}$
para cualquier valor de $\omega$.
\end{itemize}
La manera en que atiende el servidor $m$-\'esimo, es la siguiente:
\begin{itemize}
\item Al t\'ermino de la visita a la cola $j$, el servidor cambia
a la cola $j^{'}$ con probabilidad $r_{j,j^{'}}^{m}$

\item La $n$-\'esima vez que el servidor cambia de la cola $j$ a
$j'$, va acompa\~nada con el tiempo de cambio de longitud
$\delta_{j,j^{'}}^{m}\left(n\right)$, con
$\delta_{j,j^{'}}^{m}\left(n\right)$, $n\geq1$, variables
aleatorias independientes e id\'enticamente distribuidas, tales
que $\esp\left[\delta_{j,j^{'}}^{m}\left(1\right)\right]\geq0$.

\item Sea $\left\{p_{j}^{m}\right\}$ la distribuci\'on invariante
estacionaria \'unica para la Cadena de Markov con matriz de
transici\'on $\left(r_{j,j^{'}}^{m}\right)$, se supone que \'esta
existe.

\item Finalmente, se define el tiempo promedio total de traslado
entre las colas como
\begin{equation}
\delta^{*}:=\sum_{j,j^{'}}p_{j}^{m}r_{j,j^{'}}^{m}\esp\left[\delta_{j,j^{'}}^{m}\left(i\right)\right].
\end{equation}
\end{itemize}

Consideremos el caso donde los tiempos entre arribo a cada una de
las colas, $\left\{\xi_{k}\left(n\right)\right\}_{n\geq1}$ son
variables aleatorias independientes a id\'enticamente
distribuidas, y los tiempos de servicio en cada una de las colas
se distribuyen de manera independiente e id\'enticamente
distribuidas $\left\{\eta_{k}\left(n\right)\right\}_{n\geq1}$;
adem\'as ambos procesos cumplen la condici\'on de ser
independientes entre s\'i. Para la $k$-\'esima cola se define la
tasa de arribo por
$\lambda_{k}=1/\esp\left[\xi_{k}\left(1\right)\right]$ y la tasa
de servicio como
$\mu_{k}=1/\esp\left[\eta_{k}\left(1\right)\right]$, finalmente se
define la carga de la cola como $\rho_{k}=\lambda_{k}/\mu_{k}$,
donde se pide que $\rho=\sum_{k=1}^{K}\rho_{k}<1$, para garantizar
la estabilidad del sistema, esto es cierto para las pol\'iticas de
servicio exhaustiva y cerrada, ver Geetor \cite{Getoor}.\\

Si denotamos por
\begin{itemize}
\item $Q_{k}\left(t\right)$ el n\'umero de usuarios presentes en
la cola $k$ al tiempo $t$; \item $A_{k}\left(t\right)$ los
residuales de los tiempos entre arribos a la cola $k$; para cada
servidor $m$; \item $B_{m}\left(t\right)$ denota a los residuales
de los tiempos de servicio al tiempo $t$; \item
$B_{m}^{0}\left(t\right)$ los residuales de los tiempos de
traslado de la cola $k$ a la pr\'oxima por atender al tiempo $t$,

\item sea
$C_{m}\left(t\right)$ el n\'umero de usuarios atendidos durante la
visita del servidor a la cola $k$ al tiempo $t$.
\end{itemize}


En este sentido, el proceso para el sistema de visitas se puede
definir como:

\begin{equation}\label{Esp.Edos.Down}
X\left(t\right)^{T}=\left(Q_{k}\left(t\right),A_{k}\left(t\right),B_{m}\left(t\right),B_{m}^{0}\left(t\right),C_{m}\left(t\right)\right),
\end{equation}
para $k=1,\ldots,K$ y $m=1,2,\ldots,M$, donde $T$ indica que es el
transpuesto del vector que se est\'a definiendo. El proceso $X$
evoluciona en el espacio de estados:
$\mathbb{X}=\ent_{+}^{K}\times\rea_{+}^{K}\times\left(\left\{1,2,\ldots,K\right\}\times\left\{1,2,\ldots,S\right\}\right)^{M}\times\rea_{+}^{K}\times\ent_{+}^{K}$.\\

El sistema aqu\'i descrito debe de cumplir con los siguientes supuestos b\'asicos de un sistema de visitas:
%__________________________________________________________________________
\subsubsection{Supuestos B\'asicos}
%__________________________________________________________________________
\begin{itemize}
\item[A1)] Los procesos
$\xi_{1},\ldots,\xi_{K},\eta_{1},\ldots,\eta_{K}$ son mutuamente
independientes y son sucesiones independientes e id\'enticamente
distribuidas.

\item[A2)] Para alg\'un entero $p\geq1$
\begin{eqnarray*}
\esp\left[\xi_{l}\left(1\right)^{p+1}\right]&<&\infty\textrm{ para }l=1,\ldots,K\textrm{ y }\\
\esp\left[\eta_{k}\left(1\right)^{p+1}\right]&<&\infty\textrm{
para }k=1,\ldots,K.
\end{eqnarray*}
donde $\mathcal{A}$ es la clase de posibles arribos.

\item[A3)] Para cada $k=1,2,\ldots,K$ existe una funci\'on
positiva $q_{k}\left(\cdot\right)$ definida en $\rea_{+}$, y un
entero $j_{k}$, tal que
\begin{eqnarray}
P\left(\xi_{k}\left(1\right)\geq x\right)&>&0\textrm{, para todo }x>0,\\
P\left\{a\leq\sum_{i=1}^{j_{k}}\xi_{k}\left(i\right)\leq
b\right\}&\geq&\int_{a}^{b}q_{k}\left(x\right)dx, \textrm{ }0\leq
a<b.
\end{eqnarray}
\end{itemize}

En lo que respecta al supuesto (A3), en Dai y Meyn \cite{DaiSean}
hacen ver que este se puede sustituir por

\begin{itemize}
\item[A3')] Para el Proceso de Markov $X$, cada subconjunto
compacto del espacio de estados de $X$ es un conjunto peque\~no,
ver definici\'on \ref{Def.Cto.Peq.}.
\end{itemize}

Es por esta raz\'on que con la finalidad de poder hacer uso de
$A3^{'})$ es necesario recurrir a los Procesos de Harris y en
particular a los Procesos Harris Recurrente, ver \cite{Dai,
DaiSean}.
%_______________________________________________________________________
\subsection{Procesos Harris Recurrente}
%_______________________________________________________________________

Por el supuesto (A1) conforme a Davis \cite{Davis}, se puede
definir el proceso de saltos correspondiente de manera tal que
satisfaga el supuesto (A3'), de hecho la demostraci\'on est\'a
basada en la l\'inea de argumentaci\'on de Davis, \cite{Davis},
p\'aginas 362-364.\\

Entonces se tiene un espacio de estados en el cual el proceso $X$
satisface la Propiedad Fuerte de Markov, ver Dai y Meyn
\cite{DaiSean}, dado por

\[\left(\Omega,\mathcal{F},\mathcal{F}_{t},X\left(t\right),\theta_{t},P_{x}\right),\]
adem\'as de ser un proceso de Borel Derecho (Sharpe \cite{Sharpe})
en el espacio de estados medible
$\left(\mathbb{X},\mathcal{B}_\mathbb{X}\right)$. El Proceso
$X=\left\{X\left(t\right),t\geq0\right\}$ tiene trayectorias
continuas por la derecha, est\'a definido en
$\left(\Omega,\mathcal{F}\right)$ y est\'a adaptado a
$\left\{\mathcal{F}_{t},t\geq0\right\}$; la colecci\'on
$\left\{P_{x},x\in \mathbb{X}\right\}$ son medidas de probabilidad
en $\left(\Omega,\mathcal{F}\right)$ tales que para todo $x\in
\mathbb{X}$
\[P_{x}\left\{X\left(0\right)=x\right\}=1,\] y
\[E_{x}\left\{f\left(X\circ\theta_{t}\right)|\mathcal{F}_{t}\right\}=E_{X}\left(\tau\right)f\left(X\right),\]
en $\left\{\tau<\infty\right\}$, $P_{x}$-c.s., con $\theta_{t}$
definido como el operador shift.


Donde $\tau$ es un $\mathcal{F}_{t}$-tiempo de paro
\[\left(X\circ\theta_{\tau}\right)\left(w\right)=\left\{X\left(\tau\left(w\right)+t,w\right),t\geq0\right\},\]
y $f$ es una funci\'on de valores reales acotada y medible, ver \cite{Dai, KaspiMandelbaum}.\\

Sea $P^{t}\left(x,D\right)$, $D\in\mathcal{B}_{\mathbb{X}}$,
$t\geq0$ la probabilidad de transici\'on de $X$ queda definida
como:
\[P^{t}\left(x,D\right)=P_{x}\left(X\left(t\right)\in
D\right).\]


\begin{Def}
Una medida no cero $\pi$ en
$\left(\mathbb{X},\mathcal{B}_{\mathbb{X}}\right)$ es invariante
para $X$ si $\pi$ es $\sigma$-finita y
\[\pi\left(D\right)=\int_{\mathbb{X}}P^{t}\left(x,D\right)\pi\left(dx\right),\]
para todo $D\in \mathcal{B}_{\mathbb{X}}$, con $t\geq0$.
\end{Def}

\begin{Def}
El proceso de Markov $X$ es llamado Harris Recurrente si existe
una medida de probabilidad $\nu$ en
$\left(\mathbb{X},\mathcal{B}_{\mathbb{X}}\right)$, tal que si
$\nu\left(D\right)>0$ y $D\in\mathcal{B}_{\mathbb{X}}$
\[P_{x}\left\{\tau_{D}<\infty\right\}\equiv1,\] cuando
$\tau_{D}=inf\left\{t\geq0:X_{t}\in D\right\}$.
\end{Def}

\begin{Note}
\begin{itemize}
\item[i)] Si $X$ es Harris recurrente, entonces existe una \'unica
medida invariante $\pi$ (Getoor \cite{Getoor}).

\item[ii)] Si la medida invariante es finita, entonces puede
normalizarse a una medida de probabilidad, en este caso al proceso
$X$ se le llama Harris recurrente positivo.


\item[iii)] Cuando $X$ es Harris recurrente positivo se dice que
la disciplina de servicio es estable. En este caso $\pi$ denota la
distribuci\'on estacionaria y hacemos
\[P_{\pi}\left(\cdot\right)=\int_{\mathbf{X}}P_{x}\left(\cdot\right)\pi\left(dx\right),\]
y se utiliza $E_{\pi}$ para denotar el operador esperanza
correspondiente, ver \cite{DaiSean}.
\end{itemize}
\end{Note}

\begin{Def}\label{Def.Cto.Peq.}
Un conjunto $D\in\mathcal{B_{\mathbb{X}}}$ es llamado peque\~no si
existe un $t>0$, una medida de probabilidad $\nu$ en
$\mathcal{B_{\mathbb{X}}}$, y un $\delta>0$ tal que
\[P^{t}\left(x,A\right)\geq\delta\nu\left(A\right),\] para $x\in
D,A\in\mathcal{B_{\mathbb{X}}}$.
\end{Def}

La siguiente serie de resultados vienen enunciados y demostrados
en Dai \cite{Dai}:
\begin{Lema}[Lema 3.1, Dai \cite{Dai}]
Sea $B$ conjunto peque\~no cerrado, supongamos que
$P_{x}\left(\tau_{B}<\infty\right)\equiv1$ y que para alg\'un
$\delta>0$ se cumple que
\begin{equation}\label{Eq.3.1}
\sup\esp_{x}\left[\tau_{B}\left(\delta\right)\right]<\infty,
\end{equation}
donde
$\tau_{B}\left(\delta\right)=inf\left\{t\geq\delta:X\left(t\right)\in
B\right\}$. Entonces, $X$ es un proceso Harris recurrente
positivo.
\end{Lema}

\begin{Lema}[Lema 3.1, Dai \cite{Dai}]\label{Lema.3.}
Bajo el supuesto (A3), el conjunto
$B=\left\{x\in\mathbb{X}:|x|\leq k\right\}$ es un conjunto
peque\~no cerrado para cualquier $k>0$.
\end{Lema}

\begin{Teo}[Teorema 3.1, Dai \cite{Dai}]\label{Tma.3.1}
Si existe un $\delta>0$ tal que
\begin{equation}
lim_{|x|\rightarrow\infty}\frac{1}{|x|}\esp|X^{x}\left(|x|\delta\right)|=0,
\end{equation}
donde $X^{x}$ se utiliza para denotar que el proceso $X$ comienza
a partir de $x$, entonces la ecuaci\'on (\ref{Eq.3.1}) se cumple
para $B=\left\{x\in\mathbb{X}:|x|\leq k\right\}$ con alg\'un
$k>0$. En particular, $X$ es Harris recurrente positivo.
\end{Teo}

Entonces, tenemos que el proceso $X$ es un proceso de Markov que
cumple con los supuestos $A1)$-$A3)$, lo que falta de hacer es
construir el Modelo de Flujo bas\'andonos en lo hasta ahora
presentado.
%_______________________________________________________________________
\subsection{Modelo de Flujo}
%_______________________________________________________________________

Dada una condici\'on inicial $x\in\mathbb{X}$, sea

\begin{itemize}
\item $Q_{k}^{x}\left(t\right)$ la longitud de la cola al tiempo
$t$,

\item $T_{m,k}^{x}\left(t\right)$ el tiempo acumulado, al tiempo
$t$, que tarda el servidor $m$ en atender a los usuarios de la
cola $k$.

\item $T_{m,k}^{x,0}\left(t\right)$ el tiempo acumulado, al tiempo
$t$, que tarda el servidor $m$ en trasladarse a otra cola a partir de la $k$-\'esima.\\
\end{itemize}

Sup\'ongase que la funci\'on
$\left(\overline{Q}\left(\cdot\right),\overline{T}_{m}
\left(\cdot\right),\overline{T}_{m}^{0} \left(\cdot\right)\right)$
para $m=1,2,\ldots,M$ es un punto l\'imite de
\begin{equation}\label{Eq.Punto.Limite}
\left(\frac{1}{|x|}Q^{x}\left(|x|t\right),\frac{1}{|x|}T_{m}^{x}\left(|x|t\right),\frac{1}{|x|}T_{m}^{x,0}\left(|x|t\right)\right)
\end{equation}
para $m=1,2,\ldots,M$, cuando $x\rightarrow\infty$, ver
\cite{Down}. Entonces
$\left(\overline{Q}\left(t\right),\overline{T}_{m}
\left(t\right),\overline{T}_{m}^{0} \left(t\right)\right)$ es un
flujo l\'imite del sistema. Al conjunto de todos las posibles
flujos l\'imite se le llama {\emph{Modelo de Flujo}} y se le
denotar\'a por $\mathcal{Q}$, ver \cite{Down, Dai, DaiSean}.\\

El modelo de flujo satisface el siguiente conjunto de ecuaciones:

\begin{equation}\label{Eq.MF.1}
\overline{Q}_{k}\left(t\right)=\overline{Q}_{k}\left(0\right)+\lambda_{k}t-\sum_{m=1}^{M}\mu_{k}\overline{T}_{m,k}\left(t\right),\\
\end{equation}
para $k=1,2,\ldots,K$.\\
\begin{equation}\label{Eq.MF.2}
\overline{Q}_{k}\left(t\right)\geq0\textrm{ para
}k=1,2,\ldots,K.\\
\end{equation}

\begin{equation}\label{Eq.MF.3}
\overline{T}_{m,k}\left(0\right)=0,\textrm{ y }\overline{T}_{m,k}\left(\cdot\right)\textrm{ es no decreciente},\\
\end{equation}
para $k=1,2,\ldots,K$ y $m=1,2,\ldots,M$.\\
\begin{equation}\label{Eq.MF.4}
\sum_{k=1}^{K}\overline{T}_{m,k}^{0}\left(t\right)+\overline{T}_{m,k}\left(t\right)=t\textrm{
para }m=1,2,\ldots,M.\\
\end{equation}


\begin{Def}[Definici\'on 4.1, Dai \cite{Dai}]\label{Def.Modelo.Flujo}
Sea una disciplina de servicio espec\'ifica. Cualquier l\'imite
$\left(\overline{Q}\left(\cdot\right),\overline{T}\left(\cdot\right),\overline{T}^{0}\left(\cdot\right)\right)$
en (\ref{Eq.Punto.Limite}) es un {\em flujo l\'imite} de la
disciplina. Cualquier soluci\'on (\ref{Eq.MF.1})-(\ref{Eq.MF.4})
es llamado flujo soluci\'on de la disciplina.
\end{Def}

\begin{Def}
Se dice que el modelo de flujo l\'imite, modelo de flujo, de la
disciplina de la cola es estable si existe una constante
$\delta>0$ que depende de $\mu,\lambda$ y $P$ solamente, tal que
cualquier flujo l\'imite con
$|\overline{Q}\left(0\right)|+|\overline{U}|+|\overline{V}|=1$, se
tiene que $\overline{Q}\left(\cdot+\delta\right)\equiv0$.
\end{Def}

Si se hace $|x|\rightarrow\infty$ sin restringir ninguna de las
componentes, tambi\'en se obtienen un modelo de flujo, pero en
este caso el residual de los procesos de arribo y servicio
introducen un retraso:
\begin{Teo}[Teorema 4.2, Dai \cite{Dai}]\label{Tma.4.2.Dai}
Sea una disciplina fija para la cola, suponga que se cumplen las
condiciones (A1)-(A3). Si el modelo de flujo l\'imite de la
disciplina de la cola es estable, entonces la cadena de Markov $X$
que describe la din\'amica de la red bajo la disciplina es Harris
recurrente positiva.
\end{Teo}

Ahora se procede a escalar el espacio y el tiempo para reducir la
aparente fluctuaci\'on del modelo. Consid\'erese el proceso
\begin{equation}\label{Eq.3.7}
\overline{Q}^{x}\left(t\right)=\frac{1}{|x|}Q^{x}\left(|x|t\right).
\end{equation}
A este proceso se le conoce como el flujo escalado, y cualquier
l\'imite $\overline{Q}^{x}\left(t\right)$ es llamado flujo
l\'imite del proceso de longitud de la cola. Haciendo
$|q|\rightarrow\infty$ mientras se mantiene el resto de las
componentes fijas, cualquier punto l\'imite del proceso de
longitud de la cola normalizado $\overline{Q}^{x}$ es soluci\'on
del siguiente modelo de flujo.


\begin{Def}[Definici\'on 3.3, Dai y Meyn \cite{DaiSean}]
El modelo de flujo es estable si existe un tiempo fijo $t_{0}$ tal
que $\overline{Q}\left(t\right)=0$, con $t\geq t_{0}$, para
cualquier $\overline{Q}\left(\cdot\right)\in\mathcal{Q}$ que
cumple con $|\overline{Q}\left(0\right)|=1$.
\end{Def}

\begin{Lemma}[Lema 3.1, Dai y Meyn \cite{DaiSean}]
Si el modelo de flujo definido por (\ref{Eq.MF.1})-(\ref{Eq.MF.4})
es estable, entonces el modelo de flujo retrasado es tambi\'en
estable, es decir, existe $t_{0}>0$ tal que
$\overline{Q}\left(t\right)=0$ para cualquier $t\geq t_{0}$, para
cualquier soluci\'on del modelo de flujo retrasado cuya
condici\'on inicial $\overline{x}$ satisface que
$|\overline{x}|=|\overline{Q}\left(0\right)|+|\overline{A}\left(0\right)|+|\overline{B}\left(0\right)|\leq1$.
\end{Lemma}


Ahora ya estamos en condiciones de enunciar los resultados principales:


\begin{Teo}[Teorema 2.1, Down \cite{Down}]\label{Tma2.1.Down}
Suponga que el modelo de flujo es estable, y que se cumplen los supuestos (A1) y (A2), entonces
\begin{itemize}
\item[i)] Para alguna constante $\kappa_{p}$, y para cada
condici\'on inicial $x\in X$
\begin{equation}\label{Estability.Eq1}
\limsup_{t\rightarrow\infty}\frac{1}{t}\int_{0}^{t}\esp_{x}\left[|Q\left(s\right)|^{p}\right]ds\leq\kappa_{p},
\end{equation}
donde $p$ es el entero dado en (A2).
\end{itemize}
Si adem\'as se cumple la condici\'on (A3), entonces para cada
condici\'on inicial:
\begin{itemize}
\item[ii)] Los momentos transitorios convergen a su estado
estacionario:
 \begin{equation}\label{Estability.Eq2}
lim_{t\rightarrow\infty}\esp_{x}\left[Q_{k}\left(t\right)^{r}\right]=\esp_{\pi}\left[Q_{k}\left(0\right)^{r}\right]\leq\kappa_{r},
\end{equation}
para $r=1,2,\ldots,p$ y $k=1,2,\ldots,K$. Donde $\pi$ es la
probabilidad invariante para $X$.

\item[iii)]  El primer momento converge con raz\'on $t^{p-1}$:
\begin{equation}\label{Estability.Eq3}
lim_{t\rightarrow\infty}t^{p-1}|\esp_{x}\left[Q_{k}\left(t\right)\right]-\esp_{\pi}\left[Q_{k}\left(0\right)\right]|=0.
\end{equation}

\item[iv)] La {\em Ley Fuerte de los grandes n\'umeros} se cumple:
\begin{equation}\label{Estability.Eq4}
lim_{t\rightarrow\infty}\frac{1}{t}\int_{0}^{t}Q_{k}^{r}\left(s\right)ds=\esp_{\pi}\left[Q_{k}\left(0\right)^{r}\right],\textrm{
}\prob_{x}\textrm{-c.s.}
\end{equation}
para $r=1,2,\ldots,p$ y $k=1,2,\ldots,K$.
\end{itemize}
\end{Teo}

La contribuci\'on de Down a la teor\'ia de los {\emph {sistemas de
visitas c\'iclicas}}, es la relaci\'on que hay entre la
estabilidad del sistema con el comportamiento de las medidas de
desempe\~no, es decir, la condici\'on suficiente para poder
garantizar la convergencia del proceso de la longitud de la cola
as\'i como de por los menos los dos primeros momentos adem\'as de
una versi\'on de la Ley Fuerte de los Grandes N\'umeros para los
sistemas de visitas.


\begin{Teo}[Teorema 2.3, Down \cite{Down}]\label{Tma2.3.Down}
Considere el siguiente valor:
\begin{equation}\label{Eq.Rho.1serv}
\rho=\sum_{k=1}^{K}\rho_{k}+max_{1\leq j\leq K}\left(\frac{\lambda_{j}}{\sum_{s=1}^{S}p_{js}\overline{N}_{s}}\right)\delta^{*}
\end{equation}
\begin{itemize}
\item[i)] Si $\rho<1$ entonces la red es estable, es decir, se
cumple el Teorema \ref{Tma2.1.Down}.

\item[ii)] Si $\rho>1$ entonces la red es inestable, es decir, se
cumple el Teorema \ref{Tma2.2.Down}
\end{itemize}
\end{Teo}



%_________________________________________________________________________
\subsection{Modelo de Flujo}
%_________________________________________________________________________
Sup\'ongase que el sistema consta de varias colas a los cuales
llegan uno o varios servidores a dar servicio a los usuarios
esperando en la cola.\\


Sea $x$ el n\'umero de usuarios en la cola esperando por servicio
y $N\left(x\right)$ es el n\'umero de usuarios que son atendidos
con una pol\'itica dada y fija mientras el servidor permanece
dando servicio, entonces se asume que:
\begin{itemize}
\item[(S1.)]
\begin{equation}\label{S1}
lim_{x\rightarrow\infty}\esp\left[N\left(x\right)\right]=\overline{N}>0.
\end{equation}
\item[(S2.)]
\begin{equation}\label{S2}
\esp\left[N\left(x\right)\right]\leq \overline{N},
\end{equation}

para cualquier valor de $x$.
\end{itemize}

El tiempo que tarda un servidor en volver a dar servicio despu\'es
de abandonar la cola inmediata anterior y llegar a la pr\'oxima se
llama tiempo de traslado o de cambio  de cola, al momento de la
$n$-\'esima visita del servidor a la cola $j$ se genera una
sucesi\'on de variables aleatorias $\delta_{j,j+1}\left(n\right)$,
independientes e id\'enticamente distribuidas, con la propiedad de
que $\esp\left[\delta_{j,j+1}\left(1\right)\right]\geq0$.\\


Se define
\begin{equation}
\delta^{*}:=\sum_{j,j+1}\esp\left[\delta_{j,j+1}\left(1\right)\right].
\end{equation}
%\begin{figure}[H]
%\centering
%\includegraphics[width=7cm]{switchovertime.jpg}
%\caption{Sistema de Visitas C\'iclicas}
%\end{figure}

Los tiempos entre arribos a la cola $k$, son de la forma
$\left\{\xi_{k}\left(n\right)\right\}_{n\geq1}$, con la propiedad
de que son independientes e id\'enticamente distribuidos. Los
tiempos de servicio
$\left\{\eta_{k}\left(n\right)\right\}_{n\geq1}$ tienen la
propiedad de ser independientes e id\'enticamente distribuidos.
Para la $k$-\'esima cola se define la tasa de arribo a la como
$\lambda_{k}=1/\esp\left[\xi_{k}\left(1\right)\right]$ y la tasa
de servicio como
$\mu_{k}=1/\esp\left[\eta_{k}\left(1\right)\right]$, finalmente se
define la carga de la cola como $\rho_{k}=\lambda_{k}/\mu_{k}$,
donde se pide que $\rho<1$, para garantizar la estabilidad del sistema.\\

%_____________________________________________________________________
%\subsubsection{Proceso de Estados}
%_____________________________________________________________________

Para el caso m\'as sencillo podemos definir un proceso de estados
para la red que depende de la pol\'itica de servicio utilizada, el
estado $\mathbb{X}\left(t\right)$ a cualquier tiempo $t$ puede
definirse como
\begin{equation}\label{Eq.Esp.Estados}
\mathbb{X}\left(t\right)=\left(Q_{k}\left(t\right),A_{l}\left(t\right),B_{k}\left(t\right):k=1,2,\ldots,K,l\in\mathcal{A}\right),
\end{equation}

donde $Q_{k}\left(t\right)$ es la longitud de la cola $k$ para los
usuarios esperando servicio, incluyendo aquellos que est\'an
siendo atendidos, $B_{k}\left(t\right)$ son los tiempos de
servicio residuales para los usuarios de la clase $k$ que est\'an
en servicio.\\

Los tiempos entre arribos residuales, que son el tiempo que queda
hasta que el pr\'oximo usuario llega a la cola para recibir
servicio, se denotan por $A_{k}\left(t\right)$. Tanto
$B_{k}\left(t\right)$ como $A_{k}\left(t\right)$ se suponen
continuos por la derecha \cite{Dai2}.\\

Sea $\mathcal{X}$ el espacio de estados para el proceso de estados
que por definici\'on es igual  al conjunto de posibles valores
para el estado $\mathbb{X}\left(t\right)$, y sea
$x=\left(q,a,b\right)$ un estado gen\'erico en $\mathbb{X}$, la
componente $q$ determina la posici\'on del usuario en la red,
$|q|$ denota la longitud total de la cola en la red.\\

Para un estado $x=\left(q,a,b\right)\in\mathbb{X}$ definimos la
{\em norma} de $x$ como $\left\|x\right\|=|q|+|a|+|b|$. En
\cite{Dai} se muestra que para una amplia serie de disciplinas de
servicio el proceso $\mathbb{X}$ es un Proceso Fuerte de Markov, y
por tanto se puede asumir que
\[\left(\left(\Omega,\mathcal{F}\right),\mathcal{F}_{t},\mathbb{X}\left(t\right),\theta_{t},P_{x}\right)\]
es un proceso de {\em Borel Derecho} en el espacio de estados
medible $\left(\mathcal{X},\mathcal{B}_{\mathcal{X}}\right)$.\\

Sea $P^{t}\left(x,D\right)$, $D\in\mathcal{B}_{\mathbb{X}}$,
$t\geq0$ probabilidad de transici\'on de $X$ definida como
\[P^{t}\left(x,D\right)=P_{x}\left(\mathbb{X}\left(t\right)\in
D\right).\]

\begin{Def}
Una medida no cero $\pi$ en
$\left(\mathbb{X},\mathcal{B}_{\mathbb{X}}\right)$ es {\em
invariante} para $X$ si $\pi$ es $\sigma$-finita y
\[\pi\left(D\right)=\int_{X}P^{t}\left(x,D\right)\pi\left(dx\right),\]
para todo $D\in \mathcal{B}_{\mathbb{X}}$, con $t\geq0$.
\end{Def}

\begin{Def}
El proceso de Markov $X$ es llamado {\em Harris recurrente} si
existe una medida de probabilidad $\nu$ en
$\left(\mathbb{X},\mathcal{B}_{\mathbb{X}}\right)$, tal que si
$\nu\left(D\right)>0$ y $D\in\mathcal{B}_{\mathbb{X}}$
\[P_{x}\left\{\tau_{D}<\infty\right\}\equiv1,\] cuando
$\tau_{D}=inf\left\{t\geq0:\mathbb{X}_{t}\in D\right\}$.
\end{Def}

\begin{Def}
Un conjunto $D\in\mathcal{B}_\mathbb{X}$ es llamado peque\~no si
existe un $t>0$, una medida de probabilidad $\nu$ en
$\mathcal{B}_\mathbb{X}$, y un $\delta>0$ tal que
\[P^{t}\left(x,A\right)\geq\delta\nu\left(A\right),\] para $x\in
D,A\in\mathcal{B}_\mathbb{X}$.
\end{Def}
\begin{Note}
\begin{itemize}

\item[i)] Si $X$ es Harris recurrente, entonces existe una \'unica medida
invariante $\pi$ (\cite{Getoor}).

\item[ii)] Si la medida invariante es finita, entonces puede
normalizarse a una medida de probabilidad, en este caso a la
medida se le llama \textbf{Harris recurrente positiva}.

\item[iii)] Cuando $X$ es Harris recurrente positivo se dice que
la disciplina de servicio es estable. En este caso $\pi$ denota la
ditribuci\'on estacionaria; se define
\[P_{\pi}\left(\cdot\right)=\int_{X}P_{x}\left(\cdot\right)\pi\left(dx\right).\]
Se utiliza $E_{\pi}$ para denotar el operador esperanza
correspondiente, as\'i, el proceso
$X=\left\{\mathbb{X}\left(t\right),t\geq0\right\}$ es un proceso
estrictamente estacionario bajo $P_{\pi}$.

\item[iv)] En \cite{MeynTweedie} se muestra que si
$P_{x}\left\{\tau_{D}<\infty\right\}=1$ incluso para solamente un
conjunto peque\~no, entonces el proceso de Harris es recurrente.
\end{itemize}
\end{Note}


%_________________________________________________________________________
%\newpage
%_________________________________________________________________________
%\subsection{Modelo de Flujo}
%_____________________________________________________________________
Las Colas C\'iclicas se pueden describir por medio de un proceso
de Markov $\left(X\left(t\right)\right)_{t\in\rea}$, donde el
estado del sistema al tiempo $t\geq0$ est\'a dado por
\begin{equation}
X\left(t\right)=\left(Q\left(t\right),A\left(t\right),H\left(t\right),B\left(t\right),B^{0}\left(t\right),C\left(t\right)\right)
\end{equation}
definido en el espacio producto:
\begin{equation}
\mathcal{X}=\mathbb{Z}^{K}\times\rea_{+}^{K}\times\left(\left\{1,2,\ldots,K\right\}\times\left\{1,2,\ldots,S\right\}\right)^{M}\times\rea_{+}^{K}\times\rea_{+}^{K}\times\mathbb{Z}^{K},
\end{equation}

\begin{itemize}
\item $Q\left(t\right)=\left(Q_{k}\left(t\right),1\leq k\leq
K\right)$, es el n\'umero de usuarios en la cola $k$, incluyendo
aquellos que est\'an siendo atendidos provenientes de la
$k$-\'esima cola.

\item $A\left(t\right)=\left(A_{k}\left(t\right),1\leq k\leq
K\right)$, son los residuales de los tiempos de arribo en la cola
$k$. \item $H\left(t\right)$ es el par ordenado que consiste en la
cola que esta siendo atendida y la pol\'itica de servicio que se
utilizar\'a.

\item $B\left(t\right)$ es el tiempo de servicio residual.

\item $B^{0}\left(t\right)$ es el tiempo residual del cambio de
cola.

\item $C\left(t\right)$ indica el n\'umero de usuarios atendidos
durante la visita del servidor a la cola dada en
$H\left(t\right)$.
\end{itemize}

$A_{k}\left(t\right),B_{m}\left(t\right)$ y
$B_{m}^{0}\left(t\right)$ se suponen continuas por la derecha y
que satisfacen la propiedad fuerte de Markov, (\cite{Dai}).

Dada una condici\'on inicial $x\in\mathcal{X}$, $Q_{k}^{x}\left(t\right)$ es la longitud de la cola $k$ al tiempo $t$
y $T_{m,k}^{x}\left(t\right)$  el tiempo acumulado al tiempo $t$ que el servidor tarda en atender a los usuarios de la cola $k$.
De igual manera se define $T_{m,k}^{x,0}\left(t\right)$ el tiempo acumulado al tiempo $t$ que el servidor tarda en
cambiar de cola para volver a atender a los usuarios.

Para reducir la fluctuaci\'on del modelo se escala tanto el espacio como el tiempo, entonces se
tiene el proceso:

\begin{eqnarray}
\overline{Q}^{x}\left(t\right)=\frac{1}{|x|}Q^{x}\left(|x|t\right),\\
\overline{T}_{m}^{x}\left(t\right)=\frac{1}{|x|}T_{m}^{x}\left(|x|t\right),\\
\overline{T}_{m}^{x,0}\left(t\right)=\frac{1}{|x|}T_{m}^{x,0}\left(|x|t\right).
\end{eqnarray}
Cualquier l\'imite $\overline{Q}\left(t\right)$ es llamado un
flujo l\'imite del proceso longitud de la cola, al conjunto de todos los posibles flujos l\'imite
se le llamar\'a \textbf{modelo de flujo}, (\cite{MaynDown}).

\begin{Def}
Un flujo l\'imite para un sistema de visitas bajo una disciplina de
servicio espec\'ifica se define como cualquier soluci\'on
 $\left(\overline{Q}\left(\cdot\right),\overline{T}_{m}\left(\cdot\right),\overline{T}_{m}^{0}\left(\cdot\right)\right)$
 de las siguientes ecuaciones, donde
$\overline{Q}\left(t\right)=\left(\overline{Q}_{1}\left(t\right),\ldots,\overline{Q}_{K}\left(t\right)\right)$
y
$\overline{T}\left(t\right)=\left(\overline{T}_{1}\left(t\right),\ldots,\overline{T}_{K}\left(t\right)\right)$
\begin{equation}\label{Eq.3.8}
\overline{Q}_{k}\left(t\right)=\overline{Q}_{k}\left(0\right)+\lambda_{k}t-\sum_{m=1}^{M}\mu_{k}\overline{T}_{m,k}\left(t\right)\\
\end{equation}
\begin{equation}\label{Eq.3.9}
\overline{Q}_{k}\left(t\right)\geq0\textrm{ para }k=1,2,\ldots,K,\\
\end{equation}
\begin{equation}\label{Eq.3.10}
\overline{T}_{m,k}\left(0\right)=0,\textrm{ y }\overline{T}_{m,k}\left(\cdot\right)\textrm{ es no decreciente},\\
\end{equation}
\begin{equation}\label{Eq.3.11}
\sum_{k=1}^{K}\overline{T}_{m,k}^{0}\left(t\right)+\overline{T}_{m,k}\left(t\right)=t\textrm{ para}m=1,2,\ldots,M\\
\end{equation}
\end{Def}

Al conjunto de ecuaciones dadas en (\ref{Eq.3.8})-(\ref{Eq.3.11}) se
le llama {\em Modelo de flujo} y al conjunto de todas las
soluciones del modelo de flujo
$\left(\overline{Q}\left(\cdot\right),\overline{T}
\left(\cdot\right)\right)$ se le denotar\'a por $\mathcal{Q}$.


\begin{Def}
El modelo de flujo es estable si existe un tiempo fijo $t_{0}$ tal
que $\overline{Q}\left(t\right)=0$, con $t\geq t_{0}$, para
cualquier $\overline{Q}\left(\cdot\right)\in\mathcal{Q}$ que
cumple con $|\overline{Q}\left(0\right)|=1$.
\end{Def}


%_____________________________________________________________________________________
%
\subsection{Estabilidad de los Sistemas de Visitas C\'iclicas}
%_________________________________________________________________________

Es necesario realizar los siguientes supuestos, ver (\cite{Dai2}) y (\cite{DaiSean}):

\begin{itemize}
\item[A1)] $\xi_{1},\ldots,\xi_{K},\eta_{1},\ldots,\eta_{K}$ son
mutuamente independientes y son sucesiones independientes e
id\'enticamente distribuidas.

\item[A2)] Para alg\'un entero $p\geq1$
\begin{eqnarray*}
\esp\left[\xi_{k}\left(1\right)^{p+1}\right]&<&\infty\textrm{ para }l\in\mathcal{A}\textrm{ y }\\
\esp\left[\eta_{k}\left(1\right)^{p+1}\right]&<&\infty\textrm{ para
}k=1,\ldots,K.
\end{eqnarray*}
\item[A3)] El conjunto $\left\{x\in X:|x|=0\right\}$ es un
singleton, y para cada $k\in\mathcal{A}$, existe una funci\'on
positiva $q_{k}\left(x\right)$ definida en $\rea_{+}$, y un entero
$j_{k}$, tal que
\begin{eqnarray}
P\left(\xi_{k}\left(1\right)\geq x\right)&>&0\textrm{, para todo }x>0\\
P\left(\xi_{k}\left(1\right)+\ldots\xi_{k}\left(j_{k}\right)\in dx\right)&\geq& q_{k}\left(x\right)dx0\textrm{ y }\\
\int_{0}^{\infty}q_{k}\left(x\right)dx>0
\end{eqnarray}
\end{itemize}


En \cite{MaynDown} ser da un argumento para deducir que todos los
subconjuntos compactos de $X$ son peque\~nos. Entonces la
condici\'on A3) se puede generalizar a
\begin{itemize}
\item[A3')] Para el proceso de Markov $X$, cada subconjunto
compacto de $X$ es peque\~no.
\end{itemize}

\begin{Teo}\label{Tma2.1}
Suponga que el modelo de flujo para una disciplina de servicio es
estable, y suponga adem\'as que las condiciones A1) y A2) se
satisfacen. Entonces:
\begin{itemize}
\item[i)] Para alguna constante $\kappa_{p}$, y para cada
condici\'on inicial $x\in X$
\begin{equation}
\limsup_{t\rightarrow\infty}\frac{1}{t}\int_{0}^{t}\esp_{x}\left[|Q\left(t\right)|^{p}\right]ds\leq\kappa_{p}
\end{equation}
donde $p$ es el entero dado por A2).
\end{itemize}

Suponga adem\'as que A3) o A3')
se cumple, entonces la disciplina de servicio es estable y adem\'as
para cada condici\'on inicial se tiene lo siguiente:
\begin{itemize}

\item[ii)] Los momentos transitorios convergen a sus valores en estado
estacionario:
\begin{equation}
\lim_{t\rightarrow\infty}\esp_{x}\left[Q_{k}\left(t\right)^{r}\right]=\esp_{\pi}\left[Q_{k}\left(0\right)^{r}\right]\leq\kappa_{r}
\end{equation}
para $r=1,\ldots,p$ y $k=1,\ldots,K$. \item[iii)] El primer
momento converge con raz\'on $t^{p-1}$:
\begin{equation}
\lim_{t\rightarrow\infty}t^{p-1}|\esp_{x}\left[Q\left(t\right)\right]-\esp_{\pi}\left[Q\left(0\right)\right]|=0.
\end{equation}
\item[iv)] Se cumple la Ley Fuerte de los Grandes N\'umeros:
\begin{equation}
\lim_{t\rightarrow\infty}\frac{1}{t}\int_{0}^{t}Q_{k}^{r}\left(s\right)ds=\esp_{\pi}\left[Q_{k}\left(0\right)^{r}\right]
\end{equation}
$\prob$-c.s., para $r=1,\ldots,p$ y $k=1,\ldots,K$.
\end{itemize}
\end{Teo}


\begin{Teo}\label{Tma2.2}
Suponga que el fluido modelo es inestable en el sentido de que
para alguna $\epsilon_{0},c_{0}\geq0$,
\begin{equation}\label{Eq.Inestability}
|Q\left(T\right)|\geq\epsilon_{0}T-c_{0}\textrm{, con }T\geq0,
\end{equation}
para cualquier condici\'on inicial $Q\left(0\right)$, con
$|Q\left(0\right)|=1$. Entonces para cualquier $0<q\leq1$, existe
$B<0$ tal que para cualquier $|x|\geq B$,
\begin{equation}
\prob_{x}\left\{\mathbb{X}\rightarrow\infty\right\}\geq q.
\end{equation}
\end{Teo}

%_____________________________________________________________________________________
%

%_____________________________________________________________________
\subsection{Resultados principales}
%_____________________________________________________________________
En el caso particular de un modelo con un solo servidor, $M=1$, se
tiene que si se define
\begin{Def}\label{Def.Ro}
\begin{equation}\label{RoM1}
\rho=\sum_{k=1}^{K}\rho_{k}+\max_{1\leq j\leq
K}\left(\frac{\lambda_{j}}{\overline{N}}\right)\delta^{*}.
\end{equation}
\end{Def}
entonces

\begin{Teo}\label{Teo.Down}
\begin{itemize}
\item[i)] Si $\rho<1$, entonces la red es estable, es decir el teorema
(\ref{Tma2.1}) se cumple. \item[ii)] De lo contrario, es decir, si
$\rho>1$ entonces la red es inestable, es decir, el teorema
(\ref{Tma2.2}).
\end{itemize}
\end{Teo}



%_________________________________________________________________________
\subsection{Supuestos}
%_________________________________________________________________________
Consideremos el caso en el que se tienen varias colas a las cuales
llegan uno o varios servidores para dar servicio a los usuarios
que se encuentran presentes en la cola, como ya se mencion\'o hay
varios tipos de pol\'iticas de servicio, incluso podr\'ia ocurrir
que la manera en que atiende al resto de las colas sea distinta a
como lo hizo en las anteriores.\\

Para ejemplificar los sistemas de visitas c\'iclicas se
considerar\'a el caso en que a las colas los usuarios son atendidos con
una s\'ola pol\'itica de servicio.\\


Si $\omega$ es el n\'umero de usuarios en la cola al comienzo del
periodo de servicio y $N\left(\omega\right)$ es el n\'umero de
usuarios que son atendidos con una pol\'itica en espec\'ifico
durante el periodo de servicio, entonces se asume que:
\begin{itemize}
\item[1)]\label{S1}$lim_{\omega\rightarrow\infty}\esp\left[N\left(\omega\right)\right]=\overline{N}>0$;
\item[2)]\label{S2}$\esp\left[N\left(\omega\right)\right]\leq\overline{N}$
para cualquier valor de $\omega$.
\end{itemize}
La manera en que atiende el servidor $m$-\'esimo, es la siguiente:
\begin{itemize}
\item Al t\'ermino de la visita a la cola $j$, el servidor cambia
a la cola $j^{'}$ con probabilidad $r_{j,j^{'}}^{m}$

\item La $n$-\'esima vez que el servidor cambia de la cola $j$ a
$j'$, va acompa\~nada con el tiempo de cambio de longitud
$\delta_{j,j^{'}}^{m}\left(n\right)$, con
$\delta_{j,j^{'}}^{m}\left(n\right)$, $n\geq1$, variables
aleatorias independientes e id\'enticamente distribuidas, tales
que $\esp\left[\delta_{j,j^{'}}^{m}\left(1\right)\right]\geq0$.

\item Sea $\left\{p_{j}^{m}\right\}$ la distribuci\'on invariante
estacionaria \'unica para la Cadena de Markov con matriz de
transici\'on $\left(r_{j,j^{'}}^{m}\right)$, se supone que \'esta
existe.

\item Finalmente, se define el tiempo promedio total de traslado
entre las colas como
\begin{equation}
\delta^{*}:=\sum_{j,j^{'}}p_{j}^{m}r_{j,j^{'}}^{m}\esp\left[\delta_{j,j^{'}}^{m}\left(i\right)\right].
\end{equation}
\end{itemize}

Consideremos el caso donde los tiempos entre arribo a cada una de
las colas, $\left\{\xi_{k}\left(n\right)\right\}_{n\geq1}$ son
variables aleatorias independientes a id\'enticamente
distribuidas, y los tiempos de servicio en cada una de las colas
se distribuyen de manera independiente e id\'enticamente
distribuidas $\left\{\eta_{k}\left(n\right)\right\}_{n\geq1}$;
adem\'as ambos procesos cumplen la condici\'on de ser
independientes entre s\'i. Para la $k$-\'esima cola se define la
tasa de arribo por
$\lambda_{k}=1/\esp\left[\xi_{k}\left(1\right)\right]$ y la tasa
de servicio como
$\mu_{k}=1/\esp\left[\eta_{k}\left(1\right)\right]$, finalmente se
define la carga de la cola como $\rho_{k}=\lambda_{k}/\mu_{k}$,
donde se pide que $\rho=\sum_{k=1}^{K}\rho_{k}<1$, para garantizar
la estabilidad del sistema, esto es cierto para las pol\'iticas de
servicio exhaustiva y cerrada, ver Geetor \cite{Getoor}.\\

Si denotamos por
\begin{itemize}
\item $Q_{k}\left(t\right)$ el n\'umero de usuarios presentes en
la cola $k$ al tiempo $t$; \item $A_{k}\left(t\right)$ los
residuales de los tiempos entre arribos a la cola $k$; para cada
servidor $m$; \item $B_{m}\left(t\right)$ denota a los residuales
de los tiempos de servicio al tiempo $t$; \item
$B_{m}^{0}\left(t\right)$ los residuales de los tiempos de
traslado de la cola $k$ a la pr\'oxima por atender al tiempo $t$,

\item sea
$C_{m}\left(t\right)$ el n\'umero de usuarios atendidos durante la
visita del servidor a la cola $k$ al tiempo $t$.
\end{itemize}


En este sentido, el proceso para el sistema de visitas se puede
definir como:

\begin{equation}\label{Esp.Edos.Down}
X\left(t\right)^{T}=\left(Q_{k}\left(t\right),A_{k}\left(t\right),B_{m}\left(t\right),B_{m}^{0}\left(t\right),C_{m}\left(t\right)\right),
\end{equation}
para $k=1,\ldots,K$ y $m=1,2,\ldots,M$, donde $T$ indica que es el
transpuesto del vector que se est\'a definiendo. El proceso $X$
evoluciona en el espacio de estados:
$\mathbb{X}=\ent_{+}^{K}\times\rea_{+}^{K}\times\left(\left\{1,2,\ldots,K\right\}\times\left\{1,2,\ldots,S\right\}\right)^{M}\times\rea_{+}^{K}\times\ent_{+}^{K}$.\\

El sistema aqu\'i descrito debe de cumplir con los siguientes supuestos b\'asicos de un sistema de visitas:
%__________________________________________________________________________
\subsubsection{Supuestos B\'asicos}
%__________________________________________________________________________
\begin{itemize}
\item[A1)] Los procesos
$\xi_{1},\ldots,\xi_{K},\eta_{1},\ldots,\eta_{K}$ son mutuamente
independientes y son sucesiones independientes e id\'enticamente
distribuidas.

\item[A2)] Para alg\'un entero $p\geq1$
\begin{eqnarray*}
\esp\left[\xi_{l}\left(1\right)^{p+1}\right]&<&\infty\textrm{ para }l=1,\ldots,K\textrm{ y }\\
\esp\left[\eta_{k}\left(1\right)^{p+1}\right]&<&\infty\textrm{
para }k=1,\ldots,K.
\end{eqnarray*}
donde $\mathcal{A}$ es la clase de posibles arribos.

\item[A3)] Para cada $k=1,2,\ldots,K$ existe una funci\'on
positiva $q_{k}\left(\cdot\right)$ definida en $\rea_{+}$, y un
entero $j_{k}$, tal que
\begin{eqnarray}
P\left(\xi_{k}\left(1\right)\geq x\right)&>&0\textrm{, para todo }x>0,\\
P\left\{a\leq\sum_{i=1}^{j_{k}}\xi_{k}\left(i\right)\leq
b\right\}&\geq&\int_{a}^{b}q_{k}\left(x\right)dx, \textrm{ }0\leq
a<b.
\end{eqnarray}
\end{itemize}

En lo que respecta al supuesto (A3), en Dai y Meyn \cite{DaiSean}
hacen ver que este se puede sustituir por

\begin{itemize}
\item[A3')] Para el Proceso de Markov $X$, cada subconjunto
compacto del espacio de estados de $X$ es un conjunto peque\~no,
ver definici\'on \ref{Def.Cto.Peq.}.
\end{itemize}

Es por esta raz\'on que con la finalidad de poder hacer uso de
$A3^{'})$ es necesario recurrir a los Procesos de Harris y en
particular a los Procesos Harris Recurrente, ver \cite{Dai,
DaiSean}.
%_______________________________________________________________________
\subsection{Procesos Harris Recurrente}
%_______________________________________________________________________

Por el supuesto (A1) conforme a Davis \cite{Davis}, se puede
definir el proceso de saltos correspondiente de manera tal que
satisfaga el supuesto (A3'), de hecho la demostraci\'on est\'a
basada en la l\'inea de argumentaci\'on de Davis, \cite{Davis},
p\'aginas 362-364.\\

Entonces se tiene un espacio de estados en el cual el proceso $X$
satisface la Propiedad Fuerte de Markov, ver Dai y Meyn
\cite{DaiSean}, dado por

\[\left(\Omega,\mathcal{F},\mathcal{F}_{t},X\left(t\right),\theta_{t},P_{x}\right),\]
adem\'as de ser un proceso de Borel Derecho (Sharpe \cite{Sharpe})
en el espacio de estados medible
$\left(\mathbb{X},\mathcal{B}_\mathbb{X}\right)$. El Proceso
$X=\left\{X\left(t\right),t\geq0\right\}$ tiene trayectorias
continuas por la derecha, est\'a definido en
$\left(\Omega,\mathcal{F}\right)$ y est\'a adaptado a
$\left\{\mathcal{F}_{t},t\geq0\right\}$; la colecci\'on
$\left\{P_{x},x\in \mathbb{X}\right\}$ son medidas de probabilidad
en $\left(\Omega,\mathcal{F}\right)$ tales que para todo $x\in
\mathbb{X}$
\[P_{x}\left\{X\left(0\right)=x\right\}=1,\] y
\[E_{x}\left\{f\left(X\circ\theta_{t}\right)|\mathcal{F}_{t}\right\}=E_{X}\left(\tau\right)f\left(X\right),\]
en $\left\{\tau<\infty\right\}$, $P_{x}$-c.s., con $\theta_{t}$
definido como el operador shift.


Donde $\tau$ es un $\mathcal{F}_{t}$-tiempo de paro
\[\left(X\circ\theta_{\tau}\right)\left(w\right)=\left\{X\left(\tau\left(w\right)+t,w\right),t\geq0\right\},\]
y $f$ es una funci\'on de valores reales acotada y medible, ver \cite{Dai, KaspiMandelbaum}.\\

Sea $P^{t}\left(x,D\right)$, $D\in\mathcal{B}_{\mathbb{X}}$,
$t\geq0$ la probabilidad de transici\'on de $X$ queda definida
como:
\[P^{t}\left(x,D\right)=P_{x}\left(X\left(t\right)\in
D\right).\]


\begin{Def}
Una medida no cero $\pi$ en
$\left(\mathbb{X},\mathcal{B}_{\mathbb{X}}\right)$ es invariante
para $X$ si $\pi$ es $\sigma$-finita y
\[\pi\left(D\right)=\int_{\mathbb{X}}P^{t}\left(x,D\right)\pi\left(dx\right),\]
para todo $D\in \mathcal{B}_{\mathbb{X}}$, con $t\geq0$.
\end{Def}

\begin{Def}
El proceso de Markov $X$ es llamado Harris Recurrente si existe
una medida de probabilidad $\nu$ en
$\left(\mathbb{X},\mathcal{B}_{\mathbb{X}}\right)$, tal que si
$\nu\left(D\right)>0$ y $D\in\mathcal{B}_{\mathbb{X}}$
\[P_{x}\left\{\tau_{D}<\infty\right\}\equiv1,\] cuando
$\tau_{D}=inf\left\{t\geq0:X_{t}\in D\right\}$.
\end{Def}

\begin{Note}
\begin{itemize}
\item[i)] Si $X$ es Harris recurrente, entonces existe una \'unica
medida invariante $\pi$ (Getoor \cite{Getoor}).

\item[ii)] Si la medida invariante es finita, entonces puede
normalizarse a una medida de probabilidad, en este caso al proceso
$X$ se le llama Harris recurrente positivo.


\item[iii)] Cuando $X$ es Harris recurrente positivo se dice que
la disciplina de servicio es estable. En este caso $\pi$ denota la
distribuci\'on estacionaria y hacemos
\[P_{\pi}\left(\cdot\right)=\int_{\mathbf{X}}P_{x}\left(\cdot\right)\pi\left(dx\right),\]
y se utiliza $E_{\pi}$ para denotar el operador esperanza
correspondiente, ver \cite{DaiSean}.
\end{itemize}
\end{Note}

\begin{Def}\label{Def.Cto.Peq.}
Un conjunto $D\in\mathcal{B_{\mathbb{X}}}$ es llamado peque\~no si
existe un $t>0$, una medida de probabilidad $\nu$ en
$\mathcal{B_{\mathbb{X}}}$, y un $\delta>0$ tal que
\[P^{t}\left(x,A\right)\geq\delta\nu\left(A\right),\] para $x\in
D,A\in\mathcal{B_{\mathbb{X}}}$.
\end{Def}

La siguiente serie de resultados vienen enunciados y demostrados
en Dai \cite{Dai}:
\begin{Lema}[Lema 3.1, Dai \cite{Dai}]
Sea $B$ conjunto peque\~no cerrado, supongamos que
$P_{x}\left(\tau_{B}<\infty\right)\equiv1$ y que para alg\'un
$\delta>0$ se cumple que
\begin{equation}\label{Eq.3.1}
\sup\esp_{x}\left[\tau_{B}\left(\delta\right)\right]<\infty,
\end{equation}
donde
$\tau_{B}\left(\delta\right)=inf\left\{t\geq\delta:X\left(t\right)\in
B\right\}$. Entonces, $X$ es un proceso Harris recurrente
positivo.
\end{Lema}

\begin{Lema}[Lema 3.1, Dai \cite{Dai}]\label{Lema.3.}
Bajo el supuesto (A3), el conjunto
$B=\left\{x\in\mathbb{X}:|x|\leq k\right\}$ es un conjunto
peque\~no cerrado para cualquier $k>0$.
\end{Lema}

\begin{Teo}[Teorema 3.1, Dai \cite{Dai}]\label{Tma.3.1}
Si existe un $\delta>0$ tal que
\begin{equation}
lim_{|x|\rightarrow\infty}\frac{1}{|x|}\esp|X^{x}\left(|x|\delta\right)|=0,
\end{equation}
donde $X^{x}$ se utiliza para denotar que el proceso $X$ comienza
a partir de $x$, entonces la ecuaci\'on (\ref{Eq.3.1}) se cumple
para $B=\left\{x\in\mathbb{X}:|x|\leq k\right\}$ con alg\'un
$k>0$. En particular, $X$ es Harris recurrente positivo.
\end{Teo}

Entonces, tenemos que el proceso $X$ es un proceso de Markov que
cumple con los supuestos $A1)$-$A3)$, lo que falta de hacer es
construir el Modelo de Flujo bas\'andonos en lo hasta ahora
presentado.
%_______________________________________________________________________
\subsection{Modelo de Flujo}
%_______________________________________________________________________

Dada una condici\'on inicial $x\in\mathbb{X}$, sea

\begin{itemize}
\item $Q_{k}^{x}\left(t\right)$ la longitud de la cola al tiempo
$t$,

\item $T_{m,k}^{x}\left(t\right)$ el tiempo acumulado, al tiempo
$t$, que tarda el servidor $m$ en atender a los usuarios de la
cola $k$.

\item $T_{m,k}^{x,0}\left(t\right)$ el tiempo acumulado, al tiempo
$t$, que tarda el servidor $m$ en trasladarse a otra cola a partir de la $k$-\'esima.\\
\end{itemize}

Sup\'ongase que la funci\'on
$\left(\overline{Q}\left(\cdot\right),\overline{T}_{m}
\left(\cdot\right),\overline{T}_{m}^{0} \left(\cdot\right)\right)$
para $m=1,2,\ldots,M$ es un punto l\'imite de
\begin{equation}\label{Eq.Punto.Limite}
\left(\frac{1}{|x|}Q^{x}\left(|x|t\right),\frac{1}{|x|}T_{m}^{x}\left(|x|t\right),\frac{1}{|x|}T_{m}^{x,0}\left(|x|t\right)\right)
\end{equation}
para $m=1,2,\ldots,M$, cuando $x\rightarrow\infty$, ver
\cite{Down}. Entonces
$\left(\overline{Q}\left(t\right),\overline{T}_{m}
\left(t\right),\overline{T}_{m}^{0} \left(t\right)\right)$ es un
flujo l\'imite del sistema. Al conjunto de todos las posibles
flujos l\'imite se le llama {\emph{Modelo de Flujo}} y se le
denotar\'a por $\mathcal{Q}$, ver \cite{Down, Dai, DaiSean}.\\

El modelo de flujo satisface el siguiente conjunto de ecuaciones:

\begin{equation}\label{Eq.MF.1}
\overline{Q}_{k}\left(t\right)=\overline{Q}_{k}\left(0\right)+\lambda_{k}t-\sum_{m=1}^{M}\mu_{k}\overline{T}_{m,k}\left(t\right),\\
\end{equation}
para $k=1,2,\ldots,K$.\\
\begin{equation}\label{Eq.MF.2}
\overline{Q}_{k}\left(t\right)\geq0\textrm{ para
}k=1,2,\ldots,K.\\
\end{equation}

\begin{equation}\label{Eq.MF.3}
\overline{T}_{m,k}\left(0\right)=0,\textrm{ y }\overline{T}_{m,k}\left(\cdot\right)\textrm{ es no decreciente},\\
\end{equation}
para $k=1,2,\ldots,K$ y $m=1,2,\ldots,M$.\\
\begin{equation}\label{Eq.MF.4}
\sum_{k=1}^{K}\overline{T}_{m,k}^{0}\left(t\right)+\overline{T}_{m,k}\left(t\right)=t\textrm{
para }m=1,2,\ldots,M.\\
\end{equation}


\begin{Def}[Definici\'on 4.1, Dai \cite{Dai}]\label{Def.Modelo.Flujo}
Sea una disciplina de servicio espec\'ifica. Cualquier l\'imite
$\left(\overline{Q}\left(\cdot\right),\overline{T}\left(\cdot\right),\overline{T}^{0}\left(\cdot\right)\right)$
en (\ref{Eq.Punto.Limite}) es un {\em flujo l\'imite} de la
disciplina. Cualquier soluci\'on (\ref{Eq.MF.1})-(\ref{Eq.MF.4})
es llamado flujo soluci\'on de la disciplina.
\end{Def}

\begin{Def}
Se dice que el modelo de flujo l\'imite, modelo de flujo, de la
disciplina de la cola es estable si existe una constante
$\delta>0$ que depende de $\mu,\lambda$ y $P$ solamente, tal que
cualquier flujo l\'imite con
$|\overline{Q}\left(0\right)|+|\overline{U}|+|\overline{V}|=1$, se
tiene que $\overline{Q}\left(\cdot+\delta\right)\equiv0$.
\end{Def}

Si se hace $|x|\rightarrow\infty$ sin restringir ninguna de las
componentes, tambi\'en se obtienen un modelo de flujo, pero en
este caso el residual de los procesos de arribo y servicio
introducen un retraso:
\begin{Teo}[Teorema 4.2, Dai \cite{Dai}]\label{Tma.4.2.Dai}
Sea una disciplina fija para la cola, suponga que se cumplen las
condiciones (A1)-(A3). Si el modelo de flujo l\'imite de la
disciplina de la cola es estable, entonces la cadena de Markov $X$
que describe la din\'amica de la red bajo la disciplina es Harris
recurrente positiva.
\end{Teo}

Ahora se procede a escalar el espacio y el tiempo para reducir la
aparente fluctuaci\'on del modelo. Consid\'erese el proceso
\begin{equation}\label{Eq.3.7}
\overline{Q}^{x}\left(t\right)=\frac{1}{|x|}Q^{x}\left(|x|t\right).
\end{equation}
A este proceso se le conoce como el flujo escalado, y cualquier
l\'imite $\overline{Q}^{x}\left(t\right)$ es llamado flujo
l\'imite del proceso de longitud de la cola. Haciendo
$|q|\rightarrow\infty$ mientras se mantiene el resto de las
componentes fijas, cualquier punto l\'imite del proceso de
longitud de la cola normalizado $\overline{Q}^{x}$ es soluci\'on
del siguiente modelo de flujo.


\begin{Def}[Definici\'on 3.3, Dai y Meyn \cite{DaiSean}]
El modelo de flujo es estable si existe un tiempo fijo $t_{0}$ tal
que $\overline{Q}\left(t\right)=0$, con $t\geq t_{0}$, para
cualquier $\overline{Q}\left(\cdot\right)\in\mathcal{Q}$ que
cumple con $|\overline{Q}\left(0\right)|=1$.
\end{Def}

\begin{Lemma}[Lema 3.1, Dai y Meyn \cite{DaiSean}]
Si el modelo de flujo definido por (\ref{Eq.MF.1})-(\ref{Eq.MF.4})
es estable, entonces el modelo de flujo retrasado es tambi\'en
estable, es decir, existe $t_{0}>0$ tal que
$\overline{Q}\left(t\right)=0$ para cualquier $t\geq t_{0}$, para
cualquier soluci\'on del modelo de flujo retrasado cuya
condici\'on inicial $\overline{x}$ satisface que
$|\overline{x}|=|\overline{Q}\left(0\right)|+|\overline{A}\left(0\right)|+|\overline{B}\left(0\right)|\leq1$.
\end{Lemma}


Ahora ya estamos en condiciones de enunciar los resultados principales:


\begin{Teo}[Teorema 2.1, Down \cite{Down}]\label{Tma2.1.Down}
Suponga que el modelo de flujo es estable, y que se cumplen los supuestos (A1) y (A2), entonces
\begin{itemize}
\item[i)] Para alguna constante $\kappa_{p}$, y para cada
condici\'on inicial $x\in X$
\begin{equation}\label{Estability.Eq1}
\limsup_{t\rightarrow\infty}\frac{1}{t}\int_{0}^{t}\esp_{x}\left[|Q\left(s\right)|^{p}\right]ds\leq\kappa_{p},
\end{equation}
donde $p$ es el entero dado en (A2).
\end{itemize}
Si adem\'as se cumple la condici\'on (A3), entonces para cada
condici\'on inicial:
\begin{itemize}
\item[ii)] Los momentos transitorios convergen a su estado
estacionario:
 \begin{equation}\label{Estability.Eq2}
lim_{t\rightarrow\infty}\esp_{x}\left[Q_{k}\left(t\right)^{r}\right]=\esp_{\pi}\left[Q_{k}\left(0\right)^{r}\right]\leq\kappa_{r},
\end{equation}
para $r=1,2,\ldots,p$ y $k=1,2,\ldots,K$. Donde $\pi$ es la
probabilidad invariante para $X$.

\item[iii)]  El primer momento converge con raz\'on $t^{p-1}$:
\begin{equation}\label{Estability.Eq3}
lim_{t\rightarrow\infty}t^{p-1}|\esp_{x}\left[Q_{k}\left(t\right)\right]-\esp_{\pi}\left[Q_{k}\left(0\right)\right]|=0.
\end{equation}

\item[iv)] La {\em Ley Fuerte de los grandes n\'umeros} se cumple:
\begin{equation}\label{Estability.Eq4}
lim_{t\rightarrow\infty}\frac{1}{t}\int_{0}^{t}Q_{k}^{r}\left(s\right)ds=\esp_{\pi}\left[Q_{k}\left(0\right)^{r}\right],\textrm{
}\prob_{x}\textrm{-c.s.}
\end{equation}
para $r=1,2,\ldots,p$ y $k=1,2,\ldots,K$.
\end{itemize}
\end{Teo}

La contribuci\'on de Down a la teor\'ia de los {\emph {sistemas de
visitas c\'iclicas}}, es la relaci\'on que hay entre la
estabilidad del sistema con el comportamiento de las medidas de
desempe\~no, es decir, la condici\'on suficiente para poder
garantizar la convergencia del proceso de la longitud de la cola
as\'i como de por los menos los dos primeros momentos adem\'as de
una versi\'on de la Ley Fuerte de los Grandes N\'umeros para los
sistemas de visitas.


\begin{Teo}[Teorema 2.3, Down \cite{Down}]\label{Tma2.3.Down}
Considere el siguiente valor:
\begin{equation}\label{Eq.Rho.1serv}
\rho=\sum_{k=1}^{K}\rho_{k}+max_{1\leq j\leq K}\left(\frac{\lambda_{j}}{\sum_{s=1}^{S}p_{js}\overline{N}_{s}}\right)\delta^{*}
\end{equation}
\begin{itemize}
\item[i)] Si $\rho<1$ entonces la red es estable, es decir, se
cumple el Teorema \ref{Tma2.1.Down}.

\item[ii)] Si $\rho>1$ entonces la red es inestable, es decir, se
cumple el Teorema \ref{Tma2.2.Down}
\end{itemize}
\end{Teo}

%_________________________________________________________________________
%\section{DESARROLLO DEL TEMA Y/O METODOLOG\'IA}
%_________________________________________________________________________
\subsection{Supuestos}
%_________________________________________________________________________
Consideremos el caso en el que se tienen varias colas a las cuales
llegan uno o varios servidores para dar servicio a los usuarios
que se encuentran presentes en la cola, como ya se mencion\'o hay
varios tipos de pol\'iticas de servicio, incluso podr\'ia ocurrir
que la manera en que atiende al resto de las colas sea distinta a
como lo hizo en las anteriores.\\

Para ejemplificar los sistemas de visitas c\'iclicas se
considerar\'a el caso en que a las colas los usuarios son atendidos con
una s\'ola pol\'itica de servicio.\\


Si $\omega$ es el n\'umero de usuarios en la cola al comienzo del
periodo de servicio y $N\left(\omega\right)$ es el n\'umero de
usuarios que son atendidos con una pol\'itica en espec\'ifico
durante el periodo de servicio, entonces se asume que:
\begin{itemize}
\item[1)]\label{S1}$lim_{\omega\rightarrow\infty}\esp\left[N\left(\omega\right)\right]=\overline{N}>0$;
\item[2)]\label{S2}$\esp\left[N\left(\omega\right)\right]\leq\overline{N}$
para cualquier valor de $\omega$.
\end{itemize}
La manera en que atiende el servidor $m$-\'esimo, es la siguiente:
\begin{itemize}
\item Al t\'ermino de la visita a la cola $j$, el servidor cambia
a la cola $j^{'}$ con probabilidad $r_{j,j^{'}}^{m}$

\item La $n$-\'esima vez que el servidor cambia de la cola $j$ a
$j'$, va acompa\~nada con el tiempo de cambio de longitud
$\delta_{j,j^{'}}^{m}\left(n\right)$, con
$\delta_{j,j^{'}}^{m}\left(n\right)$, $n\geq1$, variables
aleatorias independientes e id\'enticamente distribuidas, tales
que $\esp\left[\delta_{j,j^{'}}^{m}\left(1\right)\right]\geq0$.

\item Sea $\left\{p_{j}^{m}\right\}$ la distribuci\'on invariante
estacionaria \'unica para la Cadena de Markov con matriz de
transici\'on $\left(r_{j,j^{'}}^{m}\right)$, se supone que \'esta
existe.

\item Finalmente, se define el tiempo promedio total de traslado
entre las colas como
\begin{equation}
\delta^{*}:=\sum_{j,j^{'}}p_{j}^{m}r_{j,j^{'}}^{m}\esp\left[\delta_{j,j^{'}}^{m}\left(i\right)\right].
\end{equation}
\end{itemize}

Consideremos el caso donde los tiempos entre arribo a cada una de
las colas, $\left\{\xi_{k}\left(n\right)\right\}_{n\geq1}$ son
variables aleatorias independientes a id\'enticamente
distribuidas, y los tiempos de servicio en cada una de las colas
se distribuyen de manera independiente e id\'enticamente
distribuidas $\left\{\eta_{k}\left(n\right)\right\}_{n\geq1}$;
adem\'as ambos procesos cumplen la condici\'on de ser
independientes entre s\'i. Para la $k$-\'esima cola se define la
tasa de arribo por
$\lambda_{k}=1/\esp\left[\xi_{k}\left(1\right)\right]$ y la tasa
de servicio como
$\mu_{k}=1/\esp\left[\eta_{k}\left(1\right)\right]$, finalmente se
define la carga de la cola como $\rho_{k}=\lambda_{k}/\mu_{k}$,
donde se pide que $\rho=\sum_{k=1}^{K}\rho_{k}<1$, para garantizar
la estabilidad del sistema, esto es cierto para las pol\'iticas de
servicio exhaustiva y cerrada, ver Geetor \cite{Getoor}.\\

Si denotamos por
\begin{itemize}
\item $Q_{k}\left(t\right)$ el n\'umero de usuarios presentes en
la cola $k$ al tiempo $t$; \item $A_{k}\left(t\right)$ los
residuales de los tiempos entre arribos a la cola $k$; para cada
servidor $m$; \item $B_{m}\left(t\right)$ denota a los residuales
de los tiempos de servicio al tiempo $t$; \item
$B_{m}^{0}\left(t\right)$ los residuales de los tiempos de
traslado de la cola $k$ a la pr\'oxima por atender al tiempo $t$,

\item sea
$C_{m}\left(t\right)$ el n\'umero de usuarios atendidos durante la
visita del servidor a la cola $k$ al tiempo $t$.
\end{itemize}


En este sentido, el proceso para el sistema de visitas se puede
definir como:

\begin{equation}\label{Esp.Edos.Down}
X\left(t\right)^{T}=\left(Q_{k}\left(t\right),A_{k}\left(t\right),B_{m}\left(t\right),B_{m}^{0}\left(t\right),C_{m}\left(t\right)\right),
\end{equation}
para $k=1,\ldots,K$ y $m=1,2,\ldots,M$, donde $T$ indica que es el
transpuesto del vector que se est\'a definiendo. El proceso $X$
evoluciona en el espacio de estados:
$\mathbb{X}=\ent_{+}^{K}\times\rea_{+}^{K}\times\left(\left\{1,2,\ldots,K\right\}\times\left\{1,2,\ldots,S\right\}\right)^{M}\times\rea_{+}^{K}\times\ent_{+}^{K}$.\\

El sistema aqu\'i descrito debe de cumplir con los siguientes supuestos b\'asicos de un sistema de visitas:
%__________________________________________________________________________
\subsubsection{Supuestos B\'asicos}
%__________________________________________________________________________
\begin{itemize}
\item[A1)] Los procesos
$\xi_{1},\ldots,\xi_{K},\eta_{1},\ldots,\eta_{K}$ son mutuamente
independientes y son sucesiones independientes e id\'enticamente
distribuidas.

\item[A2)] Para alg\'un entero $p\geq1$
\begin{eqnarray*}
\esp\left[\xi_{l}\left(1\right)^{p+1}\right]&<&\infty\textrm{ para }l=1,\ldots,K\textrm{ y }\\
\esp\left[\eta_{k}\left(1\right)^{p+1}\right]&<&\infty\textrm{
para }k=1,\ldots,K.
\end{eqnarray*}
donde $\mathcal{A}$ es la clase de posibles arribos.

\item[A3)] Para cada $k=1,2,\ldots,K$ existe una funci\'on
positiva $q_{k}\left(\cdot\right)$ definida en $\rea_{+}$, y un
entero $j_{k}$, tal que
\begin{eqnarray}
P\left(\xi_{k}\left(1\right)\geq x\right)&>&0\textrm{, para todo }x>0,\\
P\left\{a\leq\sum_{i=1}^{j_{k}}\xi_{k}\left(i\right)\leq
b\right\}&\geq&\int_{a}^{b}q_{k}\left(x\right)dx, \textrm{ }0\leq
a<b.
\end{eqnarray}
\end{itemize}

En lo que respecta al supuesto (A3), en Dai y Meyn \cite{DaiSean}
hacen ver que este se puede sustituir por

\begin{itemize}
\item[A3')] Para el Proceso de Markov $X$, cada subconjunto
compacto del espacio de estados de $X$ es un conjunto peque\~no,
ver definici\'on \ref{Def.Cto.Peq.}.
\end{itemize}

Es por esta raz\'on que con la finalidad de poder hacer uso de
$A3^{'})$ es necesario recurrir a los Procesos de Harris y en
particular a los Procesos Harris Recurrente, ver \cite{Dai,
DaiSean}.
%_______________________________________________________________________
\subsection{Procesos Harris Recurrente}
%_______________________________________________________________________

Por el supuesto (A1) conforme a Davis \cite{Davis}, se puede
definir el proceso de saltos correspondiente de manera tal que
satisfaga el supuesto (A3'), de hecho la demostraci\'on est\'a
basada en la l\'inea de argumentaci\'on de Davis, \cite{Davis},
p\'aginas 362-364.\\

Entonces se tiene un espacio de estados en el cual el proceso $X$
satisface la Propiedad Fuerte de Markov, ver Dai y Meyn
\cite{DaiSean}, dado por

\[\left(\Omega,\mathcal{F},\mathcal{F}_{t},X\left(t\right),\theta_{t},P_{x}\right),\]
adem\'as de ser un proceso de Borel Derecho (Sharpe \cite{Sharpe})
en el espacio de estados medible
$\left(\mathbb{X},\mathcal{B}_\mathbb{X}\right)$. El Proceso
$X=\left\{X\left(t\right),t\geq0\right\}$ tiene trayectorias
continuas por la derecha, est\'a definido en
$\left(\Omega,\mathcal{F}\right)$ y est\'a adaptado a
$\left\{\mathcal{F}_{t},t\geq0\right\}$; la colecci\'on
$\left\{P_{x},x\in \mathbb{X}\right\}$ son medidas de probabilidad
en $\left(\Omega,\mathcal{F}\right)$ tales que para todo $x\in
\mathbb{X}$
\[P_{x}\left\{X\left(0\right)=x\right\}=1,\] y
\[E_{x}\left\{f\left(X\circ\theta_{t}\right)|\mathcal{F}_{t}\right\}=E_{X}\left(\tau\right)f\left(X\right),\]
en $\left\{\tau<\infty\right\}$, $P_{x}$-c.s., con $\theta_{t}$
definido como el operador shift.


Donde $\tau$ es un $\mathcal{F}_{t}$-tiempo de paro
\[\left(X\circ\theta_{\tau}\right)\left(w\right)=\left\{X\left(\tau\left(w\right)+t,w\right),t\geq0\right\},\]
y $f$ es una funci\'on de valores reales acotada y medible, ver \cite{Dai, KaspiMandelbaum}.\\

Sea $P^{t}\left(x,D\right)$, $D\in\mathcal{B}_{\mathbb{X}}$,
$t\geq0$ la probabilidad de transici\'on de $X$ queda definida
como:
\[P^{t}\left(x,D\right)=P_{x}\left(X\left(t\right)\in
D\right).\]


\begin{Def}
Una medida no cero $\pi$ en
$\left(\mathbb{X},\mathcal{B}_{\mathbb{X}}\right)$ es invariante
para $X$ si $\pi$ es $\sigma$-finita y
\[\pi\left(D\right)=\int_{\mathbb{X}}P^{t}\left(x,D\right)\pi\left(dx\right),\]
para todo $D\in \mathcal{B}_{\mathbb{X}}$, con $t\geq0$.
\end{Def}

\begin{Def}
El proceso de Markov $X$ es llamado Harris Recurrente si existe
una medida de probabilidad $\nu$ en
$\left(\mathbb{X},\mathcal{B}_{\mathbb{X}}\right)$, tal que si
$\nu\left(D\right)>0$ y $D\in\mathcal{B}_{\mathbb{X}}$
\[P_{x}\left\{\tau_{D}<\infty\right\}\equiv1,\] cuando
$\tau_{D}=inf\left\{t\geq0:X_{t}\in D\right\}$.
\end{Def}

\begin{Note}
\begin{itemize}
\item[i)] Si $X$ es Harris recurrente, entonces existe una \'unica
medida invariante $\pi$ (Getoor \cite{Getoor}).

\item[ii)] Si la medida invariante es finita, entonces puede
normalizarse a una medida de probabilidad, en este caso al proceso
$X$ se le llama Harris recurrente positivo.


\item[iii)] Cuando $X$ es Harris recurrente positivo se dice que
la disciplina de servicio es estable. En este caso $\pi$ denota la
distribuci\'on estacionaria y hacemos
\[P_{\pi}\left(\cdot\right)=\int_{\mathbf{X}}P_{x}\left(\cdot\right)\pi\left(dx\right),\]
y se utiliza $E_{\pi}$ para denotar el operador esperanza
correspondiente, ver \cite{DaiSean}.
\end{itemize}
\end{Note}

\begin{Def}\label{Def.Cto.Peq.}
Un conjunto $D\in\mathcal{B_{\mathbb{X}}}$ es llamado peque\~no si
existe un $t>0$, una medida de probabilidad $\nu$ en
$\mathcal{B_{\mathbb{X}}}$, y un $\delta>0$ tal que
\[P^{t}\left(x,A\right)\geq\delta\nu\left(A\right),\] para $x\in
D,A\in\mathcal{B_{\mathbb{X}}}$.
\end{Def}

La siguiente serie de resultados vienen enunciados y demostrados
en Dai \cite{Dai}:
\begin{Lema}[Lema 3.1, Dai \cite{Dai}]
Sea $B$ conjunto peque\~no cerrado, supongamos que
$P_{x}\left(\tau_{B}<\infty\right)\equiv1$ y que para alg\'un
$\delta>0$ se cumple que
\begin{equation}\label{Eq.3.1}
\sup\esp_{x}\left[\tau_{B}\left(\delta\right)\right]<\infty,
\end{equation}
donde
$\tau_{B}\left(\delta\right)=inf\left\{t\geq\delta:X\left(t\right)\in
B\right\}$. Entonces, $X$ es un proceso Harris recurrente
positivo.
\end{Lema}

\begin{Lema}[Lema 3.1, Dai \cite{Dai}]\label{Lema.3.}
Bajo el supuesto (A3), el conjunto
$B=\left\{x\in\mathbb{X}:|x|\leq k\right\}$ es un conjunto
peque\~no cerrado para cualquier $k>0$.
\end{Lema}

\begin{Teo}[Teorema 3.1, Dai \cite{Dai}]\label{Tma.3.1}
Si existe un $\delta>0$ tal que
\begin{equation}
lim_{|x|\rightarrow\infty}\frac{1}{|x|}\esp|X^{x}\left(|x|\delta\right)|=0,
\end{equation}
donde $X^{x}$ se utiliza para denotar que el proceso $X$ comienza
a partir de $x$, entonces la ecuaci\'on (\ref{Eq.3.1}) se cumple
para $B=\left\{x\in\mathbb{X}:|x|\leq k\right\}$ con alg\'un
$k>0$. En particular, $X$ es Harris recurrente positivo.
\end{Teo}

Entonces, tenemos que el proceso $X$ es un proceso de Markov que
cumple con los supuestos $A1)$-$A3)$, lo que falta de hacer es
construir el Modelo de Flujo bas\'andonos en lo hasta ahora
presentado.
%_______________________________________________________________________
\subsection{Modelo de Flujo}
%_______________________________________________________________________

Dada una condici\'on inicial $x\in\mathbb{X}$, sea

\begin{itemize}
\item $Q_{k}^{x}\left(t\right)$ la longitud de la cola al tiempo
$t$,

\item $T_{m,k}^{x}\left(t\right)$ el tiempo acumulado, al tiempo
$t$, que tarda el servidor $m$ en atender a los usuarios de la
cola $k$.

\item $T_{m,k}^{x,0}\left(t\right)$ el tiempo acumulado, al tiempo
$t$, que tarda el servidor $m$ en trasladarse a otra cola a partir de la $k$-\'esima.\\
\end{itemize}

Sup\'ongase que la funci\'on
$\left(\overline{Q}\left(\cdot\right),\overline{T}_{m}
\left(\cdot\right),\overline{T}_{m}^{0} \left(\cdot\right)\right)$
para $m=1,2,\ldots,M$ es un punto l\'imite de
\begin{equation}\label{Eq.Punto.Limite}
\left(\frac{1}{|x|}Q^{x}\left(|x|t\right),\frac{1}{|x|}T_{m}^{x}\left(|x|t\right),\frac{1}{|x|}T_{m}^{x,0}\left(|x|t\right)\right)
\end{equation}
para $m=1,2,\ldots,M$, cuando $x\rightarrow\infty$, ver
\cite{Down}. Entonces
$\left(\overline{Q}\left(t\right),\overline{T}_{m}
\left(t\right),\overline{T}_{m}^{0} \left(t\right)\right)$ es un
flujo l\'imite del sistema. Al conjunto de todos las posibles
flujos l\'imite se le llama {\emph{Modelo de Flujo}} y se le
denotar\'a por $\mathcal{Q}$, ver \cite{Down, Dai, DaiSean}.\\

El modelo de flujo satisface el siguiente conjunto de ecuaciones:

\begin{equation}\label{Eq.MF.1}
\overline{Q}_{k}\left(t\right)=\overline{Q}_{k}\left(0\right)+\lambda_{k}t-\sum_{m=1}^{M}\mu_{k}\overline{T}_{m,k}\left(t\right),\\
\end{equation}
para $k=1,2,\ldots,K$.\\
\begin{equation}\label{Eq.MF.2}
\overline{Q}_{k}\left(t\right)\geq0\textrm{ para
}k=1,2,\ldots,K.\\
\end{equation}

\begin{equation}\label{Eq.MF.3}
\overline{T}_{m,k}\left(0\right)=0,\textrm{ y }\overline{T}_{m,k}\left(\cdot\right)\textrm{ es no decreciente},\\
\end{equation}
para $k=1,2,\ldots,K$ y $m=1,2,\ldots,M$.\\
\begin{equation}\label{Eq.MF.4}
\sum_{k=1}^{K}\overline{T}_{m,k}^{0}\left(t\right)+\overline{T}_{m,k}\left(t\right)=t\textrm{
para }m=1,2,\ldots,M.\\
\end{equation}


\begin{Def}[Definici\'on 4.1, Dai \cite{Dai}]\label{Def.Modelo.Flujo}
Sea una disciplina de servicio espec\'ifica. Cualquier l\'imite
$\left(\overline{Q}\left(\cdot\right),\overline{T}\left(\cdot\right),\overline{T}^{0}\left(\cdot\right)\right)$
en (\ref{Eq.Punto.Limite}) es un {\em flujo l\'imite} de la
disciplina. Cualquier soluci\'on (\ref{Eq.MF.1})-(\ref{Eq.MF.4})
es llamado flujo soluci\'on de la disciplina.
\end{Def}

\begin{Def}
Se dice que el modelo de flujo l\'imite, modelo de flujo, de la
disciplina de la cola es estable si existe una constante
$\delta>0$ que depende de $\mu,\lambda$ y $P$ solamente, tal que
cualquier flujo l\'imite con
$|\overline{Q}\left(0\right)|+|\overline{U}|+|\overline{V}|=1$, se
tiene que $\overline{Q}\left(\cdot+\delta\right)\equiv0$.
\end{Def}

Si se hace $|x|\rightarrow\infty$ sin restringir ninguna de las
componentes, tambi\'en se obtienen un modelo de flujo, pero en
este caso el residual de los procesos de arribo y servicio
introducen un retraso:
\begin{Teo}[Teorema 4.2, Dai \cite{Dai}]\label{Tma.4.2.Dai}
Sea una disciplina fija para la cola, suponga que se cumplen las
condiciones (A1)-(A3). Si el modelo de flujo l\'imite de la
disciplina de la cola es estable, entonces la cadena de Markov $X$
que describe la din\'amica de la red bajo la disciplina es Harris
recurrente positiva.
\end{Teo}

Ahora se procede a escalar el espacio y el tiempo para reducir la
aparente fluctuaci\'on del modelo. Consid\'erese el proceso
\begin{equation}\label{Eq.3.7}
\overline{Q}^{x}\left(t\right)=\frac{1}{|x|}Q^{x}\left(|x|t\right).
\end{equation}
A este proceso se le conoce como el flujo escalado, y cualquier
l\'imite $\overline{Q}^{x}\left(t\right)$ es llamado flujo
l\'imite del proceso de longitud de la cola. Haciendo
$|q|\rightarrow\infty$ mientras se mantiene el resto de las
componentes fijas, cualquier punto l\'imite del proceso de
longitud de la cola normalizado $\overline{Q}^{x}$ es soluci\'on
del siguiente modelo de flujo.


\begin{Def}[Definici\'on 3.3, Dai y Meyn \cite{DaiSean}]
El modelo de flujo es estable si existe un tiempo fijo $t_{0}$ tal
que $\overline{Q}\left(t\right)=0$, con $t\geq t_{0}$, para
cualquier $\overline{Q}\left(\cdot\right)\in\mathcal{Q}$ que
cumple con $|\overline{Q}\left(0\right)|=1$.
\end{Def}

\begin{Lemma}[Lema 3.1, Dai y Meyn \cite{DaiSean}]
Si el modelo de flujo definido por (\ref{Eq.MF.1})-(\ref{Eq.MF.4})
es estable, entonces el modelo de flujo retrasado es tambi\'en
estable, es decir, existe $t_{0}>0$ tal que
$\overline{Q}\left(t\right)=0$ para cualquier $t\geq t_{0}$, para
cualquier soluci\'on del modelo de flujo retrasado cuya
condici\'on inicial $\overline{x}$ satisface que
$|\overline{x}|=|\overline{Q}\left(0\right)|+|\overline{A}\left(0\right)|+|\overline{B}\left(0\right)|\leq1$.
\end{Lemma}


Ahora ya estamos en condiciones de enunciar los resultados principales:


\begin{Teo}[Teorema 2.1, Down \cite{Down}]\label{Tma2.1.Down}
Suponga que el modelo de flujo es estable, y que se cumplen los supuestos (A1) y (A2), entonces
\begin{itemize}
\item[i)] Para alguna constante $\kappa_{p}$, y para cada
condici\'on inicial $x\in X$
\begin{equation}\label{Estability.Eq1}
\limsup_{t\rightarrow\infty}\frac{1}{t}\int_{0}^{t}\esp_{x}\left[|Q\left(s\right)|^{p}\right]ds\leq\kappa_{p},
\end{equation}
donde $p$ es el entero dado en (A2).
\end{itemize}
Si adem\'as se cumple la condici\'on (A3), entonces para cada
condici\'on inicial:
\begin{itemize}
\item[ii)] Los momentos transitorios convergen a su estado
estacionario:
 \begin{equation}\label{Estability.Eq2}
lim_{t\rightarrow\infty}\esp_{x}\left[Q_{k}\left(t\right)^{r}\right]=\esp_{\pi}\left[Q_{k}\left(0\right)^{r}\right]\leq\kappa_{r},
\end{equation}
para $r=1,2,\ldots,p$ y $k=1,2,\ldots,K$. Donde $\pi$ es la
probabilidad invariante para $X$.

\item[iii)]  El primer momento converge con raz\'on $t^{p-1}$:
\begin{equation}\label{Estability.Eq3}
lim_{t\rightarrow\infty}t^{p-1}|\esp_{x}\left[Q_{k}\left(t\right)\right]-\esp_{\pi}\left[Q_{k}\left(0\right)\right]|=0.
\end{equation}

\item[iv)] La {\em Ley Fuerte de los grandes n\'umeros} se cumple:
\begin{equation}\label{Estability.Eq4}
lim_{t\rightarrow\infty}\frac{1}{t}\int_{0}^{t}Q_{k}^{r}\left(s\right)ds=\esp_{\pi}\left[Q_{k}\left(0\right)^{r}\right],\textrm{
}\prob_{x}\textrm{-c.s.}
\end{equation}
para $r=1,2,\ldots,p$ y $k=1,2,\ldots,K$.
\end{itemize}
\end{Teo}

La contribuci\'on de Down a la teor\'ia de los {\emph {sistemas de
visitas c\'iclicas}}, es la relaci\'on que hay entre la
estabilidad del sistema con el comportamiento de las medidas de
desempe\~no, es decir, la condici\'on suficiente para poder
garantizar la convergencia del proceso de la longitud de la cola
as\'i como de por los menos los dos primeros momentos adem\'as de
una versi\'on de la Ley Fuerte de los Grandes N\'umeros para los
sistemas de visitas.


\begin{Teo}[Teorema 2.3, Down \cite{Down}]\label{Tma2.3.Down}
Considere el siguiente valor:
\begin{equation}\label{Eq.Rho.1serv}
\rho=\sum_{k=1}^{K}\rho_{k}+max_{1\leq j\leq K}\left(\frac{\lambda_{j}}{\sum_{s=1}^{S}p_{js}\overline{N}_{s}}\right)\delta^{*}
\end{equation}
\begin{itemize}
\item[i)] Si $\rho<1$ entonces la red es estable, es decir, se
cumple el Teorema \ref{Tma2.1.Down}.

\item[ii)] Si $\rho>1$ entonces la red es inestable, es decir, se
cumple el Teorema \ref{Tma2.2.Down}
\end{itemize}
\end{Teo}



El sistema aqu\'i descrito debe de cumplir con los siguientes supuestos b\'asicos de un sistema de visitas:
%__________________________________________________________________________
\subsubsection{Supuestos B\'asicos}
%__________________________________________________________________________
\begin{itemize}
\item[A1)] Los procesos
$\xi_{1},\ldots,\xi_{K},\eta_{1},\ldots,\eta_{K}$ son mutuamente
independientes y son sucesiones independientes e id\'enticamente
distribuidas.

\item[A2)] Para alg\'un entero $p\geq1$
\begin{eqnarray*}
\esp\left[\xi_{l}\left(1\right)^{p+1}\right]&<&\infty\textrm{ para }l=1,\ldots,K\textrm{ y }\\
\esp\left[\eta_{k}\left(1\right)^{p+1}\right]&<&\infty\textrm{
para }k=1,\ldots,K.
\end{eqnarray*}
donde $\mathcal{A}$ es la clase de posibles arribos.

\item[A3)] Para cada $k=1,2,\ldots,K$ existe una funci\'on
positiva $q_{k}\left(\cdot\right)$ definida en $\rea_{+}$, y un
entero $j_{k}$, tal que
\begin{eqnarray}
P\left(\xi_{k}\left(1\right)\geq x\right)&>&0\textrm{, para todo }x>0,\\
P\left\{a\leq\sum_{i=1}^{j_{k}}\xi_{k}\left(i\right)\leq
b\right\}&\geq&\int_{a}^{b}q_{k}\left(x\right)dx, \textrm{ }0\leq
a<b.
\end{eqnarray}
\end{itemize}

En lo que respecta al supuesto (A3), en Dai y Meyn \cite{DaiSean}
hacen ver que este se puede sustituir por

\begin{itemize}
\item[A3')] Para el Proceso de Markov $X$, cada subconjunto
compacto del espacio de estados de $X$ es un conjunto peque\~no,
ver definici\'on \ref{Def.Cto.Peq.}.
\end{itemize}

Es por esta raz\'on que con la finalidad de poder hacer uso de
$A3^{'})$ es necesario recurrir a los Procesos de Harris y en
particular a los Procesos Harris Recurrente, ver \cite{Dai,
DaiSean}.
%_______________________________________________________________________
\subsection{Procesos Harris Recurrente}
%_______________________________________________________________________

Por el supuesto (A1) conforme a Davis \cite{Davis}, se puede
definir el proceso de saltos correspondiente de manera tal que
satisfaga el supuesto (A3'), de hecho la demostraci\'on est\'a
basada en la l\'inea de argumentaci\'on de Davis, \cite{Davis},
p\'aginas 362-364.\\

Entonces se tiene un espacio de estados en el cual el proceso $X$
satisface la Propiedad Fuerte de Markov, ver Dai y Meyn
\cite{DaiSean}, dado por

\[\left(\Omega,\mathcal{F},\mathcal{F}_{t},X\left(t\right),\theta_{t},P_{x}\right),\]
adem\'as de ser un proceso de Borel Derecho (Sharpe \cite{Sharpe})
en el espacio de estados medible
$\left(\mathbb{X},\mathcal{B}_\mathbb{X}\right)$. El Proceso
$X=\left\{X\left(t\right),t\geq0\right\}$ tiene trayectorias
continuas por la derecha, est\'a definido en
$\left(\Omega,\mathcal{F}\right)$ y est\'a adaptado a
$\left\{\mathcal{F}_{t},t\geq0\right\}$; la colecci\'on
$\left\{P_{x},x\in \mathbb{X}\right\}$ son medidas de probabilidad
en $\left(\Omega,\mathcal{F}\right)$ tales que para todo $x\in
\mathbb{X}$
\[P_{x}\left\{X\left(0\right)=x\right\}=1,\] y
\[E_{x}\left\{f\left(X\circ\theta_{t}\right)|\mathcal{F}_{t}\right\}=E_{X}\left(\tau\right)f\left(X\right),\]
en $\left\{\tau<\infty\right\}$, $P_{x}$-c.s., con $\theta_{t}$
definido como el operador shift.


Donde $\tau$ es un $\mathcal{F}_{t}$-tiempo de paro
\[\left(X\circ\theta_{\tau}\right)\left(w\right)=\left\{X\left(\tau\left(w\right)+t,w\right),t\geq0\right\},\]
y $f$ es una funci\'on de valores reales acotada y medible, ver \cite{Dai, KaspiMandelbaum}.\\

Sea $P^{t}\left(x,D\right)$, $D\in\mathcal{B}_{\mathbb{X}}$,
$t\geq0$ la probabilidad de transici\'on de $X$ queda definida
como:
\[P^{t}\left(x,D\right)=P_{x}\left(X\left(t\right)\in
D\right).\]


\begin{Def}
Una medida no cero $\pi$ en
$\left(\mathbb{X},\mathcal{B}_{\mathbb{X}}\right)$ es invariante
para $X$ si $\pi$ es $\sigma$-finita y
\[\pi\left(D\right)=\int_{\mathbb{X}}P^{t}\left(x,D\right)\pi\left(dx\right),\]
para todo $D\in \mathcal{B}_{\mathbb{X}}$, con $t\geq0$.
\end{Def}

\begin{Def}
El proceso de Markov $X$ es llamado Harris Recurrente si existe
una medida de probabilidad $\nu$ en
$\left(\mathbb{X},\mathcal{B}_{\mathbb{X}}\right)$, tal que si
$\nu\left(D\right)>0$ y $D\in\mathcal{B}_{\mathbb{X}}$
\[P_{x}\left\{\tau_{D}<\infty\right\}\equiv1,\] cuando
$\tau_{D}=inf\left\{t\geq0:X_{t}\in D\right\}$.
\end{Def}

\begin{Note}
\begin{itemize}
\item[i)] Si $X$ es Harris recurrente, entonces existe una \'unica
medida invariante $\pi$ (Getoor \cite{Getoor}).

\item[ii)] Si la medida invariante es finita, entonces puede
normalizarse a una medida de probabilidad, en este caso al proceso
$X$ se le llama Harris recurrente positivo.


\item[iii)] Cuando $X$ es Harris recurrente positivo se dice que
la disciplina de servicio es estable. En este caso $\pi$ denota la
distribuci\'on estacionaria y hacemos
\[P_{\pi}\left(\cdot\right)=\int_{\mathbf{X}}P_{x}\left(\cdot\right)\pi\left(dx\right),\]
y se utiliza $E_{\pi}$ para denotar el operador esperanza
correspondiente, ver \cite{DaiSean}.
\end{itemize}
\end{Note}

\begin{Def}\label{Def.Cto.Peq.}
Un conjunto $D\in\mathcal{B_{\mathbb{X}}}$ es llamado peque\~no si
existe un $t>0$, una medida de probabilidad $\nu$ en
$\mathcal{B_{\mathbb{X}}}$, y un $\delta>0$ tal que
\[P^{t}\left(x,A\right)\geq\delta\nu\left(A\right),\] para $x\in
D,A\in\mathcal{B_{\mathbb{X}}}$.
\end{Def}

La siguiente serie de resultados vienen enunciados y demostrados
en Dai \cite{Dai}:
\begin{Lema}[Lema 3.1, Dai \cite{Dai}]
Sea $B$ conjunto peque\~no cerrado, supongamos que
$P_{x}\left(\tau_{B}<\infty\right)\equiv1$ y que para alg\'un
$\delta>0$ se cumple que
\begin{equation}\label{Eq.3.1}
\sup\esp_{x}\left[\tau_{B}\left(\delta\right)\right]<\infty,
\end{equation}
donde
$\tau_{B}\left(\delta\right)=inf\left\{t\geq\delta:X\left(t\right)\in
B\right\}$. Entonces, $X$ es un proceso Harris recurrente
positivo.
\end{Lema}

\begin{Lema}[Lema 3.1, Dai \cite{Dai}]\label{Lema.3.}
Bajo el supuesto (A3), el conjunto
$B=\left\{x\in\mathbb{X}:|x|\leq k\right\}$ es un conjunto
peque\~no cerrado para cualquier $k>0$.
\end{Lema}

\begin{Teo}[Teorema 3.1, Dai \cite{Dai}]\label{Tma.3.1}
Si existe un $\delta>0$ tal que
\begin{equation}
lim_{|x|\rightarrow\infty}\frac{1}{|x|}\esp|X^{x}\left(|x|\delta\right)|=0,
\end{equation}
donde $X^{x}$ se utiliza para denotar que el proceso $X$ comienza
a partir de $x$, entonces la ecuaci\'on (\ref{Eq.3.1}) se cumple
para $B=\left\{x\in\mathbb{X}:|x|\leq k\right\}$ con alg\'un
$k>0$. En particular, $X$ es Harris recurrente positivo.
\end{Teo}

Entonces, tenemos que el proceso $X$ es un proceso de Markov que
cumple con los supuestos $A1)$-$A3)$, lo que falta de hacer es
construir el Modelo de Flujo bas\'andonos en lo hasta ahora
presentado.
%_______________________________________________________________________
\subsection{Modelo de Flujo}
%_______________________________________________________________________

Dada una condici\'on inicial $x\in\mathbb{X}$, sea

\begin{itemize}
\item $Q_{k}^{x}\left(t\right)$ la longitud de la cola al tiempo
$t$,

\item $T_{m,k}^{x}\left(t\right)$ el tiempo acumulado, al tiempo
$t$, que tarda el servidor $m$ en atender a los usuarios de la
cola $k$.

\item $T_{m,k}^{x,0}\left(t\right)$ el tiempo acumulado, al tiempo
$t$, que tarda el servidor $m$ en trasladarse a otra cola a partir de la $k$-\'esima.\\
\end{itemize}

Sup\'ongase que la funci\'on
$\left(\overline{Q}\left(\cdot\right),\overline{T}_{m}
\left(\cdot\right),\overline{T}_{m}^{0} \left(\cdot\right)\right)$
para $m=1,2,\ldots,M$ es un punto l\'imite de
\begin{equation}\label{Eq.Punto.Limite}
\left(\frac{1}{|x|}Q^{x}\left(|x|t\right),\frac{1}{|x|}T_{m}^{x}\left(|x|t\right),\frac{1}{|x|}T_{m}^{x,0}\left(|x|t\right)\right)
\end{equation}
para $m=1,2,\ldots,M$, cuando $x\rightarrow\infty$, ver
\cite{Down}. Entonces
$\left(\overline{Q}\left(t\right),\overline{T}_{m}
\left(t\right),\overline{T}_{m}^{0} \left(t\right)\right)$ es un
flujo l\'imite del sistema. Al conjunto de todos las posibles
flujos l\'imite se le llama {\emph{Modelo de Flujo}} y se le
denotar\'a por $\mathcal{Q}$, ver \cite{Down, Dai, DaiSean}.\\

El modelo de flujo satisface el siguiente conjunto de ecuaciones:

\begin{equation}\label{Eq.MF.1}
\overline{Q}_{k}\left(t\right)=\overline{Q}_{k}\left(0\right)+\lambda_{k}t-\sum_{m=1}^{M}\mu_{k}\overline{T}_{m,k}\left(t\right),\\
\end{equation}
para $k=1,2,\ldots,K$.\\
\begin{equation}\label{Eq.MF.2}
\overline{Q}_{k}\left(t\right)\geq0\textrm{ para
}k=1,2,\ldots,K.\\
\end{equation}

\begin{equation}\label{Eq.MF.3}
\overline{T}_{m,k}\left(0\right)=0,\textrm{ y }\overline{T}_{m,k}\left(\cdot\right)\textrm{ es no decreciente},\\
\end{equation}
para $k=1,2,\ldots,K$ y $m=1,2,\ldots,M$.\\
\begin{equation}\label{Eq.MF.4}
\sum_{k=1}^{K}\overline{T}_{m,k}^{0}\left(t\right)+\overline{T}_{m,k}\left(t\right)=t\textrm{
para }m=1,2,\ldots,M.\\
\end{equation}


\begin{Def}[Definici\'on 4.1, Dai \cite{Dai}]\label{Def.Modelo.Flujo}
Sea una disciplina de servicio espec\'ifica. Cualquier l\'imite
$\left(\overline{Q}\left(\cdot\right),\overline{T}\left(\cdot\right),\overline{T}^{0}\left(\cdot\right)\right)$
en (\ref{Eq.Punto.Limite}) es un {\em flujo l\'imite} de la
disciplina. Cualquier soluci\'on (\ref{Eq.MF.1})-(\ref{Eq.MF.4})
es llamado flujo soluci\'on de la disciplina.
\end{Def}

\begin{Def}
Se dice que el modelo de flujo l\'imite, modelo de flujo, de la
disciplina de la cola es estable si existe una constante
$\delta>0$ que depende de $\mu,\lambda$ y $P$ solamente, tal que
cualquier flujo l\'imite con
$|\overline{Q}\left(0\right)|+|\overline{U}|+|\overline{V}|=1$, se
tiene que $\overline{Q}\left(\cdot+\delta\right)\equiv0$.
\end{Def}

Si se hace $|x|\rightarrow\infty$ sin restringir ninguna de las
componentes, tambi\'en se obtienen un modelo de flujo, pero en
este caso el residual de los procesos de arribo y servicio
introducen un retraso:
\begin{Teo}[Teorema 4.2, Dai \cite{Dai}]\label{Tma.4.2.Dai}
Sea una disciplina fija para la cola, suponga que se cumplen las
condiciones (A1)-(A3). Si el modelo de flujo l\'imite de la
disciplina de la cola es estable, entonces la cadena de Markov $X$
que describe la din\'amica de la red bajo la disciplina es Harris
recurrente positiva.
\end{Teo}

Ahora se procede a escalar el espacio y el tiempo para reducir la
aparente fluctuaci\'on del modelo. Consid\'erese el proceso
\begin{equation}\label{Eq.3.7}
\overline{Q}^{x}\left(t\right)=\frac{1}{|x|}Q^{x}\left(|x|t\right).
\end{equation}
A este proceso se le conoce como el flujo escalado, y cualquier
l\'imite $\overline{Q}^{x}\left(t\right)$ es llamado flujo
l\'imite del proceso de longitud de la cola. Haciendo
$|q|\rightarrow\infty$ mientras se mantiene el resto de las
componentes fijas, cualquier punto l\'imite del proceso de
longitud de la cola normalizado $\overline{Q}^{x}$ es soluci\'on
del siguiente modelo de flujo.


\begin{Def}[Definici\'on 3.3, Dai y Meyn \cite{DaiSean}]
El modelo de flujo es estable si existe un tiempo fijo $t_{0}$ tal
que $\overline{Q}\left(t\right)=0$, con $t\geq t_{0}$, para
cualquier $\overline{Q}\left(\cdot\right)\in\mathcal{Q}$ que
cumple con $|\overline{Q}\left(0\right)|=1$.
\end{Def}

\begin{Lemma}[Lema 3.1, Dai y Meyn \cite{DaiSean}]
Si el modelo de flujo definido por (\ref{Eq.MF.1})-(\ref{Eq.MF.4})
es estable, entonces el modelo de flujo retrasado es tambi\'en
estable, es decir, existe $t_{0}>0$ tal que
$\overline{Q}\left(t\right)=0$ para cualquier $t\geq t_{0}$, para
cualquier soluci\'on del modelo de flujo retrasado cuya
condici\'on inicial $\overline{x}$ satisface que
$|\overline{x}|=|\overline{Q}\left(0\right)|+|\overline{A}\left(0\right)|+|\overline{B}\left(0\right)|\leq1$.
\end{Lemma}


Ahora ya estamos en condiciones de enunciar los resultados principales:


\begin{Teo}[Teorema 2.1, Down \cite{Down}]\label{Tma2.1.Down}
Suponga que el modelo de flujo es estable, y que se cumplen los supuestos (A1) y (A2), entonces
\begin{itemize}
\item[i)] Para alguna constante $\kappa_{p}$, y para cada
condici\'on inicial $x\in X$
\begin{equation}\label{Estability.Eq1}
\limsup_{t\rightarrow\infty}\frac{1}{t}\int_{0}^{t}\esp_{x}\left[|Q\left(s\right)|^{p}\right]ds\leq\kappa_{p},
\end{equation}
donde $p$ es el entero dado en (A2).
\end{itemize}
Si adem\'as se cumple la condici\'on (A3), entonces para cada
condici\'on inicial:
\begin{itemize}
\item[ii)] Los momentos transitorios convergen a su estado
estacionario:
 \begin{equation}\label{Estability.Eq2}
lim_{t\rightarrow\infty}\esp_{x}\left[Q_{k}\left(t\right)^{r}\right]=\esp_{\pi}\left[Q_{k}\left(0\right)^{r}\right]\leq\kappa_{r},
\end{equation}
para $r=1,2,\ldots,p$ y $k=1,2,\ldots,K$. Donde $\pi$ es la
probabilidad invariante para $X$.

\item[iii)]  El primer momento converge con raz\'on $t^{p-1}$:
\begin{equation}\label{Estability.Eq3}
lim_{t\rightarrow\infty}t^{p-1}|\esp_{x}\left[Q_{k}\left(t\right)\right]-\esp_{\pi}\left[Q_{k}\left(0\right)\right]|=0.
\end{equation}

\item[iv)] La {\em Ley Fuerte de los grandes n\'umeros} se cumple:
\begin{equation}\label{Estability.Eq4}
lim_{t\rightarrow\infty}\frac{1}{t}\int_{0}^{t}Q_{k}^{r}\left(s\right)ds=\esp_{\pi}\left[Q_{k}\left(0\right)^{r}\right],\textrm{
}\prob_{x}\textrm{-c.s.}
\end{equation}
para $r=1,2,\ldots,p$ y $k=1,2,\ldots,K$.
\end{itemize}
\end{Teo}

La contribuci\'on de Down a la teor\'ia de los {\emph {sistemas de
visitas c\'iclicas}}, es la relaci\'on que hay entre la
estabilidad del sistema con el comportamiento de las medidas de
desempe\~no, es decir, la condici\'on suficiente para poder
garantizar la convergencia del proceso de la longitud de la cola
as\'i como de por los menos los dos primeros momentos adem\'as de
una versi\'on de la Ley Fuerte de los Grandes N\'umeros para los
sistemas de visitas.


\begin{Teo}[Teorema 2.3, Down \cite{Down}]\label{Tma2.3.Down}
Considere el siguiente valor:
\begin{equation}\label{Eq.Rho.1serv}
\rho=\sum_{k=1}^{K}\rho_{k}+max_{1\leq j\leq K}\left(\frac{\lambda_{j}}{\sum_{s=1}^{S}p_{js}\overline{N}_{s}}\right)\delta^{*}
\end{equation}
\begin{itemize}
\item[i)] Si $\rho<1$ entonces la red es estable, es decir, se
cumple el Teorema \ref{Tma2.1.Down}.

\item[ii)] Si $\rho>1$ entonces la red es inestable, es decir, se
cumple el Teorema \ref{Tma2.2.Down}
\end{itemize}
\end{Teo}


%_________________________________________________________________________
\subsection{Supuestos}
%_________________________________________________________________________
Consideremos el caso en el que se tienen varias colas a las cuales
llegan uno o varios servidores para dar servicio a los usuarios
que se encuentran presentes en la cola, como ya se mencion\'o hay
varios tipos de pol\'iticas de servicio, incluso podr\'ia ocurrir
que la manera en que atiende al resto de las colas sea distinta a
como lo hizo en las anteriores.\\

Para ejemplificar los sistemas de visitas c\'iclicas se
considerar\'a el caso en que a las colas los usuarios son atendidos con
una s\'ola pol\'itica de servicio.\\



Si $\omega$ es el n\'umero de usuarios en la cola al comienzo del
periodo de servicio y $N\left(\omega\right)$ es el n\'umero de
usuarios que son atendidos con una pol\'itica en espec\'ifico
durante el periodo de servicio, entonces se asume que:
\begin{itemize}
\item[1)]\label{S1}$lim_{\omega\rightarrow\infty}\esp\left[N\left(\omega\right)\right]=\overline{N}>0$;
\item[2)]\label{S2}$\esp\left[N\left(\omega\right)\right]\leq\overline{N}$
para cualquier valor de $\omega$.
\end{itemize}
La manera en que atiende el servidor $m$-\'esimo, es la siguiente:
\begin{itemize}
\item Al t\'ermino de la visita a la cola $j$, el servidor cambia
a la cola $j^{'}$ con probabilidad $r_{j,j^{'}}^{m}$

\item La $n$-\'esima vez que el servidor cambia de la cola $j$ a
$j'$, va acompa\~nada con el tiempo de cambio de longitud
$\delta_{j,j^{'}}^{m}\left(n\right)$, con
$\delta_{j,j^{'}}^{m}\left(n\right)$, $n\geq1$, variables
aleatorias independientes e id\'enticamente distribuidas, tales
que $\esp\left[\delta_{j,j^{'}}^{m}\left(1\right)\right]\geq0$.

\item Sea $\left\{p_{j}^{m}\right\}$ la distribuci\'on invariante
estacionaria \'unica para la Cadena de Markov con matriz de
transici\'on $\left(r_{j,j^{'}}^{m}\right)$, se supone que \'esta
existe.

\item Finalmente, se define el tiempo promedio total de traslado
entre las colas como
\begin{equation}
\delta^{*}:=\sum_{j,j^{'}}p_{j}^{m}r_{j,j^{'}}^{m}\esp\left[\delta_{j,j^{'}}^{m}\left(i\right)\right].
\end{equation}
\end{itemize}

Consideremos el caso donde los tiempos entre arribo a cada una de
las colas, $\left\{\xi_{k}\left(n\right)\right\}_{n\geq1}$ son
variables aleatorias independientes a id\'enticamente
distribuidas, y los tiempos de servicio en cada una de las colas
se distribuyen de manera independiente e id\'enticamente
distribuidas $\left\{\eta_{k}\left(n\right)\right\}_{n\geq1}$;
adem\'as ambos procesos cumplen la condici\'on de ser
independientes entre s\'i. Para la $k$-\'esima cola se define la
tasa de arribo por
$\lambda_{k}=1/\esp\left[\xi_{k}\left(1\right)\right]$ y la tasa
de servicio como
$\mu_{k}=1/\esp\left[\eta_{k}\left(1\right)\right]$, finalmente se
define la carga de la cola como $\rho_{k}=\lambda_{k}/\mu_{k}$,
donde se pide que $\rho=\sum_{k=1}^{K}\rho_{k}<1$, para garantizar
la estabilidad del sistema, esto es cierto para las pol\'iticas de
servicio exhaustiva y cerrada, ver Geetor \cite{Getoor}.\\

Si denotamos por
\begin{itemize}
\item $Q_{k}\left(t\right)$ el n\'umero de usuarios presentes en
la cola $k$ al tiempo $t$; \item $A_{k}\left(t\right)$ los
residuales de los tiempos entre arribos a la cola $k$; para cada
servidor $m$; \item $B_{m}\left(t\right)$ denota a los residuales
de los tiempos de servicio al tiempo $t$; \item
$B_{m}^{0}\left(t\right)$ los residuales de los tiempos de
traslado de la cola $k$ a la pr\'oxima por atender al tiempo $t$,

\item sea
$C_{m}\left(t\right)$ el n\'umero de usuarios atendidos durante la
visita del servidor a la cola $k$ al tiempo $t$.
\end{itemize}


En este sentido, el proceso para el sistema de visitas se puede
definir como:

\begin{equation}\label{Esp.Edos.Down}
X\left(t\right)^{T}=\left(Q_{k}\left(t\right),A_{k}\left(t\right),B_{m}\left(t\right),B_{m}^{0}\left(t\right),C_{m}\left(t\right)\right),
\end{equation}
para $k=1,\ldots,K$ y $m=1,2,\ldots,M$, donde $T$ indica que es el
transpuesto del vector que se est\'a definiendo. El proceso $X$
evoluciona en el espacio de estados:
$\mathbb{X}=\ent_{+}^{K}\times\rea_{+}^{K}\times\left(\left\{1,2,\ldots,K\right\}\times\left\{1,2,\ldots,S\right\}\right)^{M}\times\rea_{+}^{K}\times\ent_{+}^{K}$.\\

El sistema aqu\'i descrito debe de cumplir con los siguientes supuestos b\'asicos de un sistema de visitas:
%__________________________________________________________________________
\subsubsection{Supuestos B\'asicos}
%__________________________________________________________________________
\begin{itemize}
\item[A1)] Los procesos
$\xi_{1},\ldots,\xi_{K},\eta_{1},\ldots,\eta_{K}$ son mutuamente
independientes y son sucesiones independientes e id\'enticamente
distribuidas.

\item[A2)] Para alg\'un entero $p\geq1$
\begin{eqnarray*}
\esp\left[\xi_{l}\left(1\right)^{p+1}\right]&<&\infty\textrm{ para }l=1,\ldots,K\textrm{ y }\\
\esp\left[\eta_{k}\left(1\right)^{p+1}\right]&<&\infty\textrm{
para }k=1,\ldots,K.
\end{eqnarray*}
donde $\mathcal{A}$ es la clase de posibles arribos.

\item[A3)] Para cada $k=1,2,\ldots,K$ existe una funci\'on
positiva $q_{k}\left(\cdot\right)$ definida en $\rea_{+}$, y un
entero $j_{k}$, tal que
\begin{eqnarray}
P\left(\xi_{k}\left(1\right)\geq x\right)&>&0\textrm{, para todo }x>0,\\
P\left\{a\leq\sum_{i=1}^{j_{k}}\xi_{k}\left(i\right)\leq
b\right\}&\geq&\int_{a}^{b}q_{k}\left(x\right)dx, \textrm{ }0\leq
a<b.
\end{eqnarray}
\end{itemize}

En lo que respecta al supuesto (A3), en Dai y Meyn \cite{DaiSean}
hacen ver que este se puede sustituir por

\begin{itemize}
\item[A3')] Para el Proceso de Markov $X$, cada subconjunto
compacto del espacio de estados de $X$ es un conjunto peque\~no,
ver definici\'on \ref{Def.Cto.Peq.}.
\end{itemize}

Es por esta raz\'on que con la finalidad de poder hacer uso de
$A3^{'})$ es necesario recurrir a los Procesos de Harris y en
particular a los Procesos Harris Recurrente, ver \cite{Dai,
DaiSean}.
%_______________________________________________________________________
\subsection{Procesos Harris Recurrente}
%_______________________________________________________________________

Por el supuesto (A1) conforme a Davis \cite{Davis}, se puede
definir el proceso de saltos correspondiente de manera tal que
satisfaga el supuesto (A3'), de hecho la demostraci\'on est\'a
basada en la l\'inea de argumentaci\'on de Davis, \cite{Davis},
p\'aginas 362-364.\\

Entonces se tiene un espacio de estados en el cual el proceso $X$
satisface la Propiedad Fuerte de Markov, ver Dai y Meyn
\cite{DaiSean}, dado por

\[\left(\Omega,\mathcal{F},\mathcal{F}_{t},X\left(t\right),\theta_{t},P_{x}\right),\]
adem\'as de ser un proceso de Borel Derecho (Sharpe \cite{Sharpe})
en el espacio de estados medible
$\left(\mathbb{X},\mathcal{B}_\mathbb{X}\right)$. El Proceso
$X=\left\{X\left(t\right),t\geq0\right\}$ tiene trayectorias
continuas por la derecha, est\'a definido en
$\left(\Omega,\mathcal{F}\right)$ y est\'a adaptado a
$\left\{\mathcal{F}_{t},t\geq0\right\}$; la colecci\'on
$\left\{P_{x},x\in \mathbb{X}\right\}$ son medidas de probabilidad
en $\left(\Omega,\mathcal{F}\right)$ tales que para todo $x\in
\mathbb{X}$
\[P_{x}\left\{X\left(0\right)=x\right\}=1,\] y
\[E_{x}\left\{f\left(X\circ\theta_{t}\right)|\mathcal{F}_{t}\right\}=E_{X}\left(\tau\right)f\left(X\right),\]
en $\left\{\tau<\infty\right\}$, $P_{x}$-c.s., con $\theta_{t}$
definido como el operador shift.


Donde $\tau$ es un $\mathcal{F}_{t}$-tiempo de paro
\[\left(X\circ\theta_{\tau}\right)\left(w\right)=\left\{X\left(\tau\left(w\right)+t,w\right),t\geq0\right\},\]
y $f$ es una funci\'on de valores reales acotada y medible, ver \cite{Dai, KaspiMandelbaum}.\\

Sea $P^{t}\left(x,D\right)$, $D\in\mathcal{B}_{\mathbb{X}}$,
$t\geq0$ la probabilidad de transici\'on de $X$ queda definida
como:
\[P^{t}\left(x,D\right)=P_{x}\left(X\left(t\right)\in
D\right).\]


\begin{Def}
Una medida no cero $\pi$ en
$\left(\mathbb{X},\mathcal{B}_{\mathbb{X}}\right)$ es invariante
para $X$ si $\pi$ es $\sigma$-finita y
\[\pi\left(D\right)=\int_{\mathbb{X}}P^{t}\left(x,D\right)\pi\left(dx\right),\]
para todo $D\in \mathcal{B}_{\mathbb{X}}$, con $t\geq0$.
\end{Def}

\begin{Def}
El proceso de Markov $X$ es llamado Harris Recurrente si existe
una medida de probabilidad $\nu$ en
$\left(\mathbb{X},\mathcal{B}_{\mathbb{X}}\right)$, tal que si
$\nu\left(D\right)>0$ y $D\in\mathcal{B}_{\mathbb{X}}$
\[P_{x}\left\{\tau_{D}<\infty\right\}\equiv1,\] cuando
$\tau_{D}=inf\left\{t\geq0:X_{t}\in D\right\}$.
\end{Def}

\begin{Note}
\begin{itemize}
\item[i)] Si $X$ es Harris recurrente, entonces existe una \'unica
medida invariante $\pi$ (Getoor \cite{Getoor}).

\item[ii)] Si la medida invariante es finita, entonces puede
normalizarse a una medida de probabilidad, en este caso al proceso
$X$ se le llama Harris recurrente positivo.


\item[iii)] Cuando $X$ es Harris recurrente positivo se dice que
la disciplina de servicio es estable. En este caso $\pi$ denota la
distribuci\'on estacionaria y hacemos
\[P_{\pi}\left(\cdot\right)=\int_{\mathbf{X}}P_{x}\left(\cdot\right)\pi\left(dx\right),\]
y se utiliza $E_{\pi}$ para denotar el operador esperanza
correspondiente, ver \cite{DaiSean}.
\end{itemize}
\end{Note}

\begin{Def}\label{Def.Cto.Peq.}
Un conjunto $D\in\mathcal{B_{\mathbb{X}}}$ es llamado peque\~no si
existe un $t>0$, una medida de probabilidad $\nu$ en
$\mathcal{B_{\mathbb{X}}}$, y un $\delta>0$ tal que
\[P^{t}\left(x,A\right)\geq\delta\nu\left(A\right),\] para $x\in
D,A\in\mathcal{B_{\mathbb{X}}}$.
\end{Def}

La siguiente serie de resultados vienen enunciados y demostrados
en Dai \cite{Dai}:
\begin{Lema}[Lema 3.1, Dai \cite{Dai}]
Sea $B$ conjunto peque\~no cerrado, supongamos que
$P_{x}\left(\tau_{B}<\infty\right)\equiv1$ y que para alg\'un
$\delta>0$ se cumple que
\begin{equation}\label{Eq.3.1}
\sup\esp_{x}\left[\tau_{B}\left(\delta\right)\right]<\infty,
\end{equation}
donde
$\tau_{B}\left(\delta\right)=inf\left\{t\geq\delta:X\left(t\right)\in
B\right\}$. Entonces, $X$ es un proceso Harris recurrente
positivo.
\end{Lema}

\begin{Lema}[Lema 3.1, Dai \cite{Dai}]\label{Lema.3.}
Bajo el supuesto (A3), el conjunto
$B=\left\{x\in\mathbb{X}:|x|\leq k\right\}$ es un conjunto
peque\~no cerrado para cualquier $k>0$.
\end{Lema}

\begin{Teo}[Teorema 3.1, Dai \cite{Dai}]\label{Tma.3.1}
Si existe un $\delta>0$ tal que
\begin{equation}
lim_{|x|\rightarrow\infty}\frac{1}{|x|}\esp|X^{x}\left(|x|\delta\right)|=0,
\end{equation}
donde $X^{x}$ se utiliza para denotar que el proceso $X$ comienza
a partir de $x$, entonces la ecuaci\'on (\ref{Eq.3.1}) se cumple
para $B=\left\{x\in\mathbb{X}:|x|\leq k\right\}$ con alg\'un
$k>0$. En particular, $X$ es Harris recurrente positivo.
\end{Teo}

Entonces, tenemos que el proceso $X$ es un proceso de Markov que
cumple con los supuestos $A1)$-$A3)$, lo que falta de hacer es
construir el Modelo de Flujo bas\'andonos en lo hasta ahora
presentado.
%_______________________________________________________________________
\subsection{Modelo de Flujo}
%_______________________________________________________________________

Dada una condici\'on inicial $x\in\mathbb{X}$, sea

\begin{itemize}
\item $Q_{k}^{x}\left(t\right)$ la longitud de la cola al tiempo
$t$,

\item $T_{m,k}^{x}\left(t\right)$ el tiempo acumulado, al tiempo
$t$, que tarda el servidor $m$ en atender a los usuarios de la
cola $k$.

\item $T_{m,k}^{x,0}\left(t\right)$ el tiempo acumulado, al tiempo
$t$, que tarda el servidor $m$ en trasladarse a otra cola a partir de la $k$-\'esima.\\
\end{itemize}

Sup\'ongase que la funci\'on
$\left(\overline{Q}\left(\cdot\right),\overline{T}_{m}
\left(\cdot\right),\overline{T}_{m}^{0} \left(\cdot\right)\right)$
para $m=1,2,\ldots,M$ es un punto l\'imite de
\begin{equation}\label{Eq.Punto.Limite}
\left(\frac{1}{|x|}Q^{x}\left(|x|t\right),\frac{1}{|x|}T_{m}^{x}\left(|x|t\right),\frac{1}{|x|}T_{m}^{x,0}\left(|x|t\right)\right)
\end{equation}
para $m=1,2,\ldots,M$, cuando $x\rightarrow\infty$, ver
\cite{Down}. Entonces
$\left(\overline{Q}\left(t\right),\overline{T}_{m}
\left(t\right),\overline{T}_{m}^{0} \left(t\right)\right)$ es un
flujo l\'imite del sistema. Al conjunto de todos las posibles
flujos l\'imite se le llama {\emph{Modelo de Flujo}} y se le
denotar\'a por $\mathcal{Q}$, ver \cite{Down, Dai, DaiSean}.\\

El modelo de flujo satisface el siguiente conjunto de ecuaciones:

\begin{equation}\label{Eq.MF.1}
\overline{Q}_{k}\left(t\right)=\overline{Q}_{k}\left(0\right)+\lambda_{k}t-\sum_{m=1}^{M}\mu_{k}\overline{T}_{m,k}\left(t\right),\\
\end{equation}
para $k=1,2,\ldots,K$.\\
\begin{equation}\label{Eq.MF.2}
\overline{Q}_{k}\left(t\right)\geq0\textrm{ para
}k=1,2,\ldots,K.\\
\end{equation}

\begin{equation}\label{Eq.MF.3}
\overline{T}_{m,k}\left(0\right)=0,\textrm{ y }\overline{T}_{m,k}\left(\cdot\right)\textrm{ es no decreciente},\\
\end{equation}
para $k=1,2,\ldots,K$ y $m=1,2,\ldots,M$.\\
\begin{equation}\label{Eq.MF.4}
\sum_{k=1}^{K}\overline{T}_{m,k}^{0}\left(t\right)+\overline{T}_{m,k}\left(t\right)=t\textrm{
para }m=1,2,\ldots,M.\\
\end{equation}


\begin{Def}[Definici\'on 4.1, Dai \cite{Dai}]\label{Def.Modelo.Flujo}
Sea una disciplina de servicio espec\'ifica. Cualquier l\'imite
$\left(\overline{Q}\left(\cdot\right),\overline{T}\left(\cdot\right),\overline{T}^{0}\left(\cdot\right)\right)$
en (\ref{Eq.Punto.Limite}) es un {\em flujo l\'imite} de la
disciplina. Cualquier soluci\'on (\ref{Eq.MF.1})-(\ref{Eq.MF.4})
es llamado flujo soluci\'on de la disciplina.
\end{Def}

\begin{Def}
Se dice que el modelo de flujo l\'imite, modelo de flujo, de la
disciplina de la cola es estable si existe una constante
$\delta>0$ que depende de $\mu,\lambda$ y $P$ solamente, tal que
cualquier flujo l\'imite con
$|\overline{Q}\left(0\right)|+|\overline{U}|+|\overline{V}|=1$, se
tiene que $\overline{Q}\left(\cdot+\delta\right)\equiv0$.
\end{Def}

Si se hace $|x|\rightarrow\infty$ sin restringir ninguna de las
componentes, tambi\'en se obtienen un modelo de flujo, pero en
este caso el residual de los procesos de arribo y servicio
introducen un retraso:
\begin{Teo}[Teorema 4.2, Dai \cite{Dai}]\label{Tma.4.2.Dai}
Sea una disciplina fija para la cola, suponga que se cumplen las
condiciones (A1)-(A3). Si el modelo de flujo l\'imite de la
disciplina de la cola es estable, entonces la cadena de Markov $X$
que describe la din\'amica de la red bajo la disciplina es Harris
recurrente positiva.
\end{Teo}

Ahora se procede a escalar el espacio y el tiempo para reducir la
aparente fluctuaci\'on del modelo. Consid\'erese el proceso
\begin{equation}\label{Eq.3.7}
\overline{Q}^{x}\left(t\right)=\frac{1}{|x|}Q^{x}\left(|x|t\right).
\end{equation}
A este proceso se le conoce como el flujo escalado, y cualquier
l\'imite $\overline{Q}^{x}\left(t\right)$ es llamado flujo
l\'imite del proceso de longitud de la cola. Haciendo
$|q|\rightarrow\infty$ mientras se mantiene el resto de las
componentes fijas, cualquier punto l\'imite del proceso de
longitud de la cola normalizado $\overline{Q}^{x}$ es soluci\'on
del siguiente modelo de flujo.


\begin{Def}[Definici\'on 3.3, Dai y Meyn \cite{DaiSean}]
El modelo de flujo es estable si existe un tiempo fijo $t_{0}$ tal
que $\overline{Q}\left(t\right)=0$, con $t\geq t_{0}$, para
cualquier $\overline{Q}\left(\cdot\right)\in\mathcal{Q}$ que
cumple con $|\overline{Q}\left(0\right)|=1$.
\end{Def}

\begin{Lemma}[Lema 3.1, Dai y Meyn \cite{DaiSean}]
Si el modelo de flujo definido por (\ref{Eq.MF.1})-(\ref{Eq.MF.4})
es estable, entonces el modelo de flujo retrasado es tambi\'en
estable, es decir, existe $t_{0}>0$ tal que
$\overline{Q}\left(t\right)=0$ para cualquier $t\geq t_{0}$, para
cualquier soluci\'on del modelo de flujo retrasado cuya
condici\'on inicial $\overline{x}$ satisface que
$|\overline{x}|=|\overline{Q}\left(0\right)|+|\overline{A}\left(0\right)|+|\overline{B}\left(0\right)|\leq1$.
\end{Lemma}


Ahora ya estamos en condiciones de enunciar los resultados principales:


\begin{Teo}[Teorema 2.1, Down \cite{Down}]\label{Tma2.1.Down}
Suponga que el modelo de flujo es estable, y que se cumplen los supuestos (A1) y (A2), entonces
\begin{itemize}
\item[i)] Para alguna constante $\kappa_{p}$, y para cada
condici\'on inicial $x\in X$
\begin{equation}\label{Estability.Eq1}
\limsup_{t\rightarrow\infty}\frac{1}{t}\int_{0}^{t}\esp_{x}\left[|Q\left(s\right)|^{p}\right]ds\leq\kappa_{p},
\end{equation}
donde $p$ es el entero dado en (A2).
\end{itemize}
Si adem\'as se cumple la condici\'on (A3), entonces para cada
condici\'on inicial:
\begin{itemize}
\item[ii)] Los momentos transitorios convergen a su estado
estacionario:
 \begin{equation}\label{Estability.Eq2}
lim_{t\rightarrow\infty}\esp_{x}\left[Q_{k}\left(t\right)^{r}\right]=\esp_{\pi}\left[Q_{k}\left(0\right)^{r}\right]\leq\kappa_{r},
\end{equation}
para $r=1,2,\ldots,p$ y $k=1,2,\ldots,K$. Donde $\pi$ es la
probabilidad invariante para $X$.

\item[iii)]  El primer momento converge con raz\'on $t^{p-1}$:
\begin{equation}\label{Estability.Eq3}
lim_{t\rightarrow\infty}t^{p-1}|\esp_{x}\left[Q_{k}\left(t\right)\right]-\esp_{\pi}\left[Q_{k}\left(0\right)\right]|=0.
\end{equation}

\item[iv)] La {\em Ley Fuerte de los grandes n\'umeros} se cumple:
\begin{equation}\label{Estability.Eq4}
lim_{t\rightarrow\infty}\frac{1}{t}\int_{0}^{t}Q_{k}^{r}\left(s\right)ds=\esp_{\pi}\left[Q_{k}\left(0\right)^{r}\right],\textrm{
}\prob_{x}\textrm{-c.s.}
\end{equation}
para $r=1,2,\ldots,p$ y $k=1,2,\ldots,K$.
\end{itemize}
\end{Teo}

La contribuci\'on de Down a la teor\'ia de los {\emph {sistemas de
visitas c\'iclicas}}, es la relaci\'on que hay entre la
estabilidad del sistema con el comportamiento de las medidas de
desempe\~no, es decir, la condici\'on suficiente para poder
garantizar la convergencia del proceso de la longitud de la cola
as\'i como de por los menos los dos primeros momentos adem\'as de
una versi\'on de la Ley Fuerte de los Grandes N\'umeros para los
sistemas de visitas.


\begin{Teo}[Teorema 2.3, Down \cite{Down}]\label{Tma2.3.Down}
Considere el siguiente valor:
\begin{equation}\label{Eq.Rho.1serv}
\rho=\sum_{k=1}^{K}\rho_{k}+max_{1\leq j\leq K}\left(\frac{\lambda_{j}}{\sum_{s=1}^{S}p_{js}\overline{N}_{s}}\right)\delta^{*}
\end{equation}
\begin{itemize}
\item[i)] Si $\rho<1$ entonces la red es estable, es decir, se
cumple el Teorema \ref{Tma2.1.Down}.

\item[ii)] Si $\rho>1$ entonces la red es inestable, es decir, se
cumple el Teorema \ref{Tma2.2.Down}
\end{itemize}
\end{Teo}



%________________________________________________________________________

%________________________________________________________________________
\subsection{Procesos Regenerativos Sigman, Thorisson y Wolff \cite{Sigman2}}
%________________________________________________________________________


\begin{Def}[Definici\'on Cl\'asica]
Un proceso estoc\'astico $X=\left\{X\left(t\right):t\geq0\right\}$ es llamado regenerativo is existe una variable aleatoria $R_{1}>0$ tal que
\begin{itemize}
\item[i)] $\left\{X\left(t+R_{1}\right):t\geq0\right\}$ es independiente de $\left\{\left\{X\left(t\right):t<R_{1}\right\},\right\}$
\item[ii)] $\left\{X\left(t+R_{1}\right):t\geq0\right\}$ es estoc\'asticamente equivalente a $\left\{X\left(t\right):t>0\right\}$
\end{itemize}

Llamamos a $R_{1}$ tiempo de regeneraci\'on, y decimos que $X$ se regenera en este punto.
\end{Def}

$\left\{X\left(t+R_{1}\right)\right\}$ es regenerativo con tiempo de regeneraci\'on $R_{2}$, independiente de $R_{1}$ pero con la misma distribuci\'on que $R_{1}$. Procediendo de esta manera se obtiene una secuencia de variables aleatorias independientes e id\'enticamente distribuidas $\left\{R_{n}\right\}$ llamados longitudes de ciclo. Si definimos a $Z_{k}\equiv R_{1}+R_{2}+\cdots+R_{k}$, se tiene un proceso de renovaci\'on llamado proceso de renovaci\'on encajado para $X$.


\begin{Note}
La existencia de un primer tiempo de regeneraci\'on, $R_{1}$, implica la existencia de una sucesi\'on completa de estos tiempos $R_{1},R_{2}\ldots,$ que satisfacen la propiedad deseada \cite{Sigman2}.
\end{Note}


\begin{Note} Para la cola $GI/GI/1$ los usuarios arriban con tiempos $t_{n}$ y son atendidos con tiempos de servicio $S_{n}$, los tiempos de arribo forman un proceso de renovaci\'on  con tiempos entre arribos independientes e identicamente distribuidos (\texttt{i.i.d.})$T_{n}=t_{n}-t_{n-1}$, adem\'as los tiempos de servicio son \texttt{i.i.d.} e independientes de los procesos de arribo. Por \textit{estable} se entiende que $\esp S_{n}<\esp T_{n}<\infty$.
\end{Note}
 


\begin{Def}
Para $x$ fijo y para cada $t\geq0$, sea $I_{x}\left(t\right)=1$ si $X\left(t\right)\leq x$,  $I_{x}\left(t\right)=0$ en caso contrario, y def\'inanse los tiempos promedio
\begin{eqnarray*}
\overline{X}&=&lim_{t\rightarrow\infty}\frac{1}{t}\int_{0}^{\infty}X\left(u\right)du\\
\prob\left(X_{\infty}\leq x\right)&=&lim_{t\rightarrow\infty}\frac{1}{t}\int_{0}^{\infty}I_{x}\left(u\right)du,
\end{eqnarray*}
cuando estos l\'imites existan.
\end{Def}

Como consecuencia del teorema de Renovaci\'on-Recompensa, se tiene que el primer l\'imite  existe y es igual a la constante
\begin{eqnarray*}
\overline{X}&=&\frac{\esp\left[\int_{0}^{R_{1}}X\left(t\right)dt\right]}{\esp\left[R_{1}\right]},
\end{eqnarray*}
suponiendo que ambas esperanzas son finitas.
 
\begin{Note}
Funciones de procesos regenerativos son regenerativas, es decir, si $X\left(t\right)$ es regenerativo y se define el proceso $Y\left(t\right)$ por $Y\left(t\right)=f\left(X\left(t\right)\right)$ para alguna funci\'on Borel medible $f\left(\cdot\right)$. Adem\'as $Y$ es regenerativo con los mismos tiempos de renovaci\'on que $X$. 

En general, los tiempos de renovaci\'on, $Z_{k}$ de un proceso regenerativo no requieren ser tiempos de paro con respecto a la evoluci\'on de $X\left(t\right)$.
\end{Note} 

\begin{Note}
Una funci\'on de un proceso de Markov, usualmente no ser\'a un proceso de Markov, sin embargo ser\'a regenerativo si el proceso de Markov lo es.
\end{Note}

 
\begin{Note}
Un proceso regenerativo con media de la longitud de ciclo finita es llamado positivo recurrente.
\end{Note}


\begin{Note}
\begin{itemize}
\item[a)] Si el proceso regenerativo $X$ es positivo recurrente y tiene trayectorias muestrales no negativas, entonces la ecuaci\'on anterior es v\'alida.
\item[b)] Si $X$ es positivo recurrente regenerativo, podemos construir una \'unica versi\'on estacionaria de este proceso, $X_{e}=\left\{X_{e}\left(t\right)\right\}$, donde $X_{e}$ es un proceso estoc\'astico regenerativo y estrictamente estacionario, con distribuci\'on marginal distribuida como $X_{\infty}$
\end{itemize}
\end{Note}


%__________________________________________________________________________________________
\subsection{Procesos Regenerativos Estacionarios - Stidham \cite{Stidham}}
%__________________________________________________________________________________________


Un proceso estoc\'astico a tiempo continuo $\left\{V\left(t\right),t\geq0\right\}$ es un proceso regenerativo si existe una sucesi\'on de variables aleatorias independientes e id\'enticamente distribuidas $\left\{X_{1},X_{2},\ldots\right\}$, sucesi\'on de renovaci\'on, tal que para cualquier conjunto de Borel $A$, 

\begin{eqnarray*}
\prob\left\{V\left(t\right)\in A|X_{1}+X_{2}+\cdots+X_{R\left(t\right)}=s,\left\{V\left(\tau\right),\tau<s\right\}\right\}=\prob\left\{V\left(t-s\right)\in A|X_{1}>t-s\right\},
\end{eqnarray*}
para todo $0\leq s\leq t$, donde $R\left(t\right)=\max\left\{X_{1}+X_{2}+\cdots+X_{j}\leq t\right\}=$n\'umero de renovaciones ({\emph{puntos de regeneraci\'on}}) que ocurren en $\left[0,t\right]$. El intervalo $\left[0,X_{1}\right)$ es llamado {\emph{primer ciclo de regeneraci\'on}} de $\left\{V\left(t \right),t\geq0\right\}$, $\left[X_{1},X_{1}+X_{2}\right)$ el {\emph{segundo ciclo de regeneraci\'on}}, y as\'i sucesivamente.

Sea $X=X_{1}$ y sea $F$ la funci\'on de distrbuci\'on de $X$


\begin{Def}
Se define el proceso estacionario, $\left\{V^{*}\left(t\right),t\geq0\right\}$, para $\left\{V\left(t\right),t\geq0\right\}$ por

\begin{eqnarray*}
\prob\left\{V\left(t\right)\in A\right\}=\frac{1}{\esp\left[X\right]}\int_{0}^{\infty}\prob\left\{V\left(t+x\right)\in A|X>x\right\}\left(1-F\left(x\right)\right)dx,
\end{eqnarray*} 
para todo $t\geq0$ y todo conjunto de Borel $A$.
\end{Def}

\begin{Def}
Una distribuci\'on se dice que es {\emph{aritm\'etica}} si todos sus puntos de incremento son m\'ultiplos de la forma $0,\lambda, 2\lambda,\ldots$ para alguna $\lambda>0$ entera.
\end{Def}


\begin{Def}
Una modificaci\'on medible de un proceso $\left\{V\left(t\right),t\geq0\right\}$, es una versi\'on de este, $\left\{V\left(t,w\right)\right\}$ conjuntamente medible para $t\geq0$ y para $w\in S$, $S$ espacio de estados para $\left\{V\left(t\right),t\geq0\right\}$.
\end{Def}

\begin{Teo}
Sea $\left\{V\left(t\right),t\geq\right\}$ un proceso regenerativo no negativo con modificaci\'on medible. Sea $\esp\left[X\right]<\infty$. Entonces el proceso estacionario dado por la ecuaci\'on anterior est\'a bien definido y tiene funci\'on de distribuci\'on independiente de $t$, adem\'as
\begin{itemize}
\item[i)] \begin{eqnarray*}
\esp\left[V^{*}\left(0\right)\right]&=&\frac{\esp\left[\int_{0}^{X}V\left(s\right)ds\right]}{\esp\left[X\right]}\end{eqnarray*}
\item[ii)] Si $\esp\left[V^{*}\left(0\right)\right]<\infty$, equivalentemente, si $\esp\left[\int_{0}^{X}V\left(s\right)ds\right]<\infty$,entonces
\begin{eqnarray*}
\frac{\int_{0}^{t}V\left(s\right)ds}{t}\rightarrow\frac{\esp\left[\int_{0}^{X}V\left(s\right)ds\right]}{\esp\left[X\right]}
\end{eqnarray*}
con probabilidad 1 y en media, cuando $t\rightarrow\infty$.
\end{itemize}
\end{Teo}

\begin{Coro}
Sea $\left\{V\left(t\right),t\geq0\right\}$ un proceso regenerativo no negativo, con modificaci\'on medible. Si $\esp <\infty$, $F$ es no-aritm\'etica, y para todo $x\geq0$, $P\left\{V\left(t\right)\leq x,C>x\right\}$ es de variaci\'on acotada como funci\'on de $t$ en cada intervalo finito $\left[0,\tau\right]$, entonces $V\left(t\right)$ converge en distribuci\'on  cuando $t\rightarrow\infty$ y $$\esp V=\frac{\esp \int_{0}^{X}V\left(s\right)ds}{\esp X}$$
Donde $V$ tiene la distribuci\'on l\'imite de $V\left(t\right)$ cuando $t\rightarrow\infty$.

\end{Coro}

Para el caso discreto se tienen resultados similares.



%______________________________________________________________________
\subsubsection{Procesos de Renovaci\'on}
%______________________________________________________________________

\begin{Def}%\label{Def.Tn}
Sean $0\leq T_{1}\leq T_{2}\leq \ldots$ son tiempos aleatorios infinitos en los cuales ocurren ciertos eventos. El n\'umero de tiempos $T_{n}$ en el intervalo $\left[0,t\right)$ es

\begin{eqnarray}
N\left(t\right)=\sum_{n=1}^{\infty}\indora\left(T_{n}\leq t\right),
\end{eqnarray}
para $t\geq0$.
\end{Def}

Si se consideran los puntos $T_{n}$ como elementos de $\rea_{+}$, y $N\left(t\right)$ es el n\'umero de puntos en $\rea$. El proceso denotado por $\left\{N\left(t\right):t\geq0\right\}$, denotado por $N\left(t\right)$, es un proceso puntual en $\rea_{+}$. Los $T_{n}$ son los tiempos de ocurrencia, el proceso puntual $N\left(t\right)$ es simple si su n\'umero de ocurrencias son distintas: $0<T_{1}<T_{2}<\ldots$ casi seguramente.

\begin{Def}
Un proceso puntual $N\left(t\right)$ es un proceso de renovaci\'on si los tiempos de interocurrencia $\xi_{n}=T_{n}-T_{n-1}$, para $n\geq1$, son independientes e identicamente distribuidos con distribuci\'on $F$, donde $F\left(0\right)=0$ y $T_{0}=0$. Los $T_{n}$ son llamados tiempos de renovaci\'on, referente a la independencia o renovaci\'on de la informaci\'on estoc\'astica en estos tiempos. Los $\xi_{n}$ son los tiempos de inter-renovaci\'on, y $N\left(t\right)$ es el n\'umero de renovaciones en el intervalo $\left[0,t\right)$
\end{Def}


\begin{Note}
Para definir un proceso de renovaci\'on para cualquier contexto, solamente hay que especificar una distribuci\'on $F$, con $F\left(0\right)=0$, para los tiempos de inter-renovaci\'on. La funci\'on $F$ en turno degune las otra variables aleatorias. De manera formal, existe un espacio de probabilidad y una sucesi\'on de variables aleatorias $\xi_{1},\xi_{2},\ldots$ definidas en este con distribuci\'on $F$. Entonces las otras cantidades son $T_{n}=\sum_{k=1}^{n}\xi_{k}$ y $N\left(t\right)=\sum_{n=1}^{\infty}\indora\left(T_{n}\leq t\right)$, donde $T_{n}\rightarrow\infty$ casi seguramente por la Ley Fuerte de los Grandes Números.
\end{Note}

%___________________________________________________________________________________________
%
\subsubsection{Teorema Principal de Renovaci\'on}
%___________________________________________________________________________________________
%

\begin{Note} Una funci\'on $h:\rea_{+}\rightarrow\rea$ es Directamente Riemann Integrable en los siguientes casos:
\begin{itemize}
\item[a)] $h\left(t\right)\geq0$ es decreciente y Riemann Integrable.
\item[b)] $h$ es continua excepto posiblemente en un conjunto de Lebesgue de medida 0, y $|h\left(t\right)|\leq b\left(t\right)$, donde $b$ es DRI.
\end{itemize}
\end{Note}

\begin{Teo}[Teorema Principal de Renovaci\'on]
Si $F$ es no aritm\'etica y $h\left(t\right)$ es Directamente Riemann Integrable (DRI), entonces

\begin{eqnarray*}
lim_{t\rightarrow\infty}U\star h=\frac{1}{\mu}\int_{\rea_{+}}h\left(s\right)ds.
\end{eqnarray*}
\end{Teo}

\begin{Prop}
Cualquier funci\'on $H\left(t\right)$ acotada en intervalos finitos y que es 0 para $t<0$ puede expresarse como
\begin{eqnarray*}
H\left(t\right)=U\star h\left(t\right)\textrm{,  donde }h\left(t\right)=H\left(t\right)-F\star H\left(t\right)
\end{eqnarray*}
\end{Prop}

\begin{Def}
Un proceso estoc\'astico $X\left(t\right)$ es crudamente regenerativo en un tiempo aleatorio positivo $T$ si
\begin{eqnarray*}
\esp\left[X\left(T+t\right)|T\right]=\esp\left[X\left(t\right)\right]\textrm{, para }t\geq0,\end{eqnarray*}
y con las esperanzas anteriores finitas.
\end{Def}

\begin{Prop}
Sup\'ongase que $X\left(t\right)$ es un proceso crudamente regenerativo en $T$, que tiene distribuci\'on $F$. Si $\esp\left[X\left(t\right)\right]$ es acotado en intervalos finitos, entonces
\begin{eqnarray*}
\esp\left[X\left(t\right)\right]=U\star h\left(t\right)\textrm{,  donde }h\left(t\right)=\esp\left[X\left(t\right)\indora\left(T>t\right)\right].
\end{eqnarray*}
\end{Prop}

\begin{Teo}[Regeneraci\'on Cruda]
Sup\'ongase que $X\left(t\right)$ es un proceso con valores positivo crudamente regenerativo en $T$, y def\'inase $M=\sup\left\{|X\left(t\right)|:t\leq T\right\}$. Si $T$ es no aritm\'etico y $M$ y $MT$ tienen media finita, entonces
\begin{eqnarray*}
lim_{t\rightarrow\infty}\esp\left[X\left(t\right)\right]=\frac{1}{\mu}\int_{\rea_{+}}h\left(s\right)ds,
\end{eqnarray*}
donde $h\left(t\right)=\esp\left[X\left(t\right)\indora\left(T>t\right)\right]$.
\end{Teo}

%___________________________________________________________________________________________
%
\subsubsection{Propiedades de los Procesos de Renovaci\'on}
%___________________________________________________________________________________________
%

Los tiempos $T_{n}$ est\'an relacionados con los conteos de $N\left(t\right)$ por

\begin{eqnarray*}
\left\{N\left(t\right)\geq n\right\}&=&\left\{T_{n}\leq t\right\}\\
T_{N\left(t\right)}\leq &t&<T_{N\left(t\right)+1},
\end{eqnarray*}

adem\'as $N\left(T_{n}\right)=n$, y 

\begin{eqnarray*}
N\left(t\right)=\max\left\{n:T_{n}\leq t\right\}=\min\left\{n:T_{n+1}>t\right\}
\end{eqnarray*}

Por propiedades de la convoluci\'on se sabe que

\begin{eqnarray*}
P\left\{T_{n}\leq t\right\}=F^{n\star}\left(t\right)
\end{eqnarray*}
que es la $n$-\'esima convoluci\'on de $F$. Entonces 

\begin{eqnarray*}
\left\{N\left(t\right)\geq n\right\}&=&\left\{T_{n}\leq t\right\}\\
P\left\{N\left(t\right)\leq n\right\}&=&1-F^{\left(n+1\right)\star}\left(t\right)
\end{eqnarray*}

Adem\'as usando el hecho de que $\esp\left[N\left(t\right)\right]=\sum_{n=1}^{\infty}P\left\{N\left(t\right)\geq n\right\}$
se tiene que

\begin{eqnarray*}
\esp\left[N\left(t\right)\right]=\sum_{n=1}^{\infty}F^{n\star}\left(t\right)
\end{eqnarray*}

\begin{Prop}
Para cada $t\geq0$, la funci\'on generadora de momentos $\esp\left[e^{\alpha N\left(t\right)}\right]$ existe para alguna $\alpha$ en una vecindad del 0, y de aqu\'i que $\esp\left[N\left(t\right)^{m}\right]<\infty$, para $m\geq1$.
\end{Prop}


\begin{Note}
Si el primer tiempo de renovaci\'on $\xi_{1}$ no tiene la misma distribuci\'on que el resto de las $\xi_{n}$, para $n\geq2$, a $N\left(t\right)$ se le llama Proceso de Renovaci\'on retardado, donde si $\xi$ tiene distribuci\'on $G$, entonces el tiempo $T_{n}$ de la $n$-\'esima renovaci\'on tiene distribuci\'on $G\star F^{\left(n-1\right)\star}\left(t\right)$
\end{Note}


\begin{Teo}
Para una constante $\mu\leq\infty$ ( o variable aleatoria), las siguientes expresiones son equivalentes:

\begin{eqnarray}
lim_{n\rightarrow\infty}n^{-1}T_{n}&=&\mu,\textrm{ c.s.}\\
lim_{t\rightarrow\infty}t^{-1}N\left(t\right)&=&1/\mu,\textrm{ c.s.}
\end{eqnarray}
\end{Teo}


Es decir, $T_{n}$ satisface la Ley Fuerte de los Grandes N\'umeros s\'i y s\'olo s\'i $N\left/t\right)$ la cumple.


\begin{Coro}[Ley Fuerte de los Grandes N\'umeros para Procesos de Renovaci\'on]
Si $N\left(t\right)$ es un proceso de renovaci\'on cuyos tiempos de inter-renovaci\'on tienen media $\mu\leq\infty$, entonces
\begin{eqnarray}
t^{-1}N\left(t\right)\rightarrow 1/\mu,\textrm{ c.s. cuando }t\rightarrow\infty.
\end{eqnarray}

\end{Coro}


Considerar el proceso estoc\'astico de valores reales $\left\{Z\left(t\right):t\geq0\right\}$ en el mismo espacio de probabilidad que $N\left(t\right)$

\begin{Def}
Para el proceso $\left\{Z\left(t\right):t\geq0\right\}$ se define la fluctuaci\'on m\'axima de $Z\left(t\right)$ en el intervalo $\left(T_{n-1},T_{n}\right]$:
\begin{eqnarray*}
M_{n}=\sup_{T_{n-1}<t\leq T_{n}}|Z\left(t\right)-Z\left(T_{n-1}\right)|
\end{eqnarray*}
\end{Def}

\begin{Teo}
Sup\'ongase que $n^{-1}T_{n}\rightarrow\mu$ c.s. cuando $n\rightarrow\infty$, donde $\mu\leq\infty$ es una constante o variable aleatoria. Sea $a$ una constante o variable aleatoria que puede ser infinita cuando $\mu$ es finita, y considere las expresiones l\'imite:
\begin{eqnarray}
lim_{n\rightarrow\infty}n^{-1}Z\left(T_{n}\right)&=&a,\textrm{ c.s.}\\
lim_{t\rightarrow\infty}t^{-1}Z\left(t\right)&=&a/\mu,\textrm{ c.s.}
\end{eqnarray}
La segunda expresi\'on implica la primera. Conversamente, la primera implica la segunda si el proceso $Z\left(t\right)$ es creciente, o si $lim_{n\rightarrow\infty}n^{-1}M_{n}=0$ c.s.
\end{Teo}

\begin{Coro}
Si $N\left(t\right)$ es un proceso de renovaci\'on, y $\left(Z\left(T_{n}\right)-Z\left(T_{n-1}\right),M_{n}\right)$, para $n\geq1$, son variables aleatorias independientes e id\'enticamente distribuidas con media finita, entonces,
\begin{eqnarray}
lim_{t\rightarrow\infty}t^{-1}Z\left(t\right)\rightarrow\frac{\esp\left[Z\left(T_{1}\right)-Z\left(T_{0}\right)\right]}{\esp\left[T_{1}\right]},\textrm{ c.s. cuando  }t\rightarrow\infty.
\end{eqnarray}
\end{Coro}



%___________________________________________________________________________________________
%
%\subsection{Propiedades de los Procesos de Renovaci\'on}
%___________________________________________________________________________________________
%

Los tiempos $T_{n}$ est\'an relacionados con los conteos de $N\left(t\right)$ por

\begin{eqnarray*}
\left\{N\left(t\right)\geq n\right\}&=&\left\{T_{n}\leq t\right\}\\
T_{N\left(t\right)}\leq &t&<T_{N\left(t\right)+1},
\end{eqnarray*}

adem\'as $N\left(T_{n}\right)=n$, y 

\begin{eqnarray*}
N\left(t\right)=\max\left\{n:T_{n}\leq t\right\}=\min\left\{n:T_{n+1}>t\right\}
\end{eqnarray*}

Por propiedades de la convoluci\'on se sabe que

\begin{eqnarray*}
P\left\{T_{n}\leq t\right\}=F^{n\star}\left(t\right)
\end{eqnarray*}
que es la $n$-\'esima convoluci\'on de $F$. Entonces 

\begin{eqnarray*}
\left\{N\left(t\right)\geq n\right\}&=&\left\{T_{n}\leq t\right\}\\
P\left\{N\left(t\right)\leq n\right\}&=&1-F^{\left(n+1\right)\star}\left(t\right)
\end{eqnarray*}

Adem\'as usando el hecho de que $\esp\left[N\left(t\right)\right]=\sum_{n=1}^{\infty}P\left\{N\left(t\right)\geq n\right\}$
se tiene que

\begin{eqnarray*}
\esp\left[N\left(t\right)\right]=\sum_{n=1}^{\infty}F^{n\star}\left(t\right)
\end{eqnarray*}

\begin{Prop}
Para cada $t\geq0$, la funci\'on generadora de momentos $\esp\left[e^{\alpha N\left(t\right)}\right]$ existe para alguna $\alpha$ en una vecindad del 0, y de aqu\'i que $\esp\left[N\left(t\right)^{m}\right]<\infty$, para $m\geq1$.
\end{Prop}


\begin{Note}
Si el primer tiempo de renovaci\'on $\xi_{1}$ no tiene la misma distribuci\'on que el resto de las $\xi_{n}$, para $n\geq2$, a $N\left(t\right)$ se le llama Proceso de Renovaci\'on retardado, donde si $\xi$ tiene distribuci\'on $G$, entonces el tiempo $T_{n}$ de la $n$-\'esima renovaci\'on tiene distribuci\'on $G\star F^{\left(n-1\right)\star}\left(t\right)$
\end{Note}


\begin{Teo}
Para una constante $\mu\leq\infty$ ( o variable aleatoria), las siguientes expresiones son equivalentes:

\begin{eqnarray}
lim_{n\rightarrow\infty}n^{-1}T_{n}&=&\mu,\textrm{ c.s.}\\
lim_{t\rightarrow\infty}t^{-1}N\left(t\right)&=&1/\mu,\textrm{ c.s.}
\end{eqnarray}
\end{Teo}


Es decir, $T_{n}$ satisface la Ley Fuerte de los Grandes N\'umeros s\'i y s\'olo s\'i $N\left/t\right)$ la cumple.


\begin{Coro}[Ley Fuerte de los Grandes N\'umeros para Procesos de Renovaci\'on]
Si $N\left(t\right)$ es un proceso de renovaci\'on cuyos tiempos de inter-renovaci\'on tienen media $\mu\leq\infty$, entonces
\begin{eqnarray}
t^{-1}N\left(t\right)\rightarrow 1/\mu,\textrm{ c.s. cuando }t\rightarrow\infty.
\end{eqnarray}

\end{Coro}


Considerar el proceso estoc\'astico de valores reales $\left\{Z\left(t\right):t\geq0\right\}$ en el mismo espacio de probabilidad que $N\left(t\right)$

\begin{Def}
Para el proceso $\left\{Z\left(t\right):t\geq0\right\}$ se define la fluctuaci\'on m\'axima de $Z\left(t\right)$ en el intervalo $\left(T_{n-1},T_{n}\right]$:
\begin{eqnarray*}
M_{n}=\sup_{T_{n-1}<t\leq T_{n}}|Z\left(t\right)-Z\left(T_{n-1}\right)|
\end{eqnarray*}
\end{Def}

\begin{Teo}
Sup\'ongase que $n^{-1}T_{n}\rightarrow\mu$ c.s. cuando $n\rightarrow\infty$, donde $\mu\leq\infty$ es una constante o variable aleatoria. Sea $a$ una constante o variable aleatoria que puede ser infinita cuando $\mu$ es finita, y considere las expresiones l\'imite:
\begin{eqnarray}
lim_{n\rightarrow\infty}n^{-1}Z\left(T_{n}\right)&=&a,\textrm{ c.s.}\\
lim_{t\rightarrow\infty}t^{-1}Z\left(t\right)&=&a/\mu,\textrm{ c.s.}
\end{eqnarray}
La segunda expresi\'on implica la primera. Conversamente, la primera implica la segunda si el proceso $Z\left(t\right)$ es creciente, o si $lim_{n\rightarrow\infty}n^{-1}M_{n}=0$ c.s.
\end{Teo}

\begin{Coro}
Si $N\left(t\right)$ es un proceso de renovaci\'on, y $\left(Z\left(T_{n}\right)-Z\left(T_{n-1}\right),M_{n}\right)$, para $n\geq1$, son variables aleatorias independientes e id\'enticamente distribuidas con media finita, entonces,
\begin{eqnarray}
lim_{t\rightarrow\infty}t^{-1}Z\left(t\right)\rightarrow\frac{\esp\left[Z\left(T_{1}\right)-Z\left(T_{0}\right)\right]}{\esp\left[T_{1}\right]},\textrm{ c.s. cuando  }t\rightarrow\infty.
\end{eqnarray}
\end{Coro}


%___________________________________________________________________________________________
%
%\subsection{Propiedades de los Procesos de Renovaci\'on}
%___________________________________________________________________________________________
%

Los tiempos $T_{n}$ est\'an relacionados con los conteos de $N\left(t\right)$ por

\begin{eqnarray*}
\left\{N\left(t\right)\geq n\right\}&=&\left\{T_{n}\leq t\right\}\\
T_{N\left(t\right)}\leq &t&<T_{N\left(t\right)+1},
\end{eqnarray*}

adem\'as $N\left(T_{n}\right)=n$, y 

\begin{eqnarray*}
N\left(t\right)=\max\left\{n:T_{n}\leq t\right\}=\min\left\{n:T_{n+1}>t\right\}
\end{eqnarray*}

Por propiedades de la convoluci\'on se sabe que

\begin{eqnarray*}
P\left\{T_{n}\leq t\right\}=F^{n\star}\left(t\right)
\end{eqnarray*}
que es la $n$-\'esima convoluci\'on de $F$. Entonces 

\begin{eqnarray*}
\left\{N\left(t\right)\geq n\right\}&=&\left\{T_{n}\leq t\right\}\\
P\left\{N\left(t\right)\leq n\right\}&=&1-F^{\left(n+1\right)\star}\left(t\right)
\end{eqnarray*}

Adem\'as usando el hecho de que $\esp\left[N\left(t\right)\right]=\sum_{n=1}^{\infty}P\left\{N\left(t\right)\geq n\right\}$
se tiene que

\begin{eqnarray*}
\esp\left[N\left(t\right)\right]=\sum_{n=1}^{\infty}F^{n\star}\left(t\right)
\end{eqnarray*}

\begin{Prop}
Para cada $t\geq0$, la funci\'on generadora de momentos $\esp\left[e^{\alpha N\left(t\right)}\right]$ existe para alguna $\alpha$ en una vecindad del 0, y de aqu\'i que $\esp\left[N\left(t\right)^{m}\right]<\infty$, para $m\geq1$.
\end{Prop}


\begin{Note}
Si el primer tiempo de renovaci\'on $\xi_{1}$ no tiene la misma distribuci\'on que el resto de las $\xi_{n}$, para $n\geq2$, a $N\left(t\right)$ se le llama Proceso de Renovaci\'on retardado, donde si $\xi$ tiene distribuci\'on $G$, entonces el tiempo $T_{n}$ de la $n$-\'esima renovaci\'on tiene distribuci\'on $G\star F^{\left(n-1\right)\star}\left(t\right)$
\end{Note}


\begin{Teo}
Para una constante $\mu\leq\infty$ ( o variable aleatoria), las siguientes expresiones son equivalentes:

\begin{eqnarray}
lim_{n\rightarrow\infty}n^{-1}T_{n}&=&\mu,\textrm{ c.s.}\\
lim_{t\rightarrow\infty}t^{-1}N\left(t\right)&=&1/\mu,\textrm{ c.s.}
\end{eqnarray}
\end{Teo}


Es decir, $T_{n}$ satisface la Ley Fuerte de los Grandes N\'umeros s\'i y s\'olo s\'i $N\left/t\right)$ la cumple.


\begin{Coro}[Ley Fuerte de los Grandes N\'umeros para Procesos de Renovaci\'on]
Si $N\left(t\right)$ es un proceso de renovaci\'on cuyos tiempos de inter-renovaci\'on tienen media $\mu\leq\infty$, entonces
\begin{eqnarray}
t^{-1}N\left(t\right)\rightarrow 1/\mu,\textrm{ c.s. cuando }t\rightarrow\infty.
\end{eqnarray}

\end{Coro}


Considerar el proceso estoc\'astico de valores reales $\left\{Z\left(t\right):t\geq0\right\}$ en el mismo espacio de probabilidad que $N\left(t\right)$

\begin{Def}
Para el proceso $\left\{Z\left(t\right):t\geq0\right\}$ se define la fluctuaci\'on m\'axima de $Z\left(t\right)$ en el intervalo $\left(T_{n-1},T_{n}\right]$:
\begin{eqnarray*}
M_{n}=\sup_{T_{n-1}<t\leq T_{n}}|Z\left(t\right)-Z\left(T_{n-1}\right)|
\end{eqnarray*}
\end{Def}

\begin{Teo}
Sup\'ongase que $n^{-1}T_{n}\rightarrow\mu$ c.s. cuando $n\rightarrow\infty$, donde $\mu\leq\infty$ es una constante o variable aleatoria. Sea $a$ una constante o variable aleatoria que puede ser infinita cuando $\mu$ es finita, y considere las expresiones l\'imite:
\begin{eqnarray}
lim_{n\rightarrow\infty}n^{-1}Z\left(T_{n}\right)&=&a,\textrm{ c.s.}\\
lim_{t\rightarrow\infty}t^{-1}Z\left(t\right)&=&a/\mu,\textrm{ c.s.}
\end{eqnarray}
La segunda expresi\'on implica la primera. Conversamente, la primera implica la segunda si el proceso $Z\left(t\right)$ es creciente, o si $lim_{n\rightarrow\infty}n^{-1}M_{n}=0$ c.s.
\end{Teo}

\begin{Coro}
Si $N\left(t\right)$ es un proceso de renovaci\'on, y $\left(Z\left(T_{n}\right)-Z\left(T_{n-1}\right),M_{n}\right)$, para $n\geq1$, son variables aleatorias independientes e id\'enticamente distribuidas con media finita, entonces,
\begin{eqnarray}
lim_{t\rightarrow\infty}t^{-1}Z\left(t\right)\rightarrow\frac{\esp\left[Z\left(T_{1}\right)-Z\left(T_{0}\right)\right]}{\esp\left[T_{1}\right]},\textrm{ c.s. cuando  }t\rightarrow\infty.
\end{eqnarray}
\end{Coro}

%___________________________________________________________________________________________
%
%\subsection{Propiedades de los Procesos de Renovaci\'on}
%___________________________________________________________________________________________
%

Los tiempos $T_{n}$ est\'an relacionados con los conteos de $N\left(t\right)$ por

\begin{eqnarray*}
\left\{N\left(t\right)\geq n\right\}&=&\left\{T_{n}\leq t\right\}\\
T_{N\left(t\right)}\leq &t&<T_{N\left(t\right)+1},
\end{eqnarray*}

adem\'as $N\left(T_{n}\right)=n$, y 

\begin{eqnarray*}
N\left(t\right)=\max\left\{n:T_{n}\leq t\right\}=\min\left\{n:T_{n+1}>t\right\}
\end{eqnarray*}

Por propiedades de la convoluci\'on se sabe que

\begin{eqnarray*}
P\left\{T_{n}\leq t\right\}=F^{n\star}\left(t\right)
\end{eqnarray*}
que es la $n$-\'esima convoluci\'on de $F$. Entonces 

\begin{eqnarray*}
\left\{N\left(t\right)\geq n\right\}&=&\left\{T_{n}\leq t\right\}\\
P\left\{N\left(t\right)\leq n\right\}&=&1-F^{\left(n+1\right)\star}\left(t\right)
\end{eqnarray*}

Adem\'as usando el hecho de que $\esp\left[N\left(t\right)\right]=\sum_{n=1}^{\infty}P\left\{N\left(t\right)\geq n\right\}$
se tiene que

\begin{eqnarray*}
\esp\left[N\left(t\right)\right]=\sum_{n=1}^{\infty}F^{n\star}\left(t\right)
\end{eqnarray*}

\begin{Prop}
Para cada $t\geq0$, la funci\'on generadora de momentos $\esp\left[e^{\alpha N\left(t\right)}\right]$ existe para alguna $\alpha$ en una vecindad del 0, y de aqu\'i que $\esp\left[N\left(t\right)^{m}\right]<\infty$, para $m\geq1$.
\end{Prop}


\begin{Note}
Si el primer tiempo de renovaci\'on $\xi_{1}$ no tiene la misma distribuci\'on que el resto de las $\xi_{n}$, para $n\geq2$, a $N\left(t\right)$ se le llama Proceso de Renovaci\'on retardado, donde si $\xi$ tiene distribuci\'on $G$, entonces el tiempo $T_{n}$ de la $n$-\'esima renovaci\'on tiene distribuci\'on $G\star F^{\left(n-1\right)\star}\left(t\right)$
\end{Note}


\begin{Teo}
Para una constante $\mu\leq\infty$ ( o variable aleatoria), las siguientes expresiones son equivalentes:

\begin{eqnarray}
lim_{n\rightarrow\infty}n^{-1}T_{n}&=&\mu,\textrm{ c.s.}\\
lim_{t\rightarrow\infty}t^{-1}N\left(t\right)&=&1/\mu,\textrm{ c.s.}
\end{eqnarray}
\end{Teo}


Es decir, $T_{n}$ satisface la Ley Fuerte de los Grandes N\'umeros s\'i y s\'olo s\'i $N\left/t\right)$ la cumple.


\begin{Coro}[Ley Fuerte de los Grandes N\'umeros para Procesos de Renovaci\'on]
Si $N\left(t\right)$ es un proceso de renovaci\'on cuyos tiempos de inter-renovaci\'on tienen media $\mu\leq\infty$, entonces
\begin{eqnarray}
t^{-1}N\left(t\right)\rightarrow 1/\mu,\textrm{ c.s. cuando }t\rightarrow\infty.
\end{eqnarray}

\end{Coro}


Considerar el proceso estoc\'astico de valores reales $\left\{Z\left(t\right):t\geq0\right\}$ en el mismo espacio de probabilidad que $N\left(t\right)$

\begin{Def}
Para el proceso $\left\{Z\left(t\right):t\geq0\right\}$ se define la fluctuaci\'on m\'axima de $Z\left(t\right)$ en el intervalo $\left(T_{n-1},T_{n}\right]$:
\begin{eqnarray*}
M_{n}=\sup_{T_{n-1}<t\leq T_{n}}|Z\left(t\right)-Z\left(T_{n-1}\right)|
\end{eqnarray*}
\end{Def}

\begin{Teo}
Sup\'ongase que $n^{-1}T_{n}\rightarrow\mu$ c.s. cuando $n\rightarrow\infty$, donde $\mu\leq\infty$ es una constante o variable aleatoria. Sea $a$ una constante o variable aleatoria que puede ser infinita cuando $\mu$ es finita, y considere las expresiones l\'imite:
\begin{eqnarray}
lim_{n\rightarrow\infty}n^{-1}Z\left(T_{n}\right)&=&a,\textrm{ c.s.}\\
lim_{t\rightarrow\infty}t^{-1}Z\left(t\right)&=&a/\mu,\textrm{ c.s.}
\end{eqnarray}
La segunda expresi\'on implica la primera. Conversamente, la primera implica la segunda si el proceso $Z\left(t\right)$ es creciente, o si $lim_{n\rightarrow\infty}n^{-1}M_{n}=0$ c.s.
\end{Teo}

\begin{Coro}
Si $N\left(t\right)$ es un proceso de renovaci\'on, y $\left(Z\left(T_{n}\right)-Z\left(T_{n-1}\right),M_{n}\right)$, para $n\geq1$, son variables aleatorias independientes e id\'enticamente distribuidas con media finita, entonces,
\begin{eqnarray}
lim_{t\rightarrow\infty}t^{-1}Z\left(t\right)\rightarrow\frac{\esp\left[Z\left(T_{1}\right)-Z\left(T_{0}\right)\right]}{\esp\left[T_{1}\right]},\textrm{ c.s. cuando  }t\rightarrow\infty.
\end{eqnarray}
\end{Coro}
%___________________________________________________________________________________________
%
%\subsection{Propiedades de los Procesos de Renovaci\'on}
%___________________________________________________________________________________________
%

Los tiempos $T_{n}$ est\'an relacionados con los conteos de $N\left(t\right)$ por

\begin{eqnarray*}
\left\{N\left(t\right)\geq n\right\}&=&\left\{T_{n}\leq t\right\}\\
T_{N\left(t\right)}\leq &t&<T_{N\left(t\right)+1},
\end{eqnarray*}

adem\'as $N\left(T_{n}\right)=n$, y 

\begin{eqnarray*}
N\left(t\right)=\max\left\{n:T_{n}\leq t\right\}=\min\left\{n:T_{n+1}>t\right\}
\end{eqnarray*}

Por propiedades de la convoluci\'on se sabe que

\begin{eqnarray*}
P\left\{T_{n}\leq t\right\}=F^{n\star}\left(t\right)
\end{eqnarray*}
que es la $n$-\'esima convoluci\'on de $F$. Entonces 

\begin{eqnarray*}
\left\{N\left(t\right)\geq n\right\}&=&\left\{T_{n}\leq t\right\}\\
P\left\{N\left(t\right)\leq n\right\}&=&1-F^{\left(n+1\right)\star}\left(t\right)
\end{eqnarray*}

Adem\'as usando el hecho de que $\esp\left[N\left(t\right)\right]=\sum_{n=1}^{\infty}P\left\{N\left(t\right)\geq n\right\}$
se tiene que

\begin{eqnarray*}
\esp\left[N\left(t\right)\right]=\sum_{n=1}^{\infty}F^{n\star}\left(t\right)
\end{eqnarray*}

\begin{Prop}
Para cada $t\geq0$, la funci\'on generadora de momentos $\esp\left[e^{\alpha N\left(t\right)}\right]$ existe para alguna $\alpha$ en una vecindad del 0, y de aqu\'i que $\esp\left[N\left(t\right)^{m}\right]<\infty$, para $m\geq1$.
\end{Prop}


\begin{Note}
Si el primer tiempo de renovaci\'on $\xi_{1}$ no tiene la misma distribuci\'on que el resto de las $\xi_{n}$, para $n\geq2$, a $N\left(t\right)$ se le llama Proceso de Renovaci\'on retardado, donde si $\xi$ tiene distribuci\'on $G$, entonces el tiempo $T_{n}$ de la $n$-\'esima renovaci\'on tiene distribuci\'on $G\star F^{\left(n-1\right)\star}\left(t\right)$
\end{Note}


\begin{Teo}
Para una constante $\mu\leq\infty$ ( o variable aleatoria), las siguientes expresiones son equivalentes:

\begin{eqnarray}
lim_{n\rightarrow\infty}n^{-1}T_{n}&=&\mu,\textrm{ c.s.}\\
lim_{t\rightarrow\infty}t^{-1}N\left(t\right)&=&1/\mu,\textrm{ c.s.}
\end{eqnarray}
\end{Teo}


Es decir, $T_{n}$ satisface la Ley Fuerte de los Grandes N\'umeros s\'i y s\'olo s\'i $N\left/t\right)$ la cumple.


\begin{Coro}[Ley Fuerte de los Grandes N\'umeros para Procesos de Renovaci\'on]
Si $N\left(t\right)$ es un proceso de renovaci\'on cuyos tiempos de inter-renovaci\'on tienen media $\mu\leq\infty$, entonces
\begin{eqnarray}
t^{-1}N\left(t\right)\rightarrow 1/\mu,\textrm{ c.s. cuando }t\rightarrow\infty.
\end{eqnarray}

\end{Coro}


Considerar el proceso estoc\'astico de valores reales $\left\{Z\left(t\right):t\geq0\right\}$ en el mismo espacio de probabilidad que $N\left(t\right)$

\begin{Def}
Para el proceso $\left\{Z\left(t\right):t\geq0\right\}$ se define la fluctuaci\'on m\'axima de $Z\left(t\right)$ en el intervalo $\left(T_{n-1},T_{n}\right]$:
\begin{eqnarray*}
M_{n}=\sup_{T_{n-1}<t\leq T_{n}}|Z\left(t\right)-Z\left(T_{n-1}\right)|
\end{eqnarray*}
\end{Def}

\begin{Teo}
Sup\'ongase que $n^{-1}T_{n}\rightarrow\mu$ c.s. cuando $n\rightarrow\infty$, donde $\mu\leq\infty$ es una constante o variable aleatoria. Sea $a$ una constante o variable aleatoria que puede ser infinita cuando $\mu$ es finita, y considere las expresiones l\'imite:
\begin{eqnarray}
lim_{n\rightarrow\infty}n^{-1}Z\left(T_{n}\right)&=&a,\textrm{ c.s.}\\
lim_{t\rightarrow\infty}t^{-1}Z\left(t\right)&=&a/\mu,\textrm{ c.s.}
\end{eqnarray}
La segunda expresi\'on implica la primera. Conversamente, la primera implica la segunda si el proceso $Z\left(t\right)$ es creciente, o si $lim_{n\rightarrow\infty}n^{-1}M_{n}=0$ c.s.
\end{Teo}

\begin{Coro}
Si $N\left(t\right)$ es un proceso de renovaci\'on, y $\left(Z\left(T_{n}\right)-Z\left(T_{n-1}\right),M_{n}\right)$, para $n\geq1$, son variables aleatorias independientes e id\'enticamente distribuidas con media finita, entonces,
\begin{eqnarray}
lim_{t\rightarrow\infty}t^{-1}Z\left(t\right)\rightarrow\frac{\esp\left[Z\left(T_{1}\right)-Z\left(T_{0}\right)\right]}{\esp\left[T_{1}\right]},\textrm{ c.s. cuando  }t\rightarrow\infty.
\end{eqnarray}
\end{Coro}


%___________________________________________________________________________________________
%
\subsubsection{Funci\'on de Renovaci\'on}
%___________________________________________________________________________________________
%


\begin{Def}
Sea $h\left(t\right)$ funci\'on de valores reales en $\rea$ acotada en intervalos finitos e igual a cero para $t<0$ La ecuaci\'on de renovaci\'on para $h\left(t\right)$ y la distribuci\'on $F$ es

\begin{eqnarray}%\label{Ec.Renovacion}
H\left(t\right)=h\left(t\right)+\int_{\left[0,t\right]}H\left(t-s\right)dF\left(s\right)\textrm{,    }t\geq0,
\end{eqnarray}
donde $H\left(t\right)$ es una funci\'on de valores reales. Esto es $H=h+F\star H$. Decimos que $H\left(t\right)$ es soluci\'on de esta ecuaci\'on si satisface la ecuaci\'on, y es acotada en intervalos finitos e iguales a cero para $t<0$.
\end{Def}

\begin{Prop}
La funci\'on $U\star h\left(t\right)$ es la \'unica soluci\'on de la ecuaci\'on de renovaci\'on (\ref{Ec.Renovacion}).
\end{Prop}

\begin{Teo}[Teorema Renovaci\'on Elemental]
\begin{eqnarray*}
t^{-1}U\left(t\right)\rightarrow 1/\mu\textrm{,    cuando }t\rightarrow\infty.
\end{eqnarray*}
\end{Teo}

%___________________________________________________________________________________________
%
%\subsection{Funci\'on de Renovaci\'on}
%___________________________________________________________________________________________
%


Sup\'ongase que $N\left(t\right)$ es un proceso de renovaci\'on con distribuci\'on $F$ con media finita $\mu$.

\begin{Def}
La funci\'on de renovaci\'on asociada con la distribuci\'on $F$, del proceso $N\left(t\right)$, es
\begin{eqnarray*}
U\left(t\right)=\sum_{n=1}^{\infty}F^{n\star}\left(t\right),\textrm{   }t\geq0,
\end{eqnarray*}
donde $F^{0\star}\left(t\right)=\indora\left(t\geq0\right)$.
\end{Def}


\begin{Prop}
Sup\'ongase que la distribuci\'on de inter-renovaci\'on $F$ tiene densidad $f$. Entonces $U\left(t\right)$ tambi\'en tiene densidad, para $t>0$, y es $U^{'}\left(t\right)=\sum_{n=0}^{\infty}f^{n\star}\left(t\right)$. Adem\'as
\begin{eqnarray*}
\prob\left\{N\left(t\right)>N\left(t-\right)\right\}=0\textrm{,   }t\geq0.
\end{eqnarray*}
\end{Prop}

\begin{Def}
La Transformada de Laplace-Stieljes de $F$ est\'a dada por

\begin{eqnarray*}
\hat{F}\left(\alpha\right)=\int_{\rea_{+}}e^{-\alpha t}dF\left(t\right)\textrm{,  }\alpha\geq0.
\end{eqnarray*}
\end{Def}

Entonces

\begin{eqnarray*}
\hat{U}\left(\alpha\right)=\sum_{n=0}^{\infty}\hat{F^{n\star}}\left(\alpha\right)=\sum_{n=0}^{\infty}\hat{F}\left(\alpha\right)^{n}=\frac{1}{1-\hat{F}\left(\alpha\right)}.
\end{eqnarray*}


\begin{Prop}
La Transformada de Laplace $\hat{U}\left(\alpha\right)$ y $\hat{F}\left(\alpha\right)$ determina una a la otra de manera \'unica por la relaci\'on $\hat{U}\left(\alpha\right)=\frac{1}{1-\hat{F}\left(\alpha\right)}$.
\end{Prop}


\begin{Note}
Un proceso de renovaci\'on $N\left(t\right)$ cuyos tiempos de inter-renovaci\'on tienen media finita, es un proceso Poisson con tasa $\lambda$ si y s\'olo s\'i $\esp\left[U\left(t\right)\right]=\lambda t$, para $t\geq0$.
\end{Note}


\begin{Teo}
Sea $N\left(t\right)$ un proceso puntual simple con puntos de localizaci\'on $T_{n}$ tal que $\eta\left(t\right)=\esp\left[N\left(\right)\right]$ es finita para cada $t$. Entonces para cualquier funci\'on $f:\rea_{+}\rightarrow\rea$,
\begin{eqnarray*}
\esp\left[\sum_{n=1}^{N\left(\right)}f\left(T_{n}\right)\right]=\int_{\left(0,t\right]}f\left(s\right)d\eta\left(s\right)\textrm{,  }t\geq0,
\end{eqnarray*}
suponiendo que la integral exista. Adem\'as si $X_{1},X_{2},\ldots$ son variables aleatorias definidas en el mismo espacio de probabilidad que el proceso $N\left(t\right)$ tal que $\esp\left[X_{n}|T_{n}=s\right]=f\left(s\right)$, independiente de $n$. Entonces
\begin{eqnarray*}
\esp\left[\sum_{n=1}^{N\left(t\right)}X_{n}\right]=\int_{\left(0,t\right]}f\left(s\right)d\eta\left(s\right)\textrm{,  }t\geq0,
\end{eqnarray*} 
suponiendo que la integral exista. 
\end{Teo}

\begin{Coro}[Identidad de Wald para Renovaciones]
Para el proceso de renovaci\'on $N\left(t\right)$,
\begin{eqnarray*}
\esp\left[T_{N\left(t\right)+1}\right]=\mu\esp\left[N\left(t\right)+1\right]\textrm{,  }t\geq0,
\end{eqnarray*}  
\end{Coro}

%______________________________________________________________________
%\subsection{Procesos de Renovaci\'on}
%______________________________________________________________________

\begin{Def}%\label{Def.Tn}
Sean $0\leq T_{1}\leq T_{2}\leq \ldots$ son tiempos aleatorios infinitos en los cuales ocurren ciertos eventos. El n\'umero de tiempos $T_{n}$ en el intervalo $\left[0,t\right)$ es

\begin{eqnarray}
N\left(t\right)=\sum_{n=1}^{\infty}\indora\left(T_{n}\leq t\right),
\end{eqnarray}
para $t\geq0$.
\end{Def}

Si se consideran los puntos $T_{n}$ como elementos de $\rea_{+}$, y $N\left(t\right)$ es el n\'umero de puntos en $\rea$. El proceso denotado por $\left\{N\left(t\right):t\geq0\right\}$, denotado por $N\left(t\right)$, es un proceso puntual en $\rea_{+}$. Los $T_{n}$ son los tiempos de ocurrencia, el proceso puntual $N\left(t\right)$ es simple si su n\'umero de ocurrencias son distintas: $0<T_{1}<T_{2}<\ldots$ casi seguramente.

\begin{Def}
Un proceso puntual $N\left(t\right)$ es un proceso de renovaci\'on si los tiempos de interocurrencia $\xi_{n}=T_{n}-T_{n-1}$, para $n\geq1$, son independientes e identicamente distribuidos con distribuci\'on $F$, donde $F\left(0\right)=0$ y $T_{0}=0$. Los $T_{n}$ son llamados tiempos de renovaci\'on, referente a la independencia o renovaci\'on de la informaci\'on estoc\'astica en estos tiempos. Los $\xi_{n}$ son los tiempos de inter-renovaci\'on, y $N\left(t\right)$ es el n\'umero de renovaciones en el intervalo $\left[0,t\right)$
\end{Def}


\begin{Note}
Para definir un proceso de renovaci\'on para cualquier contexto, solamente hay que especificar una distribuci\'on $F$, con $F\left(0\right)=0$, para los tiempos de inter-renovaci\'on. La funci\'on $F$ en turno degune las otra variables aleatorias. De manera formal, existe un espacio de probabilidad y una sucesi\'on de variables aleatorias $\xi_{1},\xi_{2},\ldots$ definidas en este con distribuci\'on $F$. Entonces las otras cantidades son $T_{n}=\sum_{k=1}^{n}\xi_{k}$ y $N\left(t\right)=\sum_{n=1}^{\infty}\indora\left(T_{n}\leq t\right)$, donde $T_{n}\rightarrow\infty$ casi seguramente por la Ley Fuerte de los Grandes Números.
\end{Note}

%___________________________________________________________________________________________
%
\subsubsection{Renewal and Regenerative Processes: Serfozo\cite{Serfozo}}
%___________________________________________________________________________________________
%
\begin{Def}%\label{Def.Tn}
Sean $0\leq T_{1}\leq T_{2}\leq \ldots$ son tiempos aleatorios infinitos en los cuales ocurren ciertos eventos. El n\'umero de tiempos $T_{n}$ en el intervalo $\left[0,t\right)$ es

\begin{eqnarray}
N\left(t\right)=\sum_{n=1}^{\infty}\indora\left(T_{n}\leq t\right),
\end{eqnarray}
para $t\geq0$.
\end{Def}

Si se consideran los puntos $T_{n}$ como elementos de $\rea_{+}$, y $N\left(t\right)$ es el n\'umero de puntos en $\rea$. El proceso denotado por $\left\{N\left(t\right):t\geq0\right\}$, denotado por $N\left(t\right)$, es un proceso puntual en $\rea_{+}$. Los $T_{n}$ son los tiempos de ocurrencia, el proceso puntual $N\left(t\right)$ es simple si su n\'umero de ocurrencias son distintas: $0<T_{1}<T_{2}<\ldots$ casi seguramente.

\begin{Def}
Un proceso puntual $N\left(t\right)$ es un proceso de renovaci\'on si los tiempos de interocurrencia $\xi_{n}=T_{n}-T_{n-1}$, para $n\geq1$, son independientes e identicamente distribuidos con distribuci\'on $F$, donde $F\left(0\right)=0$ y $T_{0}=0$. Los $T_{n}$ son llamados tiempos de renovaci\'on, referente a la independencia o renovaci\'on de la informaci\'on estoc\'astica en estos tiempos. Los $\xi_{n}$ son los tiempos de inter-renovaci\'on, y $N\left(t\right)$ es el n\'umero de renovaciones en el intervalo $\left[0,t\right)$
\end{Def}


\begin{Note}
Para definir un proceso de renovaci\'on para cualquier contexto, solamente hay que especificar una distribuci\'on $F$, con $F\left(0\right)=0$, para los tiempos de inter-renovaci\'on. La funci\'on $F$ en turno degune las otra variables aleatorias. De manera formal, existe un espacio de probabilidad y una sucesi\'on de variables aleatorias $\xi_{1},\xi_{2},\ldots$ definidas en este con distribuci\'on $F$. Entonces las otras cantidades son $T_{n}=\sum_{k=1}^{n}\xi_{k}$ y $N\left(t\right)=\sum_{n=1}^{\infty}\indora\left(T_{n}\leq t\right)$, donde $T_{n}\rightarrow\infty$ casi seguramente por la Ley Fuerte de los Grandes N\'umeros.
\end{Note}

Los tiempos $T_{n}$ est\'an relacionados con los conteos de $N\left(t\right)$ por

\begin{eqnarray*}
\left\{N\left(t\right)\geq n\right\}&=&\left\{T_{n}\leq t\right\}\\
T_{N\left(t\right)}\leq &t&<T_{N\left(t\right)+1},
\end{eqnarray*}

adem\'as $N\left(T_{n}\right)=n$, y 

\begin{eqnarray*}
N\left(t\right)=\max\left\{n:T_{n}\leq t\right\}=\min\left\{n:T_{n+1}>t\right\}
\end{eqnarray*}

Por propiedades de la convoluci\'on se sabe que

\begin{eqnarray*}
P\left\{T_{n}\leq t\right\}=F^{n\star}\left(t\right)
\end{eqnarray*}
que es la $n$-\'esima convoluci\'on de $F$. Entonces 

\begin{eqnarray*}
\left\{N\left(t\right)\geq n\right\}&=&\left\{T_{n}\leq t\right\}\\
P\left\{N\left(t\right)\leq n\right\}&=&1-F^{\left(n+1\right)\star}\left(t\right)
\end{eqnarray*}

Adem\'as usando el hecho de que $\esp\left[N\left(t\right)\right]=\sum_{n=1}^{\infty}P\left\{N\left(t\right)\geq n\right\}$
se tiene que

\begin{eqnarray*}
\esp\left[N\left(t\right)\right]=\sum_{n=1}^{\infty}F^{n\star}\left(t\right)
\end{eqnarray*}

\begin{Prop}
Para cada $t\geq0$, la funci\'on generadora de momentos $\esp\left[e^{\alpha N\left(t\right)}\right]$ existe para alguna $\alpha$ en una vecindad del 0, y de aqu\'i que $\esp\left[N\left(t\right)^{m}\right]<\infty$, para $m\geq1$.
\end{Prop}

\begin{Ejem}[\textbf{Proceso Poisson}]

Suponga que se tienen tiempos de inter-renovaci\'on \textit{i.i.d.} del proceso de renovaci\'on $N\left(t\right)$ tienen distribuci\'on exponencial $F\left(t\right)=q-e^{-\lambda t}$ con tasa $\lambda$. Entonces $N\left(t\right)$ es un proceso Poisson con tasa $\lambda$.

\end{Ejem}


\begin{Note}
Si el primer tiempo de renovaci\'on $\xi_{1}$ no tiene la misma distribuci\'on que el resto de las $\xi_{n}$, para $n\geq2$, a $N\left(t\right)$ se le llama Proceso de Renovaci\'on retardado, donde si $\xi$ tiene distribuci\'on $G$, entonces el tiempo $T_{n}$ de la $n$-\'esima renovaci\'on tiene distribuci\'on $G\star F^{\left(n-1\right)\star}\left(t\right)$
\end{Note}


\begin{Teo}
Para una constante $\mu\leq\infty$ ( o variable aleatoria), las siguientes expresiones son equivalentes:

\begin{eqnarray}
lim_{n\rightarrow\infty}n^{-1}T_{n}&=&\mu,\textrm{ c.s.}\\
lim_{t\rightarrow\infty}t^{-1}N\left(t\right)&=&1/\mu,\textrm{ c.s.}
\end{eqnarray}
\end{Teo}


Es decir, $T_{n}$ satisface la Ley Fuerte de los Grandes N\'umeros s\'i y s\'olo s\'i $N\left/t\right)$ la cumple.


\begin{Coro}[Ley Fuerte de los Grandes N\'umeros para Procesos de Renovaci\'on]
Si $N\left(t\right)$ es un proceso de renovaci\'on cuyos tiempos de inter-renovaci\'on tienen media $\mu\leq\infty$, entonces
\begin{eqnarray}
t^{-1}N\left(t\right)\rightarrow 1/\mu,\textrm{ c.s. cuando }t\rightarrow\infty.
\end{eqnarray}

\end{Coro}


Considerar el proceso estoc\'astico de valores reales $\left\{Z\left(t\right):t\geq0\right\}$ en el mismo espacio de probabilidad que $N\left(t\right)$

\begin{Def}
Para el proceso $\left\{Z\left(t\right):t\geq0\right\}$ se define la fluctuaci\'on m\'axima de $Z\left(t\right)$ en el intervalo $\left(T_{n-1},T_{n}\right]$:
\begin{eqnarray*}
M_{n}=\sup_{T_{n-1}<t\leq T_{n}}|Z\left(t\right)-Z\left(T_{n-1}\right)|
\end{eqnarray*}
\end{Def}

\begin{Teo}
Sup\'ongase que $n^{-1}T_{n}\rightarrow\mu$ c.s. cuando $n\rightarrow\infty$, donde $\mu\leq\infty$ es una constante o variable aleatoria. Sea $a$ una constante o variable aleatoria que puede ser infinita cuando $\mu$ es finita, y considere las expresiones l\'imite:
\begin{eqnarray}
lim_{n\rightarrow\infty}n^{-1}Z\left(T_{n}\right)&=&a,\textrm{ c.s.}\\
lim_{t\rightarrow\infty}t^{-1}Z\left(t\right)&=&a/\mu,\textrm{ c.s.}
\end{eqnarray}
La segunda expresi\'on implica la primera. Conversamente, la primera implica la segunda si el proceso $Z\left(t\right)$ es creciente, o si $lim_{n\rightarrow\infty}n^{-1}M_{n}=0$ c.s.
\end{Teo}

\begin{Coro}
Si $N\left(t\right)$ es un proceso de renovaci\'on, y $\left(Z\left(T_{n}\right)-Z\left(T_{n-1}\right),M_{n}\right)$, para $n\geq1$, son variables aleatorias independientes e id\'enticamente distribuidas con media finita, entonces,
\begin{eqnarray}
lim_{t\rightarrow\infty}t^{-1}Z\left(t\right)\rightarrow\frac{\esp\left[Z\left(T_{1}\right)-Z\left(T_{0}\right)\right]}{\esp\left[T_{1}\right]},\textrm{ c.s. cuando  }t\rightarrow\infty.
\end{eqnarray}
\end{Coro}


Sup\'ongase que $N\left(t\right)$ es un proceso de renovaci\'on con distribuci\'on $F$ con media finita $\mu$.

\begin{Def}
La funci\'on de renovaci\'on asociada con la distribuci\'on $F$, del proceso $N\left(t\right)$, es
\begin{eqnarray*}
U\left(t\right)=\sum_{n=1}^{\infty}F^{n\star}\left(t\right),\textrm{   }t\geq0,
\end{eqnarray*}
donde $F^{0\star}\left(t\right)=\indora\left(t\geq0\right)$.
\end{Def}


\begin{Prop}
Sup\'ongase que la distribuci\'on de inter-renovaci\'on $F$ tiene densidad $f$. Entonces $U\left(t\right)$ tambi\'en tiene densidad, para $t>0$, y es $U^{'}\left(t\right)=\sum_{n=0}^{\infty}f^{n\star}\left(t\right)$. Adem\'as
\begin{eqnarray*}
\prob\left\{N\left(t\right)>N\left(t-\right)\right\}=0\textrm{,   }t\geq0.
\end{eqnarray*}
\end{Prop}

\begin{Def}
La Transformada de Laplace-Stieljes de $F$ est\'a dada por

\begin{eqnarray*}
\hat{F}\left(\alpha\right)=\int_{\rea_{+}}e^{-\alpha t}dF\left(t\right)\textrm{,  }\alpha\geq0.
\end{eqnarray*}
\end{Def}

Entonces

\begin{eqnarray*}
\hat{U}\left(\alpha\right)=\sum_{n=0}^{\infty}\hat{F^{n\star}}\left(\alpha\right)=\sum_{n=0}^{\infty}\hat{F}\left(\alpha\right)^{n}=\frac{1}{1-\hat{F}\left(\alpha\right)}.
\end{eqnarray*}


\begin{Prop}
La Transformada de Laplace $\hat{U}\left(\alpha\right)$ y $\hat{F}\left(\alpha\right)$ determina una a la otra de manera \'unica por la relaci\'on $\hat{U}\left(\alpha\right)=\frac{1}{1-\hat{F}\left(\alpha\right)}$.
\end{Prop}


\begin{Note}
Un proceso de renovaci\'on $N\left(t\right)$ cuyos tiempos de inter-renovaci\'on tienen media finita, es un proceso Poisson con tasa $\lambda$ si y s\'olo s\'i $\esp\left[U\left(t\right)\right]=\lambda t$, para $t\geq0$.
\end{Note}


\begin{Teo}
Sea $N\left(t\right)$ un proceso puntual simple con puntos de localizaci\'on $T_{n}$ tal que $\eta\left(t\right)=\esp\left[N\left(\right)\right]$ es finita para cada $t$. Entonces para cualquier funci\'on $f:\rea_{+}\rightarrow\rea$,
\begin{eqnarray*}
\esp\left[\sum_{n=1}^{N\left(\right)}f\left(T_{n}\right)\right]=\int_{\left(0,t\right]}f\left(s\right)d\eta\left(s\right)\textrm{,  }t\geq0,
\end{eqnarray*}
suponiendo que la integral exista. Adem\'as si $X_{1},X_{2},\ldots$ son variables aleatorias definidas en el mismo espacio de probabilidad que el proceso $N\left(t\right)$ tal que $\esp\left[X_{n}|T_{n}=s\right]=f\left(s\right)$, independiente de $n$. Entonces
\begin{eqnarray*}
\esp\left[\sum_{n=1}^{N\left(t\right)}X_{n}\right]=\int_{\left(0,t\right]}f\left(s\right)d\eta\left(s\right)\textrm{,  }t\geq0,
\end{eqnarray*} 
suponiendo que la integral exista. 
\end{Teo}

\begin{Coro}[Identidad de Wald para Renovaciones]
Para el proceso de renovaci\'on $N\left(t\right)$,
\begin{eqnarray*}
\esp\left[T_{N\left(t\right)+1}\right]=\mu\esp\left[N\left(t\right)+1\right]\textrm{,  }t\geq0,
\end{eqnarray*}  
\end{Coro}


\begin{Def}
Sea $h\left(t\right)$ funci\'on de valores reales en $\rea$ acotada en intervalos finitos e igual a cero para $t<0$ La ecuaci\'on de renovaci\'on para $h\left(t\right)$ y la distribuci\'on $F$ es

\begin{eqnarray}%\label{Ec.Renovacion}
H\left(t\right)=h\left(t\right)+\int_{\left[0,t\right]}H\left(t-s\right)dF\left(s\right)\textrm{,    }t\geq0,
\end{eqnarray}
donde $H\left(t\right)$ es una funci\'on de valores reales. Esto es $H=h+F\star H$. Decimos que $H\left(t\right)$ es soluci\'on de esta ecuaci\'on si satisface la ecuaci\'on, y es acotada en intervalos finitos e iguales a cero para $t<0$.
\end{Def}

\begin{Prop}
La funci\'on $U\star h\left(t\right)$ es la \'unica soluci\'on de la ecuaci\'on de renovaci\'on (\ref{Ec.Renovacion}).
\end{Prop}

\begin{Teo}[Teorema Renovaci\'on Elemental]
\begin{eqnarray*}
t^{-1}U\left(t\right)\rightarrow 1/\mu\textrm{,    cuando }t\rightarrow\infty.
\end{eqnarray*}
\end{Teo}



Sup\'ongase que $N\left(t\right)$ es un proceso de renovaci\'on con distribuci\'on $F$ con media finita $\mu$.

\begin{Def}
La funci\'on de renovaci\'on asociada con la distribuci\'on $F$, del proceso $N\left(t\right)$, es
\begin{eqnarray*}
U\left(t\right)=\sum_{n=1}^{\infty}F^{n\star}\left(t\right),\textrm{   }t\geq0,
\end{eqnarray*}
donde $F^{0\star}\left(t\right)=\indora\left(t\geq0\right)$.
\end{Def}


\begin{Prop}
Sup\'ongase que la distribuci\'on de inter-renovaci\'on $F$ tiene densidad $f$. Entonces $U\left(t\right)$ tambi\'en tiene densidad, para $t>0$, y es $U^{'}\left(t\right)=\sum_{n=0}^{\infty}f^{n\star}\left(t\right)$. Adem\'as
\begin{eqnarray*}
\prob\left\{N\left(t\right)>N\left(t-\right)\right\}=0\textrm{,   }t\geq0.
\end{eqnarray*}
\end{Prop}

\begin{Def}
La Transformada de Laplace-Stieljes de $F$ est\'a dada por

\begin{eqnarray*}
\hat{F}\left(\alpha\right)=\int_{\rea_{+}}e^{-\alpha t}dF\left(t\right)\textrm{,  }\alpha\geq0.
\end{eqnarray*}
\end{Def}

Entonces

\begin{eqnarray*}
\hat{U}\left(\alpha\right)=\sum_{n=0}^{\infty}\hat{F^{n\star}}\left(\alpha\right)=\sum_{n=0}^{\infty}\hat{F}\left(\alpha\right)^{n}=\frac{1}{1-\hat{F}\left(\alpha\right)}.
\end{eqnarray*}


\begin{Prop}
La Transformada de Laplace $\hat{U}\left(\alpha\right)$ y $\hat{F}\left(\alpha\right)$ determina una a la otra de manera \'unica por la relaci\'on $\hat{U}\left(\alpha\right)=\frac{1}{1-\hat{F}\left(\alpha\right)}$.
\end{Prop}


\begin{Note}
Un proceso de renovaci\'on $N\left(t\right)$ cuyos tiempos de inter-renovaci\'on tienen media finita, es un proceso Poisson con tasa $\lambda$ si y s\'olo s\'i $\esp\left[U\left(t\right)\right]=\lambda t$, para $t\geq0$.
\end{Note}


\begin{Teo}
Sea $N\left(t\right)$ un proceso puntual simple con puntos de localizaci\'on $T_{n}$ tal que $\eta\left(t\right)=\esp\left[N\left(\right)\right]$ es finita para cada $t$. Entonces para cualquier funci\'on $f:\rea_{+}\rightarrow\rea$,
\begin{eqnarray*}
\esp\left[\sum_{n=1}^{N\left(\right)}f\left(T_{n}\right)\right]=\int_{\left(0,t\right]}f\left(s\right)d\eta\left(s\right)\textrm{,  }t\geq0,
\end{eqnarray*}
suponiendo que la integral exista. Adem\'as si $X_{1},X_{2},\ldots$ son variables aleatorias definidas en el mismo espacio de probabilidad que el proceso $N\left(t\right)$ tal que $\esp\left[X_{n}|T_{n}=s\right]=f\left(s\right)$, independiente de $n$. Entonces
\begin{eqnarray*}
\esp\left[\sum_{n=1}^{N\left(t\right)}X_{n}\right]=\int_{\left(0,t\right]}f\left(s\right)d\eta\left(s\right)\textrm{,  }t\geq0,
\end{eqnarray*} 
suponiendo que la integral exista. 
\end{Teo}

\begin{Coro}[Identidad de Wald para Renovaciones]
Para el proceso de renovaci\'on $N\left(t\right)$,
\begin{eqnarray*}
\esp\left[T_{N\left(t\right)+1}\right]=\mu\esp\left[N\left(t\right)+1\right]\textrm{,  }t\geq0,
\end{eqnarray*}  
\end{Coro}


\begin{Def}
Sea $h\left(t\right)$ funci\'on de valores reales en $\rea$ acotada en intervalos finitos e igual a cero para $t<0$ La ecuaci\'on de renovaci\'on para $h\left(t\right)$ y la distribuci\'on $F$ es

\begin{eqnarray}%\label{Ec.Renovacion}
H\left(t\right)=h\left(t\right)+\int_{\left[0,t\right]}H\left(t-s\right)dF\left(s\right)\textrm{,    }t\geq0,
\end{eqnarray}
donde $H\left(t\right)$ es una funci\'on de valores reales. Esto es $H=h+F\star H$. Decimos que $H\left(t\right)$ es soluci\'on de esta ecuaci\'on si satisface la ecuaci\'on, y es acotada en intervalos finitos e iguales a cero para $t<0$.
\end{Def}

\begin{Prop}
La funci\'on $U\star h\left(t\right)$ es la \'unica soluci\'on de la ecuaci\'on de renovaci\'on (\ref{Ec.Renovacion}).
\end{Prop}

\begin{Teo}[Teorema Renovaci\'on Elemental]
\begin{eqnarray*}
t^{-1}U\left(t\right)\rightarrow 1/\mu\textrm{,    cuando }t\rightarrow\infty.
\end{eqnarray*}
\end{Teo}


\begin{Note} Una funci\'on $h:\rea_{+}\rightarrow\rea$ es Directamente Riemann Integrable en los siguientes casos:
\begin{itemize}
\item[a)] $h\left(t\right)\geq0$ es decreciente y Riemann Integrable.
\item[b)] $h$ es continua excepto posiblemente en un conjunto de Lebesgue de medida 0, y $|h\left(t\right)|\leq b\left(t\right)$, donde $b$ es DRI.
\end{itemize}
\end{Note}

\begin{Teo}[Teorema Principal de Renovaci\'on]
Si $F$ es no aritm\'etica y $h\left(t\right)$ es Directamente Riemann Integrable (DRI), entonces

\begin{eqnarray*}
lim_{t\rightarrow\infty}U\star h=\frac{1}{\mu}\int_{\rea_{+}}h\left(s\right)ds.
\end{eqnarray*}
\end{Teo}

\begin{Prop}
Cualquier funci\'on $H\left(t\right)$ acotada en intervalos finitos y que es 0 para $t<0$ puede expresarse como
\begin{eqnarray*}
H\left(t\right)=U\star h\left(t\right)\textrm{,  donde }h\left(t\right)=H\left(t\right)-F\star H\left(t\right)
\end{eqnarray*}
\end{Prop}

\begin{Def}
Un proceso estoc\'astico $X\left(t\right)$ es crudamente regenerativo en un tiempo aleatorio positivo $T$ si
\begin{eqnarray*}
\esp\left[X\left(T+t\right)|T\right]=\esp\left[X\left(t\right)\right]\textrm{, para }t\geq0,\end{eqnarray*}
y con las esperanzas anteriores finitas.
\end{Def}

\begin{Prop}
Sup\'ongase que $X\left(t\right)$ es un proceso crudamente regenerativo en $T$, que tiene distribuci\'on $F$. Si $\esp\left[X\left(t\right)\right]$ es acotado en intervalos finitos, entonces
\begin{eqnarray*}
\esp\left[X\left(t\right)\right]=U\star h\left(t\right)\textrm{,  donde }h\left(t\right)=\esp\left[X\left(t\right)\indora\left(T>t\right)\right].
\end{eqnarray*}
\end{Prop}

\begin{Teo}[Regeneraci\'on Cruda]
Sup\'ongase que $X\left(t\right)$ es un proceso con valores positivo crudamente regenerativo en $T$, y def\'inase $M=\sup\left\{|X\left(t\right)|:t\leq T\right\}$. Si $T$ es no aritm\'etico y $M$ y $MT$ tienen media finita, entonces
\begin{eqnarray*}
lim_{t\rightarrow\infty}\esp\left[X\left(t\right)\right]=\frac{1}{\mu}\int_{\rea_{+}}h\left(s\right)ds,
\end{eqnarray*}
donde $h\left(t\right)=\esp\left[X\left(t\right)\indora\left(T>t\right)\right]$.
\end{Teo}


\begin{Note} Una funci\'on $h:\rea_{+}\rightarrow\rea$ es Directamente Riemann Integrable en los siguientes casos:
\begin{itemize}
\item[a)] $h\left(t\right)\geq0$ es decreciente y Riemann Integrable.
\item[b)] $h$ es continua excepto posiblemente en un conjunto de Lebesgue de medida 0, y $|h\left(t\right)|\leq b\left(t\right)$, donde $b$ es DRI.
\end{itemize}
\end{Note}

\begin{Teo}[Teorema Principal de Renovaci\'on]
Si $F$ es no aritm\'etica y $h\left(t\right)$ es Directamente Riemann Integrable (DRI), entonces

\begin{eqnarray*}
lim_{t\rightarrow\infty}U\star h=\frac{1}{\mu}\int_{\rea_{+}}h\left(s\right)ds.
\end{eqnarray*}
\end{Teo}

\begin{Prop}
Cualquier funci\'on $H\left(t\right)$ acotada en intervalos finitos y que es 0 para $t<0$ puede expresarse como
\begin{eqnarray*}
H\left(t\right)=U\star h\left(t\right)\textrm{,  donde }h\left(t\right)=H\left(t\right)-F\star H\left(t\right)
\end{eqnarray*}
\end{Prop}

\begin{Def}
Un proceso estoc\'astico $X\left(t\right)$ es crudamente regenerativo en un tiempo aleatorio positivo $T$ si
\begin{eqnarray*}
\esp\left[X\left(T+t\right)|T\right]=\esp\left[X\left(t\right)\right]\textrm{, para }t\geq0,\end{eqnarray*}
y con las esperanzas anteriores finitas.
\end{Def}

\begin{Prop}
Sup\'ongase que $X\left(t\right)$ es un proceso crudamente regenerativo en $T$, que tiene distribuci\'on $F$. Si $\esp\left[X\left(t\right)\right]$ es acotado en intervalos finitos, entonces
\begin{eqnarray*}
\esp\left[X\left(t\right)\right]=U\star h\left(t\right)\textrm{,  donde }h\left(t\right)=\esp\left[X\left(t\right)\indora\left(T>t\right)\right].
\end{eqnarray*}
\end{Prop}

\begin{Teo}[Regeneraci\'on Cruda]
Sup\'ongase que $X\left(t\right)$ es un proceso con valores positivo crudamente regenerativo en $T$, y def\'inase $M=\sup\left\{|X\left(t\right)|:t\leq T\right\}$. Si $T$ es no aritm\'etico y $M$ y $MT$ tienen media finita, entonces
\begin{eqnarray*}
lim_{t\rightarrow\infty}\esp\left[X\left(t\right)\right]=\frac{1}{\mu}\int_{\rea_{+}}h\left(s\right)ds,
\end{eqnarray*}
donde $h\left(t\right)=\esp\left[X\left(t\right)\indora\left(T>t\right)\right]$.
\end{Teo}

\begin{Def}
Para el proceso $\left\{\left(N\left(t\right),X\left(t\right)\right):t\geq0\right\}$, sus trayectoria muestrales en el intervalo de tiempo $\left[T_{n-1},T_{n}\right)$ est\'an descritas por
\begin{eqnarray*}
\zeta_{n}=\left(\xi_{n},\left\{X\left(T_{n-1}+t\right):0\leq t<\xi_{n}\right\}\right)
\end{eqnarray*}
Este $\zeta_{n}$ es el $n$-\'esimo segmento del proceso. El proceso es regenerativo sobre los tiempos $T_{n}$ si sus segmentos $\zeta_{n}$ son independientes e id\'enticamennte distribuidos.
\end{Def}


\begin{Note}
Si $\tilde{X}\left(t\right)$ con espacio de estados $\tilde{S}$ es regenerativo sobre $T_{n}$, entonces $X\left(t\right)=f\left(\tilde{X}\left(t\right)\right)$ tambi\'en es regenerativo sobre $T_{n}$, para cualquier funci\'on $f:\tilde{S}\rightarrow S$.
\end{Note}

\begin{Note}
Los procesos regenerativos son crudamente regenerativos, pero no al rev\'es.
\end{Note}


\begin{Note}
Un proceso estoc\'astico a tiempo continuo o discreto es regenerativo si existe un proceso de renovaci\'on  tal que los segmentos del proceso entre tiempos de renovaci\'on sucesivos son i.i.d., es decir, para $\left\{X\left(t\right):t\geq0\right\}$ proceso estoc\'astico a tiempo continuo con espacio de estados $S$, espacio m\'etrico.
\end{Note}

Para $\left\{X\left(t\right):t\geq0\right\}$ Proceso Estoc\'astico a tiempo continuo con estado de espacios $S$, que es un espacio m\'etrico, con trayectorias continuas por la derecha y con l\'imites por la izquierda c.s. Sea $N\left(t\right)$ un proceso de renovaci\'on en $\rea_{+}$ definido en el mismo espacio de probabilidad que $X\left(t\right)$, con tiempos de renovaci\'on $T$ y tiempos de inter-renovaci\'on $\xi_{n}=T_{n}-T_{n-1}$, con misma distribuci\'on $F$ de media finita $\mu$.



\begin{Def}
Para el proceso $\left\{\left(N\left(t\right),X\left(t\right)\right):t\geq0\right\}$, sus trayectoria muestrales en el intervalo de tiempo $\left[T_{n-1},T_{n}\right)$ est\'an descritas por
\begin{eqnarray*}
\zeta_{n}=\left(\xi_{n},\left\{X\left(T_{n-1}+t\right):0\leq t<\xi_{n}\right\}\right)
\end{eqnarray*}
Este $\zeta_{n}$ es el $n$-\'esimo segmento del proceso. El proceso es regenerativo sobre los tiempos $T_{n}$ si sus segmentos $\zeta_{n}$ son independientes e id\'enticamennte distribuidos.
\end{Def}

\begin{Note}
Un proceso regenerativo con media de la longitud de ciclo finita es llamado positivo recurrente.
\end{Note}

\begin{Teo}[Procesos Regenerativos]
Suponga que el proceso
\end{Teo}


\begin{Def}[Renewal Process Trinity]
Para un proceso de renovaci\'on $N\left(t\right)$, los siguientes procesos proveen de informaci\'on sobre los tiempos de renovaci\'on.
\begin{itemize}
\item $A\left(t\right)=t-T_{N\left(t\right)}$, el tiempo de recurrencia hacia atr\'as al tiempo $t$, que es el tiempo desde la \'ultima renovaci\'on para $t$.

\item $B\left(t\right)=T_{N\left(t\right)+1}-t$, el tiempo de recurrencia hacia adelante al tiempo $t$, residual del tiempo de renovaci\'on, que es el tiempo para la pr\'oxima renovaci\'on despu\'es de $t$.

\item $L\left(t\right)=\xi_{N\left(t\right)+1}=A\left(t\right)+B\left(t\right)$, la longitud del intervalo de renovaci\'on que contiene a $t$.
\end{itemize}
\end{Def}

\begin{Note}
El proceso tridimensional $\left(A\left(t\right),B\left(t\right),L\left(t\right)\right)$ es regenerativo sobre $T_{n}$, y por ende cada proceso lo es. Cada proceso $A\left(t\right)$ y $B\left(t\right)$ son procesos de MArkov a tiempo continuo con trayectorias continuas por partes en el espacio de estados $\rea_{+}$. Una expresi\'on conveniente para su distribuci\'on conjunta es, para $0\leq x<t,y\geq0$
\begin{equation}\label{NoRenovacion}
P\left\{A\left(t\right)>x,B\left(t\right)>y\right\}=
P\left\{N\left(t+y\right)-N\left((t-x)\right)=0\right\}
\end{equation}
\end{Note}

\begin{Ejem}[Tiempos de recurrencia Poisson]
Si $N\left(t\right)$ es un proceso Poisson con tasa $\lambda$, entonces de la expresi\'on (\ref{NoRenovacion}) se tiene que

\begin{eqnarray*}
\begin{array}{lc}
P\left\{A\left(t\right)>x,B\left(t\right)>y\right\}=e^{-\lambda\left(x+y\right)},&0\leq x<t,y\geq0,
\end{array}
\end{eqnarray*}
que es la probabilidad Poisson de no renovaciones en un intervalo de longitud $x+y$.

\end{Ejem}

%\begin{Note}
Una cadena de Markov erg\'odica tiene la propiedad de ser estacionaria si la distribuci\'on de su estado al tiempo $0$ es su distribuci\'on estacionaria.
%\end{Note}


\begin{Def}
Un proceso estoc\'astico a tiempo continuo $\left\{X\left(t\right):t\geq0\right\}$ en un espacio general es estacionario si sus distribuciones finito dimensionales son invariantes bajo cualquier  traslado: para cada $0\leq s_{1}<s_{2}<\cdots<s_{k}$ y $t\geq0$,
\begin{eqnarray*}
\left(X\left(s_{1}+t\right),\ldots,X\left(s_{k}+t\right)\right)=_{d}\left(X\left(s_{1}\right),\ldots,X\left(s_{k}\right)\right).
\end{eqnarray*}
\end{Def}

\begin{Note}
Un proceso de Markov es estacionario si $X\left(t\right)=_{d}X\left(0\right)$, $t\geq0$.
\end{Note}

Considerese el proceso $N\left(t\right)=\sum_{n}\indora\left(\tau_{n}\leq t\right)$ en $\rea_{+}$, con puntos $0<\tau_{1}<\tau_{2}<\cdots$.

\begin{Prop}
Si $N$ es un proceso puntual estacionario y $\esp\left[N\left(1\right)\right]<\infty$, entonces $\esp\left[N\left(t\right)\right]=t\esp\left[N\left(1\right)\right]$, $t\geq0$

\end{Prop}

\begin{Teo}
Los siguientes enunciados son equivalentes
\begin{itemize}
\item[i)] El proceso retardado de renovaci\'on $N$ es estacionario.

\item[ii)] EL proceso de tiempos de recurrencia hacia adelante $B\left(t\right)$ es estacionario.


\item[iii)] $\esp\left[N\left(t\right)\right]=t/\mu$,


\item[iv)] $G\left(t\right)=F_{e}\left(t\right)=\frac{1}{\mu}\int_{0}^{t}\left[1-F\left(s\right)\right]ds$
\end{itemize}
Cuando estos enunciados son ciertos, $P\left\{B\left(t\right)\leq x\right\}=F_{e}\left(x\right)$, para $t,x\geq0$.

\end{Teo}

\begin{Note}
Una consecuencia del teorema anterior es que el Proceso Poisson es el \'unico proceso sin retardo que es estacionario.
\end{Note}

\begin{Coro}
El proceso de renovaci\'on $N\left(t\right)$ sin retardo, y cuyos tiempos de inter renonaci\'on tienen media finita, es estacionario si y s\'olo si es un proceso Poisson.

\end{Coro}


%________________________________________________________________________
\subsubsection{Procesos Regenerativos}
%________________________________________________________________________



\begin{Note}
Si $\tilde{X}\left(t\right)$ con espacio de estados $\tilde{S}$ es regenerativo sobre $T_{n}$, entonces $X\left(t\right)=f\left(\tilde{X}\left(t\right)\right)$ tambi\'en es regenerativo sobre $T_{n}$, para cualquier funci\'on $f:\tilde{S}\rightarrow S$.
\end{Note}

\begin{Note}
Los procesos regenerativos son crudamente regenerativos, pero no al rev\'es.
\end{Note}
%\subsection*{Procesos Regenerativos: Sigman\cite{Sigman1}}
\begin{Def}[Definici\'on Cl\'asica]
Un proceso estoc\'astico $X=\left\{X\left(t\right):t\geq0\right\}$ es llamado regenerativo is existe una variable aleatoria $R_{1}>0$ tal que
\begin{itemize}
\item[i)] $\left\{X\left(t+R_{1}\right):t\geq0\right\}$ es independiente de $\left\{\left\{X\left(t\right):t<R_{1}\right\},\right\}$
\item[ii)] $\left\{X\left(t+R_{1}\right):t\geq0\right\}$ es estoc\'asticamente equivalente a $\left\{X\left(t\right):t>0\right\}$
\end{itemize}

Llamamos a $R_{1}$ tiempo de regeneraci\'on, y decimos que $X$ se regenera en este punto.
\end{Def}

$\left\{X\left(t+R_{1}\right)\right\}$ es regenerativo con tiempo de regeneraci\'on $R_{2}$, independiente de $R_{1}$ pero con la misma distribuci\'on que $R_{1}$. Procediendo de esta manera se obtiene una secuencia de variables aleatorias independientes e id\'enticamente distribuidas $\left\{R_{n}\right\}$ llamados longitudes de ciclo. Si definimos a $Z_{k}\equiv R_{1}+R_{2}+\cdots+R_{k}$, se tiene un proceso de renovaci\'on llamado proceso de renovaci\'on encajado para $X$.




\begin{Def}
Para $x$ fijo y para cada $t\geq0$, sea $I_{x}\left(t\right)=1$ si $X\left(t\right)\leq x$,  $I_{x}\left(t\right)=0$ en caso contrario, y def\'inanse los tiempos promedio
\begin{eqnarray*}
\overline{X}&=&lim_{t\rightarrow\infty}\frac{1}{t}\int_{0}^{\infty}X\left(u\right)du\\
\prob\left(X_{\infty}\leq x\right)&=&lim_{t\rightarrow\infty}\frac{1}{t}\int_{0}^{\infty}I_{x}\left(u\right)du,
\end{eqnarray*}
cuando estos l\'imites existan.
\end{Def}

Como consecuencia del teorema de Renovaci\'on-Recompensa, se tiene que el primer l\'imite  existe y es igual a la constante
\begin{eqnarray*}
\overline{X}&=&\frac{\esp\left[\int_{0}^{R_{1}}X\left(t\right)dt\right]}{\esp\left[R_{1}\right]},
\end{eqnarray*}
suponiendo que ambas esperanzas son finitas.

\begin{Note}
\begin{itemize}
\item[a)] Si el proceso regenerativo $X$ es positivo recurrente y tiene trayectorias muestrales no negativas, entonces la ecuaci\'on anterior es v\'alida.
\item[b)] Si $X$ es positivo recurrente regenerativo, podemos construir una \'unica versi\'on estacionaria de este proceso, $X_{e}=\left\{X_{e}\left(t\right)\right\}$, donde $X_{e}$ es un proceso estoc\'astico regenerativo y estrictamente estacionario, con distribuci\'on marginal distribuida como $X_{\infty}$
\end{itemize}
\end{Note}

%________________________________________________________________________
%\subsection{Procesos Regenerativos}
%________________________________________________________________________

Para $\left\{X\left(t\right):t\geq0\right\}$ Proceso Estoc\'astico a tiempo continuo con estado de espacios $S$, que es un espacio m\'etrico, con trayectorias continuas por la derecha y con l\'imites por la izquierda c.s. Sea $N\left(t\right)$ un proceso de renovaci\'on en $\rea_{+}$ definido en el mismo espacio de probabilidad que $X\left(t\right)$, con tiempos de renovaci\'on $T$ y tiempos de inter-renovaci\'on $\xi_{n}=T_{n}-T_{n-1}$, con misma distribuci\'on $F$ de media finita $\mu$.



\begin{Def}
Para el proceso $\left\{\left(N\left(t\right),X\left(t\right)\right):t\geq0\right\}$, sus trayectoria muestrales en el intervalo de tiempo $\left[T_{n-1},T_{n}\right)$ est\'an descritas por
\begin{eqnarray*}
\zeta_{n}=\left(\xi_{n},\left\{X\left(T_{n-1}+t\right):0\leq t<\xi_{n}\right\}\right)
\end{eqnarray*}
Este $\zeta_{n}$ es el $n$-\'esimo segmento del proceso. El proceso es regenerativo sobre los tiempos $T_{n}$ si sus segmentos $\zeta_{n}$ son independientes e id\'enticamennte distribuidos.
\end{Def}


\begin{Note}
Si $\tilde{X}\left(t\right)$ con espacio de estados $\tilde{S}$ es regenerativo sobre $T_{n}$, entonces $X\left(t\right)=f\left(\tilde{X}\left(t\right)\right)$ tambi\'en es regenerativo sobre $T_{n}$, para cualquier funci\'on $f:\tilde{S}\rightarrow S$.
\end{Note}

\begin{Note}
Los procesos regenerativos son crudamente regenerativos, pero no al rev\'es.
\end{Note}

\begin{Def}[Definici\'on Cl\'asica]
Un proceso estoc\'astico $X=\left\{X\left(t\right):t\geq0\right\}$ es llamado regenerativo is existe una variable aleatoria $R_{1}>0$ tal que
\begin{itemize}
\item[i)] $\left\{X\left(t+R_{1}\right):t\geq0\right\}$ es independiente de $\left\{\left\{X\left(t\right):t<R_{1}\right\},\right\}$
\item[ii)] $\left\{X\left(t+R_{1}\right):t\geq0\right\}$ es estoc\'asticamente equivalente a $\left\{X\left(t\right):t>0\right\}$
\end{itemize}

Llamamos a $R_{1}$ tiempo de regeneraci\'on, y decimos que $X$ se regenera en este punto.
\end{Def}

$\left\{X\left(t+R_{1}\right)\right\}$ es regenerativo con tiempo de regeneraci\'on $R_{2}$, independiente de $R_{1}$ pero con la misma distribuci\'on que $R_{1}$. Procediendo de esta manera se obtiene una secuencia de variables aleatorias independientes e id\'enticamente distribuidas $\left\{R_{n}\right\}$ llamados longitudes de ciclo. Si definimos a $Z_{k}\equiv R_{1}+R_{2}+\cdots+R_{k}$, se tiene un proceso de renovaci\'on llamado proceso de renovaci\'on encajado para $X$.

\begin{Note}
Un proceso regenerativo con media de la longitud de ciclo finita es llamado positivo recurrente.
\end{Note}


\begin{Def}
Para $x$ fijo y para cada $t\geq0$, sea $I_{x}\left(t\right)=1$ si $X\left(t\right)\leq x$,  $I_{x}\left(t\right)=0$ en caso contrario, y def\'inanse los tiempos promedio
\begin{eqnarray*}
\overline{X}&=&lim_{t\rightarrow\infty}\frac{1}{t}\int_{0}^{\infty}X\left(u\right)du\\
\prob\left(X_{\infty}\leq x\right)&=&lim_{t\rightarrow\infty}\frac{1}{t}\int_{0}^{\infty}I_{x}\left(u\right)du,
\end{eqnarray*}
cuando estos l\'imites existan.
\end{Def}

Como consecuencia del teorema de Renovaci\'on-Recompensa, se tiene que el primer l\'imite  existe y es igual a la constante
\begin{eqnarray*}
\overline{X}&=&\frac{\esp\left[\int_{0}^{R_{1}}X\left(t\right)dt\right]}{\esp\left[R_{1}\right]},
\end{eqnarray*}
suponiendo que ambas esperanzas son finitas.

\begin{Note}
\begin{itemize}
\item[a)] Si el proceso regenerativo $X$ es positivo recurrente y tiene trayectorias muestrales no negativas, entonces la ecuaci\'on anterior es v\'alida.
\item[b)] Si $X$ es positivo recurrente regenerativo, podemos construir una \'unica versi\'on estacionaria de este proceso, $X_{e}=\left\{X_{e}\left(t\right)\right\}$, donde $X_{e}$ es un proceso estoc\'astico regenerativo y estrictamente estacionario, con distribuci\'on marginal distribuida como $X_{\infty}$
\end{itemize}
\end{Note}

%__________________________________________________________________________________________
%\subsection{Procesos Regenerativos Estacionarios - Stidham \cite{Stidham}}
%__________________________________________________________________________________________


Un proceso estoc\'astico a tiempo continuo $\left\{V\left(t\right),t\geq0\right\}$ es un proceso regenerativo si existe una sucesi\'on de variables aleatorias independientes e id\'enticamente distribuidas $\left\{X_{1},X_{2},\ldots\right\}$, sucesi\'on de renovaci\'on, tal que para cualquier conjunto de Borel $A$, 

\begin{eqnarray*}
\prob\left\{V\left(t\right)\in A|X_{1}+X_{2}+\cdots+X_{R\left(t\right)}=s,\left\{V\left(\tau\right),\tau<s\right\}\right\}=\prob\left\{V\left(t-s\right)\in A|X_{1}>t-s\right\},
\end{eqnarray*}
para todo $0\leq s\leq t$, donde $R\left(t\right)=\max\left\{X_{1}+X_{2}+\cdots+X_{j}\leq t\right\}=$n\'umero de renovaciones ({\emph{puntos de regeneraci\'on}}) que ocurren en $\left[0,t\right]$. El intervalo $\left[0,X_{1}\right)$ es llamado {\emph{primer ciclo de regeneraci\'on}} de $\left\{V\left(t \right),t\geq0\right\}$, $\left[X_{1},X_{1}+X_{2}\right)$ el {\emph{segundo ciclo de regeneraci\'on}}, y as\'i sucesivamente.

Sea $X=X_{1}$ y sea $F$ la funci\'on de distrbuci\'on de $X$


\begin{Def}
Se define el proceso estacionario, $\left\{V^{*}\left(t\right),t\geq0\right\}$, para $\left\{V\left(t\right),t\geq0\right\}$ por

\begin{eqnarray*}
\prob\left\{V\left(t\right)\in A\right\}=\frac{1}{\esp\left[X\right]}\int_{0}^{\infty}\prob\left\{V\left(t+x\right)\in A|X>x\right\}\left(1-F\left(x\right)\right)dx,
\end{eqnarray*} 
para todo $t\geq0$ y todo conjunto de Borel $A$.
\end{Def}

\begin{Def}
Una distribuci\'on se dice que es {\emph{aritm\'etica}} si todos sus puntos de incremento son m\'ultiplos de la forma $0,\lambda, 2\lambda,\ldots$ para alguna $\lambda>0$ entera.
\end{Def}


\begin{Def}
Una modificaci\'on medible de un proceso $\left\{V\left(t\right),t\geq0\right\}$, es una versi\'on de este, $\left\{V\left(t,w\right)\right\}$ conjuntamente medible para $t\geq0$ y para $w\in S$, $S$ espacio de estados para $\left\{V\left(t\right),t\geq0\right\}$.
\end{Def}

\begin{Teo}
Sea $\left\{V\left(t\right),t\geq\right\}$ un proceso regenerativo no negativo con modificaci\'on medible. Sea $\esp\left[X\right]<\infty$. Entonces el proceso estacionario dado por la ecuaci\'on anterior est\'a bien definido y tiene funci\'on de distribuci\'on independiente de $t$, adem\'as
\begin{itemize}
\item[i)] \begin{eqnarray*}
\esp\left[V^{*}\left(0\right)\right]&=&\frac{\esp\left[\int_{0}^{X}V\left(s\right)ds\right]}{\esp\left[X\right]}\end{eqnarray*}
\item[ii)] Si $\esp\left[V^{*}\left(0\right)\right]<\infty$, equivalentemente, si $\esp\left[\int_{0}^{X}V\left(s\right)ds\right]<\infty$,entonces
\begin{eqnarray*}
\frac{\int_{0}^{t}V\left(s\right)ds}{t}\rightarrow\frac{\esp\left[\int_{0}^{X}V\left(s\right)ds\right]}{\esp\left[X\right]}
\end{eqnarray*}
con probabilidad 1 y en media, cuando $t\rightarrow\infty$.
\end{itemize}
\end{Teo}

%__________________________________________________________________________________________
%\subsection{Procesos Regenerativos Estacionarios - Stidham \cite{Stidham}}
%__________________________________________________________________________________________


Un proceso estoc\'astico a tiempo continuo $\left\{V\left(t\right),t\geq0\right\}$ es un proceso regenerativo si existe una sucesi\'on de variables aleatorias independientes e id\'enticamente distribuidas $\left\{X_{1},X_{2},\ldots\right\}$, sucesi\'on de renovaci\'on, tal que para cualquier conjunto de Borel $A$, 

\begin{eqnarray*}
\prob\left\{V\left(t\right)\in A|X_{1}+X_{2}+\cdots+X_{R\left(t\right)}=s,\left\{V\left(\tau\right),\tau<s\right\}\right\}=\prob\left\{V\left(t-s\right)\in A|X_{1}>t-s\right\},
\end{eqnarray*}
para todo $0\leq s\leq t$, donde $R\left(t\right)=\max\left\{X_{1}+X_{2}+\cdots+X_{j}\leq t\right\}=$n\'umero de renovaciones ({\emph{puntos de regeneraci\'on}}) que ocurren en $\left[0,t\right]$. El intervalo $\left[0,X_{1}\right)$ es llamado {\emph{primer ciclo de regeneraci\'on}} de $\left\{V\left(t \right),t\geq0\right\}$, $\left[X_{1},X_{1}+X_{2}\right)$ el {\emph{segundo ciclo de regeneraci\'on}}, y as\'i sucesivamente.

Sea $X=X_{1}$ y sea $F$ la funci\'on de distrbuci\'on de $X$


\begin{Def}
Se define el proceso estacionario, $\left\{V^{*}\left(t\right),t\geq0\right\}$, para $\left\{V\left(t\right),t\geq0\right\}$ por

\begin{eqnarray*}
\prob\left\{V\left(t\right)\in A\right\}=\frac{1}{\esp\left[X\right]}\int_{0}^{\infty}\prob\left\{V\left(t+x\right)\in A|X>x\right\}\left(1-F\left(x\right)\right)dx,
\end{eqnarray*} 
para todo $t\geq0$ y todo conjunto de Borel $A$.
\end{Def}

\begin{Def}
Una distribuci\'on se dice que es {\emph{aritm\'etica}} si todos sus puntos de incremento son m\'ultiplos de la forma $0,\lambda, 2\lambda,\ldots$ para alguna $\lambda>0$ entera.
\end{Def}


\begin{Def}
Una modificaci\'on medible de un proceso $\left\{V\left(t\right),t\geq0\right\}$, es una versi\'on de este, $\left\{V\left(t,w\right)\right\}$ conjuntamente medible para $t\geq0$ y para $w\in S$, $S$ espacio de estados para $\left\{V\left(t\right),t\geq0\right\}$.
\end{Def}

\begin{Teo}
Sea $\left\{V\left(t\right),t\geq\right\}$ un proceso regenerativo no negativo con modificaci\'on medible. Sea $\esp\left[X\right]<\infty$. Entonces el proceso estacionario dado por la ecuaci\'on anterior est\'a bien definido y tiene funci\'on de distribuci\'on independiente de $t$, adem\'as
\begin{itemize}
\item[i)] \begin{eqnarray*}
\esp\left[V^{*}\left(0\right)\right]&=&\frac{\esp\left[\int_{0}^{X}V\left(s\right)ds\right]}{\esp\left[X\right]}\end{eqnarray*}
\item[ii)] Si $\esp\left[V^{*}\left(0\right)\right]<\infty$, equivalentemente, si $\esp\left[\int_{0}^{X}V\left(s\right)ds\right]<\infty$,entonces
\begin{eqnarray*}
\frac{\int_{0}^{t}V\left(s\right)ds}{t}\rightarrow\frac{\esp\left[\int_{0}^{X}V\left(s\right)ds\right]}{\esp\left[X\right]}
\end{eqnarray*}
con probabilidad 1 y en media, cuando $t\rightarrow\infty$.
\end{itemize}
\end{Teo}

Para $\left\{X\left(t\right):t\geq0\right\}$ Proceso Estoc\'astico a tiempo continuo con estado de espacios $S$, que es un espacio m\'etrico, con trayectorias continuas por la derecha y con l\'imites por la izquierda c.s. Sea $N\left(t\right)$ un proceso de renovaci\'on en $\rea_{+}$ definido en el mismo espacio de probabilidad que $X\left(t\right)$, con tiempos de renovaci\'on $T$ y tiempos de inter-renovaci\'on $\xi_{n}=T_{n}-T_{n-1}$, con misma distribuci\'on $F$ de media finita $\mu$.
%_____________________________________________________
\subsection{Puntos de Renovaci\'on}
%_____________________________________________________

Para cada cola $Q_{i}$ se tienen los procesos de arribo a la cola, para estas, los tiempos de arribo est\'an dados por $$\left\{T_{1}^{i},T_{2}^{i},\ldots,T_{k}^{i},\ldots\right\},$$ entonces, consideremos solamente los primeros tiempos de arribo a cada una de las colas, es decir, $$\left\{T_{1}^{1},T_{1}^{2},T_{1}^{3},T_{1}^{4}\right\},$$ se sabe que cada uno de estos tiempos se distribuye de manera exponencial con par\'ametro $1/mu_{i}$. Adem\'as se sabe que para $$T^{*}=\min\left\{T_{1}^{1},T_{1}^{2},T_{1}^{3},T_{1}^{4}\right\},$$ $T^{*}$ se distribuye de manera exponencial con par\'ametro $$\mu^{*}=\sum_{i=1}^{4}\mu_{i}.$$ Ahora, dado que 
\begin{center}
\begin{tabular}{lcl}
$\tilde{r}=r_{1}+r_{2}$ & y &$\hat{r}=r_{3}+r_{4}.$
\end{tabular}
\end{center}


Supongamos que $$\tilde{r},\hat{r}<\mu^{*},$$ entonces si tomamos $$r^{*}=\min\left\{\tilde{r},\hat{r}\right\},$$ se tiene que para  $$t^{*}\in\left(0,r^{*}\right)$$ se cumple que 
\begin{center}
\begin{tabular}{lcl}
$\tau_{1}\left(1\right)=0$ & y por tanto & $\overline{\tau}_{1}=0,$
\end{tabular}
\end{center}
entonces para la segunda cola en este primer ciclo se cumple que $$\tau_{2}=\overline{\tau}_{1}+r_{1}=r_{1}<\mu^{*},$$ y por tanto se tiene que  $$\overline{\tau}_{2}=\tau_{2}.$$ Por lo tanto, nuevamente para la primer cola en el segundo ciclo $$\tau_{1}\left(2\right)=\tau_{2}\left(1\right)+r_{2}=\tilde{r}<\mu^{*}.$$ An\'alogamente para el segundo sistema se tiene que ambas colas est\'an vac\'ias, es decir, existe un valor $t^{*}$ tal que en el intervalo $\left(0,t^{*}\right)$ no ha llegado ning\'un usuario, es decir, $$L_{i}\left(t^{*}\right)=0$$ para $i=1,2,3,4$.

\subsection{Resultados para Procesos de Salida}




%________________________________________________________________________
\subsection{Procesos Regenerativos}
%________________________________________________________________________

Para $\left\{X\left(t\right):t\geq0\right\}$ Proceso Estoc\'astico a tiempo continuo con estado de espacios $S$, que es un espacio m\'etrico, con trayectorias continuas por la derecha y con l\'imites por la izquierda c.s. Sea $N\left(t\right)$ un proceso de renovaci\'on en $\rea_{+}$ definido en el mismo espacio de probabilidad que $X\left(t\right)$, con tiempos de renovaci\'on $T$ y tiempos de inter-renovaci\'on $\xi_{n}=T_{n}-T_{n-1}$, con misma distribuci\'on $F$ de media finita $\mu$.



\begin{Def}
Para el proceso $\left\{\left(N\left(t\right),X\left(t\right)\right):t\geq0\right\}$, sus trayectoria muestrales en el intervalo de tiempo $\left[T_{n-1},T_{n}\right)$ est\'an descritas por
\begin{eqnarray*}
\zeta_{n}=\left(\xi_{n},\left\{X\left(T_{n-1}+t\right):0\leq t<\xi_{n}\right\}\right)
\end{eqnarray*}
Este $\zeta_{n}$ es el $n$-\'esimo segmento del proceso. El proceso es regenerativo sobre los tiempos $T_{n}$ si sus segmentos $\zeta_{n}$ son independientes e id\'enticamennte distribuidos.
\end{Def}


\begin{Obs}
Si $\tilde{X}\left(t\right)$ con espacio de estados $\tilde{S}$ es regenerativo sobre $T_{n}$, entonces $X\left(t\right)=f\left(\tilde{X}\left(t\right)\right)$ tambi\'en es regenerativo sobre $T_{n}$, para cualquier funci\'on $f:\tilde{S}\rightarrow S$.
\end{Obs}

\begin{Obs}
Los procesos regenerativos son crudamente regenerativos, pero no al rev\'es.
\end{Obs}

\begin{Def}[Definici\'on Cl\'asica]
Un proceso estoc\'astico $X=\left\{X\left(t\right):t\geq0\right\}$ es llamado regenerativo is existe una variable aleatoria $R_{1}>0$ tal que
\begin{itemize}
\item[i)] $\left\{X\left(t+R_{1}\right):t\geq0\right\}$ es independiente de $\left\{\left\{X\left(t\right):t<R_{1}\right\},\right\}$
\item[ii)] $\left\{X\left(t+R_{1}\right):t\geq0\right\}$ es estoc\'asticamente equivalente a $\left\{X\left(t\right):t>0\right\}$
\end{itemize}

Llamamos a $R_{1}$ tiempo de regeneraci\'on, y decimos que $X$ se regenera en este punto.
\end{Def}

$\left\{X\left(t+R_{1}\right)\right\}$ es regenerativo con tiempo de regeneraci\'on $R_{2}$, independiente de $R_{1}$ pero con la misma distribuci\'on que $R_{1}$. Procediendo de esta manera se obtiene una secuencia de variables aleatorias independientes e id\'enticamente distribuidas $\left\{R_{n}\right\}$ llamados longitudes de ciclo. Si definimos a $Z_{k}\equiv R_{1}+R_{2}+\cdots+R_{k}$, se tiene un proceso de renovaci\'on llamado proceso de renovaci\'on encajado para $X$.

\begin{Note}
Un proceso regenerativo con media de la longitud de ciclo finita es llamado positivo recurrente.
\end{Note}


\begin{Def}
Para $x$ fijo y para cada $t\geq0$, sea $I_{x}\left(t\right)=1$ si $X\left(t\right)\leq x$,  $I_{x}\left(t\right)=0$ en caso contrario, y def\'inanse los tiempos promedio
\begin{eqnarray*}
\overline{X}&=&lim_{t\rightarrow\infty}\frac{1}{t}\int_{0}^{\infty}X\left(u\right)du\\
\prob\left(X_{\infty}\leq x\right)&=&lim_{t\rightarrow\infty}\frac{1}{t}\int_{0}^{\infty}I_{x}\left(u\right)du,
\end{eqnarray*}
cuando estos l\'imites existan.
\end{Def}

Como consecuencia del teorema de Renovaci\'on-Recompensa, se tiene que el primer l\'imite  existe y es igual a la constante
\begin{eqnarray*}
\overline{X}&=&\frac{\esp\left[\int_{0}^{R_{1}}X\left(t\right)dt\right]}{\esp\left[R_{1}\right]},
\end{eqnarray*}
suponiendo que ambas esperanzas son finitas.

\begin{Note}
\begin{itemize}
\item[a)] Si el proceso regenerativo $X$ es positivo recurrente y tiene trayectorias muestrales no negativas, entonces la ecuaci\'on anterior es v\'alida.
\item[b)] Si $X$ es positivo recurrente regenerativo, podemos construir una \'unica versi\'on estacionaria de este proceso, $X_{e}=\left\{X_{e}\left(t\right)\right\}$, donde $X_{e}$ es un proceso estoc\'astico regenerativo y estrictamente estacionario, con distribuci\'on marginal distribuida como $X_{\infty}$
\end{itemize}
\end{Note}


%___________________________________________________________________________________________
%
\section{Renewal and Regenerative Processes: Serfozo\cite{Serfozo}}
%___________________________________________________________________________________________
%
\begin{Def}%\label{Def.Tn}
Sean $0\leq T_{1}\leq T_{2}\leq \ldots$ son tiempos aleatorios infinitos en los cuales ocurren ciertos eventos. El n\'umero de tiempos $T_{n}$ en el intervalo $\left[0,t\right)$ es

\begin{eqnarray}
N\left(t\right)=\sum_{n=1}^{\infty}\indora\left(T_{n}\leq t\right),
\end{eqnarray}
para $t\geq0$.
\end{Def}

Si se consideran los puntos $T_{n}$ como elementos de $\rea_{+}$, y $N\left(t\right)$ es el n\'umero de puntos en $\rea$. El proceso denotado por $\left\{N\left(t\right):t\geq0\right\}$, denotado por $N\left(t\right)$, es un proceso puntual en $\rea_{+}$. Los $T_{n}$ son los tiempos de ocurrencia, el proceso puntual $N\left(t\right)$ es simple si su n\'umero de ocurrencias son distintas: $0<T_{1}<T_{2}<\ldots$ casi seguramente.

\begin{Def}
Un proceso puntual $N\left(t\right)$ es un proceso de renovaci\'on si los tiempos de interocurrencia $\xi_{n}=T_{n}-T_{n-1}$, para $n\geq1$, son independientes e identicamente distribuidos con distribuci\'on $F$, donde $F\left(0\right)=0$ y $T_{0}=0$. Los $T_{n}$ son llamados tiempos de renovaci\'on, referente a la independencia o renovaci\'on de la informaci\'on estoc\'astica en estos tiempos. Los $\xi_{n}$ son los tiempos de inter-renovaci\'on, y $N\left(t\right)$ es el n\'umero de renovaciones en el intervalo $\left[0,t\right)$
\end{Def}


\begin{Note}
Para definir un proceso de renovaci\'on para cualquier contexto, solamente hay que especificar una distribuci\'on $F$, con $F\left(0\right)=0$, para los tiempos de inter-renovaci\'on. La funci\'on $F$ en turno degune las otra variables aleatorias. De manera formal, existe un espacio de probabilidad y una sucesi\'on de variables aleatorias $\xi_{1},\xi_{2},\ldots$ definidas en este con distribuci\'on $F$. Entonces las otras cantidades son $T_{n}=\sum_{k=1}^{n}\xi_{k}$ y $N\left(t\right)=\sum_{n=1}^{\infty}\indora\left(T_{n}\leq t\right)$, donde $T_{n}\rightarrow\infty$ casi seguramente por la Ley Fuerte de los Grandes N\'umeros.
\end{Note}







Los tiempos $T_{n}$ est\'an relacionados con los conteos de $N\left(t\right)$ por

\begin{eqnarray*}
\left\{N\left(t\right)\geq n\right\}&=&\left\{T_{n}\leq t\right\}\\
T_{N\left(t\right)}\leq &t&<T_{N\left(t\right)+1},
\end{eqnarray*}

adem\'as $N\left(T_{n}\right)=n$, y 

\begin{eqnarray*}
N\left(t\right)=\max\left\{n:T_{n}\leq t\right\}=\min\left\{n:T_{n+1}>t\right\}
\end{eqnarray*}

Por propiedades de la convoluci\'on se sabe que

\begin{eqnarray*}
P\left\{T_{n}\leq t\right\}=F^{n\star}\left(t\right)
\end{eqnarray*}
que es la $n$-\'esima convoluci\'on de $F$. Entonces 

\begin{eqnarray*}
\left\{N\left(t\right)\geq n\right\}&=&\left\{T_{n}\leq t\right\}\\
P\left\{N\left(t\right)\leq n\right\}&=&1-F^{\left(n+1\right)\star}\left(t\right)
\end{eqnarray*}

Adem\'as usando el hecho de que $\esp\left[N\left(t\right)\right]=\sum_{n=1}^{\infty}P\left\{N\left(t\right)\geq n\right\}$
se tiene que

\begin{eqnarray*}
\esp\left[N\left(t\right)\right]=\sum_{n=1}^{\infty}F^{n\star}\left(t\right)
\end{eqnarray*}

\begin{Prop}
Para cada $t\geq0$, la funci\'on generadora de momentos $\esp\left[e^{\alpha N\left(t\right)}\right]$ existe para alguna $\alpha$ en una vecindad del 0, y de aqu\'i que $\esp\left[N\left(t\right)^{m}\right]<\infty$, para $m\geq1$.
\end{Prop}

\begin{Ejem}[\textbf{Proceso Poisson}]

Suponga que se tienen tiempos de inter-renovaci\'on \textit{i.i.d.} del proceso de renovaci\'on $N\left(t\right)$ tienen distribuci\'on exponencial $F\left(t\right)=q-e^{-\lambda t}$ con tasa $\lambda$. Entonces $N\left(t\right)$ es un proceso Poisson con tasa $\lambda$.

\end{Ejem}


\begin{Note}
Si el primer tiempo de renovaci\'on $\xi_{1}$ no tiene la misma distribuci\'on que el resto de las $\xi_{n}$, para $n\geq2$, a $N\left(t\right)$ se le llama Proceso de Renovaci\'on retardado, donde si $\xi$ tiene distribuci\'on $G$, entonces el tiempo $T_{n}$ de la $n$-\'esima renovaci\'on tiene distribuci\'on $G\star F^{\left(n-1\right)\star}\left(t\right)$
\end{Note}


\begin{Teo}
Para una constante $\mu\leq\infty$ ( o variable aleatoria), las siguientes expresiones son equivalentes:

\begin{eqnarray}
lim_{n\rightarrow\infty}n^{-1}T_{n}&=&\mu,\textrm{ c.s.}\\
lim_{t\rightarrow\infty}t^{-1}N\left(t\right)&=&1/\mu,\textrm{ c.s.}
\end{eqnarray}
\end{Teo}


Es decir, $T_{n}$ satisface la Ley Fuerte de los Grandes N\'umeros s\'i y s\'olo s\'i $N\left/t\right)$ la cumple.


\begin{Coro}[Ley Fuerte de los Grandes N\'umeros para Procesos de Renovaci\'on]
Si $N\left(t\right)$ es un proceso de renovaci\'on cuyos tiempos de inter-renovaci\'on tienen media $\mu\leq\infty$, entonces
\begin{eqnarray}
t^{-1}N\left(t\right)\rightarrow 1/\mu,\textrm{ c.s. cuando }t\rightarrow\infty.
\end{eqnarray}

\end{Coro}


Considerar el proceso estoc\'astico de valores reales $\left\{Z\left(t\right):t\geq0\right\}$ en el mismo espacio de probabilidad que $N\left(t\right)$

\begin{Def}
Para el proceso $\left\{Z\left(t\right):t\geq0\right\}$ se define la fluctuaci\'on m\'axima de $Z\left(t\right)$ en el intervalo $\left(T_{n-1},T_{n}\right]$:
\begin{eqnarray*}
M_{n}=\sup_{T_{n-1}<t\leq T_{n}}|Z\left(t\right)-Z\left(T_{n-1}\right)|
\end{eqnarray*}
\end{Def}

\begin{Teo}
Sup\'ongase que $n^{-1}T_{n}\rightarrow\mu$ c.s. cuando $n\rightarrow\infty$, donde $\mu\leq\infty$ es una constante o variable aleatoria. Sea $a$ una constante o variable aleatoria que puede ser infinita cuando $\mu$ es finita, y considere las expresiones l\'imite:
\begin{eqnarray}
lim_{n\rightarrow\infty}n^{-1}Z\left(T_{n}\right)&=&a,\textrm{ c.s.}\\
lim_{t\rightarrow\infty}t^{-1}Z\left(t\right)&=&a/\mu,\textrm{ c.s.}
\end{eqnarray}
La segunda expresi\'on implica la primera. Conversamente, la primera implica la segunda si el proceso $Z\left(t\right)$ es creciente, o si $lim_{n\rightarrow\infty}n^{-1}M_{n}=0$ c.s.
\end{Teo}

\begin{Coro}
Si $N\left(t\right)$ es un proceso de renovaci\'on, y $\left(Z\left(T_{n}\right)-Z\left(T_{n-1}\right),M_{n}\right)$, para $n\geq1$, son variables aleatorias independientes e id\'enticamente distribuidas con media finita, entonces,
\begin{eqnarray}
lim_{t\rightarrow\infty}t^{-1}Z\left(t\right)\rightarrow\frac{\esp\left[Z\left(T_{1}\right)-Z\left(T_{0}\right)\right]}{\esp\left[T_{1}\right]},\textrm{ c.s. cuando  }t\rightarrow\infty.
\end{eqnarray}
\end{Coro}


Sup\'ongase que $N\left(t\right)$ es un proceso de renovaci\'on con distribuci\'on $F$ con media finita $\mu$.

\begin{Def}
La funci\'on de renovaci\'on asociada con la distribuci\'on $F$, del proceso $N\left(t\right)$, es
\begin{eqnarray*}
U\left(t\right)=\sum_{n=1}^{\infty}F^{n\star}\left(t\right),\textrm{   }t\geq0,
\end{eqnarray*}
donde $F^{0\star}\left(t\right)=\indora\left(t\geq0\right)$.
\end{Def}


\begin{Prop}
Sup\'ongase que la distribuci\'on de inter-renovaci\'on $F$ tiene densidad $f$. Entonces $U\left(t\right)$ tambi\'en tiene densidad, para $t>0$, y es $U^{'}\left(t\right)=\sum_{n=0}^{\infty}f^{n\star}\left(t\right)$. Adem\'as
\begin{eqnarray*}
\prob\left\{N\left(t\right)>N\left(t-\right)\right\}=0\textrm{,   }t\geq0.
\end{eqnarray*}
\end{Prop}

\begin{Def}
La Transformada de Laplace-Stieljes de $F$ est\'a dada por

\begin{eqnarray*}
\hat{F}\left(\alpha\right)=\int_{\rea_{+}}e^{-\alpha t}dF\left(t\right)\textrm{,  }\alpha\geq0.
\end{eqnarray*}
\end{Def}

Entonces

\begin{eqnarray*}
\hat{U}\left(\alpha\right)=\sum_{n=0}^{\infty}\hat{F^{n\star}}\left(\alpha\right)=\sum_{n=0}^{\infty}\hat{F}\left(\alpha\right)^{n}=\frac{1}{1-\hat{F}\left(\alpha\right)}.
\end{eqnarray*}


\begin{Prop}
La Transformada de Laplace $\hat{U}\left(\alpha\right)$ y $\hat{F}\left(\alpha\right)$ determina una a la otra de manera \'unica por la relaci\'on $\hat{U}\left(\alpha\right)=\frac{1}{1-\hat{F}\left(\alpha\right)}$.
\end{Prop}


\begin{Note}
Un proceso de renovaci\'on $N\left(t\right)$ cuyos tiempos de inter-renovaci\'on tienen media finita, es un proceso Poisson con tasa $\lambda$ si y s\'olo s\'i $\esp\left[U\left(t\right)\right]=\lambda t$, para $t\geq0$.
\end{Note}


\begin{Teo}
Sea $N\left(t\right)$ un proceso puntual simple con puntos de localizaci\'on $T_{n}$ tal que $\eta\left(t\right)=\esp\left[N\left(\right)\right]$ es finita para cada $t$. Entonces para cualquier funci\'on $f:\rea_{+}\rightarrow\rea$,
\begin{eqnarray*}
\esp\left[\sum_{n=1}^{N\left(\right)}f\left(T_{n}\right)\right]=\int_{\left(0,t\right]}f\left(s\right)d\eta\left(s\right)\textrm{,  }t\geq0,
\end{eqnarray*}
suponiendo que la integral exista. Adem\'as si $X_{1},X_{2},\ldots$ son variables aleatorias definidas en el mismo espacio de probabilidad que el proceso $N\left(t\right)$ tal que $\esp\left[X_{n}|T_{n}=s\right]=f\left(s\right)$, independiente de $n$. Entonces
\begin{eqnarray*}
\esp\left[\sum_{n=1}^{N\left(t\right)}X_{n}\right]=\int_{\left(0,t\right]}f\left(s\right)d\eta\left(s\right)\textrm{,  }t\geq0,
\end{eqnarray*} 
suponiendo que la integral exista. 
\end{Teo}

\begin{Coro}[Identidad de Wald para Renovaciones]
Para el proceso de renovaci\'on $N\left(t\right)$,
\begin{eqnarray*}
\esp\left[T_{N\left(t\right)+1}\right]=\mu\esp\left[N\left(t\right)+1\right]\textrm{,  }t\geq0,
\end{eqnarray*}  
\end{Coro}


\begin{Def}
Sea $h\left(t\right)$ funci\'on de valores reales en $\rea$ acotada en intervalos finitos e igual a cero para $t<0$ La ecuaci\'on de renovaci\'on para $h\left(t\right)$ y la distribuci\'on $F$ es

\begin{eqnarray}%\label{Ec.Renovacion}
H\left(t\right)=h\left(t\right)+\int_{\left[0,t\right]}H\left(t-s\right)dF\left(s\right)\textrm{,    }t\geq0,
\end{eqnarray}
donde $H\left(t\right)$ es una funci\'on de valores reales. Esto es $H=h+F\star H$. Decimos que $H\left(t\right)$ es soluci\'on de esta ecuaci\'on si satisface la ecuaci\'on, y es acotada en intervalos finitos e iguales a cero para $t<0$.
\end{Def}

\begin{Prop}
La funci\'on $U\star h\left(t\right)$ es la \'unica soluci\'on de la ecuaci\'on de renovaci\'on (\ref{Ec.Renovacion}).
\end{Prop}

\begin{Teo}[Teorema Renovaci\'on Elemental]
\begin{eqnarray*}
t^{-1}U\left(t\right)\rightarrow 1/\mu\textrm{,    cuando }t\rightarrow\infty.
\end{eqnarray*}
\end{Teo}

Sup\'ongase que $N\left(t\right)$ es un proceso de renovaci\'on con distribuci\'on $F$ con media finita $\mu$.

\begin{Def}
La funci\'on de renovaci\'on asociada con la distribuci\'on $F$, del proceso $N\left(t\right)$, es
\begin{eqnarray*}
U\left(t\right)=\sum_{n=1}^{\infty}F^{n\star}\left(t\right),\textrm{   }t\geq0,
\end{eqnarray*}
donde $F^{0\star}\left(t\right)=\indora\left(t\geq0\right)$.
\end{Def}


\begin{Prop}
Sup\'ongase que la distribuci\'on de inter-renovaci\'on $F$ tiene densidad $f$. Entonces $U\left(t\right)$ tambi\'en tiene densidad, para $t>0$, y es $U^{'}\left(t\right)=\sum_{n=0}^{\infty}f^{n\star}\left(t\right)$. Adem\'as
\begin{eqnarray*}
\prob\left\{N\left(t\right)>N\left(t-\right)\right\}=0\textrm{,   }t\geq0.
\end{eqnarray*}
\end{Prop}

\begin{Def}
La Transformada de Laplace-Stieljes de $F$ est\'a dada por

\begin{eqnarray*}
\hat{F}\left(\alpha\right)=\int_{\rea_{+}}e^{-\alpha t}dF\left(t\right)\textrm{,  }\alpha\geq0.
\end{eqnarray*}
\end{Def}

Entonces

\begin{eqnarray*}
\hat{U}\left(\alpha\right)=\sum_{n=0}^{\infty}\hat{F^{n\star}}\left(\alpha\right)=\sum_{n=0}^{\infty}\hat{F}\left(\alpha\right)^{n}=\frac{1}{1-\hat{F}\left(\alpha\right)}.
\end{eqnarray*}


\begin{Prop}
La Transformada de Laplace $\hat{U}\left(\alpha\right)$ y $\hat{F}\left(\alpha\right)$ determina una a la otra de manera \'unica por la relaci\'on $\hat{U}\left(\alpha\right)=\frac{1}{1-\hat{F}\left(\alpha\right)}$.
\end{Prop}


\begin{Note}
Un proceso de renovaci\'on $N\left(t\right)$ cuyos tiempos de inter-renovaci\'on tienen media finita, es un proceso Poisson con tasa $\lambda$ si y s\'olo s\'i $\esp\left[U\left(t\right)\right]=\lambda t$, para $t\geq0$.
\end{Note}


\begin{Teo}
Sea $N\left(t\right)$ un proceso puntual simple con puntos de localizaci\'on $T_{n}$ tal que $\eta\left(t\right)=\esp\left[N\left(\right)\right]$ es finita para cada $t$. Entonces para cualquier funci\'on $f:\rea_{+}\rightarrow\rea$,
\begin{eqnarray*}
\esp\left[\sum_{n=1}^{N\left(\right)}f\left(T_{n}\right)\right]=\int_{\left(0,t\right]}f\left(s\right)d\eta\left(s\right)\textrm{,  }t\geq0,
\end{eqnarray*}
suponiendo que la integral exista. Adem\'as si $X_{1},X_{2},\ldots$ son variables aleatorias definidas en el mismo espacio de probabilidad que el proceso $N\left(t\right)$ tal que $\esp\left[X_{n}|T_{n}=s\right]=f\left(s\right)$, independiente de $n$. Entonces
\begin{eqnarray*}
\esp\left[\sum_{n=1}^{N\left(t\right)}X_{n}\right]=\int_{\left(0,t\right]}f\left(s\right)d\eta\left(s\right)\textrm{,  }t\geq0,
\end{eqnarray*} 
suponiendo que la integral exista. 
\end{Teo}

\begin{Coro}[Identidad de Wald para Renovaciones]
Para el proceso de renovaci\'on $N\left(t\right)$,
\begin{eqnarray*}
\esp\left[T_{N\left(t\right)+1}\right]=\mu\esp\left[N\left(t\right)+1\right]\textrm{,  }t\geq0,
\end{eqnarray*}  
\end{Coro}


\begin{Def}
Sea $h\left(t\right)$ funci\'on de valores reales en $\rea$ acotada en intervalos finitos e igual a cero para $t<0$ La ecuaci\'on de renovaci\'on para $h\left(t\right)$ y la distribuci\'on $F$ es

\begin{eqnarray}%\label{Ec.Renovacion}
H\left(t\right)=h\left(t\right)+\int_{\left[0,t\right]}H\left(t-s\right)dF\left(s\right)\textrm{,    }t\geq0,
\end{eqnarray}
donde $H\left(t\right)$ es una funci\'on de valores reales. Esto es $H=h+F\star H$. Decimos que $H\left(t\right)$ es soluci\'on de esta ecuaci\'on si satisface la ecuaci\'on, y es acotada en intervalos finitos e iguales a cero para $t<0$.
\end{Def}

\begin{Prop}
La funci\'on $U\star h\left(t\right)$ es la \'unica soluci\'on de la ecuaci\'on de renovaci\'on (\ref{Ec.Renovacion}).
\end{Prop}

\begin{Teo}[Teorema Renovaci\'on Elemental]
\begin{eqnarray*}
t^{-1}U\left(t\right)\rightarrow 1/\mu\textrm{,    cuando }t\rightarrow\infty.
\end{eqnarray*}
\end{Teo}


\begin{Note} Una funci\'on $h:\rea_{+}\rightarrow\rea$ es Directamente Riemann Integrable en los siguientes casos:
\begin{itemize}
\item[a)] $h\left(t\right)\geq0$ es decreciente y Riemann Integrable.
\item[b)] $h$ es continua excepto posiblemente en un conjunto de Lebesgue de medida 0, y $|h\left(t\right)|\leq b\left(t\right)$, donde $b$ es DRI.
\end{itemize}
\end{Note}

\begin{Teo}[Teorema Principal de Renovaci\'on]
Si $F$ es no aritm\'etica y $h\left(t\right)$ es Directamente Riemann Integrable (DRI), entonces

\begin{eqnarray*}
lim_{t\rightarrow\infty}U\star h=\frac{1}{\mu}\int_{\rea_{+}}h\left(s\right)ds.
\end{eqnarray*}
\end{Teo}

\begin{Prop}
Cualquier funci\'on $H\left(t\right)$ acotada en intervalos finitos y que es 0 para $t<0$ puede expresarse como
\begin{eqnarray*}
H\left(t\right)=U\star h\left(t\right)\textrm{,  donde }h\left(t\right)=H\left(t\right)-F\star H\left(t\right)
\end{eqnarray*}
\end{Prop}

\begin{Def}
Un proceso estoc\'astico $X\left(t\right)$ es crudamente regenerativo en un tiempo aleatorio positivo $T$ si
\begin{eqnarray*}
\esp\left[X\left(T+t\right)|T\right]=\esp\left[X\left(t\right)\right]\textrm{, para }t\geq0,\end{eqnarray*}
y con las esperanzas anteriores finitas.
\end{Def}

\begin{Prop}
Sup\'ongase que $X\left(t\right)$ es un proceso crudamente regenerativo en $T$, que tiene distribuci\'on $F$. Si $\esp\left[X\left(t\right)\right]$ es acotado en intervalos finitos, entonces
\begin{eqnarray*}
\esp\left[X\left(t\right)\right]=U\star h\left(t\right)\textrm{,  donde }h\left(t\right)=\esp\left[X\left(t\right)\indora\left(T>t\right)\right].
\end{eqnarray*}
\end{Prop}

\begin{Teo}[Regeneraci\'on Cruda]
Sup\'ongase que $X\left(t\right)$ es un proceso con valores positivo crudamente regenerativo en $T$, y def\'inase $M=\sup\left\{|X\left(t\right)|:t\leq T\right\}$. Si $T$ es no aritm\'etico y $M$ y $MT$ tienen media finita, entonces
\begin{eqnarray*}
lim_{t\rightarrow\infty}\esp\left[X\left(t\right)\right]=\frac{1}{\mu}\int_{\rea_{+}}h\left(s\right)ds,
\end{eqnarray*}
donde $h\left(t\right)=\esp\left[X\left(t\right)\indora\left(T>t\right)\right]$.
\end{Teo}


\begin{Note} Una funci\'on $h:\rea_{+}\rightarrow\rea$ es Directamente Riemann Integrable en los siguientes casos:
\begin{itemize}
\item[a)] $h\left(t\right)\geq0$ es decreciente y Riemann Integrable.
\item[b)] $h$ es continua excepto posiblemente en un conjunto de Lebesgue de medida 0, y $|h\left(t\right)|\leq b\left(t\right)$, donde $b$ es DRI.
\end{itemize}
\end{Note}

\begin{Teo}[Teorema Principal de Renovaci\'on]
Si $F$ es no aritm\'etica y $h\left(t\right)$ es Directamente Riemann Integrable (DRI), entonces

\begin{eqnarray*}
lim_{t\rightarrow\infty}U\star h=\frac{1}{\mu}\int_{\rea_{+}}h\left(s\right)ds.
\end{eqnarray*}
\end{Teo}

\begin{Prop}
Cualquier funci\'on $H\left(t\right)$ acotada en intervalos finitos y que es 0 para $t<0$ puede expresarse como
\begin{eqnarray*}
H\left(t\right)=U\star h\left(t\right)\textrm{,  donde }h\left(t\right)=H\left(t\right)-F\star H\left(t\right)
\end{eqnarray*}
\end{Prop}

\begin{Def}
Un proceso estoc\'astico $X\left(t\right)$ es crudamente regenerativo en un tiempo aleatorio positivo $T$ si
\begin{eqnarray*}
\esp\left[X\left(T+t\right)|T\right]=\esp\left[X\left(t\right)\right]\textrm{, para }t\geq0,\end{eqnarray*}
y con las esperanzas anteriores finitas.
\end{Def}

\begin{Prop}
Sup\'ongase que $X\left(t\right)$ es un proceso crudamente regenerativo en $T$, que tiene distribuci\'on $F$. Si $\esp\left[X\left(t\right)\right]$ es acotado en intervalos finitos, entonces
\begin{eqnarray*}
\esp\left[X\left(t\right)\right]=U\star h\left(t\right)\textrm{,  donde }h\left(t\right)=\esp\left[X\left(t\right)\indora\left(T>t\right)\right].
\end{eqnarray*}
\end{Prop}

\begin{Teo}[Regeneraci\'on Cruda]
Sup\'ongase que $X\left(t\right)$ es un proceso con valores positivo crudamente regenerativo en $T$, y def\'inase $M=\sup\left\{|X\left(t\right)|:t\leq T\right\}$. Si $T$ es no aritm\'etico y $M$ y $MT$ tienen media finita, entonces
\begin{eqnarray*}
lim_{t\rightarrow\infty}\esp\left[X\left(t\right)\right]=\frac{1}{\mu}\int_{\rea_{+}}h\left(s\right)ds,
\end{eqnarray*}
donde $h\left(t\right)=\esp\left[X\left(t\right)\indora\left(T>t\right)\right]$.
\end{Teo}

\begin{Def}
Para el proceso $\left\{\left(N\left(t\right),X\left(t\right)\right):t\geq0\right\}$, sus trayectoria muestrales en el intervalo de tiempo $\left[T_{n-1},T_{n}\right)$ est\'an descritas por
\begin{eqnarray*}
\zeta_{n}=\left(\xi_{n},\left\{X\left(T_{n-1}+t\right):0\leq t<\xi_{n}\right\}\right)
\end{eqnarray*}
Este $\zeta_{n}$ es el $n$-\'esimo segmento del proceso. El proceso es regenerativo sobre los tiempos $T_{n}$ si sus segmentos $\zeta_{n}$ son independientes e id\'enticamennte distribuidos.
\end{Def}


\begin{Note}
Si $\tilde{X}\left(t\right)$ con espacio de estados $\tilde{S}$ es regenerativo sobre $T_{n}$, entonces $X\left(t\right)=f\left(\tilde{X}\left(t\right)\right)$ tambi\'en es regenerativo sobre $T_{n}$, para cualquier funci\'on $f:\tilde{S}\rightarrow S$.
\end{Note}

\begin{Note}
Los procesos regenerativos son crudamente regenerativos, pero no al rev\'es.
\end{Note}


\begin{Note}
Un proceso estoc\'astico a tiempo continuo o discreto es regenerativo si existe un proceso de renovaci\'on  tal que los segmentos del proceso entre tiempos de renovaci\'on sucesivos son i.i.d., es decir, para $\left\{X\left(t\right):t\geq0\right\}$ proceso estoc\'astico a tiempo continuo con espacio de estados $S$, espacio m\'etrico.
\end{Note}

Para $\left\{X\left(t\right):t\geq0\right\}$ Proceso Estoc\'astico a tiempo continuo con estado de espacios $S$, que es un espacio m\'etrico, con trayectorias continuas por la derecha y con l\'imites por la izquierda c.s. Sea $N\left(t\right)$ un proceso de renovaci\'on en $\rea_{+}$ definido en el mismo espacio de probabilidad que $X\left(t\right)$, con tiempos de renovaci\'on $T$ y tiempos de inter-renovaci\'on $\xi_{n}=T_{n}-T_{n-1}$, con misma distribuci\'on $F$ de media finita $\mu$.



\begin{Def}
Para el proceso $\left\{\left(N\left(t\right),X\left(t\right)\right):t\geq0\right\}$, sus trayectoria muestrales en el intervalo de tiempo $\left[T_{n-1},T_{n}\right)$ est\'an descritas por
\begin{eqnarray*}
\zeta_{n}=\left(\xi_{n},\left\{X\left(T_{n-1}+t\right):0\leq t<\xi_{n}\right\}\right)
\end{eqnarray*}
Este $\zeta_{n}$ es el $n$-\'esimo segmento del proceso. El proceso es regenerativo sobre los tiempos $T_{n}$ si sus segmentos $\zeta_{n}$ son independientes e id\'enticamennte distribuidos.
\end{Def}

\begin{Note}
Un proceso regenerativo con media de la longitud de ciclo finita es llamado positivo recurrente.
\end{Note}

\begin{Teo}[Procesos Regenerativos]
Suponga que el proceso
\end{Teo}


\begin{Def}[Renewal Process Trinity]
Para un proceso de renovaci\'on $N\left(t\right)$, los siguientes procesos proveen de informaci\'on sobre los tiempos de renovaci\'on.
\begin{itemize}
\item $A\left(t\right)=t-T_{N\left(t\right)}$, el tiempo de recurrencia hacia atr\'as al tiempo $t$, que es el tiempo desde la \'ultima renovaci\'on para $t$.

\item $B\left(t\right)=T_{N\left(t\right)+1}-t$, el tiempo de recurrencia hacia adelante al tiempo $t$, residual del tiempo de renovaci\'on, que es el tiempo para la pr\'oxima renovaci\'on despu\'es de $t$.

\item $L\left(t\right)=\xi_{N\left(t\right)+1}=A\left(t\right)+B\left(t\right)$, la longitud del intervalo de renovaci\'on que contiene a $t$.
\end{itemize}
\end{Def}

\begin{Note}
El proceso tridimensional $\left(A\left(t\right),B\left(t\right),L\left(t\right)\right)$ es regenerativo sobre $T_{n}$, y por ende cada proceso lo es. Cada proceso $A\left(t\right)$ y $B\left(t\right)$ son procesos de MArkov a tiempo continuo con trayectorias continuas por partes en el espacio de estados $\rea_{+}$. Una expresi\'on conveniente para su distribuci\'on conjunta es, para $0\leq x<t,y\geq0$
\begin{equation}\label{NoRenovacion}
P\left\{A\left(t\right)>x,B\left(t\right)>y\right\}=
P\left\{N\left(t+y\right)-N\left((t-x)\right)=0\right\}
\end{equation}
\end{Note}

\begin{Ejem}[Tiempos de recurrencia Poisson]
Si $N\left(t\right)$ es un proceso Poisson con tasa $\lambda$, entonces de la expresi\'on (\ref{NoRenovacion}) se tiene que

\begin{eqnarray*}
\begin{array}{lc}
P\left\{A\left(t\right)>x,B\left(t\right)>y\right\}=e^{-\lambda\left(x+y\right)},&0\leq x<t,y\geq0,
\end{array}
\end{eqnarray*}
que es la probabilidad Poisson de no renovaciones en un intervalo de longitud $x+y$.

\end{Ejem}

%\begin{Note}
Una cadena de Markov erg\'odica tiene la propiedad de ser estacionaria si la distribuci\'on de su estado al tiempo $0$ es su distribuci\'on estacionaria.
%\end{Note}


\begin{Def}
Un proceso estoc\'astico a tiempo continuo $\left\{X\left(t\right):t\geq0\right\}$ en un espacio general es estacionario si sus distribuciones finito dimensionales son invariantes bajo cualquier  traslado: para cada $0\leq s_{1}<s_{2}<\cdots<s_{k}$ y $t\geq0$,
\begin{eqnarray*}
\left(X\left(s_{1}+t\right),\ldots,X\left(s_{k}+t\right)\right)=_{d}\left(X\left(s_{1}\right),\ldots,X\left(s_{k}\right)\right).
\end{eqnarray*}
\end{Def}

\begin{Note}
Un proceso de Markov es estacionario si $X\left(t\right)=_{d}X\left(0\right)$, $t\geq0$.
\end{Note}

Considerese el proceso $N\left(t\right)=\sum_{n}\indora\left(\tau_{n}\leq t\right)$ en $\rea_{+}$, con puntos $0<\tau_{1}<\tau_{2}<\cdots$.

\begin{Prop}
Si $N$ es un proceso puntual estacionario y $\esp\left[N\left(1\right)\right]<\infty$, entonces $\esp\left[N\left(t\right)\right]=t\esp\left[N\left(1\right)\right]$, $t\geq0$

\end{Prop}

\begin{Teo}
Los siguientes enunciados son equivalentes
\begin{itemize}
\item[i)] El proceso retardado de renovaci\'on $N$ es estacionario.

\item[ii)] EL proceso de tiempos de recurrencia hacia adelante $B\left(t\right)$ es estacionario.


\item[iii)] $\esp\left[N\left(t\right)\right]=t/\mu$,


\item[iv)] $G\left(t\right)=F_{e}\left(t\right)=\frac{1}{\mu}\int_{0}^{t}\left[1-F\left(s\right)\right]ds$
\end{itemize}
Cuando estos enunciados son ciertos, $P\left\{B\left(t\right)\leq x\right\}=F_{e}\left(x\right)$, para $t,x\geq0$.

\end{Teo}

\begin{Note}
Una consecuencia del teorema anterior es que el Proceso Poisson es el \'unico proceso sin retardo que es estacionario.
\end{Note}

\begin{Coro}
El proceso de renovaci\'on $N\left(t\right)$ sin retardo, y cuyos tiempos de inter renonaci\'on tienen media finita, es estacionario si y s\'olo si es un proceso Poisson.

\end{Coro}



\subsection{Renewal and Regenerative Processes: Serfozo\cite{Serfozo}}
\begin{Def}\label{Def.Tn}
Sean $0\leq T_{1}\leq T_{2}\leq \ldots$ son tiempos aleatorios infinitos en los cuales ocurren ciertos eventos. El n\'umero de tiempos $T_{n}$ en el intervalo $\left[0,t\right)$ es

\begin{eqnarray}
N\left(t\right)=\sum_{n=1}^{\infty}\indora\left(T_{n}\leq t\right),
\end{eqnarray}
para $t\geq0$.
\end{Def}

Si se consideran los puntos $T_{n}$ como elementos de $\rea_{+}$, y $N\left(t\right)$ es el n\'umero de puntos en $\rea$. El proceso denotado por $\left\{N\left(t\right):t\geq0\right\}$, denotado por $N\left(t\right)$, es un proceso puntual en $\rea_{+}$. Los $T_{n}$ son los tiempos de ocurrencia, el proceso puntual $N\left(t\right)$ es simple si su n\'umero de ocurrencias son distintas: $0<T_{1}<T_{2}<\ldots$ casi seguramente.

\begin{Def}
Un proceso puntual $N\left(t\right)$ es un proceso de renovaci\'on si los tiempos de interocurrencia $\xi_{n}=T_{n}-T_{n-1}$, para $n\geq1$, son independientes e identicamente distribuidos con distribuci\'on $F$, donde $F\left(0\right)=0$ y $T_{0}=0$. Los $T_{n}$ son llamados tiempos de renovaci\'on, referente a la independencia o renovaci\'on de la informaci\'on estoc\'astica en estos tiempos. Los $\xi_{n}$ son los tiempos de inter-renovaci\'on, y $N\left(t\right)$ es el n\'umero de renovaciones en el intervalo $\left[0,t\right)$
\end{Def}


\begin{Note}
Para definir un proceso de renovaci\'on para cualquier contexto, solamente hay que especificar una distribuci\'on $F$, con $F\left(0\right)=0$, para los tiempos de inter-renovaci\'on. La funci\'on $F$ en turno degune las otra variables aleatorias. De manera formal, existe un espacio de probabilidad y una sucesi\'on de variables aleatorias $\xi_{1},\xi_{2},\ldots$ definidas en este con distribuci\'on $F$. Entonces las otras cantidades son $T_{n}=\sum_{k=1}^{n}\xi_{k}$ y $N\left(t\right)=\sum_{n=1}^{\infty}\indora\left(T_{n}\leq t\right)$, donde $T_{n}\rightarrow\infty$ casi seguramente por la Ley Fuerte de los Grandes N\'umeros.
\end{Note}


Los tiempos $T_{n}$ est\'an relacionados con los conteos de $N\left(t\right)$ por

\begin{eqnarray*}
\left\{N\left(t\right)\geq n\right\}&=&\left\{T_{n}\leq t\right\}\\
T_{N\left(t\right)}\leq &t&<T_{N\left(t\right)+1},
\end{eqnarray*}

adem\'as $N\left(T_{n}\right)=n$, y 

\begin{eqnarray*}
N\left(t\right)=\max\left\{n:T_{n}\leq t\right\}=\min\left\{n:T_{n+1}>t\right\}
\end{eqnarray*}

Por propiedades de la convoluci\'on se sabe que

\begin{eqnarray*}
P\left\{T_{n}\leq t\right\}=F^{n\star}\left(t\right)
\end{eqnarray*}
que es la $n$-\'esima convoluci\'on de $F$. Entonces 

\begin{eqnarray*}
\left\{N\left(t\right)\geq n\right\}&=&\left\{T_{n}\leq t\right\}\\
P\left\{N\left(t\right)\leq n\right\}&=&1-F^{\left(n+1\right)\star}\left(t\right)
\end{eqnarray*}

Adem\'as usando el hecho de que $\esp\left[N\left(t\right)\right]=\sum_{n=1}^{\infty}P\left\{N\left(t\right)\geq n\right\}$
se tiene que

\begin{eqnarray*}
\esp\left[N\left(t\right)\right]=\sum_{n=1}^{\infty}F^{n\star}\left(t\right)
\end{eqnarray*}

\begin{Prop}
Para cada $t\geq0$, la funci\'on generadora de momentos $\esp\left[e^{\alpha N\left(t\right)}\right]$ existe para alguna $\alpha$ en una vecindad del 0, y de aqu\'i que $\esp\left[N\left(t\right)^{m}\right]<\infty$, para $m\geq1$.
\end{Prop}


\begin{Note}
Si el primer tiempo de renovaci\'on $\xi_{1}$ no tiene la misma distribuci\'on que el resto de las $\xi_{n}$, para $n\geq2$, a $N\left(t\right)$ se le llama Proceso de Renovaci\'on retardado, donde si $\xi$ tiene distribuci\'on $G$, entonces el tiempo $T_{n}$ de la $n$-\'esima renovaci\'on tiene distribuci\'on $G\star F^{\left(n-1\right)\star}\left(t\right)$
\end{Note}


\begin{Teo}
Para una constante $\mu\leq\infty$ ( o variable aleatoria), las siguientes expresiones son equivalentes:

\begin{eqnarray}
lim_{n\rightarrow\infty}n^{-1}T_{n}&=&\mu,\textrm{ c.s.}\\
lim_{t\rightarrow\infty}t^{-1}N\left(t\right)&=&1/\mu,\textrm{ c.s.}
\end{eqnarray}
\end{Teo}


Es decir, $T_{n}$ satisface la Ley Fuerte de los Grandes N\'umeros s\'i y s\'olo s\'i $N\left/t\right)$ la cumple.


\begin{Coro}[Ley Fuerte de los Grandes N\'umeros para Procesos de Renovaci\'on]
Si $N\left(t\right)$ es un proceso de renovaci\'on cuyos tiempos de inter-renovaci\'on tienen media $\mu\leq\infty$, entonces
\begin{eqnarray}
t^{-1}N\left(t\right)\rightarrow 1/\mu,\textrm{ c.s. cuando }t\rightarrow\infty.
\end{eqnarray}

\end{Coro}


Considerar el proceso estoc\'astico de valores reales $\left\{Z\left(t\right):t\geq0\right\}$ en el mismo espacio de probabilidad que $N\left(t\right)$

\begin{Def}
Para el proceso $\left\{Z\left(t\right):t\geq0\right\}$ se define la fluctuaci\'on m\'axima de $Z\left(t\right)$ en el intervalo $\left(T_{n-1},T_{n}\right]$:
\begin{eqnarray*}
M_{n}=\sup_{T_{n-1}<t\leq T_{n}}|Z\left(t\right)-Z\left(T_{n-1}\right)|
\end{eqnarray*}
\end{Def}

\begin{Teo}
Sup\'ongase que $n^{-1}T_{n}\rightarrow\mu$ c.s. cuando $n\rightarrow\infty$, donde $\mu\leq\infty$ es una constante o variable aleatoria. Sea $a$ una constante o variable aleatoria que puede ser infinita cuando $\mu$ es finita, y considere las expresiones l\'imite:
\begin{eqnarray}
lim_{n\rightarrow\infty}n^{-1}Z\left(T_{n}\right)&=&a,\textrm{ c.s.}\\
lim_{t\rightarrow\infty}t^{-1}Z\left(t\right)&=&a/\mu,\textrm{ c.s.}
\end{eqnarray}
La segunda expresi\'on implica la primera. Conversamente, la primera implica la segunda si el proceso $Z\left(t\right)$ es creciente, o si $lim_{n\rightarrow\infty}n^{-1}M_{n}=0$ c.s.
\end{Teo}

\begin{Coro}
Si $N\left(t\right)$ es un proceso de renovaci\'on, y $\left(Z\left(T_{n}\right)-Z\left(T_{n-1}\right),M_{n}\right)$, para $n\geq1$, son variables aleatorias independientes e id\'enticamente distribuidas con media finita, entonces,
\begin{eqnarray}
lim_{t\rightarrow\infty}t^{-1}Z\left(t\right)\rightarrow\frac{\esp\left[Z\left(T_{1}\right)-Z\left(T_{0}\right)\right]}{\esp\left[T_{1}\right]},\textrm{ c.s. cuando  }t\rightarrow\infty.
\end{eqnarray}
\end{Coro}


Sup\'ongase que $N\left(t\right)$ es un proceso de renovaci\'on con distribuci\'on $F$ con media finita $\mu$.

\begin{Def}
La funci\'on de renovaci\'on asociada con la distribuci\'on $F$, del proceso $N\left(t\right)$, es
\begin{eqnarray*}
U\left(t\right)=\sum_{n=1}^{\infty}F^{n\star}\left(t\right),\textrm{   }t\geq0,
\end{eqnarray*}
donde $F^{0\star}\left(t\right)=\indora\left(t\geq0\right)$.
\end{Def}


\begin{Prop}
Sup\'ongase que la distribuci\'on de inter-renovaci\'on $F$ tiene densidad $f$. Entonces $U\left(t\right)$ tambi\'en tiene densidad, para $t>0$, y es $U^{'}\left(t\right)=\sum_{n=0}^{\infty}f^{n\star}\left(t\right)$. Adem\'as
\begin{eqnarray*}
\prob\left\{N\left(t\right)>N\left(t-\right)\right\}=0\textrm{,   }t\geq0.
\end{eqnarray*}
\end{Prop}

\begin{Def}
La Transformada de Laplace-Stieljes de $F$ est\'a dada por

\begin{eqnarray*}
\hat{F}\left(\alpha\right)=\int_{\rea_{+}}e^{-\alpha t}dF\left(t\right)\textrm{,  }\alpha\geq0.
\end{eqnarray*}
\end{Def}

Entonces

\begin{eqnarray*}
\hat{U}\left(\alpha\right)=\sum_{n=0}^{\infty}\hat{F^{n\star}}\left(\alpha\right)=\sum_{n=0}^{\infty}\hat{F}\left(\alpha\right)^{n}=\frac{1}{1-\hat{F}\left(\alpha\right)}.
\end{eqnarray*}


\begin{Prop}
La Transformada de Laplace $\hat{U}\left(\alpha\right)$ y $\hat{F}\left(\alpha\right)$ determina una a la otra de manera \'unica por la relaci\'on $\hat{U}\left(\alpha\right)=\frac{1}{1-\hat{F}\left(\alpha\right)}$.
\end{Prop}


\begin{Note}
Un proceso de renovaci\'on $N\left(t\right)$ cuyos tiempos de inter-renovaci\'on tienen media finita, es un proceso Poisson con tasa $\lambda$ si y s\'olo s\'i $\esp\left[U\left(t\right)\right]=\lambda t$, para $t\geq0$.
\end{Note}


\begin{Teo}
Sea $N\left(t\right)$ un proceso puntual simple con puntos de localizaci\'on $T_{n}$ tal que $\eta\left(t\right)=\esp\left[N\left(\right)\right]$ es finita para cada $t$. Entonces para cualquier funci\'on $f:\rea_{+}\rightarrow\rea$,
\begin{eqnarray*}
\esp\left[\sum_{n=1}^{N\left(\right)}f\left(T_{n}\right)\right]=\int_{\left(0,t\right]}f\left(s\right)d\eta\left(s\right)\textrm{,  }t\geq0,
\end{eqnarray*}
suponiendo que la integral exista. Adem\'as si $X_{1},X_{2},\ldots$ son variables aleatorias definidas en el mismo espacio de probabilidad que el proceso $N\left(t\right)$ tal que $\esp\left[X_{n}|T_{n}=s\right]=f\left(s\right)$, independiente de $n$. Entonces
\begin{eqnarray*}
\esp\left[\sum_{n=1}^{N\left(t\right)}X_{n}\right]=\int_{\left(0,t\right]}f\left(s\right)d\eta\left(s\right)\textrm{,  }t\geq0,
\end{eqnarray*} 
suponiendo que la integral exista. 
\end{Teo}

\begin{Coro}[Identidad de Wald para Renovaciones]
Para el proceso de renovaci\'on $N\left(t\right)$,
\begin{eqnarray*}
\esp\left[T_{N\left(t\right)+1}\right]=\mu\esp\left[N\left(t\right)+1\right]\textrm{,  }t\geq0,
\end{eqnarray*}  
\end{Coro}


\begin{Def}
Sea $h\left(t\right)$ funci\'on de valores reales en $\rea$ acotada en intervalos finitos e igual a cero para $t<0$ La ecuaci\'on de renovaci\'on para $h\left(t\right)$ y la distribuci\'on $F$ es

\begin{eqnarray}\label{Ec.Renovacion}
H\left(t\right)=h\left(t\right)+\int_{\left[0,t\right]}H\left(t-s\right)dF\left(s\right)\textrm{,    }t\geq0,
\end{eqnarray}
donde $H\left(t\right)$ es una funci\'on de valores reales. Esto es $H=h+F\star H$. Decimos que $H\left(t\right)$ es soluci\'on de esta ecuaci\'on si satisface la ecuaci\'on, y es acotada en intervalos finitos e iguales a cero para $t<0$.
\end{Def}

\begin{Prop}
La funci\'on $U\star h\left(t\right)$ es la \'unica soluci\'on de la ecuaci\'on de renovaci\'on (\ref{Ec.Renovacion}).
\end{Prop}

\begin{Teo}[Teorema Renovaci\'on Elemental]
\begin{eqnarray*}
t^{-1}U\left(t\right)\rightarrow 1/\mu\textrm{,    cuando }t\rightarrow\infty.
\end{eqnarray*}
\end{Teo}



Sup\'ongase que $N\left(t\right)$ es un proceso de renovaci\'on con distribuci\'on $F$ con media finita $\mu$.

\begin{Def}
La funci\'on de renovaci\'on asociada con la distribuci\'on $F$, del proceso $N\left(t\right)$, es
\begin{eqnarray*}
U\left(t\right)=\sum_{n=1}^{\infty}F^{n\star}\left(t\right),\textrm{   }t\geq0,
\end{eqnarray*}
donde $F^{0\star}\left(t\right)=\indora\left(t\geq0\right)$.
\end{Def}


\begin{Prop}
Sup\'ongase que la distribuci\'on de inter-renovaci\'on $F$ tiene densidad $f$. Entonces $U\left(t\right)$ tambi\'en tiene densidad, para $t>0$, y es $U^{'}\left(t\right)=\sum_{n=0}^{\infty}f^{n\star}\left(t\right)$. Adem\'as
\begin{eqnarray*}
\prob\left\{N\left(t\right)>N\left(t-\right)\right\}=0\textrm{,   }t\geq0.
\end{eqnarray*}
\end{Prop}

\begin{Def}
La Transformada de Laplace-Stieljes de $F$ est\'a dada por

\begin{eqnarray*}
\hat{F}\left(\alpha\right)=\int_{\rea_{+}}e^{-\alpha t}dF\left(t\right)\textrm{,  }\alpha\geq0.
\end{eqnarray*}
\end{Def}

Entonces

\begin{eqnarray*}
\hat{U}\left(\alpha\right)=\sum_{n=0}^{\infty}\hat{F^{n\star}}\left(\alpha\right)=\sum_{n=0}^{\infty}\hat{F}\left(\alpha\right)^{n}=\frac{1}{1-\hat{F}\left(\alpha\right)}.
\end{eqnarray*}


\begin{Prop}
La Transformada de Laplace $\hat{U}\left(\alpha\right)$ y $\hat{F}\left(\alpha\right)$ determina una a la otra de manera \'unica por la relaci\'on $\hat{U}\left(\alpha\right)=\frac{1}{1-\hat{F}\left(\alpha\right)}$.
\end{Prop}


\begin{Note}
Un proceso de renovaci\'on $N\left(t\right)$ cuyos tiempos de inter-renovaci\'on tienen media finita, es un proceso Poisson con tasa $\lambda$ si y s\'olo s\'i $\esp\left[U\left(t\right)\right]=\lambda t$, para $t\geq0$.
\end{Note}


\begin{Teo}
Sea $N\left(t\right)$ un proceso puntual simple con puntos de localizaci\'on $T_{n}$ tal que $\eta\left(t\right)=\esp\left[N\left(\right)\right]$ es finita para cada $t$. Entonces para cualquier funci\'on $f:\rea_{+}\rightarrow\rea$,
\begin{eqnarray*}
\esp\left[\sum_{n=1}^{N\left(\right)}f\left(T_{n}\right)\right]=\int_{\left(0,t\right]}f\left(s\right)d\eta\left(s\right)\textrm{,  }t\geq0,
\end{eqnarray*}
suponiendo que la integral exista. Adem\'as si $X_{1},X_{2},\ldots$ son variables aleatorias definidas en el mismo espacio de probabilidad que el proceso $N\left(t\right)$ tal que $\esp\left[X_{n}|T_{n}=s\right]=f\left(s\right)$, independiente de $n$. Entonces
\begin{eqnarray*}
\esp\left[\sum_{n=1}^{N\left(t\right)}X_{n}\right]=\int_{\left(0,t\right]}f\left(s\right)d\eta\left(s\right)\textrm{,  }t\geq0,
\end{eqnarray*} 
suponiendo que la integral exista. 
\end{Teo}

\begin{Coro}[Identidad de Wald para Renovaciones]
Para el proceso de renovaci\'on $N\left(t\right)$,
\begin{eqnarray*}
\esp\left[T_{N\left(t\right)+1}\right]=\mu\esp\left[N\left(t\right)+1\right]\textrm{,  }t\geq0,
\end{eqnarray*}  
\end{Coro}


\begin{Def}
Sea $h\left(t\right)$ funci\'on de valores reales en $\rea$ acotada en intervalos finitos e igual a cero para $t<0$ La ecuaci\'on de renovaci\'on para $h\left(t\right)$ y la distribuci\'on $F$ es

\begin{eqnarray}\label{Ec.Renovacion}
H\left(t\right)=h\left(t\right)+\int_{\left[0,t\right]}H\left(t-s\right)dF\left(s\right)\textrm{,    }t\geq0,
\end{eqnarray}
donde $H\left(t\right)$ es una funci\'on de valores reales. Esto es $H=h+F\star H$. Decimos que $H\left(t\right)$ es soluci\'on de esta ecuaci\'on si satisface la ecuaci\'on, y es acotada en intervalos finitos e iguales a cero para $t<0$.
\end{Def}

\begin{Prop}
La funci\'on $U\star h\left(t\right)$ es la \'unica soluci\'on de la ecuaci\'on de renovaci\'on (\ref{Ec.Renovacion}).
\end{Prop}

\begin{Teo}[Teorema Renovaci\'on Elemental]
\begin{eqnarray*}
t^{-1}U\left(t\right)\rightarrow 1/\mu\textrm{,    cuando }t\rightarrow\infty.
\end{eqnarray*}
\end{Teo}


\begin{Note} Una funci\'on $h:\rea_{+}\rightarrow\rea$ es Directamente Riemann Integrable en los siguientes casos:
\begin{itemize}
\item[a)] $h\left(t\right)\geq0$ es decreciente y Riemann Integrable.
\item[b)] $h$ es continua excepto posiblemente en un conjunto de Lebesgue de medida 0, y $|h\left(t\right)|\leq b\left(t\right)$, donde $b$ es DRI.
\end{itemize}
\end{Note}

\begin{Teo}[Teorema Principal de Renovaci\'on]
Si $F$ es no aritm\'etica y $h\left(t\right)$ es Directamente Riemann Integrable (DRI), entonces

\begin{eqnarray*}
lim_{t\rightarrow\infty}U\star h=\frac{1}{\mu}\int_{\rea_{+}}h\left(s\right)ds.
\end{eqnarray*}
\end{Teo}

\begin{Prop}
Cualquier funci\'on $H\left(t\right)$ acotada en intervalos finitos y que es 0 para $t<0$ puede expresarse como
\begin{eqnarray*}
H\left(t\right)=U\star h\left(t\right)\textrm{,  donde }h\left(t\right)=H\left(t\right)-F\star H\left(t\right)
\end{eqnarray*}
\end{Prop}

\begin{Def}
Un proceso estoc\'astico $X\left(t\right)$ es crudamente regenerativo en un tiempo aleatorio positivo $T$ si
\begin{eqnarray*}
\esp\left[X\left(T+t\right)|T\right]=\esp\left[X\left(t\right)\right]\textrm{, para }t\geq0,\end{eqnarray*}
y con las esperanzas anteriores finitas.
\end{Def}

\begin{Prop}
Sup\'ongase que $X\left(t\right)$ es un proceso crudamente regenerativo en $T$, que tiene distribuci\'on $F$. Si $\esp\left[X\left(t\right)\right]$ es acotado en intervalos finitos, entonces
\begin{eqnarray*}
\esp\left[X\left(t\right)\right]=U\star h\left(t\right)\textrm{,  donde }h\left(t\right)=\esp\left[X\left(t\right)\indora\left(T>t\right)\right].
\end{eqnarray*}
\end{Prop}

\begin{Teo}[Regeneraci\'on Cruda]
Sup\'ongase que $X\left(t\right)$ es un proceso con valores positivo crudamente regenerativo en $T$, y def\'inase $M=\sup\left\{|X\left(t\right)|:t\leq T\right\}$. Si $T$ es no aritm\'etico y $M$ y $MT$ tienen media finita, entonces
\begin{eqnarray*}
lim_{t\rightarrow\infty}\esp\left[X\left(t\right)\right]=\frac{1}{\mu}\int_{\rea_{+}}h\left(s\right)ds,
\end{eqnarray*}
donde $h\left(t\right)=\esp\left[X\left(t\right)\indora\left(T>t\right)\right]$.
\end{Teo}


\begin{Note} Una funci\'on $h:\rea_{+}\rightarrow\rea$ es Directamente Riemann Integrable en los siguientes casos:
\begin{itemize}
\item[a)] $h\left(t\right)\geq0$ es decreciente y Riemann Integrable.
\item[b)] $h$ es continua excepto posiblemente en un conjunto de Lebesgue de medida 0, y $|h\left(t\right)|\leq b\left(t\right)$, donde $b$ es DRI.
\end{itemize}
\end{Note}

\begin{Teo}[Teorema Principal de Renovaci\'on]
Si $F$ es no aritm\'etica y $h\left(t\right)$ es Directamente Riemann Integrable (DRI), entonces

\begin{eqnarray*}
lim_{t\rightarrow\infty}U\star h=\frac{1}{\mu}\int_{\rea_{+}}h\left(s\right)ds.
\end{eqnarray*}
\end{Teo}

\begin{Prop}
Cualquier funci\'on $H\left(t\right)$ acotada en intervalos finitos y que es 0 para $t<0$ puede expresarse como
\begin{eqnarray*}
H\left(t\right)=U\star h\left(t\right)\textrm{,  donde }h\left(t\right)=H\left(t\right)-F\star H\left(t\right)
\end{eqnarray*}
\end{Prop}

\begin{Def}
Un proceso estoc\'astico $X\left(t\right)$ es crudamente regenerativo en un tiempo aleatorio positivo $T$ si
\begin{eqnarray*}
\esp\left[X\left(T+t\right)|T\right]=\esp\left[X\left(t\right)\right]\textrm{, para }t\geq0,\end{eqnarray*}
y con las esperanzas anteriores finitas.
\end{Def}

\begin{Prop}
Sup\'ongase que $X\left(t\right)$ es un proceso crudamente regenerativo en $T$, que tiene distribuci\'on $F$. Si $\esp\left[X\left(t\right)\right]$ es acotado en intervalos finitos, entonces
\begin{eqnarray*}
\esp\left[X\left(t\right)\right]=U\star h\left(t\right)\textrm{,  donde }h\left(t\right)=\esp\left[X\left(t\right)\indora\left(T>t\right)\right].
\end{eqnarray*}
\end{Prop}

\begin{Teo}[Regeneraci\'on Cruda]
Sup\'ongase que $X\left(t\right)$ es un proceso con valores positivo crudamente regenerativo en $T$, y def\'inase $M=\sup\left\{|X\left(t\right)|:t\leq T\right\}$. Si $T$ es no aritm\'etico y $M$ y $MT$ tienen media finita, entonces
\begin{eqnarray*}
lim_{t\rightarrow\infty}\esp\left[X\left(t\right)\right]=\frac{1}{\mu}\int_{\rea_{+}}h\left(s\right)ds,
\end{eqnarray*}
donde $h\left(t\right)=\esp\left[X\left(t\right)\indora\left(T>t\right)\right]$.
\end{Teo}

%________________________________________________________________________
\subsection{Procesos Regenerativos}
%________________________________________________________________________

Para $\left\{X\left(t\right):t\geq0\right\}$ Proceso Estoc\'astico a tiempo continuo con estado de espacios $S$, que es un espacio m\'etrico, con trayectorias continuas por la derecha y con l\'imites por la izquierda c.s. Sea $N\left(t\right)$ un proceso de renovaci\'on en $\rea_{+}$ definido en el mismo espacio de probabilidad que $X\left(t\right)$, con tiempos de renovaci\'on $T$ y tiempos de inter-renovaci\'on $\xi_{n}=T_{n}-T_{n-1}$, con misma distribuci\'on $F$ de media finita $\mu$.



\begin{Def}
Para el proceso $\left\{\left(N\left(t\right),X\left(t\right)\right):t\geq0\right\}$, sus trayectoria muestrales en el intervalo de tiempo $\left[T_{n-1},T_{n}\right)$ est\'an descritas por
\begin{eqnarray*}
\zeta_{n}=\left(\xi_{n},\left\{X\left(T_{n-1}+t\right):0\leq t<\xi_{n}\right\}\right)
\end{eqnarray*}
Este $\zeta_{n}$ es el $n$-\'esimo segmento del proceso. El proceso es regenerativo sobre los tiempos $T_{n}$ si sus segmentos $\zeta_{n}$ son independientes e id\'enticamennte distribuidos.
\end{Def}


\begin{Obs}
Si $\tilde{X}\left(t\right)$ con espacio de estados $\tilde{S}$ es regenerativo sobre $T_{n}$, entonces $X\left(t\right)=f\left(\tilde{X}\left(t\right)\right)$ tambi\'en es regenerativo sobre $T_{n}$, para cualquier funci\'on $f:\tilde{S}\rightarrow S$.
\end{Obs}

\begin{Obs}
Los procesos regenerativos son crudamente regenerativos, pero no al rev\'es.
\end{Obs}

\begin{Def}[Definici\'on Cl\'asica]
Un proceso estoc\'astico $X=\left\{X\left(t\right):t\geq0\right\}$ es llamado regenerativo is existe una variable aleatoria $R_{1}>0$ tal que
\begin{itemize}
\item[i)] $\left\{X\left(t+R_{1}\right):t\geq0\right\}$ es independiente de $\left\{\left\{X\left(t\right):t<R_{1}\right\},\right\}$
\item[ii)] $\left\{X\left(t+R_{1}\right):t\geq0\right\}$ es estoc\'asticamente equivalente a $\left\{X\left(t\right):t>0\right\}$
\end{itemize}

Llamamos a $R_{1}$ tiempo de regeneraci\'on, y decimos que $X$ se regenera en este punto.
\end{Def}

$\left\{X\left(t+R_{1}\right)\right\}$ es regenerativo con tiempo de regeneraci\'on $R_{2}$, independiente de $R_{1}$ pero con la misma distribuci\'on que $R_{1}$. Procediendo de esta manera se obtiene una secuencia de variables aleatorias independientes e id\'enticamente distribuidas $\left\{R_{n}\right\}$ llamados longitudes de ciclo. Si definimos a $Z_{k}\equiv R_{1}+R_{2}+\cdots+R_{k}$, se tiene un proceso de renovaci\'on llamado proceso de renovaci\'on encajado para $X$.

\begin{Note}
Un proceso regenerativo con media de la longitud de ciclo finita es llamado positivo recurrente.
\end{Note}


\begin{Def}
Para $x$ fijo y para cada $t\geq0$, sea $I_{x}\left(t\right)=1$ si $X\left(t\right)\leq x$,  $I_{x}\left(t\right)=0$ en caso contrario, y def\'inanse los tiempos promedio
\begin{eqnarray*}
\overline{X}&=&lim_{t\rightarrow\infty}\frac{1}{t}\int_{0}^{\infty}X\left(u\right)du\\
\prob\left(X_{\infty}\leq x\right)&=&lim_{t\rightarrow\infty}\frac{1}{t}\int_{0}^{\infty}I_{x}\left(u\right)du,
\end{eqnarray*}
cuando estos l\'imites existan.
\end{Def}

Como consecuencia del teorema de Renovaci\'on-Recompensa, se tiene que el primer l\'imite  existe y es igual a la constante
\begin{eqnarray*}
\overline{X}&=&\frac{\esp\left[\int_{0}^{R_{1}}X\left(t\right)dt\right]}{\esp\left[R_{1}\right]},
\end{eqnarray*}
suponiendo que ambas esperanzas son finitas.

\begin{Note}
\begin{itemize}
\item[a)] Si el proceso regenerativo $X$ es positivo recurrente y tiene trayectorias muestrales no negativas, entonces la ecuaci\'on anterior es v\'alida.
\item[b)] Si $X$ es positivo recurrente regenerativo, podemos construir una \'unica versi\'on estacionaria de este proceso, $X_{e}=\left\{X_{e}\left(t\right)\right\}$, donde $X_{e}$ es un proceso estoc\'astico regenerativo y estrictamente estacionario, con distribuci\'on marginal distribuida como $X_{\infty}$
\end{itemize}
\end{Note}

%________________________________________________________________________
\subsection{Procesos Regenerativos}
%________________________________________________________________________

Para $\left\{X\left(t\right):t\geq0\right\}$ Proceso Estoc\'astico a tiempo continuo con estado de espacios $S$, que es un espacio m\'etrico, con trayectorias continuas por la derecha y con l\'imites por la izquierda c.s. Sea $N\left(t\right)$ un proceso de renovaci\'on en $\rea_{+}$ definido en el mismo espacio de probabilidad que $X\left(t\right)$, con tiempos de renovaci\'on $T$ y tiempos de inter-renovaci\'on $\xi_{n}=T_{n}-T_{n-1}$, con misma distribuci\'on $F$ de media finita $\mu$.



\begin{Def}
Para el proceso $\left\{\left(N\left(t\right),X\left(t\right)\right):t\geq0\right\}$, sus trayectoria muestrales en el intervalo de tiempo $\left[T_{n-1},T_{n}\right)$ est\'an descritas por
\begin{eqnarray*}
\zeta_{n}=\left(\xi_{n},\left\{X\left(T_{n-1}+t\right):0\leq t<\xi_{n}\right\}\right)
\end{eqnarray*}
Este $\zeta_{n}$ es el $n$-\'esimo segmento del proceso. El proceso es regenerativo sobre los tiempos $T_{n}$ si sus segmentos $\zeta_{n}$ son independientes e id\'enticamennte distribuidos.
\end{Def}


\begin{Obs}
Si $\tilde{X}\left(t\right)$ con espacio de estados $\tilde{S}$ es regenerativo sobre $T_{n}$, entonces $X\left(t\right)=f\left(\tilde{X}\left(t\right)\right)$ tambi\'en es regenerativo sobre $T_{n}$, para cualquier funci\'on $f:\tilde{S}\rightarrow S$.
\end{Obs}

\begin{Obs}
Los procesos regenerativos son crudamente regenerativos, pero no al rev\'es.
\end{Obs}

\begin{Def}[Definici\'on Cl\'asica]
Un proceso estoc\'astico $X=\left\{X\left(t\right):t\geq0\right\}$ es llamado regenerativo is existe una variable aleatoria $R_{1}>0$ tal que
\begin{itemize}
\item[i)] $\left\{X\left(t+R_{1}\right):t\geq0\right\}$ es independiente de $\left\{\left\{X\left(t\right):t<R_{1}\right\},\right\}$
\item[ii)] $\left\{X\left(t+R_{1}\right):t\geq0\right\}$ es estoc\'asticamente equivalente a $\left\{X\left(t\right):t>0\right\}$
\end{itemize}

Llamamos a $R_{1}$ tiempo de regeneraci\'on, y decimos que $X$ se regenera en este punto.
\end{Def}

$\left\{X\left(t+R_{1}\right)\right\}$ es regenerativo con tiempo de regeneraci\'on $R_{2}$, independiente de $R_{1}$ pero con la misma distribuci\'on que $R_{1}$. Procediendo de esta manera se obtiene una secuencia de variables aleatorias independientes e id\'enticamente distribuidas $\left\{R_{n}\right\}$ llamados longitudes de ciclo. Si definimos a $Z_{k}\equiv R_{1}+R_{2}+\cdots+R_{k}$, se tiene un proceso de renovaci\'on llamado proceso de renovaci\'on encajado para $X$.

\begin{Note}
Un proceso regenerativo con media de la longitud de ciclo finita es llamado positivo recurrente.
\end{Note}


\begin{Def}
Para $x$ fijo y para cada $t\geq0$, sea $I_{x}\left(t\right)=1$ si $X\left(t\right)\leq x$,  $I_{x}\left(t\right)=0$ en caso contrario, y def\'inanse los tiempos promedio
\begin{eqnarray*}
\overline{X}&=&lim_{t\rightarrow\infty}\frac{1}{t}\int_{0}^{\infty}X\left(u\right)du\\
\prob\left(X_{\infty}\leq x\right)&=&lim_{t\rightarrow\infty}\frac{1}{t}\int_{0}^{\infty}I_{x}\left(u\right)du,
\end{eqnarray*}
cuando estos l\'imites existan.
\end{Def}

Como consecuencia del teorema de Renovaci\'on-Recompensa, se tiene que el primer l\'imite  existe y es igual a la constante
\begin{eqnarray*}
\overline{X}&=&\frac{\esp\left[\int_{0}^{R_{1}}X\left(t\right)dt\right]}{\esp\left[R_{1}\right]},
\end{eqnarray*}
suponiendo que ambas esperanzas son finitas.

\begin{Note}
\begin{itemize}
\item[a)] Si el proceso regenerativo $X$ es positivo recurrente y tiene trayectorias muestrales no negativas, entonces la ecuaci\'on anterior es v\'alida.
\item[b)] Si $X$ es positivo recurrente regenerativo, podemos construir una \'unica versi\'on estacionaria de este proceso, $X_{e}=\left\{X_{e}\left(t\right)\right\}$, donde $X_{e}$ es un proceso estoc\'astico regenerativo y estrictamente estacionario, con distribuci\'on marginal distribuida como $X_{\infty}$
\end{itemize}
\end{Note}
%__________________________________________________________________________________________
\subsection{Procesos Regenerativos Estacionarios - Stidham \cite{Stidham}}
%__________________________________________________________________________________________


Un proceso estoc\'astico a tiempo continuo $\left\{V\left(t\right),t\geq0\right\}$ es un proceso regenerativo si existe una sucesi\'on de variables aleatorias independientes e id\'enticamente distribuidas $\left\{X_{1},X_{2},\ldots\right\}$, sucesi\'on de renovaci\'on, tal que para cualquier conjunto de Borel $A$, 

\begin{eqnarray*}
\prob\left\{V\left(t\right)\in A|X_{1}+X_{2}+\cdots+X_{R\left(t\right)}=s,\left\{V\left(\tau\right),\tau<s\right\}\right\}=\prob\left\{V\left(t-s\right)\in A|X_{1}>t-s\right\},
\end{eqnarray*}
para todo $0\leq s\leq t$, donde $R\left(t\right)=\max\left\{X_{1}+X_{2}+\cdots+X_{j}\leq t\right\}=$n\'umero de renovaciones ({\emph{puntos de regeneraci\'on}}) que ocurren en $\left[0,t\right]$. El intervalo $\left[0,X_{1}\right)$ es llamado {\emph{primer ciclo de regeneraci\'on}} de $\left\{V\left(t \right),t\geq0\right\}$, $\left[X_{1},X_{1}+X_{2}\right)$ el {\emph{segundo ciclo de regeneraci\'on}}, y as\'i sucesivamente.

Sea $X=X_{1}$ y sea $F$ la funci\'on de distrbuci\'on de $X$


\begin{Def}
Se define el proceso estacionario, $\left\{V^{*}\left(t\right),t\geq0\right\}$, para $\left\{V\left(t\right),t\geq0\right\}$ por

\begin{eqnarray*}
\prob\left\{V\left(t\right)\in A\right\}=\frac{1}{\esp\left[X\right]}\int_{0}^{\infty}\prob\left\{V\left(t+x\right)\in A|X>x\right\}\left(1-F\left(x\right)\right)dx,
\end{eqnarray*} 
para todo $t\geq0$ y todo conjunto de Borel $A$.
\end{Def}

\begin{Def}
Una distribuci\'on se dice que es {\emph{aritm\'etica}} si todos sus puntos de incremento son m\'ultiplos de la forma $0,\lambda, 2\lambda,\ldots$ para alguna $\lambda>0$ entera.
\end{Def}


\begin{Def}
Una modificaci\'on medible de un proceso $\left\{V\left(t\right),t\geq0\right\}$, es una versi\'on de este, $\left\{V\left(t,w\right)\right\}$ conjuntamente medible para $t\geq0$ y para $w\in S$, $S$ espacio de estados para $\left\{V\left(t\right),t\geq0\right\}$.
\end{Def}

\begin{Teo}
Sea $\left\{V\left(t\right),t\geq\right\}$ un proceso regenerativo no negativo con modificaci\'on medible. Sea $\esp\left[X\right]<\infty$. Entonces el proceso estacionario dado por la ecuaci\'on anterior est\'a bien definido y tiene funci\'on de distribuci\'on independiente de $t$, adem\'as
\begin{itemize}
\item[i)] \begin{eqnarray*}
\esp\left[V^{*}\left(0\right)\right]&=&\frac{\esp\left[\int_{0}^{X}V\left(s\right)ds\right]}{\esp\left[X\right]}\end{eqnarray*}
\item[ii)] Si $\esp\left[V^{*}\left(0\right)\right]<\infty$, equivalentemente, si $\esp\left[\int_{0}^{X}V\left(s\right)ds\right]<\infty$,entonces
\begin{eqnarray*}
\frac{\int_{0}^{t}V\left(s\right)ds}{t}\rightarrow\frac{\esp\left[\int_{0}^{X}V\left(s\right)ds\right]}{\esp\left[X\right]}
\end{eqnarray*}
con probabilidad 1 y en media, cuando $t\rightarrow\infty$.
\end{itemize}
\end{Teo}


%__________________________________________________________________________________________
\subsection{Procesos Regenerativos Estacionarios - Stidham \cite{Stidham}}
%__________________________________________________________________________________________


Un proceso estoc\'astico a tiempo continuo $\left\{V\left(t\right),t\geq0\right\}$ es un proceso regenerativo si existe una sucesi\'on de variables aleatorias independientes e id\'enticamente distribuidas $\left\{X_{1},X_{2},\ldots\right\}$, sucesi\'on de renovaci\'on, tal que para cualquier conjunto de Borel $A$, 

\begin{eqnarray*}
\prob\left\{V\left(t\right)\in A|X_{1}+X_{2}+\cdots+X_{R\left(t\right)}=s,\left\{V\left(\tau\right),\tau<s\right\}\right\}=\prob\left\{V\left(t-s\right)\in A|X_{1}>t-s\right\},
\end{eqnarray*}
para todo $0\leq s\leq t$, donde $R\left(t\right)=\max\left\{X_{1}+X_{2}+\cdots+X_{j}\leq t\right\}=$n\'umero de renovaciones ({\emph{puntos de regeneraci\'on}}) que ocurren en $\left[0,t\right]$. El intervalo $\left[0,X_{1}\right)$ es llamado {\emph{primer ciclo de regeneraci\'on}} de $\left\{V\left(t \right),t\geq0\right\}$, $\left[X_{1},X_{1}+X_{2}\right)$ el {\emph{segundo ciclo de regeneraci\'on}}, y as\'i sucesivamente.

Sea $X=X_{1}$ y sea $F$ la funci\'on de distrbuci\'on de $X$


\begin{Def}
Se define el proceso estacionario, $\left\{V^{*}\left(t\right),t\geq0\right\}$, para $\left\{V\left(t\right),t\geq0\right\}$ por

\begin{eqnarray*}
\prob\left\{V\left(t\right)\in A\right\}=\frac{1}{\esp\left[X\right]}\int_{0}^{\infty}\prob\left\{V\left(t+x\right)\in A|X>x\right\}\left(1-F\left(x\right)\right)dx,
\end{eqnarray*} 
para todo $t\geq0$ y todo conjunto de Borel $A$.
\end{Def}

\begin{Def}
Una distribuci\'on se dice que es {\emph{aritm\'etica}} si todos sus puntos de incremento son m\'ultiplos de la forma $0,\lambda, 2\lambda,\ldots$ para alguna $\lambda>0$ entera.
\end{Def}


\begin{Def}
Una modificaci\'on medible de un proceso $\left\{V\left(t\right),t\geq0\right\}$, es una versi\'on de este, $\left\{V\left(t,w\right)\right\}$ conjuntamente medible para $t\geq0$ y para $w\in S$, $S$ espacio de estados para $\left\{V\left(t\right),t\geq0\right\}$.
\end{Def}

\begin{Teo}
Sea $\left\{V\left(t\right),t\geq\right\}$ un proceso regenerativo no negativo con modificaci\'on medible. Sea $\esp\left[X\right]<\infty$. Entonces el proceso estacionario dado por la ecuaci\'on anterior est\'a bien definido y tiene funci\'on de distribuci\'on independiente de $t$, adem\'as
\begin{itemize}
\item[i)] \begin{eqnarray*}
\esp\left[V^{*}\left(0\right)\right]&=&\frac{\esp\left[\int_{0}^{X}V\left(s\right)ds\right]}{\esp\left[X\right]}\end{eqnarray*}
\item[ii)] Si $\esp\left[V^{*}\left(0\right)\right]<\infty$, equivalentemente, si $\esp\left[\int_{0}^{X}V\left(s\right)ds\right]<\infty$,entonces
\begin{eqnarray*}
\frac{\int_{0}^{t}V\left(s\right)ds}{t}\rightarrow\frac{\esp\left[\int_{0}^{X}V\left(s\right)ds\right]}{\esp\left[X\right]}
\end{eqnarray*}
con probabilidad 1 y en media, cuando $t\rightarrow\infty$.
\end{itemize}
\end{Teo}
%___________________________________________________________________________________________
%
\subsection{Propiedades de los Procesos de Renovaci\'on}
%___________________________________________________________________________________________
%

Los tiempos $T_{n}$ est\'an relacionados con los conteos de $N\left(t\right)$ por

\begin{eqnarray*}
\left\{N\left(t\right)\geq n\right\}&=&\left\{T_{n}\leq t\right\}\\
T_{N\left(t\right)}\leq &t&<T_{N\left(t\right)+1},
\end{eqnarray*}

adem\'as $N\left(T_{n}\right)=n$, y 

\begin{eqnarray*}
N\left(t\right)=\max\left\{n:T_{n}\leq t\right\}=\min\left\{n:T_{n+1}>t\right\}
\end{eqnarray*}

Por propiedades de la convoluci\'on se sabe que

\begin{eqnarray*}
P\left\{T_{n}\leq t\right\}=F^{n\star}\left(t\right)
\end{eqnarray*}
que es la $n$-\'esima convoluci\'on de $F$. Entonces 

\begin{eqnarray*}
\left\{N\left(t\right)\geq n\right\}&=&\left\{T_{n}\leq t\right\}\\
P\left\{N\left(t\right)\leq n\right\}&=&1-F^{\left(n+1\right)\star}\left(t\right)
\end{eqnarray*}

Adem\'as usando el hecho de que $\esp\left[N\left(t\right)\right]=\sum_{n=1}^{\infty}P\left\{N\left(t\right)\geq n\right\}$
se tiene que

\begin{eqnarray*}
\esp\left[N\left(t\right)\right]=\sum_{n=1}^{\infty}F^{n\star}\left(t\right)
\end{eqnarray*}

\begin{Prop}
Para cada $t\geq0$, la funci\'on generadora de momentos $\esp\left[e^{\alpha N\left(t\right)}\right]$ existe para alguna $\alpha$ en una vecindad del 0, y de aqu\'i que $\esp\left[N\left(t\right)^{m}\right]<\infty$, para $m\geq1$.
\end{Prop}


\begin{Note}
Si el primer tiempo de renovaci\'on $\xi_{1}$ no tiene la misma distribuci\'on que el resto de las $\xi_{n}$, para $n\geq2$, a $N\left(t\right)$ se le llama Proceso de Renovaci\'on retardado, donde si $\xi$ tiene distribuci\'on $G$, entonces el tiempo $T_{n}$ de la $n$-\'esima renovaci\'on tiene distribuci\'on $G\star F^{\left(n-1\right)\star}\left(t\right)$
\end{Note}


\begin{Teo}
Para una constante $\mu\leq\infty$ ( o variable aleatoria), las siguientes expresiones son equivalentes:

\begin{eqnarray}
lim_{n\rightarrow\infty}n^{-1}T_{n}&=&\mu,\textrm{ c.s.}\\
lim_{t\rightarrow\infty}t^{-1}N\left(t\right)&=&1/\mu,\textrm{ c.s.}
\end{eqnarray}
\end{Teo}


Es decir, $T_{n}$ satisface la Ley Fuerte de los Grandes N\'umeros s\'i y s\'olo s\'i $N\left/t\right)$ la cumple.


\begin{Coro}[Ley Fuerte de los Grandes N\'umeros para Procesos de Renovaci\'on]
Si $N\left(t\right)$ es un proceso de renovaci\'on cuyos tiempos de inter-renovaci\'on tienen media $\mu\leq\infty$, entonces
\begin{eqnarray}
t^{-1}N\left(t\right)\rightarrow 1/\mu,\textrm{ c.s. cuando }t\rightarrow\infty.
\end{eqnarray}

\end{Coro}


Considerar el proceso estoc\'astico de valores reales $\left\{Z\left(t\right):t\geq0\right\}$ en el mismo espacio de probabilidad que $N\left(t\right)$

\begin{Def}
Para el proceso $\left\{Z\left(t\right):t\geq0\right\}$ se define la fluctuaci\'on m\'axima de $Z\left(t\right)$ en el intervalo $\left(T_{n-1},T_{n}\right]$:
\begin{eqnarray*}
M_{n}=\sup_{T_{n-1}<t\leq T_{n}}|Z\left(t\right)-Z\left(T_{n-1}\right)|
\end{eqnarray*}
\end{Def}

\begin{Teo}
Sup\'ongase que $n^{-1}T_{n}\rightarrow\mu$ c.s. cuando $n\rightarrow\infty$, donde $\mu\leq\infty$ es una constante o variable aleatoria. Sea $a$ una constante o variable aleatoria que puede ser infinita cuando $\mu$ es finita, y considere las expresiones l\'imite:
\begin{eqnarray}
lim_{n\rightarrow\infty}n^{-1}Z\left(T_{n}\right)&=&a,\textrm{ c.s.}\\
lim_{t\rightarrow\infty}t^{-1}Z\left(t\right)&=&a/\mu,\textrm{ c.s.}
\end{eqnarray}
La segunda expresi\'on implica la primera. Conversamente, la primera implica la segunda si el proceso $Z\left(t\right)$ es creciente, o si $lim_{n\rightarrow\infty}n^{-1}M_{n}=0$ c.s.
\end{Teo}

\begin{Coro}
Si $N\left(t\right)$ es un proceso de renovaci\'on, y $\left(Z\left(T_{n}\right)-Z\left(T_{n-1}\right),M_{n}\right)$, para $n\geq1$, son variables aleatorias independientes e id\'enticamente distribuidas con media finita, entonces,
\begin{eqnarray}
lim_{t\rightarrow\infty}t^{-1}Z\left(t\right)\rightarrow\frac{\esp\left[Z\left(T_{1}\right)-Z\left(T_{0}\right)\right]}{\esp\left[T_{1}\right]},\textrm{ c.s. cuando  }t\rightarrow\infty.
\end{eqnarray}
\end{Coro}


%___________________________________________________________________________________________
%
\subsection{Propiedades de los Procesos de Renovaci\'on}
%___________________________________________________________________________________________
%

Los tiempos $T_{n}$ est\'an relacionados con los conteos de $N\left(t\right)$ por

\begin{eqnarray*}
\left\{N\left(t\right)\geq n\right\}&=&\left\{T_{n}\leq t\right\}\\
T_{N\left(t\right)}\leq &t&<T_{N\left(t\right)+1},
\end{eqnarray*}

adem\'as $N\left(T_{n}\right)=n$, y 

\begin{eqnarray*}
N\left(t\right)=\max\left\{n:T_{n}\leq t\right\}=\min\left\{n:T_{n+1}>t\right\}
\end{eqnarray*}

Por propiedades de la convoluci\'on se sabe que

\begin{eqnarray*}
P\left\{T_{n}\leq t\right\}=F^{n\star}\left(t\right)
\end{eqnarray*}
que es la $n$-\'esima convoluci\'on de $F$. Entonces 

\begin{eqnarray*}
\left\{N\left(t\right)\geq n\right\}&=&\left\{T_{n}\leq t\right\}\\
P\left\{N\left(t\right)\leq n\right\}&=&1-F^{\left(n+1\right)\star}\left(t\right)
\end{eqnarray*}

Adem\'as usando el hecho de que $\esp\left[N\left(t\right)\right]=\sum_{n=1}^{\infty}P\left\{N\left(t\right)\geq n\right\}$
se tiene que

\begin{eqnarray*}
\esp\left[N\left(t\right)\right]=\sum_{n=1}^{\infty}F^{n\star}\left(t\right)
\end{eqnarray*}

\begin{Prop}
Para cada $t\geq0$, la funci\'on generadora de momentos $\esp\left[e^{\alpha N\left(t\right)}\right]$ existe para alguna $\alpha$ en una vecindad del 0, y de aqu\'i que $\esp\left[N\left(t\right)^{m}\right]<\infty$, para $m\geq1$.
\end{Prop}


\begin{Note}
Si el primer tiempo de renovaci\'on $\xi_{1}$ no tiene la misma distribuci\'on que el resto de las $\xi_{n}$, para $n\geq2$, a $N\left(t\right)$ se le llama Proceso de Renovaci\'on retardado, donde si $\xi$ tiene distribuci\'on $G$, entonces el tiempo $T_{n}$ de la $n$-\'esima renovaci\'on tiene distribuci\'on $G\star F^{\left(n-1\right)\star}\left(t\right)$
\end{Note}


\begin{Teo}
Para una constante $\mu\leq\infty$ ( o variable aleatoria), las siguientes expresiones son equivalentes:

\begin{eqnarray}
lim_{n\rightarrow\infty}n^{-1}T_{n}&=&\mu,\textrm{ c.s.}\\
lim_{t\rightarrow\infty}t^{-1}N\left(t\right)&=&1/\mu,\textrm{ c.s.}
\end{eqnarray}
\end{Teo}


Es decir, $T_{n}$ satisface la Ley Fuerte de los Grandes N\'umeros s\'i y s\'olo s\'i $N\left/t\right)$ la cumple.


\begin{Coro}[Ley Fuerte de los Grandes N\'umeros para Procesos de Renovaci\'on]
Si $N\left(t\right)$ es un proceso de renovaci\'on cuyos tiempos de inter-renovaci\'on tienen media $\mu\leq\infty$, entonces
\begin{eqnarray}
t^{-1}N\left(t\right)\rightarrow 1/\mu,\textrm{ c.s. cuando }t\rightarrow\infty.
\end{eqnarray}

\end{Coro}


Considerar el proceso estoc\'astico de valores reales $\left\{Z\left(t\right):t\geq0\right\}$ en el mismo espacio de probabilidad que $N\left(t\right)$

\begin{Def}
Para el proceso $\left\{Z\left(t\right):t\geq0\right\}$ se define la fluctuaci\'on m\'axima de $Z\left(t\right)$ en el intervalo $\left(T_{n-1},T_{n}\right]$:
\begin{eqnarray*}
M_{n}=\sup_{T_{n-1}<t\leq T_{n}}|Z\left(t\right)-Z\left(T_{n-1}\right)|
\end{eqnarray*}
\end{Def}

\begin{Teo}
Sup\'ongase que $n^{-1}T_{n}\rightarrow\mu$ c.s. cuando $n\rightarrow\infty$, donde $\mu\leq\infty$ es una constante o variable aleatoria. Sea $a$ una constante o variable aleatoria que puede ser infinita cuando $\mu$ es finita, y considere las expresiones l\'imite:
\begin{eqnarray}
lim_{n\rightarrow\infty}n^{-1}Z\left(T_{n}\right)&=&a,\textrm{ c.s.}\\
lim_{t\rightarrow\infty}t^{-1}Z\left(t\right)&=&a/\mu,\textrm{ c.s.}
\end{eqnarray}
La segunda expresi\'on implica la primera. Conversamente, la primera implica la segunda si el proceso $Z\left(t\right)$ es creciente, o si $lim_{n\rightarrow\infty}n^{-1}M_{n}=0$ c.s.
\end{Teo}

\begin{Coro}
Si $N\left(t\right)$ es un proceso de renovaci\'on, y $\left(Z\left(T_{n}\right)-Z\left(T_{n-1}\right),M_{n}\right)$, para $n\geq1$, son variables aleatorias independientes e id\'enticamente distribuidas con media finita, entonces,
\begin{eqnarray}
lim_{t\rightarrow\infty}t^{-1}Z\left(t\right)\rightarrow\frac{\esp\left[Z\left(T_{1}\right)-Z\left(T_{0}\right)\right]}{\esp\left[T_{1}\right]},\textrm{ c.s. cuando  }t\rightarrow\infty.
\end{eqnarray}
\end{Coro}

%___________________________________________________________________________________________
%
\subsection{Propiedades de los Procesos de Renovaci\'on}
%___________________________________________________________________________________________
%

Los tiempos $T_{n}$ est\'an relacionados con los conteos de $N\left(t\right)$ por

\begin{eqnarray*}
\left\{N\left(t\right)\geq n\right\}&=&\left\{T_{n}\leq t\right\}\\
T_{N\left(t\right)}\leq &t&<T_{N\left(t\right)+1},
\end{eqnarray*}

adem\'as $N\left(T_{n}\right)=n$, y 

\begin{eqnarray*}
N\left(t\right)=\max\left\{n:T_{n}\leq t\right\}=\min\left\{n:T_{n+1}>t\right\}
\end{eqnarray*}

Por propiedades de la convoluci\'on se sabe que

\begin{eqnarray*}
P\left\{T_{n}\leq t\right\}=F^{n\star}\left(t\right)
\end{eqnarray*}
que es la $n$-\'esima convoluci\'on de $F$. Entonces 

\begin{eqnarray*}
\left\{N\left(t\right)\geq n\right\}&=&\left\{T_{n}\leq t\right\}\\
P\left\{N\left(t\right)\leq n\right\}&=&1-F^{\left(n+1\right)\star}\left(t\right)
\end{eqnarray*}

Adem\'as usando el hecho de que $\esp\left[N\left(t\right)\right]=\sum_{n=1}^{\infty}P\left\{N\left(t\right)\geq n\right\}$
se tiene que

\begin{eqnarray*}
\esp\left[N\left(t\right)\right]=\sum_{n=1}^{\infty}F^{n\star}\left(t\right)
\end{eqnarray*}

\begin{Prop}
Para cada $t\geq0$, la funci\'on generadora de momentos $\esp\left[e^{\alpha N\left(t\right)}\right]$ existe para alguna $\alpha$ en una vecindad del 0, y de aqu\'i que $\esp\left[N\left(t\right)^{m}\right]<\infty$, para $m\geq1$.
\end{Prop}


\begin{Note}
Si el primer tiempo de renovaci\'on $\xi_{1}$ no tiene la misma distribuci\'on que el resto de las $\xi_{n}$, para $n\geq2$, a $N\left(t\right)$ se le llama Proceso de Renovaci\'on retardado, donde si $\xi$ tiene distribuci\'on $G$, entonces el tiempo $T_{n}$ de la $n$-\'esima renovaci\'on tiene distribuci\'on $G\star F^{\left(n-1\right)\star}\left(t\right)$
\end{Note}


\begin{Teo}
Para una constante $\mu\leq\infty$ ( o variable aleatoria), las siguientes expresiones son equivalentes:

\begin{eqnarray}
lim_{n\rightarrow\infty}n^{-1}T_{n}&=&\mu,\textrm{ c.s.}\\
lim_{t\rightarrow\infty}t^{-1}N\left(t\right)&=&1/\mu,\textrm{ c.s.}
\end{eqnarray}
\end{Teo}


Es decir, $T_{n}$ satisface la Ley Fuerte de los Grandes N\'umeros s\'i y s\'olo s\'i $N\left/t\right)$ la cumple.


\begin{Coro}[Ley Fuerte de los Grandes N\'umeros para Procesos de Renovaci\'on]
Si $N\left(t\right)$ es un proceso de renovaci\'on cuyos tiempos de inter-renovaci\'on tienen media $\mu\leq\infty$, entonces
\begin{eqnarray}
t^{-1}N\left(t\right)\rightarrow 1/\mu,\textrm{ c.s. cuando }t\rightarrow\infty.
\end{eqnarray}

\end{Coro}


Considerar el proceso estoc\'astico de valores reales $\left\{Z\left(t\right):t\geq0\right\}$ en el mismo espacio de probabilidad que $N\left(t\right)$

\begin{Def}
Para el proceso $\left\{Z\left(t\right):t\geq0\right\}$ se define la fluctuaci\'on m\'axima de $Z\left(t\right)$ en el intervalo $\left(T_{n-1},T_{n}\right]$:
\begin{eqnarray*}
M_{n}=\sup_{T_{n-1}<t\leq T_{n}}|Z\left(t\right)-Z\left(T_{n-1}\right)|
\end{eqnarray*}
\end{Def}

\begin{Teo}
Sup\'ongase que $n^{-1}T_{n}\rightarrow\mu$ c.s. cuando $n\rightarrow\infty$, donde $\mu\leq\infty$ es una constante o variable aleatoria. Sea $a$ una constante o variable aleatoria que puede ser infinita cuando $\mu$ es finita, y considere las expresiones l\'imite:
\begin{eqnarray}
lim_{n\rightarrow\infty}n^{-1}Z\left(T_{n}\right)&=&a,\textrm{ c.s.}\\
lim_{t\rightarrow\infty}t^{-1}Z\left(t\right)&=&a/\mu,\textrm{ c.s.}
\end{eqnarray}
La segunda expresi\'on implica la primera. Conversamente, la primera implica la segunda si el proceso $Z\left(t\right)$ es creciente, o si $lim_{n\rightarrow\infty}n^{-1}M_{n}=0$ c.s.
\end{Teo}

\begin{Coro}
Si $N\left(t\right)$ es un proceso de renovaci\'on, y $\left(Z\left(T_{n}\right)-Z\left(T_{n-1}\right),M_{n}\right)$, para $n\geq1$, son variables aleatorias independientes e id\'enticamente distribuidas con media finita, entonces,
\begin{eqnarray}
lim_{t\rightarrow\infty}t^{-1}Z\left(t\right)\rightarrow\frac{\esp\left[Z\left(T_{1}\right)-Z\left(T_{0}\right)\right]}{\esp\left[T_{1}\right]},\textrm{ c.s. cuando  }t\rightarrow\infty.
\end{eqnarray}
\end{Coro}

%___________________________________________________________________________________________
%
\subsection{Propiedades de los Procesos de Renovaci\'on}
%___________________________________________________________________________________________
%

Los tiempos $T_{n}$ est\'an relacionados con los conteos de $N\left(t\right)$ por

\begin{eqnarray*}
\left\{N\left(t\right)\geq n\right\}&=&\left\{T_{n}\leq t\right\}\\
T_{N\left(t\right)}\leq &t&<T_{N\left(t\right)+1},
\end{eqnarray*}

adem\'as $N\left(T_{n}\right)=n$, y 

\begin{eqnarray*}
N\left(t\right)=\max\left\{n:T_{n}\leq t\right\}=\min\left\{n:T_{n+1}>t\right\}
\end{eqnarray*}

Por propiedades de la convoluci\'on se sabe que

\begin{eqnarray*}
P\left\{T_{n}\leq t\right\}=F^{n\star}\left(t\right)
\end{eqnarray*}
que es la $n$-\'esima convoluci\'on de $F$. Entonces 

\begin{eqnarray*}
\left\{N\left(t\right)\geq n\right\}&=&\left\{T_{n}\leq t\right\}\\
P\left\{N\left(t\right)\leq n\right\}&=&1-F^{\left(n+1\right)\star}\left(t\right)
\end{eqnarray*}

Adem\'as usando el hecho de que $\esp\left[N\left(t\right)\right]=\sum_{n=1}^{\infty}P\left\{N\left(t\right)\geq n\right\}$
se tiene que

\begin{eqnarray*}
\esp\left[N\left(t\right)\right]=\sum_{n=1}^{\infty}F^{n\star}\left(t\right)
\end{eqnarray*}

\begin{Prop}
Para cada $t\geq0$, la funci\'on generadora de momentos $\esp\left[e^{\alpha N\left(t\right)}\right]$ existe para alguna $\alpha$ en una vecindad del 0, y de aqu\'i que $\esp\left[N\left(t\right)^{m}\right]<\infty$, para $m\geq1$.
\end{Prop}


\begin{Note}
Si el primer tiempo de renovaci\'on $\xi_{1}$ no tiene la misma distribuci\'on que el resto de las $\xi_{n}$, para $n\geq2$, a $N\left(t\right)$ se le llama Proceso de Renovaci\'on retardado, donde si $\xi$ tiene distribuci\'on $G$, entonces el tiempo $T_{n}$ de la $n$-\'esima renovaci\'on tiene distribuci\'on $G\star F^{\left(n-1\right)\star}\left(t\right)$
\end{Note}


\begin{Teo}
Para una constante $\mu\leq\infty$ ( o variable aleatoria), las siguientes expresiones son equivalentes:

\begin{eqnarray}
lim_{n\rightarrow\infty}n^{-1}T_{n}&=&\mu,\textrm{ c.s.}\\
lim_{t\rightarrow\infty}t^{-1}N\left(t\right)&=&1/\mu,\textrm{ c.s.}
\end{eqnarray}
\end{Teo}


Es decir, $T_{n}$ satisface la Ley Fuerte de los Grandes N\'umeros s\'i y s\'olo s\'i $N\left/t\right)$ la cumple.


\begin{Coro}[Ley Fuerte de los Grandes N\'umeros para Procesos de Renovaci\'on]
Si $N\left(t\right)$ es un proceso de renovaci\'on cuyos tiempos de inter-renovaci\'on tienen media $\mu\leq\infty$, entonces
\begin{eqnarray}
t^{-1}N\left(t\right)\rightarrow 1/\mu,\textrm{ c.s. cuando }t\rightarrow\infty.
\end{eqnarray}

\end{Coro}


Considerar el proceso estoc\'astico de valores reales $\left\{Z\left(t\right):t\geq0\right\}$ en el mismo espacio de probabilidad que $N\left(t\right)$

\begin{Def}
Para el proceso $\left\{Z\left(t\right):t\geq0\right\}$ se define la fluctuaci\'on m\'axima de $Z\left(t\right)$ en el intervalo $\left(T_{n-1},T_{n}\right]$:
\begin{eqnarray*}
M_{n}=\sup_{T_{n-1}<t\leq T_{n}}|Z\left(t\right)-Z\left(T_{n-1}\right)|
\end{eqnarray*}
\end{Def}

\begin{Teo}
Sup\'ongase que $n^{-1}T_{n}\rightarrow\mu$ c.s. cuando $n\rightarrow\infty$, donde $\mu\leq\infty$ es una constante o variable aleatoria. Sea $a$ una constante o variable aleatoria que puede ser infinita cuando $\mu$ es finita, y considere las expresiones l\'imite:
\begin{eqnarray}
lim_{n\rightarrow\infty}n^{-1}Z\left(T_{n}\right)&=&a,\textrm{ c.s.}\\
lim_{t\rightarrow\infty}t^{-1}Z\left(t\right)&=&a/\mu,\textrm{ c.s.}
\end{eqnarray}
La segunda expresi\'on implica la primera. Conversamente, la primera implica la segunda si el proceso $Z\left(t\right)$ es creciente, o si $lim_{n\rightarrow\infty}n^{-1}M_{n}=0$ c.s.
\end{Teo}

\begin{Coro}
Si $N\left(t\right)$ es un proceso de renovaci\'on, y $\left(Z\left(T_{n}\right)-Z\left(T_{n-1}\right),M_{n}\right)$, para $n\geq1$, son variables aleatorias independientes e id\'enticamente distribuidas con media finita, entonces,
\begin{eqnarray}
lim_{t\rightarrow\infty}t^{-1}Z\left(t\right)\rightarrow\frac{\esp\left[Z\left(T_{1}\right)-Z\left(T_{0}\right)\right]}{\esp\left[T_{1}\right]},\textrm{ c.s. cuando  }t\rightarrow\infty.
\end{eqnarray}
\end{Coro}


%__________________________________________________________________________________________
\subsection{Procesos Regenerativos Estacionarios - Stidham \cite{Stidham}}
%__________________________________________________________________________________________


Un proceso estoc\'astico a tiempo continuo $\left\{V\left(t\right),t\geq0\right\}$ es un proceso regenerativo si existe una sucesi\'on de variables aleatorias independientes e id\'enticamente distribuidas $\left\{X_{1},X_{2},\ldots\right\}$, sucesi\'on de renovaci\'on, tal que para cualquier conjunto de Borel $A$, 

\begin{eqnarray*}
\prob\left\{V\left(t\right)\in A|X_{1}+X_{2}+\cdots+X_{R\left(t\right)}=s,\left\{V\left(\tau\right),\tau<s\right\}\right\}=\prob\left\{V\left(t-s\right)\in A|X_{1}>t-s\right\},
\end{eqnarray*}
para todo $0\leq s\leq t$, donde $R\left(t\right)=\max\left\{X_{1}+X_{2}+\cdots+X_{j}\leq t\right\}=$n\'umero de renovaciones ({\emph{puntos de regeneraci\'on}}) que ocurren en $\left[0,t\right]$. El intervalo $\left[0,X_{1}\right)$ es llamado {\emph{primer ciclo de regeneraci\'on}} de $\left\{V\left(t \right),t\geq0\right\}$, $\left[X_{1},X_{1}+X_{2}\right)$ el {\emph{segundo ciclo de regeneraci\'on}}, y as\'i sucesivamente.

Sea $X=X_{1}$ y sea $F$ la funci\'on de distrbuci\'on de $X$


\begin{Def}
Se define el proceso estacionario, $\left\{V^{*}\left(t\right),t\geq0\right\}$, para $\left\{V\left(t\right),t\geq0\right\}$ por

\begin{eqnarray*}
\prob\left\{V\left(t\right)\in A\right\}=\frac{1}{\esp\left[X\right]}\int_{0}^{\infty}\prob\left\{V\left(t+x\right)\in A|X>x\right\}\left(1-F\left(x\right)\right)dx,
\end{eqnarray*} 
para todo $t\geq0$ y todo conjunto de Borel $A$.
\end{Def}

\begin{Def}
Una distribuci\'on se dice que es {\emph{aritm\'etica}} si todos sus puntos de incremento son m\'ultiplos de la forma $0,\lambda, 2\lambda,\ldots$ para alguna $\lambda>0$ entera.
\end{Def}


\begin{Def}
Una modificaci\'on medible de un proceso $\left\{V\left(t\right),t\geq0\right\}$, es una versi\'on de este, $\left\{V\left(t,w\right)\right\}$ conjuntamente medible para $t\geq0$ y para $w\in S$, $S$ espacio de estados para $\left\{V\left(t\right),t\geq0\right\}$.
\end{Def}

\begin{Teo}
Sea $\left\{V\left(t\right),t\geq\right\}$ un proceso regenerativo no negativo con modificaci\'on medible. Sea $\esp\left[X\right]<\infty$. Entonces el proceso estacionario dado por la ecuaci\'on anterior est\'a bien definido y tiene funci\'on de distribuci\'on independiente de $t$, adem\'as
\begin{itemize}
\item[i)] \begin{eqnarray*}
\esp\left[V^{*}\left(0\right)\right]&=&\frac{\esp\left[\int_{0}^{X}V\left(s\right)ds\right]}{\esp\left[X\right]}\end{eqnarray*}
\item[ii)] Si $\esp\left[V^{*}\left(0\right)\right]<\infty$, equivalentemente, si $\esp\left[\int_{0}^{X}V\left(s\right)ds\right]<\infty$,entonces
\begin{eqnarray*}
\frac{\int_{0}^{t}V\left(s\right)ds}{t}\rightarrow\frac{\esp\left[\int_{0}^{X}V\left(s\right)ds\right]}{\esp\left[X\right]}
\end{eqnarray*}
con probabilidad 1 y en media, cuando $t\rightarrow\infty$.
\end{itemize}
\end{Teo}

%______________________________________________________________________
\subsection{Procesos de Renovaci\'on}
%______________________________________________________________________

\begin{Def}\label{Def.Tn}
Sean $0\leq T_{1}\leq T_{2}\leq \ldots$ son tiempos aleatorios infinitos en los cuales ocurren ciertos eventos. El n\'umero de tiempos $T_{n}$ en el intervalo $\left[0,t\right)$ es

\begin{eqnarray}
N\left(t\right)=\sum_{n=1}^{\infty}\indora\left(T_{n}\leq t\right),
\end{eqnarray}
para $t\geq0$.
\end{Def}

Si se consideran los puntos $T_{n}$ como elementos de $\rea_{+}$, y $N\left(t\right)$ es el n\'umero de puntos en $\rea$. El proceso denotado por $\left\{N\left(t\right):t\geq0\right\}$, denotado por $N\left(t\right)$, es un proceso puntual en $\rea_{+}$. Los $T_{n}$ son los tiempos de ocurrencia, el proceso puntual $N\left(t\right)$ es simple si su n\'umero de ocurrencias son distintas: $0<T_{1}<T_{2}<\ldots$ casi seguramente.

\begin{Def}
Un proceso puntual $N\left(t\right)$ es un proceso de renovaci\'on si los tiempos de interocurrencia $\xi_{n}=T_{n}-T_{n-1}$, para $n\geq1$, son independientes e identicamente distribuidos con distribuci\'on $F$, donde $F\left(0\right)=0$ y $T_{0}=0$. Los $T_{n}$ son llamados tiempos de renovaci\'on, referente a la independencia o renovaci\'on de la informaci\'on estoc\'astica en estos tiempos. Los $\xi_{n}$ son los tiempos de inter-renovaci\'on, y $N\left(t\right)$ es el n\'umero de renovaciones en el intervalo $\left[0,t\right)$
\end{Def}


\begin{Note}
Para definir un proceso de renovaci\'on para cualquier contexto, solamente hay que especificar una distribuci\'on $F$, con $F\left(0\right)=0$, para los tiempos de inter-renovaci\'on. La funci\'on $F$ en turno degune las otra variables aleatorias. De manera formal, existe un espacio de probabilidad y una sucesi\'on de variables aleatorias $\xi_{1},\xi_{2},\ldots$ definidas en este con distribuci\'on $F$. Entonces las otras cantidades son $T_{n}=\sum_{k=1}^{n}\xi_{k}$ y $N\left(t\right)=\sum_{n=1}^{\infty}\indora\left(T_{n}\leq t\right)$, donde $T_{n}\rightarrow\infty$ casi seguramente por la Ley Fuerte de los Grandes Números.
\end{Note}

%___________________________________________________________________________________________
%
\subsection{Teorema Principal de Renovaci\'on}
%___________________________________________________________________________________________
%

\begin{Note} Una funci\'on $h:\rea_{+}\rightarrow\rea$ es Directamente Riemann Integrable en los siguientes casos:
\begin{itemize}
\item[a)] $h\left(t\right)\geq0$ es decreciente y Riemann Integrable.
\item[b)] $h$ es continua excepto posiblemente en un conjunto de Lebesgue de medida 0, y $|h\left(t\right)|\leq b\left(t\right)$, donde $b$ es DRI.
\end{itemize}
\end{Note}

\begin{Teo}[Teorema Principal de Renovaci\'on]
Si $F$ es no aritm\'etica y $h\left(t\right)$ es Directamente Riemann Integrable (DRI), entonces

\begin{eqnarray*}
lim_{t\rightarrow\infty}U\star h=\frac{1}{\mu}\int_{\rea_{+}}h\left(s\right)ds.
\end{eqnarray*}
\end{Teo}

\begin{Prop}
Cualquier funci\'on $H\left(t\right)$ acotada en intervalos finitos y que es 0 para $t<0$ puede expresarse como
\begin{eqnarray*}
H\left(t\right)=U\star h\left(t\right)\textrm{,  donde }h\left(t\right)=H\left(t\right)-F\star H\left(t\right)
\end{eqnarray*}
\end{Prop}

\begin{Def}
Un proceso estoc\'astico $X\left(t\right)$ es crudamente regenerativo en un tiempo aleatorio positivo $T$ si
\begin{eqnarray*}
\esp\left[X\left(T+t\right)|T\right]=\esp\left[X\left(t\right)\right]\textrm{, para }t\geq0,\end{eqnarray*}
y con las esperanzas anteriores finitas.
\end{Def}

\begin{Prop}
Sup\'ongase que $X\left(t\right)$ es un proceso crudamente regenerativo en $T$, que tiene distribuci\'on $F$. Si $\esp\left[X\left(t\right)\right]$ es acotado en intervalos finitos, entonces
\begin{eqnarray*}
\esp\left[X\left(t\right)\right]=U\star h\left(t\right)\textrm{,  donde }h\left(t\right)=\esp\left[X\left(t\right)\indora\left(T>t\right)\right].
\end{eqnarray*}
\end{Prop}

\begin{Teo}[Regeneraci\'on Cruda]
Sup\'ongase que $X\left(t\right)$ es un proceso con valores positivo crudamente regenerativo en $T$, y def\'inase $M=\sup\left\{|X\left(t\right)|:t\leq T\right\}$. Si $T$ es no aritm\'etico y $M$ y $MT$ tienen media finita, entonces
\begin{eqnarray*}
lim_{t\rightarrow\infty}\esp\left[X\left(t\right)\right]=\frac{1}{\mu}\int_{\rea_{+}}h\left(s\right)ds,
\end{eqnarray*}
donde $h\left(t\right)=\esp\left[X\left(t\right)\indora\left(T>t\right)\right]$.
\end{Teo}



%___________________________________________________________________________________________
%
\subsection{Funci\'on de Renovaci\'on}
%___________________________________________________________________________________________
%


\begin{Def}
Sea $h\left(t\right)$ funci\'on de valores reales en $\rea$ acotada en intervalos finitos e igual a cero para $t<0$ La ecuaci\'on de renovaci\'on para $h\left(t\right)$ y la distribuci\'on $F$ es

\begin{eqnarray}\label{Ec.Renovacion}
H\left(t\right)=h\left(t\right)+\int_{\left[0,t\right]}H\left(t-s\right)dF\left(s\right)\textrm{,    }t\geq0,
\end{eqnarray}
donde $H\left(t\right)$ es una funci\'on de valores reales. Esto es $H=h+F\star H$. Decimos que $H\left(t\right)$ es soluci\'on de esta ecuaci\'on si satisface la ecuaci\'on, y es acotada en intervalos finitos e iguales a cero para $t<0$.
\end{Def}

\begin{Prop}
La funci\'on $U\star h\left(t\right)$ es la \'unica soluci\'on de la ecuaci\'on de renovaci\'on (\ref{Ec.Renovacion}).
\end{Prop}

\begin{Teo}[Teorema Renovaci\'on Elemental]
\begin{eqnarray*}
t^{-1}U\left(t\right)\rightarrow 1/\mu\textrm{,    cuando }t\rightarrow\infty.
\end{eqnarray*}
\end{Teo}

%___________________________________________________________________________________________
%
\subsection{Propiedades de los Procesos de Renovaci\'on}
%___________________________________________________________________________________________
%

Los tiempos $T_{n}$ est\'an relacionados con los conteos de $N\left(t\right)$ por

\begin{eqnarray*}
\left\{N\left(t\right)\geq n\right\}&=&\left\{T_{n}\leq t\right\}\\
T_{N\left(t\right)}\leq &t&<T_{N\left(t\right)+1},
\end{eqnarray*}

adem\'as $N\left(T_{n}\right)=n$, y 

\begin{eqnarray*}
N\left(t\right)=\max\left\{n:T_{n}\leq t\right\}=\min\left\{n:T_{n+1}>t\right\}
\end{eqnarray*}

Por propiedades de la convoluci\'on se sabe que

\begin{eqnarray*}
P\left\{T_{n}\leq t\right\}=F^{n\star}\left(t\right)
\end{eqnarray*}
que es la $n$-\'esima convoluci\'on de $F$. Entonces 

\begin{eqnarray*}
\left\{N\left(t\right)\geq n\right\}&=&\left\{T_{n}\leq t\right\}\\
P\left\{N\left(t\right)\leq n\right\}&=&1-F^{\left(n+1\right)\star}\left(t\right)
\end{eqnarray*}

Adem\'as usando el hecho de que $\esp\left[N\left(t\right)\right]=\sum_{n=1}^{\infty}P\left\{N\left(t\right)\geq n\right\}$
se tiene que

\begin{eqnarray*}
\esp\left[N\left(t\right)\right]=\sum_{n=1}^{\infty}F^{n\star}\left(t\right)
\end{eqnarray*}

\begin{Prop}
Para cada $t\geq0$, la funci\'on generadora de momentos $\esp\left[e^{\alpha N\left(t\right)}\right]$ existe para alguna $\alpha$ en una vecindad del 0, y de aqu\'i que $\esp\left[N\left(t\right)^{m}\right]<\infty$, para $m\geq1$.
\end{Prop}


\begin{Note}
Si el primer tiempo de renovaci\'on $\xi_{1}$ no tiene la misma distribuci\'on que el resto de las $\xi_{n}$, para $n\geq2$, a $N\left(t\right)$ se le llama Proceso de Renovaci\'on retardado, donde si $\xi$ tiene distribuci\'on $G$, entonces el tiempo $T_{n}$ de la $n$-\'esima renovaci\'on tiene distribuci\'on $G\star F^{\left(n-1\right)\star}\left(t\right)$
\end{Note}


\begin{Teo}
Para una constante $\mu\leq\infty$ ( o variable aleatoria), las siguientes expresiones son equivalentes:

\begin{eqnarray}
lim_{n\rightarrow\infty}n^{-1}T_{n}&=&\mu,\textrm{ c.s.}\\
lim_{t\rightarrow\infty}t^{-1}N\left(t\right)&=&1/\mu,\textrm{ c.s.}
\end{eqnarray}
\end{Teo}


Es decir, $T_{n}$ satisface la Ley Fuerte de los Grandes N\'umeros s\'i y s\'olo s\'i $N\left/t\right)$ la cumple.


\begin{Coro}[Ley Fuerte de los Grandes N\'umeros para Procesos de Renovaci\'on]
Si $N\left(t\right)$ es un proceso de renovaci\'on cuyos tiempos de inter-renovaci\'on tienen media $\mu\leq\infty$, entonces
\begin{eqnarray}
t^{-1}N\left(t\right)\rightarrow 1/\mu,\textrm{ c.s. cuando }t\rightarrow\infty.
\end{eqnarray}

\end{Coro}


Considerar el proceso estoc\'astico de valores reales $\left\{Z\left(t\right):t\geq0\right\}$ en el mismo espacio de probabilidad que $N\left(t\right)$

\begin{Def}
Para el proceso $\left\{Z\left(t\right):t\geq0\right\}$ se define la fluctuaci\'on m\'axima de $Z\left(t\right)$ en el intervalo $\left(T_{n-1},T_{n}\right]$:
\begin{eqnarray*}
M_{n}=\sup_{T_{n-1}<t\leq T_{n}}|Z\left(t\right)-Z\left(T_{n-1}\right)|
\end{eqnarray*}
\end{Def}

\begin{Teo}
Sup\'ongase que $n^{-1}T_{n}\rightarrow\mu$ c.s. cuando $n\rightarrow\infty$, donde $\mu\leq\infty$ es una constante o variable aleatoria. Sea $a$ una constante o variable aleatoria que puede ser infinita cuando $\mu$ es finita, y considere las expresiones l\'imite:
\begin{eqnarray}
lim_{n\rightarrow\infty}n^{-1}Z\left(T_{n}\right)&=&a,\textrm{ c.s.}\\
lim_{t\rightarrow\infty}t^{-1}Z\left(t\right)&=&a/\mu,\textrm{ c.s.}
\end{eqnarray}
La segunda expresi\'on implica la primera. Conversamente, la primera implica la segunda si el proceso $Z\left(t\right)$ es creciente, o si $lim_{n\rightarrow\infty}n^{-1}M_{n}=0$ c.s.
\end{Teo}

\begin{Coro}
Si $N\left(t\right)$ es un proceso de renovaci\'on, y $\left(Z\left(T_{n}\right)-Z\left(T_{n-1}\right),M_{n}\right)$, para $n\geq1$, son variables aleatorias independientes e id\'enticamente distribuidas con media finita, entonces,
\begin{eqnarray}
lim_{t\rightarrow\infty}t^{-1}Z\left(t\right)\rightarrow\frac{\esp\left[Z\left(T_{1}\right)-Z\left(T_{0}\right)\right]}{\esp\left[T_{1}\right]},\textrm{ c.s. cuando  }t\rightarrow\infty.
\end{eqnarray}
\end{Coro}

%___________________________________________________________________________________________
%
\subsection{Funci\'on de Renovaci\'on}
%___________________________________________________________________________________________
%


Sup\'ongase que $N\left(t\right)$ es un proceso de renovaci\'on con distribuci\'on $F$ con media finita $\mu$.

\begin{Def}
La funci\'on de renovaci\'on asociada con la distribuci\'on $F$, del proceso $N\left(t\right)$, es
\begin{eqnarray*}
U\left(t\right)=\sum_{n=1}^{\infty}F^{n\star}\left(t\right),\textrm{   }t\geq0,
\end{eqnarray*}
donde $F^{0\star}\left(t\right)=\indora\left(t\geq0\right)$.
\end{Def}


\begin{Prop}
Sup\'ongase que la distribuci\'on de inter-renovaci\'on $F$ tiene densidad $f$. Entonces $U\left(t\right)$ tambi\'en tiene densidad, para $t>0$, y es $U^{'}\left(t\right)=\sum_{n=0}^{\infty}f^{n\star}\left(t\right)$. Adem\'as
\begin{eqnarray*}
\prob\left\{N\left(t\right)>N\left(t-\right)\right\}=0\textrm{,   }t\geq0.
\end{eqnarray*}
\end{Prop}

\begin{Def}
La Transformada de Laplace-Stieljes de $F$ est\'a dada por

\begin{eqnarray*}
\hat{F}\left(\alpha\right)=\int_{\rea_{+}}e^{-\alpha t}dF\left(t\right)\textrm{,  }\alpha\geq0.
\end{eqnarray*}
\end{Def}

Entonces

\begin{eqnarray*}
\hat{U}\left(\alpha\right)=\sum_{n=0}^{\infty}\hat{F^{n\star}}\left(\alpha\right)=\sum_{n=0}^{\infty}\hat{F}\left(\alpha\right)^{n}=\frac{1}{1-\hat{F}\left(\alpha\right)}.
\end{eqnarray*}


\begin{Prop}
La Transformada de Laplace $\hat{U}\left(\alpha\right)$ y $\hat{F}\left(\alpha\right)$ determina una a la otra de manera \'unica por la relaci\'on $\hat{U}\left(\alpha\right)=\frac{1}{1-\hat{F}\left(\alpha\right)}$.
\end{Prop}


\begin{Note}
Un proceso de renovaci\'on $N\left(t\right)$ cuyos tiempos de inter-renovaci\'on tienen media finita, es un proceso Poisson con tasa $\lambda$ si y s\'olo s\'i $\esp\left[U\left(t\right)\right]=\lambda t$, para $t\geq0$.
\end{Note}


\begin{Teo}
Sea $N\left(t\right)$ un proceso puntual simple con puntos de localizaci\'on $T_{n}$ tal que $\eta\left(t\right)=\esp\left[N\left(\right)\right]$ es finita para cada $t$. Entonces para cualquier funci\'on $f:\rea_{+}\rightarrow\rea$,
\begin{eqnarray*}
\esp\left[\sum_{n=1}^{N\left(\right)}f\left(T_{n}\right)\right]=\int_{\left(0,t\right]}f\left(s\right)d\eta\left(s\right)\textrm{,  }t\geq0,
\end{eqnarray*}
suponiendo que la integral exista. Adem\'as si $X_{1},X_{2},\ldots$ son variables aleatorias definidas en el mismo espacio de probabilidad que el proceso $N\left(t\right)$ tal que $\esp\left[X_{n}|T_{n}=s\right]=f\left(s\right)$, independiente de $n$. Entonces
\begin{eqnarray*}
\esp\left[\sum_{n=1}^{N\left(t\right)}X_{n}\right]=\int_{\left(0,t\right]}f\left(s\right)d\eta\left(s\right)\textrm{,  }t\geq0,
\end{eqnarray*} 
suponiendo que la integral exista. 
\end{Teo}

\begin{Coro}[Identidad de Wald para Renovaciones]
Para el proceso de renovaci\'on $N\left(t\right)$,
\begin{eqnarray*}
\esp\left[T_{N\left(t\right)+1}\right]=\mu\esp\left[N\left(t\right)+1\right]\textrm{,  }t\geq0,
\end{eqnarray*}  
\end{Coro}

%______________________________________________________________________
\subsection{Procesos de Renovaci\'on}
%______________________________________________________________________

\begin{Def}\label{Def.Tn}
Sean $0\leq T_{1}\leq T_{2}\leq \ldots$ son tiempos aleatorios infinitos en los cuales ocurren ciertos eventos. El n\'umero de tiempos $T_{n}$ en el intervalo $\left[0,t\right)$ es

\begin{eqnarray}
N\left(t\right)=\sum_{n=1}^{\infty}\indora\left(T_{n}\leq t\right),
\end{eqnarray}
para $t\geq0$.
\end{Def}

Si se consideran los puntos $T_{n}$ como elementos de $\rea_{+}$, y $N\left(t\right)$ es el n\'umero de puntos en $\rea$. El proceso denotado por $\left\{N\left(t\right):t\geq0\right\}$, denotado por $N\left(t\right)$, es un proceso puntual en $\rea_{+}$. Los $T_{n}$ son los tiempos de ocurrencia, el proceso puntual $N\left(t\right)$ es simple si su n\'umero de ocurrencias son distintas: $0<T_{1}<T_{2}<\ldots$ casi seguramente.

\begin{Def}
Un proceso puntual $N\left(t\right)$ es un proceso de renovaci\'on si los tiempos de interocurrencia $\xi_{n}=T_{n}-T_{n-1}$, para $n\geq1$, son independientes e identicamente distribuidos con distribuci\'on $F$, donde $F\left(0\right)=0$ y $T_{0}=0$. Los $T_{n}$ son llamados tiempos de renovaci\'on, referente a la independencia o renovaci\'on de la informaci\'on estoc\'astica en estos tiempos. Los $\xi_{n}$ son los tiempos de inter-renovaci\'on, y $N\left(t\right)$ es el n\'umero de renovaciones en el intervalo $\left[0,t\right)$
\end{Def}


\begin{Note}
Para definir un proceso de renovaci\'on para cualquier contexto, solamente hay que especificar una distribuci\'on $F$, con $F\left(0\right)=0$, para los tiempos de inter-renovaci\'on. La funci\'on $F$ en turno degune las otra variables aleatorias. De manera formal, existe un espacio de probabilidad y una sucesi\'on de variables aleatorias $\xi_{1},\xi_{2},\ldots$ definidas en este con distribuci\'on $F$. Entonces las otras cantidades son $T_{n}=\sum_{k=1}^{n}\xi_{k}$ y $N\left(t\right)=\sum_{n=1}^{\infty}\indora\left(T_{n}\leq t\right)$, donde $T_{n}\rightarrow\infty$ casi seguramente por la Ley Fuerte de los Grandes Números.
\end{Note}
%_____________________________________________________
\subsection{Puntos de Renovaci\'on}
%_____________________________________________________

Para cada cola $Q_{i}$ se tienen los procesos de arribo a la cola, para estas, los tiempos de arribo est\'an dados por $$\left\{T_{1}^{i},T_{2}^{i},\ldots,T_{k}^{i},\ldots\right\},$$ entonces, consideremos solamente los primeros tiempos de arribo a cada una de las colas, es decir, $$\left\{T_{1}^{1},T_{1}^{2},T_{1}^{3},T_{1}^{4}\right\},$$ se sabe que cada uno de estos tiempos se distribuye de manera exponencial con par\'ametro $1/mu_{i}$. Adem\'as se sabe que para $$T^{*}=\min\left\{T_{1}^{1},T_{1}^{2},T_{1}^{3},T_{1}^{4}\right\},$$ $T^{*}$ se distribuye de manera exponencial con par\'ametro $$\mu^{*}=\sum_{i=1}^{4}\mu_{i}.$$ Ahora, dado que 
\begin{center}
\begin{tabular}{lcl}
$\tilde{r}=r_{1}+r_{2}$ & y &$\hat{r}=r_{3}+r_{4}.$
\end{tabular}
\end{center}


Supongamos que $$\tilde{r},\hat{r}<\mu^{*},$$ entonces si tomamos $$r^{*}=\min\left\{\tilde{r},\hat{r}\right\},$$ se tiene que para  $$t^{*}\in\left(0,r^{*}\right)$$ se cumple que 
\begin{center}
\begin{tabular}{lcl}
$\tau_{1}\left(1\right)=0$ & y por tanto & $\overline{\tau}_{1}=0,$
\end{tabular}
\end{center}
entonces para la segunda cola en este primer ciclo se cumple que $$\tau_{2}=\overline{\tau}_{1}+r_{1}=r_{1}<\mu^{*},$$ y por tanto se tiene que  $$\overline{\tau}_{2}=\tau_{2}.$$ Por lo tanto, nuevamente para la primer cola en el segundo ciclo $$\tau_{1}\left(2\right)=\tau_{2}\left(1\right)+r_{2}=\tilde{r}<\mu^{*}.$$ An\'alogamente para el segundo sistema se tiene que ambas colas est\'an vac\'ias, es decir, existe un valor $t^{*}$ tal que en el intervalo $\left(0,t^{*}\right)$ no ha llegado ning\'un usuario, es decir, $$L_{i}\left(t^{*}\right)=0$$ para $i=1,2,3,4$.

\subsection{Resultados para Procesos de Salida}

En \cite{Sigman2} prueban que para la existencia de un una sucesi\'on infinita no decreciente de tiempos de regeneraci\'on $\tau_{1}\leq\tau_{2}\leq\cdots$ en los cuales el proceso se regenera, basta un tiempo de regeneraci\'on $R_{1}$, donde $R_{j}=\tau_{j}-\tau_{j-1}$. Para tal efecto se requiere la existencia de un espacio de probabilidad $\left(\Omega,\mathcal{F},\prob\right)$, y proceso estoc\'astico $\textit{X}=\left\{X\left(t\right):t\geq0\right\}$ con espacio de estados $\left(S,\mathcal{R}\right)$, con $\mathcal{R}$ $\sigma$-\'algebra.

\begin{Prop}
Si existe una variable aleatoria no negativa $R_{1}$ tal que $\theta_{R\footnotesize{1}}X=_{D}X$, entonces $\left(\Omega,\mathcal{F},\prob\right)$ puede extenderse para soportar una sucesi\'on estacionaria de variables aleatorias $R=\left\{R_{k}:k\geq1\right\}$, tal que para $k\geq1$,
\begin{eqnarray*}
\theta_{k}\left(X,R\right)=_{D}\left(X,R\right).
\end{eqnarray*}

Adem\'as, para $k\geq1$, $\theta_{k}R$ es condicionalmente independiente de $\left(X,R_{1},\ldots,R_{k}\right)$, dado $\theta_{\tau k}X$.

\end{Prop}


\begin{itemize}
\item Doob en 1953 demostr\'o que el estado estacionario de un proceso de partida en un sistema de espera $M/G/\infty$, es Poisson con la misma tasa que el proceso de arribos.

\item Burke en 1968, fue el primero en demostrar que el estado estacionario de un proceso de salida de una cola $M/M/s$ es un proceso Poisson.

\item Disney en 1973 obtuvo el siguiente resultado:

\begin{Teo}
Para el sistema de espera $M/G/1/L$ con disciplina FIFO, el proceso $\textbf{I}$ es un proceso de renovaci\'on si y s\'olo si el proceso denominado longitud de la cola es estacionario y se cumple cualquiera de los siguientes casos:

\begin{itemize}
\item[a)] Los tiempos de servicio son identicamente cero;
\item[b)] $L=0$, para cualquier proceso de servicio $S$;
\item[c)] $L=1$ y $G=D$;
\item[d)] $L=\infty$ y $G=M$.
\end{itemize}
En estos casos, respectivamente, las distribuciones de interpartida $P\left\{T_{n+1}-T_{n}\leq t\right\}$ son


\begin{itemize}
\item[a)] $1-e^{-\lambda t}$, $t\geq0$;
\item[b)] $1-e^{-\lambda t}*F\left(t\right)$, $t\geq0$;
\item[c)] $1-e^{-\lambda t}*\indora_{d}\left(t\right)$, $t\geq0$;
\item[d)] $1-e^{-\lambda t}*F\left(t\right)$, $t\geq0$.
\end{itemize}
\end{Teo}


\item Finch (1959) mostr\'o que para los sistemas $M/G/1/L$, con $1\leq L\leq \infty$ con distribuciones de servicio dos veces diferenciable, solamente el sistema $M/M/1/\infty$ tiene proceso de salida de renovaci\'on estacionario.

\item King (1971) demostro que un sistema de colas estacionario $M/G/1/1$ tiene sus tiempos de interpartida sucesivas $D_{n}$ y $D_{n+1}$ son independientes, si y s\'olo si, $G=D$, en cuyo caso le proceso de salida es de renovaci\'on.

\item Disney (1973) demostr\'o que el \'unico sistema estacionario $M/G/1/L$, que tiene proceso de salida de renovaci\'on  son los sistemas $M/M/1$ y $M/D/1/1$.



\item El siguiente resultado es de Disney y Koning (1985)
\begin{Teo}
En un sistema de espera $M/G/s$, el estado estacionario del proceso de salida es un proceso Poisson para cualquier distribuci\'on de los tiempos de servicio si el sistema tiene cualquiera de las siguientes cuatro propiedades.

\begin{itemize}
\item[a)] $s=\infty$
\item[b)] La disciplina de servicio es de procesador compartido.
\item[c)] La disciplina de servicio es LCFS y preemptive resume, esto se cumple para $L<\infty$
\item[d)] $G=M$.
\end{itemize}

\end{Teo}

\item El siguiente resultado es de Alamatsaz (1983)

\begin{Teo}
En cualquier sistema de colas $GI/G/1/L$ con $1\leq L<\infty$ y distribuci\'on de interarribos $A$ y distribuci\'on de los tiempos de servicio $B$, tal que $A\left(0\right)=0$, $A\left(t\right)\left(1-B\left(t\right)\right)>0$ para alguna $t>0$ y $B\left(t\right)$ para toda $t>0$, es imposible que el proceso de salida estacionario sea de renovaci\'on.
\end{Teo}

\end{itemize}

Estos resultados aparecen en Daley (1968) \cite{Daley68} para $\left\{T_{n}\right\}$ intervalos de inter-arribo, $\left\{D_{n}\right\}$ intervalos de inter-salida y $\left\{S_{n}\right\}$ tiempos de servicio.

\begin{itemize}
\item Si el proceso $\left\{T_{n}\right\}$ es Poisson, el proceso $\left\{D_{n}\right\}$ es no correlacionado si y s\'olo si es un proceso Poisso, lo cual ocurre si y s\'olo si $\left\{S_{n}\right\}$ son exponenciales negativas.

\item Si $\left\{S_{n}\right\}$ son exponenciales negativas, $\left\{D_{n}\right\}$ es un proceso de renovaci\'on  si y s\'olo si es un proceso Poisson, lo cual ocurre si y s\'olo si $\left\{T_{n}\right\}$ es un proceso Poisson.

\item $\esp\left(D_{n}\right)=\esp\left(T_{n}\right)$.

\item Para un sistema de visitas $GI/M/1$ se tiene el siguiente teorema:

\begin{Teo}
En un sistema estacionario $GI/M/1$ los intervalos de interpartida tienen
\begin{eqnarray*}
\esp\left(e^{-\theta D_{n}}\right)&=&\mu\left(\mu+\theta\right)^{-1}\left[\delta\theta
-\mu\left(1-\delta\right)\alpha\left(\theta\right)\right]
\left[\theta-\mu\left(1-\delta\right)^{-1}\right]\\
\alpha\left(\theta\right)&=&\esp\left[e^{-\theta T_{0}}\right]\\
var\left(D_{n}\right)&=&var\left(T_{0}\right)-\left(\tau^{-1}-\delta^{-1}\right)
2\delta\left(\esp\left(S_{0}\right)\right)^{2}\left(1-\delta\right)^{-1}.
\end{eqnarray*}
\end{Teo}



\begin{Teo}
El proceso de salida de un sistema de colas estacionario $GI/M/1$ es un proceso de renovaci\'on si y s\'olo si el proceso de entrada es un proceso Poisson, en cuyo caso el proceso de salida es un proceso Poisson.
\end{Teo}


\begin{Teo}
Los intervalos de interpartida $\left\{D_{n}\right\}$ de un sistema $M/G/1$ estacionario son no correlacionados si y s\'olo si la distribuci\'on de los tiempos de servicio es exponencial negativa, es decir, el sistema es de tipo  $M/M/1$.

\end{Teo}



\end{itemize}


%________________________________________________________________________
\subsection{Procesos Regenerativos}
%________________________________________________________________________

Para $\left\{X\left(t\right):t\geq0\right\}$ Proceso Estoc\'astico a tiempo continuo con estado de espacios $S$, que es un espacio m\'etrico, con trayectorias continuas por la derecha y con l\'imites por la izquierda c.s. Sea $N\left(t\right)$ un proceso de renovaci\'on en $\rea_{+}$ definido en el mismo espacio de probabilidad que $X\left(t\right)$, con tiempos de renovaci\'on $T$ y tiempos de inter-renovaci\'on $\xi_{n}=T_{n}-T_{n-1}$, con misma distribuci\'on $F$ de media finita $\mu$.



\begin{Def}
Para el proceso $\left\{\left(N\left(t\right),X\left(t\right)\right):t\geq0\right\}$, sus trayectoria muestrales en el intervalo de tiempo $\left[T_{n-1},T_{n}\right)$ est\'an descritas por
\begin{eqnarray*}
\zeta_{n}=\left(\xi_{n},\left\{X\left(T_{n-1}+t\right):0\leq t<\xi_{n}\right\}\right)
\end{eqnarray*}
Este $\zeta_{n}$ es el $n$-\'esimo segmento del proceso. El proceso es regenerativo sobre los tiempos $T_{n}$ si sus segmentos $\zeta_{n}$ son independientes e id\'enticamennte distribuidos.
\end{Def}


\begin{Obs}
Si $\tilde{X}\left(t\right)$ con espacio de estados $\tilde{S}$ es regenerativo sobre $T_{n}$, entonces $X\left(t\right)=f\left(\tilde{X}\left(t\right)\right)$ tambi\'en es regenerativo sobre $T_{n}$, para cualquier funci\'on $f:\tilde{S}\rightarrow S$.
\end{Obs}

\begin{Obs}
Los procesos regenerativos son crudamente regenerativos, pero no al rev\'es.
\end{Obs}

\begin{Def}[Definici\'on Cl\'asica]
Un proceso estoc\'astico $X=\left\{X\left(t\right):t\geq0\right\}$ es llamado regenerativo is existe una variable aleatoria $R_{1}>0$ tal que
\begin{itemize}
\item[i)] $\left\{X\left(t+R_{1}\right):t\geq0\right\}$ es independiente de $\left\{\left\{X\left(t\right):t<R_{1}\right\},\right\}$
\item[ii)] $\left\{X\left(t+R_{1}\right):t\geq0\right\}$ es estoc\'asticamente equivalente a $\left\{X\left(t\right):t>0\right\}$
\end{itemize}

Llamamos a $R_{1}$ tiempo de regeneraci\'on, y decimos que $X$ se regenera en este punto.
\end{Def}

$\left\{X\left(t+R_{1}\right)\right\}$ es regenerativo con tiempo de regeneraci\'on $R_{2}$, independiente de $R_{1}$ pero con la misma distribuci\'on que $R_{1}$. Procediendo de esta manera se obtiene una secuencia de variables aleatorias independientes e id\'enticamente distribuidas $\left\{R_{n}\right\}$ llamados longitudes de ciclo. Si definimos a $Z_{k}\equiv R_{1}+R_{2}+\cdots+R_{k}$, se tiene un proceso de renovaci\'on llamado proceso de renovaci\'on encajado para $X$.

\begin{Note}
Un proceso regenerativo con media de la longitud de ciclo finita es llamado positivo recurrente.
\end{Note}


\begin{Def}
Para $x$ fijo y para cada $t\geq0$, sea $I_{x}\left(t\right)=1$ si $X\left(t\right)\leq x$,  $I_{x}\left(t\right)=0$ en caso contrario, y def\'inanse los tiempos promedio
\begin{eqnarray*}
\overline{X}&=&lim_{t\rightarrow\infty}\frac{1}{t}\int_{0}^{\infty}X\left(u\right)du\\
\prob\left(X_{\infty}\leq x\right)&=&lim_{t\rightarrow\infty}\frac{1}{t}\int_{0}^{\infty}I_{x}\left(u\right)du,
\end{eqnarray*}
cuando estos l\'imites existan.
\end{Def}

Como consecuencia del teorema de Renovaci\'on-Recompensa, se tiene que el primer l\'imite  existe y es igual a la constante
\begin{eqnarray*}
\overline{X}&=&\frac{\esp\left[\int_{0}^{R_{1}}X\left(t\right)dt\right]}{\esp\left[R_{1}\right]},
\end{eqnarray*}
suponiendo que ambas esperanzas son finitas.

\begin{Note}
\begin{itemize}
\item[a)] Si el proceso regenerativo $X$ es positivo recurrente y tiene trayectorias muestrales no negativas, entonces la ecuaci\'on anterior es v\'alida.
\item[b)] Si $X$ es positivo recurrente regenerativo, podemos construir una \'unica versi\'on estacionaria de este proceso, $X_{e}=\left\{X_{e}\left(t\right)\right\}$, donde $X_{e}$ es un proceso estoc\'astico regenerativo y estrictamente estacionario, con distribuci\'on marginal distribuida como $X_{\infty}$
\end{itemize}
\end{Note}

\subsection{Renewal and Regenerative Processes: Serfozo\cite{Serfozo}}
\begin{Def}\label{Def.Tn}
Sean $0\leq T_{1}\leq T_{2}\leq \ldots$ son tiempos aleatorios infinitos en los cuales ocurren ciertos eventos. El n\'umero de tiempos $T_{n}$ en el intervalo $\left[0,t\right)$ es

\begin{eqnarray}
N\left(t\right)=\sum_{n=1}^{\infty}\indora\left(T_{n}\leq t\right),
\end{eqnarray}
para $t\geq0$.
\end{Def}

Si se consideran los puntos $T_{n}$ como elementos de $\rea_{+}$, y $N\left(t\right)$ es el n\'umero de puntos en $\rea$. El proceso denotado por $\left\{N\left(t\right):t\geq0\right\}$, denotado por $N\left(t\right)$, es un proceso puntual en $\rea_{+}$. Los $T_{n}$ son los tiempos de ocurrencia, el proceso puntual $N\left(t\right)$ es simple si su n\'umero de ocurrencias son distintas: $0<T_{1}<T_{2}<\ldots$ casi seguramente.

\begin{Def}
Un proceso puntual $N\left(t\right)$ es un proceso de renovaci\'on si los tiempos de interocurrencia $\xi_{n}=T_{n}-T_{n-1}$, para $n\geq1$, son independientes e identicamente distribuidos con distribuci\'on $F$, donde $F\left(0\right)=0$ y $T_{0}=0$. Los $T_{n}$ son llamados tiempos de renovaci\'on, referente a la independencia o renovaci\'on de la informaci\'on estoc\'astica en estos tiempos. Los $\xi_{n}$ son los tiempos de inter-renovaci\'on, y $N\left(t\right)$ es el n\'umero de renovaciones en el intervalo $\left[0,t\right)$
\end{Def}


\begin{Note}
Para definir un proceso de renovaci\'on para cualquier contexto, solamente hay que especificar una distribuci\'on $F$, con $F\left(0\right)=0$, para los tiempos de inter-renovaci\'on. La funci\'on $F$ en turno degune las otra variables aleatorias. De manera formal, existe un espacio de probabilidad y una sucesi\'on de variables aleatorias $\xi_{1},\xi_{2},\ldots$ definidas en este con distribuci\'on $F$. Entonces las otras cantidades son $T_{n}=\sum_{k=1}^{n}\xi_{k}$ y $N\left(t\right)=\sum_{n=1}^{\infty}\indora\left(T_{n}\leq t\right)$, donde $T_{n}\rightarrow\infty$ casi seguramente por la Ley Fuerte de los Grandes N\'umeros.
\end{Note}







Los tiempos $T_{n}$ est\'an relacionados con los conteos de $N\left(t\right)$ por

\begin{eqnarray*}
\left\{N\left(t\right)\geq n\right\}&=&\left\{T_{n}\leq t\right\}\\
T_{N\left(t\right)}\leq &t&<T_{N\left(t\right)+1},
\end{eqnarray*}

adem\'as $N\left(T_{n}\right)=n$, y 

\begin{eqnarray*}
N\left(t\right)=\max\left\{n:T_{n}\leq t\right\}=\min\left\{n:T_{n+1}>t\right\}
\end{eqnarray*}

Por propiedades de la convoluci\'on se sabe que

\begin{eqnarray*}
P\left\{T_{n}\leq t\right\}=F^{n\star}\left(t\right)
\end{eqnarray*}
que es la $n$-\'esima convoluci\'on de $F$. Entonces 

\begin{eqnarray*}
\left\{N\left(t\right)\geq n\right\}&=&\left\{T_{n}\leq t\right\}\\
P\left\{N\left(t\right)\leq n\right\}&=&1-F^{\left(n+1\right)\star}\left(t\right)
\end{eqnarray*}

Adem\'as usando el hecho de que $\esp\left[N\left(t\right)\right]=\sum_{n=1}^{\infty}P\left\{N\left(t\right)\geq n\right\}$
se tiene que

\begin{eqnarray*}
\esp\left[N\left(t\right)\right]=\sum_{n=1}^{\infty}F^{n\star}\left(t\right)
\end{eqnarray*}

\begin{Prop}
Para cada $t\geq0$, la funci\'on generadora de momentos $\esp\left[e^{\alpha N\left(t\right)}\right]$ existe para alguna $\alpha$ en una vecindad del 0, y de aqu\'i que $\esp\left[N\left(t\right)^{m}\right]<\infty$, para $m\geq1$.
\end{Prop}


\begin{Note}
Si el primer tiempo de renovaci\'on $\xi_{1}$ no tiene la misma distribuci\'on que el resto de las $\xi_{n}$, para $n\geq2$, a $N\left(t\right)$ se le llama Proceso de Renovaci\'on retardado, donde si $\xi$ tiene distribuci\'on $G$, entonces el tiempo $T_{n}$ de la $n$-\'esima renovaci\'on tiene distribuci\'on $G\star F^{\left(n-1\right)\star}\left(t\right)$
\end{Note}


\begin{Teo}
Para una constante $\mu\leq\infty$ ( o variable aleatoria), las siguientes expresiones son equivalentes:

\begin{eqnarray}
lim_{n\rightarrow\infty}n^{-1}T_{n}&=&\mu,\textrm{ c.s.}\\
lim_{t\rightarrow\infty}t^{-1}N\left(t\right)&=&1/\mu,\textrm{ c.s.}
\end{eqnarray}
\end{Teo}


Es decir, $T_{n}$ satisface la Ley Fuerte de los Grandes N\'umeros s\'i y s\'olo s\'i $N\left/t\right)$ la cumple.


\begin{Coro}[Ley Fuerte de los Grandes N\'umeros para Procesos de Renovaci\'on]
Si $N\left(t\right)$ es un proceso de renovaci\'on cuyos tiempos de inter-renovaci\'on tienen media $\mu\leq\infty$, entonces
\begin{eqnarray}
t^{-1}N\left(t\right)\rightarrow 1/\mu,\textrm{ c.s. cuando }t\rightarrow\infty.
\end{eqnarray}

\end{Coro}


Considerar el proceso estoc\'astico de valores reales $\left\{Z\left(t\right):t\geq0\right\}$ en el mismo espacio de probabilidad que $N\left(t\right)$

\begin{Def}
Para el proceso $\left\{Z\left(t\right):t\geq0\right\}$ se define la fluctuaci\'on m\'axima de $Z\left(t\right)$ en el intervalo $\left(T_{n-1},T_{n}\right]$:
\begin{eqnarray*}
M_{n}=\sup_{T_{n-1}<t\leq T_{n}}|Z\left(t\right)-Z\left(T_{n-1}\right)|
\end{eqnarray*}
\end{Def}

\begin{Teo}
Sup\'ongase que $n^{-1}T_{n}\rightarrow\mu$ c.s. cuando $n\rightarrow\infty$, donde $\mu\leq\infty$ es una constante o variable aleatoria. Sea $a$ una constante o variable aleatoria que puede ser infinita cuando $\mu$ es finita, y considere las expresiones l\'imite:
\begin{eqnarray}
lim_{n\rightarrow\infty}n^{-1}Z\left(T_{n}\right)&=&a,\textrm{ c.s.}\\
lim_{t\rightarrow\infty}t^{-1}Z\left(t\right)&=&a/\mu,\textrm{ c.s.}
\end{eqnarray}
La segunda expresi\'on implica la primera. Conversamente, la primera implica la segunda si el proceso $Z\left(t\right)$ es creciente, o si $lim_{n\rightarrow\infty}n^{-1}M_{n}=0$ c.s.
\end{Teo}

\begin{Coro}
Si $N\left(t\right)$ es un proceso de renovaci\'on, y $\left(Z\left(T_{n}\right)-Z\left(T_{n-1}\right),M_{n}\right)$, para $n\geq1$, son variables aleatorias independientes e id\'enticamente distribuidas con media finita, entonces,
\begin{eqnarray}
lim_{t\rightarrow\infty}t^{-1}Z\left(t\right)\rightarrow\frac{\esp\left[Z\left(T_{1}\right)-Z\left(T_{0}\right)\right]}{\esp\left[T_{1}\right]},\textrm{ c.s. cuando  }t\rightarrow\infty.
\end{eqnarray}
\end{Coro}


Sup\'ongase que $N\left(t\right)$ es un proceso de renovaci\'on con distribuci\'on $F$ con media finita $\mu$.

\begin{Def}
La funci\'on de renovaci\'on asociada con la distribuci\'on $F$, del proceso $N\left(t\right)$, es
\begin{eqnarray*}
U\left(t\right)=\sum_{n=1}^{\infty}F^{n\star}\left(t\right),\textrm{   }t\geq0,
\end{eqnarray*}
donde $F^{0\star}\left(t\right)=\indora\left(t\geq0\right)$.
\end{Def}


\begin{Prop}
Sup\'ongase que la distribuci\'on de inter-renovaci\'on $F$ tiene densidad $f$. Entonces $U\left(t\right)$ tambi\'en tiene densidad, para $t>0$, y es $U^{'}\left(t\right)=\sum_{n=0}^{\infty}f^{n\star}\left(t\right)$. Adem\'as
\begin{eqnarray*}
\prob\left\{N\left(t\right)>N\left(t-\right)\right\}=0\textrm{,   }t\geq0.
\end{eqnarray*}
\end{Prop}

\begin{Def}
La Transformada de Laplace-Stieljes de $F$ est\'a dada por

\begin{eqnarray*}
\hat{F}\left(\alpha\right)=\int_{\rea_{+}}e^{-\alpha t}dF\left(t\right)\textrm{,  }\alpha\geq0.
\end{eqnarray*}
\end{Def}

Entonces

\begin{eqnarray*}
\hat{U}\left(\alpha\right)=\sum_{n=0}^{\infty}\hat{F^{n\star}}\left(\alpha\right)=\sum_{n=0}^{\infty}\hat{F}\left(\alpha\right)^{n}=\frac{1}{1-\hat{F}\left(\alpha\right)}.
\end{eqnarray*}


\begin{Prop}
La Transformada de Laplace $\hat{U}\left(\alpha\right)$ y $\hat{F}\left(\alpha\right)$ determina una a la otra de manera \'unica por la relaci\'on $\hat{U}\left(\alpha\right)=\frac{1}{1-\hat{F}\left(\alpha\right)}$.
\end{Prop}


\begin{Note}
Un proceso de renovaci\'on $N\left(t\right)$ cuyos tiempos de inter-renovaci\'on tienen media finita, es un proceso Poisson con tasa $\lambda$ si y s\'olo s\'i $\esp\left[U\left(t\right)\right]=\lambda t$, para $t\geq0$.
\end{Note}


\begin{Teo}
Sea $N\left(t\right)$ un proceso puntual simple con puntos de localizaci\'on $T_{n}$ tal que $\eta\left(t\right)=\esp\left[N\left(\right)\right]$ es finita para cada $t$. Entonces para cualquier funci\'on $f:\rea_{+}\rightarrow\rea$,
\begin{eqnarray*}
\esp\left[\sum_{n=1}^{N\left(\right)}f\left(T_{n}\right)\right]=\int_{\left(0,t\right]}f\left(s\right)d\eta\left(s\right)\textrm{,  }t\geq0,
\end{eqnarray*}
suponiendo que la integral exista. Adem\'as si $X_{1},X_{2},\ldots$ son variables aleatorias definidas en el mismo espacio de probabilidad que el proceso $N\left(t\right)$ tal que $\esp\left[X_{n}|T_{n}=s\right]=f\left(s\right)$, independiente de $n$. Entonces
\begin{eqnarray*}
\esp\left[\sum_{n=1}^{N\left(t\right)}X_{n}\right]=\int_{\left(0,t\right]}f\left(s\right)d\eta\left(s\right)\textrm{,  }t\geq0,
\end{eqnarray*} 
suponiendo que la integral exista. 
\end{Teo}

\begin{Coro}[Identidad de Wald para Renovaciones]
Para el proceso de renovaci\'on $N\left(t\right)$,
\begin{eqnarray*}
\esp\left[T_{N\left(t\right)+1}\right]=\mu\esp\left[N\left(t\right)+1\right]\textrm{,  }t\geq0,
\end{eqnarray*}  
\end{Coro}


\begin{Def}
Sea $h\left(t\right)$ funci\'on de valores reales en $\rea$ acotada en intervalos finitos e igual a cero para $t<0$ La ecuaci\'on de renovaci\'on para $h\left(t\right)$ y la distribuci\'on $F$ es

\begin{eqnarray}\label{Ec.Renovacion}
H\left(t\right)=h\left(t\right)+\int_{\left[0,t\right]}H\left(t-s\right)dF\left(s\right)\textrm{,    }t\geq0,
\end{eqnarray}
donde $H\left(t\right)$ es una funci\'on de valores reales. Esto es $H=h+F\star H$. Decimos que $H\left(t\right)$ es soluci\'on de esta ecuaci\'on si satisface la ecuaci\'on, y es acotada en intervalos finitos e iguales a cero para $t<0$.
\end{Def}

\begin{Prop}
La funci\'on $U\star h\left(t\right)$ es la \'unica soluci\'on de la ecuaci\'on de renovaci\'on (\ref{Ec.Renovacion}).
\end{Prop}

\begin{Teo}[Teorema Renovaci\'on Elemental]
\begin{eqnarray*}
t^{-1}U\left(t\right)\rightarrow 1/\mu\textrm{,    cuando }t\rightarrow\infty.
\end{eqnarray*}
\end{Teo}



Sup\'ongase que $N\left(t\right)$ es un proceso de renovaci\'on con distribuci\'on $F$ con media finita $\mu$.

\begin{Def}
La funci\'on de renovaci\'on asociada con la distribuci\'on $F$, del proceso $N\left(t\right)$, es
\begin{eqnarray*}
U\left(t\right)=\sum_{n=1}^{\infty}F^{n\star}\left(t\right),\textrm{   }t\geq0,
\end{eqnarray*}
donde $F^{0\star}\left(t\right)=\indora\left(t\geq0\right)$.
\end{Def}


\begin{Prop}
Sup\'ongase que la distribuci\'on de inter-renovaci\'on $F$ tiene densidad $f$. Entonces $U\left(t\right)$ tambi\'en tiene densidad, para $t>0$, y es $U^{'}\left(t\right)=\sum_{n=0}^{\infty}f^{n\star}\left(t\right)$. Adem\'as
\begin{eqnarray*}
\prob\left\{N\left(t\right)>N\left(t-\right)\right\}=0\textrm{,   }t\geq0.
\end{eqnarray*}
\end{Prop}

\begin{Def}
La Transformada de Laplace-Stieljes de $F$ est\'a dada por

\begin{eqnarray*}
\hat{F}\left(\alpha\right)=\int_{\rea_{+}}e^{-\alpha t}dF\left(t\right)\textrm{,  }\alpha\geq0.
\end{eqnarray*}
\end{Def}

Entonces

\begin{eqnarray*}
\hat{U}\left(\alpha\right)=\sum_{n=0}^{\infty}\hat{F^{n\star}}\left(\alpha\right)=\sum_{n=0}^{\infty}\hat{F}\left(\alpha\right)^{n}=\frac{1}{1-\hat{F}\left(\alpha\right)}.
\end{eqnarray*}


\begin{Prop}
La Transformada de Laplace $\hat{U}\left(\alpha\right)$ y $\hat{F}\left(\alpha\right)$ determina una a la otra de manera \'unica por la relaci\'on $\hat{U}\left(\alpha\right)=\frac{1}{1-\hat{F}\left(\alpha\right)}$.
\end{Prop}


\begin{Note}
Un proceso de renovaci\'on $N\left(t\right)$ cuyos tiempos de inter-renovaci\'on tienen media finita, es un proceso Poisson con tasa $\lambda$ si y s\'olo s\'i $\esp\left[U\left(t\right)\right]=\lambda t$, para $t\geq0$.
\end{Note}


\begin{Teo}
Sea $N\left(t\right)$ un proceso puntual simple con puntos de localizaci\'on $T_{n}$ tal que $\eta\left(t\right)=\esp\left[N\left(\right)\right]$ es finita para cada $t$. Entonces para cualquier funci\'on $f:\rea_{+}\rightarrow\rea$,
\begin{eqnarray*}
\esp\left[\sum_{n=1}^{N\left(\right)}f\left(T_{n}\right)\right]=\int_{\left(0,t\right]}f\left(s\right)d\eta\left(s\right)\textrm{,  }t\geq0,
\end{eqnarray*}
suponiendo que la integral exista. Adem\'as si $X_{1},X_{2},\ldots$ son variables aleatorias definidas en el mismo espacio de probabilidad que el proceso $N\left(t\right)$ tal que $\esp\left[X_{n}|T_{n}=s\right]=f\left(s\right)$, independiente de $n$. Entonces
\begin{eqnarray*}
\esp\left[\sum_{n=1}^{N\left(t\right)}X_{n}\right]=\int_{\left(0,t\right]}f\left(s\right)d\eta\left(s\right)\textrm{,  }t\geq0,
\end{eqnarray*} 
suponiendo que la integral exista. 
\end{Teo}

\begin{Coro}[Identidad de Wald para Renovaciones]
Para el proceso de renovaci\'on $N\left(t\right)$,
\begin{eqnarray*}
\esp\left[T_{N\left(t\right)+1}\right]=\mu\esp\left[N\left(t\right)+1\right]\textrm{,  }t\geq0,
\end{eqnarray*}  
\end{Coro}


\begin{Def}
Sea $h\left(t\right)$ funci\'on de valores reales en $\rea$ acotada en intervalos finitos e igual a cero para $t<0$ La ecuaci\'on de renovaci\'on para $h\left(t\right)$ y la distribuci\'on $F$ es

\begin{eqnarray}\label{Ec.Renovacion}
H\left(t\right)=h\left(t\right)+\int_{\left[0,t\right]}H\left(t-s\right)dF\left(s\right)\textrm{,    }t\geq0,
\end{eqnarray}
donde $H\left(t\right)$ es una funci\'on de valores reales. Esto es $H=h+F\star H$. Decimos que $H\left(t\right)$ es soluci\'on de esta ecuaci\'on si satisface la ecuaci\'on, y es acotada en intervalos finitos e iguales a cero para $t<0$.
\end{Def}

\begin{Prop}
La funci\'on $U\star h\left(t\right)$ es la \'unica soluci\'on de la ecuaci\'on de renovaci\'on (\ref{Ec.Renovacion}).
\end{Prop}

\begin{Teo}[Teorema Renovaci\'on Elemental]
\begin{eqnarray*}
t^{-1}U\left(t\right)\rightarrow 1/\mu\textrm{,    cuando }t\rightarrow\infty.
\end{eqnarray*}
\end{Teo}


\begin{Note} Una funci\'on $h:\rea_{+}\rightarrow\rea$ es Directamente Riemann Integrable en los siguientes casos:
\begin{itemize}
\item[a)] $h\left(t\right)\geq0$ es decreciente y Riemann Integrable.
\item[b)] $h$ es continua excepto posiblemente en un conjunto de Lebesgue de medida 0, y $|h\left(t\right)|\leq b\left(t\right)$, donde $b$ es DRI.
\end{itemize}
\end{Note}

\begin{Teo}[Teorema Principal de Renovaci\'on]
Si $F$ es no aritm\'etica y $h\left(t\right)$ es Directamente Riemann Integrable (DRI), entonces

\begin{eqnarray*}
lim_{t\rightarrow\infty}U\star h=\frac{1}{\mu}\int_{\rea_{+}}h\left(s\right)ds.
\end{eqnarray*}
\end{Teo}

\begin{Prop}
Cualquier funci\'on $H\left(t\right)$ acotada en intervalos finitos y que es 0 para $t<0$ puede expresarse como
\begin{eqnarray*}
H\left(t\right)=U\star h\left(t\right)\textrm{,  donde }h\left(t\right)=H\left(t\right)-F\star H\left(t\right)
\end{eqnarray*}
\end{Prop}

\begin{Def}
Un proceso estoc\'astico $X\left(t\right)$ es crudamente regenerativo en un tiempo aleatorio positivo $T$ si
\begin{eqnarray*}
\esp\left[X\left(T+t\right)|T\right]=\esp\left[X\left(t\right)\right]\textrm{, para }t\geq0,\end{eqnarray*}
y con las esperanzas anteriores finitas.
\end{Def}

\begin{Prop}
Sup\'ongase que $X\left(t\right)$ es un proceso crudamente regenerativo en $T$, que tiene distribuci\'on $F$. Si $\esp\left[X\left(t\right)\right]$ es acotado en intervalos finitos, entonces
\begin{eqnarray*}
\esp\left[X\left(t\right)\right]=U\star h\left(t\right)\textrm{,  donde }h\left(t\right)=\esp\left[X\left(t\right)\indora\left(T>t\right)\right].
\end{eqnarray*}
\end{Prop}

\begin{Teo}[Regeneraci\'on Cruda]
Sup\'ongase que $X\left(t\right)$ es un proceso con valores positivo crudamente regenerativo en $T$, y def\'inase $M=\sup\left\{|X\left(t\right)|:t\leq T\right\}$. Si $T$ es no aritm\'etico y $M$ y $MT$ tienen media finita, entonces
\begin{eqnarray*}
lim_{t\rightarrow\infty}\esp\left[X\left(t\right)\right]=\frac{1}{\mu}\int_{\rea_{+}}h\left(s\right)ds,
\end{eqnarray*}
donde $h\left(t\right)=\esp\left[X\left(t\right)\indora\left(T>t\right)\right]$.
\end{Teo}


\begin{Note} Una funci\'on $h:\rea_{+}\rightarrow\rea$ es Directamente Riemann Integrable en los siguientes casos:
\begin{itemize}
\item[a)] $h\left(t\right)\geq0$ es decreciente y Riemann Integrable.
\item[b)] $h$ es continua excepto posiblemente en un conjunto de Lebesgue de medida 0, y $|h\left(t\right)|\leq b\left(t\right)$, donde $b$ es DRI.
\end{itemize}
\end{Note}

\begin{Teo}[Teorema Principal de Renovaci\'on]
Si $F$ es no aritm\'etica y $h\left(t\right)$ es Directamente Riemann Integrable (DRI), entonces

\begin{eqnarray*}
lim_{t\rightarrow\infty}U\star h=\frac{1}{\mu}\int_{\rea_{+}}h\left(s\right)ds.
\end{eqnarray*}
\end{Teo}

\begin{Prop}
Cualquier funci\'on $H\left(t\right)$ acotada en intervalos finitos y que es 0 para $t<0$ puede expresarse como
\begin{eqnarray*}
H\left(t\right)=U\star h\left(t\right)\textrm{,  donde }h\left(t\right)=H\left(t\right)-F\star H\left(t\right)
\end{eqnarray*}
\end{Prop}

\begin{Def}
Un proceso estoc\'astico $X\left(t\right)$ es crudamente regenerativo en un tiempo aleatorio positivo $T$ si
\begin{eqnarray*}
\esp\left[X\left(T+t\right)|T\right]=\esp\left[X\left(t\right)\right]\textrm{, para }t\geq0,\end{eqnarray*}
y con las esperanzas anteriores finitas.
\end{Def}

\begin{Prop}
Sup\'ongase que $X\left(t\right)$ es un proceso crudamente regenerativo en $T$, que tiene distribuci\'on $F$. Si $\esp\left[X\left(t\right)\right]$ es acotado en intervalos finitos, entonces
\begin{eqnarray*}
\esp\left[X\left(t\right)\right]=U\star h\left(t\right)\textrm{,  donde }h\left(t\right)=\esp\left[X\left(t\right)\indora\left(T>t\right)\right].
\end{eqnarray*}
\end{Prop}

\begin{Teo}[Regeneraci\'on Cruda]
Sup\'ongase que $X\left(t\right)$ es un proceso con valores positivo crudamente regenerativo en $T$, y def\'inase $M=\sup\left\{|X\left(t\right)|:t\leq T\right\}$. Si $T$ es no aritm\'etico y $M$ y $MT$ tienen media finita, entonces
\begin{eqnarray*}
lim_{t\rightarrow\infty}\esp\left[X\left(t\right)\right]=\frac{1}{\mu}\int_{\rea_{+}}h\left(s\right)ds,
\end{eqnarray*}
donde $h\left(t\right)=\esp\left[X\left(t\right)\indora\left(T>t\right)\right]$.
\end{Teo}

%________________________________________________________________________
\subsection{Procesos Regenerativos}
%________________________________________________________________________

Para $\left\{X\left(t\right):t\geq0\right\}$ Proceso Estoc\'astico a tiempo continuo con estado de espacios $S$, que es un espacio m\'etrico, con trayectorias continuas por la derecha y con l\'imites por la izquierda c.s. Sea $N\left(t\right)$ un proceso de renovaci\'on en $\rea_{+}$ definido en el mismo espacio de probabilidad que $X\left(t\right)$, con tiempos de renovaci\'on $T$ y tiempos de inter-renovaci\'on $\xi_{n}=T_{n}-T_{n-1}$, con misma distribuci\'on $F$ de media finita $\mu$.



\begin{Def}
Para el proceso $\left\{\left(N\left(t\right),X\left(t\right)\right):t\geq0\right\}$, sus trayectoria muestrales en el intervalo de tiempo $\left[T_{n-1},T_{n}\right)$ est\'an descritas por
\begin{eqnarray*}
\zeta_{n}=\left(\xi_{n},\left\{X\left(T_{n-1}+t\right):0\leq t<\xi_{n}\right\}\right)
\end{eqnarray*}
Este $\zeta_{n}$ es el $n$-\'esimo segmento del proceso. El proceso es regenerativo sobre los tiempos $T_{n}$ si sus segmentos $\zeta_{n}$ son independientes e id\'enticamennte distribuidos.
\end{Def}


\begin{Obs}
Si $\tilde{X}\left(t\right)$ con espacio de estados $\tilde{S}$ es regenerativo sobre $T_{n}$, entonces $X\left(t\right)=f\left(\tilde{X}\left(t\right)\right)$ tambi\'en es regenerativo sobre $T_{n}$, para cualquier funci\'on $f:\tilde{S}\rightarrow S$.
\end{Obs}

\begin{Obs}
Los procesos regenerativos son crudamente regenerativos, pero no al rev\'es.
\end{Obs}

\begin{Def}[Definici\'on Cl\'asica]
Un proceso estoc\'astico $X=\left\{X\left(t\right):t\geq0\right\}$ es llamado regenerativo is existe una variable aleatoria $R_{1}>0$ tal que
\begin{itemize}
\item[i)] $\left\{X\left(t+R_{1}\right):t\geq0\right\}$ es independiente de $\left\{\left\{X\left(t\right):t<R_{1}\right\},\right\}$
\item[ii)] $\left\{X\left(t+R_{1}\right):t\geq0\right\}$ es estoc\'asticamente equivalente a $\left\{X\left(t\right):t>0\right\}$
\end{itemize}

Llamamos a $R_{1}$ tiempo de regeneraci\'on, y decimos que $X$ se regenera en este punto.
\end{Def}

$\left\{X\left(t+R_{1}\right)\right\}$ es regenerativo con tiempo de regeneraci\'on $R_{2}$, independiente de $R_{1}$ pero con la misma distribuci\'on que $R_{1}$. Procediendo de esta manera se obtiene una secuencia de variables aleatorias independientes e id\'enticamente distribuidas $\left\{R_{n}\right\}$ llamados longitudes de ciclo. Si definimos a $Z_{k}\equiv R_{1}+R_{2}+\cdots+R_{k}$, se tiene un proceso de renovaci\'on llamado proceso de renovaci\'on encajado para $X$.

\begin{Note}
Un proceso regenerativo con media de la longitud de ciclo finita es llamado positivo recurrente.
\end{Note}


\begin{Def}
Para $x$ fijo y para cada $t\geq0$, sea $I_{x}\left(t\right)=1$ si $X\left(t\right)\leq x$,  $I_{x}\left(t\right)=0$ en caso contrario, y def\'inanse los tiempos promedio
\begin{eqnarray*}
\overline{X}&=&lim_{t\rightarrow\infty}\frac{1}{t}\int_{0}^{\infty}X\left(u\right)du\\
\prob\left(X_{\infty}\leq x\right)&=&lim_{t\rightarrow\infty}\frac{1}{t}\int_{0}^{\infty}I_{x}\left(u\right)du,
\end{eqnarray*}
cuando estos l\'imites existan.
\end{Def}

Como consecuencia del teorema de Renovaci\'on-Recompensa, se tiene que el primer l\'imite  existe y es igual a la constante
\begin{eqnarray*}
\overline{X}&=&\frac{\esp\left[\int_{0}^{R_{1}}X\left(t\right)dt\right]}{\esp\left[R_{1}\right]},
\end{eqnarray*}
suponiendo que ambas esperanzas son finitas.

\begin{Note}
\begin{itemize}
\item[a)] Si el proceso regenerativo $X$ es positivo recurrente y tiene trayectorias muestrales no negativas, entonces la ecuaci\'on anterior es v\'alida.
\item[b)] Si $X$ es positivo recurrente regenerativo, podemos construir una \'unica versi\'on estacionaria de este proceso, $X_{e}=\left\{X_{e}\left(t\right)\right\}$, donde $X_{e}$ es un proceso estoc\'astico regenerativo y estrictamente estacionario, con distribuci\'on marginal distribuida como $X_{\infty}$
\end{itemize}
\end{Note}

%________________________________________________________________________
\subsection{Procesos Regenerativos}
%________________________________________________________________________

Para $\left\{X\left(t\right):t\geq0\right\}$ Proceso Estoc\'astico a tiempo continuo con estado de espacios $S$, que es un espacio m\'etrico, con trayectorias continuas por la derecha y con l\'imites por la izquierda c.s. Sea $N\left(t\right)$ un proceso de renovaci\'on en $\rea_{+}$ definido en el mismo espacio de probabilidad que $X\left(t\right)$, con tiempos de renovaci\'on $T$ y tiempos de inter-renovaci\'on $\xi_{n}=T_{n}-T_{n-1}$, con misma distribuci\'on $F$ de media finita $\mu$.



\begin{Def}
Para el proceso $\left\{\left(N\left(t\right),X\left(t\right)\right):t\geq0\right\}$, sus trayectoria muestrales en el intervalo de tiempo $\left[T_{n-1},T_{n}\right)$ est\'an descritas por
\begin{eqnarray*}
\zeta_{n}=\left(\xi_{n},\left\{X\left(T_{n-1}+t\right):0\leq t<\xi_{n}\right\}\right)
\end{eqnarray*}
Este $\zeta_{n}$ es el $n$-\'esimo segmento del proceso. El proceso es regenerativo sobre los tiempos $T_{n}$ si sus segmentos $\zeta_{n}$ son independientes e id\'enticamennte distribuidos.
\end{Def}


\begin{Obs}
Si $\tilde{X}\left(t\right)$ con espacio de estados $\tilde{S}$ es regenerativo sobre $T_{n}$, entonces $X\left(t\right)=f\left(\tilde{X}\left(t\right)\right)$ tambi\'en es regenerativo sobre $T_{n}$, para cualquier funci\'on $f:\tilde{S}\rightarrow S$.
\end{Obs}

\begin{Obs}
Los procesos regenerativos son crudamente regenerativos, pero no al rev\'es.
\end{Obs}

\begin{Def}[Definici\'on Cl\'asica]
Un proceso estoc\'astico $X=\left\{X\left(t\right):t\geq0\right\}$ es llamado regenerativo is existe una variable aleatoria $R_{1}>0$ tal que
\begin{itemize}
\item[i)] $\left\{X\left(t+R_{1}\right):t\geq0\right\}$ es independiente de $\left\{\left\{X\left(t\right):t<R_{1}\right\},\right\}$
\item[ii)] $\left\{X\left(t+R_{1}\right):t\geq0\right\}$ es estoc\'asticamente equivalente a $\left\{X\left(t\right):t>0\right\}$
\end{itemize}

Llamamos a $R_{1}$ tiempo de regeneraci\'on, y decimos que $X$ se regenera en este punto.
\end{Def}

$\left\{X\left(t+R_{1}\right)\right\}$ es regenerativo con tiempo de regeneraci\'on $R_{2}$, independiente de $R_{1}$ pero con la misma distribuci\'on que $R_{1}$. Procediendo de esta manera se obtiene una secuencia de variables aleatorias independientes e id\'enticamente distribuidas $\left\{R_{n}\right\}$ llamados longitudes de ciclo. Si definimos a $Z_{k}\equiv R_{1}+R_{2}+\cdots+R_{k}$, se tiene un proceso de renovaci\'on llamado proceso de renovaci\'on encajado para $X$.

\begin{Note}
Un proceso regenerativo con media de la longitud de ciclo finita es llamado positivo recurrente.
\end{Note}


\begin{Def}
Para $x$ fijo y para cada $t\geq0$, sea $I_{x}\left(t\right)=1$ si $X\left(t\right)\leq x$,  $I_{x}\left(t\right)=0$ en caso contrario, y def\'inanse los tiempos promedio
\begin{eqnarray*}
\overline{X}&=&lim_{t\rightarrow\infty}\frac{1}{t}\int_{0}^{\infty}X\left(u\right)du\\
\prob\left(X_{\infty}\leq x\right)&=&lim_{t\rightarrow\infty}\frac{1}{t}\int_{0}^{\infty}I_{x}\left(u\right)du,
\end{eqnarray*}
cuando estos l\'imites existan.
\end{Def}

Como consecuencia del teorema de Renovaci\'on-Recompensa, se tiene que el primer l\'imite  existe y es igual a la constante
\begin{eqnarray*}
\overline{X}&=&\frac{\esp\left[\int_{0}^{R_{1}}X\left(t\right)dt\right]}{\esp\left[R_{1}\right]},
\end{eqnarray*}
suponiendo que ambas esperanzas son finitas.

\begin{Note}
\begin{itemize}
\item[a)] Si el proceso regenerativo $X$ es positivo recurrente y tiene trayectorias muestrales no negativas, entonces la ecuaci\'on anterior es v\'alida.
\item[b)] Si $X$ es positivo recurrente regenerativo, podemos construir una \'unica versi\'on estacionaria de este proceso, $X_{e}=\left\{X_{e}\left(t\right)\right\}$, donde $X_{e}$ es un proceso estoc\'astico regenerativo y estrictamente estacionario, con distribuci\'on marginal distribuida como $X_{\infty}$
\end{itemize}
\end{Note}
%__________________________________________________________________________________________
\subsection{Procesos Regenerativos Estacionarios - Stidham \cite{Stidham}}
%__________________________________________________________________________________________


Un proceso estoc\'astico a tiempo continuo $\left\{V\left(t\right),t\geq0\right\}$ es un proceso regenerativo si existe una sucesi\'on de variables aleatorias independientes e id\'enticamente distribuidas $\left\{X_{1},X_{2},\ldots\right\}$, sucesi\'on de renovaci\'on, tal que para cualquier conjunto de Borel $A$, 

\begin{eqnarray*}
\prob\left\{V\left(t\right)\in A|X_{1}+X_{2}+\cdots+X_{R\left(t\right)}=s,\left\{V\left(\tau\right),\tau<s\right\}\right\}=\prob\left\{V\left(t-s\right)\in A|X_{1}>t-s\right\},
\end{eqnarray*}
para todo $0\leq s\leq t$, donde $R\left(t\right)=\max\left\{X_{1}+X_{2}+\cdots+X_{j}\leq t\right\}=$n\'umero de renovaciones ({\emph{puntos de regeneraci\'on}}) que ocurren en $\left[0,t\right]$. El intervalo $\left[0,X_{1}\right)$ es llamado {\emph{primer ciclo de regeneraci\'on}} de $\left\{V\left(t \right),t\geq0\right\}$, $\left[X_{1},X_{1}+X_{2}\right)$ el {\emph{segundo ciclo de regeneraci\'on}}, y as\'i sucesivamente.

Sea $X=X_{1}$ y sea $F$ la funci\'on de distrbuci\'on de $X$


\begin{Def}
Se define el proceso estacionario, $\left\{V^{*}\left(t\right),t\geq0\right\}$, para $\left\{V\left(t\right),t\geq0\right\}$ por

\begin{eqnarray*}
\prob\left\{V\left(t\right)\in A\right\}=\frac{1}{\esp\left[X\right]}\int_{0}^{\infty}\prob\left\{V\left(t+x\right)\in A|X>x\right\}\left(1-F\left(x\right)\right)dx,
\end{eqnarray*} 
para todo $t\geq0$ y todo conjunto de Borel $A$.
\end{Def}

\begin{Def}
Una distribuci\'on se dice que es {\emph{aritm\'etica}} si todos sus puntos de incremento son m\'ultiplos de la forma $0,\lambda, 2\lambda,\ldots$ para alguna $\lambda>0$ entera.
\end{Def}


\begin{Def}
Una modificaci\'on medible de un proceso $\left\{V\left(t\right),t\geq0\right\}$, es una versi\'on de este, $\left\{V\left(t,w\right)\right\}$ conjuntamente medible para $t\geq0$ y para $w\in S$, $S$ espacio de estados para $\left\{V\left(t\right),t\geq0\right\}$.
\end{Def}

\begin{Teo}
Sea $\left\{V\left(t\right),t\geq\right\}$ un proceso regenerativo no negativo con modificaci\'on medible. Sea $\esp\left[X\right]<\infty$. Entonces el proceso estacionario dado por la ecuaci\'on anterior est\'a bien definido y tiene funci\'on de distribuci\'on independiente de $t$, adem\'as
\begin{itemize}
\item[i)] \begin{eqnarray*}
\esp\left[V^{*}\left(0\right)\right]&=&\frac{\esp\left[\int_{0}^{X}V\left(s\right)ds\right]}{\esp\left[X\right]}\end{eqnarray*}
\item[ii)] Si $\esp\left[V^{*}\left(0\right)\right]<\infty$, equivalentemente, si $\esp\left[\int_{0}^{X}V\left(s\right)ds\right]<\infty$,entonces
\begin{eqnarray*}
\frac{\int_{0}^{t}V\left(s\right)ds}{t}\rightarrow\frac{\esp\left[\int_{0}^{X}V\left(s\right)ds\right]}{\esp\left[X\right]}
\end{eqnarray*}
con probabilidad 1 y en media, cuando $t\rightarrow\infty$.
\end{itemize}
\end{Teo}


%__________________________________________________________________________________________
\subsection{Procesos Regenerativos Estacionarios - Stidham \cite{Stidham}}
%__________________________________________________________________________________________


Un proceso estoc\'astico a tiempo continuo $\left\{V\left(t\right),t\geq0\right\}$ es un proceso regenerativo si existe una sucesi\'on de variables aleatorias independientes e id\'enticamente distribuidas $\left\{X_{1},X_{2},\ldots\right\}$, sucesi\'on de renovaci\'on, tal que para cualquier conjunto de Borel $A$, 

\begin{eqnarray*}
\prob\left\{V\left(t\right)\in A|X_{1}+X_{2}+\cdots+X_{R\left(t\right)}=s,\left\{V\left(\tau\right),\tau<s\right\}\right\}=\prob\left\{V\left(t-s\right)\in A|X_{1}>t-s\right\},
\end{eqnarray*}
para todo $0\leq s\leq t$, donde $R\left(t\right)=\max\left\{X_{1}+X_{2}+\cdots+X_{j}\leq t\right\}=$n\'umero de renovaciones ({\emph{puntos de regeneraci\'on}}) que ocurren en $\left[0,t\right]$. El intervalo $\left[0,X_{1}\right)$ es llamado {\emph{primer ciclo de regeneraci\'on}} de $\left\{V\left(t \right),t\geq0\right\}$, $\left[X_{1},X_{1}+X_{2}\right)$ el {\emph{segundo ciclo de regeneraci\'on}}, y as\'i sucesivamente.

Sea $X=X_{1}$ y sea $F$ la funci\'on de distrbuci\'on de $X$


\begin{Def}
Se define el proceso estacionario, $\left\{V^{*}\left(t\right),t\geq0\right\}$, para $\left\{V\left(t\right),t\geq0\right\}$ por

\begin{eqnarray*}
\prob\left\{V\left(t\right)\in A\right\}=\frac{1}{\esp\left[X\right]}\int_{0}^{\infty}\prob\left\{V\left(t+x\right)\in A|X>x\right\}\left(1-F\left(x\right)\right)dx,
\end{eqnarray*} 
para todo $t\geq0$ y todo conjunto de Borel $A$.
\end{Def}

\begin{Def}
Una distribuci\'on se dice que es {\emph{aritm\'etica}} si todos sus puntos de incremento son m\'ultiplos de la forma $0,\lambda, 2\lambda,\ldots$ para alguna $\lambda>0$ entera.
\end{Def}


\begin{Def}
Una modificaci\'on medible de un proceso $\left\{V\left(t\right),t\geq0\right\}$, es una versi\'on de este, $\left\{V\left(t,w\right)\right\}$ conjuntamente medible para $t\geq0$ y para $w\in S$, $S$ espacio de estados para $\left\{V\left(t\right),t\geq0\right\}$.
\end{Def}

\begin{Teo}
Sea $\left\{V\left(t\right),t\geq\right\}$ un proceso regenerativo no negativo con modificaci\'on medible. Sea $\esp\left[X\right]<\infty$. Entonces el proceso estacionario dado por la ecuaci\'on anterior est\'a bien definido y tiene funci\'on de distribuci\'on independiente de $t$, adem\'as
\begin{itemize}
\item[i)] \begin{eqnarray*}
\esp\left[V^{*}\left(0\right)\right]&=&\frac{\esp\left[\int_{0}^{X}V\left(s\right)ds\right]}{\esp\left[X\right]}\end{eqnarray*}
\item[ii)] Si $\esp\left[V^{*}\left(0\right)\right]<\infty$, equivalentemente, si $\esp\left[\int_{0}^{X}V\left(s\right)ds\right]<\infty$,entonces
\begin{eqnarray*}
\frac{\int_{0}^{t}V\left(s\right)ds}{t}\rightarrow\frac{\esp\left[\int_{0}^{X}V\left(s\right)ds\right]}{\esp\left[X\right]}
\end{eqnarray*}
con probabilidad 1 y en media, cuando $t\rightarrow\infty$.
\end{itemize}
\end{Teo}
%
%___________________________________________________________________________________________
%\vspace{5.5cm}
%\chapter{Cadenas de Markov estacionarias}
%\vspace{-1.0cm}
%___________________________________________________________________________________________
%
\subsection{Propiedades de los Procesos de Renovaci\'on}
%___________________________________________________________________________________________
%

Los tiempos $T_{n}$ est\'an relacionados con los conteos de $N\left(t\right)$ por

\begin{eqnarray*}
\left\{N\left(t\right)\geq n\right\}&=&\left\{T_{n}\leq t\right\}\\
T_{N\left(t\right)}\leq &t&<T_{N\left(t\right)+1},
\end{eqnarray*}

adem\'as $N\left(T_{n}\right)=n$, y 

\begin{eqnarray*}
N\left(t\right)=\max\left\{n:T_{n}\leq t\right\}=\min\left\{n:T_{n+1}>t\right\}
\end{eqnarray*}

Por propiedades de la convoluci\'on se sabe que

\begin{eqnarray*}
P\left\{T_{n}\leq t\right\}=F^{n\star}\left(t\right)
\end{eqnarray*}
que es la $n$-\'esima convoluci\'on de $F$. Entonces 

\begin{eqnarray*}
\left\{N\left(t\right)\geq n\right\}&=&\left\{T_{n}\leq t\right\}\\
P\left\{N\left(t\right)\leq n\right\}&=&1-F^{\left(n+1\right)\star}\left(t\right)
\end{eqnarray*}

Adem\'as usando el hecho de que $\esp\left[N\left(t\right)\right]=\sum_{n=1}^{\infty}P\left\{N\left(t\right)\geq n\right\}$
se tiene que

\begin{eqnarray*}
\esp\left[N\left(t\right)\right]=\sum_{n=1}^{\infty}F^{n\star}\left(t\right)
\end{eqnarray*}

\begin{Prop}
Para cada $t\geq0$, la funci\'on generadora de momentos $\esp\left[e^{\alpha N\left(t\right)}\right]$ existe para alguna $\alpha$ en una vecindad del 0, y de aqu\'i que $\esp\left[N\left(t\right)^{m}\right]<\infty$, para $m\geq1$.
\end{Prop}


\begin{Note}
Si el primer tiempo de renovaci\'on $\xi_{1}$ no tiene la misma distribuci\'on que el resto de las $\xi_{n}$, para $n\geq2$, a $N\left(t\right)$ se le llama Proceso de Renovaci\'on retardado, donde si $\xi$ tiene distribuci\'on $G$, entonces el tiempo $T_{n}$ de la $n$-\'esima renovaci\'on tiene distribuci\'on $G\star F^{\left(n-1\right)\star}\left(t\right)$
\end{Note}


\begin{Teo}
Para una constante $\mu\leq\infty$ ( o variable aleatoria), las siguientes expresiones son equivalentes:

\begin{eqnarray}
lim_{n\rightarrow\infty}n^{-1}T_{n}&=&\mu,\textrm{ c.s.}\\
lim_{t\rightarrow\infty}t^{-1}N\left(t\right)&=&1/\mu,\textrm{ c.s.}
\end{eqnarray}
\end{Teo}


Es decir, $T_{n}$ satisface la Ley Fuerte de los Grandes N\'umeros s\'i y s\'olo s\'i $N\left/t\right)$ la cumple.


\begin{Coro}[Ley Fuerte de los Grandes N\'umeros para Procesos de Renovaci\'on]
Si $N\left(t\right)$ es un proceso de renovaci\'on cuyos tiempos de inter-renovaci\'on tienen media $\mu\leq\infty$, entonces
\begin{eqnarray}
t^{-1}N\left(t\right)\rightarrow 1/\mu,\textrm{ c.s. cuando }t\rightarrow\infty.
\end{eqnarray}

\end{Coro}


Considerar el proceso estoc\'astico de valores reales $\left\{Z\left(t\right):t\geq0\right\}$ en el mismo espacio de probabilidad que $N\left(t\right)$

\begin{Def}
Para el proceso $\left\{Z\left(t\right):t\geq0\right\}$ se define la fluctuaci\'on m\'axima de $Z\left(t\right)$ en el intervalo $\left(T_{n-1},T_{n}\right]$:
\begin{eqnarray*}
M_{n}=\sup_{T_{n-1}<t\leq T_{n}}|Z\left(t\right)-Z\left(T_{n-1}\right)|
\end{eqnarray*}
\end{Def}

\begin{Teo}
Sup\'ongase que $n^{-1}T_{n}\rightarrow\mu$ c.s. cuando $n\rightarrow\infty$, donde $\mu\leq\infty$ es una constante o variable aleatoria. Sea $a$ una constante o variable aleatoria que puede ser infinita cuando $\mu$ es finita, y considere las expresiones l\'imite:
\begin{eqnarray}
lim_{n\rightarrow\infty}n^{-1}Z\left(T_{n}\right)&=&a,\textrm{ c.s.}\\
lim_{t\rightarrow\infty}t^{-1}Z\left(t\right)&=&a/\mu,\textrm{ c.s.}
\end{eqnarray}
La segunda expresi\'on implica la primera. Conversamente, la primera implica la segunda si el proceso $Z\left(t\right)$ es creciente, o si $lim_{n\rightarrow\infty}n^{-1}M_{n}=0$ c.s.
\end{Teo}

\begin{Coro}
Si $N\left(t\right)$ es un proceso de renovaci\'on, y $\left(Z\left(T_{n}\right)-Z\left(T_{n-1}\right),M_{n}\right)$, para $n\geq1$, son variables aleatorias independientes e id\'enticamente distribuidas con media finita, entonces,
\begin{eqnarray}
lim_{t\rightarrow\infty}t^{-1}Z\left(t\right)\rightarrow\frac{\esp\left[Z\left(T_{1}\right)-Z\left(T_{0}\right)\right]}{\esp\left[T_{1}\right]},\textrm{ c.s. cuando  }t\rightarrow\infty.
\end{eqnarray}
\end{Coro}


%___________________________________________________________________________________________
%
%\subsection{Propiedades de los Procesos de Renovaci\'on}
%___________________________________________________________________________________________
%

Los tiempos $T_{n}$ est\'an relacionados con los conteos de $N\left(t\right)$ por

\begin{eqnarray*}
\left\{N\left(t\right)\geq n\right\}&=&\left\{T_{n}\leq t\right\}\\
T_{N\left(t\right)}\leq &t&<T_{N\left(t\right)+1},
\end{eqnarray*}

adem\'as $N\left(T_{n}\right)=n$, y 

\begin{eqnarray*}
N\left(t\right)=\max\left\{n:T_{n}\leq t\right\}=\min\left\{n:T_{n+1}>t\right\}
\end{eqnarray*}

Por propiedades de la convoluci\'on se sabe que

\begin{eqnarray*}
P\left\{T_{n}\leq t\right\}=F^{n\star}\left(t\right)
\end{eqnarray*}
que es la $n$-\'esima convoluci\'on de $F$. Entonces 

\begin{eqnarray*}
\left\{N\left(t\right)\geq n\right\}&=&\left\{T_{n}\leq t\right\}\\
P\left\{N\left(t\right)\leq n\right\}&=&1-F^{\left(n+1\right)\star}\left(t\right)
\end{eqnarray*}

Adem\'as usando el hecho de que $\esp\left[N\left(t\right)\right]=\sum_{n=1}^{\infty}P\left\{N\left(t\right)\geq n\right\}$
se tiene que

\begin{eqnarray*}
\esp\left[N\left(t\right)\right]=\sum_{n=1}^{\infty}F^{n\star}\left(t\right)
\end{eqnarray*}

\begin{Prop}
Para cada $t\geq0$, la funci\'on generadora de momentos $\esp\left[e^{\alpha N\left(t\right)}\right]$ existe para alguna $\alpha$ en una vecindad del 0, y de aqu\'i que $\esp\left[N\left(t\right)^{m}\right]<\infty$, para $m\geq1$.
\end{Prop}


\begin{Note}
Si el primer tiempo de renovaci\'on $\xi_{1}$ no tiene la misma distribuci\'on que el resto de las $\xi_{n}$, para $n\geq2$, a $N\left(t\right)$ se le llama Proceso de Renovaci\'on retardado, donde si $\xi$ tiene distribuci\'on $G$, entonces el tiempo $T_{n}$ de la $n$-\'esima renovaci\'on tiene distribuci\'on $G\star F^{\left(n-1\right)\star}\left(t\right)$
\end{Note}


\begin{Teo}
Para una constante $\mu\leq\infty$ ( o variable aleatoria), las siguientes expresiones son equivalentes:

\begin{eqnarray}
lim_{n\rightarrow\infty}n^{-1}T_{n}&=&\mu,\textrm{ c.s.}\\
lim_{t\rightarrow\infty}t^{-1}N\left(t\right)&=&1/\mu,\textrm{ c.s.}
\end{eqnarray}
\end{Teo}


Es decir, $T_{n}$ satisface la Ley Fuerte de los Grandes N\'umeros s\'i y s\'olo s\'i $N\left/t\right)$ la cumple.


\begin{Coro}[Ley Fuerte de los Grandes N\'umeros para Procesos de Renovaci\'on]
Si $N\left(t\right)$ es un proceso de renovaci\'on cuyos tiempos de inter-renovaci\'on tienen media $\mu\leq\infty$, entonces
\begin{eqnarray}
t^{-1}N\left(t\right)\rightarrow 1/\mu,\textrm{ c.s. cuando }t\rightarrow\infty.
\end{eqnarray}

\end{Coro}


Considerar el proceso estoc\'astico de valores reales $\left\{Z\left(t\right):t\geq0\right\}$ en el mismo espacio de probabilidad que $N\left(t\right)$

\begin{Def}
Para el proceso $\left\{Z\left(t\right):t\geq0\right\}$ se define la fluctuaci\'on m\'axima de $Z\left(t\right)$ en el intervalo $\left(T_{n-1},T_{n}\right]$:
\begin{eqnarray*}
M_{n}=\sup_{T_{n-1}<t\leq T_{n}}|Z\left(t\right)-Z\left(T_{n-1}\right)|
\end{eqnarray*}
\end{Def}

\begin{Teo}
Sup\'ongase que $n^{-1}T_{n}\rightarrow\mu$ c.s. cuando $n\rightarrow\infty$, donde $\mu\leq\infty$ es una constante o variable aleatoria. Sea $a$ una constante o variable aleatoria que puede ser infinita cuando $\mu$ es finita, y considere las expresiones l\'imite:
\begin{eqnarray}
lim_{n\rightarrow\infty}n^{-1}Z\left(T_{n}\right)&=&a,\textrm{ c.s.}\\
lim_{t\rightarrow\infty}t^{-1}Z\left(t\right)&=&a/\mu,\textrm{ c.s.}
\end{eqnarray}
La segunda expresi\'on implica la primera. Conversamente, la primera implica la segunda si el proceso $Z\left(t\right)$ es creciente, o si $lim_{n\rightarrow\infty}n^{-1}M_{n}=0$ c.s.
\end{Teo}

\begin{Coro}
Si $N\left(t\right)$ es un proceso de renovaci\'on, y $\left(Z\left(T_{n}\right)-Z\left(T_{n-1}\right),M_{n}\right)$, para $n\geq1$, son variables aleatorias independientes e id\'enticamente distribuidas con media finita, entonces,
\begin{eqnarray}
lim_{t\rightarrow\infty}t^{-1}Z\left(t\right)\rightarrow\frac{\esp\left[Z\left(T_{1}\right)-Z\left(T_{0}\right)\right]}{\esp\left[T_{1}\right]},\textrm{ c.s. cuando  }t\rightarrow\infty.
\end{eqnarray}
\end{Coro}

%___________________________________________________________________________________________
%
%\subsection{Propiedades de los Procesos de Renovaci\'on}
%___________________________________________________________________________________________
%

Los tiempos $T_{n}$ est\'an relacionados con los conteos de $N\left(t\right)$ por

\begin{eqnarray*}
\left\{N\left(t\right)\geq n\right\}&=&\left\{T_{n}\leq t\right\}\\
T_{N\left(t\right)}\leq &t&<T_{N\left(t\right)+1},
\end{eqnarray*}

adem\'as $N\left(T_{n}\right)=n$, y 

\begin{eqnarray*}
N\left(t\right)=\max\left\{n:T_{n}\leq t\right\}=\min\left\{n:T_{n+1}>t\right\}
\end{eqnarray*}

Por propiedades de la convoluci\'on se sabe que

\begin{eqnarray*}
P\left\{T_{n}\leq t\right\}=F^{n\star}\left(t\right)
\end{eqnarray*}
que es la $n$-\'esima convoluci\'on de $F$. Entonces 

\begin{eqnarray*}
\left\{N\left(t\right)\geq n\right\}&=&\left\{T_{n}\leq t\right\}\\
P\left\{N\left(t\right)\leq n\right\}&=&1-F^{\left(n+1\right)\star}\left(t\right)
\end{eqnarray*}

Adem\'as usando el hecho de que $\esp\left[N\left(t\right)\right]=\sum_{n=1}^{\infty}P\left\{N\left(t\right)\geq n\right\}$
se tiene que

\begin{eqnarray*}
\esp\left[N\left(t\right)\right]=\sum_{n=1}^{\infty}F^{n\star}\left(t\right)
\end{eqnarray*}

\begin{Prop}
Para cada $t\geq0$, la funci\'on generadora de momentos $\esp\left[e^{\alpha N\left(t\right)}\right]$ existe para alguna $\alpha$ en una vecindad del 0, y de aqu\'i que $\esp\left[N\left(t\right)^{m}\right]<\infty$, para $m\geq1$.
\end{Prop}


\begin{Note}
Si el primer tiempo de renovaci\'on $\xi_{1}$ no tiene la misma distribuci\'on que el resto de las $\xi_{n}$, para $n\geq2$, a $N\left(t\right)$ se le llama Proceso de Renovaci\'on retardado, donde si $\xi$ tiene distribuci\'on $G$, entonces el tiempo $T_{n}$ de la $n$-\'esima renovaci\'on tiene distribuci\'on $G\star F^{\left(n-1\right)\star}\left(t\right)$
\end{Note}


\begin{Teo}
Para una constante $\mu\leq\infty$ ( o variable aleatoria), las siguientes expresiones son equivalentes:

\begin{eqnarray}
lim_{n\rightarrow\infty}n^{-1}T_{n}&=&\mu,\textrm{ c.s.}\\
lim_{t\rightarrow\infty}t^{-1}N\left(t\right)&=&1/\mu,\textrm{ c.s.}
\end{eqnarray}
\end{Teo}


Es decir, $T_{n}$ satisface la Ley Fuerte de los Grandes N\'umeros s\'i y s\'olo s\'i $N\left/t\right)$ la cumple.


\begin{Coro}[Ley Fuerte de los Grandes N\'umeros para Procesos de Renovaci\'on]
Si $N\left(t\right)$ es un proceso de renovaci\'on cuyos tiempos de inter-renovaci\'on tienen media $\mu\leq\infty$, entonces
\begin{eqnarray}
t^{-1}N\left(t\right)\rightarrow 1/\mu,\textrm{ c.s. cuando }t\rightarrow\infty.
\end{eqnarray}

\end{Coro}


Considerar el proceso estoc\'astico de valores reales $\left\{Z\left(t\right):t\geq0\right\}$ en el mismo espacio de probabilidad que $N\left(t\right)$

\begin{Def}
Para el proceso $\left\{Z\left(t\right):t\geq0\right\}$ se define la fluctuaci\'on m\'axima de $Z\left(t\right)$ en el intervalo $\left(T_{n-1},T_{n}\right]$:
\begin{eqnarray*}
M_{n}=\sup_{T_{n-1}<t\leq T_{n}}|Z\left(t\right)-Z\left(T_{n-1}\right)|
\end{eqnarray*}
\end{Def}

\begin{Teo}
Sup\'ongase que $n^{-1}T_{n}\rightarrow\mu$ c.s. cuando $n\rightarrow\infty$, donde $\mu\leq\infty$ es una constante o variable aleatoria. Sea $a$ una constante o variable aleatoria que puede ser infinita cuando $\mu$ es finita, y considere las expresiones l\'imite:
\begin{eqnarray}
lim_{n\rightarrow\infty}n^{-1}Z\left(T_{n}\right)&=&a,\textrm{ c.s.}\\
lim_{t\rightarrow\infty}t^{-1}Z\left(t\right)&=&a/\mu,\textrm{ c.s.}
\end{eqnarray}
La segunda expresi\'on implica la primera. Conversamente, la primera implica la segunda si el proceso $Z\left(t\right)$ es creciente, o si $lim_{n\rightarrow\infty}n^{-1}M_{n}=0$ c.s.
\end{Teo}

\begin{Coro}
Si $N\left(t\right)$ es un proceso de renovaci\'on, y $\left(Z\left(T_{n}\right)-Z\left(T_{n-1}\right),M_{n}\right)$, para $n\geq1$, son variables aleatorias independientes e id\'enticamente distribuidas con media finita, entonces,
\begin{eqnarray}
lim_{t\rightarrow\infty}t^{-1}Z\left(t\right)\rightarrow\frac{\esp\left[Z\left(T_{1}\right)-Z\left(T_{0}\right)\right]}{\esp\left[T_{1}\right]},\textrm{ c.s. cuando  }t\rightarrow\infty.
\end{eqnarray}
\end{Coro}



%___________________________________________________________________________________________
%
\subsection{Propiedades de los Procesos de Renovaci\'on}
%___________________________________________________________________________________________
%

Los tiempos $T_{n}$ est\'an relacionados con los conteos de $N\left(t\right)$ por

\begin{eqnarray*}
\left\{N\left(t\right)\geq n\right\}&=&\left\{T_{n}\leq t\right\}\\
T_{N\left(t\right)}\leq &t&<T_{N\left(t\right)+1},
\end{eqnarray*}

adem\'as $N\left(T_{n}\right)=n$, y 

\begin{eqnarray*}
N\left(t\right)=\max\left\{n:T_{n}\leq t\right\}=\min\left\{n:T_{n+1}>t\right\}
\end{eqnarray*}

Por propiedades de la convoluci\'on se sabe que

\begin{eqnarray*}
P\left\{T_{n}\leq t\right\}=F^{n\star}\left(t\right)
\end{eqnarray*}
que es la $n$-\'esima convoluci\'on de $F$. Entonces 

\begin{eqnarray*}
\left\{N\left(t\right)\geq n\right\}&=&\left\{T_{n}\leq t\right\}\\
P\left\{N\left(t\right)\leq n\right\}&=&1-F^{\left(n+1\right)\star}\left(t\right)
\end{eqnarray*}

Adem\'as usando el hecho de que $\esp\left[N\left(t\right)\right]=\sum_{n=1}^{\infty}P\left\{N\left(t\right)\geq n\right\}$
se tiene que

\begin{eqnarray*}
\esp\left[N\left(t\right)\right]=\sum_{n=1}^{\infty}F^{n\star}\left(t\right)
\end{eqnarray*}

\begin{Prop}
Para cada $t\geq0$, la funci\'on generadora de momentos $\esp\left[e^{\alpha N\left(t\right)}\right]$ existe para alguna $\alpha$ en una vecindad del 0, y de aqu\'i que $\esp\left[N\left(t\right)^{m}\right]<\infty$, para $m\geq1$.
\end{Prop}


\begin{Note}
Si el primer tiempo de renovaci\'on $\xi_{1}$ no tiene la misma distribuci\'on que el resto de las $\xi_{n}$, para $n\geq2$, a $N\left(t\right)$ se le llama Proceso de Renovaci\'on retardado, donde si $\xi$ tiene distribuci\'on $G$, entonces el tiempo $T_{n}$ de la $n$-\'esima renovaci\'on tiene distribuci\'on $G\star F^{\left(n-1\right)\star}\left(t\right)$
\end{Note}


\begin{Teo}
Para una constante $\mu\leq\infty$ ( o variable aleatoria), las siguientes expresiones son equivalentes:

\begin{eqnarray}
lim_{n\rightarrow\infty}n^{-1}T_{n}&=&\mu,\textrm{ c.s.}\\
lim_{t\rightarrow\infty}t^{-1}N\left(t\right)&=&1/\mu,\textrm{ c.s.}
\end{eqnarray}
\end{Teo}


Es decir, $T_{n}$ satisface la Ley Fuerte de los Grandes N\'umeros s\'i y s\'olo s\'i $N\left/t\right)$ la cumple.


\begin{Coro}[Ley Fuerte de los Grandes N\'umeros para Procesos de Renovaci\'on]
Si $N\left(t\right)$ es un proceso de renovaci\'on cuyos tiempos de inter-renovaci\'on tienen media $\mu\leq\infty$, entonces
\begin{eqnarray}
t^{-1}N\left(t\right)\rightarrow 1/\mu,\textrm{ c.s. cuando }t\rightarrow\infty.
\end{eqnarray}

\end{Coro}


Considerar el proceso estoc\'astico de valores reales $\left\{Z\left(t\right):t\geq0\right\}$ en el mismo espacio de probabilidad que $N\left(t\right)$

\begin{Def}
Para el proceso $\left\{Z\left(t\right):t\geq0\right\}$ se define la fluctuaci\'on m\'axima de $Z\left(t\right)$ en el intervalo $\left(T_{n-1},T_{n}\right]$:
\begin{eqnarray*}
M_{n}=\sup_{T_{n-1}<t\leq T_{n}}|Z\left(t\right)-Z\left(T_{n-1}\right)|
\end{eqnarray*}
\end{Def}

\begin{Teo}
Sup\'ongase que $n^{-1}T_{n}\rightarrow\mu$ c.s. cuando $n\rightarrow\infty$, donde $\mu\leq\infty$ es una constante o variable aleatoria. Sea $a$ una constante o variable aleatoria que puede ser infinita cuando $\mu$ es finita, y considere las expresiones l\'imite:
\begin{eqnarray}
lim_{n\rightarrow\infty}n^{-1}Z\left(T_{n}\right)&=&a,\textrm{ c.s.}\\
lim_{t\rightarrow\infty}t^{-1}Z\left(t\right)&=&a/\mu,\textrm{ c.s.}
\end{eqnarray}
La segunda expresi\'on implica la primera. Conversamente, la primera implica la segunda si el proceso $Z\left(t\right)$ es creciente, o si $lim_{n\rightarrow\infty}n^{-1}M_{n}=0$ c.s.
\end{Teo}

\begin{Coro}
Si $N\left(t\right)$ es un proceso de renovaci\'on, y $\left(Z\left(T_{n}\right)-Z\left(T_{n-1}\right),M_{n}\right)$, para $n\geq1$, son variables aleatorias independientes e id\'enticamente distribuidas con media finita, entonces,
\begin{eqnarray}
lim_{t\rightarrow\infty}t^{-1}Z\left(t\right)\rightarrow\frac{\esp\left[Z\left(T_{1}\right)-Z\left(T_{0}\right)\right]}{\esp\left[T_{1}\right]},\textrm{ c.s. cuando  }t\rightarrow\infty.
\end{eqnarray}
\end{Coro}




%__________________________________________________________________________________________
\subsection{Procesos Regenerativos Estacionarios - Stidham \cite{Stidham}}
%__________________________________________________________________________________________


Un proceso estoc\'astico a tiempo continuo $\left\{V\left(t\right),t\geq0\right\}$ es un proceso regenerativo si existe una sucesi\'on de variables aleatorias independientes e id\'enticamente distribuidas $\left\{X_{1},X_{2},\ldots\right\}$, sucesi\'on de renovaci\'on, tal que para cualquier conjunto de Borel $A$, 

\begin{eqnarray*}
\prob\left\{V\left(t\right)\in A|X_{1}+X_{2}+\cdots+X_{R\left(t\right)}=s,\left\{V\left(\tau\right),\tau<s\right\}\right\}=\prob\left\{V\left(t-s\right)\in A|X_{1}>t-s\right\},
\end{eqnarray*}
para todo $0\leq s\leq t$, donde $R\left(t\right)=\max\left\{X_{1}+X_{2}+\cdots+X_{j}\leq t\right\}=$n\'umero de renovaciones ({\emph{puntos de regeneraci\'on}}) que ocurren en $\left[0,t\right]$. El intervalo $\left[0,X_{1}\right)$ es llamado {\emph{primer ciclo de regeneraci\'on}} de $\left\{V\left(t \right),t\geq0\right\}$, $\left[X_{1},X_{1}+X_{2}\right)$ el {\emph{segundo ciclo de regeneraci\'on}}, y as\'i sucesivamente.

Sea $X=X_{1}$ y sea $F$ la funci\'on de distrbuci\'on de $X$


\begin{Def}
Se define el proceso estacionario, $\left\{V^{*}\left(t\right),t\geq0\right\}$, para $\left\{V\left(t\right),t\geq0\right\}$ por

\begin{eqnarray*}
\prob\left\{V\left(t\right)\in A\right\}=\frac{1}{\esp\left[X\right]}\int_{0}^{\infty}\prob\left\{V\left(t+x\right)\in A|X>x\right\}\left(1-F\left(x\right)\right)dx,
\end{eqnarray*} 
para todo $t\geq0$ y todo conjunto de Borel $A$.
\end{Def}

\begin{Def}
Una distribuci\'on se dice que es {\emph{aritm\'etica}} si todos sus puntos de incremento son m\'ultiplos de la forma $0,\lambda, 2\lambda,\ldots$ para alguna $\lambda>0$ entera.
\end{Def}


\begin{Def}
Una modificaci\'on medible de un proceso $\left\{V\left(t\right),t\geq0\right\}$, es una versi\'on de este, $\left\{V\left(t,w\right)\right\}$ conjuntamente medible para $t\geq0$ y para $w\in S$, $S$ espacio de estados para $\left\{V\left(t\right),t\geq0\right\}$.
\end{Def}

\begin{Teo}
Sea $\left\{V\left(t\right),t\geq\right\}$ un proceso regenerativo no negativo con modificaci\'on medible. Sea $\esp\left[X\right]<\infty$. Entonces el proceso estacionario dado por la ecuaci\'on anterior est\'a bien definido y tiene funci\'on de distribuci\'on independiente de $t$, adem\'as
\begin{itemize}
\item[i)] \begin{eqnarray*}
\esp\left[V^{*}\left(0\right)\right]&=&\frac{\esp\left[\int_{0}^{X}V\left(s\right)ds\right]}{\esp\left[X\right]}\end{eqnarray*}
\item[ii)] Si $\esp\left[V^{*}\left(0\right)\right]<\infty$, equivalentemente, si $\esp\left[\int_{0}^{X}V\left(s\right)ds\right]<\infty$,entonces
\begin{eqnarray*}
\frac{\int_{0}^{t}V\left(s\right)ds}{t}\rightarrow\frac{\esp\left[\int_{0}^{X}V\left(s\right)ds\right]}{\esp\left[X\right]}
\end{eqnarray*}
con probabilidad 1 y en media, cuando $t\rightarrow\infty$.
\end{itemize}
\end{Teo}

%______________________________________________________________________
\subsection{Procesos de Renovaci\'on}
%______________________________________________________________________

\begin{Def}\label{Def.Tn}
Sean $0\leq T_{1}\leq T_{2}\leq \ldots$ son tiempos aleatorios infinitos en los cuales ocurren ciertos eventos. El n\'umero de tiempos $T_{n}$ en el intervalo $\left[0,t\right)$ es

\begin{eqnarray}
N\left(t\right)=\sum_{n=1}^{\infty}\indora\left(T_{n}\leq t\right),
\end{eqnarray}
para $t\geq0$.
\end{Def}

Si se consideran los puntos $T_{n}$ como elementos de $\rea_{+}$, y $N\left(t\right)$ es el n\'umero de puntos en $\rea$. El proceso denotado por $\left\{N\left(t\right):t\geq0\right\}$, denotado por $N\left(t\right)$, es un proceso puntual en $\rea_{+}$. Los $T_{n}$ son los tiempos de ocurrencia, el proceso puntual $N\left(t\right)$ es simple si su n\'umero de ocurrencias son distintas: $0<T_{1}<T_{2}<\ldots$ casi seguramente.

\begin{Def}
Un proceso puntual $N\left(t\right)$ es un proceso de renovaci\'on si los tiempos de interocurrencia $\xi_{n}=T_{n}-T_{n-1}$, para $n\geq1$, son independientes e identicamente distribuidos con distribuci\'on $F$, donde $F\left(0\right)=0$ y $T_{0}=0$. Los $T_{n}$ son llamados tiempos de renovaci\'on, referente a la independencia o renovaci\'on de la informaci\'on estoc\'astica en estos tiempos. Los $\xi_{n}$ son los tiempos de inter-renovaci\'on, y $N\left(t\right)$ es el n\'umero de renovaciones en el intervalo $\left[0,t\right)$
\end{Def}


\begin{Note}
Para definir un proceso de renovaci\'on para cualquier contexto, solamente hay que especificar una distribuci\'on $F$, con $F\left(0\right)=0$, para los tiempos de inter-renovaci\'on. La funci\'on $F$ en turno degune las otra variables aleatorias. De manera formal, existe un espacio de probabilidad y una sucesi\'on de variables aleatorias $\xi_{1},\xi_{2},\ldots$ definidas en este con distribuci\'on $F$. Entonces las otras cantidades son $T_{n}=\sum_{k=1}^{n}\xi_{k}$ y $N\left(t\right)=\sum_{n=1}^{\infty}\indora\left(T_{n}\leq t\right)$, donde $T_{n}\rightarrow\infty$ casi seguramente por la Ley Fuerte de los Grandes Números.
\end{Note}

%___________________________________________________________________________________________
%
\subsection{Teorema Principal de Renovaci\'on}
%___________________________________________________________________________________________
%

\begin{Note} Una funci\'on $h:\rea_{+}\rightarrow\rea$ es Directamente Riemann Integrable en los siguientes casos:
\begin{itemize}
\item[a)] $h\left(t\right)\geq0$ es decreciente y Riemann Integrable.
\item[b)] $h$ es continua excepto posiblemente en un conjunto de Lebesgue de medida 0, y $|h\left(t\right)|\leq b\left(t\right)$, donde $b$ es DRI.
\end{itemize}
\end{Note}

\begin{Teo}[Teorema Principal de Renovaci\'on]
Si $F$ es no aritm\'etica y $h\left(t\right)$ es Directamente Riemann Integrable (DRI), entonces

\begin{eqnarray*}
lim_{t\rightarrow\infty}U\star h=\frac{1}{\mu}\int_{\rea_{+}}h\left(s\right)ds.
\end{eqnarray*}
\end{Teo}

\begin{Prop}
Cualquier funci\'on $H\left(t\right)$ acotada en intervalos finitos y que es 0 para $t<0$ puede expresarse como
\begin{eqnarray*}
H\left(t\right)=U\star h\left(t\right)\textrm{,  donde }h\left(t\right)=H\left(t\right)-F\star H\left(t\right)
\end{eqnarray*}
\end{Prop}

\begin{Def}
Un proceso estoc\'astico $X\left(t\right)$ es crudamente regenerativo en un tiempo aleatorio positivo $T$ si
\begin{eqnarray*}
\esp\left[X\left(T+t\right)|T\right]=\esp\left[X\left(t\right)\right]\textrm{, para }t\geq0,\end{eqnarray*}
y con las esperanzas anteriores finitas.
\end{Def}

\begin{Prop}
Sup\'ongase que $X\left(t\right)$ es un proceso crudamente regenerativo en $T$, que tiene distribuci\'on $F$. Si $\esp\left[X\left(t\right)\right]$ es acotado en intervalos finitos, entonces
\begin{eqnarray*}
\esp\left[X\left(t\right)\right]=U\star h\left(t\right)\textrm{,  donde }h\left(t\right)=\esp\left[X\left(t\right)\indora\left(T>t\right)\right].
\end{eqnarray*}
\end{Prop}

\begin{Teo}[Regeneraci\'on Cruda]
Sup\'ongase que $X\left(t\right)$ es un proceso con valores positivo crudamente regenerativo en $T$, y def\'inase $M=\sup\left\{|X\left(t\right)|:t\leq T\right\}$. Si $T$ es no aritm\'etico y $M$ y $MT$ tienen media finita, entonces
\begin{eqnarray*}
lim_{t\rightarrow\infty}\esp\left[X\left(t\right)\right]=\frac{1}{\mu}\int_{\rea_{+}}h\left(s\right)ds,
\end{eqnarray*}
donde $h\left(t\right)=\esp\left[X\left(t\right)\indora\left(T>t\right)\right]$.
\end{Teo}



%___________________________________________________________________________________________
%
\subsection{Funci\'on de Renovaci\'on}
%___________________________________________________________________________________________
%


\begin{Def}
Sea $h\left(t\right)$ funci\'on de valores reales en $\rea$ acotada en intervalos finitos e igual a cero para $t<0$ La ecuaci\'on de renovaci\'on para $h\left(t\right)$ y la distribuci\'on $F$ es

\begin{eqnarray}\label{Ec.Renovacion}
H\left(t\right)=h\left(t\right)+\int_{\left[0,t\right]}H\left(t-s\right)dF\left(s\right)\textrm{,    }t\geq0,
\end{eqnarray}
donde $H\left(t\right)$ es una funci\'on de valores reales. Esto es $H=h+F\star H$. Decimos que $H\left(t\right)$ es soluci\'on de esta ecuaci\'on si satisface la ecuaci\'on, y es acotada en intervalos finitos e iguales a cero para $t<0$.
\end{Def}

\begin{Prop}
La funci\'on $U\star h\left(t\right)$ es la \'unica soluci\'on de la ecuaci\'on de renovaci\'on (\ref{Ec.Renovacion}).
\end{Prop}

\begin{Teo}[Teorema Renovaci\'on Elemental]
\begin{eqnarray*}
t^{-1}U\left(t\right)\rightarrow 1/\mu\textrm{,    cuando }t\rightarrow\infty.
\end{eqnarray*}
\end{Teo}

%___________________________________________________________________________________________
%
\subsection{Propiedades de los Procesos de Renovaci\'on}
%___________________________________________________________________________________________
%

Los tiempos $T_{n}$ est\'an relacionados con los conteos de $N\left(t\right)$ por

\begin{eqnarray*}
\left\{N\left(t\right)\geq n\right\}&=&\left\{T_{n}\leq t\right\}\\
T_{N\left(t\right)}\leq &t&<T_{N\left(t\right)+1},
\end{eqnarray*}

adem\'as $N\left(T_{n}\right)=n$, y 

\begin{eqnarray*}
N\left(t\right)=\max\left\{n:T_{n}\leq t\right\}=\min\left\{n:T_{n+1}>t\right\}
\end{eqnarray*}

Por propiedades de la convoluci\'on se sabe que

\begin{eqnarray*}
P\left\{T_{n}\leq t\right\}=F^{n\star}\left(t\right)
\end{eqnarray*}
que es la $n$-\'esima convoluci\'on de $F$. Entonces 

\begin{eqnarray*}
\left\{N\left(t\right)\geq n\right\}&=&\left\{T_{n}\leq t\right\}\\
P\left\{N\left(t\right)\leq n\right\}&=&1-F^{\left(n+1\right)\star}\left(t\right)
\end{eqnarray*}

Adem\'as usando el hecho de que $\esp\left[N\left(t\right)\right]=\sum_{n=1}^{\infty}P\left\{N\left(t\right)\geq n\right\}$
se tiene que

\begin{eqnarray*}
\esp\left[N\left(t\right)\right]=\sum_{n=1}^{\infty}F^{n\star}\left(t\right)
\end{eqnarray*}

\begin{Prop}
Para cada $t\geq0$, la funci\'on generadora de momentos $\esp\left[e^{\alpha N\left(t\right)}\right]$ existe para alguna $\alpha$ en una vecindad del 0, y de aqu\'i que $\esp\left[N\left(t\right)^{m}\right]<\infty$, para $m\geq1$.
\end{Prop}


\begin{Note}
Si el primer tiempo de renovaci\'on $\xi_{1}$ no tiene la misma distribuci\'on que el resto de las $\xi_{n}$, para $n\geq2$, a $N\left(t\right)$ se le llama Proceso de Renovaci\'on retardado, donde si $\xi$ tiene distribuci\'on $G$, entonces el tiempo $T_{n}$ de la $n$-\'esima renovaci\'on tiene distribuci\'on $G\star F^{\left(n-1\right)\star}\left(t\right)$
\end{Note}


\begin{Teo}
Para una constante $\mu\leq\infty$ ( o variable aleatoria), las siguientes expresiones son equivalentes:

\begin{eqnarray}
lim_{n\rightarrow\infty}n^{-1}T_{n}&=&\mu,\textrm{ c.s.}\\
lim_{t\rightarrow\infty}t^{-1}N\left(t\right)&=&1/\mu,\textrm{ c.s.}
\end{eqnarray}
\end{Teo}


Es decir, $T_{n}$ satisface la Ley Fuerte de los Grandes N\'umeros s\'i y s\'olo s\'i $N\left/t\right)$ la cumple.


\begin{Coro}[Ley Fuerte de los Grandes N\'umeros para Procesos de Renovaci\'on]
Si $N\left(t\right)$ es un proceso de renovaci\'on cuyos tiempos de inter-renovaci\'on tienen media $\mu\leq\infty$, entonces
\begin{eqnarray}
t^{-1}N\left(t\right)\rightarrow 1/\mu,\textrm{ c.s. cuando }t\rightarrow\infty.
\end{eqnarray}

\end{Coro}


Considerar el proceso estoc\'astico de valores reales $\left\{Z\left(t\right):t\geq0\right\}$ en el mismo espacio de probabilidad que $N\left(t\right)$

\begin{Def}
Para el proceso $\left\{Z\left(t\right):t\geq0\right\}$ se define la fluctuaci\'on m\'axima de $Z\left(t\right)$ en el intervalo $\left(T_{n-1},T_{n}\right]$:
\begin{eqnarray*}
M_{n}=\sup_{T_{n-1}<t\leq T_{n}}|Z\left(t\right)-Z\left(T_{n-1}\right)|
\end{eqnarray*}
\end{Def}

\begin{Teo}
Sup\'ongase que $n^{-1}T_{n}\rightarrow\mu$ c.s. cuando $n\rightarrow\infty$, donde $\mu\leq\infty$ es una constante o variable aleatoria. Sea $a$ una constante o variable aleatoria que puede ser infinita cuando $\mu$ es finita, y considere las expresiones l\'imite:
\begin{eqnarray}
lim_{n\rightarrow\infty}n^{-1}Z\left(T_{n}\right)&=&a,\textrm{ c.s.}\\
lim_{t\rightarrow\infty}t^{-1}Z\left(t\right)&=&a/\mu,\textrm{ c.s.}
\end{eqnarray}
La segunda expresi\'on implica la primera. Conversamente, la primera implica la segunda si el proceso $Z\left(t\right)$ es creciente, o si $lim_{n\rightarrow\infty}n^{-1}M_{n}=0$ c.s.
\end{Teo}

\begin{Coro}
Si $N\left(t\right)$ es un proceso de renovaci\'on, y $\left(Z\left(T_{n}\right)-Z\left(T_{n-1}\right),M_{n}\right)$, para $n\geq1$, son variables aleatorias independientes e id\'enticamente distribuidas con media finita, entonces,
\begin{eqnarray}
lim_{t\rightarrow\infty}t^{-1}Z\left(t\right)\rightarrow\frac{\esp\left[Z\left(T_{1}\right)-Z\left(T_{0}\right)\right]}{\esp\left[T_{1}\right]},\textrm{ c.s. cuando  }t\rightarrow\infty.
\end{eqnarray}
\end{Coro}


%___________________________________________________________________________________________
%
%\subsection{Propiedades de los Procesos de Renovaci\'on}
%___________________________________________________________________________________________
%

Los tiempos $T_{n}$ est\'an relacionados con los conteos de $N\left(t\right)$ por

\begin{eqnarray*}
\left\{N\left(t\right)\geq n\right\}&=&\left\{T_{n}\leq t\right\}\\
T_{N\left(t\right)}\leq &t&<T_{N\left(t\right)+1},
\end{eqnarray*}

adem\'as $N\left(T_{n}\right)=n$, y 

\begin{eqnarray*}
N\left(t\right)=\max\left\{n:T_{n}\leq t\right\}=\min\left\{n:T_{n+1}>t\right\}
\end{eqnarray*}

Por propiedades de la convoluci\'on se sabe que

\begin{eqnarray*}
P\left\{T_{n}\leq t\right\}=F^{n\star}\left(t\right)
\end{eqnarray*}
que es la $n$-\'esima convoluci\'on de $F$. Entonces 

\begin{eqnarray*}
\left\{N\left(t\right)\geq n\right\}&=&\left\{T_{n}\leq t\right\}\\
P\left\{N\left(t\right)\leq n\right\}&=&1-F^{\left(n+1\right)\star}\left(t\right)
\end{eqnarray*}

Adem\'as usando el hecho de que $\esp\left[N\left(t\right)\right]=\sum_{n=1}^{\infty}P\left\{N\left(t\right)\geq n\right\}$
se tiene que

\begin{eqnarray*}
\esp\left[N\left(t\right)\right]=\sum_{n=1}^{\infty}F^{n\star}\left(t\right)
\end{eqnarray*}

\begin{Prop}
Para cada $t\geq0$, la funci\'on generadora de momentos $\esp\left[e^{\alpha N\left(t\right)}\right]$ existe para alguna $\alpha$ en una vecindad del 0, y de aqu\'i que $\esp\left[N\left(t\right)^{m}\right]<\infty$, para $m\geq1$.
\end{Prop}


\begin{Note}
Si el primer tiempo de renovaci\'on $\xi_{1}$ no tiene la misma distribuci\'on que el resto de las $\xi_{n}$, para $n\geq2$, a $N\left(t\right)$ se le llama Proceso de Renovaci\'on retardado, donde si $\xi$ tiene distribuci\'on $G$, entonces el tiempo $T_{n}$ de la $n$-\'esima renovaci\'on tiene distribuci\'on $G\star F^{\left(n-1\right)\star}\left(t\right)$
\end{Note}


\begin{Teo}
Para una constante $\mu\leq\infty$ ( o variable aleatoria), las siguientes expresiones son equivalentes:

\begin{eqnarray}
lim_{n\rightarrow\infty}n^{-1}T_{n}&=&\mu,\textrm{ c.s.}\\
lim_{t\rightarrow\infty}t^{-1}N\left(t\right)&=&1/\mu,\textrm{ c.s.}
\end{eqnarray}
\end{Teo}


Es decir, $T_{n}$ satisface la Ley Fuerte de los Grandes N\'umeros s\'i y s\'olo s\'i $N\left/t\right)$ la cumple.


\begin{Coro}[Ley Fuerte de los Grandes N\'umeros para Procesos de Renovaci\'on]
Si $N\left(t\right)$ es un proceso de renovaci\'on cuyos tiempos de inter-renovaci\'on tienen media $\mu\leq\infty$, entonces
\begin{eqnarray}
t^{-1}N\left(t\right)\rightarrow 1/\mu,\textrm{ c.s. cuando }t\rightarrow\infty.
\end{eqnarray}

\end{Coro}


Considerar el proceso estoc\'astico de valores reales $\left\{Z\left(t\right):t\geq0\right\}$ en el mismo espacio de probabilidad que $N\left(t\right)$

\begin{Def}
Para el proceso $\left\{Z\left(t\right):t\geq0\right\}$ se define la fluctuaci\'on m\'axima de $Z\left(t\right)$ en el intervalo $\left(T_{n-1},T_{n}\right]$:
\begin{eqnarray*}
M_{n}=\sup_{T_{n-1}<t\leq T_{n}}|Z\left(t\right)-Z\left(T_{n-1}\right)|
\end{eqnarray*}
\end{Def}

\begin{Teo}
Sup\'ongase que $n^{-1}T_{n}\rightarrow\mu$ c.s. cuando $n\rightarrow\infty$, donde $\mu\leq\infty$ es una constante o variable aleatoria. Sea $a$ una constante o variable aleatoria que puede ser infinita cuando $\mu$ es finita, y considere las expresiones l\'imite:
\begin{eqnarray}
lim_{n\rightarrow\infty}n^{-1}Z\left(T_{n}\right)&=&a,\textrm{ c.s.}\\
lim_{t\rightarrow\infty}t^{-1}Z\left(t\right)&=&a/\mu,\textrm{ c.s.}
\end{eqnarray}
La segunda expresi\'on implica la primera. Conversamente, la primera implica la segunda si el proceso $Z\left(t\right)$ es creciente, o si $lim_{n\rightarrow\infty}n^{-1}M_{n}=0$ c.s.
\end{Teo}

\begin{Coro}
Si $N\left(t\right)$ es un proceso de renovaci\'on, y $\left(Z\left(T_{n}\right)-Z\left(T_{n-1}\right),M_{n}\right)$, para $n\geq1$, son variables aleatorias independientes e id\'enticamente distribuidas con media finita, entonces,
\begin{eqnarray}
lim_{t\rightarrow\infty}t^{-1}Z\left(t\right)\rightarrow\frac{\esp\left[Z\left(T_{1}\right)-Z\left(T_{0}\right)\right]}{\esp\left[T_{1}\right]},\textrm{ c.s. cuando  }t\rightarrow\infty.
\end{eqnarray}
\end{Coro}


%___________________________________________________________________________________________
%
%\subsection{Propiedades de los Procesos de Renovaci\'on}
%___________________________________________________________________________________________
%

Los tiempos $T_{n}$ est\'an relacionados con los conteos de $N\left(t\right)$ por

\begin{eqnarray*}
\left\{N\left(t\right)\geq n\right\}&=&\left\{T_{n}\leq t\right\}\\
T_{N\left(t\right)}\leq &t&<T_{N\left(t\right)+1},
\end{eqnarray*}

adem\'as $N\left(T_{n}\right)=n$, y 

\begin{eqnarray*}
N\left(t\right)=\max\left\{n:T_{n}\leq t\right\}=\min\left\{n:T_{n+1}>t\right\}
\end{eqnarray*}

Por propiedades de la convoluci\'on se sabe que

\begin{eqnarray*}
P\left\{T_{n}\leq t\right\}=F^{n\star}\left(t\right)
\end{eqnarray*}
que es la $n$-\'esima convoluci\'on de $F$. Entonces 

\begin{eqnarray*}
\left\{N\left(t\right)\geq n\right\}&=&\left\{T_{n}\leq t\right\}\\
P\left\{N\left(t\right)\leq n\right\}&=&1-F^{\left(n+1\right)\star}\left(t\right)
\end{eqnarray*}

Adem\'as usando el hecho de que $\esp\left[N\left(t\right)\right]=\sum_{n=1}^{\infty}P\left\{N\left(t\right)\geq n\right\}$
se tiene que

\begin{eqnarray*}
\esp\left[N\left(t\right)\right]=\sum_{n=1}^{\infty}F^{n\star}\left(t\right)
\end{eqnarray*}

\begin{Prop}
Para cada $t\geq0$, la funci\'on generadora de momentos $\esp\left[e^{\alpha N\left(t\right)}\right]$ existe para alguna $\alpha$ en una vecindad del 0, y de aqu\'i que $\esp\left[N\left(t\right)^{m}\right]<\infty$, para $m\geq1$.
\end{Prop}


\begin{Note}
Si el primer tiempo de renovaci\'on $\xi_{1}$ no tiene la misma distribuci\'on que el resto de las $\xi_{n}$, para $n\geq2$, a $N\left(t\right)$ se le llama Proceso de Renovaci\'on retardado, donde si $\xi$ tiene distribuci\'on $G$, entonces el tiempo $T_{n}$ de la $n$-\'esima renovaci\'on tiene distribuci\'on $G\star F^{\left(n-1\right)\star}\left(t\right)$
\end{Note}


\begin{Teo}
Para una constante $\mu\leq\infty$ ( o variable aleatoria), las siguientes expresiones son equivalentes:

\begin{eqnarray}
lim_{n\rightarrow\infty}n^{-1}T_{n}&=&\mu,\textrm{ c.s.}\\
lim_{t\rightarrow\infty}t^{-1}N\left(t\right)&=&1/\mu,\textrm{ c.s.}
\end{eqnarray}
\end{Teo}


Es decir, $T_{n}$ satisface la Ley Fuerte de los Grandes N\'umeros s\'i y s\'olo s\'i $N\left/t\right)$ la cumple.


\begin{Coro}[Ley Fuerte de los Grandes N\'umeros para Procesos de Renovaci\'on]
Si $N\left(t\right)$ es un proceso de renovaci\'on cuyos tiempos de inter-renovaci\'on tienen media $\mu\leq\infty$, entonces
\begin{eqnarray}
t^{-1}N\left(t\right)\rightarrow 1/\mu,\textrm{ c.s. cuando }t\rightarrow\infty.
\end{eqnarray}

\end{Coro}


Considerar el proceso estoc\'astico de valores reales $\left\{Z\left(t\right):t\geq0\right\}$ en el mismo espacio de probabilidad que $N\left(t\right)$

\begin{Def}
Para el proceso $\left\{Z\left(t\right):t\geq0\right\}$ se define la fluctuaci\'on m\'axima de $Z\left(t\right)$ en el intervalo $\left(T_{n-1},T_{n}\right]$:
\begin{eqnarray*}
M_{n}=\sup_{T_{n-1}<t\leq T_{n}}|Z\left(t\right)-Z\left(T_{n-1}\right)|
\end{eqnarray*}
\end{Def}

\begin{Teo}
Sup\'ongase que $n^{-1}T_{n}\rightarrow\mu$ c.s. cuando $n\rightarrow\infty$, donde $\mu\leq\infty$ es una constante o variable aleatoria. Sea $a$ una constante o variable aleatoria que puede ser infinita cuando $\mu$ es finita, y considere las expresiones l\'imite:
\begin{eqnarray}
lim_{n\rightarrow\infty}n^{-1}Z\left(T_{n}\right)&=&a,\textrm{ c.s.}\\
lim_{t\rightarrow\infty}t^{-1}Z\left(t\right)&=&a/\mu,\textrm{ c.s.}
\end{eqnarray}
La segunda expresi\'on implica la primera. Conversamente, la primera implica la segunda si el proceso $Z\left(t\right)$ es creciente, o si $lim_{n\rightarrow\infty}n^{-1}M_{n}=0$ c.s.
\end{Teo}

\begin{Coro}
Si $N\left(t\right)$ es un proceso de renovaci\'on, y $\left(Z\left(T_{n}\right)-Z\left(T_{n-1}\right),M_{n}\right)$, para $n\geq1$, son variables aleatorias independientes e id\'enticamente distribuidas con media finita, entonces,
\begin{eqnarray}
lim_{t\rightarrow\infty}t^{-1}Z\left(t\right)\rightarrow\frac{\esp\left[Z\left(T_{1}\right)-Z\left(T_{0}\right)\right]}{\esp\left[T_{1}\right]},\textrm{ c.s. cuando  }t\rightarrow\infty.
\end{eqnarray}
\end{Coro}

%___________________________________________________________________________________________
%
%\subsection{Propiedades de los Procesos de Renovaci\'on}
%___________________________________________________________________________________________
%

Los tiempos $T_{n}$ est\'an relacionados con los conteos de $N\left(t\right)$ por

\begin{eqnarray*}
\left\{N\left(t\right)\geq n\right\}&=&\left\{T_{n}\leq t\right\}\\
T_{N\left(t\right)}\leq &t&<T_{N\left(t\right)+1},
\end{eqnarray*}

adem\'as $N\left(T_{n}\right)=n$, y 

\begin{eqnarray*}
N\left(t\right)=\max\left\{n:T_{n}\leq t\right\}=\min\left\{n:T_{n+1}>t\right\}
\end{eqnarray*}

Por propiedades de la convoluci\'on se sabe que

\begin{eqnarray*}
P\left\{T_{n}\leq t\right\}=F^{n\star}\left(t\right)
\end{eqnarray*}
que es la $n$-\'esima convoluci\'on de $F$. Entonces 

\begin{eqnarray*}
\left\{N\left(t\right)\geq n\right\}&=&\left\{T_{n}\leq t\right\}\\
P\left\{N\left(t\right)\leq n\right\}&=&1-F^{\left(n+1\right)\star}\left(t\right)
\end{eqnarray*}

Adem\'as usando el hecho de que $\esp\left[N\left(t\right)\right]=\sum_{n=1}^{\infty}P\left\{N\left(t\right)\geq n\right\}$
se tiene que

\begin{eqnarray*}
\esp\left[N\left(t\right)\right]=\sum_{n=1}^{\infty}F^{n\star}\left(t\right)
\end{eqnarray*}

\begin{Prop}
Para cada $t\geq0$, la funci\'on generadora de momentos $\esp\left[e^{\alpha N\left(t\right)}\right]$ existe para alguna $\alpha$ en una vecindad del 0, y de aqu\'i que $\esp\left[N\left(t\right)^{m}\right]<\infty$, para $m\geq1$.
\end{Prop}


\begin{Note}
Si el primer tiempo de renovaci\'on $\xi_{1}$ no tiene la misma distribuci\'on que el resto de las $\xi_{n}$, para $n\geq2$, a $N\left(t\right)$ se le llama Proceso de Renovaci\'on retardado, donde si $\xi$ tiene distribuci\'on $G$, entonces el tiempo $T_{n}$ de la $n$-\'esima renovaci\'on tiene distribuci\'on $G\star F^{\left(n-1\right)\star}\left(t\right)$
\end{Note}


\begin{Teo}
Para una constante $\mu\leq\infty$ ( o variable aleatoria), las siguientes expresiones son equivalentes:

\begin{eqnarray}
lim_{n\rightarrow\infty}n^{-1}T_{n}&=&\mu,\textrm{ c.s.}\\
lim_{t\rightarrow\infty}t^{-1}N\left(t\right)&=&1/\mu,\textrm{ c.s.}
\end{eqnarray}
\end{Teo}


Es decir, $T_{n}$ satisface la Ley Fuerte de los Grandes N\'umeros s\'i y s\'olo s\'i $N\left/t\right)$ la cumple.


\begin{Coro}[Ley Fuerte de los Grandes N\'umeros para Procesos de Renovaci\'on]
Si $N\left(t\right)$ es un proceso de renovaci\'on cuyos tiempos de inter-renovaci\'on tienen media $\mu\leq\infty$, entonces
\begin{eqnarray}
t^{-1}N\left(t\right)\rightarrow 1/\mu,\textrm{ c.s. cuando }t\rightarrow\infty.
\end{eqnarray}

\end{Coro}


Considerar el proceso estoc\'astico de valores reales $\left\{Z\left(t\right):t\geq0\right\}$ en el mismo espacio de probabilidad que $N\left(t\right)$

\begin{Def}
Para el proceso $\left\{Z\left(t\right):t\geq0\right\}$ se define la fluctuaci\'on m\'axima de $Z\left(t\right)$ en el intervalo $\left(T_{n-1},T_{n}\right]$:
\begin{eqnarray*}
M_{n}=\sup_{T_{n-1}<t\leq T_{n}}|Z\left(t\right)-Z\left(T_{n-1}\right)|
\end{eqnarray*}
\end{Def}

\begin{Teo}
Sup\'ongase que $n^{-1}T_{n}\rightarrow\mu$ c.s. cuando $n\rightarrow\infty$, donde $\mu\leq\infty$ es una constante o variable aleatoria. Sea $a$ una constante o variable aleatoria que puede ser infinita cuando $\mu$ es finita, y considere las expresiones l\'imite:
\begin{eqnarray}
lim_{n\rightarrow\infty}n^{-1}Z\left(T_{n}\right)&=&a,\textrm{ c.s.}\\
lim_{t\rightarrow\infty}t^{-1}Z\left(t\right)&=&a/\mu,\textrm{ c.s.}
\end{eqnarray}
La segunda expresi\'on implica la primera. Conversamente, la primera implica la segunda si el proceso $Z\left(t\right)$ es creciente, o si $lim_{n\rightarrow\infty}n^{-1}M_{n}=0$ c.s.
\end{Teo}

\begin{Coro}
Si $N\left(t\right)$ es un proceso de renovaci\'on, y $\left(Z\left(T_{n}\right)-Z\left(T_{n-1}\right),M_{n}\right)$, para $n\geq1$, son variables aleatorias independientes e id\'enticamente distribuidas con media finita, entonces,
\begin{eqnarray}
lim_{t\rightarrow\infty}t^{-1}Z\left(t\right)\rightarrow\frac{\esp\left[Z\left(T_{1}\right)-Z\left(T_{0}\right)\right]}{\esp\left[T_{1}\right]},\textrm{ c.s. cuando  }t\rightarrow\infty.
\end{eqnarray}
\end{Coro}
%___________________________________________________________________________________________
%
%\subsection{Propiedades de los Procesos de Renovaci\'on}
%___________________________________________________________________________________________
%

Los tiempos $T_{n}$ est\'an relacionados con los conteos de $N\left(t\right)$ por

\begin{eqnarray*}
\left\{N\left(t\right)\geq n\right\}&=&\left\{T_{n}\leq t\right\}\\
T_{N\left(t\right)}\leq &t&<T_{N\left(t\right)+1},
\end{eqnarray*}

adem\'as $N\left(T_{n}\right)=n$, y 

\begin{eqnarray*}
N\left(t\right)=\max\left\{n:T_{n}\leq t\right\}=\min\left\{n:T_{n+1}>t\right\}
\end{eqnarray*}

Por propiedades de la convoluci\'on se sabe que

\begin{eqnarray*}
P\left\{T_{n}\leq t\right\}=F^{n\star}\left(t\right)
\end{eqnarray*}
que es la $n$-\'esima convoluci\'on de $F$. Entonces 

\begin{eqnarray*}
\left\{N\left(t\right)\geq n\right\}&=&\left\{T_{n}\leq t\right\}\\
P\left\{N\left(t\right)\leq n\right\}&=&1-F^{\left(n+1\right)\star}\left(t\right)
\end{eqnarray*}

Adem\'as usando el hecho de que $\esp\left[N\left(t\right)\right]=\sum_{n=1}^{\infty}P\left\{N\left(t\right)\geq n\right\}$
se tiene que

\begin{eqnarray*}
\esp\left[N\left(t\right)\right]=\sum_{n=1}^{\infty}F^{n\star}\left(t\right)
\end{eqnarray*}

\begin{Prop}
Para cada $t\geq0$, la funci\'on generadora de momentos $\esp\left[e^{\alpha N\left(t\right)}\right]$ existe para alguna $\alpha$ en una vecindad del 0, y de aqu\'i que $\esp\left[N\left(t\right)^{m}\right]<\infty$, para $m\geq1$.
\end{Prop}


\begin{Note}
Si el primer tiempo de renovaci\'on $\xi_{1}$ no tiene la misma distribuci\'on que el resto de las $\xi_{n}$, para $n\geq2$, a $N\left(t\right)$ se le llama Proceso de Renovaci\'on retardado, donde si $\xi$ tiene distribuci\'on $G$, entonces el tiempo $T_{n}$ de la $n$-\'esima renovaci\'on tiene distribuci\'on $G\star F^{\left(n-1\right)\star}\left(t\right)$
\end{Note}


\begin{Teo}
Para una constante $\mu\leq\infty$ ( o variable aleatoria), las siguientes expresiones son equivalentes:

\begin{eqnarray}
lim_{n\rightarrow\infty}n^{-1}T_{n}&=&\mu,\textrm{ c.s.}\\
lim_{t\rightarrow\infty}t^{-1}N\left(t\right)&=&1/\mu,\textrm{ c.s.}
\end{eqnarray}
\end{Teo}


Es decir, $T_{n}$ satisface la Ley Fuerte de los Grandes N\'umeros s\'i y s\'olo s\'i $N\left/t\right)$ la cumple.


\begin{Coro}[Ley Fuerte de los Grandes N\'umeros para Procesos de Renovaci\'on]
Si $N\left(t\right)$ es un proceso de renovaci\'on cuyos tiempos de inter-renovaci\'on tienen media $\mu\leq\infty$, entonces
\begin{eqnarray}
t^{-1}N\left(t\right)\rightarrow 1/\mu,\textrm{ c.s. cuando }t\rightarrow\infty.
\end{eqnarray}

\end{Coro}


Considerar el proceso estoc\'astico de valores reales $\left\{Z\left(t\right):t\geq0\right\}$ en el mismo espacio de probabilidad que $N\left(t\right)$

\begin{Def}
Para el proceso $\left\{Z\left(t\right):t\geq0\right\}$ se define la fluctuaci\'on m\'axima de $Z\left(t\right)$ en el intervalo $\left(T_{n-1},T_{n}\right]$:
\begin{eqnarray*}
M_{n}=\sup_{T_{n-1}<t\leq T_{n}}|Z\left(t\right)-Z\left(T_{n-1}\right)|
\end{eqnarray*}
\end{Def}

\begin{Teo}
Sup\'ongase que $n^{-1}T_{n}\rightarrow\mu$ c.s. cuando $n\rightarrow\infty$, donde $\mu\leq\infty$ es una constante o variable aleatoria. Sea $a$ una constante o variable aleatoria que puede ser infinita cuando $\mu$ es finita, y considere las expresiones l\'imite:
\begin{eqnarray}
lim_{n\rightarrow\infty}n^{-1}Z\left(T_{n}\right)&=&a,\textrm{ c.s.}\\
lim_{t\rightarrow\infty}t^{-1}Z\left(t\right)&=&a/\mu,\textrm{ c.s.}
\end{eqnarray}
La segunda expresi\'on implica la primera. Conversamente, la primera implica la segunda si el proceso $Z\left(t\right)$ es creciente, o si $lim_{n\rightarrow\infty}n^{-1}M_{n}=0$ c.s.
\end{Teo}

\begin{Coro}
Si $N\left(t\right)$ es un proceso de renovaci\'on, y $\left(Z\left(T_{n}\right)-Z\left(T_{n-1}\right),M_{n}\right)$, para $n\geq1$, son variables aleatorias independientes e id\'enticamente distribuidas con media finita, entonces,
\begin{eqnarray}
lim_{t\rightarrow\infty}t^{-1}Z\left(t\right)\rightarrow\frac{\esp\left[Z\left(T_{1}\right)-Z\left(T_{0}\right)\right]}{\esp\left[T_{1}\right]},\textrm{ c.s. cuando  }t\rightarrow\infty.
\end{eqnarray}
\end{Coro}


%___________________________________________________________________________________________
%
%\subsection*{Funci\'on de Renovaci\'on}
%___________________________________________________________________________________________
%


\begin{Def}
Sea $h\left(t\right)$ funci\'on de valores reales en $\rea$ acotada en intervalos finitos e igual a cero para $t<0$ La ecuaci\'on de renovaci\'on para $h\left(t\right)$ y la distribuci\'on $F$ es

\begin{eqnarray}\label{Ec.Renovacion}
H\left(t\right)=h\left(t\right)+\int_{\left[0,t\right]}H\left(t-s\right)dF\left(s\right)\textrm{,    }t\geq0,
\end{eqnarray}
donde $H\left(t\right)$ es una funci\'on de valores reales. Esto es $H=h+F\star H$. Decimos que $H\left(t\right)$ es soluci\'on de esta ecuaci\'on si satisface la ecuaci\'on, y es acotada en intervalos finitos e iguales a cero para $t<0$.
\end{Def}

\begin{Prop}
La funci\'on $U\star h\left(t\right)$ es la \'unica soluci\'on de la ecuaci\'on de renovaci\'on (\ref{Ec.Renovacion}).
\end{Prop}

\begin{Teo}[Teorema Renovaci\'on Elemental]
\begin{eqnarray*}
t^{-1}U\left(t\right)\rightarrow 1/\mu\textrm{,    cuando }t\rightarrow\infty.
\end{eqnarray*}
\end{Teo}

%___________________________________________________________________________________________
%
%\subsection{Funci\'on de Renovaci\'on}
%___________________________________________________________________________________________
%


Sup\'ongase que $N\left(t\right)$ es un proceso de renovaci\'on con distribuci\'on $F$ con media finita $\mu$.

\begin{Def}
La funci\'on de renovaci\'on asociada con la distribuci\'on $F$, del proceso $N\left(t\right)$, es
\begin{eqnarray*}
U\left(t\right)=\sum_{n=1}^{\infty}F^{n\star}\left(t\right),\textrm{   }t\geq0,
\end{eqnarray*}
donde $F^{0\star}\left(t\right)=\indora\left(t\geq0\right)$.
\end{Def}


\begin{Prop}
Sup\'ongase que la distribuci\'on de inter-renovaci\'on $F$ tiene densidad $f$. Entonces $U\left(t\right)$ tambi\'en tiene densidad, para $t>0$, y es $U^{'}\left(t\right)=\sum_{n=0}^{\infty}f^{n\star}\left(t\right)$. Adem\'as
\begin{eqnarray*}
\prob\left\{N\left(t\right)>N\left(t-\right)\right\}=0\textrm{,   }t\geq0.
\end{eqnarray*}
\end{Prop}

\begin{Def}
La Transformada de Laplace-Stieljes de $F$ est\'a dada por

\begin{eqnarray*}
\hat{F}\left(\alpha\right)=\int_{\rea_{+}}e^{-\alpha t}dF\left(t\right)\textrm{,  }\alpha\geq0.
\end{eqnarray*}
\end{Def}

Entonces

\begin{eqnarray*}
\hat{U}\left(\alpha\right)=\sum_{n=0}^{\infty}\hat{F^{n\star}}\left(\alpha\right)=\sum_{n=0}^{\infty}\hat{F}\left(\alpha\right)^{n}=\frac{1}{1-\hat{F}\left(\alpha\right)}.
\end{eqnarray*}


\begin{Prop}
La Transformada de Laplace $\hat{U}\left(\alpha\right)$ y $\hat{F}\left(\alpha\right)$ determina una a la otra de manera \'unica por la relaci\'on $\hat{U}\left(\alpha\right)=\frac{1}{1-\hat{F}\left(\alpha\right)}$.
\end{Prop}


\begin{Note}
Un proceso de renovaci\'on $N\left(t\right)$ cuyos tiempos de inter-renovaci\'on tienen media finita, es un proceso Poisson con tasa $\lambda$ si y s\'olo s\'i $\esp\left[U\left(t\right)\right]=\lambda t$, para $t\geq0$.
\end{Note}


\begin{Teo}
Sea $N\left(t\right)$ un proceso puntual simple con puntos de localizaci\'on $T_{n}$ tal que $\eta\left(t\right)=\esp\left[N\left(\right)\right]$ es finita para cada $t$. Entonces para cualquier funci\'on $f:\rea_{+}\rightarrow\rea$,
\begin{eqnarray*}
\esp\left[\sum_{n=1}^{N\left(\right)}f\left(T_{n}\right)\right]=\int_{\left(0,t\right]}f\left(s\right)d\eta\left(s\right)\textrm{,  }t\geq0,
\end{eqnarray*}
suponiendo que la integral exista. Adem\'as si $X_{1},X_{2},\ldots$ son variables aleatorias definidas en el mismo espacio de probabilidad que el proceso $N\left(t\right)$ tal que $\esp\left[X_{n}|T_{n}=s\right]=f\left(s\right)$, independiente de $n$. Entonces
\begin{eqnarray*}
\esp\left[\sum_{n=1}^{N\left(t\right)}X_{n}\right]=\int_{\left(0,t\right]}f\left(s\right)d\eta\left(s\right)\textrm{,  }t\geq0,
\end{eqnarray*} 
suponiendo que la integral exista. 
\end{Teo}

\begin{Coro}[Identidad de Wald para Renovaciones]
Para el proceso de renovaci\'on $N\left(t\right)$,
\begin{eqnarray*}
\esp\left[T_{N\left(t\right)+1}\right]=\mu\esp\left[N\left(t\right)+1\right]\textrm{,  }t\geq0,
\end{eqnarray*}  
\end{Coro}

%______________________________________________________________________
%\subsection{Procesos de Renovaci\'on}
%______________________________________________________________________

\begin{Def}\label{Def.Tn}
Sean $0\leq T_{1}\leq T_{2}\leq \ldots$ son tiempos aleatorios infinitos en los cuales ocurren ciertos eventos. El n\'umero de tiempos $T_{n}$ en el intervalo $\left[0,t\right)$ es

\begin{eqnarray}
N\left(t\right)=\sum_{n=1}^{\infty}\indora\left(T_{n}\leq t\right),
\end{eqnarray}
para $t\geq0$.
\end{Def}

Si se consideran los puntos $T_{n}$ como elementos de $\rea_{+}$, y $N\left(t\right)$ es el n\'umero de puntos en $\rea$. El proceso denotado por $\left\{N\left(t\right):t\geq0\right\}$, denotado por $N\left(t\right)$, es un proceso puntual en $\rea_{+}$. Los $T_{n}$ son los tiempos de ocurrencia, el proceso puntual $N\left(t\right)$ es simple si su n\'umero de ocurrencias son distintas: $0<T_{1}<T_{2}<\ldots$ casi seguramente.

\begin{Def}
Un proceso puntual $N\left(t\right)$ es un proceso de renovaci\'on si los tiempos de interocurrencia $\xi_{n}=T_{n}-T_{n-1}$, para $n\geq1$, son independientes e identicamente distribuidos con distribuci\'on $F$, donde $F\left(0\right)=0$ y $T_{0}=0$. Los $T_{n}$ son llamados tiempos de renovaci\'on, referente a la independencia o renovaci\'on de la informaci\'on estoc\'astica en estos tiempos. Los $\xi_{n}$ son los tiempos de inter-renovaci\'on, y $N\left(t\right)$ es el n\'umero de renovaciones en el intervalo $\left[0,t\right)$
\end{Def}


\begin{Note}
Para definir un proceso de renovaci\'on para cualquier contexto, solamente hay que especificar una distribuci\'on $F$, con $F\left(0\right)=0$, para los tiempos de inter-renovaci\'on. La funci\'on $F$ en turno degune las otra variables aleatorias. De manera formal, existe un espacio de probabilidad y una sucesi\'on de variables aleatorias $\xi_{1},\xi_{2},\ldots$ definidas en este con distribuci\'on $F$. Entonces las otras cantidades son $T_{n}=\sum_{k=1}^{n}\xi_{k}$ y $N\left(t\right)=\sum_{n=1}^{\infty}\indora\left(T_{n}\leq t\right)$, donde $T_{n}\rightarrow\infty$ casi seguramente por la Ley Fuerte de los Grandes Números.
\end{Note}

\begin{Def}\label{Def.Tn}
Sean $0\leq T_{1}\leq T_{2}\leq \ldots$ son tiempos aleatorios infinitos en los cuales ocurren ciertos eventos. El n\'umero de tiempos $T_{n}$ en el intervalo $\left[0,t\right)$ es

\begin{eqnarray}
N\left(t\right)=\sum_{n=1}^{\infty}\indora\left(T_{n}\leq t\right),
\end{eqnarray}
para $t\geq0$.
\end{Def}

Si se consideran los puntos $T_{n}$ como elementos de $\rea_{+}$, y $N\left(t\right)$ es el n\'umero de puntos en $\rea$. El proceso denotado por $\left\{N\left(t\right):t\geq0\right\}$, denotado por $N\left(t\right)$, es un proceso puntual en $\rea_{+}$. Los $T_{n}$ son los tiempos de ocurrencia, el proceso puntual $N\left(t\right)$ es simple si su n\'umero de ocurrencias son distintas: $0<T_{1}<T_{2}<\ldots$ casi seguramente.

\begin{Def}
Un proceso puntual $N\left(t\right)$ es un proceso de renovaci\'on si los tiempos de interocurrencia $\xi_{n}=T_{n}-T_{n-1}$, para $n\geq1$, son independientes e identicamente distribuidos con distribuci\'on $F$, donde $F\left(0\right)=0$ y $T_{0}=0$. Los $T_{n}$ son llamados tiempos de renovaci\'on, referente a la independencia o renovaci\'on de la informaci\'on estoc\'astica en estos tiempos. Los $\xi_{n}$ son los tiempos de inter-renovaci\'on, y $N\left(t\right)$ es el n\'umero de renovaciones en el intervalo $\left[0,t\right)$
\end{Def}


\begin{Note}
Para definir un proceso de renovaci\'on para cualquier contexto, solamente hay que especificar una distribuci\'on $F$, con $F\left(0\right)=0$, para los tiempos de inter-renovaci\'on. La funci\'on $F$ en turno degune las otra variables aleatorias. De manera formal, existe un espacio de probabilidad y una sucesi\'on de variables aleatorias $\xi_{1},\xi_{2},\ldots$ definidas en este con distribuci\'on $F$. Entonces las otras cantidades son $T_{n}=\sum_{k=1}^{n}\xi_{k}$ y $N\left(t\right)=\sum_{n=1}^{\infty}\indora\left(T_{n}\leq t\right)$, donde $T_{n}\rightarrow\infty$ casi seguramente por la Ley Fuerte de los Grandes N\'umeros.
\end{Note}







Los tiempos $T_{n}$ est\'an relacionados con los conteos de $N\left(t\right)$ por

\begin{eqnarray*}
\left\{N\left(t\right)\geq n\right\}&=&\left\{T_{n}\leq t\right\}\\
T_{N\left(t\right)}\leq &t&<T_{N\left(t\right)+1},
\end{eqnarray*}

adem\'as $N\left(T_{n}\right)=n$, y 

\begin{eqnarray*}
N\left(t\right)=\max\left\{n:T_{n}\leq t\right\}=\min\left\{n:T_{n+1}>t\right\}
\end{eqnarray*}

Por propiedades de la convoluci\'on se sabe que

\begin{eqnarray*}
P\left\{T_{n}\leq t\right\}=F^{n\star}\left(t\right)
\end{eqnarray*}
que es la $n$-\'esima convoluci\'on de $F$. Entonces 

\begin{eqnarray*}
\left\{N\left(t\right)\geq n\right\}&=&\left\{T_{n}\leq t\right\}\\
P\left\{N\left(t\right)\leq n\right\}&=&1-F^{\left(n+1\right)\star}\left(t\right)
\end{eqnarray*}

Adem\'as usando el hecho de que $\esp\left[N\left(t\right)\right]=\sum_{n=1}^{\infty}P\left\{N\left(t\right)\geq n\right\}$
se tiene que

\begin{eqnarray*}
\esp\left[N\left(t\right)\right]=\sum_{n=1}^{\infty}F^{n\star}\left(t\right)
\end{eqnarray*}

\begin{Prop}
Para cada $t\geq0$, la funci\'on generadora de momentos $\esp\left[e^{\alpha N\left(t\right)}\right]$ existe para alguna $\alpha$ en una vecindad del 0, y de aqu\'i que $\esp\left[N\left(t\right)^{m}\right]<\infty$, para $m\geq1$.
\end{Prop}

\begin{Ejem}[\textbf{Proceso Poisson}]

Suponga que se tienen tiempos de inter-renovaci\'on \textit{i.i.d.} del proceso de renovaci\'on $N\left(t\right)$ tienen distribuci\'on exponencial $F\left(t\right)=q-e^{-\lambda t}$ con tasa $\lambda$. Entonces $N\left(t\right)$ es un proceso Poisson con tasa $\lambda$.

\end{Ejem}


\begin{Note}
Si el primer tiempo de renovaci\'on $\xi_{1}$ no tiene la misma distribuci\'on que el resto de las $\xi_{n}$, para $n\geq2$, a $N\left(t\right)$ se le llama Proceso de Renovaci\'on retardado, donde si $\xi$ tiene distribuci\'on $G$, entonces el tiempo $T_{n}$ de la $n$-\'esima renovaci\'on tiene distribuci\'on $G\star F^{\left(n-1\right)\star}\left(t\right)$
\end{Note}


\begin{Teo}
Para una constante $\mu\leq\infty$ ( o variable aleatoria), las siguientes expresiones son equivalentes:

\begin{eqnarray}
lim_{n\rightarrow\infty}n^{-1}T_{n}&=&\mu,\textrm{ c.s.}\\
lim_{t\rightarrow\infty}t^{-1}N\left(t\right)&=&1/\mu,\textrm{ c.s.}
\end{eqnarray}
\end{Teo}


Es decir, $T_{n}$ satisface la Ley Fuerte de los Grandes N\'umeros s\'i y s\'olo s\'i $N\left/t\right)$ la cumple.


\begin{Coro}[Ley Fuerte de los Grandes N\'umeros para Procesos de Renovaci\'on]
Si $N\left(t\right)$ es un proceso de renovaci\'on cuyos tiempos de inter-renovaci\'on tienen media $\mu\leq\infty$, entonces
\begin{eqnarray}
t^{-1}N\left(t\right)\rightarrow 1/\mu,\textrm{ c.s. cuando }t\rightarrow\infty.
\end{eqnarray}

\end{Coro}


Considerar el proceso estoc\'astico de valores reales $\left\{Z\left(t\right):t\geq0\right\}$ en el mismo espacio de probabilidad que $N\left(t\right)$

\begin{Def}
Para el proceso $\left\{Z\left(t\right):t\geq0\right\}$ se define la fluctuaci\'on m\'axima de $Z\left(t\right)$ en el intervalo $\left(T_{n-1},T_{n}\right]$:
\begin{eqnarray*}
M_{n}=\sup_{T_{n-1}<t\leq T_{n}}|Z\left(t\right)-Z\left(T_{n-1}\right)|
\end{eqnarray*}
\end{Def}

\begin{Teo}
Sup\'ongase que $n^{-1}T_{n}\rightarrow\mu$ c.s. cuando $n\rightarrow\infty$, donde $\mu\leq\infty$ es una constante o variable aleatoria. Sea $a$ una constante o variable aleatoria que puede ser infinita cuando $\mu$ es finita, y considere las expresiones l\'imite:
\begin{eqnarray}
lim_{n\rightarrow\infty}n^{-1}Z\left(T_{n}\right)&=&a,\textrm{ c.s.}\\
lim_{t\rightarrow\infty}t^{-1}Z\left(t\right)&=&a/\mu,\textrm{ c.s.}
\end{eqnarray}
La segunda expresi\'on implica la primera. Conversamente, la primera implica la segunda si el proceso $Z\left(t\right)$ es creciente, o si $lim_{n\rightarrow\infty}n^{-1}M_{n}=0$ c.s.
\end{Teo}

\begin{Coro}
Si $N\left(t\right)$ es un proceso de renovaci\'on, y $\left(Z\left(T_{n}\right)-Z\left(T_{n-1}\right),M_{n}\right)$, para $n\geq1$, son variables aleatorias independientes e id\'enticamente distribuidas con media finita, entonces,
\begin{eqnarray}
lim_{t\rightarrow\infty}t^{-1}Z\left(t\right)\rightarrow\frac{\esp\left[Z\left(T_{1}\right)-Z\left(T_{0}\right)\right]}{\esp\left[T_{1}\right]},\textrm{ c.s. cuando  }t\rightarrow\infty.
\end{eqnarray}
\end{Coro}


Sup\'ongase que $N\left(t\right)$ es un proceso de renovaci\'on con distribuci\'on $F$ con media finita $\mu$.

\begin{Def}
La funci\'on de renovaci\'on asociada con la distribuci\'on $F$, del proceso $N\left(t\right)$, es
\begin{eqnarray*}
U\left(t\right)=\sum_{n=1}^{\infty}F^{n\star}\left(t\right),\textrm{   }t\geq0,
\end{eqnarray*}
donde $F^{0\star}\left(t\right)=\indora\left(t\geq0\right)$.
\end{Def}


\begin{Prop}
Sup\'ongase que la distribuci\'on de inter-renovaci\'on $F$ tiene densidad $f$. Entonces $U\left(t\right)$ tambi\'en tiene densidad, para $t>0$, y es $U^{'}\left(t\right)=\sum_{n=0}^{\infty}f^{n\star}\left(t\right)$. Adem\'as
\begin{eqnarray*}
\prob\left\{N\left(t\right)>N\left(t-\right)\right\}=0\textrm{,   }t\geq0.
\end{eqnarray*}
\end{Prop}

\begin{Def}
La Transformada de Laplace-Stieljes de $F$ est\'a dada por

\begin{eqnarray*}
\hat{F}\left(\alpha\right)=\int_{\rea_{+}}e^{-\alpha t}dF\left(t\right)\textrm{,  }\alpha\geq0.
\end{eqnarray*}
\end{Def}

Entonces

\begin{eqnarray*}
\hat{U}\left(\alpha\right)=\sum_{n=0}^{\infty}\hat{F^{n\star}}\left(\alpha\right)=\sum_{n=0}^{\infty}\hat{F}\left(\alpha\right)^{n}=\frac{1}{1-\hat{F}\left(\alpha\right)}.
\end{eqnarray*}


\begin{Prop}
La Transformada de Laplace $\hat{U}\left(\alpha\right)$ y $\hat{F}\left(\alpha\right)$ determina una a la otra de manera \'unica por la relaci\'on $\hat{U}\left(\alpha\right)=\frac{1}{1-\hat{F}\left(\alpha\right)}$.
\end{Prop}


\begin{Note}
Un proceso de renovaci\'on $N\left(t\right)$ cuyos tiempos de inter-renovaci\'on tienen media finita, es un proceso Poisson con tasa $\lambda$ si y s\'olo s\'i $\esp\left[U\left(t\right)\right]=\lambda t$, para $t\geq0$.
\end{Note}


\begin{Teo}
Sea $N\left(t\right)$ un proceso puntual simple con puntos de localizaci\'on $T_{n}$ tal que $\eta\left(t\right)=\esp\left[N\left(\right)\right]$ es finita para cada $t$. Entonces para cualquier funci\'on $f:\rea_{+}\rightarrow\rea$,
\begin{eqnarray*}
\esp\left[\sum_{n=1}^{N\left(\right)}f\left(T_{n}\right)\right]=\int_{\left(0,t\right]}f\left(s\right)d\eta\left(s\right)\textrm{,  }t\geq0,
\end{eqnarray*}
suponiendo que la integral exista. Adem\'as si $X_{1},X_{2},\ldots$ son variables aleatorias definidas en el mismo espacio de probabilidad que el proceso $N\left(t\right)$ tal que $\esp\left[X_{n}|T_{n}=s\right]=f\left(s\right)$, independiente de $n$. Entonces
\begin{eqnarray*}
\esp\left[\sum_{n=1}^{N\left(t\right)}X_{n}\right]=\int_{\left(0,t\right]}f\left(s\right)d\eta\left(s\right)\textrm{,  }t\geq0,
\end{eqnarray*} 
suponiendo que la integral exista. 
\end{Teo}

\begin{Coro}[Identidad de Wald para Renovaciones]
Para el proceso de renovaci\'on $N\left(t\right)$,
\begin{eqnarray*}
\esp\left[T_{N\left(t\right)+1}\right]=\mu\esp\left[N\left(t\right)+1\right]\textrm{,  }t\geq0,
\end{eqnarray*}  
\end{Coro}


\begin{Def}
Sea $h\left(t\right)$ funci\'on de valores reales en $\rea$ acotada en intervalos finitos e igual a cero para $t<0$ La ecuaci\'on de renovaci\'on para $h\left(t\right)$ y la distribuci\'on $F$ es

\begin{eqnarray}\label{Ec.Renovacion}
H\left(t\right)=h\left(t\right)+\int_{\left[0,t\right]}H\left(t-s\right)dF\left(s\right)\textrm{,    }t\geq0,
\end{eqnarray}
donde $H\left(t\right)$ es una funci\'on de valores reales. Esto es $H=h+F\star H$. Decimos que $H\left(t\right)$ es soluci\'on de esta ecuaci\'on si satisface la ecuaci\'on, y es acotada en intervalos finitos e iguales a cero para $t<0$.
\end{Def}

\begin{Prop}
La funci\'on $U\star h\left(t\right)$ es la \'unica soluci\'on de la ecuaci\'on de renovaci\'on (\ref{Ec.Renovacion}).
\end{Prop}

\begin{Teo}[Teorema Renovaci\'on Elemental]
\begin{eqnarray*}
t^{-1}U\left(t\right)\rightarrow 1/\mu\textrm{,    cuando }t\rightarrow\infty.
\end{eqnarray*}
\end{Teo}



Sup\'ongase que $N\left(t\right)$ es un proceso de renovaci\'on con distribuci\'on $F$ con media finita $\mu$.

\begin{Def}
La funci\'on de renovaci\'on asociada con la distribuci\'on $F$, del proceso $N\left(t\right)$, es
\begin{eqnarray*}
U\left(t\right)=\sum_{n=1}^{\infty}F^{n\star}\left(t\right),\textrm{   }t\geq0,
\end{eqnarray*}
donde $F^{0\star}\left(t\right)=\indora\left(t\geq0\right)$.
\end{Def}


\begin{Prop}
Sup\'ongase que la distribuci\'on de inter-renovaci\'on $F$ tiene densidad $f$. Entonces $U\left(t\right)$ tambi\'en tiene densidad, para $t>0$, y es $U^{'}\left(t\right)=\sum_{n=0}^{\infty}f^{n\star}\left(t\right)$. Adem\'as
\begin{eqnarray*}
\prob\left\{N\left(t\right)>N\left(t-\right)\right\}=0\textrm{,   }t\geq0.
\end{eqnarray*}
\end{Prop}

\begin{Def}
La Transformada de Laplace-Stieljes de $F$ est\'a dada por

\begin{eqnarray*}
\hat{F}\left(\alpha\right)=\int_{\rea_{+}}e^{-\alpha t}dF\left(t\right)\textrm{,  }\alpha\geq0.
\end{eqnarray*}
\end{Def}

Entonces

\begin{eqnarray*}
\hat{U}\left(\alpha\right)=\sum_{n=0}^{\infty}\hat{F^{n\star}}\left(\alpha\right)=\sum_{n=0}^{\infty}\hat{F}\left(\alpha\right)^{n}=\frac{1}{1-\hat{F}\left(\alpha\right)}.
\end{eqnarray*}


\begin{Prop}
La Transformada de Laplace $\hat{U}\left(\alpha\right)$ y $\hat{F}\left(\alpha\right)$ determina una a la otra de manera \'unica por la relaci\'on $\hat{U}\left(\alpha\right)=\frac{1}{1-\hat{F}\left(\alpha\right)}$.
\end{Prop}


\begin{Note}
Un proceso de renovaci\'on $N\left(t\right)$ cuyos tiempos de inter-renovaci\'on tienen media finita, es un proceso Poisson con tasa $\lambda$ si y s\'olo s\'i $\esp\left[U\left(t\right)\right]=\lambda t$, para $t\geq0$.
\end{Note}


\begin{Teo}
Sea $N\left(t\right)$ un proceso puntual simple con puntos de localizaci\'on $T_{n}$ tal que $\eta\left(t\right)=\esp\left[N\left(\right)\right]$ es finita para cada $t$. Entonces para cualquier funci\'on $f:\rea_{+}\rightarrow\rea$,
\begin{eqnarray*}
\esp\left[\sum_{n=1}^{N\left(\right)}f\left(T_{n}\right)\right]=\int_{\left(0,t\right]}f\left(s\right)d\eta\left(s\right)\textrm{,  }t\geq0,
\end{eqnarray*}
suponiendo que la integral exista. Adem\'as si $X_{1},X_{2},\ldots$ son variables aleatorias definidas en el mismo espacio de probabilidad que el proceso $N\left(t\right)$ tal que $\esp\left[X_{n}|T_{n}=s\right]=f\left(s\right)$, independiente de $n$. Entonces
\begin{eqnarray*}
\esp\left[\sum_{n=1}^{N\left(t\right)}X_{n}\right]=\int_{\left(0,t\right]}f\left(s\right)d\eta\left(s\right)\textrm{,  }t\geq0,
\end{eqnarray*} 
suponiendo que la integral exista. 
\end{Teo}

\begin{Coro}[Identidad de Wald para Renovaciones]
Para el proceso de renovaci\'on $N\left(t\right)$,
\begin{eqnarray*}
\esp\left[T_{N\left(t\right)+1}\right]=\mu\esp\left[N\left(t\right)+1\right]\textrm{,  }t\geq0,
\end{eqnarray*}  
\end{Coro}


\begin{Def}
Sea $h\left(t\right)$ funci\'on de valores reales en $\rea$ acotada en intervalos finitos e igual a cero para $t<0$ La ecuaci\'on de renovaci\'on para $h\left(t\right)$ y la distribuci\'on $F$ es

\begin{eqnarray}\label{Ec.Renovacion}
H\left(t\right)=h\left(t\right)+\int_{\left[0,t\right]}H\left(t-s\right)dF\left(s\right)\textrm{,    }t\geq0,
\end{eqnarray}
donde $H\left(t\right)$ es una funci\'on de valores reales. Esto es $H=h+F\star H$. Decimos que $H\left(t\right)$ es soluci\'on de esta ecuaci\'on si satisface la ecuaci\'on, y es acotada en intervalos finitos e iguales a cero para $t<0$.
\end{Def}

\begin{Prop}
La funci\'on $U\star h\left(t\right)$ es la \'unica soluci\'on de la ecuaci\'on de renovaci\'on (\ref{Ec.Renovacion}).
\end{Prop}

\begin{Teo}[Teorema Renovaci\'on Elemental]
\begin{eqnarray*}
t^{-1}U\left(t\right)\rightarrow 1/\mu\textrm{,    cuando }t\rightarrow\infty.
\end{eqnarray*}
\end{Teo}


\begin{Note} Una funci\'on $h:\rea_{+}\rightarrow\rea$ es Directamente Riemann Integrable en los siguientes casos:
\begin{itemize}
\item[a)] $h\left(t\right)\geq0$ es decreciente y Riemann Integrable.
\item[b)] $h$ es continua excepto posiblemente en un conjunto de Lebesgue de medida 0, y $|h\left(t\right)|\leq b\left(t\right)$, donde $b$ es DRI.
\end{itemize}
\end{Note}

\begin{Teo}[Teorema Principal de Renovaci\'on]
Si $F$ es no aritm\'etica y $h\left(t\right)$ es Directamente Riemann Integrable (DRI), entonces

\begin{eqnarray*}
lim_{t\rightarrow\infty}U\star h=\frac{1}{\mu}\int_{\rea_{+}}h\left(s\right)ds.
\end{eqnarray*}
\end{Teo}

\begin{Prop}
Cualquier funci\'on $H\left(t\right)$ acotada en intervalos finitos y que es 0 para $t<0$ puede expresarse como
\begin{eqnarray*}
H\left(t\right)=U\star h\left(t\right)\textrm{,  donde }h\left(t\right)=H\left(t\right)-F\star H\left(t\right)
\end{eqnarray*}
\end{Prop}

\begin{Def}
Un proceso estoc\'astico $X\left(t\right)$ es crudamente regenerativo en un tiempo aleatorio positivo $T$ si
\begin{eqnarray*}
\esp\left[X\left(T+t\right)|T\right]=\esp\left[X\left(t\right)\right]\textrm{, para }t\geq0,\end{eqnarray*}
y con las esperanzas anteriores finitas.
\end{Def}

\begin{Prop}
Sup\'ongase que $X\left(t\right)$ es un proceso crudamente regenerativo en $T$, que tiene distribuci\'on $F$. Si $\esp\left[X\left(t\right)\right]$ es acotado en intervalos finitos, entonces
\begin{eqnarray*}
\esp\left[X\left(t\right)\right]=U\star h\left(t\right)\textrm{,  donde }h\left(t\right)=\esp\left[X\left(t\right)\indora\left(T>t\right)\right].
\end{eqnarray*}
\end{Prop}

\begin{Teo}[Regeneraci\'on Cruda]
Sup\'ongase que $X\left(t\right)$ es un proceso con valores positivo crudamente regenerativo en $T$, y def\'inase $M=\sup\left\{|X\left(t\right)|:t\leq T\right\}$. Si $T$ es no aritm\'etico y $M$ y $MT$ tienen media finita, entonces
\begin{eqnarray*}
lim_{t\rightarrow\infty}\esp\left[X\left(t\right)\right]=\frac{1}{\mu}\int_{\rea_{+}}h\left(s\right)ds,
\end{eqnarray*}
donde $h\left(t\right)=\esp\left[X\left(t\right)\indora\left(T>t\right)\right]$.
\end{Teo}


\begin{Note} Una funci\'on $h:\rea_{+}\rightarrow\rea$ es Directamente Riemann Integrable en los siguientes casos:
\begin{itemize}
\item[a)] $h\left(t\right)\geq0$ es decreciente y Riemann Integrable.
\item[b)] $h$ es continua excepto posiblemente en un conjunto de Lebesgue de medida 0, y $|h\left(t\right)|\leq b\left(t\right)$, donde $b$ es DRI.
\end{itemize}
\end{Note}

\begin{Teo}[Teorema Principal de Renovaci\'on]
Si $F$ es no aritm\'etica y $h\left(t\right)$ es Directamente Riemann Integrable (DRI), entonces

\begin{eqnarray*}
lim_{t\rightarrow\infty}U\star h=\frac{1}{\mu}\int_{\rea_{+}}h\left(s\right)ds.
\end{eqnarray*}
\end{Teo}

\begin{Prop}
Cualquier funci\'on $H\left(t\right)$ acotada en intervalos finitos y que es 0 para $t<0$ puede expresarse como
\begin{eqnarray*}
H\left(t\right)=U\star h\left(t\right)\textrm{,  donde }h\left(t\right)=H\left(t\right)-F\star H\left(t\right)
\end{eqnarray*}
\end{Prop}

\begin{Def}
Un proceso estoc\'astico $X\left(t\right)$ es crudamente regenerativo en un tiempo aleatorio positivo $T$ si
\begin{eqnarray*}
\esp\left[X\left(T+t\right)|T\right]=\esp\left[X\left(t\right)\right]\textrm{, para }t\geq0,\end{eqnarray*}
y con las esperanzas anteriores finitas.
\end{Def}

\begin{Prop}
Sup\'ongase que $X\left(t\right)$ es un proceso crudamente regenerativo en $T$, que tiene distribuci\'on $F$. Si $\esp\left[X\left(t\right)\right]$ es acotado en intervalos finitos, entonces
\begin{eqnarray*}
\esp\left[X\left(t\right)\right]=U\star h\left(t\right)\textrm{,  donde }h\left(t\right)=\esp\left[X\left(t\right)\indora\left(T>t\right)\right].
\end{eqnarray*}
\end{Prop}

\begin{Teo}[Regeneraci\'on Cruda]
Sup\'ongase que $X\left(t\right)$ es un proceso con valores positivo crudamente regenerativo en $T$, y def\'inase $M=\sup\left\{|X\left(t\right)|:t\leq T\right\}$. Si $T$ es no aritm\'etico y $M$ y $MT$ tienen media finita, entonces
\begin{eqnarray*}
lim_{t\rightarrow\infty}\esp\left[X\left(t\right)\right]=\frac{1}{\mu}\int_{\rea_{+}}h\left(s\right)ds,
\end{eqnarray*}
donde $h\left(t\right)=\esp\left[X\left(t\right)\indora\left(T>t\right)\right]$.
\end{Teo}

\begin{Def}
Para el proceso $\left\{\left(N\left(t\right),X\left(t\right)\right):t\geq0\right\}$, sus trayectoria muestrales en el intervalo de tiempo $\left[T_{n-1},T_{n}\right)$ est\'an descritas por
\begin{eqnarray*}
\zeta_{n}=\left(\xi_{n},\left\{X\left(T_{n-1}+t\right):0\leq t<\xi_{n}\right\}\right)
\end{eqnarray*}
Este $\zeta_{n}$ es el $n$-\'esimo segmento del proceso. El proceso es regenerativo sobre los tiempos $T_{n}$ si sus segmentos $\zeta_{n}$ son independientes e id\'enticamennte distribuidos.
\end{Def}


\begin{Note}
Si $\tilde{X}\left(t\right)$ con espacio de estados $\tilde{S}$ es regenerativo sobre $T_{n}$, entonces $X\left(t\right)=f\left(\tilde{X}\left(t\right)\right)$ tambi\'en es regenerativo sobre $T_{n}$, para cualquier funci\'on $f:\tilde{S}\rightarrow S$.
\end{Note}

\begin{Note}
Los procesos regenerativos son crudamente regenerativos, pero no al rev\'es.
\end{Note}


\begin{Note}
Un proceso estoc\'astico a tiempo continuo o discreto es regenerativo si existe un proceso de renovaci\'on  tal que los segmentos del proceso entre tiempos de renovaci\'on sucesivos son i.i.d., es decir, para $\left\{X\left(t\right):t\geq0\right\}$ proceso estoc\'astico a tiempo continuo con espacio de estados $S$, espacio m\'etrico.
\end{Note}

Para $\left\{X\left(t\right):t\geq0\right\}$ Proceso Estoc\'astico a tiempo continuo con estado de espacios $S$, que es un espacio m\'etrico, con trayectorias continuas por la derecha y con l\'imites por la izquierda c.s. Sea $N\left(t\right)$ un proceso de renovaci\'on en $\rea_{+}$ definido en el mismo espacio de probabilidad que $X\left(t\right)$, con tiempos de renovaci\'on $T$ y tiempos de inter-renovaci\'on $\xi_{n}=T_{n}-T_{n-1}$, con misma distribuci\'on $F$ de media finita $\mu$.



\begin{Def}
Para el proceso $\left\{\left(N\left(t\right),X\left(t\right)\right):t\geq0\right\}$, sus trayectoria muestrales en el intervalo de tiempo $\left[T_{n-1},T_{n}\right)$ est\'an descritas por
\begin{eqnarray*}
\zeta_{n}=\left(\xi_{n},\left\{X\left(T_{n-1}+t\right):0\leq t<\xi_{n}\right\}\right)
\end{eqnarray*}
Este $\zeta_{n}$ es el $n$-\'esimo segmento del proceso. El proceso es regenerativo sobre los tiempos $T_{n}$ si sus segmentos $\zeta_{n}$ son independientes e id\'enticamennte distribuidos.
\end{Def}

\begin{Note}
Un proceso regenerativo con media de la longitud de ciclo finita es llamado positivo recurrente.
\end{Note}

\begin{Teo}[Procesos Regenerativos]
Suponga que el proceso
\end{Teo}


\begin{Def}[Renewal Process Trinity]
Para un proceso de renovaci\'on $N\left(t\right)$, los siguientes procesos proveen de informaci\'on sobre los tiempos de renovaci\'on.
\begin{itemize}
\item $A\left(t\right)=t-T_{N\left(t\right)}$, el tiempo de recurrencia hacia atr\'as al tiempo $t$, que es el tiempo desde la \'ultima renovaci\'on para $t$.

\item $B\left(t\right)=T_{N\left(t\right)+1}-t$, el tiempo de recurrencia hacia adelante al tiempo $t$, residual del tiempo de renovaci\'on, que es el tiempo para la pr\'oxima renovaci\'on despu\'es de $t$.

\item $L\left(t\right)=\xi_{N\left(t\right)+1}=A\left(t\right)+B\left(t\right)$, la longitud del intervalo de renovaci\'on que contiene a $t$.
\end{itemize}
\end{Def}

\begin{Note}
El proceso tridimensional $\left(A\left(t\right),B\left(t\right),L\left(t\right)\right)$ es regenerativo sobre $T_{n}$, y por ende cada proceso lo es. Cada proceso $A\left(t\right)$ y $B\left(t\right)$ son procesos de MArkov a tiempo continuo con trayectorias continuas por partes en el espacio de estados $\rea_{+}$. Una expresi\'on conveniente para su distribuci\'on conjunta es, para $0\leq x<t,y\geq0$
\begin{equation}\label{NoRenovacion}
P\left\{A\left(t\right)>x,B\left(t\right)>y\right\}=
P\left\{N\left(t+y\right)-N\left((t-x)\right)=0\right\}
\end{equation}
\end{Note}

\begin{Ejem}[Tiempos de recurrencia Poisson]
Si $N\left(t\right)$ es un proceso Poisson con tasa $\lambda$, entonces de la expresi\'on (\ref{NoRenovacion}) se tiene que

\begin{eqnarray*}
\begin{array}{lc}
P\left\{A\left(t\right)>x,B\left(t\right)>y\right\}=e^{-\lambda\left(x+y\right)},&0\leq x<t,y\geq0,
\end{array}
\end{eqnarray*}
que es la probabilidad Poisson de no renovaciones en un intervalo de longitud $x+y$.

\end{Ejem}

\begin{Note}
Una cadena de Markov erg\'odica tiene la propiedad de ser estacionaria si la distribuci\'on de su estado al tiempo $0$ es su distribuci\'on estacionaria.
\end{Note}


\begin{Def}
Un proceso estoc\'astico a tiempo continuo $\left\{X\left(t\right):t\geq0\right\}$ en un espacio general es estacionario si sus distribuciones finito dimensionales son invariantes bajo cualquier  traslado: para cada $0\leq s_{1}<s_{2}<\cdots<s_{k}$ y $t\geq0$,
\begin{eqnarray*}
\left(X\left(s_{1}+t\right),\ldots,X\left(s_{k}+t\right)\right)=_{d}\left(X\left(s_{1}\right),\ldots,X\left(s_{k}\right)\right).
\end{eqnarray*}
\end{Def}

\begin{Note}
Un proceso de Markov es estacionario si $X\left(t\right)=_{d}X\left(0\right)$, $t\geq0$.
\end{Note}

Considerese el proceso $N\left(t\right)=\sum_{n}\indora\left(\tau_{n}\leq t\right)$ en $\rea_{+}$, con puntos $0<\tau_{1}<\tau_{2}<\cdots$.

\begin{Prop}
Si $N$ es un proceso puntual estacionario y $\esp\left[N\left(1\right)\right]<\infty$, entonces $\esp\left[N\left(t\right)\right]=t\esp\left[N\left(1\right)\right]$, $t\geq0$

\end{Prop}

\begin{Teo}
Los siguientes enunciados son equivalentes
\begin{itemize}
\item[i)] El proceso retardado de renovaci\'on $N$ es estacionario.

\item[ii)] EL proceso de tiempos de recurrencia hacia adelante $B\left(t\right)$ es estacionario.


\item[iii)] $\esp\left[N\left(t\right)\right]=t/\mu$,


\item[iv)] $G\left(t\right)=F_{e}\left(t\right)=\frac{1}{\mu}\int_{0}^{t}\left[1-F\left(s\right)\right]ds$
\end{itemize}
Cuando estos enunciados son ciertos, $P\left\{B\left(t\right)\leq x\right\}=F_{e}\left(x\right)$, para $t,x\geq0$.

\end{Teo}

\begin{Note}
Una consecuencia del teorema anterior es que el Proceso Poisson es el \'unico proceso sin retardo que es estacionario.
\end{Note}

\begin{Coro}
El proceso de renovaci\'on $N\left(t\right)$ sin retardo, y cuyos tiempos de inter renonaci\'on tienen media finita, es estacionario si y s\'olo si es un proceso Poisson.

\end{Coro}

%______________________________________________________________________

%\section{Ejemplos, Notas importantes}
%______________________________________________________________________
%\section*{Ap\'endice A}
%__________________________________________________________________

%________________________________________________________________________
%\subsection*{Procesos Regenerativos}
%________________________________________________________________________



\begin{Note}
Si $\tilde{X}\left(t\right)$ con espacio de estados $\tilde{S}$ es regenerativo sobre $T_{n}$, entonces $X\left(t\right)=f\left(\tilde{X}\left(t\right)\right)$ tambi\'en es regenerativo sobre $T_{n}$, para cualquier funci\'on $f:\tilde{S}\rightarrow S$.
\end{Note}

\begin{Note}
Los procesos regenerativos son crudamente regenerativos, pero no al rev\'es.
\end{Note}
%\subsection*{Procesos Regenerativos: Sigman\cite{Sigman1}}
\begin{Def}[Definici\'on Cl\'asica]
Un proceso estoc\'astico $X=\left\{X\left(t\right):t\geq0\right\}$ es llamado regenerativo is existe una variable aleatoria $R_{1}>0$ tal que
\begin{itemize}
\item[i)] $\left\{X\left(t+R_{1}\right):t\geq0\right\}$ es independiente de $\left\{\left\{X\left(t\right):t<R_{1}\right\},\right\}$
\item[ii)] $\left\{X\left(t+R_{1}\right):t\geq0\right\}$ es estoc\'asticamente equivalente a $\left\{X\left(t\right):t>0\right\}$
\end{itemize}

Llamamos a $R_{1}$ tiempo de regeneraci\'on, y decimos que $X$ se regenera en este punto.
\end{Def}

$\left\{X\left(t+R_{1}\right)\right\}$ es regenerativo con tiempo de regeneraci\'on $R_{2}$, independiente de $R_{1}$ pero con la misma distribuci\'on que $R_{1}$. Procediendo de esta manera se obtiene una secuencia de variables aleatorias independientes e id\'enticamente distribuidas $\left\{R_{n}\right\}$ llamados longitudes de ciclo. Si definimos a $Z_{k}\equiv R_{1}+R_{2}+\cdots+R_{k}$, se tiene un proceso de renovaci\'on llamado proceso de renovaci\'on encajado para $X$.




\begin{Def}
Para $x$ fijo y para cada $t\geq0$, sea $I_{x}\left(t\right)=1$ si $X\left(t\right)\leq x$,  $I_{x}\left(t\right)=0$ en caso contrario, y def\'inanse los tiempos promedio
\begin{eqnarray*}
\overline{X}&=&lim_{t\rightarrow\infty}\frac{1}{t}\int_{0}^{\infty}X\left(u\right)du\\
\prob\left(X_{\infty}\leq x\right)&=&lim_{t\rightarrow\infty}\frac{1}{t}\int_{0}^{\infty}I_{x}\left(u\right)du,
\end{eqnarray*}
cuando estos l\'imites existan.
\end{Def}

Como consecuencia del teorema de Renovaci\'on-Recompensa, se tiene que el primer l\'imite  existe y es igual a la constante
\begin{eqnarray*}
\overline{X}&=&\frac{\esp\left[\int_{0}^{R_{1}}X\left(t\right)dt\right]}{\esp\left[R_{1}\right]},
\end{eqnarray*}
suponiendo que ambas esperanzas son finitas.

\begin{Note}
\begin{itemize}
\item[a)] Si el proceso regenerativo $X$ es positivo recurrente y tiene trayectorias muestrales no negativas, entonces la ecuaci\'on anterior es v\'alida.
\item[b)] Si $X$ es positivo recurrente regenerativo, podemos construir una \'unica versi\'on estacionaria de este proceso, $X_{e}=\left\{X_{e}\left(t\right)\right\}$, donde $X_{e}$ es un proceso estoc\'astico regenerativo y estrictamente estacionario, con distribuci\'on marginal distribuida como $X_{\infty}$
\end{itemize}
\end{Note}

Para $\left\{X\left(t\right):t\geq0\right\}$ Proceso Estoc\'astico a tiempo continuo con estado de espacios $S$, que es un espacio m\'etrico, con trayectorias continuas por la derecha y con l\'imites por la izquierda c.s. Sea $N\left(t\right)$ un proceso de renovaci\'on en $\rea_{+}$ definido en el mismo espacio de probabilidad que $X\left(t\right)$, con tiempos de renovaci\'on $T$ y tiempos de inter-renovaci\'on $\xi_{n}=T_{n}-T_{n-1}$, con misma distribuci\'on $F$ de media finita $\mu$.


\begin{Def}
Para el proceso $\left\{\left(N\left(t\right),X\left(t\right)\right):t\geq0\right\}$, sus trayectoria muestrales en el intervalo de tiempo $\left[T_{n-1},T_{n}\right)$ est\'an descritas por
\begin{eqnarray*}
\zeta_{n}=\left(\xi_{n},\left\{X\left(T_{n-1}+t\right):0\leq t<\xi_{n}\right\}\right)
\end{eqnarray*}
Este $\zeta_{n}$ es el $n$-\'esimo segmento del proceso. El proceso es regenerativo sobre los tiempos $T_{n}$ si sus segmentos $\zeta_{n}$ son independientes e id\'enticamennte distribuidos.
\end{Def}


\begin{Note}
Si $\tilde{X}\left(t\right)$ con espacio de estados $\tilde{S}$ es regenerativo sobre $T_{n}$, entonces $X\left(t\right)=f\left(\tilde{X}\left(t\right)\right)$ tambi\'en es regenerativo sobre $T_{n}$, para cualquier funci\'on $f:\tilde{S}\rightarrow S$.
\end{Note}

\begin{Note}
Los procesos regenerativos son crudamente regenerativos, pero no al rev\'es.
\end{Note}

\begin{Def}[Definici\'on Cl\'asica]
Un proceso estoc\'astico $X=\left\{X\left(t\right):t\geq0\right\}$ es llamado regenerativo is existe una variable aleatoria $R_{1}>0$ tal que
\begin{itemize}
\item[i)] $\left\{X\left(t+R_{1}\right):t\geq0\right\}$ es independiente de $\left\{\left\{X\left(t\right):t<R_{1}\right\},\right\}$
\item[ii)] $\left\{X\left(t+R_{1}\right):t\geq0\right\}$ es estoc\'asticamente equivalente a $\left\{X\left(t\right):t>0\right\}$
\end{itemize}

Llamamos a $R_{1}$ tiempo de regeneraci\'on, y decimos que $X$ se regenera en este punto.
\end{Def}

$\left\{X\left(t+R_{1}\right)\right\}$ es regenerativo con tiempo de regeneraci\'on $R_{2}$, independiente de $R_{1}$ pero con la misma distribuci\'on que $R_{1}$. Procediendo de esta manera se obtiene una secuencia de variables aleatorias independientes e id\'enticamente distribuidas $\left\{R_{n}\right\}$ llamados longitudes de ciclo. Si definimos a $Z_{k}\equiv R_{1}+R_{2}+\cdots+R_{k}$, se tiene un proceso de renovaci\'on llamado proceso de renovaci\'on encajado para $X$.

\begin{Note}
Un proceso regenerativo con media de la longitud de ciclo finita es llamado positivo recurrente.
\end{Note}


\begin{Def}
Para $x$ fijo y para cada $t\geq0$, sea $I_{x}\left(t\right)=1$ si $X\left(t\right)\leq x$,  $I_{x}\left(t\right)=0$ en caso contrario, y def\'inanse los tiempos promedio
\begin{eqnarray*}
\overline{X}&=&lim_{t\rightarrow\infty}\frac{1}{t}\int_{0}^{\infty}X\left(u\right)du\\
\prob\left(X_{\infty}\leq x\right)&=&lim_{t\rightarrow\infty}\frac{1}{t}\int_{0}^{\infty}I_{x}\left(u\right)du,
\end{eqnarray*}
cuando estos l\'imites existan.
\end{Def}

Como consecuencia del teorema de Renovaci\'on-Recompensa, se tiene que el primer l\'imite  existe y es igual a la constante
\begin{eqnarray*}
\overline{X}&=&\frac{\esp\left[\int_{0}^{R_{1}}X\left(t\right)dt\right]}{\esp\left[R_{1}\right]},
\end{eqnarray*}
suponiendo que ambas esperanzas son finitas.

\begin{Note}
\begin{itemize}
\item[a)] Si el proceso regenerativo $X$ es positivo recurrente y tiene trayectorias muestrales no negativas, entonces la ecuaci\'on anterior es v\'alida.
\item[b)] Si $X$ es positivo recurrente regenerativo, podemos construir una \'unica versi\'on estacionaria de este proceso, $X_{e}=\left\{X_{e}\left(t\right)\right\}$, donde $X_{e}$ es un proceso estoc\'astico regenerativo y estrictamente estacionario, con distribuci\'on marginal distribuida como $X_{\infty}$
\end{itemize}
\end{Note}

%__________________________________________________________________________________________
%\subsection{Procesos Regenerativos Estacionarios - Stidham \cite{Stidham}}
%__________________________________________________________________________________________


Un proceso estoc\'astico a tiempo continuo $\left\{V\left(t\right),t\geq0\right\}$ es un proceso regenerativo si existe una sucesi\'on de variables aleatorias independientes e id\'enticamente distribuidas $\left\{X_{1},X_{2},\ldots\right\}$, sucesi\'on de renovaci\'on, tal que para cualquier conjunto de Borel $A$, 

\begin{eqnarray*}
\prob\left\{V\left(t\right)\in A|X_{1}+X_{2}+\cdots+X_{R\left(t\right)}=s,\left\{V\left(\tau\right),\tau<s\right\}\right\}=\prob\left\{V\left(t-s\right)\in A|X_{1}>t-s\right\},
\end{eqnarray*}
para todo $0\leq s\leq t$, donde $R\left(t\right)=\max\left\{X_{1}+X_{2}+\cdots+X_{j}\leq t\right\}=$n\'umero de renovaciones ({\emph{puntos de regeneraci\'on}}) que ocurren en $\left[0,t\right]$. El intervalo $\left[0,X_{1}\right)$ es llamado {\emph{primer ciclo de regeneraci\'on}} de $\left\{V\left(t \right),t\geq0\right\}$, $\left[X_{1},X_{1}+X_{2}\right)$ el {\emph{segundo ciclo de regeneraci\'on}}, y as\'i sucesivamente.

Sea $X=X_{1}$ y sea $F$ la funci\'on de distrbuci\'on de $X$


\begin{Def}
Se define el proceso estacionario, $\left\{V^{*}\left(t\right),t\geq0\right\}$, para $\left\{V\left(t\right),t\geq0\right\}$ por

\begin{eqnarray*}
\prob\left\{V\left(t\right)\in A\right\}=\frac{1}{\esp\left[X\right]}\int_{0}^{\infty}\prob\left\{V\left(t+x\right)\in A|X>x\right\}\left(1-F\left(x\right)\right)dx,
\end{eqnarray*} 
para todo $t\geq0$ y todo conjunto de Borel $A$.
\end{Def}

\begin{Def}
Una distribuci\'on se dice que es {\emph{aritm\'etica}} si todos sus puntos de incremento son m\'ultiplos de la forma $0,\lambda, 2\lambda,\ldots$ para alguna $\lambda>0$ entera.
\end{Def}


\begin{Def}
Una modificaci\'on medible de un proceso $\left\{V\left(t\right),t\geq0\right\}$, es una versi\'on de este, $\left\{V\left(t,w\right)\right\}$ conjuntamente medible para $t\geq0$ y para $w\in S$, $S$ espacio de estados para $\left\{V\left(t\right),t\geq0\right\}$.
\end{Def}

\begin{Teo}
Sea $\left\{V\left(t\right),t\geq\right\}$ un proceso regenerativo no negativo con modificaci\'on medible. Sea $\esp\left[X\right]<\infty$. Entonces el proceso estacionario dado por la ecuaci\'on anterior est\'a bien definido y tiene funci\'on de distribuci\'on independiente de $t$, adem\'as
\begin{itemize}
\item[i)] \begin{eqnarray*}
\esp\left[V^{*}\left(0\right)\right]&=&\frac{\esp\left[\int_{0}^{X}V\left(s\right)ds\right]}{\esp\left[X\right]}\end{eqnarray*}
\item[ii)] Si $\esp\left[V^{*}\left(0\right)\right]<\infty$, equivalentemente, si $\esp\left[\int_{0}^{X}V\left(s\right)ds\right]<\infty$,entonces
\begin{eqnarray*}
\frac{\int_{0}^{t}V\left(s\right)ds}{t}\rightarrow\frac{\esp\left[\int_{0}^{X}V\left(s\right)ds\right]}{\esp\left[X\right]}
\end{eqnarray*}
con probabilidad 1 y en media, cuando $t\rightarrow\infty$.
\end{itemize}
\end{Teo}
%
%___________________________________________________________________________________________
%\vspace{5.5cm}
%\chapter{Cadenas de Markov estacionarias}
%\vspace{-1.0cm}


%__________________________________________________________________________________________
%\subsection{Procesos Regenerativos Estacionarios - Stidham \cite{Stidham}}
%__________________________________________________________________________________________


Un proceso estoc\'astico a tiempo continuo $\left\{V\left(t\right),t\geq0\right\}$ es un proceso regenerativo si existe una sucesi\'on de variables aleatorias independientes e id\'enticamente distribuidas $\left\{X_{1},X_{2},\ldots\right\}$, sucesi\'on de renovaci\'on, tal que para cualquier conjunto de Borel $A$, 

\begin{eqnarray*}
\prob\left\{V\left(t\right)\in A|X_{1}+X_{2}+\cdots+X_{R\left(t\right)}=s,\left\{V\left(\tau\right),\tau<s\right\}\right\}=\prob\left\{V\left(t-s\right)\in A|X_{1}>t-s\right\},
\end{eqnarray*}
para todo $0\leq s\leq t$, donde $R\left(t\right)=\max\left\{X_{1}+X_{2}+\cdots+X_{j}\leq t\right\}=$n\'umero de renovaciones ({\emph{puntos de regeneraci\'on}}) que ocurren en $\left[0,t\right]$. El intervalo $\left[0,X_{1}\right)$ es llamado {\emph{primer ciclo de regeneraci\'on}} de $\left\{V\left(t \right),t\geq0\right\}$, $\left[X_{1},X_{1}+X_{2}\right)$ el {\emph{segundo ciclo de regeneraci\'on}}, y as\'i sucesivamente.

Sea $X=X_{1}$ y sea $F$ la funci\'on de distrbuci\'on de $X$


\begin{Def}
Se define el proceso estacionario, $\left\{V^{*}\left(t\right),t\geq0\right\}$, para $\left\{V\left(t\right),t\geq0\right\}$ por

\begin{eqnarray*}
\prob\left\{V\left(t\right)\in A\right\}=\frac{1}{\esp\left[X\right]}\int_{0}^{\infty}\prob\left\{V\left(t+x\right)\in A|X>x\right\}\left(1-F\left(x\right)\right)dx,
\end{eqnarray*} 
para todo $t\geq0$ y todo conjunto de Borel $A$.
\end{Def}

\begin{Def}
Una distribuci\'on se dice que es {\emph{aritm\'etica}} si todos sus puntos de incremento son m\'ultiplos de la forma $0,\lambda, 2\lambda,\ldots$ para alguna $\lambda>0$ entera.
\end{Def}


\begin{Def}
Una modificaci\'on medible de un proceso $\left\{V\left(t\right),t\geq0\right\}$, es una versi\'on de este, $\left\{V\left(t,w\right)\right\}$ conjuntamente medible para $t\geq0$ y para $w\in S$, $S$ espacio de estados para $\left\{V\left(t\right),t\geq0\right\}$.
\end{Def}

\begin{Teo}
Sea $\left\{V\left(t\right),t\geq\right\}$ un proceso regenerativo no negativo con modificaci\'on medible. Sea $\esp\left[X\right]<\infty$. Entonces el proceso estacionario dado por la ecuaci\'on anterior est\'a bien definido y tiene funci\'on de distribuci\'on independiente de $t$, adem\'as
\begin{itemize}
\item[i)] \begin{eqnarray*}
\esp\left[V^{*}\left(0\right)\right]&=&\frac{\esp\left[\int_{0}^{X}V\left(s\right)ds\right]}{\esp\left[X\right]}\end{eqnarray*}
\item[ii)] Si $\esp\left[V^{*}\left(0\right)\right]<\infty$, equivalentemente, si $\esp\left[\int_{0}^{X}V\left(s\right)ds\right]<\infty$,entonces
\begin{eqnarray*}
\frac{\int_{0}^{t}V\left(s\right)ds}{t}\rightarrow\frac{\esp\left[\int_{0}^{X}V\left(s\right)ds\right]}{\esp\left[X\right]}
\end{eqnarray*}
con probabilidad 1 y en media, cuando $t\rightarrow\infty$.
\end{itemize}
\end{Teo}

Sea la funci\'on generadora de momentos para $L_{i}$, el n\'umero de usuarios en la cola $Q_{i}\left(z\right)$ en cualquier momento, est\'a dada por el tiempo promedio de $z^{L_{i}\left(t\right)}$ sobre el ciclo regenerativo definido anteriormente. Entonces 



Es decir, es posible determinar las longitudes de las colas a cualquier tiempo $t$. Entonces, determinando el primer momento es posible ver que


\begin{Def}
El tiempo de Ciclo $C_{i}$ es el periodo de tiempo que comienza cuando la cola $i$ es visitada por primera vez en un ciclo, y termina cuando es visitado nuevamente en el pr\'oximo ciclo. La duraci\'on del mismo est\'a dada por $\tau_{i}\left(m+1\right)-\tau_{i}\left(m\right)$, o equivalentemente $\overline{\tau}_{i}\left(m+1\right)-\overline{\tau}_{i}\left(m\right)$ bajo condiciones de estabilidad.
\end{Def}


\begin{Def}
El tiempo de intervisita $I_{i}$ es el periodo de tiempo que comienza cuando se ha completado el servicio en un ciclo y termina cuando es visitada nuevamente en el pr\'oximo ciclo. Su  duraci\'on del mismo est\'a dada por $\tau_{i}\left(m+1\right)-\overline{\tau}_{i}\left(m\right)$.
\end{Def}

La duraci\'on del tiempo de intervisita es $\tau_{i}\left(m+1\right)-\overline{\tau}\left(m\right)$. Dado que el n\'umero de usuarios presentes en $Q_{i}$ al tiempo $t=\tau_{i}\left(m+1\right)$ es igual al n\'umero de arribos durante el intervalo de tiempo $\left[\overline{\tau}\left(m\right),\tau_{i}\left(m+1\right)\right]$ se tiene que


\begin{eqnarray*}
\esp\left[z_{i}^{L_{i}\left(\tau_{i}\left(m+1\right)\right)}\right]=\esp\left[\left\{P_{i}\left(z_{i}\right)\right\}^{\tau_{i}\left(m+1\right)-\overline{\tau}\left(m\right)}\right]
\end{eqnarray*}

entonces, si $I_{i}\left(z\right)=\esp\left[z^{\tau_{i}\left(m+1\right)-\overline{\tau}\left(m\right)}\right]$
se tiene que $F_{i}\left(z\right)=I_{i}\left[P_{i}\left(z\right)\right]$
para $i=1,2$.

Conforme a la definici\'on dada al principio del cap\'itulo, definici\'on (\ref{Def.Tn}), sean $T_{1},T_{2},\ldots$ los puntos donde las longitudes de las colas de la red de sistemas de visitas c\'iclicas son cero simult\'aneamente, cuando la cola $Q_{j}$ es visitada por el servidor para dar servicio, es decir, $L_{1}\left(T_{i}\right)=0,L_{2}\left(T_{i}\right)=0,\hat{L}_{1}\left(T_{i}\right)=0$ y $\hat{L}_{2}\left(T_{i}\right)=0$, a estos puntos se les denominar\'a puntos regenerativos. Entonces, 

\begin{Def}
Al intervalo de tiempo entre dos puntos regenerativos se le llamar\'a ciclo regenerativo.
\end{Def}

\begin{Def}
Para $T_{i}$ se define, $M_{i}$, el n\'umero de ciclos de visita a la cola $Q_{l}$, durante el ciclo regenerativo, es decir, $M_{i}$ es un proceso de renovaci\'on.
\end{Def}

\begin{Def}
Para cada uno de los $M_{i}$'s, se definen a su vez la duraci\'on de cada uno de estos ciclos de visita en el ciclo regenerativo, $C_{i}^{(m)}$, para $m=1,2,\ldots,M_{i}$, que a su vez, tambi\'en es n proceso de renovaci\'on.
\end{Def}

\footnote{In Stidham and  Heyman \cite{Stidham} shows that is sufficient for the regenerative process to be stationary that the mean regenerative cycle time is finite: $\esp\left[\sum_{m=1}^{M_{i}}C_{i}^{(m)}\right]<\infty$, 


 como cada $C_{i}^{(m)}$ contiene intervalos de r\'eplica positivos, se tiene que $\esp\left[M_{i}\right]<\infty$, adem\'as, como $M_{i}>0$, se tiene que la condici\'on anterior es equivalente a tener que $\esp\left[C_{i}\right]<\infty$,
por lo tanto una condici\'on suficiente para la existencia del proceso regenerativo est\'a dada por $\sum_{k=1}^{N}\mu_{k}<1.$}

Para $\left\{X\left(t\right):t\geq0\right\}$ Proceso Estoc\'astico a tiempo continuo con estado de espacios $S$, que es un espacio m\'etrico, con trayectorias continuas por la derecha y con l\'imites por la izquierda c.s. Sea $N\left(t\right)$ un proceso de renovaci\'on en $\rea_{+}$ definido en el mismo espacio de probabilidad que $X\left(t\right)$, con tiempos de renovaci\'on $T$ y tiempos de inter-renovaci\'on $\xi_{n}=T_{n}-T_{n-1}$, con misma distribuci\'on $F$ de media finita $\mu$.

\begin{Def}
Un elemento aleatorio en un espacio medible $\left(E,\mathcal{E}\right)$ en un espacio de probabilidad $\left(\Omega,\mathcal{F},\prob\right)$ a $\left(E,\mathcal{E}\right)$, es decir,
para $A\in \mathcal{E}$,  se tiene que $\left\{Y\in A\right\}\in\mathcal{F}$, donde $\left\{Y\in A\right\}:=\left\{w\in\Omega:Y\left(w\right)\in A\right\}=:Y^{-1}A$.
\end{Def}

\begin{Note}
Tambi\'en se dice que $Y$ est\'a soportado por el espacio de probabilidad $\left(\Omega,\mathcal{F},\prob\right)$ y que $Y$ es un mapeo medible de $\Omega$ en $E$, es decir, es $\mathcal{F}/\mathcal{E}$ medible.
\end{Note}

\begin{Def}
Para cada $i\in \mathbb{I}$ sea $P_{i}$ una medida de probabilidad en un espacio medible $\left(E_{i},\mathcal{E}_{i}\right)$. Se define el espacio producto
$\otimes_{i\in\mathbb{I}}\left(E_{i},\mathcal{E}_{i}\right):=\left(\prod_{i\in\mathbb{I}}E_{i},\otimes_{i\in\mathbb{I}}\mathcal{E}_{i}\right)$, donde $\prod_{i\in\mathbb{I}}E_{i}$ es el producto cartesiano de los $E_{i}$'s, y $\otimes_{i\in\mathbb{I}}\mathcal{E}_{i}$ es la $\sigma$-\'algebra producto, es decir, es la $\sigma$-\'algebra m\'as peque\~na en $\prod_{i\in\mathbb{I}}E_{i}$ que hace al $i$-\'esimo mapeo proyecci\'on en $E_{i}$ medible para toda $i\in\mathbb{I}$ es la $\sigma$-\'algebra inducida por los mapeos proyecci\'on. $$\otimes_{i\in\mathbb{I}}\mathcal{E}_{i}:=\sigma\left\{\left\{y:y_{i}\in A\right\}:i\in\mathbb{I}\textrm{ y }A\in\mathcal{E}_{i}\right\}.$$
\end{Def}

\begin{Def}
Un espacio de probabilidad $\left(\tilde{\Omega},\tilde{\mathcal{F}},\tilde{\prob}\right)$ es una extensi\'on de otro espacio de probabilidad $\left(\Omega,\mathcal{F},\prob\right)$ si $\left(\tilde{\Omega},\tilde{\mathcal{F}},\tilde{\prob}\right)$ soporta un elemento aleatorio $\xi\in\left(\Omega,\mathcal{F}\right)$ que tienen a $\prob$ como distribuci\'on.
\end{Def}

\begin{Teo}
Sea $\mathbb{I}$ un conjunto de \'indices arbitrario. Para cada $i\in\mathbb{I}$ sea $P_{i}$ una medida de probabilidad en un espacio medible $\left(E_{i},\mathcal{E}_{i}\right)$. Entonces existe una \'unica medida de probabilidad $\otimes_{i\in\mathbb{I}}P_{i}$ en $\otimes_{i\in\mathbb{I}}\left(E_{i},\mathcal{E}_{i}\right)$ tal que 

\begin{eqnarray*}
\otimes_{i\in\mathbb{I}}P_{i}\left(y\in\prod_{i\in\mathbb{I}}E_{i}:y_{i}\in A_{i_{1}},\ldots,y_{n}\in A_{i_{n}}\right)=P_{i_{1}}\left(A_{i_{n}}\right)\cdots P_{i_{n}}\left(A_{i_{n}}\right)
\end{eqnarray*}
para todos los enteros $n>0$, toda $i_{1},\ldots,i_{n}\in\mathbb{I}$ y todo $A_{i_{1}}\in\mathcal{E}_{i_{1}},\ldots,A_{i_{n}}\in\mathcal{E}_{i_{n}}$
\end{Teo}

La medida $\otimes_{i\in\mathbb{I}}P_{i}$ es llamada la medida producto y $\otimes_{i\in\mathbb{I}}\left(E_{i},\mathcal{E}_{i},P_{i}\right):=\left(\prod_{i\in\mathbb{I}},E_{i},\otimes_{i\in\mathbb{I}}\mathcal{E}_{i},\otimes_{i\in\mathbb{I}}P_{i}\right)$, es llamado espacio de probabilidad producto.


\begin{Def}
Un espacio medible $\left(E,\mathcal{E}\right)$ es \textit{Polaco} si existe una m\'etrica en $E$ tal que $E$ es completo, es decir cada sucesi\'on de Cauchy converge a un l\'imite en $E$, y \textit{separable}, $E$ tienen un subconjunto denso numerable, y tal que $\mathcal{E}$ es generado por conjuntos abiertos.
\end{Def}


\begin{Def}
Dos espacios medibles $\left(E,\mathcal{E}\right)$ y $\left(G,\mathcal{G}\right)$ son Borel equivalentes \textit{isomorfos} si existe una biyecci\'on $f:E\rightarrow G$ tal que $f$ es $\mathcal{E}/\mathcal{G}$ medible y su inversa $f^{-1}$ es $\mathcal{G}/\mathcal{E}$ medible. La biyecci\'on es una equivalencia de Borel.
\end{Def}

\begin{Def}
Un espacio medible  $\left(E,\mathcal{E}\right)$ es un \textit{espacio est\'andar} si es Borel equivalente a $\left(G,\mathcal{G}\right)$, donde $G$ es un subconjunto de Borel de $\left[0,1\right]$ y $\mathcal{G}$ son los subconjuntos de Borel de $G$.
\end{Def}

\begin{Note}
Cualquier espacio Polaco es un espacio est\'andar.
\end{Note}


\begin{Def}
Un proceso estoc\'astico con conjunto de \'indices $\mathbb{I}$ y espacio de estados $\left(E,\mathcal{E}\right)$ es una familia $Z=\left(\mathbb{Z}_{s}\right)_{s\in\mathbb{I}}$ donde $\mathbb{Z}_{s}$ son elementos aleatorios definidos en un espacio de probabilidad com\'un $\left(\Omega,\mathcal{F},\prob\right)$ y todos toman valores en $\left(E,\mathcal{E}\right)$.
\end{Def}

\begin{Def}
Un proceso estoc\'astico \textit{one-sided contiuous time} (\textbf{PEOSCT}) es un proceso estoc\'astico con conjunto de \'indices $\mathbb{I}=\left[0,\infty\right)$.
\end{Def}


Sea $\left(E^{\mathbb{I}},\mathcal{E}^{\mathbb{I}}\right)$ denota el espacio producto $\left(E^{\mathbb{I}},\mathcal{E}^{\mathbb{I}}\right):=\otimes_{s\in\mathbb{I}}\left(E,\mathcal{E}\right)$. Vamos a considerar $\mathbb{Z}$ como un mapeo aleatorio, es decir, como un elemento aleatorio en $\left(E^{\mathbb{I}},\mathcal{E}^{\mathbb{I}}\right)$ definido por $Z\left(w\right)=\left(Z_{s}\left(w\right)\right)_{s\in\mathbb{I}}$ y $w\in\Omega$.

\begin{Note}
La distribuci\'on de un proceso estoc\'astico $Z$ es la distribuci\'on de $Z$ como un elemento aleatorio en $\left(E^{\mathbb{I}},\mathcal{E}^{\mathbb{I}}\right)$. La distribuci\'on de $Z$ esta determinada de manera \'unica por las distribuciones finito dimensionales.
\end{Note}

\begin{Note}
En particular cuando $Z$ toma valores reales, es decir, $\left(E,\mathcal{E}\right)=\left(\mathbb{R},\mathcal{B}\right)$ las distribuciones finito dimensionales est\'an determinadas por las funciones de distribuci\'on finito dimensionales

\begin{eqnarray}
\prob\left(Z_{t_{1}}\leq x_{1},\ldots,Z_{t_{n}}\leq x_{n}\right),x_{1},\ldots,x_{n}\in\mathbb{R},t_{1},\ldots,t_{n}\in\mathbb{I},n\geq1.
\end{eqnarray}
\end{Note}

\begin{Note}
Para espacios polacos $\left(E,\mathcal{E}\right)$ el Teorema de Consistencia de Kolmogorov asegura que dada una colecci\'on de distribuciones finito dimensionales consistentes, siempre existe un proceso estoc\'astico que posee tales distribuciones finito dimensionales.
\end{Note}


\begin{Def}
Las trayectorias de $Z$ son las realizaciones $Z\left(w\right)$ para $w\in\Omega$ del mapeo aleatorio $Z$.
\end{Def}

\begin{Note}
Algunas restricciones se imponen sobre las trayectorias, por ejemplo que sean continuas por la derecha, o continuas por la derecha con l\'imites por la izquierda, o de manera m\'as general, se pedir\'a que caigan en alg\'un subconjunto $H$ de $E^{\mathbb{I}}$. En este caso es natural considerar a $Z$ como un elemento aleatorio que no est\'a en $\left(E^{\mathbb{I}},\mathcal{E}^{\mathbb{I}}\right)$ sino en $\left(H,\mathcal{H}\right)$, donde $\mathcal{H}$ es la $\sigma$-\'algebra generada por los mapeos proyecci\'on que toman a $z\in H$ a $z_{t}\in E$ para $t\in\mathbb{I}$. A $\mathcal{H}$ se le conoce como la traza de $H$ en $E^{\mathbb{I}}$, es decir,
\begin{eqnarray}
\mathcal{H}:=E^{\mathbb{I}}\cap H:=\left\{A\cap H:A\in E^{\mathbb{I}}\right\}.
\end{eqnarray}
\end{Note}


\begin{Note}
$Z$ tiene trayectorias con valores en $H$ y cada $Z_{t}$ es un mapeo medible de $\left(\Omega,\mathcal{F}\right)$ a $\left(H,\mathcal{H}\right)$. Cuando se considera un espacio de trayectorias en particular $H$, al espacio $\left(H,\mathcal{H}\right)$ se le llama el espacio de trayectorias de $Z$.
\end{Note}

\begin{Note}
La distribuci\'on del proceso estoc\'astico $Z$ con espacio de trayectorias $\left(H,\mathcal{H}\right)$ es la distribuci\'on de $Z$ como  un elemento aleatorio en $\left(H,\mathcal{H}\right)$. La distribuci\'on, nuevemente, est\'a determinada de manera \'unica por las distribuciones finito dimensionales.
\end{Note}


\begin{Def}
Sea $Z$ un PEOSCT  con espacio de estados $\left(E,\mathcal{E}\right)$ y sea $T$ un tiempo aleatorio en $\left[0,\infty\right)$. Por $Z_{T}$ se entiende el mapeo con valores en $E$ definido en $\Omega$ en la manera obvia:
\begin{eqnarray*}
Z_{T}\left(w\right):=Z_{T\left(w\right)}\left(w\right). w\in\Omega.
\end{eqnarray*}
\end{Def}

\begin{Def}
Un PEOSCT $Z$ es conjuntamente medible (\textbf{CM}) si el mapeo que toma $\left(w,t\right)\in\Omega\times\left[0,\infty\right)$ a $Z_{t}\left(w\right)\in E$ es $\mathcal{F}\otimes\mathcal{B}\left[0,\infty\right)/\mathcal{E}$ medible.
\end{Def}

\begin{Note}
Un PEOSCT-CM implica que el proceso es medible, dado que $Z_{T}$ es una composici\'on  de dos mapeos continuos: el primero que toma $w$ en $\left(w,T\left(w\right)\right)$ es $\mathcal{F}/\mathcal{F}\otimes\mathcal{B}\left[0,\infty\right)$ medible, mientras que el segundo toma $\left(w,T\left(w\right)\right)$ en $Z_{T\left(w\right)}\left(w\right)$ es $\mathcal{F}\otimes\mathcal{B}\left[0,\infty\right)/\mathcal{E}$ medible.
\end{Note}


\begin{Def}
Un PEOSCT con espacio de estados $\left(H,\mathcal{H}\right)$ es can\'onicamente conjuntamente medible (\textbf{CCM}) si el mapeo $\left(z,t\right)\in H\times\left[0,\infty\right)$ en $Z_{t}\in E$ es $\mathcal{H}\otimes\mathcal{B}\left[0,\infty\right)/\mathcal{E}$ medible.
\end{Def}

\begin{Note}
Un PEOSCTCCM implica que el proceso es CM, dado que un PECCM $Z$ es un mapeo de $\Omega\times\left[0,\infty\right)$ a $E$, es la composici\'on de dos mapeos medibles: el primero, toma $\left(w,t\right)$ en $\left(Z\left(w\right),t\right)$ es $\mathcal{F}\otimes\mathcal{B}\left[0,\infty\right)/\mathcal{H}\otimes\mathcal{B}\left[0,\infty\right)$ medible, y el segundo que toma $\left(Z\left(w\right),t\right)$  en $Z_{t}\left(w\right)$ es $\mathcal{H}\otimes\mathcal{B}\left[0,\infty\right)/\mathcal{E}$ medible. Por tanto CCM es una condici\'on m\'as fuerte que CM.
\end{Note}

\begin{Def}
Un conjunto de trayectorias $H$ de un PEOSCT $Z$, es internamente shift-invariante (\textbf{ISI}) si 
\begin{eqnarray*}
\left\{\left(z_{t+s}\right)_{s\in\left[0,\infty\right)}:z\in H\right\}=H\textrm{, }t\in\left[0,\infty\right).
\end{eqnarray*}
\end{Def}


\begin{Def}
Dado un PEOSCTISI, se define el mapeo-shift $\theta_{t}$, $t\in\left[0,\infty\right)$, de $H$ a $H$ por 
\begin{eqnarray*}
\theta_{t}z=\left(z_{t+s}\right)_{s\in\left[0,\infty\right)}\textrm{, }z\in H.
\end{eqnarray*}
\end{Def}

\begin{Def}
Se dice que un proceso $Z$ es shift-medible (\textbf{SM}) si $Z$ tiene un conjunto de trayectorias $H$ que es ISI y adem\'as el mapeo que toma $\left(z,t\right)\in H\times\left[0,\infty\right)$ en $\theta_{t}z\in H$ es $\mathcal{H}\otimes\mathcal{B}\left[0,\infty\right)/\mathcal{H}$ medible.
\end{Def}

\begin{Note}
Un proceso estoc\'astico con conjunto de trayectorias $H$ ISI es shift-medible si y s\'olo si es CCM
\end{Note}

\begin{Note}
\begin{itemize}
\item Dado el espacio polaco $\left(E,\mathcal{E}\right)$ se tiene el  conjunto de trayectorias $D_{E}\left[0,\infty\right)$ que es ISI, entonces cumpe con ser CCM.

\item Si $G$ es abierto, podemos cubrirlo por bolas abiertas cuay cerradura este contenida en $G$, y como $G$ es segundo numerable como subespacio de $E$, lo podemos cubrir por una cantidad numerable de bolas abiertas.

\end{itemize}
\end{Note}


\begin{Note}
Los procesos estoc\'asticos $Z$ a tiempo discreto con espacio de estados polaco, tambi\'en tiene un espacio de trayectorias polaco y por tanto tiene distribuciones condicionales regulares.
\end{Note}

\begin{Teo}
El producto numerable de espacios polacos es polaco.
\end{Teo}


\begin{Def}
Sea $\left(\Omega,\mathcal{F},\prob\right)$ espacio de probabilidad que soporta al proceso $Z=\left(Z_{s}\right)_{s\in\left[0,\infty\right)}$ y $S=\left(S_{k}\right)_{0}^{\infty}$ donde $Z$ es un PEOSCTM con espacio de estados $\left(E,\mathcal{E}\right)$  y espacio de trayectorias $\left(H,\mathcal{H}\right)$  y adem\'as $S$ es una sucesi\'on de tiempos aleatorios one-sided que satisfacen la condici\'on $0\leq S_{0}<S_{1}<\cdots\rightarrow\infty$. Considerando $S$ como un mapeo medible de $\left(\Omega,\mathcal{F}\right)$ al espacio sucesi\'on $\left(L,\mathcal{L}\right)$, donde 
\begin{eqnarray*}
L=\left\{\left(s_{k}\right)_{0}^{\infty}\in\left[0,\infty\right)^{\left\{0,1,\ldots\right\}}:s_{0}<s_{1}<\cdots\rightarrow\infty\right\},
\end{eqnarray*}
donde $\mathcal{L}$ son los subconjuntos de Borel de $L$, es decir, $\mathcal{L}=L\cap\mathcal{B}^{\left\{0,1,\ldots\right\}}$.

As\'i el par $\left(Z,S\right)$ es un mapeo medible de  $\left(\Omega,\mathcal{F}\right)$ en $\left(H\times L,\mathcal{H}\otimes\mathcal{L}\right)$. El par $\mathcal{H}\otimes\mathcal{L}^{+}$ denotar\'a la clase de todas las funciones medibles de $\left(H\times L,\mathcal{H}\otimes\mathcal{L}\right)$ en $\left(\left[0,\infty\right),\mathcal{B}\left[0,\infty\right)\right)$.
\end{Def}


\begin{Def}
Sea $\theta_{t}$ el mapeo-shift conjunto de $H\times L$ en $H\times L$ dado por
\begin{eqnarray*}
\theta_{t}\left(z,\left(s_{k}\right)_{0}^{\infty}\right)=\theta_{t}\left(z,\left(s_{n_{t-}+k}-t\right)_{0}^{\infty}\right)
\end{eqnarray*}
donde 
$n_{t-}=inf\left\{n\geq1:s_{n}\geq t\right\}$.
\end{Def}

\begin{Note}
Con la finalidad de poder realizar los shift's sin complicaciones de medibilidad, se supondr\'a que $Z$ es shit-medible, es decir, el conjunto de trayectorias $H$ es invariante bajo shifts del tiempo y el mapeo que toma $\left(z,t\right)\in H\times\left[0,\infty\right)$ en $z_{t}\in E$ es $\mathcal{H}\otimes\mathcal{B}\left[0,\infty\right)/\mathcal{E}$ medible.
\end{Note}

\begin{Def}
Dado un proceso \textbf{PEOSSM} (Proceso Estoc\'astico One Side Shift Medible) $Z$, se dice regenerativo cl\'asico con tiempos de regeneraci\'on $S$ si 

\begin{eqnarray*}
\theta_{S_{n}}\left(Z,S\right)=\left(Z^{0},S^{0}\right),n\geq0
\end{eqnarray*}
y adem\'as $\theta_{S_{n}}\left(Z,S\right)$ es independiente de $\left(\left(Z_{s}\right)s\in\left[0,S_{n}\right),S_{0},\ldots,S_{n}\right)$
Si lo anterior se cumple, al par $\left(Z,S\right)$ se le llama regenerativo cl\'asico.
\end{Def}

\begin{Note}
Si el par $\left(Z,S\right)$ es regenerativo cl\'asico, entonces las longitudes de los ciclos $X_{1},X_{2},\ldots,$ son i.i.d. e independientes de la longitud del retraso $S_{0}$, es decir, $S$ es un proceso de renovaci\'on. Las longitudes de los ciclos tambi\'en son llamados tiempos de inter-regeneraci\'on y tiempos de ocurrencia.

\end{Note}

\begin{Teo}
Sup\'ongase que el par $\left(Z,S\right)$ es regenerativo cl\'asico con $\esp\left[X_{1}\right]<\infty$. Entonces $\left(Z^{*},S^{*}\right)$ en el teorema 2.1 es una versi\'on estacionaria de $\left(Z,S\right)$. Adem\'as, si $X_{1}$ es lattice con span $d$, entonces $\left(Z^{**},S^{**}\right)$ en el teorema 2.2 es una versi\'on periodicamente estacionaria de $\left(Z,S\right)$ con periodo $d$.

\end{Teo}

\begin{Def}
Una variable aleatoria $X_{1}$ es \textit{spread out} si existe una $n\geq1$ y una  funci\'on $f\in\mathcal{B}^{+}$ tal que $\int_{\rea}f\left(x\right)dx>0$ con $X_{2},X_{3},\ldots,X_{n}$ copias i.i.d  de $X_{1}$, $$\prob\left(X_{1}+\cdots+X_{n}\in B\right)\geq\int_{B}f\left(x\right)dx$$ para $B\in\mathcal{B}$.

\end{Def}



\begin{Def}
Dado un proceso estoc\'astico $Z$ se le llama \textit{wide-sense regenerative} (\textbf{WSR}) con tiempos de regeneraci\'on $S$ si $\theta_{S_{n}}\left(Z,S\right)=\left(Z^{0},S^{0}\right)$ para $n\geq0$ en distribuci\'on y $\theta_{S_{n}}\left(Z,S\right)$ es independiente de $\left(S_{0},S_{1},\ldots,S_{n}\right)$ para $n\geq0$.
Se dice que el par $\left(Z,S\right)$ es WSR si lo anterior se cumple.
\end{Def}


\begin{Note}
\begin{itemize}
\item El proceso de trayectorias $\left(\theta_{s}Z\right)_{s\in\left[0,\infty\right)}$ es WSR con tiempos de regeneraci\'on $S$ pero no es regenerativo cl\'asico.

\item Si $Z$ es cualquier proceso estacionario y $S$ es un proceso de renovaci\'on que es independiente de $Z$, entonces $\left(Z,S\right)$ es WSR pero en general no es regenerativo cl\'asico

\end{itemize}

\end{Note}


\begin{Note}
Para cualquier proceso estoc\'astico $Z$, el proceso de trayectorias $\left(\theta_{s}Z\right)_{s\in\left[0,\infty\right)}$ es siempre un proceso de Markov.
\end{Note}



\begin{Teo}
Supongase que el par $\left(Z,S\right)$ es WSR con $\esp\left[X_{1}\right]<\infty$. Entonces $\left(Z^{*},S^{*}\right)$ en el teorema 2.1 es una versi\'on estacionaria de 
$\left(Z,S\right)$.
\end{Teo}


\begin{Teo}
Supongase que $\left(Z,S\right)$ es cycle-stationary con $\esp\left[X_{1}\right]<\infty$. Sea $U$ distribuida uniformemente en $\left[0,1\right)$ e independiente de $\left(Z^{0},S^{0}\right)$ y sea $\prob^{*}$ la medida de probabilidad en $\left(\Omega,\prob\right)$ definida por $$d\prob^{*}=\frac{X_{1}}{\esp\left[X_{1}\right]}d\prob$$. Sea $\left(Z^{*},S^{*}\right)$ con distribuci\'on $\prob^{*}\left(\theta_{UX_{1}}\left(Z^{0},S^{0}\right)\in\cdot\right)$. Entonces $\left(Z^{}*,S^{*}\right)$ es estacionario,
\begin{eqnarray*}
\esp\left[f\left(Z^{*},S^{*}\right)\right]=\esp\left[\int_{0}^{X_{1}}f\left(\theta_{s}\left(Z^{0},S^{0}\right)\right)ds\right]/\esp\left[X_{1}\right]
\end{eqnarray*}
$f\in\mathcal{H}\otimes\mathcal{L}^{+}$, and $S_{0}^{*}$ es continuo con funci\'on distribuci\'on $G_{\infty}$ definida por $$G_{\infty}\left(x\right):=\frac{\esp\left[X_{1}\right]\wedge x}{\esp\left[X_{1}\right]}$$ para $x\geq0$ y densidad $\prob\left[X_{1}>x\right]/\esp\left[X_{1}\right]$, con $x\geq0$.

\end{Teo}


\begin{Teo}
Sea $Z$ un Proceso Estoc\'astico un lado shift-medible \textit{one-sided shift-measurable stochastic process}, (PEOSSM),
y $S_{0}$ y $S_{1}$ tiempos aleatorios tales que $0\leq S_{0}<S_{1}$ y
\begin{equation}
\theta_{S_{1}}Z=\theta_{S_{0}}Z\textrm{ en distribuci\'on}.
\end{equation}

Entonces el espacio de probabilidad subyacente $\left(\Omega,\mathcal{F},\prob\right)$ puede extenderse para soportar una sucesi\'on de tiempos aleatorios $S$ tales que

\begin{eqnarray}
\theta_{S_{n}}\left(Z,S\right)=\left(Z^{0},S^{0}\right),n\geq0,\textrm{ en distribuci\'on},\\
\left(Z,S_{0},S_{1}\right)\textrm{ depende de }\left(X_{2},X_{3},\ldots\right)\textrm{ solamente a traves de }\theta_{S_{1}}Z.
\end{eqnarray}
\end{Teo}




%__________________________________________________________________________________________
\subsection{Procesos Regenerativos Estacionarios - Stidham \cite{Stidham}}
%__________________________________________________________________________________________


Un proceso estoc\'astico a tiempo continuo $\left\{V\left(t\right),t\geq0\right\}$ es un proceso regenerativo si existe una sucesi\'on de variables aleatorias independientes e id\'enticamente distribuidas $\left\{X_{1},X_{2},\ldots\right\}$, sucesi\'on de renovaci\'on, tal que para cualquier conjunto de Borel $A$, 

\begin{eqnarray*}
\prob\left\{V\left(t\right)\in A|X_{1}+X_{2}+\cdots+X_{R\left(t\right)}=s,\left\{V\left(\tau\right),\tau<s\right\}\right\}=\prob\left\{V\left(t-s\right)\in A|X_{1}>t-s\right\},
\end{eqnarray*}
para todo $0\leq s\leq t$, donde $R\left(t\right)=\max\left\{X_{1}+X_{2}+\cdots+X_{j}\leq t\right\}=$n\'umero de renovaciones ({\emph{puntos de regeneraci\'on}}) que ocurren en $\left[0,t\right]$. El intervalo $\left[0,X_{1}\right)$ es llamado {\emph{primer ciclo de regeneraci\'on}} de $\left\{V\left(t \right),t\geq0\right\}$, $\left[X_{1},X_{1}+X_{2}\right)$ el {\emph{segundo ciclo de regeneraci\'on}}, y as\'i sucesivamente.

Sea $X=X_{1}$ y sea $F$ la funci\'on de distrbuci\'on de $X$


\begin{Def}
Se define el proceso estacionario, $\left\{V^{*}\left(t\right),t\geq0\right\}$, para $\left\{V\left(t\right),t\geq0\right\}$ por

\begin{eqnarray*}
\prob\left\{V\left(t\right)\in A\right\}=\frac{1}{\esp\left[X\right]}\int_{0}^{\infty}\prob\left\{V\left(t+x\right)\in A|X>x\right\}\left(1-F\left(x\right)\right)dx,
\end{eqnarray*} 
para todo $t\geq0$ y todo conjunto de Borel $A$.
\end{Def}

\begin{Def}
Una distribuci\'on se dice que es {\emph{aritm\'etica}} si todos sus puntos de incremento son m\'ultiplos de la forma $0,\lambda, 2\lambda,\ldots$ para alguna $\lambda>0$ entera.
\end{Def}


\begin{Def}
Una modificaci\'on medible de un proceso $\left\{V\left(t\right),t\geq0\right\}$, es una versi\'on de este, $\left\{V\left(t,w\right)\right\}$ conjuntamente medible para $t\geq0$ y para $w\in S$, $S$ espacio de estados para $\left\{V\left(t\right),t\geq0\right\}$.
\end{Def}

\begin{Teo}
Sea $\left\{V\left(t\right),t\geq\right\}$ un proceso regenerativo no negativo con modificaci\'on medible. Sea $\esp\left[X\right]<\infty$. Entonces el proceso estacionario dado por la ecuaci\'on anterior est\'a bien definido y tiene funci\'on de distribuci\'on independiente de $t$, adem\'as
\begin{itemize}
\item[i)] \begin{eqnarray*}
\esp\left[V^{*}\left(0\right)\right]&=&\frac{\esp\left[\int_{0}^{X}V\left(s\right)ds\right]}{\esp\left[X\right]}\end{eqnarray*}
\item[ii)] Si $\esp\left[V^{*}\left(0\right)\right]<\infty$, equivalentemente, si $\esp\left[\int_{0}^{X}V\left(s\right)ds\right]<\infty$,entonces
\begin{eqnarray*}
\frac{\int_{0}^{t}V\left(s\right)ds}{t}\rightarrow\frac{\esp\left[\int_{0}^{X}V\left(s\right)ds\right]}{\esp\left[X\right]}
\end{eqnarray*}
con probabilidad 1 y en media, cuando $t\rightarrow\infty$.
\end{itemize}
\end{Teo}

\begin{Coro}
Sea $\left\{V\left(t\right),t\geq0\right\}$ un proceso regenerativo no negativo, con modificaci\'on medible. Si $\esp <\infty$, $F$ es no-aritm\'etica, y para todo $x\geq0$, $P\left\{V\left(t\right)\leq x,C>x\right\}$ es de variaci\'on acotada como funci\'on de $t$ en cada intervalo finito $\left[0,\tau\right]$, entonces $V\left(t\right)$ converge en distribuci\'on  cuando $t\rightarrow\infty$ y $$\esp V=\frac{\esp \int_{0}^{X}V\left(s\right)ds}{\esp X}$$
Donde $V$ tiene la distribuci\'on l\'imite de $V\left(t\right)$ cuando $t\rightarrow\infty$.

\end{Coro}

Para el caso discreto se tienen resultados similares.



%______________________________________________________________________
%\subsection{Procesos de Renovaci\'on}
%______________________________________________________________________

\begin{Def}%\label{Def.Tn}
Sean $0\leq T_{1}\leq T_{2}\leq \ldots$ son tiempos aleatorios infinitos en los cuales ocurren ciertos eventos. El n\'umero de tiempos $T_{n}$ en el intervalo $\left[0,t\right)$ es

\begin{eqnarray}
N\left(t\right)=\sum_{n=1}^{\infty}\indora\left(T_{n}\leq t\right),
\end{eqnarray}
para $t\geq0$.
\end{Def}

Si se consideran los puntos $T_{n}$ como elementos de $\rea_{+}$, y $N\left(t\right)$ es el n\'umero de puntos en $\rea$. El proceso denotado por $\left\{N\left(t\right):t\geq0\right\}$, denotado por $N\left(t\right)$, es un proceso puntual en $\rea_{+}$. Los $T_{n}$ son los tiempos de ocurrencia, el proceso puntual $N\left(t\right)$ es simple si su n\'umero de ocurrencias son distintas: $0<T_{1}<T_{2}<\ldots$ casi seguramente.

\begin{Def}
Un proceso puntual $N\left(t\right)$ es un proceso de renovaci\'on si los tiempos de interocurrencia $\xi_{n}=T_{n}-T_{n-1}$, para $n\geq1$, son independientes e identicamente distribuidos con distribuci\'on $F$, donde $F\left(0\right)=0$ y $T_{0}=0$. Los $T_{n}$ son llamados tiempos de renovaci\'on, referente a la independencia o renovaci\'on de la informaci\'on estoc\'astica en estos tiempos. Los $\xi_{n}$ son los tiempos de inter-renovaci\'on, y $N\left(t\right)$ es el n\'umero de renovaciones en el intervalo $\left[0,t\right)$
\end{Def}


\begin{Note}
Para definir un proceso de renovaci\'on para cualquier contexto, solamente hay que especificar una distribuci\'on $F$, con $F\left(0\right)=0$, para los tiempos de inter-renovaci\'on. La funci\'on $F$ en turno degune las otra variables aleatorias. De manera formal, existe un espacio de probabilidad y una sucesi\'on de variables aleatorias $\xi_{1},\xi_{2},\ldots$ definidas en este con distribuci\'on $F$. Entonces las otras cantidades son $T_{n}=\sum_{k=1}^{n}\xi_{k}$ y $N\left(t\right)=\sum_{n=1}^{\infty}\indora\left(T_{n}\leq t\right)$, donde $T_{n}\rightarrow\infty$ casi seguramente por la Ley Fuerte de los Grandes Números.
\end{Note}

%___________________________________________________________________________________________
%
%\subsection{Teorema Principal de Renovaci\'on}
%___________________________________________________________________________________________
%

\begin{Note} Una funci\'on $h:\rea_{+}\rightarrow\rea$ es Directamente Riemann Integrable en los siguientes casos:
\begin{itemize}
\item[a)] $h\left(t\right)\geq0$ es decreciente y Riemann Integrable.
\item[b)] $h$ es continua excepto posiblemente en un conjunto de Lebesgue de medida 0, y $|h\left(t\right)|\leq b\left(t\right)$, donde $b$ es DRI.
\end{itemize}
\end{Note}

\begin{Teo}[Teorema Principal de Renovaci\'on]
Si $F$ es no aritm\'etica y $h\left(t\right)$ es Directamente Riemann Integrable (DRI), entonces

\begin{eqnarray*}
lim_{t\rightarrow\infty}U\star h=\frac{1}{\mu}\int_{\rea_{+}}h\left(s\right)ds.
\end{eqnarray*}
\end{Teo}

\begin{Prop}
Cualquier funci\'on $H\left(t\right)$ acotada en intervalos finitos y que es 0 para $t<0$ puede expresarse como
\begin{eqnarray*}
H\left(t\right)=U\star h\left(t\right)\textrm{,  donde }h\left(t\right)=H\left(t\right)-F\star H\left(t\right)
\end{eqnarray*}
\end{Prop}

\begin{Def}
Un proceso estoc\'astico $X\left(t\right)$ es crudamente regenerativo en un tiempo aleatorio positivo $T$ si
\begin{eqnarray*}
\esp\left[X\left(T+t\right)|T\right]=\esp\left[X\left(t\right)\right]\textrm{, para }t\geq0,\end{eqnarray*}
y con las esperanzas anteriores finitas.
\end{Def}

\begin{Prop}
Sup\'ongase que $X\left(t\right)$ es un proceso crudamente regenerativo en $T$, que tiene distribuci\'on $F$. Si $\esp\left[X\left(t\right)\right]$ es acotado en intervalos finitos, entonces
\begin{eqnarray*}
\esp\left[X\left(t\right)\right]=U\star h\left(t\right)\textrm{,  donde }h\left(t\right)=\esp\left[X\left(t\right)\indora\left(T>t\right)\right].
\end{eqnarray*}
\end{Prop}

\begin{Teo}[Regeneraci\'on Cruda]
Sup\'ongase que $X\left(t\right)$ es un proceso con valores positivo crudamente regenerativo en $T$, y def\'inase $M=\sup\left\{|X\left(t\right)|:t\leq T\right\}$. Si $T$ es no aritm\'etico y $M$ y $MT$ tienen media finita, entonces
\begin{eqnarray*}
lim_{t\rightarrow\infty}\esp\left[X\left(t\right)\right]=\frac{1}{\mu}\int_{\rea_{+}}h\left(s\right)ds,
\end{eqnarray*}
donde $h\left(t\right)=\esp\left[X\left(t\right)\indora\left(T>t\right)\right]$.
\end{Teo}

%___________________________________________________________________________________________
%
%\subsection{Propiedades de los Procesos de Renovaci\'on}
%___________________________________________________________________________________________
%

Los tiempos $T_{n}$ est\'an relacionados con los conteos de $N\left(t\right)$ por

\begin{eqnarray*}
\left\{N\left(t\right)\geq n\right\}&=&\left\{T_{n}\leq t\right\}\\
T_{N\left(t\right)}\leq &t&<T_{N\left(t\right)+1},
\end{eqnarray*}

adem\'as $N\left(T_{n}\right)=n$, y 

\begin{eqnarray*}
N\left(t\right)=\max\left\{n:T_{n}\leq t\right\}=\min\left\{n:T_{n+1}>t\right\}
\end{eqnarray*}

Por propiedades de la convoluci\'on se sabe que

\begin{eqnarray*}
P\left\{T_{n}\leq t\right\}=F^{n\star}\left(t\right)
\end{eqnarray*}
que es la $n$-\'esima convoluci\'on de $F$. Entonces 

\begin{eqnarray*}
\left\{N\left(t\right)\geq n\right\}&=&\left\{T_{n}\leq t\right\}\\
P\left\{N\left(t\right)\leq n\right\}&=&1-F^{\left(n+1\right)\star}\left(t\right)
\end{eqnarray*}

Adem\'as usando el hecho de que $\esp\left[N\left(t\right)\right]=\sum_{n=1}^{\infty}P\left\{N\left(t\right)\geq n\right\}$
se tiene que

\begin{eqnarray*}
\esp\left[N\left(t\right)\right]=\sum_{n=1}^{\infty}F^{n\star}\left(t\right)
\end{eqnarray*}

\begin{Prop}
Para cada $t\geq0$, la funci\'on generadora de momentos $\esp\left[e^{\alpha N\left(t\right)}\right]$ existe para alguna $\alpha$ en una vecindad del 0, y de aqu\'i que $\esp\left[N\left(t\right)^{m}\right]<\infty$, para $m\geq1$.
\end{Prop}


\begin{Note}
Si el primer tiempo de renovaci\'on $\xi_{1}$ no tiene la misma distribuci\'on que el resto de las $\xi_{n}$, para $n\geq2$, a $N\left(t\right)$ se le llama Proceso de Renovaci\'on retardado, donde si $\xi$ tiene distribuci\'on $G$, entonces el tiempo $T_{n}$ de la $n$-\'esima renovaci\'on tiene distribuci\'on $G\star F^{\left(n-1\right)\star}\left(t\right)$
\end{Note}


\begin{Teo}
Para una constante $\mu\leq\infty$ ( o variable aleatoria), las siguientes expresiones son equivalentes:

\begin{eqnarray}
lim_{n\rightarrow\infty}n^{-1}T_{n}&=&\mu,\textrm{ c.s.}\\
lim_{t\rightarrow\infty}t^{-1}N\left(t\right)&=&1/\mu,\textrm{ c.s.}
\end{eqnarray}
\end{Teo}


Es decir, $T_{n}$ satisface la Ley Fuerte de los Grandes N\'umeros s\'i y s\'olo s\'i $N\left/t\right)$ la cumple.


\begin{Coro}[Ley Fuerte de los Grandes N\'umeros para Procesos de Renovaci\'on]
Si $N\left(t\right)$ es un proceso de renovaci\'on cuyos tiempos de inter-renovaci\'on tienen media $\mu\leq\infty$, entonces
\begin{eqnarray}
t^{-1}N\left(t\right)\rightarrow 1/\mu,\textrm{ c.s. cuando }t\rightarrow\infty.
\end{eqnarray}

\end{Coro}


Considerar el proceso estoc\'astico de valores reales $\left\{Z\left(t\right):t\geq0\right\}$ en el mismo espacio de probabilidad que $N\left(t\right)$

\begin{Def}
Para el proceso $\left\{Z\left(t\right):t\geq0\right\}$ se define la fluctuaci\'on m\'axima de $Z\left(t\right)$ en el intervalo $\left(T_{n-1},T_{n}\right]$:
\begin{eqnarray*}
M_{n}=\sup_{T_{n-1}<t\leq T_{n}}|Z\left(t\right)-Z\left(T_{n-1}\right)|
\end{eqnarray*}
\end{Def}

\begin{Teo}
Sup\'ongase que $n^{-1}T_{n}\rightarrow\mu$ c.s. cuando $n\rightarrow\infty$, donde $\mu\leq\infty$ es una constante o variable aleatoria. Sea $a$ una constante o variable aleatoria que puede ser infinita cuando $\mu$ es finita, y considere las expresiones l\'imite:
\begin{eqnarray}
lim_{n\rightarrow\infty}n^{-1}Z\left(T_{n}\right)&=&a,\textrm{ c.s.}\\
lim_{t\rightarrow\infty}t^{-1}Z\left(t\right)&=&a/\mu,\textrm{ c.s.}
\end{eqnarray}
La segunda expresi\'on implica la primera. Conversamente, la primera implica la segunda si el proceso $Z\left(t\right)$ es creciente, o si $lim_{n\rightarrow\infty}n^{-1}M_{n}=0$ c.s.
\end{Teo}

\begin{Coro}
Si $N\left(t\right)$ es un proceso de renovaci\'on, y $\left(Z\left(T_{n}\right)-Z\left(T_{n-1}\right),M_{n}\right)$, para $n\geq1$, son variables aleatorias independientes e id\'enticamente distribuidas con media finita, entonces,
\begin{eqnarray}
lim_{t\rightarrow\infty}t^{-1}Z\left(t\right)\rightarrow\frac{\esp\left[Z\left(T_{1}\right)-Z\left(T_{0}\right)\right]}{\esp\left[T_{1}\right]},\textrm{ c.s. cuando  }t\rightarrow\infty.
\end{eqnarray}
\end{Coro}



%___________________________________________________________________________________________
%
%\subsection{Propiedades de los Procesos de Renovaci\'on}
%___________________________________________________________________________________________
%

Los tiempos $T_{n}$ est\'an relacionados con los conteos de $N\left(t\right)$ por

\begin{eqnarray*}
\left\{N\left(t\right)\geq n\right\}&=&\left\{T_{n}\leq t\right\}\\
T_{N\left(t\right)}\leq &t&<T_{N\left(t\right)+1},
\end{eqnarray*}

adem\'as $N\left(T_{n}\right)=n$, y 

\begin{eqnarray*}
N\left(t\right)=\max\left\{n:T_{n}\leq t\right\}=\min\left\{n:T_{n+1}>t\right\}
\end{eqnarray*}

Por propiedades de la convoluci\'on se sabe que

\begin{eqnarray*}
P\left\{T_{n}\leq t\right\}=F^{n\star}\left(t\right)
\end{eqnarray*}
que es la $n$-\'esima convoluci\'on de $F$. Entonces 

\begin{eqnarray*}
\left\{N\left(t\right)\geq n\right\}&=&\left\{T_{n}\leq t\right\}\\
P\left\{N\left(t\right)\leq n\right\}&=&1-F^{\left(n+1\right)\star}\left(t\right)
\end{eqnarray*}

Adem\'as usando el hecho de que $\esp\left[N\left(t\right)\right]=\sum_{n=1}^{\infty}P\left\{N\left(t\right)\geq n\right\}$
se tiene que

\begin{eqnarray*}
\esp\left[N\left(t\right)\right]=\sum_{n=1}^{\infty}F^{n\star}\left(t\right)
\end{eqnarray*}

\begin{Prop}
Para cada $t\geq0$, la funci\'on generadora de momentos $\esp\left[e^{\alpha N\left(t\right)}\right]$ existe para alguna $\alpha$ en una vecindad del 0, y de aqu\'i que $\esp\left[N\left(t\right)^{m}\right]<\infty$, para $m\geq1$.
\end{Prop}


\begin{Note}
Si el primer tiempo de renovaci\'on $\xi_{1}$ no tiene la misma distribuci\'on que el resto de las $\xi_{n}$, para $n\geq2$, a $N\left(t\right)$ se le llama Proceso de Renovaci\'on retardado, donde si $\xi$ tiene distribuci\'on $G$, entonces el tiempo $T_{n}$ de la $n$-\'esima renovaci\'on tiene distribuci\'on $G\star F^{\left(n-1\right)\star}\left(t\right)$
\end{Note}


\begin{Teo}
Para una constante $\mu\leq\infty$ ( o variable aleatoria), las siguientes expresiones son equivalentes:

\begin{eqnarray}
lim_{n\rightarrow\infty}n^{-1}T_{n}&=&\mu,\textrm{ c.s.}\\
lim_{t\rightarrow\infty}t^{-1}N\left(t\right)&=&1/\mu,\textrm{ c.s.}
\end{eqnarray}
\end{Teo}


Es decir, $T_{n}$ satisface la Ley Fuerte de los Grandes N\'umeros s\'i y s\'olo s\'i $N\left/t\right)$ la cumple.


\begin{Coro}[Ley Fuerte de los Grandes N\'umeros para Procesos de Renovaci\'on]
Si $N\left(t\right)$ es un proceso de renovaci\'on cuyos tiempos de inter-renovaci\'on tienen media $\mu\leq\infty$, entonces
\begin{eqnarray}
t^{-1}N\left(t\right)\rightarrow 1/\mu,\textrm{ c.s. cuando }t\rightarrow\infty.
\end{eqnarray}

\end{Coro}


Considerar el proceso estoc\'astico de valores reales $\left\{Z\left(t\right):t\geq0\right\}$ en el mismo espacio de probabilidad que $N\left(t\right)$

\begin{Def}
Para el proceso $\left\{Z\left(t\right):t\geq0\right\}$ se define la fluctuaci\'on m\'axima de $Z\left(t\right)$ en el intervalo $\left(T_{n-1},T_{n}\right]$:
\begin{eqnarray*}
M_{n}=\sup_{T_{n-1}<t\leq T_{n}}|Z\left(t\right)-Z\left(T_{n-1}\right)|
\end{eqnarray*}
\end{Def}

\begin{Teo}
Sup\'ongase que $n^{-1}T_{n}\rightarrow\mu$ c.s. cuando $n\rightarrow\infty$, donde $\mu\leq\infty$ es una constante o variable aleatoria. Sea $a$ una constante o variable aleatoria que puede ser infinita cuando $\mu$ es finita, y considere las expresiones l\'imite:
\begin{eqnarray}
lim_{n\rightarrow\infty}n^{-1}Z\left(T_{n}\right)&=&a,\textrm{ c.s.}\\
lim_{t\rightarrow\infty}t^{-1}Z\left(t\right)&=&a/\mu,\textrm{ c.s.}
\end{eqnarray}
La segunda expresi\'on implica la primera. Conversamente, la primera implica la segunda si el proceso $Z\left(t\right)$ es creciente, o si $lim_{n\rightarrow\infty}n^{-1}M_{n}=0$ c.s.
\end{Teo}

\begin{Coro}
Si $N\left(t\right)$ es un proceso de renovaci\'on, y $\left(Z\left(T_{n}\right)-Z\left(T_{n-1}\right),M_{n}\right)$, para $n\geq1$, son variables aleatorias independientes e id\'enticamente distribuidas con media finita, entonces,
\begin{eqnarray}
lim_{t\rightarrow\infty}t^{-1}Z\left(t\right)\rightarrow\frac{\esp\left[Z\left(T_{1}\right)-Z\left(T_{0}\right)\right]}{\esp\left[T_{1}\right]},\textrm{ c.s. cuando  }t\rightarrow\infty.
\end{eqnarray}
\end{Coro}


%___________________________________________________________________________________________
%
%\subsection{Propiedades de los Procesos de Renovaci\'on}
%___________________________________________________________________________________________
%

Los tiempos $T_{n}$ est\'an relacionados con los conteos de $N\left(t\right)$ por

\begin{eqnarray*}
\left\{N\left(t\right)\geq n\right\}&=&\left\{T_{n}\leq t\right\}\\
T_{N\left(t\right)}\leq &t&<T_{N\left(t\right)+1},
\end{eqnarray*}

adem\'as $N\left(T_{n}\right)=n$, y 

\begin{eqnarray*}
N\left(t\right)=\max\left\{n:T_{n}\leq t\right\}=\min\left\{n:T_{n+1}>t\right\}
\end{eqnarray*}

Por propiedades de la convoluci\'on se sabe que

\begin{eqnarray*}
P\left\{T_{n}\leq t\right\}=F^{n\star}\left(t\right)
\end{eqnarray*}
que es la $n$-\'esima convoluci\'on de $F$. Entonces 

\begin{eqnarray*}
\left\{N\left(t\right)\geq n\right\}&=&\left\{T_{n}\leq t\right\}\\
P\left\{N\left(t\right)\leq n\right\}&=&1-F^{\left(n+1\right)\star}\left(t\right)
\end{eqnarray*}

Adem\'as usando el hecho de que $\esp\left[N\left(t\right)\right]=\sum_{n=1}^{\infty}P\left\{N\left(t\right)\geq n\right\}$
se tiene que

\begin{eqnarray*}
\esp\left[N\left(t\right)\right]=\sum_{n=1}^{\infty}F^{n\star}\left(t\right)
\end{eqnarray*}

\begin{Prop}
Para cada $t\geq0$, la funci\'on generadora de momentos $\esp\left[e^{\alpha N\left(t\right)}\right]$ existe para alguna $\alpha$ en una vecindad del 0, y de aqu\'i que $\esp\left[N\left(t\right)^{m}\right]<\infty$, para $m\geq1$.
\end{Prop}


\begin{Note}
Si el primer tiempo de renovaci\'on $\xi_{1}$ no tiene la misma distribuci\'on que el resto de las $\xi_{n}$, para $n\geq2$, a $N\left(t\right)$ se le llama Proceso de Renovaci\'on retardado, donde si $\xi$ tiene distribuci\'on $G$, entonces el tiempo $T_{n}$ de la $n$-\'esima renovaci\'on tiene distribuci\'on $G\star F^{\left(n-1\right)\star}\left(t\right)$
\end{Note}


\begin{Teo}
Para una constante $\mu\leq\infty$ ( o variable aleatoria), las siguientes expresiones son equivalentes:

\begin{eqnarray}
lim_{n\rightarrow\infty}n^{-1}T_{n}&=&\mu,\textrm{ c.s.}\\
lim_{t\rightarrow\infty}t^{-1}N\left(t\right)&=&1/\mu,\textrm{ c.s.}
\end{eqnarray}
\end{Teo}


Es decir, $T_{n}$ satisface la Ley Fuerte de los Grandes N\'umeros s\'i y s\'olo s\'i $N\left/t\right)$ la cumple.


\begin{Coro}[Ley Fuerte de los Grandes N\'umeros para Procesos de Renovaci\'on]
Si $N\left(t\right)$ es un proceso de renovaci\'on cuyos tiempos de inter-renovaci\'on tienen media $\mu\leq\infty$, entonces
\begin{eqnarray}
t^{-1}N\left(t\right)\rightarrow 1/\mu,\textrm{ c.s. cuando }t\rightarrow\infty.
\end{eqnarray}

\end{Coro}


Considerar el proceso estoc\'astico de valores reales $\left\{Z\left(t\right):t\geq0\right\}$ en el mismo espacio de probabilidad que $N\left(t\right)$

\begin{Def}
Para el proceso $\left\{Z\left(t\right):t\geq0\right\}$ se define la fluctuaci\'on m\'axima de $Z\left(t\right)$ en el intervalo $\left(T_{n-1},T_{n}\right]$:
\begin{eqnarray*}
M_{n}=\sup_{T_{n-1}<t\leq T_{n}}|Z\left(t\right)-Z\left(T_{n-1}\right)|
\end{eqnarray*}
\end{Def}

\begin{Teo}
Sup\'ongase que $n^{-1}T_{n}\rightarrow\mu$ c.s. cuando $n\rightarrow\infty$, donde $\mu\leq\infty$ es una constante o variable aleatoria. Sea $a$ una constante o variable aleatoria que puede ser infinita cuando $\mu$ es finita, y considere las expresiones l\'imite:
\begin{eqnarray}
lim_{n\rightarrow\infty}n^{-1}Z\left(T_{n}\right)&=&a,\textrm{ c.s.}\\
lim_{t\rightarrow\infty}t^{-1}Z\left(t\right)&=&a/\mu,\textrm{ c.s.}
\end{eqnarray}
La segunda expresi\'on implica la primera. Conversamente, la primera implica la segunda si el proceso $Z\left(t\right)$ es creciente, o si $lim_{n\rightarrow\infty}n^{-1}M_{n}=0$ c.s.
\end{Teo}

\begin{Coro}
Si $N\left(t\right)$ es un proceso de renovaci\'on, y $\left(Z\left(T_{n}\right)-Z\left(T_{n-1}\right),M_{n}\right)$, para $n\geq1$, son variables aleatorias independientes e id\'enticamente distribuidas con media finita, entonces,
\begin{eqnarray}
lim_{t\rightarrow\infty}t^{-1}Z\left(t\right)\rightarrow\frac{\esp\left[Z\left(T_{1}\right)-Z\left(T_{0}\right)\right]}{\esp\left[T_{1}\right]},\textrm{ c.s. cuando  }t\rightarrow\infty.
\end{eqnarray}
\end{Coro}

%___________________________________________________________________________________________
%
%\subsection{Propiedades de los Procesos de Renovaci\'on}
%___________________________________________________________________________________________
%

Los tiempos $T_{n}$ est\'an relacionados con los conteos de $N\left(t\right)$ por

\begin{eqnarray*}
\left\{N\left(t\right)\geq n\right\}&=&\left\{T_{n}\leq t\right\}\\
T_{N\left(t\right)}\leq &t&<T_{N\left(t\right)+1},
\end{eqnarray*}

adem\'as $N\left(T_{n}\right)=n$, y 

\begin{eqnarray*}
N\left(t\right)=\max\left\{n:T_{n}\leq t\right\}=\min\left\{n:T_{n+1}>t\right\}
\end{eqnarray*}

Por propiedades de la convoluci\'on se sabe que

\begin{eqnarray*}
P\left\{T_{n}\leq t\right\}=F^{n\star}\left(t\right)
\end{eqnarray*}
que es la $n$-\'esima convoluci\'on de $F$. Entonces 

\begin{eqnarray*}
\left\{N\left(t\right)\geq n\right\}&=&\left\{T_{n}\leq t\right\}\\
P\left\{N\left(t\right)\leq n\right\}&=&1-F^{\left(n+1\right)\star}\left(t\right)
\end{eqnarray*}

Adem\'as usando el hecho de que $\esp\left[N\left(t\right)\right]=\sum_{n=1}^{\infty}P\left\{N\left(t\right)\geq n\right\}$
se tiene que

\begin{eqnarray*}
\esp\left[N\left(t\right)\right]=\sum_{n=1}^{\infty}F^{n\star}\left(t\right)
\end{eqnarray*}

\begin{Prop}
Para cada $t\geq0$, la funci\'on generadora de momentos $\esp\left[e^{\alpha N\left(t\right)}\right]$ existe para alguna $\alpha$ en una vecindad del 0, y de aqu\'i que $\esp\left[N\left(t\right)^{m}\right]<\infty$, para $m\geq1$.
\end{Prop}


\begin{Note}
Si el primer tiempo de renovaci\'on $\xi_{1}$ no tiene la misma distribuci\'on que el resto de las $\xi_{n}$, para $n\geq2$, a $N\left(t\right)$ se le llama Proceso de Renovaci\'on retardado, donde si $\xi$ tiene distribuci\'on $G$, entonces el tiempo $T_{n}$ de la $n$-\'esima renovaci\'on tiene distribuci\'on $G\star F^{\left(n-1\right)\star}\left(t\right)$
\end{Note}


\begin{Teo}
Para una constante $\mu\leq\infty$ ( o variable aleatoria), las siguientes expresiones son equivalentes:

\begin{eqnarray}
lim_{n\rightarrow\infty}n^{-1}T_{n}&=&\mu,\textrm{ c.s.}\\
lim_{t\rightarrow\infty}t^{-1}N\left(t\right)&=&1/\mu,\textrm{ c.s.}
\end{eqnarray}
\end{Teo}


Es decir, $T_{n}$ satisface la Ley Fuerte de los Grandes N\'umeros s\'i y s\'olo s\'i $N\left/t\right)$ la cumple.


\begin{Coro}[Ley Fuerte de los Grandes N\'umeros para Procesos de Renovaci\'on]
Si $N\left(t\right)$ es un proceso de renovaci\'on cuyos tiempos de inter-renovaci\'on tienen media $\mu\leq\infty$, entonces
\begin{eqnarray}
t^{-1}N\left(t\right)\rightarrow 1/\mu,\textrm{ c.s. cuando }t\rightarrow\infty.
\end{eqnarray}

\end{Coro}


Considerar el proceso estoc\'astico de valores reales $\left\{Z\left(t\right):t\geq0\right\}$ en el mismo espacio de probabilidad que $N\left(t\right)$

\begin{Def}
Para el proceso $\left\{Z\left(t\right):t\geq0\right\}$ se define la fluctuaci\'on m\'axima de $Z\left(t\right)$ en el intervalo $\left(T_{n-1},T_{n}\right]$:
\begin{eqnarray*}
M_{n}=\sup_{T_{n-1}<t\leq T_{n}}|Z\left(t\right)-Z\left(T_{n-1}\right)|
\end{eqnarray*}
\end{Def}

\begin{Teo}
Sup\'ongase que $n^{-1}T_{n}\rightarrow\mu$ c.s. cuando $n\rightarrow\infty$, donde $\mu\leq\infty$ es una constante o variable aleatoria. Sea $a$ una constante o variable aleatoria que puede ser infinita cuando $\mu$ es finita, y considere las expresiones l\'imite:
\begin{eqnarray}
lim_{n\rightarrow\infty}n^{-1}Z\left(T_{n}\right)&=&a,\textrm{ c.s.}\\
lim_{t\rightarrow\infty}t^{-1}Z\left(t\right)&=&a/\mu,\textrm{ c.s.}
\end{eqnarray}
La segunda expresi\'on implica la primera. Conversamente, la primera implica la segunda si el proceso $Z\left(t\right)$ es creciente, o si $lim_{n\rightarrow\infty}n^{-1}M_{n}=0$ c.s.
\end{Teo}

\begin{Coro}
Si $N\left(t\right)$ es un proceso de renovaci\'on, y $\left(Z\left(T_{n}\right)-Z\left(T_{n-1}\right),M_{n}\right)$, para $n\geq1$, son variables aleatorias independientes e id\'enticamente distribuidas con media finita, entonces,
\begin{eqnarray}
lim_{t\rightarrow\infty}t^{-1}Z\left(t\right)\rightarrow\frac{\esp\left[Z\left(T_{1}\right)-Z\left(T_{0}\right)\right]}{\esp\left[T_{1}\right]},\textrm{ c.s. cuando  }t\rightarrow\infty.
\end{eqnarray}
\end{Coro}
%___________________________________________________________________________________________
%
%\subsection{Propiedades de los Procesos de Renovaci\'on}
%___________________________________________________________________________________________
%

Los tiempos $T_{n}$ est\'an relacionados con los conteos de $N\left(t\right)$ por

\begin{eqnarray*}
\left\{N\left(t\right)\geq n\right\}&=&\left\{T_{n}\leq t\right\}\\
T_{N\left(t\right)}\leq &t&<T_{N\left(t\right)+1},
\end{eqnarray*}

adem\'as $N\left(T_{n}\right)=n$, y 

\begin{eqnarray*}
N\left(t\right)=\max\left\{n:T_{n}\leq t\right\}=\min\left\{n:T_{n+1}>t\right\}
\end{eqnarray*}

Por propiedades de la convoluci\'on se sabe que

\begin{eqnarray*}
P\left\{T_{n}\leq t\right\}=F^{n\star}\left(t\right)
\end{eqnarray*}
que es la $n$-\'esima convoluci\'on de $F$. Entonces 

\begin{eqnarray*}
\left\{N\left(t\right)\geq n\right\}&=&\left\{T_{n}\leq t\right\}\\
P\left\{N\left(t\right)\leq n\right\}&=&1-F^{\left(n+1\right)\star}\left(t\right)
\end{eqnarray*}

Adem\'as usando el hecho de que $\esp\left[N\left(t\right)\right]=\sum_{n=1}^{\infty}P\left\{N\left(t\right)\geq n\right\}$
se tiene que

\begin{eqnarray*}
\esp\left[N\left(t\right)\right]=\sum_{n=1}^{\infty}F^{n\star}\left(t\right)
\end{eqnarray*}

\begin{Prop}
Para cada $t\geq0$, la funci\'on generadora de momentos $\esp\left[e^{\alpha N\left(t\right)}\right]$ existe para alguna $\alpha$ en una vecindad del 0, y de aqu\'i que $\esp\left[N\left(t\right)^{m}\right]<\infty$, para $m\geq1$.
\end{Prop}


\begin{Note}
Si el primer tiempo de renovaci\'on $\xi_{1}$ no tiene la misma distribuci\'on que el resto de las $\xi_{n}$, para $n\geq2$, a $N\left(t\right)$ se le llama Proceso de Renovaci\'on retardado, donde si $\xi$ tiene distribuci\'on $G$, entonces el tiempo $T_{n}$ de la $n$-\'esima renovaci\'on tiene distribuci\'on $G\star F^{\left(n-1\right)\star}\left(t\right)$
\end{Note}


\begin{Teo}
Para una constante $\mu\leq\infty$ ( o variable aleatoria), las siguientes expresiones son equivalentes:

\begin{eqnarray}
lim_{n\rightarrow\infty}n^{-1}T_{n}&=&\mu,\textrm{ c.s.}\\
lim_{t\rightarrow\infty}t^{-1}N\left(t\right)&=&1/\mu,\textrm{ c.s.}
\end{eqnarray}
\end{Teo}


Es decir, $T_{n}$ satisface la Ley Fuerte de los Grandes N\'umeros s\'i y s\'olo s\'i $N\left/t\right)$ la cumple.


\begin{Coro}[Ley Fuerte de los Grandes N\'umeros para Procesos de Renovaci\'on]
Si $N\left(t\right)$ es un proceso de renovaci\'on cuyos tiempos de inter-renovaci\'on tienen media $\mu\leq\infty$, entonces
\begin{eqnarray}
t^{-1}N\left(t\right)\rightarrow 1/\mu,\textrm{ c.s. cuando }t\rightarrow\infty.
\end{eqnarray}

\end{Coro}


Considerar el proceso estoc\'astico de valores reales $\left\{Z\left(t\right):t\geq0\right\}$ en el mismo espacio de probabilidad que $N\left(t\right)$

\begin{Def}
Para el proceso $\left\{Z\left(t\right):t\geq0\right\}$ se define la fluctuaci\'on m\'axima de $Z\left(t\right)$ en el intervalo $\left(T_{n-1},T_{n}\right]$:
\begin{eqnarray*}
M_{n}=\sup_{T_{n-1}<t\leq T_{n}}|Z\left(t\right)-Z\left(T_{n-1}\right)|
\end{eqnarray*}
\end{Def}

\begin{Teo}
Sup\'ongase que $n^{-1}T_{n}\rightarrow\mu$ c.s. cuando $n\rightarrow\infty$, donde $\mu\leq\infty$ es una constante o variable aleatoria. Sea $a$ una constante o variable aleatoria que puede ser infinita cuando $\mu$ es finita, y considere las expresiones l\'imite:
\begin{eqnarray}
lim_{n\rightarrow\infty}n^{-1}Z\left(T_{n}\right)&=&a,\textrm{ c.s.}\\
lim_{t\rightarrow\infty}t^{-1}Z\left(t\right)&=&a/\mu,\textrm{ c.s.}
\end{eqnarray}
La segunda expresi\'on implica la primera. Conversamente, la primera implica la segunda si el proceso $Z\left(t\right)$ es creciente, o si $lim_{n\rightarrow\infty}n^{-1}M_{n}=0$ c.s.
\end{Teo}

\begin{Coro}
Si $N\left(t\right)$ es un proceso de renovaci\'on, y $\left(Z\left(T_{n}\right)-Z\left(T_{n-1}\right),M_{n}\right)$, para $n\geq1$, son variables aleatorias independientes e id\'enticamente distribuidas con media finita, entonces,
\begin{eqnarray}
lim_{t\rightarrow\infty}t^{-1}Z\left(t\right)\rightarrow\frac{\esp\left[Z\left(T_{1}\right)-Z\left(T_{0}\right)\right]}{\esp\left[T_{1}\right]},\textrm{ c.s. cuando  }t\rightarrow\infty.
\end{eqnarray}
\end{Coro}


%___________________________________________________________________________________________
%
%\subsection{Funci\'on de Renovaci\'on}
%___________________________________________________________________________________________
%


\begin{Def}
Sea $h\left(t\right)$ funci\'on de valores reales en $\rea$ acotada en intervalos finitos e igual a cero para $t<0$ La ecuaci\'on de renovaci\'on para $h\left(t\right)$ y la distribuci\'on $F$ es

\begin{eqnarray}%\label{Ec.Renovacion}
H\left(t\right)=h\left(t\right)+\int_{\left[0,t\right]}H\left(t-s\right)dF\left(s\right)\textrm{,    }t\geq0,
\end{eqnarray}
donde $H\left(t\right)$ es una funci\'on de valores reales. Esto es $H=h+F\star H$. Decimos que $H\left(t\right)$ es soluci\'on de esta ecuaci\'on si satisface la ecuaci\'on, y es acotada en intervalos finitos e iguales a cero para $t<0$.
\end{Def}

\begin{Prop}
La funci\'on $U\star h\left(t\right)$ es la \'unica soluci\'on de la ecuaci\'on de renovaci\'on (\ref{Ec.Renovacion}).
\end{Prop}

\begin{Teo}[Teorema Renovaci\'on Elemental]
\begin{eqnarray*}
t^{-1}U\left(t\right)\rightarrow 1/\mu\textrm{,    cuando }t\rightarrow\infty.
\end{eqnarray*}
\end{Teo}

%___________________________________________________________________________________________
%
%\subsection{Funci\'on de Renovaci\'on}
%___________________________________________________________________________________________
%


Sup\'ongase que $N\left(t\right)$ es un proceso de renovaci\'on con distribuci\'on $F$ con media finita $\mu$.

\begin{Def}
La funci\'on de renovaci\'on asociada con la distribuci\'on $F$, del proceso $N\left(t\right)$, es
\begin{eqnarray*}
U\left(t\right)=\sum_{n=1}^{\infty}F^{n\star}\left(t\right),\textrm{   }t\geq0,
\end{eqnarray*}
donde $F^{0\star}\left(t\right)=\indora\left(t\geq0\right)$.
\end{Def}


\begin{Prop}
Sup\'ongase que la distribuci\'on de inter-renovaci\'on $F$ tiene densidad $f$. Entonces $U\left(t\right)$ tambi\'en tiene densidad, para $t>0$, y es $U^{'}\left(t\right)=\sum_{n=0}^{\infty}f^{n\star}\left(t\right)$. Adem\'as
\begin{eqnarray*}
\prob\left\{N\left(t\right)>N\left(t-\right)\right\}=0\textrm{,   }t\geq0.
\end{eqnarray*}
\end{Prop}

\begin{Def}
La Transformada de Laplace-Stieljes de $F$ est\'a dada por

\begin{eqnarray*}
\hat{F}\left(\alpha\right)=\int_{\rea_{+}}e^{-\alpha t}dF\left(t\right)\textrm{,  }\alpha\geq0.
\end{eqnarray*}
\end{Def}

Entonces

\begin{eqnarray*}
\hat{U}\left(\alpha\right)=\sum_{n=0}^{\infty}\hat{F^{n\star}}\left(\alpha\right)=\sum_{n=0}^{\infty}\hat{F}\left(\alpha\right)^{n}=\frac{1}{1-\hat{F}\left(\alpha\right)}.
\end{eqnarray*}


\begin{Prop}
La Transformada de Laplace $\hat{U}\left(\alpha\right)$ y $\hat{F}\left(\alpha\right)$ determina una a la otra de manera \'unica por la relaci\'on $\hat{U}\left(\alpha\right)=\frac{1}{1-\hat{F}\left(\alpha\right)}$.
\end{Prop}


\begin{Note}
Un proceso de renovaci\'on $N\left(t\right)$ cuyos tiempos de inter-renovaci\'on tienen media finita, es un proceso Poisson con tasa $\lambda$ si y s\'olo s\'i $\esp\left[U\left(t\right)\right]=\lambda t$, para $t\geq0$.
\end{Note}


\begin{Teo}
Sea $N\left(t\right)$ un proceso puntual simple con puntos de localizaci\'on $T_{n}$ tal que $\eta\left(t\right)=\esp\left[N\left(\right)\right]$ es finita para cada $t$. Entonces para cualquier funci\'on $f:\rea_{+}\rightarrow\rea$,
\begin{eqnarray*}
\esp\left[\sum_{n=1}^{N\left(\right)}f\left(T_{n}\right)\right]=\int_{\left(0,t\right]}f\left(s\right)d\eta\left(s\right)\textrm{,  }t\geq0,
\end{eqnarray*}
suponiendo que la integral exista. Adem\'as si $X_{1},X_{2},\ldots$ son variables aleatorias definidas en el mismo espacio de probabilidad que el proceso $N\left(t\right)$ tal que $\esp\left[X_{n}|T_{n}=s\right]=f\left(s\right)$, independiente de $n$. Entonces
\begin{eqnarray*}
\esp\left[\sum_{n=1}^{N\left(t\right)}X_{n}\right]=\int_{\left(0,t\right]}f\left(s\right)d\eta\left(s\right)\textrm{,  }t\geq0,
\end{eqnarray*} 
suponiendo que la integral exista. 
\end{Teo}

\begin{Coro}[Identidad de Wald para Renovaciones]
Para el proceso de renovaci\'on $N\left(t\right)$,
\begin{eqnarray*}
\esp\left[T_{N\left(t\right)+1}\right]=\mu\esp\left[N\left(t\right)+1\right]\textrm{,  }t\geq0,
\end{eqnarray*}  
\end{Coro}

%______________________________________________________________________
%\subsection{Procesos de Renovaci\'on}
%______________________________________________________________________

\begin{Def}%\label{Def.Tn}
Sean $0\leq T_{1}\leq T_{2}\leq \ldots$ son tiempos aleatorios infinitos en los cuales ocurren ciertos eventos. El n\'umero de tiempos $T_{n}$ en el intervalo $\left[0,t\right)$ es

\begin{eqnarray}
N\left(t\right)=\sum_{n=1}^{\infty}\indora\left(T_{n}\leq t\right),
\end{eqnarray}
para $t\geq0$.
\end{Def}

Si se consideran los puntos $T_{n}$ como elementos de $\rea_{+}$, y $N\left(t\right)$ es el n\'umero de puntos en $\rea$. El proceso denotado por $\left\{N\left(t\right):t\geq0\right\}$, denotado por $N\left(t\right)$, es un proceso puntual en $\rea_{+}$. Los $T_{n}$ son los tiempos de ocurrencia, el proceso puntual $N\left(t\right)$ es simple si su n\'umero de ocurrencias son distintas: $0<T_{1}<T_{2}<\ldots$ casi seguramente.

\begin{Def}
Un proceso puntual $N\left(t\right)$ es un proceso de renovaci\'on si los tiempos de interocurrencia $\xi_{n}=T_{n}-T_{n-1}$, para $n\geq1$, son independientes e identicamente distribuidos con distribuci\'on $F$, donde $F\left(0\right)=0$ y $T_{0}=0$. Los $T_{n}$ son llamados tiempos de renovaci\'on, referente a la independencia o renovaci\'on de la informaci\'on estoc\'astica en estos tiempos. Los $\xi_{n}$ son los tiempos de inter-renovaci\'on, y $N\left(t\right)$ es el n\'umero de renovaciones en el intervalo $\left[0,t\right)$
\end{Def}


\begin{Note}
Para definir un proceso de renovaci\'on para cualquier contexto, solamente hay que especificar una distribuci\'on $F$, con $F\left(0\right)=0$, para los tiempos de inter-renovaci\'on. La funci\'on $F$ en turno degune las otra variables aleatorias. De manera formal, existe un espacio de probabilidad y una sucesi\'on de variables aleatorias $\xi_{1},\xi_{2},\ldots$ definidas en este con distribuci\'on $F$. Entonces las otras cantidades son $T_{n}=\sum_{k=1}^{n}\xi_{k}$ y $N\left(t\right)=\sum_{n=1}^{\infty}\indora\left(T_{n}\leq t\right)$, donde $T_{n}\rightarrow\infty$ casi seguramente por la Ley Fuerte de los Grandes Números.
\end{Note}

%___________________________________________________________________________________________
%
%\subsection{Renewal and Regenerative Processes: Serfozo\cite{Serfozo}}
%___________________________________________________________________________________________
%
\begin{Def}%\label{Def.Tn}
Sean $0\leq T_{1}\leq T_{2}\leq \ldots$ son tiempos aleatorios infinitos en los cuales ocurren ciertos eventos. El n\'umero de tiempos $T_{n}$ en el intervalo $\left[0,t\right)$ es

\begin{eqnarray}
N\left(t\right)=\sum_{n=1}^{\infty}\indora\left(T_{n}\leq t\right),
\end{eqnarray}
para $t\geq0$.
\end{Def}

Si se consideran los puntos $T_{n}$ como elementos de $\rea_{+}$, y $N\left(t\right)$ es el n\'umero de puntos en $\rea$. El proceso denotado por $\left\{N\left(t\right):t\geq0\right\}$, denotado por $N\left(t\right)$, es un proceso puntual en $\rea_{+}$. Los $T_{n}$ son los tiempos de ocurrencia, el proceso puntual $N\left(t\right)$ es simple si su n\'umero de ocurrencias son distintas: $0<T_{1}<T_{2}<\ldots$ casi seguramente.

\begin{Def}
Un proceso puntual $N\left(t\right)$ es un proceso de renovaci\'on si los tiempos de interocurrencia $\xi_{n}=T_{n}-T_{n-1}$, para $n\geq1$, son independientes e identicamente distribuidos con distribuci\'on $F$, donde $F\left(0\right)=0$ y $T_{0}=0$. Los $T_{n}$ son llamados tiempos de renovaci\'on, referente a la independencia o renovaci\'on de la informaci\'on estoc\'astica en estos tiempos. Los $\xi_{n}$ son los tiempos de inter-renovaci\'on, y $N\left(t\right)$ es el n\'umero de renovaciones en el intervalo $\left[0,t\right)$
\end{Def}


\begin{Note}
Para definir un proceso de renovaci\'on para cualquier contexto, solamente hay que especificar una distribuci\'on $F$, con $F\left(0\right)=0$, para los tiempos de inter-renovaci\'on. La funci\'on $F$ en turno degune las otra variables aleatorias. De manera formal, existe un espacio de probabilidad y una sucesi\'on de variables aleatorias $\xi_{1},\xi_{2},\ldots$ definidas en este con distribuci\'on $F$. Entonces las otras cantidades son $T_{n}=\sum_{k=1}^{n}\xi_{k}$ y $N\left(t\right)=\sum_{n=1}^{\infty}\indora\left(T_{n}\leq t\right)$, donde $T_{n}\rightarrow\infty$ casi seguramente por la Ley Fuerte de los Grandes N\'umeros.
\end{Note}







Los tiempos $T_{n}$ est\'an relacionados con los conteos de $N\left(t\right)$ por

\begin{eqnarray*}
\left\{N\left(t\right)\geq n\right\}&=&\left\{T_{n}\leq t\right\}\\
T_{N\left(t\right)}\leq &t&<T_{N\left(t\right)+1},
\end{eqnarray*}

adem\'as $N\left(T_{n}\right)=n$, y 

\begin{eqnarray*}
N\left(t\right)=\max\left\{n:T_{n}\leq t\right\}=\min\left\{n:T_{n+1}>t\right\}
\end{eqnarray*}

Por propiedades de la convoluci\'on se sabe que

\begin{eqnarray*}
P\left\{T_{n}\leq t\right\}=F^{n\star}\left(t\right)
\end{eqnarray*}
que es la $n$-\'esima convoluci\'on de $F$. Entonces 

\begin{eqnarray*}
\left\{N\left(t\right)\geq n\right\}&=&\left\{T_{n}\leq t\right\}\\
P\left\{N\left(t\right)\leq n\right\}&=&1-F^{\left(n+1\right)\star}\left(t\right)
\end{eqnarray*}

Adem\'as usando el hecho de que $\esp\left[N\left(t\right)\right]=\sum_{n=1}^{\infty}P\left\{N\left(t\right)\geq n\right\}$
se tiene que

\begin{eqnarray*}
\esp\left[N\left(t\right)\right]=\sum_{n=1}^{\infty}F^{n\star}\left(t\right)
\end{eqnarray*}

\begin{Prop}
Para cada $t\geq0$, la funci\'on generadora de momentos $\esp\left[e^{\alpha N\left(t\right)}\right]$ existe para alguna $\alpha$ en una vecindad del 0, y de aqu\'i que $\esp\left[N\left(t\right)^{m}\right]<\infty$, para $m\geq1$.
\end{Prop}

\begin{Ejem}[\textbf{Proceso Poisson}]

Suponga que se tienen tiempos de inter-renovaci\'on \textit{i.i.d.} del proceso de renovaci\'on $N\left(t\right)$ tienen distribuci\'on exponencial $F\left(t\right)=q-e^{-\lambda t}$ con tasa $\lambda$. Entonces $N\left(t\right)$ es un proceso Poisson con tasa $\lambda$.

\end{Ejem}


\begin{Note}
Si el primer tiempo de renovaci\'on $\xi_{1}$ no tiene la misma distribuci\'on que el resto de las $\xi_{n}$, para $n\geq2$, a $N\left(t\right)$ se le llama Proceso de Renovaci\'on retardado, donde si $\xi$ tiene distribuci\'on $G$, entonces el tiempo $T_{n}$ de la $n$-\'esima renovaci\'on tiene distribuci\'on $G\star F^{\left(n-1\right)\star}\left(t\right)$
\end{Note}


\begin{Teo}
Para una constante $\mu\leq\infty$ ( o variable aleatoria), las siguientes expresiones son equivalentes:

\begin{eqnarray}
lim_{n\rightarrow\infty}n^{-1}T_{n}&=&\mu,\textrm{ c.s.}\\
lim_{t\rightarrow\infty}t^{-1}N\left(t\right)&=&1/\mu,\textrm{ c.s.}
\end{eqnarray}
\end{Teo}


Es decir, $T_{n}$ satisface la Ley Fuerte de los Grandes N\'umeros s\'i y s\'olo s\'i $N\left/t\right)$ la cumple.


\begin{Coro}[Ley Fuerte de los Grandes N\'umeros para Procesos de Renovaci\'on]
Si $N\left(t\right)$ es un proceso de renovaci\'on cuyos tiempos de inter-renovaci\'on tienen media $\mu\leq\infty$, entonces
\begin{eqnarray}
t^{-1}N\left(t\right)\rightarrow 1/\mu,\textrm{ c.s. cuando }t\rightarrow\infty.
\end{eqnarray}

\end{Coro}


Considerar el proceso estoc\'astico de valores reales $\left\{Z\left(t\right):t\geq0\right\}$ en el mismo espacio de probabilidad que $N\left(t\right)$

\begin{Def}
Para el proceso $\left\{Z\left(t\right):t\geq0\right\}$ se define la fluctuaci\'on m\'axima de $Z\left(t\right)$ en el intervalo $\left(T_{n-1},T_{n}\right]$:
\begin{eqnarray*}
M_{n}=\sup_{T_{n-1}<t\leq T_{n}}|Z\left(t\right)-Z\left(T_{n-1}\right)|
\end{eqnarray*}
\end{Def}

\begin{Teo}
Sup\'ongase que $n^{-1}T_{n}\rightarrow\mu$ c.s. cuando $n\rightarrow\infty$, donde $\mu\leq\infty$ es una constante o variable aleatoria. Sea $a$ una constante o variable aleatoria que puede ser infinita cuando $\mu$ es finita, y considere las expresiones l\'imite:
\begin{eqnarray}
lim_{n\rightarrow\infty}n^{-1}Z\left(T_{n}\right)&=&a,\textrm{ c.s.}\\
lim_{t\rightarrow\infty}t^{-1}Z\left(t\right)&=&a/\mu,\textrm{ c.s.}
\end{eqnarray}
La segunda expresi\'on implica la primera. Conversamente, la primera implica la segunda si el proceso $Z\left(t\right)$ es creciente, o si $lim_{n\rightarrow\infty}n^{-1}M_{n}=0$ c.s.
\end{Teo}

\begin{Coro}
Si $N\left(t\right)$ es un proceso de renovaci\'on, y $\left(Z\left(T_{n}\right)-Z\left(T_{n-1}\right),M_{n}\right)$, para $n\geq1$, son variables aleatorias independientes e id\'enticamente distribuidas con media finita, entonces,
\begin{eqnarray}
lim_{t\rightarrow\infty}t^{-1}Z\left(t\right)\rightarrow\frac{\esp\left[Z\left(T_{1}\right)-Z\left(T_{0}\right)\right]}{\esp\left[T_{1}\right]},\textrm{ c.s. cuando  }t\rightarrow\infty.
\end{eqnarray}
\end{Coro}


Sup\'ongase que $N\left(t\right)$ es un proceso de renovaci\'on con distribuci\'on $F$ con media finita $\mu$.

\begin{Def}
La funci\'on de renovaci\'on asociada con la distribuci\'on $F$, del proceso $N\left(t\right)$, es
\begin{eqnarray*}
U\left(t\right)=\sum_{n=1}^{\infty}F^{n\star}\left(t\right),\textrm{   }t\geq0,
\end{eqnarray*}
donde $F^{0\star}\left(t\right)=\indora\left(t\geq0\right)$.
\end{Def}


\begin{Prop}
Sup\'ongase que la distribuci\'on de inter-renovaci\'on $F$ tiene densidad $f$. Entonces $U\left(t\right)$ tambi\'en tiene densidad, para $t>0$, y es $U^{'}\left(t\right)=\sum_{n=0}^{\infty}f^{n\star}\left(t\right)$. Adem\'as
\begin{eqnarray*}
\prob\left\{N\left(t\right)>N\left(t-\right)\right\}=0\textrm{,   }t\geq0.
\end{eqnarray*}
\end{Prop}

\begin{Def}
La Transformada de Laplace-Stieljes de $F$ est\'a dada por

\begin{eqnarray*}
\hat{F}\left(\alpha\right)=\int_{\rea_{+}}e^{-\alpha t}dF\left(t\right)\textrm{,  }\alpha\geq0.
\end{eqnarray*}
\end{Def}

Entonces

\begin{eqnarray*}
\hat{U}\left(\alpha\right)=\sum_{n=0}^{\infty}\hat{F^{n\star}}\left(\alpha\right)=\sum_{n=0}^{\infty}\hat{F}\left(\alpha\right)^{n}=\frac{1}{1-\hat{F}\left(\alpha\right)}.
\end{eqnarray*}


\begin{Prop}
La Transformada de Laplace $\hat{U}\left(\alpha\right)$ y $\hat{F}\left(\alpha\right)$ determina una a la otra de manera \'unica por la relaci\'on $\hat{U}\left(\alpha\right)=\frac{1}{1-\hat{F}\left(\alpha\right)}$.
\end{Prop}


\begin{Note}
Un proceso de renovaci\'on $N\left(t\right)$ cuyos tiempos de inter-renovaci\'on tienen media finita, es un proceso Poisson con tasa $\lambda$ si y s\'olo s\'i $\esp\left[U\left(t\right)\right]=\lambda t$, para $t\geq0$.
\end{Note}


\begin{Teo}
Sea $N\left(t\right)$ un proceso puntual simple con puntos de localizaci\'on $T_{n}$ tal que $\eta\left(t\right)=\esp\left[N\left(\right)\right]$ es finita para cada $t$. Entonces para cualquier funci\'on $f:\rea_{+}\rightarrow\rea$,
\begin{eqnarray*}
\esp\left[\sum_{n=1}^{N\left(\right)}f\left(T_{n}\right)\right]=\int_{\left(0,t\right]}f\left(s\right)d\eta\left(s\right)\textrm{,  }t\geq0,
\end{eqnarray*}
suponiendo que la integral exista. Adem\'as si $X_{1},X_{2},\ldots$ son variables aleatorias definidas en el mismo espacio de probabilidad que el proceso $N\left(t\right)$ tal que $\esp\left[X_{n}|T_{n}=s\right]=f\left(s\right)$, independiente de $n$. Entonces
\begin{eqnarray*}
\esp\left[\sum_{n=1}^{N\left(t\right)}X_{n}\right]=\int_{\left(0,t\right]}f\left(s\right)d\eta\left(s\right)\textrm{,  }t\geq0,
\end{eqnarray*} 
suponiendo que la integral exista. 
\end{Teo}

\begin{Coro}[Identidad de Wald para Renovaciones]
Para el proceso de renovaci\'on $N\left(t\right)$,
\begin{eqnarray*}
\esp\left[T_{N\left(t\right)+1}\right]=\mu\esp\left[N\left(t\right)+1\right]\textrm{,  }t\geq0,
\end{eqnarray*}  
\end{Coro}


\begin{Def}
Sea $h\left(t\right)$ funci\'on de valores reales en $\rea$ acotada en intervalos finitos e igual a cero para $t<0$ La ecuaci\'on de renovaci\'on para $h\left(t\right)$ y la distribuci\'on $F$ es

\begin{eqnarray}%\label{Ec.Renovacion}
H\left(t\right)=h\left(t\right)+\int_{\left[0,t\right]}H\left(t-s\right)dF\left(s\right)\textrm{,    }t\geq0,
\end{eqnarray}
donde $H\left(t\right)$ es una funci\'on de valores reales. Esto es $H=h+F\star H$. Decimos que $H\left(t\right)$ es soluci\'on de esta ecuaci\'on si satisface la ecuaci\'on, y es acotada en intervalos finitos e iguales a cero para $t<0$.
\end{Def}

\begin{Prop}
La funci\'on $U\star h\left(t\right)$ es la \'unica soluci\'on de la ecuaci\'on de renovaci\'on (\ref{Ec.Renovacion}).
\end{Prop}

\begin{Teo}[Teorema Renovaci\'on Elemental]
\begin{eqnarray*}
t^{-1}U\left(t\right)\rightarrow 1/\mu\textrm{,    cuando }t\rightarrow\infty.
\end{eqnarray*}
\end{Teo}



Sup\'ongase que $N\left(t\right)$ es un proceso de renovaci\'on con distribuci\'on $F$ con media finita $\mu$.

\begin{Def}
La funci\'on de renovaci\'on asociada con la distribuci\'on $F$, del proceso $N\left(t\right)$, es
\begin{eqnarray*}
U\left(t\right)=\sum_{n=1}^{\infty}F^{n\star}\left(t\right),\textrm{   }t\geq0,
\end{eqnarray*}
donde $F^{0\star}\left(t\right)=\indora\left(t\geq0\right)$.
\end{Def}


\begin{Prop}
Sup\'ongase que la distribuci\'on de inter-renovaci\'on $F$ tiene densidad $f$. Entonces $U\left(t\right)$ tambi\'en tiene densidad, para $t>0$, y es $U^{'}\left(t\right)=\sum_{n=0}^{\infty}f^{n\star}\left(t\right)$. Adem\'as
\begin{eqnarray*}
\prob\left\{N\left(t\right)>N\left(t-\right)\right\}=0\textrm{,   }t\geq0.
\end{eqnarray*}
\end{Prop}

\begin{Def}
La Transformada de Laplace-Stieljes de $F$ est\'a dada por

\begin{eqnarray*}
\hat{F}\left(\alpha\right)=\int_{\rea_{+}}e^{-\alpha t}dF\left(t\right)\textrm{,  }\alpha\geq0.
\end{eqnarray*}
\end{Def}

Entonces

\begin{eqnarray*}
\hat{U}\left(\alpha\right)=\sum_{n=0}^{\infty}\hat{F^{n\star}}\left(\alpha\right)=\sum_{n=0}^{\infty}\hat{F}\left(\alpha\right)^{n}=\frac{1}{1-\hat{F}\left(\alpha\right)}.
\end{eqnarray*}


\begin{Prop}
La Transformada de Laplace $\hat{U}\left(\alpha\right)$ y $\hat{F}\left(\alpha\right)$ determina una a la otra de manera \'unica por la relaci\'on $\hat{U}\left(\alpha\right)=\frac{1}{1-\hat{F}\left(\alpha\right)}$.
\end{Prop}


\begin{Note}
Un proceso de renovaci\'on $N\left(t\right)$ cuyos tiempos de inter-renovaci\'on tienen media finita, es un proceso Poisson con tasa $\lambda$ si y s\'olo s\'i $\esp\left[U\left(t\right)\right]=\lambda t$, para $t\geq0$.
\end{Note}


\begin{Teo}
Sea $N\left(t\right)$ un proceso puntual simple con puntos de localizaci\'on $T_{n}$ tal que $\eta\left(t\right)=\esp\left[N\left(\right)\right]$ es finita para cada $t$. Entonces para cualquier funci\'on $f:\rea_{+}\rightarrow\rea$,
\begin{eqnarray*}
\esp\left[\sum_{n=1}^{N\left(\right)}f\left(T_{n}\right)\right]=\int_{\left(0,t\right]}f\left(s\right)d\eta\left(s\right)\textrm{,  }t\geq0,
\end{eqnarray*}
suponiendo que la integral exista. Adem\'as si $X_{1},X_{2},\ldots$ son variables aleatorias definidas en el mismo espacio de probabilidad que el proceso $N\left(t\right)$ tal que $\esp\left[X_{n}|T_{n}=s\right]=f\left(s\right)$, independiente de $n$. Entonces
\begin{eqnarray*}
\esp\left[\sum_{n=1}^{N\left(t\right)}X_{n}\right]=\int_{\left(0,t\right]}f\left(s\right)d\eta\left(s\right)\textrm{,  }t\geq0,
\end{eqnarray*} 
suponiendo que la integral exista. 
\end{Teo}

\begin{Coro}[Identidad de Wald para Renovaciones]
Para el proceso de renovaci\'on $N\left(t\right)$,
\begin{eqnarray*}
\esp\left[T_{N\left(t\right)+1}\right]=\mu\esp\left[N\left(t\right)+1\right]\textrm{,  }t\geq0,
\end{eqnarray*}  
\end{Coro}


\begin{Def}
Sea $h\left(t\right)$ funci\'on de valores reales en $\rea$ acotada en intervalos finitos e igual a cero para $t<0$ La ecuaci\'on de renovaci\'on para $h\left(t\right)$ y la distribuci\'on $F$ es

\begin{eqnarray}%\label{Ec.Renovacion}
H\left(t\right)=h\left(t\right)+\int_{\left[0,t\right]}H\left(t-s\right)dF\left(s\right)\textrm{,    }t\geq0,
\end{eqnarray}
donde $H\left(t\right)$ es una funci\'on de valores reales. Esto es $H=h+F\star H$. Decimos que $H\left(t\right)$ es soluci\'on de esta ecuaci\'on si satisface la ecuaci\'on, y es acotada en intervalos finitos e iguales a cero para $t<0$.
\end{Def}

\begin{Prop}
La funci\'on $U\star h\left(t\right)$ es la \'unica soluci\'on de la ecuaci\'on de renovaci\'on (\ref{Ec.Renovacion}).
\end{Prop}

\begin{Teo}[Teorema Renovaci\'on Elemental]
\begin{eqnarray*}
t^{-1}U\left(t\right)\rightarrow 1/\mu\textrm{,    cuando }t\rightarrow\infty.
\end{eqnarray*}
\end{Teo}


\begin{Note} Una funci\'on $h:\rea_{+}\rightarrow\rea$ es Directamente Riemann Integrable en los siguientes casos:
\begin{itemize}
\item[a)] $h\left(t\right)\geq0$ es decreciente y Riemann Integrable.
\item[b)] $h$ es continua excepto posiblemente en un conjunto de Lebesgue de medida 0, y $|h\left(t\right)|\leq b\left(t\right)$, donde $b$ es DRI.
\end{itemize}
\end{Note}

\begin{Teo}[Teorema Principal de Renovaci\'on]
Si $F$ es no aritm\'etica y $h\left(t\right)$ es Directamente Riemann Integrable (DRI), entonces

\begin{eqnarray*}
lim_{t\rightarrow\infty}U\star h=\frac{1}{\mu}\int_{\rea_{+}}h\left(s\right)ds.
\end{eqnarray*}
\end{Teo}

\begin{Prop}
Cualquier funci\'on $H\left(t\right)$ acotada en intervalos finitos y que es 0 para $t<0$ puede expresarse como
\begin{eqnarray*}
H\left(t\right)=U\star h\left(t\right)\textrm{,  donde }h\left(t\right)=H\left(t\right)-F\star H\left(t\right)
\end{eqnarray*}
\end{Prop}

\begin{Def}
Un proceso estoc\'astico $X\left(t\right)$ es crudamente regenerativo en un tiempo aleatorio positivo $T$ si
\begin{eqnarray*}
\esp\left[X\left(T+t\right)|T\right]=\esp\left[X\left(t\right)\right]\textrm{, para }t\geq0,\end{eqnarray*}
y con las esperanzas anteriores finitas.
\end{Def}

\begin{Prop}
Sup\'ongase que $X\left(t\right)$ es un proceso crudamente regenerativo en $T$, que tiene distribuci\'on $F$. Si $\esp\left[X\left(t\right)\right]$ es acotado en intervalos finitos, entonces
\begin{eqnarray*}
\esp\left[X\left(t\right)\right]=U\star h\left(t\right)\textrm{,  donde }h\left(t\right)=\esp\left[X\left(t\right)\indora\left(T>t\right)\right].
\end{eqnarray*}
\end{Prop}

\begin{Teo}[Regeneraci\'on Cruda]
Sup\'ongase que $X\left(t\right)$ es un proceso con valores positivo crudamente regenerativo en $T$, y def\'inase $M=\sup\left\{|X\left(t\right)|:t\leq T\right\}$. Si $T$ es no aritm\'etico y $M$ y $MT$ tienen media finita, entonces
\begin{eqnarray*}
lim_{t\rightarrow\infty}\esp\left[X\left(t\right)\right]=\frac{1}{\mu}\int_{\rea_{+}}h\left(s\right)ds,
\end{eqnarray*}
donde $h\left(t\right)=\esp\left[X\left(t\right)\indora\left(T>t\right)\right]$.
\end{Teo}


\begin{Note} Una funci\'on $h:\rea_{+}\rightarrow\rea$ es Directamente Riemann Integrable en los siguientes casos:
\begin{itemize}
\item[a)] $h\left(t\right)\geq0$ es decreciente y Riemann Integrable.
\item[b)] $h$ es continua excepto posiblemente en un conjunto de Lebesgue de medida 0, y $|h\left(t\right)|\leq b\left(t\right)$, donde $b$ es DRI.
\end{itemize}
\end{Note}

\begin{Teo}[Teorema Principal de Renovaci\'on]
Si $F$ es no aritm\'etica y $h\left(t\right)$ es Directamente Riemann Integrable (DRI), entonces

\begin{eqnarray*}
lim_{t\rightarrow\infty}U\star h=\frac{1}{\mu}\int_{\rea_{+}}h\left(s\right)ds.
\end{eqnarray*}
\end{Teo}

\begin{Prop}
Cualquier funci\'on $H\left(t\right)$ acotada en intervalos finitos y que es 0 para $t<0$ puede expresarse como
\begin{eqnarray*}
H\left(t\right)=U\star h\left(t\right)\textrm{,  donde }h\left(t\right)=H\left(t\right)-F\star H\left(t\right)
\end{eqnarray*}
\end{Prop}

\begin{Def}
Un proceso estoc\'astico $X\left(t\right)$ es crudamente regenerativo en un tiempo aleatorio positivo $T$ si
\begin{eqnarray*}
\esp\left[X\left(T+t\right)|T\right]=\esp\left[X\left(t\right)\right]\textrm{, para }t\geq0,\end{eqnarray*}
y con las esperanzas anteriores finitas.
\end{Def}

\begin{Prop}
Sup\'ongase que $X\left(t\right)$ es un proceso crudamente regenerativo en $T$, que tiene distribuci\'on $F$. Si $\esp\left[X\left(t\right)\right]$ es acotado en intervalos finitos, entonces
\begin{eqnarray*}
\esp\left[X\left(t\right)\right]=U\star h\left(t\right)\textrm{,  donde }h\left(t\right)=\esp\left[X\left(t\right)\indora\left(T>t\right)\right].
\end{eqnarray*}
\end{Prop}

\begin{Teo}[Regeneraci\'on Cruda]
Sup\'ongase que $X\left(t\right)$ es un proceso con valores positivo crudamente regenerativo en $T$, y def\'inase $M=\sup\left\{|X\left(t\right)|:t\leq T\right\}$. Si $T$ es no aritm\'etico y $M$ y $MT$ tienen media finita, entonces
\begin{eqnarray*}
lim_{t\rightarrow\infty}\esp\left[X\left(t\right)\right]=\frac{1}{\mu}\int_{\rea_{+}}h\left(s\right)ds,
\end{eqnarray*}
donde $h\left(t\right)=\esp\left[X\left(t\right)\indora\left(T>t\right)\right]$.
\end{Teo}

\begin{Def}
Para el proceso $\left\{\left(N\left(t\right),X\left(t\right)\right):t\geq0\right\}$, sus trayectoria muestrales en el intervalo de tiempo $\left[T_{n-1},T_{n}\right)$ est\'an descritas por
\begin{eqnarray*}
\zeta_{n}=\left(\xi_{n},\left\{X\left(T_{n-1}+t\right):0\leq t<\xi_{n}\right\}\right)
\end{eqnarray*}
Este $\zeta_{n}$ es el $n$-\'esimo segmento del proceso. El proceso es regenerativo sobre los tiempos $T_{n}$ si sus segmentos $\zeta_{n}$ son independientes e id\'enticamennte distribuidos.
\end{Def}


\begin{Note}
Si $\tilde{X}\left(t\right)$ con espacio de estados $\tilde{S}$ es regenerativo sobre $T_{n}$, entonces $X\left(t\right)=f\left(\tilde{X}\left(t\right)\right)$ tambi\'en es regenerativo sobre $T_{n}$, para cualquier funci\'on $f:\tilde{S}\rightarrow S$.
\end{Note}

\begin{Note}
Los procesos regenerativos son crudamente regenerativos, pero no al rev\'es.
\end{Note}


\begin{Note}
Un proceso estoc\'astico a tiempo continuo o discreto es regenerativo si existe un proceso de renovaci\'on  tal que los segmentos del proceso entre tiempos de renovaci\'on sucesivos son i.i.d., es decir, para $\left\{X\left(t\right):t\geq0\right\}$ proceso estoc\'astico a tiempo continuo con espacio de estados $S$, espacio m\'etrico.
\end{Note}

Para $\left\{X\left(t\right):t\geq0\right\}$ Proceso Estoc\'astico a tiempo continuo con estado de espacios $S$, que es un espacio m\'etrico, con trayectorias continuas por la derecha y con l\'imites por la izquierda c.s. Sea $N\left(t\right)$ un proceso de renovaci\'on en $\rea_{+}$ definido en el mismo espacio de probabilidad que $X\left(t\right)$, con tiempos de renovaci\'on $T$ y tiempos de inter-renovaci\'on $\xi_{n}=T_{n}-T_{n-1}$, con misma distribuci\'on $F$ de media finita $\mu$.



\begin{Def}
Para el proceso $\left\{\left(N\left(t\right),X\left(t\right)\right):t\geq0\right\}$, sus trayectoria muestrales en el intervalo de tiempo $\left[T_{n-1},T_{n}\right)$ est\'an descritas por
\begin{eqnarray*}
\zeta_{n}=\left(\xi_{n},\left\{X\left(T_{n-1}+t\right):0\leq t<\xi_{n}\right\}\right)
\end{eqnarray*}
Este $\zeta_{n}$ es el $n$-\'esimo segmento del proceso. El proceso es regenerativo sobre los tiempos $T_{n}$ si sus segmentos $\zeta_{n}$ son independientes e id\'enticamennte distribuidos.
\end{Def}

\begin{Note}
Un proceso regenerativo con media de la longitud de ciclo finita es llamado positivo recurrente.
\end{Note}

\begin{Teo}[Procesos Regenerativos]
Suponga que el proceso
\end{Teo}


\begin{Def}[Renewal Process Trinity]
Para un proceso de renovaci\'on $N\left(t\right)$, los siguientes procesos proveen de informaci\'on sobre los tiempos de renovaci\'on.
\begin{itemize}
\item $A\left(t\right)=t-T_{N\left(t\right)}$, el tiempo de recurrencia hacia atr\'as al tiempo $t$, que es el tiempo desde la \'ultima renovaci\'on para $t$.

\item $B\left(t\right)=T_{N\left(t\right)+1}-t$, el tiempo de recurrencia hacia adelante al tiempo $t$, residual del tiempo de renovaci\'on, que es el tiempo para la pr\'oxima renovaci\'on despu\'es de $t$.

\item $L\left(t\right)=\xi_{N\left(t\right)+1}=A\left(t\right)+B\left(t\right)$, la longitud del intervalo de renovaci\'on que contiene a $t$.
\end{itemize}
\end{Def}

\begin{Note}
El proceso tridimensional $\left(A\left(t\right),B\left(t\right),L\left(t\right)\right)$ es regenerativo sobre $T_{n}$, y por ende cada proceso lo es. Cada proceso $A\left(t\right)$ y $B\left(t\right)$ son procesos de MArkov a tiempo continuo con trayectorias continuas por partes en el espacio de estados $\rea_{+}$. Una expresi\'on conveniente para su distribuci\'on conjunta es, para $0\leq x<t,y\geq0$
\begin{equation}\label{NoRenovacion}
P\left\{A\left(t\right)>x,B\left(t\right)>y\right\}=
P\left\{N\left(t+y\right)-N\left((t-x)\right)=0\right\}
\end{equation}
\end{Note}

\begin{Ejem}[Tiempos de recurrencia Poisson]
Si $N\left(t\right)$ es un proceso Poisson con tasa $\lambda$, entonces de la expresi\'on (\ref{NoRenovacion}) se tiene que

\begin{eqnarray*}
\begin{array}{lc}
P\left\{A\left(t\right)>x,B\left(t\right)>y\right\}=e^{-\lambda\left(x+y\right)},&0\leq x<t,y\geq0,
\end{array}
\end{eqnarray*}
que es la probabilidad Poisson de no renovaciones en un intervalo de longitud $x+y$.

\end{Ejem}

\begin{Note}
Una cadena de Markov erg\'odica tiene la propiedad de ser estacionaria si la distribuci\'on de su estado al tiempo $0$ es su distribuci\'on estacionaria.
\end{Note}


\begin{Def}
Un proceso estoc\'astico a tiempo continuo $\left\{X\left(t\right):t\geq0\right\}$ en un espacio general es estacionario si sus distribuciones finito dimensionales son invariantes bajo cualquier  traslado: para cada $0\leq s_{1}<s_{2}<\cdots<s_{k}$ y $t\geq0$,
\begin{eqnarray*}
\left(X\left(s_{1}+t\right),\ldots,X\left(s_{k}+t\right)\right)=_{d}\left(X\left(s_{1}\right),\ldots,X\left(s_{k}\right)\right).
\end{eqnarray*}
\end{Def}

\begin{Note}
Un proceso de Markov es estacionario si $X\left(t\right)=_{d}X\left(0\right)$, $t\geq0$.
\end{Note}

Considerese el proceso $N\left(t\right)=\sum_{n}\indora\left(\tau_{n}\leq t\right)$ en $\rea_{+}$, con puntos $0<\tau_{1}<\tau_{2}<\cdots$.

\begin{Prop}
Si $N$ es un proceso puntual estacionario y $\esp\left[N\left(1\right)\right]<\infty$, entonces $\esp\left[N\left(t\right)\right]=t\esp\left[N\left(1\right)\right]$, $t\geq0$

\end{Prop}

\begin{Teo}
Los siguientes enunciados son equivalentes
\begin{itemize}
\item[i)] El proceso retardado de renovaci\'on $N$ es estacionario.

\item[ii)] EL proceso de tiempos de recurrencia hacia adelante $B\left(t\right)$ es estacionario.


\item[iii)] $\esp\left[N\left(t\right)\right]=t/\mu$,


\item[iv)] $G\left(t\right)=F_{e}\left(t\right)=\frac{1}{\mu}\int_{0}^{t}\left[1-F\left(s\right)\right]ds$
\end{itemize}
Cuando estos enunciados son ciertos, $P\left\{B\left(t\right)\leq x\right\}=F_{e}\left(x\right)$, para $t,x\geq0$.

\end{Teo}

\begin{Note}
Una consecuencia del teorema anterior es que el Proceso Poisson es el \'unico proceso sin retardo que es estacionario.
\end{Note}

\begin{Coro}
El proceso de renovaci\'on $N\left(t\right)$ sin retardo, y cuyos tiempos de inter renonaci\'on tienen media finita, es estacionario si y s\'olo si es un proceso Poisson.

\end{Coro}


%________________________________________________________________________
\subsection{Procesos Regenerativos}
%________________________________________________________________________



\begin{Note}
Si $\tilde{X}\left(t\right)$ con espacio de estados $\tilde{S}$ es regenerativo sobre $T_{n}$, entonces $X\left(t\right)=f\left(\tilde{X}\left(t\right)\right)$ tambi\'en es regenerativo sobre $T_{n}$, para cualquier funci\'on $f:\tilde{S}\rightarrow S$.
\end{Note}

\begin{Note}
Los procesos regenerativos son crudamente regenerativos, pero no al rev\'es.
\end{Note}
%\subsection*{Procesos Regenerativos: Sigman\cite{Sigman1}}
\begin{Def}[Definici\'on Cl\'asica]
Un proceso estoc\'astico $X=\left\{X\left(t\right):t\geq0\right\}$ es llamado regenerativo is existe una variable aleatoria $R_{1}>0$ tal que
\begin{itemize}
\item[i)] $\left\{X\left(t+R_{1}\right):t\geq0\right\}$ es independiente de $\left\{\left\{X\left(t\right):t<R_{1}\right\},\right\}$
\item[ii)] $\left\{X\left(t+R_{1}\right):t\geq0\right\}$ es estoc\'asticamente equivalente a $\left\{X\left(t\right):t>0\right\}$
\end{itemize}

Llamamos a $R_{1}$ tiempo de regeneraci\'on, y decimos que $X$ se regenera en este punto.
\end{Def}

$\left\{X\left(t+R_{1}\right)\right\}$ es regenerativo con tiempo de regeneraci\'on $R_{2}$, independiente de $R_{1}$ pero con la misma distribuci\'on que $R_{1}$. Procediendo de esta manera se obtiene una secuencia de variables aleatorias independientes e id\'enticamente distribuidas $\left\{R_{n}\right\}$ llamados longitudes de ciclo. Si definimos a $Z_{k}\equiv R_{1}+R_{2}+\cdots+R_{k}$, se tiene un proceso de renovaci\'on llamado proceso de renovaci\'on encajado para $X$.




\begin{Def}
Para $x$ fijo y para cada $t\geq0$, sea $I_{x}\left(t\right)=1$ si $X\left(t\right)\leq x$,  $I_{x}\left(t\right)=0$ en caso contrario, y def\'inanse los tiempos promedio
\begin{eqnarray*}
\overline{X}&=&lim_{t\rightarrow\infty}\frac{1}{t}\int_{0}^{\infty}X\left(u\right)du\\
\prob\left(X_{\infty}\leq x\right)&=&lim_{t\rightarrow\infty}\frac{1}{t}\int_{0}^{\infty}I_{x}\left(u\right)du,
\end{eqnarray*}
cuando estos l\'imites existan.
\end{Def}

Como consecuencia del teorema de Renovaci\'on-Recompensa, se tiene que el primer l\'imite  existe y es igual a la constante
\begin{eqnarray*}
\overline{X}&=&\frac{\esp\left[\int_{0}^{R_{1}}X\left(t\right)dt\right]}{\esp\left[R_{1}\right]},
\end{eqnarray*}
suponiendo que ambas esperanzas son finitas.

\begin{Note}
\begin{itemize}
\item[a)] Si el proceso regenerativo $X$ es positivo recurrente y tiene trayectorias muestrales no negativas, entonces la ecuaci\'on anterior es v\'alida.
\item[b)] Si $X$ es positivo recurrente regenerativo, podemos construir una \'unica versi\'on estacionaria de este proceso, $X_{e}=\left\{X_{e}\left(t\right)\right\}$, donde $X_{e}$ es un proceso estoc\'astico regenerativo y estrictamente estacionario, con distribuci\'on marginal distribuida como $X_{\infty}$
\end{itemize}
\end{Note}

%________________________________________________________________________
\subsection{Procesos Regenerativos}
%________________________________________________________________________

Para $\left\{X\left(t\right):t\geq0\right\}$ Proceso Estoc\'astico a tiempo continuo con estado de espacios $S$, que es un espacio m\'etrico, con trayectorias continuas por la derecha y con l\'imites por la izquierda c.s. Sea $N\left(t\right)$ un proceso de renovaci\'on en $\rea_{+}$ definido en el mismo espacio de probabilidad que $X\left(t\right)$, con tiempos de renovaci\'on $T$ y tiempos de inter-renovaci\'on $\xi_{n}=T_{n}-T_{n-1}$, con misma distribuci\'on $F$ de media finita $\mu$.



\begin{Def}
Para el proceso $\left\{\left(N\left(t\right),X\left(t\right)\right):t\geq0\right\}$, sus trayectoria muestrales en el intervalo de tiempo $\left[T_{n-1},T_{n}\right)$ est\'an descritas por
\begin{eqnarray*}
\zeta_{n}=\left(\xi_{n},\left\{X\left(T_{n-1}+t\right):0\leq t<\xi_{n}\right\}\right)
\end{eqnarray*}
Este $\zeta_{n}$ es el $n$-\'esimo segmento del proceso. El proceso es regenerativo sobre los tiempos $T_{n}$ si sus segmentos $\zeta_{n}$ son independientes e id\'enticamennte distribuidos.
\end{Def}


\begin{Note}
Si $\tilde{X}\left(t\right)$ con espacio de estados $\tilde{S}$ es regenerativo sobre $T_{n}$, entonces $X\left(t\right)=f\left(\tilde{X}\left(t\right)\right)$ tambi\'en es regenerativo sobre $T_{n}$, para cualquier funci\'on $f:\tilde{S}\rightarrow S$.
\end{Note}

\begin{Note}
Los procesos regenerativos son crudamente regenerativos, pero no al rev\'es.
\end{Note}

\begin{Def}[Definici\'on Cl\'asica]
Un proceso estoc\'astico $X=\left\{X\left(t\right):t\geq0\right\}$ es llamado regenerativo is existe una variable aleatoria $R_{1}>0$ tal que
\begin{itemize}
\item[i)] $\left\{X\left(t+R_{1}\right):t\geq0\right\}$ es independiente de $\left\{\left\{X\left(t\right):t<R_{1}\right\},\right\}$
\item[ii)] $\left\{X\left(t+R_{1}\right):t\geq0\right\}$ es estoc\'asticamente equivalente a $\left\{X\left(t\right):t>0\right\}$
\end{itemize}

Llamamos a $R_{1}$ tiempo de regeneraci\'on, y decimos que $X$ se regenera en este punto.
\end{Def}

$\left\{X\left(t+R_{1}\right)\right\}$ es regenerativo con tiempo de regeneraci\'on $R_{2}$, independiente de $R_{1}$ pero con la misma distribuci\'on que $R_{1}$. Procediendo de esta manera se obtiene una secuencia de variables aleatorias independientes e id\'enticamente distribuidas $\left\{R_{n}\right\}$ llamados longitudes de ciclo. Si definimos a $Z_{k}\equiv R_{1}+R_{2}+\cdots+R_{k}$, se tiene un proceso de renovaci\'on llamado proceso de renovaci\'on encajado para $X$.

\begin{Note}
Un proceso regenerativo con media de la longitud de ciclo finita es llamado positivo recurrente.
\end{Note}


\begin{Def}
Para $x$ fijo y para cada $t\geq0$, sea $I_{x}\left(t\right)=1$ si $X\left(t\right)\leq x$,  $I_{x}\left(t\right)=0$ en caso contrario, y def\'inanse los tiempos promedio
\begin{eqnarray*}
\overline{X}&=&lim_{t\rightarrow\infty}\frac{1}{t}\int_{0}^{\infty}X\left(u\right)du\\
\prob\left(X_{\infty}\leq x\right)&=&lim_{t\rightarrow\infty}\frac{1}{t}\int_{0}^{\infty}I_{x}\left(u\right)du,
\end{eqnarray*}
cuando estos l\'imites existan.
\end{Def}

Como consecuencia del teorema de Renovaci\'on-Recompensa, se tiene que el primer l\'imite  existe y es igual a la constante
\begin{eqnarray*}
\overline{X}&=&\frac{\esp\left[\int_{0}^{R_{1}}X\left(t\right)dt\right]}{\esp\left[R_{1}\right]},
\end{eqnarray*}
suponiendo que ambas esperanzas son finitas.

\begin{Note}
\begin{itemize}
\item[a)] Si el proceso regenerativo $X$ es positivo recurrente y tiene trayectorias muestrales no negativas, entonces la ecuaci\'on anterior es v\'alida.
\item[b)] Si $X$ es positivo recurrente regenerativo, podemos construir una \'unica versi\'on estacionaria de este proceso, $X_{e}=\left\{X_{e}\left(t\right)\right\}$, donde $X_{e}$ es un proceso estoc\'astico regenerativo y estrictamente estacionario, con distribuci\'on marginal distribuida como $X_{\infty}$
\end{itemize}
\end{Note}





