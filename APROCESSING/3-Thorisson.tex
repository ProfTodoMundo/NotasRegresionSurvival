\documentclass{article}
\usepackage[utf8]{inputenc}
\usepackage[spanish,english]{babel}
\usepackage{amsmath,amssymb,amsthm,amsfonts}
\usepackage{geometry}
\usepackage{hyperref}
\usepackage{fancyhdr}
\usepackage{titlesec}
\usepackage{listings}
\usepackage{graphicx,graphics}
\usepackage{multicol}
\usepackage{multirow}
\usepackage{color}
\usepackage{float} 
\usepackage{subfig}
\usepackage[figuresright]{rotating}
\usepackage{enumerate}
\usepackage{anysize} 
\usepackage{url}

\title{Thorisson: Revisi\'on}
\author{Carlos E. Martínez-Rodríguez}
\date{Julio 2024}

\geometry{
  a4paper,
  left=25mm,
  right=25mm,
  top=30mm,
  bottom=30mm,
}

% Configuración de encabezados y pies de página
\pagestyle{fancy}
\fancyhf{}
\fancyhead[L]{\leftmark}
\fancyfoot[C]{\thepage}
\fancyfoot[R]{\rightmark}
\fancyfoot[L]{Carlos E. Martínez-Rodríguez}

% Definiciones de nuevos entornos
\newtheorem{Algthm}{Algoritmo}
\newtheorem{Def}{Definición}
\newtheorem{Ejem}{Ejemplo}
\newtheorem{Teo}{Teorema}
\newtheorem{Dem}{Demostración}
\newtheorem{Note}{Nota}
\newtheorem{Sol}{Solución}
\newtheorem{Prop}{Proposición}
\newtheorem{Cor}{Corolario}
\newtheorem{Col}{Corolario}
\newtheorem{Coro}{Corolario}
\newtheorem{Lemma}{Lema}
\newtheorem{Lem}{Lema}
\newtheorem{Lema}{Lema}
\newtheorem{Sup}{Supuestos}
\newtheorem{Assumption}{Supuestos}
\newtheorem{Remark}{Observación}
\newtheorem{Condition}{Condición}
\newtheorem{Theorem}{Teorema}
\newtheorem{Corollary}{Corolario}
\newtheorem{Ejemplo}{Ejemplo}
\newtheorem{Example}{Ejemplo}
\newtheorem{Obs}{Observación}

% Nuevos comandos
\def\RR{\mathbb{R}}
\def\ZZ{\mathbb{Z}}
\newcommand{\nat}{\mathbb{N}}
\newcommand{\ent}{\mathbb{Z}}
\newcommand{\rea}{\mathbb{R}}
\newcommand{\Eb}{\mathbf{E}}
\newcommand{\esp}{\mathbb{E}}
\newcommand{\prob}{\mathbb{P}}
\newcommand{\indora}{\mbox{$1$\hspace{-0.8ex}$1$}}
\newcommand{\ER}{\left(E,\mathcal{E}\right)}
\newcommand{\KM}{\left(P_{s,t}\right)}
\newcommand{\Xt}{\left(X_{t}\right)_{t\in I}}
\newcommand{\PE}{\left(X_{t}\right)_{t\in I}}
\newcommand{\SG}{\left(P_{t}\right)}
\newcommand{\CM}{\mathbf{P}^{x}}
\newcommand\mypar{\par\vspace{\baselineskip}}

\begin{document}

\maketitle

\tableofcontents
%<>===<>==<>===<>==<>===<>==<>===<>==<>===<>==<>===<>==<>===<>==<>===<>==<>===


\begin{thebibliography}{99}

\bibitem{ISL}
James, G., Witten, D., Hastie, T., and Tibshirani, R. (2013). \textit{An Introduction to Statistical Learning: with Applications in R}. Springer.

\bibitem{Logistic}
Hosmer, D. W., Lemeshow, S., and Sturdivant, R. X. (2013). \textit{Applied Logistic Regression} (3rd ed.). Wiley.

\bibitem{PatternRecognition}
Bishop, C. M. (2006). \textit{Pattern Recognition and Machine Learning}. Springer.

\bibitem{Harrell}
Harrell, F. E. (2015). \textit{Regression Modeling Strategies: With Applications to Linear Models, Logistic and Ordinal Regression, and Survival Analysis}. Springer.

\bibitem{RDocumentation}
R Documentation and Tutorials: \url{https://cran.r-project.org/manuals.html}

\bibitem{RBlogger}
Tutorials on R-bloggers: \url{https://www.r-bloggers.com/}

\bibitem{CourseraML}
Coursera: \textit{Machine Learning} by Andrew Ng.

\bibitem{edXDS}
edX: \textit{Data Science and Machine Learning Essentials} by Microsoft.

% Libros adicionales
\bibitem{Ross}
Ross, S. M. (2014). \textit{Introduction to Probability and Statistics for Engineers and Scientists}. Academic Press.

\bibitem{DeGroot}
DeGroot, M. H., and Schervish, M. J. (2012). \textit{Probability and Statistics} (4th ed.). Pearson.

\bibitem{Hogg}
Hogg, R. V., McKean, J., and Craig, A. T. (2019). \textit{Introduction to Mathematical Statistics} (8th ed.). Pearson.

\bibitem{Kleinbaum}
Kleinbaum, D. G., and Klein, M. (2010). \textit{Logistic Regression: A Self-Learning Text} (3rd ed.). Springer.

% Artículos y tutoriales adicionales
\bibitem{Wasserman}
Wasserman, L. (2004). \textit{All of Statistics: A Concise Course in Statistical Inference}. Springer.

\bibitem{KhanAcademy}
Probability and Statistics Tutorials on Khan Academy: \url{https://www.khanacademy.org/math/statistics-probability}

\bibitem{OnlineStatBook}
Online Statistics Education: \url{http://onlinestatbook.com/}

\bibitem{Peng}
Peng, C. Y. J., Lee, K. L., and Ingersoll, G. M. (2002). \textit{An Introduction to Logistic Regression Analysis and Reporting}. The Journal of Educational Research.

\bibitem{Agresti}
Agresti, A. (2007). \textit{An Introduction to Categorical Data Analysis} (2nd ed.). Wiley.

\bibitem{Han}
Han, J., Pei, J., and Kamber, M. (2011). \textit{Data Mining: Concepts and Techniques}. Morgan Kaufmann.

\bibitem{TowardsDataScience}
Data Cleaning and Preprocessing on Towards Data Science: \url{https://towardsdatascience.com/data-cleaning-and-preprocessing}

\bibitem{Molinaro}
Molinaro, A. M., Simon, R., and Pfeiffer, R. M. (2005). \textit{Prediction error estimation: a comparison of resampling methods}. Bioinformatics.

\bibitem{EvaluatingModels}
Evaluating Machine Learning Models on Towards Data Science: \url{https://towardsdatascience.com/evaluating-machine-learning-models}

\bibitem{LogisticRegressionGuide}
Practical Guide to Logistic Regression in R on Towards Data Science: \url{https://towardsdatascience.com/practical-guide-to-logistic-regression-in-r}

% Cursos en línea adicionales
\bibitem{CourseraStatistics}
Coursera: \textit{Statistics with R} by Duke University.

\bibitem{edXProbability}
edX: \textit{Data Science: Probability} by Harvard University.

\bibitem{CourseraLogistic}
Coursera: \textit{Logistic Regression} by Stanford University.

\bibitem{edXInference}
edX: \textit{Data Science: Inference and Modeling} by Harvard University.

\bibitem{CourseraWrangling}
Coursera: \textit{Data Science: Wrangling and Cleaning} by Johns Hopkins University.

\bibitem{edXRBasics}
edX: \textit{Data Science: R Basics} by Harvard University.

\bibitem{CourseraRegression}
Coursera: \textit{Regression Models} by Johns Hopkins University.

\bibitem{edXStatInference}
edX: \textit{Data Science: Statistical Inference} by Harvard University.

\bibitem{SurvivalAnalysis}
An Introduction to Survival Analysis on Towards Data Science: \url{https://towardsdatascience.com/an-introduction-to-survival-analysis}

\bibitem{MultinomialLogistic}
Multinomial Logistic Regression on DataCamp: \url{https://www.datacamp.com/community/tutorials/multinomial-logistic-regression-R}

\bibitem{CourseraSurvival}
Coursera: \textit{Survival Analysis} by Johns Hopkins University.

\bibitem{edXHighthroughput}
edX: \textit{Data Science: Statistical Inference and Modeling for High-throughput Experiments} by Harvard University.

\end{thebibliography}


\end{document}
